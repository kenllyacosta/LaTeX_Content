\RequirePackage{hyphsubst}
\documentclass[fontsize=11pt,paper=A4,BCOR=12mm,DIV=13,open=any,listof=totoc]{scrbook}
\input{../headers/paper}
% Standard für Formatierung
%\usepackage[utf8]{inputenc} % use \usepackage[utf8]{inputenc} for tex4ht
\usepackage[usenames]{color}
\usepackage{textcomp} 
\usepackage{parskip} 
\usepackage[normalem]{ulem}
\usepackage[unicode=true]{hyperref}
\usepackage{tocstyle}
\usepackage[defblank]{paralist}
\usepackage{trace}
% Minitoc
%\usepackage{minitoc}

% Keystroke
\usepackage{keystroke}



\input{../headers/babel}
\input{../headers/svg}
\input{../headers/packages2}
\input{../headers/defaultcolors}
\input{../headers/hyphenation}
\input{../headers/commands}
\publishers{en.wikibooks.org}
\mymaintitle{LaTeX
}
% Festlegungen für minitoc
% \renewcommand{\myminitoc}{\minitoc}
% \renewcommand{\mtctitle}{Überblick}
% \setcounter{minitocdepth}{1}
% \dominitoc   % diese Zeile aktiviert das Erstellen der minitocs, sie muss vor \tableofcontents kommen

% Seitenformat
% ------------
%\KOMAoption{paper}{A5}          % zulässig: letter, legal, executive; A-, B-, C-, D-Reihen
\KOMAoption{open}{right}			% zulässig: right (jedes Kapitel beginnt rechts), left, any
\KOMAoption{numbers}{auto}
% Satzspiegel jetzt neu berechnen, damit er bei Kopf- und Fußzeilen beachtet wird
\KOMAoptions{DIV=13}

% Kopf- und Fusszeilen
% --------------------
% Breite und Trennlinie
%\setheadwidth[-6mm]{textwithmarginpar}
%\setheadsepline[textwithmarginpar]{0.4pt}
\setheadwidth{text}
\setheadsepline[text]{0.4pt}

% Variante 1: Kopf: links Kapitel, rechts Abschnitt (ohne Nummer); Fuß: außen die Seitenzahl
\ohead{\headmark}
\renewcommand{\chaptermark}[1]{\markleft{#1}{}}
\renewcommand{\sectionmark}[1]{\markright{#1}{}}
\ofoot[\pagemark]{\pagemark}

% Variante 2: Kopf außen die Seitenzahl, Fuß nichts
%\ohead{\pagemark}
%\ofoot{}

% Standardschriften
% -----------------
%\KOMAoption{fontsize}{18pt}
\addtokomafont{disposition}{\rmfamily}
\addtokomafont{title}{\rmfamily} 
\setkomafont{pageheadfoot}{\normalfont\rmfamily\mdseries}

% vertikaler Ausgleich
% -------------------- 
% nein -> \raggedbottom
% ja   -> \flushbottom    aber ungeeignet bei Fußnoten
%\raggedbottom
\flushbottom

% Tiefe des Inhaltsverzeichnisses bestimmen
% -----------------------------------------
% -1   nur \part{}
%  0   bis \chapter{}
%  1   bis \section{}
%  2   bis \subsection{} usw.
\newcommand{\mytocdepth}{1}

% mypart - Teile des Buches und Inhaltsverzeichnis
% ------------------------------------------------
% Standard: nur im Inhaltsverzeichnis, zusätzlicher Eintrag ohne Seitenzahl
% Variante: nur im Inhaltsverzeichnis, zusätzlicher Eintrag mit Seitenzahl 
%\renewcommand{\mypart}[1]{\addcontentsline{toc}{part}{#1}}
% Variante: mit eigener Seite vor dem ersten Kapitel, mit Eintrag und Seitenzahl im Inhaltsverzeichnis
\renewcommand{\mypart}[1]{\part{#1}}


% maketitle
% -----------------------------------------------
% Bestandteile des Innentitels
%\title{Einführung in SQL}
%\author{Jürgen Thomas}
%\subtitle{Datenbanken bearbeiten}
\date{}
% Bestandteile von Impressum und CR
% Bestandteile von Impressum und CR

\uppertitleback{
%Detaillierte Daten zu dieser Publikation sind bei Wikibooks zu erhalten:\newline{} \url{http://de.wikibooks.org/}
%Diese Publikation ist bei der Deutschen Nationalbibliothek registriert. Detaillierte Daten sind im Internet  zu erhalten: \newline{}\url{https://portal.d-nb.de/opac.htm?method=showSearchForm#top}
%Diese Publikation ist bei der Deutschen Nationalbibliothek registriert. Detaillierte Daten sind im Internet unter der Katalog-Nr. 1008575860 zu erhalten: \newline{}\url{http://d-nb.info/1008575860}

%Namen von Programmen und Produkten sowie sonstige Angaben sind häufig geschützt. Da es auch freie Bezeichnungen gibt, wird das Symbol \textregistered{} nicht verwendet.

%Erstellt am 
\today{}
}

\lowertitleback{
{\footnotesize
On the 28th of April 2012 the contents of the English as well as German Wikibooks and Wikipedia projects were licensed under Creative Commons Attribution-ShareAlike 3.0 Unported license.
A URI to this license is given in the list of figures on page \pageref{ListOfFigures}.
If this document is a derived work from the contents of one of these projects and the content was still licensed by the project under this license at the time of derivation this document has to be licensed under the same, a similar or a compatible license, as stated in section 4b of the license.
The list of contributors is included in chapter Contributors on page \pageref{Contributors}.
The licenses GPL, LGPL and GFDL are included in chapter Licenses on page \pageref{Licenses}, since this book and/or parts of it may or may not be licensed under one or more of these licenses, and thus require inclusion of these licenses.
The licenses of the figures are given in the list of figures on page \pageref{ListOfFigures}.
This PDF was generated by the \LaTeX{} typesetting  software.
The \LaTeX{} source code is included as an attachment ({\tt source.7z.txt}) in this PDF file.
To extract the source from the PDF file, you can use the \texttt{pdfdetach} tool
including in the \texttt{poppler} suite, or the
\url{http://www.pdflabs.com/tools/pdftk-the-pdf-toolkit/} utility.
Some PDF viewers may also let you save the attachment to a file.
After extracting it from the PDF file you have to rename it to {\tt source.7z}.
To uncompress the resulting archive we recommend the use of \url{http://www.7-zip.org/}.
The \LaTeX{} source itself was generated by a program written by Dirk Hünniger, which is freely available under an open source license from \url{http://de.wikibooks.org/wiki/Benutzer:Dirk_Huenniger/wb2pdf}.
}}


\renewcommand{\mysubtitle}[1]{}
\renewcommand{\mymaintitle}[1]{}
\renewcommand{\myauthor}[1]{}


\newenvironment{myshaded}{%
  \def\FrameCommand{ \hskip-2pt \fboxsep=\FrameSep \colorbox{shadecolor}}%
  \MakeFramed {\advance\hsize-\width \FrameRestore}}%
 {\endMakeFramed}


\input{../headers/formattings}
\input{../headers/unicodes}
\input{../headers/templates}
\input{../headers/templates-dirk}
\input{../headers/templates-chemie}

\usepackage{lmodern}
\usepackage{xltxtra}
\usepackage{fontspec}

\setmainfont[Path=/usr/share/fonts/truetype/cmu/,UprightFont=cmunrm.ttf,BoldFont=cmunbx.ttf,ItalicFont=cmunti.ttf,BoldItalicFont=cmunbi.ttf]{cmunbx.ttf}
\setmonofont[Path=/usr/share/fonts/truetype/cmu/,UprightFont=cmuntt.ttf,BoldFont=cmuntb.ttf,ItalicFont=cmunit.ttf,BoldItalicFont=cmuntx.ttf]{cmunbx.ttf}

\begin{document}
\allowdisplaybreaks
\usetocstyle{standard}
\raggedbottom
\thispagestyle{empty}
\pagestyle{empty}
%\include{coverfrontpage}

%\cleardoublepage
\pagenumbering{Roman}
\maketitle
\pagestyle{scrheadings}

\setcounter{tocdepth}{\mytocdepth}
\tableofcontents 

%\cleardoublepage
\pagenumbering{arabic}

%\include{kap-vorwort}


\mymaintitle{LaTeX
}




\mypart{Getting Started}\chapter{Introduction}

\myminitoc
\label{0}

\label{1}

\section{What is TeX?}
\label{2}

{\bfseries \myhref{https://en.wikibooks.org/wiki/TeX}{\setmainfont[Path=/usr/share/fonts/truetype/cmu/,UprightFont=cmunrm.ttf,BoldFont=cmunbx.ttf,ItalicFont=cmunti.ttf,BoldItalicFont=cmunbi.ttf]{cmunbx.ttf}\setmonofont[Path=/usr/share/fonts/truetype/cmu/,UprightFont=cmuntt.ttf,BoldFont=cmuntb.ttf,ItalicFont=cmunit.ttf,BoldItalicFont=cmuntx.ttf]{cmunbx.ttf}\bfseries TeX}} is a low-{}level markup and programming language created by \myhref{https://en.wikipedia.org/wiki/Donald\%20Knuth}{Donald Knuth} to typeset documents attractively and consistently. Knuth started writing the TeX typesetting engine in 1977 to explore the potential of the digital printing equipment that was beginning to infiltrate the publishing industry at that time, especially in the hope that he could reverse the trend of deteriorating typographical quality that he saw affecting his own books and articles. With the release of 8-{}bit 
character support in 1989, TeX development has been essentially frozen with only bug fixes released periodically. TeX is a programming language in the sense that it supports the if-{}else construct: you can make calculations with it (that are performed while compiling the document), etc., but you would find it very hard to do anything else but typesetting with it. The fine control TeX offers over document structure and formatting makes it a powerful—and formidable—tool. TeX is renowned for being extremely stable, for running on many different kinds of computers, and for being virtually bug free. The version numbers of TeX are converging toward {$\pi$}, with a current version number of 3.1415926.  

The name TeX is intended by its developer to be \LaTeXIdentityTemplate{/\textquotesingle{}tɛx/}, with the final consonant of {\itshape \setmainfont[Path=/usr/share/fonts/truetype/cmu/,UprightFont=cmunrm.ttf,BoldFont=cmunbx.ttf,ItalicFont=cmunti.ttf,BoldItalicFont=cmunbi.ttf]{cmunti.ttf}\setmonofont[Path=/usr/share/fonts/truetype/cmu/,UprightFont=cmuntt.ttf,BoldFont=cmuntb.ttf,ItalicFont=cmunit.ttf,BoldItalicFont=cmuntx.ttf]{cmunti.ttf}\itshape loch}{$\text{ }$}\setmainfont[Path=/usr/share/fonts/truetype/cmu/,UprightFont=cmunrm.ttf,BoldFont=cmunbx.ttf,ItalicFont=cmunti.ttf,BoldItalicFont=cmunbi.ttf]{cmunrm.ttf}\setmonofont[Path=/usr/share/fonts/truetype/cmu/,UprightFont=cmuntt.ttf,BoldFont=cmuntb.ttf,ItalicFont=cmunit.ttf,BoldItalicFont=cmuntx.ttf]{cmunrm.ttf} or {\itshape \setmainfont[Path=/usr/share/fonts/truetype/cmu/,UprightFont=cmunrm.ttf,BoldFont=cmunbx.ttf,ItalicFont=cmunti.ttf,BoldItalicFont=cmunbi.ttf]{cmunti.ttf}\setmonofont[Path=/usr/share/fonts/truetype/cmu/,UprightFont=cmuntt.ttf,BoldFont=cmuntb.ttf,ItalicFont=cmunit.ttf,BoldItalicFont=cmuntx.ttf]{cmunti.ttf}\itshape Bach}\setmainfont[Path=/usr/share/fonts/truetype/cmu/,UprightFont=cmunrm.ttf,BoldFont=cmunbx.ttf,ItalicFont=cmunti.ttf,BoldItalicFont=cmunbi.ttf]{cmunrm.ttf}\setmonofont[Path=/usr/share/fonts/truetype/cmu/,UprightFont=cmuntt.ttf,BoldFont=cmuntb.ttf,ItalicFont=cmunit.ttf,BoldItalicFont=cmuntx.ttf]{cmunrm.ttf}. (Donald E. Knuth, {\itshape \setmainfont[Path=/usr/share/fonts/truetype/cmu/,UprightFont=cmunrm.ttf,BoldFont=cmunbx.ttf,ItalicFont=cmunti.ttf,BoldItalicFont=cmunbi.ttf]{cmunti.ttf}\setmonofont[Path=/usr/share/fonts/truetype/cmu/,UprightFont=cmuntt.ttf,BoldFont=cmuntb.ttf,ItalicFont=cmunit.ttf,BoldItalicFont=cmuntx.ttf]{cmunti.ttf}\itshape The TeXbook}\setmainfont[Path=/usr/share/fonts/truetype/cmu/,UprightFont=cmunrm.ttf,BoldFont=cmunbx.ttf,ItalicFont=cmunti.ttf,BoldItalicFont=cmunbi.ttf]{cmunrm.ttf}\setmonofont[Path=/usr/share/fonts/truetype/cmu/,UprightFont=cmuntt.ttf,BoldFont=cmuntb.ttf,ItalicFont=cmunit.ttf,BoldItalicFont=cmuntx.ttf]{cmunrm.ttf}) The letters of the name are meant to represent the capital Greek letters tau, epsilon, and chi, as TeX is an abbreviation of {\itshape \setmainfont[Path=/usr/share/fonts/truetype/cmu/,UprightFont=cmunrm.ttf,BoldFont=cmunbx.ttf,ItalicFont=cmunti.ttf,BoldItalicFont=cmunbi.ttf]{cmunti.ttf}\setmonofont[Path=/usr/share/fonts/truetype/cmu/,UprightFont=cmuntt.ttf,BoldFont=cmuntb.ttf,ItalicFont=cmunit.ttf,BoldItalicFont=cmuntx.ttf]{cmunti.ttf}\itshape τέχνη}{$\text{ }$}\setmainfont[Path=/usr/share/fonts/truetype/cmu/,UprightFont=cmunrm.ttf,BoldFont=cmunbx.ttf,ItalicFont=cmunti.ttf,BoldItalicFont=cmunbi.ttf]{cmunrm.ttf}\setmonofont[Path=/usr/share/fonts/truetype/cmu/,UprightFont=cmuntt.ttf,BoldFont=cmuntb.ttf,ItalicFont=cmunit.ttf,BoldItalicFont=cmuntx.ttf]{cmunrm.ttf} (ΤΕΧΝΗ – technē), Greek for both \symbol{34}art\symbol{34} and \symbol{34}craft\symbol{34}, which is also the root word of {\itshape \setmainfont[Path=/usr/share/fonts/truetype/cmu/,UprightFont=cmunrm.ttf,BoldFont=cmunbx.ttf,ItalicFont=cmunti.ttf,BoldItalicFont=cmunbi.ttf]{cmunti.ttf}\setmonofont[Path=/usr/share/fonts/truetype/cmu/,UprightFont=cmuntt.ttf,BoldFont=cmuntb.ttf,ItalicFont=cmunit.ttf,BoldItalicFont=cmuntx.ttf]{cmunti.ttf}\itshape technical}\setmainfont[Path=/usr/share/fonts/truetype/cmu/,UprightFont=cmunrm.ttf,BoldFont=cmunbx.ttf,ItalicFont=cmunti.ttf,BoldItalicFont=cmunbi.ttf]{cmunrm.ttf}\setmonofont[Path=/usr/share/fonts/truetype/cmu/,UprightFont=cmuntt.ttf,BoldFont=cmuntb.ttf,ItalicFont=cmunit.ttf,BoldItalicFont=cmuntx.ttf]{cmunrm.ttf}. English speakers often pronounce it \LaTeXIdentityTemplate{/\textquotesingle{}tɛk/}, like the first syllable of {\itshape \setmainfont[Path=/usr/share/fonts/truetype/cmu/,UprightFont=cmunrm.ttf,BoldFont=cmunbx.ttf,ItalicFont=cmunti.ttf,BoldItalicFont=cmunbi.ttf]{cmunti.ttf}\setmonofont[Path=/usr/share/fonts/truetype/cmu/,UprightFont=cmuntt.ttf,BoldFont=cmuntb.ttf,ItalicFont=cmunit.ttf,BoldItalicFont=cmuntx.ttf]{cmunti.ttf}\itshape technical}\setmainfont[Path=/usr/share/fonts/truetype/cmu/,UprightFont=cmunrm.ttf,BoldFont=cmunbx.ttf,ItalicFont=cmunti.ttf,BoldItalicFont=cmunbi.ttf]{cmunrm.ttf}\setmonofont[Path=/usr/share/fonts/truetype/cmu/,UprightFont=cmuntt.ttf,BoldFont=cmuntb.ttf,ItalicFont=cmunit.ttf,BoldItalicFont=cmuntx.ttf]{cmunrm.ttf}.

Programming in TeX generally progresses along a very gradual learning curve, requiring a significant investment of time to build custom macros for text formatting. Fortunately, document preparation systems based on TeX, consisting of collections of pre-{}built macros, do exist. These pre-{}built macros are time saving, and automate certain repetitive tasks and help reduce user introduced errors; however, this convenience comes at the cost of complete design flexibility. One of the most popular macro packages is called {\bfseries \setmainfont[Path=/usr/share/fonts/truetype/cmu/,UprightFont=cmunrm.ttf,BoldFont=cmunbx.ttf,ItalicFont=cmunti.ttf,BoldItalicFont=cmunbi.ttf]{cmunbx.ttf}\setmonofont[Path=/usr/share/fonts/truetype/cmu/,UprightFont=cmuntt.ttf,BoldFont=cmuntb.ttf,ItalicFont=cmunit.ttf,BoldItalicFont=cmuntx.ttf]{cmunbx.ttf}\bfseries LaTeX}\setmainfont[Path=/usr/share/fonts/truetype/cmu/,UprightFont=cmunrm.ttf,BoldFont=cmunbx.ttf,ItalicFont=cmunti.ttf,BoldItalicFont=cmunbi.ttf]{cmunrm.ttf}\setmonofont[Path=/usr/share/fonts/truetype/cmu/,UprightFont=cmuntt.ttf,BoldFont=cmuntb.ttf,ItalicFont=cmunit.ttf,BoldItalicFont=cmuntx.ttf]{cmunrm.ttf}.
\section{What is LaTeX?}
\label{3}

{\bfseries \setmainfont[Path=/usr/share/fonts/truetype/cmu/,UprightFont=cmunrm.ttf,BoldFont=cmunbx.ttf,ItalicFont=cmunti.ttf,BoldItalicFont=cmunbi.ttf]{cmunbx.ttf}\setmonofont[Path=/usr/share/fonts/truetype/cmu/,UprightFont=cmuntt.ttf,BoldFont=cmuntb.ttf,ItalicFont=cmunit.ttf,BoldItalicFont=cmuntx.ttf]{cmunbx.ttf}\bfseries LaTeX}{$\text{ }$}\setmainfont[Path=/usr/share/fonts/truetype/cmu/,UprightFont=cmunrm.ttf,BoldFont=cmunbx.ttf,ItalicFont=cmunti.ttf,BoldItalicFont=cmunbi.ttf]{cmunrm.ttf}\setmonofont[Path=/usr/share/fonts/truetype/cmu/,UprightFont=cmuntt.ttf,BoldFont=cmuntb.ttf,ItalicFont=cmunit.ttf,BoldItalicFont=cmuntx.ttf]{cmunrm.ttf} (pronounced either \symbol{34}Lah-{}tech\symbol{34} or \symbol{34}Lay-{}tech\symbol{34}) is a macro package based on TeX created by \myhref{https://en.wikipedia.org/wiki/Leslie\%20Lamport}{Leslie Lamport}. Its purpose is to simplify TeX typesetting, especially for documents containing mathematical formulae. Within the typesetting system, its name is formatted as \LatexSymbol{}.

Many later authors have contributed extensions, called {\itshape \setmainfont[Path=/usr/share/fonts/truetype/cmu/,UprightFont=cmunrm.ttf,BoldFont=cmunbx.ttf,ItalicFont=cmunti.ttf,BoldItalicFont=cmunbi.ttf]{cmunti.ttf}\setmonofont[Path=/usr/share/fonts/truetype/cmu/,UprightFont=cmuntt.ttf,BoldFont=cmuntb.ttf,ItalicFont=cmunit.ttf,BoldItalicFont=cmuntx.ttf]{cmunti.ttf}\itshape packages}{$\text{ }$}\setmainfont[Path=/usr/share/fonts/truetype/cmu/,UprightFont=cmunrm.ttf,BoldFont=cmunbx.ttf,ItalicFont=cmunti.ttf,BoldItalicFont=cmunbi.ttf]{cmunrm.ttf}\setmonofont[Path=/usr/share/fonts/truetype/cmu/,UprightFont=cmuntt.ttf,BoldFont=cmuntb.ttf,ItalicFont=cmunit.ttf,BoldItalicFont=cmuntx.ttf]{cmunrm.ttf} or {\itshape \setmainfont[Path=/usr/share/fonts/truetype/cmu/,UprightFont=cmunrm.ttf,BoldFont=cmunbx.ttf,ItalicFont=cmunti.ttf,BoldItalicFont=cmunbi.ttf]{cmunti.ttf}\setmonofont[Path=/usr/share/fonts/truetype/cmu/,UprightFont=cmuntt.ttf,BoldFont=cmuntb.ttf,ItalicFont=cmunit.ttf,BoldItalicFont=cmuntx.ttf]{cmunti.ttf}\itshape styles}\setmainfont[Path=/usr/share/fonts/truetype/cmu/,UprightFont=cmunrm.ttf,BoldFont=cmunbx.ttf,ItalicFont=cmunti.ttf,BoldItalicFont=cmunbi.ttf]{cmunrm.ttf}\setmonofont[Path=/usr/share/fonts/truetype/cmu/,UprightFont=cmuntt.ttf,BoldFont=cmuntb.ttf,ItalicFont=cmunit.ttf,BoldItalicFont=cmuntx.ttf]{cmunrm.ttf}, to LaTeX. Some of these are bundled with most TeX/LaTeX software distributions; more can be found in the Comprehensive TeX Archive Network (\myhref{http://www.ctan.org}{CTAN}).

Since LaTeX comprises a group of \myhref{https://en.wikibooks.org/wiki/TeX}{TeX} commands, LaTeX document processing is essentially programming. You create a text file in LaTeX markup, which LaTeX reads to produce the final document.

This approach has some disadvantages in comparison with a \myhref{https://en.wikipedia.org/wiki/WYSIWYG}{WYSIWYG} (What You See Is What You Get) program such as \myhref{https://en.wikipedia.org/wiki/Openoffice.org}{Openoffice.org} Writer or \myhref{https://en.wikipedia.org/wiki/Microsoft\%20Word}{Microsoft Word}.

In LaTeX:

\begin{myitemize}
\item{}  You don\textquotesingle{}t (usually) see the final version of the document when editing it.
\item{}  You generally need to know the necessary commands for LaTeX markup.
\item{}  It can sometimes be difficult to obtain a certain look for the document.
\end{myitemize}


On the other hand, there are certain advantages to the LaTeX approach:

\begin{myitemize}
\item{}  Document sources can be read with any text editor and understood, unlike the complex binary and \myhref{https://en.wikipedia.org/wiki/XML}{XML} formats used with WYSIWYG programs.
\item{}  You can concentrate purely on the structure and contents of the document, not get caught up with superficial layout issues.
\item{}  You don\textquotesingle{}t need to manually adjust fonts, text sizes, line heights, or text flow for readability, as LaTeX takes care of them automatically.
\item{}  In LaTeX the document structure is visible to the user, and can be easily copied to another document. In WYSIWYG applications it is often not obvious how a certain formatting was produced, and it might be impossible to copy it directly for use in another document.
\item{}  The layout, fonts, tables and so on are consistent throughout the document.
\item{}  Mathematical formulae can be easily typeset.
\item{}  Indexes, footnotes, citations and references are generated easily.
\item{}  Since the document source is plain text, tables, figures, equations, etc. can be generated programmatically with any language.
\item{}  You are forced to structure your documents correctly.
\end{myitemize}


The LaTeX document is a plain text file containing the content of the document, with additional markup.  When the source file is processed by the macro package, it can produce documents in several formats. LaTeX natively supports \myhref{https://en.wikipedia.org/wiki/DVI\%20file\%20format}{DVI} and PDF, but by using other software you can easily create PostScript, PNG, JPEG, etc.
\section{Philosophy of use}
\label{4}\subsection{Flexibility and modularity}
\label{5}

One of the most frustrating things beginners and even advanced users might encounter using LaTeX is the lack of flexibility regarding the document design and layout. If you want to design your document in a very specific way, you may have trouble accomplishing this. Keep in mind that LaTeX {\itshape \setmainfont[Path=/usr/share/fonts/truetype/cmu/,UprightFont=cmunrm.ttf,BoldFont=cmunbx.ttf,ItalicFont=cmunti.ttf,BoldItalicFont=cmunbi.ttf]{cmunti.ttf}\setmonofont[Path=/usr/share/fonts/truetype/cmu/,UprightFont=cmuntt.ttf,BoldFont=cmuntb.ttf,ItalicFont=cmunit.ttf,BoldItalicFont=cmuntx.ttf]{cmunti.ttf}\itshape does}{$\text{ }$}\setmainfont[Path=/usr/share/fonts/truetype/cmu/,UprightFont=cmunrm.ttf,BoldFont=cmunbx.ttf,ItalicFont=cmunti.ttf,BoldItalicFont=cmunbi.ttf]{cmunrm.ttf}\setmonofont[Path=/usr/share/fonts/truetype/cmu/,UprightFont=cmuntt.ttf,BoldFont=cmuntb.ttf,ItalicFont=cmunit.ttf,BoldItalicFont=cmuntx.ttf]{cmunrm.ttf} the formatting for you, and mostly the right way. If it is not exactly what you desired, then the LaTeX way is at least not worse, if not better. One way to look at it is that LaTeX is a bundle of macros for TeX that aims to carry out everything regarding document formatting, so that the writer only needs to care about content. If you really want flexibility, use plain TeX instead.

One solution to this dilemma is to make use of the modular possibilities of LaTeX. You can build your own macros, or use macros developed by others. You are likely not the first person to face some particular formatting problem, and someone who encountered a similar problem before may have published their solution as a package.

\myhref{http://www.ctan.org/}{CTAN} is a good place to find many resources regarding TeX and derivative packages. It is the first place where you should begin searching.
\subsection{Questions and documentation}
\label{6}

Besides internet resources being plentiful, the best documentation source remains the official manual for
every specific package, and the reference documentation, i.e., the {\itshape \setmainfont[Path=/usr/share/fonts/truetype/cmu/,UprightFont=cmunrm.ttf,BoldFont=cmunbx.ttf,ItalicFont=cmunti.ttf,BoldItalicFont=cmunbi.ttf]{cmunti.ttf}\setmonofont[Path=/usr/share/fonts/truetype/cmu/,UprightFont=cmuntt.ttf,BoldFont=cmuntb.ttf,ItalicFont=cmunit.ttf,BoldItalicFont=cmuntx.ttf]{cmunti.ttf}\itshape TeXbook}{$\text{ }$}\setmainfont[Path=/usr/share/fonts/truetype/cmu/,UprightFont=cmunrm.ttf,BoldFont=cmunbx.ttf,ItalicFont=cmunti.ttf,BoldItalicFont=cmunbi.ttf]{cmunrm.ttf}\setmonofont[Path=/usr/share/fonts/truetype/cmu/,UprightFont=cmuntt.ttf,BoldFont=cmuntb.ttf,ItalicFont=cmunit.ttf,BoldItalicFont=cmuntx.ttf]{cmunrm.ttf} by D.{\mbox{$~$}}Knuth and {\itshape \setmainfont[Path=/usr/share/fonts/truetype/cmu/,UprightFont=cmunrm.ttf,BoldFont=cmunbx.ttf,ItalicFont=cmunti.ttf,BoldItalicFont=cmunbi.ttf]{cmunti.ttf}\setmonofont[Path=/usr/share/fonts/truetype/cmu/,UprightFont=cmuntt.ttf,BoldFont=cmuntb.ttf,ItalicFont=cmunit.ttf,BoldItalicFont=cmuntx.ttf]{cmunti.ttf}\itshape LaTeX: A document preparation system}{$\text{ }$}\setmainfont[Path=/usr/share/fonts/truetype/cmu/,UprightFont=cmunrm.ttf,BoldFont=cmunbx.ttf,ItalicFont=cmunti.ttf,BoldItalicFont=cmunbi.ttf]{cmunrm.ttf}\setmonofont[Path=/usr/share/fonts/truetype/cmu/,UprightFont=cmuntt.ttf,BoldFont=cmuntb.ttf,ItalicFont=cmunit.ttf,BoldItalicFont=cmuntx.ttf]{cmunrm.ttf} by L.{\mbox{$~$}}Lamport.

Therefore before rushing on your favorite web search engine, we really urge you to have a look at the package documentation that causes troubles. This official documentation is most commonly installed along your TeX distribution, or may be found on \myhref{http://www.ctan.org/}{CTAN}.
\section{Terms regarding TeX}
\label{7}
{\bfseries
\begin{mydescription}Document preparation systems
\end{mydescription}
}

LaTeX is a document preparation system based on TeX. So the system is the combination of the language and the macros.
{\bfseries
\begin{mydescription}Distributions
\end{mydescription}
}

TeX distributions are collections of packages and programs (compilers, fonts, and macro packages) that enable you to typeset without having to manually fetch files and configure things. 
{\bfseries
\begin{mydescription}Engines
\end{mydescription}
}

An engine is an executable that can turn your source code into a printable output format. The engine by itself only handles the syntax, it also needs to load fonts and macros to fully understand the source code and generate output properly. The engine will determine what kind of source code it can read, and what format it can output (usually DVI or PDF).

All in all, distributions are an easy way to install what you need to use the engines and the systems you want. Distributions usually target specific operating systems. You can use different systems on different engines, but sometimes there are restrictions.
Code written for TeX, LaTeX or ConTeXt are (mostly) not compatible.
Additionally, engine-{}specific code (like font for XeTeX) may not be compiled by every engine.

When searching for information on LaTeX, you might also stumble upon \myhref{https://en.wikipedia.org/wiki/XeTeX}{XeTeX}, \myhref{https://en.wikipedia.org/wiki/ConTeXt}{ConTeXt}, \myhref{https://en.wikipedia.org/wiki/LuaTeX}{LuaTeX} or other names with a -{}TeX suffix.
Let\textquotesingle{}s recap most of the terms in this table.

\begin{longtable}{|>{\RaggedRight}p{0.11904\linewidth}|>{\RaggedRight}p{0.82382\linewidth}|} \hline 
{\bfseries \hspace*{0pt}\ignorespaces{}\hspace*{0pt} Systems}&{\bfseries \hspace*{0pt}\ignorespaces{}\hspace*{0pt} Descriptions}\endhead  \hline \hspace*{0pt}\ignorespaces{}\hspace*{0pt} ConTeXt &\hspace*{0pt}\ignorespaces{}\hspace*{0pt} A TeX-{}based document preparation system (as LaTeX is) with a very consistent and easy syntax and support for pdfTeX, XeTeX and LuaTeX engines.It does not have the same objective as LaTeX however.\\ \hline \hspace*{0pt}\ignorespaces{}\hspace*{0pt} LaTeX &\hspace*{0pt}\ignorespaces{}\hspace*{0pt} A TeX-{}based document preparation system designed by Leslie Lamport. It is actually a set of macros for TeX. It aims at taking care of the formatting process.\\ \hline \hspace*{0pt}\ignorespaces{}\hspace*{0pt} MetaFont &\hspace*{0pt}\ignorespaces{}\hspace*{0pt} A high-{}quality font system designed by Donald Knuth along TeX.\\ \hline \hspace*{0pt}\ignorespaces{}\hspace*{0pt} MetaPost &\hspace*{0pt}\ignorespaces{}\hspace*{0pt} A descriptive vector graphics language based on MetaFont.\\ \hline \hspace*{0pt}\ignorespaces{}\hspace*{0pt} TeX &\hspace*{0pt}\ignorespaces{}\hspace*{0pt} The original language designed by Donald Knuth.\\ \hline 
\end{longtable}


\begin{longtable}{|>{\RaggedRight}p{0.22646\linewidth}|>{\RaggedRight}p{0.71640\linewidth}|} \hline 
{\bfseries \hspace*{0pt}\ignorespaces{}\hspace*{0pt}Engines}&{\bfseries \hspace*{0pt}\ignorespaces{}\hspace*{0pt}Descriptions}\endhead  \hline \hspace*{0pt}\ignorespaces{}\hspace*{0pt} {\ttfamily \setmainfont[Path=/usr/share/fonts/truetype/cmu/,UprightFont=cmunrm.ttf,BoldFont=cmunbx.ttf,ItalicFont=cmunti.ttf,BoldItalicFont=cmunbi.ttf]{cmuntt.ttf}\setmonofont[Path=/usr/share/fonts/truetype/cmu/,UprightFont=cmuntt.ttf,BoldFont=cmuntb.ttf,ItalicFont=cmunit.ttf,BoldItalicFont=cmuntx.ttf]{cmuntt.ttf}\ttfamily luatex}\setmainfont[Path=/usr/share/fonts/truetype/cmu/,UprightFont=cmunrm.ttf,BoldFont=cmunbx.ttf,ItalicFont=cmunti.ttf,BoldItalicFont=cmunbi.ttf]{cmunrm.ttf}\setmonofont[Path=/usr/share/fonts/truetype/cmu/,UprightFont=cmuntt.ttf,BoldFont=cmuntb.ttf,ItalicFont=cmunit.ttf,BoldItalicFont=cmuntx.ttf]{cmunrm.ttf}, {\ttfamily \setmainfont[Path=/usr/share/fonts/truetype/cmu/,UprightFont=cmunrm.ttf,BoldFont=cmunbx.ttf,ItalicFont=cmunti.ttf,BoldItalicFont=cmunbi.ttf]{cmuntt.ttf}\setmonofont[Path=/usr/share/fonts/truetype/cmu/,UprightFont=cmuntt.ttf,BoldFont=cmuntb.ttf,ItalicFont=cmunit.ttf,BoldItalicFont=cmuntx.ttf]{cmuntt.ttf}\ttfamily lualatex}{$\text{ }$}\setmainfont[Path=/usr/share/fonts/truetype/cmu/,UprightFont=cmunrm.ttf,BoldFont=cmunbx.ttf,ItalicFont=cmunti.ttf,BoldItalicFont=cmunbi.ttf]{cmunrm.ttf}\setmonofont[Path=/usr/share/fonts/truetype/cmu/,UprightFont=cmuntt.ttf,BoldFont=cmuntb.ttf,ItalicFont=cmunit.ttf,BoldItalicFont=cmuntx.ttf]{cmunrm.ttf} &\hspace*{0pt}\ignorespaces{}\hspace*{0pt} A TeX engine with Lua scripting engine embedded aiming at making TeX internals more flexible.\\ \hline \hspace*{0pt}\ignorespaces{}\hspace*{0pt} {\ttfamily \setmainfont[Path=/usr/share/fonts/truetype/cmu/,UprightFont=cmunrm.ttf,BoldFont=cmunbx.ttf,ItalicFont=cmunti.ttf,BoldItalicFont=cmunbi.ttf]{cmuntt.ttf}\setmonofont[Path=/usr/share/fonts/truetype/cmu/,UprightFont=cmuntt.ttf,BoldFont=cmuntb.ttf,ItalicFont=cmunit.ttf,BoldItalicFont=cmuntx.ttf]{cmuntt.ttf}\ttfamily pdftex}\setmainfont[Path=/usr/share/fonts/truetype/cmu/,UprightFont=cmunrm.ttf,BoldFont=cmunbx.ttf,ItalicFont=cmunti.ttf,BoldItalicFont=cmunbi.ttf]{cmunrm.ttf}\setmonofont[Path=/usr/share/fonts/truetype/cmu/,UprightFont=cmuntt.ttf,BoldFont=cmuntb.ttf,ItalicFont=cmunit.ttf,BoldItalicFont=cmuntx.ttf]{cmunrm.ttf}, {\ttfamily \setmainfont[Path=/usr/share/fonts/truetype/cmu/,UprightFont=cmunrm.ttf,BoldFont=cmunbx.ttf,ItalicFont=cmunti.ttf,BoldItalicFont=cmunbi.ttf]{cmuntt.ttf}\setmonofont[Path=/usr/share/fonts/truetype/cmu/,UprightFont=cmuntt.ttf,BoldFont=cmuntb.ttf,ItalicFont=cmunit.ttf,BoldItalicFont=cmuntx.ttf]{cmuntt.ttf}\ttfamily pdflatex}{$\text{ }$}\setmainfont[Path=/usr/share/fonts/truetype/cmu/,UprightFont=cmunrm.ttf,BoldFont=cmunbx.ttf,ItalicFont=cmunti.ttf,BoldItalicFont=cmunbi.ttf]{cmunrm.ttf}\setmonofont[Path=/usr/share/fonts/truetype/cmu/,UprightFont=cmuntt.ttf,BoldFont=cmuntb.ttf,ItalicFont=cmunit.ttf,BoldItalicFont=cmuntx.ttf]{cmunrm.ttf} &\hspace*{0pt}\ignorespaces{}\hspace*{0pt} The engines (PDF compilers).\\ \hline \hspace*{0pt}\ignorespaces{}\hspace*{0pt} {\ttfamily \setmainfont[Path=/usr/share/fonts/truetype/cmu/,UprightFont=cmunrm.ttf,BoldFont=cmunbx.ttf,ItalicFont=cmunti.ttf,BoldItalicFont=cmunbi.ttf]{cmuntt.ttf}\setmonofont[Path=/usr/share/fonts/truetype/cmu/,UprightFont=cmuntt.ttf,BoldFont=cmuntb.ttf,ItalicFont=cmunit.ttf,BoldItalicFont=cmuntx.ttf]{cmuntt.ttf}\ttfamily tex}\setmainfont[Path=/usr/share/fonts/truetype/cmu/,UprightFont=cmunrm.ttf,BoldFont=cmunbx.ttf,ItalicFont=cmunti.ttf,BoldItalicFont=cmunbi.ttf]{cmunrm.ttf}\setmonofont[Path=/usr/share/fonts/truetype/cmu/,UprightFont=cmuntt.ttf,BoldFont=cmuntb.ttf,ItalicFont=cmunit.ttf,BoldItalicFont=cmuntx.ttf]{cmunrm.ttf}, {\ttfamily \setmainfont[Path=/usr/share/fonts/truetype/cmu/,UprightFont=cmunrm.ttf,BoldFont=cmunbx.ttf,ItalicFont=cmunti.ttf,BoldItalicFont=cmunbi.ttf]{cmuntt.ttf}\setmonofont[Path=/usr/share/fonts/truetype/cmu/,UprightFont=cmuntt.ttf,BoldFont=cmuntb.ttf,ItalicFont=cmunit.ttf,BoldItalicFont=cmuntx.ttf]{cmuntt.ttf}\ttfamily latex}{$\text{ }$}\setmainfont[Path=/usr/share/fonts/truetype/cmu/,UprightFont=cmunrm.ttf,BoldFont=cmunbx.ttf,ItalicFont=cmunti.ttf,BoldItalicFont=cmunbi.ttf]{cmunrm.ttf}\setmonofont[Path=/usr/share/fonts/truetype/cmu/,UprightFont=cmuntt.ttf,BoldFont=cmuntb.ttf,ItalicFont=cmunit.ttf,BoldItalicFont=cmuntx.ttf]{cmunrm.ttf} &\hspace*{0pt}\ignorespaces{}\hspace*{0pt} The engines (DVI compilers).\\ \hline \hspace*{0pt}\ignorespaces{}\hspace*{0pt} {\ttfamily \setmainfont[Path=/usr/share/fonts/truetype/cmu/,UprightFont=cmunrm.ttf,BoldFont=cmunbx.ttf,ItalicFont=cmunti.ttf,BoldItalicFont=cmunbi.ttf]{cmuntt.ttf}\setmonofont[Path=/usr/share/fonts/truetype/cmu/,UprightFont=cmuntt.ttf,BoldFont=cmuntb.ttf,ItalicFont=cmunit.ttf,BoldItalicFont=cmuntx.ttf]{cmuntt.ttf}\ttfamily xetex}\setmainfont[Path=/usr/share/fonts/truetype/cmu/,UprightFont=cmunrm.ttf,BoldFont=cmunbx.ttf,ItalicFont=cmunti.ttf,BoldItalicFont=cmunbi.ttf]{cmunrm.ttf}\setmonofont[Path=/usr/share/fonts/truetype/cmu/,UprightFont=cmuntt.ttf,BoldFont=cmuntb.ttf,ItalicFont=cmunit.ttf,BoldItalicFont=cmuntx.ttf]{cmunrm.ttf}, {\ttfamily \setmainfont[Path=/usr/share/fonts/truetype/cmu/,UprightFont=cmunrm.ttf,BoldFont=cmunbx.ttf,ItalicFont=cmunti.ttf,BoldItalicFont=cmunbi.ttf]{cmuntt.ttf}\setmonofont[Path=/usr/share/fonts/truetype/cmu/,UprightFont=cmuntt.ttf,BoldFont=cmuntb.ttf,ItalicFont=cmunit.ttf,BoldItalicFont=cmuntx.ttf]{cmuntt.ttf}\ttfamily xelatex}{$\text{ }$}\setmainfont[Path=/usr/share/fonts/truetype/cmu/,UprightFont=cmunrm.ttf,BoldFont=cmunbx.ttf,ItalicFont=cmunti.ttf,BoldItalicFont=cmunbi.ttf]{cmunrm.ttf}\setmonofont[Path=/usr/share/fonts/truetype/cmu/,UprightFont=cmuntt.ttf,BoldFont=cmuntb.ttf,ItalicFont=cmunit.ttf,BoldItalicFont=cmuntx.ttf]{cmunrm.ttf} &\hspace*{0pt}\ignorespaces{}\hspace*{0pt} a TeX engine which uses Unicode and supports widely popular {\ttfamily \setmainfont[Path=/usr/share/fonts/truetype/cmu/,UprightFont=cmunrm.ttf,BoldFont=cmunbx.ttf,ItalicFont=cmunti.ttf,BoldItalicFont=cmunbi.ttf]{cmuntt.ttf}\setmonofont[Path=/usr/share/fonts/truetype/cmu/,UprightFont=cmuntt.ttf,BoldFont=cmuntb.ttf,ItalicFont=cmunit.ttf,BoldItalicFont=cmuntx.ttf]{cmuntt.ttf}\ttfamily .ttf}{$\text{ }$}\setmainfont[Path=/usr/share/fonts/truetype/cmu/,UprightFont=cmunrm.ttf,BoldFont=cmunbx.ttf,ItalicFont=cmunti.ttf,BoldItalicFont=cmunbi.ttf]{cmunrm.ttf}\setmonofont[Path=/usr/share/fonts/truetype/cmu/,UprightFont=cmuntt.ttf,BoldFont=cmuntb.ttf,ItalicFont=cmunit.ttf,BoldItalicFont=cmuntx.ttf]{cmunrm.ttf} and {\ttfamily \setmainfont[Path=/usr/share/fonts/truetype/cmu/,UprightFont=cmunrm.ttf,BoldFont=cmunbx.ttf,ItalicFont=cmunti.ttf,BoldItalicFont=cmunbi.ttf]{cmuntt.ttf}\setmonofont[Path=/usr/share/fonts/truetype/cmu/,UprightFont=cmuntt.ttf,BoldFont=cmuntb.ttf,ItalicFont=cmunit.ttf,BoldItalicFont=cmuntx.ttf]{cmuntt.ttf}\ttfamily .otf}{$\text{ }$}\setmainfont[Path=/usr/share/fonts/truetype/cmu/,UprightFont=cmunrm.ttf,BoldFont=cmunbx.ttf,ItalicFont=cmunti.ttf,BoldItalicFont=cmunbi.ttf]{cmunrm.ttf}\setmonofont[Path=/usr/share/fonts/truetype/cmu/,UprightFont=cmuntt.ttf,BoldFont=cmuntb.ttf,ItalicFont=cmunit.ttf,BoldItalicFont=cmuntx.ttf]{cmunrm.ttf} fonts. See \mylref{163}{Fonts}.\\ \hline 
\end{longtable}


\begin{longtable}{|>{\RaggedRight}p{0.27070\linewidth}|>{\RaggedRight}p{0.67215\linewidth}|} \hline 
{\bfseries \hspace*{0pt}\ignorespaces{}\hspace*{0pt}TeX Distributions}&{\bfseries \hspace*{0pt}\ignorespaces{}\hspace*{0pt}Descriptions}\endhead  \hline \hspace*{0pt}\ignorespaces{}\hspace*{0pt} MacTeX &\hspace*{0pt}\ignorespaces{}\hspace*{0pt} A TeX Live based distribution targetting Mac OS X.\\ \hline \hspace*{0pt}\ignorespaces{}\hspace*{0pt} MiKTeX &\hspace*{0pt}\ignorespaces{}\hspace*{0pt} A TeX distribution for Windows.\\ \hline \hspace*{0pt}\ignorespaces{}\hspace*{0pt} TeX Live &\hspace*{0pt}\ignorespaces{}\hspace*{0pt} A cross-{}platform TeX distribution.\\ \hline 
\end{longtable}

\section{What next?}
\label{8}

In the next chapter we will proceed to the \mylref{10}{installation}. Then we will compile our \mylref{63}{first LaTeX file}. 

Throughout this book you should also utilise other means for learning about LaTeX. Good sources are:

\begin{myitemize}
\item{}  the {\ttfamily \myhref{http://webchat.freenode.net?channels=latex}{\setmainfont[Path=/usr/share/fonts/truetype/cmu/,UprightFont=cmunrm.ttf,BoldFont=cmunbx.ttf,ItalicFont=cmunti.ttf,BoldItalicFont=cmunbi.ttf]{cmuntt.ttf}\setmonofont[Path=/usr/share/fonts/truetype/cmu/,UprightFont=cmuntt.ttf,BoldFont=cmuntb.ttf,ItalicFont=cmunit.ttf,BoldItalicFont=cmuntx.ttf]{cmuntt.ttf}\ttfamily \#latex}} IRC channel on Freenode,
\item{}  \myhref{http://tex.stackexchange.com/}{the TeX Stack Exchange} Q\&A,
\item{}  \myhref{http://www.tex.ac.uk/}{the TeX} FAQ,
\item{}  and \myhref{http://www.texample.net/}{the TeXample.net} Community.
\end{myitemize}




\myhref{https://de.wikibooks.org/wiki/LaTeX\%2F_Einleitung}{de:LaTeX/\_Einleitung}
\myhref{https://sr.wikibooks.org/wiki/LaTeX\%2F\%D0\%A3\%D0\%B2\%D0\%BE\%D0\%B4}{sr:LaTeX/Увод}\chapter{Installation}

\myminitoc
\label{9}

\label{10}


If this is the first time you are trying out LaTeX, you don\textquotesingle{}t even need to install anything. For quick testing purpose you may just create a user account with an \mylref{45}{online LaTeX editor} and continue this tutorial in the next chapter. These websites offer collaboration capabilities while allowing you to experiment with LaTeX syntax without having to bother with installing and configuring a distribution and an editor. When you later feel that you would benefit from having a standalone LaTeX installation, you can return to this chapter and follow the instructions below.

LaTeX is not a program by itself; it is a language. {\itshape \setmainfont[Path=/usr/share/fonts/truetype/cmu/,UprightFont=cmunrm.ttf,BoldFont=cmunbx.ttf,ItalicFont=cmunti.ttf,BoldItalicFont=cmunbi.ttf]{cmunti.ttf}\setmonofont[Path=/usr/share/fonts/truetype/cmu/,UprightFont=cmuntt.ttf,BoldFont=cmuntb.ttf,ItalicFont=cmunit.ttf,BoldItalicFont=cmuntx.ttf]{cmunti.ttf}\itshape Using LaTeX}{$\text{ }$}\setmainfont[Path=/usr/share/fonts/truetype/cmu/,UprightFont=cmunrm.ttf,BoldFont=cmunbx.ttf,ItalicFont=cmunti.ttf,BoldItalicFont=cmunbi.ttf]{cmunrm.ttf}\setmonofont[Path=/usr/share/fonts/truetype/cmu/,UprightFont=cmuntt.ttf,BoldFont=cmuntb.ttf,ItalicFont=cmunit.ttf,BoldItalicFont=cmuntx.ttf]{cmunrm.ttf} requires a bunch of tools. Acquiring them manually would result in downloading and installing multiple programs in order to have a suitable computer system that can be used to create LaTeX output, such as PDFs. {\bfseries \setmainfont[Path=/usr/share/fonts/truetype/cmu/,UprightFont=cmunrm.ttf,BoldFont=cmunbx.ttf,ItalicFont=cmunti.ttf,BoldItalicFont=cmunbi.ttf]{cmunbx.ttf}\setmonofont[Path=/usr/share/fonts/truetype/cmu/,UprightFont=cmuntt.ttf,BoldFont=cmuntb.ttf,ItalicFont=cmunit.ttf,BoldItalicFont=cmuntx.ttf]{cmunbx.ttf}\bfseries TeX Distributions}{$\text{ }$}\setmainfont[Path=/usr/share/fonts/truetype/cmu/,UprightFont=cmunrm.ttf,BoldFont=cmunbx.ttf,ItalicFont=cmunti.ttf,BoldItalicFont=cmunbi.ttf]{cmunrm.ttf}\setmonofont[Path=/usr/share/fonts/truetype/cmu/,UprightFont=cmuntt.ttf,BoldFont=cmuntb.ttf,ItalicFont=cmunit.ttf,BoldItalicFont=cmuntx.ttf]{cmunrm.ttf} help the user in this way, in that it is a single step installation process that provides (almost) everything.

At a minimum, you\textquotesingle{}ll need a TeX distribution, a good text editor and a DVI or PDF viewer. More specifically, the basic requirement is to have a TeX compiler (which is used to generate output files from source), fonts, and the LaTeX macro set. Optional, and recommended installations include an attractive editor to write LaTeX source documents (this is probably where you will spend most of your time), and a bibliographic management program to manage references if you use them a lot.
\section{Distributions}
\label{11}
TeX and LaTeX are available for most computer platforms, since they were programmed to be very portable. They are most commonly installed using a distribution, such as teTeX, MiKTeX, or MacTeX. TeX distributions are collections of packages and programs (compilers, fonts, and macro packages) that enable you to typeset without having to manually fetch files and configure things. LaTeX is just a set of macro packages built for TeX.

The recommended distributions for each of the major operating systems are:

\begin{myitemize}
\item{}  \myhref{http://www.tug.org/texlive/}{TeX Live} is a major TeX distribution for *BSD, GNU/Linux, Mac OS X and Windows.
\item{}  \myhref{http://www.miktex.org/}{MiKTeX} is a Windows-{}specific distribution.
\item{}  \myhref{http://www.tug.org/mactex/}{MacTeX} is a Mac OS-{}specific distribution based on TeX Live.
\end{myitemize}


These, however, do not necessarily include an editor. You might be interested in other programs that are not part of the distribution, which will help you in writing and preparing TeX and LaTeX files.
\subsection{*BSD and GNU/Linux}
\label{12}

In the past, the most common distribution used to be {\bfseries \setmainfont[Path=/usr/share/fonts/truetype/cmu/,UprightFont=cmunrm.ttf,BoldFont=cmunbx.ttf,ItalicFont=cmunti.ttf,BoldItalicFont=cmunbi.ttf]{cmunbx.ttf}\setmonofont[Path=/usr/share/fonts/truetype/cmu/,UprightFont=cmuntt.ttf,BoldFont=cmuntb.ttf,ItalicFont=cmunit.ttf,BoldItalicFont=cmuntx.ttf]{cmunbx.ttf}\bfseries teTeX}\setmainfont[Path=/usr/share/fonts/truetype/cmu/,UprightFont=cmunrm.ttf,BoldFont=cmunbx.ttf,ItalicFont=cmunti.ttf,BoldItalicFont=cmunbi.ttf]{cmunrm.ttf}\setmonofont[Path=/usr/share/fonts/truetype/cmu/,UprightFont=cmuntt.ttf,BoldFont=cmuntb.ttf,ItalicFont=cmunit.ttf,BoldItalicFont=cmuntx.ttf]{cmunrm.ttf}. As of May 2006 teTeX is no longer actively maintained and its former maintainer Thomas Esser recommended TeX Live as the replacement.\myfootnote{\myfnhref{http://www.tug.org/tetex/}{teTeX Home Page} (Retrieved January 31, 2007)} 

The easy way to get TeX Live is to use the package manager or portage tree coming with your operating system. Usually it comes as several packages, with some of them being essential, other optional. The {\itshape \setmainfont[Path=/usr/share/fonts/truetype/cmu/,UprightFont=cmunrm.ttf,BoldFont=cmunbx.ttf,ItalicFont=cmunti.ttf,BoldItalicFont=cmunbi.ttf]{cmunti.ttf}\setmonofont[Path=/usr/share/fonts/truetype/cmu/,UprightFont=cmuntt.ttf,BoldFont=cmuntb.ttf,ItalicFont=cmunit.ttf,BoldItalicFont=cmuntx.ttf]{cmunti.ttf}\itshape core}{$\text{ }$}\setmainfont[Path=/usr/share/fonts/truetype/cmu/,UprightFont=cmunrm.ttf,BoldFont=cmunbx.ttf,ItalicFont=cmunti.ttf,BoldItalicFont=cmunbi.ttf]{cmunrm.ttf}\setmonofont[Path=/usr/share/fonts/truetype/cmu/,UprightFont=cmuntt.ttf,BoldFont=cmuntb.ttf,ItalicFont=cmunit.ttf,BoldItalicFont=cmuntx.ttf]{cmunrm.ttf} TeX Live packages should be around 200-{}300 MB.

If your *BSD or GNU/Linux distribution does not have the TeX Live packages, you should report a wish to the bug tracking system. In that case you will need to \myhref{http://www.tug.org/texlive/acquire.html}{download TeX Live} yourself and run the installer by hand.

You may wish to install the content of TeX Live more selectively. See \mylref{15}{below}.
\subsection{Mac OS X}
\label{13}
Mac OS X users may use \myhref{http://tug.org/mactex/}{MacTeX}, a TeX Live-{}based distribution supporting TeX, LaTeX, AMSTeX, ConTeXt, XeTeX and many other core packages. Download MacTeX.mpkg.zip on the \myhref{http://www.tug.org/mactex/}{MacTeX page}, unzip it and follow the instructions. Further information for Mac OS X users can be found on the \myhref{http://mactex-wiki.tug.org/}{TeX on Mac OS X Wiki}.

Since Mac OS X is also a Unix-{}based system, TeX Live is naturally available through \myhref{http://www.macports.org/}{MacPorts} and \myhref{http://www.finkproject.org/}{Fink}. \myhref{http://brew.sh/}{Homebrew} users should use the official \myhref{http://www.tug.org/mactex/}{MacTeX installer} because of the \myhref{https://github.com/Homebrew/homebrew/issues/1087}{unique directory structure used by TeX Live}. Further information for Mac OS X users can be found on the \myhref{http://mactex-wiki.tug.org/}{TeX on Mac OS X Wiki}.
\subsection{Microsoft Windows}
\label{14}
Microsoft Windows users can install \myhref{http://miktex.org/}{MiKTeX} onto their computer. It has an easy installer that takes care of setting up the environment and downloading core packages. This distribution has advanced features, such as automatic installation of packages, and simple interfaces to modify settings, such as default paper sizes.

There is also a port of TeX Live available for Windows.
\section{Custom installation with TeX Live}
\label{15}

This section targets users who want fine-{}grained control over their TeX distribution, like an installation with a minimum of disk space usage. If it is none of your concern, you may want to jump to the \mylref{22}{next section}.

Picky users may wish to have more control over their installation. Common distributions might be tedious for the user caring about disk space. In fact, MikTeX and MacTeX and packaged TeX Live features hundreds of LaTeX packages, most of them which you will never use. Most Unix with a package manager will offer TeX Live as a set of several big packages, and you often have to install 300–400 MB for a functional system.

TeX Live features a manual installation with a lot of possible customizations. You can get the network installer at \myhref{http://www.tug.org/texlive/acquire-netinstall.html}{tug.org}. This installer allows you to select precisely the packages you want to install. As a result, you may have everything you need for less than 100 MB. TeX Live is then managed through its own package manager, {\itshape \setmainfont[Path=/usr/share/fonts/truetype/cmu/,UprightFont=cmunrm.ttf,BoldFont=cmunbx.ttf,ItalicFont=cmunti.ttf,BoldItalicFont=cmunbi.ttf]{cmunti.ttf}\setmonofont[Path=/usr/share/fonts/truetype/cmu/,UprightFont=cmuntt.ttf,BoldFont=cmuntb.ttf,ItalicFont=cmunit.ttf,BoldItalicFont=cmuntx.ttf]{cmunti.ttf}\itshape tlmgr}\setmainfont[Path=/usr/share/fonts/truetype/cmu/,UprightFont=cmunrm.ttf,BoldFont=cmunbx.ttf,ItalicFont=cmunti.ttf,BoldItalicFont=cmunbi.ttf]{cmunrm.ttf}\setmonofont[Path=/usr/share/fonts/truetype/cmu/,UprightFont=cmuntt.ttf,BoldFont=cmuntb.ttf,ItalicFont=cmunit.ttf,BoldItalicFont=cmuntx.ttf]{cmunrm.ttf}. It will let you configure the distributions, install or remove extra packages and so on.

You will need a Unix-{}based operating system for the following. Mac OS X, GNU/Linux or *BSD are fine. It may work for Windows but the process must be quite different.

TeX Live groups features and packages into different concepts:
\begin{myitemize}
\item{}  {\itshape \setmainfont[Path=/usr/share/fonts/truetype/cmu/,UprightFont=cmunrm.ttf,BoldFont=cmunbx.ttf,ItalicFont=cmunti.ttf,BoldItalicFont=cmunbi.ttf]{cmunti.ttf}\setmonofont[Path=/usr/share/fonts/truetype/cmu/,UprightFont=cmuntt.ttf,BoldFont=cmuntb.ttf,ItalicFont=cmunit.ttf,BoldItalicFont=cmuntx.ttf]{cmunti.ttf}\itshape Collections}{$\text{ }$}\setmainfont[Path=/usr/share/fonts/truetype/cmu/,UprightFont=cmunrm.ttf,BoldFont=cmunbx.ttf,ItalicFont=cmunti.ttf,BoldItalicFont=cmunbi.ttf]{cmunrm.ttf}\setmonofont[Path=/usr/share/fonts/truetype/cmu/,UprightFont=cmuntt.ttf,BoldFont=cmuntb.ttf,ItalicFont=cmunit.ttf,BoldItalicFont=cmuntx.ttf]{cmunrm.ttf} are groups of packages that can always be installed individually, except for the {\itshape \setmainfont[Path=/usr/share/fonts/truetype/cmu/,UprightFont=cmunrm.ttf,BoldFont=cmunbx.ttf,ItalicFont=cmunti.ttf,BoldItalicFont=cmunbi.ttf]{cmunti.ttf}\setmonofont[Path=/usr/share/fonts/truetype/cmu/,UprightFont=cmuntt.ttf,BoldFont=cmuntb.ttf,ItalicFont=cmunit.ttf,BoldItalicFont=cmuntx.ttf]{cmunti.ttf}\itshape Essential programs and files}{$\text{ }$}\setmainfont[Path=/usr/share/fonts/truetype/cmu/,UprightFont=cmunrm.ttf,BoldFont=cmunbx.ttf,ItalicFont=cmunti.ttf,BoldItalicFont=cmunbi.ttf]{cmunrm.ttf}\setmonofont[Path=/usr/share/fonts/truetype/cmu/,UprightFont=cmuntt.ttf,BoldFont=cmuntb.ttf,ItalicFont=cmunit.ttf,BoldItalicFont=cmuntx.ttf]{cmunrm.ttf} collection. You can install collections at any time.
\item{}  {\itshape \setmainfont[Path=/usr/share/fonts/truetype/cmu/,UprightFont=cmunrm.ttf,BoldFont=cmunbx.ttf,ItalicFont=cmunti.ttf,BoldItalicFont=cmunbi.ttf]{cmunti.ttf}\setmonofont[Path=/usr/share/fonts/truetype/cmu/,UprightFont=cmuntt.ttf,BoldFont=cmuntb.ttf,ItalicFont=cmunit.ttf,BoldItalicFont=cmuntx.ttf]{cmunti.ttf}\itshape Installation Schemes}{$\text{ }$}\setmainfont[Path=/usr/share/fonts/truetype/cmu/,UprightFont=cmunrm.ttf,BoldFont=cmunbx.ttf,ItalicFont=cmunti.ttf,BoldItalicFont=cmunbi.ttf]{cmunrm.ttf}\setmonofont[Path=/usr/share/fonts/truetype/cmu/,UprightFont=cmuntt.ttf,BoldFont=cmuntb.ttf,ItalicFont=cmunit.ttf,BoldItalicFont=cmuntx.ttf]{cmunrm.ttf} group collections and packages. Schemes can only be used at installation time. You can select only one scheme at a time.
\end{myitemize}

\subsection{Minimal installation}
\label{16}

We will give you general guidelines to install a minimal TeX distribution ({\itshape \setmainfont[Path=/usr/share/fonts/truetype/cmu/,UprightFont=cmunrm.ttf,BoldFont=cmunbx.ttf,ItalicFont=cmunti.ttf,BoldItalicFont=cmunbi.ttf]{cmunti.ttf}\setmonofont[Path=/usr/share/fonts/truetype/cmu/,UprightFont=cmuntt.ttf,BoldFont=cmuntb.ttf,ItalicFont=cmunit.ttf,BoldItalicFont=cmuntx.ttf]{cmunti.ttf}\itshape i.e.}\setmainfont[Path=/usr/share/fonts/truetype/cmu/,UprightFont=cmunrm.ttf,BoldFont=cmunbx.ttf,ItalicFont=cmunti.ttf,BoldItalicFont=cmunbi.ttf]{cmunrm.ttf}\setmonofont[Path=/usr/share/fonts/truetype/cmu/,UprightFont=cmuntt.ttf,BoldFont=cmuntb.ttf,ItalicFont=cmunit.ttf,BoldItalicFont=cmuntx.ttf]{cmunrm.ttf}, only for plain TeX).
\begin{myenumerate}
\item{}  Download the installer at \myplainurl{http://mirror.ctan.org/systems/texlive/tlnet/install-tl-unx.tar.gz} and extract it to a temporary folder.
\item{}  Open a terminal in the extracted folder and log in as root.
\item{}  Change the \myhref{https://en.wikipedia.org/wiki/umask}{umask} permissions to 022 (user read/write/execute, group/others read/execute only)  to make sure other users will have read-{}only access to the installed distribution.
\end{myenumerate}
\\

\TemplateSpaceIndent{$\text{ }${}\#$\text{ }${}umask$\text{ }${}022}

\begin{TemplateInfo}{\danger}{Warning}All administration operations for TeX Live should be made with a 022 umask. Otherwise you will not be able to use TeX at all with an unprivileged user.\end{TemplateInfo}
\begin{myenumerate}
\item{}  Launch {\ttfamily \setmainfont[Path=/usr/share/fonts/truetype/cmu/,UprightFont=cmunrm.ttf,BoldFont=cmunbx.ttf,ItalicFont=cmunti.ttf,BoldItalicFont=cmunbi.ttf]{cmuntt.ttf}\setmonofont[Path=/usr/share/fonts/truetype/cmu/,UprightFont=cmuntt.ttf,BoldFont=cmuntb.ttf,ItalicFont=cmunit.ttf,BoldItalicFont=cmuntx.ttf]{cmuntt.ttf}\ttfamily install-{}tl}\setmainfont[Path=/usr/share/fonts/truetype/cmu/,UprightFont=cmunrm.ttf,BoldFont=cmunbx.ttf,ItalicFont=cmunti.ttf,BoldItalicFont=cmunbi.ttf]{cmunrm.ttf}\setmonofont[Path=/usr/share/fonts/truetype/cmu/,UprightFont=cmuntt.ttf,BoldFont=cmuntb.ttf,ItalicFont=cmunit.ttf,BoldItalicFont=cmuntx.ttf]{cmunrm.ttf}.
\item{}  Select the {\itshape \setmainfont[Path=/usr/share/fonts/truetype/cmu/,UprightFont=cmunrm.ttf,BoldFont=cmunbx.ttf,ItalicFont=cmunti.ttf,BoldItalicFont=cmunbi.ttf]{cmunti.ttf}\setmonofont[Path=/usr/share/fonts/truetype/cmu/,UprightFont=cmuntt.ttf,BoldFont=cmuntb.ttf,ItalicFont=cmunit.ttf,BoldItalicFont=cmuntx.ttf]{cmunti.ttf}\itshape minimal scheme (plain only)}\setmainfont[Path=/usr/share/fonts/truetype/cmu/,UprightFont=cmunrm.ttf,BoldFont=cmunbx.ttf,ItalicFont=cmunti.ttf,BoldItalicFont=cmunbi.ttf]{cmunrm.ttf}\setmonofont[Path=/usr/share/fonts/truetype/cmu/,UprightFont=cmuntt.ttf,BoldFont=cmuntb.ttf,ItalicFont=cmunit.ttf,BoldItalicFont=cmuntx.ttf]{cmunrm.ttf}.
\item{}  You may want to change the directory options. For example you may want to hide your personal macro folder which is located at TEXMFHOME. It is {\ttfamily \setmainfont[Path=/usr/share/fonts/truetype/cmu/,UprightFont=cmunrm.ttf,BoldFont=cmunbx.ttf,ItalicFont=cmunti.ttf,BoldItalicFont=cmunbi.ttf]{cmuntt.ttf}\setmonofont[Path=/usr/share/fonts/truetype/cmu/,UprightFont=cmuntt.ttf,BoldFont=cmuntb.ttf,ItalicFont=cmunit.ttf,BoldItalicFont=cmuntx.ttf]{cmuntt.ttf}\ttfamily \~{}/texmf}{$\text{ }$}\setmainfont[Path=/usr/share/fonts/truetype/cmu/,UprightFont=cmunrm.ttf,BoldFont=cmunbx.ttf,ItalicFont=cmunti.ttf,BoldItalicFont=cmunbi.ttf]{cmunrm.ttf}\setmonofont[Path=/usr/share/fonts/truetype/cmu/,UprightFont=cmuntt.ttf,BoldFont=cmuntb.ttf,ItalicFont=cmunit.ttf,BoldItalicFont=cmuntx.ttf]{cmunrm.ttf} by default. Replace it by {\ttfamily \setmainfont[Path=/usr/share/fonts/truetype/cmu/,UprightFont=cmunrm.ttf,BoldFont=cmunbx.ttf,ItalicFont=cmunti.ttf,BoldItalicFont=cmunbi.ttf]{cmuntt.ttf}\setmonofont[Path=/usr/share/fonts/truetype/cmu/,UprightFont=cmuntt.ttf,BoldFont=cmuntb.ttf,ItalicFont=cmunit.ttf,BoldItalicFont=cmuntx.ttf]{cmuntt.ttf}\ttfamily \~{}/.texmf}{$\text{ }$}\setmainfont[Path=/usr/share/fonts/truetype/cmu/,UprightFont=cmunrm.ttf,BoldFont=cmunbx.ttf,ItalicFont=cmunti.ttf,BoldItalicFont=cmunbi.ttf]{cmunrm.ttf}\setmonofont[Path=/usr/share/fonts/truetype/cmu/,UprightFont=cmuntt.ttf,BoldFont=cmuntb.ttf,ItalicFont=cmunit.ttf,BoldItalicFont=cmuntx.ttf]{cmunrm.ttf} to hide it.
\item{}  Now the options:
\begin{myenumerate}
\item{}  {\bfseries \setmainfont[Path=/usr/share/fonts/truetype/cmu/,UprightFont=cmunrm.ttf,BoldFont=cmunbx.ttf,ItalicFont=cmunti.ttf,BoldItalicFont=cmunbi.ttf]{cmunbx.ttf}\setmonofont[Path=/usr/share/fonts/truetype/cmu/,UprightFont=cmuntt.ttf,BoldFont=cmuntb.ttf,ItalicFont=cmunit.ttf,BoldItalicFont=cmuntx.ttf]{cmunbx.ttf}\bfseries use letter size instead of A4 by default:}{$\text{ }$}\setmainfont[Path=/usr/share/fonts/truetype/cmu/,UprightFont=cmunrm.ttf,BoldFont=cmunbx.ttf,ItalicFont=cmunti.ttf,BoldItalicFont=cmunbi.ttf]{cmunrm.ttf}\setmonofont[Path=/usr/share/fonts/truetype/cmu/,UprightFont=cmuntt.ttf,BoldFont=cmuntb.ttf,ItalicFont=cmunit.ttf,BoldItalicFont=cmuntx.ttf]{cmunrm.ttf} mostly for users from the USA.
\item{}  {\bfseries \setmainfont[Path=/usr/share/fonts/truetype/cmu/,UprightFont=cmunrm.ttf,BoldFont=cmunbx.ttf,ItalicFont=cmunti.ttf,BoldItalicFont=cmunbi.ttf]{cmunbx.ttf}\setmonofont[Path=/usr/share/fonts/truetype/cmu/,UprightFont=cmuntt.ttf,BoldFont=cmuntb.ttf,ItalicFont=cmunit.ttf,BoldItalicFont=cmuntx.ttf]{cmunbx.ttf}\bfseries execution of restricted list of programs:}{$\text{ }$}\setmainfont[Path=/usr/share/fonts/truetype/cmu/,UprightFont=cmunrm.ttf,BoldFont=cmunbx.ttf,ItalicFont=cmunti.ttf,BoldItalicFont=cmunbi.ttf]{cmunrm.ttf}\setmonofont[Path=/usr/share/fonts/truetype/cmu/,UprightFont=cmuntt.ttf,BoldFont=cmuntb.ttf,ItalicFont=cmunit.ttf,BoldItalicFont=cmuntx.ttf]{cmunrm.ttf} it is recommended to select it for security reasons. Otherwise it allows the TeX engines to call any external program. You may still configure the list afterwards.
\item{}  {\bfseries \setmainfont[Path=/usr/share/fonts/truetype/cmu/,UprightFont=cmunrm.ttf,BoldFont=cmunbx.ttf,ItalicFont=cmunti.ttf,BoldItalicFont=cmunbi.ttf]{cmunbx.ttf}\setmonofont[Path=/usr/share/fonts/truetype/cmu/,UprightFont=cmuntt.ttf,BoldFont=cmuntb.ttf,ItalicFont=cmunit.ttf,BoldItalicFont=cmuntx.ttf]{cmunbx.ttf}\bfseries create format files:}{$\text{ }$}\setmainfont[Path=/usr/share/fonts/truetype/cmu/,UprightFont=cmunrm.ttf,BoldFont=cmunbx.ttf,ItalicFont=cmunti.ttf,BoldItalicFont=cmunbi.ttf]{cmunrm.ttf}\setmonofont[Path=/usr/share/fonts/truetype/cmu/,UprightFont=cmuntt.ttf,BoldFont=cmuntb.ttf,ItalicFont=cmunit.ttf,BoldItalicFont=cmuntx.ttf]{cmunrm.ttf} targetting a minimal disk space, the best choice depends on whether there is only one user on the system, then deselecting it is better, otherwise select it. From the help menu: {\itshape \setmainfont[Path=/usr/share/fonts/truetype/cmu/,UprightFont=cmunrm.ttf,BoldFont=cmunbx.ttf,ItalicFont=cmunti.ttf,BoldItalicFont=cmunbi.ttf]{cmunti.ttf}\setmonofont[Path=/usr/share/fonts/truetype/cmu/,UprightFont=cmuntt.ttf,BoldFont=cmuntb.ttf,ItalicFont=cmunit.ttf,BoldItalicFont=cmuntx.ttf]{cmunti.ttf}\itshape \symbol{34}If this option is set, format files are created for system-{}wide use by the installer. Otherwise they will be created automatically when needed. In the latter case format files are stored in user\textquotesingle{}s directory trees and in some cases have to be re-{}created when new packages are installed.\symbol{34}}
\item{} {$\text{ }$}\setmainfont[Path=/usr/share/fonts/truetype/cmu/,UprightFont=cmunrm.ttf,BoldFont=cmunbx.ttf,ItalicFont=cmunti.ttf,BoldItalicFont=cmunbi.ttf]{cmunrm.ttf}\setmonofont[Path=/usr/share/fonts/truetype/cmu/,UprightFont=cmuntt.ttf,BoldFont=cmuntb.ttf,ItalicFont=cmunit.ttf,BoldItalicFont=cmuntx.ttf]{cmunrm.ttf} {\bfseries \setmainfont[Path=/usr/share/fonts/truetype/cmu/,UprightFont=cmunrm.ttf,BoldFont=cmunbx.ttf,ItalicFont=cmunti.ttf,BoldItalicFont=cmunbi.ttf]{cmunbx.ttf}\setmonofont[Path=/usr/share/fonts/truetype/cmu/,UprightFont=cmuntt.ttf,BoldFont=cmuntb.ttf,ItalicFont=cmunit.ttf,BoldItalicFont=cmuntx.ttf]{cmunbx.ttf}\bfseries install font/macro doc tree:}{$\text{ }$}\setmainfont[Path=/usr/share/fonts/truetype/cmu/,UprightFont=cmunrm.ttf,BoldFont=cmunbx.ttf,ItalicFont=cmunti.ttf,BoldItalicFont=cmunbi.ttf]{cmunrm.ttf}\setmonofont[Path=/usr/share/fonts/truetype/cmu/,UprightFont=cmuntt.ttf,BoldFont=cmuntb.ttf,ItalicFont=cmunit.ttf,BoldItalicFont=cmuntx.ttf]{cmunrm.ttf} useful if you are a developer, but very space consuming. Turn it off if you want to save space.
\item{}  {\bfseries \setmainfont[Path=/usr/share/fonts/truetype/cmu/,UprightFont=cmunrm.ttf,BoldFont=cmunbx.ttf,ItalicFont=cmunti.ttf,BoldItalicFont=cmunbi.ttf]{cmunbx.ttf}\setmonofont[Path=/usr/share/fonts/truetype/cmu/,UprightFont=cmuntt.ttf,BoldFont=cmuntb.ttf,ItalicFont=cmunit.ttf,BoldItalicFont=cmuntx.ttf]{cmunbx.ttf}\bfseries install font/macro source tree:}{$\text{ }$}\setmainfont[Path=/usr/share/fonts/truetype/cmu/,UprightFont=cmunrm.ttf,BoldFont=cmunbx.ttf,ItalicFont=cmunti.ttf,BoldItalicFont=cmunbi.ttf]{cmunrm.ttf}\setmonofont[Path=/usr/share/fonts/truetype/cmu/,UprightFont=cmuntt.ttf,BoldFont=cmuntb.ttf,ItalicFont=cmunit.ttf,BoldItalicFont=cmuntx.ttf]{cmunrm.ttf} same as above.
\item{}  Symlinks are fine by default, change it if you know what you are doing.
\end{myenumerate}

\item{}  Select portable installation if you install the distribution to an optical disc, or any kind of external media. Leave to default for a traditional installation on the system hard drive.
\end{myenumerate}


At this point it should display\\

\TemplateSpaceIndent{$\text{ }${}1$\text{ }${}collections$\text{ }${}out$\text{ }${}of$\text{ }${}85,$\text{ }${}disk$\text{ }${}space$\text{ }${}required:$\text{ }${}40$\text{ }${}MB}

or a similar space usage.

You can now proceed to installation: {\itshape \setmainfont[Path=/usr/share/fonts/truetype/cmu/,UprightFont=cmunrm.ttf,BoldFont=cmunbx.ttf,ItalicFont=cmunti.ttf,BoldItalicFont=cmunbi.ttf]{cmunti.ttf}\setmonofont[Path=/usr/share/fonts/truetype/cmu/,UprightFont=cmuntt.ttf,BoldFont=cmuntb.ttf,ItalicFont=cmunit.ttf,BoldItalicFont=cmuntx.ttf]{cmunti.ttf}\itshape start installation to hard disk}\setmainfont[Path=/usr/share/fonts/truetype/cmu/,UprightFont=cmunrm.ttf,BoldFont=cmunbx.ttf,ItalicFont=cmunti.ttf,BoldItalicFont=cmunbi.ttf]{cmunrm.ttf}\setmonofont[Path=/usr/share/fonts/truetype/cmu/,UprightFont=cmuntt.ttf,BoldFont=cmuntb.ttf,ItalicFont=cmunit.ttf,BoldItalicFont=cmuntx.ttf]{cmunrm.ttf}.

Don\textquotesingle{}t forget to add the binaries to your \myhref{https://en.wikipedia.org/wiki/PATH\%20\%28variable\%29}{PATH} as it\textquotesingle{}s noticed at the end of the installation procedure.
\subsection{First test}
\label{17}

In a terminal write\\

\TemplateSpaceIndent{$\text{ }${}\${}$\text{ }${}tex$\text{ }${}\textquotesingle{}\textbackslash{}empty$\text{ }${}Hello$\text{ }${}world!\textbackslash{}bye\textquotesingle{}$\text{ }$\newline{}
$\text{ }${}\${}$\text{ }${}pdftex$\text{ }${}\textquotesingle{}\textbackslash{}empty$\text{ }${}Hello$\text{ }${}world!\textbackslash{}bye\textquotesingle{}}

You should get a DVI or a PDF file accordingly.
\subsection{Configuration}
\label{18}

Formerly, TeX distributions used to be configured with the {\ttfamily \setmainfont[Path=/usr/share/fonts/truetype/cmu/,UprightFont=cmunrm.ttf,BoldFont=cmunbx.ttf,ItalicFont=cmunti.ttf,BoldItalicFont=cmunbi.ttf]{cmuntt.ttf}\setmonofont[Path=/usr/share/fonts/truetype/cmu/,UprightFont=cmuntt.ttf,BoldFont=cmuntb.ttf,ItalicFont=cmunit.ttf,BoldItalicFont=cmuntx.ttf]{cmuntt.ttf}\ttfamily texconfig}{$\text{ }$}\setmainfont[Path=/usr/share/fonts/truetype/cmu/,UprightFont=cmunrm.ttf,BoldFont=cmunbx.ttf,ItalicFont=cmunti.ttf,BoldItalicFont=cmunbi.ttf]{cmunrm.ttf}\setmonofont[Path=/usr/share/fonts/truetype/cmu/,UprightFont=cmuntt.ttf,BoldFont=cmuntb.ttf,ItalicFont=cmunit.ttf,BoldItalicFont=cmuntx.ttf]{cmunrm.ttf} tool from the teTeX distribution. TeX Live still features this tool, but recommends using its own tool instead: {\ttfamily \setmainfont[Path=/usr/share/fonts/truetype/cmu/,UprightFont=cmunrm.ttf,BoldFont=cmunbx.ttf,ItalicFont=cmunti.ttf,BoldItalicFont=cmunbi.ttf]{cmuntt.ttf}\setmonofont[Path=/usr/share/fonts/truetype/cmu/,UprightFont=cmuntt.ttf,BoldFont=cmuntb.ttf,ItalicFont=cmunit.ttf,BoldItalicFont=cmuntx.ttf]{cmuntt.ttf}\ttfamily tlmgr}\setmainfont[Path=/usr/share/fonts/truetype/cmu/,UprightFont=cmunrm.ttf,BoldFont=cmunbx.ttf,ItalicFont=cmunti.ttf,BoldItalicFont=cmunbi.ttf]{cmunrm.ttf}\setmonofont[Path=/usr/share/fonts/truetype/cmu/,UprightFont=cmuntt.ttf,BoldFont=cmuntb.ttf,ItalicFont=cmunit.ttf,BoldItalicFont=cmuntx.ttf]{cmunrm.ttf}.
Note that as of January 2013 not all {\ttfamily \setmainfont[Path=/usr/share/fonts/truetype/cmu/,UprightFont=cmunrm.ttf,BoldFont=cmunbx.ttf,ItalicFont=cmunti.ttf,BoldItalicFont=cmunbi.ttf]{cmuntt.ttf}\setmonofont[Path=/usr/share/fonts/truetype/cmu/,UprightFont=cmuntt.ttf,BoldFont=cmuntb.ttf,ItalicFont=cmunit.ttf,BoldItalicFont=cmuntx.ttf]{cmuntt.ttf}\ttfamily texconfig}{$\text{ }$}\setmainfont[Path=/usr/share/fonts/truetype/cmu/,UprightFont=cmunrm.ttf,BoldFont=cmunbx.ttf,ItalicFont=cmunti.ttf,BoldItalicFont=cmunbi.ttf]{cmunrm.ttf}\setmonofont[Path=/usr/share/fonts/truetype/cmu/,UprightFont=cmuntt.ttf,BoldFont=cmuntb.ttf,ItalicFont=cmunit.ttf,BoldItalicFont=cmuntx.ttf]{cmunrm.ttf} features are implemented by {\ttfamily \setmainfont[Path=/usr/share/fonts/truetype/cmu/,UprightFont=cmunrm.ttf,BoldFont=cmunbx.ttf,ItalicFont=cmunti.ttf,BoldItalicFont=cmunbi.ttf]{cmuntt.ttf}\setmonofont[Path=/usr/share/fonts/truetype/cmu/,UprightFont=cmuntt.ttf,BoldFont=cmuntb.ttf,ItalicFont=cmunit.ttf,BoldItalicFont=cmuntx.ttf]{cmuntt.ttf}\ttfamily tlmgr}\setmainfont[Path=/usr/share/fonts/truetype/cmu/,UprightFont=cmunrm.ttf,BoldFont=cmunbx.ttf,ItalicFont=cmunti.ttf,BoldItalicFont=cmunbi.ttf]{cmunrm.ttf}\setmonofont[Path=/usr/share/fonts/truetype/cmu/,UprightFont=cmuntt.ttf,BoldFont=cmuntb.ttf,ItalicFont=cmunit.ttf,BoldItalicFont=cmuntx.ttf]{cmunrm.ttf}. Only use {\ttfamily \setmainfont[Path=/usr/share/fonts/truetype/cmu/,UprightFont=cmunrm.ttf,BoldFont=cmunbx.ttf,ItalicFont=cmunti.ttf,BoldItalicFont=cmunbi.ttf]{cmuntt.ttf}\setmonofont[Path=/usr/share/fonts/truetype/cmu/,UprightFont=cmuntt.ttf,BoldFont=cmuntb.ttf,ItalicFont=cmunit.ttf,BoldItalicFont=cmuntx.ttf]{cmuntt.ttf}\ttfamily texconfig}{$\text{ }$}\setmainfont[Path=/usr/share/fonts/truetype/cmu/,UprightFont=cmunrm.ttf,BoldFont=cmunbx.ttf,ItalicFont=cmunti.ttf,BoldItalicFont=cmunbi.ttf]{cmunrm.ttf}\setmonofont[Path=/usr/share/fonts/truetype/cmu/,UprightFont=cmuntt.ttf,BoldFont=cmuntb.ttf,ItalicFont=cmunit.ttf,BoldItalicFont=cmuntx.ttf]{cmunrm.ttf} when you cannot do what you want with {\ttfamily \setmainfont[Path=/usr/share/fonts/truetype/cmu/,UprightFont=cmunrm.ttf,BoldFont=cmunbx.ttf,ItalicFont=cmunti.ttf,BoldItalicFont=cmunbi.ttf]{cmuntt.ttf}\setmonofont[Path=/usr/share/fonts/truetype/cmu/,UprightFont=cmuntt.ttf,BoldFont=cmuntb.ttf,ItalicFont=cmunit.ttf,BoldItalicFont=cmuntx.ttf]{cmuntt.ttf}\ttfamily tlmgr}\setmainfont[Path=/usr/share/fonts/truetype/cmu/,UprightFont=cmunrm.ttf,BoldFont=cmunbx.ttf,ItalicFont=cmunti.ttf,BoldItalicFont=cmunbi.ttf]{cmunrm.ttf}\setmonofont[Path=/usr/share/fonts/truetype/cmu/,UprightFont=cmuntt.ttf,BoldFont=cmuntb.ttf,ItalicFont=cmunit.ttf,BoldItalicFont=cmuntx.ttf]{cmunrm.ttf}.

List current installation options:\\

\TemplateSpaceIndent{$\text{ }${}tlmgr$\text{ }${}option}


You can change the install options:\\

\TemplateSpaceIndent{$\text{ }${}tlmgr$\text{ }${}option$\text{ }${}src$\text{ }${}1$\text{ }$\newline{}
$\text{ }${}tlmgr$\text{ }${}option$\text{ }${}doc$\text{ }${}0$\text{ }$\newline{}
$\text{ }${}tlmgr$\text{ }${}option$\text{ }${}paper$\text{ }${}letter}


See the {\ttfamily \setmainfont[Path=/usr/share/fonts/truetype/cmu/,UprightFont=cmunrm.ttf,BoldFont=cmunbx.ttf,ItalicFont=cmunti.ttf,BoldItalicFont=cmunbi.ttf]{cmuntt.ttf}\setmonofont[Path=/usr/share/fonts/truetype/cmu/,UprightFont=cmuntt.ttf,BoldFont=cmuntb.ttf,ItalicFont=cmunit.ttf,BoldItalicFont=cmuntx.ttf]{cmuntt.ttf}\ttfamily TLMGR(1)}{$\text{ }$}\setmainfont[Path=/usr/share/fonts/truetype/cmu/,UprightFont=cmunrm.ttf,BoldFont=cmunbx.ttf,ItalicFont=cmunti.ttf,BoldItalicFont=cmunbi.ttf]{cmunrm.ttf}\setmonofont[Path=/usr/share/fonts/truetype/cmu/,UprightFont=cmuntt.ttf,BoldFont=cmuntb.ttf,ItalicFont=cmunit.ttf,BoldItalicFont=cmuntx.ttf]{cmunrm.ttf} man page for more details on its usage.
If you did not install the documents as told previously, you can still access the {\ttfamily \setmainfont[Path=/usr/share/fonts/truetype/cmu/,UprightFont=cmunrm.ttf,BoldFont=cmunbx.ttf,ItalicFont=cmunti.ttf,BoldItalicFont=cmunbi.ttf]{cmuntt.ttf}\setmonofont[Path=/usr/share/fonts/truetype/cmu/,UprightFont=cmuntt.ttf,BoldFont=cmuntb.ttf,ItalicFont=cmunit.ttf,BoldItalicFont=cmuntx.ttf]{cmuntt.ttf}\ttfamily tlmgr}{$\text{ }$}\setmainfont[Path=/usr/share/fonts/truetype/cmu/,UprightFont=cmunrm.ttf,BoldFont=cmunbx.ttf,ItalicFont=cmunti.ttf,BoldItalicFont=cmunbi.ttf]{cmunrm.ttf}\setmonofont[Path=/usr/share/fonts/truetype/cmu/,UprightFont=cmuntt.ttf,BoldFont=cmuntb.ttf,ItalicFont=cmunit.ttf,BoldItalicFont=cmuntx.ttf]{cmunrm.ttf} man page with\\

\TemplateSpaceIndent{$\text{ }${}tlmgr$\text{ }${}help}

\subsection{Installing LaTeX}
\label{19}

\begin{TemplateInfo}{\danger}{Warning}Do not forget to set the root umask to 022 for all TeX Live administration operations.\end{TemplateInfo}
Now we have a running plain TeX environment, let\textquotesingle{}s install the base packages for LaTeX.
\\

\TemplateSpaceIndent{$\text{ }${}\#$\text{ }${}tlmgr$\text{ }${}install$\text{ }${}latex$\text{ }${}latex-{}bin$\text{ }${}latexconfig$\text{ }${}latex-{}fonts}


In this case you can omit {\ttfamily \setmainfont[Path=/usr/share/fonts/truetype/cmu/,UprightFont=cmunrm.ttf,BoldFont=cmunbx.ttf,ItalicFont=cmunti.ttf,BoldItalicFont=cmunbi.ttf]{cmuntt.ttf}\setmonofont[Path=/usr/share/fonts/truetype/cmu/,UprightFont=cmuntt.ttf,BoldFont=cmuntb.ttf,ItalicFont=cmunit.ttf,BoldItalicFont=cmuntx.ttf]{cmuntt.ttf}\ttfamily latexconfig latex-{}fonts}{$\text{ }$}\setmainfont[Path=/usr/share/fonts/truetype/cmu/,UprightFont=cmunrm.ttf,BoldFont=cmunbx.ttf,ItalicFont=cmunti.ttf,BoldItalicFont=cmunbi.ttf]{cmunrm.ttf}\setmonofont[Path=/usr/share/fonts/truetype/cmu/,UprightFont=cmuntt.ttf,BoldFont=cmuntb.ttf,ItalicFont=cmunit.ttf,BoldItalicFont=cmuntx.ttf]{cmunrm.ttf} as they are auto-{}resolved dependencies to LaTeX.
Note that {\ttfamily \setmainfont[Path=/usr/share/fonts/truetype/cmu/,UprightFont=cmunrm.ttf,BoldFont=cmunbx.ttf,ItalicFont=cmunti.ttf,BoldItalicFont=cmunbi.ttf]{cmuntt.ttf}\setmonofont[Path=/usr/share/fonts/truetype/cmu/,UprightFont=cmuntt.ttf,BoldFont=cmuntb.ttf,ItalicFont=cmunit.ttf,BoldItalicFont=cmuntx.ttf]{cmuntt.ttf}\ttfamily tlmgr}{$\text{ }$}\setmainfont[Path=/usr/share/fonts/truetype/cmu/,UprightFont=cmunrm.ttf,BoldFont=cmunbx.ttf,ItalicFont=cmunti.ttf,BoldItalicFont=cmunbi.ttf]{cmunrm.ttf}\setmonofont[Path=/usr/share/fonts/truetype/cmu/,UprightFont=cmuntt.ttf,BoldFont=cmuntb.ttf,ItalicFont=cmunit.ttf,BoldItalicFont=cmuntx.ttf]{cmunrm.ttf} resolves some dependencies, but not all. You may need to install dependencies manually. Thankfully this is rarely too cumbersome.

Other interesting packages:\\

\TemplateSpaceIndent{$\text{ }${}\#$\text{ }${}tlmgr$\text{ }${}install$\text{ }${}amsmath$\text{ }${}babel$\text{ }${}carlisle$\text{ }${}ec$\text{ }${}geometry$\text{ }${}graphics$\text{ }${}hyperref$\text{ }${}lm$\text{ }${}$\text{ }$\newline{}
$\text{ }${}marvosym$\text{ }${}oberdiek$\text{ }${}parskip$\text{ }${}pdftex-{}def$\text{ }${}url}


\begin{longtable}{>{\RaggedRight}p{0.17133\linewidth}>{\RaggedRight}p{0.77153\linewidth}} 
\hspace*{0pt}\ignorespaces{}\hspace*{0pt} \LaTeXTT{amsmath}&\hspace*{0pt}\ignorespaces{}\hspace*{0pt} The essentials for math typesetting.\\ \hspace*{0pt}\ignorespaces{}\hspace*{0pt} \LaTeXTT{babel}&\hspace*{0pt}\ignorespaces{}\hspace*{0pt} Internationalization support.\\ \hspace*{0pt}\ignorespaces{}\hspace*{0pt} \LaTeXTT{carlisle}&\hspace*{0pt}\ignorespaces{}\hspace*{0pt} Bundle package required for some \LaTeXTT{babel} features.\\ \hspace*{0pt}\ignorespaces{}\hspace*{0pt} \LaTeXTT{ec}&\hspace*{0pt}\ignorespaces{}\hspace*{0pt} Required for T1 encoding.\\ \hspace*{0pt}\ignorespaces{}\hspace*{0pt} \LaTeXTT{geometry}&\hspace*{0pt}\ignorespaces{}\hspace*{0pt} For page layout.\\ \hspace*{0pt}\ignorespaces{}\hspace*{0pt} \LaTeXTT{graphics}&\hspace*{0pt}\ignorespaces{}\hspace*{0pt} The essentials to import graphics.\\ \hspace*{0pt}\ignorespaces{}\hspace*{0pt} \LaTeXTT{hyperref}&\hspace*{0pt}\ignorespaces{}\hspace*{0pt} PDF bookmarks, PDF followable links, link style, TOC links, etc.\\ \hspace*{0pt}\ignorespaces{}\hspace*{0pt} \LaTeXTT{lm}&\hspace*{0pt}\ignorespaces{}\hspace*{0pt} One of the best Computer Modern style font available for several font encodings (such as T1).\\ \hspace*{0pt}\ignorespaces{}\hspace*{0pt} \LaTeXTT{marvosym}&\hspace*{0pt}\ignorespaces{}\hspace*{0pt} Several symbols, such as the official euro.\\ \hspace*{0pt}\ignorespaces{}\hspace*{0pt} \LaTeXTT{oberdiek}&\hspace*{0pt}\ignorespaces{}\hspace*{0pt} Bundle package required for some \LaTeXTT{geometry} features.\\ \hspace*{0pt}\ignorespaces{}\hspace*{0pt} \LaTeXTT{parskip}&\hspace*{0pt}\ignorespaces{}\hspace*{0pt} Let you configure paragraph breaks and indents properly.\\ \hspace*{0pt}\ignorespaces{}\hspace*{0pt} \LaTeXTT{pdftex-{}def}&\hspace*{0pt}\ignorespaces{}\hspace*{0pt} Required for some \LaTeXTT{graphics} features.\\ \hspace*{0pt}\ignorespaces{}\hspace*{0pt} \LaTeXTT{url}&\hspace*{0pt}\ignorespaces{}\hspace*{0pt} Required for some \LaTeXTT{hyperref} features. 
\end{longtable}


If you installed a package you do not need anymore, use\\

\TemplateSpaceIndent{$\text{ }${}\#$\text{ }${}tlmgr$\text{ }${}remove$\text{ }${}<{}package>{}}

\subsection{Hyphenation}
\label{20}

If you are using Babel for non-{}English documents, you need to install the hyphenation patterns for every language you are going to use. They are all packaged individually. For instance, use
\\

\TemplateSpaceIndent{$\text{ }${}\#$\text{ }${}tlmgr$\text{ }${}install$\text{ }${}hyphen-{}\{finnish,sanskrit\}}


for finnish and sanskrit hyphenation patterns.

Note that if you have been using another TeX distribution beforehand, you may still have hyphenation cache stored in you home folder. You need to remove it so that the new packages are taken into account. The TeX Live cache is usually stored in the {\ttfamily \setmainfont[Path=/usr/share/fonts/truetype/cmu/,UprightFont=cmunrm.ttf,BoldFont=cmunbx.ttf,ItalicFont=cmunti.ttf,BoldItalicFont=cmunbi.ttf]{cmuntt.ttf}\setmonofont[Path=/usr/share/fonts/truetype/cmu/,UprightFont=cmuntt.ttf,BoldFont=cmuntb.ttf,ItalicFont=cmunit.ttf,BoldItalicFont=cmuntx.ttf]{cmuntt.ttf}\ttfamily \~{}/.texliveYYYY}{$\text{ }$}\setmainfont[Path=/usr/share/fonts/truetype/cmu/,UprightFont=cmunrm.ttf,BoldFont=cmunbx.ttf,ItalicFont=cmunti.ttf,BoldItalicFont=cmunbi.ttf]{cmunrm.ttf}\setmonofont[Path=/usr/share/fonts/truetype/cmu/,UprightFont=cmuntt.ttf,BoldFont=cmuntb.ttf,ItalicFont=cmunit.ttf,BoldItalicFont=cmuntx.ttf]{cmunrm.ttf} folder ({\ttfamily \setmainfont[Path=/usr/share/fonts/truetype/cmu/,UprightFont=cmunrm.ttf,BoldFont=cmunbx.ttf,ItalicFont=cmunti.ttf,BoldItalicFont=cmunbi.ttf]{cmuntt.ttf}\setmonofont[Path=/usr/share/fonts/truetype/cmu/,UprightFont=cmuntt.ttf,BoldFont=cmuntb.ttf,ItalicFont=cmunit.ttf,BoldItalicFont=cmuntx.ttf]{cmuntt.ttf}\ttfamily YYYY}{$\text{ }$}\setmainfont[Path=/usr/share/fonts/truetype/cmu/,UprightFont=cmunrm.ttf,BoldFont=cmunbx.ttf,ItalicFont=cmunti.ttf,BoldItalicFont=cmunbi.ttf]{cmunrm.ttf}\setmonofont[Path=/usr/share/fonts/truetype/cmu/,UprightFont=cmuntt.ttf,BoldFont=cmuntb.ttf,ItalicFont=cmunit.ttf,BoldItalicFont=cmuntx.ttf]{cmunrm.ttf} stands for the year). You may safely remove this folder as it contains only generated data. TeX compilers will re-{}generate the cache accordingly on next compilation.
\subsection{Uninstallation}
\label{21}
By default TeX Live will install in {\ttfamily \setmainfont[Path=/usr/share/fonts/truetype/cmu/,UprightFont=cmunrm.ttf,BoldFont=cmunbx.ttf,ItalicFont=cmunti.ttf,BoldItalicFont=cmunbi.ttf]{cmuntt.ttf}\setmonofont[Path=/usr/share/fonts/truetype/cmu/,UprightFont=cmuntt.ttf,BoldFont=cmuntb.ttf,ItalicFont=cmunit.ttf,BoldItalicFont=cmuntx.ttf]{cmuntt.ttf}\ttfamily /usr/local/texlive}\setmainfont[Path=/usr/share/fonts/truetype/cmu/,UprightFont=cmunrm.ttf,BoldFont=cmunbx.ttf,ItalicFont=cmunti.ttf,BoldItalicFont=cmunbi.ttf]{cmunrm.ttf}\setmonofont[Path=/usr/share/fonts/truetype/cmu/,UprightFont=cmuntt.ttf,BoldFont=cmuntb.ttf,ItalicFont=cmunit.ttf,BoldItalicFont=cmuntx.ttf]{cmunrm.ttf}. The distribution is quite proper as it will not write any file outside its folder, except for the cache (like font cache, hyphenation patters, etc.). By default,
\begin{myitemize}
\item{}  the system cache goes  in {\ttfamily \setmainfont[Path=/usr/share/fonts/truetype/cmu/,UprightFont=cmunrm.ttf,BoldFont=cmunbx.ttf,ItalicFont=cmunti.ttf,BoldItalicFont=cmunbi.ttf]{cmuntt.ttf}\setmonofont[Path=/usr/share/fonts/truetype/cmu/,UprightFont=cmuntt.ttf,BoldFont=cmuntb.ttf,ItalicFont=cmunit.ttf,BoldItalicFont=cmuntx.ttf]{cmuntt.ttf}\ttfamily /var/lib/texmf}\setmainfont[Path=/usr/share/fonts/truetype/cmu/,UprightFont=cmunrm.ttf,BoldFont=cmunbx.ttf,ItalicFont=cmunti.ttf,BoldItalicFont=cmunbi.ttf]{cmunrm.ttf}\setmonofont[Path=/usr/share/fonts/truetype/cmu/,UprightFont=cmuntt.ttf,BoldFont=cmuntb.ttf,ItalicFont=cmunit.ttf,BoldItalicFont=cmuntx.ttf]{cmunrm.ttf};
\item{}  the user cache goes in {\ttfamily \setmainfont[Path=/usr/share/fonts/truetype/cmu/,UprightFont=cmunrm.ttf,BoldFont=cmunbx.ttf,ItalicFont=cmunti.ttf,BoldItalicFont=cmunbi.ttf]{cmuntt.ttf}\setmonofont[Path=/usr/share/fonts/truetype/cmu/,UprightFont=cmuntt.ttf,BoldFont=cmuntb.ttf,ItalicFont=cmunit.ttf,BoldItalicFont=cmuntx.ttf]{cmuntt.ttf}\ttfamily \~{}/.texliveYYYY}\setmainfont[Path=/usr/share/fonts/truetype/cmu/,UprightFont=cmunrm.ttf,BoldFont=cmunbx.ttf,ItalicFont=cmunti.ttf,BoldItalicFont=cmunbi.ttf]{cmunrm.ttf}\setmonofont[Path=/usr/share/fonts/truetype/cmu/,UprightFont=cmuntt.ttf,BoldFont=cmuntb.ttf,ItalicFont=cmunit.ttf,BoldItalicFont=cmuntx.ttf]{cmunrm.ttf}.
\end{myitemize}


Therefore TeX Live can be installed and uninstalled safely by removing the aforementioned folders.

Still, TeX Live provides a more convenient way to do this:\\

\TemplateSpaceIndent{$\text{ }${}\#$\text{ }${}tlmgr$\text{ }${}uninstall}

You may still have to wipe out the folders if you put untracked files in them.
\section{Editors}
\label{22}

TeX and LaTeX source documents (as well as related files) are all text files, and can be opened and modified in almost any text editor.
You should use a text editor (e.g. Notepad), not a word processor (Word, OpenOffice). Dedicated LaTeX editors are more useful than generic plain text editors, because they usually have autocompletion of commands, spell and error checking and handy macros.
\subsection{Cross-{}platform}
\label{23}
\subsubsection{BaKoMa TeX}
\label{24}

\myhref{http://bakoma-tex.com/menu/about.php}{BaKoMa TeX} is an editor for Windows and Mac OS with WYSIWYG-{}like features. It takes care of compiling the LaTeX source and updating it constantly to view changes to document almost in real time. You can take an evaluation copy for 28 days.
\subsubsection{Emacs}
\label{25}

\myhref{http://www.gnu.org/software/emacs}{Emacs} is a general purpose, extensible text processing system. Advanced users can program it (in elisp) to make Emacs the best LaTeX environment that will fit their needs. In turn beginners may prefer using it in combination with \myhref{http://www.gnu.org/software/auctex/}{AUCTeX} and Reftex (extensions that may be installed into the Emacs program). Depending on your configuration, Emacs can provide a complete LaTeX editing environment with auto-{}completion, spell-{}checking, a complete set of keyboard shortcuts, table of contents view, document preview and many other features.
\subsubsection{gedit-{}latex-{}plugin}
\label{26}

Gedit with \myhref{https://wiki.gnome.org/Apps/Gedit/LaTeXPlugin}{gedit-{}latex-{}plugin} is also worth trying out for users of GNOME. GEdit is a cross-{}platform application for Windows, Mac, and Linux
\subsubsection{Gummi}
\label{27}


\begin{minipage}{1.0\linewidth}
\begin{center}
\includegraphics[width=1.0\linewidth,height=6.5in,keepaspectratio]{../images/1.png}
\end{center}
\raggedright{}\myfigurewithcaption{1}{Screenshot of \myhref{https://en.wikipedia.org/wiki/Gummi\%20\%28software\%29}{Gummi}.}
\end{minipage}\vspace{0.75cm}


\myhref{https://en.wikipedia.org/wiki/Gummi\%20\%28software\%29}{Gummi} is a LaTeX editor for Linux, which compiles the output of pdflatex in realtime and shows it on the right half of the screen\myfootnote{\myfnhref{http://gummi.midnightcoding.org/}{Gummi}}.
\subsubsection{LyX}
\label{28}


\begin{minipage}{1.0\linewidth}
\begin{center}
\includegraphics[width=1.0\linewidth,height=6.5in,keepaspectratio]{../images/2.png}
\end{center}
\raggedright{}\myfigurewithcaption{2}{LyX1.6.3}
\end{minipage}\vspace{0.75cm}



\myhref{https://en.wikipedia.org/wiki/LyX}{LyX} is a popular document preparation system for Windows, Linux and Mac OS. It provides a graphical interface to LaTeX, including several popular packages. It contains formula and table editors and shows visual clues of the final document on the screen enabling users to write LaTeX documents without worrying about the actual syntax. LyX calls this a What You See Is What You Mean (WYSIWYM) approach.\myfootnote{\myfnhref{http://www.lyx.org/}{LyX}}

LyX saves its documents in their own markup, and generates LaTeX code based on this. The user is mostly isolated from the LaTeX code and not in complete control of it, and as such LyX is not a normal LaTeX editor. However, as LaTeX is underlying system, knowledge of how that works is useful also for a LyX user. In addition, if one wants to do something that is not supported in the GUI, using LaTeX code may be required.
\subsubsection{TeXmaker}
\label{29}

\myhref{http://www.xm1math.net/texmaker/}{TeXmaker} is a cross-{}platform editor very similar to Kile in features and user interface. In addition it has its own PDF viewer.
\subsubsection{TeXstudio}
\label{30}

\myhref{http://texstudio.sourceforge.net/}{TeXstudio} is a cross-{}platform open source LaTeX editor forked from Texmaker.
\subsubsection{TeXworks}
\label{31}


\begin{minipage}{1.0\linewidth}
\begin{center}
\includegraphics[width=1.0\linewidth,height=6.5in,keepaspectratio]{../images/3.png}
\end{center}
\raggedright{}\myfigurewithcaption{3}{Screenshot of TeXworks on Ubuntu 12.10.}
\end{minipage}\vspace{0.75cm}


\myhref{https://en.wikipedia.org/wiki/TeXworks}{TeXworks} is a dedicated TeX editor that is included in MiKTeX and TeX Live. It was developed with the idea that a simple interface is better than a cluttered one, and thus to make it easier for people in their early days with LaTeX to get to what they want to do: write their documents. TeXworks originally came about precisely because a math professor wanted his students to have a better initial experience with LaTeX.

You can install TeXworks with the package manager of your Linux distribution or choose it as an install option in the Windows or Mac installer.
\subsubsection{Vim}
\label{32}
\myhref{https://en.wikipedia.org/wiki/Vim\%20\%28text\%20editor\%29}{Vim} is another general purpose text editor for a wide variety of platforms including UNIX, Mac OS X and Windows. A variety of extensions exist including \myhref{http://www.vim.org/scripts/script.php?script_id=3109}{LaTeX Box} and \myhref{http://vim-latex.sourceforge.net/}{Vim-{}LaTeX}.
\subsection{*BSD and GNU/Linux-{}only}
\label{33}
\subsubsection{Kile}
\label{34}


\begin{minipage}{1.0\linewidth}
\begin{center}
\includegraphics[width=1.0\linewidth,height=6.5in,keepaspectratio]{../images/4.png}
\end{center}
\raggedright{}\myfigurewithcaption{4}{Screenshot of \myhref{https://en.wikipedia.org/wiki/Kile}{Kile}.}
\end{minipage}\vspace{0.75cm}


\myhref{http://kile.sourceforge.net/}{Kile} is a LaTeX editor for \myhref{http://en.wikipedia.org/wiki/KDE_Software_Compilation_4}{KDE} (cross platform), providing a powerful GUI for editing multiple documents and compiling them with many different TeX compilers. Kile is based on Kate editor, has a quick access toolbar for symbols, document structure viewer, a console and customizable build options. Kile can be run in all operating systems that can run KDE.
\subsubsection{LaTeXila}
\label{35}

\myhref{http://projects.gnome.org/latexila/}{LaTeXila} is another text editor for Linux (Gnome).
\subsection{Mac OS X-{}only}
\label{36}
\subsubsection{TeXShop}
\label{37}

\myhref{http://www.uoregon.edu/~koch/texshop/}{TeXShop} is a TeXworks-{}like editor and previewer for Mac OS that is bundled with the MacTeX distribution.  It uses multiple windows, one for editing the source, one for the preview, and one as a console for error messages.  It offers one-{}click updating of the preview and allows easy crossfinding between the code and the preview by using CMD-{}click.
\subsubsection{TeXnicle}
\label{38}
\myhref{http://www.bobsoft-mac.de/texnicle/texnicle.html}{TeXnicle} is a free editor for Mac OS that includes the ability to perform live updates. It includes a code library for the swift insertion of code and the ability to execute detailed word counts on documents. It also performs code highlighting and the editing window is customisable, permitting the user to select the font, colour, background colour of the editing environment. It is in active development.
\subsubsection{Archimedes}
\label{39}

\myhref{http://www.mattrajca.com/archimedes}{Archimedes} is an easy-{}to-{}use LaTeX and Markdown editor designed from the ground up for Mac OS X. It includes a built-{}in LaTeX library, code completion support, live previews, macro support, integration with sharing services, and PDF and HTML export options. Archimedes\textquotesingle{}s Magic Type feature lets users insert mathematical symbols just by drawing them on their MacBook\textquotesingle{}s trackpad or Magic Trackpad.
\subsection{Windows-{}only}
\label{40}
\subsubsection{LEd}
\label{41}
\myhref{http://www.latexeditor.org/}{LEd}
\subsubsection{TeXnicCenter}
\label{42}

\myhref{http://www.texniccenter.org/}{TeXnicCenter} is a popular free and open source LaTeX editor for Windows. It also has a similar user interface to TeXmaker and Kile.
\subsubsection{WinEdt}
\label{43}

\myhref{http://www.winedt.com/}{WinEdt} is a powerful and versatile text editor with strong predisposition towards creation of LaTeX/TeX documents for Windows. It has been designed and configured to integrate with TeX Systems such as MiTeX or TeX Live. Its built-{}in macro helps in  compiling the LaTeX source to the WYSIWYG-{}like DVI or PDF or PS and also in exporting the document to other mark-{}up languages as HTML or XML.
\subsubsection{WinShell}
\label{44}
\myhref{http://www.winshell.de/}{WinShell}
\subsection{Online solutions}
\label{45}

To get started without needing to install anything, you can use a web-{}hosted service featuring a full TeX distribution and a web LaTeX editor.

\begin{myitemize}
\item{}  \myhref{https://authorea.com}{Authorea} is an integrated online framework for the creation of technical documents in collaboration. Authorea\textquotesingle{}s frontend allows you to enter text in LaTeX or Markdown, as well as figures, and equations (in LaTeX or MathML). Authorea\textquotesingle{}s versioning control system is entirely based on Git (every article is a Git repository).
\end{myitemize}


\begin{myitemize}
\item{}  \myhref{https://www.overleaf.com}{Overleaf} is a secure, easy to use online LaTeX editor with integrated rapid preview -{} like \myhref{http://en.wikipedia.org/wiki/Etherpad}{EtherPad} for LaTeX. Start writing with one click (no signup required) and share the link. It supports real time preview, figures, bibliographies and custom styles.
\end{myitemize}


\begin{myitemize}
\item{}  \myhref{http://blue.publications.li}{publications.li} is a real-{}time collaborative LaTeX editor running the open-{}source editor \myhref{https://github.com/gnieh/bluelatex}{\textbackslash{}BlueLatex}.
\end{myitemize}


\begin{myitemize}
\item{}  \myhref{https://www.sharelatex.com}{ShareLaTeX.com} is a secure cloud-{}based LaTeX editor offering unlimited free project. Premium accounts are available for extra features such as version control and Dropbox integration.
\end{myitemize}


\begin{myitemize}
\item{}  \myhref{http://simplelatex.com}{SimpleLaTeX} is an online editor and previewer for short LaTeX notes, which can be optionally cached or shared. Previews are available in SVG, PNG, and PDF. It also includes a simple GUI for editing tables.
\end{myitemize}


\begin{myitemize}
\item{}  \myhref{http://www.verbosus.com}{Verbosus} is a professional Online LaTeX Editor that supports collaboration with other users and is free to use. Merge conflicts can easily resolved by using a built-{}in merge tool that uses an implementation of the diff-{}algorithm to generate information required for a successful merge.
\end{myitemize}

\section{Bibliography management}
\label{46}
Bibliography files ({\ttfamily \setmainfont[Path=/usr/share/fonts/truetype/cmu/,UprightFont=cmunrm.ttf,BoldFont=cmunbx.ttf,ItalicFont=cmunti.ttf,BoldItalicFont=cmunbi.ttf]{cmuntt.ttf}\setmonofont[Path=/usr/share/fonts/truetype/cmu/,UprightFont=cmuntt.ttf,BoldFont=cmuntb.ttf,ItalicFont=cmunit.ttf,BoldItalicFont=cmuntx.ttf]{cmuntt.ttf}\ttfamily *.bib}\setmainfont[Path=/usr/share/fonts/truetype/cmu/,UprightFont=cmunrm.ttf,BoldFont=cmunbx.ttf,ItalicFont=cmunti.ttf,BoldItalicFont=cmunbi.ttf]{cmunrm.ttf}\setmonofont[Path=/usr/share/fonts/truetype/cmu/,UprightFont=cmuntt.ttf,BoldFont=cmuntb.ttf,ItalicFont=cmunit.ttf,BoldItalicFont=cmuntx.ttf]{cmunrm.ttf}) are most easily edited and modified using a management system. These graphical user interfaces all feature a database form, where information is entered for each reference item, and the resulting text file can be used directly by BibTeX.
\subsection{Cross-{}platform}
\label{47}


\begin{minipage}{1.0\linewidth}
\begin{center}
\includegraphics[width=1.0\linewidth,height=6.5in,keepaspectratio]{../images/5.png}
\end{center}
\raggedright{}\myfigurewithcaption{5}{Screenshot of \myhref{https://en.wikipedia.org/wiki/JabRef}{JabRef}.}
\end{minipage}\vspace{0.75cm}


\begin{myitemize}
\item{}  \myhref{http://jabref.sourceforge.net/}{JabRef}
\item{}  \myhref{http://www.mendeley.com//}{Mendeley}
\item{}  \myhref{https://zotero.org/}{Zotero}
\end{myitemize}

\subsection{Mac OS X-{}only}
\label{48}


\begin{minipage}{1.0\linewidth}
\begin{center}
\includegraphics[width=1.0\linewidth,height=6.5in,keepaspectratio]{../images/6.jpg}
\end{center}
\raggedright{}\myfigurewithcaption{6}{Screenshot of BibDesk}
\end{minipage}\vspace{0.75cm}


\begin{myitemize}
\item{}  \myhref{https://en.wikipedia.org/wiki/BibDesk}{BibDesk} is a bibliography manager based on a BibTeX file. It imports references from the internet and makes it easy to organize references using tags and categories\myfootnote{\myfnhref{http://bibdesk.sourceforge.net/}{BibDesk}}.
\end{myitemize}

\section{Viewers}
\label{49}

Finally, you will need a viewer for the files LaTeX outputs. Normally LaTeX saves the final document as a {\ttfamily \setmainfont[Path=/usr/share/fonts/truetype/cmu/,UprightFont=cmunrm.ttf,BoldFont=cmunbx.ttf,ItalicFont=cmunti.ttf,BoldItalicFont=cmunbi.ttf]{cmuntt.ttf}\setmonofont[Path=/usr/share/fonts/truetype/cmu/,UprightFont=cmuntt.ttf,BoldFont=cmuntb.ttf,ItalicFont=cmunit.ttf,BoldItalicFont=cmuntx.ttf]{cmuntt.ttf}\ttfamily .dvi}{$\text{ }$}\setmainfont[Path=/usr/share/fonts/truetype/cmu/,UprightFont=cmunrm.ttf,BoldFont=cmunbx.ttf,ItalicFont=cmunti.ttf,BoldItalicFont=cmunbi.ttf]{cmunrm.ttf}\setmonofont[Path=/usr/share/fonts/truetype/cmu/,UprightFont=cmuntt.ttf,BoldFont=cmuntb.ttf,ItalicFont=cmunit.ttf,BoldItalicFont=cmuntx.ttf]{cmunrm.ttf} (Device independent file format), but you will rarely want it to. DVI files do not contain embedded fonts and many document viewers are unable to open them.

Usually you will use a LaTeX compiler like {\ttfamily \setmainfont[Path=/usr/share/fonts/truetype/cmu/,UprightFont=cmunrm.ttf,BoldFont=cmunbx.ttf,ItalicFont=cmunti.ttf,BoldItalicFont=cmunbi.ttf]{cmuntt.ttf}\setmonofont[Path=/usr/share/fonts/truetype/cmu/,UprightFont=cmuntt.ttf,BoldFont=cmuntb.ttf,ItalicFont=cmunit.ttf,BoldItalicFont=cmuntx.ttf]{cmuntt.ttf}\ttfamily pdflatex}{$\text{ }$}\setmainfont[Path=/usr/share/fonts/truetype/cmu/,UprightFont=cmunrm.ttf,BoldFont=cmunbx.ttf,ItalicFont=cmunti.ttf,BoldItalicFont=cmunbi.ttf]{cmunrm.ttf}\setmonofont[Path=/usr/share/fonts/truetype/cmu/,UprightFont=cmuntt.ttf,BoldFont=cmuntb.ttf,ItalicFont=cmunit.ttf,BoldItalicFont=cmuntx.ttf]{cmunrm.ttf} to produce a PDF file directly, or a tool like {\ttfamily \setmainfont[Path=/usr/share/fonts/truetype/cmu/,UprightFont=cmunrm.ttf,BoldFont=cmunbx.ttf,ItalicFont=cmunti.ttf,BoldItalicFont=cmunbi.ttf]{cmuntt.ttf}\setmonofont[Path=/usr/share/fonts/truetype/cmu/,UprightFont=cmuntt.ttf,BoldFont=cmuntb.ttf,ItalicFont=cmunit.ttf,BoldItalicFont=cmuntx.ttf]{cmuntt.ttf}\ttfamily dvi2pdf}{$\text{ }$}\setmainfont[Path=/usr/share/fonts/truetype/cmu/,UprightFont=cmunrm.ttf,BoldFont=cmunbx.ttf,ItalicFont=cmunti.ttf,BoldItalicFont=cmunbi.ttf]{cmunrm.ttf}\setmonofont[Path=/usr/share/fonts/truetype/cmu/,UprightFont=cmuntt.ttf,BoldFont=cmuntb.ttf,ItalicFont=cmunit.ttf,BoldItalicFont=cmuntx.ttf]{cmunrm.ttf} to convert the DVI file to PDF format. Then you can view the result with any PDF viewer.

Practically all LaTeX distributions have a DVI viewer for viewing the default output of {\ttfamily \setmainfont[Path=/usr/share/fonts/truetype/cmu/,UprightFont=cmunrm.ttf,BoldFont=cmunbx.ttf,ItalicFont=cmunti.ttf,BoldItalicFont=cmunbi.ttf]{cmuntt.ttf}\setmonofont[Path=/usr/share/fonts/truetype/cmu/,UprightFont=cmuntt.ttf,BoldFont=cmuntb.ttf,ItalicFont=cmunit.ttf,BoldItalicFont=cmuntx.ttf]{cmuntt.ttf}\ttfamily latex}\setmainfont[Path=/usr/share/fonts/truetype/cmu/,UprightFont=cmunrm.ttf,BoldFont=cmunbx.ttf,ItalicFont=cmunti.ttf,BoldItalicFont=cmunbi.ttf]{cmunrm.ttf}\setmonofont[Path=/usr/share/fonts/truetype/cmu/,UprightFont=cmuntt.ttf,BoldFont=cmuntb.ttf,ItalicFont=cmunit.ttf,BoldItalicFont=cmuntx.ttf]{cmunrm.ttf}, and also tools such as {\ttfamily \setmainfont[Path=/usr/share/fonts/truetype/cmu/,UprightFont=cmunrm.ttf,BoldFont=cmunbx.ttf,ItalicFont=cmunti.ttf,BoldItalicFont=cmunbi.ttf]{cmuntt.ttf}\setmonofont[Path=/usr/share/fonts/truetype/cmu/,UprightFont=cmuntt.ttf,BoldFont=cmuntb.ttf,ItalicFont=cmunit.ttf,BoldItalicFont=cmuntx.ttf]{cmuntt.ttf}\ttfamily dvi2pdf}{$\text{ }$}\setmainfont[Path=/usr/share/fonts/truetype/cmu/,UprightFont=cmunrm.ttf,BoldFont=cmunbx.ttf,ItalicFont=cmunti.ttf,BoldItalicFont=cmunbi.ttf]{cmunrm.ttf}\setmonofont[Path=/usr/share/fonts/truetype/cmu/,UprightFont=cmuntt.ttf,BoldFont=cmuntb.ttf,ItalicFont=cmunit.ttf,BoldItalicFont=cmuntx.ttf]{cmunrm.ttf} for converting the result automatically to PDF and PS formats.

Here follows a list of various PDF viewers.
\begin{myitemize}
\item{}  PDF.js (built-{}in modern browsers)
\item{}  \myhref{https://wiki.gnome.org/Apps/Evince}{Evince} (Linux GNOME)
\item{}  \myhref{http://www.foxitsoftware.com/Secure_PDF_Reader/}{Foxit} (Windows)
\item{}  \myhref{https://okular.kde.org/}{Okular} (Linux KDE)
\item{}  Preview (built-{}in Mac OS X)
\item{}  \myhref{http://skim-app.sourceforge.net/}{Skim} (Mac OS X)
\item{}  \myhref{http://www.sumatrapdfreader.org/free-pdf-reader.html}{Sumatra PDF} (Windows)
\item{}  \myhref{http://www.foolabs.com/xpdf/about.html}{Xpdf} (Linux)
\item{}  \myhref{https://pwmt.org/projects/zathura/}{Zathura} (Linux)
\end{myitemize}

\section{Tables and graphics tools}
\label{50}

LaTeX is a document preparation system, it does not aim at being a spreadsheet tool nor a vector graphics tool.

If LaTeX can render beautiful tables in a dynamic and flexible manner, it will not handle the handy features you could get with a spreadsheet like dynamic cells and calculus. Other tools are better at that. The ideal solution is to combine the strength of both tools: build your dynamic table with a spreadsheet, and export it to LaTeX to get a beautiful table seamlessly integrated to your document. See \mylref{248}{Tables} for more details.

The graphics topic is a bit different since it is possible to write \mylref{774}{procedural graphics} from within your LaTeX document. Procedural graphics produce state-{}of-{}the-{}art results that integrates perfectly to LaTeX ({\itshape \setmainfont[Path=/usr/share/fonts/truetype/cmu/,UprightFont=cmunrm.ttf,BoldFont=cmunbx.ttf,ItalicFont=cmunti.ttf,BoldItalicFont=cmunbi.ttf]{cmunti.ttf}\setmonofont[Path=/usr/share/fonts/truetype/cmu/,UprightFont=cmuntt.ttf,BoldFont=cmuntb.ttf,ItalicFont=cmunit.ttf,BoldItalicFont=cmuntx.ttf]{cmunti.ttf}\itshape e.g.}{$\text{ }$}\setmainfont[Path=/usr/share/fonts/truetype/cmu/,UprightFont=cmunrm.ttf,BoldFont=cmunbx.ttf,ItalicFont=cmunti.ttf,BoldItalicFont=cmunbi.ttf]{cmunrm.ttf}\setmonofont[Path=/usr/share/fonts/truetype/cmu/,UprightFont=cmuntt.ttf,BoldFont=cmuntb.ttf,ItalicFont=cmunit.ttf,BoldItalicFont=cmuntx.ttf]{cmunrm.ttf} no font change), but have a steep learning curve and require a lot of time to draw.

For easier and quicker drawings, you may want to use a WYSIWYG tool and export the result to a vector format like PDF. The drawback is that it will contrast in style with the rest of your document (font, size, etc.). Some tools have the capability to export to LaTeX, which will partially solve this issue. See \mylref{336}{Importing Graphics} for more details.
\section{References}
\label{51}




\myhref{https://de.wikibooks.org/wiki/LaTeX\%2F_Installation}{de:LaTeX/\_Installation}
\myhref{https://sr.wikibooks.org/wiki/LaTeX\%2F\%D0\%98\%D0\%BD\%D1\%81\%D1\%82\%D0\%B0\%D0\%BB\%D0\%B0\%D1\%86\%D0\%B8\%D1\%98\%D0\%B0}{sr:LaTeX/Инсталација}\chapter{Installing Extra Packages}

\myminitoc
\label{52}

\label{53}


Add-{}on features for LaTeX are known as packages. Dozens of these are pre-{}installed with LaTeX and can be used in your documents immediately. They should all be stored in subdirectories of {\ttfamily \setmainfont[Path=/usr/share/fonts/truetype/cmu/,UprightFont=cmunrm.ttf,BoldFont=cmunbx.ttf,ItalicFont=cmunti.ttf,BoldItalicFont=cmunbi.ttf]{cmuntt.ttf}\setmonofont[Path=/usr/share/fonts/truetype/cmu/,UprightFont=cmuntt.ttf,BoldFont=cmuntb.ttf,ItalicFont=cmunit.ttf,BoldItalicFont=cmuntx.ttf]{cmuntt.ttf}\ttfamily texmf/tex/latex}{$\text{ }$}\setmainfont[Path=/usr/share/fonts/truetype/cmu/,UprightFont=cmunrm.ttf,BoldFont=cmunbx.ttf,ItalicFont=cmunti.ttf,BoldItalicFont=cmunbi.ttf]{cmunrm.ttf}\setmonofont[Path=/usr/share/fonts/truetype/cmu/,UprightFont=cmuntt.ttf,BoldFont=cmuntb.ttf,ItalicFont=cmunit.ttf,BoldItalicFont=cmuntx.ttf]{cmunrm.ttf} named after each package. The directory name \symbol{34}texmf\symbol{34} stands for “TEX and METAFONT”. To find out what other packages are available and what they do, you should use the \myhref{http://www.ctan.org/search.html}{CTAN search page} which includes a link to Graham Williams\textquotesingle{} comprehensive package catalogue.

A package is a file or collection of files containing extra LaTeX commands and programming which add new styling features or modify those already existing. There are two main file types: class files with {\ttfamily \setmainfont[Path=/usr/share/fonts/truetype/cmu/,UprightFont=cmunrm.ttf,BoldFont=cmunbx.ttf,ItalicFont=cmunti.ttf,BoldItalicFont=cmunbi.ttf]{cmuntt.ttf}\setmonofont[Path=/usr/share/fonts/truetype/cmu/,UprightFont=cmuntt.ttf,BoldFont=cmuntb.ttf,ItalicFont=cmunit.ttf,BoldItalicFont=cmuntx.ttf]{cmuntt.ttf}\ttfamily .cls}{$\text{ }$}\setmainfont[Path=/usr/share/fonts/truetype/cmu/,UprightFont=cmunrm.ttf,BoldFont=cmunbx.ttf,ItalicFont=cmunti.ttf,BoldItalicFont=cmunbi.ttf]{cmunrm.ttf}\setmonofont[Path=/usr/share/fonts/truetype/cmu/,UprightFont=cmuntt.ttf,BoldFont=cmuntb.ttf,ItalicFont=cmunit.ttf,BoldItalicFont=cmuntx.ttf]{cmunrm.ttf} extension, and style files with {\ttfamily \setmainfont[Path=/usr/share/fonts/truetype/cmu/,UprightFont=cmunrm.ttf,BoldFont=cmunbx.ttf,ItalicFont=cmunti.ttf,BoldItalicFont=cmunbi.ttf]{cmuntt.ttf}\setmonofont[Path=/usr/share/fonts/truetype/cmu/,UprightFont=cmuntt.ttf,BoldFont=cmuntb.ttf,ItalicFont=cmunit.ttf,BoldItalicFont=cmuntx.ttf]{cmuntt.ttf}\ttfamily .sty}{$\text{ }$}\setmainfont[Path=/usr/share/fonts/truetype/cmu/,UprightFont=cmunrm.ttf,BoldFont=cmunbx.ttf,ItalicFont=cmunti.ttf,BoldItalicFont=cmunbi.ttf]{cmunrm.ttf}\setmonofont[Path=/usr/share/fonts/truetype/cmu/,UprightFont=cmuntt.ttf,BoldFont=cmuntb.ttf,ItalicFont=cmunit.ttf,BoldItalicFont=cmuntx.ttf]{cmunrm.ttf} extension. There may be ancillary files as well. When you try to typeset a document which requires a package which is not installed on your system, LaTeX will warn you with an error message that it is missing. You can download updates to packages you already have (both the ones that were installed along with your version of LaTeX as well as ones you added). There is no limit to the number of packages you can have installed on your computer (apart from disk space!), but there is a configurable limit to the number that can be used inside any one LaTeX document at the same time, although it depends on how big each package is. In practice there is no problem in having even a couple of dozen packages active.

Most LaTeX installations come with a large set of pre-{}installed style packages, so you can use the package manager of the TeX distribution or the one on your system to manage them. See the automatic installation. But many more are available on the net. The main place to look for style packages on the Internet is \myhref{http://www.ctan.org/}{CTAN}. Once you have identified a package you need that is not in your distribution, use the indexes on any CTAN server to find the package you need and the directory where it can be downloaded from. See the manual installation.
\section{Automatic installation}
\label{54}
If on an operating system with a package manager or a portage tree, you can often find packages in repositories.

With MikTeX there is a package manager that allows you to pick the package you want individually. As a convenient feature, upon the compilation of a file requiring non-{}installed packages, MikTeX will automatically prompt to install the missing ones.

With TeX Live, it is common to have the distribution packed into a few big packages. For example, to install something related to internationalization, you might have to install a package like {\ttfamily \setmainfont[Path=/usr/share/fonts/truetype/cmu/,UprightFont=cmunrm.ttf,BoldFont=cmunbx.ttf,ItalicFont=cmunti.ttf,BoldItalicFont=cmunbi.ttf]{cmuntt.ttf}\setmonofont[Path=/usr/share/fonts/truetype/cmu/,UprightFont=cmuntt.ttf,BoldFont=cmuntb.ttf,ItalicFont=cmunit.ttf,BoldItalicFont=cmuntx.ttf]{cmuntt.ttf}\ttfamily texlive-{}lang}\setmainfont[Path=/usr/share/fonts/truetype/cmu/,UprightFont=cmunrm.ttf,BoldFont=cmunbx.ttf,ItalicFont=cmunti.ttf,BoldItalicFont=cmunbi.ttf]{cmunrm.ttf}\setmonofont[Path=/usr/share/fonts/truetype/cmu/,UprightFont=cmuntt.ttf,BoldFont=cmuntb.ttf,ItalicFont=cmunit.ttf,BoldItalicFont=cmuntx.ttf]{cmunrm.ttf}.
With TeX Live manually installed, use {\ttfamily \setmainfont[Path=/usr/share/fonts/truetype/cmu/,UprightFont=cmunrm.ttf,BoldFont=cmunbx.ttf,ItalicFont=cmunti.ttf,BoldItalicFont=cmunbi.ttf]{cmuntt.ttf}\setmonofont[Path=/usr/share/fonts/truetype/cmu/,UprightFont=cmuntt.ttf,BoldFont=cmuntb.ttf,ItalicFont=cmunit.ttf,BoldItalicFont=cmuntx.ttf]{cmuntt.ttf}\ttfamily tlmgr}{$\text{ }$}\setmainfont[Path=/usr/share/fonts/truetype/cmu/,UprightFont=cmunrm.ttf,BoldFont=cmunbx.ttf,ItalicFont=cmunti.ttf,BoldItalicFont=cmunbi.ttf]{cmunrm.ttf}\setmonofont[Path=/usr/share/fonts/truetype/cmu/,UprightFont=cmuntt.ttf,BoldFont=cmuntb.ttf,ItalicFont=cmunit.ttf,BoldItalicFont=cmuntx.ttf]{cmunrm.ttf} to manage packages individually.
\\

\TemplateSpaceIndent{$\text{ }${}tlmgr$\text{ }${}install$\text{ }${}<{}package1>{}$\text{ }${}<{}package2>{}$\text{ }${}...$\text{ }$\newline{}
$\text{ }${}tlmgr$\text{ }${}remove$\text{ }${}<{}package1>{}$\text{ }${}<{}package2>{}$\text{ }${}...}


The use of {\ttfamily \setmainfont[Path=/usr/share/fonts/truetype/cmu/,UprightFont=cmunrm.ttf,BoldFont=cmunbx.ttf,ItalicFont=cmunti.ttf,BoldItalicFont=cmunbi.ttf]{cmuntt.ttf}\setmonofont[Path=/usr/share/fonts/truetype/cmu/,UprightFont=cmuntt.ttf,BoldFont=cmuntb.ttf,ItalicFont=cmunit.ttf,BoldItalicFont=cmuntx.ttf]{cmuntt.ttf}\ttfamily tlmgr}{$\text{ }$}\setmainfont[Path=/usr/share/fonts/truetype/cmu/,UprightFont=cmunrm.ttf,BoldFont=cmunbx.ttf,ItalicFont=cmunti.ttf,BoldItalicFont=cmunbi.ttf]{cmunrm.ttf}\setmonofont[Path=/usr/share/fonts/truetype/cmu/,UprightFont=cmuntt.ttf,BoldFont=cmuntb.ttf,ItalicFont=cmunit.ttf,BoldItalicFont=cmuntx.ttf]{cmunrm.ttf} is covered in the \mylref{10}{Installation} chapter.

If you cannot find the wanted package with any of the previous methods, see the manual installation.
\section{Manual installation}
\label{55}
\subsection{Downloading packages}
\label{56}

What you need to look for is usually two files, one ending in {\ttfamily \setmainfont[Path=/usr/share/fonts/truetype/cmu/,UprightFont=cmunrm.ttf,BoldFont=cmunbx.ttf,ItalicFont=cmunti.ttf,BoldItalicFont=cmunbi.ttf]{cmuntt.ttf}\setmonofont[Path=/usr/share/fonts/truetype/cmu/,UprightFont=cmuntt.ttf,BoldFont=cmuntb.ttf,ItalicFont=cmunit.ttf,BoldItalicFont=cmuntx.ttf]{cmuntt.ttf}\ttfamily .dtx}{$\text{ }$}\setmainfont[Path=/usr/share/fonts/truetype/cmu/,UprightFont=cmunrm.ttf,BoldFont=cmunbx.ttf,ItalicFont=cmunti.ttf,BoldItalicFont=cmunbi.ttf]{cmunrm.ttf}\setmonofont[Path=/usr/share/fonts/truetype/cmu/,UprightFont=cmuntt.ttf,BoldFont=cmuntb.ttf,ItalicFont=cmunit.ttf,BoldItalicFont=cmuntx.ttf]{cmunrm.ttf} and the other in {\ttfamily \setmainfont[Path=/usr/share/fonts/truetype/cmu/,UprightFont=cmunrm.ttf,BoldFont=cmunbx.ttf,ItalicFont=cmunti.ttf,BoldItalicFont=cmunbi.ttf]{cmuntt.ttf}\setmonofont[Path=/usr/share/fonts/truetype/cmu/,UprightFont=cmuntt.ttf,BoldFont=cmuntb.ttf,ItalicFont=cmunit.ttf,BoldItalicFont=cmuntx.ttf]{cmuntt.ttf}\ttfamily .ins}\setmainfont[Path=/usr/share/fonts/truetype/cmu/,UprightFont=cmunrm.ttf,BoldFont=cmunbx.ttf,ItalicFont=cmunti.ttf,BoldItalicFont=cmunbi.ttf]{cmunrm.ttf}\setmonofont[Path=/usr/share/fonts/truetype/cmu/,UprightFont=cmuntt.ttf,BoldFont=cmuntb.ttf,ItalicFont=cmunit.ttf,BoldItalicFont=cmuntx.ttf]{cmunrm.ttf}. The first is a DOCTeX file, which combines the package program and its documentation in a single file. The second is the installation routine (much smaller). You must always download both files. If the two files are not there, it means one of two things:
\begin{myitemize}
\item{}  {\itshape \setmainfont[Path=/usr/share/fonts/truetype/cmu/,UprightFont=cmunrm.ttf,BoldFont=cmunbx.ttf,ItalicFont=cmunti.ttf,BoldItalicFont=cmunbi.ttf]{cmunti.ttf}\setmonofont[Path=/usr/share/fonts/truetype/cmu/,UprightFont=cmuntt.ttf,BoldFont=cmuntb.ttf,ItalicFont=cmunit.ttf,BoldItalicFont=cmuntx.ttf]{cmunti.ttf}\itshape Either}{$\text{ }$}\setmainfont[Path=/usr/share/fonts/truetype/cmu/,UprightFont=cmunrm.ttf,BoldFont=cmunbx.ttf,ItalicFont=cmunti.ttf,BoldItalicFont=cmunbi.ttf]{cmunrm.ttf}\setmonofont[Path=/usr/share/fonts/truetype/cmu/,UprightFont=cmuntt.ttf,BoldFont=cmuntb.ttf,ItalicFont=cmunit.ttf,BoldItalicFont=cmuntx.ttf]{cmunrm.ttf} the package is part of a much larger bundle which you shouldn\textquotesingle{}t normally update unless you change 

UNKNOWN TEMPLATE  
FULLBOOKNAME

{}

version of LaTeX;
\item{}  {\itshape \setmainfont[Path=/usr/share/fonts/truetype/cmu/,UprightFont=cmunrm.ttf,BoldFont=cmunbx.ttf,ItalicFont=cmunti.ttf,BoldItalicFont=cmunbi.ttf]{cmunti.ttf}\setmonofont[Path=/usr/share/fonts/truetype/cmu/,UprightFont=cmuntt.ttf,BoldFont=cmuntb.ttf,ItalicFont=cmunit.ttf,BoldItalicFont=cmuntx.ttf]{cmunti.ttf}\itshape or}{$\text{ }$}\setmainfont[Path=/usr/share/fonts/truetype/cmu/,UprightFont=cmunrm.ttf,BoldFont=cmunbx.ttf,ItalicFont=cmunti.ttf,BoldItalicFont=cmunbi.ttf]{cmunrm.ttf}\setmonofont[Path=/usr/share/fonts/truetype/cmu/,UprightFont=cmuntt.ttf,BoldFont=cmuntb.ttf,ItalicFont=cmunit.ttf,BoldItalicFont=cmuntx.ttf]{cmunrm.ttf} it\textquotesingle{}s an older or relatively simple package written by an author who did not use a {\ttfamily \setmainfont[Path=/usr/share/fonts/truetype/cmu/,UprightFont=cmunrm.ttf,BoldFont=cmunbx.ttf,ItalicFont=cmunti.ttf,BoldItalicFont=cmunbi.ttf]{cmuntt.ttf}\setmonofont[Path=/usr/share/fonts/truetype/cmu/,UprightFont=cmuntt.ttf,BoldFont=cmuntb.ttf,ItalicFont=cmunit.ttf,BoldItalicFont=cmuntx.ttf]{cmuntt.ttf}\ttfamily .dtx}{$\text{ }$}\setmainfont[Path=/usr/share/fonts/truetype/cmu/,UprightFont=cmunrm.ttf,BoldFont=cmunbx.ttf,ItalicFont=cmunti.ttf,BoldItalicFont=cmunbi.ttf]{cmunrm.ttf}\setmonofont[Path=/usr/share/fonts/truetype/cmu/,UprightFont=cmuntt.ttf,BoldFont=cmuntb.ttf,ItalicFont=cmunit.ttf,BoldItalicFont=cmuntx.ttf]{cmunrm.ttf} file. 
\end{myitemize}


Download the package files to a temporary directory. There will often be a {\ttfamily \setmainfont[Path=/usr/share/fonts/truetype/cmu/,UprightFont=cmunrm.ttf,BoldFont=cmunbx.ttf,ItalicFont=cmunti.ttf,BoldItalicFont=cmunbi.ttf]{cmuntt.ttf}\setmonofont[Path=/usr/share/fonts/truetype/cmu/,UprightFont=cmuntt.ttf,BoldFont=cmuntb.ttf,ItalicFont=cmunit.ttf,BoldItalicFont=cmuntx.ttf]{cmuntt.ttf}\ttfamily readme.txt}{$\text{ }$}\setmainfont[Path=/usr/share/fonts/truetype/cmu/,UprightFont=cmunrm.ttf,BoldFont=cmunbx.ttf,ItalicFont=cmunti.ttf,BoldItalicFont=cmunbi.ttf]{cmunrm.ttf}\setmonofont[Path=/usr/share/fonts/truetype/cmu/,UprightFont=cmuntt.ttf,BoldFont=cmuntb.ttf,ItalicFont=cmunit.ttf,BoldItalicFont=cmuntx.ttf]{cmunrm.ttf} with a brief description of the package. You should of course read this file first.
\subsection{Installing a package}
\label{57}
There are five steps to installing a LaTeX package. (These steps can also be used on the pieces of a complicated package you wrote yourself; in this case, skip straight to Step 3.)

1. {\bfseries \setmainfont[Path=/usr/share/fonts/truetype/cmu/,UprightFont=cmunrm.ttf,BoldFont=cmunbx.ttf,ItalicFont=cmunti.ttf,BoldItalicFont=cmunbi.ttf]{cmunbx.ttf}\setmonofont[Path=/usr/share/fonts/truetype/cmu/,UprightFont=cmuntt.ttf,BoldFont=cmuntb.ttf,ItalicFont=cmunit.ttf,BoldItalicFont=cmuntx.ttf]{cmunbx.ttf}\bfseries Extract the files}{$\text{ }$}\setmainfont[Path=/usr/share/fonts/truetype/cmu/,UprightFont=cmunrm.ttf,BoldFont=cmunbx.ttf,ItalicFont=cmunti.ttf,BoldItalicFont=cmunbi.ttf]{cmunrm.ttf}\setmonofont[Path=/usr/share/fonts/truetype/cmu/,UprightFont=cmuntt.ttf,BoldFont=cmuntb.ttf,ItalicFont=cmunit.ttf,BoldItalicFont=cmuntx.ttf]{cmunrm.ttf} Run LaTeX on the {\ttfamily \setmainfont[Path=/usr/share/fonts/truetype/cmu/,UprightFont=cmunrm.ttf,BoldFont=cmunbx.ttf,ItalicFont=cmunti.ttf,BoldItalicFont=cmunbi.ttf]{cmuntt.ttf}\setmonofont[Path=/usr/share/fonts/truetype/cmu/,UprightFont=cmuntt.ttf,BoldFont=cmuntb.ttf,ItalicFont=cmunit.ttf,BoldItalicFont=cmuntx.ttf]{cmuntt.ttf}\ttfamily .ins}{$\text{ }$}\setmainfont[Path=/usr/share/fonts/truetype/cmu/,UprightFont=cmunrm.ttf,BoldFont=cmunbx.ttf,ItalicFont=cmunti.ttf,BoldItalicFont=cmunbi.ttf]{cmunrm.ttf}\setmonofont[Path=/usr/share/fonts/truetype/cmu/,UprightFont=cmuntt.ttf,BoldFont=cmuntb.ttf,ItalicFont=cmunit.ttf,BoldItalicFont=cmuntx.ttf]{cmunrm.ttf} file. That is, open the file in your editor and process it as if it were a LaTeX document (which it is), or if you prefer, type latex followed by the {\ttfamily \setmainfont[Path=/usr/share/fonts/truetype/cmu/,UprightFont=cmunrm.ttf,BoldFont=cmunbx.ttf,ItalicFont=cmunti.ttf,BoldItalicFont=cmunbi.ttf]{cmuntt.ttf}\setmonofont[Path=/usr/share/fonts/truetype/cmu/,UprightFont=cmuntt.ttf,BoldFont=cmuntb.ttf,ItalicFont=cmunit.ttf,BoldItalicFont=cmuntx.ttf]{cmuntt.ttf}\ttfamily .ins}{$\text{ }$}\setmainfont[Path=/usr/share/fonts/truetype/cmu/,UprightFont=cmunrm.ttf,BoldFont=cmunbx.ttf,ItalicFont=cmunti.ttf,BoldItalicFont=cmunbi.ttf]{cmunrm.ttf}\setmonofont[Path=/usr/share/fonts/truetype/cmu/,UprightFont=cmuntt.ttf,BoldFont=cmuntb.ttf,ItalicFont=cmunit.ttf,BoldItalicFont=cmuntx.ttf]{cmunrm.ttf} filename in a command window in your temporary directory. This will extract all the files needed from the {\ttfamily \setmainfont[Path=/usr/share/fonts/truetype/cmu/,UprightFont=cmunrm.ttf,BoldFont=cmunbx.ttf,ItalicFont=cmunti.ttf,BoldItalicFont=cmunbi.ttf]{cmuntt.ttf}\setmonofont[Path=/usr/share/fonts/truetype/cmu/,UprightFont=cmuntt.ttf,BoldFont=cmuntb.ttf,ItalicFont=cmunit.ttf,BoldItalicFont=cmuntx.ttf]{cmuntt.ttf}\ttfamily .dtx}{$\text{ }$}\setmainfont[Path=/usr/share/fonts/truetype/cmu/,UprightFont=cmunrm.ttf,BoldFont=cmunbx.ttf,ItalicFont=cmunti.ttf,BoldItalicFont=cmunbi.ttf]{cmunrm.ttf}\setmonofont[Path=/usr/share/fonts/truetype/cmu/,UprightFont=cmuntt.ttf,BoldFont=cmuntb.ttf,ItalicFont=cmunit.ttf,BoldItalicFont=cmuntx.ttf]{cmunrm.ttf} file (which is why you must have both of them present in the temporary directory). Note down or print the names of the files created if there are a lot of them (read the log file if you want to see their names again).

2. {\bfseries \setmainfont[Path=/usr/share/fonts/truetype/cmu/,UprightFont=cmunrm.ttf,BoldFont=cmunbx.ttf,ItalicFont=cmunti.ttf,BoldItalicFont=cmunbi.ttf]{cmunbx.ttf}\setmonofont[Path=/usr/share/fonts/truetype/cmu/,UprightFont=cmuntt.ttf,BoldFont=cmuntb.ttf,ItalicFont=cmunit.ttf,BoldItalicFont=cmuntx.ttf]{cmunbx.ttf}\bfseries Create the documentation}{$\text{ }$}\setmainfont[Path=/usr/share/fonts/truetype/cmu/,UprightFont=cmunrm.ttf,BoldFont=cmunbx.ttf,ItalicFont=cmunti.ttf,BoldItalicFont=cmunbi.ttf]{cmunrm.ttf}\setmonofont[Path=/usr/share/fonts/truetype/cmu/,UprightFont=cmuntt.ttf,BoldFont=cmuntb.ttf,ItalicFont=cmunit.ttf,BoldItalicFont=cmuntx.ttf]{cmunrm.ttf} Run LaTeX on the {\ttfamily \setmainfont[Path=/usr/share/fonts/truetype/cmu/,UprightFont=cmunrm.ttf,BoldFont=cmunbx.ttf,ItalicFont=cmunti.ttf,BoldItalicFont=cmunbi.ttf]{cmuntt.ttf}\setmonofont[Path=/usr/share/fonts/truetype/cmu/,UprightFont=cmuntt.ttf,BoldFont=cmuntb.ttf,ItalicFont=cmunit.ttf,BoldItalicFont=cmuntx.ttf]{cmuntt.ttf}\ttfamily .dtx}{$\text{ }$}\setmainfont[Path=/usr/share/fonts/truetype/cmu/,UprightFont=cmunrm.ttf,BoldFont=cmunbx.ttf,ItalicFont=cmunti.ttf,BoldItalicFont=cmunbi.ttf]{cmunrm.ttf}\setmonofont[Path=/usr/share/fonts/truetype/cmu/,UprightFont=cmuntt.ttf,BoldFont=cmuntb.ttf,ItalicFont=cmunit.ttf,BoldItalicFont=cmuntx.ttf]{cmunrm.ttf} file. You might need to run it twice or more, to get the cross-{}references right (just like any other LaTeX document). This will create a {\ttfamily \setmainfont[Path=/usr/share/fonts/truetype/cmu/,UprightFont=cmunrm.ttf,BoldFont=cmunbx.ttf,ItalicFont=cmunti.ttf,BoldItalicFont=cmunbi.ttf]{cmuntt.ttf}\setmonofont[Path=/usr/share/fonts/truetype/cmu/,UprightFont=cmuntt.ttf,BoldFont=cmuntb.ttf,ItalicFont=cmunit.ttf,BoldItalicFont=cmuntx.ttf]{cmuntt.ttf}\ttfamily .dvi}{$\text{ }$}\setmainfont[Path=/usr/share/fonts/truetype/cmu/,UprightFont=cmunrm.ttf,BoldFont=cmunbx.ttf,ItalicFont=cmunti.ttf,BoldItalicFont=cmunbi.ttf]{cmunrm.ttf}\setmonofont[Path=/usr/share/fonts/truetype/cmu/,UprightFont=cmuntt.ttf,BoldFont=cmuntb.ttf,ItalicFont=cmunit.ttf,BoldItalicFont=cmuntx.ttf]{cmunrm.ttf} file of documentation explaining what the package is for and how to use it. If you prefer to create PDF then run pdfLaTeX instead. If you created a {\ttfamily \setmainfont[Path=/usr/share/fonts/truetype/cmu/,UprightFont=cmunrm.ttf,BoldFont=cmunbx.ttf,ItalicFont=cmunti.ttf,BoldItalicFont=cmunbi.ttf]{cmuntt.ttf}\setmonofont[Path=/usr/share/fonts/truetype/cmu/,UprightFont=cmuntt.ttf,BoldFont=cmuntb.ttf,ItalicFont=cmunit.ttf,BoldItalicFont=cmuntx.ttf]{cmuntt.ttf}\ttfamily .idx}{$\text{ }$}\setmainfont[Path=/usr/share/fonts/truetype/cmu/,UprightFont=cmunrm.ttf,BoldFont=cmunbx.ttf,ItalicFont=cmunti.ttf,BoldItalicFont=cmunbi.ttf]{cmunrm.ttf}\setmonofont[Path=/usr/share/fonts/truetype/cmu/,UprightFont=cmuntt.ttf,BoldFont=cmuntb.ttf,ItalicFont=cmunit.ttf,BoldItalicFont=cmuntx.ttf]{cmunrm.ttf} as well, it means that the document contains an index, too. If you want the index to be created properly, follow the steps in the \mylref{626}{indexing} section. Sometimes you will see that a {\ttfamily \setmainfont[Path=/usr/share/fonts/truetype/cmu/,UprightFont=cmunrm.ttf,BoldFont=cmunbx.ttf,ItalicFont=cmunti.ttf,BoldItalicFont=cmunbi.ttf]{cmuntt.ttf}\setmonofont[Path=/usr/share/fonts/truetype/cmu/,UprightFont=cmuntt.ttf,BoldFont=cmuntb.ttf,ItalicFont=cmunit.ttf,BoldItalicFont=cmuntx.ttf]{cmuntt.ttf}\ttfamily .glo}{$\text{ }$}\setmainfont[Path=/usr/share/fonts/truetype/cmu/,UprightFont=cmunrm.ttf,BoldFont=cmunbx.ttf,ItalicFont=cmunti.ttf,BoldItalicFont=cmunbi.ttf]{cmunrm.ttf}\setmonofont[Path=/usr/share/fonts/truetype/cmu/,UprightFont=cmuntt.ttf,BoldFont=cmuntb.ttf,ItalicFont=cmunit.ttf,BoldItalicFont=cmuntx.ttf]{cmunrm.ttf} (glossary) file has been produced. Run the following command instead:
\\

\TemplateSpaceIndent{$\text{ }${}makeindex$\text{ }${}-{}s$\text{ }${}gglo.ist$\text{ }${}-{}o$\text{ }${}name.gls$\text{ }${}name.glo}


3. {\bfseries \setmainfont[Path=/usr/share/fonts/truetype/cmu/,UprightFont=cmunrm.ttf,BoldFont=cmunbx.ttf,ItalicFont=cmunti.ttf,BoldItalicFont=cmunbi.ttf]{cmunbx.ttf}\setmonofont[Path=/usr/share/fonts/truetype/cmu/,UprightFont=cmuntt.ttf,BoldFont=cmuntb.ttf,ItalicFont=cmunit.ttf,BoldItalicFont=cmuntx.ttf]{cmunbx.ttf}\bfseries Install the files}{$\text{ }$}\setmainfont[Path=/usr/share/fonts/truetype/cmu/,UprightFont=cmunrm.ttf,BoldFont=cmunbx.ttf,ItalicFont=cmunti.ttf,BoldItalicFont=cmunbi.ttf]{cmunrm.ttf}\setmonofont[Path=/usr/share/fonts/truetype/cmu/,UprightFont=cmuntt.ttf,BoldFont=cmuntb.ttf,ItalicFont=cmunit.ttf,BoldItalicFont=cmuntx.ttf]{cmunrm.ttf} While the documentation is printing, move or copy the package files from your temporary directory to the right place{$\text{[}$}s{$\text{]}$} in your TeX local installation directory tree. Packages installed by hand should always be placed in your \symbol{34}local\symbol{34} directory tree, {\itshape \setmainfont[Path=/usr/share/fonts/truetype/cmu/,UprightFont=cmunrm.ttf,BoldFont=cmunbx.ttf,ItalicFont=cmunti.ttf,BoldItalicFont=cmunbi.ttf]{cmunti.ttf}\setmonofont[Path=/usr/share/fonts/truetype/cmu/,UprightFont=cmuntt.ttf,BoldFont=cmuntb.ttf,ItalicFont=cmunit.ttf,BoldItalicFont=cmuntx.ttf]{cmunti.ttf}\itshape not}{$\text{ }$}\setmainfont[Path=/usr/share/fonts/truetype/cmu/,UprightFont=cmunrm.ttf,BoldFont=cmunbx.ttf,ItalicFont=cmunti.ttf,BoldItalicFont=cmunbi.ttf]{cmunrm.ttf}\setmonofont[Path=/usr/share/fonts/truetype/cmu/,UprightFont=cmuntt.ttf,BoldFont=cmuntb.ttf,ItalicFont=cmunit.ttf,BoldItalicFont=cmuntx.ttf]{cmunrm.ttf} in the directory tree containing all the pre-{}installed packages. This is done to {\itshape \setmainfont[Path=/usr/share/fonts/truetype/cmu/,UprightFont=cmunrm.ttf,BoldFont=cmunbx.ttf,ItalicFont=cmunti.ttf,BoldItalicFont=cmunbi.ttf]{cmunti.ttf}\setmonofont[Path=/usr/share/fonts/truetype/cmu/,UprightFont=cmuntt.ttf,BoldFont=cmuntb.ttf,ItalicFont=cmunit.ttf,BoldItalicFont=cmuntx.ttf]{cmunti.ttf}\itshape a)}{$\text{ }$}\setmainfont[Path=/usr/share/fonts/truetype/cmu/,UprightFont=cmunrm.ttf,BoldFont=cmunbx.ttf,ItalicFont=cmunti.ttf,BoldItalicFont=cmunbi.ttf]{cmunrm.ttf}\setmonofont[Path=/usr/share/fonts/truetype/cmu/,UprightFont=cmuntt.ttf,BoldFont=cmuntb.ttf,ItalicFont=cmunit.ttf,BoldItalicFont=cmuntx.ttf]{cmunrm.ttf} prevent your new package accidentally overwriting files in the main TeX directories; and {\itshape \setmainfont[Path=/usr/share/fonts/truetype/cmu/,UprightFont=cmunrm.ttf,BoldFont=cmunbx.ttf,ItalicFont=cmunti.ttf,BoldItalicFont=cmunbi.ttf]{cmunti.ttf}\setmonofont[Path=/usr/share/fonts/truetype/cmu/,UprightFont=cmuntt.ttf,BoldFont=cmuntb.ttf,ItalicFont=cmunit.ttf,BoldItalicFont=cmuntx.ttf]{cmunti.ttf}\itshape b)}{$\text{ }$}\setmainfont[Path=/usr/share/fonts/truetype/cmu/,UprightFont=cmunrm.ttf,BoldFont=cmunbx.ttf,ItalicFont=cmunti.ttf,BoldItalicFont=cmunbi.ttf]{cmunrm.ttf}\setmonofont[Path=/usr/share/fonts/truetype/cmu/,UprightFont=cmuntt.ttf,BoldFont=cmuntb.ttf,ItalicFont=cmunit.ttf,BoldItalicFont=cmuntx.ttf]{cmunrm.ttf} avoid your newly-{}installed files being overwritten when you next update your version of TeX. 

For a TDS(TeX Directory Structure)-{}conformant system, your \symbol{34}local installation directory tree\symbol{34} is a folder and its subfolders. The outermost folder should probably be called {\ttfamily \setmainfont[Path=/usr/share/fonts/truetype/cmu/,UprightFont=cmunrm.ttf,BoldFont=cmunbx.ttf,ItalicFont=cmunti.ttf,BoldItalicFont=cmunbi.ttf]{cmuntt.ttf}\setmonofont[Path=/usr/share/fonts/truetype/cmu/,UprightFont=cmuntt.ttf,BoldFont=cmuntb.ttf,ItalicFont=cmunit.ttf,BoldItalicFont=cmuntx.ttf]{cmuntt.ttf}\ttfamily texmf-{}local/}{$\text{ }$}\setmainfont[Path=/usr/share/fonts/truetype/cmu/,UprightFont=cmunrm.ttf,BoldFont=cmunbx.ttf,ItalicFont=cmunti.ttf,BoldItalicFont=cmunbi.ttf]{cmunrm.ttf}\setmonofont[Path=/usr/share/fonts/truetype/cmu/,UprightFont=cmuntt.ttf,BoldFont=cmuntb.ttf,ItalicFont=cmunit.ttf,BoldItalicFont=cmuntx.ttf]{cmunrm.ttf} or {\ttfamily \setmainfont[Path=/usr/share/fonts/truetype/cmu/,UprightFont=cmunrm.ttf,BoldFont=cmunbx.ttf,ItalicFont=cmunti.ttf,BoldItalicFont=cmunbi.ttf]{cmuntt.ttf}\setmonofont[Path=/usr/share/fonts/truetype/cmu/,UprightFont=cmuntt.ttf,BoldFont=cmuntb.ttf,ItalicFont=cmunit.ttf,BoldItalicFont=cmuntx.ttf]{cmuntt.ttf}\ttfamily texmf/}\setmainfont[Path=/usr/share/fonts/truetype/cmu/,UprightFont=cmunrm.ttf,BoldFont=cmunbx.ttf,ItalicFont=cmunti.ttf,BoldItalicFont=cmunbi.ttf]{cmunrm.ttf}\setmonofont[Path=/usr/share/fonts/truetype/cmu/,UprightFont=cmuntt.ttf,BoldFont=cmuntb.ttf,ItalicFont=cmunit.ttf,BoldItalicFont=cmuntx.ttf]{cmunrm.ttf}. Its location depends on your system:

\begin{myquote}
\item{} 
\begin{myitemize}
\item{}  MacTeX: {\ttfamily \setmainfont[Path=/usr/share/fonts/truetype/cmu/,UprightFont=cmunrm.ttf,BoldFont=cmunbx.ttf,ItalicFont=cmunti.ttf,BoldItalicFont=cmunbi.ttf]{cmuntt.ttf}\setmonofont[Path=/usr/share/fonts/truetype/cmu/,UprightFont=cmuntt.ttf,BoldFont=cmuntb.ttf,ItalicFont=cmunit.ttf,BoldItalicFont=cmuntx.ttf]{cmuntt.ttf}\ttfamily Users/{\itshape \setmainfont[Path=/usr/share/fonts/truetype/cmu/,UprightFont=cmunrm.ttf,BoldFont=cmunbx.ttf,ItalicFont=cmunti.ttf,BoldItalicFont=cmunbi.ttf]{cmunit.ttf}\setmonofont[Path=/usr/share/fonts/truetype/cmu/,UprightFont=cmuntt.ttf,BoldFont=cmuntb.ttf,ItalicFont=cmunit.ttf,BoldItalicFont=cmuntx.ttf]{cmunit.ttf}\ttfamily \itshape username}\setmainfont[Path=/usr/share/fonts/truetype/cmu/,UprightFont=cmunrm.ttf,BoldFont=cmunbx.ttf,ItalicFont=cmunti.ttf,BoldItalicFont=cmunbi.ttf]{cmuntt.ttf}\setmonofont[Path=/usr/share/fonts/truetype/cmu/,UprightFont=cmuntt.ttf,BoldFont=cmuntb.ttf,ItalicFont=cmunit.ttf,BoldItalicFont=cmuntx.ttf]{cmuntt.ttf}\ttfamily /Library/texmf/}\setmainfont[Path=/usr/share/fonts/truetype/cmu/,UprightFont=cmunrm.ttf,BoldFont=cmunbx.ttf,ItalicFont=cmunti.ttf,BoldItalicFont=cmunbi.ttf]{cmunrm.ttf}\setmonofont[Path=/usr/share/fonts/truetype/cmu/,UprightFont=cmuntt.ttf,BoldFont=cmuntb.ttf,ItalicFont=cmunit.ttf,BoldItalicFont=cmuntx.ttf]{cmunrm.ttf}.
\item{}  Unix-{}type systems: Usually {\ttfamily \setmainfont[Path=/usr/share/fonts/truetype/cmu/,UprightFont=cmunrm.ttf,BoldFont=cmunbx.ttf,ItalicFont=cmunti.ttf,BoldItalicFont=cmunbi.ttf]{cmuntt.ttf}\setmonofont[Path=/usr/share/fonts/truetype/cmu/,UprightFont=cmuntt.ttf,BoldFont=cmuntb.ttf,ItalicFont=cmunit.ttf,BoldItalicFont=cmuntx.ttf]{cmuntt.ttf}\ttfamily \~{}/texmf/}\setmainfont[Path=/usr/share/fonts/truetype/cmu/,UprightFont=cmunrm.ttf,BoldFont=cmunbx.ttf,ItalicFont=cmunti.ttf,BoldItalicFont=cmunbi.ttf]{cmunrm.ttf}\setmonofont[Path=/usr/share/fonts/truetype/cmu/,UprightFont=cmuntt.ttf,BoldFont=cmuntb.ttf,ItalicFont=cmunit.ttf,BoldItalicFont=cmuntx.ttf]{cmunrm.ttf}.
\item{}  MikTeX: Your local directory tree can be any folder you like, as long as you then register it as a user-{}managed texmf directory (see \myplainurl{http://docs.miktex.org/manual/localadditions.html\#id573803)}
\end{myitemize}

\end{myquote}


The \symbol{34}right place\symbol{34} sometimes causes confusion, especially if your TeX installation is old or does not conform to the TeX Directory Structure(TDS). For a TDS-{}conformant system, the \symbol{34}right place\symbol{34} for a LaTeX {\ttfamily \setmainfont[Path=/usr/share/fonts/truetype/cmu/,UprightFont=cmunrm.ttf,BoldFont=cmunbx.ttf,ItalicFont=cmunti.ttf,BoldItalicFont=cmunbi.ttf]{cmuntt.ttf}\setmonofont[Path=/usr/share/fonts/truetype/cmu/,UprightFont=cmuntt.ttf,BoldFont=cmuntb.ttf,ItalicFont=cmunit.ttf,BoldItalicFont=cmuntx.ttf]{cmuntt.ttf}\ttfamily .sty}{$\text{ }$}\setmainfont[Path=/usr/share/fonts/truetype/cmu/,UprightFont=cmunrm.ttf,BoldFont=cmunbx.ttf,ItalicFont=cmunti.ttf,BoldItalicFont=cmunbi.ttf]{cmunrm.ttf}\setmonofont[Path=/usr/share/fonts/truetype/cmu/,UprightFont=cmuntt.ttf,BoldFont=cmuntb.ttf,ItalicFont=cmunit.ttf,BoldItalicFont=cmuntx.ttf]{cmunrm.ttf} file is a suitably-{}named subdirectory of {\ttfamily \setmainfont[Path=/usr/share/fonts/truetype/cmu/,UprightFont=cmunrm.ttf,BoldFont=cmunbx.ttf,ItalicFont=cmunti.ttf,BoldItalicFont=cmunbi.ttf]{cmuntt.ttf}\setmonofont[Path=/usr/share/fonts/truetype/cmu/,UprightFont=cmuntt.ttf,BoldFont=cmuntb.ttf,ItalicFont=cmunit.ttf,BoldItalicFont=cmuntx.ttf]{cmuntt.ttf}\ttfamily texmf/tex/latex/}\setmainfont[Path=/usr/share/fonts/truetype/cmu/,UprightFont=cmunrm.ttf,BoldFont=cmunbx.ttf,ItalicFont=cmunti.ttf,BoldItalicFont=cmunbi.ttf]{cmunrm.ttf}\setmonofont[Path=/usr/share/fonts/truetype/cmu/,UprightFont=cmuntt.ttf,BoldFont=cmuntb.ttf,ItalicFont=cmunit.ttf,BoldItalicFont=cmuntx.ttf]{cmunrm.ttf}. \symbol{34}Suitably-{}named\symbol{34} means sensible and meaningful (and probably short). For a package like paralist, for example, I\textquotesingle{}d call the directory {\ttfamily \setmainfont[Path=/usr/share/fonts/truetype/cmu/,UprightFont=cmunrm.ttf,BoldFont=cmunbx.ttf,ItalicFont=cmunti.ttf,BoldItalicFont=cmunbi.ttf]{cmuntt.ttf}\setmonofont[Path=/usr/share/fonts/truetype/cmu/,UprightFont=cmuntt.ttf,BoldFont=cmuntb.ttf,ItalicFont=cmunit.ttf,BoldItalicFont=cmuntx.ttf]{cmuntt.ttf}\ttfamily texmf/tex/latex/paralist}\setmainfont[Path=/usr/share/fonts/truetype/cmu/,UprightFont=cmunrm.ttf,BoldFont=cmunbx.ttf,ItalicFont=cmunti.ttf,BoldItalicFont=cmunbi.ttf]{cmunrm.ttf}\setmonofont[Path=/usr/share/fonts/truetype/cmu/,UprightFont=cmuntt.ttf,BoldFont=cmuntb.ttf,ItalicFont=cmunit.ttf,BoldItalicFont=cmuntx.ttf]{cmunrm.ttf}. 

Often there is just a {\ttfamily \setmainfont[Path=/usr/share/fonts/truetype/cmu/,UprightFont=cmunrm.ttf,BoldFont=cmunbx.ttf,ItalicFont=cmunti.ttf,BoldItalicFont=cmunbi.ttf]{cmuntt.ttf}\setmonofont[Path=/usr/share/fonts/truetype/cmu/,UprightFont=cmuntt.ttf,BoldFont=cmuntb.ttf,ItalicFont=cmunit.ttf,BoldItalicFont=cmuntx.ttf]{cmuntt.ttf}\ttfamily .sty}{$\text{ }$}\setmainfont[Path=/usr/share/fonts/truetype/cmu/,UprightFont=cmunrm.ttf,BoldFont=cmunbx.ttf,ItalicFont=cmunti.ttf,BoldItalicFont=cmunbi.ttf]{cmunrm.ttf}\setmonofont[Path=/usr/share/fonts/truetype/cmu/,UprightFont=cmuntt.ttf,BoldFont=cmuntb.ttf,ItalicFont=cmunit.ttf,BoldItalicFont=cmuntx.ttf]{cmunrm.ttf} file to move, but in the case of complex packages there may be more, and they may belong in different locations. For example, new BibTeX packages or font packages will typically have several files to install. This is why it is a good idea to create a sub-{}directory for the package rather than dump the files into misc along with other unrelated stuff. If there are configuration or other files, read the documentation to find out if there is a special or preferred location to move them to.

\begin{longtable}{|>{\RaggedRight}p{0.09132\linewidth}|>{\RaggedRight}p{0.47659\linewidth}|>{\RaggedRight}p{0.34638\linewidth}|} \hline 
\multicolumn{3}{|>{\RaggedRight}p{0.97143\linewidth}|}{{\bfseries \hspace*{0pt}\ignorespaces{}\hspace*{0pt}Where to put files from packages}}\\ \hline {\bfseries \hspace*{0pt}\ignorespaces{}\hspace*{0pt}Type}&{\bfseries \hspace*{0pt}\ignorespaces{}\hspace*{0pt}Directory (under {\ttfamily \setmainfont[Path=/usr/share/fonts/truetype/cmu/,UprightFont=cmunrm.ttf,BoldFont=cmunbx.ttf,ItalicFont=cmunti.ttf,BoldItalicFont=cmunbi.ttf]{cmuntt.ttf}\setmonofont[Path=/usr/share/fonts/truetype/cmu/,UprightFont=cmuntt.ttf,BoldFont=cmuntb.ttf,ItalicFont=cmunit.ttf,BoldItalicFont=cmuntx.ttf]{cmuntt.ttf}\ttfamily texmf/}{$\text{ }$}\setmainfont[Path=/usr/share/fonts/truetype/cmu/,UprightFont=cmunrm.ttf,BoldFont=cmunbx.ttf,ItalicFont=cmunti.ttf,BoldItalicFont=cmunbi.ttf]{cmunrm.ttf}\setmonofont[Path=/usr/share/fonts/truetype/cmu/,UprightFont=cmuntt.ttf,BoldFont=cmuntb.ttf,ItalicFont=cmunit.ttf,BoldItalicFont=cmuntx.ttf]{cmunrm.ttf} or {\ttfamily \setmainfont[Path=/usr/share/fonts/truetype/cmu/,UprightFont=cmunrm.ttf,BoldFont=cmunbx.ttf,ItalicFont=cmunti.ttf,BoldItalicFont=cmunbi.ttf]{cmuntt.ttf}\setmonofont[Path=/usr/share/fonts/truetype/cmu/,UprightFont=cmuntt.ttf,BoldFont=cmuntb.ttf,ItalicFont=cmunit.ttf,BoldItalicFont=cmuntx.ttf]{cmuntt.ttf}\ttfamily texmf-{}local/}\setmainfont[Path=/usr/share/fonts/truetype/cmu/,UprightFont=cmunrm.ttf,BoldFont=cmunbx.ttf,ItalicFont=cmunti.ttf,BoldItalicFont=cmunbi.ttf]{cmunrm.ttf}\setmonofont[Path=/usr/share/fonts/truetype/cmu/,UprightFont=cmuntt.ttf,BoldFont=cmuntb.ttf,ItalicFont=cmunit.ttf,BoldItalicFont=cmuntx.ttf]{cmunrm.ttf})}&{\bfseries \hspace*{0pt}\ignorespaces{}\hspace*{0pt}Description}\endhead  \hline \hspace*{0pt}\ignorespaces{}\hspace*{0pt}{\ttfamily \setmainfont[Path=/usr/share/fonts/truetype/cmu/,UprightFont=cmunrm.ttf,BoldFont=cmunbx.ttf,ItalicFont=cmunti.ttf,BoldItalicFont=cmunbi.ttf]{cmuntt.ttf}\setmonofont[Path=/usr/share/fonts/truetype/cmu/,UprightFont=cmuntt.ttf,BoldFont=cmuntb.ttf,ItalicFont=cmunit.ttf,BoldItalicFont=cmuntx.ttf]{cmuntt.ttf}\ttfamily .afm}&\hspace*{0pt}\ignorespaces{}\hspace*{0pt}{\ttfamily fonts/afm/{\itshape \setmainfont[Path=/usr/share/fonts/truetype/cmu/,UprightFont=cmunrm.ttf,BoldFont=cmunbx.ttf,ItalicFont=cmunti.ttf,BoldItalicFont=cmunbi.ttf]{cmunit.ttf}\setmonofont[Path=/usr/share/fonts/truetype/cmu/,UprightFont=cmuntt.ttf,BoldFont=cmuntb.ttf,ItalicFont=cmunit.ttf,BoldItalicFont=cmuntx.ttf]{cmunit.ttf}\ttfamily \itshape foundry}\setmainfont[Path=/usr/share/fonts/truetype/cmu/,UprightFont=cmunrm.ttf,BoldFont=cmunbx.ttf,ItalicFont=cmunti.ttf,BoldItalicFont=cmunbi.ttf]{cmuntt.ttf}\setmonofont[Path=/usr/share/fonts/truetype/cmu/,UprightFont=cmuntt.ttf,BoldFont=cmuntb.ttf,ItalicFont=cmunit.ttf,BoldItalicFont=cmuntx.ttf]{cmuntt.ttf}\ttfamily /{\itshape \setmainfont[Path=/usr/share/fonts/truetype/cmu/,UprightFont=cmunrm.ttf,BoldFont=cmunbx.ttf,ItalicFont=cmunti.ttf,BoldItalicFont=cmunbi.ttf]{cmunit.ttf}\setmonofont[Path=/usr/share/fonts/truetype/cmu/,UprightFont=cmuntt.ttf,BoldFont=cmuntb.ttf,ItalicFont=cmunit.ttf,BoldItalicFont=cmuntx.ttf]{cmunit.ttf}\ttfamily \itshape typeface}}&\hspace*{0pt}\ignorespaces{}\hspace*{0pt}\setmainfont[Path=/usr/share/fonts/truetype/cmu/,UprightFont=cmunrm.ttf,BoldFont=cmunbx.ttf,ItalicFont=cmunti.ttf,BoldItalicFont=cmunbi.ttf]{cmunrm.ttf}\setmonofont[Path=/usr/share/fonts/truetype/cmu/,UprightFont=cmuntt.ttf,BoldFont=cmuntb.ttf,ItalicFont=cmunit.ttf,BoldItalicFont=cmuntx.ttf]{cmunrm.ttf}Adobe Font Metrics for Type 1 fonts\\ \hline \hspace*{0pt}\ignorespaces{}\hspace*{0pt}{\ttfamily \setmainfont[Path=/usr/share/fonts/truetype/cmu/,UprightFont=cmunrm.ttf,BoldFont=cmunbx.ttf,ItalicFont=cmunti.ttf,BoldItalicFont=cmunbi.ttf]{cmuntt.ttf}\setmonofont[Path=/usr/share/fonts/truetype/cmu/,UprightFont=cmuntt.ttf,BoldFont=cmuntb.ttf,ItalicFont=cmunit.ttf,BoldItalicFont=cmuntx.ttf]{cmuntt.ttf}\ttfamily .bib}&\hspace*{0pt}\ignorespaces{}\hspace*{0pt}{\ttfamily bibtex/bib/{\itshape \setmainfont[Path=/usr/share/fonts/truetype/cmu/,UprightFont=cmunrm.ttf,BoldFont=cmunbx.ttf,ItalicFont=cmunti.ttf,BoldItalicFont=cmunbi.ttf]{cmunit.ttf}\setmonofont[Path=/usr/share/fonts/truetype/cmu/,UprightFont=cmuntt.ttf,BoldFont=cmuntb.ttf,ItalicFont=cmunit.ttf,BoldItalicFont=cmuntx.ttf]{cmunit.ttf}\ttfamily \itshape bibliography}}&\hspace*{0pt}\ignorespaces{}\hspace*{0pt}\setmainfont[Path=/usr/share/fonts/truetype/cmu/,UprightFont=cmunrm.ttf,BoldFont=cmunbx.ttf,ItalicFont=cmunti.ttf,BoldItalicFont=cmunbi.ttf]{cmunrm.ttf}\setmonofont[Path=/usr/share/fonts/truetype/cmu/,UprightFont=cmuntt.ttf,BoldFont=cmuntb.ttf,ItalicFont=cmunit.ttf,BoldItalicFont=cmuntx.ttf]{cmunrm.ttf}BibTeX bibliography\\ \hline \hspace*{0pt}\ignorespaces{}\hspace*{0pt}{\ttfamily \setmainfont[Path=/usr/share/fonts/truetype/cmu/,UprightFont=cmunrm.ttf,BoldFont=cmunbx.ttf,ItalicFont=cmunti.ttf,BoldItalicFont=cmunbi.ttf]{cmuntt.ttf}\setmonofont[Path=/usr/share/fonts/truetype/cmu/,UprightFont=cmuntt.ttf,BoldFont=cmuntb.ttf,ItalicFont=cmunit.ttf,BoldItalicFont=cmuntx.ttf]{cmuntt.ttf}\ttfamily .bst}&\hspace*{0pt}\ignorespaces{}\hspace*{0pt}{\ttfamily bibtex/bst/{\itshape \setmainfont[Path=/usr/share/fonts/truetype/cmu/,UprightFont=cmunrm.ttf,BoldFont=cmunbx.ttf,ItalicFont=cmunti.ttf,BoldItalicFont=cmunbi.ttf]{cmunit.ttf}\setmonofont[Path=/usr/share/fonts/truetype/cmu/,UprightFont=cmuntt.ttf,BoldFont=cmuntb.ttf,ItalicFont=cmunit.ttf,BoldItalicFont=cmuntx.ttf]{cmunit.ttf}\ttfamily \itshape packagename}}&\hspace*{0pt}\ignorespaces{}\hspace*{0pt}\setmainfont[Path=/usr/share/fonts/truetype/cmu/,UprightFont=cmunrm.ttf,BoldFont=cmunbx.ttf,ItalicFont=cmunti.ttf,BoldItalicFont=cmunbi.ttf]{cmunrm.ttf}\setmonofont[Path=/usr/share/fonts/truetype/cmu/,UprightFont=cmuntt.ttf,BoldFont=cmuntb.ttf,ItalicFont=cmunit.ttf,BoldItalicFont=cmuntx.ttf]{cmunrm.ttf}BibTeX style\\ \hline \hspace*{0pt}\ignorespaces{}\hspace*{0pt}{\ttfamily \setmainfont[Path=/usr/share/fonts/truetype/cmu/,UprightFont=cmunrm.ttf,BoldFont=cmunbx.ttf,ItalicFont=cmunti.ttf,BoldItalicFont=cmunbi.ttf]{cmuntt.ttf}\setmonofont[Path=/usr/share/fonts/truetype/cmu/,UprightFont=cmuntt.ttf,BoldFont=cmuntb.ttf,ItalicFont=cmunit.ttf,BoldItalicFont=cmuntx.ttf]{cmuntt.ttf}\ttfamily .cls}&\hspace*{0pt}\ignorespaces{}\hspace*{0pt}{\ttfamily tex/latex/base}&\hspace*{0pt}\ignorespaces{}\hspace*{0pt}\setmainfont[Path=/usr/share/fonts/truetype/cmu/,UprightFont=cmunrm.ttf,BoldFont=cmunbx.ttf,ItalicFont=cmunti.ttf,BoldItalicFont=cmunbi.ttf]{cmunrm.ttf}\setmonofont[Path=/usr/share/fonts/truetype/cmu/,UprightFont=cmuntt.ttf,BoldFont=cmuntb.ttf,ItalicFont=cmunit.ttf,BoldItalicFont=cmuntx.ttf]{cmunrm.ttf}Document class file\\ \hline \hspace*{0pt}\ignorespaces{}\hspace*{0pt}{\ttfamily \setmainfont[Path=/usr/share/fonts/truetype/cmu/,UprightFont=cmunrm.ttf,BoldFont=cmunbx.ttf,ItalicFont=cmunti.ttf,BoldItalicFont=cmunbi.ttf]{cmuntt.ttf}\setmonofont[Path=/usr/share/fonts/truetype/cmu/,UprightFont=cmuntt.ttf,BoldFont=cmuntb.ttf,ItalicFont=cmunit.ttf,BoldItalicFont=cmuntx.ttf]{cmuntt.ttf}\ttfamily .dvi}&\hspace*{0pt}\ignorespaces{}\hspace*{0pt}{\ttfamily doc}&\hspace*{0pt}\ignorespaces{}\hspace*{0pt}\setmainfont[Path=/usr/share/fonts/truetype/cmu/,UprightFont=cmunrm.ttf,BoldFont=cmunbx.ttf,ItalicFont=cmunti.ttf,BoldItalicFont=cmunbi.ttf]{cmunrm.ttf}\setmonofont[Path=/usr/share/fonts/truetype/cmu/,UprightFont=cmuntt.ttf,BoldFont=cmuntb.ttf,ItalicFont=cmunit.ttf,BoldItalicFont=cmuntx.ttf]{cmunrm.ttf}package documentation\\ \hline \hspace*{0pt}\ignorespaces{}\hspace*{0pt}{\ttfamily \setmainfont[Path=/usr/share/fonts/truetype/cmu/,UprightFont=cmunrm.ttf,BoldFont=cmunbx.ttf,ItalicFont=cmunti.ttf,BoldItalicFont=cmunbi.ttf]{cmuntt.ttf}\setmonofont[Path=/usr/share/fonts/truetype/cmu/,UprightFont=cmuntt.ttf,BoldFont=cmuntb.ttf,ItalicFont=cmunit.ttf,BoldItalicFont=cmuntx.ttf]{cmuntt.ttf}\ttfamily .enc}&\hspace*{0pt}\ignorespaces{}\hspace*{0pt}{\ttfamily fonts/enc}&\hspace*{0pt}\ignorespaces{}\hspace*{0pt}\setmainfont[Path=/usr/share/fonts/truetype/cmu/,UprightFont=cmunrm.ttf,BoldFont=cmunbx.ttf,ItalicFont=cmunti.ttf,BoldItalicFont=cmunbi.ttf]{cmunrm.ttf}\setmonofont[Path=/usr/share/fonts/truetype/cmu/,UprightFont=cmuntt.ttf,BoldFont=cmuntb.ttf,ItalicFont=cmunit.ttf,BoldItalicFont=cmuntx.ttf]{cmunrm.ttf}Font encoding\\ \hline \hspace*{0pt}\ignorespaces{}\hspace*{0pt}{\ttfamily \setmainfont[Path=/usr/share/fonts/truetype/cmu/,UprightFont=cmunrm.ttf,BoldFont=cmunbx.ttf,ItalicFont=cmunti.ttf,BoldItalicFont=cmunbi.ttf]{cmuntt.ttf}\setmonofont[Path=/usr/share/fonts/truetype/cmu/,UprightFont=cmuntt.ttf,BoldFont=cmuntb.ttf,ItalicFont=cmunit.ttf,BoldItalicFont=cmuntx.ttf]{cmuntt.ttf}\ttfamily .fd}&\hspace*{0pt}\ignorespaces{}\hspace*{0pt}{\ttfamily tex/latex/mfnfss}&\hspace*{0pt}\ignorespaces{}\hspace*{0pt}\setmainfont[Path=/usr/share/fonts/truetype/cmu/,UprightFont=cmunrm.ttf,BoldFont=cmunbx.ttf,ItalicFont=cmunti.ttf,BoldItalicFont=cmunbi.ttf]{cmunrm.ttf}\setmonofont[Path=/usr/share/fonts/truetype/cmu/,UprightFont=cmuntt.ttf,BoldFont=cmuntb.ttf,ItalicFont=cmunit.ttf,BoldItalicFont=cmuntx.ttf]{cmunrm.ttf}Font Definition files for METAFONT fonts\\ \hline \hspace*{0pt}\ignorespaces{}\hspace*{0pt}{\ttfamily \setmainfont[Path=/usr/share/fonts/truetype/cmu/,UprightFont=cmunrm.ttf,BoldFont=cmunbx.ttf,ItalicFont=cmunti.ttf,BoldItalicFont=cmunbi.ttf]{cmuntt.ttf}\setmonofont[Path=/usr/share/fonts/truetype/cmu/,UprightFont=cmuntt.ttf,BoldFont=cmuntb.ttf,ItalicFont=cmunit.ttf,BoldItalicFont=cmuntx.ttf]{cmuntt.ttf}\ttfamily .fd}&\hspace*{0pt}\ignorespaces{}\hspace*{0pt}{\ttfamily tex/latex/psnfss}&\hspace*{0pt}\ignorespaces{}\hspace*{0pt}\setmainfont[Path=/usr/share/fonts/truetype/cmu/,UprightFont=cmunrm.ttf,BoldFont=cmunbx.ttf,ItalicFont=cmunti.ttf,BoldItalicFont=cmunbi.ttf]{cmunrm.ttf}\setmonofont[Path=/usr/share/fonts/truetype/cmu/,UprightFont=cmuntt.ttf,BoldFont=cmuntb.ttf,ItalicFont=cmunit.ttf,BoldItalicFont=cmuntx.ttf]{cmunrm.ttf}Font Definition files for PostScript Type 1 fonts\\ \hline \hspace*{0pt}\ignorespaces{}\hspace*{0pt}{\ttfamily \setmainfont[Path=/usr/share/fonts/truetype/cmu/,UprightFont=cmunrm.ttf,BoldFont=cmunbx.ttf,ItalicFont=cmunti.ttf,BoldItalicFont=cmunbi.ttf]{cmuntt.ttf}\setmonofont[Path=/usr/share/fonts/truetype/cmu/,UprightFont=cmuntt.ttf,BoldFont=cmuntb.ttf,ItalicFont=cmunit.ttf,BoldItalicFont=cmuntx.ttf]{cmuntt.ttf}\ttfamily .map}&\hspace*{0pt}\ignorespaces{}\hspace*{0pt}{\ttfamily fonts/map}&\hspace*{0pt}\ignorespaces{}\hspace*{0pt}\setmainfont[Path=/usr/share/fonts/truetype/cmu/,UprightFont=cmunrm.ttf,BoldFont=cmunbx.ttf,ItalicFont=cmunti.ttf,BoldItalicFont=cmunbi.ttf]{cmunrm.ttf}\setmonofont[Path=/usr/share/fonts/truetype/cmu/,UprightFont=cmuntt.ttf,BoldFont=cmuntb.ttf,ItalicFont=cmunit.ttf,BoldItalicFont=cmuntx.ttf]{cmunrm.ttf}Font mapping files\\ \hline \hspace*{0pt}\ignorespaces{}\hspace*{0pt}{\ttfamily \setmainfont[Path=/usr/share/fonts/truetype/cmu/,UprightFont=cmunrm.ttf,BoldFont=cmunbx.ttf,ItalicFont=cmunti.ttf,BoldItalicFont=cmunbi.ttf]{cmuntt.ttf}\setmonofont[Path=/usr/share/fonts/truetype/cmu/,UprightFont=cmuntt.ttf,BoldFont=cmuntb.ttf,ItalicFont=cmunit.ttf,BoldItalicFont=cmuntx.ttf]{cmuntt.ttf}\ttfamily .mf}&\hspace*{0pt}\ignorespaces{}\hspace*{0pt}{\ttfamily fonts/source/public/{\itshape \setmainfont[Path=/usr/share/fonts/truetype/cmu/,UprightFont=cmunrm.ttf,BoldFont=cmunbx.ttf,ItalicFont=cmunti.ttf,BoldItalicFont=cmunbi.ttf]{cmunit.ttf}\setmonofont[Path=/usr/share/fonts/truetype/cmu/,UprightFont=cmuntt.ttf,BoldFont=cmuntb.ttf,ItalicFont=cmunit.ttf,BoldItalicFont=cmuntx.ttf]{cmunit.ttf}\ttfamily \itshape typeface}}&\hspace*{0pt}\ignorespaces{}\hspace*{0pt}\setmainfont[Path=/usr/share/fonts/truetype/cmu/,UprightFont=cmunrm.ttf,BoldFont=cmunbx.ttf,ItalicFont=cmunti.ttf,BoldItalicFont=cmunbi.ttf]{cmunrm.ttf}\setmonofont[Path=/usr/share/fonts/truetype/cmu/,UprightFont=cmuntt.ttf,BoldFont=cmuntb.ttf,ItalicFont=cmunit.ttf,BoldItalicFont=cmuntx.ttf]{cmunrm.ttf}METAFONT outline\\ \hline \hspace*{0pt}\ignorespaces{}\hspace*{0pt}{\ttfamily \setmainfont[Path=/usr/share/fonts/truetype/cmu/,UprightFont=cmunrm.ttf,BoldFont=cmunbx.ttf,ItalicFont=cmunti.ttf,BoldItalicFont=cmunbi.ttf]{cmuntt.ttf}\setmonofont[Path=/usr/share/fonts/truetype/cmu/,UprightFont=cmuntt.ttf,BoldFont=cmuntb.ttf,ItalicFont=cmunit.ttf,BoldItalicFont=cmuntx.ttf]{cmuntt.ttf}\ttfamily .pdf}&\hspace*{0pt}\ignorespaces{}\hspace*{0pt}{\ttfamily doc}&\hspace*{0pt}\ignorespaces{}\hspace*{0pt}\setmainfont[Path=/usr/share/fonts/truetype/cmu/,UprightFont=cmunrm.ttf,BoldFont=cmunbx.ttf,ItalicFont=cmunti.ttf,BoldItalicFont=cmunbi.ttf]{cmunrm.ttf}\setmonofont[Path=/usr/share/fonts/truetype/cmu/,UprightFont=cmuntt.ttf,BoldFont=cmuntb.ttf,ItalicFont=cmunit.ttf,BoldItalicFont=cmuntx.ttf]{cmunrm.ttf}package documentation\\ \hline \hspace*{0pt}\ignorespaces{}\hspace*{0pt}{\ttfamily \setmainfont[Path=/usr/share/fonts/truetype/cmu/,UprightFont=cmunrm.ttf,BoldFont=cmunbx.ttf,ItalicFont=cmunti.ttf,BoldItalicFont=cmunbi.ttf]{cmuntt.ttf}\setmonofont[Path=/usr/share/fonts/truetype/cmu/,UprightFont=cmuntt.ttf,BoldFont=cmuntb.ttf,ItalicFont=cmunit.ttf,BoldItalicFont=cmuntx.ttf]{cmuntt.ttf}\ttfamily .pfb}&\hspace*{0pt}\ignorespaces{}\hspace*{0pt}{\ttfamily fonts/type1/{\itshape \setmainfont[Path=/usr/share/fonts/truetype/cmu/,UprightFont=cmunrm.ttf,BoldFont=cmunbx.ttf,ItalicFont=cmunti.ttf,BoldItalicFont=cmunbi.ttf]{cmunit.ttf}\setmonofont[Path=/usr/share/fonts/truetype/cmu/,UprightFont=cmuntt.ttf,BoldFont=cmuntb.ttf,ItalicFont=cmunit.ttf,BoldItalicFont=cmuntx.ttf]{cmunit.ttf}\ttfamily \itshape foundry}\setmainfont[Path=/usr/share/fonts/truetype/cmu/,UprightFont=cmunrm.ttf,BoldFont=cmunbx.ttf,ItalicFont=cmunti.ttf,BoldItalicFont=cmunbi.ttf]{cmuntt.ttf}\setmonofont[Path=/usr/share/fonts/truetype/cmu/,UprightFont=cmuntt.ttf,BoldFont=cmuntb.ttf,ItalicFont=cmunit.ttf,BoldItalicFont=cmuntx.ttf]{cmuntt.ttf}\ttfamily /{\itshape \setmainfont[Path=/usr/share/fonts/truetype/cmu/,UprightFont=cmunrm.ttf,BoldFont=cmunbx.ttf,ItalicFont=cmunti.ttf,BoldItalicFont=cmunbi.ttf]{cmunit.ttf}\setmonofont[Path=/usr/share/fonts/truetype/cmu/,UprightFont=cmuntt.ttf,BoldFont=cmuntb.ttf,ItalicFont=cmunit.ttf,BoldItalicFont=cmuntx.ttf]{cmunit.ttf}\ttfamily \itshape typeface}}&\hspace*{0pt}\ignorespaces{}\hspace*{0pt}\setmainfont[Path=/usr/share/fonts/truetype/cmu/,UprightFont=cmunrm.ttf,BoldFont=cmunbx.ttf,ItalicFont=cmunti.ttf,BoldItalicFont=cmunbi.ttf]{cmunrm.ttf}\setmonofont[Path=/usr/share/fonts/truetype/cmu/,UprightFont=cmuntt.ttf,BoldFont=cmuntb.ttf,ItalicFont=cmunit.ttf,BoldItalicFont=cmuntx.ttf]{cmunrm.ttf}PostScript Type 1 outline\\ \hline \hspace*{0pt}\ignorespaces{}\hspace*{0pt}{\ttfamily \setmainfont[Path=/usr/share/fonts/truetype/cmu/,UprightFont=cmunrm.ttf,BoldFont=cmunbx.ttf,ItalicFont=cmunti.ttf,BoldItalicFont=cmunbi.ttf]{cmuntt.ttf}\setmonofont[Path=/usr/share/fonts/truetype/cmu/,UprightFont=cmuntt.ttf,BoldFont=cmuntb.ttf,ItalicFont=cmunit.ttf,BoldItalicFont=cmuntx.ttf]{cmuntt.ttf}\ttfamily .sty}&\hspace*{0pt}\ignorespaces{}\hspace*{0pt}{\ttfamily tex/latex/{\itshape \setmainfont[Path=/usr/share/fonts/truetype/cmu/,UprightFont=cmunrm.ttf,BoldFont=cmunbx.ttf,ItalicFont=cmunti.ttf,BoldItalicFont=cmunbi.ttf]{cmunit.ttf}\setmonofont[Path=/usr/share/fonts/truetype/cmu/,UprightFont=cmuntt.ttf,BoldFont=cmuntb.ttf,ItalicFont=cmunit.ttf,BoldItalicFont=cmuntx.ttf]{cmunit.ttf}\ttfamily \itshape packagename}}&\hspace*{0pt}\ignorespaces{}\hspace*{0pt}\setmainfont[Path=/usr/share/fonts/truetype/cmu/,UprightFont=cmunrm.ttf,BoldFont=cmunbx.ttf,ItalicFont=cmunti.ttf,BoldItalicFont=cmunbi.ttf]{cmunrm.ttf}\setmonofont[Path=/usr/share/fonts/truetype/cmu/,UprightFont=cmuntt.ttf,BoldFont=cmuntb.ttf,ItalicFont=cmunit.ttf,BoldItalicFont=cmuntx.ttf]{cmunrm.ttf}Style file: the normal package content\\ \hline \hspace*{0pt}\ignorespaces{}\hspace*{0pt}{\ttfamily \setmainfont[Path=/usr/share/fonts/truetype/cmu/,UprightFont=cmunrm.ttf,BoldFont=cmunbx.ttf,ItalicFont=cmunti.ttf,BoldItalicFont=cmunbi.ttf]{cmuntt.ttf}\setmonofont[Path=/usr/share/fonts/truetype/cmu/,UprightFont=cmuntt.ttf,BoldFont=cmuntb.ttf,ItalicFont=cmunit.ttf,BoldItalicFont=cmuntx.ttf]{cmuntt.ttf}\ttfamily .tex}&\hspace*{0pt}\ignorespaces{}\hspace*{0pt}{\ttfamily doc}&\hspace*{0pt}\ignorespaces{}\hspace*{0pt}\setmainfont[Path=/usr/share/fonts/truetype/cmu/,UprightFont=cmunrm.ttf,BoldFont=cmunbx.ttf,ItalicFont=cmunti.ttf,BoldItalicFont=cmunbi.ttf]{cmunrm.ttf}\setmonofont[Path=/usr/share/fonts/truetype/cmu/,UprightFont=cmuntt.ttf,BoldFont=cmuntb.ttf,ItalicFont=cmunit.ttf,BoldItalicFont=cmuntx.ttf]{cmunrm.ttf}TeX source for package documentation\\ \hline \hspace*{0pt}\ignorespaces{}\hspace*{0pt}{\ttfamily \setmainfont[Path=/usr/share/fonts/truetype/cmu/,UprightFont=cmunrm.ttf,BoldFont=cmunbx.ttf,ItalicFont=cmunti.ttf,BoldItalicFont=cmunbi.ttf]{cmuntt.ttf}\setmonofont[Path=/usr/share/fonts/truetype/cmu/,UprightFont=cmuntt.ttf,BoldFont=cmuntb.ttf,ItalicFont=cmunit.ttf,BoldItalicFont=cmuntx.ttf]{cmuntt.ttf}\ttfamily .tex}&\hspace*{0pt}\ignorespaces{}\hspace*{0pt}{\ttfamily tex/plain/{\itshape \setmainfont[Path=/usr/share/fonts/truetype/cmu/,UprightFont=cmunrm.ttf,BoldFont=cmunbx.ttf,ItalicFont=cmunti.ttf,BoldItalicFont=cmunbi.ttf]{cmunit.ttf}\setmonofont[Path=/usr/share/fonts/truetype/cmu/,UprightFont=cmuntt.ttf,BoldFont=cmuntb.ttf,ItalicFont=cmunit.ttf,BoldItalicFont=cmuntx.ttf]{cmunit.ttf}\ttfamily \itshape packagename}}&\hspace*{0pt}\ignorespaces{}\hspace*{0pt}\setmainfont[Path=/usr/share/fonts/truetype/cmu/,UprightFont=cmunrm.ttf,BoldFont=cmunbx.ttf,ItalicFont=cmunti.ttf,BoldItalicFont=cmunbi.ttf]{cmunrm.ttf}\setmonofont[Path=/usr/share/fonts/truetype/cmu/,UprightFont=cmuntt.ttf,BoldFont=cmuntb.ttf,ItalicFont=cmunit.ttf,BoldItalicFont=cmuntx.ttf]{cmunrm.ttf}Plain TeX macro files\\ \hline \hspace*{0pt}\ignorespaces{}\hspace*{0pt}{\ttfamily \setmainfont[Path=/usr/share/fonts/truetype/cmu/,UprightFont=cmunrm.ttf,BoldFont=cmunbx.ttf,ItalicFont=cmunti.ttf,BoldItalicFont=cmunbi.ttf]{cmuntt.ttf}\setmonofont[Path=/usr/share/fonts/truetype/cmu/,UprightFont=cmuntt.ttf,BoldFont=cmuntb.ttf,ItalicFont=cmunit.ttf,BoldItalicFont=cmuntx.ttf]{cmuntt.ttf}\ttfamily .tfm}&\hspace*{0pt}\ignorespaces{}\hspace*{0pt}{\ttfamily fonts/tfm/{\itshape \setmainfont[Path=/usr/share/fonts/truetype/cmu/,UprightFont=cmunrm.ttf,BoldFont=cmunbx.ttf,ItalicFont=cmunti.ttf,BoldItalicFont=cmunbi.ttf]{cmunit.ttf}\setmonofont[Path=/usr/share/fonts/truetype/cmu/,UprightFont=cmuntt.ttf,BoldFont=cmuntb.ttf,ItalicFont=cmunit.ttf,BoldItalicFont=cmuntx.ttf]{cmunit.ttf}\ttfamily \itshape foundry}\setmainfont[Path=/usr/share/fonts/truetype/cmu/,UprightFont=cmunrm.ttf,BoldFont=cmunbx.ttf,ItalicFont=cmunti.ttf,BoldItalicFont=cmunbi.ttf]{cmuntt.ttf}\setmonofont[Path=/usr/share/fonts/truetype/cmu/,UprightFont=cmuntt.ttf,BoldFont=cmuntb.ttf,ItalicFont=cmunit.ttf,BoldItalicFont=cmuntx.ttf]{cmuntt.ttf}\ttfamily /{\itshape \setmainfont[Path=/usr/share/fonts/truetype/cmu/,UprightFont=cmunrm.ttf,BoldFont=cmunbx.ttf,ItalicFont=cmunti.ttf,BoldItalicFont=cmunbi.ttf]{cmunit.ttf}\setmonofont[Path=/usr/share/fonts/truetype/cmu/,UprightFont=cmuntt.ttf,BoldFont=cmuntb.ttf,ItalicFont=cmunit.ttf,BoldItalicFont=cmuntx.ttf]{cmunit.ttf}\ttfamily \itshape typeface}}&\hspace*{0pt}\ignorespaces{}\hspace*{0pt}\setmainfont[Path=/usr/share/fonts/truetype/cmu/,UprightFont=cmunrm.ttf,BoldFont=cmunbx.ttf,ItalicFont=cmunti.ttf,BoldItalicFont=cmunbi.ttf]{cmunrm.ttf}\setmonofont[Path=/usr/share/fonts/truetype/cmu/,UprightFont=cmuntt.ttf,BoldFont=cmuntb.ttf,ItalicFont=cmunit.ttf,BoldItalicFont=cmuntx.ttf]{cmunrm.ttf}TeX Font Metrics for METAFONT and Type 1 fonts\\ \hline \hspace*{0pt}\ignorespaces{}\hspace*{0pt}{\ttfamily \setmainfont[Path=/usr/share/fonts/truetype/cmu/,UprightFont=cmunrm.ttf,BoldFont=cmunbx.ttf,ItalicFont=cmunti.ttf,BoldItalicFont=cmunbi.ttf]{cmuntt.ttf}\setmonofont[Path=/usr/share/fonts/truetype/cmu/,UprightFont=cmuntt.ttf,BoldFont=cmuntb.ttf,ItalicFont=cmunit.ttf,BoldItalicFont=cmuntx.ttf]{cmuntt.ttf}\ttfamily .ttf}&\hspace*{0pt}\ignorespaces{}\hspace*{0pt}{\ttfamily fonts/truetype/{\itshape \setmainfont[Path=/usr/share/fonts/truetype/cmu/,UprightFont=cmunrm.ttf,BoldFont=cmunbx.ttf,ItalicFont=cmunti.ttf,BoldItalicFont=cmunbi.ttf]{cmunit.ttf}\setmonofont[Path=/usr/share/fonts/truetype/cmu/,UprightFont=cmuntt.ttf,BoldFont=cmuntb.ttf,ItalicFont=cmunit.ttf,BoldItalicFont=cmuntx.ttf]{cmunit.ttf}\ttfamily \itshape foundry}\setmainfont[Path=/usr/share/fonts/truetype/cmu/,UprightFont=cmunrm.ttf,BoldFont=cmunbx.ttf,ItalicFont=cmunti.ttf,BoldItalicFont=cmunbi.ttf]{cmuntt.ttf}\setmonofont[Path=/usr/share/fonts/truetype/cmu/,UprightFont=cmuntt.ttf,BoldFont=cmuntb.ttf,ItalicFont=cmunit.ttf,BoldItalicFont=cmuntx.ttf]{cmuntt.ttf}\ttfamily /{\itshape \setmainfont[Path=/usr/share/fonts/truetype/cmu/,UprightFont=cmunrm.ttf,BoldFont=cmunbx.ttf,ItalicFont=cmunti.ttf,BoldItalicFont=cmunbi.ttf]{cmunit.ttf}\setmonofont[Path=/usr/share/fonts/truetype/cmu/,UprightFont=cmuntt.ttf,BoldFont=cmuntb.ttf,ItalicFont=cmunit.ttf,BoldItalicFont=cmuntx.ttf]{cmunit.ttf}\ttfamily \itshape typeface}}&\hspace*{0pt}\ignorespaces{}\hspace*{0pt}\setmainfont[Path=/usr/share/fonts/truetype/cmu/,UprightFont=cmunrm.ttf,BoldFont=cmunbx.ttf,ItalicFont=cmunti.ttf,BoldItalicFont=cmunbi.ttf]{cmunrm.ttf}\setmonofont[Path=/usr/share/fonts/truetype/cmu/,UprightFont=cmuntt.ttf,BoldFont=cmuntb.ttf,ItalicFont=cmunit.ttf,BoldItalicFont=cmuntx.ttf]{cmunrm.ttf}TrueType font\\ \hline \hspace*{0pt}\ignorespaces{}\hspace*{0pt}{\ttfamily \setmainfont[Path=/usr/share/fonts/truetype/cmu/,UprightFont=cmunrm.ttf,BoldFont=cmunbx.ttf,ItalicFont=cmunti.ttf,BoldItalicFont=cmunbi.ttf]{cmuntt.ttf}\setmonofont[Path=/usr/share/fonts/truetype/cmu/,UprightFont=cmuntt.ttf,BoldFont=cmuntb.ttf,ItalicFont=cmunit.ttf,BoldItalicFont=cmuntx.ttf]{cmuntt.ttf}\ttfamily .vf}&\hspace*{0pt}\ignorespaces{}\hspace*{0pt}{\ttfamily fonts/vf/{\itshape \setmainfont[Path=/usr/share/fonts/truetype/cmu/,UprightFont=cmunrm.ttf,BoldFont=cmunbx.ttf,ItalicFont=cmunti.ttf,BoldItalicFont=cmunbi.ttf]{cmunit.ttf}\setmonofont[Path=/usr/share/fonts/truetype/cmu/,UprightFont=cmuntt.ttf,BoldFont=cmuntb.ttf,ItalicFont=cmunit.ttf,BoldItalicFont=cmuntx.ttf]{cmunit.ttf}\ttfamily \itshape foundry}\setmainfont[Path=/usr/share/fonts/truetype/cmu/,UprightFont=cmunrm.ttf,BoldFont=cmunbx.ttf,ItalicFont=cmunti.ttf,BoldItalicFont=cmunbi.ttf]{cmuntt.ttf}\setmonofont[Path=/usr/share/fonts/truetype/cmu/,UprightFont=cmuntt.ttf,BoldFont=cmuntb.ttf,ItalicFont=cmunit.ttf,BoldItalicFont=cmuntx.ttf]{cmuntt.ttf}\ttfamily /{\itshape \setmainfont[Path=/usr/share/fonts/truetype/cmu/,UprightFont=cmunrm.ttf,BoldFont=cmunbx.ttf,ItalicFont=cmunti.ttf,BoldItalicFont=cmunbi.ttf]{cmunit.ttf}\setmonofont[Path=/usr/share/fonts/truetype/cmu/,UprightFont=cmuntt.ttf,BoldFont=cmuntb.ttf,ItalicFont=cmunit.ttf,BoldItalicFont=cmuntx.ttf]{cmunit.ttf}\ttfamily \itshape typeface}}&\hspace*{0pt}\ignorespaces{}\hspace*{0pt}\setmainfont[Path=/usr/share/fonts/truetype/cmu/,UprightFont=cmunrm.ttf,BoldFont=cmunbx.ttf,ItalicFont=cmunti.ttf,BoldItalicFont=cmunbi.ttf]{cmunrm.ttf}\setmonofont[Path=/usr/share/fonts/truetype/cmu/,UprightFont=cmuntt.ttf,BoldFont=cmuntb.ttf,ItalicFont=cmunit.ttf,BoldItalicFont=cmuntx.ttf]{cmunrm.ttf}TeX virtual fonts\\ \hline \hspace*{0pt}\ignorespaces{}\hspace*{0pt}others &\hspace*{0pt}\ignorespaces{}\hspace*{0pt}{\ttfamily \setmainfont[Path=/usr/share/fonts/truetype/cmu/,UprightFont=cmunrm.ttf,BoldFont=cmunbx.ttf,ItalicFont=cmunti.ttf,BoldItalicFont=cmunbi.ttf]{cmuntt.ttf}\setmonofont[Path=/usr/share/fonts/truetype/cmu/,UprightFont=cmuntt.ttf,BoldFont=cmuntb.ttf,ItalicFont=cmunit.ttf,BoldItalicFont=cmuntx.ttf]{cmuntt.ttf}\ttfamily tex/latex/{\itshape \setmainfont[Path=/usr/share/fonts/truetype/cmu/,UprightFont=cmunrm.ttf,BoldFont=cmunbx.ttf,ItalicFont=cmunti.ttf,BoldItalicFont=cmunbi.ttf]{cmunit.ttf}\setmonofont[Path=/usr/share/fonts/truetype/cmu/,UprightFont=cmuntt.ttf,BoldFont=cmuntb.ttf,ItalicFont=cmunit.ttf,BoldItalicFont=cmuntx.ttf]{cmunit.ttf}\ttfamily \itshape packagename}}&\hspace*{0pt}\ignorespaces{}\hspace*{0pt}\setmainfont[Path=/usr/share/fonts/truetype/cmu/,UprightFont=cmunrm.ttf,BoldFont=cmunbx.ttf,ItalicFont=cmunti.ttf,BoldItalicFont=cmunbi.ttf]{cmunrm.ttf}\setmonofont[Path=/usr/share/fonts/truetype/cmu/,UprightFont=cmuntt.ttf,BoldFont=cmuntb.ttf,ItalicFont=cmunit.ttf,BoldItalicFont=cmuntx.ttf]{cmunrm.ttf}other types of file unless instructed otherwise\\ \hline 
\end{longtable}


For most fonts on CTAN, the {\ttfamily {\itshape \setmainfont[Path=/usr/share/fonts/truetype/cmu/,UprightFont=cmunrm.ttf,BoldFont=cmunbx.ttf,ItalicFont=cmunti.ttf,BoldItalicFont=cmunbi.ttf]{cmunit.ttf}\setmonofont[Path=/usr/share/fonts/truetype/cmu/,UprightFont=cmuntt.ttf,BoldFont=cmuntb.ttf,ItalicFont=cmunit.ttf,BoldItalicFont=cmuntx.ttf]{cmunit.ttf}\ttfamily \itshape foundry}}{$\text{ }$}\setmainfont[Path=/usr/share/fonts/truetype/cmu/,UprightFont=cmunrm.ttf,BoldFont=cmunbx.ttf,ItalicFont=cmunti.ttf,BoldItalicFont=cmunbi.ttf]{cmunrm.ttf}\setmonofont[Path=/usr/share/fonts/truetype/cmu/,UprightFont=cmuntt.ttf,BoldFont=cmuntb.ttf,ItalicFont=cmunit.ttf,BoldItalicFont=cmuntx.ttf]{cmunrm.ttf} is {\ttfamily \setmainfont[Path=/usr/share/fonts/truetype/cmu/,UprightFont=cmunrm.ttf,BoldFont=cmunbx.ttf,ItalicFont=cmunti.ttf,BoldItalicFont=cmunbi.ttf]{cmuntt.ttf}\setmonofont[Path=/usr/share/fonts/truetype/cmu/,UprightFont=cmuntt.ttf,BoldFont=cmuntb.ttf,ItalicFont=cmunit.ttf,BoldItalicFont=cmuntx.ttf]{cmuntt.ttf}\ttfamily public}\setmainfont[Path=/usr/share/fonts/truetype/cmu/,UprightFont=cmunrm.ttf,BoldFont=cmunbx.ttf,ItalicFont=cmunti.ttf,BoldItalicFont=cmunbi.ttf]{cmunrm.ttf}\setmonofont[Path=/usr/share/fonts/truetype/cmu/,UprightFont=cmuntt.ttf,BoldFont=cmuntb.ttf,ItalicFont=cmunit.ttf,BoldItalicFont=cmuntx.ttf]{cmunrm.ttf}.

4. {\bfseries \setmainfont[Path=/usr/share/fonts/truetype/cmu/,UprightFont=cmunrm.ttf,BoldFont=cmunbx.ttf,ItalicFont=cmunti.ttf,BoldItalicFont=cmunbi.ttf]{cmunbx.ttf}\setmonofont[Path=/usr/share/fonts/truetype/cmu/,UprightFont=cmuntt.ttf,BoldFont=cmuntb.ttf,ItalicFont=cmunit.ttf,BoldItalicFont=cmuntx.ttf]{cmunbx.ttf}\bfseries Update your index}{$\text{ }$}\setmainfont[Path=/usr/share/fonts/truetype/cmu/,UprightFont=cmunrm.ttf,BoldFont=cmunbx.ttf,ItalicFont=cmunti.ttf,BoldItalicFont=cmunbi.ttf]{cmunrm.ttf}\setmonofont[Path=/usr/share/fonts/truetype/cmu/,UprightFont=cmuntt.ttf,BoldFont=cmuntb.ttf,ItalicFont=cmunit.ttf,BoldItalicFont=cmuntx.ttf]{cmunrm.ttf} Finally, run your TeX indexer program to update the package database. This program comes with every modern version of TeX and has various names depending on the LaTeX distribution you use. (Read the documentation that came with your installation to find out which it is, or consult \myplainurl{http://www.tug.org/fonts/fontinstall.html\#fndb):}
\begin{myquote}
\item{} 
\begin{myitemize}
\item{}  teTeX, TeX Live, fpTeX: {\ttfamily \setmainfont[Path=/usr/share/fonts/truetype/cmu/,UprightFont=cmunrm.ttf,BoldFont=cmunbx.ttf,ItalicFont=cmunti.ttf,BoldItalicFont=cmunbi.ttf]{cmuntt.ttf}\setmonofont[Path=/usr/share/fonts/truetype/cmu/,UprightFont=cmuntt.ttf,BoldFont=cmuntb.ttf,ItalicFont=cmunit.ttf,BoldItalicFont=cmuntx.ttf]{cmuntt.ttf}\ttfamily texhash}
\item{} {$\text{ }$}\setmainfont[Path=/usr/share/fonts/truetype/cmu/,UprightFont=cmunrm.ttf,BoldFont=cmunbx.ttf,ItalicFont=cmunti.ttf,BoldItalicFont=cmunbi.ttf]{cmunrm.ttf}\setmonofont[Path=/usr/share/fonts/truetype/cmu/,UprightFont=cmuntt.ttf,BoldFont=cmuntb.ttf,ItalicFont=cmunit.ttf,BoldItalicFont=cmuntx.ttf]{cmunrm.ttf} web2c: {\ttfamily \setmainfont[Path=/usr/share/fonts/truetype/cmu/,UprightFont=cmunrm.ttf,BoldFont=cmunbx.ttf,ItalicFont=cmunti.ttf,BoldItalicFont=cmunbi.ttf]{cmuntt.ttf}\setmonofont[Path=/usr/share/fonts/truetype/cmu/,UprightFont=cmuntt.ttf,BoldFont=cmuntb.ttf,ItalicFont=cmunit.ttf,BoldItalicFont=cmuntx.ttf]{cmuntt.ttf}\ttfamily mktexlsr}
\item{} {$\text{ }$}\setmainfont[Path=/usr/share/fonts/truetype/cmu/,UprightFont=cmunrm.ttf,BoldFont=cmunbx.ttf,ItalicFont=cmunti.ttf,BoldItalicFont=cmunbi.ttf]{cmunrm.ttf}\setmonofont[Path=/usr/share/fonts/truetype/cmu/,UprightFont=cmuntt.ttf,BoldFont=cmuntb.ttf,ItalicFont=cmunit.ttf,BoldItalicFont=cmuntx.ttf]{cmunrm.ttf} MacTeX: MacTeX appears to do this for you.
\item{}  MikTeX: {\ttfamily \setmainfont[Path=/usr/share/fonts/truetype/cmu/,UprightFont=cmunrm.ttf,BoldFont=cmunbx.ttf,ItalicFont=cmunti.ttf,BoldItalicFont=cmunbi.ttf]{cmuntt.ttf}\setmonofont[Path=/usr/share/fonts/truetype/cmu/,UprightFont=cmuntt.ttf,BoldFont=cmuntb.ttf,ItalicFont=cmunit.ttf,BoldItalicFont=cmuntx.ttf]{cmuntt.ttf}\ttfamily initexmf -{}-{}update-{}fndb}{$\text{ }$}\setmainfont[Path=/usr/share/fonts/truetype/cmu/,UprightFont=cmunrm.ttf,BoldFont=cmunbx.ttf,ItalicFont=cmunti.ttf,BoldItalicFont=cmunbi.ttf]{cmunrm.ttf}\setmonofont[Path=/usr/share/fonts/truetype/cmu/,UprightFont=cmuntt.ttf,BoldFont=cmuntb.ttf,ItalicFont=cmunit.ttf,BoldItalicFont=cmuntx.ttf]{cmunrm.ttf} (or use the GUI)
\item{}  MiKTeX 2.7 or later versions, installed on Windows XP through Windows 7: Start -{}>{} All Programs -{}>{} MikTex -{}>{} Settings.  In Windows 8 use the keyword {\itshape \setmainfont[Path=/usr/share/fonts/truetype/cmu/,UprightFont=cmunrm.ttf,BoldFont=cmunbx.ttf,ItalicFont=cmunti.ttf,BoldItalicFont=cmunbi.ttf]{cmunti.ttf}\setmonofont[Path=/usr/share/fonts/truetype/cmu/,UprightFont=cmuntt.ttf,BoldFont=cmuntb.ttf,ItalicFont=cmunit.ttf,BoldItalicFont=cmuntx.ttf]{cmunti.ttf}\itshape Settings}{$\text{ }$}\setmainfont[Path=/usr/share/fonts/truetype/cmu/,UprightFont=cmunrm.ttf,BoldFont=cmunbx.ttf,ItalicFont=cmunti.ttf,BoldItalicFont=cmunbi.ttf]{cmunrm.ttf}\setmonofont[Path=/usr/share/fonts/truetype/cmu/,UprightFont=cmuntt.ttf,BoldFont=cmuntb.ttf,ItalicFont=cmunit.ttf,BoldItalicFont=cmuntx.ttf]{cmunrm.ttf} and choose the option of {\itshape \setmainfont[Path=/usr/share/fonts/truetype/cmu/,UprightFont=cmunrm.ttf,BoldFont=cmunbx.ttf,ItalicFont=cmunti.ttf,BoldItalicFont=cmunbi.ttf]{cmunti.ttf}\setmonofont[Path=/usr/share/fonts/truetype/cmu/,UprightFont=cmuntt.ttf,BoldFont=cmuntb.ttf,ItalicFont=cmunit.ttf,BoldItalicFont=cmuntx.ttf]{cmunti.ttf}\itshape Settings}{$\text{ }$}\setmainfont[Path=/usr/share/fonts/truetype/cmu/,UprightFont=cmunrm.ttf,BoldFont=cmunbx.ttf,ItalicFont=cmunti.ttf,BoldItalicFont=cmunbi.ttf]{cmunrm.ttf}\setmonofont[Path=/usr/share/fonts/truetype/cmu/,UprightFont=cmuntt.ttf,BoldFont=cmuntb.ttf,ItalicFont=cmunit.ttf,BoldItalicFont=cmuntx.ttf]{cmunrm.ttf} with the MiKTex logo.  In {\itshape \setmainfont[Path=/usr/share/fonts/truetype/cmu/,UprightFont=cmunrm.ttf,BoldFont=cmunbx.ttf,ItalicFont=cmunti.ttf,BoldItalicFont=cmunbi.ttf]{cmunti.ttf}\setmonofont[Path=/usr/share/fonts/truetype/cmu/,UprightFont=cmuntt.ttf,BoldFont=cmuntb.ttf,ItalicFont=cmunit.ttf,BoldItalicFont=cmuntx.ttf]{cmunti.ttf}\itshape Settings}{$\text{ }$}\setmainfont[Path=/usr/share/fonts/truetype/cmu/,UprightFont=cmunrm.ttf,BoldFont=cmunbx.ttf,ItalicFont=cmunti.ttf,BoldItalicFont=cmunbi.ttf]{cmunrm.ttf}\setmonofont[Path=/usr/share/fonts/truetype/cmu/,UprightFont=cmuntt.ttf,BoldFont=cmuntb.ttf,ItalicFont=cmunit.ttf,BoldItalicFont=cmuntx.ttf]{cmunrm.ttf} menu choose the first tab and click on {\itshape \setmainfont[Path=/usr/share/fonts/truetype/cmu/,UprightFont=cmunrm.ttf,BoldFont=cmunbx.ttf,ItalicFont=cmunti.ttf,BoldItalicFont=cmunbi.ttf]{cmunti.ttf}\setmonofont[Path=/usr/share/fonts/truetype/cmu/,UprightFont=cmuntt.ttf,BoldFont=cmuntb.ttf,ItalicFont=cmunit.ttf,BoldItalicFont=cmuntx.ttf]{cmunti.ttf}\itshape Refresh FNDB}\setmainfont[Path=/usr/share/fonts/truetype/cmu/,UprightFont=cmunrm.ttf,BoldFont=cmunbx.ttf,ItalicFont=cmunti.ttf,BoldItalicFont=cmunbi.ttf]{cmunrm.ttf}\setmonofont[Path=/usr/share/fonts/truetype/cmu/,UprightFont=cmuntt.ttf,BoldFont=cmuntb.ttf,ItalicFont=cmunit.ttf,BoldItalicFont=cmuntx.ttf]{cmunrm.ttf}-{}button (MikTex will then check the Program Files directory and update the list of File Name DataBase). After that just verify by clicking \textquotesingle{}OK\textquotesingle{}.
\end{myitemize}

\end{myquote}


\begin{TemplateInfo}{\danger}{Warning}This step is utterly essential, otherwise nothing will work.\end{TemplateInfo}

5. {\bfseries \setmainfont[Path=/usr/share/fonts/truetype/cmu/,UprightFont=cmunrm.ttf,BoldFont=cmunbx.ttf,ItalicFont=cmunti.ttf,BoldItalicFont=cmunbi.ttf]{cmunbx.ttf}\setmonofont[Path=/usr/share/fonts/truetype/cmu/,UprightFont=cmuntt.ttf,BoldFont=cmuntb.ttf,ItalicFont=cmunit.ttf,BoldItalicFont=cmuntx.ttf]{cmunbx.ttf}\bfseries Update font maps}{$\text{ }$}\setmainfont[Path=/usr/share/fonts/truetype/cmu/,UprightFont=cmunrm.ttf,BoldFont=cmunbx.ttf,ItalicFont=cmunti.ttf,BoldItalicFont=cmunbi.ttf]{cmunrm.ttf}\setmonofont[Path=/usr/share/fonts/truetype/cmu/,UprightFont=cmuntt.ttf,BoldFont=cmuntb.ttf,ItalicFont=cmunit.ttf,BoldItalicFont=cmuntx.ttf]{cmunrm.ttf} If your package installed any TrueType or Type 1 fonts, you need to update the font mapping files {\itshape \setmainfont[Path=/usr/share/fonts/truetype/cmu/,UprightFont=cmunrm.ttf,BoldFont=cmunbx.ttf,ItalicFont=cmunti.ttf,BoldItalicFont=cmunbi.ttf]{cmunti.ttf}\setmonofont[Path=/usr/share/fonts/truetype/cmu/,UprightFont=cmuntt.ttf,BoldFont=cmuntb.ttf,ItalicFont=cmunit.ttf,BoldItalicFont=cmuntx.ttf]{cmunti.ttf}\itshape in addition}{$\text{ }$}\setmainfont[Path=/usr/share/fonts/truetype/cmu/,UprightFont=cmunrm.ttf,BoldFont=cmunbx.ttf,ItalicFont=cmunti.ttf,BoldItalicFont=cmunbi.ttf]{cmunrm.ttf}\setmonofont[Path=/usr/share/fonts/truetype/cmu/,UprightFont=cmuntt.ttf,BoldFont=cmuntb.ttf,ItalicFont=cmunit.ttf,BoldItalicFont=cmuntx.ttf]{cmunrm.ttf} to updating the index. Your package author should have included a {\ttfamily \setmainfont[Path=/usr/share/fonts/truetype/cmu/,UprightFont=cmunrm.ttf,BoldFont=cmunbx.ttf,ItalicFont=cmunti.ttf,BoldItalicFont=cmunbi.ttf]{cmuntt.ttf}\setmonofont[Path=/usr/share/fonts/truetype/cmu/,UprightFont=cmuntt.ttf,BoldFont=cmuntb.ttf,ItalicFont=cmunit.ttf,BoldItalicFont=cmuntx.ttf]{cmuntt.ttf}\ttfamily .map}{$\text{ }$}\setmainfont[Path=/usr/share/fonts/truetype/cmu/,UprightFont=cmunrm.ttf,BoldFont=cmunbx.ttf,ItalicFont=cmunti.ttf,BoldItalicFont=cmunbi.ttf]{cmunrm.ttf}\setmonofont[Path=/usr/share/fonts/truetype/cmu/,UprightFont=cmuntt.ttf,BoldFont=cmuntb.ttf,ItalicFont=cmunit.ttf,BoldItalicFont=cmuntx.ttf]{cmunrm.ttf} file for the fonts. The map updating program is usually some variant on {\ttfamily \setmainfont[Path=/usr/share/fonts/truetype/cmu/,UprightFont=cmunrm.ttf,BoldFont=cmunbx.ttf,ItalicFont=cmunti.ttf,BoldItalicFont=cmunbi.ttf]{cmuntt.ttf}\setmonofont[Path=/usr/share/fonts/truetype/cmu/,UprightFont=cmuntt.ttf,BoldFont=cmuntb.ttf,ItalicFont=cmunit.ttf,BoldItalicFont=cmuntx.ttf]{cmuntt.ttf}\ttfamily updmap}\setmainfont[Path=/usr/share/fonts/truetype/cmu/,UprightFont=cmunrm.ttf,BoldFont=cmunbx.ttf,ItalicFont=cmunti.ttf,BoldItalicFont=cmunbi.ttf]{cmunrm.ttf}\setmonofont[Path=/usr/share/fonts/truetype/cmu/,UprightFont=cmuntt.ttf,BoldFont=cmuntb.ttf,ItalicFont=cmunit.ttf,BoldItalicFont=cmuntx.ttf]{cmunrm.ttf}, depending on your distribution:
\begin{myquote}
\item{} 
\begin{myitemize}
\item{}  TeX Live and MacTeX: {\ttfamily \setmainfont[Path=/usr/share/fonts/truetype/cmu/,UprightFont=cmunrm.ttf,BoldFont=cmunbx.ttf,ItalicFont=cmunti.ttf,BoldItalicFont=cmunbi.ttf]{cmuntt.ttf}\setmonofont[Path=/usr/share/fonts/truetype/cmu/,UprightFont=cmuntt.ttf,BoldFont=cmuntb.ttf,ItalicFont=cmunit.ttf,BoldItalicFont=cmuntx.ttf]{cmuntt.ttf}\ttfamily updmap -{}-{}enable Map={\itshape \setmainfont[Path=/usr/share/fonts/truetype/cmu/,UprightFont=cmunrm.ttf,BoldFont=cmunbx.ttf,ItalicFont=cmunti.ttf,BoldItalicFont=cmunbi.ttf]{cmunit.ttf}\setmonofont[Path=/usr/share/fonts/truetype/cmu/,UprightFont=cmuntt.ttf,BoldFont=cmuntb.ttf,ItalicFont=cmunit.ttf,BoldItalicFont=cmuntx.ttf]{cmunit.ttf}\ttfamily \itshape mapfile}\setmainfont[Path=/usr/share/fonts/truetype/cmu/,UprightFont=cmunrm.ttf,BoldFont=cmunbx.ttf,ItalicFont=cmunti.ttf,BoldItalicFont=cmunbi.ttf]{cmuntt.ttf}\setmonofont[Path=/usr/share/fonts/truetype/cmu/,UprightFont=cmuntt.ttf,BoldFont=cmuntb.ttf,ItalicFont=cmunit.ttf,BoldItalicFont=cmuntx.ttf]{cmuntt.ttf}\ttfamily .map}{$\text{ }$}\setmainfont[Path=/usr/share/fonts/truetype/cmu/,UprightFont=cmunrm.ttf,BoldFont=cmunbx.ttf,ItalicFont=cmunti.ttf,BoldItalicFont=cmunbi.ttf]{cmunrm.ttf}\setmonofont[Path=/usr/share/fonts/truetype/cmu/,UprightFont=cmuntt.ttf,BoldFont=cmuntb.ttf,ItalicFont=cmunit.ttf,BoldItalicFont=cmuntx.ttf]{cmunrm.ttf} (if you installed the files in a personal tree) or {\ttfamily \setmainfont[Path=/usr/share/fonts/truetype/cmu/,UprightFont=cmunrm.ttf,BoldFont=cmunbx.ttf,ItalicFont=cmunti.ttf,BoldItalicFont=cmunbi.ttf]{cmuntt.ttf}\setmonofont[Path=/usr/share/fonts/truetype/cmu/,UprightFont=cmuntt.ttf,BoldFont=cmuntb.ttf,ItalicFont=cmunit.ttf,BoldItalicFont=cmuntx.ttf]{cmuntt.ttf}\ttfamily updmap-{}sys -{}-{}enable Map={\itshape \setmainfont[Path=/usr/share/fonts/truetype/cmu/,UprightFont=cmunrm.ttf,BoldFont=cmunbx.ttf,ItalicFont=cmunti.ttf,BoldItalicFont=cmunbi.ttf]{cmunit.ttf}\setmonofont[Path=/usr/share/fonts/truetype/cmu/,UprightFont=cmuntt.ttf,BoldFont=cmuntb.ttf,ItalicFont=cmunit.ttf,BoldItalicFont=cmuntx.ttf]{cmunit.ttf}\ttfamily \itshape mapfile}\setmainfont[Path=/usr/share/fonts/truetype/cmu/,UprightFont=cmunrm.ttf,BoldFont=cmunbx.ttf,ItalicFont=cmunti.ttf,BoldItalicFont=cmunbi.ttf]{cmuntt.ttf}\setmonofont[Path=/usr/share/fonts/truetype/cmu/,UprightFont=cmuntt.ttf,BoldFont=cmuntb.ttf,ItalicFont=cmunit.ttf,BoldItalicFont=cmuntx.ttf]{cmuntt.ttf}\ttfamily .map}{$\text{ }$}\setmainfont[Path=/usr/share/fonts/truetype/cmu/,UprightFont=cmunrm.ttf,BoldFont=cmunbx.ttf,ItalicFont=cmunti.ttf,BoldItalicFont=cmunbi.ttf]{cmunrm.ttf}\setmonofont[Path=/usr/share/fonts/truetype/cmu/,UprightFont=cmuntt.ttf,BoldFont=cmuntb.ttf,ItalicFont=cmunit.ttf,BoldItalicFont=cmuntx.ttf]{cmunrm.ttf} (if you installed the files in a system directory).
\item{}  MikTeX: Run {\ttfamily \setmainfont[Path=/usr/share/fonts/truetype/cmu/,UprightFont=cmunrm.ttf,BoldFont=cmunbx.ttf,ItalicFont=cmunti.ttf,BoldItalicFont=cmunbi.ttf]{cmuntt.ttf}\setmonofont[Path=/usr/share/fonts/truetype/cmu/,UprightFont=cmuntt.ttf,BoldFont=cmuntb.ttf,ItalicFont=cmunit.ttf,BoldItalicFont=cmuntx.ttf]{cmuntt.ttf}\ttfamily initexmf -{}-{}edit-{}config-{}file updmap}\setmainfont[Path=/usr/share/fonts/truetype/cmu/,UprightFont=cmunrm.ttf,BoldFont=cmunbx.ttf,ItalicFont=cmunti.ttf,BoldItalicFont=cmunbi.ttf]{cmunrm.ttf}\setmonofont[Path=/usr/share/fonts/truetype/cmu/,UprightFont=cmuntt.ttf,BoldFont=cmuntb.ttf,ItalicFont=cmunit.ttf,BoldItalicFont=cmuntx.ttf]{cmunrm.ttf}, add the line \symbol{34}{\ttfamily \setmainfont[Path=/usr/share/fonts/truetype/cmu/,UprightFont=cmunrm.ttf,BoldFont=cmunbx.ttf,ItalicFont=cmunti.ttf,BoldItalicFont=cmunbi.ttf]{cmuntt.ttf}\setmonofont[Path=/usr/share/fonts/truetype/cmu/,UprightFont=cmuntt.ttf,BoldFont=cmuntb.ttf,ItalicFont=cmunit.ttf,BoldItalicFont=cmuntx.ttf]{cmuntt.ttf}\ttfamily Map {\itshape \setmainfont[Path=/usr/share/fonts/truetype/cmu/,UprightFont=cmunrm.ttf,BoldFont=cmunbx.ttf,ItalicFont=cmunti.ttf,BoldItalicFont=cmunbi.ttf]{cmunit.ttf}\setmonofont[Path=/usr/share/fonts/truetype/cmu/,UprightFont=cmuntt.ttf,BoldFont=cmuntb.ttf,ItalicFont=cmunit.ttf,BoldItalicFont=cmuntx.ttf]{cmunit.ttf}\ttfamily \itshape mapfile}\setmainfont[Path=/usr/share/fonts/truetype/cmu/,UprightFont=cmunrm.ttf,BoldFont=cmunbx.ttf,ItalicFont=cmunti.ttf,BoldItalicFont=cmunbi.ttf]{cmuntt.ttf}\setmonofont[Path=/usr/share/fonts/truetype/cmu/,UprightFont=cmuntt.ttf,BoldFont=cmuntb.ttf,ItalicFont=cmunit.ttf,BoldItalicFont=cmuntx.ttf]{cmuntt.ttf}\ttfamily .map}{$\text{ }$}\setmainfont[Path=/usr/share/fonts/truetype/cmu/,UprightFont=cmunrm.ttf,BoldFont=cmunbx.ttf,ItalicFont=cmunti.ttf,BoldItalicFont=cmunbi.ttf]{cmunrm.ttf}\setmonofont[Path=/usr/share/fonts/truetype/cmu/,UprightFont=cmuntt.ttf,BoldFont=cmuntb.ttf,ItalicFont=cmunit.ttf,BoldItalicFont=cmuntx.ttf]{cmunrm.ttf} to the file that opens, then run {\ttfamily \setmainfont[Path=/usr/share/fonts/truetype/cmu/,UprightFont=cmunrm.ttf,BoldFont=cmunbx.ttf,ItalicFont=cmunti.ttf,BoldItalicFont=cmunbi.ttf]{cmuntt.ttf}\setmonofont[Path=/usr/share/fonts/truetype/cmu/,UprightFont=cmuntt.ttf,BoldFont=cmuntb.ttf,ItalicFont=cmunit.ttf,BoldItalicFont=cmuntx.ttf]{cmuntt.ttf}\ttfamily initexmf -{}-{}mkmaps}\setmainfont[Path=/usr/share/fonts/truetype/cmu/,UprightFont=cmunrm.ttf,BoldFont=cmunbx.ttf,ItalicFont=cmunti.ttf,BoldItalicFont=cmunbi.ttf]{cmunrm.ttf}\setmonofont[Path=/usr/share/fonts/truetype/cmu/,UprightFont=cmuntt.ttf,BoldFont=cmuntb.ttf,ItalicFont=cmunit.ttf,BoldItalicFont=cmuntx.ttf]{cmunrm.ttf}.
\end{myitemize}

\end{myquote}


See \myplainurl{http://www.tug.org/fonts/fontinstall.html.}

The reason this process has not been automated widely is that there are still thousands of installations which do not conform to the TDS, such as old shared Unix systems and some Microsoft Windows systems, so there is no way for an installation program to guess where to put the files: you have to know this. There are
also systems where the owner, user, or installer has chosen not to follow the recommended TDS directory structure, or is unable
to do so for political or security reasons (such as a shared system where the user cannot write to a protected directory). The reason for having the {\ttfamily \setmainfont[Path=/usr/share/fonts/truetype/cmu/,UprightFont=cmunrm.ttf,BoldFont=cmunbx.ttf,ItalicFont=cmunti.ttf,BoldItalicFont=cmunbi.ttf]{cmuntt.ttf}\setmonofont[Path=/usr/share/fonts/truetype/cmu/,UprightFont=cmuntt.ttf,BoldFont=cmuntb.ttf,ItalicFont=cmunit.ttf,BoldItalicFont=cmuntx.ttf]{cmuntt.ttf}\ttfamily texmf-{}local}{$\text{ }$}\setmainfont[Path=/usr/share/fonts/truetype/cmu/,UprightFont=cmunrm.ttf,BoldFont=cmunbx.ttf,ItalicFont=cmunti.ttf,BoldItalicFont=cmunbi.ttf]{cmunrm.ttf}\setmonofont[Path=/usr/share/fonts/truetype/cmu/,UprightFont=cmuntt.ttf,BoldFont=cmuntb.ttf,ItalicFont=cmunit.ttf,BoldItalicFont=cmuntx.ttf]{cmunrm.ttf} directory (called {\ttfamily \setmainfont[Path=/usr/share/fonts/truetype/cmu/,UprightFont=cmunrm.ttf,BoldFont=cmunbx.ttf,ItalicFont=cmunti.ttf,BoldItalicFont=cmunbi.ttf]{cmuntt.ttf}\setmonofont[Path=/usr/share/fonts/truetype/cmu/,UprightFont=cmuntt.ttf,BoldFont=cmuntb.ttf,ItalicFont=cmunit.ttf,BoldItalicFont=cmuntx.ttf]{cmuntt.ttf}\ttfamily texmf.local}{$\text{ }$}\setmainfont[Path=/usr/share/fonts/truetype/cmu/,UprightFont=cmunrm.ttf,BoldFont=cmunbx.ttf,ItalicFont=cmunti.ttf,BoldItalicFont=cmunbi.ttf]{cmunrm.ttf}\setmonofont[Path=/usr/share/fonts/truetype/cmu/,UprightFont=cmuntt.ttf,BoldFont=cmuntb.ttf,ItalicFont=cmunit.ttf,BoldItalicFont=cmuntx.ttf]{cmunrm.ttf} on some systems) is to provide a place for local modifications or personal updates, especially if you are a user on a shared or managed system (Unix, Linux, VMS, Windows NT/2000/XP, etc.) where you may not have write-{}access to the main TeX installation directory tree. You can also have a personal {\ttfamily \setmainfont[Path=/usr/share/fonts/truetype/cmu/,UprightFont=cmunrm.ttf,BoldFont=cmunbx.ttf,ItalicFont=cmunti.ttf,BoldItalicFont=cmunbi.ttf]{cmuntt.ttf}\setmonofont[Path=/usr/share/fonts/truetype/cmu/,UprightFont=cmuntt.ttf,BoldFont=cmuntb.ttf,ItalicFont=cmunit.ttf,BoldItalicFont=cmuntx.ttf]{cmuntt.ttf}\ttfamily texmf}{$\text{ }$}\setmainfont[Path=/usr/share/fonts/truetype/cmu/,UprightFont=cmunrm.ttf,BoldFont=cmunbx.ttf,ItalicFont=cmunti.ttf,BoldItalicFont=cmunbi.ttf]{cmunrm.ttf}\setmonofont[Path=/usr/share/fonts/truetype/cmu/,UprightFont=cmuntt.ttf,BoldFont=cmuntb.ttf,ItalicFont=cmunit.ttf,BoldItalicFont=cmuntx.ttf]{cmunrm.ttf} subdirectory in your own login directory. Your installation must be configured to look in these directories first, however, so that any updates to standard packages will be found there before the superseded copies in the main {\ttfamily \setmainfont[Path=/usr/share/fonts/truetype/cmu/,UprightFont=cmunrm.ttf,BoldFont=cmunbx.ttf,ItalicFont=cmunti.ttf,BoldItalicFont=cmunbi.ttf]{cmuntt.ttf}\setmonofont[Path=/usr/share/fonts/truetype/cmu/,UprightFont=cmuntt.ttf,BoldFont=cmuntb.ttf,ItalicFont=cmunit.ttf,BoldItalicFont=cmuntx.ttf]{cmuntt.ttf}\ttfamily texmf}{$\text{ }$}\setmainfont[Path=/usr/share/fonts/truetype/cmu/,UprightFont=cmunrm.ttf,BoldFont=cmunbx.ttf,ItalicFont=cmunti.ttf,BoldItalicFont=cmunbi.ttf]{cmunrm.ttf}\setmonofont[Path=/usr/share/fonts/truetype/cmu/,UprightFont=cmuntt.ttf,BoldFont=cmuntb.ttf,ItalicFont=cmunit.ttf,BoldItalicFont=cmuntx.ttf]{cmunrm.ttf} tree. All modern TeX installations should do this anyway, but if not, you can edit {\ttfamily \setmainfont[Path=/usr/share/fonts/truetype/cmu/,UprightFont=cmunrm.ttf,BoldFont=cmunbx.ttf,ItalicFont=cmunti.ttf,BoldItalicFont=cmunbi.ttf]{cmuntt.ttf}\setmonofont[Path=/usr/share/fonts/truetype/cmu/,UprightFont=cmuntt.ttf,BoldFont=cmuntb.ttf,ItalicFont=cmunit.ttf,BoldItalicFont=cmuntx.ttf]{cmuntt.ttf}\ttfamily texmf/web2c/texmf.cnf}{$\text{ }$}\setmainfont[Path=/usr/share/fonts/truetype/cmu/,UprightFont=cmunrm.ttf,BoldFont=cmunbx.ttf,ItalicFont=cmunti.ttf,BoldItalicFont=cmunbi.ttf]{cmunrm.ttf}\setmonofont[Path=/usr/share/fonts/truetype/cmu/,UprightFont=cmuntt.ttf,BoldFont=cmuntb.ttf,ItalicFont=cmunit.ttf,BoldItalicFont=cmuntx.ttf]{cmunrm.ttf} yourself.
\section{Checking package status}
\label{58}

The universal way to check if a file is available to TeX compilers is the command-{}line tool {\ttfamily \setmainfont[Path=/usr/share/fonts/truetype/cmu/,UprightFont=cmunrm.ttf,BoldFont=cmunbx.ttf,ItalicFont=cmunti.ttf,BoldItalicFont=cmunbi.ttf]{cmuntt.ttf}\setmonofont[Path=/usr/share/fonts/truetype/cmu/,UprightFont=cmuntt.ttf,BoldFont=cmuntb.ttf,ItalicFont=cmunit.ttf,BoldItalicFont=cmuntx.ttf]{cmuntt.ttf}\ttfamily kpsewhich}\setmainfont[Path=/usr/share/fonts/truetype/cmu/,UprightFont=cmunrm.ttf,BoldFont=cmunbx.ttf,ItalicFont=cmunti.ttf,BoldItalicFont=cmunbi.ttf]{cmunrm.ttf}\setmonofont[Path=/usr/share/fonts/truetype/cmu/,UprightFont=cmuntt.ttf,BoldFont=cmuntb.ttf,ItalicFont=cmunit.ttf,BoldItalicFont=cmuntx.ttf]{cmunrm.ttf}.\\

\TemplateSpaceIndent{$\text{ }${}\${}$\text{ }${}kpsewhich$\text{ }${}tikz$\text{ }$\newline{}
$\text{ }${}/usr/local/texlive/2012/texmf-{}dist/tex/plain/pgf/frontendlayer/tikz.tex}


{\ttfamily \setmainfont[Path=/usr/share/fonts/truetype/cmu/,UprightFont=cmunrm.ttf,BoldFont=cmunbx.ttf,ItalicFont=cmunti.ttf,BoldItalicFont=cmunbi.ttf]{cmuntt.ttf}\setmonofont[Path=/usr/share/fonts/truetype/cmu/,UprightFont=cmuntt.ttf,BoldFont=cmuntb.ttf,ItalicFont=cmunit.ttf,BoldItalicFont=cmuntx.ttf]{cmuntt.ttf}\ttfamily kpsewhich}{$\text{ }$}\setmainfont[Path=/usr/share/fonts/truetype/cmu/,UprightFont=cmunrm.ttf,BoldFont=cmunbx.ttf,ItalicFont=cmunti.ttf,BoldItalicFont=cmunbi.ttf]{cmunrm.ttf}\setmonofont[Path=/usr/share/fonts/truetype/cmu/,UprightFont=cmuntt.ttf,BoldFont=cmuntb.ttf,ItalicFont=cmunit.ttf,BoldItalicFont=cmuntx.ttf]{cmunrm.ttf} will actually search for files only, not for packages. It returns the path to the file.
For more details on a specific  package use the command-{}line tool {\ttfamily \setmainfont[Path=/usr/share/fonts/truetype/cmu/,UprightFont=cmunrm.ttf,BoldFont=cmunbx.ttf,ItalicFont=cmunti.ttf,BoldItalicFont=cmunbi.ttf]{cmuntt.ttf}\setmonofont[Path=/usr/share/fonts/truetype/cmu/,UprightFont=cmuntt.ttf,BoldFont=cmuntb.ttf,ItalicFont=cmunit.ttf,BoldItalicFont=cmuntx.ttf]{cmuntt.ttf}\ttfamily tlmgr}{$\text{ }$}\setmainfont[Path=/usr/share/fonts/truetype/cmu/,UprightFont=cmunrm.ttf,BoldFont=cmunbx.ttf,ItalicFont=cmunti.ttf,BoldItalicFont=cmunbi.ttf]{cmunrm.ttf}\setmonofont[Path=/usr/share/fonts/truetype/cmu/,UprightFont=cmuntt.ttf,BoldFont=cmuntb.ttf,ItalicFont=cmunit.ttf,BoldItalicFont=cmuntx.ttf]{cmunrm.ttf} (TeX Live only):
\\

\TemplateSpaceIndent{$\text{ }${}tlmgr$\text{ }${}info$\text{ }${}<{}package>{}}


The {\ttfamily \setmainfont[Path=/usr/share/fonts/truetype/cmu/,UprightFont=cmunrm.ttf,BoldFont=cmunbx.ttf,ItalicFont=cmunti.ttf,BoldItalicFont=cmunbi.ttf]{cmuntt.ttf}\setmonofont[Path=/usr/share/fonts/truetype/cmu/,UprightFont=cmuntt.ttf,BoldFont=cmuntb.ttf,ItalicFont=cmunit.ttf,BoldItalicFont=cmuntx.ttf]{cmuntt.ttf}\ttfamily tlmgr}{$\text{ }$}\setmainfont[Path=/usr/share/fonts/truetype/cmu/,UprightFont=cmunrm.ttf,BoldFont=cmunbx.ttf,ItalicFont=cmunti.ttf,BoldItalicFont=cmunbi.ttf]{cmunrm.ttf}\setmonofont[Path=/usr/share/fonts/truetype/cmu/,UprightFont=cmuntt.ttf,BoldFont=cmuntb.ttf,ItalicFont=cmunit.ttf,BoldItalicFont=cmuntx.ttf]{cmunrm.ttf} tool has lot more options. To consult the documentation:\\

\TemplateSpaceIndent{$\text{ }${}tlmgr$\text{ }${}help}

\section{Package documentation}
\label{59}

To find out what commands a package provides (and thus how to use it), you need to read the documentation. In the {\ttfamily \setmainfont[Path=/usr/share/fonts/truetype/cmu/,UprightFont=cmunrm.ttf,BoldFont=cmunbx.ttf,ItalicFont=cmunti.ttf,BoldItalicFont=cmunbi.ttf]{cmuntt.ttf}\setmonofont[Path=/usr/share/fonts/truetype/cmu/,UprightFont=cmuntt.ttf,BoldFont=cmuntb.ttf,ItalicFont=cmunit.ttf,BoldItalicFont=cmuntx.ttf]{cmuntt.ttf}\ttfamily texmf/doc}{$\text{ }$}\setmainfont[Path=/usr/share/fonts/truetype/cmu/,UprightFont=cmunrm.ttf,BoldFont=cmunbx.ttf,ItalicFont=cmunti.ttf,BoldItalicFont=cmunbi.ttf]{cmunrm.ttf}\setmonofont[Path=/usr/share/fonts/truetype/cmu/,UprightFont=cmuntt.ttf,BoldFont=cmuntb.ttf,ItalicFont=cmunit.ttf,BoldItalicFont=cmuntx.ttf]{cmunrm.ttf} subdirectory of your installation there should be directories full of .dvi files, one for every package installed. This location is distribution-{}specific, but is {\itshape \setmainfont[Path=/usr/share/fonts/truetype/cmu/,UprightFont=cmunrm.ttf,BoldFont=cmunbx.ttf,ItalicFont=cmunti.ttf,BoldItalicFont=cmunbi.ttf]{cmunti.ttf}\setmonofont[Path=/usr/share/fonts/truetype/cmu/,UprightFont=cmuntt.ttf,BoldFont=cmuntb.ttf,ItalicFont=cmunit.ttf,BoldItalicFont=cmuntx.ttf]{cmunti.ttf}\itshape typically}{$\text{ }$}\setmainfont[Path=/usr/share/fonts/truetype/cmu/,UprightFont=cmunrm.ttf,BoldFont=cmunbx.ttf,ItalicFont=cmunti.ttf,BoldItalicFont=cmunbi.ttf]{cmunrm.ttf}\setmonofont[Path=/usr/share/fonts/truetype/cmu/,UprightFont=cmuntt.ttf,BoldFont=cmuntb.ttf,ItalicFont=cmunit.ttf,BoldItalicFont=cmuntx.ttf]{cmunrm.ttf} found in:
\begin{longtable}{|>{\RaggedRight}p{0.20281\linewidth}|>{\RaggedRight}p{0.74005\linewidth}|} \hline 
{\bfseries \hspace*{0pt}\ignorespaces{}\hspace*{0pt}Distribution}&{\bfseries \hspace*{0pt}\ignorespaces{}\hspace*{0pt}Path}\endhead  \hline \hspace*{0pt}\ignorespaces{}\hspace*{0pt}MacTeX &\hspace*{0pt}\ignorespaces{}\hspace*{0pt} {\ttfamily \setmainfont[Path=/usr/share/fonts/truetype/cmu/,UprightFont=cmunrm.ttf,BoldFont=cmunbx.ttf,ItalicFont=cmunti.ttf,BoldItalicFont=cmunbi.ttf]{cmuntt.ttf}\setmonofont[Path=/usr/share/fonts/truetype/cmu/,UprightFont=cmuntt.ttf,BoldFont=cmuntb.ttf,ItalicFont=cmunit.ttf,BoldItalicFont=cmuntx.ttf]{cmuntt.ttf}\ttfamily /Library/TeX/Documentation/texmf-{}doc/latex}\\ \hline \hspace*{0pt}\ignorespaces{}\hspace*{0pt}{$\text{ }$}\setmainfont[Path=/usr/share/fonts/truetype/cmu/,UprightFont=cmunrm.ttf,BoldFont=cmunbx.ttf,ItalicFont=cmunti.ttf,BoldItalicFont=cmunbi.ttf]{cmunrm.ttf}\setmonofont[Path=/usr/share/fonts/truetype/cmu/,UprightFont=cmuntt.ttf,BoldFont=cmuntb.ttf,ItalicFont=cmunit.ttf,BoldItalicFont=cmuntx.ttf]{cmunrm.ttf} MiKTeX &\hspace*{0pt}\ignorespaces{}\hspace*{0pt} {\ttfamily \setmainfont[Path=/usr/share/fonts/truetype/cmu/,UprightFont=cmunrm.ttf,BoldFont=cmunbx.ttf,ItalicFont=cmunti.ttf,BoldItalicFont=cmunbi.ttf]{cmuntt.ttf}\setmonofont[Path=/usr/share/fonts/truetype/cmu/,UprightFont=cmuntt.ttf,BoldFont=cmuntb.ttf,ItalicFont=cmunit.ttf,BoldItalicFont=cmuntx.ttf]{cmuntt.ttf}\ttfamily \%MIKTEX\_DIR\%\textbackslash{}doc\textbackslash{}latex}\\ \hline \hspace*{0pt}\ignorespaces{}\hspace*{0pt}{$\text{ }$}\setmainfont[Path=/usr/share/fonts/truetype/cmu/,UprightFont=cmunrm.ttf,BoldFont=cmunbx.ttf,ItalicFont=cmunti.ttf,BoldItalicFont=cmunbi.ttf]{cmunrm.ttf}\setmonofont[Path=/usr/share/fonts/truetype/cmu/,UprightFont=cmuntt.ttf,BoldFont=cmuntb.ttf,ItalicFont=cmunit.ttf,BoldItalicFont=cmuntx.ttf]{cmunrm.ttf} TeX Live &\hspace*{0pt}\ignorespaces{}\hspace*{0pt} {\ttfamily \setmainfont[Path=/usr/share/fonts/truetype/cmu/,UprightFont=cmunrm.ttf,BoldFont=cmunbx.ttf,ItalicFont=cmunti.ttf,BoldItalicFont=cmunbi.ttf]{cmuntt.ttf}\setmonofont[Path=/usr/share/fonts/truetype/cmu/,UprightFont=cmuntt.ttf,BoldFont=cmuntb.ttf,ItalicFont=cmunit.ttf,BoldItalicFont=cmuntx.ttf]{cmuntt.ttf}\ttfamily \${}TEXMFDIST/doc/latex}\\ \hline 
\end{longtable}
\setmainfont[Path=/usr/share/fonts/truetype/cmu/,UprightFont=cmunrm.ttf,BoldFont=cmunbx.ttf,ItalicFont=cmunti.ttf,BoldItalicFont=cmunbi.ttf]{cmunrm.ttf}\setmonofont[Path=/usr/share/fonts/truetype/cmu/,UprightFont=cmuntt.ttf,BoldFont=cmuntb.ttf,ItalicFont=cmunit.ttf,BoldItalicFont=cmuntx.ttf]{cmunrm.ttf}
Generally, {\itshape \setmainfont[Path=/usr/share/fonts/truetype/cmu/,UprightFont=cmunrm.ttf,BoldFont=cmunbx.ttf,ItalicFont=cmunti.ttf,BoldItalicFont=cmunbi.ttf]{cmunti.ttf}\setmonofont[Path=/usr/share/fonts/truetype/cmu/,UprightFont=cmuntt.ttf,BoldFont=cmuntb.ttf,ItalicFont=cmunit.ttf,BoldItalicFont=cmuntx.ttf]{cmunti.ttf}\itshape most}{$\text{ }$}\setmainfont[Path=/usr/share/fonts/truetype/cmu/,UprightFont=cmunrm.ttf,BoldFont=cmunbx.ttf,ItalicFont=cmunti.ttf,BoldItalicFont=cmunbi.ttf]{cmunrm.ttf}\setmonofont[Path=/usr/share/fonts/truetype/cmu/,UprightFont=cmuntt.ttf,BoldFont=cmuntb.ttf,ItalicFont=cmunit.ttf,BoldItalicFont=cmuntx.ttf]{cmunrm.ttf} of the packages are in the {\ttfamily \setmainfont[Path=/usr/share/fonts/truetype/cmu/,UprightFont=cmunrm.ttf,BoldFont=cmunbx.ttf,ItalicFont=cmunti.ttf,BoldItalicFont=cmunbi.ttf]{cmuntt.ttf}\setmonofont[Path=/usr/share/fonts/truetype/cmu/,UprightFont=cmuntt.ttf,BoldFont=cmuntb.ttf,ItalicFont=cmunit.ttf,BoldItalicFont=cmuntx.ttf]{cmuntt.ttf}\ttfamily latex}{$\text{ }$}\setmainfont[Path=/usr/share/fonts/truetype/cmu/,UprightFont=cmunrm.ttf,BoldFont=cmunbx.ttf,ItalicFont=cmunti.ttf,BoldItalicFont=cmunbi.ttf]{cmunrm.ttf}\setmonofont[Path=/usr/share/fonts/truetype/cmu/,UprightFont=cmuntt.ttf,BoldFont=cmuntb.ttf,ItalicFont=cmunit.ttf,BoldItalicFont=cmuntx.ttf]{cmunrm.ttf} subdirectory, although other packages (such as BibTeX and font packages) are found in other subdirectories in {\ttfamily \setmainfont[Path=/usr/share/fonts/truetype/cmu/,UprightFont=cmunrm.ttf,BoldFont=cmunbx.ttf,ItalicFont=cmunti.ttf,BoldItalicFont=cmunbi.ttf]{cmuntt.ttf}\setmonofont[Path=/usr/share/fonts/truetype/cmu/,UprightFont=cmuntt.ttf,BoldFont=cmuntb.ttf,ItalicFont=cmunit.ttf,BoldItalicFont=cmuntx.ttf]{cmuntt.ttf}\ttfamily doc}\setmainfont[Path=/usr/share/fonts/truetype/cmu/,UprightFont=cmunrm.ttf,BoldFont=cmunbx.ttf,ItalicFont=cmunti.ttf,BoldItalicFont=cmunbi.ttf]{cmunrm.ttf}\setmonofont[Path=/usr/share/fonts/truetype/cmu/,UprightFont=cmuntt.ttf,BoldFont=cmuntb.ttf,ItalicFont=cmunit.ttf,BoldItalicFont=cmuntx.ttf]{cmunrm.ttf}. The documentation directories have the same name of the package (e.g. {\ttfamily \setmainfont[Path=/usr/share/fonts/truetype/cmu/,UprightFont=cmunrm.ttf,BoldFont=cmunbx.ttf,ItalicFont=cmunti.ttf,BoldItalicFont=cmunbi.ttf]{cmuntt.ttf}\setmonofont[Path=/usr/share/fonts/truetype/cmu/,UprightFont=cmuntt.ttf,BoldFont=cmuntb.ttf,ItalicFont=cmunit.ttf,BoldItalicFont=cmuntx.ttf]{cmuntt.ttf}\ttfamily amsmath}\setmainfont[Path=/usr/share/fonts/truetype/cmu/,UprightFont=cmunrm.ttf,BoldFont=cmunbx.ttf,ItalicFont=cmunti.ttf,BoldItalicFont=cmunbi.ttf]{cmunrm.ttf}\setmonofont[Path=/usr/share/fonts/truetype/cmu/,UprightFont=cmuntt.ttf,BoldFont=cmuntb.ttf,ItalicFont=cmunit.ttf,BoldItalicFont=cmuntx.ttf]{cmunrm.ttf}), which generally have one or more relevant documents in a variety of formats ({\ttfamily \setmainfont[Path=/usr/share/fonts/truetype/cmu/,UprightFont=cmunrm.ttf,BoldFont=cmunbx.ttf,ItalicFont=cmunti.ttf,BoldItalicFont=cmunbi.ttf]{cmuntt.ttf}\setmonofont[Path=/usr/share/fonts/truetype/cmu/,UprightFont=cmuntt.ttf,BoldFont=cmuntb.ttf,ItalicFont=cmunit.ttf,BoldItalicFont=cmuntx.ttf]{cmuntt.ttf}\ttfamily dvi}\setmainfont[Path=/usr/share/fonts/truetype/cmu/,UprightFont=cmunrm.ttf,BoldFont=cmunbx.ttf,ItalicFont=cmunti.ttf,BoldItalicFont=cmunbi.ttf]{cmunrm.ttf}\setmonofont[Path=/usr/share/fonts/truetype/cmu/,UprightFont=cmuntt.ttf,BoldFont=cmuntb.ttf,ItalicFont=cmunit.ttf,BoldItalicFont=cmuntx.ttf]{cmunrm.ttf}, {\ttfamily \setmainfont[Path=/usr/share/fonts/truetype/cmu/,UprightFont=cmunrm.ttf,BoldFont=cmunbx.ttf,ItalicFont=cmunti.ttf,BoldItalicFont=cmunbi.ttf]{cmuntt.ttf}\setmonofont[Path=/usr/share/fonts/truetype/cmu/,UprightFont=cmuntt.ttf,BoldFont=cmuntb.ttf,ItalicFont=cmunit.ttf,BoldItalicFont=cmuntx.ttf]{cmuntt.ttf}\ttfamily txt}\setmainfont[Path=/usr/share/fonts/truetype/cmu/,UprightFont=cmunrm.ttf,BoldFont=cmunbx.ttf,ItalicFont=cmunti.ttf,BoldItalicFont=cmunbi.ttf]{cmunrm.ttf}\setmonofont[Path=/usr/share/fonts/truetype/cmu/,UprightFont=cmuntt.ttf,BoldFont=cmuntb.ttf,ItalicFont=cmunit.ttf,BoldItalicFont=cmuntx.ttf]{cmunrm.ttf}, {\ttfamily \setmainfont[Path=/usr/share/fonts/truetype/cmu/,UprightFont=cmunrm.ttf,BoldFont=cmunbx.ttf,ItalicFont=cmunti.ttf,BoldItalicFont=cmunbi.ttf]{cmuntt.ttf}\setmonofont[Path=/usr/share/fonts/truetype/cmu/,UprightFont=cmuntt.ttf,BoldFont=cmuntb.ttf,ItalicFont=cmunit.ttf,BoldItalicFont=cmuntx.ttf]{cmuntt.ttf}\ttfamily pdf}\setmainfont[Path=/usr/share/fonts/truetype/cmu/,UprightFont=cmunrm.ttf,BoldFont=cmunbx.ttf,ItalicFont=cmunti.ttf,BoldItalicFont=cmunbi.ttf]{cmunrm.ttf}\setmonofont[Path=/usr/share/fonts/truetype/cmu/,UprightFont=cmuntt.ttf,BoldFont=cmuntb.ttf,ItalicFont=cmunit.ttf,BoldItalicFont=cmuntx.ttf]{cmunrm.ttf}, etc.). The documents generally have the same name as the package, but there are exceptions (for example, the documentation for {\ttfamily \setmainfont[Path=/usr/share/fonts/truetype/cmu/,UprightFont=cmunrm.ttf,BoldFont=cmunbx.ttf,ItalicFont=cmunti.ttf,BoldItalicFont=cmunbi.ttf]{cmuntt.ttf}\setmonofont[Path=/usr/share/fonts/truetype/cmu/,UprightFont=cmuntt.ttf,BoldFont=cmuntb.ttf,ItalicFont=cmunit.ttf,BoldItalicFont=cmuntx.ttf]{cmuntt.ttf}\ttfamily amsmath}{$\text{ }$}\setmainfont[Path=/usr/share/fonts/truetype/cmu/,UprightFont=cmunrm.ttf,BoldFont=cmunbx.ttf,ItalicFont=cmunti.ttf,BoldItalicFont=cmunbi.ttf]{cmunrm.ttf}\setmonofont[Path=/usr/share/fonts/truetype/cmu/,UprightFont=cmuntt.ttf,BoldFont=cmuntb.ttf,ItalicFont=cmunit.ttf,BoldItalicFont=cmuntx.ttf]{cmunrm.ttf} is found at {\ttfamily \setmainfont[Path=/usr/share/fonts/truetype/cmu/,UprightFont=cmunrm.ttf,BoldFont=cmunbx.ttf,ItalicFont=cmunti.ttf,BoldItalicFont=cmunbi.ttf]{cmuntt.ttf}\setmonofont[Path=/usr/share/fonts/truetype/cmu/,UprightFont=cmuntt.ttf,BoldFont=cmuntb.ttf,ItalicFont=cmunit.ttf,BoldItalicFont=cmuntx.ttf]{cmuntt.ttf}\ttfamily latex/amsmath/amsdoc.dvi}\setmainfont[Path=/usr/share/fonts/truetype/cmu/,UprightFont=cmunrm.ttf,BoldFont=cmunbx.ttf,ItalicFont=cmunti.ttf,BoldItalicFont=cmunbi.ttf]{cmunrm.ttf}\setmonofont[Path=/usr/share/fonts/truetype/cmu/,UprightFont=cmuntt.ttf,BoldFont=cmuntb.ttf,ItalicFont=cmunit.ttf,BoldItalicFont=cmuntx.ttf]{cmunrm.ttf}). If your installation procedure has not installed the documentation, the DVI files can all be downloaded from CTAN. Before using a package, you should read the documentation carefully, especially the subsection usually called \symbol{34}User Interface\symbol{34}, which describes the commands the package makes available. You cannot just guess and hope it will work: you have to read it and find out.

You can usually automatically open any installed package documentation with the {\itshape \setmainfont[Path=/usr/share/fonts/truetype/cmu/,UprightFont=cmunrm.ttf,BoldFont=cmunbx.ttf,ItalicFont=cmunti.ttf,BoldItalicFont=cmunbi.ttf]{cmunti.ttf}\setmonofont[Path=/usr/share/fonts/truetype/cmu/,UprightFont=cmuntt.ttf,BoldFont=cmuntb.ttf,ItalicFont=cmunit.ttf,BoldItalicFont=cmuntx.ttf]{cmunti.ttf}\itshape texdoc}{$\text{ }$}\setmainfont[Path=/usr/share/fonts/truetype/cmu/,UprightFont=cmunrm.ttf,BoldFont=cmunbx.ttf,ItalicFont=cmunti.ttf,BoldItalicFont=cmunbi.ttf]{cmunrm.ttf}\setmonofont[Path=/usr/share/fonts/truetype/cmu/,UprightFont=cmuntt.ttf,BoldFont=cmuntb.ttf,ItalicFont=cmunit.ttf,BoldItalicFont=cmuntx.ttf]{cmunrm.ttf} command:

\begin{Shaded}
\begin{Highlighting}[]

\KeywordTok{texdoc}\ensuremath{\text{ }}\KeywordTok{<}\NormalTok{package-name}\KeywordTok{>}\newline
\end{Highlighting}
\end{Shaded}

\section{External resources}
\label{60}
The best way to look for LaTeX packages is the already mentioned \myhref{http://tug.ctan.org/search.html}{CTAN: Search}.
Additional resources form \myhref{http://www.ctan.org/tex-archive/help/Catalogue/catalogue.html}{The TeX Catalogue Online}:
\begin{myitemize}
\item{}  \myhref{http://www.ctan.org/tex-archive/help/Catalogue/alpha.html}{Alphabetic catalogue}
\item{}  \myhref{http://www.ctan.org/tex-archive/help/Catalogue/brief.html}{With brief descriptions}
\item{}  \myhref{http://www.ctan.org/tex-archive/help/Catalogue/bytopic.html}{Topical catalogue} with packages sorted systematically
\item{}  \myhref{http://www.ctan.org/tex-archive/help/Catalogue/hier.html}{Hierarchical} mirroring the CTAN folder hierarchy
\end{myitemize}

\section{See Also}
\label{61}
\begin{myitemize}
\item{}  \mylref{991}{LaTeX/Package Reference}
\end{myitemize}




\myhref{https://sr.wikibooks.org/wiki/LaTeX\%2F\%D0\%98\%D0\%BD\%D1\%81\%D1\%82\%D0\%B0\%D0\%BB\%D0\%B0\%D1\%86\%D0\%B8\%D1\%98\%D0\%B0\%20\%D0\%B4\%D0\%BE\%D0\%B4\%D0\%B0\%D1\%82\%D0\%BD\%D0\%B8\%D1\%85\%20\%D0\%BF\%D0\%B0\%D0\%BA\%D0\%B5\%D1\%82\%D0\%B0}{sr:LaTeX/Инсталација додатних пакета}\chapter{Basics}

\myminitoc
\label{62}

\label{63}


This tutorial is aimed at getting familiar with the bare bones of \mylref{1}{LaTeX}.

Before starting, ensure you have LaTeX installed on your computer (see \mylref{10}{Installation} for instructions of what you will need).
\begin{myitemize}
\item{}  We will first have a look at the LaTeX syntax.
\item{}  We will create our first LaTeX document.
\item{}  Then we will take you through how to feed this file through the LaTeX system to produce quality output, such as postscript or PDF.
\item{}  Finally we will have a look at the file names and types.
\end{myitemize}

\section{The LaTeX syntax}
\label{64}

LaTeX uses a markup language in order to describe document structure and presentation. LaTeX converts your source text, combined with the markup, into a high quality document. For the purpose of analogy, web pages work in a similar way: the HTML is used to describe the document, but it is your browser that presents it in its full glory -{} with different colours, fonts, sizes, etc.

The input for LaTeX is a \myhref{https://en.wikipedia.org/wiki/plain\%20text}{plain text} file. You can create it with any text editor. It contains the text of the document, as well as the commands that tell LaTeX how to typeset the text.

A minimal example looks something like the following (the commands will be explained later):

\begin{Shaded}
\begin{Highlighting}[]

\NormalTok{\textbackslash{}documentclass\{article\}}
 
\NormalTok{\textbackslash{}begin\{document\}}
\NormalTok{Hello world!}
\NormalTok{\textbackslash{}end\{document\}}
\end{Highlighting}
\end{Shaded}

\subsection{Spaces}
\label{65}

The LaTeX compiler normalises whitespace so that whitespace characters, such as {$\text{[}$}space{$\text{]}$} or {$\text{[}$}tab{$\text{]}$}, are treated uniformly as \symbol{34}space\symbol{34}: several consecutive \symbol{34}spaces\symbol{34} are treated as one, \symbol{34}space\symbol{34} opening a line is generally ignored, and a single line break also yields “space”. A double line break (an empty line), however, defines the end of a paragraph; multiple empty lines are also treated as the end of a paragraph. An example of applying these rules is presented below: the left-{}hand side shows the user\textquotesingle{}s input (.tex), while the right-{}hand side depicts the rendered output (.dvi/.pdf/.ps).

\begin{longtable}{p{1.0\linewidth}}
\begin{Shaded}
\begin{Highlighting}[]

\NormalTok{It does not matter whether you}
\NormalTok{enter one or several             spaces}
\NormalTok{after a word.}
 
\NormalTok{An empty line starts a new}
\NormalTok{paragraph.}
\end{Highlighting}
\end{Shaded}
\\
It does not matter whether you enter one or several spaces after a word.

An empty line starts a new paragraph.

\end{longtable}
\subsection{Reserved Characters}
\label{66}
The following symbols are reserved characters that either have a special meaning under LaTeX or are unavailable in all the fonts. If you enter them directly in your text, they will normally not print but rather make LaTeX do things you did not intend.
\\

\TemplateSpaceIndent{$\text{ }${}\#$\text{ }${}\${}$\text{ }${}\%$\text{ }${}\^{}$\text{ }${}\&$\text{ }${}\_$\text{ }${}\{$\text{ }${}\}$\text{ }${}\~{}$\text{ }${}\textbackslash{}}


As you will see, these characters can be used in your documents all the same by adding a prefix backslash:
\begin{Shaded}
\begin{Highlighting}[]

\NormalTok{\textbackslash{}# \textbackslash{}$ \textbackslash{}% \textbackslash{}^\{\} \textbackslash{}& \textbackslash{}_ \textbackslash{}\{ \textbackslash{}\} \textbackslash{}~\{\} \textbackslash{}textbackslash\{\}}
\end{Highlighting}
\end{Shaded}


The backslash character \LaTeXTT{\textbackslash{}} {\itshape \setmainfont[Path=/usr/share/fonts/truetype/cmu/,UprightFont=cmunrm.ttf,BoldFont=cmunbx.ttf,ItalicFont=cmunti.ttf,BoldItalicFont=cmunbi.ttf]{cmunti.ttf}\setmonofont[Path=/usr/share/fonts/truetype/cmu/,UprightFont=cmuntt.ttf,BoldFont=cmuntb.ttf,ItalicFont=cmunit.ttf,BoldItalicFont=cmuntx.ttf]{cmunti.ttf}\itshape cannot}{$\text{ }$}\setmainfont[Path=/usr/share/fonts/truetype/cmu/,UprightFont=cmunrm.ttf,BoldFont=cmunbx.ttf,ItalicFont=cmunti.ttf,BoldItalicFont=cmunbi.ttf]{cmunrm.ttf}\setmonofont[Path=/usr/share/fonts/truetype/cmu/,UprightFont=cmuntt.ttf,BoldFont=cmuntb.ttf,ItalicFont=cmunit.ttf,BoldItalicFont=cmuntx.ttf]{cmunrm.ttf} be entered by adding another backslash in front of it (\LaTeXTT{\textbackslash{}\textbackslash{}}); this sequence is used for line breaking. For introducing a backslash in math mode, you can use \LaTeXTT{\textbackslash{}backslash} instead.

The commands \LaTeXTT{\textbackslash{}\~{}} and \LaTeXTT{\textbackslash{}\^{}} produce respectively a tilde and  a hat which is placed over the next letter. For example \LaTeXTT{\textbackslash{}\~{}n} gives ñ. That\textquotesingle{}s why you need braces to specify there is no letter as argument. You can also use \LaTeXTT{\textbackslash{}textasciitilde} and \LaTeXTT{\textbackslash{}textasciicircum} to enter these characters; or \myhref{http://tex.stackexchange.com/questions/9363/how-does-one-insert-a-backslash-or-a-tilde-into-latex}{other commands }.  

If you want to insert text that might contain several particular symbols (such as URIs), you can consider using the \LaTeXTT{\textbackslash{}verb} command, which will be discussed later in the section on \myhref{https://en.wikibooks.org/wiki/LaTeX\%2FFormatting}{formatting}. For source code, see \mylref{593}{Source Code Listings}

The \textquotesingle{}less than\textquotesingle{} (<{}) and \textquotesingle{}greater than\textquotesingle{} (>{}) characters are the only visible ASCII characters (not reserved) that will not print correctly. See \mylref{198}{Special Characters} for an explanation and a workaround.

Non-{}ASCII characters ({\itshape \setmainfont[Path=/usr/share/fonts/truetype/cmu/,UprightFont=cmunrm.ttf,BoldFont=cmunbx.ttf,ItalicFont=cmunti.ttf,BoldItalicFont=cmunbi.ttf]{cmunti.ttf}\setmonofont[Path=/usr/share/fonts/truetype/cmu/,UprightFont=cmuntt.ttf,BoldFont=cmuntb.ttf,ItalicFont=cmunit.ttf,BoldItalicFont=cmuntx.ttf]{cmunti.ttf}\itshape e.g.}{$\text{ }$}\setmainfont[Path=/usr/share/fonts/truetype/cmu/,UprightFont=cmunrm.ttf,BoldFont=cmunbx.ttf,ItalicFont=cmunti.ttf,BoldItalicFont=cmunbi.ttf]{cmunrm.ttf}\setmonofont[Path=/usr/share/fonts/truetype/cmu/,UprightFont=cmuntt.ttf,BoldFont=cmuntb.ttf,ItalicFont=cmunit.ttf,BoldItalicFont=cmuntx.ttf]{cmunrm.ttf} accents, diacritics) can be typed in directly for most cases. However you must {\itshape \setmainfont[Path=/usr/share/fonts/truetype/cmu/,UprightFont=cmunrm.ttf,BoldFont=cmunbx.ttf,ItalicFont=cmunti.ttf,BoldItalicFont=cmunbi.ttf]{cmunti.ttf}\setmonofont[Path=/usr/share/fonts/truetype/cmu/,UprightFont=cmuntt.ttf,BoldFont=cmuntb.ttf,ItalicFont=cmunit.ttf,BoldItalicFont=cmuntx.ttf]{cmunti.ttf}\itshape configure}{$\text{ }$}\setmainfont[Path=/usr/share/fonts/truetype/cmu/,UprightFont=cmunrm.ttf,BoldFont=cmunbx.ttf,ItalicFont=cmunti.ttf,BoldItalicFont=cmunbi.ttf]{cmunrm.ttf}\setmonofont[Path=/usr/share/fonts/truetype/cmu/,UprightFont=cmuntt.ttf,BoldFont=cmuntb.ttf,ItalicFont=cmunit.ttf,BoldItalicFont=cmuntx.ttf]{cmunrm.ttf} the document appropriately. The other symbols and many more can be printed with special commands as in mathematical formulae or as accents. We will tackle this issue in \mylref{192}{Special Characters}.
\subsection{LaTeX groups}
\label{67}
Sometimes a certain state shall be kept local, {\itshape \setmainfont[Path=/usr/share/fonts/truetype/cmu/,UprightFont=cmunrm.ttf,BoldFont=cmunbx.ttf,ItalicFont=cmunti.ttf,BoldItalicFont=cmunbi.ttf]{cmunti.ttf}\setmonofont[Path=/usr/share/fonts/truetype/cmu/,UprightFont=cmuntt.ttf,BoldFont=cmuntb.ttf,ItalicFont=cmunit.ttf,BoldItalicFont=cmuntx.ttf]{cmunti.ttf}\itshape i.e.}{$\text{ }$}\setmainfont[Path=/usr/share/fonts/truetype/cmu/,UprightFont=cmunrm.ttf,BoldFont=cmunbx.ttf,ItalicFont=cmunti.ttf,BoldItalicFont=cmunbi.ttf]{cmunrm.ttf}\setmonofont[Path=/usr/share/fonts/truetype/cmu/,UprightFont=cmuntt.ttf,BoldFont=cmuntb.ttf,ItalicFont=cmunit.ttf,BoldItalicFont=cmuntx.ttf]{cmunrm.ttf} limiting its
scope. This can be done by enclosing the part to be changed locally in curly 
braces. In certain occasions, using braces won\textquotesingle{}t be possible. LaTeX provides
\LaTeXTT{\textbackslash{}bgroup} and \LaTeXTT{\textbackslash{}egroup} to begin and
end a group, respectively.

\begin{Shaded}
\begin{Highlighting}[]

\NormalTok{\textbackslash{}documentclass\{article\}}
\NormalTok{\textbackslash{}begin\{document\}}
\NormalTok{normal text \{\textbackslash{}itshape walzing \textbackslash{}bfseries Wombat\} more normal text}
 
\NormalTok{normal text \textbackslash{}bgroup\textbackslash{}itshape walzing \textbackslash{}bfseries Wombat\textbackslash{}egroup\{\} more normal text}
\NormalTok{\textbackslash{}end\{document\}}
\end{Highlighting}
\end{Shaded}


Environments form an implicit group.
\subsection{LaTeX environments}
\label{68}

{\itshape \setmainfont[Path=/usr/share/fonts/truetype/cmu/,UprightFont=cmunrm.ttf,BoldFont=cmunbx.ttf,ItalicFont=cmunti.ttf,BoldItalicFont=cmunbi.ttf]{cmunti.ttf}\setmonofont[Path=/usr/share/fonts/truetype/cmu/,UprightFont=cmuntt.ttf,BoldFont=cmuntb.ttf,ItalicFont=cmunit.ttf,BoldItalicFont=cmuntx.ttf]{cmunti.ttf}\itshape Environments}{$\text{ }$}\setmainfont[Path=/usr/share/fonts/truetype/cmu/,UprightFont=cmunrm.ttf,BoldFont=cmunbx.ttf,ItalicFont=cmunti.ttf,BoldItalicFont=cmunbi.ttf]{cmunrm.ttf}\setmonofont[Path=/usr/share/fonts/truetype/cmu/,UprightFont=cmuntt.ttf,BoldFont=cmuntb.ttf,ItalicFont=cmunit.ttf,BoldItalicFont=cmuntx.ttf]{cmunrm.ttf} in LaTeX have a role that is quite similar to commands, but they usually have effect on a wider part of the document. Their syntax is:
\begin{Shaded}
\begin{Highlighting}[]

\NormalTok{\textbackslash{}begin\{environmentname\}}
\NormalTok{text to be influenced}
\NormalTok{\textbackslash{}end\{environmentname\}}
\end{Highlighting}
\end{Shaded}


Between the \LaTeXTT{\textbackslash{}begin} and the \LaTeXTT{\textbackslash{}end} you can put other commands and nested environments. The internal mechanism of environments defines a group, which makes its usage safe (no influence on the other parts of the document).
In general, environments can accept arguments as well, but this feature is not commonly used and so it will be discussed in more advanced parts of the document. 

Anything in LaTeX can be expressed in terms of commands and environments.
\subsection{LaTeX commands}
\label{69}
LaTeX commands are case sensitive, and take one of the following two formats:
\begin{myitemize}
\item{}  They start with a backslash \LaTeXTT{\textbackslash{}} and then have a name consisting of letters only. Command names are terminated by a space, a number or any other \symbol{34}non-{}letter\symbol{34}.
\item{}  They consist of a backslash \LaTeXTT{\textbackslash{}} and exactly one non-{}letter.
\end{myitemize}


Some commands need an argument, which has to be given between curly braces \LaTeXTT{\{ \}} after the command name. Some commands support optional parameters, which are added after the command name in square brackets \LaTeXTT{{$\text{[}$} {$\text{]}$}}. The general syntax is:
\begin{Shaded}
\begin{Highlighting}[]

\NormalTok{\textbackslash{}commandname[option1,option2,...]\{argument1\}\{argument2\}...}
\end{Highlighting}
\end{Shaded}


Most standard LaTeX commands have a {\itshape \setmainfont[Path=/usr/share/fonts/truetype/cmu/,UprightFont=cmunrm.ttf,BoldFont=cmunbx.ttf,ItalicFont=cmunti.ttf,BoldItalicFont=cmunbi.ttf]{cmunti.ttf}\setmonofont[Path=/usr/share/fonts/truetype/cmu/,UprightFont=cmuntt.ttf,BoldFont=cmuntb.ttf,ItalicFont=cmunit.ttf,BoldItalicFont=cmuntx.ttf]{cmunti.ttf}\itshape switch}{$\text{ }$}\setmainfont[Path=/usr/share/fonts/truetype/cmu/,UprightFont=cmunrm.ttf,BoldFont=cmunbx.ttf,ItalicFont=cmunti.ttf,BoldItalicFont=cmunbi.ttf]{cmunrm.ttf}\setmonofont[Path=/usr/share/fonts/truetype/cmu/,UprightFont=cmuntt.ttf,BoldFont=cmuntb.ttf,ItalicFont=cmunit.ttf,BoldItalicFont=cmuntx.ttf]{cmunrm.ttf} equivalent. Switches have no arguments but apply on the rest of the scope, {\itshape \setmainfont[Path=/usr/share/fonts/truetype/cmu/,UprightFont=cmunrm.ttf,BoldFont=cmunbx.ttf,ItalicFont=cmunti.ttf,BoldItalicFont=cmunbi.ttf]{cmunti.ttf}\setmonofont[Path=/usr/share/fonts/truetype/cmu/,UprightFont=cmuntt.ttf,BoldFont=cmuntb.ttf,ItalicFont=cmunit.ttf,BoldItalicFont=cmuntx.ttf]{cmunti.ttf}\itshape i.e.}{$\text{ }$}\setmainfont[Path=/usr/share/fonts/truetype/cmu/,UprightFont=cmunrm.ttf,BoldFont=cmunbx.ttf,ItalicFont=cmunti.ttf,BoldItalicFont=cmunbi.ttf]{cmunrm.ttf}\setmonofont[Path=/usr/share/fonts/truetype/cmu/,UprightFont=cmuntt.ttf,BoldFont=cmuntb.ttf,ItalicFont=cmunit.ttf,BoldItalicFont=cmuntx.ttf]{cmunrm.ttf} the current group or environment. A switch should (almost) never be called outside of any scope, otherwise it will apply on the rest of the document.

\begin{TemplateInfo}{\danger}{Warning}Commands with arguments and switches should not be confused. This is a very common error!\end{TemplateInfo}

Example:
\begin{Shaded}
\begin{Highlighting}[]

\CommentTok{% \textbackslash{}emph is a command with argument, \textbackslash{}em is a switch.}
\NormalTok{\textbackslash{}emph\{emphasized text\}, this part is normal }\CommentTok{% Correct}
\NormalTok{\{\textbackslash{}em emphasized text\}, this part is normal }\CommentTok{% Correct}
 
\NormalTok{\textbackslash{}em emphasized text, this part is normal }\CommentTok{% Incorrect}
\NormalTok{\textbackslash{}em\{emphasized text\}, this part is normal }\CommentTok character while processing an input file, it ignores the rest of the current line, the line break, and all whitespace at the beginning of the next line.

This can be used to write notes into the input file, which will not show up in the printed version.
\begin{longtable}{p{1.0\linewidth}}
\begin{Shaded}
\begin{Highlighting}[]
\NormalTok{This is an }\CommentTok{% stupid}
\CommentTok
            \NormalTok{ifragilist}\CommentTok character can be used to split long input lines that do not allow whitespace or line breaks, as with Supercalifragilisticexpialidocious above.

The core LaTeX language does not have a predefined syntax for commenting out regions spanning multiple lines. Refer to \mylref{141}{multiline comments} for simple workarounds.
\section{Our first document}
\label{71}

Now we can create our first document. We will produce the absolute bare minimum that is needed in order to get some output; the well known {\bfseries \setmainfont[Path=/usr/share/fonts/truetype/cmu/,UprightFont=cmunrm.ttf,BoldFont=cmunbx.ttf,ItalicFont=cmunti.ttf,BoldItalicFont=cmunbi.ttf]{cmunbx.ttf}\setmonofont[Path=/usr/share/fonts/truetype/cmu/,UprightFont=cmuntt.ttf,BoldFont=cmuntb.ttf,ItalicFont=cmunit.ttf,BoldItalicFont=cmuntx.ttf]{cmunbx.ttf}\bfseries Hello World!}{$\text{ }$}\setmainfont[Path=/usr/share/fonts/truetype/cmu/,UprightFont=cmunrm.ttf,BoldFont=cmunbx.ttf,ItalicFont=cmunti.ttf,BoldItalicFont=cmunbi.ttf]{cmunrm.ttf}\setmonofont[Path=/usr/share/fonts/truetype/cmu/,UprightFont=cmuntt.ttf,BoldFont=cmuntb.ttf,ItalicFont=cmunit.ttf,BoldItalicFont=cmuntx.ttf]{cmunrm.ttf} approach will be suitable here.

\begin{myitemize}
\item{}  Open your favorite text-{}editor. \myhref{https://en.wikibooks.org/wiki/Learning\%20the\%20vi\%20Editor\%2FVim}{vim}, \myhref{https://en.wikibooks.org/wiki/emacs}{emacs}, Notepad++, and other text editors will have syntax highlighting that will help to write your files.
\item{}  Reproduce the following text in your editor. This is the LaTeX source.
\end{myitemize}

\begin{Shaded}
\begin{Highlighting}[]

\CommentTok{% hello.tex - Our first LaTeX example!}
\NormalTok{\textbackslash{}documentclass\{article\}}
\NormalTok{\textbackslash{}begin\{document\}}
\NormalTok{Hello World!}
\NormalTok{\textbackslash{}end\{document\}}
\end{Highlighting}
\end{Shaded}


\begin{myitemize}
\item{}  Save your file as {\ttfamily \setmainfont[Path=/usr/share/fonts/truetype/cmu/,UprightFont=cmunrm.ttf,BoldFont=cmunbx.ttf,ItalicFont=cmunti.ttf,BoldItalicFont=cmunbi.ttf]{cmuntt.ttf}\setmonofont[Path=/usr/share/fonts/truetype/cmu/,UprightFont=cmuntt.ttf,BoldFont=cmuntb.ttf,ItalicFont=cmunit.ttf,BoldItalicFont=cmuntx.ttf]{cmuntt.ttf}\ttfamily hello.tex}\setmainfont[Path=/usr/share/fonts/truetype/cmu/,UprightFont=cmunrm.ttf,BoldFont=cmunbx.ttf,ItalicFont=cmunti.ttf,BoldItalicFont=cmunbi.ttf]{cmunrm.ttf}\setmonofont[Path=/usr/share/fonts/truetype/cmu/,UprightFont=cmuntt.ttf,BoldFont=cmuntb.ttf,ItalicFont=cmunit.ttf,BoldItalicFont=cmuntx.ttf]{cmunrm.ttf}.
\end{myitemize}

When picking a name for your file, make sure it bears a {\ttfamily \setmainfont[Path=/usr/share/fonts/truetype/cmu/,UprightFont=cmunrm.ttf,BoldFont=cmunbx.ttf,ItalicFont=cmunti.ttf,BoldItalicFont=cmunbi.ttf]{cmuntt.ttf}\setmonofont[Path=/usr/share/fonts/truetype/cmu/,UprightFont=cmuntt.ttf,BoldFont=cmuntb.ttf,ItalicFont=cmunit.ttf,BoldItalicFont=cmuntx.ttf]{cmuntt.ttf}\ttfamily .tex}{$\text{ }$}\setmainfont[Path=/usr/share/fonts/truetype/cmu/,UprightFont=cmunrm.ttf,BoldFont=cmunbx.ttf,ItalicFont=cmunti.ttf,BoldItalicFont=cmunbi.ttf]{cmunrm.ttf}\setmonofont[Path=/usr/share/fonts/truetype/cmu/,UprightFont=cmuntt.ttf,BoldFont=cmuntb.ttf,ItalicFont=cmunit.ttf,BoldItalicFont=cmuntx.ttf]{cmunrm.ttf} extension.
\subsection{What does it all mean?}
\label{72}

{\scalefont{0.78046}\begin{longtable}{|>{\RaggedRight}p{0.46853\linewidth}|>{\RaggedRight}p{0.47433\linewidth}|} \hline 
\hspace*{0pt}\ignorespaces{}\hspace*{0pt} \LaTeXTT{\% hello.tex -{} Our first LaTeX example!}&\hspace*{0pt}\ignorespaces{}\hspace*{0pt} The first line is a {\itshape \setmainfont[Path=/usr/share/fonts/truetype/cmu/,UprightFont=cmunrm.ttf,BoldFont=cmunbx.ttf,ItalicFont=cmunti.ttf,BoldItalicFont=cmunbi.ttf]{cmunti.ttf}\setmonofont[Path=/usr/share/fonts/truetype/cmu/,UprightFont=cmuntt.ttf,BoldFont=cmuntb.ttf,ItalicFont=cmunit.ttf,BoldItalicFont=cmuntx.ttf]{cmunti.ttf}\itshape comment}\setmainfont[Path=/usr/share/fonts/truetype/cmu/,UprightFont=cmunrm.ttf,BoldFont=cmunbx.ttf,ItalicFont=cmunti.ttf,BoldItalicFont=cmunbi.ttf]{cmunrm.ttf}\setmonofont[Path=/usr/share/fonts/truetype/cmu/,UprightFont=cmuntt.ttf,BoldFont=cmuntb.ttf,ItalicFont=cmunit.ttf,BoldItalicFont=cmuntx.ttf]{cmunrm.ttf}. This is because it begins with the percent symbol (\%); when LaTeX sees this, it simply ignores the rest of the line. Comments are useful for people to annotate parts of the source file. For example, you could put information about the author and the date, or whatever you wish.\\ \hline \hspace*{0pt}\ignorespaces{}\hspace*{0pt} \LaTeXTT{\textbackslash{}documentclass\{article\}}&\hspace*{0pt}\ignorespaces{}\hspace*{0pt} This line is a command and tells LaTeX to use the \LaTeXTT{article} document class. A document class file defines the formatting, which in this case is a generic article format. The handy thing is that if you want to change the appearance of your document, substitute article for another class file that exists.\\ \hline \hspace*{0pt}\ignorespaces{}\hspace*{0pt} \LaTeXTT{\textbackslash{}begin\{document\}}&\hspace*{0pt}\ignorespaces{}\hspace*{0pt} This line is the beginning of the environment called \LaTeXTT{document}; it alerts LaTeX that content of the document is about to commence. Anything above this command is known generally to belong in the {\itshape \setmainfont[Path=/usr/share/fonts/truetype/cmu/,UprightFont=cmunrm.ttf,BoldFont=cmunbx.ttf,ItalicFont=cmunti.ttf,BoldItalicFont=cmunbi.ttf]{cmunti.ttf}\setmonofont[Path=/usr/share/fonts/truetype/cmu/,UprightFont=cmuntt.ttf,BoldFont=cmuntb.ttf,ItalicFont=cmunit.ttf,BoldItalicFont=cmuntx.ttf]{cmunti.ttf}\itshape preamble}\setmainfont[Path=/usr/share/fonts/truetype/cmu/,UprightFont=cmunrm.ttf,BoldFont=cmunbx.ttf,ItalicFont=cmunti.ttf,BoldItalicFont=cmunbi.ttf]{cmunrm.ttf}\setmonofont[Path=/usr/share/fonts/truetype/cmu/,UprightFont=cmuntt.ttf,BoldFont=cmuntb.ttf,ItalicFont=cmunit.ttf,BoldItalicFont=cmuntx.ttf]{cmunrm.ttf}.\\ \hline \hspace*{0pt}\ignorespaces{}\hspace*{0pt} \LaTeXTT{Hello World!}&\hspace*{0pt}\ignorespaces{}\hspace*{0pt} This was the only actual line containing real content -{} the text that we wanted displayed on the page.\\ \hline \hspace*{0pt}\ignorespaces{}\hspace*{0pt} \LaTeXTT{\textbackslash{}end\{document\}}&\hspace*{0pt}\ignorespaces{}\hspace*{0pt} The \LaTeXTT{document} environment ends here. It tells LaTeX that the document source is complete, anything after this line will be ignored.\\ \hline 
\end{longtable}
}

As we have said before, each of the LaTeX commands begins with a backslash (\LaTeXTT{\textbackslash{}}). This is LaTeX\textquotesingle{}s way of knowing that whenever it sees a backslash, to expect some commands. Comments are not classed as a command, since all they tell LaTeX is to ignore the line. Comments never affect the output of the document.
\section{Compilation}
\label{73}\subsection{Compilation process}
\label{74}

The general concept is to transform a plain text document into a publishable
format, mostly a DVI, PS or PDF file. This process is called {\itshape \setmainfont[Path=/usr/share/fonts/truetype/cmu/,UprightFont=cmunrm.ttf,BoldFont=cmunbx.ttf,ItalicFont=cmunti.ttf,BoldItalicFont=cmunbi.ttf]{cmunti.ttf}\setmonofont[Path=/usr/share/fonts/truetype/cmu/,UprightFont=cmuntt.ttf,BoldFont=cmuntb.ttf,ItalicFont=cmunit.ttf,BoldItalicFont=cmuntx.ttf]{cmunti.ttf}\itshape compilation}\setmainfont[Path=/usr/share/fonts/truetype/cmu/,UprightFont=cmunrm.ttf,BoldFont=cmunbx.ttf,ItalicFont=cmunti.ttf,BoldItalicFont=cmunbi.ttf]{cmunrm.ttf}\setmonofont[Path=/usr/share/fonts/truetype/cmu/,UprightFont=cmuntt.ttf,BoldFont=cmuntb.ttf,ItalicFont=cmunit.ttf,BoldItalicFont=cmuntx.ttf]{cmunrm.ttf}, which is done by an executable file called a {\itshape \setmainfont[Path=/usr/share/fonts/truetype/cmu/,UprightFont=cmunrm.ttf,BoldFont=cmunbx.ttf,ItalicFont=cmunti.ttf,BoldItalicFont=cmunbi.ttf]{cmunti.ttf}\setmonofont[Path=/usr/share/fonts/truetype/cmu/,UprightFont=cmuntt.ttf,BoldFont=cmuntb.ttf,ItalicFont=cmunit.ttf,BoldItalicFont=cmuntx.ttf]{cmunti.ttf}\itshape compiler}\setmainfont[Path=/usr/share/fonts/truetype/cmu/,UprightFont=cmunrm.ttf,BoldFont=cmunbx.ttf,ItalicFont=cmunti.ttf,BoldItalicFont=cmunbi.ttf]{cmunrm.ttf}\setmonofont[Path=/usr/share/fonts/truetype/cmu/,UprightFont=cmuntt.ttf,BoldFont=cmuntb.ttf,ItalicFont=cmunit.ttf,BoldItalicFont=cmuntx.ttf]{cmunrm.ttf}.

There are two main compilers.
\begin{myitemize}
\item{}  {\ttfamily \setmainfont[Path=/usr/share/fonts/truetype/cmu/,UprightFont=cmunrm.ttf,BoldFont=cmunbx.ttf,ItalicFont=cmunti.ttf,BoldItalicFont=cmunbi.ttf]{cmuntt.ttf}\setmonofont[Path=/usr/share/fonts/truetype/cmu/,UprightFont=cmuntt.ttf,BoldFont=cmuntb.ttf,ItalicFont=cmunit.ttf,BoldItalicFont=cmuntx.ttf]{cmuntt.ttf}\ttfamily tex}{$\text{ }$}\setmainfont[Path=/usr/share/fonts/truetype/cmu/,UprightFont=cmunrm.ttf,BoldFont=cmunbx.ttf,ItalicFont=cmunti.ttf,BoldItalicFont=cmunbi.ttf]{cmunrm.ttf}\setmonofont[Path=/usr/share/fonts/truetype/cmu/,UprightFont=cmuntt.ttf,BoldFont=cmuntb.ttf,ItalicFont=cmunit.ttf,BoldItalicFont=cmuntx.ttf]{cmunrm.ttf} compiler reads a TeX {\ttfamily \setmainfont[Path=/usr/share/fonts/truetype/cmu/,UprightFont=cmunrm.ttf,BoldFont=cmunbx.ttf,ItalicFont=cmunti.ttf,BoldItalicFont=cmunbi.ttf]{cmuntt.ttf}\setmonofont[Path=/usr/share/fonts/truetype/cmu/,UprightFont=cmuntt.ttf,BoldFont=cmuntb.ttf,ItalicFont=cmunit.ttf,BoldItalicFont=cmuntx.ttf]{cmuntt.ttf}\ttfamily .tex}{$\text{ }$}\setmainfont[Path=/usr/share/fonts/truetype/cmu/,UprightFont=cmunrm.ttf,BoldFont=cmunbx.ttf,ItalicFont=cmunti.ttf,BoldItalicFont=cmunbi.ttf]{cmunrm.ttf}\setmonofont[Path=/usr/share/fonts/truetype/cmu/,UprightFont=cmuntt.ttf,BoldFont=cmuntb.ttf,ItalicFont=cmunit.ttf,BoldItalicFont=cmuntx.ttf]{cmunrm.ttf} file and creates a {\ttfamily \setmainfont[Path=/usr/share/fonts/truetype/cmu/,UprightFont=cmunrm.ttf,BoldFont=cmunbx.ttf,ItalicFont=cmunti.ttf,BoldItalicFont=cmunbi.ttf]{cmuntt.ttf}\setmonofont[Path=/usr/share/fonts/truetype/cmu/,UprightFont=cmuntt.ttf,BoldFont=cmuntb.ttf,ItalicFont=cmunit.ttf,BoldItalicFont=cmuntx.ttf]{cmuntt.ttf}\ttfamily .dvi}\setmainfont[Path=/usr/share/fonts/truetype/cmu/,UprightFont=cmunrm.ttf,BoldFont=cmunbx.ttf,ItalicFont=cmunti.ttf,BoldItalicFont=cmunbi.ttf]{cmunrm.ttf}\setmonofont[Path=/usr/share/fonts/truetype/cmu/,UprightFont=cmuntt.ttf,BoldFont=cmuntb.ttf,ItalicFont=cmunit.ttf,BoldItalicFont=cmuntx.ttf]{cmunrm.ttf}.
\item{}  {\ttfamily \setmainfont[Path=/usr/share/fonts/truetype/cmu/,UprightFont=cmunrm.ttf,BoldFont=cmunbx.ttf,ItalicFont=cmunti.ttf,BoldItalicFont=cmunbi.ttf]{cmuntt.ttf}\setmonofont[Path=/usr/share/fonts/truetype/cmu/,UprightFont=cmuntt.ttf,BoldFont=cmuntb.ttf,ItalicFont=cmunit.ttf,BoldItalicFont=cmuntx.ttf]{cmuntt.ttf}\ttfamily pdftex}{$\text{ }$}\setmainfont[Path=/usr/share/fonts/truetype/cmu/,UprightFont=cmunrm.ttf,BoldFont=cmunbx.ttf,ItalicFont=cmunti.ttf,BoldItalicFont=cmunbi.ttf]{cmunrm.ttf}\setmonofont[Path=/usr/share/fonts/truetype/cmu/,UprightFont=cmuntt.ttf,BoldFont=cmuntb.ttf,ItalicFont=cmunit.ttf,BoldItalicFont=cmuntx.ttf]{cmunrm.ttf} compiler reads a TeX {\ttfamily \setmainfont[Path=/usr/share/fonts/truetype/cmu/,UprightFont=cmunrm.ttf,BoldFont=cmunbx.ttf,ItalicFont=cmunti.ttf,BoldItalicFont=cmunbi.ttf]{cmuntt.ttf}\setmonofont[Path=/usr/share/fonts/truetype/cmu/,UprightFont=cmuntt.ttf,BoldFont=cmuntb.ttf,ItalicFont=cmunit.ttf,BoldItalicFont=cmuntx.ttf]{cmuntt.ttf}\ttfamily .tex}{$\text{ }$}\setmainfont[Path=/usr/share/fonts/truetype/cmu/,UprightFont=cmunrm.ttf,BoldFont=cmunbx.ttf,ItalicFont=cmunti.ttf,BoldItalicFont=cmunbi.ttf]{cmunrm.ttf}\setmonofont[Path=/usr/share/fonts/truetype/cmu/,UprightFont=cmuntt.ttf,BoldFont=cmuntb.ttf,ItalicFont=cmunit.ttf,BoldItalicFont=cmuntx.ttf]{cmunrm.ttf} file and creates a {\ttfamily \setmainfont[Path=/usr/share/fonts/truetype/cmu/,UprightFont=cmunrm.ttf,BoldFont=cmunbx.ttf,ItalicFont=cmunti.ttf,BoldItalicFont=cmunbi.ttf]{cmuntt.ttf}\setmonofont[Path=/usr/share/fonts/truetype/cmu/,UprightFont=cmuntt.ttf,BoldFont=cmuntb.ttf,ItalicFont=cmunit.ttf,BoldItalicFont=cmuntx.ttf]{cmuntt.ttf}\ttfamily .pdf}\setmainfont[Path=/usr/share/fonts/truetype/cmu/,UprightFont=cmunrm.ttf,BoldFont=cmunbx.ttf,ItalicFont=cmunti.ttf,BoldItalicFont=cmunbi.ttf]{cmunrm.ttf}\setmonofont[Path=/usr/share/fonts/truetype/cmu/,UprightFont=cmuntt.ttf,BoldFont=cmuntb.ttf,ItalicFont=cmunit.ttf,BoldItalicFont=cmuntx.ttf]{cmunrm.ttf}.
\end{myitemize}


These compilers are basically used to compile Plain TeX, not LaTeX. There is no such LaTeX compiler since LaTeX is just a bunch of macros for TeX. However, there are two executables related to the previous compilers:
\begin{myitemize}
\item{}  {\ttfamily \setmainfont[Path=/usr/share/fonts/truetype/cmu/,UprightFont=cmunrm.ttf,BoldFont=cmunbx.ttf,ItalicFont=cmunti.ttf,BoldItalicFont=cmunbi.ttf]{cmuntt.ttf}\setmonofont[Path=/usr/share/fonts/truetype/cmu/,UprightFont=cmuntt.ttf,BoldFont=cmuntb.ttf,ItalicFont=cmunit.ttf,BoldItalicFont=cmuntx.ttf]{cmuntt.ttf}\ttfamily latex}{$\text{ }$}\setmainfont[Path=/usr/share/fonts/truetype/cmu/,UprightFont=cmunrm.ttf,BoldFont=cmunbx.ttf,ItalicFont=cmunti.ttf,BoldItalicFont=cmunbi.ttf]{cmunrm.ttf}\setmonofont[Path=/usr/share/fonts/truetype/cmu/,UprightFont=cmuntt.ttf,BoldFont=cmuntb.ttf,ItalicFont=cmunit.ttf,BoldItalicFont=cmuntx.ttf]{cmunrm.ttf} executable calls {\ttfamily \setmainfont[Path=/usr/share/fonts/truetype/cmu/,UprightFont=cmunrm.ttf,BoldFont=cmunbx.ttf,ItalicFont=cmunti.ttf,BoldItalicFont=cmunbi.ttf]{cmuntt.ttf}\setmonofont[Path=/usr/share/fonts/truetype/cmu/,UprightFont=cmuntt.ttf,BoldFont=cmuntb.ttf,ItalicFont=cmunit.ttf,BoldItalicFont=cmuntx.ttf]{cmuntt.ttf}\ttfamily tex}{$\text{ }$}\setmainfont[Path=/usr/share/fonts/truetype/cmu/,UprightFont=cmunrm.ttf,BoldFont=cmunbx.ttf,ItalicFont=cmunti.ttf,BoldItalicFont=cmunbi.ttf]{cmunrm.ttf}\setmonofont[Path=/usr/share/fonts/truetype/cmu/,UprightFont=cmuntt.ttf,BoldFont=cmuntb.ttf,ItalicFont=cmunit.ttf,BoldItalicFont=cmuntx.ttf]{cmunrm.ttf} with LaTeX initialization files, reads a LaTeX {\ttfamily \setmainfont[Path=/usr/share/fonts/truetype/cmu/,UprightFont=cmunrm.ttf,BoldFont=cmunbx.ttf,ItalicFont=cmunti.ttf,BoldItalicFont=cmunbi.ttf]{cmuntt.ttf}\setmonofont[Path=/usr/share/fonts/truetype/cmu/,UprightFont=cmuntt.ttf,BoldFont=cmuntb.ttf,ItalicFont=cmunit.ttf,BoldItalicFont=cmuntx.ttf]{cmuntt.ttf}\ttfamily .tex}{$\text{ }$}\setmainfont[Path=/usr/share/fonts/truetype/cmu/,UprightFont=cmunrm.ttf,BoldFont=cmunbx.ttf,ItalicFont=cmunti.ttf,BoldItalicFont=cmunbi.ttf]{cmunrm.ttf}\setmonofont[Path=/usr/share/fonts/truetype/cmu/,UprightFont=cmuntt.ttf,BoldFont=cmuntb.ttf,ItalicFont=cmunit.ttf,BoldItalicFont=cmuntx.ttf]{cmunrm.ttf} file and creates a {\ttfamily \setmainfont[Path=/usr/share/fonts/truetype/cmu/,UprightFont=cmunrm.ttf,BoldFont=cmunbx.ttf,ItalicFont=cmunti.ttf,BoldItalicFont=cmunbi.ttf]{cmuntt.ttf}\setmonofont[Path=/usr/share/fonts/truetype/cmu/,UprightFont=cmuntt.ttf,BoldFont=cmuntb.ttf,ItalicFont=cmunit.ttf,BoldItalicFont=cmuntx.ttf]{cmuntt.ttf}\ttfamily .dvi}
\item{} {$\text{ }$}\setmainfont[Path=/usr/share/fonts/truetype/cmu/,UprightFont=cmunrm.ttf,BoldFont=cmunbx.ttf,ItalicFont=cmunti.ttf,BoldItalicFont=cmunbi.ttf]{cmunrm.ttf}\setmonofont[Path=/usr/share/fonts/truetype/cmu/,UprightFont=cmuntt.ttf,BoldFont=cmuntb.ttf,ItalicFont=cmunit.ttf,BoldItalicFont=cmuntx.ttf]{cmunrm.ttf} {\ttfamily \setmainfont[Path=/usr/share/fonts/truetype/cmu/,UprightFont=cmunrm.ttf,BoldFont=cmunbx.ttf,ItalicFont=cmunti.ttf,BoldItalicFont=cmunbi.ttf]{cmuntt.ttf}\setmonofont[Path=/usr/share/fonts/truetype/cmu/,UprightFont=cmuntt.ttf,BoldFont=cmuntb.ttf,ItalicFont=cmunit.ttf,BoldItalicFont=cmuntx.ttf]{cmuntt.ttf}\ttfamily pdflatex}{$\text{ }$}\setmainfont[Path=/usr/share/fonts/truetype/cmu/,UprightFont=cmunrm.ttf,BoldFont=cmunbx.ttf,ItalicFont=cmunti.ttf,BoldItalicFont=cmunbi.ttf]{cmunrm.ttf}\setmonofont[Path=/usr/share/fonts/truetype/cmu/,UprightFont=cmuntt.ttf,BoldFont=cmuntb.ttf,ItalicFont=cmunit.ttf,BoldItalicFont=cmuntx.ttf]{cmunrm.ttf} executable calls {\ttfamily \setmainfont[Path=/usr/share/fonts/truetype/cmu/,UprightFont=cmunrm.ttf,BoldFont=cmunbx.ttf,ItalicFont=cmunti.ttf,BoldItalicFont=cmunbi.ttf]{cmuntt.ttf}\setmonofont[Path=/usr/share/fonts/truetype/cmu/,UprightFont=cmuntt.ttf,BoldFont=cmuntb.ttf,ItalicFont=cmunit.ttf,BoldItalicFont=cmuntx.ttf]{cmuntt.ttf}\ttfamily pdftex}{$\text{ }$}\setmainfont[Path=/usr/share/fonts/truetype/cmu/,UprightFont=cmunrm.ttf,BoldFont=cmunbx.ttf,ItalicFont=cmunti.ttf,BoldItalicFont=cmunbi.ttf]{cmunrm.ttf}\setmonofont[Path=/usr/share/fonts/truetype/cmu/,UprightFont=cmuntt.ttf,BoldFont=cmuntb.ttf,ItalicFont=cmunit.ttf,BoldItalicFont=cmuntx.ttf]{cmunrm.ttf} with LaTeX initialization files, reads a LaTeX {\ttfamily \setmainfont[Path=/usr/share/fonts/truetype/cmu/,UprightFont=cmunrm.ttf,BoldFont=cmunbx.ttf,ItalicFont=cmunti.ttf,BoldItalicFont=cmunbi.ttf]{cmuntt.ttf}\setmonofont[Path=/usr/share/fonts/truetype/cmu/,UprightFont=cmuntt.ttf,BoldFont=cmuntb.ttf,ItalicFont=cmunit.ttf,BoldItalicFont=cmuntx.ttf]{cmuntt.ttf}\ttfamily .tex}{$\text{ }$}\setmainfont[Path=/usr/share/fonts/truetype/cmu/,UprightFont=cmunrm.ttf,BoldFont=cmunbx.ttf,ItalicFont=cmunti.ttf,BoldItalicFont=cmunbi.ttf]{cmunrm.ttf}\setmonofont[Path=/usr/share/fonts/truetype/cmu/,UprightFont=cmuntt.ttf,BoldFont=cmuntb.ttf,ItalicFont=cmunit.ttf,BoldItalicFont=cmuntx.ttf]{cmunrm.ttf} file and creates a {\ttfamily \setmainfont[Path=/usr/share/fonts/truetype/cmu/,UprightFont=cmunrm.ttf,BoldFont=cmunbx.ttf,ItalicFont=cmunti.ttf,BoldItalicFont=cmunbi.ttf]{cmuntt.ttf}\setmonofont[Path=/usr/share/fonts/truetype/cmu/,UprightFont=cmuntt.ttf,BoldFont=cmuntb.ttf,ItalicFont=cmunit.ttf,BoldItalicFont=cmuntx.ttf]{cmuntt.ttf}\ttfamily .pdf}
\end{myitemize}
\setmainfont[Path=/usr/share/fonts/truetype/cmu/,UprightFont=cmunrm.ttf,BoldFont=cmunbx.ttf,ItalicFont=cmunti.ttf,BoldItalicFont=cmunbi.ttf]{cmunrm.ttf}\setmonofont[Path=/usr/share/fonts/truetype/cmu/,UprightFont=cmuntt.ttf,BoldFont=cmuntb.ttf,ItalicFont=cmunit.ttf,BoldItalicFont=cmuntx.ttf]{cmunrm.ttf}

If you compile a Plain TeX document with a LaTeX compiler (such as {\ttfamily \setmainfont[Path=/usr/share/fonts/truetype/cmu/,UprightFont=cmunrm.ttf,BoldFont=cmunbx.ttf,ItalicFont=cmunti.ttf,BoldItalicFont=cmunbi.ttf]{cmuntt.ttf}\setmonofont[Path=/usr/share/fonts/truetype/cmu/,UprightFont=cmuntt.ttf,BoldFont=cmuntb.ttf,ItalicFont=cmunit.ttf,BoldItalicFont=cmuntx.ttf]{cmuntt.ttf}\ttfamily pdflatex}\setmainfont[Path=/usr/share/fonts/truetype/cmu/,UprightFont=cmunrm.ttf,BoldFont=cmunbx.ttf,ItalicFont=cmunti.ttf,BoldItalicFont=cmunbi.ttf]{cmunrm.ttf}\setmonofont[Path=/usr/share/fonts/truetype/cmu/,UprightFont=cmuntt.ttf,BoldFont=cmuntb.ttf,ItalicFont=cmunit.ttf,BoldItalicFont=cmuntx.ttf]{cmunrm.ttf}) it will work while the opposite is not true: if you try to compile a LaTeX source with a TeX compiler you will get many errors. 

As a matter of fact, following your operating system {\ttfamily \setmainfont[Path=/usr/share/fonts/truetype/cmu/,UprightFont=cmunrm.ttf,BoldFont=cmunbx.ttf,ItalicFont=cmunti.ttf,BoldItalicFont=cmunbi.ttf]{cmuntt.ttf}\setmonofont[Path=/usr/share/fonts/truetype/cmu/,UprightFont=cmuntt.ttf,BoldFont=cmuntb.ttf,ItalicFont=cmunit.ttf,BoldItalicFont=cmuntx.ttf]{cmuntt.ttf}\ttfamily latex}{$\text{ }$}\setmainfont[Path=/usr/share/fonts/truetype/cmu/,UprightFont=cmunrm.ttf,BoldFont=cmunbx.ttf,ItalicFont=cmunti.ttf,BoldItalicFont=cmunbi.ttf]{cmunrm.ttf}\setmonofont[Path=/usr/share/fonts/truetype/cmu/,UprightFont=cmuntt.ttf,BoldFont=cmuntb.ttf,ItalicFont=cmunit.ttf,BoldItalicFont=cmuntx.ttf]{cmunrm.ttf} and {\ttfamily \setmainfont[Path=/usr/share/fonts/truetype/cmu/,UprightFont=cmunrm.ttf,BoldFont=cmunbx.ttf,ItalicFont=cmunti.ttf,BoldItalicFont=cmunbi.ttf]{cmuntt.ttf}\setmonofont[Path=/usr/share/fonts/truetype/cmu/,UprightFont=cmuntt.ttf,BoldFont=cmuntb.ttf,ItalicFont=cmunit.ttf,BoldItalicFont=cmuntx.ttf]{cmuntt.ttf}\ttfamily pdflatex}{$\text{ }$}\setmainfont[Path=/usr/share/fonts/truetype/cmu/,UprightFont=cmunrm.ttf,BoldFont=cmunbx.ttf,ItalicFont=cmunti.ttf,BoldItalicFont=cmunbi.ttf]{cmunrm.ttf}\setmonofont[Path=/usr/share/fonts/truetype/cmu/,UprightFont=cmuntt.ttf,BoldFont=cmuntb.ttf,ItalicFont=cmunit.ttf,BoldItalicFont=cmuntx.ttf]{cmunrm.ttf} are simple scripts or symbolic links.

Most of the programs should be already within your LaTeX distribution; the others come with \myhref{https://en.wikibooks.org/wiki/Ghostscript}{Ghostscript}, which is a free and multi-{}platform software as well.
Here are common programs you expect to find in any LaTeX distribution:
\begin{myitemize}
\item{}  {\ttfamily \setmainfont[Path=/usr/share/fonts/truetype/cmu/,UprightFont=cmunrm.ttf,BoldFont=cmunbx.ttf,ItalicFont=cmunti.ttf,BoldItalicFont=cmunbi.ttf]{cmuntt.ttf}\setmonofont[Path=/usr/share/fonts/truetype/cmu/,UprightFont=cmuntt.ttf,BoldFont=cmuntb.ttf,ItalicFont=cmunit.ttf,BoldItalicFont=cmuntx.ttf]{cmuntt.ttf}\ttfamily dvi2ps}{$\text{ }$}\setmainfont[Path=/usr/share/fonts/truetype/cmu/,UprightFont=cmunrm.ttf,BoldFont=cmunbx.ttf,ItalicFont=cmunti.ttf,BoldItalicFont=cmunbi.ttf]{cmunrm.ttf}\setmonofont[Path=/usr/share/fonts/truetype/cmu/,UprightFont=cmuntt.ttf,BoldFont=cmuntb.ttf,ItalicFont=cmunit.ttf,BoldItalicFont=cmuntx.ttf]{cmunrm.ttf} converts the {\ttfamily \setmainfont[Path=/usr/share/fonts/truetype/cmu/,UprightFont=cmunrm.ttf,BoldFont=cmunbx.ttf,ItalicFont=cmunti.ttf,BoldItalicFont=cmunbi.ttf]{cmuntt.ttf}\setmonofont[Path=/usr/share/fonts/truetype/cmu/,UprightFont=cmuntt.ttf,BoldFont=cmuntb.ttf,ItalicFont=cmunit.ttf,BoldItalicFont=cmuntx.ttf]{cmuntt.ttf}\ttfamily .dvi}{$\text{ }$}\setmainfont[Path=/usr/share/fonts/truetype/cmu/,UprightFont=cmunrm.ttf,BoldFont=cmunbx.ttf,ItalicFont=cmunti.ttf,BoldItalicFont=cmunbi.ttf]{cmunrm.ttf}\setmonofont[Path=/usr/share/fonts/truetype/cmu/,UprightFont=cmuntt.ttf,BoldFont=cmuntb.ttf,ItalicFont=cmunit.ttf,BoldItalicFont=cmuntx.ttf]{cmunrm.ttf} file to {\ttfamily \setmainfont[Path=/usr/share/fonts/truetype/cmu/,UprightFont=cmunrm.ttf,BoldFont=cmunbx.ttf,ItalicFont=cmunti.ttf,BoldItalicFont=cmunbi.ttf]{cmuntt.ttf}\setmonofont[Path=/usr/share/fonts/truetype/cmu/,UprightFont=cmuntt.ttf,BoldFont=cmuntb.ttf,ItalicFont=cmunit.ttf,BoldItalicFont=cmuntx.ttf]{cmuntt.ttf}\ttfamily .ps}{$\text{ }$}\setmainfont[Path=/usr/share/fonts/truetype/cmu/,UprightFont=cmunrm.ttf,BoldFont=cmunbx.ttf,ItalicFont=cmunti.ttf,BoldItalicFont=cmunbi.ttf]{cmunrm.ttf}\setmonofont[Path=/usr/share/fonts/truetype/cmu/,UprightFont=cmuntt.ttf,BoldFont=cmuntb.ttf,ItalicFont=cmunit.ttf,BoldItalicFont=cmuntx.ttf]{cmunrm.ttf} (postscript).
\item{}  {\ttfamily \setmainfont[Path=/usr/share/fonts/truetype/cmu/,UprightFont=cmunrm.ttf,BoldFont=cmunbx.ttf,ItalicFont=cmunti.ttf,BoldItalicFont=cmunbi.ttf]{cmuntt.ttf}\setmonofont[Path=/usr/share/fonts/truetype/cmu/,UprightFont=cmuntt.ttf,BoldFont=cmuntb.ttf,ItalicFont=cmunit.ttf,BoldItalicFont=cmuntx.ttf]{cmuntt.ttf}\ttfamily dvi2pdf}{$\text{ }$}\setmainfont[Path=/usr/share/fonts/truetype/cmu/,UprightFont=cmunrm.ttf,BoldFont=cmunbx.ttf,ItalicFont=cmunti.ttf,BoldItalicFont=cmunbi.ttf]{cmunrm.ttf}\setmonofont[Path=/usr/share/fonts/truetype/cmu/,UprightFont=cmuntt.ttf,BoldFont=cmuntb.ttf,ItalicFont=cmunit.ttf,BoldItalicFont=cmuntx.ttf]{cmunrm.ttf} converts the {\ttfamily \setmainfont[Path=/usr/share/fonts/truetype/cmu/,UprightFont=cmunrm.ttf,BoldFont=cmunbx.ttf,ItalicFont=cmunti.ttf,BoldItalicFont=cmunbi.ttf]{cmuntt.ttf}\setmonofont[Path=/usr/share/fonts/truetype/cmu/,UprightFont=cmuntt.ttf,BoldFont=cmuntb.ttf,ItalicFont=cmunit.ttf,BoldItalicFont=cmuntx.ttf]{cmuntt.ttf}\ttfamily .dvi}{$\text{ }$}\setmainfont[Path=/usr/share/fonts/truetype/cmu/,UprightFont=cmunrm.ttf,BoldFont=cmunbx.ttf,ItalicFont=cmunti.ttf,BoldItalicFont=cmunbi.ttf]{cmunrm.ttf}\setmonofont[Path=/usr/share/fonts/truetype/cmu/,UprightFont=cmuntt.ttf,BoldFont=cmuntb.ttf,ItalicFont=cmunit.ttf,BoldItalicFont=cmuntx.ttf]{cmunrm.ttf} file to {\ttfamily \setmainfont[Path=/usr/share/fonts/truetype/cmu/,UprightFont=cmunrm.ttf,BoldFont=cmunbx.ttf,ItalicFont=cmunti.ttf,BoldItalicFont=cmunbi.ttf]{cmuntt.ttf}\setmonofont[Path=/usr/share/fonts/truetype/cmu/,UprightFont=cmuntt.ttf,BoldFont=cmuntb.ttf,ItalicFont=cmunit.ttf,BoldItalicFont=cmuntx.ttf]{cmuntt.ttf}\ttfamily .pdf}{$\text{ }$}\setmainfont[Path=/usr/share/fonts/truetype/cmu/,UprightFont=cmunrm.ttf,BoldFont=cmunbx.ttf,ItalicFont=cmunti.ttf,BoldItalicFont=cmunbi.ttf]{cmunrm.ttf}\setmonofont[Path=/usr/share/fonts/truetype/cmu/,UprightFont=cmuntt.ttf,BoldFont=cmuntb.ttf,ItalicFont=cmunit.ttf,BoldItalicFont=cmuntx.ttf]{cmunrm.ttf} ({\ttfamily \setmainfont[Path=/usr/share/fonts/truetype/cmu/,UprightFont=cmunrm.ttf,BoldFont=cmunbx.ttf,ItalicFont=cmunti.ttf,BoldItalicFont=cmunbi.ttf]{cmuntt.ttf}\setmonofont[Path=/usr/share/fonts/truetype/cmu/,UprightFont=cmuntt.ttf,BoldFont=cmuntb.ttf,ItalicFont=cmunit.ttf,BoldItalicFont=cmuntx.ttf]{cmuntt.ttf}\ttfamily dvi2pdfm}{$\text{ }$}\setmainfont[Path=/usr/share/fonts/truetype/cmu/,UprightFont=cmunrm.ttf,BoldFont=cmunbx.ttf,ItalicFont=cmunti.ttf,BoldItalicFont=cmunbi.ttf]{cmunrm.ttf}\setmonofont[Path=/usr/share/fonts/truetype/cmu/,UprightFont=cmuntt.ttf,BoldFont=cmuntb.ttf,ItalicFont=cmunit.ttf,BoldItalicFont=cmuntx.ttf]{cmunrm.ttf} is an improved version).
\end{myitemize}

and with Ghostscript:
\begin{myitemize}
\item{}  {\ttfamily \setmainfont[Path=/usr/share/fonts/truetype/cmu/,UprightFont=cmunrm.ttf,BoldFont=cmunbx.ttf,ItalicFont=cmunti.ttf,BoldItalicFont=cmunbi.ttf]{cmuntt.ttf}\setmonofont[Path=/usr/share/fonts/truetype/cmu/,UprightFont=cmuntt.ttf,BoldFont=cmuntb.ttf,ItalicFont=cmunit.ttf,BoldItalicFont=cmuntx.ttf]{cmuntt.ttf}\ttfamily ps2pdf}{$\text{ }$}\setmainfont[Path=/usr/share/fonts/truetype/cmu/,UprightFont=cmunrm.ttf,BoldFont=cmunbx.ttf,ItalicFont=cmunti.ttf,BoldItalicFont=cmunbi.ttf]{cmunrm.ttf}\setmonofont[Path=/usr/share/fonts/truetype/cmu/,UprightFont=cmuntt.ttf,BoldFont=cmuntb.ttf,ItalicFont=cmunit.ttf,BoldItalicFont=cmuntx.ttf]{cmunrm.ttf} and {\ttfamily \setmainfont[Path=/usr/share/fonts/truetype/cmu/,UprightFont=cmunrm.ttf,BoldFont=cmunbx.ttf,ItalicFont=cmunti.ttf,BoldItalicFont=cmunbi.ttf]{cmuntt.ttf}\setmonofont[Path=/usr/share/fonts/truetype/cmu/,UprightFont=cmuntt.ttf,BoldFont=cmuntb.ttf,ItalicFont=cmunit.ttf,BoldItalicFont=cmuntx.ttf]{cmuntt.ttf}\ttfamily pdf2ps}{$\text{ }$}\setmainfont[Path=/usr/share/fonts/truetype/cmu/,UprightFont=cmunrm.ttf,BoldFont=cmunbx.ttf,ItalicFont=cmunti.ttf,BoldItalicFont=cmunbi.ttf]{cmunrm.ttf}\setmonofont[Path=/usr/share/fonts/truetype/cmu/,UprightFont=cmuntt.ttf,BoldFont=cmuntb.ttf,ItalicFont=cmunit.ttf,BoldItalicFont=cmuntx.ttf]{cmunrm.ttf} converts the {\ttfamily \setmainfont[Path=/usr/share/fonts/truetype/cmu/,UprightFont=cmunrm.ttf,BoldFont=cmunbx.ttf,ItalicFont=cmunti.ttf,BoldItalicFont=cmunbi.ttf]{cmuntt.ttf}\setmonofont[Path=/usr/share/fonts/truetype/cmu/,UprightFont=cmuntt.ttf,BoldFont=cmuntb.ttf,ItalicFont=cmunit.ttf,BoldItalicFont=cmuntx.ttf]{cmuntt.ttf}\ttfamily .ps}{$\text{ }$}\setmainfont[Path=/usr/share/fonts/truetype/cmu/,UprightFont=cmunrm.ttf,BoldFont=cmunbx.ttf,ItalicFont=cmunti.ttf,BoldItalicFont=cmunbi.ttf]{cmunrm.ttf}\setmonofont[Path=/usr/share/fonts/truetype/cmu/,UprightFont=cmuntt.ttf,BoldFont=cmuntb.ttf,ItalicFont=cmunit.ttf,BoldItalicFont=cmuntx.ttf]{cmunrm.ttf} file to {\ttfamily \setmainfont[Path=/usr/share/fonts/truetype/cmu/,UprightFont=cmunrm.ttf,BoldFont=cmunbx.ttf,ItalicFont=cmunti.ttf,BoldItalicFont=cmunbi.ttf]{cmuntt.ttf}\setmonofont[Path=/usr/share/fonts/truetype/cmu/,UprightFont=cmuntt.ttf,BoldFont=cmuntb.ttf,ItalicFont=cmunit.ttf,BoldItalicFont=cmuntx.ttf]{cmuntt.ttf}\ttfamily .pdf}{$\text{ }$}\setmainfont[Path=/usr/share/fonts/truetype/cmu/,UprightFont=cmunrm.ttf,BoldFont=cmunbx.ttf,ItalicFont=cmunti.ttf,BoldItalicFont=cmunbi.ttf]{cmunrm.ttf}\setmonofont[Path=/usr/share/fonts/truetype/cmu/,UprightFont=cmuntt.ttf,BoldFont=cmuntb.ttf,ItalicFont=cmunit.ttf,BoldItalicFont=cmuntx.ttf]{cmunrm.ttf} and vice-{}versa.
\end{myitemize}


When LaTeX was created, the only format it could create was DVI; later PDF support was added by {\ttfamily \setmainfont[Path=/usr/share/fonts/truetype/cmu/,UprightFont=cmunrm.ttf,BoldFont=cmunbx.ttf,ItalicFont=cmunti.ttf,BoldItalicFont=cmunbi.ttf]{cmuntt.ttf}\setmonofont[Path=/usr/share/fonts/truetype/cmu/,UprightFont=cmuntt.ttf,BoldFont=cmuntb.ttf,ItalicFont=cmunit.ttf,BoldItalicFont=cmuntx.ttf]{cmuntt.ttf}\ttfamily pdflatex}\setmainfont[Path=/usr/share/fonts/truetype/cmu/,UprightFont=cmunrm.ttf,BoldFont=cmunbx.ttf,ItalicFont=cmunti.ttf,BoldItalicFont=cmunbi.ttf]{cmunrm.ttf}\setmonofont[Path=/usr/share/fonts/truetype/cmu/,UprightFont=cmuntt.ttf,BoldFont=cmuntb.ttf,ItalicFont=cmunit.ttf,BoldItalicFont=cmuntx.ttf]{cmunrm.ttf}. PDF files can be created with both {\ttfamily \setmainfont[Path=/usr/share/fonts/truetype/cmu/,UprightFont=cmunrm.ttf,BoldFont=cmunbx.ttf,ItalicFont=cmunti.ttf,BoldItalicFont=cmunbi.ttf]{cmuntt.ttf}\setmonofont[Path=/usr/share/fonts/truetype/cmu/,UprightFont=cmuntt.ttf,BoldFont=cmuntb.ttf,ItalicFont=cmunit.ttf,BoldItalicFont=cmuntx.ttf]{cmuntt.ttf}\ttfamily pdflatex}{$\text{ }$}\setmainfont[Path=/usr/share/fonts/truetype/cmu/,UprightFont=cmunrm.ttf,BoldFont=cmunbx.ttf,ItalicFont=cmunti.ttf,BoldItalicFont=cmunbi.ttf]{cmunrm.ttf}\setmonofont[Path=/usr/share/fonts/truetype/cmu/,UprightFont=cmuntt.ttf,BoldFont=cmuntb.ttf,ItalicFont=cmunit.ttf,BoldItalicFont=cmuntx.ttf]{cmunrm.ttf} and {\ttfamily \setmainfont[Path=/usr/share/fonts/truetype/cmu/,UprightFont=cmunrm.ttf,BoldFont=cmunbx.ttf,ItalicFont=cmunti.ttf,BoldItalicFont=cmunbi.ttf]{cmuntt.ttf}\setmonofont[Path=/usr/share/fonts/truetype/cmu/,UprightFont=cmuntt.ttf,BoldFont=cmuntb.ttf,ItalicFont=cmunit.ttf,BoldItalicFont=cmuntx.ttf]{cmuntt.ttf}\ttfamily dvipdfm}\setmainfont[Path=/usr/share/fonts/truetype/cmu/,UprightFont=cmunrm.ttf,BoldFont=cmunbx.ttf,ItalicFont=cmunti.ttf,BoldItalicFont=cmunbi.ttf]{cmunrm.ttf}\setmonofont[Path=/usr/share/fonts/truetype/cmu/,UprightFont=cmuntt.ttf,BoldFont=cmuntb.ttf,ItalicFont=cmunit.ttf,BoldItalicFont=cmuntx.ttf]{cmunrm.ttf}.  The output of {\ttfamily \setmainfont[Path=/usr/share/fonts/truetype/cmu/,UprightFont=cmunrm.ttf,BoldFont=cmunbx.ttf,ItalicFont=cmunti.ttf,BoldItalicFont=cmunbi.ttf]{cmuntt.ttf}\setmonofont[Path=/usr/share/fonts/truetype/cmu/,UprightFont=cmuntt.ttf,BoldFont=cmuntb.ttf,ItalicFont=cmunit.ttf,BoldItalicFont=cmuntx.ttf]{cmuntt.ttf}\ttfamily pdflatex}{$\text{ }$}\setmainfont[Path=/usr/share/fonts/truetype/cmu/,UprightFont=cmunrm.ttf,BoldFont=cmunbx.ttf,ItalicFont=cmunti.ttf,BoldItalicFont=cmunbi.ttf]{cmunrm.ttf}\setmonofont[Path=/usr/share/fonts/truetype/cmu/,UprightFont=cmuntt.ttf,BoldFont=cmuntb.ttf,ItalicFont=cmunit.ttf,BoldItalicFont=cmuntx.ttf]{cmunrm.ttf} takes direct advantage of modern features of PDF such as hyperlinks and embedded fonts, which are not part of DVI.  Passing through DVI imposes limitations of its older format.  On the other hand, some packages, such as PSTricks, exploit the process of conversion to DVI, and therefore will not work with pdflatex. Some of those packages embed information in the DVI that doesn\textquotesingle{}t appear when the DVI is viewed, but reemerges when the DVI is converted to another, newer format.

You would write your document slightly differently depending on the compiler you are using ({\ttfamily \setmainfont[Path=/usr/share/fonts/truetype/cmu/,UprightFont=cmunrm.ttf,BoldFont=cmunbx.ttf,ItalicFont=cmunti.ttf,BoldItalicFont=cmunbi.ttf]{cmuntt.ttf}\setmonofont[Path=/usr/share/fonts/truetype/cmu/,UprightFont=cmuntt.ttf,BoldFont=cmuntb.ttf,ItalicFont=cmunit.ttf,BoldItalicFont=cmuntx.ttf]{cmuntt.ttf}\ttfamily latex}{$\text{ }$}\setmainfont[Path=/usr/share/fonts/truetype/cmu/,UprightFont=cmunrm.ttf,BoldFont=cmunbx.ttf,ItalicFont=cmunti.ttf,BoldItalicFont=cmunbi.ttf]{cmunrm.ttf}\setmonofont[Path=/usr/share/fonts/truetype/cmu/,UprightFont=cmuntt.ttf,BoldFont=cmuntb.ttf,ItalicFont=cmunit.ttf,BoldItalicFont=cmuntx.ttf]{cmunrm.ttf} or {\ttfamily \setmainfont[Path=/usr/share/fonts/truetype/cmu/,UprightFont=cmunrm.ttf,BoldFont=cmunbx.ttf,ItalicFont=cmunti.ttf,BoldItalicFont=cmunbi.ttf]{cmuntt.ttf}\setmonofont[Path=/usr/share/fonts/truetype/cmu/,UprightFont=cmuntt.ttf,BoldFont=cmuntb.ttf,ItalicFont=cmunit.ttf,BoldItalicFont=cmuntx.ttf]{cmuntt.ttf}\ttfamily pdflatex}\setmainfont[Path=/usr/share/fonts/truetype/cmu/,UprightFont=cmunrm.ttf,BoldFont=cmunbx.ttf,ItalicFont=cmunti.ttf,BoldItalicFont=cmunbi.ttf]{cmunrm.ttf}\setmonofont[Path=/usr/share/fonts/truetype/cmu/,UprightFont=cmuntt.ttf,BoldFont=cmuntb.ttf,ItalicFont=cmunit.ttf,BoldItalicFont=cmuntx.ttf]{cmunrm.ttf}).  But as we will see later it is possible to add a sort of abstraction layer to hide the details of which compiler you\textquotesingle{}re using, while the compiler can handle the translation itself.

The following diagram shows the relationships between the LaTeX source code and the formats you can create from it:



\begin{minipage}{1.0\linewidth}
\begin{center}
\includegraphics[width=1.0\linewidth,height=6.5in,keepaspectratio]{../images/7.\SVGExtension}
\end{center}
\raggedright{}\myfigurewithoutcaption{7}
\end{minipage}\vspace{0.75cm}



The boxed red text represents the file formats, the blue text on the arrows represents the commands you have to use, the small dark green text under the boxes represents the image formats that are supported. Any time you pass through an arrow you lose some information, which might decrease the features of your document. Therefore, you should choose the shortest route to reach your target format. This is probably the most convenient way to obtain an output in your desired format anyway. Starting from a LaTeX source, the best way is to use only {\itshape \setmainfont[Path=/usr/share/fonts/truetype/cmu/,UprightFont=cmunrm.ttf,BoldFont=cmunbx.ttf,ItalicFont=cmunti.ttf,BoldItalicFont=cmunbi.ttf]{cmunti.ttf}\setmonofont[Path=/usr/share/fonts/truetype/cmu/,UprightFont=cmuntt.ttf,BoldFont=cmuntb.ttf,ItalicFont=cmunit.ttf,BoldItalicFont=cmuntx.ttf]{cmunti.ttf}\itshape latex}{$\text{ }$}\setmainfont[Path=/usr/share/fonts/truetype/cmu/,UprightFont=cmunrm.ttf,BoldFont=cmunbx.ttf,ItalicFont=cmunti.ttf,BoldItalicFont=cmunbi.ttf]{cmunrm.ttf}\setmonofont[Path=/usr/share/fonts/truetype/cmu/,UprightFont=cmuntt.ttf,BoldFont=cmuntb.ttf,ItalicFont=cmunit.ttf,BoldItalicFont=cmuntx.ttf]{cmunrm.ttf} for a DVI output or {\itshape \setmainfont[Path=/usr/share/fonts/truetype/cmu/,UprightFont=cmunrm.ttf,BoldFont=cmunbx.ttf,ItalicFont=cmunti.ttf,BoldItalicFont=cmunbi.ttf]{cmunti.ttf}\setmonofont[Path=/usr/share/fonts/truetype/cmu/,UprightFont=cmuntt.ttf,BoldFont=cmuntb.ttf,ItalicFont=cmunit.ttf,BoldItalicFont=cmuntx.ttf]{cmunti.ttf}\itshape pdflatex}{$\text{ }$}\setmainfont[Path=/usr/share/fonts/truetype/cmu/,UprightFont=cmunrm.ttf,BoldFont=cmunbx.ttf,ItalicFont=cmunti.ttf,BoldItalicFont=cmunbi.ttf]{cmunrm.ttf}\setmonofont[Path=/usr/share/fonts/truetype/cmu/,UprightFont=cmuntt.ttf,BoldFont=cmuntb.ttf,ItalicFont=cmunit.ttf,BoldItalicFont=cmuntx.ttf]{cmunrm.ttf} for a PDF output, converting to PostScript only when it is necessary to print the document.

Chapter \mylref{925}{../Export To Other Formats/} discusses more about exporting LaTeX source to other file formats.
\subsection{Generating the document}
\label{75}

It is clearly not going to be the most exciting document you have ever seen, but we want to see it nonetheless. I am assuming that you are at a command prompt, already in the directory where {\ttfamily \setmainfont[Path=/usr/share/fonts/truetype/cmu/,UprightFont=cmunrm.ttf,BoldFont=cmunbx.ttf,ItalicFont=cmunti.ttf,BoldItalicFont=cmunbi.ttf]{cmuntt.ttf}\setmonofont[Path=/usr/share/fonts/truetype/cmu/,UprightFont=cmuntt.ttf,BoldFont=cmuntb.ttf,ItalicFont=cmunit.ttf,BoldItalicFont=cmuntx.ttf]{cmuntt.ttf}\ttfamily hello.tex}{$\text{ }$}\setmainfont[Path=/usr/share/fonts/truetype/cmu/,UprightFont=cmunrm.ttf,BoldFont=cmunbx.ttf,ItalicFont=cmunti.ttf,BoldItalicFont=cmunbi.ttf]{cmunrm.ttf}\setmonofont[Path=/usr/share/fonts/truetype/cmu/,UprightFont=cmuntt.ttf,BoldFont=cmuntb.ttf,ItalicFont=cmunit.ttf,BoldItalicFont=cmuntx.ttf]{cmunrm.ttf} is stored.
LaTeX itself does not have a GUI (graphical user interface), since it is just a program that crunches away at your input
files, and produces either a DVI or PDF file. Some LaTeX installations feature a graphical front-{}end where you can click LaTeX into compiling your input file. On other systems there might
be some typing involved, so here is how to coax LaTeX into compiling your input file on a text based system. Please note: this description assumes that you already have a working LaTeX installation on your computer.

\begin{myenumerate}
\item{}  Type the command: {\ttfamily \setmainfont[Path=/usr/share/fonts/truetype/cmu/,UprightFont=cmunrm.ttf,BoldFont=cmunbx.ttf,ItalicFont=cmunti.ttf,BoldItalicFont=cmunbi.ttf]{cmuntt.ttf}\setmonofont[Path=/usr/share/fonts/truetype/cmu/,UprightFont=cmuntt.ttf,BoldFont=cmuntb.ttf,ItalicFont=cmunit.ttf,BoldItalicFont=cmuntx.ttf]{cmuntt.ttf}\ttfamily latex hello}{$\text{ }$}\setmainfont[Path=/usr/share/fonts/truetype/cmu/,UprightFont=cmunrm.ttf,BoldFont=cmunbx.ttf,ItalicFont=cmunti.ttf,BoldItalicFont=cmunbi.ttf]{cmunrm.ttf}\setmonofont[Path=/usr/share/fonts/truetype/cmu/,UprightFont=cmuntt.ttf,BoldFont=cmuntb.ttf,ItalicFont=cmunit.ttf,BoldItalicFont=cmuntx.ttf]{cmunrm.ttf} (the {\ttfamily \setmainfont[Path=/usr/share/fonts/truetype/cmu/,UprightFont=cmunrm.ttf,BoldFont=cmunbx.ttf,ItalicFont=cmunti.ttf,BoldItalicFont=cmunbi.ttf]{cmuntt.ttf}\setmonofont[Path=/usr/share/fonts/truetype/cmu/,UprightFont=cmuntt.ttf,BoldFont=cmuntb.ttf,ItalicFont=cmunit.ttf,BoldItalicFont=cmuntx.ttf]{cmuntt.ttf}\ttfamily .tex}{$\text{ }$}\setmainfont[Path=/usr/share/fonts/truetype/cmu/,UprightFont=cmunrm.ttf,BoldFont=cmunbx.ttf,ItalicFont=cmunti.ttf,BoldItalicFont=cmunbi.ttf]{cmunrm.ttf}\setmonofont[Path=/usr/share/fonts/truetype/cmu/,UprightFont=cmuntt.ttf,BoldFont=cmuntb.ttf,ItalicFont=cmunit.ttf,BoldItalicFont=cmuntx.ttf]{cmunrm.ttf} extension is not required, although you can include it if you wish)
\item{}  Various bits of info about LaTeX and its progress will be displayed. If all went well, the last two lines displayed in the console will be: 
\end{myenumerate}

\\

\TemplateSpaceIndent{$\text{ }${}Output$\text{ }${}written$\text{ }${}on$\text{ }${}hello.dvi$\text{ }${}(1$\text{ }${}page,$\text{ }${}232$\text{ }${}bytes).$\text{ }$\newline{}
$\text{ }${}Transcript$\text{ }${}written$\text{ }${}on$\text{ }${}hello.log.}


This means that your source file has been processed and the resulting document is called {\itshape \setmainfont[Path=/usr/share/fonts/truetype/cmu/,UprightFont=cmunrm.ttf,BoldFont=cmunbx.ttf,ItalicFont=cmunti.ttf,BoldItalicFont=cmunbi.ttf]{cmunti.ttf}\setmonofont[Path=/usr/share/fonts/truetype/cmu/,UprightFont=cmuntt.ttf,BoldFont=cmuntb.ttf,ItalicFont=cmunit.ttf,BoldItalicFont=cmuntx.ttf]{cmunti.ttf}\itshape hello.dvi}\setmainfont[Path=/usr/share/fonts/truetype/cmu/,UprightFont=cmunrm.ttf,BoldFont=cmunbx.ttf,ItalicFont=cmunti.ttf,BoldItalicFont=cmunbi.ttf]{cmunrm.ttf}\setmonofont[Path=/usr/share/fonts/truetype/cmu/,UprightFont=cmuntt.ttf,BoldFont=cmuntb.ttf,ItalicFont=cmunit.ttf,BoldItalicFont=cmuntx.ttf]{cmunrm.ttf}, which takes up 1 page and 232 bytes of space.
Now you may view the DVI file. On Unix with X11 you can type {\ttfamily \setmainfont[Path=/usr/share/fonts/truetype/cmu/,UprightFont=cmunrm.ttf,BoldFont=cmunbx.ttf,ItalicFont=cmunti.ttf,BoldItalicFont=cmunbi.ttf]{cmuntt.ttf}\setmonofont[Path=/usr/share/fonts/truetype/cmu/,UprightFont=cmuntt.ttf,BoldFont=cmuntb.ttf,ItalicFont=cmunit.ttf,BoldItalicFont=cmuntx.ttf]{cmuntt.ttf}\ttfamily xdvi foo.dvi}\setmainfont[Path=/usr/share/fonts/truetype/cmu/,UprightFont=cmunrm.ttf,BoldFont=cmunbx.ttf,ItalicFont=cmunti.ttf,BoldItalicFont=cmunbi.ttf]{cmunrm.ttf}\setmonofont[Path=/usr/share/fonts/truetype/cmu/,UprightFont=cmuntt.ttf,BoldFont=cmuntb.ttf,ItalicFont=cmunit.ttf,BoldItalicFont=cmuntx.ttf]{cmunrm.ttf}, on Windows you can use a program called {\itshape \setmainfont[Path=/usr/share/fonts/truetype/cmu/,UprightFont=cmunrm.ttf,BoldFont=cmunbx.ttf,ItalicFont=cmunti.ttf,BoldItalicFont=cmunbi.ttf]{cmunti.ttf}\setmonofont[Path=/usr/share/fonts/truetype/cmu/,UprightFont=cmuntt.ttf,BoldFont=cmuntb.ttf,ItalicFont=cmunit.ttf,BoldItalicFont=cmuntx.ttf]{cmunti.ttf}\itshape yap}{$\text{ }$}\setmainfont[Path=/usr/share/fonts/truetype/cmu/,UprightFont=cmunrm.ttf,BoldFont=cmunbx.ttf,ItalicFont=cmunti.ttf,BoldItalicFont=cmunbi.ttf]{cmunrm.ttf}\setmonofont[Path=/usr/share/fonts/truetype/cmu/,UprightFont=cmuntt.ttf,BoldFont=cmuntb.ttf,ItalicFont=cmunit.ttf,BoldItalicFont=cmuntx.ttf]{cmunrm.ttf} (yet another previewer). (Now evince and okular, the standard document viewers for many Linux distributions are able to view DVI files.)

This way you created the DVI file, but with the same source file you can create a PDF document. The steps are exactly the same as before, but you have to replace the command {\ttfamily \setmainfont[Path=/usr/share/fonts/truetype/cmu/,UprightFont=cmunrm.ttf,BoldFont=cmunbx.ttf,ItalicFont=cmunti.ttf,BoldItalicFont=cmunbi.ttf]{cmuntt.ttf}\setmonofont[Path=/usr/share/fonts/truetype/cmu/,UprightFont=cmuntt.ttf,BoldFont=cmuntb.ttf,ItalicFont=cmunit.ttf,BoldItalicFont=cmuntx.ttf]{cmuntt.ttf}\ttfamily latex}{$\text{ }$}\setmainfont[Path=/usr/share/fonts/truetype/cmu/,UprightFont=cmunrm.ttf,BoldFont=cmunbx.ttf,ItalicFont=cmunti.ttf,BoldItalicFont=cmunbi.ttf]{cmunrm.ttf}\setmonofont[Path=/usr/share/fonts/truetype/cmu/,UprightFont=cmuntt.ttf,BoldFont=cmuntb.ttf,ItalicFont=cmunit.ttf,BoldItalicFont=cmuntx.ttf]{cmunrm.ttf} with {\ttfamily \setmainfont[Path=/usr/share/fonts/truetype/cmu/,UprightFont=cmunrm.ttf,BoldFont=cmunbx.ttf,ItalicFont=cmunti.ttf,BoldItalicFont=cmunbi.ttf]{cmuntt.ttf}\setmonofont[Path=/usr/share/fonts/truetype/cmu/,UprightFont=cmuntt.ttf,BoldFont=cmuntb.ttf,ItalicFont=cmunit.ttf,BoldItalicFont=cmuntx.ttf]{cmuntt.ttf}\ttfamily pdflatex}\setmainfont[Path=/usr/share/fonts/truetype/cmu/,UprightFont=cmunrm.ttf,BoldFont=cmunbx.ttf,ItalicFont=cmunti.ttf,BoldItalicFont=cmunbi.ttf]{cmunrm.ttf}\setmonofont[Path=/usr/share/fonts/truetype/cmu/,UprightFont=cmuntt.ttf,BoldFont=cmuntb.ttf,ItalicFont=cmunit.ttf,BoldItalicFont=cmuntx.ttf]{cmunrm.ttf}:
\begin{myenumerate}
\item{}  Type the command: {\ttfamily \setmainfont[Path=/usr/share/fonts/truetype/cmu/,UprightFont=cmunrm.ttf,BoldFont=cmunbx.ttf,ItalicFont=cmunti.ttf,BoldItalicFont=cmunbi.ttf]{cmuntt.ttf}\setmonofont[Path=/usr/share/fonts/truetype/cmu/,UprightFont=cmuntt.ttf,BoldFont=cmuntb.ttf,ItalicFont=cmunit.ttf,BoldItalicFont=cmuntx.ttf]{cmuntt.ttf}\ttfamily pdflatex hello}{$\text{ }$}\setmainfont[Path=/usr/share/fonts/truetype/cmu/,UprightFont=cmunrm.ttf,BoldFont=cmunbx.ttf,ItalicFont=cmunti.ttf,BoldItalicFont=cmunbi.ttf]{cmunrm.ttf}\setmonofont[Path=/usr/share/fonts/truetype/cmu/,UprightFont=cmuntt.ttf,BoldFont=cmuntb.ttf,ItalicFont=cmunit.ttf,BoldItalicFont=cmuntx.ttf]{cmunrm.ttf} (as before, the {\ttfamily \setmainfont[Path=/usr/share/fonts/truetype/cmu/,UprightFont=cmunrm.ttf,BoldFont=cmunbx.ttf,ItalicFont=cmunti.ttf,BoldItalicFont=cmunbi.ttf]{cmuntt.ttf}\setmonofont[Path=/usr/share/fonts/truetype/cmu/,UprightFont=cmuntt.ttf,BoldFont=cmuntb.ttf,ItalicFont=cmunit.ttf,BoldItalicFont=cmuntx.ttf]{cmuntt.ttf}\ttfamily .tex}{$\text{ }$}\setmainfont[Path=/usr/share/fonts/truetype/cmu/,UprightFont=cmunrm.ttf,BoldFont=cmunbx.ttf,ItalicFont=cmunti.ttf,BoldItalicFont=cmunbi.ttf]{cmunrm.ttf}\setmonofont[Path=/usr/share/fonts/truetype/cmu/,UprightFont=cmuntt.ttf,BoldFont=cmuntb.ttf,ItalicFont=cmunit.ttf,BoldItalicFont=cmuntx.ttf]{cmunrm.ttf} extension is not required)
\item{}  Various bits of info about LaTeX and its progress will be displayed. If all went well, the last two lines displayed in the console will be: 
\end{myenumerate}

\\

\TemplateSpaceIndent{$\text{ }${}Output$\text{ }${}written$\text{ }${}on$\text{ }${}hello.pdf$\text{ }${}(1$\text{ }${}page,$\text{ }${}5548$\text{ }${}bytes).$\text{ }$\newline{}
$\text{ }${}Transcript$\text{ }${}written$\text{ }${}on$\text{ }${}hello.log.}


you can notice that the PDF document is bigger than the DVI, even if it contains exactly the same information. The main differences between the DVI and PDF formats are:
\begin{myitemize}
\item{}  {\bfseries \setmainfont[Path=/usr/share/fonts/truetype/cmu/,UprightFont=cmunrm.ttf,BoldFont=cmunbx.ttf,ItalicFont=cmunti.ttf,BoldItalicFont=cmunbi.ttf]{cmunbx.ttf}\setmonofont[Path=/usr/share/fonts/truetype/cmu/,UprightFont=cmuntt.ttf,BoldFont=cmuntb.ttf,ItalicFont=cmunit.ttf,BoldItalicFont=cmuntx.ttf]{cmunbx.ttf}\bfseries DVI}{$\text{ }$}\setmainfont[Path=/usr/share/fonts/truetype/cmu/,UprightFont=cmunrm.ttf,BoldFont=cmunbx.ttf,ItalicFont=cmunti.ttf,BoldItalicFont=cmunbi.ttf]{cmunrm.ttf}\setmonofont[Path=/usr/share/fonts/truetype/cmu/,UprightFont=cmuntt.ttf,BoldFont=cmuntb.ttf,ItalicFont=cmunit.ttf,BoldItalicFont=cmuntx.ttf]{cmunrm.ttf} needs less disk space and it is faster to create. It does not include the fonts within the document, so if you want the document to be viewed properly on another computer, there must be all the necessary fonts installed. It does not support any interactivity such as hyperlinks or animated images. DVI viewers are not very common, so you can consider using it for previewing your document while typesetting.
\item{}  {\bfseries \setmainfont[Path=/usr/share/fonts/truetype/cmu/,UprightFont=cmunrm.ttf,BoldFont=cmunbx.ttf,ItalicFont=cmunti.ttf,BoldItalicFont=cmunbi.ttf]{cmunbx.ttf}\setmonofont[Path=/usr/share/fonts/truetype/cmu/,UprightFont=cmuntt.ttf,BoldFont=cmuntb.ttf,ItalicFont=cmunit.ttf,BoldItalicFont=cmuntx.ttf]{cmunbx.ttf}\bfseries PDF}{$\text{ }$}\setmainfont[Path=/usr/share/fonts/truetype/cmu/,UprightFont=cmunrm.ttf,BoldFont=cmunbx.ttf,ItalicFont=cmunti.ttf,BoldItalicFont=cmunbi.ttf]{cmunrm.ttf}\setmonofont[Path=/usr/share/fonts/truetype/cmu/,UprightFont=cmuntt.ttf,BoldFont=cmuntb.ttf,ItalicFont=cmunit.ttf,BoldItalicFont=cmuntx.ttf]{cmunrm.ttf} needs more disk space and it is slower to create, but it includes all the necessary fonts within the document, so you will not have any problem of portability. It supports internal and external hyperlinks. It also supports advanced typographic features: \myhref{https://en.wikipedia.org/wiki/Hanging\%20punctuation}{hanging punctuation}, font expansion and margin kerning resulting in more flexibility available to the TeX engine and better looking output. Nowadays it is the {\itshape \setmainfont[Path=/usr/share/fonts/truetype/cmu/,UprightFont=cmunrm.ttf,BoldFont=cmunbx.ttf,ItalicFont=cmunti.ttf,BoldItalicFont=cmunbi.ttf]{cmunti.ttf}\setmonofont[Path=/usr/share/fonts/truetype/cmu/,UprightFont=cmuntt.ttf,BoldFont=cmuntb.ttf,ItalicFont=cmunit.ttf,BoldItalicFont=cmuntx.ttf]{cmunti.ttf}\itshape de facto}{$\text{ }$}\setmainfont[Path=/usr/share/fonts/truetype/cmu/,UprightFont=cmunrm.ttf,BoldFont=cmunbx.ttf,ItalicFont=cmunti.ttf,BoldItalicFont=cmunbi.ttf]{cmunrm.ttf}\setmonofont[Path=/usr/share/fonts/truetype/cmu/,UprightFont=cmuntt.ttf,BoldFont=cmuntb.ttf,ItalicFont=cmunit.ttf,BoldItalicFont=cmuntx.ttf]{cmunrm.ttf} standard for sharing and publishing documents, so you can consider using it for the final version of your document.
\end{myitemize}


About now, you saw you can create both DVI and PDF document from the same source. This is true, but it gets a bit more complicated if you want to introduce images or links. This will be explained in detail in the next chapters, but for now assume you can compile in both DVI and PDF without any problem.

Note, in this instance, due to the simplicity of the file, you only need to run the LaTeX command once. However, if you begin to create complex documents, including bibliographies and cross-{}references, etc, LaTeX needs to be executed multiple times to resolve the references. But this will be discussed in the future when it comes up.
\subsection{Autobuild Systems}
\label{76}

Compiling using only the {\ttfamily \setmainfont[Path=/usr/share/fonts/truetype/cmu/,UprightFont=cmunrm.ttf,BoldFont=cmunbx.ttf,ItalicFont=cmunti.ttf,BoldItalicFont=cmunbi.ttf]{cmuntt.ttf}\setmonofont[Path=/usr/share/fonts/truetype/cmu/,UprightFont=cmuntt.ttf,BoldFont=cmuntb.ttf,ItalicFont=cmunit.ttf,BoldItalicFont=cmuntx.ttf]{cmuntt.ttf}\ttfamily latex}{$\text{ }$}\setmainfont[Path=/usr/share/fonts/truetype/cmu/,UprightFont=cmunrm.ttf,BoldFont=cmunbx.ttf,ItalicFont=cmunti.ttf,BoldItalicFont=cmunbi.ttf]{cmunrm.ttf}\setmonofont[Path=/usr/share/fonts/truetype/cmu/,UprightFont=cmuntt.ttf,BoldFont=cmuntb.ttf,ItalicFont=cmunit.ttf,BoldItalicFont=cmuntx.ttf]{cmunrm.ttf} binary can be quite tricky as soon as you start working on more complex documents as previously stated. A number of programs exist to automatically read in a TeX document and run the appropriate compilers the appropriate number of times. For example, {\ttfamily \setmainfont[Path=/usr/share/fonts/truetype/cmu/,UprightFont=cmunrm.ttf,BoldFont=cmunbx.ttf,ItalicFont=cmunti.ttf,BoldItalicFont=cmunbi.ttf]{cmuntt.ttf}\setmonofont[Path=/usr/share/fonts/truetype/cmu/,UprightFont=cmuntt.ttf,BoldFont=cmuntb.ttf,ItalicFont=cmunit.ttf,BoldItalicFont=cmuntx.ttf]{cmuntt.ttf}\ttfamily latexmk}{$\text{ }$}\setmainfont[Path=/usr/share/fonts/truetype/cmu/,UprightFont=cmunrm.ttf,BoldFont=cmunbx.ttf,ItalicFont=cmunti.ttf,BoldItalicFont=cmunbi.ttf]{cmunrm.ttf}\setmonofont[Path=/usr/share/fonts/truetype/cmu/,UprightFont=cmuntt.ttf,BoldFont=cmuntb.ttf,ItalicFont=cmunit.ttf,BoldItalicFont=cmuntx.ttf]{cmunrm.ttf} can generate a PDF from most TeX files simply: \\

\TemplateSpaceIndent{$\text{ }${}\${}$\text{ }${}latexmk$\text{ }${}-{}pdf$\text{ }${}file.tex}

Note that most \mylref{22}{editors} will take care of it for you.
\subsection{Compressed PDF}
\label{77}
For a PDF output, you may have noticed that the output PDF file is not always the same size depending on the engine you used to compile the file. So {\ttfamily \setmainfont[Path=/usr/share/fonts/truetype/cmu/,UprightFont=cmunrm.ttf,BoldFont=cmunbx.ttf,ItalicFont=cmunti.ttf,BoldItalicFont=cmunbi.ttf]{cmuntt.ttf}\setmonofont[Path=/usr/share/fonts/truetype/cmu/,UprightFont=cmuntt.ttf,BoldFont=cmuntb.ttf,ItalicFont=cmunit.ttf,BoldItalicFont=cmuntx.ttf]{cmuntt.ttf}\ttfamily latex → dvips → ps2pdf}{$\text{ }$}\setmainfont[Path=/usr/share/fonts/truetype/cmu/,UprightFont=cmunrm.ttf,BoldFont=cmunbx.ttf,ItalicFont=cmunti.ttf,BoldItalicFont=cmunbi.ttf]{cmunrm.ttf}\setmonofont[Path=/usr/share/fonts/truetype/cmu/,UprightFont=cmuntt.ttf,BoldFont=cmuntb.ttf,ItalicFont=cmunit.ttf,BoldItalicFont=cmuntx.ttf]{cmunrm.ttf} will usually be much smaller than {\ttfamily \setmainfont[Path=/usr/share/fonts/truetype/cmu/,UprightFont=cmunrm.ttf,BoldFont=cmunbx.ttf,ItalicFont=cmunti.ttf,BoldItalicFont=cmunbi.ttf]{cmuntt.ttf}\setmonofont[Path=/usr/share/fonts/truetype/cmu/,UprightFont=cmuntt.ttf,BoldFont=cmuntb.ttf,ItalicFont=cmunit.ttf,BoldItalicFont=cmuntx.ttf]{cmuntt.ttf}\ttfamily pdflatex}\setmainfont[Path=/usr/share/fonts/truetype/cmu/,UprightFont=cmunrm.ttf,BoldFont=cmunbx.ttf,ItalicFont=cmunti.ttf,BoldItalicFont=cmunbi.ttf]{cmunrm.ttf}\setmonofont[Path=/usr/share/fonts/truetype/cmu/,UprightFont=cmuntt.ttf,BoldFont=cmuntb.ttf,ItalicFont=cmunit.ttf,BoldItalicFont=cmuntx.ttf]{cmunrm.ttf}.
If you want {\ttfamily \setmainfont[Path=/usr/share/fonts/truetype/cmu/,UprightFont=cmunrm.ttf,BoldFont=cmunbx.ttf,ItalicFont=cmunti.ttf,BoldItalicFont=cmunbi.ttf]{cmuntt.ttf}\setmonofont[Path=/usr/share/fonts/truetype/cmu/,UprightFont=cmuntt.ttf,BoldFont=cmuntb.ttf,ItalicFont=cmunit.ttf,BoldItalicFont=cmuntx.ttf]{cmuntt.ttf}\ttfamily pdflatex}{$\text{ }$}\setmainfont[Path=/usr/share/fonts/truetype/cmu/,UprightFont=cmunrm.ttf,BoldFont=cmunbx.ttf,ItalicFont=cmunti.ttf,BoldItalicFont=cmunbi.ttf]{cmunrm.ttf}\setmonofont[Path=/usr/share/fonts/truetype/cmu/,UprightFont=cmuntt.ttf,BoldFont=cmuntb.ttf,ItalicFont=cmunit.ttf,BoldItalicFont=cmuntx.ttf]{cmunrm.ttf} features along with a small output file size, you can use the Ghostscript command:
\\

\TemplateSpaceIndent{$\text{ }${}\${}$\text{ }${}gs$\text{ }${}-{}dBATCH$\text{ }${}-{}dNOPAUSE$\text{ }${}-{}q$\text{ }${}-{}sDEVICE=pdfwrite$\text{ }${}-{}sOutputFile=\symbol{34}Compressed.pdf\symbol{34}$\text{ }$\newline{}
$\text{ }${}\symbol{34}Original.pdf\symbol{34}}

\section{Files}
\label{78}\subsection{Picking suitable filenames}
\label{79}

Never, ever use directories (folders) or file names that contain spaces. Although your operating system probably supports them, some don\textquotesingle{}t, and they will only cause grief and tears with TeX. Make filenames as short or as long as you wish, but strictly avoid spaces. Stick to lower-{}case letters without accents (a-{}z), the digits 0-{}9, the hyphen ({\mbox{$-$}}), and only one full point or period (.) to separate the file extension (somewhat similar to the conventions for a good Web URL): it will let you refer to TeX files over the Web more easily and make your files more portable. Some operating systems do not distinguish between upper-{}case and lower-{}case letters, others do. Therefore it\textquotesingle{}s best not to mix them.
\subsection{Ancillary files}
\label{80}
The TeX compilers are single-{}pass processes. It means that there is no way for a compiler to {\itshape \setmainfont[Path=/usr/share/fonts/truetype/cmu/,UprightFont=cmunrm.ttf,BoldFont=cmunbx.ttf,ItalicFont=cmunti.ttf,BoldItalicFont=cmunbi.ttf]{cmunti.ttf}\setmonofont[Path=/usr/share/fonts/truetype/cmu/,UprightFont=cmuntt.ttf,BoldFont=cmuntb.ttf,ItalicFont=cmunit.ttf,BoldItalicFont=cmuntx.ttf]{cmunti.ttf}\itshape jump}{$\text{ }$}\setmainfont[Path=/usr/share/fonts/truetype/cmu/,UprightFont=cmunrm.ttf,BoldFont=cmunbx.ttf,ItalicFont=cmunti.ttf,BoldItalicFont=cmunbi.ttf]{cmunrm.ttf}\setmonofont[Path=/usr/share/fonts/truetype/cmu/,UprightFont=cmuntt.ttf,BoldFont=cmuntb.ttf,ItalicFont=cmunit.ttf,BoldItalicFont=cmuntx.ttf]{cmunrm.ttf} around the document, which would be useful for the table of contents and references. Indeed the compiler cannot guess at which page a specific section is going to be printed, so when the table of contents is printed before the upcoming sections, it cannot set the page numbers.

To circumvent this issue, many LaTeX commands which need to {\itshape \setmainfont[Path=/usr/share/fonts/truetype/cmu/,UprightFont=cmunrm.ttf,BoldFont=cmunbx.ttf,ItalicFont=cmunti.ttf,BoldItalicFont=cmunbi.ttf]{cmunti.ttf}\setmonofont[Path=/usr/share/fonts/truetype/cmu/,UprightFont=cmuntt.ttf,BoldFont=cmuntb.ttf,ItalicFont=cmunit.ttf,BoldItalicFont=cmuntx.ttf]{cmunti.ttf}\itshape jump}{$\text{ }$}\setmainfont[Path=/usr/share/fonts/truetype/cmu/,UprightFont=cmunrm.ttf,BoldFont=cmunbx.ttf,ItalicFont=cmunti.ttf,BoldItalicFont=cmunbi.ttf]{cmunrm.ttf}\setmonofont[Path=/usr/share/fonts/truetype/cmu/,UprightFont=cmuntt.ttf,BoldFont=cmuntb.ttf,ItalicFont=cmunit.ttf,BoldItalicFont=cmuntx.ttf]{cmunrm.ttf} use ancillary files which usually have the same file name as the current document but a different extension. It stores temporary data into these files and use them for the next compilation. So to have an up-{}to-{}date table of contents, you need to compile the document twice. There is no need to re-{}compile if no section moved.

For example, the temporary file for the table of contents data is {\ttfamily \setmainfont[Path=/usr/share/fonts/truetype/cmu/,UprightFont=cmunrm.ttf,BoldFont=cmunbx.ttf,ItalicFont=cmunti.ttf,BoldItalicFont=cmunbi.ttf]{cmuntt.ttf}\setmonofont[Path=/usr/share/fonts/truetype/cmu/,UprightFont=cmuntt.ttf,BoldFont=cmuntb.ttf,ItalicFont=cmunit.ttf,BoldItalicFont=cmuntx.ttf]{cmuntt.ttf}\ttfamily filename.toc}\setmainfont[Path=/usr/share/fonts/truetype/cmu/,UprightFont=cmunrm.ttf,BoldFont=cmunbx.ttf,ItalicFont=cmunti.ttf,BoldItalicFont=cmunbi.ttf]{cmunrm.ttf}\setmonofont[Path=/usr/share/fonts/truetype/cmu/,UprightFont=cmuntt.ttf,BoldFont=cmuntb.ttf,ItalicFont=cmunit.ttf,BoldItalicFont=cmuntx.ttf]{cmunrm.ttf}.

None of these files contains unrecoverable information. It means you can delete them safely, compiling will regenerate them automatically.
\begin{TemplateInfo}{\danger}{Warning} The only important file types are {\ttfamily \setmainfont[Path=/usr/share/fonts/truetype/cmu/,UprightFont=cmunrm.ttf,BoldFont=cmunbx.ttf,ItalicFont=cmunti.ttf,BoldItalicFont=cmunbi.ttf]{cmuntt.ttf}\setmonofont[Path=/usr/share/fonts/truetype/cmu/,UprightFont=cmuntt.ttf,BoldFont=cmuntb.ttf,ItalicFont=cmunit.ttf,BoldItalicFont=cmuntx.ttf]{cmuntt.ttf}\ttfamily .tex}\setmainfont[Path=/usr/share/fonts/truetype/cmu/,UprightFont=cmunrm.ttf,BoldFont=cmunbx.ttf,ItalicFont=cmunti.ttf,BoldItalicFont=cmunbi.ttf]{cmunrm.ttf}\setmonofont[Path=/usr/share/fonts/truetype/cmu/,UprightFont=cmuntt.ttf,BoldFont=cmuntb.ttf,ItalicFont=cmunit.ttf,BoldItalicFont=cmuntx.ttf]{cmunrm.ttf}, {\ttfamily \setmainfont[Path=/usr/share/fonts/truetype/cmu/,UprightFont=cmunrm.ttf,BoldFont=cmunbx.ttf,ItalicFont=cmunti.ttf,BoldItalicFont=cmunbi.ttf]{cmuntt.ttf}\setmonofont[Path=/usr/share/fonts/truetype/cmu/,UprightFont=cmuntt.ttf,BoldFont=cmuntb.ttf,ItalicFont=cmunit.ttf,BoldItalicFont=cmuntx.ttf]{cmuntt.ttf}\ttfamily .cls}{$\text{ }$}\setmainfont[Path=/usr/share/fonts/truetype/cmu/,UprightFont=cmunrm.ttf,BoldFont=cmunbx.ttf,ItalicFont=cmunti.ttf,BoldItalicFont=cmunbi.ttf]{cmunrm.ttf}\setmonofont[Path=/usr/share/fonts/truetype/cmu/,UprightFont=cmuntt.ttf,BoldFont=cmuntb.ttf,ItalicFont=cmunit.ttf,BoldItalicFont=cmuntx.ttf]{cmunrm.ttf} and {\ttfamily \setmainfont[Path=/usr/share/fonts/truetype/cmu/,UprightFont=cmunrm.ttf,BoldFont=cmunbx.ttf,ItalicFont=cmunti.ttf,BoldItalicFont=cmunbi.ttf]{cmuntt.ttf}\setmonofont[Path=/usr/share/fonts/truetype/cmu/,UprightFont=cmuntt.ttf,BoldFont=cmuntb.ttf,ItalicFont=cmunit.ttf,BoldItalicFont=cmuntx.ttf]{cmuntt.ttf}\ttfamily .sty}\setmainfont[Path=/usr/share/fonts/truetype/cmu/,UprightFont=cmunrm.ttf,BoldFont=cmunbx.ttf,ItalicFont=cmunti.ttf,BoldItalicFont=cmunbi.ttf]{cmunrm.ttf}\setmonofont[Path=/usr/share/fonts/truetype/cmu/,UprightFont=cmuntt.ttf,BoldFont=cmuntb.ttf,ItalicFont=cmunit.ttf,BoldItalicFont=cmuntx.ttf]{cmunrm.ttf}, {\ttfamily \setmainfont[Path=/usr/share/fonts/truetype/cmu/,UprightFont=cmunrm.ttf,BoldFont=cmunbx.ttf,ItalicFont=cmunti.ttf,BoldItalicFont=cmunbi.ttf]{cmuntt.ttf}\setmonofont[Path=/usr/share/fonts/truetype/cmu/,UprightFont=cmuntt.ttf,BoldFont=cmuntb.ttf,ItalicFont=cmunit.ttf,BoldItalicFont=cmuntx.ttf]{cmuntt.ttf}\ttfamily .bib}{$\text{ }$}\setmainfont[Path=/usr/share/fonts/truetype/cmu/,UprightFont=cmunrm.ttf,BoldFont=cmunbx.ttf,ItalicFont=cmunti.ttf,BoldItalicFont=cmunbi.ttf]{cmunrm.ttf}\setmonofont[Path=/usr/share/fonts/truetype/cmu/,UprightFont=cmuntt.ttf,BoldFont=cmuntb.ttf,ItalicFont=cmunit.ttf,BoldItalicFont=cmuntx.ttf]{cmunrm.ttf} and {\ttfamily \setmainfont[Path=/usr/share/fonts/truetype/cmu/,UprightFont=cmunrm.ttf,BoldFont=cmunbx.ttf,ItalicFont=cmunti.ttf,BoldItalicFont=cmunbi.ttf]{cmuntt.ttf}\setmonofont[Path=/usr/share/fonts/truetype/cmu/,UprightFont=cmuntt.ttf,BoldFont=cmuntb.ttf,ItalicFont=cmunit.ttf,BoldItalicFont=cmuntx.ttf]{cmuntt.ttf}\ttfamily .bst}{$\text{ }$}\setmainfont[Path=/usr/share/fonts/truetype/cmu/,UprightFont=cmunrm.ttf,BoldFont=cmunbx.ttf,ItalicFont=cmunti.ttf,BoldItalicFont=cmunbi.ttf]{cmunrm.ttf}\setmonofont[Path=/usr/share/fonts/truetype/cmu/,UprightFont=cmuntt.ttf,BoldFont=cmuntb.ttf,ItalicFont=cmunit.ttf,BoldItalicFont=cmuntx.ttf]{cmunrm.ttf} for BibTeX, these are not temporary and should not be deleted.\end{TemplateInfo}

When you work with various capabilities of LaTeX (index, glossaries, bibliographies, etc.) you will soon find yourself in a maze of files with various extensions and probably no clue. The following list explains the most common file types you might encounter when working with TeX:

\begin{longtable}{|>{\RaggedRight}p{0.09158\linewidth}|>{\RaggedRight}p{0.85127\linewidth}|} \hline 
\multicolumn{2}{|>{\RaggedRight}p{0.97143\linewidth}|}{{\bfseries \hspace*{0pt}\ignorespaces{}\hspace*{0pt}Common file extensions in LaTeX}}\endhead  \hline \hspace*{0pt}\ignorespaces{}\hspace*{0pt} {\ttfamily \setmainfont[Path=/usr/share/fonts/truetype/cmu/,UprightFont=cmunrm.ttf,BoldFont=cmunbx.ttf,ItalicFont=cmunti.ttf,BoldItalicFont=cmunbi.ttf]{cmuntt.ttf}\setmonofont[Path=/usr/share/fonts/truetype/cmu/,UprightFont=cmuntt.ttf,BoldFont=cmuntb.ttf,ItalicFont=cmunit.ttf,BoldItalicFont=cmuntx.ttf]{cmuntt.ttf}\ttfamily .aux}&\hspace*{0pt}\ignorespaces{}\hspace*{0pt}{$\text{ }$}\setmainfont[Path=/usr/share/fonts/truetype/cmu/,UprightFont=cmunrm.ttf,BoldFont=cmunbx.ttf,ItalicFont=cmunti.ttf,BoldItalicFont=cmunbi.ttf]{cmunrm.ttf}\setmonofont[Path=/usr/share/fonts/truetype/cmu/,UprightFont=cmuntt.ttf,BoldFont=cmuntb.ttf,ItalicFont=cmunit.ttf,BoldItalicFont=cmuntx.ttf]{cmunrm.ttf} A file that transports information from one compiler run to the next. Among other things, the {\ttfamily \setmainfont[Path=/usr/share/fonts/truetype/cmu/,UprightFont=cmunrm.ttf,BoldFont=cmunbx.ttf,ItalicFont=cmunti.ttf,BoldItalicFont=cmunbi.ttf]{cmuntt.ttf}\setmonofont[Path=/usr/share/fonts/truetype/cmu/,UprightFont=cmuntt.ttf,BoldFont=cmuntb.ttf,ItalicFont=cmunit.ttf,BoldItalicFont=cmuntx.ttf]{cmuntt.ttf}\ttfamily .aux}{$\text{ }$}\setmainfont[Path=/usr/share/fonts/truetype/cmu/,UprightFont=cmunrm.ttf,BoldFont=cmunbx.ttf,ItalicFont=cmunti.ttf,BoldItalicFont=cmunbi.ttf]{cmunrm.ttf}\setmonofont[Path=/usr/share/fonts/truetype/cmu/,UprightFont=cmuntt.ttf,BoldFont=cmuntb.ttf,ItalicFont=cmunit.ttf,BoldItalicFont=cmuntx.ttf]{cmunrm.ttf} file is used to store information associated with cross-{}references.\\ \hline \hspace*{0pt}\ignorespaces{}\hspace*{0pt} {\ttfamily \setmainfont[Path=/usr/share/fonts/truetype/cmu/,UprightFont=cmunrm.ttf,BoldFont=cmunbx.ttf,ItalicFont=cmunti.ttf,BoldItalicFont=cmunbi.ttf]{cmuntt.ttf}\setmonofont[Path=/usr/share/fonts/truetype/cmu/,UprightFont=cmuntt.ttf,BoldFont=cmuntb.ttf,ItalicFont=cmunit.ttf,BoldItalicFont=cmuntx.ttf]{cmuntt.ttf}\ttfamily .bbl}&\hspace*{0pt}\ignorespaces{}\hspace*{0pt}{$\text{ }$}\setmainfont[Path=/usr/share/fonts/truetype/cmu/,UprightFont=cmunrm.ttf,BoldFont=cmunbx.ttf,ItalicFont=cmunti.ttf,BoldItalicFont=cmunbi.ttf]{cmunrm.ttf}\setmonofont[Path=/usr/share/fonts/truetype/cmu/,UprightFont=cmuntt.ttf,BoldFont=cmuntb.ttf,ItalicFont=cmunit.ttf,BoldItalicFont=cmuntx.ttf]{cmunrm.ttf} Bibliography file output by BiBTeX and used by LaTeX\\ \hline \hspace*{0pt}\ignorespaces{}\hspace*{0pt} {\ttfamily \setmainfont[Path=/usr/share/fonts/truetype/cmu/,UprightFont=cmunrm.ttf,BoldFont=cmunbx.ttf,ItalicFont=cmunti.ttf,BoldItalicFont=cmunbi.ttf]{cmuntt.ttf}\setmonofont[Path=/usr/share/fonts/truetype/cmu/,UprightFont=cmuntt.ttf,BoldFont=cmuntb.ttf,ItalicFont=cmunit.ttf,BoldItalicFont=cmuntx.ttf]{cmuntt.ttf}\ttfamily .bib}{$\text{ }$}\setmainfont[Path=/usr/share/fonts/truetype/cmu/,UprightFont=cmunrm.ttf,BoldFont=cmunbx.ttf,ItalicFont=cmunti.ttf,BoldItalicFont=cmunbi.ttf]{cmunrm.ttf}\setmonofont[Path=/usr/share/fonts/truetype/cmu/,UprightFont=cmuntt.ttf,BoldFont=cmuntb.ttf,ItalicFont=cmunit.ttf,BoldItalicFont=cmuntx.ttf]{cmunrm.ttf} &\hspace*{0pt}\ignorespaces{}\hspace*{0pt} Bibliography database file. (where you can store a list of full bibliographic citations) \\ \hline \hspace*{0pt}\ignorespaces{}\hspace*{0pt} {\ttfamily \setmainfont[Path=/usr/share/fonts/truetype/cmu/,UprightFont=cmunrm.ttf,BoldFont=cmunbx.ttf,ItalicFont=cmunti.ttf,BoldItalicFont=cmunbi.ttf]{cmuntt.ttf}\setmonofont[Path=/usr/share/fonts/truetype/cmu/,UprightFont=cmuntt.ttf,BoldFont=cmuntb.ttf,ItalicFont=cmunit.ttf,BoldItalicFont=cmuntx.ttf]{cmuntt.ttf}\ttfamily .blg}&\hspace*{0pt}\ignorespaces{}\hspace*{0pt}{$\text{ }$}\setmainfont[Path=/usr/share/fonts/truetype/cmu/,UprightFont=cmunrm.ttf,BoldFont=cmunbx.ttf,ItalicFont=cmunti.ttf,BoldItalicFont=cmunbi.ttf]{cmunrm.ttf}\setmonofont[Path=/usr/share/fonts/truetype/cmu/,UprightFont=cmuntt.ttf,BoldFont=cmuntb.ttf,ItalicFont=cmunit.ttf,BoldItalicFont=cmuntx.ttf]{cmunrm.ttf} BiBTeX log file. (errors are logged here)\\ \hline \hspace*{0pt}\ignorespaces{}\hspace*{0pt} {\ttfamily \setmainfont[Path=/usr/share/fonts/truetype/cmu/,UprightFont=cmunrm.ttf,BoldFont=cmunbx.ttf,ItalicFont=cmunti.ttf,BoldItalicFont=cmunbi.ttf]{cmuntt.ttf}\setmonofont[Path=/usr/share/fonts/truetype/cmu/,UprightFont=cmuntt.ttf,BoldFont=cmuntb.ttf,ItalicFont=cmunit.ttf,BoldItalicFont=cmuntx.ttf]{cmuntt.ttf}\ttfamily .bst}&\hspace*{0pt}\ignorespaces{}\hspace*{0pt}{$\text{ }$}\setmainfont[Path=/usr/share/fonts/truetype/cmu/,UprightFont=cmunrm.ttf,BoldFont=cmunbx.ttf,ItalicFont=cmunti.ttf,BoldItalicFont=cmunbi.ttf]{cmunrm.ttf}\setmonofont[Path=/usr/share/fonts/truetype/cmu/,UprightFont=cmuntt.ttf,BoldFont=cmuntb.ttf,ItalicFont=cmunit.ttf,BoldItalicFont=cmuntx.ttf]{cmunrm.ttf} BiBTeX style file.\\ \hline \hspace*{0pt}\ignorespaces{}\hspace*{0pt} {\ttfamily \setmainfont[Path=/usr/share/fonts/truetype/cmu/,UprightFont=cmunrm.ttf,BoldFont=cmunbx.ttf,ItalicFont=cmunti.ttf,BoldItalicFont=cmunbi.ttf]{cmuntt.ttf}\setmonofont[Path=/usr/share/fonts/truetype/cmu/,UprightFont=cmuntt.ttf,BoldFont=cmuntb.ttf,ItalicFont=cmunit.ttf,BoldItalicFont=cmuntx.ttf]{cmuntt.ttf}\ttfamily .cls}&\hspace*{0pt}\ignorespaces{}\hspace*{0pt}{$\text{ }$}\setmainfont[Path=/usr/share/fonts/truetype/cmu/,UprightFont=cmunrm.ttf,BoldFont=cmunbx.ttf,ItalicFont=cmunti.ttf,BoldItalicFont=cmunbi.ttf]{cmunrm.ttf}\setmonofont[Path=/usr/share/fonts/truetype/cmu/,UprightFont=cmuntt.ttf,BoldFont=cmuntb.ttf,ItalicFont=cmunit.ttf,BoldItalicFont=cmuntx.ttf]{cmunrm.ttf} Class files define what your document looks like. They are selected with the {\ttfamily \setmainfont[Path=/usr/share/fonts/truetype/cmu/,UprightFont=cmunrm.ttf,BoldFont=cmunbx.ttf,ItalicFont=cmunti.ttf,BoldItalicFont=cmunbi.ttf]{cmuntt.ttf}\setmonofont[Path=/usr/share/fonts/truetype/cmu/,UprightFont=cmuntt.ttf,BoldFont=cmuntb.ttf,ItalicFont=cmunit.ttf,BoldItalicFont=cmuntx.ttf]{cmuntt.ttf}\ttfamily \textbackslash{}documentclass}{$\text{ }$}\setmainfont[Path=/usr/share/fonts/truetype/cmu/,UprightFont=cmunrm.ttf,BoldFont=cmunbx.ttf,ItalicFont=cmunti.ttf,BoldItalicFont=cmunbi.ttf]{cmunrm.ttf}\setmonofont[Path=/usr/share/fonts/truetype/cmu/,UprightFont=cmuntt.ttf,BoldFont=cmuntb.ttf,ItalicFont=cmunit.ttf,BoldItalicFont=cmuntx.ttf]{cmunrm.ttf} command.\\ \hline \hspace*{0pt}\ignorespaces{}\hspace*{0pt} {\ttfamily \setmainfont[Path=/usr/share/fonts/truetype/cmu/,UprightFont=cmunrm.ttf,BoldFont=cmunbx.ttf,ItalicFont=cmunti.ttf,BoldItalicFont=cmunbi.ttf]{cmuntt.ttf}\setmonofont[Path=/usr/share/fonts/truetype/cmu/,UprightFont=cmuntt.ttf,BoldFont=cmuntb.ttf,ItalicFont=cmunit.ttf,BoldItalicFont=cmuntx.ttf]{cmuntt.ttf}\ttfamily .dtx}&\hspace*{0pt}\ignorespaces{}\hspace*{0pt}{$\text{ }$}\setmainfont[Path=/usr/share/fonts/truetype/cmu/,UprightFont=cmunrm.ttf,BoldFont=cmunbx.ttf,ItalicFont=cmunti.ttf,BoldItalicFont=cmunbi.ttf]{cmunrm.ttf}\setmonofont[Path=/usr/share/fonts/truetype/cmu/,UprightFont=cmuntt.ttf,BoldFont=cmuntb.ttf,ItalicFont=cmunit.ttf,BoldItalicFont=cmuntx.ttf]{cmunrm.ttf} Documented TeX. This is the main distribution format for LaTeX style files. If you process a {\ttfamily \setmainfont[Path=/usr/share/fonts/truetype/cmu/,UprightFont=cmunrm.ttf,BoldFont=cmunbx.ttf,ItalicFont=cmunti.ttf,BoldItalicFont=cmunbi.ttf]{cmuntt.ttf}\setmonofont[Path=/usr/share/fonts/truetype/cmu/,UprightFont=cmuntt.ttf,BoldFont=cmuntb.ttf,ItalicFont=cmunit.ttf,BoldItalicFont=cmuntx.ttf]{cmuntt.ttf}\ttfamily .dtx}{$\text{ }$}\setmainfont[Path=/usr/share/fonts/truetype/cmu/,UprightFont=cmunrm.ttf,BoldFont=cmunbx.ttf,ItalicFont=cmunti.ttf,BoldItalicFont=cmunbi.ttf]{cmunrm.ttf}\setmonofont[Path=/usr/share/fonts/truetype/cmu/,UprightFont=cmuntt.ttf,BoldFont=cmuntb.ttf,ItalicFont=cmunit.ttf,BoldItalicFont=cmuntx.ttf]{cmunrm.ttf} file you get documented macro code of the LaTeX package contained in the {\ttfamily \setmainfont[Path=/usr/share/fonts/truetype/cmu/,UprightFont=cmunrm.ttf,BoldFont=cmunbx.ttf,ItalicFont=cmunti.ttf,BoldItalicFont=cmunbi.ttf]{cmuntt.ttf}\setmonofont[Path=/usr/share/fonts/truetype/cmu/,UprightFont=cmuntt.ttf,BoldFont=cmuntb.ttf,ItalicFont=cmunit.ttf,BoldItalicFont=cmuntx.ttf]{cmuntt.ttf}\ttfamily .dtx}{$\text{ }$}\setmainfont[Path=/usr/share/fonts/truetype/cmu/,UprightFont=cmunrm.ttf,BoldFont=cmunbx.ttf,ItalicFont=cmunti.ttf,BoldItalicFont=cmunbi.ttf]{cmunrm.ttf}\setmonofont[Path=/usr/share/fonts/truetype/cmu/,UprightFont=cmuntt.ttf,BoldFont=cmuntb.ttf,ItalicFont=cmunit.ttf,BoldItalicFont=cmuntx.ttf]{cmunrm.ttf} file.\\ \hline \hspace*{0pt}\ignorespaces{}\hspace*{0pt} {\ttfamily \setmainfont[Path=/usr/share/fonts/truetype/cmu/,UprightFont=cmunrm.ttf,BoldFont=cmunbx.ttf,ItalicFont=cmunti.ttf,BoldItalicFont=cmunbi.ttf]{cmuntt.ttf}\setmonofont[Path=/usr/share/fonts/truetype/cmu/,UprightFont=cmuntt.ttf,BoldFont=cmuntb.ttf,ItalicFont=cmunit.ttf,BoldItalicFont=cmuntx.ttf]{cmuntt.ttf}\ttfamily .ins}&\hspace*{0pt}\ignorespaces{}\hspace*{0pt}{$\text{ }$}\setmainfont[Path=/usr/share/fonts/truetype/cmu/,UprightFont=cmunrm.ttf,BoldFont=cmunbx.ttf,ItalicFont=cmunti.ttf,BoldItalicFont=cmunbi.ttf]{cmunrm.ttf}\setmonofont[Path=/usr/share/fonts/truetype/cmu/,UprightFont=cmuntt.ttf,BoldFont=cmuntb.ttf,ItalicFont=cmunit.ttf,BoldItalicFont=cmuntx.ttf]{cmunrm.ttf} The installer for the files contained in the matching {\ttfamily \setmainfont[Path=/usr/share/fonts/truetype/cmu/,UprightFont=cmunrm.ttf,BoldFont=cmunbx.ttf,ItalicFont=cmunti.ttf,BoldItalicFont=cmunbi.ttf]{cmuntt.ttf}\setmonofont[Path=/usr/share/fonts/truetype/cmu/,UprightFont=cmuntt.ttf,BoldFont=cmuntb.ttf,ItalicFont=cmunit.ttf,BoldItalicFont=cmuntx.ttf]{cmuntt.ttf}\ttfamily .dtx}{$\text{ }$}\setmainfont[Path=/usr/share/fonts/truetype/cmu/,UprightFont=cmunrm.ttf,BoldFont=cmunbx.ttf,ItalicFont=cmunti.ttf,BoldItalicFont=cmunbi.ttf]{cmunrm.ttf}\setmonofont[Path=/usr/share/fonts/truetype/cmu/,UprightFont=cmuntt.ttf,BoldFont=cmuntb.ttf,ItalicFont=cmunit.ttf,BoldItalicFont=cmuntx.ttf]{cmunrm.ttf} file. If you download a LaTeX package from the net, you will normally get a {\ttfamily \setmainfont[Path=/usr/share/fonts/truetype/cmu/,UprightFont=cmunrm.ttf,BoldFont=cmunbx.ttf,ItalicFont=cmunti.ttf,BoldItalicFont=cmunbi.ttf]{cmuntt.ttf}\setmonofont[Path=/usr/share/fonts/truetype/cmu/,UprightFont=cmuntt.ttf,BoldFont=cmuntb.ttf,ItalicFont=cmunit.ttf,BoldItalicFont=cmuntx.ttf]{cmuntt.ttf}\ttfamily .dtx}{$\text{ }$}\setmainfont[Path=/usr/share/fonts/truetype/cmu/,UprightFont=cmunrm.ttf,BoldFont=cmunbx.ttf,ItalicFont=cmunti.ttf,BoldItalicFont=cmunbi.ttf]{cmunrm.ttf}\setmonofont[Path=/usr/share/fonts/truetype/cmu/,UprightFont=cmuntt.ttf,BoldFont=cmuntb.ttf,ItalicFont=cmunit.ttf,BoldItalicFont=cmuntx.ttf]{cmunrm.ttf} and a {\ttfamily \setmainfont[Path=/usr/share/fonts/truetype/cmu/,UprightFont=cmunrm.ttf,BoldFont=cmunbx.ttf,ItalicFont=cmunti.ttf,BoldItalicFont=cmunbi.ttf]{cmuntt.ttf}\setmonofont[Path=/usr/share/fonts/truetype/cmu/,UprightFont=cmuntt.ttf,BoldFont=cmuntb.ttf,ItalicFont=cmunit.ttf,BoldItalicFont=cmuntx.ttf]{cmuntt.ttf}\ttfamily .ins}{$\text{ }$}\setmainfont[Path=/usr/share/fonts/truetype/cmu/,UprightFont=cmunrm.ttf,BoldFont=cmunbx.ttf,ItalicFont=cmunti.ttf,BoldItalicFont=cmunbi.ttf]{cmunrm.ttf}\setmonofont[Path=/usr/share/fonts/truetype/cmu/,UprightFont=cmuntt.ttf,BoldFont=cmuntb.ttf,ItalicFont=cmunit.ttf,BoldItalicFont=cmuntx.ttf]{cmunrm.ttf} file. Run LaTeX on the {\ttfamily \setmainfont[Path=/usr/share/fonts/truetype/cmu/,UprightFont=cmunrm.ttf,BoldFont=cmunbx.ttf,ItalicFont=cmunti.ttf,BoldItalicFont=cmunbi.ttf]{cmuntt.ttf}\setmonofont[Path=/usr/share/fonts/truetype/cmu/,UprightFont=cmuntt.ttf,BoldFont=cmuntb.ttf,ItalicFont=cmunit.ttf,BoldItalicFont=cmuntx.ttf]{cmuntt.ttf}\ttfamily .ins}{$\text{ }$}\setmainfont[Path=/usr/share/fonts/truetype/cmu/,UprightFont=cmunrm.ttf,BoldFont=cmunbx.ttf,ItalicFont=cmunti.ttf,BoldItalicFont=cmunbi.ttf]{cmunrm.ttf}\setmonofont[Path=/usr/share/fonts/truetype/cmu/,UprightFont=cmuntt.ttf,BoldFont=cmuntb.ttf,ItalicFont=cmunit.ttf,BoldItalicFont=cmuntx.ttf]{cmunrm.ttf} file to unpack the {\ttfamily \setmainfont[Path=/usr/share/fonts/truetype/cmu/,UprightFont=cmunrm.ttf,BoldFont=cmunbx.ttf,ItalicFont=cmunti.ttf,BoldItalicFont=cmunbi.ttf]{cmuntt.ttf}\setmonofont[Path=/usr/share/fonts/truetype/cmu/,UprightFont=cmuntt.ttf,BoldFont=cmuntb.ttf,ItalicFont=cmunit.ttf,BoldItalicFont=cmuntx.ttf]{cmuntt.ttf}\ttfamily .dtx}{$\text{ }$}\setmainfont[Path=/usr/share/fonts/truetype/cmu/,UprightFont=cmunrm.ttf,BoldFont=cmunbx.ttf,ItalicFont=cmunti.ttf,BoldItalicFont=cmunbi.ttf]{cmunrm.ttf}\setmonofont[Path=/usr/share/fonts/truetype/cmu/,UprightFont=cmuntt.ttf,BoldFont=cmuntb.ttf,ItalicFont=cmunit.ttf,BoldItalicFont=cmuntx.ttf]{cmunrm.ttf} file.\\ \hline \hspace*{0pt}\ignorespaces{}\hspace*{0pt} {\ttfamily \setmainfont[Path=/usr/share/fonts/truetype/cmu/,UprightFont=cmunrm.ttf,BoldFont=cmunbx.ttf,ItalicFont=cmunti.ttf,BoldItalicFont=cmunbi.ttf]{cmuntt.ttf}\setmonofont[Path=/usr/share/fonts/truetype/cmu/,UprightFont=cmuntt.ttf,BoldFont=cmuntb.ttf,ItalicFont=cmunit.ttf,BoldItalicFont=cmuntx.ttf]{cmuntt.ttf}\ttfamily .fd}&\hspace*{0pt}\ignorespaces{}\hspace*{0pt}{$\text{ }$}\setmainfont[Path=/usr/share/fonts/truetype/cmu/,UprightFont=cmunrm.ttf,BoldFont=cmunbx.ttf,ItalicFont=cmunti.ttf,BoldItalicFont=cmunbi.ttf]{cmunrm.ttf}\setmonofont[Path=/usr/share/fonts/truetype/cmu/,UprightFont=cmuntt.ttf,BoldFont=cmuntb.ttf,ItalicFont=cmunit.ttf,BoldItalicFont=cmuntx.ttf]{cmunrm.ttf} Font description file telling LaTeX about new fonts.\\ \hline \hspace*{0pt}\ignorespaces{}\hspace*{0pt} {\ttfamily \setmainfont[Path=/usr/share/fonts/truetype/cmu/,UprightFont=cmunrm.ttf,BoldFont=cmunbx.ttf,ItalicFont=cmunti.ttf,BoldItalicFont=cmunbi.ttf]{cmuntt.ttf}\setmonofont[Path=/usr/share/fonts/truetype/cmu/,UprightFont=cmuntt.ttf,BoldFont=cmuntb.ttf,ItalicFont=cmunit.ttf,BoldItalicFont=cmuntx.ttf]{cmuntt.ttf}\ttfamily .dvi}&\hspace*{0pt}\ignorespaces{}\hspace*{0pt}{$\text{ }$}\setmainfont[Path=/usr/share/fonts/truetype/cmu/,UprightFont=cmunrm.ttf,BoldFont=cmunbx.ttf,ItalicFont=cmunti.ttf,BoldItalicFont=cmunbi.ttf]{cmunrm.ttf}\setmonofont[Path=/usr/share/fonts/truetype/cmu/,UprightFont=cmuntt.ttf,BoldFont=cmuntb.ttf,ItalicFont=cmunit.ttf,BoldItalicFont=cmuntx.ttf]{cmunrm.ttf} Device Independent File. This is the main result of a LaTeX compile run with {\itshape \setmainfont[Path=/usr/share/fonts/truetype/cmu/,UprightFont=cmunrm.ttf,BoldFont=cmunbx.ttf,ItalicFont=cmunti.ttf,BoldItalicFont=cmunbi.ttf]{cmunti.ttf}\setmonofont[Path=/usr/share/fonts/truetype/cmu/,UprightFont=cmuntt.ttf,BoldFont=cmuntb.ttf,ItalicFont=cmunit.ttf,BoldItalicFont=cmuntx.ttf]{cmunti.ttf}\itshape latex}\setmainfont[Path=/usr/share/fonts/truetype/cmu/,UprightFont=cmunrm.ttf,BoldFont=cmunbx.ttf,ItalicFont=cmunti.ttf,BoldItalicFont=cmunbi.ttf]{cmunrm.ttf}\setmonofont[Path=/usr/share/fonts/truetype/cmu/,UprightFont=cmuntt.ttf,BoldFont=cmuntb.ttf,ItalicFont=cmunit.ttf,BoldItalicFont=cmuntx.ttf]{cmunrm.ttf}. You can look at its content with a DVI previewer program or you can send it to a printer with dvips or a similar application.\\ \hline \hspace*{0pt}\ignorespaces{}\hspace*{0pt} {\ttfamily \setmainfont[Path=/usr/share/fonts/truetype/cmu/,UprightFont=cmunrm.ttf,BoldFont=cmunbx.ttf,ItalicFont=cmunti.ttf,BoldItalicFont=cmunbi.ttf]{cmuntt.ttf}\setmonofont[Path=/usr/share/fonts/truetype/cmu/,UprightFont=cmuntt.ttf,BoldFont=cmuntb.ttf,ItalicFont=cmunit.ttf,BoldItalicFont=cmuntx.ttf]{cmuntt.ttf}\ttfamily .pdf}&\hspace*{0pt}\ignorespaces{}\hspace*{0pt}{$\text{ }$}\setmainfont[Path=/usr/share/fonts/truetype/cmu/,UprightFont=cmunrm.ttf,BoldFont=cmunbx.ttf,ItalicFont=cmunti.ttf,BoldItalicFont=cmunbi.ttf]{cmunrm.ttf}\setmonofont[Path=/usr/share/fonts/truetype/cmu/,UprightFont=cmuntt.ttf,BoldFont=cmuntb.ttf,ItalicFont=cmunit.ttf,BoldItalicFont=cmuntx.ttf]{cmunrm.ttf} Portable Document Format. This is the main result of a LaTeX compile run with {\itshape \setmainfont[Path=/usr/share/fonts/truetype/cmu/,UprightFont=cmunrm.ttf,BoldFont=cmunbx.ttf,ItalicFont=cmunti.ttf,BoldItalicFont=cmunbi.ttf]{cmunti.ttf}\setmonofont[Path=/usr/share/fonts/truetype/cmu/,UprightFont=cmuntt.ttf,BoldFont=cmuntb.ttf,ItalicFont=cmunit.ttf,BoldItalicFont=cmuntx.ttf]{cmunti.ttf}\itshape pdflatex}\setmainfont[Path=/usr/share/fonts/truetype/cmu/,UprightFont=cmunrm.ttf,BoldFont=cmunbx.ttf,ItalicFont=cmunti.ttf,BoldItalicFont=cmunbi.ttf]{cmunrm.ttf}\setmonofont[Path=/usr/share/fonts/truetype/cmu/,UprightFont=cmuntt.ttf,BoldFont=cmuntb.ttf,ItalicFont=cmunit.ttf,BoldItalicFont=cmuntx.ttf]{cmunrm.ttf}. You can look at its content or print it with any PDF viewer.\\ \hline \hspace*{0pt}\ignorespaces{}\hspace*{0pt} {\ttfamily \setmainfont[Path=/usr/share/fonts/truetype/cmu/,UprightFont=cmunrm.ttf,BoldFont=cmunbx.ttf,ItalicFont=cmunti.ttf,BoldItalicFont=cmunbi.ttf]{cmuntt.ttf}\setmonofont[Path=/usr/share/fonts/truetype/cmu/,UprightFont=cmuntt.ttf,BoldFont=cmuntb.ttf,ItalicFont=cmunit.ttf,BoldItalicFont=cmuntx.ttf]{cmuntt.ttf}\ttfamily .log}&\hspace*{0pt}\ignorespaces{}\hspace*{0pt}{$\text{ }$}\setmainfont[Path=/usr/share/fonts/truetype/cmu/,UprightFont=cmunrm.ttf,BoldFont=cmunbx.ttf,ItalicFont=cmunti.ttf,BoldItalicFont=cmunbi.ttf]{cmunrm.ttf}\setmonofont[Path=/usr/share/fonts/truetype/cmu/,UprightFont=cmuntt.ttf,BoldFont=cmuntb.ttf,ItalicFont=cmunit.ttf,BoldItalicFont=cmuntx.ttf]{cmunrm.ttf} Gives a detailed account of what happened during the last compiler run.\\ \hline \hspace*{0pt}\ignorespaces{}\hspace*{0pt} {\ttfamily \setmainfont[Path=/usr/share/fonts/truetype/cmu/,UprightFont=cmunrm.ttf,BoldFont=cmunbx.ttf,ItalicFont=cmunti.ttf,BoldItalicFont=cmunbi.ttf]{cmuntt.ttf}\setmonofont[Path=/usr/share/fonts/truetype/cmu/,UprightFont=cmuntt.ttf,BoldFont=cmuntb.ttf,ItalicFont=cmunit.ttf,BoldItalicFont=cmuntx.ttf]{cmuntt.ttf}\ttfamily .toc}&\hspace*{0pt}\ignorespaces{}\hspace*{0pt}{$\text{ }$}\setmainfont[Path=/usr/share/fonts/truetype/cmu/,UprightFont=cmunrm.ttf,BoldFont=cmunbx.ttf,ItalicFont=cmunti.ttf,BoldItalicFont=cmunbi.ttf]{cmunrm.ttf}\setmonofont[Path=/usr/share/fonts/truetype/cmu/,UprightFont=cmuntt.ttf,BoldFont=cmuntb.ttf,ItalicFont=cmunit.ttf,BoldItalicFont=cmuntx.ttf]{cmunrm.ttf} Stores all your section headers. It gets read in for the next compiler run and is used to produce the table of contents.\\ \hline \hspace*{0pt}\ignorespaces{}\hspace*{0pt} {\ttfamily \setmainfont[Path=/usr/share/fonts/truetype/cmu/,UprightFont=cmunrm.ttf,BoldFont=cmunbx.ttf,ItalicFont=cmunti.ttf,BoldItalicFont=cmunbi.ttf]{cmuntt.ttf}\setmonofont[Path=/usr/share/fonts/truetype/cmu/,UprightFont=cmuntt.ttf,BoldFont=cmuntb.ttf,ItalicFont=cmunit.ttf,BoldItalicFont=cmuntx.ttf]{cmuntt.ttf}\ttfamily .lof}&\hspace*{0pt}\ignorespaces{}\hspace*{0pt}{$\text{ }$}\setmainfont[Path=/usr/share/fonts/truetype/cmu/,UprightFont=cmunrm.ttf,BoldFont=cmunbx.ttf,ItalicFont=cmunti.ttf,BoldItalicFont=cmunbi.ttf]{cmunrm.ttf}\setmonofont[Path=/usr/share/fonts/truetype/cmu/,UprightFont=cmuntt.ttf,BoldFont=cmuntb.ttf,ItalicFont=cmunit.ttf,BoldItalicFont=cmuntx.ttf]{cmunrm.ttf} This is like {\ttfamily \setmainfont[Path=/usr/share/fonts/truetype/cmu/,UprightFont=cmunrm.ttf,BoldFont=cmunbx.ttf,ItalicFont=cmunti.ttf,BoldItalicFont=cmunbi.ttf]{cmuntt.ttf}\setmonofont[Path=/usr/share/fonts/truetype/cmu/,UprightFont=cmuntt.ttf,BoldFont=cmuntb.ttf,ItalicFont=cmunit.ttf,BoldItalicFont=cmuntx.ttf]{cmuntt.ttf}\ttfamily .toc}{$\text{ }$}\setmainfont[Path=/usr/share/fonts/truetype/cmu/,UprightFont=cmunrm.ttf,BoldFont=cmunbx.ttf,ItalicFont=cmunti.ttf,BoldItalicFont=cmunbi.ttf]{cmunrm.ttf}\setmonofont[Path=/usr/share/fonts/truetype/cmu/,UprightFont=cmuntt.ttf,BoldFont=cmuntb.ttf,ItalicFont=cmunit.ttf,BoldItalicFont=cmuntx.ttf]{cmunrm.ttf} but for the list of figures.\\ \hline \hspace*{0pt}\ignorespaces{}\hspace*{0pt} {\ttfamily \setmainfont[Path=/usr/share/fonts/truetype/cmu/,UprightFont=cmunrm.ttf,BoldFont=cmunbx.ttf,ItalicFont=cmunti.ttf,BoldItalicFont=cmunbi.ttf]{cmuntt.ttf}\setmonofont[Path=/usr/share/fonts/truetype/cmu/,UprightFont=cmuntt.ttf,BoldFont=cmuntb.ttf,ItalicFont=cmunit.ttf,BoldItalicFont=cmuntx.ttf]{cmuntt.ttf}\ttfamily .lot}&\hspace*{0pt}\ignorespaces{}\hspace*{0pt}{$\text{ }$}\setmainfont[Path=/usr/share/fonts/truetype/cmu/,UprightFont=cmunrm.ttf,BoldFont=cmunbx.ttf,ItalicFont=cmunti.ttf,BoldItalicFont=cmunbi.ttf]{cmunrm.ttf}\setmonofont[Path=/usr/share/fonts/truetype/cmu/,UprightFont=cmuntt.ttf,BoldFont=cmuntb.ttf,ItalicFont=cmunit.ttf,BoldItalicFont=cmuntx.ttf]{cmunrm.ttf} And again the same for the list of tables.\\ \hline \hspace*{0pt}\ignorespaces{}\hspace*{0pt} {\ttfamily \setmainfont[Path=/usr/share/fonts/truetype/cmu/,UprightFont=cmunrm.ttf,BoldFont=cmunbx.ttf,ItalicFont=cmunti.ttf,BoldItalicFont=cmunbi.ttf]{cmuntt.ttf}\setmonofont[Path=/usr/share/fonts/truetype/cmu/,UprightFont=cmuntt.ttf,BoldFont=cmuntb.ttf,ItalicFont=cmunit.ttf,BoldItalicFont=cmuntx.ttf]{cmuntt.ttf}\ttfamily .idx}&\hspace*{0pt}\ignorespaces{}\hspace*{0pt}{$\text{ }$}\setmainfont[Path=/usr/share/fonts/truetype/cmu/,UprightFont=cmunrm.ttf,BoldFont=cmunbx.ttf,ItalicFont=cmunti.ttf,BoldItalicFont=cmunbi.ttf]{cmunrm.ttf}\setmonofont[Path=/usr/share/fonts/truetype/cmu/,UprightFont=cmuntt.ttf,BoldFont=cmuntb.ttf,ItalicFont=cmunit.ttf,BoldItalicFont=cmuntx.ttf]{cmunrm.ttf} If your document contains an index. LaTeX stores all the words that go into the index in this file. Process this file with makeindex.\\ \hline \hspace*{0pt}\ignorespaces{}\hspace*{0pt} {\ttfamily \setmainfont[Path=/usr/share/fonts/truetype/cmu/,UprightFont=cmunrm.ttf,BoldFont=cmunbx.ttf,ItalicFont=cmunti.ttf,BoldItalicFont=cmunbi.ttf]{cmuntt.ttf}\setmonofont[Path=/usr/share/fonts/truetype/cmu/,UprightFont=cmuntt.ttf,BoldFont=cmuntb.ttf,ItalicFont=cmunit.ttf,BoldItalicFont=cmuntx.ttf]{cmuntt.ttf}\ttfamily .ind}&\hspace*{0pt}\ignorespaces{}\hspace*{0pt}{$\text{ }$}\setmainfont[Path=/usr/share/fonts/truetype/cmu/,UprightFont=cmunrm.ttf,BoldFont=cmunbx.ttf,ItalicFont=cmunti.ttf,BoldItalicFont=cmunbi.ttf]{cmunrm.ttf}\setmonofont[Path=/usr/share/fonts/truetype/cmu/,UprightFont=cmuntt.ttf,BoldFont=cmuntb.ttf,ItalicFont=cmunit.ttf,BoldItalicFont=cmuntx.ttf]{cmunrm.ttf} The processed .idx file, ready for inclusion into your document on the next compile cycle.\\ \hline \hspace*{0pt}\ignorespaces{}\hspace*{0pt} {\ttfamily \setmainfont[Path=/usr/share/fonts/truetype/cmu/,UprightFont=cmunrm.ttf,BoldFont=cmunbx.ttf,ItalicFont=cmunti.ttf,BoldItalicFont=cmunbi.ttf]{cmuntt.ttf}\setmonofont[Path=/usr/share/fonts/truetype/cmu/,UprightFont=cmuntt.ttf,BoldFont=cmuntb.ttf,ItalicFont=cmunit.ttf,BoldItalicFont=cmuntx.ttf]{cmuntt.ttf}\ttfamily .ilg}&\hspace*{0pt}\ignorespaces{}\hspace*{0pt}{$\text{ }$}\setmainfont[Path=/usr/share/fonts/truetype/cmu/,UprightFont=cmunrm.ttf,BoldFont=cmunbx.ttf,ItalicFont=cmunti.ttf,BoldItalicFont=cmunbi.ttf]{cmunrm.ttf}\setmonofont[Path=/usr/share/fonts/truetype/cmu/,UprightFont=cmuntt.ttf,BoldFont=cmuntb.ttf,ItalicFont=cmunit.ttf,BoldItalicFont=cmuntx.ttf]{cmunrm.ttf} Logfile telling what makeindex did.\\ \hline \hspace*{0pt}\ignorespaces{}\hspace*{0pt} {\ttfamily \setmainfont[Path=/usr/share/fonts/truetype/cmu/,UprightFont=cmunrm.ttf,BoldFont=cmunbx.ttf,ItalicFont=cmunti.ttf,BoldItalicFont=cmunbi.ttf]{cmuntt.ttf}\setmonofont[Path=/usr/share/fonts/truetype/cmu/,UprightFont=cmuntt.ttf,BoldFont=cmuntb.ttf,ItalicFont=cmunit.ttf,BoldItalicFont=cmuntx.ttf]{cmuntt.ttf}\ttfamily .sty}&\hspace*{0pt}\ignorespaces{}\hspace*{0pt}{$\text{ }$}\setmainfont[Path=/usr/share/fonts/truetype/cmu/,UprightFont=cmunrm.ttf,BoldFont=cmunbx.ttf,ItalicFont=cmunti.ttf,BoldItalicFont=cmunbi.ttf]{cmunrm.ttf}\setmonofont[Path=/usr/share/fonts/truetype/cmu/,UprightFont=cmuntt.ttf,BoldFont=cmuntb.ttf,ItalicFont=cmunit.ttf,BoldItalicFont=cmuntx.ttf]{cmunrm.ttf} LaTeX Macro package. This is a file you can load into your LaTeX document using the \textbackslash{}usepackage command.\\ \hline \hspace*{0pt}\ignorespaces{}\hspace*{0pt} {\ttfamily \setmainfont[Path=/usr/share/fonts/truetype/cmu/,UprightFont=cmunrm.ttf,BoldFont=cmunbx.ttf,ItalicFont=cmunti.ttf,BoldItalicFont=cmunbi.ttf]{cmuntt.ttf}\setmonofont[Path=/usr/share/fonts/truetype/cmu/,UprightFont=cmuntt.ttf,BoldFont=cmuntb.ttf,ItalicFont=cmunit.ttf,BoldItalicFont=cmuntx.ttf]{cmuntt.ttf}\ttfamily .tex}&\hspace*{0pt}\ignorespaces{}\hspace*{0pt}{$\text{ }$}\setmainfont[Path=/usr/share/fonts/truetype/cmu/,UprightFont=cmunrm.ttf,BoldFont=cmunbx.ttf,ItalicFont=cmunti.ttf,BoldItalicFont=cmunbi.ttf]{cmunrm.ttf}\setmonofont[Path=/usr/share/fonts/truetype/cmu/,UprightFont=cmuntt.ttf,BoldFont=cmuntb.ttf,ItalicFont=cmunit.ttf,BoldItalicFont=cmuntx.ttf]{cmunrm.ttf} LaTeX or TeX input file. It can be compiled with latex.\\ \hline \hspace*{0pt}\ignorespaces{}\hspace*{0pt} {\ttfamily \setmainfont[Path=/usr/share/fonts/truetype/cmu/,UprightFont=cmunrm.ttf,BoldFont=cmunbx.ttf,ItalicFont=cmunti.ttf,BoldItalicFont=cmunbi.ttf]{cmuntt.ttf}\setmonofont[Path=/usr/share/fonts/truetype/cmu/,UprightFont=cmuntt.ttf,BoldFont=cmuntb.ttf,ItalicFont=cmunit.ttf,BoldItalicFont=cmuntx.ttf]{cmuntt.ttf}\ttfamily .out}&\hspace*{0pt}\ignorespaces{}\hspace*{0pt}{$\text{ }$}\setmainfont[Path=/usr/share/fonts/truetype/cmu/,UprightFont=cmunrm.ttf,BoldFont=cmunbx.ttf,ItalicFont=cmunti.ttf,BoldItalicFont=cmunbi.ttf]{cmunrm.ttf}\setmonofont[Path=/usr/share/fonts/truetype/cmu/,UprightFont=cmuntt.ttf,BoldFont=cmuntb.ttf,ItalicFont=cmunit.ttf,BoldItalicFont=cmuntx.ttf]{cmunrm.ttf} hyperref package file, just one for the master file.\\ \hline 
\end{longtable}

\section{And what now?}
\label{81}
\subsection{Common Elements}
\label{82}

See \mylref{89}{Document Structure} and the {\bfseries \setmainfont[Path=/usr/share/fonts/truetype/cmu/,UprightFont=cmunrm.ttf,BoldFont=cmunbx.ttf,ItalicFont=cmunti.ttf,BoldItalicFont=cmunbi.ttf]{cmunbx.ttf}\setmonofont[Path=/usr/share/fonts/truetype/cmu/,UprightFont=cmuntt.ttf,BoldFont=cmuntb.ttf,ItalicFont=cmunit.ttf,BoldItalicFont=cmuntx.ttf]{cmunbx.ttf}\bfseries Common Elements}{$\text{ }$}\setmainfont[Path=/usr/share/fonts/truetype/cmu/,UprightFont=cmunrm.ttf,BoldFont=cmunbx.ttf,ItalicFont=cmunti.ttf,BoldItalicFont=cmunbi.ttf]{cmunrm.ttf}\setmonofont[Path=/usr/share/fonts/truetype/cmu/,UprightFont=cmuntt.ttf,BoldFont=cmuntb.ttf,ItalicFont=cmunit.ttf,BoldItalicFont=cmuntx.ttf]{cmunrm.ttf} part for all the common features that belong to every type of document.
\subsection{Non-{}English documents and special characters}
\label{83}
LaTeX has some nice features for most languages in the world. You can tell LaTeX to follow typography rules of the target language, ease special characters input, and so on. See \mylref{192}{Special Characters} and \mylref{209}{Internationalization}.
\subsection{Modular document}
\label{84}
See \mylref{895}{Modular Documents} for good recommendations about the way to organize big projects into multiple files.
\subsection{Questions and Issues}
\label{85}
We highly urge you to read the \mylref{945}{FAQ} if you have issues about basic features, or if you want to read essential recommendations.
For the more specific questions and issues, refer to the \mylref{968}{Tips and Tricks} page.

\subsection{Macros for the utmost efficiency}
\label{86}
The full power of LaTeX resides in macros. They make your documents very dynamic and flexible. See the \mylref{837}{dedicated part}.
\subsection{Working in a team}
\label{87}
See chapter \mylref{911}{../Collaborative Writing of LaTeX Documents/}.



\myhref{https://sr.wikibooks.org/wiki/LaTeX\%2F\%D0\%9E\%D1\%81\%D0\%BD\%D0\%BE\%D0\%B2\%D0\%B5}{sr:LaTeX/Основе}
\mypart{Common Elements}\chapter{Document Structure}

\myminitoc
\label{88}

\label{89}


The main point of writing a text is to convey ideas, information, or knowledge to the reader. The reader will
understand the text better if these ideas are well-{}structured, and will see and feel this structure much better if the typographical form reflects the logical and semantic structure of the content.

LaTeX is different from other typesetting systems in that you just have to tell it the logical and semantical structure of a text. It then derives
the typographical form of the text according to the “rules” given in the document class file and in various style files. LaTeX allows users to structure their documents with a variety of hierarchical constructs, including chapters, sections, subsections and paragraphs. 
\section{Global structure}
\label{90}

When LaTeX processes an input file, it expects it to follow a certain structure. Thus every input file must contain the commands

\begin{Shaded}
\begin{Highlighting}[]

\NormalTok{\textbackslash{}documentclass\{...\}}
 
\NormalTok{\textbackslash{}begin\{document\}}
\NormalTok{...}
\NormalTok{\textbackslash{}end\{document\}}
\end{Highlighting}
\end{Shaded}


The area between \LaTeXTT{\textbackslash{}documentclass\{...\}} and \LaTeXTT{\textbackslash{}begin\{document\}} is called the {\itshape \setmainfont[Path=/usr/share/fonts/truetype/cmu/,UprightFont=cmunrm.ttf,BoldFont=cmunbx.ttf,ItalicFont=cmunti.ttf,BoldItalicFont=cmunbi.ttf]{cmunti.ttf}\setmonofont[Path=/usr/share/fonts/truetype/cmu/,UprightFont=cmuntt.ttf,BoldFont=cmuntb.ttf,ItalicFont=cmunit.ttf,BoldItalicFont=cmuntx.ttf]{cmunti.ttf}\itshape preamble}\setmainfont[Path=/usr/share/fonts/truetype/cmu/,UprightFont=cmunrm.ttf,BoldFont=cmunbx.ttf,ItalicFont=cmunti.ttf,BoldItalicFont=cmunbi.ttf]{cmunrm.ttf}\setmonofont[Path=/usr/share/fonts/truetype/cmu/,UprightFont=cmuntt.ttf,BoldFont=cmuntb.ttf,ItalicFont=cmunit.ttf,BoldItalicFont=cmuntx.ttf]{cmunrm.ttf}. It normally contains commands that affect the entire document.

After the preamble, the text of your document is enclosed between two commands which identify the beginning and end of the actual document:

\begin{Shaded}
\begin{Highlighting}[]

\NormalTok{\textbackslash{}begin\{document\}}
\NormalTok{...}
\NormalTok{\textbackslash{}end\{document\}}
\end{Highlighting}
\end{Shaded}


You would put your text where the dots are. The reason for marking off the beginning of your text is that LaTeX allows you to insert extra setup specifications before it (where the blank line is in the example above: we\textquotesingle{}ll be using this soon). The reason for marking off the end of your text is to provide a place for LaTeX to be programmed to do extra stuff automatically at the end of the document, like making an index.

A useful side-{}effect of marking the end of the document text is that you can store comments or temporary text underneath the \LaTeXTT{\textbackslash{}end\{document\}} in the knowledge that LaTeX will never try to typeset them:

\begin{Shaded}
\begin{Highlighting}[]

\NormalTok{\textbackslash{}end\{document\}}
\NormalTok{...}
 
\end{Highlighting}
\end{Shaded}

\section{Preamble}
\label{91}
\subsection{Document classes}
\label{92}

When processing an input file, LaTeX needs to know the type of document the author wants to create. This is specified with the {\ttfamily \setmainfont[Path=/usr/share/fonts/truetype/cmu/,UprightFont=cmunrm.ttf,BoldFont=cmunbx.ttf,ItalicFont=cmunti.ttf,BoldItalicFont=cmunbi.ttf]{cmuntt.ttf}\setmonofont[Path=/usr/share/fonts/truetype/cmu/,UprightFont=cmuntt.ttf,BoldFont=cmuntb.ttf,ItalicFont=cmunit.ttf,BoldItalicFont=cmuntx.ttf]{cmuntt.ttf}\ttfamily \textbackslash{}documentclass}{$\text{ }$}\setmainfont[Path=/usr/share/fonts/truetype/cmu/,UprightFont=cmunrm.ttf,BoldFont=cmunbx.ttf,ItalicFont=cmunti.ttf,BoldItalicFont=cmunbi.ttf]{cmunrm.ttf}\setmonofont[Path=/usr/share/fonts/truetype/cmu/,UprightFont=cmuntt.ttf,BoldFont=cmuntb.ttf,ItalicFont=cmunit.ttf,BoldItalicFont=cmuntx.ttf]{cmunrm.ttf} command. It is recommended to put this declaration at the very beginning.

\begin{Shaded}
\begin{Highlighting}[]

\NormalTok{\textbackslash{}documentclass[options]\{class\}}
\end{Highlighting}
\end{Shaded}


Here, {\ttfamily \setmainfont[Path=/usr/share/fonts/truetype/cmu/,UprightFont=cmunrm.ttf,BoldFont=cmunbx.ttf,ItalicFont=cmunti.ttf,BoldItalicFont=cmunbi.ttf]{cmuntt.ttf}\setmonofont[Path=/usr/share/fonts/truetype/cmu/,UprightFont=cmuntt.ttf,BoldFont=cmuntb.ttf,ItalicFont=cmunit.ttf,BoldItalicFont=cmuntx.ttf]{cmuntt.ttf}\ttfamily class}{$\text{ }$}\setmainfont[Path=/usr/share/fonts/truetype/cmu/,UprightFont=cmunrm.ttf,BoldFont=cmunbx.ttf,ItalicFont=cmunti.ttf,BoldItalicFont=cmunbi.ttf]{cmunrm.ttf}\setmonofont[Path=/usr/share/fonts/truetype/cmu/,UprightFont=cmuntt.ttf,BoldFont=cmuntb.ttf,ItalicFont=cmunit.ttf,BoldItalicFont=cmuntx.ttf]{cmunrm.ttf} specifies the type of document to be created. The LaTeX distribution provides additional classes for other documents, including letters and slides. It is also possible to create your own, as is often done by journal publishers, who simply provide you with their own class file, which tells LaTeX how to format your content. But we\textquotesingle{}ll be happy with the standard article class for now. The {\ttfamily \setmainfont[Path=/usr/share/fonts/truetype/cmu/,UprightFont=cmunrm.ttf,BoldFont=cmunbx.ttf,ItalicFont=cmunti.ttf,BoldItalicFont=cmunbi.ttf]{cmuntt.ttf}\setmonofont[Path=/usr/share/fonts/truetype/cmu/,UprightFont=cmuntt.ttf,BoldFont=cmuntb.ttf,ItalicFont=cmunit.ttf,BoldItalicFont=cmuntx.ttf]{cmuntt.ttf}\ttfamily options}{$\text{ }$}\setmainfont[Path=/usr/share/fonts/truetype/cmu/,UprightFont=cmunrm.ttf,BoldFont=cmunbx.ttf,ItalicFont=cmunti.ttf,BoldItalicFont=cmunbi.ttf]{cmunrm.ttf}\setmonofont[Path=/usr/share/fonts/truetype/cmu/,UprightFont=cmuntt.ttf,BoldFont=cmuntb.ttf,ItalicFont=cmunit.ttf,BoldItalicFont=cmuntx.ttf]{cmunrm.ttf} parameter customizes the behavior of the document class. The options have to be separated by commas. 

Example: an input file for a LaTeX document could start with the line

\begin{Shaded}
\begin{Highlighting}[]

\NormalTok{\textbackslash{}documentclass[11pt,twoside,a4paper]\{article\}}
\end{Highlighting}
\end{Shaded}


which instructs LaTeX to typeset the document as an article with a base font size of 11 points, and to produce a layout suitable for double sided printing on A4 paper.

Here are some document classes that can be used with LaTeX:

\begin{longtable}{|>{\RaggedRight}p{0.14248\linewidth}|>{\RaggedRight}p{0.80038\linewidth}|} \hline 
\multicolumn{2}{|>{\RaggedRight}p{0.97143\linewidth}|}{{\bfseries \hspace*{0pt}\ignorespaces{}\hspace*{0pt}Document Classes}}\endhead  \hline \hspace*{0pt}\ignorespaces{}\hspace*{0pt} \LaTeXTT{article}&\hspace*{0pt}\ignorespaces{}\hspace*{0pt} For articles in scientific journals, presentations, short reports, program documentation, invitations, ...\\ \hline \hspace*{0pt}\ignorespaces{}\hspace*{0pt} \LaTeXTT{IEEEtran}&\hspace*{0pt}\ignorespaces{}\hspace*{0pt} For articles with the IEEE Transactions format.\\ \hline \hspace*{0pt}\ignorespaces{}\hspace*{0pt} \LaTeXTT{proc}&\hspace*{0pt}\ignorespaces{}\hspace*{0pt} A class for proceedings based on the article class.\\ \hline \hspace*{0pt}\ignorespaces{}\hspace*{0pt} \LaTeXTT{report}&\hspace*{0pt}\ignorespaces{}\hspace*{0pt} For longer reports containing several chapters, small books, thesis, ...\\ \hline \hspace*{0pt}\ignorespaces{}\hspace*{0pt} \LaTeXTT{book}&\hspace*{0pt}\ignorespaces{}\hspace*{0pt} For real books.\\ \hline \hspace*{0pt}\ignorespaces{}\hspace*{0pt} \LaTeXTT{slides}&\hspace*{0pt}\ignorespaces{}\hspace*{0pt} For slides. The class uses big sans serif letters.\\ \hline \hspace*{0pt}\ignorespaces{}\hspace*{0pt} \LaTeXTT{memoir}&\hspace*{0pt}\ignorespaces{}\hspace*{0pt} For changing sensibly the output of the document. It is based on the \LaTeXTT{book} class, but you can create any kind of document with it \myplainurl{http://www.ctan.org/tex-archive/macros/latex/contrib/memoir/memman.pdf}\\ \hline \hspace*{0pt}\ignorespaces{}\hspace*{0pt} \LaTeXTT{letter}&\hspace*{0pt}\ignorespaces{}\hspace*{0pt} For writing letters. \\ \hline \hspace*{0pt}\ignorespaces{}\hspace*{0pt} \LaTeXTT{beamer}&\hspace*{0pt}\ignorespaces{}\hspace*{0pt} For writing presentations (see \mylref{729}{LaTeX/Presentations}). \\ \hline 
\end{longtable}


The standard document classes that are a part of LaTeX are built to be fairly generic, which is why they have a lot of options in common. Other classes may have different options (or none at all). Normally, third party classes come with some documentation to let you know. The most common options for the standard document classes are listed in the following table:

\begin{longtable}{|>{\RaggedRight}p{0.34804\linewidth}|>{\RaggedRight}p{0.59482\linewidth}|} \hline 
\multicolumn{2}{|>{\RaggedRight}p{0.97143\linewidth}|}{{\bfseries \hspace*{0pt}\ignorespaces{}\hspace*{0pt}Document Class Options}}\endhead  \hline \hspace*{0pt}\ignorespaces{}\hspace*{0pt} \LaTeXTT{10pt, 11pt, 12pt}&\hspace*{0pt}\ignorespaces{}\hspace*{0pt}Sets the size of the main font in the document. If no option is specified, 10pt is assumed.\\ \hline \hspace*{0pt}\ignorespaces{}\hspace*{0pt} \LaTeXTT{a4paper, letterpaper,}...&\hspace*{0pt}\ignorespaces{}\hspace*{0pt}Defines the paper size. The default size is \LaTeXTT{letterpaper}; However, many European distributions of TeX now come pre-{}set for A4, not Letter, and this is also true of all distributions of pdfLaTeX. Besides that, \LaTeXTT{a5paper, b5paper, executivepaper}, and \LaTeXTT{legalpaper} can be specified.\\ \hline \hspace*{0pt}\ignorespaces{}\hspace*{0pt} \LaTeXTT{fleqn}&\hspace*{0pt}\ignorespaces{}\hspace*{0pt}Typesets displayed formulas left-{}aligned instead of centered.\\ \hline \hspace*{0pt}\ignorespaces{}\hspace*{0pt} \LaTeXTT{leqno}&\hspace*{0pt}\ignorespaces{}\hspace*{0pt}Places the numbering of formulas on the left hand side instead of the right.\\ \hline \hspace*{0pt}\ignorespaces{}\hspace*{0pt} \LaTeXTT{titlepage, notitlepage}&\hspace*{0pt}\ignorespaces{}\hspace*{0pt}Specifies whether a new page should be started after the document title or not. The article class does not start a new page by default, while report and book do.\\ \hline \hspace*{0pt}\ignorespaces{}\hspace*{0pt} \LaTeXTT{twocolumn}&\hspace*{0pt}\ignorespaces{}\hspace*{0pt}Instructs LaTeX to typeset the document in two columns instead of one.\\ \hline \hspace*{0pt}\ignorespaces{}\hspace*{0pt} \LaTeXTT{twoside, oneside}&\hspace*{0pt}\ignorespaces{}\hspace*{0pt}Specifies whether double or single sided output should be generated. The classes \LaTeXTT{article} and \LaTeXTT{report} are single sided and the \LaTeXTT{book} class is double sided by default. Note that this option concerns the style of the document only. The option \LaTeXTT{twoside} does not tell the printer you use that it should actually make a two-{}sided printout.\\ \hline \hspace*{0pt}\ignorespaces{}\hspace*{0pt} \LaTeXTT{landscape}&\hspace*{0pt}\ignorespaces{}\hspace*{0pt}Changes the layout of the document to print in landscape mode.\\ \hline \hspace*{0pt}\ignorespaces{}\hspace*{0pt} \LaTeXTT{openright, openany}&\hspace*{0pt}\ignorespaces{}\hspace*{0pt}Makes chapters begin either only on right hand pages or on the next page available. This does not work with the \LaTeXTT{article} class, as it does not know about chapters. The \LaTeXTT{report} class by default starts chapters on the next page available and the \LaTeXTT{book} class starts them on right hand pages.\\ \hline \hspace*{0pt}\ignorespaces{}\hspace*{0pt}\LaTeXTT{draft}&\hspace*{0pt}\ignorespaces{}\hspace*{0pt}makes LaTeX indicate hyphenation and justification problems with a small square in the right-{}hand margin of the problem line so they can be located quickly by a human. It also suppresses the inclusion of images and shows only a frame where they would normally occur.\\ \hline 
\end{longtable}


For example, if you want a report to be in 12pt type on A4, but printed one-{}sided in draft mode, you would use:
	
\begin{Shaded}
\begin{Highlighting}[]

\NormalTok{\textbackslash{}documentclass[12pt,a4paper,oneside,draft]\{report\}}
\end{Highlighting}
\end{Shaded}

\subsection{Packages}
\label{93}

While writing your document, you will probably find that there are some areas where basic LaTeX cannot solve your problem. If you want to include graphics, colored text or source code from a file into your document, you need to enhance the capabilities of LaTeX. Such enhancements are called packages. Some packages come with the LaTeX base distribution. Others are provided separately. Modern TeX distributions come with a large number of packages pre-{}installed. The command to use a package is pretty simple: \LaTeXTT{\textbackslash{}usepackage}:

\begin{Shaded}
\begin{Highlighting}[]

\NormalTok{\textbackslash{}usepackage[options]\{package\}}
\end{Highlighting}
\end{Shaded}


command, where package is the name of the package and options is a list of keywords that trigger special features in the package. For example, to use the \LaTeXTT{color} package, which lets you typeset in colors, you would type:

\begin{Shaded}
\begin{Highlighting}[]

\NormalTok{\textbackslash{}documentclass\{report\}}
\NormalTok{\textbackslash{}usepackage\{color\}}
 
\NormalTok{\textbackslash{}begin\{document\}}
\NormalTok{...}
\NormalTok{\textbackslash{}end\{document\}}
\end{Highlighting}
\end{Shaded}


You can pass several options to a package, each separated by a comma.

\begin{Shaded}
\begin{Highlighting}[]

\NormalTok{\textbackslash{}usepackage[option1,option2,option3]\{''package_name''\}}
\end{Highlighting}
\end{Shaded}

\section{The {\itshape \setmainfont[Path=/usr/share/fonts/truetype/cmu/,UprightFont=cmunrm.ttf,BoldFont=cmunbx.ttf,ItalicFont=cmunti.ttf,BoldItalicFont=cmunbi.ttf]{cmunti.ttf}\setmonofont[Path=/usr/share/fonts/truetype/cmu/,UprightFont=cmuntt.ttf,BoldFont=cmuntb.ttf,ItalicFont=cmunit.ttf,BoldItalicFont=cmuntx.ttf]{cmunti.ttf}\itshape document}{$\text{ }$}\setmainfont[Path=/usr/share/fonts/truetype/cmu/,UprightFont=cmunrm.ttf,BoldFont=cmunbx.ttf,ItalicFont=cmunti.ttf,BoldItalicFont=cmunbi.ttf]{cmunrm.ttf}\setmonofont[Path=/usr/share/fonts/truetype/cmu/,UprightFont=cmuntt.ttf,BoldFont=cmuntb.ttf,ItalicFont=cmunit.ttf,BoldItalicFont=cmuntx.ttf]{cmunrm.ttf} environment}
\label{94}
\subsection{Top matter}
\label{95}

At the beginning of most documents there will be information about the document itself, such as the title and date, and also information about the authors, such as name, address, email etc. All of this type of information within LaTeX is collectively referred to as {\itshape \setmainfont[Path=/usr/share/fonts/truetype/cmu/,UprightFont=cmunrm.ttf,BoldFont=cmunbx.ttf,ItalicFont=cmunti.ttf,BoldItalicFont=cmunbi.ttf]{cmunti.ttf}\setmonofont[Path=/usr/share/fonts/truetype/cmu/,UprightFont=cmuntt.ttf,BoldFont=cmuntb.ttf,ItalicFont=cmunit.ttf,BoldItalicFont=cmuntx.ttf]{cmunti.ttf}\itshape top matter}\setmainfont[Path=/usr/share/fonts/truetype/cmu/,UprightFont=cmunrm.ttf,BoldFont=cmunbx.ttf,ItalicFont=cmunti.ttf,BoldItalicFont=cmunbi.ttf]{cmunrm.ttf}\setmonofont[Path=/usr/share/fonts/truetype/cmu/,UprightFont=cmuntt.ttf,BoldFont=cmuntb.ttf,ItalicFont=cmunit.ttf,BoldItalicFont=cmuntx.ttf]{cmunrm.ttf}. Although never explicitly specified (there is no \LaTeXTT{\textbackslash{}topmatter} command) you are likely to encounter the term within LaTeX documentation.

A simple example:
\begin{Shaded}
\begin{Highlighting}[]

\NormalTok{\textbackslash{}documentclass[11pt,a4paper]\{report\}}
 
\NormalTok{\textbackslash{}begin\{document\}}
\NormalTok{\textbackslash{}title\{How to Structure a LaTeX Document\}}
\NormalTok{\textbackslash{}author\{Andrew Roberts\}}
\NormalTok{\textbackslash{}date\{December 2004\}}
\NormalTok{\textbackslash{}maketitle}
\NormalTok{\textbackslash{}end\{document\}}
\end{Highlighting}
\end{Shaded}


The \LaTeXTT{\textbackslash{}title}, \LaTeXTT{\textbackslash{}author}, and \LaTeXTT{\textbackslash{}date} commands are self-{}explanatory. You put the title, author name, and date in curly braces after the relevant command. The title and author are usually compulsory (at least if you want LaTeX to write the title automatically); if you omit the \LaTeXTT{\textbackslash{}date} command, LaTeX uses today\textquotesingle{}s date by default. You always finish the top matter with the \LaTeXTT{\textbackslash{}maketitle} command, which tells LaTeX that it\textquotesingle{}s complete and it can typeset the title according to the information you have provided and the class (style) you are using. If you omit \LaTeXTT{\textbackslash{}maketitle}, the titling will never be typeset (unless you write your own).

Here is a more complicated example:
\begin{Shaded}
\begin{Highlighting}[]

\NormalTok{\textbackslash{}title\{How to Structure a \textbackslash{}LaTeX\{\} Document\}}
\NormalTok{\textbackslash{}author\{Joe Bloggs\textbackslash{}\textbackslash{}}
  \NormalTok{School of Computing,\textbackslash{}\textbackslash{}}
  \NormalTok{University of Study,\textbackslash{}\textbackslash{}}
  \NormalTok{Books,\textbackslash{}\textbackslash{}}
  \NormalTok{United Readdom,\textbackslash{}\textbackslash{}}
  \NormalTok{RN 1234\textbackslash{}\textbackslash{}}
  \NormalTok{\textbackslash{}texttt\{jbloggs@latex.wizard\}\}}
\NormalTok{\textbackslash{}date\{\textbackslash{}today\}}
\NormalTok{\textbackslash{}maketitle}
\end{Highlighting}
\end{Shaded}


As you can see, you can use commands as arguments of \LaTeXTT{\textbackslash{}title} and the others. The double backslash (\LaTeXTT{\textbackslash{}\textbackslash{}}) is the LaTeX command for forced linebreaks in tabular material.

If there are two authors separate them with the \LaTeXTT{\textbackslash{}and} command:
\begin{Shaded}
\begin{Highlighting}[]

\NormalTok{\textbackslash{}title\{Our Fun Document\}}
\NormalTok{\textbackslash{}author\{Jane Doe \textbackslash{}and John Doe\} }
\NormalTok{\textbackslash{}date\{\textbackslash{}today\}}
\NormalTok{\textbackslash{}maketitle}
\end{Highlighting}
\end{Shaded}




Using this approach, you can create only basic output whose layout is very hard to change. If you want to create your title freely, see the \mylref{293}{Title Creation} section.
\subsection{Abstract}
\label{96}

As most research papers have an abstract, there are predefined commands for telling LaTeX which part of the content makes up the abstract. This should appear in its logical order, therefore, after the top matter, but before the main sections of the body. This command is available for the document classes {\itshape \setmainfont[Path=/usr/share/fonts/truetype/cmu/,UprightFont=cmunrm.ttf,BoldFont=cmunbx.ttf,ItalicFont=cmunti.ttf,BoldItalicFont=cmunbi.ttf]{cmunti.ttf}\setmonofont[Path=/usr/share/fonts/truetype/cmu/,UprightFont=cmuntt.ttf,BoldFont=cmuntb.ttf,ItalicFont=cmunit.ttf,BoldItalicFont=cmuntx.ttf]{cmunti.ttf}\itshape article}{$\text{ }$}\setmainfont[Path=/usr/share/fonts/truetype/cmu/,UprightFont=cmunrm.ttf,BoldFont=cmunbx.ttf,ItalicFont=cmunti.ttf,BoldItalicFont=cmunbi.ttf]{cmunrm.ttf}\setmonofont[Path=/usr/share/fonts/truetype/cmu/,UprightFont=cmuntt.ttf,BoldFont=cmuntb.ttf,ItalicFont=cmunit.ttf,BoldItalicFont=cmuntx.ttf]{cmunrm.ttf} and {\itshape \setmainfont[Path=/usr/share/fonts/truetype/cmu/,UprightFont=cmunrm.ttf,BoldFont=cmunbx.ttf,ItalicFont=cmunti.ttf,BoldItalicFont=cmunbi.ttf]{cmunti.ttf}\setmonofont[Path=/usr/share/fonts/truetype/cmu/,UprightFont=cmuntt.ttf,BoldFont=cmuntb.ttf,ItalicFont=cmunit.ttf,BoldItalicFont=cmuntx.ttf]{cmunti.ttf}\itshape report}\setmainfont[Path=/usr/share/fonts/truetype/cmu/,UprightFont=cmunrm.ttf,BoldFont=cmunbx.ttf,ItalicFont=cmunti.ttf,BoldItalicFont=cmunbi.ttf]{cmunrm.ttf}\setmonofont[Path=/usr/share/fonts/truetype/cmu/,UprightFont=cmuntt.ttf,BoldFont=cmuntb.ttf,ItalicFont=cmunit.ttf,BoldItalicFont=cmuntx.ttf]{cmunrm.ttf}, but not {\itshape \setmainfont[Path=/usr/share/fonts/truetype/cmu/,UprightFont=cmunrm.ttf,BoldFont=cmunbx.ttf,ItalicFont=cmunti.ttf,BoldItalicFont=cmunbi.ttf]{cmunti.ttf}\setmonofont[Path=/usr/share/fonts/truetype/cmu/,UprightFont=cmuntt.ttf,BoldFont=cmuntb.ttf,ItalicFont=cmunit.ttf,BoldItalicFont=cmuntx.ttf]{cmunti.ttf}\itshape book}\setmainfont[Path=/usr/share/fonts/truetype/cmu/,UprightFont=cmunrm.ttf,BoldFont=cmunbx.ttf,ItalicFont=cmunti.ttf,BoldItalicFont=cmunbi.ttf]{cmunrm.ttf}\setmonofont[Path=/usr/share/fonts/truetype/cmu/,UprightFont=cmuntt.ttf,BoldFont=cmuntb.ttf,ItalicFont=cmunit.ttf,BoldItalicFont=cmuntx.ttf]{cmunrm.ttf}. 

\begin{Shaded}
\begin{Highlighting}[]

\NormalTok{\textbackslash{}documentclass\{article\}}
 
\NormalTok{\textbackslash{}begin\{document\}}
 
\NormalTok{\textbackslash{}begin\{abstract\}}
\NormalTok{Your abstract goes here...}
\NormalTok{...}
\NormalTok{\textbackslash{}end\{abstract\}}
\NormalTok{...}
\NormalTok{\textbackslash{}end\{document\}}
\end{Highlighting}
\end{Shaded}


By default, LaTeX will use the word \symbol{34}Abstract\symbol{34} as a title for your abstract. If you want to change it into anything else, e.g. \symbol{34}Executive Summary\symbol{34}, add the following line before you begin the abstract environment:

\begin{Shaded}
\begin{Highlighting}[]

\NormalTok{\textbackslash{}renewcommand\{\textbackslash{}abstractname\}\{Executive Summary\}}
\end{Highlighting}
\end{Shaded}

\subsection{Sectioning commands}
\label{97}

The commands for inserting sections are fairly intuitive. Of course, certain commands are appropriate to different document classes. For example, a book has chapters but an article doesn\textquotesingle{}t. Here are some of the structure commands found in {\itshape \setmainfont[Path=/usr/share/fonts/truetype/cmu/,UprightFont=cmunrm.ttf,BoldFont=cmunbx.ttf,ItalicFont=cmunti.ttf,BoldItalicFont=cmunbi.ttf]{cmunti.ttf}\setmonofont[Path=/usr/share/fonts/truetype/cmu/,UprightFont=cmuntt.ttf,BoldFont=cmuntb.ttf,ItalicFont=cmunit.ttf,BoldItalicFont=cmuntx.ttf]{cmunti.ttf}\itshape simple.tex}\setmainfont[Path=/usr/share/fonts/truetype/cmu/,UprightFont=cmunrm.ttf,BoldFont=cmunbx.ttf,ItalicFont=cmunti.ttf,BoldItalicFont=cmunbi.ttf]{cmunrm.ttf}\setmonofont[Path=/usr/share/fonts/truetype/cmu/,UprightFont=cmuntt.ttf,BoldFont=cmuntb.ttf,ItalicFont=cmunit.ttf,BoldItalicFont=cmuntx.ttf]{cmunrm.ttf}.

\begin{Shaded}
\begin{Highlighting}[]

\NormalTok{\textbackslash{}chapter\{Introduction\}}
\NormalTok{This chapter's content...}
 
\NormalTok{\textbackslash{}section\{Structure\}}
\NormalTok{This section's content...}
 
\NormalTok{\textbackslash{}subsection\{Top Matter\}}
\NormalTok{This subsection's content...}
 
\NormalTok{\textbackslash{}subsubsection\{Article Information\}}
\NormalTok{This subsubsection's content...}
\end{Highlighting}
\end{Shaded}


Notice that you do not need to specify section numbers; LaTeX will sort that out for you. Also, for sections, you do not need to use \LaTeXTT{\textbackslash{}begin} and \LaTeXTT{\textbackslash{}end} commands to indicate which content belongs to a given block. 

LaTeX provides 7 levels of depth for defining sections (see table below). Each section in this table is a subsection of the one above it.

\begin{longtable}{|>{\RaggedRight}p{0.55694\linewidth}|>{\RaggedRight}p{0.08986\linewidth}|>{\RaggedRight}p{0.26748\linewidth}|} \hline 
{\bfseries \hspace*{0pt}\ignorespaces{}\hspace*{0pt} Command}&{\bfseries \hspace*{0pt}\ignorespaces{}\hspace*{0pt} Level}&{\bfseries \hspace*{0pt}\ignorespaces{}\hspace*{0pt} Comment}\endhead  \hline \hspace*{0pt}\ignorespaces{}\hspace*{0pt} \LaTeXTT{\textbackslash{}part\{\textquotesingle{}\textquotesingle{}part\textquotesingle{}\textquotesingle{}\}}&\hspace*{0pt}\ignorespaces{}\hspace*{0pt} -{}1&\hspace*{0pt}\ignorespaces{}\hspace*{0pt} not in letters\\ \hline \hspace*{0pt}\ignorespaces{}\hspace*{0pt} \LaTeXTT{\textbackslash{}chapter\{\textquotesingle{}\textquotesingle{}chapter\textquotesingle{}\textquotesingle{}\}}&\hspace*{0pt}\ignorespaces{}\hspace*{0pt} 0&\hspace*{0pt}\ignorespaces{}\hspace*{0pt} only books and reports\\ \hline \hspace*{0pt}\ignorespaces{}\hspace*{0pt} \LaTeXTT{\textbackslash{}section\{\textquotesingle{}\textquotesingle{}section\textquotesingle{}\textquotesingle{}\}}&\hspace*{0pt}\ignorespaces{}\hspace*{0pt} 1&\hspace*{0pt}\ignorespaces{}\hspace*{0pt} not in letters\\ \hline \hspace*{0pt}\ignorespaces{}\hspace*{0pt} \LaTeXTT{\textbackslash{}subsection\{\textquotesingle{}\textquotesingle{}subsection\textquotesingle{}\textquotesingle{}\}}&\hspace*{0pt}\ignorespaces{}\hspace*{0pt} 2&\hspace*{0pt}\ignorespaces{}\hspace*{0pt} not in letters\\ \hline \hspace*{0pt}\ignorespaces{}\hspace*{0pt} \LaTeXTT{\textbackslash{}subsubsection\{\textquotesingle{}\textquotesingle{}subsubsection\textquotesingle{}\textquotesingle{}\}}&\hspace*{0pt}\ignorespaces{}\hspace*{0pt} 3&\hspace*{0pt}\ignorespaces{}\hspace*{0pt} not in letters\\ \hline \hspace*{0pt}\ignorespaces{}\hspace*{0pt} \LaTeXTT{\textbackslash{}paragraph\{\textquotesingle{}\textquotesingle{}paragraph\textquotesingle{}\textquotesingle{}\}}&\hspace*{0pt}\ignorespaces{}\hspace*{0pt} 4&\hspace*{0pt}\ignorespaces{}\hspace*{0pt} not in letters\\ \hline \hspace*{0pt}\ignorespaces{}\hspace*{0pt} \LaTeXTT{\textbackslash{}subparagraph\{\textquotesingle{}\textquotesingle{}subparagraph\textquotesingle{}\textquotesingle{}\}}&\hspace*{0pt}\ignorespaces{}\hspace*{0pt} 5&\hspace*{0pt}\ignorespaces{}\hspace*{0pt} not in letters\\ \hline 
\end{longtable}


All the titles of the sections are added automatically to the table of contents (if you decide to insert one). But if you make manual styling changes to your heading, for example a very long title, or some special line-{}breaks or unusual font-{}play, this would appear in the Table of Contents as well, which you almost certainly don\textquotesingle{}t want. LaTeX allows you to give an optional extra version of the heading text which only gets used in the Table of Contents and any running heads, if they are in effect. This optional alternative heading goes in {$\text{[}$}square brackets{$\text{]}$} before the curly braces:

\begin{Shaded}
\begin{Highlighting}[]

\NormalTok{\textbackslash{}section[Effect on staff turnover]\{An analysis of the}
\NormalTok{effect of the revised recruitment policies on staff}
\NormalTok{turnover at divisional headquarters\}}
\end{Highlighting}
\end{Shaded}

\subsubsection{Section numbering}
\label{98}

Numbering of the sections is performed automatically by LaTeX, so don\textquotesingle{}t bother adding them explicitly, just insert the heading you want between the curly braces. Parts get roman numerals (Part I, Part II, etc.); chapters and sections get decimal numbering like this document, and appendices (which are just a special case of chapters, and share the same structure) are lettered (A, B, C, etc.).

You can change the depth to which section numbering occurs, so you can turn it off selectively. By default it is set to 3. If you only want parts, chapters, and sections numbered, not subsections or subsubsections etc., you can change the value of the \LaTeXTT{secnumdepth} \mylref{469}{counter} using the \LaTeXTT{\textbackslash{}setcounter} command, giving the depth level you wish. For example, if you want to change it to \symbol{34}1\symbol{34}:

\begin{Shaded}
\begin{Highlighting}[]

\NormalTok{\textbackslash{}setcounter\{secnumdepth\}\{1\}}
\end{Highlighting}
\end{Shaded}


A related counter is \LaTeXTT{tocdepth}, which specifies what depth to take the Table of Contents to. It can be reset in exactly the
same way as \LaTeXTT{secnumdepth}. For example:

\begin{Shaded}
\begin{Highlighting}[]

\NormalTok{\textbackslash{}setcounter\{tocdepth\}\{3\}}
\end{Highlighting}
\end{Shaded}


To get an unnumbered section heading which does not go into the Table of Contents, follow the command name with an asterisk before the opening curly brace:

\begin{Shaded}
\begin{Highlighting}[]

\NormalTok{\textbackslash{}subsection*\{Introduction\}}
\end{Highlighting}
\end{Shaded}


All the divisional commands from \LaTeXTT{\textbackslash{}part*} to \LaTeXTT{\textbackslash{}subparagraph*} have this \symbol{34}starred\symbol{34} version which can be used on special occasions for an unnumbered heading when the setting of \LaTeXTT{secnumdepth} would normally mean it would be numbered.

If you want the unnumbered section to be in the table of contents anyway, use the \LaTeXTT{\textbackslash{}addcontentsline} command like this:

\begin{Shaded}
\begin{Highlighting}[]

\NormalTok{\textbackslash{}section*\{Introduction\}}
\NormalTok{\textbackslash{}addcontentsline\{toc\}\{section\}\{Introduction\}}
\end{Highlighting}
\end{Shaded}


Note that if you use PDF bookmarks you will need to add a phantom section so that bookmark will lead to the correct place in the document. The \LaTeXTT{\textbackslash{}phantomsection} command is defined in the \LaTeXTT{hyperref} package, and is implemented normally as follows:

\begin{Shaded}
\begin{Highlighting}[]

\NormalTok{\textbackslash{}phantomsection}
\NormalTok{\textbackslash{}addcontentsline\{toc\}\{section\}\{Introduction\}}
\NormalTok{\textbackslash{}section*\{Introduction\} }
\end{Highlighting}
\end{Shaded}


For chapters you will also need to clear the page (this will also correct page numbering in the ToC):

\begin{Shaded}
\begin{Highlighting}[]

\NormalTok{\textbackslash{}cleardoublepage}
\NormalTok{\textbackslash{}phantomsection}
\NormalTok{\textbackslash{}addcontentsline\{toc\}\{chapter\}\{Bibliography\}}
\NormalTok{\textbackslash{}bibliographystyle\{unsrt\}}
\NormalTok{\textbackslash{}bibliography\{my_bib_file\}}
\end{Highlighting}
\end{Shaded}


The value where the section numbering starts from can be set with the following command:

\begin{Shaded}
\begin{Highlighting}[]

\NormalTok{\textbackslash{}setcounter\{section\}\{4\}}
\end{Highlighting}
\end{Shaded}


The next section after this command will now be numbered 5.

For more details on counters, see the \mylref{469}{dedicated chapter}.
\subsubsection{Section number style}
\label{99}

See \mylref{469}{Counters}.
\subsection{Ordinary paragraphs}
\label{100}

Paragraphs of text come after section headings. Simply type the text and leave a blank line between paragraphs. The blank line means \symbol{34}start a new paragraph here\symbol{34}: it does {\bfseries \setmainfont[Path=/usr/share/fonts/truetype/cmu/,UprightFont=cmunrm.ttf,BoldFont=cmunbx.ttf,ItalicFont=cmunti.ttf,BoldItalicFont=cmunbi.ttf]{cmunbx.ttf}\setmonofont[Path=/usr/share/fonts/truetype/cmu/,UprightFont=cmuntt.ttf,BoldFont=cmuntb.ttf,ItalicFont=cmunit.ttf,BoldItalicFont=cmuntx.ttf]{cmunbx.ttf}\bfseries not}{$\text{ }$}\setmainfont[Path=/usr/share/fonts/truetype/cmu/,UprightFont=cmunrm.ttf,BoldFont=cmunbx.ttf,ItalicFont=cmunti.ttf,BoldItalicFont=cmunbi.ttf]{cmunrm.ttf}\setmonofont[Path=/usr/share/fonts/truetype/cmu/,UprightFont=cmuntt.ttf,BoldFont=cmuntb.ttf,ItalicFont=cmunit.ttf,BoldItalicFont=cmuntx.ttf]{cmunrm.ttf} mean you get a blank line in the typeset output. For formatting paragraph indents and spacing between paragraphs, refer to the \mylref{140}{Paragraph Formatting} section.
\subsection{Table of contents}
\label{101}

All auto-{}numbered headings get entered in the Table of Contents (ToC) automatically. You don\textquotesingle{}t have to print a ToC, but if you want to, just add the command \LaTeXTT{\textbackslash{}tableofcontents} at the point where you want it printed (usually after the Abstract or Summary).

Entries for the ToC are recorded each time you process your document, and reproduced the next time you process it, so you need to re-{}run LaTeX one extra time to ensure that all ToC pagenumber references are correctly calculated. We\textquotesingle{}ve already seen how to use the optional argument to the sectioning commands to add text to the ToC which is slightly different from the one printed in the body of the document. It is also possible to add extra lines to the ToC, to force extra or unnumbered section headings to be included.

The commands \LaTeXTT{\textbackslash{}listoffigures} and \LaTeXTT{\textbackslash{}listoftables} work in exactly the same way as \LaTeXTT{\textbackslash{}tableofcontents} to automatically list all your tables and figures. If you use them, they normally go after the \LaTeXTT{\textbackslash{}tableofcontents} command. The \LaTeXTT{\textbackslash{}tableofcontents} command normally shows only numbered section headings, and only down to the level defined by the \LaTeXTT{tocdepth} counter, but you can add extra entries with the \LaTeXTT{\textbackslash{}addcontentsline} command. For example if you use an unnumbered section heading command to start a preliminary piece of text like a Foreword or Preface, you can write:

\begin{Shaded}
\begin{Highlighting}[]

\NormalTok{\textbackslash{}subsection*\{Preface\}}
\NormalTok{\textbackslash{}addcontentsline\{toc\}\{subsection\}\{Preface\}}
\end{Highlighting}
\end{Shaded}


This will format an unnumbered ToC entry for \symbol{34}Preface\symbol{34} in the \symbol{34}subsection\symbol{34} style. You can use the same mechanism to add lines to the List of Figures or List of Tables by substituting \LaTeXTT{lof} or \LaTeXTT{lot} for \LaTeXTT{toc}. If the hyperref package is used and the link does not point to the correct chapter, the command \LaTeXTT{\textbackslash{}phantomsection} in combination with \LaTeXTT{\textbackslash{}clearpage} or \LaTeXTT{\textbackslash{}cleardoublepage} can be used (see also \mylref{434}{Labels and Cross-{}referencing}):

\begin{Shaded}
\begin{Highlighting}[]

\NormalTok{\textbackslash{}cleardoublepage}
\NormalTok{\textbackslash{}phantomsection}
\NormalTok{\textbackslash{}addcontentsline\{toc\}\{chapter\}\{List of Figures\}}
\NormalTok{\textbackslash{}listoffigures}
\end{Highlighting}
\end{Shaded}


To change the title of the TOC, you have to paste this command \LaTeXTT{\textbackslash{}renewcommand\{\textbackslash{}contentsname\}\{<{}New table of contents title>{}\}} in your document preamble. The List of Figures (LoF) and List of Tables (LoT) names can be changed by replacing the \LaTeXTT{\textbackslash{}contentsname} with \LaTeXTT{\textbackslash{}listfigurename} for LoF and \LaTeXTT{\textbackslash{}listtablename} for LoT.
\subsubsection{Depth}
\label{102}
The default ToC will list headings of level 3 and above. To change how deep the table of contents displays automatically the following command can be used in the preamble:

\begin{Shaded}
\begin{Highlighting}[]

\NormalTok{\textbackslash{}setcounter\{tocdepth\}\{4\}}
\end{Highlighting}
\end{Shaded}


This will make the table of contents include everything down to paragraphs.   The levels are defined above on this page.
Note that this solution does not permit changing the depth dynamically.

You can change the depth of specific section type, which could be useful for PDF bookmarks (if you are using the \LaTeXTT{hyperref} package) :

\begin{Shaded}
\begin{Highlighting}[]

\NormalTok{\textbackslash{}makeatletter}
\NormalTok{\textbackslash{}renewcommand*\{\textbackslash{}toclevel@chapter\}\{-1\} }\CommentTok{% Put chapter depth at the same level as}
 \NormalTok{\textbackslash{}part.}
\NormalTok{\textbackslash{}chapter\{Epilogue\}}
\NormalTok{\textbackslash{}renewcommand*\{\textbackslash{}toclevel@chapter\}\{0\} }\CommentTok{% Put chapter depth back to its default}
 \NormalTok{value.}
\NormalTok{\textbackslash{}makeatother}
\end{Highlighting}
\end{Shaded}


In order to further tune the display or the numbering of the table of contents, for instance if the appendix should be less detailed, you can make use of the \LaTeXTT{tocvsec2} package (\myhref{http://www.ctan.org/pkg/tocvsec2}{CTAN}, \myhref{http://mirror.ctan.org/macros/latex/contrib/tocvsec2/tocvsec2.pdf}{doc}).
\section{Book structure}
\label{103}

The standard LaTeX \LaTeXTT{book} class follows the same layout described above with some additions. By default a book will be two-{}sided, {\itshape \setmainfont[Path=/usr/share/fonts/truetype/cmu/,UprightFont=cmunrm.ttf,BoldFont=cmunbx.ttf,ItalicFont=cmunti.ttf,BoldItalicFont=cmunbi.ttf]{cmunti.ttf}\setmonofont[Path=/usr/share/fonts/truetype/cmu/,UprightFont=cmuntt.ttf,BoldFont=cmuntb.ttf,ItalicFont=cmunit.ttf,BoldItalicFont=cmuntx.ttf]{cmunti.ttf}\itshape i.e.}{$\text{ }$}\setmainfont[Path=/usr/share/fonts/truetype/cmu/,UprightFont=cmunrm.ttf,BoldFont=cmunbx.ttf,ItalicFont=cmunti.ttf,BoldItalicFont=cmunbi.ttf]{cmunrm.ttf}\setmonofont[Path=/usr/share/fonts/truetype/cmu/,UprightFont=cmuntt.ttf,BoldFont=cmuntb.ttf,ItalicFont=cmunit.ttf,BoldItalicFont=cmuntx.ttf]{cmunrm.ttf} left and right margins will change according to the page number parity. Furthermore current chapter and section will be printed in the header.

If you do not make use of chapters, it is barely useful to use the \LaTeXTT{book} class.

Additionally the class provides macros to change the formatting of some places of the document. We will give you some advice on how to use them properly.\myfootnote{\myplainurl{http://tex.stackexchange.com/questions/20538/what-is-the-right-order-when-using-frontmatter-tableofcontents-mainmatter}}

\begin{Shaded}
\begin{Highlighting}[]

\NormalTok{\textbackslash{}begin\{document\}}
\NormalTok{\textbackslash{}frontmatter}
 
\NormalTok{\textbackslash{}maketitle}
 
\CommentTok{% Introductory chapters}
\NormalTok{\textbackslash{}chapter\{Preface\}}
\CommentTok{% ...}
 
\NormalTok{\textbackslash{}mainmatter}
\NormalTok{\textbackslash{}chapter\{First chapter\}}
\CommentTok{% ...}
 
\NormalTok{\textbackslash{}appendix}
\NormalTok{\textbackslash{}chapter\{First Appendix\}}
 
\NormalTok{\textbackslash{}backmatter}
\NormalTok{\textbackslash{}chapter\{Last note\}}
\end{Highlighting}
\end{Shaded}


\begin{myitemize}
\item{}  The frontmatter chapters will not be numbered. Page numbers will be printed in roman numerals. Frontmatter is not supposed to have sections, since they will be number \LaTeXTT{0.n} because there is no chapter numbering. Check the \mylref{469}{Counters} chapter for a fix.
\item{}  The mainmatter chapters works as usual. The command resets the page numbering. Page numbers will be printed in arabic numerals.
\item{}  The \LaTeXTT{\textbackslash{}appendix} macro can be used to indicate that following sections or chapters are to be numbered as appendices. Appendices can be used for the article class too:
\end{myitemize}


\begin{Shaded}
\begin{Highlighting}[]

\NormalTok{\textbackslash{}appendix}
\NormalTok{\textbackslash{}section\{First Appendix\}}
\end{Highlighting}
\end{Shaded}


Only use the \LaTeXTT{\textbackslash{}appendix} macro once for all appendices.

\begin{myitemize}
\item{}  The backmatter behaves like the frontmatter. It has the same issue with section numbering.
\end{myitemize}


As a general rule you should avoid mixing the command order. Nonetheless all commands are optional, so you might consider using only a few.

Note that the special content like the table of contents is considered as an unnumbered chapter.
\subsection{Page order}
\label{104}

This is one traditional page order for books.
{\bfseries
\begin{mydescription}Frontmatter
\end{mydescription}
}

\begin{myenumerate}
\item{}  Half-{}title
\item{}  Empty
\item{}  Title page
\item{}  Information (copyright notice, ISBN, etc.)
\item{}  Dedication if any, else empty
\item{}  Table of contents
\item{}  List of figures (can be in the backmatter too)
\item{}  Preface chapter
\end{myenumerate}

{\bfseries
\begin{mydescription}Mainmatter
\end{mydescription}
}

\begin{myenumerate}
\item{}  Main topic
\end{myenumerate}

{\bfseries
\begin{mydescription}Appendix
\end{mydescription}
}

\begin{myenumerate}
\item{}  Some subordinate chapters
\end{myenumerate}

{\bfseries
\begin{mydescription}Backmatter
\end{mydescription}
}

\begin{myenumerate}
\item{}  Bibliography
\item{}  Glossary / Index
\end{myenumerate}



\section{Special pages}
\label{105}

Comprehensive papers often feature special pages at the end, like indices, glossaries and bibliographies. Since this is a quite complex topic, we will give you details in the dedicated part {\itshape \setmainfont[Path=/usr/share/fonts/truetype/cmu/,UprightFont=cmunrm.ttf,BoldFont=cmunbx.ttf,ItalicFont=cmunti.ttf,BoldItalicFont=cmunbi.ttf]{cmunti.ttf}\setmonofont[Path=/usr/share/fonts/truetype/cmu/,UprightFont=cmuntt.ttf,BoldFont=cmuntb.ttf,ItalicFont=cmunit.ttf,BoldItalicFont=cmuntx.ttf]{cmunti.ttf}\itshape Special Pages}\setmainfont[Path=/usr/share/fonts/truetype/cmu/,UprightFont=cmunrm.ttf,BoldFont=cmunbx.ttf,ItalicFont=cmunti.ttf,BoldItalicFont=cmunbi.ttf]{cmunrm.ttf}\setmonofont[Path=/usr/share/fonts/truetype/cmu/,UprightFont=cmuntt.ttf,BoldFont=cmuntb.ttf,ItalicFont=cmunit.ttf,BoldItalicFont=cmuntx.ttf]{cmunrm.ttf}.
\subsection{Bibliography}
\label{106}

Any good research paper will have a complete list of references. LaTeX has two ways of inserting your references into a document: 
\begin{myitemize}
\item{} you can embed them within the document itself. It\textquotesingle{}s simpler, but it can be time-{}consuming if you are writing several papers about similar subjects so that you often have to cite the same books.
\item{} you can store them in an external \myhref{http://www.bibtex.org}{BibTeX file } and then link them via a command to your current document and use a \myhref{http://www.cs.stir.ac.uk/~kjt/software/latex/showbst.html}{Bibtex style} to define how they appear. This way you can create a small database of the references you might use and simply link them, letting LaTeX work for you.
\end{myitemize}


To learn how to add a bibliography to your document, see the \mylref{667}{Bibliography Management} section.

\section{Notes and references}
\label{107}

\LaTeXNullTemplate{}



\myhref{https://ru.wikibooks.org/wiki/LaTeX\%2F\%D0\%A1\%D1\%82\%D1\%80\%D1\%83\%D0\%BA\%D1\%82\%D1\%83\%D1\%80\%D0\%B0\%20\%D0\%B4\%D0\%BE\%D0\%BA\%D1\%83\%D0\%BC\%D0\%B5\%D0\%BD\%D1\%82\%D0\%B0}{ru:LaTeX/Структура документа}
\myhref{https://sr.wikibooks.org/wiki/LaTeX\%2F\%D0\%A1\%D1\%82\%D1\%80\%D1\%83\%D0\%BA\%D1\%82\%D1\%83\%D1\%80\%D0\%B0\%20\%D0\%B4\%D0\%BE\%D0\%BA\%D1\%83\%D0\%BC\%D0\%B5\%D0\%BD\%D1\%82\%D0\%B0}{sr:LaTeX/Структура документа}\chapter{Text Formatting}

\myminitoc
\label{108}

\label{109}


This section will guide you through the formatting techniques of the text. {\itshape \setmainfont[Path=/usr/share/fonts/truetype/cmu/,UprightFont=cmunrm.ttf,BoldFont=cmunbx.ttf,ItalicFont=cmunti.ttf,BoldItalicFont=cmunbi.ttf]{cmunti.ttf}\setmonofont[Path=/usr/share/fonts/truetype/cmu/,UprightFont=cmuntt.ttf,BoldFont=cmuntb.ttf,ItalicFont=cmunit.ttf,BoldItalicFont=cmuntx.ttf]{cmunti.ttf}\itshape Formatting}{$\text{ }$}\setmainfont[Path=/usr/share/fonts/truetype/cmu/,UprightFont=cmunrm.ttf,BoldFont=cmunbx.ttf,ItalicFont=cmunti.ttf,BoldItalicFont=cmunbi.ttf]{cmunrm.ttf}\setmonofont[Path=/usr/share/fonts/truetype/cmu/,UprightFont=cmuntt.ttf,BoldFont=cmuntb.ttf,ItalicFont=cmunit.ttf,BoldItalicFont=cmuntx.ttf]{cmunrm.ttf} tends to refer to most things to do with appearance, so it makes the list of possible topics quite eclectic: text style, spacing, etc. If {\itshape \setmainfont[Path=/usr/share/fonts/truetype/cmu/,UprightFont=cmunrm.ttf,BoldFont=cmunbx.ttf,ItalicFont=cmunti.ttf,BoldItalicFont=cmunbi.ttf]{cmunti.ttf}\setmonofont[Path=/usr/share/fonts/truetype/cmu/,UprightFont=cmuntt.ttf,BoldFont=cmuntb.ttf,ItalicFont=cmunit.ttf,BoldItalicFont=cmuntx.ttf]{cmunti.ttf}\itshape formatting}{$\text{ }$}\setmainfont[Path=/usr/share/fonts/truetype/cmu/,UprightFont=cmunrm.ttf,BoldFont=cmunbx.ttf,ItalicFont=cmunti.ttf,BoldItalicFont=cmunbi.ttf]{cmunrm.ttf}\setmonofont[Path=/usr/share/fonts/truetype/cmu/,UprightFont=cmuntt.ttf,BoldFont=cmuntb.ttf,ItalicFont=cmunit.ttf,BoldItalicFont=cmuntx.ttf]{cmunrm.ttf} may also refer to paragraphs and to the page layout, we will focus on the customization of words and sentences for now.

A lot of formatting techniques are required to differentiate certain elements from the rest of the text. It is often necessary to add emphasis to key words or phrases. {\itshape \setmainfont[Path=/usr/share/fonts/truetype/cmu/,UprightFont=cmunrm.ttf,BoldFont=cmunbx.ttf,ItalicFont=cmunti.ttf,BoldItalicFont=cmunbi.ttf]{cmunti.ttf}\setmonofont[Path=/usr/share/fonts/truetype/cmu/,UprightFont=cmuntt.ttf,BoldFont=cmuntb.ttf,ItalicFont=cmunit.ttf,BoldItalicFont=cmuntx.ttf]{cmunti.ttf}\itshape Footnotes}{$\text{ }$}\setmainfont[Path=/usr/share/fonts/truetype/cmu/,UprightFont=cmunrm.ttf,BoldFont=cmunbx.ttf,ItalicFont=cmunti.ttf,BoldItalicFont=cmunbi.ttf]{cmunrm.ttf}\setmonofont[Path=/usr/share/fonts/truetype/cmu/,UprightFont=cmuntt.ttf,BoldFont=cmuntb.ttf,ItalicFont=cmunit.ttf,BoldItalicFont=cmuntx.ttf]{cmunrm.ttf} are useful for providing extra information or clarification without interrupting the main flow of text. So, for these reasons, formatting is very important. However, it is also very easy to abuse, and a document that has been over-{}done can look and read worse than one with none at all.

LaTeX is so flexible that we will actually only skim the surface, as you can have much more control over the presentation of your document if you wish. Having said that, one of the purposes of LaTeX is to take away the stress of having to deal with the physical presentation yourself, so you need not get too carried away!
\section{Spacing}
\label{110}
\subsection{Line Spacing}
\label{111}

If you want to use larger inter-{}line spacing in a document, you can change its value by putting the
\begin{Shaded}
\begin{Highlighting}[]

\NormalTok{\textbackslash{}linespread\{factor\}}
\end{Highlighting}
\end{Shaded}


command into the preamble of your document. Use \LaTeXTT{\textbackslash{}linespread\{1.3\}} for \symbol{34}one and a half\symbol{34} line spacing, and \LaTeXTT{\textbackslash{}linespread\{1.6\}} for \symbol{34}double\symbol{34} line spacing. Normally the lines are not spread, so the default line spread factor is 1. This may not be ideal in all situations: see \myplainurl{http://tex.stackexchange.com/questions/30073/why-is-the-linespread-factor-as-it-is} .

The \LaTeXTT{setspace} package allows more fine-{}grained control over line spacing. To set \symbol{34}one and a half\symbol{34} line spacing document-{}wide, but not where it is usually unnecessary (e.g. footnotes, captions):

\begin{Shaded}
\begin{Highlighting}[]

\NormalTok{\textbackslash{}usepackage\{setspace\}}
\CommentTok{%\textbackslash{}singlespacing}
\NormalTok{\textbackslash{}onehalfspacing}
\CommentTok{%\textbackslash{}doublespacing}
\CommentTok{%\textbackslash{}setstretch\{1.1\}}
\end{Highlighting}
\end{Shaded}


To change line spacing within the document, the {\ttfamily \setmainfont[Path=/usr/share/fonts/truetype/cmu/,UprightFont=cmunrm.ttf,BoldFont=cmunbx.ttf,ItalicFont=cmunti.ttf,BoldItalicFont=cmunbi.ttf]{cmuntt.ttf}\setmonofont[Path=/usr/share/fonts/truetype/cmu/,UprightFont=cmuntt.ttf,BoldFont=cmuntb.ttf,ItalicFont=cmunit.ttf,BoldItalicFont=cmuntx.ttf]{cmuntt.ttf}\ttfamily setspace}{$\text{ }$}\setmainfont[Path=/usr/share/fonts/truetype/cmu/,UprightFont=cmunrm.ttf,BoldFont=cmunbx.ttf,ItalicFont=cmunti.ttf,BoldItalicFont=cmunbi.ttf]{cmunrm.ttf}\setmonofont[Path=/usr/share/fonts/truetype/cmu/,UprightFont=cmuntt.ttf,BoldFont=cmuntb.ttf,ItalicFont=cmunit.ttf,BoldItalicFont=cmuntx.ttf]{cmunrm.ttf} package provides the environments {\ttfamily \setmainfont[Path=/usr/share/fonts/truetype/cmu/,UprightFont=cmunrm.ttf,BoldFont=cmunbx.ttf,ItalicFont=cmunti.ttf,BoldItalicFont=cmunbi.ttf]{cmuntt.ttf}\setmonofont[Path=/usr/share/fonts/truetype/cmu/,UprightFont=cmuntt.ttf,BoldFont=cmuntb.ttf,ItalicFont=cmunit.ttf,BoldItalicFont=cmuntx.ttf]{cmuntt.ttf}\ttfamily singlespace}\setmainfont[Path=/usr/share/fonts/truetype/cmu/,UprightFont=cmunrm.ttf,BoldFont=cmunbx.ttf,ItalicFont=cmunti.ttf,BoldItalicFont=cmunbi.ttf]{cmunrm.ttf}\setmonofont[Path=/usr/share/fonts/truetype/cmu/,UprightFont=cmuntt.ttf,BoldFont=cmuntb.ttf,ItalicFont=cmunit.ttf,BoldItalicFont=cmuntx.ttf]{cmunrm.ttf}, {\ttfamily \setmainfont[Path=/usr/share/fonts/truetype/cmu/,UprightFont=cmunrm.ttf,BoldFont=cmunbx.ttf,ItalicFont=cmunti.ttf,BoldItalicFont=cmunbi.ttf]{cmuntt.ttf}\setmonofont[Path=/usr/share/fonts/truetype/cmu/,UprightFont=cmuntt.ttf,BoldFont=cmuntb.ttf,ItalicFont=cmunit.ttf,BoldItalicFont=cmuntx.ttf]{cmuntt.ttf}\ttfamily onehalfspace}\setmainfont[Path=/usr/share/fonts/truetype/cmu/,UprightFont=cmunrm.ttf,BoldFont=cmunbx.ttf,ItalicFont=cmunti.ttf,BoldItalicFont=cmunbi.ttf]{cmunrm.ttf}\setmonofont[Path=/usr/share/fonts/truetype/cmu/,UprightFont=cmuntt.ttf,BoldFont=cmuntb.ttf,ItalicFont=cmunit.ttf,BoldItalicFont=cmuntx.ttf]{cmunrm.ttf}, {\ttfamily \setmainfont[Path=/usr/share/fonts/truetype/cmu/,UprightFont=cmunrm.ttf,BoldFont=cmunbx.ttf,ItalicFont=cmunti.ttf,BoldItalicFont=cmunbi.ttf]{cmuntt.ttf}\setmonofont[Path=/usr/share/fonts/truetype/cmu/,UprightFont=cmuntt.ttf,BoldFont=cmuntb.ttf,ItalicFont=cmunit.ttf,BoldItalicFont=cmuntx.ttf]{cmuntt.ttf}\ttfamily doublespace}{$\text{ }$}\setmainfont[Path=/usr/share/fonts/truetype/cmu/,UprightFont=cmunrm.ttf,BoldFont=cmunbx.ttf,ItalicFont=cmunti.ttf,BoldItalicFont=cmunbi.ttf]{cmunrm.ttf}\setmonofont[Path=/usr/share/fonts/truetype/cmu/,UprightFont=cmuntt.ttf,BoldFont=cmuntb.ttf,ItalicFont=cmunit.ttf,BoldItalicFont=cmuntx.ttf]{cmunrm.ttf} and {\ttfamily \setmainfont[Path=/usr/share/fonts/truetype/cmu/,UprightFont=cmunrm.ttf,BoldFont=cmunbx.ttf,ItalicFont=cmunti.ttf,BoldItalicFont=cmunbi.ttf]{cmuntt.ttf}\setmonofont[Path=/usr/share/fonts/truetype/cmu/,UprightFont=cmuntt.ttf,BoldFont=cmuntb.ttf,ItalicFont=cmunit.ttf,BoldItalicFont=cmuntx.ttf]{cmuntt.ttf}\ttfamily spacing}\setmainfont[Path=/usr/share/fonts/truetype/cmu/,UprightFont=cmunrm.ttf,BoldFont=cmunbx.ttf,ItalicFont=cmunti.ttf,BoldItalicFont=cmunbi.ttf]{cmunrm.ttf}\setmonofont[Path=/usr/share/fonts/truetype/cmu/,UprightFont=cmuntt.ttf,BoldFont=cmuntb.ttf,ItalicFont=cmunit.ttf,BoldItalicFont=cmuntx.ttf]{cmunrm.ttf}:

\begin{Shaded}
\begin{Highlighting}[]

\NormalTok{This paragraph has \textbackslash{}\textbackslash{} default \textbackslash{}\textbackslash{} line spacing.}
 
\NormalTok{\textbackslash{}begin\{doublespace\}}
  \NormalTok{This paragraph has \textbackslash{}\textbackslash{} double \textbackslash{}\textbackslash{} line spacing.}
\NormalTok{\textbackslash{}end\{doublespace\}}
 
\NormalTok{\textbackslash{}begin\{spacing\}\{2.5\}}
  \NormalTok{This paragraph has \textbackslash{}\textbackslash{} huge gaps \textbackslash{}\textbackslash{} between lines.}
\NormalTok{\textbackslash{}end\{spacing\}}
\end{Highlighting}
\end{Shaded}


\begin{TemplateInfo}{\danger}{Warning}The line spacing value is contained in the \LaTeXTT{\textbackslash{}baselineskip} \mylref{456}{length}, but it is not recommended to change its value since it will have an impact on other types of content than paragraphs, which will result in an undesired effect.\end{TemplateInfo}
\subsection{Non-{}breaking spaces}
\label{112}

This {\itshape \setmainfont[Path=/usr/share/fonts/truetype/cmu/,UprightFont=cmunrm.ttf,BoldFont=cmunbx.ttf,ItalicFont=cmunti.ttf,BoldItalicFont=cmunbi.ttf]{cmunti.ttf}\setmonofont[Path=/usr/share/fonts/truetype/cmu/,UprightFont=cmuntt.ttf,BoldFont=cmuntb.ttf,ItalicFont=cmunit.ttf,BoldItalicFont=cmuntx.ttf]{cmunti.ttf}\itshape essential}{$\text{ }$}\setmainfont[Path=/usr/share/fonts/truetype/cmu/,UprightFont=cmunrm.ttf,BoldFont=cmunbx.ttf,ItalicFont=cmunti.ttf,BoldItalicFont=cmunbi.ttf]{cmunrm.ttf}\setmonofont[Path=/usr/share/fonts/truetype/cmu/,UprightFont=cmuntt.ttf,BoldFont=cmuntb.ttf,ItalicFont=cmunit.ttf,BoldItalicFont=cmuntx.ttf]{cmunrm.ttf} feature is a bit unknown to newcomers, although it is available on most WYSIWYG document processors. A non-{}breaking space between two tokens (e.g. words, punctuation marks) prevents the processors from inserting a line break between them. Besides a non-{}breaking space cannot be enlarged. It is very important for a consistent reading.

LaTeX uses the \textquotesingle{}\~{}\textquotesingle{} symbol as a non-{}breaking space.
You would  usually use non-{}breaking spaces for punctuation marks in some languages, for units and currencies, for initials, etc.
In French typography, you would put a non-{}breaking space before all two-{}parts punctuation marks.

Examples:
\begin{Shaded}
\begin{Highlighting}[]

\NormalTok{D.~\textbackslash{}textsc\{Knuth\}}
\NormalTok{50~€}
\end{Highlighting}
\end{Shaded}

\subsection{Space between words and sentences}
\label{113}

To get a straight right margin in the output, LaTeX inserts varying amounts of space between the words. By default, it also inserts slightly more space at the end of a sentence. However, the extra space added at the end of sentences is generally considered typographically old-{}fashioned in English language printing. (The practice is found in nineteenth century design and in twentieth century typewriter styles.) Most modern typesetters treat the end of sentence space the same as the interword space. (See for example, Bringhurst\textquotesingle{}s {\itshape \setmainfont[Path=/usr/share/fonts/truetype/cmu/,UprightFont=cmunrm.ttf,BoldFont=cmunbx.ttf,ItalicFont=cmunti.ttf,BoldItalicFont=cmunbi.ttf]{cmunti.ttf}\setmonofont[Path=/usr/share/fonts/truetype/cmu/,UprightFont=cmuntt.ttf,BoldFont=cmuntb.ttf,ItalicFont=cmunit.ttf,BoldItalicFont=cmuntx.ttf]{cmunti.ttf}\itshape Elements of Typographic Style}\setmainfont[Path=/usr/share/fonts/truetype/cmu/,UprightFont=cmunrm.ttf,BoldFont=cmunbx.ttf,ItalicFont=cmunti.ttf,BoldItalicFont=cmunbi.ttf]{cmunrm.ttf}\setmonofont[Path=/usr/share/fonts/truetype/cmu/,UprightFont=cmuntt.ttf,BoldFont=cmuntb.ttf,ItalicFont=cmunit.ttf,BoldItalicFont=cmuntx.ttf]{cmunrm.ttf}.) The additional space after periods can be disabled with the command

\begin{Shaded}
\begin{Highlighting}[]

\NormalTok{\textbackslash{}frenchspacing}
\end{Highlighting}
\end{Shaded}


which tells LaTeX not to insert more space after a period than after ordinary character. Frenchspacing can be turned off later in your document via the \LaTeXTT{\textbackslash{}nonfrenchspacing} command.

If an author wishes to use the wider end-{}of-{}sentence spacing, care must be exercised so that punctuation marks are not misinterpreted as ends of sentences. TeX assumes that sentences end with periods, question marks or exclamation marks. Although if a period follows an uppercase letter, this is not taken as a sentence ending, since periods after uppercase letters normally occur in abbreviations. Any exception from these assumptions has to be specified by the author. A backslash in front of a space generates a space that will not be enlarged. A tilde ‘\LaTeXTT{\~{}}’ character generates a non-{}breaking space. The command \LaTeXTT{\textbackslash{}@} in front of a period specifies that this period terminates a sentence even when it follows an uppercase letter. (If you are using \LaTeXTT{\textbackslash{}frenchspacing}, then none of these exceptions need be specified.)
\subsection{Stretched spaces}
\label{114}

You can insert a horizontal stretched space with \LaTeXTT{\textbackslash{}hfill} in a line so that the rest gets \symbol{34}pushed\symbol{34} toward the right margin.
For instance this may be useful in the header.
\begin{Shaded}
\begin{Highlighting}[]

\NormalTok{Author Name \textbackslash{}hfill \textbackslash{}today}
\end{Highlighting}
\end{Shaded}


Similarly you can insert vertical stretched space with \LaTeXTT{\textbackslash{}vfill}. It may be useful for special pages.
\begin{Shaded}
\begin{Highlighting}[]

\NormalTok{\textbackslash{}maketitle}
\NormalTok{\textbackslash{}vfill}
\NormalTok{\textbackslash{}tableofcontents}
\NormalTok{\textbackslash{}clearpage}
 
\NormalTok{\textbackslash{}section\{My first section\}}
\CommentTok{% ...}
\end{Highlighting}
\end{Shaded}


See \mylref{456}{Lengths} for more details.
\subsection{Manual spacing}
\label{115}

The spaces between words and sentences, between paragraphs, sections, subsections, etc. is determined automatically by LaTeX. It is against LaTeX philosophy to insert spaces manually and will usually lead to bad formatting.
Manual spacing is a matter of macro writing and package creation.

See \mylref{456}{Lengths} for more details.
\section{Hyphenation}
\label{116}

LaTeX hyphenates words whenever necessary. Hyphenation rules will vary for different languages. LaTeX only supports English by default, so if you want to have correct hyphenation rules for your desired language, see \mylref{209}{Internationalization}.

If the hyphenation algorithm does not find the correct hyphenation points, you can remedy the situation
by using the following commands to tell TeX about the exception. The command
\begin{Shaded}
\begin{Highlighting}[]
\NormalTok{\textbackslash{}hyphenation\{word list\} }
\end{Highlighting}
\end{Shaded}

causes the words listed in the argument to be hyphenated only at the points marked by “-{}”. The argument of the command should only contain words built from normal letters, or rather characters that are considered to be normal letters by LaTeX. It is known that the hyphenation algorithm does not find all correct American English hyphenation points for several words. A log of known exceptions is published periodically in the {\itshape \setmainfont[Path=/usr/share/fonts/truetype/cmu/,UprightFont=cmunrm.ttf,BoldFont=cmunbx.ttf,ItalicFont=cmunti.ttf,BoldItalicFont=cmunbi.ttf]{cmunti.ttf}\setmonofont[Path=/usr/share/fonts/truetype/cmu/,UprightFont=cmuntt.ttf,BoldFont=cmuntb.ttf,ItalicFont=cmunit.ttf,BoldItalicFont=cmuntx.ttf]{cmunti.ttf}\itshape TUGboat}{$\text{ }$}\setmainfont[Path=/usr/share/fonts/truetype/cmu/,UprightFont=cmunrm.ttf,BoldFont=cmunbx.ttf,ItalicFont=cmunti.ttf,BoldItalicFont=cmunbi.ttf]{cmunrm.ttf}\setmonofont[Path=/usr/share/fonts/truetype/cmu/,UprightFont=cmuntt.ttf,BoldFont=cmuntb.ttf,ItalicFont=cmunit.ttf,BoldItalicFont=cmuntx.ttf]{cmunrm.ttf} journal. (2012 list: \myplainurl{https://www.tug.org/TUGboat/tb33-1/tb103hyf.pdf)}

The hyphenation hints are stored for the language that is active when the hyphenation command occurs. This means that if you place a hyphenation command into the preamble of your document it will influence the English language hyphenation. If you place the command after the \LaTeXTT{\textbackslash{}begin\{document\}} and you are using some package for national language support like babel, then the hyphenation hints will be active in the language activated through babel. The example below will allow “hyphenation” to be hyphenated as well as “Hyphenation”, and it prevents “FORTRAN”, “Fortran” and “fortran” from being hyphenated at all. No special characters or symbols are allowed in the argument. Example:
\begin{Shaded}
\begin{Highlighting}[]

\NormalTok{\textbackslash{}hyphenation\{FORTRAN Hy-phen-a-tion\}}
\end{Highlighting}
\end{Shaded}


The command \LaTeXTT{\textbackslash{}-{}} inserts a discretionary hyphen into a word.  This also becomes the only point where hyphenation is allowed in this word.  This command is especially useful for words containing special characters (e.g., accented characters), because LaTeX does not automatically hyphenate words containing special characters.
\begin{longtable}{p{1.0\linewidth}}
\begin{Shaded}
\begin{Highlighting}[]

\NormalTok{\textbackslash{}begin\{minipage\}\{2in\}}
\NormalTok{I think this is: su\textbackslash{}-per\textbackslash{}-cal\textbackslash{}-}\CommentTok
\NormalTok{al\textbackslash{}-i\textbackslash{}-do\textbackslash{}-cious}
\NormalTok{\textbackslash{}end\{minipage\}}
\end{Highlighting}
\end{Shaded}
\\
{$\begin{array}{l}\mbox{I think this is: supercalifragi-}\\ \mbox{listicexpialidocious}\end{array}$}

\end{longtable}

LaTeX does not hyphenate compound words that contain a dash\myfootnote{\LaTeXTT{hyphenat} package documentation, p3}. There are two packages that can add back flexibility. The \LaTeXTT{hyphenat} package supplies the \LaTeXTT{\textbackslash{}hyp} command. This command typesets the dash and then subjects the constituent words to automatic hyphenation.  After loading the package:
\begin{Shaded}
\begin{Highlighting}[]

\NormalTok{\textbackslash{}usepackage\{hyphenat\}}
\end{Highlighting}
\end{Shaded}


one should write, instead of electromagnetic-{}endioscopy:
\begin{Shaded}
\begin{Highlighting}[]

\NormalTok{electromagnetic\textbackslash{}hyp\{\}endioscopy}
\end{Highlighting}
\end{Shaded}


The \LaTeXTT{extdash} package also offers features for controlling the hyphenation of compound words containing dashes {\mbox{$\text{---}$}} as opposed to the words themselves which it leaves to LaTeX.  The \LaTeXTT{shortcuts} option enables a more compressed syntax:
\begin{Shaded}
\begin{Highlighting}[]

\NormalTok{\textbackslash{}usepackage[shortcuts]\{extdash\}}
\end{Highlighting}
\end{Shaded}


Typical usage is as follows, assuming the compressed syntax.  In both cases, LaTeX can break and hyphenate the constituent words, but in the latter case, it will not break after the L:
\begin{Shaded}
\begin{Highlighting}[]

\NormalTok{electromagnetic\textbackslash{}-/endioscopy}
\NormalTok{L\textbackslash{}=/approximation}
\end{Highlighting}
\end{Shaded}


One or more words can be kept together on the {\bfseries \setmainfont[Path=/usr/share/fonts/truetype/cmu/,UprightFont=cmunrm.ttf,BoldFont=cmunbx.ttf,ItalicFont=cmunti.ttf,BoldItalicFont=cmunbi.ttf]{cmunbx.ttf}\setmonofont[Path=/usr/share/fonts/truetype/cmu/,UprightFont=cmuntt.ttf,BoldFont=cmuntb.ttf,ItalicFont=cmunit.ttf,BoldItalicFont=cmuntx.ttf]{cmunbx.ttf}\bfseries one line}{$\text{ }$}\setmainfont[Path=/usr/share/fonts/truetype/cmu/,UprightFont=cmunrm.ttf,BoldFont=cmunbx.ttf,ItalicFont=cmunti.ttf,BoldItalicFont=cmunbi.ttf]{cmunrm.ttf}\setmonofont[Path=/usr/share/fonts/truetype/cmu/,UprightFont=cmuntt.ttf,BoldFont=cmuntb.ttf,ItalicFont=cmunit.ttf,BoldItalicFont=cmuntx.ttf]{cmunrm.ttf} with the standard LaTeX command:
\begin{Shaded}
\begin{Highlighting}[]

\NormalTok{\textbackslash{}mbox\{text\}}
\end{Highlighting}
\end{Shaded}


This prevents hyphenation and causes its argument to be kept together under all circumstances.  For example:
\begin{Shaded}
\begin{Highlighting}[]

\NormalTok{My phone number will change soon. It will be \textbackslash{}mbox\{0116 291 2319\}.}
\end{Highlighting}
\end{Shaded}


\LaTeXTT{\textbackslash{}fbox} is similar to \LaTeXTT{\textbackslash{}mbox}, but in addition there will be a visible box drawn around the content.

To avoid hyphenation altogether, the penalty for hyphenation can be set to an extreme value:
\begin{Shaded}
\begin{Highlighting}[]

\NormalTok{\textbackslash{}hyphenpenalty=100000}
\end{Highlighting}
\end{Shaded}


You can change the degree to which LaTeX will hyphenate by changing the value of \LaTeXTT{\textbackslash{}tolerance=1000} and \LaTeXTT{\textbackslash{}hyphenpenalty=1000}.
You\textquotesingle{}ll have to experiment with the values to achieve the desired effect. A document which has a low tolerance value will cause LaTeX not to tolerate uneven spacing between words, hyphenating words more frequently than in documents with higher tolerances.
Also note that using a higher text width will decrease the probability of encountering badly hyphenated word. For example adding
\begin{Shaded}
\begin{Highlighting}[]

\NormalTok{\textbackslash{}usepackage\{geometry\}}
\end{Highlighting}
\end{Shaded}

will widen the text width and reduce the amount of margin overruns.
\section{Quote-{}marks}
\label{117}

LaTeX treats left and right quotes as different entities. For single quotes, a grave accent, {\ttfamily \setmainfont[Path=/usr/share/fonts/truetype/cmu/,UprightFont=cmunrm.ttf,BoldFont=cmunbx.ttf,ItalicFont=cmunti.ttf,BoldItalicFont=cmunbi.ttf]{cmuntt.ttf}\setmonofont[Path=/usr/share/fonts/truetype/cmu/,UprightFont=cmuntt.ttf,BoldFont=cmuntb.ttf,ItalicFont=cmunit.ttf,BoldItalicFont=cmuntx.ttf]{cmuntt.ttf}\ttfamily `}{$\text{ }$}\setmainfont[Path=/usr/share/fonts/truetype/cmu/,UprightFont=cmunrm.ttf,BoldFont=cmunbx.ttf,ItalicFont=cmunti.ttf,BoldItalicFont=cmunbi.ttf]{cmunrm.ttf}\setmonofont[Path=/usr/share/fonts/truetype/cmu/,UprightFont=cmuntt.ttf,BoldFont=cmuntb.ttf,ItalicFont=cmunit.ttf,BoldItalicFont=cmuntx.ttf]{cmunrm.ttf} (on American keyboards, this symbol is found on the tilde key; adjacent to the number 1 key on most keyboards) gives a left quote mark, and an apostrophe, {\ttfamily \setmainfont[Path=/usr/share/fonts/truetype/cmu/,UprightFont=cmunrm.ttf,BoldFont=cmunbx.ttf,ItalicFont=cmunti.ttf,BoldItalicFont=cmunbi.ttf]{cmuntt.ttf}\setmonofont[Path=/usr/share/fonts/truetype/cmu/,UprightFont=cmuntt.ttf,BoldFont=cmuntb.ttf,ItalicFont=cmunit.ttf,BoldItalicFont=cmuntx.ttf]{cmuntt.ttf}\ttfamily \textquotesingle{}}{$\text{ }$}\setmainfont[Path=/usr/share/fonts/truetype/cmu/,UprightFont=cmunrm.ttf,BoldFont=cmunbx.ttf,ItalicFont=cmunti.ttf,BoldItalicFont=cmunbi.ttf]{cmunrm.ttf}\setmonofont[Path=/usr/share/fonts/truetype/cmu/,UprightFont=cmuntt.ttf,BoldFont=cmuntb.ttf,ItalicFont=cmunit.ttf,BoldItalicFont=cmuntx.ttf]{cmunrm.ttf} gives a right. For double quotes, simply double the symbols, and LaTeX will interpret them accordingly. (Don\textquotesingle{}t use the {\ttfamily \setmainfont[Path=/usr/share/fonts/truetype/cmu/,UprightFont=cmunrm.ttf,BoldFont=cmunbx.ttf,ItalicFont=cmunti.ttf,BoldItalicFont=cmunbi.ttf]{cmuntt.ttf}\setmonofont[Path=/usr/share/fonts/truetype/cmu/,UprightFont=cmuntt.ttf,BoldFont=cmuntb.ttf,ItalicFont=cmunit.ttf,BoldItalicFont=cmuntx.ttf]{cmuntt.ttf}\ttfamily \symbol{34}}{$\text{ }$}\setmainfont[Path=/usr/share/fonts/truetype/cmu/,UprightFont=cmunrm.ttf,BoldFont=cmunbx.ttf,ItalicFont=cmunti.ttf,BoldItalicFont=cmunbi.ttf]{cmunrm.ttf}\setmonofont[Path=/usr/share/fonts/truetype/cmu/,UprightFont=cmuntt.ttf,BoldFont=cmuntb.ttf,ItalicFont=cmunit.ttf,BoldItalicFont=cmuntx.ttf]{cmunrm.ttf} for right double quotes: when the {\ttfamily \setmainfont[Path=/usr/share/fonts/truetype/cmu/,UprightFont=cmunrm.ttf,BoldFont=cmunbx.ttf,ItalicFont=cmunti.ttf,BoldItalicFont=cmunbi.ttf]{cmuntt.ttf}\setmonofont[Path=/usr/share/fonts/truetype/cmu/,UprightFont=cmuntt.ttf,BoldFont=cmuntb.ttf,ItalicFont=cmunit.ttf,BoldItalicFont=cmuntx.ttf]{cmuntt.ttf}\ttfamily babel}{$\text{ }$}\setmainfont[Path=/usr/share/fonts/truetype/cmu/,UprightFont=cmunrm.ttf,BoldFont=cmunbx.ttf,ItalicFont=cmunti.ttf,BoldItalicFont=cmunbi.ttf]{cmunrm.ttf}\setmonofont[Path=/usr/share/fonts/truetype/cmu/,UprightFont=cmuntt.ttf,BoldFont=cmuntb.ttf,ItalicFont=cmunit.ttf,BoldItalicFont=cmuntx.ttf]{cmunrm.ttf} package is used for some languages (e.g. German), the {\ttfamily \setmainfont[Path=/usr/share/fonts/truetype/cmu/,UprightFont=cmunrm.ttf,BoldFont=cmunbx.ttf,ItalicFont=cmunti.ttf,BoldItalicFont=cmunbi.ttf]{cmuntt.ttf}\setmonofont[Path=/usr/share/fonts/truetype/cmu/,UprightFont=cmuntt.ttf,BoldFont=cmuntb.ttf,ItalicFont=cmunit.ttf,BoldItalicFont=cmuntx.ttf]{cmuntt.ttf}\ttfamily \symbol{34}}{$\text{ }$}\setmainfont[Path=/usr/share/fonts/truetype/cmu/,UprightFont=cmunrm.ttf,BoldFont=cmunbx.ttf,ItalicFont=cmunti.ttf,BoldItalicFont=cmunbi.ttf]{cmunrm.ttf}\setmonofont[Path=/usr/share/fonts/truetype/cmu/,UprightFont=cmuntt.ttf,BoldFont=cmuntb.ttf,ItalicFont=cmunit.ttf,BoldItalicFont=cmuntx.ttf]{cmunrm.ttf} is redefined to produce an umlaut accent; using {\ttfamily \setmainfont[Path=/usr/share/fonts/truetype/cmu/,UprightFont=cmunrm.ttf,BoldFont=cmunbx.ttf,ItalicFont=cmunti.ttf,BoldItalicFont=cmunbi.ttf]{cmuntt.ttf}\setmonofont[Path=/usr/share/fonts/truetype/cmu/,UprightFont=cmuntt.ttf,BoldFont=cmuntb.ttf,ItalicFont=cmunit.ttf,BoldItalicFont=cmuntx.ttf]{cmuntt.ttf}\ttfamily \symbol{34}}{$\text{ }$}\setmainfont[Path=/usr/share/fonts/truetype/cmu/,UprightFont=cmunrm.ttf,BoldFont=cmunbx.ttf,ItalicFont=cmunti.ttf,BoldItalicFont=cmunbi.ttf]{cmunrm.ttf}\setmonofont[Path=/usr/share/fonts/truetype/cmu/,UprightFont=cmuntt.ttf,BoldFont=cmuntb.ttf,ItalicFont=cmunit.ttf,BoldItalicFont=cmuntx.ttf]{cmunrm.ttf} for right double quotes will either lead to bad spacing or it being used to produce an umlaut). On British keyboards, \textquotesingle{} {\ttfamily \setmainfont[Path=/usr/share/fonts/truetype/cmu/,UprightFont=cmunrm.ttf,BoldFont=cmunbx.ttf,ItalicFont=cmunti.ttf,BoldItalicFont=cmunbi.ttf]{cmuntt.ttf}\setmonofont[Path=/usr/share/fonts/truetype/cmu/,UprightFont=cmuntt.ttf,BoldFont=cmuntb.ttf,ItalicFont=cmunit.ttf,BoldItalicFont=cmuntx.ttf]{cmuntt.ttf}\ttfamily `}{$\text{ }$}\setmainfont[Path=/usr/share/fonts/truetype/cmu/,UprightFont=cmunrm.ttf,BoldFont=cmunbx.ttf,ItalicFont=cmunti.ttf,BoldItalicFont=cmunbi.ttf]{cmunrm.ttf}\setmonofont[Path=/usr/share/fonts/truetype/cmu/,UprightFont=cmuntt.ttf,BoldFont=cmuntb.ttf,ItalicFont=cmunit.ttf,BoldItalicFont=cmuntx.ttf]{cmunrm.ttf} \textquotesingle{} is left of the \textquotesingle{} 1 \textquotesingle{} key and shares the key with \textquotesingle{} ¬ \textquotesingle{}, and sometimes \textquotesingle{} ¦ \textquotesingle{} or \textquotesingle{} | \textquotesingle{}. The apostrophe (\textquotesingle{}) key is to the right of the colon/semicolon key and shares it with the \textquotesingle{} @ \textquotesingle{} symbol.


{\scalefont{0.80877}\begin{longtable}{>{\RaggedRight}p{0.45609\linewidth}>{\RaggedRight}p{0.48677\linewidth}} 
\hspace*{0pt}\ignorespaces{}\hspace*{0pt} \LaTeXTT{To `quote\textquotesingle{} in LaTeX}&\hspace*{0pt}\ignorespaces{}\hspace*{0pt}\begin{minipage}{1.0\linewidth}\begin{center}\includegraphics[width=1.0\linewidth,height=6.5in,keepaspectratio]{../images/8.png}\end{center}\myfigurewithoutcaption{8}\end{minipage}\\ \hspace*{0pt}\ignorespaces{}\hspace*{0pt} \LaTeXTT{To ``quote\textquotesingle{}\textquotesingle{} in LaTeX}&\hspace*{0pt}\ignorespaces{}\hspace*{0pt}\begin{minipage}{1.0\linewidth}\begin{center}\includegraphics[width=1.0\linewidth,height=6.5in,keepaspectratio]{../images/9.png}\end{center}\myfigurewithoutcaption{9}\end{minipage}\\ \hspace*{0pt}\ignorespaces{}\hspace*{0pt} \LaTeXTT{To ``quote\symbol{34} in LaTeX}&\hspace*{0pt}\ignorespaces{}\hspace*{0pt}\begin{minipage}{1.0\linewidth}\begin{center}\includegraphics[width=1.0\linewidth,height=6.5in,keepaspectratio]{../images/10.png}\end{center}\myfigurewithoutcaption{10}\end{minipage}\\ \hspace*{0pt}\ignorespaces{}\hspace*{0pt} \LaTeXTT{To ,,quote\textquotesingle{}\textquotesingle{} in LaTeX}&\hspace*{0pt}\ignorespaces{}\hspace*{0pt}\begin{minipage}{1.0\linewidth}\begin{center}\includegraphics[width=1.0\linewidth,height=6.5in,keepaspectratio]{../images/11.png}\end{center}\myfigurewithoutcaption{11}\end{minipage}\\ \hspace*{0pt}\ignorespaces{}\hspace*{0pt}\LaTeXTT{,,German quotation marks``}&\hspace*{0pt}\ignorespaces{}\hspace*{0pt}\begin{minipage}{1.0\linewidth}\begin{center}\includegraphics[width=1.0\linewidth,height=6.5in,keepaspectratio]{../images/12.png}\end{center}\myfigurewithoutcaption{12}\end{minipage}\\ \hspace*{0pt}\ignorespaces{}\hspace*{0pt}\LaTeXTT{<{}<{}French quotation marks>{}>{}}&\hspace*{0pt}\ignorespaces{}\hspace*{0pt}\begin{minipage}{1.0\linewidth}\begin{center}\includegraphics[width=1.0\linewidth,height=6.5in,keepaspectratio]{../images/13.png}\end{center}\myfigurewithoutcaption{13}\end{minipage}\\ \hspace*{0pt}\ignorespaces{}\hspace*{0pt}\LaTeXTT{``Please press the `x\textquotesingle{} key.\textquotesingle{}\textquotesingle{}}&\hspace*{0pt}\ignorespaces{}\hspace*{0pt}\includegraphics[width=0.04250\textwidth,height=6.5in,keepaspectratio]{../images/14.png} \\ \hspace*{0pt}\ignorespaces{}\hspace*{0pt}\LaTeXTT{,,Proszę, naciśnij klawisz <{}<{}x>{}>{}\textquotesingle{}\textquotesingle{}.}&\hspace*{0pt}\ignorespaces{}\hspace*{0pt}\begin{minipage}{1.0\linewidth}\begin{center}\includegraphics[width=1.0\linewidth,height=6.5in,keepaspectratio]{../images/15.png}\end{center}\myfigurewithoutcaption{15}\end{minipage} 
\end{longtable}
}

The right quote is also used for apostrophe in LaTeX without trouble.

For left bottom quote and European quoting style you need to use T1 font encoding enabled by:
\begin{Shaded}
\begin{Highlighting}[]

\NormalTok{\textbackslash{}usepackage[T1]\{fontenc\}}
\end{Highlighting}
\end{Shaded}


See \mylref{163}{Fonts} for more details on font encoding.

The package \LaTeXTT{csquotes} offers a multilingual solution to quotations, with integration to citation mechanisms offered by BibTeX. This package allows one for example to switch languages and quotation styles according to babel language selections.
\section{Diacritics and accents}
\label{118}

Most accents and diacritics may be inserted with direct keyboard input by configuring the preamble properly. For symbols unavailable on your keyboard,
diacritics may be added to letters by placing special escaped metacharacters before the letter that requires the diacritic.

See \mylref{192}{Special Characters}.
\section{Margin misalignment and interword spacing}
\label{119}

Some very long words, numbers or URLs may not be hyphenated properly and move far beyond the side margin. One solution for this problem is to use \LaTeXTT{sloppypar} environment, which tells LaTeX to adjust word spacing less strictly. As a result, some spaces between words may be a bit too large, but long words will be placed properly.

\begin{longtable}{p{1.0\linewidth}}
\begin{Shaded}
\begin{Highlighting}[]

\NormalTok{This is a paragraph with}
\NormalTok{a very long word ABCDEFGHIJKLMNOPRST;}
\NormalTok{then we have another bad thing}
\NormalTok{--- a long number 1234567890123456789.}
 
\NormalTok{\textbackslash{}begin\{sloppypar\}}
\NormalTok{This is a paragraph with}
\NormalTok{a very long word ABCDEFGHIJKLMNOPRST;}
\NormalTok{then we have an another bad thing}
\NormalTok{--- a long number 1234567890123456789.}
\NormalTok{\textbackslash{}end\{sloppypar\}}
\end{Highlighting}
\end{Shaded}
\\



\begin{minipage}{1.0\linewidth}
\begin{center}
\includegraphics[width=1.0\linewidth,height=6.5in,keepaspectratio]{../images/16.png}
\end{center}
\raggedright{}\myfigurewithcaption{16}{border}
\end{minipage}\vspace{0.75cm}



\end{longtable}

Another solution is to edit the text to avoid long words, numbers or URLs approaching the side margin.
\section{Ligatures}
\label{120}
Some letter combinations are typeset not just by setting the different letters one after the other, but by actually using special symbols (like \symbol{34}ff\symbol{34}), called \myhref{https://en.wikipedia.org/wiki/Typographical\%20ligature}{ligatures}.
Ligatures can be prohibited by inserting \LaTeXTT{\{\}} or, if this does not work, \LaTeXTT{\{\textbackslash{}kern0pt\}} between the two letters in question. This might be necessary with words built from two words. Here is an example:

\begin{longtable}{p{1.0\linewidth}}
\begin{Shaded}
\begin{Highlighting}[]

\NormalTok{\textbackslash{}Large Not shelfful\textbackslash{}\textbackslash{}}
\NormalTok{but shelf\{\}ful}
\end{Highlighting}
\end{Shaded}
\\



\begin{minipage}{0.50000\textwidth}
\begin{center}
\includegraphics[width=1.0\textwidth,height=6.5in,keepaspectratio]{../images/17.png}
\end{center}
\raggedright{}\myfigurewithoutcaption{17}
\end{minipage}\vspace{0.75cm}



\end{longtable}

Ligatures can interfere with some text-{}search tools (a search for {\ttfamily \setmainfont[Path=/usr/share/fonts/truetype/cmu/,UprightFont=cmunrm.ttf,BoldFont=cmunbx.ttf,ItalicFont=cmunti.ttf,BoldItalicFont=cmunbi.ttf]{cmuntt.ttf}\setmonofont[Path=/usr/share/fonts/truetype/cmu/,UprightFont=cmuntt.ttf,BoldFont=cmuntb.ttf,ItalicFont=cmunit.ttf,BoldItalicFont=cmuntx.ttf]{cmuntt.ttf}\ttfamily \symbol{34}{\bfseries \setmainfont[Path=/usr/share/fonts/truetype/cmu/,UprightFont=cmunrm.ttf,BoldFont=cmunbx.ttf,ItalicFont=cmunti.ttf,BoldItalicFont=cmunbi.ttf]{cmuntb.ttf}\setmonofont[Path=/usr/share/fonts/truetype/cmu/,UprightFont=cmuntt.ttf,BoldFont=cmuntb.ttf,ItalicFont=cmunit.ttf,BoldItalicFont=cmuntx.ttf]{cmuntb.ttf}\ttfamily \bfseries fi}\setmainfont[Path=/usr/share/fonts/truetype/cmu/,UprightFont=cmunrm.ttf,BoldFont=cmunbx.ttf,ItalicFont=cmunti.ttf,BoldItalicFont=cmunbi.ttf]{cmuntt.ttf}\setmonofont[Path=/usr/share/fonts/truetype/cmu/,UprightFont=cmuntt.ttf,BoldFont=cmuntb.ttf,ItalicFont=cmunit.ttf,BoldItalicFont=cmuntx.ttf]{cmuntt.ttf}\ttfamily nally\symbol{34}}{$\text{ }$}\setmainfont[Path=/usr/share/fonts/truetype/cmu/,UprightFont=cmunrm.ttf,BoldFont=cmunbx.ttf,ItalicFont=cmunti.ttf,BoldItalicFont=cmunbi.ttf]{cmunrm.ttf}\setmonofont[Path=/usr/share/fonts/truetype/cmu/,UprightFont=cmuntt.ttf,BoldFont=cmuntb.ttf,ItalicFont=cmunit.ttf,BoldItalicFont=cmuntx.ttf]{cmunrm.ttf} wouldn\textquotesingle{}t find the string {\ttfamily \setmainfont[Path=/usr/share/fonts/truetype/cmu/,UprightFont=cmunrm.ttf,BoldFont=cmunbx.ttf,ItalicFont=cmunti.ttf,BoldItalicFont=cmunbi.ttf]{cmuntt.ttf}\setmonofont[Path=/usr/share/fonts/truetype/cmu/,UprightFont=cmuntt.ttf,BoldFont=cmuntb.ttf,ItalicFont=cmunit.ttf,BoldItalicFont=cmuntx.ttf]{cmuntt.ttf}\ttfamily \symbol{34}{\bfseries \setmainfont[Path=/usr/share/fonts/truetype/cmu/,UprightFont=cmunrm.ttf,BoldFont=cmunbx.ttf,ItalicFont=cmunti.ttf,BoldItalicFont=cmunbi.ttf]{cmuntb.ttf}\setmonofont[Path=/usr/share/fonts/truetype/cmu/,UprightFont=cmuntt.ttf,BoldFont=cmuntb.ttf,ItalicFont=cmunit.ttf,BoldItalicFont=cmuntx.ttf]{cmuntb.ttf}\ttfamily \bfseries fi}\setmainfont[Path=/usr/share/fonts/truetype/cmu/,UprightFont=cmunrm.ttf,BoldFont=cmunbx.ttf,ItalicFont=cmunti.ttf,BoldItalicFont=cmunbi.ttf]{cmuntt.ttf}\setmonofont[Path=/usr/share/fonts/truetype/cmu/,UprightFont=cmuntt.ttf,BoldFont=cmuntb.ttf,ItalicFont=cmunit.ttf,BoldItalicFont=cmuntx.ttf]{cmuntt.ttf}\ttfamily nally\symbol{34}}\setmainfont[Path=/usr/share/fonts/truetype/cmu/,UprightFont=cmunrm.ttf,BoldFont=cmunbx.ttf,ItalicFont=cmunti.ttf,BoldItalicFont=cmunbi.ttf]{cmunrm.ttf}\setmonofont[Path=/usr/share/fonts/truetype/cmu/,UprightFont=cmuntt.ttf,BoldFont=cmuntb.ttf,ItalicFont=cmunit.ttf,BoldItalicFont=cmuntx.ttf]{cmunrm.ttf}). The \LaTeXTT{\textbackslash{}DisableLigatures} from the \myhref{http://www.ctan.org/tex-archive/macros/latex/contrib/microtype/}{microtype package} can disable ligatures in the whole document to increase accessibility.

\begin{Shaded}
\begin{Highlighting}[]

\NormalTok{\textbackslash{}usepackage\{microtype\}}
\NormalTok{\textbackslash{}DisableLigatures\{encoding = *, family = *\}}
\end{Highlighting}
\end{Shaded}


Note that this will also disable ligatures such as \symbol{34}-{}-{}\symbol{34} to \symbol{34}–\symbol{34}, \symbol{34}-{}-{}-{}\symbol{34} to \symbol{34}—\symbol{34}, etc.

If you are using XeLaTeX and OpenType fonts, the fontspec package allows for standard ligatures to be turned off as well as fancy swash ligatures to be turned on.

Another solution is to use the \LaTeXTT{cmap} package, which will help the reader to interpret the ligatures:

\begin{Shaded}
\begin{Highlighting}[]

\NormalTok{\textbackslash{}usepackage[resetfonts]\{cmap\}}
\end{Highlighting}
\end{Shaded}

\section{Slash marks}
\label{121}
The normal typesetting of the \LaTeXTT{/} character in LaTeX does not allow following characters to be \symbol{34}broken\symbol{34} onto new lines, which often create \symbol{34}overfull\symbol{34} errors in output (where letters push off the margin). Words that use slash marks, such as \symbol{34}input/output\symbol{34} should be typeset as \symbol{34}\LaTeXTT{input\textbackslash{}slash output}\symbol{34}, which allow the line to \symbol{34}break\symbol{34} after the slash mark (if needed). The use of the \LaTeXTT{/} character in LaTeX should be restricted to units, such as \symbol{34}\LaTeXTT{mm/year}\symbol{34}, which should not be broken over multiple lines.

A word after \LaTeXTT{/} or \LaTeXTT{\textbackslash{}slash} is not automatically hyphenated. This is a similar problem to non-{}hyphenation of words with a dash described under \mylref{116}{Hyphenation}. One way to have both a line break and automatic hyphenation in both words is
\begin{Shaded}
\begin{Highlighting}[]

\NormalTok{input\textbackslash{}slash\textbackslash{}hspace\{0pt\}output}
\end{Highlighting}
\end{Shaded}

Both \LaTeXTT{/} and \LaTeXTT{\textbackslash{}slash} can be used with a zero \LaTeXTT{\textbackslash{}hspace} like this.  \LaTeXTT{\textbackslash{}slash} includes a penalty to make a line break there less desirable.  This combination can be made into a new slash macro if desired.  The \LaTeXTT{hyphenat} package includes an \LaTeXTT{\textbackslash{}fshyp} which will add a hyphen after the slash like \symbol{34}input/-{} output\symbol{34} if the line breaks there.
\section{Fonts}
\label{122}

To change the font family, emphasize text, and other font-{}related issues, see \mylref{163}{Fonts}.
\section{Formatting macros}
\label{123}

Even if you can easily change the output of your fonts using those commands, you\textquotesingle{}re better off not using explicit commands like this, because they work in opposition to the basic idea of LaTeX, which is to separate the logical and visual markup of your document. This means that if you use the same font changing command in several places in order to typeset a special kind of information, you should use \LaTeXTT{\textbackslash{}newcommand} to define a \symbol{34}logical wrapper command\symbol{34} for the font changing command.

\begin{longtable}{p{1.0\linewidth}}
\begin{Shaded}
\begin{Highlighting}[]

\NormalTok{\textbackslash{}newcommand\{\textbackslash{}oops\}[1]\{\textbackslash{}textit\{#1\}\}}
 
\NormalTok{Do not \textbackslash{}oops\{enter\} this room,}
\NormalTok{it’s occupied by \textbackslash{}oops\{machines\}}
\NormalTok{of unknown origin and purpose.}
\end{Highlighting}
\end{Shaded}
\\

Do not {\itshape \setmainfont[Path=/usr/share/fonts/truetype/cmu/,UprightFont=cmunrm.ttf,BoldFont=cmunbx.ttf,ItalicFont=cmunti.ttf,BoldItalicFont=cmunbi.ttf]{cmunti.ttf}\setmonofont[Path=/usr/share/fonts/truetype/cmu/,UprightFont=cmuntt.ttf,BoldFont=cmuntb.ttf,ItalicFont=cmunit.ttf,BoldItalicFont=cmuntx.ttf]{cmunti.ttf}\itshape enter}{$\text{ }$}\setmainfont[Path=/usr/share/fonts/truetype/cmu/,UprightFont=cmunrm.ttf,BoldFont=cmunbx.ttf,ItalicFont=cmunti.ttf,BoldItalicFont=cmunbi.ttf]{cmunrm.ttf}\setmonofont[Path=/usr/share/fonts/truetype/cmu/,UprightFont=cmuntt.ttf,BoldFont=cmuntb.ttf,ItalicFont=cmunit.ttf,BoldItalicFont=cmuntx.ttf]{cmunrm.ttf} this room, it’s occupied by {\itshape \setmainfont[Path=/usr/share/fonts/truetype/cmu/,UprightFont=cmunrm.ttf,BoldFont=cmunbx.ttf,ItalicFont=cmunti.ttf,BoldItalicFont=cmunbi.ttf]{cmunti.ttf}\setmonofont[Path=/usr/share/fonts/truetype/cmu/,UprightFont=cmuntt.ttf,BoldFont=cmuntb.ttf,ItalicFont=cmunit.ttf,BoldItalicFont=cmuntx.ttf]{cmunti.ttf}\itshape machines}{$\text{ }$}\setmainfont[Path=/usr/share/fonts/truetype/cmu/,UprightFont=cmunrm.ttf,BoldFont=cmunbx.ttf,ItalicFont=cmunti.ttf,BoldItalicFont=cmunbi.ttf]{cmunrm.ttf}\setmonofont[Path=/usr/share/fonts/truetype/cmu/,UprightFont=cmuntt.ttf,BoldFont=cmuntb.ttf,ItalicFont=cmunit.ttf,BoldItalicFont=cmuntx.ttf]{cmunrm.ttf} of unknown origin and purpose.

\end{longtable}

This approach has the advantage that you can decide at some later stage that you want to use some visual representation of danger other than \LaTeXTT{\textbackslash{}textit}, without having to wade through your document, identifying all the occurrences of \LaTeXTT{\textbackslash{}textit} and then figuring out for each one whether it was used for pointing out danger or for some other reason.

See \mylref{837}{Macros} for more details.
\section{Text mode superscript and subscript}
\label{124}

Sub and superscripting can be done quite easily using \LaTeXTT{\textbackslash{}textsubscript\{\}} and \LaTeXTT{\textbackslash{}textsuperscript\{\}}. 

\begin{Shaded}
\begin{Highlighting}[]

\NormalTok{\textbackslash{}documentclass\{article\}}
\NormalTok{\textbackslash{}begin\{document\}}
\NormalTok{Wombat\textbackslash{}textsubscript\{walzing\}}
 
\NormalTok{Michelangelo was born on March 6\textbackslash{}textsuperscript\{th\}, 1475.}
\NormalTok{\textbackslash{}end\{document\}}
\end{Highlighting}
\end{Shaded}




\begin{minipage}{1.0\linewidth}
\begin{center}
\includegraphics[width=1.0\linewidth,height=6.5in,keepaspectratio]{../images/18.png}
\end{center}
\raggedright{}\myfigurewithoutcaption{18}
\end{minipage}\vspace{0.75cm}



Note: A current LaTeX version is needed to use subscripts that way.
\section{Text figures (\symbol{34}old style\symbol{34} numerals)}
\label{125}

Many typographers prefer to use titling figures, sometimes called lining figures, when numerals are interspersed with full caps, when they appear in tables, and when they appear in equations, using \myhref{https://en.wikipedia.org/wiki/Text\%20figures}{text figures} elsewhere.  LaTeX allows this usage through the \LaTeXTT{\textbackslash{}oldstylenums\{\}} command:

\begin{Shaded}
\begin{Highlighting}[]

\NormalTok{\textbackslash{}oldstylenums\{1234567890\}}
\end{Highlighting}
\end{Shaded}


Some fonts do not have text figures built in; the \LaTeXTT{textcomp} package attempts to remedy this by effectively generating text figures from the currently-{}selected font.  Put \LaTeXTT{\textbackslash{}usepackage\{textcomp\}} in your preamble.  \LaTeXTT{textcomp} also allows you to use decimal points, properly formatted dollar signs, etc. within \LaTeXTT{\textbackslash{}oldstylenums\{\}}.

One common use for text figures is in section, paragraph, and page numbers.  These can be set to use text figures by placing some code in your preamble:

\begin{Shaded}
\begin{Highlighting}[]

\NormalTok{\textbackslash{}usepackage\{textcomp\}}
 
\CommentTok
  \CommentTok{% Make \textbackslash{}section\{\} use text figures}
  \NormalTok{\textbackslash{}let\textbackslash{}myTheSection\textbackslash{}thesection}
  \NormalTok{\textbackslash{}renewcommand\{\textbackslash{}thesection\}\{ \textbackslash{}oldstylenums\{\textbackslash{}myTheSection\} \}}
 
  \CommentTok{% Make \textbackslash{}paragraph\{\} use text figures}
  \NormalTok{\textbackslash{}let\textbackslash{}myTheParagraph\textbackslash{}theparagraph}
  \NormalTok{\textbackslash{}renewcommand\{\textbackslash{}theparagraph\}\{ \textbackslash{}oldstylenums\{\textbackslash{}myTheParagraph\} \}}
 
  \CommentTok{% Make the page numbers in text figures}
  \NormalTok{\textbackslash{}let\textbackslash{}myThePage\textbackslash{}thepage}
  \NormalTok{\textbackslash{}renewcommand\{\textbackslash{}thepage\}\{ \textbackslash{}oldstylenums\{\textbackslash{}myThePage\} \}}
\NormalTok{\}}
\end{Highlighting}
\end{Shaded}


Should you use additional sectioning or paragraphing commands, you may adapt the previous code listing to include them as well.
{\bfseries
\begin{mydescription}Note
\end{mydescription}
}

A subsequent use of the \LaTeXTT{\textbackslash{}pagenumbering} command, e.g., \LaTeXTT{\textbackslash{}pagenumbering\{arabic\}}, will reset the \LaTeXTT{\textbackslash{}thepage} command back to the original. Thus, if you use the \LaTeXTT{\textbackslash{}pagenumbering} command in your document, be sure to reinstate your \LaTeXTT{\textbackslash{}myThePage definition} from the code above:

\begin{Shaded}
\begin{Highlighting}[]

\NormalTok{...}
\NormalTok{\textbackslash{}tableofcontents}
\NormalTok{\textbackslash{}pagenumbering\{roman\}}
\NormalTok{\textbackslash{}chapter\{Preface\}}
\NormalTok{...}
\NormalTok{\textbackslash{}chapter\{Introduction\}}
\NormalTok{...}
\NormalTok{\textbackslash{}pagenumbering\{arabic\}}
\CommentTok{% without this, the \textbackslash{}thepage command will not be in oldstyle (e.g., in your}
 \NormalTok{Table of Contents\}}
\NormalTok{\textbackslash{}renewcommand\{\textbackslash{}thepage\}\{ \textbackslash{}oldstylenums\{\textbackslash{}myThePage\} \}}
\NormalTok{\textbackslash{}Chapter\{Foo\}}
\NormalTok{...}
\end{Highlighting}
\end{Shaded}

\section{Dashes and hyphens}
\label{126}
LaTeX knows four kinds of dashes: a \myhref{https://en.wikipedia.org/wiki/hyphen}{hyphen} (-{}), \myhref{https://en.wikipedia.org/wiki/Dash\%23En\%20dash}{en dash} (–), \myhref{https://en.wikipedia.org/wiki/Dash\%23Em\%20dash}{em dash} (—), or a \myhref{https://en.wikipedia.org/wiki/Plus\%20and\%20minus\%20signs\%23Minus\%20sign}{minus sign} (−). You can access three of them with different numbers of consecutive dashes. The fourth sign is actually not a dash at all—it is the mathematical minus sign:

\begin{longtable}{p{1.0\linewidth}}
\begin{Shaded}
\begin{Highlighting}[]

\NormalTok{Hyphen: daughter-in-law, X-rated\textbackslash{}\textbackslash{}}
\NormalTok{En dash: pages 13--67\textbackslash{}\textbackslash{}}
\NormalTok{Em dash: yes---or no? \textbackslash{}\textbackslash{}}
\NormalTok{Minus sign: $0$, $1$ and $-1$}
\end{Highlighting}
\end{Shaded}
\\



\begin{minipage}{0.50000\textwidth}
\begin{center}
\includegraphics[width=1.0\textwidth,height=6.5in,keepaspectratio]{../images/19.png}
\end{center}
\raggedright{}\myfigurewithoutcaption{19}
\end{minipage}\vspace{0.75cm}



\end{longtable}

The names for these dashes are: ‘-{}’(-{}) hyphen , ‘-{}-{}’({\mbox{$-$}}) en-{}dash , ‘-{}-{}-{}’({\mbox{$\text{---}$}}) em-{}dash  and ‘{$-$}’({\mbox{$-$}}) minus sign. They have different purposes:

\begin{longtable}{|>{\RaggedRight}p{0.16344\linewidth}|>{\RaggedRight}p{0.19763\linewidth}|>{\RaggedRight}p{0.55321\linewidth}|} \hline 
{\bfseries \hspace*{0pt}\ignorespaces{}\hspace*{0pt} Input}&{\bfseries \hspace*{0pt}\ignorespaces{}\hspace*{0pt} Output}&{\bfseries \hspace*{0pt}\ignorespaces{}\hspace*{0pt} Purpose}\endhead  \hline \hspace*{0pt}\ignorespaces{}\hspace*{0pt} -{}&\hspace*{0pt}\ignorespaces{}\hspace*{0pt} -{}&\hspace*{0pt}\ignorespaces{}\hspace*{0pt} inter-{}word\\ \hline \hspace*{0pt}\ignorespaces{}\hspace*{0pt} -{}-{}&\hspace*{0pt}\ignorespaces{}\hspace*{0pt} {\mbox{$-$}}&\hspace*{0pt}\ignorespaces{}\hspace*{0pt} page range, 1{\mbox{$-$}}10\\ \hline \hspace*{0pt}\ignorespaces{}\hspace*{0pt} -{}-{}-{}&\hspace*{0pt}\ignorespaces{}\hspace*{0pt} {\mbox{$\text{---}$}}&\hspace*{0pt}\ignorespaces{}\hspace*{0pt} punctuation dash{\mbox{$\text{---}$}}like this\\ \hline \hspace*{0pt}\ignorespaces{}\hspace*{0pt} \${}-{}\${}&\hspace*{0pt}\ignorespaces{}\hspace*{0pt} {\mbox{$-$}}&\hspace*{0pt}\ignorespaces{}\hspace*{0pt} minus sign\\ \hline 
\end{longtable}


Use \LaTeXTT{\textbackslash{}hyp\{\}} macro from \LaTeXTT{hyphenat} package instead of hyphen if you want LaTeX to break compound words between lines.

The commands \LaTeXTT{\textbackslash{}textendash} and \LaTeXTT{\textbackslash{}textemdash} are also used to produce en-{}dash ({\mbox{$-$}}), and em-{}dash ({\mbox{$\text{---}$}}), respectively.
\section{Ellipsis (…)}
\label{127}

A sequence of three dots is known as an {\itshape \myhref{https://en.wikipedia.org/wiki/Ellipsis}{\setmainfont[Path=/usr/share/fonts/truetype/cmu/,UprightFont=cmunrm.ttf,BoldFont=cmunbx.ttf,ItalicFont=cmunti.ttf,BoldItalicFont=cmunbi.ttf]{cmunti.ttf}\setmonofont[Path=/usr/share/fonts/truetype/cmu/,UprightFont=cmuntt.ttf,BoldFont=cmuntb.ttf,ItalicFont=cmunit.ttf,BoldItalicFont=cmuntx.ttf]{cmunti.ttf}\itshape ellipsis}}, which is commonly used to indicate omitted text. On a typewriter, a comma or a period takes the same amount of space as any other letter. In book printing, these characters occupy only a little space and are set very close to the preceding letter. Therefore, you cannot enter ‘ellipsis’ by just typing three dots, as the spacing would be wrong. Instead, there is a special command for these dots. It is called \LaTeXTT{\textbackslash{}ldots}:

\begin{longtable}{p{1.0\linewidth}}
\begin{Shaded}
\begin{Highlighting}[]

\NormalTok{Not like this ... but like this:\textbackslash{}\textbackslash{}}
\NormalTok{New York, Tokyo, Budapest, \textbackslash{}ldots}
\end{Highlighting}
\end{Shaded}
\\



\begin{minipage}{0.50000\textwidth}
\begin{center}
\includegraphics[width=1.0\textwidth,height=6.5in,keepaspectratio]{../images/20.png}
\end{center}
\raggedright{}\myfigurewithoutcaption{20}
\end{minipage}\vspace{0.75cm}



\end{longtable}

Alternatively, you can use the \LaTeXTT{\textbackslash{}textellipsis} command which allows the spacing between the dots to vary.
\section{Ready-{}made strings}
\label{128}

There are some very simple LaTeX commands for typesetting special text strings:


\begin{minipage}{0.75000\textwidth}
\begin{center}
\includegraphics[width=1.0\textwidth,height=6.5in,keepaspectratio]{../images/21.png}
\end{center}
\raggedright{}\myfigurewithoutcaption{21}
\end{minipage}\vspace{0.75cm}




\begin{myquote}
\item{}
\end{myquote}

\LaTeXNullTemplate{}
\section{Notes and References}
\label{129}
\LaTeXNullTemplate{}
\ARoberts{}




\myhref{https://ru.wikibooks.org/wiki/LaTeX\%2F\%D0\%A4\%D0\%BE\%D1\%80\%D0\%BC\%D0\%B0\%D1\%82\%D0\%B8\%D1\%80\%D0\%BE\%D0\%B2\%D0\%B0\%D0\%BD\%D0\%B8\%D0\%B5\%20\%D1\%82\%D0\%B5\%D0\%BA\%D1\%81\%D1\%82\%D0\%B0}{ru:LaTeX/Форматирование текста}
\myhref{https://sr.wikibooks.org/wiki/LateX\%2F\%D0\%A4\%D0\%BE\%D1\%80\%D0\%BC\%D0\%B0\%D1\%82\%D0\%B8\%D1\%80\%D0\%B0\%D1\%9A\%D0\%B5\%20\%D1\%82\%D0\%B5\%D0\%BA\%D1\%81\%D1\%82\%D0\%B0}{sr:LateX/Форматирање текста}\chapter{Paragraph Formatting}

\myminitoc
\label{130}

\label{131}


Altering the paragraph formatting is rarely necessary in academic writing. It is primarily used for formatting text in floats or for more exotic documents.
\section{Paragraph alignment}
\label{132}

Paragraphs in LaTeX are usually fully justified, {\itshape \setmainfont[Path=/usr/share/fonts/truetype/cmu/,UprightFont=cmunrm.ttf,BoldFont=cmunbx.ttf,ItalicFont=cmunti.ttf,BoldItalicFont=cmunbi.ttf]{cmunti.ttf}\setmonofont[Path=/usr/share/fonts/truetype/cmu/,UprightFont=cmuntt.ttf,BoldFont=cmuntb.ttf,ItalicFont=cmunit.ttf,BoldItalicFont=cmuntx.ttf]{cmunti.ttf}\itshape i.e.}{$\text{ }$}\setmainfont[Path=/usr/share/fonts/truetype/cmu/,UprightFont=cmunrm.ttf,BoldFont=cmunbx.ttf,ItalicFont=cmunti.ttf,BoldItalicFont=cmunbi.ttf]{cmunrm.ttf}\setmonofont[Path=/usr/share/fonts/truetype/cmu/,UprightFont=cmuntt.ttf,BoldFont=cmuntb.ttf,ItalicFont=cmunit.ttf,BoldItalicFont=cmuntx.ttf]{cmunrm.ttf} flush with both the left and right margins. For whatever reason, should you wish to alter the justification of a paragraph, there are three environments at hand, and also LaTeX command equivalents.

\begin{longtable}{|>{\RaggedRight}p{0.27645\linewidth}|>{\RaggedRight}p{0.29415\linewidth}|>{\RaggedRight}p{0.34369\linewidth}|} \hline 
{\bfseries \hspace*{0pt}\ignorespaces{}\hspace*{0pt} Alignment}&{\bfseries \hspace*{0pt}\ignorespaces{}\hspace*{0pt} Environment}&{\bfseries \hspace*{0pt}\ignorespaces{}\hspace*{0pt} Command}\endhead  \hline \hspace*{0pt}\ignorespaces{}\hspace*{0pt} Left justified&\hspace*{0pt}\ignorespaces{}\hspace*{0pt} \LaTeXTT{flushleft}&\hspace*{0pt}\ignorespaces{}\hspace*{0pt} \LaTeXTT{\textbackslash{}raggedright}\\ \hline \hspace*{0pt}\ignorespaces{}\hspace*{0pt} Right justified&\hspace*{0pt}\ignorespaces{}\hspace*{0pt} \LaTeXTT{flushright}&\hspace*{0pt}\ignorespaces{}\hspace*{0pt} \LaTeXTT{\textbackslash{}raggedleft}\\ \hline \hspace*{0pt}\ignorespaces{}\hspace*{0pt} Center&\hspace*{0pt}\ignorespaces{}\hspace*{0pt} \LaTeXTT{center}&\hspace*{0pt}\ignorespaces{}\hspace*{0pt} \LaTeXTT{\textbackslash{}centering}\\ \hline 
\end{longtable}


All text between the \LaTeXTT{\textbackslash{}begin} and \LaTeXTT{\textbackslash{}end} of the specified environment will be justified appropriately. The commands listed are for use within other environments. For example, \LaTeXTT{p} (paragraph) columns in \LaTeXTT{tabular}.

\begin{TemplateInfo}{\danger}{Warning} There is no way (in standard LaTeX) to set full justification explicitly. It means that if you do not enclose the previous 3 commands into a group, the rest of the document will be affected.
So the right way of doing this with commands is
\begin{Shaded}
\begin{Highlighting}[]

\NormalTok{\{\textbackslash{}raggedleft\{\}Some text flushed right.\}}
\end{Highlighting}
\end{Shaded}

\end{TemplateInfo}

However, if you {\itshape \setmainfont[Path=/usr/share/fonts/truetype/cmu/,UprightFont=cmunrm.ttf,BoldFont=cmunbx.ttf,ItalicFont=cmunti.ttf,BoldItalicFont=cmunbi.ttf]{cmunti.ttf}\setmonofont[Path=/usr/share/fonts/truetype/cmu/,UprightFont=cmuntt.ttf,BoldFont=cmuntb.ttf,ItalicFont=cmunit.ttf,BoldItalicFont=cmuntx.ttf]{cmunti.ttf}\itshape really}{$\text{ }$}\setmainfont[Path=/usr/share/fonts/truetype/cmu/,UprightFont=cmunrm.ttf,BoldFont=cmunbx.ttf,ItalicFont=cmunti.ttf,BoldItalicFont=cmunbi.ttf]{cmunrm.ttf}\setmonofont[Path=/usr/share/fonts/truetype/cmu/,UprightFont=cmuntt.ttf,BoldFont=cmuntb.ttf,ItalicFont=cmunit.ttf,BoldItalicFont=cmuntx.ttf]{cmunrm.ttf} need to disable one of the above commands locally (for example because you have to use some broken package), you can use the command \LaTeXTT{\textbackslash{}justifying} from package \LaTeXTT{ragged2e}.
\section{Paragraph indent and break}
\label{133}

By default, the first paragraph after a heading follows the standard Anglo-{}American publishers\textquotesingle{} practice of no indentation. The size of subsequent paragraph indents is determined by a parameter called \LaTeXTT{\textbackslash{}parindent}. The default length that this constant holds is set by the document class that you use. It is possible to override it by using the \LaTeXTT{\textbackslash{}setlength} command. This will set paragraph indents to 1cm:

\begin{Shaded}
\begin{Highlighting}[]

\NormalTok{\textbackslash{}setlength\{\textbackslash{}parindent\}\{1cm\} }\CommentTok{% Default is 15pt.}
\end{Highlighting}
\end{Shaded}


Whitespace in LaTeX can also be made flexible (what Lamport calls \symbol{34}rubber\symbol{34} lengths). This means that values such as extra vertical space inserted before a paragraph \LaTeXTT{\textbackslash{}parskip} can have a default dimension plus an amount of expansion minus an amount of contraction. This is useful on pages in complex documents where not every page may be an exact number of fixed-{}height lines long, so some give-{}and-{}take in vertical space is useful. You specify this in a \LaTeXTT{\textbackslash{}setlength} command like this:

\begin{Shaded}
\begin{Highlighting}[]

\NormalTok{\textbackslash{}setlength\{\textbackslash{}parskip\}\{1cm plus4mm minus3mm\}}
\end{Highlighting}
\end{Shaded}


If you want to indent a paragraph that is not indented, you can use
\begin{Shaded}
\begin{Highlighting}[]

\NormalTok{\textbackslash{}indent}
\end{Highlighting}
\end{Shaded}

at the beginning of the paragraph. Obviously, this will only have an effect when \LaTeXTT{\textbackslash{}parindent} is not set to zero. If you want to indent the beginning of every section, you can use the \LaTeXTT{indentfirst} package: once loaded, the beginning of any chapter/section is indented by the usual paragraph indentation.

To create a non-{}indented paragraph, you can use
\begin{Shaded}
\begin{Highlighting}[]

\NormalTok{\textbackslash{}noindent}
\end{Highlighting}
\end{Shaded}


as the first command of the paragraph. This might come in handy when you start a document with body text and not with a sectioning command.

Be careful, however, if you decide to set the indent to zero, then it means you will need a vertical space between paragraphs in order to make them clear. The space between paragraphs is held in \LaTeXTT{\textbackslash{}parskip}, which could be altered in a similar fashion as above. However, this parameter is used elsewhere too, such as in lists, which means you run the risk of making various parts of your document look very untidy by changing this setting. If you want to use the style of having no indentation with a space between paragraphs, use the \LaTeXTT{parskip} package, which does this for you, while making adjustments to the spacing of lists and other structures which use paragraph spacing, so they don\textquotesingle{}t get too far apart. If you want both indent and break, use

\begin{Shaded}
\begin{Highlighting}[]

\NormalTok{\textbackslash{}usepackage\{parskip\}}
\NormalTok{\textbackslash{}setlength\{\textbackslash{}parindent\}\{15pt\}}
\end{Highlighting}
\end{Shaded}


To indent subsequent lines of a paragraph, use the TeX command \LaTeXTT{\textbackslash{}hangindent}. (While the default behaviour is to apply the hanging indent after the first line, this may be changed with the \LaTeXTT{\textbackslash{}hangafter} command.)  An example follows.

\begin{Shaded}
\begin{Highlighting}[]

\NormalTok{\textbackslash{}hangindent=0.7cm This paragraph has an extra indentation at the left.}
\end{Highlighting}
\end{Shaded}


The TeX commands \LaTeXTT{\textbackslash{}leftskip} and \LaTeXTT{\textbackslash{}rightskip} add additional space to the left and right sides of each line, allowing the formatting for subsequent paragraphs to differ from the overall document margins.  This space is in addition to the indentation added by \LaTeXTT{\textbackslash{}parindent} and \LaTeXTT{\textbackslash{}hangindent}.

To change the indentation of the last line in a paragraph, use the TeX command \LaTeXTT{\textbackslash{}parfillskip}.
\section{\textbackslash{}paragraph line break}
\label{134}
Default style for \LaTeXTT{\textbackslash{}paragraph} may seem odd in the first place, as it writes the following text next to the title. If you do not like it, use a class other than the traditional article/book, or use ConTeXt or PlainTeX. Hacking of the class in use is really not the way LaTeX is intended to be used, and you may encounter a lot of frustrating issues.

Anyway, let\textquotesingle{}s analyse the problem. If you add a manual line break with \LaTeXTT{\textbackslash{}\textbackslash{}}, LaTeX will complain that 
\\

\TemplateSpaceIndent{$\text{ }${}There\textquotesingle{}s$\text{ }${}no$\text{ }${}line$\text{ }${}here$\text{ }${}to$\text{ }${}end.}


Simply adding an empty space will do it:

\begin{Shaded}
\begin{Highlighting}[]

\NormalTok{\textbackslash{}paragraph\{Title\} \textbackslash{}hspace\{0pt\} \textbackslash{}\textbackslash{}}
\NormalTok{Text...}
\end{Highlighting}
\end{Shaded}


Alternatively you can use the shorter, yet not completely equivalent syntax:

\begin{Shaded}
\begin{Highlighting}[]

\NormalTok{\textbackslash{}paragraph\{Title\} ~\textbackslash{}\textbackslash{}}
\NormalTok{Text...}
\end{Highlighting}
\end{Shaded}

\section{Line spacing}
\label{135}
To change line spacing in the whole document use the command \LaTeXTT{\textbackslash{}linespread} covered in \mylref{111}{Text Formatting}.

Alternatively, you can use the \LaTeXTT{\textbackslash{}usepackage\{setspace\}} package, which is also covered in \mylref{111}{Text Formatting}. This package provides the commands \LaTeXTT{\textbackslash{}doublespacing}, \LaTeXTT{\textbackslash{}onehalfspacing}, \LaTeXTT{\textbackslash{}singlespacing} and \LaTeXTT{\textbackslash{}setstretch\{baselinestretch\}}, which will specify the line spacing for all sections and paragraphs until another command is used. Furthermore, the package provides the following environments in order to change line spacing within the document but not document-{}wide:
\begin{myitemize}
\item{}  \LaTeXTT{doublespace}: lines are double spaced;
\item{}  \LaTeXTT{onehalfspace}: line spacing set to one-{}and-{}half spacing;
\item{}  \LaTeXTT{singlespace}: normal line spacing;
\item{}  \LaTeXTT{spacing}: customizable line spacing, e.g. \LaTeXTT{\textbackslash{}begin\{spacing\}\{\textbackslash{}baselinestretch\} ... \textbackslash{}end\{spacing\}}.
\end{myitemize}


See the section on \mylref{189}{customizing lists} for information on how to change the line spacing in lists.
\section{Manual breaks}
\label{136}

LaTeX takes care of formatting, breaks included. You should avoid manual breaking as much as possible, for it could lead to very bad formatting.

Controlling the breaks should be reserved to macro and package writers. Here follows a quick reference.

\begin{longtable}{|>{\RaggedRight}p{0.28673\linewidth}|>{\RaggedRight}p{0.65613\linewidth}|} \hline 
\hspace*{0pt}\ignorespaces{}\hspace*{0pt} \LaTeXTT{\textbackslash{}newline}&\hspace*{0pt}\ignorespaces{}\hspace*{0pt} Breaks the line at the point of the command.\\ \hline \hspace*{0pt}\ignorespaces{}\hspace*{0pt} \LaTeXTT{\textbackslash{}\textbackslash{}}&\hspace*{0pt}\ignorespaces{}\hspace*{0pt} Breaks the line at the point of the command; it is usually a shorter version of the previous command, but LaTeX sometimes redefines it for several environments. This command also features the vertical space as optional parameter.\\ \hline \hspace*{0pt}\ignorespaces{}\hspace*{0pt} \LaTeXTT{\textbackslash{}\textbackslash{}*}&\hspace*{0pt}\ignorespaces{}\hspace*{0pt} Breaks the line at the point of the command and additionally prohibits a page break after the forced line break. This command also features the vertical space as optional parameter.\\ \hline \hspace*{0pt}\ignorespaces{}\hspace*{0pt} \LaTeXTT{\textbackslash{}\textbackslash{}{$\text{[}$}extra-{}space{$\text{]}$}}&\hspace*{0pt}\ignorespaces{}\hspace*{0pt}  Command \textbackslash{}\textbackslash{} has an optional argument that specifies the amount of extra vertical space to be inserted before the next line. This amount can be  negative. \\ \hline \hspace*{0pt}\ignorespaces{}\hspace*{0pt} \LaTeXTT{\textbackslash{}linebreak{$\text{[}$}number{$\text{]}$}}&\hspace*{0pt}\ignorespaces{}\hspace*{0pt} Breaks the line at the point of the command. The {\itshape \setmainfont[Path=/usr/share/fonts/truetype/cmu/,UprightFont=cmunrm.ttf,BoldFont=cmunbx.ttf,ItalicFont=cmunti.ttf,BoldItalicFont=cmunbi.ttf]{cmunti.ttf}\setmonofont[Path=/usr/share/fonts/truetype/cmu/,UprightFont=cmuntt.ttf,BoldFont=cmuntb.ttf,ItalicFont=cmunit.ttf,BoldItalicFont=cmuntx.ttf]{cmunti.ttf}\itshape number}{$\text{ }$}\setmainfont[Path=/usr/share/fonts/truetype/cmu/,UprightFont=cmunrm.ttf,BoldFont=cmunbx.ttf,ItalicFont=cmunti.ttf,BoldItalicFont=cmunbi.ttf]{cmunrm.ttf}\setmonofont[Path=/usr/share/fonts/truetype/cmu/,UprightFont=cmuntt.ttf,BoldFont=cmuntb.ttf,ItalicFont=cmunit.ttf,BoldItalicFont=cmuntx.ttf]{cmunrm.ttf} you provide as an argument represents the priority of the command in a range from 0 (it will be easily ignored) to 4 (do it anyway). LaTeX will try to produce the best line breaks possible. If it cannot, it will decide whether including the linebreak or not according to the priority you have provided.\\ \hline \hspace*{0pt}\ignorespaces{}\hspace*{0pt} \LaTeXTT{\textbackslash{}break} (TeX)&\hspace*{0pt}\ignorespaces{}\hspace*{0pt} Breaks the line without filling the current line. This will result in an underful badness if you do not fill the line yourself, {\itshape \setmainfont[Path=/usr/share/fonts/truetype/cmu/,UprightFont=cmunrm.ttf,BoldFont=cmunbx.ttf,ItalicFont=cmunti.ttf,BoldItalicFont=cmunbi.ttf]{cmunti.ttf}\setmonofont[Path=/usr/share/fonts/truetype/cmu/,UprightFont=cmuntt.ttf,BoldFont=cmuntb.ttf,ItalicFont=cmunit.ttf,BoldItalicFont=cmuntx.ttf]{cmunti.ttf}\itshape i.e.}{$\text{ }$}\setmainfont[Path=/usr/share/fonts/truetype/cmu/,UprightFont=cmunrm.ttf,BoldFont=cmunbx.ttf,ItalicFont=cmunti.ttf,BoldItalicFont=cmunbi.ttf]{cmunrm.ttf}\setmonofont[Path=/usr/share/fonts/truetype/cmu/,UprightFont=cmuntt.ttf,BoldFont=cmuntb.ttf,ItalicFont=cmunit.ttf,BoldItalicFont=cmuntx.ttf]{cmunrm.ttf} \LaTeXTT{...\textbackslash{}hfill\textbackslash{}break ...}. Actually \LaTeXTT{\textbackslash{}hfill\textbackslash{}break} produces the same as \LaTeXTT{\textbackslash{}newline} and \LaTeXTT{\textbackslash{}\textbackslash{}}.\\ \hline \hspace*{0pt}\ignorespaces{}\hspace*{0pt} \LaTeXTT{\textbackslash{}par} (TeX)&\hspace*{0pt}\ignorespaces{}\hspace*{0pt} Starts a new paragraph. It is a horizontal mode command, so you can only use it in a paragraph.\\ \hline 
\end{longtable}


The page breaks are covered in \mylref{331}{Page Layout}. More details on manual spaces between paragraphs (such as \LaTeXTT{\textbackslash{}bigskip}) can be found in \mylref{456}{Lengths}.
\section{Special paragraphs}
\label{137}
\subsection{Verbatim text}
\label{138}

There are several ways to introduce text that won\textquotesingle{}t be interpreted by the compiler. If you use the \LaTeXTT{verbatim} environment, everything input between the {\itshape \setmainfont[Path=/usr/share/fonts/truetype/cmu/,UprightFont=cmunrm.ttf,BoldFont=cmunbx.ttf,ItalicFont=cmunti.ttf,BoldItalicFont=cmunbi.ttf]{cmunti.ttf}\setmonofont[Path=/usr/share/fonts/truetype/cmu/,UprightFont=cmuntt.ttf,BoldFont=cmuntb.ttf,ItalicFont=cmunit.ttf,BoldItalicFont=cmuntx.ttf]{cmunti.ttf}\itshape begin}{$\text{ }$}\setmainfont[Path=/usr/share/fonts/truetype/cmu/,UprightFont=cmunrm.ttf,BoldFont=cmunbx.ttf,ItalicFont=cmunti.ttf,BoldItalicFont=cmunbi.ttf]{cmunrm.ttf}\setmonofont[Path=/usr/share/fonts/truetype/cmu/,UprightFont=cmuntt.ttf,BoldFont=cmuntb.ttf,ItalicFont=cmunit.ttf,BoldItalicFont=cmuntx.ttf]{cmunrm.ttf} and {\itshape \setmainfont[Path=/usr/share/fonts/truetype/cmu/,UprightFont=cmunrm.ttf,BoldFont=cmunbx.ttf,ItalicFont=cmunti.ttf,BoldItalicFont=cmunbi.ttf]{cmunti.ttf}\setmonofont[Path=/usr/share/fonts/truetype/cmu/,UprightFont=cmuntt.ttf,BoldFont=cmuntb.ttf,ItalicFont=cmunit.ttf,BoldItalicFont=cmuntx.ttf]{cmunti.ttf}\itshape end}{$\text{ }$}\setmainfont[Path=/usr/share/fonts/truetype/cmu/,UprightFont=cmunrm.ttf,BoldFont=cmunbx.ttf,ItalicFont=cmunti.ttf,BoldItalicFont=cmunbi.ttf]{cmunrm.ttf}\setmonofont[Path=/usr/share/fonts/truetype/cmu/,UprightFont=cmuntt.ttf,BoldFont=cmuntb.ttf,ItalicFont=cmunit.ttf,BoldItalicFont=cmuntx.ttf]{cmunrm.ttf} commands are processed as if by a typewriter. All spaces and new lines are reproduced as given, and the text is displayed in an appropriate fixed-{}width font. Any LaTeX command will be ignored and handled as plain text. This is ideal for typesetting program source code. Here is an example:

\begin{longtable}{p{1.0\linewidth}}
\begin{Shaded}
\begin{Highlighting}[]

\NormalTok{\textbackslash{}begin\{verbatim\}}
\NormalTok{The verbatim environment}
  \NormalTok{simply reproduces every}
 \NormalTok{character you input,}
\NormalTok{including all  s p a c e s!}
\NormalTok{\textbackslash{}end\{verbatim\}}
\end{Highlighting}
\end{Shaded}
\\



\begin{minipage}{1.0\linewidth}
\begin{center}
\includegraphics[width=1.0\linewidth,height=6.5in,keepaspectratio]{../images/22.\SVGExtension}
\end{center}
\raggedright{}\myfigurewithoutcaption{22}
\end{minipage}\vspace{0.75cm}



\end{longtable}

Note: once in the \LaTeXTT{verbatim} environment, the only command that will be recognized is \LaTeXTT{\textbackslash{}end\{verbatim\}}. Any others will be output. The font size in the verbatim environment can be adjusted by placing a \mylref{140}{font size command} before \LaTeXTT{\textbackslash{}begin\{verbatim\}}. If this is a problem, you can use the \LaTeXTT{alltt} package instead, providing an environment with the same name:

\begin{longtable}{p{1.0\linewidth}}
\begin{Shaded}
\begin{Highlighting}[]

\NormalTok{\textbackslash{}begin\{alltt\}}
\NormalTok{Verbatim extended with the ability}
\NormalTok{to use normal commands.  Therefore, it}
\NormalTok{is possible to \textbackslash{}emph\{emphasize\} words in}
\NormalTok{this environment, for example.}
\NormalTok{\textbackslash{}end\{alltt\}}
\end{Highlighting}
\end{Shaded}
\\



\begin{minipage}{1.0\linewidth}
\begin{center}
\includegraphics[width=1.0\linewidth,height=6.5in,keepaspectratio]{../images/23.\SVGExtension}
\end{center}
\raggedright{}\myfigurewithoutcaption{23}
\end{minipage}\vspace{0.75cm}



\end{longtable}

Remember to add \LaTeXTT{\textbackslash{}usepackage\{alltt\}} to your preamble to use it though! 
Within the \LaTeXTT{alltt} environment, you can use the command \LaTeXTT{\textbackslash{}normalfont} to get back the normal font.
To write equations within the \LaTeXTT{alltt} enviroment, you can use \LaTeXTT{\textbackslash{}(} and \LaTeXTT{\textbackslash{})} to enclose them, instead of the usual \LaTeXTT{\${}}.

When using \LaTeXTT{\textbackslash{}textbf\{\}} inside the \LaTeXTT{alltt} enviroment, note that the standard font has no bold TT font. Txtfonts has bold fonts: just add \LaTeXTT{\textbackslash{}renewcommand\{\textbackslash{}ttdefault\}\{txtt\}} after \LaTeXTT{\textbackslash{}usepackage\{alltt\}}.

If you just want to introduce a short verbatim phrase, you don\textquotesingle{}t need to use the whole environment, but you have the \LaTeXTT{\textbackslash{}verb} command:

\begin{Shaded}
\begin{Highlighting}[]

\NormalTok{\textbackslash{}verb+my text+}
\end{Highlighting}
\end{Shaded}


The first character following \LaTeXTT{\textbackslash{}verb} is the delimiter: here we have used \symbol{34}+\symbol{34}, but you can use any character you like except *; \LaTeXTT{\textbackslash{}verb} will print verbatim all the text after it until it finds the next delimiter. For example, the code:

\begin{Shaded}
\begin{Highlighting}[]

\NormalTok{\textbackslash{}verb;\textbackslash{}textbf\{Hi mate!\};}
\end{Highlighting}
\end{Shaded}

will print \LaTeXTT{\textbackslash{}textbf\{Hi mate!\}}, ignoring the effect \LaTeXTT{\textbackslash{}textbf} should have on text.

For more control over formatting, however, you can try the \LaTeXTT{fancyvrb} package, which provides a \LaTeXTT{Verbatim} environment (note the capital letter) which lets you draw a rule round the verbatim text, change the font size, and even have typographic effects inside the \LaTeXTT{Verbatim}  environment. It can also be used in conjunction with the \LaTeXTT{fancybox} package and it can add reference line numbers (useful for chunks of data or programming), and it can even include entire external files.

To use verbatim in beamer, the frame needs to be made fragile: \textbackslash{}begin\{frame\}{$\text{[}$}fragile{$\text{]}$} .\subsubsection{Typesetting URLs}
\label{139}

One of either the \LaTeXTT{hyperref} or \LaTeXTT{url} packages provides the \LaTeXTT{\textbackslash{}url} command, which properly typesets URLs, for example:

\begin{Shaded}
\begin{Highlighting}[]

\NormalTok{Go to \textbackslash{}url\{http://www.uni.edu/~myname/best-website-ever.html\} for my website.}
\end{Highlighting}
\end{Shaded}


will show this URL exactly as typed (similar to the \LaTeXTT{\textbackslash{}verb} command), but the \LaTeXTT{\textbackslash{}url} command also performs a hyphenless break at punctuation characters (only in PDFLaTeX, not in plain LaTeX+ dvips). It was designed for Web URLs, so it understands their syntax and will never break midway through an unpunctuated word, only at slashes and full stops. Bear in mind, however, that spaces are forbidden in URLs, so using spaces in \LaTeXTT{\textbackslash{}url} arguments will fail, as will using other non-{}URL-{}valid characters.

When using this command through the \LaTeXTT{hyperref} package, the URL is \symbol{34}clickable\symbol{34} in the PDF document, whereas it is not linked to the web when using only the \LaTeXTT{url} package. Also when using the \LaTeXTT{hyperref} package, to remove the border placed around a URL, insert \LaTeXTT{pdfborder = \{0 0 0 0\}} inside the \LaTeXTT{\textbackslash{}hypersetup\{\}}. (Alternately \LaTeXTT{pdfborder = \{0 0 0\}} might work if the four zeroes do not.)

You can put the following code into your preamble to change the style, how URLs are displayed to the normal font:
\begin{Shaded}
\begin{Highlighting}[]

\NormalTok{\textbackslash{}urlstyle\{same\}}
\end{Highlighting}
\end{Shaded}

See also \mylref{403}{Hyperlinks}
\subsubsection{{\itshape \setmainfont[Path=/usr/share/fonts/truetype/cmu/,UprightFont=cmunrm.ttf,BoldFont=cmunbx.ttf,ItalicFont=cmunti.ttf,BoldItalicFont=cmunbi.ttf]{cmunti.ttf}\setmonofont[Path=/usr/share/fonts/truetype/cmu/,UprightFont=cmuntt.ttf,BoldFont=cmuntb.ttf,ItalicFont=cmunit.ttf,BoldItalicFont=cmuntx.ttf]{cmunti.ttf}\itshape Listing}{$\text{ }$}\setmainfont[Path=/usr/share/fonts/truetype/cmu/,UprightFont=cmunrm.ttf,BoldFont=cmunbx.ttf,ItalicFont=cmunti.ttf,BoldItalicFont=cmunbi.ttf]{cmunrm.ttf}\setmonofont[Path=/usr/share/fonts/truetype/cmu/,UprightFont=cmuntt.ttf,BoldFont=cmuntb.ttf,ItalicFont=cmunit.ttf,BoldItalicFont=cmuntx.ttf]{cmunrm.ttf} environment}
\label{140}

This is also an extension of the verbatim environment provided by the \LaTeXTT{moreverb} package. The extra functionality it provides is that it can add line numbers along side the text. The command: \LaTeXTT{\textbackslash{}begin\{listing\}{$\text{[}$}step{$\text{]}$}\{first line\}}. The mandatory {\itshape \setmainfont[Path=/usr/share/fonts/truetype/cmu/,UprightFont=cmunrm.ttf,BoldFont=cmunbx.ttf,ItalicFont=cmunti.ttf,BoldItalicFont=cmunbi.ttf]{cmunti.ttf}\setmonofont[Path=/usr/share/fonts/truetype/cmu/,UprightFont=cmuntt.ttf,BoldFont=cmuntb.ttf,ItalicFont=cmunit.ttf,BoldItalicFont=cmuntx.ttf]{cmunti.ttf}\itshape first line}{$\text{ }$}\setmainfont[Path=/usr/share/fonts/truetype/cmu/,UprightFont=cmunrm.ttf,BoldFont=cmunbx.ttf,ItalicFont=cmunti.ttf,BoldItalicFont=cmunbi.ttf]{cmunrm.ttf}\setmonofont[Path=/usr/share/fonts/truetype/cmu/,UprightFont=cmuntt.ttf,BoldFont=cmuntb.ttf,ItalicFont=cmunit.ttf,BoldItalicFont=cmuntx.ttf]{cmunrm.ttf} argument is for specifying which line the numbering shall commence. The optional {\itshape \setmainfont[Path=/usr/share/fonts/truetype/cmu/,UprightFont=cmunrm.ttf,BoldFont=cmunbx.ttf,ItalicFont=cmunti.ttf,BoldItalicFont=cmunbi.ttf]{cmunti.ttf}\setmonofont[Path=/usr/share/fonts/truetype/cmu/,UprightFont=cmuntt.ttf,BoldFont=cmuntb.ttf,ItalicFont=cmunit.ttf,BoldItalicFont=cmuntx.ttf]{cmunti.ttf}\itshape step}{$\text{ }$}\setmainfont[Path=/usr/share/fonts/truetype/cmu/,UprightFont=cmunrm.ttf,BoldFont=cmunbx.ttf,ItalicFont=cmunti.ttf,BoldItalicFont=cmunbi.ttf]{cmunrm.ttf}\setmonofont[Path=/usr/share/fonts/truetype/cmu/,UprightFont=cmuntt.ttf,BoldFont=cmuntb.ttf,ItalicFont=cmunit.ttf,BoldItalicFont=cmuntx.ttf]{cmunrm.ttf} is the step between numbered lines (the default is 1, which means every line will be numbered).

To use this environment, remember to add \LaTeXTT{\textbackslash{}usepackage\{moreverb\}} to the document preamble.
\subsection{Multiline comments}
\label{141}

As we have seen, the only way LaTeX allows you to add comments is by using the special character \LaTeXTT{\%}, that will comment out all the rest of the line after itself. This approach is really time-{}consuming if you want to insert long comments or just comment out a part of your document that you want to improve later, unless you\textquotesingle{}re using an \mylref{8}{editor} that automates this process. Alternatively, you can use the \LaTeXTT{verbatim} package, to be loaded in the preamble as usual:
\begin{Shaded}
\begin{Highlighting}[]

\NormalTok{\textbackslash{}usepackage\{verbatim\} }
\end{Highlighting}
\end{Shaded}


(you can also use the \LaTeXTT{comment} package instead)
you can use an environment called \LaTeXTT{comment} that will comment out everything within itself. Here is an example:

\begin{longtable}{p{1.0\linewidth}}
\begin{Shaded}
\begin{Highlighting}[]

\NormalTok{This is another}
\NormalTok{\textbackslash{}begin\{comment\}}
\CommentTok{rather stupid,}
\CommentTok{but helpful}
\NormalTok{\textbackslash{}end\{comment\}}
\NormalTok{example for embedding}
\NormalTok{comments in your document.}
\end{Highlighting}
\end{Shaded}
\\

This is another example for embedding comments in your document.

\end{longtable}

Note that this won’t work inside complex environments, like math for example. You may be wondering, why should I load a package called \LaTeXTT{verbatim} to have the possibility to add comments? The answer is straightforward: commented text is interpreted by the compiler just like verbatim text, the only difference is that verbatim text is introduced within the document, while the comment is just dropped.

Alternatively, you can define a \LaTeXTT{\textbackslash{}comment\{\}} command, by adding the following to the document\textquotesingle{}s preamble: 
\begin{Shaded}
\begin{Highlighting}[]

\NormalTok{\textbackslash{}newcommand\{\textbackslash{}comment\}[1]\{\}}
\end{Highlighting}
\end{Shaded}


Then, to comment out text, simply do something like this:
\begin{longtable}{p{1.0\linewidth}}
\begin{Shaded}
\begin{Highlighting}[]

\NormalTok{\textbackslash{}comment\{This is a long comment and can extend over multiple lines, etc.\} But it}
 \NormalTok{won't show.}
\end{Highlighting}
\end{Shaded}
\\
But it won\textquotesingle{}t show.

\end{longtable}

This approach can, however, produce unwanted spaces in the document, so it may work better to use
\begin{Shaded}
\begin{Highlighting}[]

\NormalTok{\textbackslash{}newcommand\{\textbackslash{}comment\}[2]\{#2\}}
\end{Highlighting}
\end{Shaded}

Then if you supply only one argument to \LaTeXTT{\textbackslash{}comment\{\}}, this has the desired effect without producing extra spaces.

Another drawback is that content is still parsed and possibly expanded, so you cannot put anything you want in it (such as LaTeX commands).\subsection{Skipping parts of the source}
\label{142}

A more robust way of making the TeX engine skip some part of the source is to use the TeX \LaTeXTT{\textbackslash{}iffalse}-{}conditional. The typical use is
\begin{longtable}{p{1.0\linewidth}}
\begin{Shaded}
\begin{Highlighting}[]

\NormalTok{This we want to keep}
 
\CommentTok{\textbackslash{}iffalse % ----- START THE CUT ---------}
\CommentTok{ }
\CommentTok{But this part }
\CommentTok{$$\textbackslash{}int_\{-\textbackslash{}infty\}^\textbackslash{}infty\textbackslash{}mathrm\{d\}x\textbackslash{},x^\{-2\}$$ }
\CommentTok{we want to skip}
\CommentTok{ }
\CommentTok{\textbackslash{}fi} \CommentTok{% ---------- END THE CUT -----------}
 
\NormalTok{Here it begins again}
\end{Highlighting}
\end{Shaded}
\\

This we want to keep

Here it begins again

\end{longtable}
The \LaTeXTT{\textbackslash{}iffalse}-{}conditional is always false.
\subsection{Quoting text}
\label{143}

LaTeX provides several environments for quoting text; they have small differences and they are aimed for different types of quotations. All of them are indented on either margin, and you will need to add your own quotation marks if you want them. The provided environments are:{\bfseries
\begin{mydescription}\LaTeXTT{quote}
\end{mydescription}
}

\begin{myquote}
\item{} for a short quotation, or a series of small quotes, separated by blank lines.
\end{myquote}
{\bfseries
\begin{mydescription}\LaTeXTT{quotation}
\end{mydescription}
}

\begin{myquote}
\item{} for use with longer quotations, of more than one paragraph, because it indents the first line of each paragraph.
\end{myquote}
{\bfseries
\begin{mydescription}\LaTeXTT{verse}
\end{mydescription}
}

\begin{myquote}
\item{} is for quotations where line breaks are important, such as poetry. Once in, new stanzas are created with a blank line, and new lines within a stanza are indicated using the newline command, 
\begin{Shaded}
\begin{Highlighting}[]

\NormalTok{\textbackslash{}\textbackslash{}}\newline
\end{Highlighting}
\end{Shaded}
. If a line takes up more than one line on the page, then all subsequent lines are indented until explicitly separated with 
\begin{Shaded}
\begin{Highlighting}[]

\NormalTok{\textbackslash{}\textbackslash{}}\newline
\end{Highlighting}
\end{Shaded}
.
\end{myquote}

\subsection{Abstracts}
\label{144}

In scientific publications it is customary to start with an abstract which gives the reader a quick overview of what to expect. See \mylref{96}{Document Structure}.
\section{Notes and References}
\label{145}
\ARoberts{}

\chapter{Colors}

\myminitoc
\label{146}

\label{147}

Adding colors to your text is supported by the \myhref{http://www.ctan.org/pkg/color}{{\ttfamily \setmainfont[Path=/usr/share/fonts/truetype/cmu/,UprightFont=cmunrm.ttf,BoldFont=cmunbx.ttf,ItalicFont=cmunti.ttf,BoldItalicFont=cmunbi.ttf]{cmuntt.ttf}\setmonofont[Path=/usr/share/fonts/truetype/cmu/,UprightFont=cmuntt.ttf,BoldFont=cmuntb.ttf,ItalicFont=cmunit.ttf,BoldItalicFont=cmuntx.ttf]{cmuntt.ttf}\ttfamily color}} package. Using this package, you can set the font color, text background, or page background. You can choose from predefined colors or define your own colors using RGB, Hex, or CMYK. Mathematical formulas can also be colored.
\section{Adding the color package}
\label{148}
To make use of these features, the color package must be imported.
\begin{Shaded}
\begin{Highlighting}[]

\NormalTok{\textbackslash{}usepackage\{color\}}
\end{Highlighting}
\end{Shaded}


Alternatively, one can write:  
\begin{Shaded}
\begin{Highlighting}[]

\NormalTok{\textbackslash{}usepackage[usenames,dvipsnames,svgnames,table]\{xcolor\}}
\end{Highlighting}
\end{Shaded}

The \LaTeXTT{\textbackslash{}usepackage} is obvious, but the initialization of additional commands like \LaTeXTT{usenames} allows you to use names of the default colors, the same 16 base colors as used in HTML. The \LaTeXTT{dvipsnames} allows you access to more colors, another 64, and \LaTeXTT{svgnames} allows access to about 150 colors. The initialization of \symbol{34}table\symbol{34}  allows colors to be added to tables by placing the color command just before the table. The package loaded here is the \myhref{http://www.ctan.org/pkg/xcolor}{{\ttfamily \setmainfont[Path=/usr/share/fonts/truetype/cmu/,UprightFont=cmunrm.ttf,BoldFont=cmunbx.ttf,ItalicFont=cmunti.ttf,BoldItalicFont=cmunbi.ttf]{cmuntt.ttf}\setmonofont[Path=/usr/share/fonts/truetype/cmu/,UprightFont=cmuntt.ttf,BoldFont=cmuntb.ttf,ItalicFont=cmunit.ttf,BoldItalicFont=cmuntx.ttf]{cmuntt.ttf}\ttfamily xcolor}} package.  

If you need more colors, then you may also want to look at adding the \LaTeXTT{x11names} to the initialization section as well, this offers more than 300 colors, but you need to make sure your xcolor package is the most recent you can download.
\section{Entering colored text}
\label{149}

The simplest way to type colored text is by:
\begin{Shaded}
\begin{Highlighting}[]

\NormalTok{\textbackslash{}textcolor\{declared-color\}\{text\}}
\end{Highlighting}
\end{Shaded}

where \LaTeXTT{declared-{}color} is a color that was defined before by \LaTeXTT{\textbackslash{}definecolor}.

Another possible way by

\begin{Shaded}
\begin{Highlighting}[]

\NormalTok{\{\textbackslash{}color\{declared-color\} some text\}}
\end{Highlighting}
\end{Shaded}

that will switch the standard text color to the color you want. It will work until the end of the current TeX group. For example:
\begin{longtable}{p{1.0\linewidth}}
\begin{Shaded}
\begin{Highlighting}[]

\NormalTok{\textbackslash{}emph\{some black text, \textbackslash{}color\{red\} followed by a red fragment\}, going black}
 \NormalTok{again.}
\end{Highlighting}
\end{Shaded}
\\



\begin{minipage}{0.95000\textwidth}
\begin{center}
\includegraphics[width=1.0\textwidth,height=6.5in,keepaspectratio]{../images/24.png}
\end{center}
\raggedright{}\myfigurewithoutcaption{24}
\end{minipage}\vspace{0.75cm}



\end{longtable}

The difference between
\LaTeXTT{\textbackslash{}textcolor}
and
\LaTeXTT{\textbackslash{}color}
is the same as that between
\LaTeXTT{\textbackslash{}texttt}
and
\LaTeXTT{\textbackslash{}ttfamily},
you can use the one you prefer. The \LaTeXTT{\textbackslash{}color} environment allows the text to run over multiple lines and other text environments whereas the text in \LaTeXTT{\textbackslash{}textcolor} must all be one paragraph and not contain other environments.

You can change the background color of the whole page by:
\begin{Shaded}
\begin{Highlighting}[]

\NormalTok{\textbackslash{}pagecolor\{declared-color\}}
\end{Highlighting}
\end{Shaded}

\section{Entering colored background for the text}
\label{150}

\begin{Shaded}
\begin{Highlighting}[]

\NormalTok{\textbackslash{}colorbox\{declared-color\}\{text\}}
\end{Highlighting}
\end{Shaded}


If the background color and the text color is changed, then:
\begin{Shaded}
\begin{Highlighting}[]

\NormalTok{\textbackslash{}colorbox\{declared-color1\}\{\textbackslash{}color\{declared-color2\}text\}}
\end{Highlighting}
\end{Shaded}


There is also \textbackslash{}fcolorbox to make framed background color in yet another color:
\begin{Shaded}
\begin{Highlighting}[]

\NormalTok{\textbackslash{}fcolorbox\{declared-color-frame\}\{declared-color-background\}\{text\}}
\end{Highlighting}
\end{Shaded}

\section{Predefined colors}
\label{151}

The predefined color names are\\

\TemplateSpaceIndent{$\text{ }${}black,$\text{ }${}blue,$\text{ }${}brown,$\text{ }${}cyan,$\text{ }${}darkgray,$\text{ }${}gray,$\text{ }${}green,$\text{ }${}lightgray,$\text{ }${}lime,$\text{ }${}magenta,$\text{ }$\newline{}
$\text{ }${}olive,$\text{ }${}orange,$\text{ }${}pink,$\text{ }${}purple,$\text{ }${}red,$\text{ }${}teal,$\text{ }${}violet,$\text{ }${}white,$\text{ }${}yellow.}

There may be other pre-{}defined colors on your system, but these should be available on all systems.

If you would like a color not pre-{}defined, you can use one of the 68 dvips colors, or define your own. These options are discussed in the following sections
\subsection{The 68 standard colors known to dvips}
\label{152}

Invoke the package with the usenames and dvipsnames option. If you are using \LaTeXTT{tikz} or \LaTeXTT{pstricks} package you must declare the xcolor package before that, otherwise it will not work.

\begin{Shaded}
\begin{Highlighting}[]

\NormalTok{\textbackslash{}usepackage[dvipsnames]\{xcolor\}}
\end{Highlighting}
\end{Shaded}


\begin{longtable}{|>{\RaggedRight}p{0.30088\linewidth}|>{\RaggedRight}p{0.11113\linewidth}|>{\RaggedRight}p{0.04357\linewidth}|>{\RaggedRight}p{0.29043\linewidth}|>{\RaggedRight}p{0.11113\linewidth}|} \hline 
{\bfseries \hspace*{0pt}\ignorespaces{}\hspace*{0pt} Name }&{\bfseries \hspace*{0pt}\ignorespaces{}\hspace*{0pt} Color }&{\bfseries \hspace*{0pt}\ignorespaces{}\hspace*{0pt} {\mbox{$~$}}}&{\bfseries \hspace*{0pt}\ignorespaces{}\hspace*{0pt} Name }&{\bfseries \hspace*{0pt}\ignorespaces{}\hspace*{0pt} Color}\endhead  \hline \hspace*{0pt}\ignorespaces{}\hspace*{0pt} {\bfseries \setmainfont[Path=/usr/share/fonts/truetype/cmu/,UprightFont=cmunrm.ttf,BoldFont=cmunbx.ttf,ItalicFont=cmunti.ttf,BoldItalicFont=cmunbi.ttf]{cmunbx.ttf}\setmonofont[Path=/usr/share/fonts/truetype/cmu/,UprightFont=cmuntt.ttf,BoldFont=cmuntb.ttf,ItalicFont=cmunit.ttf,BoldItalicFont=cmuntx.ttf]{cmunbx.ttf}\bfseries Apricot}{$\text{ }$}\setmainfont[Path=/usr/share/fonts/truetype/cmu/,UprightFont=cmunrm.ttf,BoldFont=cmunbx.ttf,ItalicFont=cmunti.ttf,BoldItalicFont=cmunbi.ttf]{cmunrm.ttf}\setmonofont[Path=/usr/share/fonts/truetype/cmu/,UprightFont=cmuntt.ttf,BoldFont=cmuntb.ttf,ItalicFont=cmunit.ttf,BoldItalicFont=cmuntx.ttf]{cmunrm.ttf} &\cellcolor[rgb]{0.98431,0.72549,0.50980}\hspace*{0pt}\ignorespaces{}\hspace*{0pt} {\mbox{$~$}}&\hspace*{0pt}\ignorespaces{}\hspace*{0pt}{\mbox{$~$}}&\hspace*{0pt}\ignorespaces{}\hspace*{0pt} {\bfseries \setmainfont[Path=/usr/share/fonts/truetype/cmu/,UprightFont=cmunrm.ttf,BoldFont=cmunbx.ttf,ItalicFont=cmunti.ttf,BoldItalicFont=cmunbi.ttf]{cmunbx.ttf}\setmonofont[Path=/usr/share/fonts/truetype/cmu/,UprightFont=cmuntt.ttf,BoldFont=cmuntb.ttf,ItalicFont=cmunit.ttf,BoldItalicFont=cmuntx.ttf]{cmunbx.ttf}\bfseries Aquamarine}{$\text{ }$}\setmainfont[Path=/usr/share/fonts/truetype/cmu/,UprightFont=cmunrm.ttf,BoldFont=cmunbx.ttf,ItalicFont=cmunti.ttf,BoldItalicFont=cmunbi.ttf]{cmunrm.ttf}\setmonofont[Path=/usr/share/fonts/truetype/cmu/,UprightFont=cmuntt.ttf,BoldFont=cmuntb.ttf,ItalicFont=cmunit.ttf,BoldItalicFont=cmuntx.ttf]{cmunrm.ttf} &\cellcolor[rgb]{0.00000,0.70980,0.74510}\hspace*{0pt}\ignorespaces{}\hspace*{0pt} {\mbox{$~$}}\\ \hline \hspace*{0pt}\ignorespaces{}\hspace*{0pt} {\bfseries \setmainfont[Path=/usr/share/fonts/truetype/cmu/,UprightFont=cmunrm.ttf,BoldFont=cmunbx.ttf,ItalicFont=cmunti.ttf,BoldItalicFont=cmunbi.ttf]{cmunbx.ttf}\setmonofont[Path=/usr/share/fonts/truetype/cmu/,UprightFont=cmuntt.ttf,BoldFont=cmuntb.ttf,ItalicFont=cmunit.ttf,BoldItalicFont=cmuntx.ttf]{cmunbx.ttf}\bfseries Bittersweet}{$\text{ }$}\setmainfont[Path=/usr/share/fonts/truetype/cmu/,UprightFont=cmunrm.ttf,BoldFont=cmunbx.ttf,ItalicFont=cmunti.ttf,BoldItalicFont=cmunbi.ttf]{cmunrm.ttf}\setmonofont[Path=/usr/share/fonts/truetype/cmu/,UprightFont=cmuntt.ttf,BoldFont=cmuntb.ttf,ItalicFont=cmunit.ttf,BoldItalicFont=cmuntx.ttf]{cmunrm.ttf}  &\cellcolor[rgb]{0.75294,0.30980,0.09020}\hspace*{0pt}\ignorespaces{}\hspace*{0pt} {\mbox{$~$}}&\hspace*{0pt}\ignorespaces{}\hspace*{0pt}{\mbox{$~$}}&\hspace*{0pt}\ignorespaces{}\hspace*{0pt} {\bfseries \setmainfont[Path=/usr/share/fonts/truetype/cmu/,UprightFont=cmunrm.ttf,BoldFont=cmunbx.ttf,ItalicFont=cmunti.ttf,BoldItalicFont=cmunbi.ttf]{cmunbx.ttf}\setmonofont[Path=/usr/share/fonts/truetype/cmu/,UprightFont=cmuntt.ttf,BoldFont=cmuntb.ttf,ItalicFont=cmunit.ttf,BoldItalicFont=cmuntx.ttf]{cmunbx.ttf}\bfseries Black}{$\text{ }$}\setmainfont[Path=/usr/share/fonts/truetype/cmu/,UprightFont=cmunrm.ttf,BoldFont=cmunbx.ttf,ItalicFont=cmunti.ttf,BoldItalicFont=cmunbi.ttf]{cmunrm.ttf}\setmonofont[Path=/usr/share/fonts/truetype/cmu/,UprightFont=cmuntt.ttf,BoldFont=cmuntb.ttf,ItalicFont=cmunit.ttf,BoldItalicFont=cmuntx.ttf]{cmunrm.ttf} &\cellcolor[rgb]{0.13333,0.11765,0.12157}\hspace*{0pt}\ignorespaces{}\hspace*{0pt} {\mbox{$~$}}\\ \hline \hspace*{0pt}\ignorespaces{}\hspace*{0pt}  {\bfseries \setmainfont[Path=/usr/share/fonts/truetype/cmu/,UprightFont=cmunrm.ttf,BoldFont=cmunbx.ttf,ItalicFont=cmunti.ttf,BoldItalicFont=cmunbi.ttf]{cmunbx.ttf}\setmonofont[Path=/usr/share/fonts/truetype/cmu/,UprightFont=cmuntt.ttf,BoldFont=cmuntb.ttf,ItalicFont=cmunit.ttf,BoldItalicFont=cmuntx.ttf]{cmunbx.ttf}\bfseries Blue}&\cellcolor[rgb]{0.17647,0.18431,0.57255}\hspace*{0pt}\ignorespaces{}\hspace*{0pt}{$\text{ }$}\setmainfont[Path=/usr/share/fonts/truetype/cmu/,UprightFont=cmunrm.ttf,BoldFont=cmunbx.ttf,ItalicFont=cmunti.ttf,BoldItalicFont=cmunbi.ttf]{cmunrm.ttf}\setmonofont[Path=/usr/share/fonts/truetype/cmu/,UprightFont=cmuntt.ttf,BoldFont=cmuntb.ttf,ItalicFont=cmunit.ttf,BoldItalicFont=cmuntx.ttf]{cmunrm.ttf} {\mbox{$~$}}&\hspace*{0pt}\ignorespaces{}\hspace*{0pt}{\mbox{$~$}}&\hspace*{0pt}\ignorespaces{}\hspace*{0pt} {\bfseries \setmainfont[Path=/usr/share/fonts/truetype/cmu/,UprightFont=cmunrm.ttf,BoldFont=cmunbx.ttf,ItalicFont=cmunti.ttf,BoldItalicFont=cmunbi.ttf]{cmunbx.ttf}\setmonofont[Path=/usr/share/fonts/truetype/cmu/,UprightFont=cmuntt.ttf,BoldFont=cmuntb.ttf,ItalicFont=cmunit.ttf,BoldItalicFont=cmuntx.ttf]{cmunbx.ttf}\bfseries BlueGreen}{$\text{ }$}\setmainfont[Path=/usr/share/fonts/truetype/cmu/,UprightFont=cmunrm.ttf,BoldFont=cmunbx.ttf,ItalicFont=cmunti.ttf,BoldItalicFont=cmunbi.ttf]{cmunrm.ttf}\setmonofont[Path=/usr/share/fonts/truetype/cmu/,UprightFont=cmuntt.ttf,BoldFont=cmuntb.ttf,ItalicFont=cmunit.ttf,BoldItalicFont=cmuntx.ttf]{cmunrm.ttf} &\cellcolor[rgb]{0.00000,0.70196,0.72157}\hspace*{0pt}\ignorespaces{}\hspace*{0pt} {\mbox{$~$}}\\ \hline \hspace*{0pt}\ignorespaces{}\hspace*{0pt} {\bfseries \setmainfont[Path=/usr/share/fonts/truetype/cmu/,UprightFont=cmunrm.ttf,BoldFont=cmunbx.ttf,ItalicFont=cmunti.ttf,BoldItalicFont=cmunbi.ttf]{cmunbx.ttf}\setmonofont[Path=/usr/share/fonts/truetype/cmu/,UprightFont=cmuntt.ttf,BoldFont=cmuntb.ttf,ItalicFont=cmunit.ttf,BoldItalicFont=cmuntx.ttf]{cmunbx.ttf}\bfseries BlueViolet}{$\text{ }$}\setmainfont[Path=/usr/share/fonts/truetype/cmu/,UprightFont=cmunrm.ttf,BoldFont=cmunbx.ttf,ItalicFont=cmunti.ttf,BoldItalicFont=cmunbi.ttf]{cmunrm.ttf}\setmonofont[Path=/usr/share/fonts/truetype/cmu/,UprightFont=cmuntt.ttf,BoldFont=cmuntb.ttf,ItalicFont=cmunit.ttf,BoldItalicFont=cmuntx.ttf]{cmunrm.ttf} &\cellcolor[rgb]{0.27843,0.22353,0.57255}\hspace*{0pt}\ignorespaces{}\hspace*{0pt} {\mbox{$~$}}&\hspace*{0pt}\ignorespaces{}\hspace*{0pt}{\mbox{$~$}}&\hspace*{0pt}\ignorespaces{}\hspace*{0pt} {\bfseries \setmainfont[Path=/usr/share/fonts/truetype/cmu/,UprightFont=cmunrm.ttf,BoldFont=cmunbx.ttf,ItalicFont=cmunti.ttf,BoldItalicFont=cmunbi.ttf]{cmunbx.ttf}\setmonofont[Path=/usr/share/fonts/truetype/cmu/,UprightFont=cmuntt.ttf,BoldFont=cmuntb.ttf,ItalicFont=cmunit.ttf,BoldItalicFont=cmuntx.ttf]{cmunbx.ttf}\bfseries BrickRed}{$\text{ }$}\setmainfont[Path=/usr/share/fonts/truetype/cmu/,UprightFont=cmunrm.ttf,BoldFont=cmunbx.ttf,ItalicFont=cmunti.ttf,BoldItalicFont=cmunbi.ttf]{cmunrm.ttf}\setmonofont[Path=/usr/share/fonts/truetype/cmu/,UprightFont=cmuntt.ttf,BoldFont=cmuntb.ttf,ItalicFont=cmunit.ttf,BoldItalicFont=cmuntx.ttf]{cmunrm.ttf} &\cellcolor[rgb]{0.71373,0.19608,0.10980}\hspace*{0pt}\ignorespaces{}\hspace*{0pt} {\mbox{$~$}}\\ \hline \hspace*{0pt}\ignorespaces{}\hspace*{0pt} {\bfseries \setmainfont[Path=/usr/share/fonts/truetype/cmu/,UprightFont=cmunrm.ttf,BoldFont=cmunbx.ttf,ItalicFont=cmunti.ttf,BoldItalicFont=cmunbi.ttf]{cmunbx.ttf}\setmonofont[Path=/usr/share/fonts/truetype/cmu/,UprightFont=cmuntt.ttf,BoldFont=cmuntb.ttf,ItalicFont=cmunit.ttf,BoldItalicFont=cmuntx.ttf]{cmunbx.ttf}\bfseries Brown}{$\text{ }$}\setmainfont[Path=/usr/share/fonts/truetype/cmu/,UprightFont=cmunrm.ttf,BoldFont=cmunbx.ttf,ItalicFont=cmunti.ttf,BoldItalicFont=cmunbi.ttf]{cmunrm.ttf}\setmonofont[Path=/usr/share/fonts/truetype/cmu/,UprightFont=cmuntt.ttf,BoldFont=cmuntb.ttf,ItalicFont=cmunit.ttf,BoldItalicFont=cmuntx.ttf]{cmunrm.ttf} &\cellcolor[rgb]{0.47451,0.14510,0.00000}\hspace*{0pt}\ignorespaces{}\hspace*{0pt} {\mbox{$~$}}&\hspace*{0pt}\ignorespaces{}\hspace*{0pt}{\mbox{$~$}}&\hspace*{0pt}\ignorespaces{}\hspace*{0pt} {\bfseries \setmainfont[Path=/usr/share/fonts/truetype/cmu/,UprightFont=cmunrm.ttf,BoldFont=cmunbx.ttf,ItalicFont=cmunti.ttf,BoldItalicFont=cmunbi.ttf]{cmunbx.ttf}\setmonofont[Path=/usr/share/fonts/truetype/cmu/,UprightFont=cmuntt.ttf,BoldFont=cmuntb.ttf,ItalicFont=cmunit.ttf,BoldItalicFont=cmuntx.ttf]{cmunbx.ttf}\bfseries BurntOrange}{$\text{ }$}\setmainfont[Path=/usr/share/fonts/truetype/cmu/,UprightFont=cmunrm.ttf,BoldFont=cmunbx.ttf,ItalicFont=cmunti.ttf,BoldItalicFont=cmunbi.ttf]{cmunrm.ttf}\setmonofont[Path=/usr/share/fonts/truetype/cmu/,UprightFont=cmuntt.ttf,BoldFont=cmuntb.ttf,ItalicFont=cmunit.ttf,BoldItalicFont=cmuntx.ttf]{cmunrm.ttf} &\cellcolor[rgb]{0.96863,0.57255,0.11373}\hspace*{0pt}\ignorespaces{}\hspace*{0pt} {\mbox{$~$}}\\ \hline \hspace*{0pt}\ignorespaces{}\hspace*{0pt} {\bfseries \setmainfont[Path=/usr/share/fonts/truetype/cmu/,UprightFont=cmunrm.ttf,BoldFont=cmunbx.ttf,ItalicFont=cmunti.ttf,BoldItalicFont=cmunbi.ttf]{cmunbx.ttf}\setmonofont[Path=/usr/share/fonts/truetype/cmu/,UprightFont=cmuntt.ttf,BoldFont=cmuntb.ttf,ItalicFont=cmunit.ttf,BoldItalicFont=cmuntx.ttf]{cmunbx.ttf}\bfseries CadetBlue}{$\text{ }$}\setmainfont[Path=/usr/share/fonts/truetype/cmu/,UprightFont=cmunrm.ttf,BoldFont=cmunbx.ttf,ItalicFont=cmunti.ttf,BoldItalicFont=cmunbi.ttf]{cmunrm.ttf}\setmonofont[Path=/usr/share/fonts/truetype/cmu/,UprightFont=cmuntt.ttf,BoldFont=cmuntb.ttf,ItalicFont=cmunit.ttf,BoldItalicFont=cmuntx.ttf]{cmunrm.ttf} &\cellcolor[rgb]{0.45490,0.44706,0.60392}\hspace*{0pt}\ignorespaces{}\hspace*{0pt} {\mbox{$~$}}&\hspace*{0pt}\ignorespaces{}\hspace*{0pt}{\mbox{$~$}}&\hspace*{0pt}\ignorespaces{}\hspace*{0pt} {\bfseries \setmainfont[Path=/usr/share/fonts/truetype/cmu/,UprightFont=cmunrm.ttf,BoldFont=cmunbx.ttf,ItalicFont=cmunti.ttf,BoldItalicFont=cmunbi.ttf]{cmunbx.ttf}\setmonofont[Path=/usr/share/fonts/truetype/cmu/,UprightFont=cmuntt.ttf,BoldFont=cmuntb.ttf,ItalicFont=cmunit.ttf,BoldItalicFont=cmuntx.ttf]{cmunbx.ttf}\bfseries CarnationPink}{$\text{ }$}\setmainfont[Path=/usr/share/fonts/truetype/cmu/,UprightFont=cmunrm.ttf,BoldFont=cmunbx.ttf,ItalicFont=cmunti.ttf,BoldItalicFont=cmunbi.ttf]{cmunrm.ttf}\setmonofont[Path=/usr/share/fonts/truetype/cmu/,UprightFont=cmuntt.ttf,BoldFont=cmuntb.ttf,ItalicFont=cmunit.ttf,BoldItalicFont=cmuntx.ttf]{cmunrm.ttf} &\cellcolor[rgb]{0.94902,0.50980,0.70588}\hspace*{0pt}\ignorespaces{}\hspace*{0pt} {\mbox{$~$}}\\ \hline \hspace*{0pt}\ignorespaces{}\hspace*{0pt} {\bfseries \setmainfont[Path=/usr/share/fonts/truetype/cmu/,UprightFont=cmunrm.ttf,BoldFont=cmunbx.ttf,ItalicFont=cmunti.ttf,BoldItalicFont=cmunbi.ttf]{cmunbx.ttf}\setmonofont[Path=/usr/share/fonts/truetype/cmu/,UprightFont=cmuntt.ttf,BoldFont=cmuntb.ttf,ItalicFont=cmunit.ttf,BoldItalicFont=cmuntx.ttf]{cmunbx.ttf}\bfseries Cerulean}{$\text{ }$}\setmainfont[Path=/usr/share/fonts/truetype/cmu/,UprightFont=cmunrm.ttf,BoldFont=cmunbx.ttf,ItalicFont=cmunti.ttf,BoldItalicFont=cmunbi.ttf]{cmunrm.ttf}\setmonofont[Path=/usr/share/fonts/truetype/cmu/,UprightFont=cmuntt.ttf,BoldFont=cmuntb.ttf,ItalicFont=cmunit.ttf,BoldItalicFont=cmuntx.ttf]{cmunrm.ttf} &\cellcolor[rgb]{0.00000,0.63529,0.89020}\hspace*{0pt}\ignorespaces{}\hspace*{0pt} {\mbox{$~$}}&\hspace*{0pt}\ignorespaces{}\hspace*{0pt}{\mbox{$~$}}&\hspace*{0pt}\ignorespaces{}\hspace*{0pt} {\bfseries \setmainfont[Path=/usr/share/fonts/truetype/cmu/,UprightFont=cmunrm.ttf,BoldFont=cmunbx.ttf,ItalicFont=cmunti.ttf,BoldItalicFont=cmunbi.ttf]{cmunbx.ttf}\setmonofont[Path=/usr/share/fonts/truetype/cmu/,UprightFont=cmuntt.ttf,BoldFont=cmuntb.ttf,ItalicFont=cmunit.ttf,BoldItalicFont=cmuntx.ttf]{cmunbx.ttf}\bfseries CornflowerBlue}{$\text{ }$}\setmainfont[Path=/usr/share/fonts/truetype/cmu/,UprightFont=cmunrm.ttf,BoldFont=cmunbx.ttf,ItalicFont=cmunti.ttf,BoldItalicFont=cmunbi.ttf]{cmunrm.ttf}\setmonofont[Path=/usr/share/fonts/truetype/cmu/,UprightFont=cmuntt.ttf,BoldFont=cmuntb.ttf,ItalicFont=cmunit.ttf,BoldItalicFont=cmuntx.ttf]{cmunrm.ttf} &\cellcolor[rgb]{0.25490,0.69020,0.89412}\hspace*{0pt}\ignorespaces{}\hspace*{0pt} {\mbox{$~$}}\\ \hline \hspace*{0pt}\ignorespaces{}\hspace*{0pt} {\bfseries \setmainfont[Path=/usr/share/fonts/truetype/cmu/,UprightFont=cmunrm.ttf,BoldFont=cmunbx.ttf,ItalicFont=cmunti.ttf,BoldItalicFont=cmunbi.ttf]{cmunbx.ttf}\setmonofont[Path=/usr/share/fonts/truetype/cmu/,UprightFont=cmuntt.ttf,BoldFont=cmuntb.ttf,ItalicFont=cmunit.ttf,BoldItalicFont=cmuntx.ttf]{cmunbx.ttf}\bfseries Cyan}{$\text{ }$}\setmainfont[Path=/usr/share/fonts/truetype/cmu/,UprightFont=cmunrm.ttf,BoldFont=cmunbx.ttf,ItalicFont=cmunti.ttf,BoldItalicFont=cmunbi.ttf]{cmunrm.ttf}\setmonofont[Path=/usr/share/fonts/truetype/cmu/,UprightFont=cmuntt.ttf,BoldFont=cmuntb.ttf,ItalicFont=cmunit.ttf,BoldItalicFont=cmuntx.ttf]{cmunrm.ttf} &\cellcolor[rgb]{0.00000,0.68235,0.93725}\hspace*{0pt}\ignorespaces{}\hspace*{0pt} {\mbox{$~$}}&\hspace*{0pt}\ignorespaces{}\hspace*{0pt}{\mbox{$~$}}&\hspace*{0pt}\ignorespaces{}\hspace*{0pt} {\bfseries \setmainfont[Path=/usr/share/fonts/truetype/cmu/,UprightFont=cmunrm.ttf,BoldFont=cmunbx.ttf,ItalicFont=cmunti.ttf,BoldItalicFont=cmunbi.ttf]{cmunbx.ttf}\setmonofont[Path=/usr/share/fonts/truetype/cmu/,UprightFont=cmuntt.ttf,BoldFont=cmuntb.ttf,ItalicFont=cmunit.ttf,BoldItalicFont=cmuntx.ttf]{cmunbx.ttf}\bfseries Dandelion}{$\text{ }$}\setmainfont[Path=/usr/share/fonts/truetype/cmu/,UprightFont=cmunrm.ttf,BoldFont=cmunbx.ttf,ItalicFont=cmunti.ttf,BoldItalicFont=cmunbi.ttf]{cmunrm.ttf}\setmonofont[Path=/usr/share/fonts/truetype/cmu/,UprightFont=cmuntt.ttf,BoldFont=cmuntb.ttf,ItalicFont=cmunit.ttf,BoldItalicFont=cmuntx.ttf]{cmunrm.ttf} &\cellcolor[rgb]{0.99216,0.73725,0.25882}\hspace*{0pt}\ignorespaces{}\hspace*{0pt} {\mbox{$~$}}\\ \hline \hspace*{0pt}\ignorespaces{}\hspace*{0pt} {\bfseries \setmainfont[Path=/usr/share/fonts/truetype/cmu/,UprightFont=cmunrm.ttf,BoldFont=cmunbx.ttf,ItalicFont=cmunti.ttf,BoldItalicFont=cmunbi.ttf]{cmunbx.ttf}\setmonofont[Path=/usr/share/fonts/truetype/cmu/,UprightFont=cmuntt.ttf,BoldFont=cmuntb.ttf,ItalicFont=cmunit.ttf,BoldItalicFont=cmuntx.ttf]{cmunbx.ttf}\bfseries DarkOrchid}{$\text{ }$}\setmainfont[Path=/usr/share/fonts/truetype/cmu/,UprightFont=cmunrm.ttf,BoldFont=cmunbx.ttf,ItalicFont=cmunti.ttf,BoldItalicFont=cmunbi.ttf]{cmunrm.ttf}\setmonofont[Path=/usr/share/fonts/truetype/cmu/,UprightFont=cmuntt.ttf,BoldFont=cmuntb.ttf,ItalicFont=cmunit.ttf,BoldItalicFont=cmuntx.ttf]{cmunrm.ttf} &\cellcolor[rgb]{0.64314,0.32549,0.54118}\hspace*{0pt}\ignorespaces{}\hspace*{0pt} {\mbox{$~$}}&\hspace*{0pt}\ignorespaces{}\hspace*{0pt}{\mbox{$~$}}&\hspace*{0pt}\ignorespaces{}\hspace*{0pt} {\bfseries \setmainfont[Path=/usr/share/fonts/truetype/cmu/,UprightFont=cmunrm.ttf,BoldFont=cmunbx.ttf,ItalicFont=cmunti.ttf,BoldItalicFont=cmunbi.ttf]{cmunbx.ttf}\setmonofont[Path=/usr/share/fonts/truetype/cmu/,UprightFont=cmuntt.ttf,BoldFont=cmuntb.ttf,ItalicFont=cmunit.ttf,BoldItalicFont=cmuntx.ttf]{cmunbx.ttf}\bfseries Emerald}{$\text{ }$}\setmainfont[Path=/usr/share/fonts/truetype/cmu/,UprightFont=cmunrm.ttf,BoldFont=cmunbx.ttf,ItalicFont=cmunti.ttf,BoldItalicFont=cmunbi.ttf]{cmunrm.ttf}\setmonofont[Path=/usr/share/fonts/truetype/cmu/,UprightFont=cmuntt.ttf,BoldFont=cmuntb.ttf,ItalicFont=cmunit.ttf,BoldItalicFont=cmuntx.ttf]{cmunrm.ttf} &\cellcolor[rgb]{0.00000,0.66275,0.61569}\hspace*{0pt}\ignorespaces{}\hspace*{0pt} {\mbox{$~$}}\\ \hline \hspace*{0pt}\ignorespaces{}\hspace*{0pt} {\bfseries \setmainfont[Path=/usr/share/fonts/truetype/cmu/,UprightFont=cmunrm.ttf,BoldFont=cmunbx.ttf,ItalicFont=cmunti.ttf,BoldItalicFont=cmunbi.ttf]{cmunbx.ttf}\setmonofont[Path=/usr/share/fonts/truetype/cmu/,UprightFont=cmuntt.ttf,BoldFont=cmuntb.ttf,ItalicFont=cmunit.ttf,BoldItalicFont=cmuntx.ttf]{cmunbx.ttf}\bfseries ForestGreen}{$\text{ }$}\setmainfont[Path=/usr/share/fonts/truetype/cmu/,UprightFont=cmunrm.ttf,BoldFont=cmunbx.ttf,ItalicFont=cmunti.ttf,BoldItalicFont=cmunbi.ttf]{cmunrm.ttf}\setmonofont[Path=/usr/share/fonts/truetype/cmu/,UprightFont=cmuntt.ttf,BoldFont=cmuntb.ttf,ItalicFont=cmunit.ttf,BoldItalicFont=cmuntx.ttf]{cmunrm.ttf} &\cellcolor[rgb]{0.00000,0.60784,0.33333}\hspace*{0pt}\ignorespaces{}\hspace*{0pt} {\mbox{$~$}}&\hspace*{0pt}\ignorespaces{}\hspace*{0pt}{\mbox{$~$}}&\hspace*{0pt}\ignorespaces{}\hspace*{0pt} {\bfseries \setmainfont[Path=/usr/share/fonts/truetype/cmu/,UprightFont=cmunrm.ttf,BoldFont=cmunbx.ttf,ItalicFont=cmunti.ttf,BoldItalicFont=cmunbi.ttf]{cmunbx.ttf}\setmonofont[Path=/usr/share/fonts/truetype/cmu/,UprightFont=cmuntt.ttf,BoldFont=cmuntb.ttf,ItalicFont=cmunit.ttf,BoldItalicFont=cmuntx.ttf]{cmunbx.ttf}\bfseries Fuchsia}{$\text{ }$}\setmainfont[Path=/usr/share/fonts/truetype/cmu/,UprightFont=cmunrm.ttf,BoldFont=cmunbx.ttf,ItalicFont=cmunti.ttf,BoldItalicFont=cmunbi.ttf]{cmunrm.ttf}\setmonofont[Path=/usr/share/fonts/truetype/cmu/,UprightFont=cmuntt.ttf,BoldFont=cmuntb.ttf,ItalicFont=cmunit.ttf,BoldItalicFont=cmuntx.ttf]{cmunrm.ttf} &\cellcolor[rgb]{0.54902,0.21176,0.54902}\hspace*{0pt}\ignorespaces{}\hspace*{0pt} {\mbox{$~$}}\\ \hline \hspace*{0pt}\ignorespaces{}\hspace*{0pt} {\bfseries \setmainfont[Path=/usr/share/fonts/truetype/cmu/,UprightFont=cmunrm.ttf,BoldFont=cmunbx.ttf,ItalicFont=cmunti.ttf,BoldItalicFont=cmunbi.ttf]{cmunbx.ttf}\setmonofont[Path=/usr/share/fonts/truetype/cmu/,UprightFont=cmuntt.ttf,BoldFont=cmuntb.ttf,ItalicFont=cmunit.ttf,BoldItalicFont=cmuntx.ttf]{cmunbx.ttf}\bfseries Goldenrod}{$\text{ }$}\setmainfont[Path=/usr/share/fonts/truetype/cmu/,UprightFont=cmunrm.ttf,BoldFont=cmunbx.ttf,ItalicFont=cmunti.ttf,BoldItalicFont=cmunbi.ttf]{cmunrm.ttf}\setmonofont[Path=/usr/share/fonts/truetype/cmu/,UprightFont=cmuntt.ttf,BoldFont=cmuntb.ttf,ItalicFont=cmunit.ttf,BoldItalicFont=cmuntx.ttf]{cmunrm.ttf} &\cellcolor[rgb]{1.00000,0.87451,0.25882}\hspace*{0pt}\ignorespaces{}\hspace*{0pt} {\mbox{$~$}}&\hspace*{0pt}\ignorespaces{}\hspace*{0pt}{\mbox{$~$}}&\hspace*{0pt}\ignorespaces{}\hspace*{0pt} {\bfseries \setmainfont[Path=/usr/share/fonts/truetype/cmu/,UprightFont=cmunrm.ttf,BoldFont=cmunbx.ttf,ItalicFont=cmunti.ttf,BoldItalicFont=cmunbi.ttf]{cmunbx.ttf}\setmonofont[Path=/usr/share/fonts/truetype/cmu/,UprightFont=cmuntt.ttf,BoldFont=cmuntb.ttf,ItalicFont=cmunit.ttf,BoldItalicFont=cmuntx.ttf]{cmunbx.ttf}\bfseries Gray}{$\text{ }$}\setmainfont[Path=/usr/share/fonts/truetype/cmu/,UprightFont=cmunrm.ttf,BoldFont=cmunbx.ttf,ItalicFont=cmunti.ttf,BoldItalicFont=cmunbi.ttf]{cmunrm.ttf}\setmonofont[Path=/usr/share/fonts/truetype/cmu/,UprightFont=cmuntt.ttf,BoldFont=cmuntb.ttf,ItalicFont=cmunit.ttf,BoldItalicFont=cmuntx.ttf]{cmunrm.ttf} &\cellcolor[rgb]{0.58039,0.58824,0.59608}\hspace*{0pt}\ignorespaces{}\hspace*{0pt} {\mbox{$~$}}\\ \hline \hspace*{0pt}\ignorespaces{}\hspace*{0pt} {\bfseries \setmainfont[Path=/usr/share/fonts/truetype/cmu/,UprightFont=cmunrm.ttf,BoldFont=cmunbx.ttf,ItalicFont=cmunti.ttf,BoldItalicFont=cmunbi.ttf]{cmunbx.ttf}\setmonofont[Path=/usr/share/fonts/truetype/cmu/,UprightFont=cmuntt.ttf,BoldFont=cmuntb.ttf,ItalicFont=cmunit.ttf,BoldItalicFont=cmuntx.ttf]{cmunbx.ttf}\bfseries Green}{$\text{ }$}\setmainfont[Path=/usr/share/fonts/truetype/cmu/,UprightFont=cmunrm.ttf,BoldFont=cmunbx.ttf,ItalicFont=cmunti.ttf,BoldItalicFont=cmunbi.ttf]{cmunrm.ttf}\setmonofont[Path=/usr/share/fonts/truetype/cmu/,UprightFont=cmuntt.ttf,BoldFont=cmuntb.ttf,ItalicFont=cmunit.ttf,BoldItalicFont=cmuntx.ttf]{cmunrm.ttf} &\cellcolor[rgb]{0.00000,0.65098,0.30980}\hspace*{0pt}\ignorespaces{}\hspace*{0pt} {\mbox{$~$}}&\hspace*{0pt}\ignorespaces{}\hspace*{0pt}{\mbox{$~$}}&\hspace*{0pt}\ignorespaces{}\hspace*{0pt} {\bfseries \setmainfont[Path=/usr/share/fonts/truetype/cmu/,UprightFont=cmunrm.ttf,BoldFont=cmunbx.ttf,ItalicFont=cmunti.ttf,BoldItalicFont=cmunbi.ttf]{cmunbx.ttf}\setmonofont[Path=/usr/share/fonts/truetype/cmu/,UprightFont=cmuntt.ttf,BoldFont=cmuntb.ttf,ItalicFont=cmunit.ttf,BoldItalicFont=cmuntx.ttf]{cmunbx.ttf}\bfseries GreenYellow}{$\text{ }$}\setmainfont[Path=/usr/share/fonts/truetype/cmu/,UprightFont=cmunrm.ttf,BoldFont=cmunbx.ttf,ItalicFont=cmunti.ttf,BoldItalicFont=cmunbi.ttf]{cmunrm.ttf}\setmonofont[Path=/usr/share/fonts/truetype/cmu/,UprightFont=cmuntt.ttf,BoldFont=cmuntb.ttf,ItalicFont=cmunit.ttf,BoldItalicFont=cmuntx.ttf]{cmunrm.ttf} &\cellcolor[rgb]{0.87451,0.90196,0.45490}\hspace*{0pt}\ignorespaces{}\hspace*{0pt} {\mbox{$~$}}\\ \hline \hspace*{0pt}\ignorespaces{}\hspace*{0pt}  {\bfseries \setmainfont[Path=/usr/share/fonts/truetype/cmu/,UprightFont=cmunrm.ttf,BoldFont=cmunbx.ttf,ItalicFont=cmunti.ttf,BoldItalicFont=cmunbi.ttf]{cmunbx.ttf}\setmonofont[Path=/usr/share/fonts/truetype/cmu/,UprightFont=cmuntt.ttf,BoldFont=cmuntb.ttf,ItalicFont=cmunit.ttf,BoldItalicFont=cmuntx.ttf]{cmunbx.ttf}\bfseries JungleGreen}&\cellcolor[rgb]{0.00000,0.66275,0.60392}\hspace*{0pt}\ignorespaces{}\hspace*{0pt}{$\text{ }$}\setmainfont[Path=/usr/share/fonts/truetype/cmu/,UprightFont=cmunrm.ttf,BoldFont=cmunbx.ttf,ItalicFont=cmunti.ttf,BoldItalicFont=cmunbi.ttf]{cmunrm.ttf}\setmonofont[Path=/usr/share/fonts/truetype/cmu/,UprightFont=cmuntt.ttf,BoldFont=cmuntb.ttf,ItalicFont=cmunit.ttf,BoldItalicFont=cmuntx.ttf]{cmunrm.ttf} {\mbox{$~$}}&\hspace*{0pt}\ignorespaces{}\hspace*{0pt}{\mbox{$~$}}&\hspace*{0pt}\ignorespaces{}\hspace*{0pt} {\bfseries \setmainfont[Path=/usr/share/fonts/truetype/cmu/,UprightFont=cmunrm.ttf,BoldFont=cmunbx.ttf,ItalicFont=cmunti.ttf,BoldItalicFont=cmunbi.ttf]{cmunbx.ttf}\setmonofont[Path=/usr/share/fonts/truetype/cmu/,UprightFont=cmuntt.ttf,BoldFont=cmuntb.ttf,ItalicFont=cmunit.ttf,BoldItalicFont=cmuntx.ttf]{cmunbx.ttf}\bfseries Lavender}{$\text{ }$}\setmainfont[Path=/usr/share/fonts/truetype/cmu/,UprightFont=cmunrm.ttf,BoldFont=cmunbx.ttf,ItalicFont=cmunti.ttf,BoldItalicFont=cmunbi.ttf]{cmunrm.ttf}\setmonofont[Path=/usr/share/fonts/truetype/cmu/,UprightFont=cmuntt.ttf,BoldFont=cmuntb.ttf,ItalicFont=cmunit.ttf,BoldItalicFont=cmuntx.ttf]{cmunrm.ttf} &\cellcolor[rgb]{0.95686,0.61961,0.76863}\hspace*{0pt}\ignorespaces{}\hspace*{0pt} {\mbox{$~$}}\\ \hline \hspace*{0pt}\ignorespaces{}\hspace*{0pt} {\bfseries \setmainfont[Path=/usr/share/fonts/truetype/cmu/,UprightFont=cmunrm.ttf,BoldFont=cmunbx.ttf,ItalicFont=cmunti.ttf,BoldItalicFont=cmunbi.ttf]{cmunbx.ttf}\setmonofont[Path=/usr/share/fonts/truetype/cmu/,UprightFont=cmuntt.ttf,BoldFont=cmuntb.ttf,ItalicFont=cmunit.ttf,BoldItalicFont=cmuntx.ttf]{cmunbx.ttf}\bfseries LimeGreen}{$\text{ }$}\setmainfont[Path=/usr/share/fonts/truetype/cmu/,UprightFont=cmunrm.ttf,BoldFont=cmunbx.ttf,ItalicFont=cmunti.ttf,BoldItalicFont=cmunbi.ttf]{cmunrm.ttf}\setmonofont[Path=/usr/share/fonts/truetype/cmu/,UprightFont=cmuntt.ttf,BoldFont=cmuntb.ttf,ItalicFont=cmunit.ttf,BoldItalicFont=cmuntx.ttf]{cmunrm.ttf} &\cellcolor[rgb]{0.55294,0.78039,0.24314}\hspace*{0pt}\ignorespaces{}\hspace*{0pt} {\mbox{$~$}}&\hspace*{0pt}\ignorespaces{}\hspace*{0pt}{\mbox{$~$}}&\hspace*{0pt}\ignorespaces{}\hspace*{0pt} {\bfseries \setmainfont[Path=/usr/share/fonts/truetype/cmu/,UprightFont=cmunrm.ttf,BoldFont=cmunbx.ttf,ItalicFont=cmunti.ttf,BoldItalicFont=cmunbi.ttf]{cmunbx.ttf}\setmonofont[Path=/usr/share/fonts/truetype/cmu/,UprightFont=cmuntt.ttf,BoldFont=cmuntb.ttf,ItalicFont=cmunit.ttf,BoldItalicFont=cmuntx.ttf]{cmunbx.ttf}\bfseries Magenta}{$\text{ }$}\setmainfont[Path=/usr/share/fonts/truetype/cmu/,UprightFont=cmunrm.ttf,BoldFont=cmunbx.ttf,ItalicFont=cmunti.ttf,BoldItalicFont=cmunbi.ttf]{cmunrm.ttf}\setmonofont[Path=/usr/share/fonts/truetype/cmu/,UprightFont=cmuntt.ttf,BoldFont=cmuntb.ttf,ItalicFont=cmunit.ttf,BoldItalicFont=cmuntx.ttf]{cmunrm.ttf} &\cellcolor[rgb]{0.92549,0.00000,0.54902}\hspace*{0pt}\ignorespaces{}\hspace*{0pt} {\mbox{$~$}}\\ \hline \hspace*{0pt}\ignorespaces{}\hspace*{0pt} {\bfseries \setmainfont[Path=/usr/share/fonts/truetype/cmu/,UprightFont=cmunrm.ttf,BoldFont=cmunbx.ttf,ItalicFont=cmunti.ttf,BoldItalicFont=cmunbi.ttf]{cmunbx.ttf}\setmonofont[Path=/usr/share/fonts/truetype/cmu/,UprightFont=cmuntt.ttf,BoldFont=cmuntb.ttf,ItalicFont=cmunit.ttf,BoldItalicFont=cmuntx.ttf]{cmunbx.ttf}\bfseries Mahogany}{$\text{ }$}\setmainfont[Path=/usr/share/fonts/truetype/cmu/,UprightFont=cmunrm.ttf,BoldFont=cmunbx.ttf,ItalicFont=cmunti.ttf,BoldItalicFont=cmunbi.ttf]{cmunrm.ttf}\setmonofont[Path=/usr/share/fonts/truetype/cmu/,UprightFont=cmuntt.ttf,BoldFont=cmuntb.ttf,ItalicFont=cmunit.ttf,BoldItalicFont=cmuntx.ttf]{cmunrm.ttf} &\cellcolor[rgb]{0.66275,0.20392,0.12157}\hspace*{0pt}\ignorespaces{}\hspace*{0pt} {\mbox{$~$}}&\hspace*{0pt}\ignorespaces{}\hspace*{0pt}{\mbox{$~$}}&\hspace*{0pt}\ignorespaces{}\hspace*{0pt} {\bfseries \setmainfont[Path=/usr/share/fonts/truetype/cmu/,UprightFont=cmunrm.ttf,BoldFont=cmunbx.ttf,ItalicFont=cmunti.ttf,BoldItalicFont=cmunbi.ttf]{cmunbx.ttf}\setmonofont[Path=/usr/share/fonts/truetype/cmu/,UprightFont=cmuntt.ttf,BoldFont=cmuntb.ttf,ItalicFont=cmunit.ttf,BoldItalicFont=cmuntx.ttf]{cmunbx.ttf}\bfseries Maroon}{$\text{ }$}\setmainfont[Path=/usr/share/fonts/truetype/cmu/,UprightFont=cmunrm.ttf,BoldFont=cmunbx.ttf,ItalicFont=cmunti.ttf,BoldItalicFont=cmunbi.ttf]{cmunrm.ttf}\setmonofont[Path=/usr/share/fonts/truetype/cmu/,UprightFont=cmuntt.ttf,BoldFont=cmuntb.ttf,ItalicFont=cmunit.ttf,BoldItalicFont=cmuntx.ttf]{cmunrm.ttf} &\cellcolor[rgb]{0.68627,0.19608,0.20784}\hspace*{0pt}\ignorespaces{}\hspace*{0pt} {\mbox{$~$}}\\ \hline \hspace*{0pt}\ignorespaces{}\hspace*{0pt} {\bfseries \setmainfont[Path=/usr/share/fonts/truetype/cmu/,UprightFont=cmunrm.ttf,BoldFont=cmunbx.ttf,ItalicFont=cmunti.ttf,BoldItalicFont=cmunbi.ttf]{cmunbx.ttf}\setmonofont[Path=/usr/share/fonts/truetype/cmu/,UprightFont=cmuntt.ttf,BoldFont=cmuntb.ttf,ItalicFont=cmunit.ttf,BoldItalicFont=cmuntx.ttf]{cmunbx.ttf}\bfseries Melon}{$\text{ }$}\setmainfont[Path=/usr/share/fonts/truetype/cmu/,UprightFont=cmunrm.ttf,BoldFont=cmunbx.ttf,ItalicFont=cmunti.ttf,BoldItalicFont=cmunbi.ttf]{cmunrm.ttf}\setmonofont[Path=/usr/share/fonts/truetype/cmu/,UprightFont=cmuntt.ttf,BoldFont=cmuntb.ttf,ItalicFont=cmunit.ttf,BoldItalicFont=cmuntx.ttf]{cmunrm.ttf} &\cellcolor[rgb]{0.97255,0.61961,0.48235}\hspace*{0pt}\ignorespaces{}\hspace*{0pt} {\mbox{$~$}}&\hspace*{0pt}\ignorespaces{}\hspace*{0pt}{\mbox{$~$}}&\hspace*{0pt}\ignorespaces{}\hspace*{0pt} {\bfseries \setmainfont[Path=/usr/share/fonts/truetype/cmu/,UprightFont=cmunrm.ttf,BoldFont=cmunbx.ttf,ItalicFont=cmunti.ttf,BoldItalicFont=cmunbi.ttf]{cmunbx.ttf}\setmonofont[Path=/usr/share/fonts/truetype/cmu/,UprightFont=cmuntt.ttf,BoldFont=cmuntb.ttf,ItalicFont=cmunit.ttf,BoldItalicFont=cmuntx.ttf]{cmunbx.ttf}\bfseries MidnightBlue}{$\text{ }$}\setmainfont[Path=/usr/share/fonts/truetype/cmu/,UprightFont=cmunrm.ttf,BoldFont=cmunbx.ttf,ItalicFont=cmunti.ttf,BoldItalicFont=cmunbi.ttf]{cmunrm.ttf}\setmonofont[Path=/usr/share/fonts/truetype/cmu/,UprightFont=cmuntt.ttf,BoldFont=cmuntb.ttf,ItalicFont=cmunit.ttf,BoldItalicFont=cmuntx.ttf]{cmunrm.ttf} &\cellcolor[rgb]{0.00000,0.40392,0.58431}\hspace*{0pt}\ignorespaces{}\hspace*{0pt} {\mbox{$~$}}\\ \hline \hspace*{0pt}\ignorespaces{}\hspace*{0pt} {\bfseries \setmainfont[Path=/usr/share/fonts/truetype/cmu/,UprightFont=cmunrm.ttf,BoldFont=cmunbx.ttf,ItalicFont=cmunti.ttf,BoldItalicFont=cmunbi.ttf]{cmunbx.ttf}\setmonofont[Path=/usr/share/fonts/truetype/cmu/,UprightFont=cmuntt.ttf,BoldFont=cmuntb.ttf,ItalicFont=cmunit.ttf,BoldItalicFont=cmuntx.ttf]{cmunbx.ttf}\bfseries Mulberry}{$\text{ }$}\setmainfont[Path=/usr/share/fonts/truetype/cmu/,UprightFont=cmunrm.ttf,BoldFont=cmunbx.ttf,ItalicFont=cmunti.ttf,BoldItalicFont=cmunbi.ttf]{cmunrm.ttf}\setmonofont[Path=/usr/share/fonts/truetype/cmu/,UprightFont=cmuntt.ttf,BoldFont=cmuntb.ttf,ItalicFont=cmunit.ttf,BoldItalicFont=cmuntx.ttf]{cmunrm.ttf} &\cellcolor{A93C93}\hspace*{0pt}\ignorespaces{}\hspace*{0pt} {\mbox{$~$}}&\hspace*{0pt}\ignorespaces{}\hspace*{0pt}{\mbox{$~$}}&\hspace*{0pt}\ignorespaces{}\hspace*{0pt} {\bfseries \setmainfont[Path=/usr/share/fonts/truetype/cmu/,UprightFont=cmunrm.ttf,BoldFont=cmunbx.ttf,ItalicFont=cmunti.ttf,BoldItalicFont=cmunbi.ttf]{cmunbx.ttf}\setmonofont[Path=/usr/share/fonts/truetype/cmu/,UprightFont=cmuntt.ttf,BoldFont=cmuntb.ttf,ItalicFont=cmunit.ttf,BoldItalicFont=cmuntx.ttf]{cmunbx.ttf}\bfseries NavyBlue}{$\text{ }$}\setmainfont[Path=/usr/share/fonts/truetype/cmu/,UprightFont=cmunrm.ttf,BoldFont=cmunbx.ttf,ItalicFont=cmunti.ttf,BoldItalicFont=cmunbi.ttf]{cmunrm.ttf}\setmonofont[Path=/usr/share/fonts/truetype/cmu/,UprightFont=cmuntt.ttf,BoldFont=cmuntb.ttf,ItalicFont=cmunit.ttf,BoldItalicFont=cmuntx.ttf]{cmunrm.ttf} &\cellcolor[rgb]{0.00000,0.43137,0.72157}\hspace*{0pt}\ignorespaces{}\hspace*{0pt} {\mbox{$~$}}\\ \hline \hspace*{0pt}\ignorespaces{}\hspace*{0pt} {\bfseries \setmainfont[Path=/usr/share/fonts/truetype/cmu/,UprightFont=cmunrm.ttf,BoldFont=cmunbx.ttf,ItalicFont=cmunti.ttf,BoldItalicFont=cmunbi.ttf]{cmunbx.ttf}\setmonofont[Path=/usr/share/fonts/truetype/cmu/,UprightFont=cmuntt.ttf,BoldFont=cmuntb.ttf,ItalicFont=cmunit.ttf,BoldItalicFont=cmuntx.ttf]{cmunbx.ttf}\bfseries OliveGreen}{$\text{ }$}\setmainfont[Path=/usr/share/fonts/truetype/cmu/,UprightFont=cmunrm.ttf,BoldFont=cmunbx.ttf,ItalicFont=cmunti.ttf,BoldItalicFont=cmunbi.ttf]{cmunrm.ttf}\setmonofont[Path=/usr/share/fonts/truetype/cmu/,UprightFont=cmuntt.ttf,BoldFont=cmuntb.ttf,ItalicFont=cmunit.ttf,BoldItalicFont=cmuntx.ttf]{cmunrm.ttf} &\cellcolor[rgb]{0.23529,0.50196,0.19216}\hspace*{0pt}\ignorespaces{}\hspace*{0pt} {\mbox{$~$}}&\hspace*{0pt}\ignorespaces{}\hspace*{0pt}{\mbox{$~$}}&\hspace*{0pt}\ignorespaces{}\hspace*{0pt} {\bfseries \setmainfont[Path=/usr/share/fonts/truetype/cmu/,UprightFont=cmunrm.ttf,BoldFont=cmunbx.ttf,ItalicFont=cmunti.ttf,BoldItalicFont=cmunbi.ttf]{cmunbx.ttf}\setmonofont[Path=/usr/share/fonts/truetype/cmu/,UprightFont=cmuntt.ttf,BoldFont=cmuntb.ttf,ItalicFont=cmunit.ttf,BoldItalicFont=cmuntx.ttf]{cmunbx.ttf}\bfseries Orange}{$\text{ }$}\setmainfont[Path=/usr/share/fonts/truetype/cmu/,UprightFont=cmunrm.ttf,BoldFont=cmunbx.ttf,ItalicFont=cmunti.ttf,BoldItalicFont=cmunbi.ttf]{cmunrm.ttf}\setmonofont[Path=/usr/share/fonts/truetype/cmu/,UprightFont=cmuntt.ttf,BoldFont=cmuntb.ttf,ItalicFont=cmunit.ttf,BoldItalicFont=cmuntx.ttf]{cmunrm.ttf} &\cellcolor[rgb]{0.96078,0.50588,0.21569}\hspace*{0pt}\ignorespaces{}\hspace*{0pt} {\mbox{$~$}}\\ \hline \hspace*{0pt}\ignorespaces{}\hspace*{0pt} {\bfseries \setmainfont[Path=/usr/share/fonts/truetype/cmu/,UprightFont=cmunrm.ttf,BoldFont=cmunbx.ttf,ItalicFont=cmunti.ttf,BoldItalicFont=cmunbi.ttf]{cmunbx.ttf}\setmonofont[Path=/usr/share/fonts/truetype/cmu/,UprightFont=cmuntt.ttf,BoldFont=cmuntb.ttf,ItalicFont=cmunit.ttf,BoldItalicFont=cmuntx.ttf]{cmunbx.ttf}\bfseries OrangeRed}{$\text{ }$}\setmainfont[Path=/usr/share/fonts/truetype/cmu/,UprightFont=cmunrm.ttf,BoldFont=cmunbx.ttf,ItalicFont=cmunti.ttf,BoldItalicFont=cmunbi.ttf]{cmunrm.ttf}\setmonofont[Path=/usr/share/fonts/truetype/cmu/,UprightFont=cmuntt.ttf,BoldFont=cmuntb.ttf,ItalicFont=cmunit.ttf,BoldItalicFont=cmuntx.ttf]{cmunrm.ttf} &\cellcolor[rgb]{0.92941,0.07451,0.35294}\hspace*{0pt}\ignorespaces{}\hspace*{0pt} {\mbox{$~$}}&\hspace*{0pt}\ignorespaces{}\hspace*{0pt}{\mbox{$~$}}&\hspace*{0pt}\ignorespaces{}\hspace*{0pt} {\bfseries \setmainfont[Path=/usr/share/fonts/truetype/cmu/,UprightFont=cmunrm.ttf,BoldFont=cmunbx.ttf,ItalicFont=cmunti.ttf,BoldItalicFont=cmunbi.ttf]{cmunbx.ttf}\setmonofont[Path=/usr/share/fonts/truetype/cmu/,UprightFont=cmuntt.ttf,BoldFont=cmuntb.ttf,ItalicFont=cmunit.ttf,BoldItalicFont=cmuntx.ttf]{cmunbx.ttf}\bfseries Orchid}{$\text{ }$}\setmainfont[Path=/usr/share/fonts/truetype/cmu/,UprightFont=cmunrm.ttf,BoldFont=cmunbx.ttf,ItalicFont=cmunti.ttf,BoldItalicFont=cmunbi.ttf]{cmunrm.ttf}\setmonofont[Path=/usr/share/fonts/truetype/cmu/,UprightFont=cmuntt.ttf,BoldFont=cmuntb.ttf,ItalicFont=cmunit.ttf,BoldItalicFont=cmuntx.ttf]{cmunrm.ttf} &\cellcolor[rgb]{0.68627,0.44706,0.69020}\hspace*{0pt}\ignorespaces{}\hspace*{0pt} {\mbox{$~$}}\\ \hline \hspace*{0pt}\ignorespaces{}\hspace*{0pt} {\bfseries \setmainfont[Path=/usr/share/fonts/truetype/cmu/,UprightFont=cmunrm.ttf,BoldFont=cmunbx.ttf,ItalicFont=cmunti.ttf,BoldItalicFont=cmunbi.ttf]{cmunbx.ttf}\setmonofont[Path=/usr/share/fonts/truetype/cmu/,UprightFont=cmuntt.ttf,BoldFont=cmuntb.ttf,ItalicFont=cmunit.ttf,BoldItalicFont=cmuntx.ttf]{cmunbx.ttf}\bfseries Peach}{$\text{ }$}\setmainfont[Path=/usr/share/fonts/truetype/cmu/,UprightFont=cmunrm.ttf,BoldFont=cmunbx.ttf,ItalicFont=cmunti.ttf,BoldItalicFont=cmunbi.ttf]{cmunrm.ttf}\setmonofont[Path=/usr/share/fonts/truetype/cmu/,UprightFont=cmuntt.ttf,BoldFont=cmuntb.ttf,ItalicFont=cmunit.ttf,BoldItalicFont=cmuntx.ttf]{cmunrm.ttf} &\cellcolor[rgb]{0.96863,0.58824,0.35294}\hspace*{0pt}\ignorespaces{}\hspace*{0pt} {\mbox{$~$}}&\hspace*{0pt}\ignorespaces{}\hspace*{0pt}{\mbox{$~$}}&\hspace*{0pt}\ignorespaces{}\hspace*{0pt}  {\bfseries \setmainfont[Path=/usr/share/fonts/truetype/cmu/,UprightFont=cmunrm.ttf,BoldFont=cmunbx.ttf,ItalicFont=cmunti.ttf,BoldItalicFont=cmunbi.ttf]{cmunbx.ttf}\setmonofont[Path=/usr/share/fonts/truetype/cmu/,UprightFont=cmuntt.ttf,BoldFont=cmuntb.ttf,ItalicFont=cmunit.ttf,BoldItalicFont=cmuntx.ttf]{cmunbx.ttf}\bfseries Periwinkle}&\cellcolor[rgb]{0.47451,0.46667,0.72157}\hspace*{0pt}\ignorespaces{}\hspace*{0pt}{$\text{ }$}\setmainfont[Path=/usr/share/fonts/truetype/cmu/,UprightFont=cmunrm.ttf,BoldFont=cmunbx.ttf,ItalicFont=cmunti.ttf,BoldItalicFont=cmunbi.ttf]{cmunrm.ttf}\setmonofont[Path=/usr/share/fonts/truetype/cmu/,UprightFont=cmuntt.ttf,BoldFont=cmuntb.ttf,ItalicFont=cmunit.ttf,BoldItalicFont=cmuntx.ttf]{cmunrm.ttf} {\mbox{$~$}}\\ \hline \hspace*{0pt}\ignorespaces{}\hspace*{0pt} {\bfseries \setmainfont[Path=/usr/share/fonts/truetype/cmu/,UprightFont=cmunrm.ttf,BoldFont=cmunbx.ttf,ItalicFont=cmunti.ttf,BoldItalicFont=cmunbi.ttf]{cmunbx.ttf}\setmonofont[Path=/usr/share/fonts/truetype/cmu/,UprightFont=cmuntt.ttf,BoldFont=cmuntb.ttf,ItalicFont=cmunit.ttf,BoldItalicFont=cmuntx.ttf]{cmunbx.ttf}\bfseries PineGreen}{$\text{ }$}\setmainfont[Path=/usr/share/fonts/truetype/cmu/,UprightFont=cmunrm.ttf,BoldFont=cmunbx.ttf,ItalicFont=cmunti.ttf,BoldItalicFont=cmunbi.ttf]{cmunrm.ttf}\setmonofont[Path=/usr/share/fonts/truetype/cmu/,UprightFont=cmuntt.ttf,BoldFont=cmuntb.ttf,ItalicFont=cmunit.ttf,BoldItalicFont=cmuntx.ttf]{cmunrm.ttf} &\cellcolor[rgb]{0.00000,0.54510,0.44706}\hspace*{0pt}\ignorespaces{}\hspace*{0pt} {\mbox{$~$}}&\hspace*{0pt}\ignorespaces{}\hspace*{0pt}{\mbox{$~$}}&\hspace*{0pt}\ignorespaces{}\hspace*{0pt} {\bfseries \setmainfont[Path=/usr/share/fonts/truetype/cmu/,UprightFont=cmunrm.ttf,BoldFont=cmunbx.ttf,ItalicFont=cmunti.ttf,BoldItalicFont=cmunbi.ttf]{cmunbx.ttf}\setmonofont[Path=/usr/share/fonts/truetype/cmu/,UprightFont=cmuntt.ttf,BoldFont=cmuntb.ttf,ItalicFont=cmunit.ttf,BoldItalicFont=cmuntx.ttf]{cmunbx.ttf}\bfseries Plum}{$\text{ }$}\setmainfont[Path=/usr/share/fonts/truetype/cmu/,UprightFont=cmunrm.ttf,BoldFont=cmunbx.ttf,ItalicFont=cmunti.ttf,BoldItalicFont=cmunbi.ttf]{cmunrm.ttf}\setmonofont[Path=/usr/share/fonts/truetype/cmu/,UprightFont=cmuntt.ttf,BoldFont=cmuntb.ttf,ItalicFont=cmunit.ttf,BoldItalicFont=cmuntx.ttf]{cmunrm.ttf} &\cellcolor[rgb]{0.57255,0.14902,0.56078}\hspace*{0pt}\ignorespaces{}\hspace*{0pt} {\mbox{$~$}}\\ \hline \hspace*{0pt}\ignorespaces{}\hspace*{0pt} {\bfseries \setmainfont[Path=/usr/share/fonts/truetype/cmu/,UprightFont=cmunrm.ttf,BoldFont=cmunbx.ttf,ItalicFont=cmunti.ttf,BoldItalicFont=cmunbi.ttf]{cmunbx.ttf}\setmonofont[Path=/usr/share/fonts/truetype/cmu/,UprightFont=cmuntt.ttf,BoldFont=cmuntb.ttf,ItalicFont=cmunit.ttf,BoldItalicFont=cmuntx.ttf]{cmunbx.ttf}\bfseries ProcessBlue}{$\text{ }$}\setmainfont[Path=/usr/share/fonts/truetype/cmu/,UprightFont=cmunrm.ttf,BoldFont=cmunbx.ttf,ItalicFont=cmunti.ttf,BoldItalicFont=cmunbi.ttf]{cmunrm.ttf}\setmonofont[Path=/usr/share/fonts/truetype/cmu/,UprightFont=cmuntt.ttf,BoldFont=cmuntb.ttf,ItalicFont=cmunit.ttf,BoldItalicFont=cmuntx.ttf]{cmunrm.ttf} &\cellcolor[rgb]{0.00000,0.69020,0.94118}\hspace*{0pt}\ignorespaces{}\hspace*{0pt} {\mbox{$~$}}&\hspace*{0pt}\ignorespaces{}\hspace*{0pt}{\mbox{$~$}}&\hspace*{0pt}\ignorespaces{}\hspace*{0pt} {\bfseries \setmainfont[Path=/usr/share/fonts/truetype/cmu/,UprightFont=cmunrm.ttf,BoldFont=cmunbx.ttf,ItalicFont=cmunti.ttf,BoldItalicFont=cmunbi.ttf]{cmunbx.ttf}\setmonofont[Path=/usr/share/fonts/truetype/cmu/,UprightFont=cmuntt.ttf,BoldFont=cmuntb.ttf,ItalicFont=cmunit.ttf,BoldItalicFont=cmuntx.ttf]{cmunbx.ttf}\bfseries Purple}{$\text{ }$}\setmainfont[Path=/usr/share/fonts/truetype/cmu/,UprightFont=cmunrm.ttf,BoldFont=cmunbx.ttf,ItalicFont=cmunti.ttf,BoldItalicFont=cmunbi.ttf]{cmunrm.ttf}\setmonofont[Path=/usr/share/fonts/truetype/cmu/,UprightFont=cmuntt.ttf,BoldFont=cmuntb.ttf,ItalicFont=cmunit.ttf,BoldItalicFont=cmuntx.ttf]{cmunrm.ttf} &\cellcolor[rgb]{0.60000,0.27843,0.60784}\hspace*{0pt}\ignorespaces{}\hspace*{0pt} {\mbox{$~$}}\\ \hline \hspace*{0pt}\ignorespaces{}\hspace*{0pt} {\bfseries \setmainfont[Path=/usr/share/fonts/truetype/cmu/,UprightFont=cmunrm.ttf,BoldFont=cmunbx.ttf,ItalicFont=cmunti.ttf,BoldItalicFont=cmunbi.ttf]{cmunbx.ttf}\setmonofont[Path=/usr/share/fonts/truetype/cmu/,UprightFont=cmuntt.ttf,BoldFont=cmuntb.ttf,ItalicFont=cmunit.ttf,BoldItalicFont=cmuntx.ttf]{cmunbx.ttf}\bfseries RawSienna}{$\text{ }$}\setmainfont[Path=/usr/share/fonts/truetype/cmu/,UprightFont=cmunrm.ttf,BoldFont=cmunbx.ttf,ItalicFont=cmunti.ttf,BoldItalicFont=cmunbi.ttf]{cmunrm.ttf}\setmonofont[Path=/usr/share/fonts/truetype/cmu/,UprightFont=cmuntt.ttf,BoldFont=cmuntb.ttf,ItalicFont=cmunit.ttf,BoldItalicFont=cmuntx.ttf]{cmunrm.ttf} &\cellcolor[rgb]{0.59216,0.25098,0.02353}\hspace*{0pt}\ignorespaces{}\hspace*{0pt} {\mbox{$~$}}&\hspace*{0pt}\ignorespaces{}\hspace*{0pt}{\mbox{$~$}}&\hspace*{0pt}\ignorespaces{}\hspace*{0pt} {\bfseries \setmainfont[Path=/usr/share/fonts/truetype/cmu/,UprightFont=cmunrm.ttf,BoldFont=cmunbx.ttf,ItalicFont=cmunti.ttf,BoldItalicFont=cmunbi.ttf]{cmunbx.ttf}\setmonofont[Path=/usr/share/fonts/truetype/cmu/,UprightFont=cmuntt.ttf,BoldFont=cmuntb.ttf,ItalicFont=cmunit.ttf,BoldItalicFont=cmuntx.ttf]{cmunbx.ttf}\bfseries Red}{$\text{ }$}\setmainfont[Path=/usr/share/fonts/truetype/cmu/,UprightFont=cmunrm.ttf,BoldFont=cmunbx.ttf,ItalicFont=cmunti.ttf,BoldItalicFont=cmunbi.ttf]{cmunrm.ttf}\setmonofont[Path=/usr/share/fonts/truetype/cmu/,UprightFont=cmuntt.ttf,BoldFont=cmuntb.ttf,ItalicFont=cmunit.ttf,BoldItalicFont=cmuntx.ttf]{cmunrm.ttf} &\cellcolor[rgb]{0.92941,0.10588,0.13725}\hspace*{0pt}\ignorespaces{}\hspace*{0pt} {\mbox{$~$}}\\ \hline \hspace*{0pt}\ignorespaces{}\hspace*{0pt} {\bfseries \setmainfont[Path=/usr/share/fonts/truetype/cmu/,UprightFont=cmunrm.ttf,BoldFont=cmunbx.ttf,ItalicFont=cmunti.ttf,BoldItalicFont=cmunbi.ttf]{cmunbx.ttf}\setmonofont[Path=/usr/share/fonts/truetype/cmu/,UprightFont=cmuntt.ttf,BoldFont=cmuntb.ttf,ItalicFont=cmunit.ttf,BoldItalicFont=cmuntx.ttf]{cmunbx.ttf}\bfseries RedOrange}{$\text{ }$}\setmainfont[Path=/usr/share/fonts/truetype/cmu/,UprightFont=cmunrm.ttf,BoldFont=cmunbx.ttf,ItalicFont=cmunti.ttf,BoldItalicFont=cmunbi.ttf]{cmunrm.ttf}\setmonofont[Path=/usr/share/fonts/truetype/cmu/,UprightFont=cmuntt.ttf,BoldFont=cmuntb.ttf,ItalicFont=cmunit.ttf,BoldItalicFont=cmuntx.ttf]{cmunrm.ttf} &\cellcolor[rgb]{0.94902,0.37647,0.20784}\hspace*{0pt}\ignorespaces{}\hspace*{0pt} {\mbox{$~$}}&\hspace*{0pt}\ignorespaces{}\hspace*{0pt}{\mbox{$~$}}&\hspace*{0pt}\ignorespaces{}\hspace*{0pt} {\bfseries \setmainfont[Path=/usr/share/fonts/truetype/cmu/,UprightFont=cmunrm.ttf,BoldFont=cmunbx.ttf,ItalicFont=cmunti.ttf,BoldItalicFont=cmunbi.ttf]{cmunbx.ttf}\setmonofont[Path=/usr/share/fonts/truetype/cmu/,UprightFont=cmuntt.ttf,BoldFont=cmuntb.ttf,ItalicFont=cmunit.ttf,BoldItalicFont=cmuntx.ttf]{cmunbx.ttf}\bfseries RedViolet}{$\text{ }$}\setmainfont[Path=/usr/share/fonts/truetype/cmu/,UprightFont=cmunrm.ttf,BoldFont=cmunbx.ttf,ItalicFont=cmunti.ttf,BoldItalicFont=cmunbi.ttf]{cmunrm.ttf}\setmonofont[Path=/usr/share/fonts/truetype/cmu/,UprightFont=cmuntt.ttf,BoldFont=cmuntb.ttf,ItalicFont=cmunit.ttf,BoldItalicFont=cmuntx.ttf]{cmunrm.ttf} &\cellcolor[rgb]{0.63137,0.14118,0.41961}\hspace*{0pt}\ignorespaces{}\hspace*{0pt} {\mbox{$~$}}\\ \hline \hspace*{0pt}\ignorespaces{}\hspace*{0pt} {\bfseries \setmainfont[Path=/usr/share/fonts/truetype/cmu/,UprightFont=cmunrm.ttf,BoldFont=cmunbx.ttf,ItalicFont=cmunti.ttf,BoldItalicFont=cmunbi.ttf]{cmunbx.ttf}\setmonofont[Path=/usr/share/fonts/truetype/cmu/,UprightFont=cmuntt.ttf,BoldFont=cmuntb.ttf,ItalicFont=cmunit.ttf,BoldItalicFont=cmuntx.ttf]{cmunbx.ttf}\bfseries Rhodamine}{$\text{ }$}\setmainfont[Path=/usr/share/fonts/truetype/cmu/,UprightFont=cmunrm.ttf,BoldFont=cmunbx.ttf,ItalicFont=cmunti.ttf,BoldItalicFont=cmunbi.ttf]{cmunrm.ttf}\setmonofont[Path=/usr/share/fonts/truetype/cmu/,UprightFont=cmuntt.ttf,BoldFont=cmuntb.ttf,ItalicFont=cmunit.ttf,BoldItalicFont=cmuntx.ttf]{cmunrm.ttf} &\cellcolor[rgb]{0.93725,0.33333,0.62353}\hspace*{0pt}\ignorespaces{}\hspace*{0pt} {\mbox{$~$}}&\hspace*{0pt}\ignorespaces{}\hspace*{0pt}{\mbox{$~$}}&\hspace*{0pt}\ignorespaces{}\hspace*{0pt} {\bfseries \setmainfont[Path=/usr/share/fonts/truetype/cmu/,UprightFont=cmunrm.ttf,BoldFont=cmunbx.ttf,ItalicFont=cmunti.ttf,BoldItalicFont=cmunbi.ttf]{cmunbx.ttf}\setmonofont[Path=/usr/share/fonts/truetype/cmu/,UprightFont=cmuntt.ttf,BoldFont=cmuntb.ttf,ItalicFont=cmunit.ttf,BoldItalicFont=cmuntx.ttf]{cmunbx.ttf}\bfseries RoyalBlue}{$\text{ }$}\setmainfont[Path=/usr/share/fonts/truetype/cmu/,UprightFont=cmunrm.ttf,BoldFont=cmunbx.ttf,ItalicFont=cmunti.ttf,BoldItalicFont=cmunbi.ttf]{cmunrm.ttf}\setmonofont[Path=/usr/share/fonts/truetype/cmu/,UprightFont=cmuntt.ttf,BoldFont=cmuntb.ttf,ItalicFont=cmunit.ttf,BoldItalicFont=cmuntx.ttf]{cmunrm.ttf} &\cellcolor[rgb]{0.00000,0.44314,0.73725}\hspace*{0pt}\ignorespaces{}\hspace*{0pt} {\mbox{$~$}}\\ \hline \hspace*{0pt}\ignorespaces{}\hspace*{0pt} {\bfseries \setmainfont[Path=/usr/share/fonts/truetype/cmu/,UprightFont=cmunrm.ttf,BoldFont=cmunbx.ttf,ItalicFont=cmunti.ttf,BoldItalicFont=cmunbi.ttf]{cmunbx.ttf}\setmonofont[Path=/usr/share/fonts/truetype/cmu/,UprightFont=cmuntt.ttf,BoldFont=cmuntb.ttf,ItalicFont=cmunit.ttf,BoldItalicFont=cmuntx.ttf]{cmunbx.ttf}\bfseries RoyalPurple}{$\text{ }$}\setmainfont[Path=/usr/share/fonts/truetype/cmu/,UprightFont=cmunrm.ttf,BoldFont=cmunbx.ttf,ItalicFont=cmunti.ttf,BoldItalicFont=cmunbi.ttf]{cmunrm.ttf}\setmonofont[Path=/usr/share/fonts/truetype/cmu/,UprightFont=cmuntt.ttf,BoldFont=cmuntb.ttf,ItalicFont=cmunit.ttf,BoldItalicFont=cmuntx.ttf]{cmunrm.ttf} &\cellcolor[rgb]{0.38039,0.24706,0.60000}\hspace*{0pt}\ignorespaces{}\hspace*{0pt} {\mbox{$~$}}&\hspace*{0pt}\ignorespaces{}\hspace*{0pt}{\mbox{$~$}}&\hspace*{0pt}\ignorespaces{}\hspace*{0pt} {\bfseries \setmainfont[Path=/usr/share/fonts/truetype/cmu/,UprightFont=cmunrm.ttf,BoldFont=cmunbx.ttf,ItalicFont=cmunti.ttf,BoldItalicFont=cmunbi.ttf]{cmunbx.ttf}\setmonofont[Path=/usr/share/fonts/truetype/cmu/,UprightFont=cmuntt.ttf,BoldFont=cmuntb.ttf,ItalicFont=cmunit.ttf,BoldItalicFont=cmuntx.ttf]{cmunbx.ttf}\bfseries RubineRed}{$\text{ }$}\setmainfont[Path=/usr/share/fonts/truetype/cmu/,UprightFont=cmunrm.ttf,BoldFont=cmunbx.ttf,ItalicFont=cmunti.ttf,BoldItalicFont=cmunbi.ttf]{cmunrm.ttf}\setmonofont[Path=/usr/share/fonts/truetype/cmu/,UprightFont=cmuntt.ttf,BoldFont=cmuntb.ttf,ItalicFont=cmunit.ttf,BoldItalicFont=cmuntx.ttf]{cmunrm.ttf} &\cellcolor[rgb]{0.92941,0.00392,0.49020}\hspace*{0pt}\ignorespaces{}\hspace*{0pt} {\mbox{$~$}}\\ \hline \hspace*{0pt}\ignorespaces{}\hspace*{0pt} {\bfseries \setmainfont[Path=/usr/share/fonts/truetype/cmu/,UprightFont=cmunrm.ttf,BoldFont=cmunbx.ttf,ItalicFont=cmunti.ttf,BoldItalicFont=cmunbi.ttf]{cmunbx.ttf}\setmonofont[Path=/usr/share/fonts/truetype/cmu/,UprightFont=cmuntt.ttf,BoldFont=cmuntb.ttf,ItalicFont=cmunit.ttf,BoldItalicFont=cmuntx.ttf]{cmunbx.ttf}\bfseries Salmon}{$\text{ }$}\setmainfont[Path=/usr/share/fonts/truetype/cmu/,UprightFont=cmunrm.ttf,BoldFont=cmunbx.ttf,ItalicFont=cmunti.ttf,BoldItalicFont=cmunbi.ttf]{cmunrm.ttf}\setmonofont[Path=/usr/share/fonts/truetype/cmu/,UprightFont=cmuntt.ttf,BoldFont=cmuntb.ttf,ItalicFont=cmunit.ttf,BoldItalicFont=cmuntx.ttf]{cmunrm.ttf} &\cellcolor[rgb]{0.96471,0.57255,0.53725}\hspace*{0pt}\ignorespaces{}\hspace*{0pt} {\mbox{$~$}}&\hspace*{0pt}\ignorespaces{}\hspace*{0pt}{\mbox{$~$}}&\hspace*{0pt}\ignorespaces{}\hspace*{0pt} {\bfseries \setmainfont[Path=/usr/share/fonts/truetype/cmu/,UprightFont=cmunrm.ttf,BoldFont=cmunbx.ttf,ItalicFont=cmunti.ttf,BoldItalicFont=cmunbi.ttf]{cmunbx.ttf}\setmonofont[Path=/usr/share/fonts/truetype/cmu/,UprightFont=cmuntt.ttf,BoldFont=cmuntb.ttf,ItalicFont=cmunit.ttf,BoldItalicFont=cmuntx.ttf]{cmunbx.ttf}\bfseries SeaGreen}{$\text{ }$}\setmainfont[Path=/usr/share/fonts/truetype/cmu/,UprightFont=cmunrm.ttf,BoldFont=cmunbx.ttf,ItalicFont=cmunti.ttf,BoldItalicFont=cmunbi.ttf]{cmunrm.ttf}\setmonofont[Path=/usr/share/fonts/truetype/cmu/,UprightFont=cmuntt.ttf,BoldFont=cmuntb.ttf,ItalicFont=cmunit.ttf,BoldItalicFont=cmuntx.ttf]{cmunrm.ttf} &\cellcolor[rgb]{0.24706,0.73725,0.61569}\hspace*{0pt}\ignorespaces{}\hspace*{0pt} {\mbox{$~$}}\\ \hline \hspace*{0pt}\ignorespaces{}\hspace*{0pt} {\bfseries \setmainfont[Path=/usr/share/fonts/truetype/cmu/,UprightFont=cmunrm.ttf,BoldFont=cmunbx.ttf,ItalicFont=cmunti.ttf,BoldItalicFont=cmunbi.ttf]{cmunbx.ttf}\setmonofont[Path=/usr/share/fonts/truetype/cmu/,UprightFont=cmuntt.ttf,BoldFont=cmuntb.ttf,ItalicFont=cmunit.ttf,BoldItalicFont=cmuntx.ttf]{cmunbx.ttf}\bfseries Sepia}{$\text{ }$}\setmainfont[Path=/usr/share/fonts/truetype/cmu/,UprightFont=cmunrm.ttf,BoldFont=cmunbx.ttf,ItalicFont=cmunti.ttf,BoldItalicFont=cmunbi.ttf]{cmunrm.ttf}\setmonofont[Path=/usr/share/fonts/truetype/cmu/,UprightFont=cmuntt.ttf,BoldFont=cmuntb.ttf,ItalicFont=cmunit.ttf,BoldItalicFont=cmuntx.ttf]{cmunrm.ttf} &\cellcolor[rgb]{0.40392,0.09412,0.00000}\hspace*{0pt}\ignorespaces{}\hspace*{0pt} {\mbox{$~$}}&\hspace*{0pt}\ignorespaces{}\hspace*{0pt}{\mbox{$~$}}&\hspace*{0pt}\ignorespaces{}\hspace*{0pt} {\bfseries \setmainfont[Path=/usr/share/fonts/truetype/cmu/,UprightFont=cmunrm.ttf,BoldFont=cmunbx.ttf,ItalicFont=cmunti.ttf,BoldItalicFont=cmunbi.ttf]{cmunbx.ttf}\setmonofont[Path=/usr/share/fonts/truetype/cmu/,UprightFont=cmuntt.ttf,BoldFont=cmuntb.ttf,ItalicFont=cmunit.ttf,BoldItalicFont=cmuntx.ttf]{cmunbx.ttf}\bfseries SkyBlue}{$\text{ }$}\setmainfont[Path=/usr/share/fonts/truetype/cmu/,UprightFont=cmunrm.ttf,BoldFont=cmunbx.ttf,ItalicFont=cmunti.ttf,BoldItalicFont=cmunbi.ttf]{cmunrm.ttf}\setmonofont[Path=/usr/share/fonts/truetype/cmu/,UprightFont=cmuntt.ttf,BoldFont=cmuntb.ttf,ItalicFont=cmunit.ttf,BoldItalicFont=cmuntx.ttf]{cmunrm.ttf} &\cellcolor[rgb]{0.27451,0.77255,0.86667}\hspace*{0pt}\ignorespaces{}\hspace*{0pt} {\mbox{$~$}}\\ \hline \hspace*{0pt}\ignorespaces{}\hspace*{0pt} {\bfseries \setmainfont[Path=/usr/share/fonts/truetype/cmu/,UprightFont=cmunrm.ttf,BoldFont=cmunbx.ttf,ItalicFont=cmunti.ttf,BoldItalicFont=cmunbi.ttf]{cmunbx.ttf}\setmonofont[Path=/usr/share/fonts/truetype/cmu/,UprightFont=cmuntt.ttf,BoldFont=cmuntb.ttf,ItalicFont=cmunit.ttf,BoldItalicFont=cmuntx.ttf]{cmunbx.ttf}\bfseries SpringGreen}{$\text{ }$}\setmainfont[Path=/usr/share/fonts/truetype/cmu/,UprightFont=cmunrm.ttf,BoldFont=cmunbx.ttf,ItalicFont=cmunti.ttf,BoldItalicFont=cmunbi.ttf]{cmunrm.ttf}\setmonofont[Path=/usr/share/fonts/truetype/cmu/,UprightFont=cmuntt.ttf,BoldFont=cmuntb.ttf,ItalicFont=cmunit.ttf,BoldItalicFont=cmuntx.ttf]{cmunrm.ttf} &\cellcolor[rgb]{0.77647,0.86275,0.40392}\hspace*{0pt}\ignorespaces{}\hspace*{0pt} {\mbox{$~$}}&\hspace*{0pt}\ignorespaces{}\hspace*{0pt}{\mbox{$~$}}&\hspace*{0pt}\ignorespaces{}\hspace*{0pt} {\bfseries \setmainfont[Path=/usr/share/fonts/truetype/cmu/,UprightFont=cmunrm.ttf,BoldFont=cmunbx.ttf,ItalicFont=cmunti.ttf,BoldItalicFont=cmunbi.ttf]{cmunbx.ttf}\setmonofont[Path=/usr/share/fonts/truetype/cmu/,UprightFont=cmuntt.ttf,BoldFont=cmuntb.ttf,ItalicFont=cmunit.ttf,BoldItalicFont=cmuntx.ttf]{cmunbx.ttf}\bfseries Tan}{$\text{ }$}\setmainfont[Path=/usr/share/fonts/truetype/cmu/,UprightFont=cmunrm.ttf,BoldFont=cmunbx.ttf,ItalicFont=cmunti.ttf,BoldItalicFont=cmunbi.ttf]{cmunrm.ttf}\setmonofont[Path=/usr/share/fonts/truetype/cmu/,UprightFont=cmuntt.ttf,BoldFont=cmuntb.ttf,ItalicFont=cmunit.ttf,BoldItalicFont=cmuntx.ttf]{cmunrm.ttf} &\cellcolor[rgb]{0.85490,0.61569,0.46275}\hspace*{0pt}\ignorespaces{}\hspace*{0pt} {\mbox{$~$}}\\ \hline \hspace*{0pt}\ignorespaces{}\hspace*{0pt} {\bfseries \setmainfont[Path=/usr/share/fonts/truetype/cmu/,UprightFont=cmunrm.ttf,BoldFont=cmunbx.ttf,ItalicFont=cmunti.ttf,BoldItalicFont=cmunbi.ttf]{cmunbx.ttf}\setmonofont[Path=/usr/share/fonts/truetype/cmu/,UprightFont=cmuntt.ttf,BoldFont=cmuntb.ttf,ItalicFont=cmunit.ttf,BoldItalicFont=cmuntx.ttf]{cmunbx.ttf}\bfseries TealBlue}{$\text{ }$}\setmainfont[Path=/usr/share/fonts/truetype/cmu/,UprightFont=cmunrm.ttf,BoldFont=cmunbx.ttf,ItalicFont=cmunti.ttf,BoldItalicFont=cmunbi.ttf]{cmunrm.ttf}\setmonofont[Path=/usr/share/fonts/truetype/cmu/,UprightFont=cmuntt.ttf,BoldFont=cmuntb.ttf,ItalicFont=cmunit.ttf,BoldItalicFont=cmuntx.ttf]{cmunrm.ttf} &\cellcolor[rgb]{0.00000,0.68235,0.70196}\hspace*{0pt}\ignorespaces{}\hspace*{0pt} {\mbox{$~$}}&\hspace*{0pt}\ignorespaces{}\hspace*{0pt}{\mbox{$~$}}&\hspace*{0pt}\ignorespaces{}\hspace*{0pt} {\bfseries \setmainfont[Path=/usr/share/fonts/truetype/cmu/,UprightFont=cmunrm.ttf,BoldFont=cmunbx.ttf,ItalicFont=cmunti.ttf,BoldItalicFont=cmunbi.ttf]{cmunbx.ttf}\setmonofont[Path=/usr/share/fonts/truetype/cmu/,UprightFont=cmuntt.ttf,BoldFont=cmuntb.ttf,ItalicFont=cmunit.ttf,BoldItalicFont=cmuntx.ttf]{cmunbx.ttf}\bfseries Thistle}{$\text{ }$}\setmainfont[Path=/usr/share/fonts/truetype/cmu/,UprightFont=cmunrm.ttf,BoldFont=cmunbx.ttf,ItalicFont=cmunti.ttf,BoldItalicFont=cmunbi.ttf]{cmunrm.ttf}\setmonofont[Path=/usr/share/fonts/truetype/cmu/,UprightFont=cmuntt.ttf,BoldFont=cmuntb.ttf,ItalicFont=cmunit.ttf,BoldItalicFont=cmuntx.ttf]{cmunrm.ttf} &\cellcolor[rgb]{0.84706,0.51373,0.71765}\hspace*{0pt}\ignorespaces{}\hspace*{0pt} {\mbox{$~$}}\\ \hline \hspace*{0pt}\ignorespaces{}\hspace*{0pt} {\bfseries \setmainfont[Path=/usr/share/fonts/truetype/cmu/,UprightFont=cmunrm.ttf,BoldFont=cmunbx.ttf,ItalicFont=cmunti.ttf,BoldItalicFont=cmunbi.ttf]{cmunbx.ttf}\setmonofont[Path=/usr/share/fonts/truetype/cmu/,UprightFont=cmuntt.ttf,BoldFont=cmuntb.ttf,ItalicFont=cmunit.ttf,BoldItalicFont=cmuntx.ttf]{cmunbx.ttf}\bfseries Turquoise}{$\text{ }$}\setmainfont[Path=/usr/share/fonts/truetype/cmu/,UprightFont=cmunrm.ttf,BoldFont=cmunbx.ttf,ItalicFont=cmunti.ttf,BoldItalicFont=cmunbi.ttf]{cmunrm.ttf}\setmonofont[Path=/usr/share/fonts/truetype/cmu/,UprightFont=cmuntt.ttf,BoldFont=cmuntb.ttf,ItalicFont=cmunit.ttf,BoldItalicFont=cmuntx.ttf]{cmunrm.ttf} &\cellcolor[rgb]{0.00000,0.70588,0.80784}\hspace*{0pt}\ignorespaces{}\hspace*{0pt} {\mbox{$~$}}&\hspace*{0pt}\ignorespaces{}\hspace*{0pt}{\mbox{$~$}}&\hspace*{0pt}\ignorespaces{}\hspace*{0pt} {\bfseries \setmainfont[Path=/usr/share/fonts/truetype/cmu/,UprightFont=cmunrm.ttf,BoldFont=cmunbx.ttf,ItalicFont=cmunti.ttf,BoldItalicFont=cmunbi.ttf]{cmunbx.ttf}\setmonofont[Path=/usr/share/fonts/truetype/cmu/,UprightFont=cmuntt.ttf,BoldFont=cmuntb.ttf,ItalicFont=cmunit.ttf,BoldItalicFont=cmuntx.ttf]{cmunbx.ttf}\bfseries Violet}{$\text{ }$}\setmainfont[Path=/usr/share/fonts/truetype/cmu/,UprightFont=cmunrm.ttf,BoldFont=cmunbx.ttf,ItalicFont=cmunti.ttf,BoldItalicFont=cmunbi.ttf]{cmunrm.ttf}\setmonofont[Path=/usr/share/fonts/truetype/cmu/,UprightFont=cmuntt.ttf,BoldFont=cmuntb.ttf,ItalicFont=cmunit.ttf,BoldItalicFont=cmuntx.ttf]{cmunrm.ttf} &\cellcolor[rgb]{0.34510,0.25882,0.60784}\hspace*{0pt}\ignorespaces{}\hspace*{0pt} {\mbox{$~$}}\\ \hline \hspace*{0pt}\ignorespaces{}\hspace*{0pt} {\bfseries \setmainfont[Path=/usr/share/fonts/truetype/cmu/,UprightFont=cmunrm.ttf,BoldFont=cmunbx.ttf,ItalicFont=cmunti.ttf,BoldItalicFont=cmunbi.ttf]{cmunbx.ttf}\setmonofont[Path=/usr/share/fonts/truetype/cmu/,UprightFont=cmuntt.ttf,BoldFont=cmuntb.ttf,ItalicFont=cmunit.ttf,BoldItalicFont=cmuntx.ttf]{cmunbx.ttf}\bfseries VioletRed}{$\text{ }$}\setmainfont[Path=/usr/share/fonts/truetype/cmu/,UprightFont=cmunrm.ttf,BoldFont=cmunbx.ttf,ItalicFont=cmunti.ttf,BoldItalicFont=cmunbi.ttf]{cmunrm.ttf}\setmonofont[Path=/usr/share/fonts/truetype/cmu/,UprightFont=cmuntt.ttf,BoldFont=cmuntb.ttf,ItalicFont=cmunit.ttf,BoldItalicFont=cmuntx.ttf]{cmunrm.ttf} &\cellcolor[rgb]{0.93725,0.34510,0.62745}\hspace*{0pt}\ignorespaces{}\hspace*{0pt} {\mbox{$~$}}&\hspace*{0pt}\ignorespaces{}\hspace*{0pt}{\mbox{$~$}}&\hspace*{0pt}\ignorespaces{}\hspace*{0pt} {\bfseries \setmainfont[Path=/usr/share/fonts/truetype/cmu/,UprightFont=cmunrm.ttf,BoldFont=cmunbx.ttf,ItalicFont=cmunti.ttf,BoldItalicFont=cmunbi.ttf]{cmunbx.ttf}\setmonofont[Path=/usr/share/fonts/truetype/cmu/,UprightFont=cmuntt.ttf,BoldFont=cmuntb.ttf,ItalicFont=cmunit.ttf,BoldItalicFont=cmuntx.ttf]{cmunbx.ttf}\bfseries White}{$\text{ }$}\setmainfont[Path=/usr/share/fonts/truetype/cmu/,UprightFont=cmunrm.ttf,BoldFont=cmunbx.ttf,ItalicFont=cmunti.ttf,BoldItalicFont=cmunbi.ttf]{cmunrm.ttf}\setmonofont[Path=/usr/share/fonts/truetype/cmu/,UprightFont=cmuntt.ttf,BoldFont=cmuntb.ttf,ItalicFont=cmunit.ttf,BoldItalicFont=cmuntx.ttf]{cmunrm.ttf} &\cellcolor{}\hspace*{0pt}\ignorespaces{}\hspace*{0pt} {\mbox{$~$}}\\ \hline \hspace*{0pt}\ignorespaces{}\hspace*{0pt} {\bfseries \setmainfont[Path=/usr/share/fonts/truetype/cmu/,UprightFont=cmunrm.ttf,BoldFont=cmunbx.ttf,ItalicFont=cmunti.ttf,BoldItalicFont=cmunbi.ttf]{cmunbx.ttf}\setmonofont[Path=/usr/share/fonts/truetype/cmu/,UprightFont=cmuntt.ttf,BoldFont=cmuntb.ttf,ItalicFont=cmunit.ttf,BoldItalicFont=cmuntx.ttf]{cmunbx.ttf}\bfseries WildStrawberry}{$\text{ }$}\setmainfont[Path=/usr/share/fonts/truetype/cmu/,UprightFont=cmunrm.ttf,BoldFont=cmunbx.ttf,ItalicFont=cmunti.ttf,BoldItalicFont=cmunbi.ttf]{cmunrm.ttf}\setmonofont[Path=/usr/share/fonts/truetype/cmu/,UprightFont=cmuntt.ttf,BoldFont=cmuntb.ttf,ItalicFont=cmunit.ttf,BoldItalicFont=cmuntx.ttf]{cmunrm.ttf} &\cellcolor[rgb]{0.93333,0.16078,0.40392}\hspace*{0pt}\ignorespaces{}\hspace*{0pt} {\mbox{$~$}}&\hspace*{0pt}\ignorespaces{}\hspace*{0pt}{\mbox{$~$}}&\hspace*{0pt}\ignorespaces{}\hspace*{0pt} {\bfseries \setmainfont[Path=/usr/share/fonts/truetype/cmu/,UprightFont=cmunrm.ttf,BoldFont=cmunbx.ttf,ItalicFont=cmunti.ttf,BoldItalicFont=cmunbi.ttf]{cmunbx.ttf}\setmonofont[Path=/usr/share/fonts/truetype/cmu/,UprightFont=cmuntt.ttf,BoldFont=cmuntb.ttf,ItalicFont=cmunit.ttf,BoldItalicFont=cmuntx.ttf]{cmunbx.ttf}\bfseries Yellow}{$\text{ }$}\setmainfont[Path=/usr/share/fonts/truetype/cmu/,UprightFont=cmunrm.ttf,BoldFont=cmunbx.ttf,ItalicFont=cmunti.ttf,BoldItalicFont=cmunbi.ttf]{cmunrm.ttf}\setmonofont[Path=/usr/share/fonts/truetype/cmu/,UprightFont=cmuntt.ttf,BoldFont=cmuntb.ttf,ItalicFont=cmunit.ttf,BoldItalicFont=cmuntx.ttf]{cmunrm.ttf} &\cellcolor[rgb]{1.00000,0.94902,0.00000}\hspace*{0pt}\ignorespaces{}\hspace*{0pt} {\mbox{$~$}}\\ \hline \hspace*{0pt}\ignorespaces{}\hspace*{0pt}{\bfseries \setmainfont[Path=/usr/share/fonts/truetype/cmu/,UprightFont=cmunrm.ttf,BoldFont=cmunbx.ttf,ItalicFont=cmunti.ttf,BoldItalicFont=cmunbi.ttf]{cmunbx.ttf}\setmonofont[Path=/usr/share/fonts/truetype/cmu/,UprightFont=cmuntt.ttf,BoldFont=cmuntb.ttf,ItalicFont=cmunit.ttf,BoldItalicFont=cmuntx.ttf]{cmunbx.ttf}\bfseries YellowGreen}{$\text{ }$}\setmainfont[Path=/usr/share/fonts/truetype/cmu/,UprightFont=cmunrm.ttf,BoldFont=cmunbx.ttf,ItalicFont=cmunti.ttf,BoldItalicFont=cmunbi.ttf]{cmunrm.ttf}\setmonofont[Path=/usr/share/fonts/truetype/cmu/,UprightFont=cmuntt.ttf,BoldFont=cmuntb.ttf,ItalicFont=cmunit.ttf,BoldItalicFont=cmuntx.ttf]{cmunrm.ttf} &\cellcolor{ #98CC70}\hspace*{0pt}\ignorespaces{}\hspace*{0pt} {\mbox{$~$}}&\hspace*{0pt}\ignorespaces{}\hspace*{0pt}{\mbox{$~$}}&\hspace*{0pt}\ignorespaces{}\hspace*{0pt} {\bfseries \setmainfont[Path=/usr/share/fonts/truetype/cmu/,UprightFont=cmunrm.ttf,BoldFont=cmunbx.ttf,ItalicFont=cmunti.ttf,BoldItalicFont=cmunbi.ttf]{cmunbx.ttf}\setmonofont[Path=/usr/share/fonts/truetype/cmu/,UprightFont=cmuntt.ttf,BoldFont=cmuntb.ttf,ItalicFont=cmunit.ttf,BoldItalicFont=cmuntx.ttf]{cmunbx.ttf}\bfseries YellowOrange}{$\text{ }$}\setmainfont[Path=/usr/share/fonts/truetype/cmu/,UprightFont=cmunrm.ttf,BoldFont=cmunbx.ttf,ItalicFont=cmunti.ttf,BoldItalicFont=cmunbi.ttf]{cmunrm.ttf}\setmonofont[Path=/usr/share/fonts/truetype/cmu/,UprightFont=cmuntt.ttf,BoldFont=cmuntb.ttf,ItalicFont=cmunit.ttf,BoldItalicFont=cmuntx.ttf]{cmunrm.ttf} &\cellcolor[rgb]{0.98039,0.63529,0.10196}\hspace*{0pt}\ignorespaces{}\hspace*{0pt} {\mbox{$~$}}\\ \hline 
\end{longtable}

\section{Defining new colors}
\label{153}
If the predefined colors are not adequate, you may wish to define your own.
\subsection{Place}
\label{154}
Define the colors in the {\itshape \setmainfont[Path=/usr/share/fonts/truetype/cmu/,UprightFont=cmunrm.ttf,BoldFont=cmunbx.ttf,ItalicFont=cmunti.ttf,BoldItalicFont=cmunbi.ttf]{cmunti.ttf}\setmonofont[Path=/usr/share/fonts/truetype/cmu/,UprightFont=cmuntt.ttf,BoldFont=cmuntb.ttf,ItalicFont=cmunit.ttf,BoldItalicFont=cmuntx.ttf]{cmunti.ttf}\itshape preamble}{$\text{ }$}\setmainfont[Path=/usr/share/fonts/truetype/cmu/,UprightFont=cmunrm.ttf,BoldFont=cmunbx.ttf,ItalicFont=cmunti.ttf,BoldItalicFont=cmunbi.ttf]{cmunrm.ttf}\setmonofont[Path=/usr/share/fonts/truetype/cmu/,UprightFont=cmuntt.ttf,BoldFont=cmuntb.ttf,ItalicFont=cmunit.ttf,BoldItalicFont=cmuntx.ttf]{cmunrm.ttf} of your document. (Reason: do so in the preamble, so that you can already refer to them in the preamble, which is useful, for instance, in an argument of another package that supports colors as arguments, such as the \mylref{593}{listings} package.)
\subsection{Method}
\label{155}

You need to include the \LaTeXTT{xcolor} package in your preamble to define new colors. In the abstract, the colors are defined following this scheme:

\begin{Shaded}
\begin{Highlighting}[]

\NormalTok{\textbackslash{}definecolor\{name\}\{model\}\{color-spec\}}
\end{Highlighting}
\end{Shaded}


where:
\begin{myitemize}
\item{}  {\itshape \setmainfont[Path=/usr/share/fonts/truetype/cmu/,UprightFont=cmunrm.ttf,BoldFont=cmunbx.ttf,ItalicFont=cmunti.ttf,BoldItalicFont=cmunbi.ttf]{cmunti.ttf}\setmonofont[Path=/usr/share/fonts/truetype/cmu/,UprightFont=cmuntt.ttf,BoldFont=cmuntb.ttf,ItalicFont=cmunit.ttf,BoldItalicFont=cmuntx.ttf]{cmunti.ttf}\itshape name}{$\text{ }$}\setmainfont[Path=/usr/share/fonts/truetype/cmu/,UprightFont=cmunrm.ttf,BoldFont=cmunbx.ttf,ItalicFont=cmunti.ttf,BoldItalicFont=cmunbi.ttf]{cmunrm.ttf}\setmonofont[Path=/usr/share/fonts/truetype/cmu/,UprightFont=cmuntt.ttf,BoldFont=cmuntb.ttf,ItalicFont=cmunit.ttf,BoldItalicFont=cmuntx.ttf]{cmunrm.ttf} is the name of the color; you can call it as you like
\item{}  {\itshape \setmainfont[Path=/usr/share/fonts/truetype/cmu/,UprightFont=cmunrm.ttf,BoldFont=cmunbx.ttf,ItalicFont=cmunti.ttf,BoldItalicFont=cmunbi.ttf]{cmunti.ttf}\setmonofont[Path=/usr/share/fonts/truetype/cmu/,UprightFont=cmuntt.ttf,BoldFont=cmuntb.ttf,ItalicFont=cmunit.ttf,BoldItalicFont=cmuntx.ttf]{cmunti.ttf}\itshape model}{$\text{ }$}\setmainfont[Path=/usr/share/fonts/truetype/cmu/,UprightFont=cmunrm.ttf,BoldFont=cmunbx.ttf,ItalicFont=cmunti.ttf,BoldItalicFont=cmunbi.ttf]{cmunrm.ttf}\setmonofont[Path=/usr/share/fonts/truetype/cmu/,UprightFont=cmuntt.ttf,BoldFont=cmuntb.ttf,ItalicFont=cmunit.ttf,BoldItalicFont=cmuntx.ttf]{cmunrm.ttf} is the way you {\itshape \setmainfont[Path=/usr/share/fonts/truetype/cmu/,UprightFont=cmunrm.ttf,BoldFont=cmunbx.ttf,ItalicFont=cmunti.ttf,BoldItalicFont=cmunbi.ttf]{cmunti.ttf}\setmonofont[Path=/usr/share/fonts/truetype/cmu/,UprightFont=cmuntt.ttf,BoldFont=cmuntb.ttf,ItalicFont=cmunit.ttf,BoldItalicFont=cmuntx.ttf]{cmunti.ttf}\itshape describe}{$\text{ }$}\setmainfont[Path=/usr/share/fonts/truetype/cmu/,UprightFont=cmunrm.ttf,BoldFont=cmunbx.ttf,ItalicFont=cmunti.ttf,BoldItalicFont=cmunbi.ttf]{cmunrm.ttf}\setmonofont[Path=/usr/share/fonts/truetype/cmu/,UprightFont=cmuntt.ttf,BoldFont=cmuntb.ttf,ItalicFont=cmunit.ttf,BoldItalicFont=cmuntx.ttf]{cmunrm.ttf} the color, and is one of {\itshape \setmainfont[Path=/usr/share/fonts/truetype/cmu/,UprightFont=cmunrm.ttf,BoldFont=cmunbx.ttf,ItalicFont=cmunti.ttf,BoldItalicFont=cmunbi.ttf]{cmunti.ttf}\setmonofont[Path=/usr/share/fonts/truetype/cmu/,UprightFont=cmuntt.ttf,BoldFont=cmuntb.ttf,ItalicFont=cmunit.ttf,BoldItalicFont=cmuntx.ttf]{cmunti.ttf}\itshape gray}\setmainfont[Path=/usr/share/fonts/truetype/cmu/,UprightFont=cmunrm.ttf,BoldFont=cmunbx.ttf,ItalicFont=cmunti.ttf,BoldItalicFont=cmunbi.ttf]{cmunrm.ttf}\setmonofont[Path=/usr/share/fonts/truetype/cmu/,UprightFont=cmuntt.ttf,BoldFont=cmuntb.ttf,ItalicFont=cmunit.ttf,BoldItalicFont=cmuntx.ttf]{cmunrm.ttf}, {\itshape \setmainfont[Path=/usr/share/fonts/truetype/cmu/,UprightFont=cmunrm.ttf,BoldFont=cmunbx.ttf,ItalicFont=cmunti.ttf,BoldItalicFont=cmunbi.ttf]{cmunti.ttf}\setmonofont[Path=/usr/share/fonts/truetype/cmu/,UprightFont=cmuntt.ttf,BoldFont=cmuntb.ttf,ItalicFont=cmunit.ttf,BoldItalicFont=cmuntx.ttf]{cmunti.ttf}\itshape rgb}\setmainfont[Path=/usr/share/fonts/truetype/cmu/,UprightFont=cmunrm.ttf,BoldFont=cmunbx.ttf,ItalicFont=cmunti.ttf,BoldItalicFont=cmunbi.ttf]{cmunrm.ttf}\setmonofont[Path=/usr/share/fonts/truetype/cmu/,UprightFont=cmuntt.ttf,BoldFont=cmuntb.ttf,ItalicFont=cmunit.ttf,BoldItalicFont=cmuntx.ttf]{cmunrm.ttf}, {\itshape \setmainfont[Path=/usr/share/fonts/truetype/cmu/,UprightFont=cmunrm.ttf,BoldFont=cmunbx.ttf,ItalicFont=cmunti.ttf,BoldItalicFont=cmunbi.ttf]{cmunti.ttf}\setmonofont[Path=/usr/share/fonts/truetype/cmu/,UprightFont=cmuntt.ttf,BoldFont=cmuntb.ttf,ItalicFont=cmunit.ttf,BoldItalicFont=cmuntx.ttf]{cmunti.ttf}\itshape RGB}\setmainfont[Path=/usr/share/fonts/truetype/cmu/,UprightFont=cmunrm.ttf,BoldFont=cmunbx.ttf,ItalicFont=cmunti.ttf,BoldItalicFont=cmunbi.ttf]{cmunrm.ttf}\setmonofont[Path=/usr/share/fonts/truetype/cmu/,UprightFont=cmuntt.ttf,BoldFont=cmuntb.ttf,ItalicFont=cmunit.ttf,BoldItalicFont=cmuntx.ttf]{cmunrm.ttf}, {\itshape \setmainfont[Path=/usr/share/fonts/truetype/cmu/,UprightFont=cmunrm.ttf,BoldFont=cmunbx.ttf,ItalicFont=cmunti.ttf,BoldItalicFont=cmunbi.ttf]{cmunti.ttf}\setmonofont[Path=/usr/share/fonts/truetype/cmu/,UprightFont=cmuntt.ttf,BoldFont=cmuntb.ttf,ItalicFont=cmunit.ttf,BoldItalicFont=cmuntx.ttf]{cmunti.ttf}\itshape HTML}\setmainfont[Path=/usr/share/fonts/truetype/cmu/,UprightFont=cmunrm.ttf,BoldFont=cmunbx.ttf,ItalicFont=cmunti.ttf,BoldItalicFont=cmunbi.ttf]{cmunrm.ttf}\setmonofont[Path=/usr/share/fonts/truetype/cmu/,UprightFont=cmuntt.ttf,BoldFont=cmuntb.ttf,ItalicFont=cmunit.ttf,BoldItalicFont=cmuntx.ttf]{cmunrm.ttf}, and {\itshape \setmainfont[Path=/usr/share/fonts/truetype/cmu/,UprightFont=cmunrm.ttf,BoldFont=cmunbx.ttf,ItalicFont=cmunti.ttf,BoldItalicFont=cmunbi.ttf]{cmunti.ttf}\setmonofont[Path=/usr/share/fonts/truetype/cmu/,UprightFont=cmuntt.ttf,BoldFont=cmuntb.ttf,ItalicFont=cmunit.ttf,BoldItalicFont=cmuntx.ttf]{cmunti.ttf}\itshape cmyk}\setmainfont[Path=/usr/share/fonts/truetype/cmu/,UprightFont=cmunrm.ttf,BoldFont=cmunbx.ttf,ItalicFont=cmunti.ttf,BoldItalicFont=cmunbi.ttf]{cmunrm.ttf}\setmonofont[Path=/usr/share/fonts/truetype/cmu/,UprightFont=cmuntt.ttf,BoldFont=cmuntb.ttf,ItalicFont=cmunit.ttf,BoldItalicFont=cmuntx.ttf]{cmunrm.ttf}.
\item{}  {\itshape \setmainfont[Path=/usr/share/fonts/truetype/cmu/,UprightFont=cmunrm.ttf,BoldFont=cmunbx.ttf,ItalicFont=cmunti.ttf,BoldItalicFont=cmunbi.ttf]{cmunti.ttf}\setmonofont[Path=/usr/share/fonts/truetype/cmu/,UprightFont=cmuntt.ttf,BoldFont=cmuntb.ttf,ItalicFont=cmunit.ttf,BoldItalicFont=cmuntx.ttf]{cmunti.ttf}\itshape color-{}spec}{$\text{ }$}\setmainfont[Path=/usr/share/fonts/truetype/cmu/,UprightFont=cmunrm.ttf,BoldFont=cmunbx.ttf,ItalicFont=cmunti.ttf,BoldItalicFont=cmunbi.ttf]{cmunrm.ttf}\setmonofont[Path=/usr/share/fonts/truetype/cmu/,UprightFont=cmuntt.ttf,BoldFont=cmuntb.ttf,ItalicFont=cmunit.ttf,BoldItalicFont=cmuntx.ttf]{cmunrm.ttf} is the description of the color
\end{myitemize}

\subsection{Color Models}
\label{156}

Among the models you can use to describe the color are the following (several more are described in the \myhref{http://mirror.ctan.org/macros/latex/contrib/xcolor/xcolor.pdf}{xcolor manual}):{\scriptsize{}
\begin{longtable}{|>{\RaggedRight}p{0.09190\linewidth}|>{\RaggedRight}p{0.31383\linewidth}|>{\RaggedRight}p{0.24000\linewidth}|>{\RaggedRight}p{0.24000\linewidth}|} \hline 
\multicolumn{4}{|>{\RaggedRight}p{0.97143\linewidth}|}{{\bfseries \hspace*{0pt}\ignorespaces{}\hspace*{0pt} Color Models}}\\ \hline {\bfseries \hspace*{0pt}\ignorespaces{}\hspace*{0pt} Model}&{\bfseries \hspace*{0pt}\ignorespaces{}\hspace*{0pt} Description}&{\bfseries \hspace*{0pt}\ignorespaces{}\hspace*{0pt}Color Specification}&{\bfseries \hspace*{0pt}\ignorespaces{}\hspace*{0pt} Example}\endhead  \hline \hspace*{0pt}\ignorespaces{}\hspace*{0pt}{\ttfamily \setmainfont[Path=/usr/share/fonts/truetype/cmu/,UprightFont=cmunrm.ttf,BoldFont=cmunbx.ttf,ItalicFont=cmunti.ttf,BoldItalicFont=cmunbi.ttf]{cmuntt.ttf}\setmonofont[Path=/usr/share/fonts/truetype/cmu/,UprightFont=cmuntt.ttf,BoldFont=cmuntb.ttf,ItalicFont=cmunit.ttf,BoldItalicFont=cmuntx.ttf]{cmuntt.ttf}\ttfamily gray}{$\text{ }$}\setmainfont[Path=/usr/share/fonts/truetype/cmu/,UprightFont=cmunrm.ttf,BoldFont=cmunbx.ttf,ItalicFont=cmunti.ttf,BoldItalicFont=cmunbi.ttf]{cmunrm.ttf}\setmonofont[Path=/usr/share/fonts/truetype/cmu/,UprightFont=cmuntt.ttf,BoldFont=cmuntb.ttf,ItalicFont=cmunit.ttf,BoldItalicFont=cmuntx.ttf]{cmunrm.ttf} &\hspace*{0pt}\ignorespaces{}\hspace*{0pt} Shades of gray \newline{} (0-{}1) &\hspace*{0pt}\ignorespaces{}\hspace*{0pt} Just one number between 0 (black) and 1 (white), so 0.95 will be very light gray, 0.30 will be dark gray. &\hspace*{0pt}\ignorespaces{}\hspace*{0pt} {\ttfamily \setmainfont[Path=/usr/share/fonts/truetype/cmu/,UprightFont=cmunrm.ttf,BoldFont=cmunbx.ttf,ItalicFont=cmunti.ttf,BoldItalicFont=cmunbi.ttf]{cmuntt.ttf}\setmonofont[Path=/usr/share/fonts/truetype/cmu/,UprightFont=cmuntt.ttf,BoldFont=cmuntb.ttf,ItalicFont=cmunit.ttf,BoldItalicFont=cmuntx.ttf]{cmuntt.ttf}\ttfamily \textbackslash{}definecolor\{light-{}gray\}\{gray\}\{0.95\}}{$\text{ }$}\setmainfont[Path=/usr/share/fonts/truetype/cmu/,UprightFont=cmunrm.ttf,BoldFont=cmunbx.ttf,ItalicFont=cmunti.ttf,BoldItalicFont=cmunbi.ttf]{cmunrm.ttf}\setmonofont[Path=/usr/share/fonts/truetype/cmu/,UprightFont=cmuntt.ttf,BoldFont=cmuntb.ttf,ItalicFont=cmunit.ttf,BoldItalicFont=cmuntx.ttf]{cmunrm.ttf} \\ \hline \hspace*{0pt}\ignorespaces{}\hspace*{0pt}{\ttfamily \setmainfont[Path=/usr/share/fonts/truetype/cmu/,UprightFont=cmunrm.ttf,BoldFont=cmunbx.ttf,ItalicFont=cmunti.ttf,BoldItalicFont=cmunbi.ttf]{cmuntt.ttf}\setmonofont[Path=/usr/share/fonts/truetype/cmu/,UprightFont=cmuntt.ttf,BoldFont=cmuntb.ttf,ItalicFont=cmunit.ttf,BoldItalicFont=cmuntx.ttf]{cmuntt.ttf}\ttfamily rgb}{$\text{ }$}\setmainfont[Path=/usr/share/fonts/truetype/cmu/,UprightFont=cmunrm.ttf,BoldFont=cmunbx.ttf,ItalicFont=cmunti.ttf,BoldItalicFont=cmunbi.ttf]{cmunrm.ttf}\setmonofont[Path=/usr/share/fonts/truetype/cmu/,UprightFont=cmuntt.ttf,BoldFont=cmuntb.ttf,ItalicFont=cmunit.ttf,BoldItalicFont=cmuntx.ttf]{cmunrm.ttf} &\hspace*{0pt}\ignorespaces{}\hspace*{0pt} Red, Green, Blue \newline{} (0-{}1) &\hspace*{0pt}\ignorespaces{}\hspace*{0pt} Three numbers given in the form {\itshape \setmainfont[Path=/usr/share/fonts/truetype/cmu/,UprightFont=cmunrm.ttf,BoldFont=cmunbx.ttf,ItalicFont=cmunti.ttf,BoldItalicFont=cmunbi.ttf]{cmunti.ttf}\setmonofont[Path=/usr/share/fonts/truetype/cmu/,UprightFont=cmuntt.ttf,BoldFont=cmuntb.ttf,ItalicFont=cmunit.ttf,BoldItalicFont=cmuntx.ttf]{cmunti.ttf}\itshape red,green,blue}\setmainfont[Path=/usr/share/fonts/truetype/cmu/,UprightFont=cmunrm.ttf,BoldFont=cmunbx.ttf,ItalicFont=cmunti.ttf,BoldItalicFont=cmunbi.ttf]{cmunrm.ttf}\setmonofont[Path=/usr/share/fonts/truetype/cmu/,UprightFont=cmuntt.ttf,BoldFont=cmuntb.ttf,ItalicFont=cmunit.ttf,BoldItalicFont=cmuntx.ttf]{cmunrm.ttf}; the quantity of each color is represented with a number between 0 and 1. &\hspace*{0pt}\ignorespaces{}\hspace*{0pt} {\ttfamily \setmainfont[Path=/usr/share/fonts/truetype/cmu/,UprightFont=cmunrm.ttf,BoldFont=cmunbx.ttf,ItalicFont=cmunti.ttf,BoldItalicFont=cmunbi.ttf]{cmuntt.ttf}\setmonofont[Path=/usr/share/fonts/truetype/cmu/,UprightFont=cmuntt.ttf,BoldFont=cmuntb.ttf,ItalicFont=cmunit.ttf,BoldItalicFont=cmuntx.ttf]{cmuntt.ttf}\ttfamily \textbackslash{}definecolor\{orange\}\{rgb\}\{1,0.5,0\}}\\ \hline \hspace*{0pt}\ignorespaces{}\hspace*{0pt}{\ttfamily RGB}{$\text{ }$}\setmainfont[Path=/usr/share/fonts/truetype/cmu/,UprightFont=cmunrm.ttf,BoldFont=cmunbx.ttf,ItalicFont=cmunti.ttf,BoldItalicFont=cmunbi.ttf]{cmunrm.ttf}\setmonofont[Path=/usr/share/fonts/truetype/cmu/,UprightFont=cmuntt.ttf,BoldFont=cmuntb.ttf,ItalicFont=cmunit.ttf,BoldItalicFont=cmuntx.ttf]{cmunrm.ttf} &\hspace*{0pt}\ignorespaces{}\hspace*{0pt} Red, Green, Blue \newline{} (0-{}255) &\hspace*{0pt}\ignorespaces{}\hspace*{0pt} Three numbers given in the form {\itshape \setmainfont[Path=/usr/share/fonts/truetype/cmu/,UprightFont=cmunrm.ttf,BoldFont=cmunbx.ttf,ItalicFont=cmunti.ttf,BoldItalicFont=cmunbi.ttf]{cmunti.ttf}\setmonofont[Path=/usr/share/fonts/truetype/cmu/,UprightFont=cmuntt.ttf,BoldFont=cmuntb.ttf,ItalicFont=cmunit.ttf,BoldItalicFont=cmuntx.ttf]{cmunti.ttf}\itshape red,green,blue}\setmainfont[Path=/usr/share/fonts/truetype/cmu/,UprightFont=cmunrm.ttf,BoldFont=cmunbx.ttf,ItalicFont=cmunti.ttf,BoldItalicFont=cmunbi.ttf]{cmunrm.ttf}\setmonofont[Path=/usr/share/fonts/truetype/cmu/,UprightFont=cmuntt.ttf,BoldFont=cmuntb.ttf,ItalicFont=cmunit.ttf,BoldItalicFont=cmuntx.ttf]{cmunrm.ttf}; the quantity of each color is represented with a number between 0 and 255. &\hspace*{0pt}\ignorespaces{}\hspace*{0pt} {\ttfamily \setmainfont[Path=/usr/share/fonts/truetype/cmu/,UprightFont=cmunrm.ttf,BoldFont=cmunbx.ttf,ItalicFont=cmunti.ttf,BoldItalicFont=cmunbi.ttf]{cmuntt.ttf}\setmonofont[Path=/usr/share/fonts/truetype/cmu/,UprightFont=cmuntt.ttf,BoldFont=cmuntb.ttf,ItalicFont=cmunit.ttf,BoldItalicFont=cmuntx.ttf]{cmuntt.ttf}\ttfamily \textbackslash{}definecolor\{orange\}\{RGB\}\{255,127,0\}}\\ \hline \hspace*{0pt}\ignorespaces{}\hspace*{0pt}{\ttfamily HTML}{$\text{ }$}\setmainfont[Path=/usr/share/fonts/truetype/cmu/,UprightFont=cmunrm.ttf,BoldFont=cmunbx.ttf,ItalicFont=cmunti.ttf,BoldItalicFont=cmunbi.ttf]{cmunrm.ttf}\setmonofont[Path=/usr/share/fonts/truetype/cmu/,UprightFont=cmuntt.ttf,BoldFont=cmuntb.ttf,ItalicFont=cmunit.ttf,BoldItalicFont=cmuntx.ttf]{cmunrm.ttf} &\hspace*{0pt}\ignorespaces{}\hspace*{0pt} Red, Green, Blue \newline{} (00-{}FF) &\hspace*{0pt}\ignorespaces{}\hspace*{0pt} Six hexadecimal numbers given in the form {\itshape \setmainfont[Path=/usr/share/fonts/truetype/cmu/,UprightFont=cmunrm.ttf,BoldFont=cmunbx.ttf,ItalicFont=cmunti.ttf,BoldItalicFont=cmunbi.ttf]{cmunti.ttf}\setmonofont[Path=/usr/share/fonts/truetype/cmu/,UprightFont=cmuntt.ttf,BoldFont=cmuntb.ttf,ItalicFont=cmunit.ttf,BoldItalicFont=cmuntx.ttf]{cmunti.ttf}\itshape RRGGBB}\setmainfont[Path=/usr/share/fonts/truetype/cmu/,UprightFont=cmunrm.ttf,BoldFont=cmunbx.ttf,ItalicFont=cmunti.ttf,BoldItalicFont=cmunbi.ttf]{cmunrm.ttf}\setmonofont[Path=/usr/share/fonts/truetype/cmu/,UprightFont=cmuntt.ttf,BoldFont=cmuntb.ttf,ItalicFont=cmunit.ttf,BoldItalicFont=cmuntx.ttf]{cmunrm.ttf}; similar to what is used in HTML. &\hspace*{0pt}\ignorespaces{}\hspace*{0pt} {\ttfamily \setmainfont[Path=/usr/share/fonts/truetype/cmu/,UprightFont=cmunrm.ttf,BoldFont=cmunbx.ttf,ItalicFont=cmunti.ttf,BoldItalicFont=cmunbi.ttf]{cmuntt.ttf}\setmonofont[Path=/usr/share/fonts/truetype/cmu/,UprightFont=cmuntt.ttf,BoldFont=cmuntb.ttf,ItalicFont=cmunit.ttf,BoldItalicFont=cmuntx.ttf]{cmuntt.ttf}\ttfamily \textbackslash{}definecolor\{orange\}\{HTML\}\{FF7F00\}}\\ \hline \hspace*{0pt}\ignorespaces{}\hspace*{0pt}{\ttfamily cmyk}{$\text{ }$}\setmainfont[Path=/usr/share/fonts/truetype/cmu/,UprightFont=cmunrm.ttf,BoldFont=cmunbx.ttf,ItalicFont=cmunti.ttf,BoldItalicFont=cmunbi.ttf]{cmunrm.ttf}\setmonofont[Path=/usr/share/fonts/truetype/cmu/,UprightFont=cmuntt.ttf,BoldFont=cmuntb.ttf,ItalicFont=cmunit.ttf,BoldItalicFont=cmuntx.ttf]{cmunrm.ttf} &\hspace*{0pt}\ignorespaces{}\hspace*{0pt} Cyan, Magenta, Yellow, Black \newline{} (0-{}1) &\hspace*{0pt}\ignorespaces{}\hspace*{0pt} Four numbers given in the form {\itshape \setmainfont[Path=/usr/share/fonts/truetype/cmu/,UprightFont=cmunrm.ttf,BoldFont=cmunbx.ttf,ItalicFont=cmunti.ttf,BoldItalicFont=cmunbi.ttf]{cmunti.ttf}\setmonofont[Path=/usr/share/fonts/truetype/cmu/,UprightFont=cmuntt.ttf,BoldFont=cmuntb.ttf,ItalicFont=cmunit.ttf,BoldItalicFont=cmuntx.ttf]{cmunti.ttf}\itshape cyan,magenta,yellow,black}\setmainfont[Path=/usr/share/fonts/truetype/cmu/,UprightFont=cmunrm.ttf,BoldFont=cmunbx.ttf,ItalicFont=cmunti.ttf,BoldItalicFont=cmunbi.ttf]{cmunrm.ttf}\setmonofont[Path=/usr/share/fonts/truetype/cmu/,UprightFont=cmuntt.ttf,BoldFont=cmuntb.ttf,ItalicFont=cmunit.ttf,BoldItalicFont=cmuntx.ttf]{cmunrm.ttf}; the quantity of each color is represented with a number between 0 and 1. &\hspace*{0pt}\ignorespaces{}\hspace*{0pt} {\ttfamily \setmainfont[Path=/usr/share/fonts/truetype/cmu/,UprightFont=cmunrm.ttf,BoldFont=cmunbx.ttf,ItalicFont=cmunti.ttf,BoldItalicFont=cmunbi.ttf]{cmuntt.ttf}\setmonofont[Path=/usr/share/fonts/truetype/cmu/,UprightFont=cmuntt.ttf,BoldFont=cmuntb.ttf,ItalicFont=cmunit.ttf,BoldItalicFont=cmuntx.ttf]{cmuntt.ttf}\ttfamily \textbackslash{}definecolor\{orange\}\{cmyk\}\{0,0.5,1,0\}}\\ \hline 
\end{longtable}
}\setmainfont[Path=/usr/share/fonts/truetype/cmu/,UprightFont=cmunrm.ttf,BoldFont=cmunbx.ttf,ItalicFont=cmunti.ttf,BoldItalicFont=cmunbi.ttf]{cmunrm.ttf}\setmonofont[Path=/usr/share/fonts/truetype/cmu/,UprightFont=cmuntt.ttf,BoldFont=cmuntb.ttf,ItalicFont=cmunit.ttf,BoldItalicFont=cmuntx.ttf]{cmunrm.ttf}
\subsection{Examples}
\label{157}

To define a new color, follow the following example, which defines orange for you, by setting the red to the maximum, the green to one half (0.5), and the blue to the minimum:

\begin{Shaded}
\begin{Highlighting}[]

\NormalTok{\textbackslash{}definecolor\{orange\}\{rgb\}\{1,0.5,0\}}
\end{Highlighting}
\end{Shaded}


The following code should give a similar results to the last code chunk.

\begin{Shaded}
\begin{Highlighting}[]

\NormalTok{\textbackslash{}definecolor\{orange\}\{RGB\}\{255,127,0\}}
\end{Highlighting}
\end{Shaded}


If you loaded the \LaTeXTT{xcolor} package, you can define colors upon previously defined ones.

The first specifies 20 percent blue and 80 percent white; the second is a mixture of 20 percent blue and 80 percent black; and the last one is a mixture of (20*0.3) percent blue, ((100-{}20)*0.3) percent black and (100-{}30) percent green.

\begin{Shaded}
\begin{Highlighting}[]

\NormalTok{\textbackslash{}color\{blue!20\}}
\NormalTok{\textbackslash{}color\{blue!20!black\}}
\NormalTok{\textbackslash{}color\{blue!20!black!30!green\}}
\end{Highlighting}
\end{Shaded}


\LaTeXTT{xcolor} also feature a handy command to define colors from color mixes:

\begin{Shaded}
\begin{Highlighting}[]

\NormalTok{\textbackslash{}colorlet\{notgreen\}\{blue!50!yellow\}}
\end{Highlighting}
\end{Shaded}

\subsection{Using color specifications directly}
\label{158}

Normally one would predeclare all the colors as above, but sometimes it is convenient to directly use a color without naming it first. To achieve this, 
\LaTeXTT{\textbackslash{}color} 
and 
\LaTeXTT{\textbackslash{}textcolor} 
have an alternative syntax specifying the model in square brackets, and the color specification in curly braces. For example:

\begin{Shaded}
\begin{Highlighting}[]

\NormalTok{\{\textbackslash{}color[rgb]\{1,0,0\} This text will appear red-colored\}}
\NormalTok{\textbackslash{}textcolor[rgb]\{0,1,0\}\{This text will appear green-colored\}}
\end{Highlighting}
\end{Shaded}

\subsection{Creating / Capturing colors}
\label{159}

You may want to use colors that appear on another document, web pages, pictures, etc.
Alternatively, you may want to play around with rgb values to create your own custom colors.

Image processing suites like the free \myhref{http://www.gimp.org/downloads/}{GIMP} suite for Linux/Windows/Mac offer color picker facilities to capture any color on your screen or synthesize colors directly from their respective rgb / hsv / hexadecimal values. 

Smaller, free utilities also exist:
\begin{myitemize}
\item{}  Linux/BSD: The \myhref{http://gcolor2.sourceforge.net/}{gcolor2} tool (usually also available in repositories)
\item{}  Microsoft Windows: The open-{}source \myhref{http://colorselector.sourceforge.net/}{Color Selector} tool.
\item{}  Apple Macs: \myhref{http://wafflesoftware.net/hexpicker/}{Hex Color Picker} for creating custom colors and the built-{}in \myhref{http://www.apple.com/uk/osx/apps/all.html\#colormeter}{DigitalColor Meter} for capturing colors on screen.
\item{}  Online utilities: See here for a \myhref{https://en.wikipedia.org/wiki/Color_tool}{Wikipedia article with several external links}
\end{myitemize}

\subsection{Spot colors}
\label{160}

Spot colors are customary in printing. They usually refer to pre-{}mixed inks based on a swatchbook (like Pantone, TruMatch or Toyo). The package \LaTeXTT{colorspace} extends xcolor to provide real spot colors. They are defined with, say:

\begin{Shaded}
\begin{Highlighting}[]

\NormalTok{\textbackslash{}definespotcolor\{mygreen\}\{PANTONE 7716 C\}\{.83, 0, .40, .11\}}
\end{Highlighting}
\end{Shaded}

\section{Sources}
\label{161}
\begin{myitemize}
\item{}  \myhref{http://mirror.ctan.org/macros/latex/contrib/xcolor/xcolor.pdf}{The xcolor manual}
\item{}  \myhref{http://mirrors.ctan.org/macros/latex/required/graphics/grfguide.pdf}{The color package documentation }
\end{myitemize}


\chapter{Fonts}

\myminitoc
\label{162}

\label{163}


Fonts are a complex topic. For common documents, only \mylref{165}{Font families}, \mylref{172}{Emphasizing text}, and \mylref{173}{Font encoding} are really needed. The other sections are more useful to macro writers or for very specific needs.
\section{Introduction}
\label{164}

The digital fonts have a long and intricate history. See \myhref{https://en.wikipedia.org/wiki/Adobe\%20Font\%20Metrics}{Adobe Font Metrics} for some more details.

Originally TeX was conceived to use its own font system, MetaFont, designed by D. Knuth. The default font family for TeX and friends is called Computer Modern. These high quality fonts are scalable, and have a wide range of typographical fine tuning capabilities.

Standard {\ttfamily \setmainfont[Path=/usr/share/fonts/truetype/cmu/,UprightFont=cmunrm.ttf,BoldFont=cmunbx.ttf,ItalicFont=cmunti.ttf,BoldItalicFont=cmunbi.ttf]{cmuntt.ttf}\setmonofont[Path=/usr/share/fonts/truetype/cmu/,UprightFont=cmuntt.ttf,BoldFont=cmuntb.ttf,ItalicFont=cmunit.ttf,BoldItalicFont=cmuntx.ttf]{cmuntt.ttf}\ttfamily tex}{$\text{ }$}\setmainfont[Path=/usr/share/fonts/truetype/cmu/,UprightFont=cmunrm.ttf,BoldFont=cmunbx.ttf,ItalicFont=cmunti.ttf,BoldItalicFont=cmunbi.ttf]{cmunrm.ttf}\setmonofont[Path=/usr/share/fonts/truetype/cmu/,UprightFont=cmuntt.ttf,BoldFont=cmuntb.ttf,ItalicFont=cmunit.ttf,BoldItalicFont=cmuntx.ttf]{cmunrm.ttf} compilers will let you use other fonts. There are many different font types, such as PostScript Type1/Type3 fonts and bitmap fonts. Type1 are outline fonts (vector graphics) which are commonly used by {\ttfamily \setmainfont[Path=/usr/share/fonts/truetype/cmu/,UprightFont=cmunrm.ttf,BoldFont=cmunbx.ttf,ItalicFont=cmunti.ttf,BoldItalicFont=cmunbi.ttf]{cmuntt.ttf}\setmonofont[Path=/usr/share/fonts/truetype/cmu/,UprightFont=cmuntt.ttf,BoldFont=cmuntb.ttf,ItalicFont=cmunit.ttf,BoldItalicFont=cmuntx.ttf]{cmuntt.ttf}\ttfamily pdftex}\setmainfont[Path=/usr/share/fonts/truetype/cmu/,UprightFont=cmunrm.ttf,BoldFont=cmunbx.ttf,ItalicFont=cmunti.ttf,BoldItalicFont=cmunbi.ttf]{cmunrm.ttf}\setmonofont[Path=/usr/share/fonts/truetype/cmu/,UprightFont=cmuntt.ttf,BoldFont=cmuntb.ttf,ItalicFont=cmunit.ttf,BoldItalicFont=cmuntx.ttf]{cmunrm.ttf}. Bitmap fonts are raster graphics, and usually have very poor quality, which can easily be seen when zooming or printing a document. Type3 is a superset of Type1 and has more functionalities from Postscript, such as embedding raster graphics. In the TeX world, Type3 fonts are often used to embed bitmap fonts.

It should be noticed that fonts get generated the first time they are required, hence the long compilation time.

However, MetaFont is internally a quite complex font system, and the most popular font systems as of this day are \myhref{https://en.wikipedia.org/wiki/TrueType}{Truetype} font (ttf) and \myhref{https://en.wikipedia.org/wiki/OpenType}{OpenType} font (otf). With modern TeX compilers such as {\ttfamily \setmainfont[Path=/usr/share/fonts/truetype/cmu/,UprightFont=cmunrm.ttf,BoldFont=cmunbx.ttf,ItalicFont=cmunti.ttf,BoldItalicFont=cmunbi.ttf]{cmuntt.ttf}\setmonofont[Path=/usr/share/fonts/truetype/cmu/,UprightFont=cmuntt.ttf,BoldFont=cmuntb.ttf,ItalicFont=cmunit.ttf,BoldItalicFont=cmuntx.ttf]{cmuntt.ttf}\ttfamily xetex}{$\text{ }$}\setmainfont[Path=/usr/share/fonts/truetype/cmu/,UprightFont=cmunrm.ttf,BoldFont=cmunbx.ttf,ItalicFont=cmunti.ttf,BoldItalicFont=cmunbi.ttf]{cmunrm.ttf}\setmonofont[Path=/usr/share/fonts/truetype/cmu/,UprightFont=cmuntt.ttf,BoldFont=cmuntb.ttf,ItalicFont=cmunit.ttf,BoldItalicFont=cmuntx.ttf]{cmunrm.ttf} and {\ttfamily \setmainfont[Path=/usr/share/fonts/truetype/cmu/,UprightFont=cmunrm.ttf,BoldFont=cmunbx.ttf,ItalicFont=cmunti.ttf,BoldItalicFont=cmunbi.ttf]{cmuntt.ttf}\setmonofont[Path=/usr/share/fonts/truetype/cmu/,UprightFont=cmuntt.ttf,BoldFont=cmuntb.ttf,ItalicFont=cmunit.ttf,BoldItalicFont=cmuntx.ttf]{cmuntt.ttf}\ttfamily luatex}{$\text{ }$}\setmainfont[Path=/usr/share/fonts/truetype/cmu/,UprightFont=cmunrm.ttf,BoldFont=cmunbx.ttf,ItalicFont=cmunti.ttf,BoldItalicFont=cmunbi.ttf]{cmunrm.ttf}\setmonofont[Path=/usr/share/fonts/truetype/cmu/,UprightFont=cmuntt.ttf,BoldFont=cmuntb.ttf,ItalicFont=cmunit.ttf,BoldItalicFont=cmuntx.ttf]{cmunrm.ttf} it is possible to make use of such fonts in LaTeX documents. If you want/have to stick with the standard compilers, the aforementioned font types must first be converted and made available to LaTeX ({\itshape \setmainfont[Path=/usr/share/fonts/truetype/cmu/,UprightFont=cmunrm.ttf,BoldFont=cmunbx.ttf,ItalicFont=cmunti.ttf,BoldItalicFont=cmunbi.ttf]{cmunti.ttf}\setmonofont[Path=/usr/share/fonts/truetype/cmu/,UprightFont=cmuntt.ttf,BoldFont=cmuntb.ttf,ItalicFont=cmunit.ttf,BoldItalicFont=cmuntx.ttf]{cmunti.ttf}\itshape e.g.}{$\text{ }$}\setmainfont[Path=/usr/share/fonts/truetype/cmu/,UprightFont=cmunrm.ttf,BoldFont=cmunbx.ttf,ItalicFont=cmunti.ttf,BoldItalicFont=cmunbi.ttf]{cmunrm.ttf}\setmonofont[Path=/usr/share/fonts/truetype/cmu/,UprightFont=cmuntt.ttf,BoldFont=cmuntb.ttf,ItalicFont=cmunit.ttf,BoldItalicFont=cmuntx.ttf]{cmunrm.ttf} converted to Type1 fonts). The external links section below has some useful resources.

In LaTeX, there are many ways to specify and control fonts. It is a very complex matter in typography.
\section{Font families}
\label{165}
There are many font families e.g. Computer Modern, Times, Arial, and Courier. Those families can be grouped into three main categories: roman (rm) or serif, sans serif (sf) and monospace (tt) (see \myhref{https://en.wikipedia.org/wiki/Typeface}{Typeface} for more details). Each font family comes with the default design which falls into one of those categories; however, it is interchangeable among them. Computer Modern Roman is the default font family for LaTeX. Fonts in each family also have different properties (size, shape, weight, etc.). Families are meant to be consistent, so it is highly discouraged to change fonts individually rather than the whole family.

The three families are defined by their respective variables:
\begin{myitemize}
\item{}  \LaTeXTT{\textbackslash{}rmdefault}
\item{}  \LaTeXTT{\textbackslash{}sfdefault}
\item{}  \LaTeXTT{\textbackslash{}ttdefault}
\end{myitemize}


The default family is contained in the \LaTeXTT{\textbackslash{}familydefault} variable, and it is meant to have one of the three aforementioned variables as value. The default is defined (in the preamble) like the following assignment:
\begin{Shaded}
\begin{Highlighting}[]

\NormalTok{\textbackslash{}renewcommand\{\textbackslash{}familydefault\}\{\textbackslash{}sfdefault\}}
\end{Highlighting}
\end{Shaded}

This will turn all the part of the document using the default font to the default sans serif, which is Computer Modern Sans Serif if you did not change the default font.

Changing font families usually works in two steps:
\begin{myenumerate}
\item{}  First specify which family you want to change (rm, sf or tt).
\item{}  Second specify the new default family if it is not rm.
\end{myenumerate}


Mathematical fonts is a more complex matter. Fonts may come with a package that will take care of defining all three families plus the math fonts. You can do it by yourself, in which case you do not have to load any package.

Below is an example\myfootnote{found at the Google discussion group {\itshape \setmainfont[Path=/usr/share/fonts/truetype/cmu/,UprightFont=cmunrm.ttf,BoldFont=cmunbx.ttf,ItalicFont=cmunti.ttf,BoldItalicFont=cmunbi.ttf]{cmunti.ttf}\setmonofont[Path=/usr/share/fonts/truetype/cmu/,UprightFont=cmuntt.ttf,BoldFont=cmuntb.ttf,ItalicFont=cmunit.ttf,BoldItalicFont=cmuntx.ttf]{cmunti.ttf}\itshape latexlovers}}{$\text{ }$}\setmainfont[Path=/usr/share/fonts/truetype/cmu/,UprightFont=cmunrm.ttf,BoldFont=cmunbx.ttf,ItalicFont=cmunti.ttf,BoldItalicFont=cmunbi.ttf]{cmunrm.ttf}\setmonofont[Path=/usr/share/fonts/truetype/cmu/,UprightFont=cmuntt.ttf,BoldFont=cmuntb.ttf,ItalicFont=cmunit.ttf,BoldItalicFont=cmuntx.ttf]{cmunrm.ttf} that demonstrates how to change a specific family.

\begin{longtable}{p{1.0\linewidth}}
\begin{Shaded}
\begin{Highlighting}[]

\CommentTok{% Default font (\textbackslash{}familydefault = \textbackslash{}rmdefault = Computer Modern Roman)}
\NormalTok{Lorem ipsum dolor sit amet, consectitur adipiscing elit.}
 
\CommentTok{% Palatino font (ppl must be installed).}
\NormalTok{\textbackslash{}renewcommand*\textbackslash{}rmdefault\{ppl\}}
\NormalTok{Lorem ipsum dolor sit amet, consectitur adipiscing elit.}
 
\CommentTok{% Iwona font (iwona must be installed).}
\NormalTok{\textbackslash{}renewcommand*\textbackslash{}rmdefault\{iwona\}}
\NormalTok{Lorem ipsum dolor sit amet, consectitur adipiscing elit.}
\end{Highlighting}
\end{Shaded}
\\


\begin{minipage}{0.75000\textwidth}
\begin{center}
\includegraphics[width=1.0\textwidth,height=6.5in,keepaspectratio]{../images/25.png}
\end{center}
\raggedright{}\myfigurewithoutcaption{25}
\end{minipage}\vspace{0.75cm}


\end{longtable}

The three default family font variables and the \LaTeXTT{\textbackslash{}familydefault} variable should not be confused with their respective switch:
\begin{myitemize}
\item{}  \LaTeXTT{\textbackslash{}normalfont}
\item{}  \LaTeXTT{\textbackslash{}rmfamily}
\item{}  \LaTeXTT{\textbackslash{}sffamily}
\item{}  \LaTeXTT{\textbackslash{}ttfamily}
\end{myitemize}

\section{Available LaTeX Fonts}
\label{166}

To choose a font of your liking, please visit \myplainurl{http://www.tug.dk/FontCatalogue/.} Here 
are some common examples. 

Below are some fonts which are installed by default.
\subsubsection{Serif Fonts}
\label{167}

\begin{longtable}{>{\RaggedRight}p{0.28714\linewidth}>{\RaggedRight}p{0.65571\linewidth}} 
{\bfseries \hspace*{0pt}\ignorespaces{}\hspace*{0pt} Abbreviation }&{\bfseries \hspace*{0pt}\ignorespaces{}\hspace*{0pt} Font Name}\endhead  \hspace*{0pt}\ignorespaces{}\hspace*{0pt} cmr &\hspace*{0pt}\ignorespaces{}\hspace*{0pt} Computer Modern Roman (default)\\ \hspace*{0pt}\ignorespaces{}\hspace*{0pt} lmr &\hspace*{0pt}\ignorespaces{}\hspace*{0pt} Latin Modern Roman\\ \hspace*{0pt}\ignorespaces{}\hspace*{0pt} pbk &\hspace*{0pt}\ignorespaces{}\hspace*{0pt} Bookman\\ \hspace*{0pt}\ignorespaces{}\hspace*{0pt} bch &\hspace*{0pt}\ignorespaces{}\hspace*{0pt} Charter\\ \hspace*{0pt}\ignorespaces{}\hspace*{0pt} pnc &\hspace*{0pt}\ignorespaces{}\hspace*{0pt} New Century Schoolbook\\ \hspace*{0pt}\ignorespaces{}\hspace*{0pt} ppl &\hspace*{0pt}\ignorespaces{}\hspace*{0pt} Palatino\\ \hspace*{0pt}\ignorespaces{}\hspace*{0pt} ptm &\hspace*{0pt}\ignorespaces{}\hspace*{0pt} Times 
\end{longtable}

\subsubsection{Sans Serif Fonts}
\label{168}

\begin{longtable}{>{\RaggedRight}p{0.27258\linewidth}>{\RaggedRight}p{0.67028\linewidth}} 
{\bfseries \hspace*{0pt}\ignorespaces{}\hspace*{0pt} Abbreviation }&{\bfseries \hspace*{0pt}\ignorespaces{}\hspace*{0pt} Font Name}\endhead  \hspace*{0pt}\ignorespaces{}\hspace*{0pt} cmss &\hspace*{0pt}\ignorespaces{}\hspace*{0pt} Computer Modern Sans Serif (default)\\ \hspace*{0pt}\ignorespaces{}\hspace*{0pt} lmss &\hspace*{0pt}\ignorespaces{}\hspace*{0pt} Latin Modern Sans Serif\\ \hspace*{0pt}\ignorespaces{}\hspace*{0pt} pag &\hspace*{0pt}\ignorespaces{}\hspace*{0pt} Avant Garde\\ \hspace*{0pt}\ignorespaces{}\hspace*{0pt} phv &\hspace*{0pt}\ignorespaces{}\hspace*{0pt} Helvetica 
\end{longtable}

\subsubsection{Typewriter Fonts}
\label{169}

\begin{longtable}{>{\RaggedRight}p{0.26869\linewidth}>{\RaggedRight}p{0.67417\linewidth}} 
{\bfseries \hspace*{0pt}\ignorespaces{}\hspace*{0pt} Abbreviation }&{\bfseries \hspace*{0pt}\ignorespaces{}\hspace*{0pt} Font Name}\endhead  \hspace*{0pt}\ignorespaces{}\hspace*{0pt} cmtt &\hspace*{0pt}\ignorespaces{}\hspace*{0pt} Computer Modern Typewriter (default)\\ \hspace*{0pt}\ignorespaces{}\hspace*{0pt} lmtt &\hspace*{0pt}\ignorespaces{}\hspace*{0pt} Latin Modern\\ \hspace*{0pt}\ignorespaces{}\hspace*{0pt} pcr &\hspace*{0pt}\ignorespaces{}\hspace*{0pt} Courier 
\end{longtable}


Furthermore, the \myhref{http://www.ctan.org/tex-archive/fonts/bera/}{Bera Mono} (BitStream Vera Mono) and \myhref{http://www.ctan.org/tex-archive/fonts/LuxiMono/}{LuxiMono} fonts were designed to look good when used in conjunction with the Computer Modern serif font.
\\

\TemplateSpaceIndent{$\text{ }${}\textbackslash{}usepackage{$\text{[}$}scaled=0.85{$\text{]}$}\{beramono\}}

\subsubsection{Cursive Fonts}
\label{170}

Since LaTeX has no generic family group for cursive fonts, these fonts are usually assigned to the roman family. 

\begin{longtable}{>{\RaggedRight}p{0.45729\linewidth}>{\RaggedRight}p{0.48557\linewidth}} 
{\bfseries \hspace*{0pt}\ignorespaces{}\hspace*{0pt} Abbreviation }&{\bfseries \hspace*{0pt}\ignorespaces{}\hspace*{0pt} Font Name}\\ \hspace*{0pt}\ignorespaces{}\hspace*{0pt} pzc &\hspace*{0pt}\ignorespaces{}\hspace*{0pt} Zapf Chancery 
\end{longtable}

\subsubsection{Mathematical Formula Fonts}
\label{171}

\begin{longtable}{>{\RaggedRight}p{0.29170\linewidth}>{\RaggedRight}p{0.65116\linewidth}} 
{\bfseries \hspace*{0pt}\ignorespaces{}\hspace*{0pt} Abbreviation }&{\bfseries \hspace*{0pt}\ignorespaces{}\hspace*{0pt} Font Name}\endhead  \hspace*{0pt}\ignorespaces{}\hspace*{0pt} cmm &\hspace*{0pt}\ignorespaces{}\hspace*{0pt} Computer Modern (math italic) \\ \hspace*{0pt}\ignorespaces{}\hspace*{0pt} cmsy &\hspace*{0pt}\ignorespaces{}\hspace*{0pt} Computer Modern (math symbols) \\ \hspace*{0pt}\ignorespaces{}\hspace*{0pt} zplm &\hspace*{0pt}\ignorespaces{}\hspace*{0pt} Palatino (math)  
\end{longtable}

\section{Emphasizing text}
\label{172}

\begin{TemplateInfo}{\danger}{Warning}Do not overuse emphasis in your paragraphs. Emphasis should be reserved for only key terms or other particularly important concepts in a text, and bold text especially used minimally.\end{TemplateInfo}

In order to add some emphasis to a word or a phrase, the simplest way is to use the \LaTeXTT{\textbackslash{}emph\{text\}} command, which usually italicizes the text. Italics may be specified explicitly with \LaTeXTT{\textbackslash{}textit\{text\}}.

\begin{longtable}{p{1.0\linewidth}}
\begin{Shaded}
\begin{Highlighting}[]

\NormalTok{I want to \textbackslash{}emph\{emphasize\} a word.}
\end{Highlighting}
\end{Shaded}
\\



\begin{minipage}{1.0\linewidth}
\begin{center}
\includegraphics[width=1.0\linewidth,height=6.5in,keepaspectratio]{../images/26.png}
\end{center}
\raggedright{}\myfigurewithoutcaption{26}
\end{minipage}\vspace{0.75cm}



\end{longtable}

Note that the \LaTeXTT{\textbackslash{}emph} command is dynamic: if you emphasize a word which is already in an emphasized sentence, it will be reverted to the upright font.

\begin{longtable}{p{1.0\linewidth}}
\begin{Shaded}
\begin{Highlighting}[]

\NormalTok{\textbackslash{}emph\{In this emphasized sentence, there is an emphasized \textbackslash{}emph\{word\} which}
 \NormalTok{looks upright.\}}
\end{Highlighting}
\end{Shaded}
\\

{\itshape \setmainfont[Path=/usr/share/fonts/truetype/cmu/,UprightFont=cmunrm.ttf,BoldFont=cmunbx.ttf,ItalicFont=cmunti.ttf,BoldItalicFont=cmunbi.ttf]{cmunti.ttf}\setmonofont[Path=/usr/share/fonts/truetype/cmu/,UprightFont=cmuntt.ttf,BoldFont=cmuntb.ttf,ItalicFont=cmunit.ttf,BoldItalicFont=cmuntx.ttf]{cmunti.ttf}\itshape In this emphasized sentence, there is an emphasized}\setmainfont[Path=/usr/share/fonts/truetype/cmu/,UprightFont=cmunrm.ttf,BoldFont=cmunbx.ttf,ItalicFont=cmunti.ttf,BoldItalicFont=cmunbi.ttf]{cmunrm.ttf}\setmonofont[Path=/usr/share/fonts/truetype/cmu/,UprightFont=cmuntt.ttf,BoldFont=cmuntb.ttf,ItalicFont=cmunit.ttf,BoldItalicFont=cmuntx.ttf]{cmunrm.ttf}word{\itshape {$\text{ }$}\setmainfont[Path=/usr/share/fonts/truetype/cmu/,UprightFont=cmunrm.ttf,BoldFont=cmunbx.ttf,ItalicFont=cmunti.ttf,BoldItalicFont=cmunbi.ttf]{cmunti.ttf}\setmonofont[Path=/usr/share/fonts/truetype/cmu/,UprightFont=cmuntt.ttf,BoldFont=cmuntb.ttf,ItalicFont=cmunit.ttf,BoldItalicFont=cmuntx.ttf]{cmunti.ttf}\itshape  which looks upright.}\setmainfont[Path=/usr/share/fonts/truetype/cmu/,UprightFont=cmunrm.ttf,BoldFont=cmunbx.ttf,ItalicFont=cmunti.ttf,BoldItalicFont=cmunbi.ttf]{cmunrm.ttf}\setmonofont[Path=/usr/share/fonts/truetype/cmu/,UprightFont=cmuntt.ttf,BoldFont=cmuntb.ttf,ItalicFont=cmunit.ttf,BoldItalicFont=cmuntx.ttf]{cmunrm.ttf}

\end{longtable}

Text may be emphasized more heavily through the use of boldface, particularly for keywords the reader may be trying to find when reading the text. As bold text is generally read before any other text in a paragraph or even on a page, it should be used sparingly. It may also be used in place of italics when using sans-{}serif typefaces to provide a greater contrast with unemphasized text. Bold text can be generated with the \LaTeXTT{\textbackslash{}textbf\{text\}} command.

\begin{longtable}{p{1.0\linewidth}}
\begin{Shaded}
\begin{Highlighting}[]

\NormalTok{\textbackslash{}textbf\{Bold text\} may be used to heavily emphasize very important words or}
 \NormalTok{phrases.}
\end{Highlighting}
\end{Shaded}
\\

{\bfseries \setmainfont[Path=/usr/share/fonts/truetype/cmu/,UprightFont=cmunrm.ttf,BoldFont=cmunbx.ttf,ItalicFont=cmunti.ttf,BoldItalicFont=cmunbi.ttf]{cmunbx.ttf}\setmonofont[Path=/usr/share/fonts/truetype/cmu/,UprightFont=cmuntt.ttf,BoldFont=cmuntb.ttf,ItalicFont=cmunit.ttf,BoldItalicFont=cmuntx.ttf]{cmunbx.ttf}\bfseries Bold text}{$\text{ }$}\setmainfont[Path=/usr/share/fonts/truetype/cmu/,UprightFont=cmunrm.ttf,BoldFont=cmunbx.ttf,ItalicFont=cmunti.ttf,BoldItalicFont=cmunbi.ttf]{cmunrm.ttf}\setmonofont[Path=/usr/share/fonts/truetype/cmu/,UprightFont=cmuntt.ttf,BoldFont=cmuntb.ttf,ItalicFont=cmunit.ttf,BoldItalicFont=cmuntx.ttf]{cmunrm.ttf} may be used to heavily emphasize very important words or phrases.

\end{longtable}
\section{Font encoding}
\label{173}

A {\itshape \setmainfont[Path=/usr/share/fonts/truetype/cmu/,UprightFont=cmunrm.ttf,BoldFont=cmunbx.ttf,ItalicFont=cmunti.ttf,BoldItalicFont=cmunbi.ttf]{cmunti.ttf}\setmonofont[Path=/usr/share/fonts/truetype/cmu/,UprightFont=cmuntt.ttf,BoldFont=cmuntb.ttf,ItalicFont=cmunit.ttf,BoldItalicFont=cmuntx.ttf]{cmunti.ttf}\itshape character}{$\text{ }$}\setmainfont[Path=/usr/share/fonts/truetype/cmu/,UprightFont=cmunrm.ttf,BoldFont=cmunbx.ttf,ItalicFont=cmunti.ttf,BoldItalicFont=cmunbi.ttf]{cmunrm.ttf}\setmonofont[Path=/usr/share/fonts/truetype/cmu/,UprightFont=cmuntt.ttf,BoldFont=cmuntb.ttf,ItalicFont=cmunit.ttf,BoldItalicFont=cmuntx.ttf]{cmunrm.ttf} is a sequence of bytes, and should not be confused with its representation, the {\itshape \setmainfont[Path=/usr/share/fonts/truetype/cmu/,UprightFont=cmunrm.ttf,BoldFont=cmunbx.ttf,ItalicFont=cmunti.ttf,BoldItalicFont=cmunbi.ttf]{cmunti.ttf}\setmonofont[Path=/usr/share/fonts/truetype/cmu/,UprightFont=cmuntt.ttf,BoldFont=cmuntb.ttf,ItalicFont=cmunit.ttf,BoldItalicFont=cmuntx.ttf]{cmunti.ttf}\itshape glyph}\setmainfont[Path=/usr/share/fonts/truetype/cmu/,UprightFont=cmunrm.ttf,BoldFont=cmunbx.ttf,ItalicFont=cmunti.ttf,BoldItalicFont=cmunbi.ttf]{cmunrm.ttf}\setmonofont[Path=/usr/share/fonts/truetype/cmu/,UprightFont=cmuntt.ttf,BoldFont=cmuntb.ttf,ItalicFont=cmunit.ttf,BoldItalicFont=cmuntx.ttf]{cmunrm.ttf}, which is what the reader sees. So the character \textquotesingle{}a\textquotesingle{} has different representations following the used font, for example the upright version, the italic version, various weights and heights, and so on.

Upon compilation, {\ttfamily \setmainfont[Path=/usr/share/fonts/truetype/cmu/,UprightFont=cmunrm.ttf,BoldFont=cmunbx.ttf,ItalicFont=cmunti.ttf,BoldItalicFont=cmunbi.ttf]{cmuntt.ttf}\setmonofont[Path=/usr/share/fonts/truetype/cmu/,UprightFont=cmuntt.ttf,BoldFont=cmuntb.ttf,ItalicFont=cmunit.ttf,BoldItalicFont=cmuntx.ttf]{cmuntt.ttf}\ttfamily tex}{$\text{ }$}\setmainfont[Path=/usr/share/fonts/truetype/cmu/,UprightFont=cmunrm.ttf,BoldFont=cmunbx.ttf,ItalicFont=cmunti.ttf,BoldItalicFont=cmunbi.ttf]{cmunrm.ttf}\setmonofont[Path=/usr/share/fonts/truetype/cmu/,UprightFont=cmuntt.ttf,BoldFont=cmuntb.ttf,ItalicFont=cmunit.ttf,BoldItalicFont=cmuntx.ttf]{cmunrm.ttf} will have to choose the right font glyph for every character. This is what is called {\itshape \setmainfont[Path=/usr/share/fonts/truetype/cmu/,UprightFont=cmunrm.ttf,BoldFont=cmunbx.ttf,ItalicFont=cmunti.ttf,BoldItalicFont=cmunbi.ttf]{cmunti.ttf}\setmonofont[Path=/usr/share/fonts/truetype/cmu/,UprightFont=cmuntt.ttf,BoldFont=cmuntb.ttf,ItalicFont=cmunit.ttf,BoldItalicFont=cmuntx.ttf]{cmunti.ttf}\itshape font encoding}\setmainfont[Path=/usr/share/fonts/truetype/cmu/,UprightFont=cmunrm.ttf,BoldFont=cmunbx.ttf,ItalicFont=cmunti.ttf,BoldItalicFont=cmunbi.ttf]{cmunrm.ttf}\setmonofont[Path=/usr/share/fonts/truetype/cmu/,UprightFont=cmuntt.ttf,BoldFont=cmuntb.ttf,ItalicFont=cmunit.ttf,BoldItalicFont=cmuntx.ttf]{cmunrm.ttf}. 
The default LaTeX font encoding is OT1, the encoding of the original Computer Modern TeX text fonts. It contains only 128 characters, many from ASCII, but leaving out some others and including a number that are not in ASCII. When accented characters are required, TeX creates them by combining a normal character with an accent. While the resulting output looks perfect, this approach has some caveats.

\begin{myitemize}
\item{}  It stops the automatic hyphenation from working inside words containing accented characters.
\item{}  Searches for words with accents in PDFs will fail.
\item{}  Extracting (\textquotesingle{}e.g.\textquotesingle{} copy paste) the umlaut \textquotesingle{}Ä\textquotesingle{} via a PDF viewer actually extracts the two characters  \textquotesingle{}\LaTeXTT{\symbol{34}A}\textquotesingle{}.
\item{}  Besides, some of Latin letters could not be created by combining a normal character with an accent, to say nothing about letters of non-{}Latin alphabets, such as Greek or Cyrillic.
\end{myitemize}


To overcome these shortcomings, several 8-{}bit CM-{}like font sets were created. {\itshape \setmainfont[Path=/usr/share/fonts/truetype/cmu/,UprightFont=cmunrm.ttf,BoldFont=cmunbx.ttf,ItalicFont=cmunti.ttf,BoldItalicFont=cmunbi.ttf]{cmunti.ttf}\setmonofont[Path=/usr/share/fonts/truetype/cmu/,UprightFont=cmuntt.ttf,BoldFont=cmuntb.ttf,ItalicFont=cmunit.ttf,BoldItalicFont=cmuntx.ttf]{cmunti.ttf}\itshape Extended Cork}{$\text{ }$}\setmainfont[Path=/usr/share/fonts/truetype/cmu/,UprightFont=cmunrm.ttf,BoldFont=cmunbx.ttf,ItalicFont=cmunti.ttf,BoldItalicFont=cmunbi.ttf]{cmunrm.ttf}\setmonofont[Path=/usr/share/fonts/truetype/cmu/,UprightFont=cmuntt.ttf,BoldFont=cmuntb.ttf,ItalicFont=cmunit.ttf,BoldItalicFont=cmuntx.ttf]{cmunrm.ttf} (EC) fonts in T1 encoding contains letters and punctuation characters for {\itshape \setmainfont[Path=/usr/share/fonts/truetype/cmu/,UprightFont=cmunrm.ttf,BoldFont=cmunbx.ttf,ItalicFont=cmunti.ttf,BoldItalicFont=cmunbi.ttf]{cmunti.ttf}\setmonofont[Path=/usr/share/fonts/truetype/cmu/,UprightFont=cmuntt.ttf,BoldFont=cmuntb.ttf,ItalicFont=cmunit.ttf,BoldItalicFont=cmuntx.ttf]{cmunti.ttf}\itshape most of the European languages}{$\text{ }$}\setmainfont[Path=/usr/share/fonts/truetype/cmu/,UprightFont=cmunrm.ttf,BoldFont=cmunbx.ttf,ItalicFont=cmunti.ttf,BoldItalicFont=cmunbi.ttf]{cmunrm.ttf}\setmonofont[Path=/usr/share/fonts/truetype/cmu/,UprightFont=cmuntt.ttf,BoldFont=cmuntb.ttf,ItalicFont=cmunit.ttf,BoldItalicFont=cmuntx.ttf]{cmunrm.ttf} based on Latin script. The LH font set contains letters necessary to typeset documents in languages using Cyrillic script. Because of the large number of Cyrillic glyphs, they are arranged into four font encodings—T2A, T2B, T2C, and X2. The CB bundle contains fonts in LGR encoding for the composition of Greek text. By using these fonts you can improve/enable hyphenation in non-{}English documents. Another advantage of using new CM-{}like fonts is that they provide fonts of CM families in all weights, shapes, and optically scaled font sizes.

All this is not possible with {\itshape \setmainfont[Path=/usr/share/fonts/truetype/cmu/,UprightFont=cmunrm.ttf,BoldFont=cmunbx.ttf,ItalicFont=cmunti.ttf,BoldItalicFont=cmunbi.ttf]{cmunti.ttf}\setmonofont[Path=/usr/share/fonts/truetype/cmu/,UprightFont=cmuntt.ttf,BoldFont=cmuntb.ttf,ItalicFont=cmunit.ttf,BoldItalicFont=cmuntx.ttf]{cmunti.ttf}\itshape OT1}\setmainfont[Path=/usr/share/fonts/truetype/cmu/,UprightFont=cmunrm.ttf,BoldFont=cmunbx.ttf,ItalicFont=cmunti.ttf,BoldItalicFont=cmunbi.ttf]{cmunrm.ttf}\setmonofont[Path=/usr/share/fonts/truetype/cmu/,UprightFont=cmuntt.ttf,BoldFont=cmuntb.ttf,ItalicFont=cmunit.ttf,BoldItalicFont=cmuntx.ttf]{cmunrm.ttf}; that\textquotesingle{}s why you may want to change the font encoding of your document.

\begin{TemplateInfo}{\danger}{Warning}If you do not have a specific font encoding issue ({\itshape \setmainfont[Path=/usr/share/fonts/truetype/cmu/,UprightFont=cmunrm.ttf,BoldFont=cmunbx.ttf,ItalicFont=cmunti.ttf,BoldItalicFont=cmunbi.ttf]{cmunti.ttf}\setmonofont[Path=/usr/share/fonts/truetype/cmu/,UprightFont=cmuntt.ttf,BoldFont=cmuntb.ttf,ItalicFont=cmunit.ttf,BoldItalicFont=cmuntx.ttf]{cmunti.ttf}\itshape e.g.}{$\text{ }$}\setmainfont[Path=/usr/share/fonts/truetype/cmu/,UprightFont=cmunrm.ttf,BoldFont=cmunbx.ttf,ItalicFont=cmunti.ttf,BoldItalicFont=cmunbi.ttf]{cmunrm.ttf}\setmonofont[Path=/usr/share/fonts/truetype/cmu/,UprightFont=cmuntt.ttf,BoldFont=cmuntb.ttf,ItalicFont=cmunit.ttf,BoldItalicFont=cmuntx.ttf]{cmunrm.ttf} writing English only), there is no need for T1. Sticking to the default font encoding is not a problem.\end{TemplateInfo}

Note that changing the font encoding will have some requirements over the fonts being used. The default Computer Modern font does not support T1. You will need Computer Modern Super (\LaTeXTT{cm-{}super}) or Latin Modern (\LaTeXTT{lmodern}), which are Computer Modern-{}like fonts with T1 support. If you have none of these, it is quite frequent (depends on your TeX installation) that {\ttfamily \setmainfont[Path=/usr/share/fonts/truetype/cmu/,UprightFont=cmunrm.ttf,BoldFont=cmunbx.ttf,ItalicFont=cmunti.ttf,BoldItalicFont=cmunbi.ttf]{cmuntt.ttf}\setmonofont[Path=/usr/share/fonts/truetype/cmu/,UprightFont=cmuntt.ttf,BoldFont=cmuntb.ttf,ItalicFont=cmunit.ttf,BoldItalicFont=cmuntx.ttf]{cmuntt.ttf}\ttfamily tex}{$\text{ }$}\setmainfont[Path=/usr/share/fonts/truetype/cmu/,UprightFont=cmunrm.ttf,BoldFont=cmunbx.ttf,ItalicFont=cmunti.ttf,BoldItalicFont=cmunbi.ttf]{cmunrm.ttf}\setmonofont[Path=/usr/share/fonts/truetype/cmu/,UprightFont=cmuntt.ttf,BoldFont=cmuntb.ttf,ItalicFont=cmunit.ttf,BoldItalicFont=cmuntx.ttf]{cmunrm.ttf} chooses a Type3 font such as the Type3 EC, which is a bitmap font. Bitmap fonts look rather ugly when zoomed or printed.

\begin{TemplateInfo}{\danger}{Warning}
If after using T1 you find yourself with very low quality fonts, it is because there is no appropriate font installed on your system. Install either \LaTeXTT{cm-{}super} or \LaTeXTT{lmodern}. This is a very common error!
\end{TemplateInfo}

The \LaTeXTT{fontenc} package tells LaTeX what font encoding to use. Font encoding is set with:
\begin{Shaded}
\begin{Highlighting}[]

\NormalTok{\textbackslash{}usepackage['encoding']\{fontenc\}}
\end{Highlighting}
\end{Shaded}

where \LaTeXTT{encoding} is the font encoding. It is possible to load several encodings simultaneously.

There is nothing to change in your document to use CM Super fonts (assuming they are installed), they will get loaded automatically if you use T1 encoding. For {\ttfamily \setmainfont[Path=/usr/share/fonts/truetype/cmu/,UprightFont=cmunrm.ttf,BoldFont=cmunbx.ttf,ItalicFont=cmunti.ttf,BoldItalicFont=cmunbi.ttf]{cmuntt.ttf}\setmonofont[Path=/usr/share/fonts/truetype/cmu/,UprightFont=cmuntt.ttf,BoldFont=cmuntb.ttf,ItalicFont=cmunit.ttf,BoldItalicFont=cmuntx.ttf]{cmuntt.ttf}\ttfamily lmodern}\setmainfont[Path=/usr/share/fonts/truetype/cmu/,UprightFont=cmunrm.ttf,BoldFont=cmunbx.ttf,ItalicFont=cmunti.ttf,BoldItalicFont=cmunbi.ttf]{cmunrm.ttf}\setmonofont[Path=/usr/share/fonts/truetype/cmu/,UprightFont=cmuntt.ttf,BoldFont=cmuntb.ttf,ItalicFont=cmunit.ttf,BoldItalicFont=cmuntx.ttf]{cmunrm.ttf}, you will need to load the package after the T1 encoding has been set:

\begin{Shaded}
\begin{Highlighting}[]

\NormalTok{\textbackslash{}usepackage[T1]\{fontenc\}}
\NormalTok{\textbackslash{}usepackage\{lmodern\}}
\end{Highlighting}
\end{Shaded}


The package \LaTeXTT{ae} (almost European) is obsolete. It provided some workarounds for hyphenation of words with special characters. These are not necessary any more with fonts like lmodern. Using the ae package leads to text encoding problems in PDF files generated via {\ttfamily \setmainfont[Path=/usr/share/fonts/truetype/cmu/,UprightFont=cmunrm.ttf,BoldFont=cmunbx.ttf,ItalicFont=cmunti.ttf,BoldItalicFont=cmunbi.ttf]{cmuntt.ttf}\setmonofont[Path=/usr/share/fonts/truetype/cmu/,UprightFont=cmuntt.ttf,BoldFont=cmuntb.ttf,ItalicFont=cmunit.ttf,BoldItalicFont=cmuntx.ttf]{cmuntt.ttf}\ttfamily pdflatex}{$\text{ }$}\setmainfont[Path=/usr/share/fonts/truetype/cmu/,UprightFont=cmunrm.ttf,BoldFont=cmunbx.ttf,ItalicFont=cmunti.ttf,BoldItalicFont=cmunbi.ttf]{cmunrm.ttf}\setmonofont[Path=/usr/share/fonts/truetype/cmu/,UprightFont=cmuntt.ttf,BoldFont=cmuntb.ttf,ItalicFont=cmunit.ttf,BoldItalicFont=cmuntx.ttf]{cmunrm.ttf} (e.g. text extraction and searching), besides typographic issues.
\section{Font styles}
\label{174}

Each family has its own font characteristics (such as italic and bold), also known as font styles, or font properties.

Font styles are usually implemented with different font files. So it is possible to build a new font family by specifying the font styles of different font families.
\subsection{Shapes}
\label{175}

The following table lists the commands you will need to access the typical font shapes:

{\scalefont{0.86850}\begin{longtable}{|>{\RaggedRight}p{0.22261\linewidth}|>{\RaggedRight}p{0.22261\linewidth}|>{\RaggedRight}p{0.21790\linewidth}|>{\RaggedRight}p{0.22261\linewidth}|} \hline 
{\bfseries \hspace*{0pt}\ignorespaces{}\hspace*{0pt} LaTeX command}&{\bfseries \hspace*{0pt}\ignorespaces{}\hspace*{0pt} Equivalent to}&{\bfseries \hspace*{0pt}\ignorespaces{}\hspace*{0pt} Output style}&{\bfseries \hspace*{0pt}\ignorespaces{}\hspace*{0pt} Remarks}\endhead  \hline \hspace*{0pt}\ignorespaces{}\hspace*{0pt} \LaTeXTT{\textbackslash{}textnormal\{...\}}&\hspace*{0pt}\ignorespaces{}\hspace*{0pt} \LaTeXTT{\{\textbackslash{}normalfont ...\}}&\hspace*{0pt}\ignorespaces{}\hspace*{0pt} document font family&\hspace*{0pt}\ignorespaces{}\hspace*{0pt} This is the default or normal font.\\ \hline \hspace*{0pt}\ignorespaces{}\hspace*{0pt} \LaTeXTT{\textbackslash{}emph\{...\}}&\hspace*{0pt}\ignorespaces{}\hspace*{0pt} \LaTeXTT{\{\textbackslash{}em ...\}}&\hspace*{0pt}\ignorespaces{}\hspace*{0pt} {\itshape \setmainfont[Path=/usr/share/fonts/truetype/cmu/,UprightFont=cmunrm.ttf,BoldFont=cmunbx.ttf,ItalicFont=cmunti.ttf,BoldItalicFont=cmunbi.ttf]{cmunti.ttf}\setmonofont[Path=/usr/share/fonts/truetype/cmu/,UprightFont=cmuntt.ttf,BoldFont=cmuntb.ttf,ItalicFont=cmunit.ttf,BoldItalicFont=cmuntx.ttf]{cmunti.ttf}\itshape emphasis}&\hspace*{0pt}\ignorespaces{}\hspace*{0pt}{$\text{ }$}\setmainfont[Path=/usr/share/fonts/truetype/cmu/,UprightFont=cmunrm.ttf,BoldFont=cmunbx.ttf,ItalicFont=cmunti.ttf,BoldItalicFont=cmunbi.ttf]{cmunrm.ttf}\setmonofont[Path=/usr/share/fonts/truetype/cmu/,UprightFont=cmuntt.ttf,BoldFont=cmuntb.ttf,ItalicFont=cmunit.ttf,BoldItalicFont=cmuntx.ttf]{cmunrm.ttf} Typically italics. Using emph\{\} inside of italic text removes the italics on the emphasized text.\\ \hline \hspace*{0pt}\ignorespaces{}\hspace*{0pt} \LaTeXTT{\textbackslash{}textrm\{...\}}&\hspace*{0pt}\ignorespaces{}\hspace*{0pt} \LaTeXTT{\{\textbackslash{}rmfamily ...\}}&\hspace*{0pt}\ignorespaces{}\hspace*{0pt} roman font family&\hspace*{0pt}\ignorespaces{}\hspace*{0pt}\\ \hline \hspace*{0pt}\ignorespaces{}\hspace*{0pt} \LaTeXTT{\textbackslash{}textsf\{...\}}&\hspace*{0pt}\ignorespaces{}\hspace*{0pt} \LaTeXTT{\{\textbackslash{}sffamily ...\}}&\hspace*{0pt}\ignorespaces{}\hspace*{0pt} sans serif font family&\hspace*{0pt}\ignorespaces{}\hspace*{0pt}\\ \hline \hspace*{0pt}\ignorespaces{}\hspace*{0pt} \LaTeXTT{\textbackslash{}texttt\{...\}}&\hspace*{0pt}\ignorespaces{}\hspace*{0pt} \LaTeXTT{\{\textbackslash{}ttfamily ...\}}&\hspace*{0pt}\ignorespaces{}\hspace*{0pt} teletypefont family&\hspace*{0pt}\ignorespaces{}\hspace*{0pt} This is a fixed-{}width or monospace font.\\ \hline \hspace*{0pt}\ignorespaces{}\hspace*{0pt} \LaTeXTT{\textbackslash{}textup\{...\}}&\hspace*{0pt}\ignorespaces{}\hspace*{0pt} \LaTeXTT{\{\textbackslash{}upshape ...\}}&\hspace*{0pt}\ignorespaces{}\hspace*{0pt} upright shape&\hspace*{0pt}\ignorespaces{}\hspace*{0pt} The same as the normal typeface.\\ \hline \hspace*{0pt}\ignorespaces{}\hspace*{0pt} \LaTeXTT{\textbackslash{}textit\{...\}}&\hspace*{0pt}\ignorespaces{}\hspace*{0pt} \LaTeXTT{\{\textbackslash{}itshape ...\}}&\hspace*{0pt}\ignorespaces{}\hspace*{0pt} italic shape&\hspace*{0pt}\ignorespaces{}\hspace*{0pt} \\ \hline \hspace*{0pt}\ignorespaces{}\hspace*{0pt} \LaTeXTT{\textbackslash{}textsl\{...\}}&\hspace*{0pt}\ignorespaces{}\hspace*{0pt} \LaTeXTT{\{\textbackslash{}slshape ...\}}&\hspace*{0pt}\ignorespaces{}\hspace*{0pt} slanted shape&\hspace*{0pt}\ignorespaces{}\hspace*{0pt} A skewed version of the normal typeface (similar to, but slightly different from, italics).\\ \hline \hspace*{0pt}\ignorespaces{}\hspace*{0pt} \LaTeXTT{\textbackslash{}textsc\{...\}}&\hspace*{0pt}\ignorespaces{}\hspace*{0pt} \LaTeXTT{\{\textbackslash{}scshape ...\}}&\hspace*{0pt}\ignorespaces{}\hspace*{0pt} Small Capitals&\hspace*{0pt}\ignorespaces{}\hspace*{0pt}\\ \hline \hspace*{0pt}\ignorespaces{}\hspace*{0pt} \LaTeXTT{\textbackslash{}uppercase\{...\}}&\hspace*{0pt}\ignorespaces{}\hspace*{0pt} &\hspace*{0pt}\ignorespaces{}\hspace*{0pt} uppercase (all caps)&\hspace*{0pt}\ignorespaces{}\hspace*{0pt} Also \LaTeXTT{\textbackslash{}lowercase}. There are some caveats, though; see \myhref{http://www.tex.ac.uk/cgi-bin/texfaq2html?label=casechange}{here}.\\ \hline \hspace*{0pt}\ignorespaces{}\hspace*{0pt} \LaTeXTT{\textbackslash{}textbf\{...\}}&\hspace*{0pt}\ignorespaces{}\hspace*{0pt} \LaTeXTT{\{\textbackslash{}bfseries ...\}}&\hspace*{0pt}\ignorespaces{}\hspace*{0pt} bold&\hspace*{0pt}\ignorespaces{}\hspace*{0pt}\\ \hline \hspace*{0pt}\ignorespaces{}\hspace*{0pt} \LaTeXTT{\textbackslash{}textmd\{...\}}&\hspace*{0pt}\ignorespaces{}\hspace*{0pt} \LaTeXTT{\{\textbackslash{}mdseries ...\}}&\hspace*{0pt}\ignorespaces{}\hspace*{0pt} medium weight&\hspace*{0pt}\ignorespaces{}\hspace*{0pt} A font weight in between normal and bold.\\ \hline \hspace*{0pt}\ignorespaces{}\hspace*{0pt} \LaTeXTT{\textbackslash{}textlf\{...\}}&\hspace*{0pt}\ignorespaces{}\hspace*{0pt} \LaTeXTT{\{\textbackslash{}lfseries ...\}}&\hspace*{0pt}\ignorespaces{}\hspace*{0pt} light&\hspace*{0pt}\ignorespaces{}\hspace*{0pt} A font weight lighter than normal. Not supported by all typefaces.\\ \hline 
\end{longtable}
}

The commands in column two are not entirely equivalent to the commands in column one: They do not correct spacing after the selected font style has ended. The commands in column one are therefore in general recommended.

You may have noticed the absence of underline. This is because underlining is not recommended for typographic reasons (it weighs the text down). You should use {\itshape \setmainfont[Path=/usr/share/fonts/truetype/cmu/,UprightFont=cmunrm.ttf,BoldFont=cmunbx.ttf,ItalicFont=cmunti.ttf,BoldItalicFont=cmunbi.ttf]{cmunti.ttf}\setmonofont[Path=/usr/share/fonts/truetype/cmu/,UprightFont=cmuntt.ttf,BoldFont=cmuntb.ttf,ItalicFont=cmunit.ttf,BoldItalicFont=cmuntx.ttf]{cmunti.ttf}\itshape emph}{$\text{ }$}\setmainfont[Path=/usr/share/fonts/truetype/cmu/,UprightFont=cmunrm.ttf,BoldFont=cmunbx.ttf,ItalicFont=cmunti.ttf,BoldItalicFont=cmunbi.ttf]{cmunrm.ttf}\setmonofont[Path=/usr/share/fonts/truetype/cmu/,UprightFont=cmuntt.ttf,BoldFont=cmuntb.ttf,ItalicFont=cmunit.ttf,BoldItalicFont=cmuntx.ttf]{cmunrm.ttf} instead. However underlining text provides a useful extra form of emphasis during the editing process, for example to draw attention to changes. Although underlining is available via the \LaTeXTT{\textbackslash{}underline\{...\}} command, text underlined in this way will not break properly. This functionality has to be added with the \LaTeXTT{ulem} (underline emphasis) package. Stick \LaTeXTT{\textbackslash{}usepackage\{ulem\}} in your preamble. By default, this overrides the \LaTeXTT{\textbackslash{}emph} command with the underline rather than the italic style. It is unlikely that you wish this to be the desired effect, so it is better to stop \LaTeXTT{ulem} taking over \LaTeXTT{\textbackslash{}emph} and simply call the underline command as and when it is needed.

\begin{myitemize}
\item{}  To restore the usual \LaTeXTT{\textbackslash{}emph} formatting, add \LaTeXTT{\textbackslash{}normalem} straight after the \LaTeXTT{document} environment begins. Alternatively, use \LaTeXTT{\textbackslash{}usepackage{$\text{[}$}normalem{$\text{]}$}\{ulem\}}.
\item{}  To underline, use \LaTeXTT{\textbackslash{}uline\{...\}} along with \LaTeXTT{\textbackslash{}usepackage{$\text{[}$}normalem{$\text{]}$}\{ulem\}}..
\item{}  To add a wavy underline, use \LaTeXTT{\textbackslash{}uwave\{...\}} along with \LaTeXTT{\textbackslash{}usepackage{$\text{[}$}normalem{$\text{]}$}\{ulem\}}..
\item{}  For a strike-{}out (strikethrough), use \LaTeXTT{\textbackslash{}sout\{...\}} along with \LaTeXTT{\textbackslash{}usepackage{$\text{[}$}normalem{$\text{]}$}\{ulem\}}..
\item{}  For a slash through each individual character \LaTeXTT{\textbackslash{}xout\{...\}} along with \LaTeXTT{\textbackslash{}usepackage{$\text{[}$}normalem{$\text{]}$}\{ulem\}}.
\end{myitemize}


Some font styles are not compatible one with the other. But some extra packages will fill this hole.
For bold small capitals, you might want to use:
\begin{Shaded}
\begin{Highlighting}[]

\NormalTok{\textbackslash{}usepackage\{bold-extra\}}
\CommentTok{% ...}
\NormalTok{\textbackslash{}textsc\{ \textbackslash{}textbf\{This is bold small capitals\} \}}
\end{Highlighting}
\end{Shaded}

\subsection{Sizing text}
\label{176}

To apply different font sizes, simply follow the commands on this table:

\begin{longtable}{|>{\RaggedRight}p{0.57952\linewidth}|>{\RaggedRight}p{0.36333\linewidth}|} \hline 
{\bfseries \hspace*{0pt}\ignorespaces{}\hspace*{0pt} Command}&{\bfseries \hspace*{0pt}\ignorespaces{}\hspace*{0pt} Output}\endhead  \hline \hspace*{0pt}\ignorespaces{}\hspace*{0pt}\LaTeXTT{\textbackslash{}tiny}&\hspace*{0pt}\ignorespaces{}\hspace*{0pt}sample text\\ \hline \hspace*{0pt}\ignorespaces{}\hspace*{0pt}\LaTeXTT{\textbackslash{}scriptsize}&\hspace*{0pt}\ignorespaces{}\hspace*{0pt}sample text\\ \hline \hspace*{0pt}\ignorespaces{}\hspace*{0pt}\LaTeXTT{\textbackslash{}footnotesize}&\hspace*{0pt}\ignorespaces{}\hspace*{0pt}sample text\\ \hline \hspace*{0pt}\ignorespaces{}\hspace*{0pt}\LaTeXTT{\textbackslash{}small}&\hspace*{0pt}\ignorespaces{}\hspace*{0pt}sample text\\ \hline \hspace*{0pt}\ignorespaces{}\hspace*{0pt}\LaTeXTT{\textbackslash{}normalsize}&\hspace*{0pt}\ignorespaces{}\hspace*{0pt}sample text\\ \hline \hspace*{0pt}\ignorespaces{}\hspace*{0pt}\LaTeXTT{\textbackslash{}large}&\hspace*{0pt}\ignorespaces{}\hspace*{0pt}sample text\\ \hline \hspace*{0pt}\ignorespaces{}\hspace*{0pt}\LaTeXTT{\textbackslash{}Large}&\hspace*{0pt}\ignorespaces{}\hspace*{0pt}sample text\\ \hline \hspace*{0pt}\ignorespaces{}\hspace*{0pt}\LaTeXTT{\textbackslash{}LARGE}&\hspace*{0pt}\ignorespaces{}\hspace*{0pt}sample text\\ \hline \hspace*{0pt}\ignorespaces{}\hspace*{0pt}\LaTeXTT{\textbackslash{}huge}&\hspace*{0pt}\ignorespaces{}\hspace*{0pt}sample text\\ \hline \hspace*{0pt}\ignorespaces{}\hspace*{0pt}\LaTeXTT{\textbackslash{}Huge}&\hspace*{0pt}\ignorespaces{}\hspace*{0pt}sample text\\ \hline 
\end{longtable}


These commands change the size within a given scope, so for instance \LaTeXTT{\{\textbackslash{}Large some words\}} will change the size of only \LaTeXTT{some words}, and does not affect the font in the rest of the document. It will work for most parts of the text.

\begin{Shaded}
\begin{Highlighting}[]

\NormalTok{\{\textbackslash{}Large\textbackslash{}tableofcontents\}}
\end{Highlighting}
\end{Shaded}


These commands cannot be used in math mode. However, part of a formula may be set in a different size by using an \textbackslash{}mbox command containing the size command.
The new size takes effect immediately after the size command; if an entire paragraph or unit is set in a certain size, the size command should include the blank line or the \LaTeXTT{\textbackslash{}end\{...\}} which delimits the unit.

The default for  \LaTeXTT{\textbackslash{}normalsize} is 10{\mbox{$~$}}point (option \LaTeXTT{10pt}), but it may differ for some Document Styles or their options. The actual size produced by these commands also depends on the Document Style and, in some styles, more than one of these size commands may produce the same actual size.

Note that the font size definitions are set by the document class.  Depending on the document style the actual font size may differ from that listed above.  And not every document class has unique sizes for all 10 size commands.

{\scriptsize{}
\begin{longtable}{|>{\RaggedRight}p{0.14937\linewidth}|>{\RaggedRight}p{0.08829\linewidth}|>{\RaggedRight}p{0.08829\linewidth}|>{\RaggedRight}p{0.08829\linewidth}|>{\RaggedRight}p{0.08829\linewidth}|>{\RaggedRight}p{0.08829\linewidth}|>{\RaggedRight}p{0.08829\linewidth}|>{\RaggedRight}p{0.09232\linewidth}|} \hline 
\multicolumn{8}{|>{\RaggedRight}p{0.97143\linewidth}|}{{\bfseries \hspace*{0pt}\ignorespaces{}\hspace*{0pt} Absolute Point Sizes}}\\ \hline \multirow{2}{\linewidth}{{\bfseries \hspace*{0pt}\ignorespaces{}\hspace*{0pt} size }}&\multicolumn{3}{|>{\RaggedRight}p{0.28143\linewidth}|}{{\bfseries \hspace*{0pt}\ignorespaces{}\hspace*{0pt} standard classes (except {\itshape \setmainfont[Path=/usr/share/fonts/truetype/cmu/,UprightFont=cmunrm.ttf,BoldFont=cmunbx.ttf,ItalicFont=cmunti.ttf,BoldItalicFont=cmunbi.ttf]{cmunti.ttf}\setmonofont[Path=/usr/share/fonts/truetype/cmu/,UprightFont=cmuntt.ttf,BoldFont=cmuntb.ttf,ItalicFont=cmunit.ttf,BoldItalicFont=cmuntx.ttf]{cmunti.ttf}\itshape slides}\setmainfont[Path=/usr/share/fonts/truetype/cmu/,UprightFont=cmunrm.ttf,BoldFont=cmunbx.ttf,ItalicFont=cmunti.ttf,BoldItalicFont=cmunbi.ttf]{cmunrm.ttf}\setmonofont[Path=/usr/share/fonts/truetype/cmu/,UprightFont=cmuntt.ttf,BoldFont=cmuntb.ttf,ItalicFont=cmunit.ttf,BoldItalicFont=cmuntx.ttf]{cmunrm.ttf}), beamer}}&\multicolumn{3}{|>{\RaggedRight}p{0.28143\linewidth}|}{{\bfseries \hspace*{0pt}\ignorespaces{}\hspace*{0pt} AMS classes, {\itshape \setmainfont[Path=/usr/share/fonts/truetype/cmu/,UprightFont=cmunrm.ttf,BoldFont=cmunbx.ttf,ItalicFont=cmunti.ttf,BoldItalicFont=cmunbi.ttf]{cmunti.ttf}\setmonofont[Path=/usr/share/fonts/truetype/cmu/,UprightFont=cmuntt.ttf,BoldFont=cmuntb.ttf,ItalicFont=cmunit.ttf,BoldItalicFont=cmuntx.ttf]{cmunti.ttf}\itshape memoir}}}&\multirow{2}{\linewidth}{{\bfseries \hspace*{0pt}\ignorespaces{}\hspace*{0pt}{$\text{ }$}\setmainfont[Path=/usr/share/fonts/truetype/cmu/,UprightFont=cmunrm.ttf,BoldFont=cmunbx.ttf,ItalicFont=cmunti.ttf,BoldItalicFont=cmunbi.ttf]{cmunrm.ttf}\setmonofont[Path=/usr/share/fonts/truetype/cmu/,UprightFont=cmuntt.ttf,BoldFont=cmuntb.ttf,ItalicFont=cmunit.ttf,BoldItalicFont=cmuntx.ttf]{cmunrm.ttf} {\itshape \setmainfont[Path=/usr/share/fonts/truetype/cmu/,UprightFont=cmunrm.ttf,BoldFont=cmunbx.ttf,ItalicFont=cmunti.ttf,BoldItalicFont=cmunbi.ttf]{cmunti.ttf}\setmonofont[Path=/usr/share/fonts/truetype/cmu/,UprightFont=cmuntt.ttf,BoldFont=cmuntb.ttf,ItalicFont=cmunit.ttf,BoldItalicFont=cmuntx.ttf]{cmunti.ttf}\itshape slides}}}\\ \cline{2-2}\cline{3-3}\cline{4-4}\cline{5-5}\cline{6-6}\cline{7-7} \multicolumn{1}{|c|}{}&{\bfseries \hspace*{0pt}\ignorespaces{}\hspace*{0pt}{$\text{ }$}\setmainfont[Path=/usr/share/fonts/truetype/cmu/,UprightFont=cmunrm.ttf,BoldFont=cmunbx.ttf,ItalicFont=cmunti.ttf,BoldItalicFont=cmunbi.ttf]{cmunrm.ttf}\setmonofont[Path=/usr/share/fonts/truetype/cmu/,UprightFont=cmuntt.ttf,BoldFont=cmuntb.ttf,ItalicFont=cmunit.ttf,BoldItalicFont=cmuntx.ttf]{cmunrm.ttf} {$\text{[}$}10pt{$\text{]}$} }&{\bfseries \hspace*{0pt}\ignorespaces{}\hspace*{0pt} {$\text{[}$}11pt{$\text{]}$} }&{\bfseries \hspace*{0pt}\ignorespaces{}\hspace*{0pt} {$\text{[}$}12pt{$\text{]}$}}&{\bfseries \hspace*{0pt}\ignorespaces{}\hspace*{0pt} {$\text{[}$}10pt{$\text{]}$} }&{\bfseries \hspace*{0pt}\ignorespaces{}\hspace*{0pt} {$\text{[}$}11pt{$\text{]}$} }&{\bfseries \hspace*{0pt}\ignorespaces{}\hspace*{0pt} {$\text{[}$}12pt{$\text{]}$}}&\multicolumn{1}{|c|}{}\\ \cline{1-1}\cline{2-2}\cline{3-3}\cline{4-4}\cline{5-5}\cline{6-6}\cline{7-7} \hspace*{0pt}\ignorespaces{}\hspace*{0pt}{\ttfamily \setmainfont[Path=/usr/share/fonts/truetype/cmu/,UprightFont=cmunrm.ttf,BoldFont=cmunbx.ttf,ItalicFont=cmunti.ttf,BoldItalicFont=cmunbi.ttf]{cmuntt.ttf}\setmonofont[Path=/usr/share/fonts/truetype/cmu/,UprightFont=cmuntt.ttf,BoldFont=cmuntb.ttf,ItalicFont=cmunit.ttf,BoldItalicFont=cmuntx.ttf]{cmuntt.ttf}\ttfamily \textbackslash{}tiny}&\hspace*{0pt}\ignorespaces{}\hspace*{0pt}{$\text{ }$}\setmainfont[Path=/usr/share/fonts/truetype/cmu/,UprightFont=cmunrm.ttf,BoldFont=cmunbx.ttf,ItalicFont=cmunti.ttf,BoldItalicFont=cmunbi.ttf]{cmunrm.ttf}\setmonofont[Path=/usr/share/fonts/truetype/cmu/,UprightFont=cmuntt.ttf,BoldFont=cmuntb.ttf,ItalicFont=cmunit.ttf,BoldItalicFont=cmuntx.ttf]{cmunrm.ttf}  5 &\hspace*{0pt}\ignorespaces{}\hspace*{0pt}  6   &\hspace*{0pt}\ignorespaces{}\hspace*{0pt}  6&\hspace*{0pt}\ignorespaces{}\hspace*{0pt}  6 &\hspace*{0pt}\ignorespaces{}\hspace*{0pt}  7   &\hspace*{0pt}\ignorespaces{}\hspace*{0pt}  8&\hspace*{0pt}\ignorespaces{}\hspace*{0pt} 13.82\endhead  \hline \hspace*{0pt}\ignorespaces{}\hspace*{0pt}{\ttfamily \setmainfont[Path=/usr/share/fonts/truetype/cmu/,UprightFont=cmunrm.ttf,BoldFont=cmunbx.ttf,ItalicFont=cmunti.ttf,BoldItalicFont=cmunbi.ttf]{cmuntt.ttf}\setmonofont[Path=/usr/share/fonts/truetype/cmu/,UprightFont=cmuntt.ttf,BoldFont=cmuntb.ttf,ItalicFont=cmunit.ttf,BoldItalicFont=cmuntx.ttf]{cmuntt.ttf}\ttfamily \textbackslash{}scriptsize}&\hspace*{0pt}\ignorespaces{}\hspace*{0pt}{$\text{ }$}\setmainfont[Path=/usr/share/fonts/truetype/cmu/,UprightFont=cmunrm.ttf,BoldFont=cmunbx.ttf,ItalicFont=cmunti.ttf,BoldItalicFont=cmunbi.ttf]{cmunrm.ttf}\setmonofont[Path=/usr/share/fonts/truetype/cmu/,UprightFont=cmuntt.ttf,BoldFont=cmuntb.ttf,ItalicFont=cmunit.ttf,BoldItalicFont=cmuntx.ttf]{cmunrm.ttf}  7 &\hspace*{0pt}\ignorespaces{}\hspace*{0pt}  8   &\hspace*{0pt}\ignorespaces{}\hspace*{0pt}   8&\hspace*{0pt}\ignorespaces{}\hspace*{0pt}  7 &\hspace*{0pt}\ignorespaces{}\hspace*{0pt}  8   &\hspace*{0pt}\ignorespaces{}\hspace*{0pt}   9&\hspace*{0pt}\ignorespaces{}\hspace*{0pt} 16.59\\ \hline \hspace*{0pt}\ignorespaces{}\hspace*{0pt}{\ttfamily \setmainfont[Path=/usr/share/fonts/truetype/cmu/,UprightFont=cmunrm.ttf,BoldFont=cmunbx.ttf,ItalicFont=cmunti.ttf,BoldItalicFont=cmunbi.ttf]{cmuntt.ttf}\setmonofont[Path=/usr/share/fonts/truetype/cmu/,UprightFont=cmuntt.ttf,BoldFont=cmuntb.ttf,ItalicFont=cmunit.ttf,BoldItalicFont=cmuntx.ttf]{cmuntt.ttf}\ttfamily \textbackslash{}footnotesize}&\hspace*{0pt}\ignorespaces{}\hspace*{0pt}{$\text{ }$}\setmainfont[Path=/usr/share/fonts/truetype/cmu/,UprightFont=cmunrm.ttf,BoldFont=cmunbx.ttf,ItalicFont=cmunti.ttf,BoldItalicFont=cmunbi.ttf]{cmunrm.ttf}\setmonofont[Path=/usr/share/fonts/truetype/cmu/,UprightFont=cmuntt.ttf,BoldFont=cmuntb.ttf,ItalicFont=cmunit.ttf,BoldItalicFont=cmuntx.ttf]{cmunrm.ttf}  8 &\hspace*{0pt}\ignorespaces{}\hspace*{0pt}   9  &\hspace*{0pt}\ignorespaces{}\hspace*{0pt} 10&\hspace*{0pt}\ignorespaces{}\hspace*{0pt}  8 &\hspace*{0pt}\ignorespaces{}\hspace*{0pt}   9  &\hspace*{0pt}\ignorespaces{}\hspace*{0pt} 10&\hspace*{0pt}\ignorespaces{}\hspace*{0pt} 16.59\\ \hline \hspace*{0pt}\ignorespaces{}\hspace*{0pt}{\ttfamily \setmainfont[Path=/usr/share/fonts/truetype/cmu/,UprightFont=cmunrm.ttf,BoldFont=cmunbx.ttf,ItalicFont=cmunti.ttf,BoldItalicFont=cmunbi.ttf]{cmuntt.ttf}\setmonofont[Path=/usr/share/fonts/truetype/cmu/,UprightFont=cmuntt.ttf,BoldFont=cmuntb.ttf,ItalicFont=cmunit.ttf,BoldItalicFont=cmuntx.ttf]{cmuntt.ttf}\ttfamily \textbackslash{}small}&\hspace*{0pt}\ignorespaces{}\hspace*{0pt}{$\text{ }$}\setmainfont[Path=/usr/share/fonts/truetype/cmu/,UprightFont=cmunrm.ttf,BoldFont=cmunbx.ttf,ItalicFont=cmunti.ttf,BoldItalicFont=cmunbi.ttf]{cmunrm.ttf}\setmonofont[Path=/usr/share/fonts/truetype/cmu/,UprightFont=cmuntt.ttf,BoldFont=cmuntb.ttf,ItalicFont=cmunit.ttf,BoldItalicFont=cmuntx.ttf]{cmunrm.ttf}  9  &\hspace*{0pt}\ignorespaces{}\hspace*{0pt} 10 &\hspace*{0pt}\ignorespaces{}\hspace*{0pt} 10.95&\hspace*{0pt}\ignorespaces{}\hspace*{0pt}  9  &\hspace*{0pt}\ignorespaces{}\hspace*{0pt} 10 &\hspace*{0pt}\ignorespaces{}\hspace*{0pt} 10.95&\hspace*{0pt}\ignorespaces{}\hspace*{0pt} 16.59\\ \hline \hspace*{0pt}\ignorespaces{}\hspace*{0pt}{\ttfamily \setmainfont[Path=/usr/share/fonts/truetype/cmu/,UprightFont=cmunrm.ttf,BoldFont=cmunbx.ttf,ItalicFont=cmunti.ttf,BoldItalicFont=cmunbi.ttf]{cmuntt.ttf}\setmonofont[Path=/usr/share/fonts/truetype/cmu/,UprightFont=cmuntt.ttf,BoldFont=cmuntb.ttf,ItalicFont=cmunit.ttf,BoldItalicFont=cmuntx.ttf]{cmuntt.ttf}\ttfamily \textbackslash{}normalsize}&\hspace*{0pt}\ignorespaces{}\hspace*{0pt}{$\text{ }$}\setmainfont[Path=/usr/share/fonts/truetype/cmu/,UprightFont=cmunrm.ttf,BoldFont=cmunbx.ttf,ItalicFont=cmunti.ttf,BoldItalicFont=cmunbi.ttf]{cmunrm.ttf}\setmonofont[Path=/usr/share/fonts/truetype/cmu/,UprightFont=cmuntt.ttf,BoldFont=cmuntb.ttf,ItalicFont=cmunit.ttf,BoldItalicFont=cmuntx.ttf]{cmunrm.ttf} 10 &\hspace*{0pt}\ignorespaces{}\hspace*{0pt} 10.95 &\hspace*{0pt}\ignorespaces{}\hspace*{0pt} 12&\hspace*{0pt}\ignorespaces{}\hspace*{0pt} 10 &\hspace*{0pt}\ignorespaces{}\hspace*{0pt} 10.95 &\hspace*{0pt}\ignorespaces{}\hspace*{0pt} 12&\hspace*{0pt}\ignorespaces{}\hspace*{0pt} 19.907\\ \hline \hspace*{0pt}\ignorespaces{}\hspace*{0pt}{\ttfamily \setmainfont[Path=/usr/share/fonts/truetype/cmu/,UprightFont=cmunrm.ttf,BoldFont=cmunbx.ttf,ItalicFont=cmunti.ttf,BoldItalicFont=cmunbi.ttf]{cmuntt.ttf}\setmonofont[Path=/usr/share/fonts/truetype/cmu/,UprightFont=cmuntt.ttf,BoldFont=cmuntb.ttf,ItalicFont=cmunit.ttf,BoldItalicFont=cmuntx.ttf]{cmuntt.ttf}\ttfamily \textbackslash{}large}&\hspace*{0pt}\ignorespaces{}\hspace*{0pt}{$\text{ }$}\setmainfont[Path=/usr/share/fonts/truetype/cmu/,UprightFont=cmunrm.ttf,BoldFont=cmunbx.ttf,ItalicFont=cmunti.ttf,BoldItalicFont=cmunbi.ttf]{cmunrm.ttf}\setmonofont[Path=/usr/share/fonts/truetype/cmu/,UprightFont=cmuntt.ttf,BoldFont=cmuntb.ttf,ItalicFont=cmunit.ttf,BoldItalicFont=cmuntx.ttf]{cmunrm.ttf} 12 &\hspace*{0pt}\ignorespaces{}\hspace*{0pt} 12 &\hspace*{0pt}\ignorespaces{}\hspace*{0pt} 14.4&\hspace*{0pt}\ignorespaces{}\hspace*{0pt} 10.95 &\hspace*{0pt}\ignorespaces{}\hspace*{0pt} 12 &\hspace*{0pt}\ignorespaces{}\hspace*{0pt} 14.4&\hspace*{0pt}\ignorespaces{}\hspace*{0pt} 23.89 \\ \hline \hspace*{0pt}\ignorespaces{}\hspace*{0pt}{\ttfamily \setmainfont[Path=/usr/share/fonts/truetype/cmu/,UprightFont=cmunrm.ttf,BoldFont=cmunbx.ttf,ItalicFont=cmunti.ttf,BoldItalicFont=cmunbi.ttf]{cmuntt.ttf}\setmonofont[Path=/usr/share/fonts/truetype/cmu/,UprightFont=cmuntt.ttf,BoldFont=cmuntb.ttf,ItalicFont=cmunit.ttf,BoldItalicFont=cmuntx.ttf]{cmuntt.ttf}\ttfamily \textbackslash{}Large}&\hspace*{0pt}\ignorespaces{}\hspace*{0pt}{$\text{ }$}\setmainfont[Path=/usr/share/fonts/truetype/cmu/,UprightFont=cmunrm.ttf,BoldFont=cmunbx.ttf,ItalicFont=cmunti.ttf,BoldItalicFont=cmunbi.ttf]{cmunrm.ttf}\setmonofont[Path=/usr/share/fonts/truetype/cmu/,UprightFont=cmuntt.ttf,BoldFont=cmuntb.ttf,ItalicFont=cmunit.ttf,BoldItalicFont=cmuntx.ttf]{cmunrm.ttf} 14.4 &\hspace*{0pt}\ignorespaces{}\hspace*{0pt} 14.4 &\hspace*{0pt}\ignorespaces{}\hspace*{0pt} 17.28&\hspace*{0pt}\ignorespaces{}\hspace*{0pt} 12 &\hspace*{0pt}\ignorespaces{}\hspace*{0pt} 14.4 &\hspace*{0pt}\ignorespaces{}\hspace*{0pt} 17.28&\hspace*{0pt}\ignorespaces{}\hspace*{0pt} 28.66\\ \hline \hspace*{0pt}\ignorespaces{}\hspace*{0pt}{\ttfamily \setmainfont[Path=/usr/share/fonts/truetype/cmu/,UprightFont=cmunrm.ttf,BoldFont=cmunbx.ttf,ItalicFont=cmunti.ttf,BoldItalicFont=cmunbi.ttf]{cmuntt.ttf}\setmonofont[Path=/usr/share/fonts/truetype/cmu/,UprightFont=cmuntt.ttf,BoldFont=cmuntb.ttf,ItalicFont=cmunit.ttf,BoldItalicFont=cmuntx.ttf]{cmuntt.ttf}\ttfamily \textbackslash{}LARGE}&\hspace*{0pt}\ignorespaces{}\hspace*{0pt}{$\text{ }$}\setmainfont[Path=/usr/share/fonts/truetype/cmu/,UprightFont=cmunrm.ttf,BoldFont=cmunbx.ttf,ItalicFont=cmunti.ttf,BoldItalicFont=cmunbi.ttf]{cmunrm.ttf}\setmonofont[Path=/usr/share/fonts/truetype/cmu/,UprightFont=cmuntt.ttf,BoldFont=cmuntb.ttf,ItalicFont=cmunit.ttf,BoldItalicFont=cmuntx.ttf]{cmunrm.ttf} 17.28 &\hspace*{0pt}\ignorespaces{}\hspace*{0pt} 17.28 &\hspace*{0pt}\ignorespaces{}\hspace*{0pt} 20.74&\hspace*{0pt}\ignorespaces{}\hspace*{0pt} 14.4 &\hspace*{0pt}\ignorespaces{}\hspace*{0pt} 17.28 &\hspace*{0pt}\ignorespaces{}\hspace*{0pt} 20.74&\hspace*{0pt}\ignorespaces{}\hspace*{0pt} 34.4\\ \hline \hspace*{0pt}\ignorespaces{}\hspace*{0pt}{\ttfamily \setmainfont[Path=/usr/share/fonts/truetype/cmu/,UprightFont=cmunrm.ttf,BoldFont=cmunbx.ttf,ItalicFont=cmunti.ttf,BoldItalicFont=cmunbi.ttf]{cmuntt.ttf}\setmonofont[Path=/usr/share/fonts/truetype/cmu/,UprightFont=cmuntt.ttf,BoldFont=cmuntb.ttf,ItalicFont=cmunit.ttf,BoldItalicFont=cmuntx.ttf]{cmuntt.ttf}\ttfamily \textbackslash{}huge}&\hspace*{0pt}\ignorespaces{}\hspace*{0pt}{$\text{ }$}\setmainfont[Path=/usr/share/fonts/truetype/cmu/,UprightFont=cmunrm.ttf,BoldFont=cmunbx.ttf,ItalicFont=cmunti.ttf,BoldItalicFont=cmunbi.ttf]{cmunrm.ttf}\setmonofont[Path=/usr/share/fonts/truetype/cmu/,UprightFont=cmuntt.ttf,BoldFont=cmuntb.ttf,ItalicFont=cmunit.ttf,BoldItalicFont=cmuntx.ttf]{cmunrm.ttf} 20.74 &\hspace*{0pt}\ignorespaces{}\hspace*{0pt} 20.74  &\hspace*{0pt}\ignorespaces{}\hspace*{0pt} 24.88&\hspace*{0pt}\ignorespaces{}\hspace*{0pt} 17.28 &\hspace*{0pt}\ignorespaces{}\hspace*{0pt} 20.74  &\hspace*{0pt}\ignorespaces{}\hspace*{0pt} 24.88&\hspace*{0pt}\ignorespaces{}\hspace*{0pt} 41.28 \\ \hline \hspace*{0pt}\ignorespaces{}\hspace*{0pt}{\ttfamily \setmainfont[Path=/usr/share/fonts/truetype/cmu/,UprightFont=cmunrm.ttf,BoldFont=cmunbx.ttf,ItalicFont=cmunti.ttf,BoldItalicFont=cmunbi.ttf]{cmuntt.ttf}\setmonofont[Path=/usr/share/fonts/truetype/cmu/,UprightFont=cmuntt.ttf,BoldFont=cmuntb.ttf,ItalicFont=cmunit.ttf,BoldItalicFont=cmuntx.ttf]{cmuntt.ttf}\ttfamily \textbackslash{}Huge}&\hspace*{0pt}\ignorespaces{}\hspace*{0pt}{$\text{ }$}\setmainfont[Path=/usr/share/fonts/truetype/cmu/,UprightFont=cmunrm.ttf,BoldFont=cmunbx.ttf,ItalicFont=cmunti.ttf,BoldItalicFont=cmunbi.ttf]{cmunrm.ttf}\setmonofont[Path=/usr/share/fonts/truetype/cmu/,UprightFont=cmuntt.ttf,BoldFont=cmuntb.ttf,ItalicFont=cmunit.ttf,BoldItalicFont=cmuntx.ttf]{cmunrm.ttf} 24.88 &\hspace*{0pt}\ignorespaces{}\hspace*{0pt} 24.88 &\hspace*{0pt}\ignorespaces{}\hspace*{0pt} 24.88&\hspace*{0pt}\ignorespaces{}\hspace*{0pt} 20.74 &\hspace*{0pt}\ignorespaces{}\hspace*{0pt} 24.88 &\hspace*{0pt}\ignorespaces{}\hspace*{0pt} 24.88&\hspace*{0pt}\ignorespaces{}\hspace*{0pt} 41.28\\ \hline 
\end{longtable}
}

As a technical note, points in TeX follow the standard American point size in which 1 pt is approximately 0.3513\myoverline{6} mm. The standard point size used in most modern computer programs (known as the {\itshape \setmainfont[Path=/usr/share/fonts/truetype/cmu/,UprightFont=cmunrm.ttf,BoldFont=cmunbx.ttf,ItalicFont=cmunti.ttf,BoldItalicFont=cmunbi.ttf]{cmunti.ttf}\setmonofont[Path=/usr/share/fonts/truetype/cmu/,UprightFont=cmuntt.ttf,BoldFont=cmuntb.ttf,ItalicFont=cmunit.ttf,BoldItalicFont=cmuntx.ttf]{cmunti.ttf}\itshape desktop publishing point}{$\text{ }$}\setmainfont[Path=/usr/share/fonts/truetype/cmu/,UprightFont=cmunrm.ttf,BoldFont=cmunbx.ttf,ItalicFont=cmunti.ttf,BoldItalicFont=cmunbi.ttf]{cmunrm.ttf}\setmonofont[Path=/usr/share/fonts/truetype/cmu/,UprightFont=cmuntt.ttf,BoldFont=cmuntb.ttf,ItalicFont=cmunit.ttf,BoldItalicFont=cmuntx.ttf]{cmunrm.ttf} or {\itshape \setmainfont[Path=/usr/share/fonts/truetype/cmu/,UprightFont=cmunrm.ttf,BoldFont=cmunbx.ttf,ItalicFont=cmunti.ttf,BoldItalicFont=cmunbi.ttf]{cmunti.ttf}\setmonofont[Path=/usr/share/fonts/truetype/cmu/,UprightFont=cmuntt.ttf,BoldFont=cmuntb.ttf,ItalicFont=cmunit.ttf,BoldItalicFont=cmuntx.ttf]{cmunti.ttf}\itshape PostScript point}\setmainfont[Path=/usr/share/fonts/truetype/cmu/,UprightFont=cmunrm.ttf,BoldFont=cmunbx.ttf,ItalicFont=cmunti.ttf,BoldItalicFont=cmunbi.ttf]{cmunrm.ttf}\setmonofont[Path=/usr/share/fonts/truetype/cmu/,UprightFont=cmuntt.ttf,BoldFont=cmuntb.ttf,ItalicFont=cmunit.ttf,BoldItalicFont=cmuntx.ttf]{cmunrm.ttf}) has 1 pt equal to approximately 0.352\myoverline{7} mm while the standard European point size (known as the {\itshape \setmainfont[Path=/usr/share/fonts/truetype/cmu/,UprightFont=cmunrm.ttf,BoldFont=cmunbx.ttf,ItalicFont=cmunti.ttf,BoldItalicFont=cmunbi.ttf]{cmunti.ttf}\setmonofont[Path=/usr/share/fonts/truetype/cmu/,UprightFont=cmuntt.ttf,BoldFont=cmuntb.ttf,ItalicFont=cmunit.ttf,BoldItalicFont=cmuntx.ttf]{cmunti.ttf}\itshape Didot point}\setmainfont[Path=/usr/share/fonts/truetype/cmu/,UprightFont=cmunrm.ttf,BoldFont=cmunbx.ttf,ItalicFont=cmunti.ttf,BoldItalicFont=cmunbi.ttf]{cmunrm.ttf}\setmonofont[Path=/usr/share/fonts/truetype/cmu/,UprightFont=cmuntt.ttf,BoldFont=cmuntb.ttf,ItalicFont=cmunit.ttf,BoldItalicFont=cmuntx.ttf]{cmunrm.ttf}) had 1 pt equal to approximately 0.37597151 mm (see: \myhref{https://en.wikipedia.org/wiki/Point_\%28typography\%29}{point (typography)}).
\section{Local font selection}
\label{177}
You can change font for a specific part of the text. There are four font properties you can change.{\bfseries
\begin{mydescription}\LaTeXTT{\textbackslash{}fontencoding}
\end{mydescription}
}
\begin{myquote}\item{} The font encoding, such as OT1 (TeX default) or T1 (extended characters support, better PDF support, widely used).
\end{myquote}
{\bfseries
\begin{mydescription}\LaTeXTT{\textbackslash{}fontfamily}
\end{mydescription}
}
\begin{myquote}\item{} The font family.
\end{myquote}
{\bfseries
\begin{mydescription}\LaTeXTT{\textbackslash{}fontseries}
\end{mydescription}
}
\begin{myquote}\item{} The series: l=light, m=medium, b=bold, bx=very bold.
\end{myquote}
{\bfseries
\begin{mydescription}\LaTeXTT{\textbackslash{}fontshape}
\end{mydescription}
}
\begin{myquote}\item{} The shape: it=italic, n=normal, sl=slanted, sc=small capitals.
\end{myquote}


\begin{Shaded}
\begin{Highlighting}[]

\NormalTok{\{}
\NormalTok{\textbackslash{}fontfamily\{anttlc\}\textbackslash{}selectfont}
\NormalTok{Some text in anttlc...}
\NormalTok{\}}
\end{Highlighting}
\end{Shaded}


The \LaTeXTT{\textbackslash{}selectfont} command is mandatory, otherwise the font will not be changed. It is highly recommended to enclose the command in a group to cleanly return to the previous font selection when desired.

You can use all these commands in a row:
\begin{Shaded}
\begin{Highlighting}[]

\NormalTok{\{}
\NormalTok{\textbackslash{}fontencoding\{T1\}\textbackslash{}fontfamily\{anttlc\}\textbackslash{}fontseries\{m\}\textbackslash{}fontshape\{n\}\textbackslash{}selectfont}
\NormalTok{Some text in anttlc...}
\NormalTok{\}}
\end{Highlighting}
\end{Shaded}


The default values are stored in \LaTeXTT{\textbackslash{}encodingdefault}, \LaTeXTT{\textbackslash{}familydefault}, \LaTeXTT{\textbackslash{}seriesdefault} and \LaTeXTT{\textbackslash{}shapedefault}. Setting back the default font properties can be done with
\begin{Shaded}
\begin{Highlighting}[]

\NormalTok{\textbackslash{}fontencoding\{\textbackslash{}encodingdefault\}}
\NormalTok{\textbackslash{}fontfamily\{\textbackslash{}familydefault\}}
\NormalTok{\textbackslash{}fontseries\{\textbackslash{}seriesdefault\}}
\NormalTok{\textbackslash{}fontshape\{\textbackslash{}shapedefault\}}
\NormalTok{\textbackslash{}selectfont}
\end{Highlighting}
\end{Shaded}


For short, you can use the \LaTeXTT{\textbackslash{}usefont\{<{}encoding>{}\}\{<{}family>{}\}\{<{}series>{}\}\{<{}shape>{}\}} command.
\begin{Shaded}
\begin{Highlighting}[]

\NormalTok{\textbackslash{}usefont\{T1\}\{cmr\}\{m\}\{n\} }\CommentTok{% Computer Modern Roman (TeX default) in T1 encoding.}
 \NormalTok{May lead to bad text quality if you do not have cm-super installed.}
\NormalTok{\textbackslash{}usefont\{T1\}\{phv\}\{m\}\{sc\} }\CommentTok{% phv family (sans serif) medium small capitals.}
\NormalTok{\textbackslash{}usefont\{T1\}\{ptm\}\{b\}\{it\} }\CommentTok{% ptm family bold italic}
\NormalTok{\textbackslash{}usefont\{U\}\{pzd\}\{m\}\{n\}   }\CommentTok{% ...}
\end{Highlighting}
\end{Shaded}

\section{Arbitrary font size}
\label{178}

The \LaTeXTT{\textbackslash{}tiny}...\LaTeXTT{\textbackslash{}Huge} commands are often enough for most contents. These are fixed sizes however. In most document processors, you can usually choose any size for any font. This is because the characters actually get magnified. If it usually looks correct for medium sizes, it will look odd at extreme sizes because of an unbalanced thickness. In TeX it is possible to change the magnification of anything, but highly discouraged for the aforementioned reason. Changing the font size is made by changing the font file. Yes, there is a file for every size: cmr10 for Computer Modern Roman 10pt, cmr12 for Computer Modern Roman 12pt, etc. This ensure the characters are correctly balanced and remain readable at all defined sizes.

You may choose a particular font size with the \LaTeXTT{\textbackslash{}fontsize\{<{}size>{}\}\{<{}line space>{}\}} command. Example:

\begin{Shaded}
\begin{Highlighting}[]

\NormalTok{\{\textbackslash{}fontsize\{5cm\}\{5.5cm\}\textbackslash{}selectfont This is big!\}}
\end{Highlighting}
\end{Shaded}


If you are using the default Computer Modern font encoding, you may get the following message:\\

\TemplateSpaceIndent{$\text{ }${}LaTeX$\text{ }${}Font$\text{ }${}Warning:$\text{ }${}Font$\text{ }${}shape$\text{ }${}`OT1/cmr/m/n\textquotesingle{}$\text{ }${}in$\text{ }${}size$\text{ }${}<{}142.26378>{}$\text{ }${}not$\text{ }${}available$\text{ }$\newline{}
$\text{ }${}(Font)$\text{ }${}$\text{ }${}$\text{ }${}$\text{ }${}$\text{ }${}$\text{ }${}$\text{ }${}$\text{ }${}$\text{ }${}$\text{ }${}$\text{ }${}$\text{ }${}$\text{ }${}$\text{ }${}size$\text{ }${}<{}24.88>{}$\text{ }${}substituted$\text{ }${}on$\text{ }${}input$\text{ }${}line$\text{ }${}103.}

In that case you will notice that the font size is by default restricted to a set of fixed sizes as noted above. You can use the fix-{}cm or type1cm packages to allow computer modern fonts to be scaled to arbitrary values.
\section{Finding fonts}
\label{179}

You will find a huge font directory along examples and configurations at \myhref{http://www.tug.dk/FontCatalogue/}{TUG Font Catalogue}.
\section{Using arbitrary system fonts}
\label{180}

If you use the \myhref{https://en.wikipedia.org/wiki/XeTeX}{XeTeX} or \myhref{https://en.wikipedia.org/wiki/LuaTeX}{LuaTeX} engine and the \myhref{http://www.ctan.org/pkg/fontspec}{fontspec} package, you\textquotesingle{}ll be able to use any font installed in the system effortlessly. XeTeX also allows using \myhref{https://en.wikipedia.org/wiki/Opentype}{OpenType} technology of modern fonts like specifying alternate glyphs and optical size variants. XeTeX also uses \myhref{https://en.wikipedia.org/wiki/Unicode}{Unicode} by default, which might be helpful for font issues.

To use the fonts, simply load the \LaTeXTT{fontspec} package and set the font:

\begin{Shaded}
\begin{Highlighting}[]

\NormalTok{\textbackslash{}documentclass\{article\}}
 
\NormalTok{\textbackslash{}usepackage\{fontspec\}}
\NormalTok{\textbackslash{}setmainfont\{Arial\}}
 
\NormalTok{\textbackslash{}begin\{document\}}
\NormalTok{Lorem ipsum...}
\NormalTok{\textbackslash{}end\{document\}}
\end{Highlighting}
\end{Shaded}


Then compile the document with {\ttfamily \setmainfont[Path=/usr/share/fonts/truetype/cmu/,UprightFont=cmunrm.ttf,BoldFont=cmunbx.ttf,ItalicFont=cmunti.ttf,BoldItalicFont=cmunbi.ttf]{cmuntt.ttf}\setmonofont[Path=/usr/share/fonts/truetype/cmu/,UprightFont=cmuntt.ttf,BoldFont=cmuntb.ttf,ItalicFont=cmunit.ttf,BoldItalicFont=cmuntx.ttf]{cmuntt.ttf}\ttfamily xelatex}{$\text{ }$}\setmainfont[Path=/usr/share/fonts/truetype/cmu/,UprightFont=cmunrm.ttf,BoldFont=cmunbx.ttf,ItalicFont=cmunti.ttf,BoldItalicFont=cmunbi.ttf]{cmunrm.ttf}\setmonofont[Path=/usr/share/fonts/truetype/cmu/,UprightFont=cmuntt.ttf,BoldFont=cmuntb.ttf,ItalicFont=cmunit.ttf,BoldItalicFont=cmuntx.ttf]{cmunrm.ttf} or {\ttfamily \setmainfont[Path=/usr/share/fonts/truetype/cmu/,UprightFont=cmunrm.ttf,BoldFont=cmunbx.ttf,ItalicFont=cmunti.ttf,BoldItalicFont=cmunbi.ttf]{cmuntt.ttf}\setmonofont[Path=/usr/share/fonts/truetype/cmu/,UprightFont=cmuntt.ttf,BoldFont=cmuntb.ttf,ItalicFont=cmunit.ttf,BoldItalicFont=cmuntx.ttf]{cmuntt.ttf}\ttfamily lualatex}\setmainfont[Path=/usr/share/fonts/truetype/cmu/,UprightFont=cmunrm.ttf,BoldFont=cmunbx.ttf,ItalicFont=cmunti.ttf,BoldItalicFont=cmunbi.ttf]{cmunrm.ttf}\setmonofont[Path=/usr/share/fonts/truetype/cmu/,UprightFont=cmuntt.ttf,BoldFont=cmuntb.ttf,ItalicFont=cmunit.ttf,BoldItalicFont=cmuntx.ttf]{cmunrm.ttf}. Note that you can only generate .pdf files, and that you need a sufficiently new TeX distribution (TeX Live 2009 should work for XeTeX and Tex Live 2010 for LuaTeX). Also you should {\itshape \setmainfont[Path=/usr/share/fonts/truetype/cmu/,UprightFont=cmunrm.ttf,BoldFont=cmunbx.ttf,ItalicFont=cmunti.ttf,BoldItalicFont=cmunbi.ttf]{cmunti.ttf}\setmonofont[Path=/usr/share/fonts/truetype/cmu/,UprightFont=cmuntt.ttf,BoldFont=cmuntb.ttf,ItalicFont=cmunit.ttf,BoldItalicFont=cmuntx.ttf]{cmunti.ttf}\itshape not}{$\text{ }$}\setmainfont[Path=/usr/share/fonts/truetype/cmu/,UprightFont=cmunrm.ttf,BoldFont=cmunbx.ttf,ItalicFont=cmunti.ttf,BoldItalicFont=cmunbi.ttf]{cmunrm.ttf}\setmonofont[Path=/usr/share/fonts/truetype/cmu/,UprightFont=cmuntt.ttf,BoldFont=cmuntb.ttf,ItalicFont=cmunit.ttf,BoldItalicFont=cmuntx.ttf]{cmunrm.ttf} load the \LaTeXTT{inputenc} or \LaTeXTT{fontenc} package. Instead make sure that your document is encoded as UTF-{}8 and load \LaTeXTT{fontspec}, which will take care of the font encoding. To make your document support both {\ttfamily \setmainfont[Path=/usr/share/fonts/truetype/cmu/,UprightFont=cmunrm.ttf,BoldFont=cmunbx.ttf,ItalicFont=cmunti.ttf,BoldItalicFont=cmunbi.ttf]{cmuntt.ttf}\setmonofont[Path=/usr/share/fonts/truetype/cmu/,UprightFont=cmuntt.ttf,BoldFont=cmuntb.ttf,ItalicFont=cmunit.ttf,BoldItalicFont=cmuntx.ttf]{cmuntt.ttf}\ttfamily pdflatex}{$\text{ }$}\setmainfont[Path=/usr/share/fonts/truetype/cmu/,UprightFont=cmunrm.ttf,BoldFont=cmunbx.ttf,ItalicFont=cmunti.ttf,BoldItalicFont=cmunbi.ttf]{cmunrm.ttf}\setmonofont[Path=/usr/share/fonts/truetype/cmu/,UprightFont=cmuntt.ttf,BoldFont=cmuntb.ttf,ItalicFont=cmunit.ttf,BoldItalicFont=cmuntx.ttf]{cmunrm.ttf} and {\ttfamily \setmainfont[Path=/usr/share/fonts/truetype/cmu/,UprightFont=cmunrm.ttf,BoldFont=cmunbx.ttf,ItalicFont=cmunti.ttf,BoldItalicFont=cmunbi.ttf]{cmuntt.ttf}\setmonofont[Path=/usr/share/fonts/truetype/cmu/,UprightFont=cmuntt.ttf,BoldFont=cmuntb.ttf,ItalicFont=cmunit.ttf,BoldItalicFont=cmuntx.ttf]{cmuntt.ttf}\ttfamily xelatex}\setmainfont[Path=/usr/share/fonts/truetype/cmu/,UprightFont=cmunrm.ttf,BoldFont=cmunbx.ttf,ItalicFont=cmunti.ttf,BoldItalicFont=cmunbi.ttf]{cmunrm.ttf}\setmonofont[Path=/usr/share/fonts/truetype/cmu/,UprightFont=cmuntt.ttf,BoldFont=cmuntb.ttf,ItalicFont=cmunit.ttf,BoldItalicFont=cmuntx.ttf]{cmunrm.ttf}/{\ttfamily \setmainfont[Path=/usr/share/fonts/truetype/cmu/,UprightFont=cmunrm.ttf,BoldFont=cmunbx.ttf,ItalicFont=cmunti.ttf,BoldItalicFont=cmunbi.ttf]{cmuntt.ttf}\setmonofont[Path=/usr/share/fonts/truetype/cmu/,UprightFont=cmuntt.ttf,BoldFont=cmuntb.ttf,ItalicFont=cmunit.ttf,BoldItalicFont=cmuntx.ttf]{cmuntt.ttf}\ttfamily lualatex}{$\text{ }$}\setmainfont[Path=/usr/share/fonts/truetype/cmu/,UprightFont=cmunrm.ttf,BoldFont=cmunbx.ttf,ItalicFont=cmunti.ttf,BoldItalicFont=cmunbi.ttf]{cmunrm.ttf}\setmonofont[Path=/usr/share/fonts/truetype/cmu/,UprightFont=cmuntt.ttf,BoldFont=cmuntb.ttf,ItalicFont=cmunit.ttf,BoldItalicFont=cmuntx.ttf]{cmunrm.ttf} you can use the \LaTeXTT{\textbackslash{}ifxetex}/ \LaTeXTT{\textbackslash{}ifluatex} macro from the \myhref{http://www.ctan.org/pkg/ifxetex/}{ifxetex}/ \myhref{http://www.ctan.org/pkg/ifluatex}{ifluatex} package. For example for {\ttfamily \setmainfont[Path=/usr/share/fonts/truetype/cmu/,UprightFont=cmunrm.ttf,BoldFont=cmunbx.ttf,ItalicFont=cmunti.ttf,BoldItalicFont=cmunbi.ttf]{cmuntt.ttf}\setmonofont[Path=/usr/share/fonts/truetype/cmu/,UprightFont=cmuntt.ttf,BoldFont=cmuntb.ttf,ItalicFont=cmunit.ttf,BoldItalicFont=cmuntx.ttf]{cmuntt.ttf}\ttfamily xelatex}\setmainfont[Path=/usr/share/fonts/truetype/cmu/,UprightFont=cmunrm.ttf,BoldFont=cmunbx.ttf,ItalicFont=cmunti.ttf,BoldItalicFont=cmunbi.ttf]{cmunrm.ttf}\setmonofont[Path=/usr/share/fonts/truetype/cmu/,UprightFont=cmuntt.ttf,BoldFont=cmuntb.ttf,ItalicFont=cmunit.ttf,BoldItalicFont=cmuntx.ttf]{cmunrm.ttf}

\begin{Shaded}
\begin{Highlighting}[]

\NormalTok{\textbackslash{}documentclass\{article\}}
\NormalTok{\textbackslash{}usepackage\{ifxetex\}}
 
\NormalTok{\textbackslash{}ifxetex}
  \NormalTok{\textbackslash{}usepackage\{fontspec\}}
  \NormalTok{\textbackslash{}defaultfontfeatures\{Ligatures=TeX\} }\CommentTok{% To support LaTeX quoting style}
  \NormalTok{\textbackslash{}setromanfont\{Hoefler Text\}}
\NormalTok{\textbackslash{}else}
  \NormalTok{\textbackslash{}usepackage[T1]\{fontenc\}}
  \NormalTok{\textbackslash{}usepackage[utf8]\{inputenc\}}
\NormalTok{\textbackslash{}fi}
 
\NormalTok{\textbackslash{}begin\{document\}}
\NormalTok{Lorem ipsum...}
\NormalTok{\textbackslash{}end\{document\}}
\end{Highlighting}
\end{Shaded}

\section{PDF fonts and properties}
\label{181}
PDF documents have the capability to embed font files. It makes them portable, hence the name {\itshape \setmainfont[Path=/usr/share/fonts/truetype/cmu/,UprightFont=cmunrm.ttf,BoldFont=cmunbx.ttf,ItalicFont=cmunti.ttf,BoldItalicFont=cmunbi.ttf]{cmunti.ttf}\setmonofont[Path=/usr/share/fonts/truetype/cmu/,UprightFont=cmuntt.ttf,BoldFont=cmuntb.ttf,ItalicFont=cmunit.ttf,BoldItalicFont=cmuntx.ttf]{cmunti.ttf}\itshape Portable Document Format}\setmainfont[Path=/usr/share/fonts/truetype/cmu/,UprightFont=cmunrm.ttf,BoldFont=cmunbx.ttf,ItalicFont=cmunti.ttf,BoldItalicFont=cmunbi.ttf]{cmunrm.ttf}\setmonofont[Path=/usr/share/fonts/truetype/cmu/,UprightFont=cmuntt.ttf,BoldFont=cmuntb.ttf,ItalicFont=cmunit.ttf,BoldItalicFont=cmuntx.ttf]{cmunrm.ttf}.

Many PDF viewers have a {\itshape \setmainfont[Path=/usr/share/fonts/truetype/cmu/,UprightFont=cmunrm.ttf,BoldFont=cmunbx.ttf,ItalicFont=cmunti.ttf,BoldItalicFont=cmunbi.ttf]{cmunti.ttf}\setmonofont[Path=/usr/share/fonts/truetype/cmu/,UprightFont=cmuntt.ttf,BoldFont=cmuntb.ttf,ItalicFont=cmunit.ttf,BoldItalicFont=cmuntx.ttf]{cmunti.ttf}\itshape Properties}{$\text{ }$}\setmainfont[Path=/usr/share/fonts/truetype/cmu/,UprightFont=cmunrm.ttf,BoldFont=cmunbx.ttf,ItalicFont=cmunti.ttf,BoldItalicFont=cmunbi.ttf]{cmunrm.ttf}\setmonofont[Path=/usr/share/fonts/truetype/cmu/,UprightFont=cmuntt.ttf,BoldFont=cmuntb.ttf,ItalicFont=cmunit.ttf,BoldItalicFont=cmuntx.ttf]{cmunrm.ttf} feature to list embedded fonts and document metadata.

Many Unix systems make use of the {\itshape \setmainfont[Path=/usr/share/fonts/truetype/cmu/,UprightFont=cmunrm.ttf,BoldFont=cmunbx.ttf,ItalicFont=cmunti.ttf,BoldItalicFont=cmunbi.ttf]{cmunti.ttf}\setmonofont[Path=/usr/share/fonts/truetype/cmu/,UprightFont=cmuntt.ttf,BoldFont=cmuntb.ttf,ItalicFont=cmunit.ttf,BoldItalicFont=cmuntx.ttf]{cmunti.ttf}\itshape poppler}{$\text{ }$}\setmainfont[Path=/usr/share/fonts/truetype/cmu/,UprightFont=cmunrm.ttf,BoldFont=cmunbx.ttf,ItalicFont=cmunti.ttf,BoldItalicFont=cmunbi.ttf]{cmunrm.ttf}\setmonofont[Path=/usr/share/fonts/truetype/cmu/,UprightFont=cmuntt.ttf,BoldFont=cmuntb.ttf,ItalicFont=cmunit.ttf,BoldItalicFont=cmuntx.ttf]{cmunrm.ttf} tool set which features {\ttfamily \setmainfont[Path=/usr/share/fonts/truetype/cmu/,UprightFont=cmunrm.ttf,BoldFont=cmunbx.ttf,ItalicFont=cmunti.ttf,BoldItalicFont=cmunbi.ttf]{cmuntt.ttf}\setmonofont[Path=/usr/share/fonts/truetype/cmu/,UprightFont=cmuntt.ttf,BoldFont=cmuntb.ttf,ItalicFont=cmunit.ttf,BoldItalicFont=cmuntx.ttf]{cmuntt.ttf}\ttfamily pdfinfo}{$\text{ }$}\setmainfont[Path=/usr/share/fonts/truetype/cmu/,UprightFont=cmunrm.ttf,BoldFont=cmunbx.ttf,ItalicFont=cmunti.ttf,BoldItalicFont=cmunbi.ttf]{cmunrm.ttf}\setmonofont[Path=/usr/share/fonts/truetype/cmu/,UprightFont=cmuntt.ttf,BoldFont=cmuntb.ttf,ItalicFont=cmunit.ttf,BoldItalicFont=cmuntx.ttf]{cmunrm.ttf} to list PDF metadata, and {\ttfamily \setmainfont[Path=/usr/share/fonts/truetype/cmu/,UprightFont=cmunrm.ttf,BoldFont=cmunbx.ttf,ItalicFont=cmunti.ttf,BoldItalicFont=cmunbi.ttf]{cmuntt.ttf}\setmonofont[Path=/usr/share/fonts/truetype/cmu/,UprightFont=cmuntt.ttf,BoldFont=cmuntb.ttf,ItalicFont=cmunit.ttf,BoldItalicFont=cmuntx.ttf]{cmuntt.ttf}\ttfamily pdffonts}{$\text{ }$}\setmainfont[Path=/usr/share/fonts/truetype/cmu/,UprightFont=cmunrm.ttf,BoldFont=cmunbx.ttf,ItalicFont=cmunti.ttf,BoldItalicFont=cmunbi.ttf]{cmunrm.ttf}\setmonofont[Path=/usr/share/fonts/truetype/cmu/,UprightFont=cmuntt.ttf,BoldFont=cmuntb.ttf,ItalicFont=cmunit.ttf,BoldItalicFont=cmuntx.ttf]{cmunrm.ttf} to list embedded fonts.
\section{Useful websites}
\label{182}

\begin{myitemize}
\item{}  \myhref{http://www.tug.dk/FontCatalogue/}{The Latex Font Catalogue}
\item{}  \myhref{http://www.cl.cam.ac.uk/~rf10/pstex/latexcommands.htm}{LaTeX font commands}
\item{}  \myhref{http://www.ee.iitb.ac.in/~trivedi/LatexHelp/latexfont.htm}{How to change fonts in Latex}
\item{}  {$\text{[}$}ftp://tug.ctan.org/tex-{}archive/fonts/utilities/fontinst/doc/talks/et99-{}font-{}tutorial.pdf Understanding the world of TEX fonts and mastering the basics of fontinst{$\text{]}$}
\item{}  \myhref{http://www.tug.org/TUGboat/Articles/tb27-1/tb86kroonenberg-fonts.pdf}{Font installation the shallow way} {\itshape \setmainfont[Path=/usr/share/fonts/truetype/cmu/,UprightFont=cmunrm.ttf,BoldFont=cmunbx.ttf,ItalicFont=cmunti.ttf,BoldItalicFont=cmunbi.ttf]{cmunti.ttf}\setmonofont[Path=/usr/share/fonts/truetype/cmu/,UprightFont=cmuntt.ttf,BoldFont=cmuntb.ttf,ItalicFont=cmunit.ttf,BoldItalicFont=cmuntx.ttf]{cmunti.ttf}\itshape \symbol{34}For one-{}off projects, you can cut corners with font installation}{$\text{ }$}\setmainfont[Path=/usr/share/fonts/truetype/cmu/,UprightFont=cmunrm.ttf,BoldFont=cmunbx.ttf,ItalicFont=cmunti.ttf,BoldItalicFont=cmunbi.ttf]{cmunrm.ttf}\setmonofont[Path=/usr/share/fonts/truetype/cmu/,UprightFont=cmuntt.ttf,BoldFont=cmuntb.ttf,ItalicFont=cmunit.ttf,BoldItalicFont=cmuntx.ttf]{cmunrm.ttf} (i.e. fontinst) {\itshape \setmainfont[Path=/usr/share/fonts/truetype/cmu/,UprightFont=cmunrm.ttf,BoldFont=cmunbx.ttf,ItalicFont=cmunti.ttf,BoldItalicFont=cmunbi.ttf]{cmunti.ttf}\setmonofont[Path=/usr/share/fonts/truetype/cmu/,UprightFont=cmuntt.ttf,BoldFont=cmuntb.ttf,ItalicFont=cmunit.ttf,BoldItalicFont=cmuntx.ttf]{cmunti.ttf}\itshape and end up with a more manageable set of files and a cleaner TEX installation. This article shows how and why\symbol{34}}
\end{myitemize}
\setmainfont[Path=/usr/share/fonts/truetype/cmu/,UprightFont=cmunrm.ttf,BoldFont=cmunbx.ttf,ItalicFont=cmunti.ttf,BoldItalicFont=cmunbi.ttf]{cmunrm.ttf}\setmonofont[Path=/usr/share/fonts/truetype/cmu/,UprightFont=cmuntt.ttf,BoldFont=cmuntb.ttf,ItalicFont=cmunit.ttf,BoldItalicFont=cmuntx.ttf]{cmunrm.ttf}
\subsection{TrueType (ttf) fonts}
\label{183}
\begin{myitemize}
\item{}  \myhref{http://c.caignaert.free.fr/Install-ttf-Font.pdf}{Step-{}by-{}step guide to manually install a ttf-{}font for PdfTeX}
\item{}  \myhref{http://www.tex.ac.uk/ctan/support/installfont/}{A bash script for installing a LaTeX font family} (\myhref{http://latex.josef-kleber.de/download/installfont-tl}{MikTeX} / \myhref{http://latex.josef-kleber.de/download/installfont-tl}{TeXLive})
\item{}  \myhref{http://xpt.sourceforge.net/techdocs/language/latex/latex33-LaTeXAndTrueTypeFont}{LaTeX And TrueType Font}
\item{}  \myhref{http://fachschaft.physik.uni-greifswald.de/~stitch/ttf.html}{True Type Fonts with LaTeX under Linux + MiKTeX 2.5}
\item{}  \myhref{http://william.famille-blum.org/software/latexttf/index.html}{Unicode Truetype font installer for LaTeX under Windows + MikTeX}
\item{}  \myhref{http://www.radamir.com/tex/ttf-tex.htm}{Using TrueType fonts with TeX (LaTeX) and pdfTeX (pdfLaTeX)} (for MikTeX)
\end{myitemize}

\section{References}
\label{184}
\LaTeXNullTemplate{}



\myhref{https://sr.wikibooks.org/wiki/LaTeX\%2F\%D0\%A1\%D0\%BB\%D0\%BE\%D0\%B2\%D0\%B0}{sr:LaTeX/Слова}\chapter{List Structures}

\myminitoc
\label{185}

\label{186}


Convenient and predictable list formatting is one of the many advantages of using LaTeX. Users of WYSIWYG word processors can sometimes be frustrated by the software\textquotesingle{}s attempts to determine when they intend lists to begin and end. As a mark-{}up language, LaTeX gives more control over the structure and content of lists.
\section{List structures}
\label{187}

Lists often appear in documents, especially academic, as their purpose is often to present information in a clear and concise fashion. List structures in LaTeX are simply environments which essentially come in three types:
\begin{myitemize}
\item{}  \LaTeXTT{itemize} for a bullet list 
\item{}  \LaTeXTT{enumerate} for an enumerated list and
\item{}  \LaTeXTT{description} for a descriptive list.
\end{myitemize}


All lists follow the basic format:

\begin{Shaded}
\begin{Highlighting}[]

\NormalTok{\textbackslash{}begin\{list}\AlertTok{_type}\NormalTok{\}  }
\NormalTok{\textbackslash{}item The first item }
\NormalTok{\textbackslash{}item The second item }
\NormalTok{\textbackslash{}item The third etc \textbackslash{}ldots }
\NormalTok{\textbackslash{}end\{list_type\}}
 
\end{Highlighting}
\end{Shaded}


All three of these types of lists can have multiple paragraphs per item: just type the additional paragraphs in the normal way, with a blank line between each. So long as they are still contained within the enclosing environment, they will automatically be indented to follow underneath their item.


Try out the examples below, to see what the lists look like in a
real document.

\begin{longtable}{p{1.0\linewidth}}
\begin{Shaded}
\begin{Highlighting}[]

\NormalTok{\textbackslash{}documentclass\{article\}}
\NormalTok{\textbackslash{}usepackage\{blindtext\}}
\NormalTok{\textbackslash{}begin\{document\}}
\NormalTok{\textbackslash{}begin\{itemize\}}
\NormalTok{\textbackslash{}item \textbackslash{}blindtext}
\NormalTok{\textbackslash{}item \textbackslash{}blindtext}
\NormalTok{\textbackslash{}end\{itemize\}}
\NormalTok{\textbackslash{}begin\{enumerate\}}
\NormalTok{\textbackslash{}item \textbackslash{}blindtext}
\NormalTok{\textbackslash{}item \textbackslash{}blindtext}
\NormalTok{\textbackslash{}end\{enumerate\}}
\NormalTok{\textbackslash{}begin\{description\}}
\NormalTok{\textbackslash{}item [Ant] \textbackslash{}blindtext}
\NormalTok{\textbackslash{}item [Elephant] \textbackslash{}blindtext}
\NormalTok{\textbackslash{}end\{description\}}
\NormalTok{\textbackslash{}end\{document\}}
\end{Highlighting}
\end{Shaded}
\\


\begin{minipage}{1.0\linewidth}
\begin{center}
\includegraphics[width=1.0\linewidth,height=6.5in,keepaspectratio]{../images/27.png}
\end{center}
\raggedright{}\myfigurewithcaption{27}{Sample output of lists in
LaTeX. Itemize, enumerate and description.}
\end{minipage}\vspace{0.75cm}


 
\end{longtable}


LaTeX will happily allow you to insert a list environment into an
existing one (up to a depth of four, more levels are available
using packages). Simply begin the appropriate environment at the
desired point within the current list. Latex will sort out the
layout and any numbering for you.

\begin{longtable}{p{1.0\linewidth}}
\begin{Shaded}
\begin{Highlighting}[]

\NormalTok{\textbackslash{}begin\{enumerate\}}
\NormalTok{\textbackslash{}item The first item}
\NormalTok{\textbackslash{}begin\{enumerate\}}
\NormalTok{\textbackslash{}item Nested item 1}
\NormalTok{\textbackslash{}item Nested item 2}
\NormalTok{\textbackslash{}end\{enumerate\}}
\NormalTok{\textbackslash{}item The second item}
\NormalTok{\textbackslash{}item The third etc \textbackslash{}ldots}
\NormalTok{\textbackslash{}end\{enumerate\}}
\end{Highlighting}
\end{Shaded}
\\



\begin{minipage}{1.0\linewidth}
\begin{center}
\includegraphics[width=1.0\linewidth,height=6.5in,keepaspectratio]{../images/28.\SVGExtension}
\end{center}
\raggedright{}\myfigurewithoutcaption{28}
\end{minipage}\vspace{0.75cm}


 
\end{longtable}
\section{Some special lists}
\label{188}

Sometimes you feel the need to better align the different list
items. If you are using a KOMA-{}script class (or package
{\ttfamily \setmainfont[Path=/usr/share/fonts/truetype/cmu/,UprightFont=cmunrm.ttf,BoldFont=cmunbx.ttf,ItalicFont=cmunti.ttf,BoldItalicFont=cmunbi.ttf]{cmuntt.ttf}\setmonofont[Path=/usr/share/fonts/truetype/cmu/,UprightFont=cmuntt.ttf,BoldFont=cmuntb.ttf,ItalicFont=cmunit.ttf,BoldItalicFont=cmuntx.ttf]{cmuntt.ttf}\ttfamily scrextend}\setmainfont[Path=/usr/share/fonts/truetype/cmu/,UprightFont=cmunrm.ttf,BoldFont=cmunbx.ttf,ItalicFont=cmunti.ttf,BoldItalicFont=cmunbi.ttf]{cmunrm.ttf}\setmonofont[Path=/usr/share/fonts/truetype/cmu/,UprightFont=cmuntt.ttf,BoldFont=cmuntb.ttf,ItalicFont=cmunit.ttf,BoldItalicFont=cmuntx.ttf]{cmunrm.ttf}), the
{\ttfamily \setmainfont[Path=/usr/share/fonts/truetype/cmu/,UprightFont=cmunrm.ttf,BoldFont=cmunbx.ttf,ItalicFont=cmunti.ttf,BoldItalicFont=cmunbi.ttf]{cmuntt.ttf}\setmonofont[Path=/usr/share/fonts/truetype/cmu/,UprightFont=cmuntt.ttf,BoldFont=cmuntb.ttf,ItalicFont=cmunit.ttf,BoldItalicFont=cmuntx.ttf]{cmuntt.ttf}\ttfamily labeling}{$\text{ }$}\setmainfont[Path=/usr/share/fonts/truetype/cmu/,UprightFont=cmunrm.ttf,BoldFont=cmunbx.ttf,ItalicFont=cmunti.ttf,BoldItalicFont=cmunbi.ttf]{cmunrm.ttf}\setmonofont[Path=/usr/share/fonts/truetype/cmu/,UprightFont=cmuntt.ttf,BoldFont=cmuntb.ttf,ItalicFont=cmunit.ttf,BoldItalicFont=cmuntx.ttf]{cmunrm.ttf} environment is handy. It takes a mandatory
argument that contains the longest of your labels.

\begin{longtable}{p{1.0\linewidth}}
\begin{Shaded}
\begin{Highlighting}[]

\NormalTok{\textbackslash{}documentclass[twocolumn]\{article\}}
\NormalTok{\textbackslash{}usepackage\{blindtext\}}
\NormalTok{\textbackslash{}usepackage\{scrextend\}}
\NormalTok{\textbackslash{}addtokomafont\{labelinglabel\}\{\textbackslash{}sffamily\}}
\NormalTok{\textbackslash{}begin\{document\}}
\NormalTok{\textbackslash{}blindtext}
\NormalTok{\textbackslash{}begin\{labeling\}\{alligator\}}
\NormalTok{\textbackslash{}item [ant] really busy all the time}
\NormalTok{\textbackslash{}item [chimp] likes bananas}
\NormalTok{\textbackslash{}item [alligator] very dangerous animal, sharp teeth, long}
\NormalTok{muscular tail and a bit of text that is longer than one}
\NormalTok{line and shows the alignment of text quite nicely}
\NormalTok{\textbackslash{}end\{labeling\}}
\NormalTok{\textbackslash{}end\{document\}}
\end{Highlighting}
\end{Shaded}
\\


\begin{minipage}{1.0\linewidth}
\begin{center}
\includegraphics[width=1.0\linewidth,height=6.5in,keepaspectratio]{../images/29.png}
\end{center}
\raggedright{}\myfigurewithoutcaption{29}
\end{minipage}\vspace{0.75cm}



\end{longtable}


If you are on tight space limitations and only have short item
descriptions, you may want to have the list inline. Please note
that the example also shows how to change the font.

\begin{longtable}{p{1.0\linewidth}}
\begin{Shaded}
\begin{Highlighting}[]

\NormalTok{\textbackslash{}documentclass[twocolumn]\{article\}}
\NormalTok{\textbackslash{}usepackage\{blindtext\}}
\NormalTok{\textbackslash{}usepackage[inline]\{enumitem\}}
\NormalTok{\textbackslash{}usepackage\{xcolor\}}
\NormalTok{\textbackslash{}begin\{document\}}
\NormalTok{\textbackslash{}blindtext Coco likes fruit. Her favorites are:}
\NormalTok{\textbackslash{}begin\{enumerate*\}[label=\{\textbackslash{}alph*)\},font=\{\textbackslash{}color\{red!50!black\}\textbackslash{}bfseries\}]}
\NormalTok{\textbackslash{}item bananas}
\NormalTok{\textbackslash{}item apples}
\NormalTok{\textbackslash{}item oranges and}
\NormalTok{\textbackslash{}item lemons.}
\NormalTok{\textbackslash{}end\{enumerate*\}}
\NormalTok{\textbackslash{}blindtext}
\NormalTok{\textbackslash{}end\{document\}}
\end{Highlighting}
\end{Shaded}
\\


\begin{minipage}{1.0\linewidth}
\begin{center}
\includegraphics[width=1.0\linewidth,height=6.5in,keepaspectratio]{../images/30.png}
\end{center}
\raggedright{}\myfigurewithoutcaption{30}
\end{minipage}\vspace{0.75cm}



\end{longtable}
\mylref{147}{Need some details on Colors?}



If you want a horizontal list, package tasks can be handy. In
combination with a package like exsheets, you can prepare exam
papers for students.

\begin{longtable}{p{1.0\linewidth}}
\begin{Shaded}
\begin{Highlighting}[]

\NormalTok{\textbackslash{}documentclass\{article\}}
\NormalTok{\textbackslash{}usepackage\{blindtext\}}
\NormalTok{\textbackslash{}usepackage\{tasks\}}
\NormalTok{\textbackslash{}begin\{document\}}
\NormalTok{Which one of the entries does not fit with the others?}
\NormalTok{\textbackslash{}begin\{tasks\}(4)}
\NormalTok{\textbackslash{}task mercury}
\NormalTok{\textbackslash{}task iron}
\NormalTok{\textbackslash{}task lead}
\NormalTok{\textbackslash{}task zinc}
\NormalTok{\textbackslash{}end\{tasks\}}
\NormalTok{\textbackslash{}end\{document\}}
\end{Highlighting}
\end{Shaded}
\\


\begin{minipage}{1.0\linewidth}
\begin{center}
\includegraphics[width=1.0\linewidth,height=6.5in,keepaspectratio]{../images/31.png}
\end{center}
\raggedright{}\myfigurewithoutcaption{31}
\end{minipage}\vspace{0.75cm}



\end{longtable}

\section{Customizing lists}
\label{189}

Especially when dealing with lists containing of just a few words
per item, the standard lists take up too much space and you want
to customize the appearance. Package \LaTeXTT{enumitem}
helps you by providing a simple interface.

You can change the appearance of lists globally in the preamble,
or just for single lists using the optional argument of the
environment. Have a look at the following example where the list
on the right is more compact using {\ttfamily \setmainfont[Path=/usr/share/fonts/truetype/cmu/,UprightFont=cmunrm.ttf,BoldFont=cmunbx.ttf,ItalicFont=cmunti.ttf,BoldItalicFont=cmunbi.ttf]{cmuntt.ttf}\setmonofont[Path=/usr/share/fonts/truetype/cmu/,UprightFont=cmuntt.ttf,BoldFont=cmuntb.ttf,ItalicFont=cmunit.ttf,BoldItalicFont=cmuntx.ttf]{cmuntt.ttf}\ttfamily noitemsep}\setmainfont[Path=/usr/share/fonts/truetype/cmu/,UprightFont=cmunrm.ttf,BoldFont=cmunbx.ttf,ItalicFont=cmunti.ttf,BoldItalicFont=cmunbi.ttf]{cmunrm.ttf}\setmonofont[Path=/usr/share/fonts/truetype/cmu/,UprightFont=cmuntt.ttf,BoldFont=cmuntb.ttf,ItalicFont=cmunit.ttf,BoldItalicFont=cmuntx.ttf]{cmunrm.ttf}.

\begin{longtable}{p{1.0\linewidth}}
\begin{Shaded}
\begin{Highlighting}[]

\NormalTok{\textbackslash{}documentclass[twocolumn]\{article\}}
\NormalTok{\textbackslash{}usepackage\{blindtext\}}
\NormalTok{\textbackslash{}usepackage\{enumitem\}}
\NormalTok{\textbackslash{}begin\{document\}}
\NormalTok{\textbackslash{}blindtext}
\NormalTok{\textbackslash{}begin\{itemize\}}
\NormalTok{\textbackslash{}item more work}
\NormalTok{\textbackslash{}item more responsibility}
\NormalTok{\textbackslash{}item more satisfaction}
\NormalTok{\textbackslash{}end\{itemize\}}
\NormalTok{\textbackslash{}blindtext}
\NormalTok{\textbackslash{}newpage}
\NormalTok{\textbackslash{}blindtext}
\NormalTok{\textbackslash{}begin\{itemize\}[noitemsep]}
\NormalTok{\textbackslash{}item more work}
\NormalTok{\textbackslash{}item more responsibility}
\NormalTok{\textbackslash{}item more satisfaction}
\NormalTok{\textbackslash{}end\{itemize\}}
\NormalTok{\textbackslash{}blindtext}
\NormalTok{\textbackslash{}end\{document\}}
\end{Highlighting}
\end{Shaded}
\\


\begin{minipage}{1.0\linewidth}
\begin{center}
\includegraphics[width=1.0\linewidth,height=6.5in,keepaspectratio]{../images/32.png}
\end{center}
\raggedright{}\myfigurewithoutcaption{32}
\end{minipage}\vspace{0.75cm}



\end{longtable}

An example for alignment and the width of the label. 
\begin{longtable}{p{1.0\linewidth}}
\begin{Shaded}
\begin{Highlighting}[]

\NormalTok{\textbackslash{}documentclass[twocolumn]\{article\}}
\NormalTok{\textbackslash{}usepackage\{blindtext\}}
\NormalTok{\textbackslash{}usepackage\{enumitem\}}
\NormalTok{\textbackslash{}begin\{document\}}
\NormalTok{\textbackslash{}blindtext Coco likes fruit. Her favourites are:}
\NormalTok{\textbackslash{}begin\{description\}[align=left]}
\NormalTok{\textbackslash{}item [Kate] some detail}
\NormalTok{\textbackslash{}item [Christina]some detail}
\NormalTok{\textbackslash{}item [Laura]some detail}
\NormalTok{\textbackslash{}end\{description\}}
\NormalTok{\textbackslash{}begin\{description\}[align=right]}
\NormalTok{\textbackslash{}item [Kate] some detail}
\NormalTok{\textbackslash{}item [Christina]some detail}
\NormalTok{\textbackslash{}item [Laura]some detail}
\NormalTok{\textbackslash{}end\{description\}}
\NormalTok{\textbackslash{}begin\{description\}[align=right,labelwidth=3cm]}
\NormalTok{\textbackslash{}item [Kate] some detail}
\NormalTok{\textbackslash{}item [Christina]some detail}
\NormalTok{\textbackslash{}item [Laura]some detail}
\NormalTok{\textbackslash{}end\{description\}}
\NormalTok{\textbackslash{}blindtext}
\NormalTok{\textbackslash{}end\{document\}}
\end{Highlighting}
\end{Shaded}
\\


\begin{minipage}{1.0\linewidth}
\begin{center}
\includegraphics[width=1.0\linewidth,height=6.5in,keepaspectratio]{../images/33.png}
\end{center}
\raggedright{}\myfigurewithoutcaption{33}
\end{minipage}\vspace{0.75cm}



\end{longtable}

The documentation of package enumitem goes into more detail with
respect to what can be changed and how. You can even define your
own lists.
Environments like {\ttfamily \setmainfont[Path=/usr/share/fonts/truetype/cmu/,UprightFont=cmunrm.ttf,BoldFont=cmunbx.ttf,ItalicFont=cmunti.ttf,BoldItalicFont=cmunbi.ttf]{cmuntt.ttf}\setmonofont[Path=/usr/share/fonts/truetype/cmu/,UprightFont=cmuntt.ttf,BoldFont=cmuntb.ttf,ItalicFont=cmunit.ttf,BoldItalicFont=cmuntx.ttf]{cmuntt.ttf}\ttfamily labeling}{$\text{ }$}\setmainfont[Path=/usr/share/fonts/truetype/cmu/,UprightFont=cmunrm.ttf,BoldFont=cmunbx.ttf,ItalicFont=cmunti.ttf,BoldItalicFont=cmunbi.ttf]{cmunrm.ttf}\setmonofont[Path=/usr/share/fonts/truetype/cmu/,UprightFont=cmuntt.ttf,BoldFont=cmuntb.ttf,ItalicFont=cmunit.ttf,BoldItalicFont=cmuntx.ttf]{cmunrm.ttf} and {\ttfamily \setmainfont[Path=/usr/share/fonts/truetype/cmu/,UprightFont=cmunrm.ttf,BoldFont=cmunbx.ttf,ItalicFont=cmunti.ttf,BoldItalicFont=cmunbi.ttf]{cmuntt.ttf}\setmonofont[Path=/usr/share/fonts/truetype/cmu/,UprightFont=cmuntt.ttf,BoldFont=cmuntb.ttf,ItalicFont=cmunit.ttf,BoldItalicFont=cmuntx.ttf]{cmuntt.ttf}\ttfamily tasks}{$\text{ }$}\setmainfont[Path=/usr/share/fonts/truetype/cmu/,UprightFont=cmunrm.ttf,BoldFont=cmunbx.ttf,ItalicFont=cmunti.ttf,BoldItalicFont=cmunbi.ttf]{cmunrm.ttf}\setmonofont[Path=/usr/share/fonts/truetype/cmu/,UprightFont=cmuntt.ttf,BoldFont=cmuntb.ttf,ItalicFont=cmunit.ttf,BoldItalicFont=cmuntx.ttf]{cmunrm.ttf} 
can be changed differently, details can be found in the package 
documentation respectively.

\section{Easylist package}
\label{190}
The \LaTeXTT{easylist} package allows you to create list using a more convenient syntax and with infinite nested levels.
It is also very customizable.

Load the package with the control character as optional argument:
\begin{Shaded}
\begin{Highlighting}[]

\NormalTok{\textbackslash{}usepackage[ampersand]\{easylist\}}
 
\end{Highlighting}
\end{Shaded}


The \LaTeXTT{easylist} environment will default to enumerations.
\begin{Shaded}
\begin{Highlighting}[]

\NormalTok{\textbackslash{}begin\{easylist\}}
\NormalTok{& Main item~:}
\NormalTok{&& Sub item.}
\NormalTok{&& Another sub item.}
\NormalTok{\textbackslash{}end\{easylist\}}
 
\end{Highlighting}
\end{Shaded}


It features predefined styles which you can set as optional argument.
\begin{Shaded}
\begin{Highlighting}[]

\NormalTok{\textbackslash{}begin\{easylist\}[itemize]}
\CommentTok{% ...}
\NormalTok{\textbackslash{}end\{easylist\}}
 
\end{Highlighting}
\end{Shaded}


Available styles:
\begin{myitemize}
\item{}  \LaTeXTT{tractatus}
\item{}  \LaTeXTT{checklist} -{} All items have empty check boxes next to them
\item{}  \LaTeXTT{booktoc} -{} Approximately the format used by the table of contents of the book class
\item{}  \LaTeXTT{articletoc} -{} Approximately the format used by the table of contents of the article class
\item{}  \LaTeXTT{enumerate} -{} The default
\item{}  \LaTeXTT{itemize}
\end{myitemize}


You can customize lists with the \LaTeXTT{\textbackslash{}ListProperties(...)} command and revert back the customization with \LaTeXTT{\textbackslash{}newlist\{\}}. Yes, that\textquotesingle{}s parentheses for \LaTeXTT{\textbackslash{}ListProperties} parameters.

The \LaTeXTT{Style} parameter sets the style of counters and text, the \LaTeXTT{Style*} parameter sets the style of counters, and the \LaTeXTT{Style**} parameter sets the style of text. The parameter \LaTeXTT{Numbers} determines the way that the numbers are displayed and the possible values are \LaTeXTT{r} or \LaTeXTT{R} (for lower and upper case Roman numerals), \LaTeXTT{l} or \LaTeXTT{L} (for lower and upper case letters), \LaTeXTT{a} (for Arabic numbers, the default), and \LaTeXTT{z} (for Zapf\textquotesingle{}s Dingbats).

The \LaTeXTT{FinalMark} parameter sets the punctuation of the final counter (Ex: \LaTeXTT{FinalMark3=\{)\}}) while \LaTeXTT{FinalSpace} sets the amount of space between the item and the item\textquotesingle{}s text. The \LaTeXTT{Margin} parameter sets the distance from the left margin (Ex: \LaTeXTT{FinalSpace2=1cm}). The \LaTeXTT{Progressive} parameter sets the distance from the left margin of all items in proportion to their level.

The \LaTeXTT{Hide = n} parameter prevents the first \LaTeXTT{n} counters from appearing in all levels. If there is a number after a parameter (Ex: \LaTeXTT{Style3*}) then this numbers indicates the level that it will affect (Ex: \LaTeXTT{Style3=\textbackslash{}color\{red\}}).

Example of custom enumerate:
\begin{Shaded}
\begin{Highlighting}[]

\NormalTok{\textbackslash{}begin\{easylist\}[enumerate]}
\NormalTok{\textbackslash{}ListProperties(Style2*=,Numbers=a,Numbers1=R,FinalMark=\{)\})}
\NormalTok{& Main item~:}
\NormalTok{&& Sub item.}
\NormalTok{&& Another sub item.}
\NormalTok{\textbackslash{}end\{easylist\}}
 
\end{Highlighting}
\end{Shaded}


Note that we put the \LaTeXTT{FinalMark} argument between \LaTeXTT{\{\}} to avoid LaTeX understanding it as the end of the properties list.
Now we change the default properties to print a custom itemize:

\begin{longtable}{p{1.0\linewidth}}
\begin{Shaded}
\begin{Highlighting}[]

\NormalTok{\textbackslash{}usepackage\{amssymb\}}
\NormalTok{\textbackslash{}ListProperties(Hide=100, Hang=true, Progressive=3ex, Style*=-- ,}
\NormalTok{Style2*=$\textbackslash{}bullet$ ,Style3*=$\textbackslash{}circ$ ,Style4*=\textbackslash{}tiny$\textbackslash{}blacksquare$ )}
\CommentTok{% ...}
 
\NormalTok{\textbackslash{}begin\{easylist\}}
\NormalTok{& Blah}
\NormalTok{& Blah}
\NormalTok{&& Blah}
\NormalTok{&&& Blah}
\NormalTok{&&&& Blah}
\NormalTok{&&&&& Blah}
\NormalTok{\textbackslash{}end\{easylist\}}
\end{Highlighting}
\end{Shaded}
\\

– Blah$\text{ }$\newline{}

{\mbox{$~$}} {$\bullet$} Blah$\text{ }$\newline{}

{\mbox{$~$}}{\mbox{$~$}} {$\circ$} Blah$\text{ }$\newline{}

{\mbox{$~$}}{\mbox{$~$}}{\mbox{$~$}} {$\blacksquare$} Blah$\text{ }$\newline{}

{\mbox{$~$}}{\mbox{$~$}}{\mbox{$~$}}{\mbox{$~$}} – Blah
 
\end{longtable}

Spaces in \LaTeXTT{Style} parameters are important. The \LaTeXTT{Style*} parameter acts as a default value and \LaTeXTT{easylist} will use a medium dash for level 1, 5 and onward. 

You can also define custom styles using LaTeX macros:
\begin{Shaded}
\begin{Highlighting}[]

\NormalTok{\textbackslash{}newcommand\textbackslash{}myitemize\{\textbackslash{}ListProperties(Hide=100, Hang=true, Progressive=3ex,}
 \NormalTok{Style*=$\textbackslash{}star$ )\}}
\NormalTok{\textbackslash{}newcommand\textbackslash{}myenumerate\{\textbackslash{}ListProperties(Space=2\textbackslash{}baselineskip)\}}
 
\CommentTok{% ...}
\NormalTok{\textbackslash{}begin\{easylist\} \textbackslash{}myitemize}
\NormalTok{& Blah}
\NormalTok{\textbackslash{}end\{easylist\}}
 
\end{Highlighting}
\end{Shaded}


Important note: \LaTeXTT{easylist} has some drawbacks.
First if you need to put an easylist inside an environment using the same control character as the one specified for easylist, you may get an error.
To circumvent it, use the following commands provided by easylist:

\begin{Shaded}
\begin{Highlighting}[]

\NormalTok{\textbackslash{}Activate}
\NormalTok{\textbackslash{}begin\{easylist\}}
\NormalTok{& ...}
\NormalTok{\textbackslash{}end\{easylist\}}
\NormalTok{\textbackslash{}Deactivate}
 
\end{Highlighting}
\end{Shaded}


Besides using \LaTeXTT{easylist} along with figures may cause some trouble to the layout and the indentation.
LaTeX lists do not have this problem.

To use easylist with Beamer, each frame that uses easylist must be marked as fragile:

\begin{Shaded}
\begin{Highlighting}[]

\NormalTok{\textbackslash{}begin\{frame\}[fragile]}
\NormalTok{...}
\NormalTok{\textbackslash{}begin\{easylist\}[itemize]}
\NormalTok{...}
\NormalTok{\textbackslash{}end\{easylist\}}
\NormalTok{...}
\NormalTok{\textbackslash{}end\{frame\}}
 
\end{Highlighting}
\end{Shaded}






\myhref{https://sr.wikibooks.org/wiki/LaTeX\%2F\%D0\%A1\%D1\%82\%D1\%80\%D1\%83\%D0\%BA\%D1\%82\%D1\%83\%D1\%80\%D0\%B0\%20\%D0\%BB\%D0\%B8\%D1\%81\%D1\%82\%D0\%B8}{sr:LaTeX/Структура листи}\chapter{Special Characters}

\myminitoc
\label{191}

\label{192}


In this chapter we will tackle matters related to input encoding, typesetting diacritics and special characters.

In the following document, we will refer to {\itshape \setmainfont[Path=/usr/share/fonts/truetype/cmu/,UprightFont=cmunrm.ttf,BoldFont=cmunbx.ttf,ItalicFont=cmunti.ttf,BoldItalicFont=cmunbi.ttf]{cmunti.ttf}\setmonofont[Path=/usr/share/fonts/truetype/cmu/,UprightFont=cmuntt.ttf,BoldFont=cmuntb.ttf,ItalicFont=cmunit.ttf,BoldItalicFont=cmuntx.ttf]{cmunti.ttf}\itshape special characters}{$\text{ }$}\setmainfont[Path=/usr/share/fonts/truetype/cmu/,UprightFont=cmunrm.ttf,BoldFont=cmunbx.ttf,ItalicFont=cmunti.ttf,BoldItalicFont=cmunbi.ttf]{cmunrm.ttf}\setmonofont[Path=/usr/share/fonts/truetype/cmu/,UprightFont=cmuntt.ttf,BoldFont=cmuntb.ttf,ItalicFont=cmunit.ttf,BoldItalicFont=cmuntx.ttf]{cmunrm.ttf} for all symbols other than {\itshape \setmainfont[Path=/usr/share/fonts/truetype/cmu/,UprightFont=cmunrm.ttf,BoldFont=cmunbx.ttf,ItalicFont=cmunti.ttf,BoldItalicFont=cmunbi.ttf]{cmunti.ttf}\setmonofont[Path=/usr/share/fonts/truetype/cmu/,UprightFont=cmuntt.ttf,BoldFont=cmuntb.ttf,ItalicFont=cmunit.ttf,BoldItalicFont=cmuntx.ttf]{cmunti.ttf}\itshape A-{}Za-{}z0-{}9}{$\text{ }$}\setmainfont[Path=/usr/share/fonts/truetype/cmu/,UprightFont=cmunrm.ttf,BoldFont=cmunbx.ttf,ItalicFont=cmunti.ttf,BoldItalicFont=cmunbi.ttf]{cmunrm.ttf}\setmonofont[Path=/usr/share/fonts/truetype/cmu/,UprightFont=cmuntt.ttf,BoldFont=cmuntb.ttf,ItalicFont=cmunit.ttf,BoldItalicFont=cmuntx.ttf]{cmunrm.ttf} and English punctuation marks.

This chapter is tightly linked with the font encoding issue. You should have a look at \mylref{163}{Fonts} on the topic.

Some languages usually need a dedicated input system to ease document writing. This is the case for Arabic, Chinese, Japanese, Korean and others. This specific matter will be tackled in \mylref{209}{Internationalization}.

The rules for producing characters with diacritical marks, such as accents, differ somewhat depending whether you are in text mode, math mode, or the tabbing environment.
\section{Input encoding}
\label{193}
\subsection{A technical matter}
\label{194}

Most of the modern computer systems allow you to input letters of national alphabets directly from the keyboard. If you tried to input these special characters in your LaTeX source file and compiled it, you may have noticed that they do not get printed at all.

A LaTeX source document is a plain text file. A computer stores data in a binary format, that is a sequence of bits (0 and 1). To display a plain text file, we need a code which tells which sequence of bits corresponds to which sequence of characters. This association is called {\itshape \setmainfont[Path=/usr/share/fonts/truetype/cmu/,UprightFont=cmunrm.ttf,BoldFont=cmunbx.ttf,ItalicFont=cmunti.ttf,BoldItalicFont=cmunbi.ttf]{cmunti.ttf}\setmonofont[Path=/usr/share/fonts/truetype/cmu/,UprightFont=cmuntt.ttf,BoldFont=cmuntb.ttf,ItalicFont=cmunit.ttf,BoldItalicFont=cmuntx.ttf]{cmunti.ttf}\itshape input encoding}\setmainfont[Path=/usr/share/fonts/truetype/cmu/,UprightFont=cmunrm.ttf,BoldFont=cmunbx.ttf,ItalicFont=cmunti.ttf,BoldItalicFont=cmunbi.ttf]{cmunrm.ttf}\setmonofont[Path=/usr/share/fonts/truetype/cmu/,UprightFont=cmuntt.ttf,BoldFont=cmuntb.ttf,ItalicFont=cmunit.ttf,BoldItalicFont=cmuntx.ttf]{cmunrm.ttf}, {\itshape \setmainfont[Path=/usr/share/fonts/truetype/cmu/,UprightFont=cmunrm.ttf,BoldFont=cmunbx.ttf,ItalicFont=cmunti.ttf,BoldItalicFont=cmunbi.ttf]{cmunti.ttf}\setmonofont[Path=/usr/share/fonts/truetype/cmu/,UprightFont=cmuntt.ttf,BoldFont=cmuntb.ttf,ItalicFont=cmunit.ttf,BoldItalicFont=cmuntx.ttf]{cmunti.ttf}\itshape character encoding}\setmainfont[Path=/usr/share/fonts/truetype/cmu/,UprightFont=cmunrm.ttf,BoldFont=cmunbx.ttf,ItalicFont=cmunti.ttf,BoldItalicFont=cmunbi.ttf]{cmunrm.ttf}\setmonofont[Path=/usr/share/fonts/truetype/cmu/,UprightFont=cmuntt.ttf,BoldFont=cmuntb.ttf,ItalicFont=cmunit.ttf,BoldItalicFont=cmuntx.ttf]{cmunrm.ttf}, or more informally {\itshape \setmainfont[Path=/usr/share/fonts/truetype/cmu/,UprightFont=cmunrm.ttf,BoldFont=cmunbx.ttf,ItalicFont=cmunti.ttf,BoldItalicFont=cmunbi.ttf]{cmunti.ttf}\setmonofont[Path=/usr/share/fonts/truetype/cmu/,UprightFont=cmuntt.ttf,BoldFont=cmuntb.ttf,ItalicFont=cmunit.ttf,BoldItalicFont=cmuntx.ttf]{cmunti.ttf}\itshape charset}\setmainfont[Path=/usr/share/fonts/truetype/cmu/,UprightFont=cmunrm.ttf,BoldFont=cmunbx.ttf,ItalicFont=cmunti.ttf,BoldItalicFont=cmunbi.ttf]{cmunrm.ttf}\setmonofont[Path=/usr/share/fonts/truetype/cmu/,UprightFont=cmuntt.ttf,BoldFont=cmuntb.ttf,ItalicFont=cmunit.ttf,BoldItalicFont=cmuntx.ttf]{cmunrm.ttf}.

For historical reasons, there are many different input encodings. There is an attempt to unify all the encoding with a specification that contains all existent symbols that are known from human history. This specification is {\itshape \setmainfont[Path=/usr/share/fonts/truetype/cmu/,UprightFont=cmunrm.ttf,BoldFont=cmunbx.ttf,ItalicFont=cmunti.ttf,BoldItalicFont=cmunbi.ttf]{cmunti.ttf}\setmonofont[Path=/usr/share/fonts/truetype/cmu/,UprightFont=cmuntt.ttf,BoldFont=cmuntb.ttf,ItalicFont=cmunit.ttf,BoldItalicFont=cmuntx.ttf]{cmunti.ttf}\itshape Unicode}\setmainfont[Path=/usr/share/fonts/truetype/cmu/,UprightFont=cmunrm.ttf,BoldFont=cmunbx.ttf,ItalicFont=cmunti.ttf,BoldItalicFont=cmunbi.ttf]{cmunrm.ttf}\setmonofont[Path=/usr/share/fonts/truetype/cmu/,UprightFont=cmuntt.ttf,BoldFont=cmuntb.ttf,ItalicFont=cmunit.ttf,BoldItalicFont=cmuntx.ttf]{cmunrm.ttf}. It only defines {\itshape \setmainfont[Path=/usr/share/fonts/truetype/cmu/,UprightFont=cmunrm.ttf,BoldFont=cmunbx.ttf,ItalicFont=cmunti.ttf,BoldItalicFont=cmunbi.ttf]{cmunti.ttf}\setmonofont[Path=/usr/share/fonts/truetype/cmu/,UprightFont=cmuntt.ttf,BoldFont=cmuntb.ttf,ItalicFont=cmunit.ttf,BoldItalicFont=cmuntx.ttf]{cmunti.ttf}\itshape code points}\setmainfont[Path=/usr/share/fonts/truetype/cmu/,UprightFont=cmunrm.ttf,BoldFont=cmunbx.ttf,ItalicFont=cmunti.ttf,BoldItalicFont=cmunbi.ttf]{cmunrm.ttf}\setmonofont[Path=/usr/share/fonts/truetype/cmu/,UprightFont=cmuntt.ttf,BoldFont=cmuntb.ttf,ItalicFont=cmunit.ttf,BoldItalicFont=cmuntx.ttf]{cmunrm.ttf}, which is a number for a symbol, but not the way symbols are represented in binary value. For that, unicode encodings are in charge. There are also several unicode encodings available, UTF-{}8 being one of them.

The ASCII encoding is an encoding which defines 128 characters on 7 bits. Its widespread use has led the vast majority of encodings to have backward compatibility with ASCII, by defining the first 128 characters the same way. The other characters are added using more bits (8 or more).

This is actually a big issue, since if you do not use the right encoding to display a file, it will show weird characters. What most programs try to do is guess statistically the encoding by analyzing the frequent sequences of bits. Sadly, it is not 100\% safe. Some text editors may not bother guessing the encoding and will just use the OS default encoding.
You should consider that other people might not be able to display directly your input files on their computer, because the default encoding for text file is different. It does not mean that the user cannot use another encoding, besides the default one, only that it has to be configured.
For example, the German umlaut ä on OS/2 is encoded as 132, with Latin1 it is encoded as 228, while in Cyrillic encoding cp1251 this letter does not exist at all. Therefore you should {\itshape \setmainfont[Path=/usr/share/fonts/truetype/cmu/,UprightFont=cmunrm.ttf,BoldFont=cmunbx.ttf,ItalicFont=cmunti.ttf,BoldItalicFont=cmunbi.ttf]{cmunti.ttf}\setmonofont[Path=/usr/share/fonts/truetype/cmu/,UprightFont=cmuntt.ttf,BoldFont=cmuntb.ttf,ItalicFont=cmunit.ttf,BoldItalicFont=cmuntx.ttf]{cmunti.ttf}\itshape consider encoding with care}\setmainfont[Path=/usr/share/fonts/truetype/cmu/,UprightFont=cmunrm.ttf,BoldFont=cmunbx.ttf,ItalicFont=cmunti.ttf,BoldItalicFont=cmunbi.ttf]{cmunrm.ttf}\setmonofont[Path=/usr/share/fonts/truetype/cmu/,UprightFont=cmuntt.ttf,BoldFont=cmuntb.ttf,ItalicFont=cmunit.ttf,BoldItalicFont=cmuntx.ttf]{cmunrm.ttf}.

The following table shows the default  encodings for some operating systems.

\begin{longtable}{|>{\RaggedRight}p{0.51450\linewidth}|>{\RaggedRight}p{0.24197\linewidth}|>{\RaggedRight}p{0.15782\linewidth}|} \hline 
\multirow{2}{\linewidth}{{\bfseries \hspace*{0pt}\ignorespaces{}\hspace*{0pt}Operating system}}&\multicolumn{2}{|>{\RaggedRight}p{0.41190\linewidth}|}{{\bfseries \hspace*{0pt}\ignorespaces{}\hspace*{0pt}Default Encodings}}\\ \cline{2-2}\cline{3-3} \multicolumn{1}{|c|}{}&{\bfseries \hspace*{0pt}\ignorespaces{}\hspace*{0pt}Western Latin}&{\bfseries \hspace*{0pt}\ignorespaces{}\hspace*{0pt}Cyrillic}\endhead  \hline \hspace*{0pt}\ignorespaces{}\hspace*{0pt} Modern Unices (*BSD, Mac OS X, GNU/Linux)&\hspace*{0pt}\ignorespaces{}\hspace*{0pt} {\ttfamily \setmainfont[Path=/usr/share/fonts/truetype/cmu/,UprightFont=cmunrm.ttf,BoldFont=cmunbx.ttf,ItalicFont=cmunti.ttf,BoldItalicFont=cmunbi.ttf]{cmuntt.ttf}\setmonofont[Path=/usr/share/fonts/truetype/cmu/,UprightFont=cmuntt.ttf,BoldFont=cmuntb.ttf,ItalicFont=cmunit.ttf,BoldItalicFont=cmuntx.ttf]{cmuntt.ttf}\ttfamily utf-{}8}&\hspace*{0pt}\ignorespaces{}\hspace*{0pt}{$\text{ }$}\setmainfont[Path=/usr/share/fonts/truetype/cmu/,UprightFont=cmunrm.ttf,BoldFont=cmunbx.ttf,ItalicFont=cmunti.ttf,BoldItalicFont=cmunbi.ttf]{cmunrm.ttf}\setmonofont[Path=/usr/share/fonts/truetype/cmu/,UprightFont=cmuntt.ttf,BoldFont=cmuntb.ttf,ItalicFont=cmunit.ttf,BoldItalicFont=cmuntx.ttf]{cmunrm.ttf} {\ttfamily \setmainfont[Path=/usr/share/fonts/truetype/cmu/,UprightFont=cmunrm.ttf,BoldFont=cmunbx.ttf,ItalicFont=cmunti.ttf,BoldItalicFont=cmunbi.ttf]{cmuntt.ttf}\setmonofont[Path=/usr/share/fonts/truetype/cmu/,UprightFont=cmuntt.ttf,BoldFont=cmuntb.ttf,ItalicFont=cmunit.ttf,BoldItalicFont=cmuntx.ttf]{cmuntt.ttf}\ttfamily utf-{}8}\\ \hline \hspace*{0pt}\ignorespaces{}\hspace*{0pt}\setmainfont[Path=/usr/share/fonts/truetype/cmu/,UprightFont=cmunrm.ttf,BoldFont=cmunbx.ttf,ItalicFont=cmunti.ttf,BoldItalicFont=cmunbi.ttf]{cmunrm.ttf}\setmonofont[Path=/usr/share/fonts/truetype/cmu/,UprightFont=cmuntt.ttf,BoldFont=cmuntb.ttf,ItalicFont=cmunit.ttf,BoldItalicFont=cmuntx.ttf]{cmunrm.ttf}Mac (before OS X)&\hspace*{0pt}\ignorespaces{}\hspace*{0pt}{\ttfamily \setmainfont[Path=/usr/share/fonts/truetype/cmu/,UprightFont=cmunrm.ttf,BoldFont=cmunbx.ttf,ItalicFont=cmunti.ttf,BoldItalicFont=cmunbi.ttf]{cmuntt.ttf}\setmonofont[Path=/usr/share/fonts/truetype/cmu/,UprightFont=cmuntt.ttf,BoldFont=cmuntb.ttf,ItalicFont=cmunit.ttf,BoldItalicFont=cmuntx.ttf]{cmuntt.ttf}\ttfamily applemac}&\hspace*{0pt}\ignorespaces{}\hspace*{0pt}{\ttfamily maccyr}\\ \hline \hspace*{0pt}\ignorespaces{}\hspace*{0pt}\setmainfont[Path=/usr/share/fonts/truetype/cmu/,UprightFont=cmunrm.ttf,BoldFont=cmunbx.ttf,ItalicFont=cmunti.ttf,BoldItalicFont=cmunbi.ttf]{cmunrm.ttf}\setmonofont[Path=/usr/share/fonts/truetype/cmu/,UprightFont=cmuntt.ttf,BoldFont=cmuntb.ttf,ItalicFont=cmunit.ttf,BoldItalicFont=cmuntx.ttf]{cmunrm.ttf}Unix (Old)&\hspace*{0pt}\ignorespaces{}\hspace*{0pt}{\ttfamily \setmainfont[Path=/usr/share/fonts/truetype/cmu/,UprightFont=cmunrm.ttf,BoldFont=cmunbx.ttf,ItalicFont=cmunti.ttf,BoldItalicFont=cmunbi.ttf]{cmuntt.ttf}\setmonofont[Path=/usr/share/fonts/truetype/cmu/,UprightFont=cmuntt.ttf,BoldFont=cmuntb.ttf,ItalicFont=cmunit.ttf,BoldItalicFont=cmuntx.ttf]{cmuntt.ttf}\ttfamily latin1}&\hspace*{0pt}\ignorespaces{}\hspace*{0pt}{\ttfamily koi8-{}ru}\\ \hline \hspace*{0pt}\ignorespaces{}\hspace*{0pt}\setmainfont[Path=/usr/share/fonts/truetype/cmu/,UprightFont=cmunrm.ttf,BoldFont=cmunbx.ttf,ItalicFont=cmunti.ttf,BoldItalicFont=cmunbi.ttf]{cmunrm.ttf}\setmonofont[Path=/usr/share/fonts/truetype/cmu/,UprightFont=cmuntt.ttf,BoldFont=cmuntb.ttf,ItalicFont=cmunit.ttf,BoldItalicFont=cmuntx.ttf]{cmunrm.ttf}Windows&\hspace*{0pt}\ignorespaces{}\hspace*{0pt}{\ttfamily \setmainfont[Path=/usr/share/fonts/truetype/cmu/,UprightFont=cmunrm.ttf,BoldFont=cmunbx.ttf,ItalicFont=cmunti.ttf,BoldItalicFont=cmunbi.ttf]{cmuntt.ttf}\setmonofont[Path=/usr/share/fonts/truetype/cmu/,UprightFont=cmuntt.ttf,BoldFont=cmuntb.ttf,ItalicFont=cmunit.ttf,BoldItalicFont=cmuntx.ttf]{cmuntt.ttf}\ttfamily ansinew}\setmainfont[Path=/usr/share/fonts/truetype/cmu/,UprightFont=cmunrm.ttf,BoldFont=cmunbx.ttf,ItalicFont=cmunti.ttf,BoldItalicFont=cmunbi.ttf]{cmunrm.ttf}\setmonofont[Path=/usr/share/fonts/truetype/cmu/,UprightFont=cmuntt.ttf,BoldFont=cmuntb.ttf,ItalicFont=cmunit.ttf,BoldItalicFont=cmuntx.ttf]{cmunrm.ttf}, {\ttfamily \setmainfont[Path=/usr/share/fonts/truetype/cmu/,UprightFont=cmunrm.ttf,BoldFont=cmunbx.ttf,ItalicFont=cmunti.ttf,BoldItalicFont=cmunbi.ttf]{cmuntt.ttf}\setmonofont[Path=/usr/share/fonts/truetype/cmu/,UprightFont=cmuntt.ttf,BoldFont=cmuntb.ttf,ItalicFont=cmunit.ttf,BoldItalicFont=cmuntx.ttf]{cmuntt.ttf}\ttfamily cp1252}&\hspace*{0pt}\ignorespaces{}\hspace*{0pt}{\ttfamily cp1251}\\ \hline \hspace*{0pt}\ignorespaces{}\hspace*{0pt}\setmainfont[Path=/usr/share/fonts/truetype/cmu/,UprightFont=cmunrm.ttf,BoldFont=cmunbx.ttf,ItalicFont=cmunti.ttf,BoldItalicFont=cmunbi.ttf]{cmunrm.ttf}\setmonofont[Path=/usr/share/fonts/truetype/cmu/,UprightFont=cmuntt.ttf,BoldFont=cmuntb.ttf,ItalicFont=cmunit.ttf,BoldItalicFont=cmuntx.ttf]{cmunrm.ttf}DOS, OS/2&\hspace*{0pt}\ignorespaces{}\hspace*{0pt}{\ttfamily \setmainfont[Path=/usr/share/fonts/truetype/cmu/,UprightFont=cmunrm.ttf,BoldFont=cmunbx.ttf,ItalicFont=cmunti.ttf,BoldItalicFont=cmunbi.ttf]{cmuntt.ttf}\setmonofont[Path=/usr/share/fonts/truetype/cmu/,UprightFont=cmuntt.ttf,BoldFont=cmuntb.ttf,ItalicFont=cmunit.ttf,BoldItalicFont=cmuntx.ttf]{cmuntt.ttf}\ttfamily cp850}&\hspace*{0pt}\ignorespaces{}\hspace*{0pt}{\ttfamily cp866nav}\\ \hline 
\end{longtable}
\setmainfont[Path=/usr/share/fonts/truetype/cmu/,UprightFont=cmunrm.ttf,BoldFont=cmunbx.ttf,ItalicFont=cmunti.ttf,BoldItalicFont=cmunbi.ttf]{cmunrm.ttf}\setmonofont[Path=/usr/share/fonts/truetype/cmu/,UprightFont=cmuntt.ttf,BoldFont=cmuntb.ttf,ItalicFont=cmunit.ttf,BoldItalicFont=cmuntx.ttf]{cmunrm.ttf}

UTF-{}8 and Latin1 are not compatible. It means that if you try to open a Latin1-{}encoded file using a UTF-{}8 decoding, it will display odd symbols only if you used accents in it, since both encoding are ASCII superset they encode the {\itshape \setmainfont[Path=/usr/share/fonts/truetype/cmu/,UprightFont=cmunrm.ttf,BoldFont=cmunbx.ttf,ItalicFont=cmunti.ttf,BoldItalicFont=cmunbi.ttf]{cmunti.ttf}\setmonofont[Path=/usr/share/fonts/truetype/cmu/,UprightFont=cmuntt.ttf,BoldFont=cmuntb.ttf,ItalicFont=cmunit.ttf,BoldItalicFont=cmuntx.ttf]{cmunti.ttf}\itshape classic}{$\text{ }$}\setmainfont[Path=/usr/share/fonts/truetype/cmu/,UprightFont=cmunrm.ttf,BoldFont=cmunbx.ttf,ItalicFont=cmunti.ttf,BoldItalicFont=cmunbi.ttf]{cmunrm.ttf}\setmonofont[Path=/usr/share/fonts/truetype/cmu/,UprightFont=cmuntt.ttf,BoldFont=cmuntb.ttf,ItalicFont=cmunit.ttf,BoldItalicFont=cmuntx.ttf]{cmunrm.ttf} letters the same way. There aren\textquotesingle{}t many advantages in using Latin1 over UTF-{}8, which is technically superior. UTF-{}8 is also becoming the most widely used encoding (on the Web, in modern Unices, etc.).

\begin{TemplateInfo}{\danger}{Warning}We really urge you to use UTF-{}8 encoding. It is technically superior to most (all?) encodings, it supports the full Unicode specification (all symbols that ever existed), and is backward compatible with ASCII. Latin1 is not universal, and having multiple encoding around has always been a source of problems.\myfootnote{For a quick explanation on character sets, see  \myfnhref{http://www.joelonsoftware.com/articles/Unicode.html}{this article on Joel Spolski\textquotesingle{}s blog}.}
\end{TemplateInfo}
\subsection{Dealing with LaTeX}
\label{195}

TeX uses ASCII by default. But 128 characters is not enough to support non-{}english languages. TeX has its own way to do that with commands for every diacritical marking (see \mylref{197}{Escaped codes}). But if we want accents and other special characters to appear directly in the source file, we have to tell TeX that we want to use a different encoding.

There are several encodings available to LaTeX:
\begin{myitemize}
\item{}  ASCII: the default. Only bare english characters are supported in the source file.
\item{}  ISO-{}8859-{}1 (a.k.a. Latin 1): 8-{}bits encoding. It supports most characters for latin languages, but that\textquotesingle{}s it.
\item{}  UTF-{}8: a Unicode multi-{}byte encoding. Supports the complete Unicode specification.
\item{}  Others...
\end{myitemize}

In the following we will assume you want to use UTF-{}8.

There are some {\itshape \setmainfont[Path=/usr/share/fonts/truetype/cmu/,UprightFont=cmunrm.ttf,BoldFont=cmunbx.ttf,ItalicFont=cmunti.ttf,BoldItalicFont=cmunbi.ttf]{cmunti.ttf}\setmonofont[Path=/usr/share/fonts/truetype/cmu/,UprightFont=cmuntt.ttf,BoldFont=cmuntb.ttf,ItalicFont=cmunit.ttf,BoldItalicFont=cmuntx.ttf]{cmunti.ttf}\itshape important steps}{$\text{ }$}\setmainfont[Path=/usr/share/fonts/truetype/cmu/,UprightFont=cmunrm.ttf,BoldFont=cmunbx.ttf,ItalicFont=cmunti.ttf,BoldItalicFont=cmunbi.ttf]{cmunrm.ttf}\setmonofont[Path=/usr/share/fonts/truetype/cmu/,UprightFont=cmuntt.ttf,BoldFont=cmuntb.ttf,ItalicFont=cmunit.ttf,BoldItalicFont=cmuntx.ttf]{cmunrm.ttf} to specify encoding.
\begin{myitemize}
\item{}  Make sure your text editor decodes the file in UTF-{}8.
\item{}  Make sure it saves your file in UTF-{}8. Most text editors do not make the distinction, but some do, such as Notepad++.
\item{}  If you are working in a terminal, make sure it is set to support UTF-{}8 input and output. Some old Unix terminals may not support UTF-{}8. \myhref{https://en.wikipedia.org/wiki/PuTTY}{PuTTY} is not set to use UTF-{}8 by default, you have to configure it.
\item{}  Tell LaTeX that the source file is UTF-{}8 encoded.
\end{myitemize}


\begin{Shaded}
\begin{Highlighting}[]

\NormalTok{\textbackslash{}usepackage[utf8]\{inputenc\}}
\end{Highlighting}
\end{Shaded}


\LaTeXTT{inputenc} \myfootnote{For a detailed information on the package, see  \myfnhref{https://www.tug.org/texmf-dist/doc/latex/base/inputenc.pdf}{complete specifications written by the package\textquotesingle{}s authors}.} package tells LaTeX what the text encoding format of your {\ttfamily \setmainfont[Path=/usr/share/fonts/truetype/cmu/,UprightFont=cmunrm.ttf,BoldFont=cmunbx.ttf,ItalicFont=cmunti.ttf,BoldItalicFont=cmunbi.ttf]{cmuntt.ttf}\setmonofont[Path=/usr/share/fonts/truetype/cmu/,UprightFont=cmuntt.ttf,BoldFont=cmuntb.ttf,ItalicFont=cmunit.ttf,BoldItalicFont=cmuntx.ttf]{cmuntt.ttf}\ttfamily .tex}{$\text{ }$}\setmainfont[Path=/usr/share/fonts/truetype/cmu/,UprightFont=cmunrm.ttf,BoldFont=cmunbx.ttf,ItalicFont=cmunti.ttf,BoldItalicFont=cmunbi.ttf]{cmunrm.ttf}\setmonofont[Path=/usr/share/fonts/truetype/cmu/,UprightFont=cmuntt.ttf,BoldFont=cmuntb.ttf,ItalicFont=cmunit.ttf,BoldItalicFont=cmuntx.ttf]{cmunrm.ttf} files is.

\begin{TemplateInfo}{\danger}{Warning}If you check the character encoding ({\itshape \setmainfont[Path=/usr/share/fonts/truetype/cmu/,UprightFont=cmunrm.ttf,BoldFont=cmunbx.ttf,ItalicFont=cmunti.ttf,BoldItalicFont=cmunbi.ttf]{cmunti.ttf}\setmonofont[Path=/usr/share/fonts/truetype/cmu/,UprightFont=cmuntt.ttf,BoldFont=cmuntb.ttf,ItalicFont=cmunit.ttf,BoldItalicFont=cmuntx.ttf]{cmunti.ttf}\itshape e.g.}{$\text{ }$}\setmainfont[Path=/usr/share/fonts/truetype/cmu/,UprightFont=cmunrm.ttf,BoldFont=cmunbx.ttf,ItalicFont=cmunti.ttf,BoldItalicFont=cmunbi.ttf]{cmunrm.ttf}\setmonofont[Path=/usr/share/fonts/truetype/cmu/,UprightFont=cmuntt.ttf,BoldFont=cmuntb.ttf,ItalicFont=cmunit.ttf,BoldItalicFont=cmuntx.ttf]{cmunrm.ttf} using the Unix {\ttfamily \setmainfont[Path=/usr/share/fonts/truetype/cmu/,UprightFont=cmunrm.ttf,BoldFont=cmunbx.ttf,ItalicFont=cmunti.ttf,BoldItalicFont=cmunbi.ttf]{cmuntt.ttf}\setmonofont[Path=/usr/share/fonts/truetype/cmu/,UprightFont=cmuntt.ttf,BoldFont=cmuntb.ttf,ItalicFont=cmunit.ttf,BoldItalicFont=cmuntx.ttf]{cmuntt.ttf}\ttfamily file}{$\text{ }$}\setmainfont[Path=/usr/share/fonts/truetype/cmu/,UprightFont=cmunrm.ttf,BoldFont=cmunbx.ttf,ItalicFont=cmunti.ttf,BoldItalicFont=cmunbi.ttf]{cmunrm.ttf}\setmonofont[Path=/usr/share/fonts/truetype/cmu/,UprightFont=cmuntt.ttf,BoldFont=cmuntb.ttf,ItalicFont=cmunit.ttf,BoldItalicFont=cmuntx.ttf]{cmunrm.ttf} command), be sure that your file contains at least one special character, otherwise it will be recognized as ASCII (which is logical since UTF-{}8 is as superset of ASCII).\end{TemplateInfo}

The inputenc package allows as well the user to change the encoding {\itshape \setmainfont[Path=/usr/share/fonts/truetype/cmu/,UprightFont=cmunrm.ttf,BoldFont=cmunbx.ttf,ItalicFont=cmunti.ttf,BoldItalicFont=cmunbi.ttf]{cmunti.ttf}\setmonofont[Path=/usr/share/fonts/truetype/cmu/,UprightFont=cmuntt.ttf,BoldFont=cmuntb.ttf,ItalicFont=cmunit.ttf,BoldItalicFont=cmuntx.ttf]{cmunti.ttf}\itshape within the document}{$\text{ }$}\setmainfont[Path=/usr/share/fonts/truetype/cmu/,UprightFont=cmunrm.ttf,BoldFont=cmunbx.ttf,ItalicFont=cmunti.ttf,BoldItalicFont=cmunbi.ttf]{cmunrm.ttf}\setmonofont[Path=/usr/share/fonts/truetype/cmu/,UprightFont=cmuntt.ttf,BoldFont=cmuntb.ttf,ItalicFont=cmunit.ttf,BoldItalicFont=cmuntx.ttf]{cmunrm.ttf} by means of the command \LaTeXTT{\textbackslash{}inputencoding\{\textquotesingle{}encoding name\textquotesingle{}\}}.

\begin{Shaded}
\begin{Highlighting}[]

\NormalTok{\textbackslash{}usepackage[utf8]\{inputenc\}}
\CommentTok{% ...}
\CommentTok{% In this area}
\CommentTok{% The UTF-8 encoding is specified.}
\CommentTok{% ...}
\NormalTok{\textbackslash{}inputencoding\{latin1\}}
\CommentTok{% ...}
\CommentTok{% Here the text encoding is specified as ISO Latin-1.}
\CommentTok{% ...}
\NormalTok{\textbackslash{}inputencoding\{utf8\}}
\CommentTok{% Back to the UTF-8 encoding.}
\CommentTok{% ...}
\end{Highlighting}
\end{Shaded}

\subsection{Extending the support}
\label{196}

The LaTeX support of UTF-{}8 is fairly specific: it includes only a limited range of unicode input characters. It only defines those symbols that are known to be available with the current {\itshape \setmainfont[Path=/usr/share/fonts/truetype/cmu/,UprightFont=cmunrm.ttf,BoldFont=cmunbx.ttf,ItalicFont=cmunti.ttf,BoldItalicFont=cmunbi.ttf]{cmunti.ttf}\setmonofont[Path=/usr/share/fonts/truetype/cmu/,UprightFont=cmuntt.ttf,BoldFont=cmuntb.ttf,ItalicFont=cmunit.ttf,BoldItalicFont=cmuntx.ttf]{cmunti.ttf}\itshape font encoding}\setmainfont[Path=/usr/share/fonts/truetype/cmu/,UprightFont=cmunrm.ttf,BoldFont=cmunbx.ttf,ItalicFont=cmunti.ttf,BoldItalicFont=cmunbi.ttf]{cmunrm.ttf}\setmonofont[Path=/usr/share/fonts/truetype/cmu/,UprightFont=cmuntt.ttf,BoldFont=cmuntb.ttf,ItalicFont=cmunit.ttf,BoldItalicFont=cmuntx.ttf]{cmunrm.ttf}. You might encounter a situation where using UTF-{}8 might result in error:
\\

\TemplateSpaceIndent{$\text{ }${}!$\text{ }${}Package$\text{ }${}inputenc$\text{ }${}Error:$\text{ }${}Unicode$\text{ }${}char$\text{ }${}\textbackslash{}u8:ũ$\text{ }${}not$\text{ }${}set$\text{ }${}up$\text{ }${}for$\text{ }${}use$\text{ }${}with$\text{ }${}LaTeX.}


This is due to the utf8 definition not necessarily having a mapping of all the character glyphs you are able to enter on your keyboard. Such characters are for example 
\\

\TemplateSpaceIndent{$\text{ }${}ŷ$\text{ }${}Ŷ$\text{ }${}ũ$\text{ }${}Ũ$\text{ }${}ẽ$\text{ }${}Ẽ$\text{ }${}ĩ$\text{ }${}Ĩ}


In such case, you may try need to use the \LaTeXTT{utf8x} option to define more character combinations. \LaTeXTT{utf8x} is not officially supported, but can be viable in some cases. However it might break up compatibility with some packages like \LaTeXTT{csquotes}.

Another possiblity is to stick with \LaTeXTT{utf8} and to define the characters yourself. This is easy:
\begin{Shaded}
\begin{Highlighting}[]

\NormalTok{\textbackslash{}DeclareUnicodeCharacter\{'codepoint'\}\{'TeX sequence'\}}
\end{Highlighting}
\end{Shaded}

where \LaTeXTT{codepoint} is the unicode codepoint of the desired character. \LaTeXTT{TeX sequence} is what to print when the character matching the codepoint is met.
You may find codepoints on this \myhref{http://www.unicode.org/charts/\#symbols}{site}. Codepoints are easy to find on the web.
Example:
\begin{Shaded}
\begin{Highlighting}[]

\NormalTok{\textbackslash{}DeclareUnicodeCharacter\{0177\}\{\textbackslash{}^y\}}
\end{Highlighting}
\end{Shaded}

Now inputting \textquotesingle{}ŷ\textquotesingle{} will effectively print \textquotesingle{}ŷ\textquotesingle{}.

With XeTeX and LuaTeX the inputenc package is no longer needed. Both engines support UTF-{}8 directly and allow the use of TTF and OpenType fonts to support Unicode characters. See the \mylref{180}{Fonts} section for more information.
\section{Escaped codes}
\label{197}

In addition to direct UTF-{}8 input, LaTeX supports the composition of special characters. This is convenient if your keyboard lacks some desired accents and other diacritics.

The following accents may be placed on letters. Although \textquotesingle{}o\textquotesingle{} letter is used in most of the examples, the accents may be placed on any letter. Accents may even be placed above a \symbol{34}missing\symbol{34} letter; for example, \LaTeXTT{\textbackslash{}\~{}\{\}} produces a tilde over a blank space.

The following commands may be used only in paragraph (default) or LR (left-{}right) mode.
\begin{longtable}{|>{\RaggedRight}p{0.20405\linewidth}|>{\RaggedRight}p{0.10272\linewidth}|>{\RaggedRight}p{0.60752\linewidth}|} \hline 
{\bfseries \hspace*{0pt}\ignorespaces{}\hspace*{0pt} LaTeX command}&{\bfseries \hspace*{0pt}\ignorespaces{}\hspace*{0pt} Sample}&{\bfseries \hspace*{0pt}\ignorespaces{}\hspace*{0pt} Description}\endhead  \hline \hspace*{0pt}\ignorespaces{}\hspace*{0pt}\LaTeXTT{\textbackslash{}`\{o\}} &\hspace*{0pt}\ignorespaces{}\hspace*{0pt} ò &\hspace*{0pt}\ignorespaces{}\hspace*{0pt} grave accent\\ \hline \hspace*{0pt}\ignorespaces{}\hspace*{0pt}\LaTeXTT{\textbackslash{}\textquotesingle{}\{o\}} &\hspace*{0pt}\ignorespaces{}\hspace*{0pt} ó &\hspace*{0pt}\ignorespaces{}\hspace*{0pt} acute accent\\ \hline \hspace*{0pt}\ignorespaces{}\hspace*{0pt}\LaTeXTT{\textbackslash{}\^{}\{o\}} &\hspace*{0pt}\ignorespaces{}\hspace*{0pt} ô &\hspace*{0pt}\ignorespaces{}\hspace*{0pt} circumflex\\ \hline \hspace*{0pt}\ignorespaces{}\hspace*{0pt}\LaTeXTT{\textbackslash{}\symbol{34}\{o\}} &\hspace*{0pt}\ignorespaces{}\hspace*{0pt} ö &\hspace*{0pt}\ignorespaces{}\hspace*{0pt} umlaut, trema or dieresis\\ \hline \hspace*{0pt}\ignorespaces{}\hspace*{0pt}\LaTeXTT{\textbackslash{}H\{o\}} &\hspace*{0pt}\ignorespaces{}\hspace*{0pt} ő &\hspace*{0pt}\ignorespaces{}\hspace*{0pt} long Hungarian umlaut (double acute)\\ \hline \hspace*{0pt}\ignorespaces{}\hspace*{0pt}\LaTeXTT{\textbackslash{}\~{}\{o\}} &\hspace*{0pt}\ignorespaces{}\hspace*{0pt} õ &\hspace*{0pt}\ignorespaces{}\hspace*{0pt} tilde\\ \hline \hspace*{0pt}\ignorespaces{}\hspace*{0pt}\LaTeXTT{\textbackslash{}c\{c\}} &\hspace*{0pt}\ignorespaces{}\hspace*{0pt} ç &\hspace*{0pt}\ignorespaces{}\hspace*{0pt} cedilla\\ \hline \hspace*{0pt}\ignorespaces{}\hspace*{0pt}\LaTeXTT{\textbackslash{}k\{a\}} &\hspace*{0pt}\ignorespaces{}\hspace*{0pt} ą &\hspace*{0pt}\ignorespaces{}\hspace*{0pt} ogonek\\ \hline \hspace*{0pt}\ignorespaces{}\hspace*{0pt}\LaTeXTT{\textbackslash{}l\{\}} &\hspace*{0pt}\ignorespaces{}\hspace*{0pt} ł &\hspace*{0pt}\ignorespaces{}\hspace*{0pt} barred l (l with stroke)\\ \hline \hspace*{0pt}\ignorespaces{}\hspace*{0pt}\LaTeXTT{\textbackslash{}=\{o\}} &\hspace*{0pt}\ignorespaces{}\hspace*{0pt} ō &\hspace*{0pt}\ignorespaces{}\hspace*{0pt} macron accent (a bar over the letter)\\ \hline \hspace*{0pt}\ignorespaces{}\hspace*{0pt}\LaTeXTT{\textbackslash{}b\{o\}} &\hspace*{0pt}\ignorespaces{}\hspace*{0pt} \uline{o} &\hspace*{0pt}\ignorespaces{}\hspace*{0pt} bar under the letter \\ \hline \hspace*{0pt}\ignorespaces{}\hspace*{0pt}\LaTeXTT{\textbackslash{}.\{o\}} &\hspace*{0pt}\ignorespaces{}\hspace*{0pt} ȯ &\hspace*{0pt}\ignorespaces{}\hspace*{0pt} dot over the letter\\ \hline \hspace*{0pt}\ignorespaces{}\hspace*{0pt}\LaTeXTT{\textbackslash{}d\{u\}} &\hspace*{0pt}\ignorespaces{}\hspace*{0pt} ụ &\hspace*{0pt}\ignorespaces{}\hspace*{0pt} dot under the letter\\ \hline \hspace*{0pt}\ignorespaces{}\hspace*{0pt}\LaTeXTT{\textbackslash{}r\{a\}} &\hspace*{0pt}\ignorespaces{}\hspace*{0pt} å &\hspace*{0pt}\ignorespaces{}\hspace*{0pt} ring over the letter (for å there is also the special command \LaTeXTT{\textbackslash{}aa})\\ \hline \hspace*{0pt}\ignorespaces{}\hspace*{0pt}\LaTeXTT{\textbackslash{}u\{o\}} &\hspace*{0pt}\ignorespaces{}\hspace*{0pt} ŏ &\hspace*{0pt}\ignorespaces{}\hspace*{0pt} breve over the letter\\ \hline \hspace*{0pt}\ignorespaces{}\hspace*{0pt}\LaTeXTT{\textbackslash{}v\{s\}} &\hspace*{0pt}\ignorespaces{}\hspace*{0pt} š &\hspace*{0pt}\ignorespaces{}\hspace*{0pt} caron/háček (\symbol{34}v\symbol{34}) over the letter\\ \hline \hspace*{0pt}\ignorespaces{}\hspace*{0pt}\LaTeXTT{\textbackslash{}t\{oo\}} &\hspace*{0pt}\ignorespaces{}\hspace*{0pt} o͡o &\hspace*{0pt}\ignorespaces{}\hspace*{0pt} \symbol{34}tie\symbol{34} (inverted u) over the two letters \\ \hline \hspace*{0pt}\ignorespaces{}\hspace*{0pt}\LaTeXTT{\textbackslash{}o} &\hspace*{0pt}\ignorespaces{}\hspace*{0pt} ø &\hspace*{0pt}\ignorespaces{}\hspace*{0pt} slashed o (o with stroke)\\ \hline 
\end{longtable}


To place a diacritic on top of an i or a j, its dot has to be removed. The dotless version of these letters is accomplished by typing \LaTeXTT{\textbackslash{}i} and \LaTeXTT{\textbackslash{}j}. For example:
\begin{myitemize}
\item{}  \LaTeXTT{\textbackslash{}\^{}\{\textbackslash{}i\}} should be used for i circumflex \textquotesingle{}î\textquotesingle{};
\item{}  \LaTeXTT{\textbackslash{}\symbol{34}\{\textbackslash{}i\}} should be used for i umlaut \textquotesingle{}ï\textquotesingle{}.
\end{myitemize}


If a document is to be written completely in a language that requires particular diacritics several times, then using the right configuration allows those characters to be written directly in the document. For example, to achieve easier coding of umlauts, the babel package can be configured as \LaTeXTT{\textbackslash{}usepackage{$\text{[}$}german{$\text{]}$}\{babel\}}. This provides the short hand \LaTeXTT{\symbol{34}o} for \LaTeXTT{\textbackslash{}\symbol{34}o}. This is very useful if one needs to use some text accents in a label, since no backslash will be accepted otherwise.

More information regarding language configuration can be found in the \mylref{209}{Internationalization} section.
\section{{\itshape \setmainfont[Path=/usr/share/fonts/truetype/cmu/,UprightFont=cmunrm.ttf,BoldFont=cmunbx.ttf,ItalicFont=cmunti.ttf,BoldItalicFont=cmunbi.ttf]{cmunti.ttf}\setmonofont[Path=/usr/share/fonts/truetype/cmu/,UprightFont=cmuntt.ttf,BoldFont=cmuntb.ttf,ItalicFont=cmunit.ttf,BoldItalicFont=cmuntx.ttf]{cmunti.ttf}\itshape Less than <{}}{$\text{ }$}\setmainfont[Path=/usr/share/fonts/truetype/cmu/,UprightFont=cmunrm.ttf,BoldFont=cmunbx.ttf,ItalicFont=cmunti.ttf,BoldItalicFont=cmunbi.ttf]{cmunrm.ttf}\setmonofont[Path=/usr/share/fonts/truetype/cmu/,UprightFont=cmuntt.ttf,BoldFont=cmuntb.ttf,ItalicFont=cmunit.ttf,BoldItalicFont=cmuntx.ttf]{cmunrm.ttf} and {\itshape \setmainfont[Path=/usr/share/fonts/truetype/cmu/,UprightFont=cmunrm.ttf,BoldFont=cmunbx.ttf,ItalicFont=cmunti.ttf,BoldItalicFont=cmunbi.ttf]{cmunti.ttf}\setmonofont[Path=/usr/share/fonts/truetype/cmu/,UprightFont=cmuntt.ttf,BoldFont=cmuntb.ttf,ItalicFont=cmunit.ttf,BoldItalicFont=cmuntx.ttf]{cmunti.ttf}\itshape greater than >{}}}
\label{198}\setmainfont[Path=/usr/share/fonts/truetype/cmu/,UprightFont=cmunrm.ttf,BoldFont=cmunbx.ttf,ItalicFont=cmunti.ttf,BoldItalicFont=cmunbi.ttf]{cmunrm.ttf}\setmonofont[Path=/usr/share/fonts/truetype/cmu/,UprightFont=cmuntt.ttf,BoldFont=cmuntb.ttf,ItalicFont=cmunit.ttf,BoldItalicFont=cmuntx.ttf]{cmunrm.ttf}

The two symbols \textquotesingle{}<{}\textquotesingle{} and \textquotesingle{}>{}\textquotesingle{} are actually ASCII characters, but you may have noticed that they will print \textquotesingle{}¡\textquotesingle{} and \textquotesingle{}¿\textquotesingle{} respectively. This is a font encoding issue. If you want them to print their real symbol, you will have to use another font encoding such as T1, loaded with the \LaTeXTT{fontenc} package. See \mylref{163}{Fonts} for more details on font encoding.

Alternatively, they can be printed with dedicated commands:
\begin{Shaded}
\begin{Highlighting}[]

\NormalTok{\textbackslash{}textless}
\NormalTok{\textbackslash{}textgreater}
\end{Highlighting}
\end{Shaded}

\section{Euro € currency symbol}
\label{199}

When writing about money these days, you need the \myhref{https://en.wikipedia.org/wiki/euro\%20sign}{euro sign}.
The \LaTeXTT{textcomp} package features a \LaTeXTT{\textbackslash{}texteuro} command which gives you the euro symbol as supplied by your current text font. Depending on your chosen font this may be quite far from the official symbol.

An official version of the euro symbol is provided by \LaTeXTT{eurosym}. Load it in the preamble (optionally with the \LaTeXTT{official} option):

\begin{Shaded}
\begin{Highlighting}[]

\NormalTok{\textbackslash{}usepackage[official]\{eurosym\}}
\end{Highlighting}
\end{Shaded}


then you can insert it with the \LaTeXTT{\textbackslash{}euro\{\}} command. Finally, if you want a euro symbol that matches with the current font style (e.g., bold, italics, etc.) you can use a different option:

\begin{Shaded}
\begin{Highlighting}[]

\NormalTok{\textbackslash{}usepackage[gen]\{eurosym\}}
\end{Highlighting}
\end{Shaded}


again you can insert the euro symbol with \LaTeXTT{\textbackslash{}euro\{\}}.

Alternatively you can use the \LaTeXTT{marvosym} package which also provides the official euro symbol.
\begin{Shaded}
\begin{Highlighting}[]

\NormalTok{\textbackslash{}usepackage\{marvosym\}}
\CommentTok{% ...}
 
\NormalTok{\textbackslash{}EUR\{\}}
\end{Highlighting}
\end{Shaded}


Now that you have succeeded in printing a euro sign, you may want the \textquotesingle{}€\textquotesingle{} on your keyboard to actually print the euro sign as above.
There is a simple method to do that. You must make sure you are using UTF-{}8 encoding along with a working \LaTeXTT{\textbackslash{}euro\{\}} or \LaTeXTT{\textbackslash{}EUR\{\}}command.

\begin{Shaded}
\begin{Highlighting}[]

\NormalTok{\textbackslash{}DeclareUnicodeCharacter\{20AC\}\{\textbackslash{}euro\}}
\CommentTok{% or}
\NormalTok{\textbackslash{}DeclareUnicodeCharacter\{20AC\}\{\textbackslash{}EUR\{\}\}}
\end{Highlighting}
\end{Shaded}


Complete example:

\begin{Shaded}
\begin{Highlighting}[]

\NormalTok{\textbackslash{}usepackage[utf8]\{inputenc\}}
\NormalTok{\textbackslash{}usepackage\{marvosym\}}
\NormalTok{\textbackslash{}DeclareUnicodeCharacter\{20AC\}\{\textbackslash{}EUR\{\}\}}
\end{Highlighting}
\end{Shaded}

\section{Degree symbol for temperature and math}
\label{200}
The easiest way to print temperature and angle values is to use the \LaTeXTT{\textbackslash{}SI\{value\}\{unit\}} command from the \LaTeXTT{siunitx} package, which works both in text and math mode:
\begin{Shaded}
\begin{Highlighting}[]

\NormalTok{\textbackslash{}usepackage\{amsmath\}}
\NormalTok{\textbackslash{}usepackage\{siunitx\}}
\CommentTok{%...}
 
\NormalTok{A \textbackslash{}SI\{45\}\{\textbackslash{}degree\} angle.}
 
\NormalTok{It is $\textbackslash{}SI\{17\}\{\textbackslash{}degreeCelsius\}$ outside.}
\end{Highlighting}
\end{Shaded}

For more information, see the \myhref{http://ctan.org/pkg/siunitx}{documentation of the \LaTeXTT{siunitx} package}.

A common mistake is to use the \LaTeXTT{\textbackslash{}circ} command. It will not print the correct character (though \LaTeXTT{\${}\^{}\textbackslash{}circ\${}} will). Use the \LaTeXTT{textcomp} package instead, which provides a \LaTeXTT{\textbackslash{}textdegree} command.

\begin{Shaded}
\begin{Highlighting}[]

\NormalTok{\textbackslash{}usepackage\{textcomp\}}
\CommentTok{%...}
 
\NormalTok{A $45$\textbackslash{}textdegree angle.}
\end{Highlighting}
\end{Shaded}


For temperature, you can use the same command or opt for the \LaTeXTT{gensymb} package and write

\begin{Shaded}
\begin{Highlighting}[]

\NormalTok{\textbackslash{}usepackage\{gensymb\}}
\NormalTok{\textbackslash{}usepackage\{textcomp\}}
\CommentTok{%...}
 
\NormalTok{17\textbackslash{},\textbackslash{}celsius }\CommentTok{% best (with textcomp)}
\end{Highlighting}
\end{Shaded}


Some keyboard layouts feature the degree symbol, you can use it directly if you are using UTF-{}8 and \LaTeXTT{textcomp}. For better results (font quality) we recommend the use of an appropriate font, like \LaTeXTT{lmodern}:

\begin{Shaded}
\begin{Highlighting}[]

\NormalTok{\textbackslash{}usepackage[utf8]\{inputenc\}}
\NormalTok{\textbackslash{}usepackage\{lmodern\}}
\NormalTok{\textbackslash{}usepackage\{textcomp\}}
 
\CommentTok{% ...}
 
\NormalTok{17\textbackslash{},°C}
 
\NormalTok{17\textbackslash{},℃ }\CommentTok{% best}
\end{Highlighting}
\end{Shaded}

\section{Other symbols}
\label{201}

LaTeX has many symbols at its disposal. The majority of them are within the mathematical domain, and later chapters will cover how to get access to them. For the more common text symbols, use the following commands:

\begin{longtable}{|>{\RaggedRight}p{0.64081\linewidth}|>{\RaggedRight}p{0.12047\linewidth}|>{\RaggedRight}p{0.15301\linewidth}|} \hline 
{\bfseries \hspace*{0pt}\ignorespaces{}\hspace*{0pt} Command}&{\bfseries \hspace*{0pt}\ignorespaces{}\hspace*{0pt} Sample}&{\bfseries \hspace*{0pt}\ignorespaces{}\hspace*{0pt} Character}\endhead  \hline \hspace*{0pt}\ignorespaces{}\hspace*{0pt}\LaTeXTT{\textbackslash{}\%}&\hspace*{0pt}\ignorespaces{}\hspace*{0pt}{$\%$}&\hspace*{0pt}\ignorespaces{}\hspace*{0pt} \%\\ \hline \hspace*{0pt}\ignorespaces{}\hspace*{0pt}\LaTeXTT{\textbackslash{}\${}}&\hspace*{0pt}\ignorespaces{}\hspace*{0pt}{$\$$}&\hspace*{0pt}\ignorespaces{}\hspace*{0pt}\${}\\ \hline \hspace*{0pt}\ignorespaces{}\hspace*{0pt}\LaTeXTT{\textbackslash{}\{}&\hspace*{0pt}\ignorespaces{}\hspace*{0pt}{$\{$}&\hspace*{0pt}\ignorespaces{}\hspace*{0pt}\{\\ \hline \hspace*{0pt}\ignorespaces{}\hspace*{0pt}\LaTeXTT{\textbackslash{}\_}&\hspace*{0pt}\ignorespaces{}\hspace*{0pt}{$\_$}&\hspace*{0pt}\ignorespaces{}\hspace*{0pt}\_\\ \hline \hspace*{0pt}\ignorespaces{}\hspace*{0pt}\LaTeXTT{\textbackslash{}P}&\hspace*{0pt}\ignorespaces{}\hspace*{0pt}{$\P$}&\hspace*{0pt}\ignorespaces{}\hspace*{0pt}¶\\ \hline \hspace*{0pt}\ignorespaces{}\hspace*{0pt}\LaTeXTT{\textbackslash{}ddag}&\hspace*{0pt}\ignorespaces{}\hspace*{0pt}n/a&\hspace*{0pt}\ignorespaces{}\hspace*{0pt}‡\\ \hline \hspace*{0pt}\ignorespaces{}\hspace*{0pt}\LaTeXTT{\textbackslash{}textbar}&\hspace*{0pt}\ignorespaces{}\hspace*{0pt}n/a&\hspace*{0pt}\ignorespaces{}\hspace*{0pt}|\\ \hline \hspace*{0pt}\ignorespaces{}\hspace*{0pt}\LaTeXTT{\textbackslash{}textgreater}&\hspace*{0pt}\ignorespaces{}\hspace*{0pt}{$>$}&\hspace*{0pt}\ignorespaces{}\hspace*{0pt}{\mbox{$>$}} \\ \hline \hspace*{0pt}\ignorespaces{}\hspace*{0pt}\LaTeXTT{\textbackslash{}textendash}&\hspace*{0pt}\ignorespaces{}\hspace*{0pt}n/a&\hspace*{0pt}\ignorespaces{}\hspace*{0pt}–\\ \hline \hspace*{0pt}\ignorespaces{}\hspace*{0pt}\LaTeXTT{\textbackslash{}texttrademark}&\hspace*{0pt}\ignorespaces{}\hspace*{0pt}n/a&\hspace*{0pt}\ignorespaces{}\hspace*{0pt}™\\ \hline \hspace*{0pt}\ignorespaces{}\hspace*{0pt}\LaTeXTT{\textbackslash{}textexclamdown}&\hspace*{0pt}\ignorespaces{}\hspace*{0pt}n/a&\hspace*{0pt}\ignorespaces{}\hspace*{0pt}¡\\ \hline \hspace*{0pt}\ignorespaces{}\hspace*{0pt}\LaTeXTT{\textbackslash{}textsuperscript<{}nowiki>{}\{a\}<{}/nowiki>{}}&\hspace*{0pt}\ignorespaces{}\hspace*{0pt}{$\mathrm{X^{a}}$}&\hspace*{0pt}\ignorespaces{}\hspace*{0pt}\setmainfont[Path=/usr/share/fonts/truetype/cmu/,UprightFont=cmunrm.ttf,BoldFont=cmunbx.ttf,ItalicFont=cmunti.ttf,BoldItalicFont=cmunbi.ttf]{cmunrm.ttf}\setmonofont[Path=/usr/share/fonts/truetype/cmu/,UprightFont=cmuntt.ttf,BoldFont=cmuntb.ttf,ItalicFont=cmunit.ttf,BoldItalicFont=cmuntx.ttf]{cmunrm.ttf}\textsuperscript{a}\\ \hline \hspace*{0pt}\ignorespaces{}\hspace*{0pt}\LaTeXTT{\textbackslash{}pounds}&\hspace*{0pt}\ignorespaces{}\hspace*{0pt}n/a&\hspace*{0pt}\ignorespaces{}\hspace*{0pt}£\\ \hline \hspace*{0pt}\ignorespaces{}\hspace*{0pt}\LaTeXTT{\textbackslash{}\#}&\hspace*{0pt}\ignorespaces{}\hspace*{0pt}{$\#$}&\hspace*{0pt}\ignorespaces{}\hspace*{0pt}\#\\ \hline \hspace*{0pt}\ignorespaces{}\hspace*{0pt}\LaTeXTT{\textbackslash{}\&}&\hspace*{0pt}\ignorespaces{}\hspace*{0pt}{$\&$}&\hspace*{0pt}\ignorespaces{}\hspace*{0pt}{\mbox{$\&$}}\\ \hline \hspace*{0pt}\ignorespaces{}\hspace*{0pt}\LaTeXTT{\textbackslash{}<{}nowiki>{}\}<{}/nowiki>{}}&\hspace*{0pt}\ignorespaces{}\hspace*{0pt}{$\}$}&\hspace*{0pt}\ignorespaces{}\hspace*{0pt}\}\\ \hline \hspace*{0pt}\ignorespaces{}\hspace*{0pt}\LaTeXTT{\textbackslash{}S}&\hspace*{0pt}\ignorespaces{}\hspace*{0pt}{$\S$}&\hspace*{0pt}\ignorespaces{}\hspace*{0pt}§\\ \hline \hspace*{0pt}\ignorespaces{}\hspace*{0pt}\LaTeXTT{\textbackslash{}dag}&\hspace*{0pt}\ignorespaces{}\hspace*{0pt}n/a&\hspace*{0pt}\ignorespaces{}\hspace*{0pt}†\\ \hline \hspace*{0pt}\ignorespaces{}\hspace*{0pt}\LaTeXTT{\textbackslash{}textbackslash}&\hspace*{0pt}\ignorespaces{}\hspace*{0pt}n/a&\hspace*{0pt}\ignorespaces{}\hspace*{0pt}\textbackslash{}\\ \hline \hspace*{0pt}\ignorespaces{}\hspace*{0pt}\LaTeXTT{\textbackslash{}textless}&\hspace*{0pt}\ignorespaces{}\hspace*{0pt}{$<$}&\hspace*{0pt}\ignorespaces{}\hspace*{0pt}{\mbox{$<$}} \\ \hline \hspace*{0pt}\ignorespaces{}\hspace*{0pt}\LaTeXTT{\textbackslash{}textemdash}&\hspace*{0pt}\ignorespaces{}\hspace*{0pt}n/a&\hspace*{0pt}\ignorespaces{}\hspace*{0pt}—\\ \hline \hspace*{0pt}\ignorespaces{}\hspace*{0pt}\LaTeXTT{\textbackslash{}textregistered}&\hspace*{0pt}\ignorespaces{}\hspace*{0pt}n/a&\hspace*{0pt}\ignorespaces{}\hspace*{0pt}®\\ \hline \hspace*{0pt}\ignorespaces{}\hspace*{0pt}\LaTeXTT{\textbackslash{}textquestiondown}&\hspace*{0pt}\ignorespaces{}\hspace*{0pt}n/a&\hspace*{0pt}\ignorespaces{}\hspace*{0pt}¿\\ \hline \hspace*{0pt}\ignorespaces{}\hspace*{0pt}\LaTeXTT{\textbackslash{}textcircled<{}nowiki>{}\{a\}<{}/nowiki>{}}&\hspace*{0pt}\ignorespaces{}\hspace*{0pt}n/a&\hspace*{0pt}\ignorespaces{}\hspace*{0pt}\setmainfont[Path=/usr/share/fonts/truetype/wqy/]{wqy-zenhei.ttc}\setmonofont[Path=/usr/share/fonts/truetype/wqy/]{wqy-zenhei.ttc}ⓐ\\ \hline \hspace*{0pt}\ignorespaces{}\hspace*{0pt}\LaTeXTT{\setmainfont[Path=/usr/share/fonts/truetype/cmu/,UprightFont=cmunrm.ttf,BoldFont=cmunbx.ttf,ItalicFont=cmunti.ttf,BoldItalicFont=cmunbi.ttf]{cmunrm.ttf}\setmonofont[Path=/usr/share/fonts/truetype/cmu/,UprightFont=cmuntt.ttf,BoldFont=cmuntb.ttf,ItalicFont=cmunit.ttf,BoldItalicFont=cmuntx.ttf]{cmunrm.ttf}\textbackslash{}copyright}&\hspace*{0pt}\ignorespaces{}\hspace*{0pt}n/a&\hspace*{0pt}\ignorespaces{}\hspace*{0pt}©\\ \hline 
\end{longtable}


Not mentioned in above table, tilde (\~{}) is used in LaTeX code to produce \mylref{113}{non-{}breakable space}. To get printed tilde sign, either write \LaTeXTT{\textbackslash{}\~{}\{\}} or \LaTeXTT{\textbackslash{}textasciitilde\{\}}. 
And a visible space ␣ can be created with \LaTeXTT{\textbackslash{}textvisiblespace}.

For some more interesting symbols, the Postscript ZipfDingbats font is available thanks to the \LaTeXTT{pifont} package. Add the declaration to your preamble: \LaTeXTT{\textbackslash{}usepackage\{pifont\}}. Next, the command \LaTeXTT{\textbackslash{}ding\{number\}}, will print the specified symbol. Here is a table of the available symbols:



\begin{minipage}{1.0\linewidth}
\begin{center}
\includegraphics[width=1.0\linewidth,height=6.5in,keepaspectratio]{../images/34.png}
\end{center}
\raggedright{}\myfigurewithcaption{34}{ZapfDingbats symbols}
\end{minipage}\vspace{0.75cm}

.
\section{In special environments}
\label{202}
\subsection{Math mode}
\label{203}

Several of the above and some similar accents can also be produced in math mode. The following commands may be used only in math mode.

\begin{longtable}{|>{\RaggedRight}p{0.22903\linewidth}|>{\RaggedRight}p{0.10272\linewidth}|>{\RaggedRight}p{0.29580\linewidth}|>{\RaggedRight}p{0.25817\linewidth}|} \hline 
{\bfseries \hspace*{0pt}\ignorespaces{}\hspace*{0pt} LaTeX command}&{\bfseries \hspace*{0pt}\ignorespaces{}\hspace*{0pt} Sample}&{\bfseries \hspace*{0pt}\ignorespaces{}\hspace*{0pt} Description}&{\bfseries \hspace*{0pt}\ignorespaces{}\hspace*{0pt} Text-{}mode equivalence}\endhead  \hline \hspace*{0pt}\ignorespaces{}\hspace*{0pt}\LaTeXTT{\textbackslash{}hat\{o\}}&\hspace*{0pt}\ignorespaces{}\hspace*{0pt}{$\hat{o}$}&\hspace*{0pt}\ignorespaces{}\hspace*{0pt}circumflex&\hspace*{0pt}\ignorespaces{}\hspace*{0pt}\LaTeXTT{\textbackslash{}\^{}}\\ \hline \hspace*{0pt}\ignorespaces{}\hspace*{0pt}\LaTeXTT{\textbackslash{}widehat\{oo\}}&\hspace*{0pt}\ignorespaces{}\hspace*{0pt}{$\widehat{oo}$}&\hspace*{0pt}\ignorespaces{}\hspace*{0pt}wide version of \LaTeXTT{\textbackslash{}hat} over several letters&\hspace*{0pt}\ignorespaces{}\hspace*{0pt}\\ \hline \hspace*{0pt}\ignorespaces{}\hspace*{0pt}\LaTeXTT{\textbackslash{}check\{o\}}&\hspace*{0pt}\ignorespaces{}\hspace*{0pt}{$\check{o}$}&\hspace*{0pt}\ignorespaces{}\hspace*{0pt}vee or check&\hspace*{0pt}\ignorespaces{}\hspace*{0pt}\LaTeXTT{\textbackslash{}v}\\ \hline \hspace*{0pt}\ignorespaces{}\hspace*{0pt}\LaTeXTT{\textbackslash{}tilde\{o\}}&\hspace*{0pt}\ignorespaces{}\hspace*{0pt}{$\tilde{o}$}&\hspace*{0pt}\ignorespaces{}\hspace*{0pt}tilde&\hspace*{0pt}\ignorespaces{}\hspace*{0pt}\LaTeXTT{\textbackslash{}\~{}}\\ \hline \hspace*{0pt}\ignorespaces{}\hspace*{0pt}\LaTeXTT{\textbackslash{}widetilde\{oo\}}&\hspace*{0pt}\ignorespaces{}\hspace*{0pt}{$\widetilde{oo}$}&\hspace*{0pt}\ignorespaces{}\hspace*{0pt}wide version of \LaTeXTT{\textbackslash{}tilde} over several letters&\hspace*{0pt}\ignorespaces{}\hspace*{0pt}\\ \hline \hspace*{0pt}\ignorespaces{}\hspace*{0pt}\LaTeXTT{\textbackslash{}acute\{o\}}&\hspace*{0pt}\ignorespaces{}\hspace*{0pt}{$\acute{o}$}&\hspace*{0pt}\ignorespaces{}\hspace*{0pt}acute accent&\hspace*{0pt}\ignorespaces{}\hspace*{0pt}\LaTeXTT{\textbackslash{}\textquotesingle{}}\\ \hline \hspace*{0pt}\ignorespaces{}\hspace*{0pt}\LaTeXTT{\textbackslash{}grave\{o\}}&\hspace*{0pt}\ignorespaces{}\hspace*{0pt}{$\grave{o}$}&\hspace*{0pt}\ignorespaces{}\hspace*{0pt}grave accent&\hspace*{0pt}\ignorespaces{}\hspace*{0pt}\LaTeXTT{\textbackslash{}`}\\ \hline \hspace*{0pt}\ignorespaces{}\hspace*{0pt}\LaTeXTT{\textbackslash{}dot\{o\}}&\hspace*{0pt}\ignorespaces{}\hspace*{0pt}{$\dot{o}$}&\hspace*{0pt}\ignorespaces{}\hspace*{0pt}dot over the letter&\hspace*{0pt}\ignorespaces{}\hspace*{0pt}\LaTeXTT{\textbackslash{}.}\\ \hline \hspace*{0pt}\ignorespaces{}\hspace*{0pt}\LaTeXTT{\textbackslash{}ddot\{o\}}&\hspace*{0pt}\ignorespaces{}\hspace*{0pt}{$\ddot{o}$}&\hspace*{0pt}\ignorespaces{}\hspace*{0pt}two dots over the letter (umlaut in text-{}mode)&\hspace*{0pt}\ignorespaces{}\hspace*{0pt}\LaTeXTT{\textbackslash{}\symbol{34}}\\ \hline \hspace*{0pt}\ignorespaces{}\hspace*{0pt}\LaTeXTT{\textbackslash{}breve\{o\}}&\hspace*{0pt}\ignorespaces{}\hspace*{0pt}{$\breve{o}$}&\hspace*{0pt}\ignorespaces{}\hspace*{0pt}breve&\hspace*{0pt}\ignorespaces{}\hspace*{0pt}\LaTeXTT{\textbackslash{}u}\\ \hline \hspace*{0pt}\ignorespaces{}\hspace*{0pt}\LaTeXTT{\textbackslash{}bar\{o\}}&\hspace*{0pt}\ignorespaces{}\hspace*{0pt}{$\bar{o}$}&\hspace*{0pt}\ignorespaces{}\hspace*{0pt}macron&\hspace*{0pt}\ignorespaces{}\hspace*{0pt}\LaTeXTT{\textbackslash{}=}\\ \hline \hspace*{0pt}\ignorespaces{}\hspace*{0pt}\LaTeXTT{\textbackslash{}vec\{o\}}&\hspace*{0pt}\ignorespaces{}\hspace*{0pt}{$\vec{o}$}&\hspace*{0pt}\ignorespaces{}\hspace*{0pt}vector (arrow) over the letter&\hspace*{0pt}\ignorespaces{}\hspace*{0pt}\\ \hline 
\end{longtable}


When applying accents to letters \LaTeXTT{i} and \LaTeXTT{j}, you can use \textbackslash{}imath and \textbackslash{}jmath to keep the dots from interfering with the accents:
\begin{longtable}{|>{\RaggedRight}p{0.20405\linewidth}|>{\RaggedRight}p{0.10272\linewidth}|>{\RaggedRight}p{0.32001\linewidth}|>{\RaggedRight}p{0.25894\linewidth}|} \hline 
{\bfseries \hspace*{0pt}\ignorespaces{}\hspace*{0pt} LaTeX command}&{\bfseries \hspace*{0pt}\ignorespaces{}\hspace*{0pt} Sample}&{\bfseries \hspace*{0pt}\ignorespaces{}\hspace*{0pt} Description}&{\bfseries \hspace*{0pt}\ignorespaces{}\hspace*{0pt} Sample with upper dot}\endhead  \hline \hspace*{0pt}\ignorespaces{}\hspace*{0pt}\LaTeXTT{\textbackslash{}hat\{\textbackslash{}imath\}}&\hspace*{0pt}\ignorespaces{}\hspace*{0pt}{$\hat{\imath}$}&\hspace*{0pt}\ignorespaces{}\hspace*{0pt}circumflex on letter \LaTeXTT{i} without upper dot&\hspace*{0pt}\ignorespaces{}\hspace*{0pt}{$\hat{i}$}\\ \hline \hspace*{0pt}\ignorespaces{}\hspace*{0pt}\LaTeXTT{\textbackslash{}vec\{\textbackslash{}jmath\}}&\hspace*{0pt}\ignorespaces{}\hspace*{0pt}{$\vec{\jmath}$}&\hspace*{0pt}\ignorespaces{}\hspace*{0pt}vector (arrow) on letter \LaTeXTT{j} without upper dot&\hspace*{0pt}\ignorespaces{}\hspace*{0pt}{$\vec{j}$}\\ \hline 
\end{longtable}

\subsection{Tabbing environment}
\label{204}

Some of the accent marks used in running text have other uses in the tabbing environment. In that case they can be created with the following command:

\begin{myitemize}
\item{}  \LaTeXTT{\textbackslash{}a\textquotesingle{}} for an acute accent
\item{}  \LaTeXTT{\textbackslash{}a`} for a grave accent
\item{}  \LaTeXTT{\textbackslash{}a=} for a macron accent
\end{myitemize}

\section{Unicode keyboard input}
\label{205}
\myhref{https://en.wikipedia.org/wiki/Unicode\%20input}{w:Unicode input}
Some operating systems provide a keyboard combination to input any Unicode code point, the so-{}called {\itshape \setmainfont[Path=/usr/share/fonts/truetype/cmu/,UprightFont=cmunrm.ttf,BoldFont=cmunbx.ttf,ItalicFont=cmunti.ttf,BoldItalicFont=cmunbi.ttf]{cmunti.ttf}\setmonofont[Path=/usr/share/fonts/truetype/cmu/,UprightFont=cmuntt.ttf,BoldFont=cmuntb.ttf,ItalicFont=cmunit.ttf,BoldItalicFont=cmuntx.ttf]{cmunti.ttf}\itshape unicode compose key}\setmainfont[Path=/usr/share/fonts/truetype/cmu/,UprightFont=cmunrm.ttf,BoldFont=cmunbx.ttf,ItalicFont=cmunti.ttf,BoldItalicFont=cmunbi.ttf]{cmunrm.ttf}\setmonofont[Path=/usr/share/fonts/truetype/cmu/,UprightFont=cmuntt.ttf,BoldFont=cmuntb.ttf,ItalicFont=cmunit.ttf,BoldItalicFont=cmuntx.ttf]{cmunrm.ttf}.

Many X applications (*BSD and GNU/Linux) support the {\ttfamily \setmainfont[Path=/usr/share/fonts/truetype/cmu/,UprightFont=cmunrm.ttf,BoldFont=cmunbx.ttf,ItalicFont=cmunti.ttf,BoldItalicFont=cmunbi.ttf]{cmuntt.ttf}\setmonofont[Path=/usr/share/fonts/truetype/cmu/,UprightFont=cmuntt.ttf,BoldFont=cmuntb.ttf,ItalicFont=cmunit.ttf,BoldItalicFont=cmuntx.ttf]{cmuntt.ttf}\ttfamily Ctrl+Shift+u}{$\text{ }$}\setmainfont[Path=/usr/share/fonts/truetype/cmu/,UprightFont=cmunrm.ttf,BoldFont=cmunbx.ttf,ItalicFont=cmunti.ttf,BoldItalicFont=cmunbi.ttf]{cmunrm.ttf}\setmonofont[Path=/usr/share/fonts/truetype/cmu/,UprightFont=cmuntt.ttf,BoldFont=cmuntb.ttf,ItalicFont=cmunit.ttf,BoldItalicFont=cmuntx.ttf]{cmunrm.ttf} combination. A \textquotesingle{}\uline{u}\textquotesingle{} symbol should appear. Type the code point and press {\ttfamily \setmainfont[Path=/usr/share/fonts/truetype/cmu/,UprightFont=cmunrm.ttf,BoldFont=cmunbx.ttf,ItalicFont=cmunti.ttf,BoldItalicFont=cmunbi.ttf]{cmuntt.ttf}\setmonofont[Path=/usr/share/fonts/truetype/cmu/,UprightFont=cmuntt.ttf,BoldFont=cmuntb.ttf,ItalicFont=cmunit.ttf,BoldItalicFont=cmuntx.ttf]{cmuntt.ttf}\ttfamily enter}{$\text{ }$}\setmainfont[Path=/usr/share/fonts/truetype/cmu/,UprightFont=cmunrm.ttf,BoldFont=cmunbx.ttf,ItalicFont=cmunti.ttf,BoldItalicFont=cmunbi.ttf]{cmunrm.ttf}\setmonofont[Path=/usr/share/fonts/truetype/cmu/,UprightFont=cmuntt.ttf,BoldFont=cmuntb.ttf,ItalicFont=cmunit.ttf,BoldItalicFont=cmuntx.ttf]{cmunrm.ttf} or {\ttfamily \setmainfont[Path=/usr/share/fonts/truetype/cmu/,UprightFont=cmunrm.ttf,BoldFont=cmunbx.ttf,ItalicFont=cmunti.ttf,BoldItalicFont=cmunbi.ttf]{cmuntt.ttf}\setmonofont[Path=/usr/share/fonts/truetype/cmu/,UprightFont=cmuntt.ttf,BoldFont=cmuntb.ttf,ItalicFont=cmunit.ttf,BoldItalicFont=cmuntx.ttf]{cmuntt.ttf}\ttfamily space}{$\text{ }$}\setmainfont[Path=/usr/share/fonts/truetype/cmu/,UprightFont=cmunrm.ttf,BoldFont=cmunbx.ttf,ItalicFont=cmunti.ttf,BoldItalicFont=cmunbi.ttf]{cmunrm.ttf}\setmonofont[Path=/usr/share/fonts/truetype/cmu/,UprightFont=cmuntt.ttf,BoldFont=cmuntb.ttf,ItalicFont=cmunit.ttf,BoldItalicFont=cmuntx.ttf]{cmunrm.ttf} to actually print the character.
Example:\\

\TemplateSpaceIndent{$\text{ }${}<{}Ctrl+Shift+u>{}$\text{ }${}20AC$\text{ }${}<{}space>{}}

will print the euro character.

Desktop environments like GNOME and KDE may feature a customizable compose key for more memorizable sequences.

Xorg features advanced keyboard layouts with variants that let you enter a lot of characters easily with combination using the aprioriate modifier, like {\ttfamily \setmainfont[Path=/usr/share/fonts/truetype/cmu/,UprightFont=cmunrm.ttf,BoldFont=cmunbx.ttf,ItalicFont=cmunti.ttf,BoldItalicFont=cmunbi.ttf]{cmuntt.ttf}\setmonofont[Path=/usr/share/fonts/truetype/cmu/,UprightFont=cmuntt.ttf,BoldFont=cmuntb.ttf,ItalicFont=cmunit.ttf,BoldItalicFont=cmuntx.ttf]{cmuntt.ttf}\ttfamily Alt Gr}\setmainfont[Path=/usr/share/fonts/truetype/cmu/,UprightFont=cmunrm.ttf,BoldFont=cmunbx.ttf,ItalicFont=cmunti.ttf,BoldItalicFont=cmunbi.ttf]{cmunrm.ttf}\setmonofont[Path=/usr/share/fonts/truetype/cmu/,UprightFont=cmuntt.ttf,BoldFont=cmuntb.ttf,ItalicFont=cmunit.ttf,BoldItalicFont=cmuntx.ttf]{cmunrm.ttf}. It highly depends on the selected layout+variant, so we suggest you to play a bit with your keyboard, preceeding every key and dead key with the {\ttfamily \setmainfont[Path=/usr/share/fonts/truetype/cmu/,UprightFont=cmunrm.ttf,BoldFont=cmunbx.ttf,ItalicFont=cmunti.ttf,BoldItalicFont=cmunbi.ttf]{cmuntt.ttf}\setmonofont[Path=/usr/share/fonts/truetype/cmu/,UprightFont=cmuntt.ttf,BoldFont=cmuntb.ttf,ItalicFont=cmunit.ttf,BoldItalicFont=cmuntx.ttf]{cmuntt.ttf}\ttfamily Alt Gr}{$\text{ }$}\setmainfont[Path=/usr/share/fonts/truetype/cmu/,UprightFont=cmunrm.ttf,BoldFont=cmunbx.ttf,ItalicFont=cmunti.ttf,BoldItalicFont=cmunbi.ttf]{cmunrm.ttf}\setmonofont[Path=/usr/share/fonts/truetype/cmu/,UprightFont=cmuntt.ttf,BoldFont=cmuntb.ttf,ItalicFont=cmunit.ttf,BoldItalicFont=cmuntx.ttf]{cmunrm.ttf} modifier.
\section{External links}
\label{206}

\begin{myitemize}
\item{}  \myhref{http://spectroscopy.mps.ohio-state.edu/symposium_53/latexinstruct.html}{A few other LaTeX accents and symbols}
\item{}  \myhref{http://www.giss.nasa.gov/tools/latex/ltx-401.html}{NASA GISS: Accents}
\item{}  {$\text{[}$}ftp://sunsite.icm.edu.pl/pub/CTAN/info/symbols/comprehensive/symbols-{}a4.pdf The Comprehensive LATEX Symbol List{$\text{]}$}
\item{}  \myhref{http://www.rpi.edu/dept/arc/training/latex/LaTeX_symbols.pdf}{PDF document with a lengthy list of symbols provided by various packages}
\end{myitemize}

\section{Notes and References}
\label{207}
\LaTeXNullTemplate{}



\myhref{https://sr.wikibooks.org/wiki/LaTeX\%2F\%D0\%9F\%D0\%BE\%D1\%81\%D0\%B5\%D0\%B1\%D0\%BD\%D0\%B8\%20\%D0\%B7\%D0\%BD\%D0\%B0\%D0\%BA\%D0\%BE\%D0\%B2\%D0\%B8}{sr:LaTeX/Посебни знакови}\chapter{Internationalization}

\myminitoc
\label{208}

\label{209}


LaTeX has to be configured and used appropriately when it is used to write documents in languages other than English.  This has to address three main areas:

\begin{myenumerate}
\item{}  LaTeX needs to know how to hyphenate the language(s) to be used.
\item{}  The user needs to use language-{}specific typographic rules. In French for example, there is a mandatory space before each colon character (:).
\item{}  The input of special characters, especially for languages using an input system (Arab, Chinese, Japanese, Korean).
\end{myenumerate}


It is convenient to be able to insert language-{}specific special characters directly from the keyboard instead of using cumbersome coding (for example, by typing {\ttfamily \setmainfont[Path=/usr/share/fonts/truetype/cmu/,UprightFont=cmunrm.ttf,BoldFont=cmunbx.ttf,ItalicFont=cmunti.ttf,BoldItalicFont=cmunbi.ttf]{cmuntt.ttf}\setmonofont[Path=/usr/share/fonts/truetype/cmu/,UprightFont=cmuntt.ttf,BoldFont=cmuntb.ttf,ItalicFont=cmunit.ttf,BoldItalicFont=cmuntx.ttf]{cmuntt.ttf}\ttfamily ä}{$\text{ }$}\setmainfont[Path=/usr/share/fonts/truetype/cmu/,UprightFont=cmunrm.ttf,BoldFont=cmunbx.ttf,ItalicFont=cmunti.ttf,BoldItalicFont=cmunbi.ttf]{cmunrm.ttf}\setmonofont[Path=/usr/share/fonts/truetype/cmu/,UprightFont=cmuntt.ttf,BoldFont=cmuntb.ttf,ItalicFont=cmunit.ttf,BoldItalicFont=cmuntx.ttf]{cmunrm.ttf} instead of {\ttfamily \setmainfont[Path=/usr/share/fonts/truetype/cmu/,UprightFont=cmunrm.ttf,BoldFont=cmunbx.ttf,ItalicFont=cmunti.ttf,BoldItalicFont=cmunbi.ttf]{cmuntt.ttf}\setmonofont[Path=/usr/share/fonts/truetype/cmu/,UprightFont=cmuntt.ttf,BoldFont=cmuntb.ttf,ItalicFont=cmunit.ttf,BoldItalicFont=cmuntx.ttf]{cmuntt.ttf}\ttfamily \textbackslash{}\symbol{34}\{a\}}\setmainfont[Path=/usr/share/fonts/truetype/cmu/,UprightFont=cmunrm.ttf,BoldFont=cmunbx.ttf,ItalicFont=cmunti.ttf,BoldItalicFont=cmunbi.ttf]{cmunrm.ttf}\setmonofont[Path=/usr/share/fonts/truetype/cmu/,UprightFont=cmuntt.ttf,BoldFont=cmuntb.ttf,ItalicFont=cmunit.ttf,BoldItalicFont=cmuntx.ttf]{cmunrm.ttf}). This can be done by configuring input encoding properly. We will not tackle this issue here: see the \mylref{192}{Special Characters} chapter.

Some languages require special fonts with the proper font encoding set. See \mylref{173}{Font encoding}.

Some of the methods described in this chapter may be useful when dealing with non-{}English author names in bibliographies.

Here is a collection of suggestions about writing a LaTeX document in a language other than English. If you have experience in a language not listed below, please add some notes about it.
\section{Prerequisites}
\label{210}

Most non-{}english language will need to input special characters very often. For a convenient writing you will need to set the input encoding and the font encoding properly.

The following configuration is optimal for many languages (most latin languages). Make sure your document is saved using the UTF-{}8 encoding.

\begin{Shaded}
\begin{Highlighting}[]

\NormalTok{\textbackslash{}usepackage[utf8]\{inputenc\}}
\NormalTok{\textbackslash{}usepackage[T1]\{fontenc\}}
\end{Highlighting}
\end{Shaded}


\begin{TemplateInfo}{\danger}{Warning}In the following document we will assume you are using this configuration unless otherwise specified.\end{TemplateInfo}

For more details check \mylref{173}{Font encoding} and \mylref{192}{Special Characters}.
\section{Babel}
\label{211}

The \LaTeXTT{babel} package by Johannes Braams and Javier Bezos will take care of everything (with XeTeX and LuaTeX you should consider \LaTeXTT{polyglossia}). You can load it in your preamble, providing as an argument name of the language you want to use (usually its English name, but not always):

\begin{Shaded}
\begin{Highlighting}[]

\NormalTok{\textbackslash{}usepackage[language]\{babel\}}
\end{Highlighting}
\end{Shaded}


You should place it soon after the \LaTeXTT{\textbackslash{}documentclass} command, so that all the other packages you load afterwards will know the language you are using. Babel will automatically activate the appropriate hyphenation rules for the language you choose. If your LaTeX format does not support hyphenation in the language of your choice, babel will still work but will disable hyphenation, which has quite a negative effect on the appearance of the typeset document. Babel also specifies new commands for some languages, which simplify the input of special characters. See the sections about languages below for more information.

If you call babel with multiple languages:
\begin{Shaded}
\begin{Highlighting}[]

\NormalTok{\textbackslash{}usepackage[languageA,languageB]\{babel\}}
\end{Highlighting}
\end{Shaded}


then the last language in the option list will be active (i.e. languageB), and you can use the command
\begin{Shaded}
\begin{Highlighting}[]

\NormalTok{\textbackslash{}selectlanguage\{languageA\}}
\end{Highlighting}
\end{Shaded}

to change the active language. You can also add short pieces of text in another language using the command
\begin{Shaded}
\begin{Highlighting}[]

\NormalTok{\textbackslash{}foreignlanguage\{languageB\}\{Text in another language\}}
\end{Highlighting}
\end{Shaded}


Babel also offers various environments for entering larger pieces of text in another language:

\begin{Shaded}
\begin{Highlighting}[]

\NormalTok{\textbackslash{}begin\{otherlanguage\}\{languageB\}}
\NormalTok{Text in language B. This environment switches all language-related definitions,}
 \NormalTok{like the language }
\NormalTok{specific names for figures, tables etc. to the other language.}
\NormalTok{\textbackslash{}end\{otherlanguage\}}
\end{Highlighting}
\end{Shaded}


The starred version of this environment typesets the main text according to the rules of the other language, but keeps the language specific string for ancillary things like figures, in the main language of the document. The environment {\ttfamily \setmainfont[Path=/usr/share/fonts/truetype/cmu/,UprightFont=cmunrm.ttf,BoldFont=cmunbx.ttf,ItalicFont=cmunti.ttf,BoldItalicFont=cmunbi.ttf]{cmuntt.ttf}\setmonofont[Path=/usr/share/fonts/truetype/cmu/,UprightFont=cmuntt.ttf,BoldFont=cmuntb.ttf,ItalicFont=cmunit.ttf,BoldItalicFont=cmuntx.ttf]{cmuntt.ttf}\ttfamily hyphenrules}{$\text{ }$}\setmainfont[Path=/usr/share/fonts/truetype/cmu/,UprightFont=cmunrm.ttf,BoldFont=cmunbx.ttf,ItalicFont=cmunti.ttf,BoldItalicFont=cmunbi.ttf]{cmunrm.ttf}\setmonofont[Path=/usr/share/fonts/truetype/cmu/,UprightFont=cmuntt.ttf,BoldFont=cmuntb.ttf,ItalicFont=cmunit.ttf,BoldItalicFont=cmuntx.ttf]{cmunrm.ttf} switches only the hyphenation patterns used; it can also be used to disallow hyphenation by using the language name \textquotesingle{}nohyphenation\textquotesingle{} (but note {\ttfamily \setmainfont[Path=/usr/share/fonts/truetype/cmu/,UprightFont=cmunrm.ttf,BoldFont=cmunbx.ttf,ItalicFont=cmunti.ttf,BoldItalicFont=cmunbi.ttf]{cmuntt.ttf}\setmonofont[Path=/usr/share/fonts/truetype/cmu/,UprightFont=cmuntt.ttf,BoldFont=cmuntb.ttf,ItalicFont=cmunit.ttf,BoldItalicFont=cmuntx.ttf]{cmuntt.ttf}\ttfamily selectlanguage*}{$\text{ }$}\setmainfont[Path=/usr/share/fonts/truetype/cmu/,UprightFont=cmunrm.ttf,BoldFont=cmunbx.ttf,ItalicFont=cmunti.ttf,BoldItalicFont=cmunbi.ttf]{cmunrm.ttf}\setmonofont[Path=/usr/share/fonts/truetype/cmu/,UprightFont=cmuntt.ttf,BoldFont=cmuntb.ttf,ItalicFont=cmunit.ttf,BoldItalicFont=cmuntx.ttf]{cmunrm.ttf} is preferred).

The \myhref{http://ftp.snt.utwente.nl/pub/software/tex/macros/latex/required/babel/base/babel.pdf}{babel manual} provides much more information on these and many other options.
\section{Multilingual versions}
\label{212}

It is possible in LaTeX to typeset the content of one document in several languages and to choose upon compilation which language to output. This might be convenient to keep a consistent sectioning and formatting across the different languages. It is also useful if you make use of multiple proper nouns and other untranslated content. Using the commands above in multilingual documents can be cumbersome, and therefore \LaTeXTT{babel}  provides a way to define shorter names. With 
\begin{Shaded}
\begin{Highlighting}[]

\NormalTok{\textbackslash{}babeltags\{de = german\}}
\end{Highlighting}
\end{Shaded}

You can write:
\begin{Shaded}
\begin{Highlighting}[]

\NormalTok{text \textbackslash{}textde\{German text\} text}
\NormalTok{text}
\NormalTok{\textbackslash{}begin\{de\}}
\NormalTok{German text}
\NormalTok{\textbackslash{}end\{de\}}
\NormalTok{text}
\end{Highlighting}
\end{Shaded}

\subsection{Alternative choice using iflang}
\label{213}
The current language can also be tested by using the \LaTeXTT{iflang} package by Heiko Oberdiek (the built-{}in feature from the babel package is not reliable). Here comes a simple example:\\

\TemplateSpaceIndent{$\text{ }${}\textbackslash{}IfLanguageName\{ngerman\}\{Hallo\}\{Hello\}}

This allows to easily distinguish between two languages without the need of defining own commands.
The babel language is changed by setting\\

\TemplateSpaceIndent{$\text{ }${}\textbackslash{}selectlanguage\{english\}}

\section{Specific languages}
\label{214}

\subsection{Arabic script}
\label{215}

For languages which use the Arabic script, including Arabic, Persian, Urdu, Pashto, Kurdish, Uyghur, etc., add the following code to your preamble:

\begin{Shaded}
\begin{Highlighting}[]

\NormalTok{\textbackslash{}usepackage\{arabtex\}}
\end{Highlighting}
\end{Shaded}

You can input text in either romanized characters or native Arabic script encodings.  Use any of the following commands and environments to enter in text:

\begin{Shaded}
\begin{Highlighting}[]

\NormalTok{\textbackslash{}< ... >}
\NormalTok{\textbackslash{}RL\{ ... \}}
\NormalTok{\textbackslash{}begin\{arabtext\} ... \textbackslash{}end\{arabtext\}.}
\end{Highlighting}
\end{Shaded}


See the \myhref{https://en.wikipedia.org/wiki/ArabTeX}{ArabTeX} Wikipedia article for further details.

You may also use the \LaTeXTT{Arabi} package within Babel to typeset Arabic and Persian

\begin{Shaded}
\begin{Highlighting}[]

\NormalTok{\textbackslash{}usepackage\{cmap\}}
\NormalTok{\textbackslash{}usepackage[LAE,LFE]\{fontenc\}}
\NormalTok{\textbackslash{}usepackage[utf8]\{inputenc\}}
\NormalTok{\textbackslash{}usepackage[arabic,farsi]\{babel\}}
\end{Highlighting}
\end{Shaded}


You may also copy and paste from PDF files produced with Arabi thanks to the support of the \LaTeXTT{cmap} package.
You may use Arabi with LyX, or with tex4ht to produce HTML. 

See \myhref{http://www.ctan.org/tex-archive/language/arabic/arabi/}{Arabi page on CTAN}
\subsection{Armenian}
\label{216}

The Armenian script uses its own characters, which will require you to install a text editor that supports \myhref{https://en.wikipedia.org/wiki/Unicode}{Unicode} and will allow you to enter UTF-{}8 text, such as \myhref{http://www.xm1math.net/texmaker/}{Texmaker} or \myhref{http://www.winedt.com/}{WinEdt}. These text editors should then be configured to compile using XeLaTeX.

Once the text editor is set up to compile with XeLaTeX, the \LaTeXTT{fontspec} package can be used to write in Armenian:

\begin{Shaded}
\begin{Highlighting}[]

\NormalTok{\textbackslash{}usepackage\{fontspec\}}
\NormalTok{\textbackslash{}setmainfont\{DejaVu Serif\}}
\end{Highlighting}
\end{Shaded}


or

\begin{Shaded}
\begin{Highlighting}[]

\NormalTok{\textbackslash{}usepackage\{fontspec\}}
\NormalTok{\textbackslash{}setmainfont\{Sylfaen\}}
\end{Highlighting}
\end{Shaded}


The Sylfaen font lacks italic and bold, but DejaVu Serif supports them.

See \myhref{https://en.wikibooks.org/wiki/\%3Ahy\%3A\%D4\%BC\%D5\%A1\%D5\%8F\%D5\%A5\%D4\%BD\%2F\%D4\%B2\%D5\%A1\%D6\%80\%D6\%87\%20\%D5\%A1\%D5\%B7\%D5\%AD\%D5\%A1\%D6\%80\%D5\%B0}{Armenian Wikibooks} for further details, especially on how to configure the Unicode supporting text editors to compile with XeLaTeX.
\subsection{Cyrillic script}
\label{217}

Version 3.7h of \LaTeXTT{babel} includes support for the {\ttfamily \setmainfont[Path=/usr/share/fonts/truetype/cmu/,UprightFont=cmunrm.ttf,BoldFont=cmunbx.ttf,ItalicFont=cmunti.ttf,BoldItalicFont=cmunbi.ttf]{cmuntt.ttf}\setmonofont[Path=/usr/share/fonts/truetype/cmu/,UprightFont=cmuntt.ttf,BoldFont=cmuntb.ttf,ItalicFont=cmunit.ttf,BoldItalicFont=cmuntx.ttf]{cmuntt.ttf}\ttfamily T2*}{$\text{ }$}\setmainfont[Path=/usr/share/fonts/truetype/cmu/,UprightFont=cmunrm.ttf,BoldFont=cmunbx.ttf,ItalicFont=cmunti.ttf,BoldItalicFont=cmunbi.ttf]{cmunrm.ttf}\setmonofont[Path=/usr/share/fonts/truetype/cmu/,UprightFont=cmuntt.ttf,BoldFont=cmuntb.ttf,ItalicFont=cmunit.ttf,BoldItalicFont=cmuntx.ttf]{cmunrm.ttf} encodings and for typesetting Bulgarian, Russian and Ukrainian texts using Cyrillic letters\myfootnote{\myfnhref{http://www.ctan.org/tex-archive/info/lshort/english/lshort.pdf}{The Not So Short Introduction to LaTeX}, 2.5.6 Support for Cyrillic, Maksym Polyakov}. Support for Cyrillic is based on standard LaTeX mechanisms plus the \LaTeXTT{fontenc} and \LaTeXTT{inputenc} packages. AMS-{}LaTeX packages should be loaded before \LaTeXTT{fontenc} and \LaTeXTT{babel}\setmainfont[Path=/usr/share/fonts/truetype/cmu/,UprightFont=cmunrm.ttf,BoldFont=cmunbx.ttf,ItalicFont=cmunti.ttf,BoldItalicFont=cmunbi.ttf]{cmunrm.ttf}\setmonofont[Path=/usr/share/fonts/truetype/cmu/,UprightFont=cmuntt.ttf,BoldFont=cmuntb.ttf,ItalicFont=cmunit.ttf,BoldItalicFont=cmuntx.ttf]{cmunrm.ttf}\textsuperscript{(Why?)}. If you are going to use Cyrillics in mathmode, you also need to load \LaTeXTT{mathtext} package before \LaTeXTT{fontenc}:

\begin{Shaded}
\begin{Highlighting}[]

\NormalTok{\textbackslash{}usepackage\{amsmath,amsthm,amssymb\}}
\NormalTok{\textbackslash{}usepackage\{mathtext\}}
 
\NormalTok{\textbackslash{}usepackage[T1,T2A]\{fontenc\}}
\NormalTok{\textbackslash{}usepackage[utf8]\{inputenc\}}
\NormalTok{\textbackslash{}usepackage[english,bulgarian,russian,ukrainian]\{babel\}}
\end{Highlighting}
\end{Shaded}


Generally, \LaTeXTT{babel} will automatically choose the default font encoding, for the above three languages this is {\ttfamily \setmainfont[Path=/usr/share/fonts/truetype/cmu/,UprightFont=cmunrm.ttf,BoldFont=cmunbx.ttf,ItalicFont=cmunti.ttf,BoldItalicFont=cmunbi.ttf]{cmuntt.ttf}\setmonofont[Path=/usr/share/fonts/truetype/cmu/,UprightFont=cmuntt.ttf,BoldFont=cmuntb.ttf,ItalicFont=cmunit.ttf,BoldItalicFont=cmuntx.ttf]{cmuntt.ttf}\ttfamily T2A.}{$\text{ }$}\setmainfont[Path=/usr/share/fonts/truetype/cmu/,UprightFont=cmunrm.ttf,BoldFont=cmunbx.ttf,ItalicFont=cmunti.ttf,BoldItalicFont=cmunbi.ttf]{cmunrm.ttf}\setmonofont[Path=/usr/share/fonts/truetype/cmu/,UprightFont=cmuntt.ttf,BoldFont=cmuntb.ttf,ItalicFont=cmunit.ttf,BoldItalicFont=cmuntx.ttf]{cmunrm.ttf} However, documents are not restricted to a single font encoding. For multilingual documents using Cyrillic and Latin-{}based languages it makes sense to include Latin font encoding explicitly. Babel will take care of switching to the appropriate font encoding when a different language is selected within the document.

On modern operating systems it is beneficial to use Unicode (\LaTeXTT{utf8} or \LaTeXTT{utf8x}) instead of KOI8-{}RU (\LaTeXTT{koi8-{}ru}) as an input encoding for Cyrillic text.

In addition to enabling hyphenations, translating automatically generated text strings, and activating some language specific typographic rules (like \LaTeXTT{\textbackslash{}frenchspacing}), \LaTeXTT{babel} provides some commands allowing typesetting according to the standards of Bulgarian, Russian, or Ukrainian languages.

For all three languages, language specific punctuation is provided: the Cyrillic dash for the text (it is little narrower than Latin dash and surrounded by tiny spaces), a dash for direct speech, quotes, and commands to facilitate hyphenation:

\begin{longtable}{|>{\RaggedRight}p{0.19898\linewidth}|>{\RaggedRight}p{0.74388\linewidth}|} \hline 
{\bfseries \hspace*{0pt}\ignorespaces{}\hspace*{0pt} Key combination }&{\bfseries \hspace*{0pt}\ignorespaces{}\hspace*{0pt} Action}\endhead  \hline \hspace*{0pt}\ignorespaces{}\hspace*{0pt} \LaTeXTT{\symbol{34}|} &\hspace*{0pt}\ignorespaces{}\hspace*{0pt} No ligature at this position.\\ \hline \hspace*{0pt}\ignorespaces{}\hspace*{0pt} \LaTeXTT{\symbol{34}-{}} &\hspace*{0pt}\ignorespaces{}\hspace*{0pt} Explicit hyphen sign, allowing hyphenation in the rest of the word.\\ \hline \hspace*{0pt}\ignorespaces{}\hspace*{0pt} \LaTeXTT{\symbol{34}-{}-{}-{}} &\hspace*{0pt}\ignorespaces{}\hspace*{0pt} Cyrillic emdash in plain text.\\ \hline \hspace*{0pt}\ignorespaces{}\hspace*{0pt} \LaTeXTT{\symbol{34}-{}-{}\~{}} &\hspace*{0pt}\ignorespaces{}\hspace*{0pt} Cyrillic emdash in compound names (surnames).\\ \hline \hspace*{0pt}\ignorespaces{}\hspace*{0pt} \LaTeXTT{\symbol{34}-{}-{}*} &\hspace*{0pt}\ignorespaces{}\hspace*{0pt} Cyrillic emdash for denoting direct speech.\\ \hline \hspace*{0pt}\ignorespaces{}\hspace*{0pt} \LaTeXTT{\symbol{34}\symbol{34}} &\hspace*{0pt}\ignorespaces{}\hspace*{0pt} Similar to \LaTeXTT{\symbol{34}-{}}, but it produces no hyphen sign (used for compound words with hyphen, e.g. \LaTeXTT{x-{}\symbol{34}\symbol{34}y} or some other signs as “disable/enable”).\\ \hline \hspace*{0pt}\ignorespaces{}\hspace*{0pt} \LaTeXTT{\symbol{34}\~{}} &\hspace*{0pt}\ignorespaces{}\hspace*{0pt} Compound word mark without a breakpoint.\\ \hline \hspace*{0pt}\ignorespaces{}\hspace*{0pt} \LaTeXTT{\symbol{34}=} &\hspace*{0pt}\ignorespaces{}\hspace*{0pt} Compound word mark with a breakpoint, allowing hyphenation in the composing words.\\ \hline \hspace*{0pt}\ignorespaces{}\hspace*{0pt} \LaTeXTT{\symbol{34},} &\hspace*{0pt}\ignorespaces{}\hspace*{0pt} Thinspace for initials with a breakpoint in a following surname.\\ \hline \hspace*{0pt}\ignorespaces{}\hspace*{0pt} \LaTeXTT{\symbol{34}‘} &\hspace*{0pt}\ignorespaces{}\hspace*{0pt} German opening double quote (,,).\\ \hline \hspace*{0pt}\ignorespaces{}\hspace*{0pt} \LaTeXTT{\symbol{34}’} &\hspace*{0pt}\ignorespaces{}\hspace*{0pt} German closing double quote (“).\\ \hline \hspace*{0pt}\ignorespaces{}\hspace*{0pt} \LaTeXTT{\symbol{34}<{}} &\hspace*{0pt}\ignorespaces{}\hspace*{0pt} French opening double quote (<{}<{}).\\ \hline \hspace*{0pt}\ignorespaces{}\hspace*{0pt} \LaTeXTT{\symbol{34}>{}} &\hspace*{0pt}\ignorespaces{}\hspace*{0pt} French closing double quote (>{}>{}).\\ \hline 
\end{longtable}


The Russian and Ukrainian options of \LaTeXTT{babel} define the commands
\begin{Shaded}
\begin{Highlighting}[]

\NormalTok{\textbackslash{}Asbuk}
\NormalTok{\textbackslash{}asbuk}
\end{Highlighting}
\end{Shaded}

which act like \LaTeXTT{\textbackslash{}Alph} and \LaTeXTT{\textbackslash{}alph} (commands for turning counters into letters, {\itshape \setmainfont[Path=/usr/share/fonts/truetype/cmu/,UprightFont=cmunrm.ttf,BoldFont=cmunbx.ttf,ItalicFont=cmunti.ttf,BoldItalicFont=cmunbi.ttf]{cmunti.ttf}\setmonofont[Path=/usr/share/fonts/truetype/cmu/,UprightFont=cmuntt.ttf,BoldFont=cmuntb.ttf,ItalicFont=cmunit.ttf,BoldItalicFont=cmuntx.ttf]{cmunti.ttf}\itshape e.g.}{$\text{ }$}\setmainfont[Path=/usr/share/fonts/truetype/cmu/,UprightFont=cmunrm.ttf,BoldFont=cmunbx.ttf,ItalicFont=cmunti.ttf,BoldItalicFont=cmunbi.ttf]{cmunrm.ttf}\setmonofont[Path=/usr/share/fonts/truetype/cmu/,UprightFont=cmuntt.ttf,BoldFont=cmuntb.ttf,ItalicFont=cmunit.ttf,BoldItalicFont=cmuntx.ttf]{cmunrm.ttf} \LaTeXTT{a, b, c...}), but produce capital and small letters of Russian or Ukrainian alphabets (whichever is the active language of the document).

The Bulgarian option of \LaTeXTT{babel} provides the commands
\begin{Shaded}
\begin{Highlighting}[]

\NormalTok{\textbackslash{}enumBul}
\NormalTok{\textbackslash{}enumLat}
\NormalTok{\textbackslash{}enumEng}
\end{Highlighting}
\end{Shaded}

which make \LaTeXTT{\textbackslash{}Alph} and \LaTeXTT{\textbackslash{}alph} produce letters of either Bulgarian or Latin (English) alphabets. The default behaviour of \LaTeXTT{\textbackslash{}Alph} and \LaTeXTT{\textbackslash{}alph} for the Bulgarian language option is to produce letters from the Bulgarian alphabet.

See the Bulgarian translation of \symbol{34}The Not So Short Introduction to LaTeX\symbol{34} \myfootnote{\myfnhref{http://www.ctan.org/tex-archive/info/lshort/bulgarian/lshort-bg.pdf}{The Not So Short Introduction to LaTeX}, Bulgarian translation} for a method to type Cyrillic letters directly from the keyboard using a different distribution.
\subsection{Chinese}
\label{218}

One possible Chinese support is made available thanks to the \LaTeXTT{CJK} package collection. If you are using a package manager or a portage tree, the CJK collection is usually in a separate package because of its size (mainly due to fonts).

Make sure your document is saved using the UTF-{}8 character encoding. See \mylref{192}{Special Characters} for more details.
Put the parts where you want to write chinese characters in a \LaTeXTT{CJK} environment.

\begin{Shaded}
\begin{Highlighting}[]

\NormalTok{\textbackslash{}documentclass\{article\}}
\NormalTok{\textbackslash{}usepackage\{CJK\}}
 
\NormalTok{\textbackslash{}begin\{document\}}
 
\NormalTok{\textbackslash{}begin\{CJK\}\{UTF8\}\{gbsn\}}
\NormalTok{你好}
\NormalTok{You can mix latin letters and chinese.}
\NormalTok{\textbackslash{}end\{CJK\}}
 
\NormalTok{\textbackslash{}end\{document\}}
\end{Highlighting}
\end{Shaded}


The last argument specifies the font. It must fit the desired language, since fonts are different for Chinese, Japanese and Korean. Possible choices for Chinese include:
\begin{myitemize}
\item{}  gbsn (\setmainfont[Path=/usr/share/fonts/truetype/wqy/]{wqy-zenhei.ttc}\setmonofont[Path=/usr/share/fonts/truetype/wqy/]{wqy-zenhei.ttc}简体宋体\setmainfont[Path=/usr/share/fonts/truetype/cmu/,UprightFont=cmunrm.ttf,BoldFont=cmunbx.ttf,ItalicFont=cmunti.ttf,BoldItalicFont=cmunbi.ttf]{cmunrm.ttf}\setmonofont[Path=/usr/share/fonts/truetype/cmu/,UprightFont=cmuntt.ttf,BoldFont=cmuntb.ttf,ItalicFont=cmunit.ttf,BoldItalicFont=cmuntx.ttf]{cmunrm.ttf}, simplified Chinese)
\item{}  gkai (\setmainfont[Path=/usr/share/fonts/truetype/wqy/]{wqy-zenhei.ttc}\setmonofont[Path=/usr/share/fonts/truetype/wqy/]{wqy-zenhei.ttc}简体楷体\setmainfont[Path=/usr/share/fonts/truetype/cmu/,UprightFont=cmunrm.ttf,BoldFont=cmunbx.ttf,ItalicFont=cmunti.ttf,BoldItalicFont=cmunbi.ttf]{cmunrm.ttf}\setmonofont[Path=/usr/share/fonts/truetype/cmu/,UprightFont=cmuntt.ttf,BoldFont=cmuntb.ttf,ItalicFont=cmunit.ttf,BoldItalicFont=cmuntx.ttf]{cmunrm.ttf}, simplified Chinese)
\item{}  bsmi (\setmainfont[Path=/usr/share/fonts/truetype/wqy/]{wqy-zenhei.ttc}\setmonofont[Path=/usr/share/fonts/truetype/wqy/]{wqy-zenhei.ttc}繁体细上海宋体\setmainfont[Path=/usr/share/fonts/truetype/cmu/,UprightFont=cmunrm.ttf,BoldFont=cmunbx.ttf,ItalicFont=cmunti.ttf,BoldItalicFont=cmunbi.ttf]{cmunrm.ttf}\setmonofont[Path=/usr/share/fonts/truetype/cmu/,UprightFont=cmuntt.ttf,BoldFont=cmuntb.ttf,ItalicFont=cmunit.ttf,BoldItalicFont=cmuntx.ttf]{cmunrm.ttf}, traditional Chinese)
\item{}  bkai (\setmainfont[Path=/usr/share/fonts/truetype/wqy/]{wqy-zenhei.ttc}\setmonofont[Path=/usr/share/fonts/truetype/wqy/]{wqy-zenhei.ttc}繁体标楷体\setmainfont[Path=/usr/share/fonts/truetype/cmu/,UprightFont=cmunrm.ttf,BoldFont=cmunbx.ttf,ItalicFont=cmunti.ttf,BoldItalicFont=cmunbi.ttf]{cmunrm.ttf}\setmonofont[Path=/usr/share/fonts/truetype/cmu/,UprightFont=cmuntt.ttf,BoldFont=cmuntb.ttf,ItalicFont=cmunit.ttf,BoldItalicFont=cmuntx.ttf]{cmunrm.ttf}, traditional Chinese)
\end{myitemize}

\subsection{Czech}
\label{219}

Czech is fine using
\begin{Shaded}
\begin{Highlighting}[]

\NormalTok{\textbackslash{}usepackage[czech]\{babel\}}
\end{Highlighting}
\end{Shaded}

UTF-{}8 allows you to have „czech quotation marks“ directly in your text. Otherwise, there are macros {\itshape \setmainfont[Path=/usr/share/fonts/truetype/cmu/,UprightFont=cmunrm.ttf,BoldFont=cmunbx.ttf,ItalicFont=cmunti.ttf,BoldItalicFont=cmunbi.ttf]{cmunti.ttf}\setmonofont[Path=/usr/share/fonts/truetype/cmu/,UprightFont=cmuntt.ttf,BoldFont=cmuntb.ttf,ItalicFont=cmunit.ttf,BoldItalicFont=cmuntx.ttf]{cmunti.ttf}\itshape \textbackslash{}clqq}{$\text{ }$}\setmainfont[Path=/usr/share/fonts/truetype/cmu/,UprightFont=cmunrm.ttf,BoldFont=cmunbx.ttf,ItalicFont=cmunti.ttf,BoldItalicFont=cmunbi.ttf]{cmunrm.ttf}\setmonofont[Path=/usr/share/fonts/truetype/cmu/,UprightFont=cmuntt.ttf,BoldFont=cmuntb.ttf,ItalicFont=cmunit.ttf,BoldItalicFont=cmuntx.ttf]{cmunrm.ttf} and {\itshape \setmainfont[Path=/usr/share/fonts/truetype/cmu/,UprightFont=cmunrm.ttf,BoldFont=cmunbx.ttf,ItalicFont=cmunti.ttf,BoldItalicFont=cmunbi.ttf]{cmunti.ttf}\setmonofont[Path=/usr/share/fonts/truetype/cmu/,UprightFont=cmuntt.ttf,BoldFont=cmuntb.ttf,ItalicFont=cmunit.ttf,BoldItalicFont=cmuntx.ttf]{cmunti.ttf}\itshape \textbackslash{}crqq}{$\text{ }$}\setmainfont[Path=/usr/share/fonts/truetype/cmu/,UprightFont=cmunrm.ttf,BoldFont=cmunbx.ttf,ItalicFont=cmunti.ttf,BoldItalicFont=cmunbi.ttf]{cmunrm.ttf}\setmonofont[Path=/usr/share/fonts/truetype/cmu/,UprightFont=cmuntt.ttf,BoldFont=cmuntb.ttf,ItalicFont=cmunit.ttf,BoldItalicFont=cmuntx.ttf]{cmunrm.ttf} to produce left and right quote. You can place quotated text inside \LaTeXTT{\textbackslash{}uv}.
\subsection{Finnish}
\label{220}

Finnish language hyphenation is enabled with:
\begin{Shaded}
\begin{Highlighting}[]

\NormalTok{\textbackslash{}usepackage[finnish]\{babel\}}
\end{Highlighting}
\end{Shaded}

This will also automatically change document language (section names, etc.) to Finnish.
\subsection{French}
\label{221}

You can load French language support with the following command:

\begin{Shaded}
\begin{Highlighting}[]

\NormalTok{\textbackslash{}usepackage[frenchb]\{babel\}}
\end{Highlighting}
\end{Shaded}


There are multiple options for typesetting French documents, depending on the flavor of French: \LaTeXTT{french}, \LaTeXTT{frenchb}, and \LaTeXTT{francais} for Parisian French, and \LaTeXTT{acadian} and \LaTeXTT{canadien} for new-{}world French. If you do not know or do not really care, we would recommend using \LaTeXTT{frenchb}.

However, as of version 3.0 of \LaTeXTT{babel-{}french}, it is advised to choose the language as a global option with the following command\myfootnote{\myfnhref{http://mirrors.ctan.org/macros/latex/contrib/babel-contrib/french/frenchb.pdf}{\LaTeXTT{babel-{}french} documentation}: \symbol{34}the French language should now be loaded as french, not as frenchb or francais and preferably as a global option of \LaTeXTT{\textbackslash{}documentclass}. Some tolerance still exists in v3.0, but do not rely on it.\symbol{34}}:

\begin{Shaded}
\begin{Highlighting}[]

\NormalTok{\textbackslash{}documentclass[french]\{article\}}
\NormalTok{\textbackslash{}usepackage\{babel\}}
\end{Highlighting}
\end{Shaded}


All enable French hyphenation, if you have configured your LaTeX system accordingly. All of these also change all automatic text into French: \LaTeXTT{\textbackslash{}chapter} prints {\itshape \setmainfont[Path=/usr/share/fonts/truetype/cmu/,UprightFont=cmunrm.ttf,BoldFont=cmunbx.ttf,ItalicFont=cmunti.ttf,BoldItalicFont=cmunbi.ttf]{cmunti.ttf}\setmonofont[Path=/usr/share/fonts/truetype/cmu/,UprightFont=cmuntt.ttf,BoldFont=cmuntb.ttf,ItalicFont=cmunit.ttf,BoldItalicFont=cmuntx.ttf]{cmunti.ttf}\itshape Chapitre}\setmainfont[Path=/usr/share/fonts/truetype/cmu/,UprightFont=cmunrm.ttf,BoldFont=cmunbx.ttf,ItalicFont=cmunti.ttf,BoldItalicFont=cmunbi.ttf]{cmunrm.ttf}\setmonofont[Path=/usr/share/fonts/truetype/cmu/,UprightFont=cmuntt.ttf,BoldFont=cmuntb.ttf,ItalicFont=cmunit.ttf,BoldItalicFont=cmuntx.ttf]{cmunrm.ttf}, \LaTeXTT{\textbackslash{}today} prints the current date in French and so on. A set of new commands also becomes available, which allows you to write French input files more easily. Check out the following table for inspiration:

\begin{longtable}{|>{\RaggedRight}p{0.65232\linewidth}|>{\RaggedRight}p{0.29054\linewidth}|} \hline 
{\bfseries \hspace*{0pt}\ignorespaces{}\hspace*{0pt}input code}&{\bfseries \hspace*{0pt}\ignorespaces{}\hspace*{0pt}rendered output}\endhead  \hline \hspace*{0pt}\ignorespaces{}\hspace*{0pt}{\ttfamily \setmainfont[Path=/usr/share/fonts/truetype/cmu/,UprightFont=cmunrm.ttf,BoldFont=cmunbx.ttf,ItalicFont=cmunti.ttf,BoldItalicFont=cmunbi.ttf]{cmuntt.ttf}\setmonofont[Path=/usr/share/fonts/truetype/cmu/,UprightFont=cmuntt.ttf,BoldFont=cmuntb.ttf,ItalicFont=cmunit.ttf,BoldItalicFont=cmuntx.ttf]{cmuntt.ttf}\ttfamily \textbackslash{}og guillemets \textbackslash{}fg\{\}}&\hspace*{0pt}\ignorespaces{}\hspace*{0pt}\setmainfont[Path=/usr/share/fonts/truetype/cmu/,UprightFont=cmunrm.ttf,BoldFont=cmunbx.ttf,ItalicFont=cmunti.ttf,BoldItalicFont=cmunbi.ttf]{cmunrm.ttf}\setmonofont[Path=/usr/share/fonts/truetype/cmu/,UprightFont=cmuntt.ttf,BoldFont=cmuntb.ttf,ItalicFont=cmunit.ttf,BoldItalicFont=cmuntx.ttf]{cmunrm.ttf}« guillemets »\\ \hline \hspace*{0pt}\ignorespaces{}\hspace*{0pt}{\ttfamily \setmainfont[Path=/usr/share/fonts/truetype/cmu/,UprightFont=cmunrm.ttf,BoldFont=cmunbx.ttf,ItalicFont=cmunti.ttf,BoldItalicFont=cmunbi.ttf]{cmuntt.ttf}\setmonofont[Path=/usr/share/fonts/truetype/cmu/,UprightFont=cmuntt.ttf,BoldFont=cmuntb.ttf,ItalicFont=cmunit.ttf,BoldItalicFont=cmuntx.ttf]{cmuntt.ttf}\ttfamily M\textbackslash{}up\{me\}, D\textbackslash{}up\{r\}}&\hspace*{0pt}\ignorespaces{}\hspace*{0pt}\setmainfont[Path=/usr/share/fonts/truetype/cmu/,UprightFont=cmunrm.ttf,BoldFont=cmunbx.ttf,ItalicFont=cmunti.ttf,BoldItalicFont=cmunbi.ttf]{cmunrm.ttf}\setmonofont[Path=/usr/share/fonts/truetype/cmu/,UprightFont=cmuntt.ttf,BoldFont=cmuntb.ttf,ItalicFont=cmunit.ttf,BoldItalicFont=cmuntx.ttf]{cmunrm.ttf}M\setmainfont[Path=/usr/share/fonts/truetype/cmu/,UprightFont=cmunrm.ttf,BoldFont=cmunbx.ttf,ItalicFont=cmunti.ttf,BoldItalicFont=cmunbi.ttf]{cmunrm.ttf}\setmonofont[Path=/usr/share/fonts/truetype/cmu/,UprightFont=cmuntt.ttf,BoldFont=cmuntb.ttf,ItalicFont=cmunit.ttf,BoldItalicFont=cmuntx.ttf]{cmunrm.ttf}\textsuperscript{me}, D\setmainfont[Path=/usr/share/fonts/truetype/cmu/,UprightFont=cmunrm.ttf,BoldFont=cmunbx.ttf,ItalicFont=cmunti.ttf,BoldItalicFont=cmunbi.ttf]{cmunrm.ttf}\setmonofont[Path=/usr/share/fonts/truetype/cmu/,UprightFont=cmuntt.ttf,BoldFont=cmuntb.ttf,ItalicFont=cmunit.ttf,BoldItalicFont=cmuntx.ttf]{cmunrm.ttf}\textsuperscript{r}\\ \hline \hspace*{0pt}\ignorespaces{}\hspace*{0pt}{\ttfamily \setmainfont[Path=/usr/share/fonts/truetype/cmu/,UprightFont=cmunrm.ttf,BoldFont=cmunbx.ttf,ItalicFont=cmunti.ttf,BoldItalicFont=cmunbi.ttf]{cmuntt.ttf}\setmonofont[Path=/usr/share/fonts/truetype/cmu/,UprightFont=cmuntt.ttf,BoldFont=cmuntb.ttf,ItalicFont=cmunit.ttf,BoldItalicFont=cmuntx.ttf]{cmuntt.ttf}\ttfamily 1\textbackslash{}ier\{\}, 1\textbackslash{}iere\{\}, 1\textbackslash{}ieres\{\}}&\hspace*{0pt}\ignorespaces{}\hspace*{0pt}\setmainfont[Path=/usr/share/fonts/truetype/cmu/,UprightFont=cmunrm.ttf,BoldFont=cmunbx.ttf,ItalicFont=cmunti.ttf,BoldItalicFont=cmunbi.ttf]{cmunrm.ttf}\setmonofont[Path=/usr/share/fonts/truetype/cmu/,UprightFont=cmuntt.ttf,BoldFont=cmuntb.ttf,ItalicFont=cmunit.ttf,BoldItalicFont=cmuntx.ttf]{cmunrm.ttf}1\setmainfont[Path=/usr/share/fonts/truetype/cmu/,UprightFont=cmunrm.ttf,BoldFont=cmunbx.ttf,ItalicFont=cmunti.ttf,BoldItalicFont=cmunbi.ttf]{cmunrm.ttf}\setmonofont[Path=/usr/share/fonts/truetype/cmu/,UprightFont=cmuntt.ttf,BoldFont=cmuntb.ttf,ItalicFont=cmunit.ttf,BoldItalicFont=cmuntx.ttf]{cmunrm.ttf}\textsuperscript{er}, 1\setmainfont[Path=/usr/share/fonts/truetype/cmu/,UprightFont=cmunrm.ttf,BoldFont=cmunbx.ttf,ItalicFont=cmunti.ttf,BoldItalicFont=cmunbi.ttf]{cmunrm.ttf}\setmonofont[Path=/usr/share/fonts/truetype/cmu/,UprightFont=cmuntt.ttf,BoldFont=cmuntb.ttf,ItalicFont=cmunit.ttf,BoldItalicFont=cmuntx.ttf]{cmunrm.ttf}\textsuperscript{re}, 1\setmainfont[Path=/usr/share/fonts/truetype/cmu/,UprightFont=cmunrm.ttf,BoldFont=cmunbx.ttf,ItalicFont=cmunti.ttf,BoldItalicFont=cmunbi.ttf]{cmunrm.ttf}\setmonofont[Path=/usr/share/fonts/truetype/cmu/,UprightFont=cmuntt.ttf,BoldFont=cmuntb.ttf,ItalicFont=cmunit.ttf,BoldItalicFont=cmuntx.ttf]{cmunrm.ttf}\textsuperscript{res}\\ \hline \hspace*{0pt}\ignorespaces{}\hspace*{0pt}{\ttfamily \setmainfont[Path=/usr/share/fonts/truetype/cmu/,UprightFont=cmunrm.ttf,BoldFont=cmunbx.ttf,ItalicFont=cmunti.ttf,BoldItalicFont=cmunbi.ttf]{cmuntt.ttf}\setmonofont[Path=/usr/share/fonts/truetype/cmu/,UprightFont=cmuntt.ttf,BoldFont=cmuntb.ttf,ItalicFont=cmunit.ttf,BoldItalicFont=cmuntx.ttf]{cmuntt.ttf}\ttfamily 2\textbackslash{}ieme\{\} 4\textbackslash{}iemes\{\}}&\hspace*{0pt}\ignorespaces{}\hspace*{0pt}\setmainfont[Path=/usr/share/fonts/truetype/cmu/,UprightFont=cmunrm.ttf,BoldFont=cmunbx.ttf,ItalicFont=cmunti.ttf,BoldItalicFont=cmunbi.ttf]{cmunrm.ttf}\setmonofont[Path=/usr/share/fonts/truetype/cmu/,UprightFont=cmuntt.ttf,BoldFont=cmuntb.ttf,ItalicFont=cmunit.ttf,BoldItalicFont=cmuntx.ttf]{cmunrm.ttf}2\setmainfont[Path=/usr/share/fonts/truetype/cmu/,UprightFont=cmunrm.ttf,BoldFont=cmunbx.ttf,ItalicFont=cmunti.ttf,BoldItalicFont=cmunbi.ttf]{cmunrm.ttf}\setmonofont[Path=/usr/share/fonts/truetype/cmu/,UprightFont=cmuntt.ttf,BoldFont=cmuntb.ttf,ItalicFont=cmunit.ttf,BoldItalicFont=cmuntx.ttf]{cmunrm.ttf}\textsuperscript{e} 4\setmainfont[Path=/usr/share/fonts/truetype/cmu/,UprightFont=cmunrm.ttf,BoldFont=cmunbx.ttf,ItalicFont=cmunti.ttf,BoldItalicFont=cmunbi.ttf]{cmunrm.ttf}\setmonofont[Path=/usr/share/fonts/truetype/cmu/,UprightFont=cmuntt.ttf,BoldFont=cmuntb.ttf,ItalicFont=cmunit.ttf,BoldItalicFont=cmuntx.ttf]{cmunrm.ttf}\textsuperscript{es}\\ \hline \hspace*{0pt}\ignorespaces{}\hspace*{0pt}{\ttfamily \setmainfont[Path=/usr/share/fonts/truetype/cmu/,UprightFont=cmunrm.ttf,BoldFont=cmunbx.ttf,ItalicFont=cmunti.ttf,BoldItalicFont=cmunbi.ttf]{cmuntt.ttf}\setmonofont[Path=/usr/share/fonts/truetype/cmu/,UprightFont=cmuntt.ttf,BoldFont=cmuntb.ttf,ItalicFont=cmunit.ttf,BoldItalicFont=cmuntx.ttf]{cmuntt.ttf}\ttfamily \textbackslash{}No 1, \textbackslash{}no 2}&\hspace*{0pt}\ignorespaces{}\hspace*{0pt}\setmainfont[Path=/usr/share/fonts/truetype/cmu/,UprightFont=cmunrm.ttf,BoldFont=cmunbx.ttf,ItalicFont=cmunti.ttf,BoldItalicFont=cmunbi.ttf]{cmunrm.ttf}\setmonofont[Path=/usr/share/fonts/truetype/cmu/,UprightFont=cmuntt.ttf,BoldFont=cmuntb.ttf,ItalicFont=cmunit.ttf,BoldItalicFont=cmuntx.ttf]{cmunrm.ttf}N° 1, n° 2\\ \hline \hspace*{0pt}\ignorespaces{}\hspace*{0pt}{\ttfamily \setmainfont[Path=/usr/share/fonts/truetype/cmu/,UprightFont=cmunrm.ttf,BoldFont=cmunbx.ttf,ItalicFont=cmunti.ttf,BoldItalicFont=cmunbi.ttf]{cmuntt.ttf}\setmonofont[Path=/usr/share/fonts/truetype/cmu/,UprightFont=cmuntt.ttf,BoldFont=cmuntb.ttf,ItalicFont=cmunit.ttf,BoldItalicFont=cmuntx.ttf]{cmuntt.ttf}\ttfamily 20\~{}\textbackslash{}degres C, 45\textbackslash{}degres}&\hspace*{0pt}\ignorespaces{}\hspace*{0pt}\setmainfont[Path=/usr/share/fonts/truetype/cmu/,UprightFont=cmunrm.ttf,BoldFont=cmunbx.ttf,ItalicFont=cmunti.ttf,BoldItalicFont=cmunbi.ttf]{cmunrm.ttf}\setmonofont[Path=/usr/share/fonts/truetype/cmu/,UprightFont=cmuntt.ttf,BoldFont=cmuntb.ttf,ItalicFont=cmunit.ttf,BoldItalicFont=cmuntx.ttf]{cmunrm.ttf}20 °C, 45°\\ \hline \hspace*{0pt}\ignorespaces{}\hspace*{0pt}{\ttfamily \setmainfont[Path=/usr/share/fonts/truetype/cmu/,UprightFont=cmunrm.ttf,BoldFont=cmunbx.ttf,ItalicFont=cmunti.ttf,BoldItalicFont=cmunbi.ttf]{cmuntt.ttf}\setmonofont[Path=/usr/share/fonts/truetype/cmu/,UprightFont=cmuntt.ttf,BoldFont=cmuntb.ttf,ItalicFont=cmunit.ttf,BoldItalicFont=cmuntx.ttf]{cmuntt.ttf}\ttfamily M. \textbackslash{}bsc\{Durand\}}&\hspace*{0pt}\ignorespaces{}\hspace*{0pt}{$\text{ }$}\setmainfont[Path=/usr/share/fonts/truetype/cmu/,UprightFont=cmunrm.ttf,BoldFont=cmunbx.ttf,ItalicFont=cmunti.ttf,BoldItalicFont=cmunbi.ttf]{cmunrm.ttf}\setmonofont[Path=/usr/share/fonts/truetype/cmu/,UprightFont=cmuntt.ttf,BoldFont=cmuntb.ttf,ItalicFont=cmunit.ttf,BoldItalicFont=cmuntx.ttf]{cmunrm.ttf} M. Durand\\ \hline \hspace*{0pt}\ignorespaces{}\hspace*{0pt}{\ttfamily \setmainfont[Path=/usr/share/fonts/truetype/cmu/,UprightFont=cmunrm.ttf,BoldFont=cmunbx.ttf,ItalicFont=cmunti.ttf,BoldItalicFont=cmunbi.ttf]{cmuntt.ttf}\setmonofont[Path=/usr/share/fonts/truetype/cmu/,UprightFont=cmuntt.ttf,BoldFont=cmuntb.ttf,ItalicFont=cmunit.ttf,BoldItalicFont=cmuntx.ttf]{cmuntt.ttf}\ttfamily \textbackslash{}nombre\{1234,56789\}}&\hspace*{0pt}\ignorespaces{}\hspace*{0pt}\setmainfont[Path=/usr/share/fonts/truetype/cmu/,UprightFont=cmunrm.ttf,BoldFont=cmunbx.ttf,ItalicFont=cmunti.ttf,BoldItalicFont=cmunbi.ttf]{cmunrm.ttf}\setmonofont[Path=/usr/share/fonts/truetype/cmu/,UprightFont=cmuntt.ttf,BoldFont=cmuntb.ttf,ItalicFont=cmunit.ttf,BoldItalicFont=cmuntx.ttf]{cmunrm.ttf}1 234,567 89\\ \hline 
\end{longtable}


You may want to typeset guillemets and other French characters directly if your keyboard have them. Running Xorg (*BSD and GNU/Linux), you may want to use the {\ttfamily \setmainfont[Path=/usr/share/fonts/truetype/cmu/,UprightFont=cmunrm.ttf,BoldFont=cmunbx.ttf,ItalicFont=cmunti.ttf,BoldItalicFont=cmunbi.ttf]{cmuntt.ttf}\setmonofont[Path=/usr/share/fonts/truetype/cmu/,UprightFont=cmuntt.ttf,BoldFont=cmuntb.ttf,ItalicFont=cmunit.ttf,BoldItalicFont=cmuntx.ttf]{cmuntt.ttf}\ttfamily oss}{$\text{ }$}\setmainfont[Path=/usr/share/fonts/truetype/cmu/,UprightFont=cmunrm.ttf,BoldFont=cmunbx.ttf,ItalicFont=cmunti.ttf,BoldItalicFont=cmunbi.ttf]{cmunrm.ttf}\setmonofont[Path=/usr/share/fonts/truetype/cmu/,UprightFont=cmuntt.ttf,BoldFont=cmuntb.ttf,ItalicFont=cmunit.ttf,BoldItalicFont=cmuntx.ttf]{cmunrm.ttf} variant which features some nice shortcuts, like

\begin{longtable}{|>{\RaggedRight}p{0.62806\linewidth}|>{\RaggedRight}p{0.31480\linewidth}|} \hline 
{\bfseries \hspace*{0pt}\ignorespaces{}\hspace*{0pt}Key combination}&{\bfseries \hspace*{0pt}\ignorespaces{}\hspace*{0pt}Character}\endhead  \hline \hspace*{0pt}\ignorespaces{}\hspace*{0pt}{\ttfamily \setmainfont[Path=/usr/share/fonts/truetype/cmu/,UprightFont=cmunrm.ttf,BoldFont=cmunbx.ttf,ItalicFont=cmunti.ttf,BoldItalicFont=cmunbi.ttf]{cmuntt.ttf}\setmonofont[Path=/usr/share/fonts/truetype/cmu/,UprightFont=cmuntt.ttf,BoldFont=cmuntb.ttf,ItalicFont=cmunit.ttf,BoldItalicFont=cmuntx.ttf]{cmuntt.ttf}\ttfamily Alt Gr + w}{$\text{ }$}\setmainfont[Path=/usr/share/fonts/truetype/cmu/,UprightFont=cmunrm.ttf,BoldFont=cmunbx.ttf,ItalicFont=cmunti.ttf,BoldItalicFont=cmunbi.ttf]{cmunrm.ttf}\setmonofont[Path=/usr/share/fonts/truetype/cmu/,UprightFont=cmuntt.ttf,BoldFont=cmuntb.ttf,ItalicFont=cmunit.ttf,BoldItalicFont=cmuntx.ttf]{cmunrm.ttf} &\hspace*{0pt}\ignorespaces{}\hspace*{0pt} «\\ \hline \hspace*{0pt}\ignorespaces{}\hspace*{0pt}{\ttfamily \setmainfont[Path=/usr/share/fonts/truetype/cmu/,UprightFont=cmunrm.ttf,BoldFont=cmunbx.ttf,ItalicFont=cmunti.ttf,BoldItalicFont=cmunbi.ttf]{cmuntt.ttf}\setmonofont[Path=/usr/share/fonts/truetype/cmu/,UprightFont=cmuntt.ttf,BoldFont=cmuntb.ttf,ItalicFont=cmunit.ttf,BoldItalicFont=cmuntx.ttf]{cmuntt.ttf}\ttfamily Alt Gr + x}{$\text{ }$}\setmainfont[Path=/usr/share/fonts/truetype/cmu/,UprightFont=cmunrm.ttf,BoldFont=cmunbx.ttf,ItalicFont=cmunti.ttf,BoldItalicFont=cmunbi.ttf]{cmunrm.ttf}\setmonofont[Path=/usr/share/fonts/truetype/cmu/,UprightFont=cmuntt.ttf,BoldFont=cmuntb.ttf,ItalicFont=cmunit.ttf,BoldItalicFont=cmuntx.ttf]{cmunrm.ttf} &\hspace*{0pt}\ignorespaces{}\hspace*{0pt} »\\ \hline \hspace*{0pt}\ignorespaces{}\hspace*{0pt}{\ttfamily \setmainfont[Path=/usr/share/fonts/truetype/cmu/,UprightFont=cmunrm.ttf,BoldFont=cmunbx.ttf,ItalicFont=cmunti.ttf,BoldItalicFont=cmunbi.ttf]{cmuntt.ttf}\setmonofont[Path=/usr/share/fonts/truetype/cmu/,UprightFont=cmuntt.ttf,BoldFont=cmuntb.ttf,ItalicFont=cmunit.ttf,BoldItalicFont=cmuntx.ttf]{cmuntt.ttf}\ttfamily Alt Gr + Shift + é}{$\text{ }$}\setmainfont[Path=/usr/share/fonts/truetype/cmu/,UprightFont=cmunrm.ttf,BoldFont=cmunbx.ttf,ItalicFont=cmunti.ttf,BoldItalicFont=cmunbi.ttf]{cmunrm.ttf}\setmonofont[Path=/usr/share/fonts/truetype/cmu/,UprightFont=cmuntt.ttf,BoldFont=cmuntb.ttf,ItalicFont=cmunit.ttf,BoldItalicFont=cmuntx.ttf]{cmunrm.ttf} &\hspace*{0pt}\ignorespaces{}\hspace*{0pt} É\\ \hline \hspace*{0pt}\ignorespaces{}\hspace*{0pt}{\ttfamily \setmainfont[Path=/usr/share/fonts/truetype/cmu/,UprightFont=cmunrm.ttf,BoldFont=cmunbx.ttf,ItalicFont=cmunti.ttf,BoldItalicFont=cmunbi.ttf]{cmuntt.ttf}\setmonofont[Path=/usr/share/fonts/truetype/cmu/,UprightFont=cmuntt.ttf,BoldFont=cmuntb.ttf,ItalicFont=cmunit.ttf,BoldItalicFont=cmuntx.ttf]{cmuntt.ttf}\ttfamily Alt Gr + Shift + è}{$\text{ }$}\setmainfont[Path=/usr/share/fonts/truetype/cmu/,UprightFont=cmunrm.ttf,BoldFont=cmunbx.ttf,ItalicFont=cmunti.ttf,BoldItalicFont=cmunbi.ttf]{cmunrm.ttf}\setmonofont[Path=/usr/share/fonts/truetype/cmu/,UprightFont=cmuntt.ttf,BoldFont=cmuntb.ttf,ItalicFont=cmunit.ttf,BoldItalicFont=cmuntx.ttf]{cmunrm.ttf} &\hspace*{0pt}\ignorespaces{}\hspace*{0pt} È\\ \hline \hspace*{0pt}\ignorespaces{}\hspace*{0pt}{\ttfamily \setmainfont[Path=/usr/share/fonts/truetype/cmu/,UprightFont=cmunrm.ttf,BoldFont=cmunbx.ttf,ItalicFont=cmunti.ttf,BoldItalicFont=cmunbi.ttf]{cmuntt.ttf}\setmonofont[Path=/usr/share/fonts/truetype/cmu/,UprightFont=cmuntt.ttf,BoldFont=cmuntb.ttf,ItalicFont=cmunit.ttf,BoldItalicFont=cmuntx.ttf]{cmuntt.ttf}\ttfamily Alt Gr + Shift + ç}{$\text{ }$}\setmainfont[Path=/usr/share/fonts/truetype/cmu/,UprightFont=cmunrm.ttf,BoldFont=cmunbx.ttf,ItalicFont=cmunti.ttf,BoldItalicFont=cmunbi.ttf]{cmunrm.ttf}\setmonofont[Path=/usr/share/fonts/truetype/cmu/,UprightFont=cmuntt.ttf,BoldFont=cmuntb.ttf,ItalicFont=cmunit.ttf,BoldItalicFont=cmuntx.ttf]{cmunrm.ttf} &\hspace*{0pt}\ignorespaces{}\hspace*{0pt} Ç\\ \hline 
\end{longtable}


You will need the T1 font encoding for guillemets to print properly.

For the degree character you will get an error like\\

\TemplateSpaceIndent{$\text{ }${}!$\text{ }${}Package$\text{ }${}inputenc$\text{ }${}Error:$\text{ }${}Unicode$\text{ }${}char$\text{ }${}\textbackslash{}u8:°$\text{ }${}not$\text{ }${}set$\text{ }${}up$\text{ }${}for$\text{ }${}use$\text{ }${}with$\text{ }${}LaTeX.}


The \LaTeXTT{textcomp} package will fix it for you.

The great advantage of Babel for French is that it will handle some elements of French typography for you, especially non-{}breaking spaces before all two-{}parts punctuation marks. So now you can write:

\begin{Shaded}
\begin{Highlighting}[]

\NormalTok{Il répondit: «Ce pain coûte-t-il 2~€?»}
\end{Highlighting}
\end{Shaded}


The non-{}breaking space before the euro symbol is still necessary because currency symbols and other units or not supported in general (that\textquotesingle{}s not specific to French).

You can use the \LaTeXTT{numprint} package along Babel. It will let you print numbers the French way.
\begin{longtable}{p{1.0\linewidth}}
\begin{Shaded}
\begin{Highlighting}[]

\NormalTok{\textbackslash{}usepackage[frenchb]\{babel\}}
\NormalTok{\textbackslash{}usepackage[autolanguage]\{numprint\} }\CommentTok{% Must be loaded *after* babel.}
 
\CommentTok{% ...}
 
\NormalTok{\textbackslash{}nombre\{123456.123456 e-17\}}
\end{Highlighting}
\end{Shaded}
\\

{$123~456,123~456\cdot 10^{-17}$}

\end{longtable}


You will also notice that the layout of lists changes when switching to the French language. This is customizable using the \LaTeXTT{\textbackslash{}frenchbsetup} command. For more information on what the \LaTeXTT{frenchb} option of \LaTeXTT{babel} does and how you can customize its behavior, run LaTeX on file {\ttfamily \setmainfont[Path=/usr/share/fonts/truetype/cmu/,UprightFont=cmunrm.ttf,BoldFont=cmunbx.ttf,ItalicFont=cmunti.ttf,BoldItalicFont=cmunbi.ttf]{cmuntt.ttf}\setmonofont[Path=/usr/share/fonts/truetype/cmu/,UprightFont=cmuntt.ttf,BoldFont=cmuntb.ttf,ItalicFont=cmunit.ttf,BoldItalicFont=cmuntx.ttf]{cmuntt.ttf}\ttfamily frenchb.dtx}{$\text{ }$}\setmainfont[Path=/usr/share/fonts/truetype/cmu/,UprightFont=cmunrm.ttf,BoldFont=cmunbx.ttf,ItalicFont=cmunti.ttf,BoldItalicFont=cmunbi.ttf]{cmunrm.ttf}\setmonofont[Path=/usr/share/fonts/truetype/cmu/,UprightFont=cmuntt.ttf,BoldFont=cmuntb.ttf,ItalicFont=cmunit.ttf,BoldItalicFont=cmuntx.ttf]{cmunrm.ttf} and read the produced file {\ttfamily \setmainfont[Path=/usr/share/fonts/truetype/cmu/,UprightFont=cmunrm.ttf,BoldFont=cmunbx.ttf,ItalicFont=cmunti.ttf,BoldItalicFont=cmunbi.ttf]{cmuntt.ttf}\setmonofont[Path=/usr/share/fonts/truetype/cmu/,UprightFont=cmuntt.ttf,BoldFont=cmuntb.ttf,ItalicFont=cmunit.ttf,BoldItalicFont=cmuntx.ttf]{cmuntt.ttf}\ttfamily frenchb.pdf}{$\text{ }$}\setmainfont[Path=/usr/share/fonts/truetype/cmu/,UprightFont=cmunrm.ttf,BoldFont=cmunbx.ttf,ItalicFont=cmunti.ttf,BoldItalicFont=cmunbi.ttf]{cmunrm.ttf}\setmonofont[Path=/usr/share/fonts/truetype/cmu/,UprightFont=cmuntt.ttf,BoldFont=cmuntb.ttf,ItalicFont=cmunit.ttf,BoldItalicFont=cmuntx.ttf]{cmunrm.ttf} or {\ttfamily \setmainfont[Path=/usr/share/fonts/truetype/cmu/,UprightFont=cmunrm.ttf,BoldFont=cmunbx.ttf,ItalicFont=cmunti.ttf,BoldItalicFont=cmunbi.ttf]{cmuntt.ttf}\setmonofont[Path=/usr/share/fonts/truetype/cmu/,UprightFont=cmuntt.ttf,BoldFont=cmuntb.ttf,ItalicFont=cmunit.ttf,BoldItalicFont=cmuntx.ttf]{cmuntt.ttf}\ttfamily frenchb.dvi}\setmainfont[Path=/usr/share/fonts/truetype/cmu/,UprightFont=cmunrm.ttf,BoldFont=cmunbx.ttf,ItalicFont=cmunti.ttf,BoldItalicFont=cmunbi.ttf]{cmunrm.ttf}\setmonofont[Path=/usr/share/fonts/truetype/cmu/,UprightFont=cmuntt.ttf,BoldFont=cmuntb.ttf,ItalicFont=cmunit.ttf,BoldItalicFont=cmuntx.ttf]{cmunrm.ttf}. You can get the PDF version on \myhref{http://mirrors.ctan.org/macros/latex/contrib/babel-contrib/frenchb/frenchb.pdf}{CTAN}.
\subsection{German}
\label{222}

You can load German language support using {\itshape \setmainfont[Path=/usr/share/fonts/truetype/cmu/,UprightFont=cmunrm.ttf,BoldFont=cmunbx.ttf,ItalicFont=cmunti.ttf,BoldItalicFont=cmunbi.ttf]{cmunti.ttf}\setmonofont[Path=/usr/share/fonts/truetype/cmu/,UprightFont=cmuntt.ttf,BoldFont=cmuntb.ttf,ItalicFont=cmunit.ttf,BoldItalicFont=cmuntx.ttf]{cmunti.ttf}\itshape either one}{$\text{ }$}\setmainfont[Path=/usr/share/fonts/truetype/cmu/,UprightFont=cmunrm.ttf,BoldFont=cmunbx.ttf,ItalicFont=cmunti.ttf,BoldItalicFont=cmunbi.ttf]{cmunrm.ttf}\setmonofont[Path=/usr/share/fonts/truetype/cmu/,UprightFont=cmuntt.ttf,BoldFont=cmuntb.ttf,ItalicFont=cmunit.ttf,BoldItalicFont=cmuntx.ttf]{cmunrm.ttf} of the two following commands.

For traditional (\symbol{34}old\symbol{34}) German orthography use
\begin{Shaded}
\begin{Highlighting}[]

\NormalTok{\textbackslash{}usepackage[german]\{babel\}}
\end{Highlighting}
\end{Shaded}


{\bfseries \setmainfont[Path=/usr/share/fonts/truetype/cmu/,UprightFont=cmunrm.ttf,BoldFont=cmunbx.ttf,ItalicFont=cmunti.ttf,BoldItalicFont=cmunbi.ttf]{cmunbx.ttf}\setmonofont[Path=/usr/share/fonts/truetype/cmu/,UprightFont=cmuntt.ttf,BoldFont=cmuntb.ttf,ItalicFont=cmunit.ttf,BoldItalicFont=cmuntx.ttf]{cmunbx.ttf}\bfseries or}{$\text{ }$}\setmainfont[Path=/usr/share/fonts/truetype/cmu/,UprightFont=cmunrm.ttf,BoldFont=cmunbx.ttf,ItalicFont=cmunti.ttf,BoldItalicFont=cmunbi.ttf]{cmunrm.ttf}\setmonofont[Path=/usr/share/fonts/truetype/cmu/,UprightFont=cmuntt.ttf,BoldFont=cmuntb.ttf,ItalicFont=cmunit.ttf,BoldItalicFont=cmuntx.ttf]{cmunrm.ttf} for reform (\symbol{34}new\symbol{34}) German orthography use

\begin{Shaded}
\begin{Highlighting}[]

\NormalTok{\textbackslash{}usepackage[ngerman]\{babel\}}
\end{Highlighting}
\end{Shaded}


This enables German hyphenation, if you have configured your LaTeX
system accordingly. It also changes all automatic text into German, e.g.
“Chapter” becomes “Kapitel”. A set of new commands also becomes available,
which allows you to write German input files more quickly even when
you don\textquotesingle{}t use the inputenc package. Check out the table below for inspiration.
With inputenc, all this becomes moot, but your text also is locked in a
particular encoding world.

\begin{longtable}{|>{\RaggedRight}p{0.61613\linewidth}|>{\RaggedRight}p{0.32673\linewidth}|} \hline 
\multicolumn{2}{|>{\RaggedRight}p{0.97143\linewidth}|}{{\bfseries \hspace*{0pt}\ignorespaces{}\hspace*{0pt}German Special Characters.}}\endhead  \hline \hspace*{0pt}\ignorespaces{}\hspace*{0pt}{\ttfamily \setmainfont[Path=/usr/share/fonts/truetype/cmu/,UprightFont=cmunrm.ttf,BoldFont=cmunbx.ttf,ItalicFont=cmunti.ttf,BoldItalicFont=cmunbi.ttf]{cmuntt.ttf}\setmonofont[Path=/usr/share/fonts/truetype/cmu/,UprightFont=cmuntt.ttf,BoldFont=cmuntb.ttf,ItalicFont=cmunit.ttf,BoldItalicFont=cmuntx.ttf]{cmuntt.ttf}\ttfamily \symbol{34}A \symbol{34}O \symbol{34}U}{$\text{ }$}\setmainfont[Path=/usr/share/fonts/truetype/cmu/,UprightFont=cmunrm.ttf,BoldFont=cmunbx.ttf,ItalicFont=cmunti.ttf,BoldItalicFont=cmunbi.ttf]{cmunrm.ttf}\setmonofont[Path=/usr/share/fonts/truetype/cmu/,UprightFont=cmuntt.ttf,BoldFont=cmuntb.ttf,ItalicFont=cmunit.ttf,BoldItalicFont=cmuntx.ttf]{cmunrm.ttf} &\hspace*{0pt}\ignorespaces{}\hspace*{0pt} Ä Ö Ü\\ \hline \hspace*{0pt}\ignorespaces{}\hspace*{0pt}{\ttfamily \setmainfont[Path=/usr/share/fonts/truetype/cmu/,UprightFont=cmunrm.ttf,BoldFont=cmunbx.ttf,ItalicFont=cmunti.ttf,BoldItalicFont=cmunbi.ttf]{cmuntt.ttf}\setmonofont[Path=/usr/share/fonts/truetype/cmu/,UprightFont=cmuntt.ttf,BoldFont=cmuntb.ttf,ItalicFont=cmunit.ttf,BoldItalicFont=cmuntx.ttf]{cmuntt.ttf}\ttfamily \symbol{34}a \symbol{34}o \symbol{34}u \symbol{34}s}{$\text{ }$}\setmainfont[Path=/usr/share/fonts/truetype/cmu/,UprightFont=cmunrm.ttf,BoldFont=cmunbx.ttf,ItalicFont=cmunti.ttf,BoldItalicFont=cmunbi.ttf]{cmunrm.ttf}\setmonofont[Path=/usr/share/fonts/truetype/cmu/,UprightFont=cmuntt.ttf,BoldFont=cmuntb.ttf,ItalicFont=cmunit.ttf,BoldItalicFont=cmuntx.ttf]{cmunrm.ttf} &\hspace*{0pt}\ignorespaces{}\hspace*{0pt} ä ö ü ß\\ \hline \hspace*{0pt}\ignorespaces{}\hspace*{0pt}{\ttfamily \setmainfont[Path=/usr/share/fonts/truetype/cmu/,UprightFont=cmunrm.ttf,BoldFont=cmunbx.ttf,ItalicFont=cmunti.ttf,BoldItalicFont=cmunbi.ttf]{cmuntt.ttf}\setmonofont[Path=/usr/share/fonts/truetype/cmu/,UprightFont=cmuntt.ttf,BoldFont=cmuntb.ttf,ItalicFont=cmunit.ttf,BoldItalicFont=cmuntx.ttf]{cmuntt.ttf}\ttfamily \symbol{34}`}{$\text{ }$}\setmainfont[Path=/usr/share/fonts/truetype/cmu/,UprightFont=cmunrm.ttf,BoldFont=cmunbx.ttf,ItalicFont=cmunti.ttf,BoldItalicFont=cmunbi.ttf]{cmunrm.ttf}\setmonofont[Path=/usr/share/fonts/truetype/cmu/,UprightFont=cmuntt.ttf,BoldFont=cmuntb.ttf,ItalicFont=cmunit.ttf,BoldItalicFont=cmuntx.ttf]{cmunrm.ttf} or {\ttfamily \setmainfont[Path=/usr/share/fonts/truetype/cmu/,UprightFont=cmunrm.ttf,BoldFont=cmunbx.ttf,ItalicFont=cmunti.ttf,BoldItalicFont=cmunbi.ttf]{cmuntt.ttf}\setmonofont[Path=/usr/share/fonts/truetype/cmu/,UprightFont=cmuntt.ttf,BoldFont=cmuntb.ttf,ItalicFont=cmunit.ttf,BoldItalicFont=cmuntx.ttf]{cmuntt.ttf}\ttfamily \textbackslash{}glqq}{$\text{ }$}\setmainfont[Path=/usr/share/fonts/truetype/cmu/,UprightFont=cmunrm.ttf,BoldFont=cmunbx.ttf,ItalicFont=cmunti.ttf,BoldItalicFont=cmunbi.ttf]{cmunrm.ttf}\setmonofont[Path=/usr/share/fonts/truetype/cmu/,UprightFont=cmuntt.ttf,BoldFont=cmuntb.ttf,ItalicFont=cmunit.ttf,BoldItalicFont=cmuntx.ttf]{cmunrm.ttf} &\hspace*{0pt}\ignorespaces{}\hspace*{0pt} „ \\ \hline \hspace*{0pt}\ignorespaces{}\hspace*{0pt}{\ttfamily \setmainfont[Path=/usr/share/fonts/truetype/cmu/,UprightFont=cmunrm.ttf,BoldFont=cmunbx.ttf,ItalicFont=cmunti.ttf,BoldItalicFont=cmunbi.ttf]{cmuntt.ttf}\setmonofont[Path=/usr/share/fonts/truetype/cmu/,UprightFont=cmuntt.ttf,BoldFont=cmuntb.ttf,ItalicFont=cmunit.ttf,BoldItalicFont=cmuntx.ttf]{cmuntt.ttf}\ttfamily \symbol{34}\textquotesingle{}}{$\text{ }$}\setmainfont[Path=/usr/share/fonts/truetype/cmu/,UprightFont=cmunrm.ttf,BoldFont=cmunbx.ttf,ItalicFont=cmunti.ttf,BoldItalicFont=cmunbi.ttf]{cmunrm.ttf}\setmonofont[Path=/usr/share/fonts/truetype/cmu/,UprightFont=cmuntt.ttf,BoldFont=cmuntb.ttf,ItalicFont=cmunit.ttf,BoldItalicFont=cmuntx.ttf]{cmunrm.ttf} or {\ttfamily \setmainfont[Path=/usr/share/fonts/truetype/cmu/,UprightFont=cmunrm.ttf,BoldFont=cmunbx.ttf,ItalicFont=cmunti.ttf,BoldItalicFont=cmunbi.ttf]{cmuntt.ttf}\setmonofont[Path=/usr/share/fonts/truetype/cmu/,UprightFont=cmuntt.ttf,BoldFont=cmuntb.ttf,ItalicFont=cmunit.ttf,BoldItalicFont=cmuntx.ttf]{cmuntt.ttf}\ttfamily \textbackslash{}grqq}{$\text{ }$}\setmainfont[Path=/usr/share/fonts/truetype/cmu/,UprightFont=cmunrm.ttf,BoldFont=cmunbx.ttf,ItalicFont=cmunti.ttf,BoldItalicFont=cmunbi.ttf]{cmunrm.ttf}\setmonofont[Path=/usr/share/fonts/truetype/cmu/,UprightFont=cmuntt.ttf,BoldFont=cmuntb.ttf,ItalicFont=cmunit.ttf,BoldItalicFont=cmuntx.ttf]{cmunrm.ttf} &\hspace*{0pt}\ignorespaces{}\hspace*{0pt} “\\ \hline \hspace*{0pt}\ignorespaces{}\hspace*{0pt}{\ttfamily \setmainfont[Path=/usr/share/fonts/truetype/cmu/,UprightFont=cmunrm.ttf,BoldFont=cmunbx.ttf,ItalicFont=cmunti.ttf,BoldItalicFont=cmunbi.ttf]{cmuntt.ttf}\setmonofont[Path=/usr/share/fonts/truetype/cmu/,UprightFont=cmuntt.ttf,BoldFont=cmuntb.ttf,ItalicFont=cmunit.ttf,BoldItalicFont=cmuntx.ttf]{cmuntt.ttf}\ttfamily \textbackslash{}glq \textbackslash{}grq}{$\text{ }$}\setmainfont[Path=/usr/share/fonts/truetype/cmu/,UprightFont=cmunrm.ttf,BoldFont=cmunbx.ttf,ItalicFont=cmunti.ttf,BoldItalicFont=cmunbi.ttf]{cmunrm.ttf}\setmonofont[Path=/usr/share/fonts/truetype/cmu/,UprightFont=cmuntt.ttf,BoldFont=cmuntb.ttf,ItalicFont=cmunit.ttf,BoldItalicFont=cmuntx.ttf]{cmunrm.ttf} &\hspace*{0pt}\ignorespaces{}\hspace*{0pt} \\ \hline \hspace*{0pt}\ignorespaces{}\hspace*{0pt}{\ttfamily \setmainfont[Path=/usr/share/fonts/truetype/cmu/,UprightFont=cmunrm.ttf,BoldFont=cmunbx.ttf,ItalicFont=cmunti.ttf,BoldItalicFont=cmunbi.ttf]{cmuntt.ttf}\setmonofont[Path=/usr/share/fonts/truetype/cmu/,UprightFont=cmuntt.ttf,BoldFont=cmuntb.ttf,ItalicFont=cmunit.ttf,BoldItalicFont=cmuntx.ttf]{cmuntt.ttf}\ttfamily \symbol{34}<{}}{$\text{ }$}\setmainfont[Path=/usr/share/fonts/truetype/cmu/,UprightFont=cmunrm.ttf,BoldFont=cmunbx.ttf,ItalicFont=cmunti.ttf,BoldItalicFont=cmunbi.ttf]{cmunrm.ttf}\setmonofont[Path=/usr/share/fonts/truetype/cmu/,UprightFont=cmuntt.ttf,BoldFont=cmuntb.ttf,ItalicFont=cmunit.ttf,BoldItalicFont=cmuntx.ttf]{cmunrm.ttf} or {\ttfamily \setmainfont[Path=/usr/share/fonts/truetype/cmu/,UprightFont=cmunrm.ttf,BoldFont=cmunbx.ttf,ItalicFont=cmunti.ttf,BoldItalicFont=cmunbi.ttf]{cmuntt.ttf}\setmonofont[Path=/usr/share/fonts/truetype/cmu/,UprightFont=cmuntt.ttf,BoldFont=cmuntb.ttf,ItalicFont=cmunit.ttf,BoldItalicFont=cmuntx.ttf]{cmuntt.ttf}\ttfamily \textbackslash{}flqq}{$\text{ }$}\setmainfont[Path=/usr/share/fonts/truetype/cmu/,UprightFont=cmunrm.ttf,BoldFont=cmunbx.ttf,ItalicFont=cmunti.ttf,BoldItalicFont=cmunbi.ttf]{cmunrm.ttf}\setmonofont[Path=/usr/share/fonts/truetype/cmu/,UprightFont=cmuntt.ttf,BoldFont=cmuntb.ttf,ItalicFont=cmunit.ttf,BoldItalicFont=cmuntx.ttf]{cmunrm.ttf} &\hspace*{0pt}\ignorespaces{}\hspace*{0pt} «\\ \hline \hspace*{0pt}\ignorespaces{}\hspace*{0pt}{\ttfamily \setmainfont[Path=/usr/share/fonts/truetype/cmu/,UprightFont=cmunrm.ttf,BoldFont=cmunbx.ttf,ItalicFont=cmunti.ttf,BoldItalicFont=cmunbi.ttf]{cmuntt.ttf}\setmonofont[Path=/usr/share/fonts/truetype/cmu/,UprightFont=cmuntt.ttf,BoldFont=cmuntb.ttf,ItalicFont=cmunit.ttf,BoldItalicFont=cmuntx.ttf]{cmuntt.ttf}\ttfamily \symbol{34}>{}}{$\text{ }$}\setmainfont[Path=/usr/share/fonts/truetype/cmu/,UprightFont=cmunrm.ttf,BoldFont=cmunbx.ttf,ItalicFont=cmunti.ttf,BoldItalicFont=cmunbi.ttf]{cmunrm.ttf}\setmonofont[Path=/usr/share/fonts/truetype/cmu/,UprightFont=cmuntt.ttf,BoldFont=cmuntb.ttf,ItalicFont=cmunit.ttf,BoldItalicFont=cmuntx.ttf]{cmunrm.ttf} or {\ttfamily \setmainfont[Path=/usr/share/fonts/truetype/cmu/,UprightFont=cmunrm.ttf,BoldFont=cmunbx.ttf,ItalicFont=cmunti.ttf,BoldItalicFont=cmunbi.ttf]{cmuntt.ttf}\setmonofont[Path=/usr/share/fonts/truetype/cmu/,UprightFont=cmuntt.ttf,BoldFont=cmuntb.ttf,ItalicFont=cmunit.ttf,BoldItalicFont=cmuntx.ttf]{cmuntt.ttf}\ttfamily \textbackslash{}frqq}{$\text{ }$}\setmainfont[Path=/usr/share/fonts/truetype/cmu/,UprightFont=cmunrm.ttf,BoldFont=cmunbx.ttf,ItalicFont=cmunti.ttf,BoldItalicFont=cmunbi.ttf]{cmunrm.ttf}\setmonofont[Path=/usr/share/fonts/truetype/cmu/,UprightFont=cmuntt.ttf,BoldFont=cmuntb.ttf,ItalicFont=cmunit.ttf,BoldItalicFont=cmuntx.ttf]{cmunrm.ttf} &\hspace*{0pt}\ignorespaces{}\hspace*{0pt} »\\ \hline \hspace*{0pt}\ignorespaces{}\hspace*{0pt}{\ttfamily \setmainfont[Path=/usr/share/fonts/truetype/cmu/,UprightFont=cmunrm.ttf,BoldFont=cmunbx.ttf,ItalicFont=cmunti.ttf,BoldItalicFont=cmunbi.ttf]{cmuntt.ttf}\setmonofont[Path=/usr/share/fonts/truetype/cmu/,UprightFont=cmuntt.ttf,BoldFont=cmuntb.ttf,ItalicFont=cmunit.ttf,BoldItalicFont=cmuntx.ttf]{cmuntt.ttf}\ttfamily \textbackslash{}flq \textbackslash{}frq}{$\text{ }$}\setmainfont[Path=/usr/share/fonts/truetype/cmu/,UprightFont=cmunrm.ttf,BoldFont=cmunbx.ttf,ItalicFont=cmunti.ttf,BoldItalicFont=cmunbi.ttf]{cmunrm.ttf}\setmonofont[Path=/usr/share/fonts/truetype/cmu/,UprightFont=cmuntt.ttf,BoldFont=cmuntb.ttf,ItalicFont=cmunit.ttf,BoldItalicFont=cmuntx.ttf]{cmunrm.ttf} &\hspace*{0pt}\ignorespaces{}\hspace*{0pt} ‹ ›\\ \hline \hspace*{0pt}\ignorespaces{}\hspace*{0pt}{\ttfamily \setmainfont[Path=/usr/share/fonts/truetype/cmu/,UprightFont=cmunrm.ttf,BoldFont=cmunbx.ttf,ItalicFont=cmunti.ttf,BoldItalicFont=cmunbi.ttf]{cmuntt.ttf}\setmonofont[Path=/usr/share/fonts/truetype/cmu/,UprightFont=cmuntt.ttf,BoldFont=cmuntb.ttf,ItalicFont=cmunit.ttf,BoldItalicFont=cmuntx.ttf]{cmuntt.ttf}\ttfamily \textbackslash{}dq}{$\text{ }$}\setmainfont[Path=/usr/share/fonts/truetype/cmu/,UprightFont=cmunrm.ttf,BoldFont=cmunbx.ttf,ItalicFont=cmunti.ttf,BoldItalicFont=cmunbi.ttf]{cmunrm.ttf}\setmonofont[Path=/usr/share/fonts/truetype/cmu/,UprightFont=cmuntt.ttf,BoldFont=cmuntb.ttf,ItalicFont=cmunit.ttf,BoldItalicFont=cmuntx.ttf]{cmunrm.ttf} &\hspace*{0pt}\ignorespaces{}\hspace*{0pt} \symbol{34}\\ \hline 
\end{longtable}


In German books you sometimes find French quotation marks («guillemets»). German typesetters, however, use them differently. A quote in a German book would look like »this«. In the German speaking part of Switzerland, typesetters use «guillemets» the same way the French do. A major problem arises from the use of commands like \LaTeXTT{\textbackslash{}flq}: If you use the OT1 font encoding (which is the default) the guillemets will look like the math symbol \symbol{34}{$\ll$}\symbol{34}, which turns a typesetter\textquotesingle{}s stomach. T1 encoded fonts, on the other hand, do contain the required symbols. So if you are using this type of quote, make sure you use the T1 encoding.

Decimal numbers usually have to be written like {\ttfamily \setmainfont[Path=/usr/share/fonts/truetype/cmu/,UprightFont=cmunrm.ttf,BoldFont=cmunbx.ttf,ItalicFont=cmunti.ttf,BoldItalicFont=cmunbi.ttf]{cmuntt.ttf}\setmonofont[Path=/usr/share/fonts/truetype/cmu/,UprightFont=cmuntt.ttf,BoldFont=cmuntb.ttf,ItalicFont=cmunit.ttf,BoldItalicFont=cmuntx.ttf]{cmuntt.ttf}\ttfamily 0\{,\}5}{$\text{ }$}\setmainfont[Path=/usr/share/fonts/truetype/cmu/,UprightFont=cmunrm.ttf,BoldFont=cmunbx.ttf,ItalicFont=cmunti.ttf,BoldItalicFont=cmunbi.ttf]{cmunrm.ttf}\setmonofont[Path=/usr/share/fonts/truetype/cmu/,UprightFont=cmuntt.ttf,BoldFont=cmuntb.ttf,ItalicFont=cmunit.ttf,BoldItalicFont=cmuntx.ttf]{cmunrm.ttf} (not just 0,5). Packages like {\ttfamily \setmainfont[Path=/usr/share/fonts/truetype/cmu/,UprightFont=cmunrm.ttf,BoldFont=cmunbx.ttf,ItalicFont=cmunti.ttf,BoldItalicFont=cmunbi.ttf]{cmuntt.ttf}\setmonofont[Path=/usr/share/fonts/truetype/cmu/,UprightFont=cmuntt.ttf,BoldFont=cmuntb.ttf,ItalicFont=cmunit.ttf,BoldItalicFont=cmuntx.ttf]{cmuntt.ttf}\ttfamily ziffer}{$\text{ }$}\setmainfont[Path=/usr/share/fonts/truetype/cmu/,UprightFont=cmunrm.ttf,BoldFont=cmunbx.ttf,ItalicFont=cmunti.ttf,BoldItalicFont=cmunbi.ttf]{cmunrm.ttf}\setmonofont[Path=/usr/share/fonts/truetype/cmu/,UprightFont=cmuntt.ttf,BoldFont=cmuntb.ttf,ItalicFont=cmunit.ttf,BoldItalicFont=cmuntx.ttf]{cmunrm.ttf} enable input like {\ttfamily \setmainfont[Path=/usr/share/fonts/truetype/cmu/,UprightFont=cmunrm.ttf,BoldFont=cmunbx.ttf,ItalicFont=cmunti.ttf,BoldItalicFont=cmunbi.ttf]{cmuntt.ttf}\setmonofont[Path=/usr/share/fonts/truetype/cmu/,UprightFont=cmuntt.ttf,BoldFont=cmuntb.ttf,ItalicFont=cmunit.ttf,BoldItalicFont=cmuntx.ttf]{cmuntt.ttf}\ttfamily 0,5}\setmainfont[Path=/usr/share/fonts/truetype/cmu/,UprightFont=cmunrm.ttf,BoldFont=cmunbx.ttf,ItalicFont=cmunti.ttf,BoldItalicFont=cmunbi.ttf]{cmunrm.ttf}\setmonofont[Path=/usr/share/fonts/truetype/cmu/,UprightFont=cmuntt.ttf,BoldFont=cmuntb.ttf,ItalicFont=cmunit.ttf,BoldItalicFont=cmuntx.ttf]{cmunrm.ttf}. Alternatively, one can use the \LaTeXTT{\textbackslash{}num} command from the \LaTeXTT{babel} and (globally) set the decimal marker using
\begin{longtable}{p{1.0\linewidth}}
\begin{Shaded}
\begin{Highlighting}[]

\NormalTok{\textbackslash{}usepackage[output-decimal-marker=\{,\}]\{siunitx\}}
\CommentTok{% ...}
\NormalTok{\textbackslash{}num\{0,5\}}
\end{Highlighting}
\end{Shaded}
\\

{$0{,}5$}

\end{longtable}

\subsection{Greek}
\label{223}

This is the preamble you need to write in the Greek language. Note the particular input encoding.

\begin{Shaded}
\begin{Highlighting}[]

\NormalTok{\textbackslash{}usepackage[english,greek]\{babel\}}
\NormalTok{\textbackslash{}usepackage[iso-8859-7]\{inputenc\}}
\end{Highlighting}
\end{Shaded}


This preamble enables hyphenation and changes all automatic text to Greek. A set of new commands also becomes available, which allows you to write Greek input files more easily. In order to temporarily switch to English and vice versa, one can use the commands \LaTeXTT{\textbackslash{}textlatin\{english text\}} and \LaTeXTT{\textbackslash{}textgreek\{greek text\}} that both take one argument which is then typeset using the requested font encoding. Otherwise you can use the command \LaTeXTT{\textbackslash{}selectlanguage\{...\}} described in a previous section. Use \LaTeXTT{\textbackslash{}euro} for the Euro symbol.
\subsection{Hungarian}
\label{224}

Use the following lines:
\begin{Shaded}
\begin{Highlighting}[]

\NormalTok{\textbackslash{}usepackage[magyar]\{babel\}}
\end{Highlighting}
\end{Shaded}


More information \myhref{http://www.math.bme.hu/latex/}{in hungarian}.
\subsection{Icelandic and Faroese}
\label{225}

The following lines can be added to write Icelandic text:

\begin{Shaded}
\begin{Highlighting}[]

\NormalTok{\textbackslash{}usepackage[icelandic]\{babel\}}
\end{Highlighting}
\end{Shaded}


This changes text like {\itshape \setmainfont[Path=/usr/share/fonts/truetype/cmu/,UprightFont=cmunrm.ttf,BoldFont=cmunbx.ttf,ItalicFont=cmunti.ttf,BoldItalicFont=cmunbi.ttf]{cmunti.ttf}\setmonofont[Path=/usr/share/fonts/truetype/cmu/,UprightFont=cmuntt.ttf,BoldFont=cmuntb.ttf,ItalicFont=cmunit.ttf,BoldItalicFont=cmuntx.ttf]{cmunti.ttf}\itshape Part}{$\text{ }$}\setmainfont[Path=/usr/share/fonts/truetype/cmu/,UprightFont=cmunrm.ttf,BoldFont=cmunbx.ttf,ItalicFont=cmunti.ttf,BoldItalicFont=cmunbi.ttf]{cmunrm.ttf}\setmonofont[Path=/usr/share/fonts/truetype/cmu/,UprightFont=cmuntt.ttf,BoldFont=cmuntb.ttf,ItalicFont=cmunit.ttf,BoldItalicFont=cmuntx.ttf]{cmunrm.ttf} into {\itshape \setmainfont[Path=/usr/share/fonts/truetype/cmu/,UprightFont=cmunrm.ttf,BoldFont=cmunbx.ttf,ItalicFont=cmunti.ttf,BoldItalicFont=cmunbi.ttf]{cmunti.ttf}\setmonofont[Path=/usr/share/fonts/truetype/cmu/,UprightFont=cmuntt.ttf,BoldFont=cmuntb.ttf,ItalicFont=cmunit.ttf,BoldItalicFont=cmuntx.ttf]{cmunti.ttf}\itshape Hluti}\setmainfont[Path=/usr/share/fonts/truetype/cmu/,UprightFont=cmunrm.ttf,BoldFont=cmunbx.ttf,ItalicFont=cmunti.ttf,BoldItalicFont=cmunbi.ttf]{cmunrm.ttf}\setmonofont[Path=/usr/share/fonts/truetype/cmu/,UprightFont=cmuntt.ttf,BoldFont=cmuntb.ttf,ItalicFont=cmunit.ttf,BoldItalicFont=cmuntx.ttf]{cmunrm.ttf}. It makes additional commands available:

\begin{longtable}{|>{\RaggedRight}p{0.73704\linewidth}|>{\RaggedRight}p{0.20582\linewidth}|} \hline 
\multicolumn{2}{|>{\RaggedRight}p{0.97143\linewidth}|}{{\bfseries \hspace*{0pt}\ignorespaces{}\hspace*{0pt}Icelandic special commands}}\endhead  \hline \hspace*{0pt}\ignorespaces{}\hspace*{0pt}{\ttfamily \setmainfont[Path=/usr/share/fonts/truetype/cmu/,UprightFont=cmunrm.ttf,BoldFont=cmunbx.ttf,ItalicFont=cmunti.ttf,BoldItalicFont=cmunbi.ttf]{cmuntt.ttf}\setmonofont[Path=/usr/share/fonts/truetype/cmu/,UprightFont=cmuntt.ttf,BoldFont=cmuntb.ttf,ItalicFont=cmunit.ttf,BoldItalicFont=cmuntx.ttf]{cmuntt.ttf}\ttfamily \symbol{34}`}{$\text{ }$}\setmainfont[Path=/usr/share/fonts/truetype/cmu/,UprightFont=cmunrm.ttf,BoldFont=cmunbx.ttf,ItalicFont=cmunti.ttf,BoldItalicFont=cmunbi.ttf]{cmunrm.ttf}\setmonofont[Path=/usr/share/fonts/truetype/cmu/,UprightFont=cmuntt.ttf,BoldFont=cmuntb.ttf,ItalicFont=cmunit.ttf,BoldItalicFont=cmuntx.ttf]{cmunrm.ttf} or {\ttfamily \setmainfont[Path=/usr/share/fonts/truetype/cmu/,UprightFont=cmunrm.ttf,BoldFont=cmunbx.ttf,ItalicFont=cmunti.ttf,BoldItalicFont=cmunbi.ttf]{cmuntt.ttf}\setmonofont[Path=/usr/share/fonts/truetype/cmu/,UprightFont=cmuntt.ttf,BoldFont=cmuntb.ttf,ItalicFont=cmunit.ttf,BoldItalicFont=cmuntx.ttf]{cmuntt.ttf}\ttfamily \textbackslash{}glqq}{$\text{ }$}\setmainfont[Path=/usr/share/fonts/truetype/cmu/,UprightFont=cmunrm.ttf,BoldFont=cmunbx.ttf,ItalicFont=cmunti.ttf,BoldItalicFont=cmunbi.ttf]{cmunrm.ttf}\setmonofont[Path=/usr/share/fonts/truetype/cmu/,UprightFont=cmuntt.ttf,BoldFont=cmuntb.ttf,ItalicFont=cmunit.ttf,BoldItalicFont=cmuntx.ttf]{cmunrm.ttf} &\hspace*{0pt}\ignorespaces{}\hspace*{0pt} „ \\ \hline \hspace*{0pt}\ignorespaces{}\hspace*{0pt}{\ttfamily \setmainfont[Path=/usr/share/fonts/truetype/cmu/,UprightFont=cmunrm.ttf,BoldFont=cmunbx.ttf,ItalicFont=cmunti.ttf,BoldItalicFont=cmunbi.ttf]{cmuntt.ttf}\setmonofont[Path=/usr/share/fonts/truetype/cmu/,UprightFont=cmuntt.ttf,BoldFont=cmuntb.ttf,ItalicFont=cmunit.ttf,BoldItalicFont=cmuntx.ttf]{cmuntt.ttf}\ttfamily \textbackslash{}grqq}{$\text{ }$}\setmainfont[Path=/usr/share/fonts/truetype/cmu/,UprightFont=cmunrm.ttf,BoldFont=cmunbx.ttf,ItalicFont=cmunti.ttf,BoldItalicFont=cmunbi.ttf]{cmunrm.ttf}\setmonofont[Path=/usr/share/fonts/truetype/cmu/,UprightFont=cmuntt.ttf,BoldFont=cmuntb.ttf,ItalicFont=cmunit.ttf,BoldItalicFont=cmuntx.ttf]{cmunrm.ttf} &\hspace*{0pt}\ignorespaces{}\hspace*{0pt} “\\ \hline \hspace*{0pt}\ignorespaces{}\hspace*{0pt}{\ttfamily \setmainfont[Path=/usr/share/fonts/truetype/cmu/,UprightFont=cmunrm.ttf,BoldFont=cmunbx.ttf,ItalicFont=cmunti.ttf,BoldItalicFont=cmunbi.ttf]{cmuntt.ttf}\setmonofont[Path=/usr/share/fonts/truetype/cmu/,UprightFont=cmuntt.ttf,BoldFont=cmuntb.ttf,ItalicFont=cmunit.ttf,BoldItalicFont=cmuntx.ttf]{cmuntt.ttf}\ttfamily \textbackslash{}TH}{$\text{ }$}\setmainfont[Path=/usr/share/fonts/truetype/cmu/,UprightFont=cmunrm.ttf,BoldFont=cmunbx.ttf,ItalicFont=cmunti.ttf,BoldItalicFont=cmunbi.ttf]{cmunrm.ttf}\setmonofont[Path=/usr/share/fonts/truetype/cmu/,UprightFont=cmuntt.ttf,BoldFont=cmuntb.ttf,ItalicFont=cmunit.ttf,BoldItalicFont=cmuntx.ttf]{cmunrm.ttf} &\hspace*{0pt}\ignorespaces{}\hspace*{0pt} Þ\\ \hline \hspace*{0pt}\ignorespaces{}\hspace*{0pt}{\ttfamily \setmainfont[Path=/usr/share/fonts/truetype/cmu/,UprightFont=cmunrm.ttf,BoldFont=cmunbx.ttf,ItalicFont=cmunti.ttf,BoldItalicFont=cmunbi.ttf]{cmuntt.ttf}\setmonofont[Path=/usr/share/fonts/truetype/cmu/,UprightFont=cmuntt.ttf,BoldFont=cmuntb.ttf,ItalicFont=cmunit.ttf,BoldItalicFont=cmuntx.ttf]{cmuntt.ttf}\ttfamily \textbackslash{}th}{$\text{ }$}\setmainfont[Path=/usr/share/fonts/truetype/cmu/,UprightFont=cmunrm.ttf,BoldFont=cmunbx.ttf,ItalicFont=cmunti.ttf,BoldItalicFont=cmunbi.ttf]{cmunrm.ttf}\setmonofont[Path=/usr/share/fonts/truetype/cmu/,UprightFont=cmuntt.ttf,BoldFont=cmuntb.ttf,ItalicFont=cmunit.ttf,BoldItalicFont=cmuntx.ttf]{cmunrm.ttf} &\hspace*{0pt}\ignorespaces{}\hspace*{0pt} þ\\ \hline \hspace*{0pt}\ignorespaces{}\hspace*{0pt}{\ttfamily \setmainfont[Path=/usr/share/fonts/truetype/cmu/,UprightFont=cmunrm.ttf,BoldFont=cmunbx.ttf,ItalicFont=cmunti.ttf,BoldItalicFont=cmunbi.ttf]{cmuntt.ttf}\setmonofont[Path=/usr/share/fonts/truetype/cmu/,UprightFont=cmuntt.ttf,BoldFont=cmuntb.ttf,ItalicFont=cmunit.ttf,BoldItalicFont=cmuntx.ttf]{cmuntt.ttf}\ttfamily \textbackslash{}DH}{$\text{ }$}\setmainfont[Path=/usr/share/fonts/truetype/cmu/,UprightFont=cmunrm.ttf,BoldFont=cmunbx.ttf,ItalicFont=cmunti.ttf,BoldItalicFont=cmunbi.ttf]{cmunrm.ttf}\setmonofont[Path=/usr/share/fonts/truetype/cmu/,UprightFont=cmuntt.ttf,BoldFont=cmuntb.ttf,ItalicFont=cmunit.ttf,BoldItalicFont=cmuntx.ttf]{cmunrm.ttf} &\hspace*{0pt}\ignorespaces{}\hspace*{0pt} Ð\\ \hline \hspace*{0pt}\ignorespaces{}\hspace*{0pt}{\ttfamily \setmainfont[Path=/usr/share/fonts/truetype/cmu/,UprightFont=cmunrm.ttf,BoldFont=cmunbx.ttf,ItalicFont=cmunti.ttf,BoldItalicFont=cmunbi.ttf]{cmuntt.ttf}\setmonofont[Path=/usr/share/fonts/truetype/cmu/,UprightFont=cmuntt.ttf,BoldFont=cmuntb.ttf,ItalicFont=cmunit.ttf,BoldItalicFont=cmuntx.ttf]{cmuntt.ttf}\ttfamily \textbackslash{}dh}{$\text{ }$}\setmainfont[Path=/usr/share/fonts/truetype/cmu/,UprightFont=cmunrm.ttf,BoldFont=cmunbx.ttf,ItalicFont=cmunti.ttf,BoldItalicFont=cmunbi.ttf]{cmunrm.ttf}\setmonofont[Path=/usr/share/fonts/truetype/cmu/,UprightFont=cmuntt.ttf,BoldFont=cmuntb.ttf,ItalicFont=cmunit.ttf,BoldItalicFont=cmuntx.ttf]{cmunrm.ttf} &\hspace*{0pt}\ignorespaces{}\hspace*{0pt} ð\\ \hline 
\end{longtable}


To make special characters such as Þ and Æ become available just add:

\begin{Shaded}
\begin{Highlighting}[]

\NormalTok{\textbackslash{}usepackage[T1]\{fontenc\}}
\end{Highlighting}
\end{Shaded}


The default LATEX font encoding is OT1, but it contains only the 128 characters. The T1 encoding contains letters and punctuation characters for most of the European languages using Latin script.
\subsection{Italian}
\label{226}

Italian is well supported by LaTeX. Just add 
\begin{Shaded}
\begin{Highlighting}[]

\NormalTok{\textbackslash{}usepackage[italian]\{babel\}}
\end{Highlighting}
\end{Shaded}

at the beginning of your document and the output of all the commands will be translated properly.
\subsection{Japanese}
\label{227}

There is a variant of TeX intended for Japanese named \myhref{http://ascii.asciimw.jp/pb/ptex/}{pTeX}, which supports vertical typesetting.

Another possible way to write in japanese is to use Lualatex and the luatex-{}ja package. Adapted example from the Luatexja documentation :
\begin{Shaded}
\begin{Highlighting}[]

\NormalTok{\textbackslash{}documentclass\{ltjsarticle\}}
\NormalTok{\textbackslash{}usepackage\{luatexja\} }\CommentTok{% This line is unnecessary when using ltjclasses or}
 \NormalTok{ltjsclasses.}
\NormalTok{\textbackslash{}begin\{document\}}
\NormalTok{\textbackslash{}section\{はじめてのLua\textbackslash{}TeX-ja\}}
\NormalTok{ちゃんと日本語が出るかな?}
\NormalTok{\textbackslash{}subsection\{出たかな?\}}
\NormalTok{長い文章を入力するとちゃんと右端のところで折り返されるかな?}
\NormalTok{大丈夫そうな気がするけど.ちょっと不安だけど何事も挑戦だよね.}
\NormalTok{\textbackslash{}end\{document\}}
\end{Highlighting}
\end{Shaded}


You can also use capabilities provided by the Fontspec package and those provided by Luatexja-{}fontspec to declare the font you want to use in your paper. Let us take an example :
\begin{Shaded}
\begin{Highlighting}[]

\CommentTok{% **********************************}
\CommentTok{% Basic setup}
\NormalTok{\textbackslash{}documentclass[10pt,a4paper]\{article\}}
\NormalTok{\textbackslash{}usepackage\{luatextra\}}\CommentTok{%this package calls fontspec, luatexbase, lualibs,}
 \NormalTok{metalogo, luacode and fixltx2e}
\NormalTok{\textbackslash{}setmainfont[Ligatures=Rare,Numbers=OldStyle]\{Arno Pro\} }\CommentTok{%setup of western font}
\NormalTok{\textbackslash{}usepackage\{luatexja\}}
\NormalTok{\textbackslash{}usepackage\{luatexja-fontspec\}}\CommentTok{%needed to call \textbackslash{}setmainjfont bellow}
\NormalTok{\textbackslash{}setmainjfont[BoldFont=KozGoPr6N-Bold]\{KozGoPr6N-Regular\} }\CommentTok{%setup of japanese}
 \NormalTok{font}
\CommentTok{%***********************************}
\NormalTok{\textbackslash{}begin\{document\}}
\NormalTok{It is a test to show japanese and english mix. テスト中です。どうですか皆さん。}
\NormalTok{\textbackslash{}end\{document\}}
\end{Highlighting}
\end{Shaded}

Use UTF-{}8 as your encoding. In case you don\textquotesingle{}t know how to do this, take a look at Texmaker, a LaTeX editor which use UTF-{}8 by default.


Another (but old) possible Japanese support is made available thanks to the \LaTeXTT{CJK} package collection. If you are using a package manager or a portage tree, the CJK collection is usually in a separate package because of its size (mainly due to fonts).

Make sure your document is saved using the UTF-{}8 character encoding. See \mylref{192}{Special Characters} for more details.
Put the parts where you want to write japanese characters in a \LaTeXTT{CJK} environment.
\begin{Shaded}
\begin{Highlighting}[]

\NormalTok{\textbackslash{}documentclass\{article\}}
\NormalTok{\textbackslash{}usepackage\{CJK\}}
 
\NormalTok{\textbackslash{}begin\{document\}}
 
\NormalTok{\textbackslash{}begin\{CJK\}\{UTF8\}\{min\}}
\NormalTok{こんにちは}
\NormalTok{You can mix latin letters as well as hiragana, katakana and kanji.}
\NormalTok{\textbackslash{}end\{CJK\}}
 
\NormalTok{\textbackslash{}end\{document\}}
\end{Highlighting}
\end{Shaded}

The last argument specifies the font. It must fit the desired language, since fonts are different for Chinese, Japanese and Korean. \LaTeXTT{min} is an example for Japanese.
\subsection{Korean}
\label{228}

The two most widely used encodings for Korean text files are EUC-{}KR and its upward compatible extension used in Korean MS-{}Windows, CP949/Windows-{}949/UHC. In these encodings each US-{}ASCII character represents its normal ASCII character similar to other ASCII compatible encodings such as ISO-{}8859-{}x, EUC-{}JP, Big5, or Shift\_JIS. On the other hand, Hangul syllables, Hanjas (Chinese characters as used in Korea), Hangul Jamos, Hiraganas, Katakanas, Greek and Cyrillic characters and other symbols and letters drawn from KS X 1001 are represented by two consecutive octets. The first has its MSB set. Until the mid-{}1990\textquotesingle{}s, it took a considerable amount of time and effort to set up a Korean-{}capable environment under a non-{}localized (non-{}Korean) operating system. You can skim through the now much-{}outdated \myplainurl{http://jshin.net/faq} to get a glimpse of what it was like to use Korean under non-{}Korean OS in mid-{}1990\textquotesingle{}s. 

TeX and LaTeX were originally written for scripts with no more than 256 characters in their alphabet. To make them work for languages with considerably more characters such as Korean or Chinese, a subfont mechanism was developed. It divides a single CJK font with thousands or tens of thousands of glyphs into a set of subfonts with 256 glyphs each.

For Korean, there are three widely used packages.
\begin{myitemize}
\item{}  HLATEX by UN Koaunghi
\item{}  hLATEXp by CHA Jaechoon 
\item{}  the CJK package by Werner Lemberg
\end{myitemize}


HLATEX and hLATEXp are specific to Korean and provide Korean localization on top of the font support. They both can process Korean input text files encoded in EUC-{}KR. HLATEX can even process input files encoded in CP949/Windows-{}949/UHC and UTF-{}8 when used along with Λ, Ω.

The CJK package is not specific to Korean. It can process input files in UTF-{}8 as well as in various CJK encodings including EUC-{}KR and CP949/Windows-{}949/UHC, it can be used to typeset documents with multilingual content (especially Chinese, Japanese and Korean). The CJK package has no Korean localization such as the one offered by HLATEX and it does not come with as many special Korean fonts as HLATEX.

The ultimate purpose of using typesetting programs like TeX and LaTeX is to get documents typeset in an {\itshape \setmainfont[Path=/usr/share/fonts/truetype/cmu/,UprightFont=cmunrm.ttf,BoldFont=cmunbx.ttf,ItalicFont=cmunti.ttf,BoldItalicFont=cmunbi.ttf]{cmunti.ttf}\setmonofont[Path=/usr/share/fonts/truetype/cmu/,UprightFont=cmuntt.ttf,BoldFont=cmuntb.ttf,ItalicFont=cmunit.ttf,BoldItalicFont=cmuntx.ttf]{cmunti.ttf}\itshape aesthetically}{$\text{ }$}\setmainfont[Path=/usr/share/fonts/truetype/cmu/,UprightFont=cmunrm.ttf,BoldFont=cmunbx.ttf,ItalicFont=cmunti.ttf,BoldItalicFont=cmunbi.ttf]{cmunrm.ttf}\setmonofont[Path=/usr/share/fonts/truetype/cmu/,UprightFont=cmuntt.ttf,BoldFont=cmuntb.ttf,ItalicFont=cmunit.ttf,BoldItalicFont=cmuntx.ttf]{cmunrm.ttf} satisfying way. Arguably the most important element in typesetting is a set of welldesigned fonts. The HLATEX distribution includes UHC PostScript fonts of 10 different families and Munhwabu fonts (TrueType) of 5 different families. The CJK package works with a set of fonts used by earlier versions of HLATEX and it can use Bitstream\textquotesingle{}s cyberbit True-{}Type font.

To use the HLATEX package for typesetting your Korean text, put the following declaration into the preamble of your document:
\begin{Shaded}
\begin{Highlighting}[]

\NormalTok{\textbackslash{}usepackage\{hangul\}}
\end{Highlighting}
\end{Shaded}

This command turns the Korean localization on. The headings of chapters, sections, subsections, table of content and table of figures are all translated into Korean and the formatting of the document is changed to follow Korean conventions. The package also provides automatic {\itshape \setmainfont[Path=/usr/share/fonts/truetype/cmu/,UprightFont=cmunrm.ttf,BoldFont=cmunbx.ttf,ItalicFont=cmunti.ttf,BoldItalicFont=cmunbi.ttf]{cmunti.ttf}\setmonofont[Path=/usr/share/fonts/truetype/cmu/,UprightFont=cmuntt.ttf,BoldFont=cmuntb.ttf,ItalicFont=cmunit.ttf,BoldItalicFont=cmuntx.ttf]{cmunti.ttf}\itshape particle selection}\setmainfont[Path=/usr/share/fonts/truetype/cmu/,UprightFont=cmunrm.ttf,BoldFont=cmunbx.ttf,ItalicFont=cmunti.ttf,BoldItalicFont=cmunbi.ttf]{cmunrm.ttf}\setmonofont[Path=/usr/share/fonts/truetype/cmu/,UprightFont=cmuntt.ttf,BoldFont=cmuntb.ttf,ItalicFont=cmunit.ttf,BoldItalicFont=cmuntx.ttf]{cmunrm.ttf}. In Korean, there are pairs of post-{}fix particles grammatically equivalent but different in form. Which of any given pair is correct depends on whether the preceding syllable ends with a vowel or a consonant. (It is a bit more complex than this, but this should give you a good picture.) Native Korean speakers have no problem picking the right particle, but it cannot be determined which particle to use for references and other automatic text that will change while you edit the document. It takes a painstaking effort to place appropriate particles manually every time you add/remove references or simply shuffle parts of your document around. HLATEX relieves its users from this boring and error-{}prone process. 

In case you don\textquotesingle{}t need Korean localization features but just want to typeset Korean text, you can put the following line in the preamble, instead.
\begin{Shaded}
\begin{Highlighting}[]

\NormalTok{\textbackslash{}usepackage\{hfont\}}
\end{Highlighting}
\end{Shaded}

For more details on typesetting Korean with HLATEX, refer to the HLATEX Guide. Check out the web site of the \myhref{http://www.ktug.or.kr/}{Korean TeX User Group (KTUG)}.

In the FAQ section of KTUG it is recommended to use the kotex package

\begin{Shaded}
\begin{Highlighting}[]

\NormalTok{\textbackslash{}usepackage\{kotex\}}
\end{Highlighting}
\end{Shaded}

\subsection{Persian script}
\label{229}

For Persian language, there is a dedicated package called XePersian which uses XeLaTeX as the typesetting engine. Just add the following code to your preamble:

\begin{Shaded}
\begin{Highlighting}[]

\NormalTok{\textbackslash{}usepackage\{xepersian\}}
\end{Highlighting}
\end{Shaded}


See \myhref{http://www.ctan.org/tex-archive/macros/xetex/latex/xepersian/}{XePersian page on CTAN}

Moreover, Arabic script can be used to type Persian as illustrated in the \mylref{240}{corresponding section}.
\subsection{Polish}
\label{230}

If you plan to use Polish in your UTF-{}8 encoded document, use the following code
\begin{Shaded}
\begin{Highlighting}[]

\NormalTok{\textbackslash{}usepackage[utf8]\{inputenc\}}
\NormalTok{\textbackslash{}usepackage\{polski\}}
\NormalTok{\textbackslash{}usepackage[polish]\{babel\}}
\end{Highlighting}
\end{Shaded}


The above code merely allows to use Polish letters and translates the automatic text to Polish, so that \symbol{34}chapter\symbol{34} becomes \symbol{34}rozdział\symbol{34}.
There are a few additional things one must remember about.
\subsubsection{Connectives}
\label{231}

Polish has many single letter connectives: \symbol{34}a\symbol{34}, \symbol{34}o\symbol{34}, \symbol{34}w\symbol{34}, \symbol{34}i\symbol{34}, \symbol{34}u\symbol{34}, \symbol{34}z\symbol{34}, etc., grammar and typography rules don\textquotesingle{}t allow for them to end a printed line.
To ensure that LaTeX won\textquotesingle{}t set them as last letter in the line, you have to use non breakable space:

\begin{Shaded}
\begin{Highlighting}[]

\NormalTok{Noc była sierpniowa, ciepła i~słodka, Księżyc oświecał srebrnem światłem}
 \NormalTok{wgłębienie, tak,}
\NormalTok{że twarze małego rycerza i~Basi były skąpane w blasku.}
\NormalTok{Poniżej, na podwórzu zamkowem, widać było uśpione kupy żołnierzy, a~także}
 \NormalTok{i~ciała zabitych}
\NormalTok{podczas dziennej strzelaniny, bo nie znaleziono dotąd czasu na ich pogrzebanie.}
\end{Highlighting}
\end{Shaded}


\subsubsection{Numerals}
\label{232}

According to Polish grammar rules, you have to put dots after numerals in chapter, section, subsection, etc. headers.

This is achieved by redefining few LaTeX macros.

For books:
\begin{Shaded}
\begin{Highlighting}[]

\NormalTok{\textbackslash{}renewcommand\textbackslash{}thechapter\{\textbackslash{}arabic\{chapter\}.\}}
\NormalTok{\textbackslash{}renewcommand\textbackslash{}thesection\{\textbackslash{}arabic\{chapter\}.\textbackslash{}arabic\{section\}.\}}
\NormalTok{\textbackslash{}re}
\NormalTok{newcommand\textbackslash{}thesubsection\{\textbackslash{}arabic\{chapter\}.\textbackslash{}arabic\{section\}.\textbackslash{}arabic\{subsection\}.\}}
\NormalTok{\textbackslash{}renewcommand\textbackslash{}thesubsubsectio}
\NormalTok{n\{\textbackslash{}arabic\{chapter\}.\textbackslash{}arabic\{section\}.\textbackslash{}arabic\{subsection\}.\textbackslash{}arabic\{subsubsection\}.\}}
\end{Highlighting}
\end{Shaded}


For articles:
\begin{Shaded}
\begin{Highlighting}[]

\NormalTok{\textbackslash{}renewcommand\textbackslash{}thesection\{\textbackslash{}arabic\{section\}.\}}
\NormalTok{\textbackslash{}renewcommand\textbackslash{}thesubsection\{\textbackslash{}arabic\{section\}.\textbackslash{}arabic\{subsection\}.\}}
\NormalTok{\textbackslash{}renewcomman}
\NormalTok{d\textbackslash{}thesubsubsection\{\textbackslash{}arabic\{section\}.\textbackslash{}arabic\{subsection\}.\textbackslash{}arabic\{subsubsection\}.\}}
\end{Highlighting}
\end{Shaded}



Alternatively you can use dedicated document classes:
\begin{myitemize}
\item{}  the \LaTeXTT{mwart} class instead of \LaTeXTT{article},
\item{}  \LaTeXTT{mwbk} instead of \LaTeXTT{book}
\item{}  and \LaTeXTT{mwrep} instead of \LaTeXTT{report}.
\end{myitemize}

Those classes have much more European typography settings but {\itshape \setmainfont[Path=/usr/share/fonts/truetype/cmu/,UprightFont=cmunrm.ttf,BoldFont=cmunbx.ttf,ItalicFont=cmunti.ttf,BoldItalicFont=cmunbi.ttf]{cmunti.ttf}\setmonofont[Path=/usr/share/fonts/truetype/cmu/,UprightFont=cmuntt.ttf,BoldFont=cmuntb.ttf,ItalicFont=cmunit.ttf,BoldItalicFont=cmuntx.ttf]{cmunti.ttf}\itshape do not}{$\text{ }$}\setmainfont[Path=/usr/share/fonts/truetype/cmu/,UprightFont=cmunrm.ttf,BoldFont=cmunbx.ttf,ItalicFont=cmunti.ttf,BoldItalicFont=cmunbi.ttf]{cmunrm.ttf}\setmonofont[Path=/usr/share/fonts/truetype/cmu/,UprightFont=cmuntt.ttf,BoldFont=cmuntb.ttf,ItalicFont=cmunit.ttf,BoldItalicFont=cmuntx.ttf]{cmunrm.ttf} require the use of Polish babel settings or character encoding.

Simple usage:
\begin{Shaded}
\begin{Highlighting}[]

\NormalTok{\textbackslash{}documentclass\{mwart\}}
\NormalTok{\textbackslash{}usepackage[polish]\{babel\}}
\NormalTok{\textbackslash{}usepackage\{polski\}}
\NormalTok{\textbackslash{}begin\{document\}}
\NormalTok{Pójdź kińże tę chmurność w głąb flaszy.}
\NormalTok{\textbackslash{}end\{document\}}
\end{Highlighting}
\end{Shaded}


Full documentation for those classes is available at \myplainurl{http://web.archive.org/web/20040609034031/http://www.ci.pwr.wroc.pl/~pmazur/LaTeX/mwclsdoc.pdf} (Polish).
\subsubsection{Indentation}
\label{233}

It may be customary (depending on publisher) to indent the first paragraph in sections and chapters:
\begin{Shaded}
\begin{Highlighting}[]

\NormalTok{\textbackslash{}usepackage\{indentfirst\}}
\end{Highlighting}
\end{Shaded}

\subsubsection{Hyphenation and typography}
\label{234}

It\textquotesingle{}s much more frowned upon to set pages with hyphenation between pages than it is customary in American typesetting.

To adjust penalties for hyphenation spanning pages, use this command:
\begin{Shaded}
\begin{Highlighting}[]

\NormalTok{\textbackslash{}brokenpenalty=1000}
\end{Highlighting}
\end{Shaded}


To adjust penalties for leaving widows and orphans (clubs in TeX nomenclature) use those commands:
\begin{Shaded}
\begin{Highlighting}[]

\NormalTok{\textbackslash{}clubpenalty=1000}
\NormalTok{\textbackslash{}widowpenalty=1000}
\end{Highlighting}
\end{Shaded}

\subsubsection{Commas in math}
\label{235}

According to Polish typography rules, fractional parts of numbers should be delimited by a comma, not a dot.
To make LaTeX not insert additional space in math mode after a comma (unless there is a space after the comma), use the \LaTeXTT{icomma} package.

\begin{Shaded}
\begin{Highlighting}[]

\NormalTok{\textbackslash{}usepackage\{icomma\}}
\end{Highlighting}
\end{Shaded}


Unfortunately, it is partially incompatible with the \LaTeXTT{dcolumn} package.
One needs to either use dots in columns with numerical data in the source file and make \LaTeXTT{dcolumn} switch them to commas for display {\itshape \setmainfont[Path=/usr/share/fonts/truetype/cmu/,UprightFont=cmunrm.ttf,BoldFont=cmunbx.ttf,ItalicFont=cmunti.ttf,BoldItalicFont=cmunbi.ttf]{cmunti.ttf}\setmonofont[Path=/usr/share/fonts/truetype/cmu/,UprightFont=cmuntt.ttf,BoldFont=cmuntb.ttf,ItalicFont=cmunit.ttf,BoldItalicFont=cmuntx.ttf]{cmunti.ttf}\itshape or}{$\text{ }$}\setmainfont[Path=/usr/share/fonts/truetype/cmu/,UprightFont=cmunrm.ttf,BoldFont=cmunbx.ttf,ItalicFont=cmunti.ttf,BoldItalicFont=cmunbi.ttf]{cmunrm.ttf}\setmonofont[Path=/usr/share/fonts/truetype/cmu/,UprightFont=cmuntt.ttf,BoldFont=cmuntb.ttf,ItalicFont=cmunit.ttf,BoldItalicFont=cmuntx.ttf]{cmunrm.ttf} define the column as follows:
\begin{Shaded}
\begin{Highlighting}[]

\NormalTok{\textbackslash{}begin\{tabular\}\{... D\{,\}\{\textbackslash{}mathord\textbackslash{}mathcomma\}\{2\} ...\}}
\end{Highlighting}
\end{Shaded}


The alternative is to use the \LaTeXTT{numprint} package, but it is much less convenient.
\subsubsection{Further information}
\label{236}

Refer the \myhref{http://so.pwn.pl/zasady.php}{Słownik Ortograficzny} (in Polish) for additional information on Polish grammar and typography rules.

Good extract is available at \myhref{http://dtp.msstudio.com.pl/typo.html}{Zasady Typograficzne Składania Tekstu} (in Polish).
\subsection{Portuguese}
\label{237}

Add the following code to your preamble:

\begin{Shaded}
\begin{Highlighting}[]

\NormalTok{\textbackslash{}usepackage[portuguese]\{babel\}}
\end{Highlighting}
\end{Shaded}


You can substitute the language for brazilian portuguese by choosing \LaTeXTT{brazilian} or \LaTeXTT{brazil}.
\subsection{Slovak}
\label{238}

Basic settings are fine when left the same as Czech, but Slovak needs special signs for \textquotesingle{}ď\textquotesingle{}, \textquotesingle{}ť\textquotesingle{}, \textquotesingle{}ľ\textquotesingle{}. To be able to type them from keyboard use the following settings:
\begin{Shaded}
\begin{Highlighting}[]

\NormalTok{\textbackslash{}usepackage[slovak]\{babel\}}
\NormalTok{\textbackslash{}usepackage[IL2]\{fontenc\}}
\end{Highlighting}
\end{Shaded}

\subsection{Spanish}
\label{239}

Include the appropriate Babel option:
\begin{Shaded}
\begin{Highlighting}[]

\NormalTok{\textbackslash{}usepackage[spanish]\{babel\}}
\end{Highlighting}
\end{Shaded}


The trick is that Spanish has several options and commands to control the layout. The options may be loaded either at the call to Babel, or before, by defining the command \LaTeXTT{\textbackslash{}spanishoptions}. Therefore, the following commands are roughly equivalent:

\begin{Shaded}
\begin{Highlighting}[]

\NormalTok{\textbackslash{}def\textbackslash{}spanishoptions\{mexico\}}
\NormalTok{\textbackslash{}usepackage[spanish]\{babel\}}
\end{Highlighting}
\end{Shaded}


\begin{Shaded}
\begin{Highlighting}[]

\NormalTok{\textbackslash{}usepackage[spanish,mexico]\{babel\}}
\end{Highlighting}
\end{Shaded}


On average, the former syntax should be preferred, as the latter is a deviation from standard Babel behavior, and thus may break other programs (LyX, latex2rtf) interacting with LaTeX.

Spanish also defines shorthands for the dot and <{}<{} >{}>{} so that they are used as logical markup: the former is used as decimal marker in math mode, and the output is typically either a comma or a dot; the latter is used for quoted text, and the output is typically either «» or “”. This allows different typographical conventions with the same input, as preferences may be quite different from, say, Spain and Mexico.

Two particularly useful options are \LaTeXTT{es-{}noquoting,es-{}nolists}: some packages and classes are known to collide with Spanish in the way they handle active characters, and these options disable the internal workings of Spanish to allow you to overcome these common pitfalls. Moreover, these options may simplify the way LyX customizes some features of the Spanish layout from inside the GUI.

The options \LaTeXTT{mexico,mexico-{}com} provide support for local custom in Mexico: the former using decimal dot, as customary, and the latter allowing decimal comma, as required by the Mexican Official Norm (NOM) of the Department of Economy for labels in foods and goods. More localizations are in the making.

The other commands modify the spanish layout after loading Babel. Two particularly useful commands are \LaTeXTT{\textbackslash{}spanishoperators} and \LaTeXTT{\textbackslash{}spanishdeactivate}. 

The macro \LaTeXTT{\textbackslash{}spanishoperators\{<{}list of operators>{}\}\{} contains a list of spanish mathematical operators, and may be redefined at will. For instance, the command
\begin{Shaded}
\begin{Highlighting}[]

\NormalTok{\textbackslash{}def\textbackslash{}spanishoperators\{sen\}}
\end{Highlighting}
\end{Shaded}

only defines \LaTeXTT{sen}, overriding all other definitions; the command \LaTeXTT{\textbackslash{}let\textbackslash{}spanishoperators\textbackslash{}relax} disables them all. This command supports accented or spaced operators: the \LaTeXTT{\textbackslash{}acute\{<{}letter>{}\}} command puts an accent, and the \LaTeXTT{\textbackslash{},} command adds a small space. 
For instance, the following operators are defined by default.

\begin{Shaded}
\begin{Highlighting}[]

\NormalTok{l\textbackslash{}acute\{i\}m l\textbackslash{}acute\{i\}m\textbackslash{},sup l\textbackslash{}acute\{i\}m\textbackslash{},inf m\textbackslash{}acute\{a\}x }
\NormalTok{\textbackslash{}acute\{i\}nf m\textbackslash{}acute\{i\}n sen tg arc\textbackslash{},sen arc\textbackslash{},cos arc\textbackslash{},tg }
\NormalTok{cotg cosec senh tgh}
\end{Highlighting}
\end{Shaded}


Finally, the macro \LaTeXTT{\textbackslash{}spanishdeactivate\{<{}list of characters>{}\}} disables some active characters, to keep you out of trouble if they are redefined by other packages. The candidates for deactivation are the set \{<{}>{}.\symbol{34}\textquotesingle{}\}. Please, beware that some option preempt the availability of some active characters. In particular, you should not combine the \LaTeXTT{es-{}noquoting} option with \LaTeXTT{\textbackslash{}spanishdeactivate\{<{}>{}\}}, or the \LaTeXTT{es-{}noshorthands} with \LaTeXTT{\textbackslash{}spanishdeactivate\{<{}>{}.\symbol{34}\}}. 

Please check the documentation for Babel or {\ttfamily \setmainfont[Path=/usr/share/fonts/truetype/cmu/,UprightFont=cmunrm.ttf,BoldFont=cmunbx.ttf,ItalicFont=cmunti.ttf,BoldItalicFont=cmunbi.ttf]{cmuntt.ttf}\setmonofont[Path=/usr/share/fonts/truetype/cmu/,UprightFont=cmuntt.ttf,BoldFont=cmuntb.ttf,ItalicFont=cmunit.ttf,BoldItalicFont=cmuntx.ttf]{cmuntt.ttf}\ttfamily spanish.dtx}{$\text{ }$}\setmainfont[Path=/usr/share/fonts/truetype/cmu/,UprightFont=cmunrm.ttf,BoldFont=cmunbx.ttf,ItalicFont=cmunti.ttf,BoldItalicFont=cmunbi.ttf]{cmunrm.ttf}\setmonofont[Path=/usr/share/fonts/truetype/cmu/,UprightFont=cmuntt.ttf,BoldFont=cmuntb.ttf,ItalicFont=cmunit.ttf,BoldItalicFont=cmuntx.ttf]{cmunrm.ttf} for further details.
\subsection{Tibetan}
\label{240}
One option to use Tibetan script in LaTeX is to add
\begin{Shaded}
\begin{Highlighting}[]

\NormalTok{\textbackslash{}usepackage\{ctib\}}
\end{Highlighting}
\end{Shaded}

to your preamble and use a slightly modified Wylie transliteration for input. Refer to the excellent package documentation for details.
More information can be found on \myplainurl{http://www.thlib.org/tools/scripts/wiki/latex.html}
\section{References}
\label{241}


\chapter{Rotations}

\myminitoc
\label{242}

\label{243}


\begin{TemplateInfo}{\danger}{Warning}Many DVI viewers do not support rotating of text and tables. The text will be displayed normally. You must convert your DVI file to a PDF document and view it in a PDF viewer to see the rotation in effect.  Take care however that printing from those PDF files may rotate the respective page {\itshape \setmainfont[Path=/usr/share/fonts/truetype/cmu/,UprightFont=cmunrm.ttf,BoldFont=cmunbx.ttf,ItalicFont=cmunti.ttf,BoldItalicFont=cmunbi.ttf]{cmunti.ttf}\setmonofont[Path=/usr/share/fonts/truetype/cmu/,UprightFont=cmuntt.ttf,BoldFont=cmuntb.ttf,ItalicFont=cmunit.ttf,BoldItalicFont=cmuntx.ttf]{cmunti.ttf}\itshape again in the same direction}{$\text{ }$}\setmainfont[Path=/usr/share/fonts/truetype/cmu/,UprightFont=cmunrm.ttf,BoldFont=cmunbx.ttf,ItalicFont=cmunti.ttf,BoldItalicFont=cmunbi.ttf]{cmunrm.ttf}\setmonofont[Path=/usr/share/fonts/truetype/cmu/,UprightFont=cmuntt.ttf,BoldFont=cmuntb.ttf,ItalicFont=cmunit.ttf,BoldItalicFont=cmuntx.ttf]{cmunrm.ttf} under certain circumstances.  This behaviour can be influenced by the settings of your dvi2pdf converter, look at your manual for further information.\end{TemplateInfo}
\section{The {\itshape \setmainfont[Path=/usr/share/fonts/truetype/cmu/,UprightFont=cmunrm.ttf,BoldFont=cmunbx.ttf,ItalicFont=cmunti.ttf,BoldItalicFont=cmunbi.ttf]{cmunti.ttf}\setmonofont[Path=/usr/share/fonts/truetype/cmu/,UprightFont=cmuntt.ttf,BoldFont=cmuntb.ttf,ItalicFont=cmunit.ttf,BoldItalicFont=cmuntx.ttf]{cmunti.ttf}\itshape rotating}{$\text{ }$}\setmainfont[Path=/usr/share/fonts/truetype/cmu/,UprightFont=cmunrm.ttf,BoldFont=cmunbx.ttf,ItalicFont=cmunti.ttf,BoldItalicFont=cmunbi.ttf]{cmunrm.ttf}\setmonofont[Path=/usr/share/fonts/truetype/cmu/,UprightFont=cmuntt.ttf,BoldFont=cmuntb.ttf,ItalicFont=cmunit.ttf,BoldItalicFont=cmuntx.ttf]{cmunrm.ttf} package}
\label{244}

The package \LaTeXTT{rotating} gives you the possibility to rotate any object of an arbitrary angle. Once you have loaded it with the standard command in the preamble:

\begin{Shaded}
\begin{Highlighting}[]

\NormalTok{\textbackslash{}usepackage\{rotating\}}
\end{Highlighting}
\end{Shaded}


you can use three new environments:

\begin{Shaded}
\begin{Highlighting}[]

\NormalTok{\textbackslash{}begin\{sideways\}}
\end{Highlighting}
\end{Shaded}


it will rotate the whole argument by 90 degrees counterclockwise. Moreover:

\begin{Shaded}
\begin{Highlighting}[]

\NormalTok{\textbackslash{}begin\{turn\}\{30\}}
\end{Highlighting}
\end{Shaded}


it will turn the argument of 30 degrees. You can give any angle as an argument, whether it is positive or negative. It will leave the necessary space to avoid any overlapping of text.

\begin{Shaded}
\begin{Highlighting}[]

\NormalTok{\textbackslash{}begin\{rotate\}\{30\}}
\end{Highlighting}
\end{Shaded}


like \LaTeXTT{turn}, but it will not add any extra space.

If you want to make a float sideways so that the caption is also rotated, you can use

\begin{Shaded}
\begin{Highlighting}[]

\NormalTok{\textbackslash{}begin\{sidewaysfigure\}}
\end{Highlighting}
\end{Shaded}


or

\begin{Shaded}
\begin{Highlighting}[]

\NormalTok{\textbackslash{}begin\{sidewaystable\}}
\end{Highlighting}
\end{Shaded}


Note, though, they will be placed on a separate page.

If you would like to rotate a TikZ picture you could use \LaTeXTT{sideways} together with \LaTeXTT{minipage}.

\begin{Shaded}
\begin{Highlighting}[]

\NormalTok{\textbackslash{}begin\{figure\}}
  \NormalTok{\textbackslash{}begin\{sideways\}}
    \NormalTok{\textbackslash{}begin\{minipage\}\{17.5cm\}}
      \NormalTok{\textbackslash{}input\{../path/to/picture\}}
    \NormalTok{\textbackslash{}end\{minipage\}}
  \NormalTok{\textbackslash{}end\{sideways\}}
  \NormalTok{\textbackslash{}centering}
  \NormalTok{\textbackslash{}caption[Caption]\{Caption.\}}
  \NormalTok{\textbackslash{}label\{pic:picture\}}
\NormalTok{\textbackslash{}end\{figure\}}
\end{Highlighting}
\end{Shaded}


You can also use the \LaTeXTT{\textbackslash{}rotatebox} command. Let\textquotesingle{}s rotate a tabular inside a table for example:

\begin{Shaded}
\begin{Highlighting}[]

\NormalTok{\textbackslash{}begin\{table\}[p]}
	\NormalTok{\textbackslash{}centering}
	\NormalTok{\textbackslash{}rotatebox\{90\}\{}
		\NormalTok{\textbackslash{}begin\{minipage\}\{\textbackslash{}textheight\}}
		\NormalTok{\textbackslash{}begin\{tabular\}\{}
\end{Highlighting}
\end{Shaded}

\subsection{Options}
\label{245}

Default is sidewaysfigures/sidewaystables are oriented depending on page number in two sided documents (takes two passes).

The rotating package takes the following options.
{\bfseries
\begin{mydescription}counterclockwise/anticlockwise
\end{mydescription}
}
\begin{myquote}\item{} In single sided documents turn sidewaysfigures/sidewaystables counterclockwise.
\end{myquote}
{\bfseries
\begin{mydescription}clockwise
\end{mydescription}
}
\begin{myquote}\item{} In single sided documents turn sidewaysfigures/sidewaystables clockwise (default).
\end{myquote}
{\bfseries
\begin{mydescription}figuresright
\end{mydescription}
}
\begin{myquote}\item{} In two sided documents all sidewaysfigures/sidewaystables are same orientation (left of figure, table now bottom of page).  This is the style preferred by the Chicago Manual of Style (broadside).
\end{myquote}
{\bfseries
\begin{mydescription}figuresleft
\end{mydescription}
}
\begin{myquote}\item{} In two sided documents all sidewaysfigures/sidewaystables are same orientation (left of figure, table now at top of page).
\end{myquote}

\section{The {\itshape \setmainfont[Path=/usr/share/fonts/truetype/cmu/,UprightFont=cmunrm.ttf,BoldFont=cmunbx.ttf,ItalicFont=cmunti.ttf,BoldItalicFont=cmunbi.ttf]{cmunti.ttf}\setmonofont[Path=/usr/share/fonts/truetype/cmu/,UprightFont=cmuntt.ttf,BoldFont=cmuntb.ttf,ItalicFont=cmunit.ttf,BoldItalicFont=cmuntx.ttf]{cmunti.ttf}\itshape rotfloat}{$\text{ }$}\setmainfont[Path=/usr/share/fonts/truetype/cmu/,UprightFont=cmunrm.ttf,BoldFont=cmunbx.ttf,ItalicFont=cmunti.ttf,BoldItalicFont=cmunbi.ttf]{cmunrm.ttf}\setmonofont[Path=/usr/share/fonts/truetype/cmu/,UprightFont=cmuntt.ttf,BoldFont=cmuntb.ttf,ItalicFont=cmunit.ttf,BoldItalicFont=cmuntx.ttf]{cmunrm.ttf} package}
\label{246}

When it is desirable to place the rotated table at the exact location where it appears in the source (.tex) file, \LaTeXTT{rotfloat} package may be used. Then one can use

\begin{Shaded}
\begin{Highlighting}[]

\NormalTok{\textbackslash{}begin\{sidewaystable\}[H]}
\end{Highlighting}
\end{Shaded}


just like for normal tables. The \LaTeXTT{H} option can not be used without this package.

\chapter{Tables}

\myminitoc
\label{247}

\label{248}

Tables are a common feature in academic writing, often used to summarize research results. Mastering the art of table construction in LaTeX is therefore necessary to produce quality papers and with sufficient practice one can print beautiful tables of any kind.

Keeping in mind that LaTeX is not a spreadsheet, it makes sense to use a dedicated tool to build tables and then to export these tables into the document. Basic tables are not too taxing, but anything more advanced can take a fair bit of construction; in these cases, more advanced packages can be very useful. However, first it is important to know the basics. Once you are comfortable with basic LaTeX tables, you might have a look at more advanced packages or the \mylref{289}{export options of your favorite spreadsheet}. Thanks to the modular nature of LaTeX, the whole process can be automated in a fairly comfortable way.

For a long time, LaTeX tables were quite a chaotic topic, with dozens of packages doing similar things, while not always being compatible with one another. Sometimes you had to make trade-{}offs. The situation changed recently (2010) with the release of the \LaTeXTT{tabu} package which combines the power of \LaTeXTT{longtable}, \LaTeXTT{tabularx} and much more. The \LaTeXTT{tabu} environment is far less fragile and restricted than the older alternatives. Nonetheless, before attempting to use this package for the first time it will be beneficial to understand how the classic environment works, since \LaTeXTT{tabu} works the same way. Note however that the author of \LaTeXTT{tabu} will not fix bugs to the current version, and that the next version introduces new syntax that will likely break existing documents.\myfootnote{\myplainurl{http://tex.stackexchange.com/questions/121841/is-the-tabu-package-obsolete}} 
\section{The {\itshape \setmainfont[Path=/usr/share/fonts/truetype/cmu/,UprightFont=cmunrm.ttf,BoldFont=cmunbx.ttf,ItalicFont=cmunti.ttf,BoldItalicFont=cmunbi.ttf]{cmunti.ttf}\setmonofont[Path=/usr/share/fonts/truetype/cmu/,UprightFont=cmuntt.ttf,BoldFont=cmuntb.ttf,ItalicFont=cmunit.ttf,BoldItalicFont=cmuntx.ttf]{cmunti.ttf}\itshape tabular}{$\text{ }$}\setmainfont[Path=/usr/share/fonts/truetype/cmu/,UprightFont=cmunrm.ttf,BoldFont=cmunbx.ttf,ItalicFont=cmunti.ttf,BoldItalicFont=cmunbi.ttf]{cmunrm.ttf}\setmonofont[Path=/usr/share/fonts/truetype/cmu/,UprightFont=cmuntt.ttf,BoldFont=cmuntb.ttf,ItalicFont=cmunit.ttf,BoldItalicFont=cmuntx.ttf]{cmunrm.ttf} environment}
\label{249}

The \LaTeXTT{tabular} environment can be used to typeset tables with optional horizontal and vertical lines. LaTeX determines the width of the columns automatically.

The first line of the environment has the form:
\begin{Shaded}
\begin{Highlighting}[]

\NormalTok{\textbackslash{}begin\{tabular\}[pos]\{table spec\}}
\end{Highlighting}
\end{Shaded}


The \LaTeXTT{table spec} argument tells LaTeX the alignment to be used in each column and the vertical lines to insert.

The number of columns does not need to be specified as it is inferred by looking at the number of arguments provided. It is also possible to add vertical lines between the columns here. The following symbols are available to describe the table columns (some of them require that the package \LaTeXTT{array} has been loaded):

\begin{longtable}{|>{\RaggedRight}p{0.17133\linewidth}|>{\RaggedRight}p{0.77153\linewidth}|} \hline 
\hspace*{0pt}\ignorespaces{}\hspace*{0pt} \LaTeXTT{l}&\hspace*{0pt}\ignorespaces{}\hspace*{0pt} left-{}justified column\\ \hline \hspace*{0pt}\ignorespaces{}\hspace*{0pt} \LaTeXTT{c}&\hspace*{0pt}\ignorespaces{}\hspace*{0pt} centered column\\ \hline \hspace*{0pt}\ignorespaces{}\hspace*{0pt} \LaTeXTT{r}&\hspace*{0pt}\ignorespaces{}\hspace*{0pt} right-{}justified column\\ \hline \hspace*{0pt}\ignorespaces{}\hspace*{0pt} \LaTeXTT{p\{\textquotesingle{}width\textquotesingle{}\}}&\hspace*{0pt}\ignorespaces{}\hspace*{0pt} paragraph column with text vertically aligned at the top\\ \hline \hspace*{0pt}\ignorespaces{}\hspace*{0pt} \LaTeXTT{m\{\textquotesingle{}width\textquotesingle{}\}}&\hspace*{0pt}\ignorespaces{}\hspace*{0pt} paragraph column with text vertically aligned in the middle (requires array package)\\ \hline \hspace*{0pt}\ignorespaces{}\hspace*{0pt} \LaTeXTT{b\{\textquotesingle{}width\textquotesingle{}\}}&\hspace*{0pt}\ignorespaces{}\hspace*{0pt} paragraph column with text vertically aligned at the bottom (requires array package)\\ \hline \hspace*{0pt}\ignorespaces{}\hspace*{0pt} {\ttfamily \setmainfont[Path=/usr/share/fonts/truetype/cmu/,UprightFont=cmunrm.ttf,BoldFont=cmunbx.ttf,ItalicFont=cmunti.ttf,BoldItalicFont=cmunbi.ttf]{cmuntt.ttf}\setmonofont[Path=/usr/share/fonts/truetype/cmu/,UprightFont=cmuntt.ttf,BoldFont=cmuntb.ttf,ItalicFont=cmunit.ttf,BoldItalicFont=cmuntx.ttf]{cmuntt.ttf}\ttfamily |}&\hspace*{0pt}\ignorespaces{}\hspace*{0pt}{$\text{ }$}\setmainfont[Path=/usr/share/fonts/truetype/cmu/,UprightFont=cmunrm.ttf,BoldFont=cmunbx.ttf,ItalicFont=cmunti.ttf,BoldItalicFont=cmunbi.ttf]{cmunrm.ttf}\setmonofont[Path=/usr/share/fonts/truetype/cmu/,UprightFont=cmuntt.ttf,BoldFont=cmuntb.ttf,ItalicFont=cmunit.ttf,BoldItalicFont=cmuntx.ttf]{cmunrm.ttf} vertical line\\ \hline \hspace*{0pt}\ignorespaces{}\hspace*{0pt} {\ttfamily \setmainfont[Path=/usr/share/fonts/truetype/cmu/,UprightFont=cmunrm.ttf,BoldFont=cmunbx.ttf,ItalicFont=cmunti.ttf,BoldItalicFont=cmunbi.ttf]{cmuntt.ttf}\setmonofont[Path=/usr/share/fonts/truetype/cmu/,UprightFont=cmuntt.ttf,BoldFont=cmuntb.ttf,ItalicFont=cmunit.ttf,BoldItalicFont=cmuntx.ttf]{cmuntt.ttf}\ttfamily ||}&\hspace*{0pt}\ignorespaces{}\hspace*{0pt}{$\text{ }$}\setmainfont[Path=/usr/share/fonts/truetype/cmu/,UprightFont=cmunrm.ttf,BoldFont=cmunbx.ttf,ItalicFont=cmunti.ttf,BoldItalicFont=cmunbi.ttf]{cmunrm.ttf}\setmonofont[Path=/usr/share/fonts/truetype/cmu/,UprightFont=cmuntt.ttf,BoldFont=cmuntb.ttf,ItalicFont=cmunit.ttf,BoldItalicFont=cmuntx.ttf]{cmunrm.ttf} double vertical line\\ \hline 
\end{longtable}


By default, if the text in a column is too wide for the page, LaTeX won’t automatically wrap it. Using \LaTeXTT{p\{\textquotesingle{}width\textquotesingle{}\}} you can define a special type of column which will wrap-{}around the text as in a normal paragraph. You can pass the width using any unit supported by LaTeX, such as \textquotesingle{}pt\textquotesingle{} and \textquotesingle{}cm\textquotesingle{}, or {\itshape \setmainfont[Path=/usr/share/fonts/truetype/cmu/,UprightFont=cmunrm.ttf,BoldFont=cmunbx.ttf,ItalicFont=cmunti.ttf,BoldItalicFont=cmunbi.ttf]{cmunti.ttf}\setmonofont[Path=/usr/share/fonts/truetype/cmu/,UprightFont=cmuntt.ttf,BoldFont=cmuntb.ttf,ItalicFont=cmunit.ttf,BoldItalicFont=cmuntx.ttf]{cmunti.ttf}\itshape command lengths}\setmainfont[Path=/usr/share/fonts/truetype/cmu/,UprightFont=cmunrm.ttf,BoldFont=cmunbx.ttf,ItalicFont=cmunti.ttf,BoldItalicFont=cmunbi.ttf]{cmunrm.ttf}\setmonofont[Path=/usr/share/fonts/truetype/cmu/,UprightFont=cmuntt.ttf,BoldFont=cmuntb.ttf,ItalicFont=cmunit.ttf,BoldItalicFont=cmuntx.ttf]{cmunrm.ttf}, such as \LaTeXTT{\textbackslash{}textwidth}. You can find a list in chapter \mylref{456}{Lengths}.

The optional parameter \LaTeXTT{pos} can be used to specify the vertical position of the table relative to the baseline of the surrounding text. In most cases, you will not need this option. It becomes relevant only if your table is not in a paragraph of its own. You can use the following letters:

\begin{longtable}{|>{\RaggedRight}p{0.18435\linewidth}|>{\RaggedRight}p{0.75851\linewidth}|} \hline 
\hspace*{0pt}\ignorespaces{}\hspace*{0pt} \LaTeXTT{b}&\hspace*{0pt}\ignorespaces{}\hspace*{0pt} bottom\\ \hline \hspace*{0pt}\ignorespaces{}\hspace*{0pt} \LaTeXTT{c}&\hspace*{0pt}\ignorespaces{}\hspace*{0pt} center (default)\\ \hline \hspace*{0pt}\ignorespaces{}\hspace*{0pt} \LaTeXTT{t}&\hspace*{0pt}\ignorespaces{}\hspace*{0pt} top\\ \hline 
\end{longtable}


To specify a font format (such as bold, italic, etc.) for an entire column, you can add \LaTeXTT{>{}\{\textbackslash{}format\}} before you declare the alignment. For example \LaTeXTT{\textbackslash{}begin\{tabular\}\{ >{}\{\textbackslash{}bfseries\}l c >{}\{\textbackslash{}itshape\}r \}} will indicate a three column table with the first one aligned to the left and in bold font, the second one aligned in the center and with normal font, and the third aligned to the right and in italic. The \symbol{34}array\symbol{34} package needs to be activated in the preamble for this to work.

In the first line you have pointed out how many columns you want, their alignment and the vertical lines to separate them. Once in the environment, you have to introduce the text you want, separating between cells and introducing new lines. The commands you have to use are the following:

\begin{longtable}{|>{\RaggedRight}p{0.18576\linewidth}|>{\RaggedRight}p{0.75710\linewidth}|} \hline 
\hspace*{0pt}\ignorespaces{}\hspace*{0pt} \LaTeXTT{\&}&\hspace*{0pt}\ignorespaces{}\hspace*{0pt} column separator\\ \hline \hspace*{0pt}\ignorespaces{}\hspace*{0pt} \LaTeXTT{\textbackslash{}\textbackslash{}}&\hspace*{0pt}\ignorespaces{}\hspace*{0pt} start new row (additional space may be specified after \LaTeXTT{\textbackslash{}\textbackslash{}} using square brackets, such as \LaTeXTT{\textbackslash{}\textbackslash{}{$\text{[}$}6pt{$\text{]}$}})\\ \hline \hspace*{0pt}\ignorespaces{}\hspace*{0pt} \LaTeXTT{\textbackslash{}hline}&\hspace*{0pt}\ignorespaces{}\hspace*{0pt} horizontal line\\ \hline \hspace*{0pt}\ignorespaces{}\hspace*{0pt} \LaTeXTT{\textbackslash{}newline}&\hspace*{0pt}\ignorespaces{}\hspace*{0pt} start a new line within a cell (in a paragraph column)\\ \hline \hspace*{0pt}\ignorespaces{}\hspace*{0pt} \LaTeXTT{\textbackslash{}cline\{i-{}j\}}&\hspace*{0pt}\ignorespaces{}\hspace*{0pt} partial horizontal line beginning in column {\itshape \setmainfont[Path=/usr/share/fonts/truetype/cmu/,UprightFont=cmunrm.ttf,BoldFont=cmunbx.ttf,ItalicFont=cmunti.ttf,BoldItalicFont=cmunbi.ttf]{cmunti.ttf}\setmonofont[Path=/usr/share/fonts/truetype/cmu/,UprightFont=cmuntt.ttf,BoldFont=cmuntb.ttf,ItalicFont=cmunit.ttf,BoldItalicFont=cmuntx.ttf]{cmunti.ttf}\itshape i}{$\text{ }$}\setmainfont[Path=/usr/share/fonts/truetype/cmu/,UprightFont=cmunrm.ttf,BoldFont=cmunbx.ttf,ItalicFont=cmunti.ttf,BoldItalicFont=cmunbi.ttf]{cmunrm.ttf}\setmonofont[Path=/usr/share/fonts/truetype/cmu/,UprightFont=cmuntt.ttf,BoldFont=cmuntb.ttf,ItalicFont=cmunit.ttf,BoldItalicFont=cmuntx.ttf]{cmunrm.ttf} and ending in column {\itshape \setmainfont[Path=/usr/share/fonts/truetype/cmu/,UprightFont=cmunrm.ttf,BoldFont=cmunbx.ttf,ItalicFont=cmunti.ttf,BoldItalicFont=cmunbi.ttf]{cmunti.ttf}\setmonofont[Path=/usr/share/fonts/truetype/cmu/,UprightFont=cmuntt.ttf,BoldFont=cmuntb.ttf,ItalicFont=cmunit.ttf,BoldItalicFont=cmuntx.ttf]{cmunti.ttf}\itshape j}\\ \hline 
\end{longtable}
\setmainfont[Path=/usr/share/fonts/truetype/cmu/,UprightFont=cmunrm.ttf,BoldFont=cmunbx.ttf,ItalicFont=cmunti.ttf,BoldItalicFont=cmunbi.ttf]{cmunrm.ttf}\setmonofont[Path=/usr/share/fonts/truetype/cmu/,UprightFont=cmuntt.ttf,BoldFont=cmuntb.ttf,ItalicFont=cmunit.ttf,BoldItalicFont=cmuntx.ttf]{cmunrm.ttf}

Note, any white space inserted between these commands is purely down to ones\textquotesingle{} preferences. I personally add spaces between to make it easier to read.
\subsection{Basic examples}
\label{250}

This example shows how to create a simple table in LaTeX. It is a three-{}by-{}three table, but without any lines.

\begin{longtable}{p{1.0\linewidth}}
\begin{Shaded}
\begin{Highlighting}[]

\NormalTok{\textbackslash{}begin\{tabular\}\{ l c r \}}
  \NormalTok{1 & 2 & 3 \textbackslash{}\textbackslash{}}
  \NormalTok{4 & 5 & 6 \textbackslash{}\textbackslash{}}
  \NormalTok{7 & 8 & 9 \textbackslash{}\textbackslash{}}
\NormalTok{\textbackslash{}end\{tabular\}}
\end{Highlighting}
\end{Shaded}
\\

{$\begin{array}{lcr} 1 & 2 & 3 \\ 4 & 5 & 6 \\ 7 & 8 & 9 \\ \end{array}$}

\end{longtable}

Expanding upon that by including some vertical lines:

\begin{longtable}{p{1.0\linewidth}}
\begin{Shaded}
\begin{Highlighting}[]

\NormalTok{\textbackslash{}begin\{tabular\}\{ l  c  r \}}
  \NormalTok{1 & 2 & 3 \textbackslash{}\textbackslash{}}
  \NormalTok{4 & 5 & 6 \textbackslash{}\textbackslash{}}
  \NormalTok{7 & 8 & 9 \textbackslash{}\textbackslash{}}
\NormalTok{\textbackslash{}end\{tabular\}}
\end{Highlighting}
\end{Shaded}
\\

{$\begin{array}{l|c||r} 1 & 2 & 3 \\ 4 & 5 & 6 \\ 7 & 8 & 9 \\ \end{array}$}

\end{longtable}

To add horizontal lines to the very top and bottom edges of the table:

\begin{longtable}{p{1.0\linewidth}}
\begin{Shaded}
\begin{Highlighting}[]

\NormalTok{\textbackslash{}begin\{tabular\}\{ l  c  r \}}
  \NormalTok{\textbackslash{}hline			}
  \NormalTok{1 & 2 & 3 \textbackslash{}\textbackslash{}}
  \NormalTok{4 & 5 & 6 \textbackslash{}\textbackslash{}}
  \NormalTok{7 & 8 & 9 \textbackslash{}\textbackslash{}}
  \NormalTok{\textbackslash{}hline  }
\NormalTok{\textbackslash{}end\{tabular\}}
\end{Highlighting}
\end{Shaded}
\\

{$\begin{array}{l|c||r}\hline 1 & 2 & 3 \\ 4 & 5 & 6 \\ 7 & 8 & 9 \\ \hline \end{array}$}

\end{longtable}

And finally, to add lines between all rows, as well as centering (notice the use of the center environment -{} of course, the result of this is not obvious from the preview on this web page):

\begin{longtable}{p{1.0\linewidth}}
\begin{Shaded}
\begin{Highlighting}[]

\NormalTok{\textbackslash{}begin\{center\}}
  \NormalTok{\textbackslash{}begin\{tabular\}\{ l  c  r \}}
    \NormalTok{\textbackslash{}hline}
    \NormalTok{1 & 2 & 3 \textbackslash{}\textbackslash{} \textbackslash{}hline}
    \NormalTok{4 & 5 & 6 \textbackslash{}\textbackslash{} \textbackslash{}hline}
    \NormalTok{7 & 8 & 9 \textbackslash{}\textbackslash{}}
    \NormalTok{\textbackslash{}hline}
  \NormalTok{\textbackslash{}end\{tabular\}}
\NormalTok{\textbackslash{}end\{center\}}
\end{Highlighting}
\end{Shaded}
\\

{$\begin{array}{l|c||r}\hline 1 & 2 & 3 \\ \hline  4 & 5 & 6 \\ \hline  7 & 8 & 9 \\ \hline \end{array}$}

\end{longtable}

\begin{longtable}{p{1.0\linewidth}}
\begin{Shaded}
\begin{Highlighting}[]

\NormalTok{\textbackslash{}begin\{center\}}
  \NormalTok{\textbackslash{}begin\{tabular\}\{  l  c  r \}}
    \NormalTok{\textbackslash{}hline}
    \NormalTok{1 & 2 & 3 \textbackslash{}\textbackslash{} \textbackslash{}hline}
    \NormalTok{4 & 5 & 6 \textbackslash{}\textbackslash{} \textbackslash{}hline \textbackslash{}hline}
    \NormalTok{7 & 8 & 9 \textbackslash{}\textbackslash{}}
    \NormalTok{\textbackslash{}hline}
  \NormalTok{\textbackslash{}end\{tabular\}}
\NormalTok{\textbackslash{}end\{center\}}
\end{Highlighting}
\end{Shaded}
\\

{$\begin{array}{|l||c|||r}\hline 1 & 2 & 3 \\ \hline  4 & 5 & 6 \\ \hline \hline 7 & 8 & 9 \\ \hline \end{array}$}

\end{longtable}

\begin{longtable}{p{1.0\linewidth}}
\begin{Shaded}
\begin{Highlighting}[]

\NormalTok{\textbackslash{}begin\{tabular\}}
  \NormalTok{\textbackslash{}hline}
  \NormalTok{7C0 & hexadecimal \textbackslash{}\textbackslash{}}
  \NormalTok{3700 & octal \textbackslash{}\textbackslash{} \textbackslash{}cline\{2-2\}}
  \NormalTok{11111000000 & binary \textbackslash{}\textbackslash{}}
  \NormalTok{\textbackslash{}hline \textbackslash{}hline}
  \NormalTok{1984 & decimal \textbackslash{}\textbackslash{}}
  \NormalTok{\textbackslash{}hline}
\NormalTok{\textbackslash{}end\{tabular\}}
\end{Highlighting}
\end{Shaded}
\\



\begin{minipage}{0.62500\textwidth}
\begin{center}
\includegraphics[width=1.0\textwidth,height=6.5in,keepaspectratio]{../images/35.\SVGExtension}
\end{center}
\raggedright{}\myfigurewithoutcaption{35}
\end{minipage}\vspace{0.75cm}



\end{longtable}
\subsection{Text wrapping in tables}
\label{251}

LaTeX\textquotesingle{}s algorithms for formatting tables have a few shortcomings. One is that it will not automatically wrap text in cells, even if it overruns the width of the page. For columns that will contain text whose length exceeds the column\textquotesingle{}s width, it is recommended that you use the \LaTeXTT{p} attribute and specify the desired width of the column (although it may take some trial-{}and-{}error to get the result you want). For a more convenient method, have a look at \mylref{272}{The tabularx package}, or \mylref{272}{The tabulary package}.

Instead of \LaTeXTT{p}, use the \LaTeXTT{m} attribute to have the lines aligned toward the middle of the box or the \LaTeXTT{b} attribute to align along the bottom of the box.

Here is a simple example. The following code creates two tables with the same code; the only difference is that the last column of the second one has a defined width of 5 centimeters, while in the first one we didn\textquotesingle{}t specify any width. Compiling this code:

\begin{Shaded}
\begin{Highlighting}[]

\NormalTok{\textbackslash{}documentclass\{article\} }
\NormalTok{\textbackslash{}usepackage[english]\{babel\}}
 
\NormalTok{\textbackslash{}begin\{document\}}
 
\NormalTok{Without specifying width for last column:}
\NormalTok{\textbackslash{}begin\{center\}}
    \NormalTok{\textbackslash{}begin\{tabular\} l  l  l  l \}}
    \NormalTok{\textbackslash{}hline}
    \NormalTok{Day & Min Temp & Max Temp & Summary \textbackslash{}\textbackslash{} \textbackslash{}hline}
    \NormalTok{Monday & 11C & 22C & A clear day with lots of sunshine.}
    \NormalTok{However, the strong breeze will bring down the temperatures. \textbackslash{}\textbackslash{} \textbackslash{}hline}
    \NormalTok{Tuesday & 9C & 19C & Cloudy with rain, across many northern regions. Clear}
 \NormalTok{spells }
    \NormalTok{across most of Scotland and Northern Ireland, }
    \NormalTok{but rain reaching the far northwest. \textbackslash{}\textbackslash{} \textbackslash{}hline}
    \NormalTok{Wednesday & 10C & 21C & Rain will still linger for the morning. }
    \NormalTok{Conditions will improve by early afternoon and continue }
    \NormalTok{throughout the evening. \textbackslash{}\textbackslash{}}
    \NormalTok{\textbackslash{}hline}
    \NormalTok{\textbackslash{}end\{tabular\}}
\NormalTok{\textbackslash{}end\{center\}}
 
\NormalTok{With width specified:}
\NormalTok{\textbackslash{}begin\{center\}}
    \NormalTok{\textbackslash{}begin\{tabular\}\{  l  l  l  p\{5cm\} \}}
    \NormalTok{\textbackslash{}hline}
    \NormalTok{Day & Min Temp & Max Temp & Summary \textbackslash{}\textbackslash{} \textbackslash{}hline}
    \NormalTok{Monday & 11C & 22C & A clear day with lots of sunshine.  }
    \NormalTok{However, the strong breeze will bring down the temperatures. \textbackslash{}\textbackslash{} \textbackslash{}hline}
    \NormalTok{Tuesday & 9C & 19C & Cloudy with rain, across many northern regions. Clear}
 \NormalTok{spells }
    \NormalTok{across most of Scotland and Northern Ireland, }
    \NormalTok{but rain reaching the far northwest. \textbackslash{}\textbackslash{} \textbackslash{}hline}
    \NormalTok{Wednesday & 10C & 21C & Rain will still linger for the morning. }
    \NormalTok{Conditions will improve by early afternoon and continue }
    \NormalTok{throughout the evening. \textbackslash{}\textbackslash{}}
    \NormalTok{\textbackslash{}hline}
    \NormalTok{\textbackslash{}end\{tabular\}}
\NormalTok{\textbackslash{}end\{center\}}
 
\NormalTok{\textbackslash{}end\{document\}}
\end{Highlighting}
\end{Shaded}


You get the following output:



\begin{minipage}{1.0\linewidth}
\begin{center}
\includegraphics[width=1.0\linewidth,height=6.5in,keepaspectratio]{../images/36.\SVGExtension}
\end{center}
\raggedright{}\myfigurewithoutcaption{36}
\end{minipage}\vspace{0.75cm}



Note that the first table has been cropped, since the output is wider than the page width.
\subsection{Manually broken paragraphs in table cells}
\label{252}
Sometimes it is necessary to not rely on the breaking algorithm when using the \LaTeXTT{p} specifier, but rather specify the line breaks by hand. In this case it is easiest to use a \LaTeXTT{\textbackslash{}parbox}:

\begin{Shaded}
\begin{Highlighting}[]

\NormalTok{\textbackslash{}begin\{tabular\}\{cc\}}
  \NormalTok{boring cell content & \textbackslash{}parbox[t]\{5cm\}\{rather long par\textbackslash{}\textbackslash{}new par\}}
\NormalTok{\textbackslash{}end\{tabular\}}
\end{Highlighting}
\end{Shaded}

\subsection{Space between columns}
\label{253}

To tweak the space between columns (LaTeX will by default choose very tight columns), one can alter the column separation: \LaTeXTT{\textbackslash{}setlength\{\textbackslash{}tabcolsep\}\{5pt\}}.
The default value is 6pt.
\subsection{Space between rows}
\label{254}

Re-{}define the \LaTeXTT{\textbackslash{}arraystretch} command to set the space between rows:
\begin{Shaded}
\begin{Highlighting}[]

\NormalTok{\textbackslash{}renewcommand\{\textbackslash{}arraystretch\}\{1.5\}}
\end{Highlighting}
\end{Shaded}

Default value is 1.0.

An alternative way to adjust the rule spacing is to add \LaTeXTT{\textbackslash{}noalign\{\textbackslash{}smallskip\}} before or after the \LaTeXTT{\textbackslash{}hline} and \LaTeXTT{\textbackslash{}cline\{i-{}j\}} commands:

\begin{Shaded}
\begin{Highlighting}[]

\NormalTok{\textbackslash{}begin\{tabular\}\{  l  l  r  \}}
  \NormalTok{\textbackslash{}hline\textbackslash{}noalign\{\textbackslash{}smallskip\}}
  \NormalTok{\textbackslash{}multicolumn\{2\}\{c\}\{Item\} \textbackslash{}\textbackslash{}}
  \NormalTok{\textbackslash{}cline\{1-2\}\textbackslash{}noalign\{\textbackslash{}smallskip\}}
  \NormalTok{Animal & Description & Price (\textbackslash{}$) \textbackslash{}\textbackslash{}}
  \NormalTok{\textbackslash{}noalign\{\textbackslash{}smallskip\}\textbackslash{}hline\textbackslash{}noalign\{\textbackslash{}smallskip\}}
  \NormalTok{Gnat  & per gram & 13.65 \textbackslash{}\textbackslash{}}
        \NormalTok{& each     &  0.01 \textbackslash{}\textbackslash{}}
  \NormalTok{Gnu   & stuffed  & 92.50 \textbackslash{}\textbackslash{}}
  \NormalTok{Emu   & stuffed  & 33.33 \textbackslash{}\textbackslash{}}
  \NormalTok{Armadillo & frozen & 8.99 \textbackslash{}\textbackslash{}}
  \NormalTok{\textbackslash{}noalign\{\textbackslash{}smallskip\}\textbackslash{}hline}
\NormalTok{\textbackslash{}end\{tabular\}}
\end{Highlighting}
\end{Shaded}


You may also specify the skip after a line explicitly using glue after the line terminator

\begin{Shaded}
\begin{Highlighting}[]

\NormalTok{\textbackslash{}begin\{tabular\}\{ll\}}
\NormalTok{\textbackslash{}hline}
\NormalTok{Mineral & Color \textbackslash{}\textbackslash{}[1cm]}
\NormalTok{Ruby & red \textbackslash{}\textbackslash{}}
\NormalTok{Sapphire & blue \textbackslash{}\textbackslash{}}
\NormalTok{\textbackslash{}hline}
\NormalTok{\textbackslash{}end\{tabular\}}
\end{Highlighting}
\end{Shaded}

\subsection{Other environments inside tables}
\label{255}

If you use some LaTeX environments inside table cells, like \LaTeXTT{verbatim} or \LaTeXTT{enumerate}:

\begin{Shaded}
\begin{Highlighting}[]

\NormalTok{\textbackslash{}begin\{tabular\}\{c c\}}
	\NormalTok{\textbackslash{}hline}
	\NormalTok{\textbackslash{}begin\{verbatim\}}
	\NormalTok{code}
	\NormalTok{\textbackslash{}end\{verbatim\}}
	\NormalTok{& description}
 	\NormalTok{\textbackslash{}\textbackslash{} \textbackslash{}hline}
\NormalTok{\textbackslash{}end\{tabular\}}
\end{Highlighting}
\end{Shaded}


you might encounter errors similar to\\

\TemplateSpaceIndent{$\text{ }${}!$\text{ }${}LaTeX$\text{ }${}Error:$\text{ }${}Something\textquotesingle{}s$\text{ }${}wrong-{}-{}perhaps$\text{ }${}a$\text{ }${}missing$\text{ }${}\textbackslash{}item.}


To solve this problem, change \mylref{272}{column specifier} to \symbol{34}paragraph\symbol{34} (\LaTeXTT{p}, \LaTeXTT{m} or \LaTeXTT{b}).

\begin{Shaded}
\begin{Highlighting}[]

\NormalTok{\textbackslash{}begin\{tabular\}\{m\{5cm\} c\}}
\end{Highlighting}
\end{Shaded}

\subsection{Defining multiple columns}
\label{256}

It is possible to define many identical columns at once using the \LaTeXTT{*\{num\}\{str\}} syntax. This is particularly useful when your table has many columns.

Here is a table with six centered columns flanked by a single column on each side:

\begin{longtable}{p{1.0\linewidth}}
\begin{Shaded}
\begin{Highlighting}[]

\NormalTok{\textbackslash{}begin\{tabular\}\{l*\{6\}\{c\}r\}}
\NormalTok{Team              & P & W & D & L & F  & A & Pts \textbackslash{}\textbackslash{}}
\NormalTok{\textbackslash{}hline}
\NormalTok{Manchester United & 6 & 4 & 0 & 2 & 10 & 5 & 12  \textbackslash{}\textbackslash{}}
\NormalTok{Celtic            & 6 & 3 & 0 & 3 &  8 & 9 &  9  \textbackslash{}\textbackslash{}}
\NormalTok{Benfica           & 6 & 2 & 1 & 3 &  7 & 8 &  7  \textbackslash{}\textbackslash{}}
\NormalTok{FC Copenhagen     & 6 & 2 & 1 & 3 &  5 & 8 &  7  \textbackslash{}\textbackslash{}}
\NormalTok{\textbackslash{}end\{tabular\}}
\end{Highlighting}
\end{Shaded}
\\



\begin{minipage}{1.0\linewidth}
\begin{center}
\includegraphics[width=1.0\linewidth,height=6.5in,keepaspectratio]{../images/37.\SVGExtension}
\end{center}
\raggedright{}\myfigurewithoutcaption{37}
\end{minipage}\vspace{0.75cm}



\end{longtable}
\subsection{Column specification using {\itshape \setmainfont[Path=/usr/share/fonts/truetype/cmu/,UprightFont=cmunrm.ttf,BoldFont=cmunbx.ttf,ItalicFont=cmunti.ttf,BoldItalicFont=cmunbi.ttf]{cmunti.ttf}\setmonofont[Path=/usr/share/fonts/truetype/cmu/,UprightFont=cmuntt.ttf,BoldFont=cmuntb.ttf,ItalicFont=cmunit.ttf,BoldItalicFont=cmuntx.ttf]{cmunti.ttf}\itshape >{}\{\textbackslash{}cmd\}}{$\text{ }$}\setmainfont[Path=/usr/share/fonts/truetype/cmu/,UprightFont=cmunrm.ttf,BoldFont=cmunbx.ttf,ItalicFont=cmunti.ttf,BoldItalicFont=cmunbi.ttf]{cmunrm.ttf}\setmonofont[Path=/usr/share/fonts/truetype/cmu/,UprightFont=cmuntt.ttf,BoldFont=cmuntb.ttf,ItalicFont=cmunit.ttf,BoldItalicFont=cmuntx.ttf]{cmunrm.ttf} and {\itshape \setmainfont[Path=/usr/share/fonts/truetype/cmu/,UprightFont=cmunrm.ttf,BoldFont=cmunbx.ttf,ItalicFont=cmunti.ttf,BoldItalicFont=cmunbi.ttf]{cmunti.ttf}\setmonofont[Path=/usr/share/fonts/truetype/cmu/,UprightFont=cmuntt.ttf,BoldFont=cmuntb.ttf,ItalicFont=cmunit.ttf,BoldItalicFont=cmuntx.ttf]{cmunti.ttf}\itshape <{}\{\textbackslash{}cmd\}}{$\text{ }$}\setmainfont[Path=/usr/share/fonts/truetype/cmu/,UprightFont=cmunrm.ttf,BoldFont=cmunbx.ttf,ItalicFont=cmunti.ttf,BoldItalicFont=cmunbi.ttf]{cmunrm.ttf}\setmonofont[Path=/usr/share/fonts/truetype/cmu/,UprightFont=cmuntt.ttf,BoldFont=cmuntb.ttf,ItalicFont=cmunit.ttf,BoldItalicFont=cmuntx.ttf]{cmunrm.ttf}}
\label{257}

The column specification can be altered using the \LaTeXTT{array} package.  This is done in the 
argument of the tabular environment using \LaTeXTT{>{}\{\textbackslash{}command\}} for commands executed right 
{\itshape \setmainfont[Path=/usr/share/fonts/truetype/cmu/,UprightFont=cmunrm.ttf,BoldFont=cmunbx.ttf,ItalicFont=cmunti.ttf,BoldItalicFont=cmunbi.ttf]{cmunti.ttf}\setmonofont[Path=/usr/share/fonts/truetype/cmu/,UprightFont=cmuntt.ttf,BoldFont=cmuntb.ttf,ItalicFont=cmunit.ttf,BoldItalicFont=cmuntx.ttf]{cmunti.ttf}\itshape before}{$\text{ }$}\setmainfont[Path=/usr/share/fonts/truetype/cmu/,UprightFont=cmunrm.ttf,BoldFont=cmunbx.ttf,ItalicFont=cmunti.ttf,BoldItalicFont=cmunbi.ttf]{cmunrm.ttf}\setmonofont[Path=/usr/share/fonts/truetype/cmu/,UprightFont=cmuntt.ttf,BoldFont=cmuntb.ttf,ItalicFont=cmunit.ttf,BoldItalicFont=cmuntx.ttf]{cmunrm.ttf} each column element and \LaTeXTT{<{}\{\textbackslash{}command\}} for commands to be executed right
{\itshape \setmainfont[Path=/usr/share/fonts/truetype/cmu/,UprightFont=cmunrm.ttf,BoldFont=cmunbx.ttf,ItalicFont=cmunti.ttf,BoldItalicFont=cmunbi.ttf]{cmunti.ttf}\setmonofont[Path=/usr/share/fonts/truetype/cmu/,UprightFont=cmuntt.ttf,BoldFont=cmuntb.ttf,ItalicFont=cmunit.ttf,BoldItalicFont=cmuntx.ttf]{cmunti.ttf}\itshape after}{$\text{ }$}\setmainfont[Path=/usr/share/fonts/truetype/cmu/,UprightFont=cmunrm.ttf,BoldFont=cmunbx.ttf,ItalicFont=cmunti.ttf,BoldItalicFont=cmunbi.ttf]{cmunrm.ttf}\setmonofont[Path=/usr/share/fonts/truetype/cmu/,UprightFont=cmuntt.ttf,BoldFont=cmuntb.ttf,ItalicFont=cmunit.ttf,BoldItalicFont=cmuntx.ttf]{cmunrm.ttf} each column element. 
As an example: to get a column in math mode enter: \LaTeXTT{\textbackslash{}begin\{tabular\}\{>{}\{\${}\}c<{}\{\${}\}\}}. 
Another example is changing the font: \LaTeXTT{\textbackslash{}begin\{tabular\}\{>{}\{\textbackslash{}small\}c\}} to print the column in a small font.

The argument of the \LaTeXTT{>{}} and \LaTeXTT{<{}} specifications must be correctly balanced when it comes to \LaTeXTT{\{} and \LaTeXTT{\}} characters. This means that \LaTeXTT{>{}\{\textbackslash{}bfseries\}} is valid, while \LaTeXTT{>{}\{\textbackslash{}textbf\}} will not work and \LaTeXTT{>{}\{\textbackslash{}textbf\{\}} is not valid. If there is the need to use the text of the table as an argument (for instance, using the \LaTeXTT{\textbackslash{}textbf} to produce bold text), one should use the \LaTeXTT{\textbackslash{}bgroup} and \LaTeXTT{\textbackslash{}egroup} commands: \LaTeXTT{>{}\{\textbackslash{}textbf\textbackslash{}bgroup\}c<{}\{\textbackslash{}egroup\}} produces the intended effect. This works only for some basic LaTeX commands. For other commands, such as \LaTeXTT{\textbackslash{}underline} to underline text, it is necessary to temporarily store the column text in a box using \LaTeXTT{lrbox}. First, you must define such a box with \LaTeXTT{\textbackslash{}newsavebox\{\textbackslash{}boxname\}} and then you can define:

\begin{Shaded}
\begin{Highlighting}[]

\NormalTok{>\{\textbackslash{}begin\{lrbox\}\{\textbackslash{}boxname\} \}}\CommentTok
\NormalTok{<\{\textbackslash{}end\{lrbox\}}\CommentTok
 \NormalTok{\}}
\end{Highlighting}
\end{Shaded}


This stores the text in a box and afterwards, takes the text out of the box with \LaTeXTT{\textbackslash{}unhbox} (this destroys the box, if the box is needed again one should use \LaTeXTT{\textbackslash{}unhcopy} instead) and passing it to \LaTeXTT{\textbackslash{}underline}. (For LaTeX2e, you may want to use \LaTeXTT{\textbackslash{}usebox\{\textbackslash{}boxname\}} instead of \LaTeXTT{\textbackslash{}unhbox\textbackslash{}boxname}.)

This same trick done with \LaTeXTT{\textbackslash{}raisebox} instead of \LaTeXTT{\textbackslash{}underline} can force all lines in a table to have equal height, instead of the natural varying height that can occur when e.g. math terms or superscripts occur in the text.

Here is an example showing the use of both \LaTeXTT{p\{...\}} and \LaTeXTT{>{}\{\textbackslash{}centering\}} :

\begin{Shaded}
\begin{Highlighting}[]

\NormalTok{\textbackslash{}begin\{tabular\}\{>\{\textbackslash{}centering\}p\{3.5cm\}<\{\textbackslash{}centering\}p\{3.5cm\} \}}
\NormalTok{Geometry  & Algebra}
\NormalTok{\textbackslash{}tabularnewline}
\NormalTok{\textbackslash{}hline}
 \NormalTok{Points & Addition }
\NormalTok{\textbackslash{}tabularnewline}
 \NormalTok{Spheres & Multiplication }
\NormalTok{\textbackslash{}end\{tabular\}}
\end{Highlighting}
\end{Shaded}


Note the use of \LaTeXTT{\textbackslash{}tabularnewline} instead of \LaTeXTT{\textbackslash{}\textbackslash{}} to avoid a \LaTeXTT{Misplaced \textbackslash{}noalign} error.
\subsection{@-{}expressions}
\label{258}

The column separator can be specified with the \LaTeXTT{@\{...\}} construct. 

It typically takes some text as its argument, and when appended to a column, it will automatically insert that text into each cell in that column before the actual data for that cell. This command kills the inter-{}column space and replaces it with whatever is between the curly braces. To add space, use \LaTeXTT{@\{\textbackslash{}hspace\{\textquotesingle{}\textquotesingle{}width\textquotesingle{}\textquotesingle{}\}\}}.

Admittedly, this is not that clear, and so will require a few examples to clarify. Sometimes, it is desirable in scientific tables to have the numbers aligned on the decimal point. This can be achieved by doing the following:

\begin{longtable}{p{1.0\linewidth}}
\begin{Shaded}
\begin{Highlighting}[]

\NormalTok{\textbackslash{}begin\{tabular\}\{r@\{.\}l\}}
  \NormalTok{3   & 14159 \textbackslash{}\textbackslash{}}
  \NormalTok{16  & 2     \textbackslash{}\textbackslash{}}
  \NormalTok{123 & 456   \textbackslash{}\textbackslash{}}
\NormalTok{\textbackslash{}end\{tabular\}}
\end{Highlighting}
\end{Shaded}
\\

{$\begin{aligned}3&.14159\\16&.2\\123&.456\end{aligned}$}

\end{longtable}

The space-{}suppressing qualities of the @-{}expression actually make it quite useful for manipulating the horizontal spacing between columns. Given a basic table, and varying the column descriptions:

\begin{Shaded}
\begin{Highlighting}[]

\NormalTok{\textbackslash{}begin\{tabular\}\{ ll \}}
  \NormalTok{\textbackslash{}hline}
  \NormalTok{stuff & stuff \textbackslash{}\textbackslash{} \textbackslash{}hline}
  \NormalTok{stuff & stuff \textbackslash{}\textbackslash{}}
  \NormalTok{\textbackslash{}hline}
\NormalTok{\textbackslash{}end\{tabular\}}
\end{Highlighting}
\end{Shaded}


\begin{longtable}{>{\RaggedRight}p{0.27291\linewidth}>{\RaggedRight}p{0.66995\linewidth}} 
\hspace*{0pt}\ignorespaces{}\hspace*{0pt} {\ttfamily \setmainfont[Path=/usr/share/fonts/truetype/cmu/,UprightFont=cmunrm.ttf,BoldFont=cmunbx.ttf,ItalicFont=cmunti.ttf,BoldItalicFont=cmunbi.ttf]{cmuntt.ttf}\setmonofont[Path=/usr/share/fonts/truetype/cmu/,UprightFont=cmuntt.ttf,BoldFont=cmuntb.ttf,ItalicFont=cmunit.ttf,BoldItalicFont=cmuntx.ttf]{cmuntt.ttf}\ttfamily \{|l|l|\}}&\hspace*{0pt}\ignorespaces{}\hspace*{0pt}\setmainfont[Path=/usr/share/fonts/truetype/cmu/,UprightFont=cmunrm.ttf,BoldFont=cmunbx.ttf,ItalicFont=cmunti.ttf,BoldItalicFont=cmunbi.ttf]{cmunrm.ttf}\setmonofont[Path=/usr/share/fonts/truetype/cmu/,UprightFont=cmuntt.ttf,BoldFont=cmuntb.ttf,ItalicFont=cmunit.ttf,BoldItalicFont=cmuntx.ttf]{cmunrm.ttf}\begin{minipage}{1.0\linewidth}\begin{center}\includegraphics[width=1.0\linewidth,height=6.5in,keepaspectratio]{../images/38.\SVGExtension}\end{center}\myfigurewithoutcaption{38}\end{minipage}\\ \hspace*{0pt}\ignorespaces{}\hspace*{0pt} {\ttfamily \setmainfont[Path=/usr/share/fonts/truetype/cmu/,UprightFont=cmunrm.ttf,BoldFont=cmunbx.ttf,ItalicFont=cmunti.ttf,BoldItalicFont=cmunbi.ttf]{cmuntt.ttf}\setmonofont[Path=/usr/share/fonts/truetype/cmu/,UprightFont=cmuntt.ttf,BoldFont=cmuntb.ttf,ItalicFont=cmunit.ttf,BoldItalicFont=cmuntx.ttf]{cmuntt.ttf}\ttfamily \{|@\{\}l|l@\{\}|\}}&\hspace*{0pt}\ignorespaces{}\hspace*{0pt}\setmainfont[Path=/usr/share/fonts/truetype/cmu/,UprightFont=cmunrm.ttf,BoldFont=cmunbx.ttf,ItalicFont=cmunti.ttf,BoldItalicFont=cmunbi.ttf]{cmunrm.ttf}\setmonofont[Path=/usr/share/fonts/truetype/cmu/,UprightFont=cmuntt.ttf,BoldFont=cmuntb.ttf,ItalicFont=cmunit.ttf,BoldItalicFont=cmuntx.ttf]{cmunrm.ttf}\begin{minipage}{1.0\linewidth}\begin{center}\includegraphics[width=1.0\linewidth,height=6.5in,keepaspectratio]{../images/39.\SVGExtension}\end{center}\myfigurewithoutcaption{39}\end{minipage}\\ \hspace*{0pt}\ignorespaces{}\hspace*{0pt} {\ttfamily \setmainfont[Path=/usr/share/fonts/truetype/cmu/,UprightFont=cmunrm.ttf,BoldFont=cmunbx.ttf,ItalicFont=cmunti.ttf,BoldItalicFont=cmunbi.ttf]{cmuntt.ttf}\setmonofont[Path=/usr/share/fonts/truetype/cmu/,UprightFont=cmuntt.ttf,BoldFont=cmuntb.ttf,ItalicFont=cmunit.ttf,BoldItalicFont=cmuntx.ttf]{cmuntt.ttf}\ttfamily \{|@\{\}l@\{\}|l@\{\}|\}}&\hspace*{0pt}\ignorespaces{}\hspace*{0pt}\setmainfont[Path=/usr/share/fonts/truetype/cmu/,UprightFont=cmunrm.ttf,BoldFont=cmunbx.ttf,ItalicFont=cmunti.ttf,BoldItalicFont=cmunbi.ttf]{cmunrm.ttf}\setmonofont[Path=/usr/share/fonts/truetype/cmu/,UprightFont=cmuntt.ttf,BoldFont=cmuntb.ttf,ItalicFont=cmunit.ttf,BoldItalicFont=cmuntx.ttf]{cmunrm.ttf}\begin{minipage}{1.0\linewidth}\begin{center}\includegraphics[width=1.0\linewidth,height=6.5in,keepaspectratio]{../images/40.\SVGExtension}\end{center}\myfigurewithoutcaption{40}\end{minipage}\\ \hspace*{0pt}\ignorespaces{}\hspace*{0pt} {\ttfamily \setmainfont[Path=/usr/share/fonts/truetype/cmu/,UprightFont=cmunrm.ttf,BoldFont=cmunbx.ttf,ItalicFont=cmunti.ttf,BoldItalicFont=cmunbi.ttf]{cmuntt.ttf}\setmonofont[Path=/usr/share/fonts/truetype/cmu/,UprightFont=cmuntt.ttf,BoldFont=cmuntb.ttf,ItalicFont=cmunit.ttf,BoldItalicFont=cmuntx.ttf]{cmuntt.ttf}\ttfamily \{|@\{\}l@\{\}|@\{\}l@\{\}|\}}&\hspace*{0pt}\ignorespaces{}\hspace*{0pt}\setmainfont[Path=/usr/share/fonts/truetype/cmu/,UprightFont=cmunrm.ttf,BoldFont=cmunbx.ttf,ItalicFont=cmunti.ttf,BoldItalicFont=cmunbi.ttf]{cmunrm.ttf}\setmonofont[Path=/usr/share/fonts/truetype/cmu/,UprightFont=cmuntt.ttf,BoldFont=cmuntb.ttf,ItalicFont=cmunit.ttf,BoldItalicFont=cmuntx.ttf]{cmunrm.ttf}\begin{minipage}{1.0\linewidth}\begin{center}\includegraphics[width=1.0\linewidth,height=6.5in,keepaspectratio]{../images/41.\SVGExtension}\end{center}\myfigurewithoutcaption{41}\end{minipage} 
\end{longtable}

\subsection{Aligning columns at decimal points using dcolumn}
\label{259}
Instead of using @-{}expressions to build columns of decimals aligned to the decimal point (or equivalent symbol), it is possible to center a column on the decimal separator using the \LaTeXTT{dcolumn} package, which provides a new column specifier for floating point data. See the \myhref{http://anorien.csc.warwick.ac.uk/mirrors/CTAN/macros/latex/required/tools/dcolumn.pdf}{dcolumn package documentation} for more information, but a simple way to use \LaTeXTT{dcolumn} is as follows.
\begin{longtable}{p{1.0\linewidth}}
\begin{Shaded}
\begin{Highlighting}[]

\NormalTok{\textbackslash{}usepackage\{dcolumn\}}
\NormalTok{\textbackslash{}ldots}
\NormalTok{\textbackslash{}newcolumntype\{d\}[1]\{D\{.\}\{\textbackslash{}cdot\}\{#1\} \}}
\CommentTok{%the argument for d specifies the maximum number of decimal places}
\NormalTok{\textbackslash{}begin\{tabular\}\{l r c d\{1\} \}}
\NormalTok{Left&Right&Center&\textbackslash{}mathrm\{Decimal\}\textbackslash{}\textbackslash{}}
\NormalTok{1&2&3&4\textbackslash{}\textbackslash{}}
\NormalTok{11&22&33&44\textbackslash{}\textbackslash{}}
\NormalTok{1.1&2.2&3.3&4.4\textbackslash{}\textbackslash{}}
\NormalTok{\textbackslash{}end\{tabular\}}
\end{Highlighting}
\end{Shaded}
\\


\begin{minipage}{0.50000\textwidth}
\begin{center}
\includegraphics[width=1.0\textwidth,height=6.5in,keepaspectratio]{../images/42.png}
\end{center}
\raggedright{}\myfigurewithoutcaption{42}
\end{minipage}\vspace{0.75cm}



\end{longtable}
A negative argument provided for the number of decimal places in the new column type allows unlimited decimal places, but may result in rather wide columns.  Rounding is not applied, so the data to be tabulated should be adjusted to the number of decimal places specified.  Note that a decimal aligned column is typeset in math mode, hence the use of \textbackslash{}mathrm for the column heading in the example above.  Also, text in a decimal aligned column (for example the header) will be right-{}aligned before the decimal separator (assuming there\textquotesingle{}s no decimal separator in the text).  While this may be fine for very short text, or numeric column headings, it looks cumbersome in the example above.  A solution to this is to use the \LaTeXTT{\textbackslash{}multicolumn} command described below, specifying a single column and its alignment. For example to center the header {\itshape \setmainfont[Path=/usr/share/fonts/truetype/cmu/,UprightFont=cmunrm.ttf,BoldFont=cmunbx.ttf,ItalicFont=cmunti.ttf,BoldItalicFont=cmunbi.ttf]{cmunti.ttf}\setmonofont[Path=/usr/share/fonts/truetype/cmu/,UprightFont=cmuntt.ttf,BoldFont=cmuntb.ttf,ItalicFont=cmunit.ttf,BoldItalicFont=cmuntx.ttf]{cmunti.ttf}\itshape Decimal}{$\text{ }$}\setmainfont[Path=/usr/share/fonts/truetype/cmu/,UprightFont=cmunrm.ttf,BoldFont=cmunbx.ttf,ItalicFont=cmunti.ttf,BoldItalicFont=cmunbi.ttf]{cmunrm.ttf}\setmonofont[Path=/usr/share/fonts/truetype/cmu/,UprightFont=cmuntt.ttf,BoldFont=cmuntb.ttf,ItalicFont=cmunit.ttf,BoldItalicFont=cmuntx.ttf]{cmunrm.ttf} over its column in the above example, the first line of the table itself would be \LaTeXTT{Left\&Right\&Center\&\textbackslash{}multicolumn\{1\}\{c\}\{Decimal\}\textbackslash{}\textbackslash{}}
\subsubsection{Bold text and dcolumn}
\label{260}
To draw attention to particular entries in a table, it may be nice to use bold text.  Ordinarily this is easy, but as dcolumn needs to {\itshape \setmainfont[Path=/usr/share/fonts/truetype/cmu/,UprightFont=cmunrm.ttf,BoldFont=cmunbx.ttf,ItalicFont=cmunti.ttf,BoldItalicFont=cmunbi.ttf]{cmunti.ttf}\setmonofont[Path=/usr/share/fonts/truetype/cmu/,UprightFont=cmuntt.ttf,BoldFont=cmuntb.ttf,ItalicFont=cmunit.ttf,BoldItalicFont=cmuntx.ttf]{cmunti.ttf}\itshape see}{$\text{ }$}\setmainfont[Path=/usr/share/fonts/truetype/cmu/,UprightFont=cmunrm.ttf,BoldFont=cmunbx.ttf,ItalicFont=cmunti.ttf,BoldItalicFont=cmunbi.ttf]{cmunrm.ttf}\setmonofont[Path=/usr/share/fonts/truetype/cmu/,UprightFont=cmuntt.ttf,BoldFont=cmuntb.ttf,ItalicFont=cmunit.ttf,BoldItalicFont=cmuntx.ttf]{cmunrm.ttf} the decimal point it is rather harder to do.  In addition, the usual bold characters are wider than their normal counterparts, meaning that although the decimals may align nicely, the figures (for more than 2-{}-{}3 digits on one side of the decimal point) will be visibly misaligned.  It is however possible to use normal width bold characters and define a new bold column type, as shown below.\myfootnote{\myhref{}{Decimals in table don\textquotesingle{}t align with dcolumn when bolded}. Stackexchange. Retrieved  }

\begin{longtable}{p{1.0\linewidth}}
\begin{Shaded}
\begin{Highlighting}[]

\NormalTok{\textbackslash{}usepackage\{dcolumn\}}
\CommentTok{%here we're setting up a version of the math fonts with normal x-width}
\NormalTok{\textbackslash{}DeclareMathVersion\{nxbold\} }
\NormalTok{\textbackslash{}SetSymbolFont\{operators\}\{nxbold\}\{OT1\}\{cmr\} \{b\}\{n\}}
\NormalTok{\textbackslash{}SetSymbolFont\{letters\}  \{nxbold\}\{OML\}\{cmm\} \{b\}\{it\}}
\NormalTok{\textbackslash{}SetSymbolFont\{symbols\}  \{nxbold\}\{OMS\}\{cmsy\}\{b\}\{n\}}
 
\NormalTok{\textbackslash{}begin\{document\}}
\NormalTok{\textbackslash{}makeatletter}
\NormalTok{\textbackslash{}newcolumntype\{d\}\{D\{.\}\{.\}\{-1\} \} }\CommentTok{%decimal column as before}
\CommentTok{%wide bold decimal column}
\NormalTok{\textbackslash{}newcolumntype\{B\}[3]\{>\{\textbackslash{}boldmath\textbackslash{}DC@\{#1\}\{#2\}\{#3\} \}c<\{\textbackslash{}DC@end\} \} }
\CommentTok{%normal width bold decimal column}
\NormalTok{\textbackslash{}newcolumntype\{Z\}[3]\{>\{\textbackslash{}mathversion\{nxbold\}\textbackslash{}DC@\{#1\}\{#2\}\{#3\} \}c<\{\textbackslash{}DC@end\} \} }
\NormalTok{\textbackslash{}makeatother}
\NormalTok{\textbackslash{}begin\{tabular\}\{l l d\}}
    \NormalTok{Type &M & \textbackslash{}multicolumn\{1\}\{c\}\{N\} \textbackslash{}\textbackslash{}}
    \NormalTok{Normal & 1 & 22222.222 \textbackslash{}\textbackslash{}}
    \NormalTok{Bold (standard)&10 & \textbackslash{}multicolumn\{1\}\{B\{.\}\{.\}\{-1\} \}\{22222.222\}\textbackslash{}\textbackslash{}}
    \NormalTok{Bold (nxbold)&100 & \textbackslash{}multicolumn\{1\}\{Z\{.\}\{.\}\{-1\} \}\{22222.222\}\textbackslash{}\textbackslash{}}
\NormalTok{\textbackslash{}end\{tabular\}}
\NormalTok{\textbackslash{}end\{document\}}
\end{Highlighting}
\end{Shaded}
\\


\begin{minipage}{0.50000\textwidth}
\begin{center}
\includegraphics[width=1.0\textwidth,height=6.5in,keepaspectratio]{../images/43.png}
\end{center}
\raggedright{}\myfigurewithoutcaption{43}
\end{minipage}\vspace{0.75cm}



\end{longtable}
\section{Row specification}
\label{261} 

It might be convenient to apply the same command over every cell of a row, just as for column. Unfortunately the \LaTeXTT{tabular} environment cannot do that by default.
We will need \LaTeXTT{tabu} instead, which provides the \LaTeXTT{\textbackslash{}rowfont} option.

\begin{Shaded}
\begin{Highlighting}[]

\NormalTok{\textbackslash{}begin\{tabular\}\{XX\}}
\NormalTok{\textbackslash{}rowfont\{\textbackslash{}bfseries\textbackslash{}itshape\textbackslash{}large\} Header1 & Header2 \textbackslash{}\textbackslash{}}
\NormalTok{\textbackslash{}hline}
\NormalTok{Cell2 & Cell2}
\NormalTok{\textbackslash{}end\{tabular\}}
\end{Highlighting}
\end{Shaded}

\section{Spanning}
\label{262}

To complete this tutorial, we take a quick look at how to generate slightly more complex tables. Unsurprisingly, the commands necessary have to be embedded within the table data itself.
\subsection{Rows spanning multiple columns}
\label{263}

The command for this looks like this: \LaTeXTT{\textbackslash{}multicolumn\{num\_cols\}\{alignment\}\{contents\}}. \LaTeXTT{num\_cols} is the number of subsequent columns to merge; \LaTeXTT{alignment} is either \LaTeXTT{l}, \LaTeXTT{c}, \LaTeXTT{r}, or to have text wrapping specify a width \LaTeXTT{p\{5.0cm\}} . And \LaTeXTT{contents} is simply the actual data you want to be contained within that cell. A simple example:

\begin{longtable}{p{1.0\linewidth}}
\begin{Shaded}
\begin{Highlighting}[]

\NormalTok{\textbackslash{}begin\{tabular\}\{ ll \}}
  \NormalTok{\textbackslash{}hline}
  \NormalTok{\textbackslash{}multicolumn\{2\}\{Team sheet\} \textbackslash{}\textbackslash{}}
  \NormalTok{\textbackslash{}hline}
  \NormalTok{GK & Paul Robinson \textbackslash{}\textbackslash{}}
  \NormalTok{LB & Lucas Radebe \textbackslash{}\textbackslash{}}
  \NormalTok{DC & Michael Duberry \textbackslash{}\textbackslash{}}
  \NormalTok{DC & Dominic Matteo \textbackslash{}\textbackslash{}}
  \NormalTok{RB & Dider Domi \textbackslash{}\textbackslash{}}
  \NormalTok{MC & David Batty \textbackslash{}\textbackslash{}}
  \NormalTok{MC & Eirik Bakke \textbackslash{}\textbackslash{}}
  \NormalTok{MC & Jody Morris \textbackslash{}\textbackslash{}}
  \NormalTok{FW & Jamie McMaster \textbackslash{}\textbackslash{}}
  \NormalTok{ST & Alan Smith \textbackslash{}\textbackslash{}}
  \NormalTok{ST & Mark Viduka \textbackslash{}\textbackslash{}}
  \NormalTok{\textbackslash{}hline}
\NormalTok{\textbackslash{}end\{tabular\}}
\end{Highlighting}
\end{Shaded}
\\



\begin{minipage}{1.0\linewidth}
\begin{center}
\includegraphics[width=1.0\linewidth,height=6.5in,keepaspectratio]{../images/44.\SVGExtension}
\end{center}
\raggedright{}\myfigurewithoutcaption{44}
\end{minipage}\vspace{0.75cm}



\end{longtable}
\subsection{Columns spanning multiple rows}
\label{264}

The first thing you need to do is add \LaTeXTT{\textbackslash{}usepackage\{multirow\}} to the preamble\myfootnote{Package multirow \myfnhref{http://www.ctan.org/tex-archive/macros/latex/contrib/multirow/}{on CTAN}}. This then provides the command needed for spanning rows: \LaTeXTT{\textbackslash{}multirow\{\textquotesingle{}\textquotesingle{}num\_rows\textquotesingle{}\textquotesingle{}\}\{\textquotesingle{}\textquotesingle{}width\textquotesingle{}\textquotesingle{}\}\{\textquotesingle{}\textquotesingle{}contents\textquotesingle{}\textquotesingle{}\}}. The arguments are pretty simple to deduce (\LaTeXTT{*} for the {\itshape \setmainfont[Path=/usr/share/fonts/truetype/cmu/,UprightFont=cmunrm.ttf,BoldFont=cmunbx.ttf,ItalicFont=cmunti.ttf,BoldItalicFont=cmunbi.ttf]{cmunti.ttf}\setmonofont[Path=/usr/share/fonts/truetype/cmu/,UprightFont=cmuntt.ttf,BoldFont=cmuntb.ttf,ItalicFont=cmunit.ttf,BoldItalicFont=cmuntx.ttf]{cmunti.ttf}\itshape width}{$\text{ }$}\setmainfont[Path=/usr/share/fonts/truetype/cmu/,UprightFont=cmunrm.ttf,BoldFont=cmunbx.ttf,ItalicFont=cmunti.ttf,BoldItalicFont=cmunbi.ttf]{cmunrm.ttf}\setmonofont[Path=/usr/share/fonts/truetype/cmu/,UprightFont=cmuntt.ttf,BoldFont=cmuntb.ttf,ItalicFont=cmunit.ttf,BoldItalicFont=cmuntx.ttf]{cmunrm.ttf} means the content\textquotesingle{}s natural width).

\begin{longtable}{p{1.0\linewidth}}
\begin{Shaded}
\begin{Highlighting}[]

\NormalTok{...}
\NormalTok{\textbackslash{}usepackage\{multirow\}}
\NormalTok{...}
 
\NormalTok{\textbackslash{}begin\{tabular\}\{ lll \}}
\NormalTok{\textbackslash{}hline}
\NormalTok{\textbackslash{}multicolumn\{3\}\{ c \}\{Team sheet\} \textbackslash{}\textbackslash{}}
\NormalTok{\textbackslash{}hline}
\NormalTok{Goalkeeper & GK & Paul Robinson \textbackslash{}\textbackslash{} \textbackslash{}hline}
\NormalTok{\textbackslash{}multirow\{4\}\{*\}\{Defenders\} & LB & Lucas Radebe \textbackslash{}\textbackslash{}}
 \NormalTok{& DC & Michael Duburry \textbackslash{}\textbackslash{}}
 \NormalTok{& DC & Dominic Matteo \textbackslash{}\textbackslash{}}
 \NormalTok{& RB & Didier Domi \textbackslash{}\textbackslash{} \textbackslash{}hline}
\NormalTok{\textbackslash{}multirow\{3\}\{*\}\{Midfielders\} & MC & David Batty \textbackslash{}\textbackslash{}}
 \NormalTok{& MC & Eirik Bakke \textbackslash{}\textbackslash{}}
 \NormalTok{& MC & Jody Morris \textbackslash{}\textbackslash{} \textbackslash{}hline}
\NormalTok{Forward & FW & Jamie McMaster \textbackslash{}\textbackslash{} \textbackslash{}hline}
\NormalTok{\textbackslash{}multirow\{2\}\{*\}\{Strikers\} & ST & Alan Smith \textbackslash{}\textbackslash{}}
 \NormalTok{& ST & Mark Viduka \textbackslash{}\textbackslash{}}
\NormalTok{\textbackslash{}hline}
\NormalTok{\textbackslash{}end\{tabular\}}
\end{Highlighting}
\end{Shaded}
\\



\begin{minipage}{1.0\linewidth}
\begin{center}
\includegraphics[width=1.0\linewidth,height=6.5in,keepaspectratio]{../images/45.\SVGExtension}
\end{center}
\raggedright{}\myfigurewithoutcaption{45}
\end{minipage}\vspace{0.75cm}



\end{longtable}

The main thing to note when using \LaTeXTT{\textbackslash{}multirow} is that a blank entry must be inserted for each appropriate cell in each subsequent row to be spanned.

If there is no data for a cell, just don\textquotesingle{}t type anything, but you still need the \symbol{34}\&\symbol{34} separating it from the next column\textquotesingle{}s data. The astute reader will already have deduced that for a table of {$n$} columns, there must always be {$n-1$} ampersands in each row (unless \LaTeXTT{\textbackslash{}multicolumn} is also used).
\subsection{Spanning in both directions simultaneously}
\label{265}

Here is a nontrivial example of how to use spanning in both directions simultaneously and have the borders of the cells drawn correctly:

\begin{longtable}{p{1.0\linewidth}}
\begin{Shaded}
\begin{Highlighting}[]

\NormalTok{\textbackslash{}usepackage\{multirow\}}
 
\NormalTok{\textbackslash{}begin\{tabular\}\{ccccccl\}}
\NormalTok{\textbackslash{}cline\{3-6\}}
\NormalTok{& & \textbackslash{}multicolumn\{4\}\{ c \}\{Primes\} \textbackslash{}\textbackslash{} \textbackslash{}cline\{3-6\}}
\NormalTok{& & 2 & 3 & 5 & 7 \textbackslash{}\textbackslash{} \textbackslash{}cline\{1-6\}}
\NormalTok{\textbackslash{}multicolumn\{1\}\{ c  \}\{\textbackslash{}multirow\{2\}\{*\}\{Powers\} \} &}
\NormalTok{\textbackslash{}multicolumn\{1\}\{ c \}\{504\} & 3 & 2 & 0 & 1 &     \textbackslash{}\textbackslash{} \textbackslash{}cline\{2-6\}}
\NormalTok{\textbackslash{}multicolumn\{1\}\{ c  \}\{\}                        &}
\NormalTok{\textbackslash{}multicolumn\{1\}\{ c \}\{540\} & 2 & 3 & 1 & 0 &     \textbackslash{}\textbackslash{} \textbackslash{}cline\{1-6\}}
\NormalTok{\textbackslash{}multicolumn\{1\}\{ c  \}\{\textbackslash{}multirow\{2\}\{*\}\{Powers\} \} &}
\NormalTok{\textbackslash{}multicolumn\{1\}\{ c \}\{gcd\} & 2 & 2 & 0 & 0 & min \textbackslash{}\textbackslash{} \textbackslash{}cline\{2-6\}}
\NormalTok{\textbackslash{}multicolumn\{1\}\{ c  \}\{\}                        &}
\NormalTok{\textbackslash{}multicolumn\{1\}\{ c \}\{lcm\} & 3 & 3 & 1 & 1 & max \textbackslash{}\textbackslash{} \textbackslash{}cline\{1-6\}}
\NormalTok{\textbackslash{}end\{tabular\}}
\end{Highlighting}
\end{Shaded}
\\



\begin{minipage}{0.75000\textwidth}
\begin{center}
\includegraphics[width=1.0\textwidth,height=6.5in,keepaspectratio]{../images/46.\SVGExtension}
\end{center}
\raggedright{}\myfigurewithoutcaption{46}
\end{minipage}\vspace{0.75cm}



\end{longtable}

The command \LaTeXTT{\textbackslash{}multicolumn\{1\}\{} is just used to draw vertical borders both on the left and on the right of the cell. Even when combined with \LaTeXTT{\textbackslash{}multirow\{2\}\{*\}\{...\}}, it still draws vertical borders that only span the first row. To compensate for that, we add \LaTeXTT{\textbackslash{}multicolumn\{1\}\{} in the following rows spanned by the multirow. Note that we cannot just use \LaTeXTT{\textbackslash{}hline} to draw horizontal lines, since we do not want the line to be drawn over the text that spans several rows. Instead we use the command \LaTeXTT{\textbackslash{}cline\{2-{}6\}} and opt out the first column that contains the text \symbol{34}Powers\symbol{34}.

Here is another example exploiting the same ideas to make
the familiar and popular \symbol{34}2x2\symbol{34} or double dichotomy:

\begin{longtable}{p{1.0\linewidth}}
\begin{Shaded}
\begin{Highlighting}[]

\NormalTok{\textbackslash{}begin\{tabular\}\{ rcc \}}
\NormalTok{\textbackslash{}multicolumn\{1\}\{r\}\{\}}
 \NormalTok{&  \textbackslash{}multicolumn\{1\}\{c\}\{noninteractive\}}
 \NormalTok{& \textbackslash{}multicolumn\{1\}\{c\}\{interactive\} \textbackslash{}\textbackslash{}}
\NormalTok{\textbackslash{}cline\{2-3\}}
\NormalTok{massively multiple & Library & University \textbackslash{}\textbackslash{}}
\NormalTok{\textbackslash{}cline\{2-3\}}
\NormalTok{one-to-one & Book & Tutor \textbackslash{}\textbackslash{}}
\NormalTok{\textbackslash{}cline\{2-3\}}
\NormalTok{\textbackslash{}end\{tabular\}}
\end{Highlighting}
\end{Shaded}
\\



\begin{minipage}{1.0\linewidth}
\begin{center}
\includegraphics[width=1.0\linewidth,height=6.5in,keepaspectratio]{../images/47.\SVGExtension}
\end{center}
\raggedright{}\myfigurewithoutcaption{47}
\end{minipage}\vspace{0.75cm}



\end{longtable}
\section{Controlling table size}
\label{266}
\subsection{Resize tables}
\label{267}

The \LaTeXTT{graphicx} packages features the command \LaTeXTT{\textbackslash{}resizebox\{width\}\{height\}\{object\}} which can be used with \LaTeXTT{tabular} to specify the height and width of a table. The following example shows how to resize a table to 8cm width while maintaining the original width/height ratio.

\begin{Shaded}
\begin{Highlighting}[]

\NormalTok{\textbackslash{}usepackage\{graphicx\}}
\CommentTok{% ...}
 
\NormalTok{\textbackslash{}resizebox\{8cm\}\{!\} \{}
  \NormalTok{\textbackslash{}begin\{tabular\}...}
  \NormalTok{\textbackslash{}end\{tabular\}}
\NormalTok{\}}
\end{Highlighting}
\end{Shaded}


Resizing table including the caption

\begin{Shaded}
\begin{Highlighting}[]

\NormalTok{\textbackslash{}begin\{table\}[h]}
\NormalTok{\textbackslash{}resizebox\{1.4\textbackslash{}textwidth\}\{!\}\{\textbackslash{}begin\{minipage\}\{\textbackslash{}textwidth\}}
\NormalTok{\textbackslash{}begin\{tabular\}\{rcc\}}
\NormalTok{& \textbackslash{}multicolumn\{1\}\{c\}\{noninteractive\}}
\NormalTok{& \textbackslash{}multicolumn\{1\}\{c\}\{interactive\} \textbackslash{}\textbackslash{}}
\NormalTok{\textbackslash{}cline\{2-3\}}
\NormalTok{massively multiple & Library & University \textbackslash{}\textbackslash{}}
\NormalTok{\textbackslash{}cline\{2-3\}}
\NormalTok{one-to-one & Book & Tutor \textbackslash{}\textbackslash{}}
\NormalTok{\textbackslash{}cline\{2-3\}}
\NormalTok{\textbackslash{}end\{tabular\}}
\NormalTok{\textbackslash{}caption[Table caption text]\{Table taken from \textbackslash{}cite[p.10]\{refid\} \}}
\NormalTok{\textbackslash{}label\{table:name\}}
\NormalTok{\textbackslash{}end\{minipage\} \}}
\NormalTok{\textbackslash{}end\{table\}}
\end{Highlighting}
\end{Shaded}


Alternatively you can use \LaTeXTT{\textbackslash{}scalebox\{ratio\}\{object\}} in the same way but with ratios rather than fixed sizes:

\begin{Shaded}
\begin{Highlighting}[]

\NormalTok{\textbackslash{}usepackage\{graphicx\}}
\CommentTok{% ...}
 
\NormalTok{\textbackslash{}scalebox\{0.7\}\{}
  \NormalTok{\textbackslash{}begin\{tabular\}...}
  \NormalTok{\textbackslash{}end\{tabular\}}
\NormalTok{\}}
\end{Highlighting}
\end{Shaded}

\subsection{Changing font size}
\label{268}

A table can be globally switched to a different font size by simply adding the desired size command (here: \LaTeXTT{\textbackslash{}footnotesize}) in the table scope, which may be after the \LaTeXTT{\textbackslash{}begin\{table\}} statement if you use floats, otherwise you need to add a group delimiter.

\begin{Shaded}
\begin{Highlighting}[]

\NormalTok{\{\textbackslash{}footnotesize}
  \NormalTok{\textbackslash{}begin\{tabular\} r  r  c  c  c \}}
      \CommentTok{% ...}
  \NormalTok{\textbackslash{}end\{tabular\}}
\NormalTok{\}}
\end{Highlighting}
\end{Shaded}


\begin{Shaded}
\begin{Highlighting}[]

\NormalTok{\textbackslash{}begin\{table\}[h]\textbackslash{}footnotesize}
  \NormalTok{\textbackslash{}caption\{Performance at peak F-measure\}}
  \NormalTok{\textbackslash{}begin\{tabular\} r  r  c  c  c \}}
      \CommentTok{% ...}
  \NormalTok{\textbackslash{}end\{tabular\}}
\NormalTok{\textbackslash{}end\{table\}}
\end{Highlighting}
\end{Shaded}


Alternatively, you can change the default font for all the tables in your document by placing the following code in the preamble:

\begin{Shaded}
\begin{Highlighting}[]

\NormalTok{\textbackslash{}let\textbackslash{}oldtabular\textbackslash{}tabular }
\NormalTok{\textbackslash{}renewcommand\{\textbackslash{}tabular\}\{\textbackslash{}footnotesize\textbackslash{}oldtabular\} }
\end{Highlighting}
\end{Shaded}


See \mylref{176}{Fonts} for named font sizes.  The table caption font size is not affected. To control the caption font size, see \mylref{379}{Caption Styles}.
\section{Colors}
\label{269}\subsection{Alternate row colors in tables}
\label{270}

The \LaTeXTT{xcolor} package provides the necessary commands to produce tables with alternate row colors, when loaded with the \LaTeXTT{table} option.
The command \LaTeXTT{\textbackslash{}rowcolors\{<{}\textquotesingle{}\textquotesingle{}starting row\textquotesingle{}\textquotesingle{}>{}\}\{<{}\textquotesingle{}\textquotesingle{}odd color\textquotesingle{}\textquotesingle{}>{}\}\{<{}\textquotesingle{}\textquotesingle{}even color\textquotesingle{}\textquotesingle{}>{}\}} has to be specified right before the \LaTeXTT{tabular} environment starts.

\begin{longtable}{p{1.0\linewidth}}
\begin{Shaded}
\begin{Highlighting}[]

\NormalTok{\textbackslash{}documentclass\{article\}}
 
\NormalTok{\textbackslash{}usepackage[table]\{xcolor\}}
 
\NormalTok{\textbackslash{}begin\{document\}}
 
\NormalTok{\textbackslash{}begin\{center\}}
\NormalTok{\textbackslash{}rowcolors\{1\}\{green\}\{pink\}}
 
\NormalTok{\textbackslash{}begin\{tabular\}\{lll\}}
\NormalTok{odd 	& odd 	& odd \textbackslash{}\textbackslash{}}
\NormalTok{even 	& even 	& even\textbackslash{}\textbackslash{}}
\NormalTok{odd 	& odd 	& odd \textbackslash{}\textbackslash{}}
\NormalTok{even 	& even 	& even\textbackslash{}\textbackslash{}}
\NormalTok{\textbackslash{}end\{tabular\}}
\NormalTok{\textbackslash{}end\{center\}}
 
\NormalTok{\textbackslash{}end\{document\}}
\end{Highlighting}
\end{Shaded}
\\



\begin{minipage}{1.0\linewidth}
\begin{center}
\includegraphics[width=1.0\linewidth,height=6.5in,keepaspectratio]{../images/48.\SVGExtension}
\end{center}
\raggedright{}\myfigurewithoutcaption{48}
\end{minipage}\vspace{0.75cm}



\end{longtable}

The command \LaTeXTT{\textbackslash{}hiderowcolors} is available to deactivate highlighting from a specified row until the end of the table.
Highlighting can be reactivated within the table via the \LaTeXTT{\textbackslash{}showrowcolors} command. If while using these commands you experience \symbol{34}misplaced \textbackslash{}noalign errors\symbol{34} then use the commands at the very beginning or end of a row in your tabular.
\begin{Shaded}
\begin{Highlighting}[]
\NormalTok{\textbackslash{}hiderowcolors odd & odd & odd \textbackslash{}\textbackslash{}}
\end{Highlighting}
\end{Shaded}

or
\begin{Shaded}
\begin{Highlighting}[]
\NormalTok{odd & odd & odd \textbackslash{}\textbackslash{} \textbackslash{}showrowcolors}
\end{Highlighting}
\end{Shaded}

\subsection{Colors of individual cells}
\label{271}

As above this uses the \LaTeXTT{xcolor} package.

\begin{Shaded}
\begin{Highlighting}[]

\CommentTok{% Include this somewhere in your document}
\NormalTok{\textbackslash{}usepackage[table]\{xcolor\}}
 
\CommentTok{% Enter this in the cell you wish to color a light grey.}
\CommentTok{% NB: the word 'gray' here denotes the grayscale color scheme, not the color}
 \NormalTok{grey. '0.9' denotes how dark the grey is.}
\NormalTok{\textbackslash{}cellcolor[gray]\{0.9\}}
\CommentTok{% The following will color the cell red.}
\NormalTok{\textbackslash{}cellcolor\{red\}}
\end{Highlighting}
\end{Shaded}

\section{Width and stretching}
\label{272}

We keep providing documentation for \LaTeXTT{tabular*} and \LaTeXTT{tabularx} although they are completely eclipsed by the much more powerful and flexible \LaTeXTT{tabu} environment. Actually \LaTeXTT{tabu} is greatly inspired by those environments, so it may be worth it to have an idea how they work, particularly for \LaTeXTT{tabularx}.
\subsection{The {\itshape \setmainfont[Path=/usr/share/fonts/truetype/cmu/,UprightFont=cmunrm.ttf,BoldFont=cmunbx.ttf,ItalicFont=cmunti.ttf,BoldItalicFont=cmunbi.ttf]{cmunti.ttf}\setmonofont[Path=/usr/share/fonts/truetype/cmu/,UprightFont=cmuntt.ttf,BoldFont=cmuntb.ttf,ItalicFont=cmunit.ttf,BoldItalicFont=cmuntx.ttf]{cmunti.ttf}\itshape tabular*}{$\text{ }$}\setmainfont[Path=/usr/share/fonts/truetype/cmu/,UprightFont=cmunrm.ttf,BoldFont=cmunbx.ttf,ItalicFont=cmunti.ttf,BoldItalicFont=cmunbi.ttf]{cmunrm.ttf}\setmonofont[Path=/usr/share/fonts/truetype/cmu/,UprightFont=cmuntt.ttf,BoldFont=cmuntb.ttf,ItalicFont=cmunit.ttf,BoldItalicFont=cmuntx.ttf]{cmunrm.ttf} environment}
\label{273}

This is basically a slight extension on the original tabular version, although it requires an extra argument (before the column descriptions) to specify the preferred width of the table.

\begin{longtable}{p{1.0\linewidth}}
\begin{Shaded}
\begin{Highlighting}[]

\NormalTok{\textbackslash{}begin\{tabular*\}\{0.75\textbackslash{}textwidth\}\{  c  c  c  r  \}}
  \NormalTok{\textbackslash{}hline}
  \NormalTok{label 1 & label 2 & label 3 & label 4 \textbackslash{}\textbackslash{}}
  \NormalTok{\textbackslash{}hline }
  \NormalTok{item 1  & item 2  & item 3  & item 4  \textbackslash{}\textbackslash{}}
  \NormalTok{\textbackslash{}hline}
\NormalTok{\textbackslash{}end\{tabular*\}}
\end{Highlighting}
\end{Shaded}
\\



\begin{minipage}{1.0\linewidth}
\begin{center}
\includegraphics[width=1.0\linewidth,height=6.5in,keepaspectratio]{../images/49.\SVGExtension}
\end{center}
\raggedright{}\myfigurewithoutcaption{49}
\end{minipage}\vspace{0.75cm}



\end{longtable}

However, that may not look quite as intended. The columns are still at their natural width (just wide enough to fit their contents) while the rows are as wide as the table width specified. If you do not like this default, you must also explicitly insert extra column space. LaTeX has {\itshape \setmainfont[Path=/usr/share/fonts/truetype/cmu/,UprightFont=cmunrm.ttf,BoldFont=cmunbx.ttf,ItalicFont=cmunti.ttf,BoldItalicFont=cmunbi.ttf]{cmunti.ttf}\setmonofont[Path=/usr/share/fonts/truetype/cmu/,UprightFont=cmuntt.ttf,BoldFont=cmuntb.ttf,ItalicFont=cmunit.ttf,BoldItalicFont=cmuntx.ttf]{cmunti.ttf}\itshape rubber lengths}\setmainfont[Path=/usr/share/fonts/truetype/cmu/,UprightFont=cmunrm.ttf,BoldFont=cmunbx.ttf,ItalicFont=cmunti.ttf,BoldItalicFont=cmunbi.ttf]{cmunrm.ttf}\setmonofont[Path=/usr/share/fonts/truetype/cmu/,UprightFont=cmuntt.ttf,BoldFont=cmuntb.ttf,ItalicFont=cmunit.ttf,BoldItalicFont=cmuntx.ttf]{cmunrm.ttf}, which, unlike others, are not fixed. LaTeX can dynamically decide how long the lengths should be. So, an example of this is the following.

\begin{longtable}{p{1.0\linewidth}}
\begin{Shaded}
\begin{Highlighting}[]

\NormalTok{\textbackslash{}begin\{tabular*\}\{0.75\textbackslash{}textwidth\}\{@\{\textbackslash{}extracolsep\{\textbackslash{}fill\} \}  c  c  c  r  \}}
  \NormalTok{\textbackslash{}hline}
  \NormalTok{label 1 & label 2 & label 3 & label 4 \textbackslash{}\textbackslash{}}
  \NormalTok{\textbackslash{}hline }
  \NormalTok{item 1  & item 2  & item 3  & item 4  \textbackslash{}\textbackslash{}}
  \NormalTok{\textbackslash{}hline}
\NormalTok{\textbackslash{}end\{tabular*\}}
\end{Highlighting}
\end{Shaded}
\\



\begin{minipage}{1.0\linewidth}
\begin{center}
\includegraphics[width=1.0\linewidth,height=6.5in,keepaspectratio]{../images/50.\SVGExtension}
\end{center}
\raggedright{}\myfigurewithoutcaption{50}
\end{minipage}\vspace{0.75cm}



\end{longtable}

You will notice the \LaTeXTT{@\{...\}} construct added at the beginning of the column description. Within it is the \LaTeXTT{\textbackslash{}extracolsep} command, which requires a width. A fixed width could have been used. However, by using a rubber length, such as \LaTeXTT{\textbackslash{}fill}, the columns are automatically spaced evenly.
\subsection{The {\itshape \setmainfont[Path=/usr/share/fonts/truetype/cmu/,UprightFont=cmunrm.ttf,BoldFont=cmunbx.ttf,ItalicFont=cmunti.ttf,BoldItalicFont=cmunbi.ttf]{cmunti.ttf}\setmonofont[Path=/usr/share/fonts/truetype/cmu/,UprightFont=cmuntt.ttf,BoldFont=cmuntb.ttf,ItalicFont=cmunit.ttf,BoldItalicFont=cmuntx.ttf]{cmunti.ttf}\itshape tabularx}{$\text{ }$}\setmainfont[Path=/usr/share/fonts/truetype/cmu/,UprightFont=cmunrm.ttf,BoldFont=cmunbx.ttf,ItalicFont=cmunti.ttf,BoldItalicFont=cmunbi.ttf]{cmunrm.ttf}\setmonofont[Path=/usr/share/fonts/truetype/cmu/,UprightFont=cmuntt.ttf,BoldFont=cmuntb.ttf,ItalicFont=cmunit.ttf,BoldItalicFont=cmuntx.ttf]{cmunrm.ttf} package}
\label{274}

This package provides a table environment called \LaTeXTT{tabularx}, which is similar to the \LaTeXTT{tabular*} environment except that it has a new column specifier \LaTeXTT{X} (in uppercase). The column(s) specified with this specifier will be stretched to make the table as wide as specified, greatly simplifying the creation of tables.

\begin{longtable}{p{1.0\linewidth}}
\begin{Shaded}
\begin{Highlighting}[]

\NormalTok{\textbackslash{}usepackage\{tabularx\}}
\CommentTok{% ...}
 
\NormalTok{\textbackslash{}begin\{tabularx\}\{\textbackslash{}textwidth\}\{ XXXX \}}
  \NormalTok{\textbackslash{}hline}
  \NormalTok{label 1 & label 2 & label 3 & label 4 \textbackslash{}\textbackslash{}}
  \NormalTok{\textbackslash{}hline }
  \NormalTok{item 1  & item 2  & item 3  & item 4  \textbackslash{}\textbackslash{}}
  \NormalTok{\textbackslash{}hline}
\NormalTok{\textbackslash{}end\{tabularx\}}
\end{Highlighting}
\end{Shaded}
\\



\begin{minipage}{1.0\linewidth}
\begin{center}
\includegraphics[width=1.0\linewidth,height=6.5in,keepaspectratio]{../images/51.\SVGExtension}
\end{center}
\raggedright{}\myfigurewithoutcaption{51}
\end{minipage}\vspace{0.75cm}



\end{longtable}

The content provided for the boxes is treated as for a \LaTeXTT{p} column, except that the width is calculated automatically. If you use the package \LaTeXTT{array}, you may also apply any \LaTeXTT{>{}\{\textbackslash{}cmd\}} or \LaTeXTT{<{}\{\textbackslash{}cmd\}} command to achieve specific behavior (like \LaTeXTT{\textbackslash{}centering}, or \LaTeXTT{\textbackslash{}raggedright\textbackslash{}arraybackslash}) as described previously.

Another option is to use \LaTeXTT{\textbackslash{}newcolumntype} to format selected columns in a different way. It defines a new column specifier, e.g. \LaTeXTT{R} (in uppercase). In this example, the second and fourth column is adjusted in a different way (\LaTeXTT{\textbackslash{}raggedleft}):

\begin{longtable}{p{1.0\linewidth}}
\begin{Shaded}
\begin{Highlighting}[]

\NormalTok{\textbackslash{}usepackage\{tabularx\}}
\CommentTok
\NormalTok{\textbackslash{}begin\{tabularx\}\{\textbackslash{}textwidth\}\{ lRlR \}}
  \NormalTok{\textbackslash{}hline}
  \NormalTok{label 1 & label 2 & label 3 & label 4 \textbackslash{}\textbackslash{}}
  \NormalTok{\textbackslash{}hline }
  \NormalTok{item 1  & item 2  & item 3  & item 4  \textbackslash{}\textbackslash{}}
  \NormalTok{\textbackslash{}hline}
\NormalTok{\textbackslash{}end\{tabularx\}}
\end{Highlighting}
\end{Shaded}
\\



\begin{minipage}{1.0\linewidth}
\begin{center}
\includegraphics[width=1.0\linewidth,height=6.5in,keepaspectratio]{../images/52.\SVGExtension}
\end{center}
\raggedright{}\myfigurewithoutcaption{52}
\end{minipage}\vspace{0.75cm}



\end{longtable}

Tabularx with rows spanning multiple columns using \LaTeXTT{\textbackslash{}multicolumn}. The two central columns are posing as one by using the \LaTeXTT{X@\{\}} option. Note that the \LaTeXTT{\textbackslash{}multicolumn} width (which in this example is 2) should equal the (in this example 1+1) width of the spanned columns:

\begin{longtable}{p{1.0\linewidth}}
\begin{Shaded}
\begin{Highlighting}[]

\NormalTok{\textbackslash{}usepackage\{tabularx\}}
\CommentTok{% ...}
 
\NormalTok{\textbackslash{}begin\{tabularx\}\{1\textbackslash{}textwidth\}\{ >\{\textbackslash{}}
\NormalTok{setlength\textbackslash{}hsize\{1\textbackslash{}hsize\}\textbackslash{}centering\}X>\{\textbackslash{}setlength\textbackslash{}hsize\{1\textbackslash{}hsize\}\textbackslash{}raggedleft\}X@\{\} }
\NormalTok{>\{}
\NormalTok{\textbackslash{}setlength\textbackslash{}hsize\{1\textbackslash{}hsize\}\textbackslash{}raggedright\}X>\{\textbackslash{}setlength\textbackslash{}hsize\{1\textbackslash{}hsize\}\textbackslash{}centering\}X}
 \NormalTok{\} }
  \NormalTok{\textbackslash{}hline}
\NormalTok{Label 1 & \textbackslash{}multicolumn\{2\}\{>\{\textbackslash{}centering\textbackslash{}setlength\textbackslash{}hsize\{2\textbackslash{}hsize\} \}X\}\{Label 2\} &}
 \NormalTok{Label 3\textbackslash{}tabularnewline}
\NormalTok{\textbackslash{}hline }
  \NormalTok{123  & 123  & 456  & 123  \textbackslash{}tabularnewline}
  \NormalTok{\textbackslash{}hline}
  \NormalTok{123  & 123  & 456  & 123  \textbackslash{}tabularnewline}
  \NormalTok{\textbackslash{}hline}
\NormalTok{\textbackslash{}end\{tabularx\}}
\end{Highlighting}
\end{Shaded}
\\



\begin{minipage}{1.0\linewidth}
\begin{center}
\includegraphics[width=1.0\linewidth,height=6.5in,keepaspectratio]{../images/53.\SVGExtension}
\end{center}
\raggedright{}\myfigurewithoutcaption{53}
\end{minipage}\vspace{0.75cm}



\end{longtable}

In a way analogous to how new commands with arguments can be created with \textbackslash{}newcommand, new column types with arguments can be created with \textbackslash{}newcolumntype as follows:

\begin{Shaded}
\begin{Highlighting}[]

\NormalTok{\textbackslash{}usepackage\{tabularx\}}
\NormalTok{\textbackslash{}usepackage[table]\{xcolor\} }\CommentTok{%Used to color the last column}
\CommentTok
\NormalTok{\textbackslash{}newcolumntype\{R\}[1]\{>\{\textbackslash{}hsize=#1\textbackslash{}hsize\textbackslash{}raggedleft\textbackslash{}arraybackslash\}X\}}\CommentTok
 
\NormalTok{\textbackslash{}begin\{tabularx\}\{\textbackslash{}textwidth\}\{  L\{1\}  R\{0.5\}  R\{0.5\}  C\{2\}\{gray\}  \}}
  \NormalTok{\textbackslash{}hline}
  \NormalTok{label 1 & label 2 & label 3 & label 4 \textbackslash{}\textbackslash{}}
  \NormalTok{\textbackslash{}hline}
  \NormalTok{item 1  & item 2  & item 3  & item 4  \textbackslash{}\textbackslash{}}
  \NormalTok{\textbackslash{}hline}
\NormalTok{\textbackslash{}end\{tabularx\}}
\end{Highlighting}
\end{Shaded}


where since there are 4 columns, the sum of the \textbackslash{}hsize\textquotesingle{}s (1 + 0.5 + 0.5 + 2) must be equal to 4. The default value used by tabularx for \textbackslash{}hsize is 1.
\subsection{The {\itshape \setmainfont[Path=/usr/share/fonts/truetype/cmu/,UprightFont=cmunrm.ttf,BoldFont=cmunbx.ttf,ItalicFont=cmunti.ttf,BoldItalicFont=cmunbi.ttf]{cmunti.ttf}\setmonofont[Path=/usr/share/fonts/truetype/cmu/,UprightFont=cmuntt.ttf,BoldFont=cmuntb.ttf,ItalicFont=cmunit.ttf,BoldItalicFont=cmuntx.ttf]{cmunti.ttf}\itshape tabulary}{$\text{ }$}\setmainfont[Path=/usr/share/fonts/truetype/cmu/,UprightFont=cmunrm.ttf,BoldFont=cmunbx.ttf,ItalicFont=cmunti.ttf,BoldItalicFont=cmunbi.ttf]{cmunrm.ttf}\setmonofont[Path=/usr/share/fonts/truetype/cmu/,UprightFont=cmuntt.ttf,BoldFont=cmuntb.ttf,ItalicFont=cmunit.ttf,BoldItalicFont=cmuntx.ttf]{cmunrm.ttf} package}
\label{275}

\myhref{http://www.ctan.org/pkg/tabulary}{{\ttfamily \setmainfont[Path=/usr/share/fonts/truetype/cmu/,UprightFont=cmunrm.ttf,BoldFont=cmunbx.ttf,ItalicFont=cmunti.ttf,BoldItalicFont=cmunbi.ttf]{cmuntt.ttf}\setmonofont[Path=/usr/share/fonts/truetype/cmu/,UprightFont=cmuntt.ttf,BoldFont=cmuntb.ttf,ItalicFont=cmunit.ttf,BoldItalicFont=cmuntx.ttf]{cmuntt.ttf}\ttfamily tabulary}} is a modified \LaTeXTT{tabular*} allowing width of columns set for equal heights.
\LaTeXTT{tabulary} allows easy and convenient writing of well balanced tables. 

The problem with \LaTeXTT{tabularx} is that it leaves much blank if your cells are almost empty. Besides, it is not easy to have different column sizes.

\LaTeXTT{tabulary} tries to balance the column widths so that each column has at least its natural width, without exceeding the maximum length.

\begin{Shaded}
\begin{Highlighting}[]

\NormalTok{\textbackslash{}usepackage\{tabulary\}}
\NormalTok{...}
 
\NormalTok{\textbackslash{}begin\{center\}}
  \NormalTok{\textbackslash{}begin\{tabulary\}\{0.7\textbackslash{}textwidth\}\{LCL\}}
    \NormalTok{Short sentences      & \textbackslash{}#  & Long sentences                                 }
                \NormalTok{\textbackslash{}\textbackslash{}}
    \NormalTok{\textbackslash{}hline}
    \NormalTok{This is short.       & 173 & This is much loooooooonger, because there are}
 \NormalTok{many more words.  \textbackslash{}\textbackslash{}}
    \NormalTok{This is not shorter. & 317 & This is still loooooooonger, because there are}
 \NormalTok{many more words. \textbackslash{}\textbackslash{}}
  \NormalTok{\textbackslash{}end\{tabulary\}  }
\NormalTok{\textbackslash{}end\{center\}}
\end{Highlighting}
\end{Shaded}


The first parameter is the maximum width. \LaTeXTT{tabulary} will try not to exceed it, but it will not stretch to it if there is not enough content, contrary to \LaTeXTT{tabularx}.

The second parameter is the column disposition. Possible values are those from the \LaTeXTT{tabular} environment, plus

\begin{longtable}{|>{\RaggedRight}p{0.09743\linewidth}|>{\RaggedRight}p{0.84542\linewidth}|} \hline 
\hspace*{0pt}\ignorespaces{}\hspace*{0pt} \LaTeXTT{L}&\hspace*{0pt}\ignorespaces{}\hspace*{0pt} left-{}justified balanced column\\ \hline \hspace*{0pt}\ignorespaces{}\hspace*{0pt} \LaTeXTT{C}&\hspace*{0pt}\ignorespaces{}\hspace*{0pt} centered balanced column\\ \hline \hspace*{0pt}\ignorespaces{}\hspace*{0pt} \LaTeXTT{R}&\hspace*{0pt}\ignorespaces{}\hspace*{0pt} right-{}justified balanced column\\ \hline \hspace*{0pt}\ignorespaces{}\hspace*{0pt} \LaTeXTT{J}&\hspace*{0pt}\ignorespaces{}\hspace*{0pt} left-{}right-{}justified balanced column\\ \hline 
\end{longtable}


These are all capitals.
\subsection{The {\itshape \setmainfont[Path=/usr/share/fonts/truetype/cmu/,UprightFont=cmunrm.ttf,BoldFont=cmunbx.ttf,ItalicFont=cmunti.ttf,BoldItalicFont=cmunbi.ttf]{cmunti.ttf}\setmonofont[Path=/usr/share/fonts/truetype/cmu/,UprightFont=cmuntt.ttf,BoldFont=cmuntb.ttf,ItalicFont=cmunit.ttf,BoldItalicFont=cmuntx.ttf]{cmunti.ttf}\itshape tabu}{$\text{ }$}\setmainfont[Path=/usr/share/fonts/truetype/cmu/,UprightFont=cmunrm.ttf,BoldFont=cmunbx.ttf,ItalicFont=cmunti.ttf,BoldItalicFont=cmunbi.ttf]{cmunrm.ttf}\setmonofont[Path=/usr/share/fonts/truetype/cmu/,UprightFont=cmuntt.ttf,BoldFont=cmuntb.ttf,ItalicFont=cmunit.ttf,BoldItalicFont=cmuntx.ttf]{cmunrm.ttf} environment}
\label{276}

It works pretty much like \LaTeXTT{tabularx}.

\begin{Shaded}
\begin{Highlighting}[]

 
\NormalTok{\textbackslash{}begin\{tabu\} to \textbackslash{}linewidth \{llX[2]lllXl\}}
\CommentTok{% ...}
\NormalTok{\textbackslash{}end\{tabu\}}
\end{Highlighting}
\end{Shaded}


\LaTeXTT{to \textbackslash{}linewidth} specifies the target width.
The \LaTeXTT{X} parameter can have an optional span factor.
\section{Table across several pages}
\label{277}

Long tables are natively supported by LaTeX thanks to the \LaTeXTT{longtable} environment. Unfortunately this environment does not support stretching (X columns).

The \LaTeXTT{tabu} packages provides the \LaTeXTT{longtabu} environment. It has most of the features of \LaTeXTT{tabu}, with the additional capability to span multiple pages.

LaTeX can do well with long tables: you can specify a header that will repeat on every page, a header for the first page only, and the same for the footer.

\begin{Shaded}
\begin{Highlighting}[]

\NormalTok{\textbackslash{}begin\{longtabu\} to \textbackslash{}linewidth \{lX[2]lXl\}}
 
\NormalTok{\textbackslash{}rowfont\textbackslash{}bfseries H1 & H2 & H3 & H4 & H5 \textbackslash{}\textbackslash{} \textbackslash{}hline }
\NormalTok{\textbackslash{}endhead}
 
\NormalTok{\textbackslash{}\textbackslash{} \textbackslash{}hline}
\NormalTok{\textbackslash{}multicolumn\{5\}\{r\}\{There is more to come\} \textbackslash{}\textbackslash{}}
\NormalTok{\textbackslash{}endfoot}
 
\NormalTok{\textbackslash{}\textbackslash{} \textbackslash{}hline}
\NormalTok{\textbackslash{}endlastfoot}
 
\CommentTok{% Content ...}
 
\end{Highlighting}
\end{Shaded}


It uses syntax similar to \LaTeXTT{longtable}, so you should have a look at its documentation if you want to know more.

Alternatively you can try one of the following packages \myhref{http://www.ctan.org/pkg/supertabular}{{\ttfamily \setmainfont[Path=/usr/share/fonts/truetype/cmu/,UprightFont=cmunrm.ttf,BoldFont=cmunbx.ttf,ItalicFont=cmunti.ttf,BoldItalicFont=cmunbi.ttf]{cmuntt.ttf}\setmonofont[Path=/usr/share/fonts/truetype/cmu/,UprightFont=cmuntt.ttf,BoldFont=cmuntb.ttf,ItalicFont=cmunit.ttf,BoldItalicFont=cmuntx.ttf]{cmuntt.ttf}\ttfamily supertabular}} or \myhref{http://www.ctan.org/pkg/xtab}{{\ttfamily \setmainfont[Path=/usr/share/fonts/truetype/cmu/,UprightFont=cmunrm.ttf,BoldFont=cmunbx.ttf,ItalicFont=cmunti.ttf,BoldItalicFont=cmunbi.ttf]{cmuntt.ttf}\setmonofont[Path=/usr/share/fonts/truetype/cmu/,UprightFont=cmuntt.ttf,BoldFont=cmuntb.ttf,ItalicFont=cmunit.ttf,BoldItalicFont=cmuntx.ttf]{cmuntt.ttf}\ttfamily xtab}}, an extended and somewhat improved version of \LaTeXTT{supertabular}.
\section{Partial vertical lines}
\label{278}

Adding a partial vertical line to an individual cell:

\begin{longtable}{p{1.0\linewidth}}
\begin{Shaded}
\begin{Highlighting}[]

\NormalTok{\textbackslash{}begin\{tabular\}\{ l c r \}}
  \NormalTok{\textbackslash{}hline}
  \NormalTok{1 & 2 & 3 \textbackslash{}\textbackslash{} \textbackslash{}hline}
  \NormalTok{4 & 5 & \textbackslash{}multicolumn\{1\}\{r\}\{6\}  \textbackslash{}\textbackslash{} \textbackslash{}hline}
  \NormalTok{7 & 8 & 9 \textbackslash{}\textbackslash{} \textbackslash{}hline}
\NormalTok{\textbackslash{}end\{tabular\}}
\end{Highlighting}
\end{Shaded}
\\



\begin{minipage}{1.0\linewidth}
\begin{center}
\includegraphics[width=1.0\linewidth,height=6.5in,keepaspectratio]{../images/54.\SVGExtension}
\end{center}
\raggedright{}\myfigurewithoutcaption{54}
\end{minipage}\vspace{0.75cm}



\end{longtable}

Removing part of a vertical line in a particular cell:

\begin{longtable}{p{1.0\linewidth}}
\begin{Shaded}
\begin{Highlighting}[]

\NormalTok{\textbackslash{}begin\{tabular\}\{  l  c  r  \}}
  \NormalTok{\textbackslash{}hline}
  \NormalTok{1 & 2 & 3 \textbackslash{}\textbackslash{} \textbackslash{}hline}
  \NormalTok{4 & 5 & \textbackslash{}multicolumn\{1\}\{r\}\{6\} \textbackslash{}\textbackslash{} \textbackslash{}hline}
  \NormalTok{7 & 8 & 9 \textbackslash{}\textbackslash{} \textbackslash{}hline}
\NormalTok{\textbackslash{}end\{tabular\}}
\end{Highlighting}
\end{Shaded}
\\



\begin{minipage}{1.0\linewidth}
\begin{center}
\includegraphics[width=1.0\linewidth,height=6.5in,keepaspectratio]{../images/55.\SVGExtension}
\end{center}
\raggedright{}\myfigurewithoutcaption{55}
\end{minipage}\vspace{0.75cm}



\end{longtable}
\section{Vertically centered images}
\label{279}

Inserting images into a table row will align it at the top of the cell.
By using the \LaTeXTT{array} package this problem can be solved.
Defining a new columntype will keep the image vertically centered.

\begin{Shaded}
\begin{Highlighting}[]

\NormalTok{\textbackslash{}newcolumntype\{V\}\{>\{\textbackslash{}centering\textbackslash{}arraybackslash\} m\{.4\textbackslash{}linewidth\} \}}
\end{Highlighting}
\end{Shaded}


Or use a parbox to center the image.

\begin{Shaded}
\begin{Highlighting}[]

\NormalTok{\textbackslash{}parbox[c]\{1em\}\{\textbackslash{}includegraphics\{image.png\} \}}
\end{Highlighting}
\end{Shaded}


A raisebox works as well, also allowing to manually fine-{}tune the alignment with its first parameter.

\begin{Shaded}
\begin{Highlighting}[]

\NormalTok{\textbackslash{}raisebox\{-.5\textbackslash{}height\}\{\textbackslash{}includegraphics\{image.png\} \}}
\end{Highlighting}
\end{Shaded}

\section{Footnotes in tables}
\label{280}

The \LaTeXTT{tabular} environment does not handle footnotes properly. The \LaTeXTT{longtabular} fixes that.

Instead of using \LaTeXTT{longtabular} we recommend \LaTeXTT{tabu} which handles footnotes properly, both in normal and long tables.
\section{Professional tables}
\label{281}

Many professionally typeset books and journals feature simple tables, which have appropriate spacing above and below lines, and almost {\itshape \setmainfont[Path=/usr/share/fonts/truetype/cmu/,UprightFont=cmunrm.ttf,BoldFont=cmunbx.ttf,ItalicFont=cmunti.ttf,BoldItalicFont=cmunbi.ttf]{cmunti.ttf}\setmonofont[Path=/usr/share/fonts/truetype/cmu/,UprightFont=cmuntt.ttf,BoldFont=cmuntb.ttf,ItalicFont=cmunit.ttf,BoldItalicFont=cmuntx.ttf]{cmunti.ttf}\itshape never}{$\text{ }$}\setmainfont[Path=/usr/share/fonts/truetype/cmu/,UprightFont=cmunrm.ttf,BoldFont=cmunbx.ttf,ItalicFont=cmunti.ttf,BoldItalicFont=cmunbi.ttf]{cmunrm.ttf}\setmonofont[Path=/usr/share/fonts/truetype/cmu/,UprightFont=cmuntt.ttf,BoldFont=cmuntb.ttf,ItalicFont=cmunit.ttf,BoldItalicFont=cmuntx.ttf]{cmunrm.ttf} use vertical rules. Many examples of LaTeX tables (including this Wikibook) showcase the use of vertical rules (using \symbol{34}|\symbol{34}), and double-{}rules (using \LaTeXTT{\textbackslash{}hline\textbackslash{}hline} or \symbol{34}||\symbol{34}), which are regarded as unnecessary and distracting in a professionally published form.  The \myhref{http://www.ctan.org/tex-archive/macros/latex/contrib/booktabs/}{booktabs} package is useful for easily providing this professionalism in LaTeX tables, and the \myhref{http://mirrors.ctan.org/macros/latex/contrib/booktabs/booktabs.pdf}{documentation} also provides guidelines on what constitutes a \symbol{34}good\symbol{34} table.

In brief, the package uses \LaTeXTT{\textbackslash{}toprule} for the uppermost rule (or line), \LaTeXTT{\textbackslash{}midrule} for the rules appearing in the middle of the table (such as under the header), and \LaTeXTT{\textbackslash{}bottomrule} for the lowermost rule. This ensures that the rule weight and spacing are acceptable. In addition, \LaTeXTT{\textbackslash{}cmidrule} can be used for mid-{}rules that span specified columns. The following example contrasts the use of booktabs and  two equivalent normal LaTeX implementations (the second example requires \LaTeXTT{\textbackslash{}usepackage\{array\}} or \LaTeXTT{\textbackslash{}usepackage\{dcolumn\}}, and the third example requires \LaTeXTT{\textbackslash{}usepackage\{booktabs\}} in the preamble).
\subsection{Normal LaTeX}
\label{282}
\begin{longtable}{p{1.0\linewidth}}
\begin{Shaded}
\begin{Highlighting}[]

\NormalTok{\textbackslash{}begin\{tabular\}\{llr\}}
\NormalTok{\textbackslash{}hline}
\NormalTok{\textbackslash{}multicolumn\{2\}\{c\}\{Item\} \textbackslash{}\textbackslash{}}
\NormalTok{\textbackslash{}cline\{1-2\}}
\NormalTok{Animal    & Description & Price (\textbackslash{}$) \textbackslash{}\textbackslash{}}
\NormalTok{\textbackslash{}hline}
\NormalTok{Gnat      & per gram    & 13.65      \textbackslash{}\textbackslash{}}
          \NormalTok{& each        & 0.01       \textbackslash{}\textbackslash{}}
\NormalTok{Gnu       & stuffed     & 92.50      \textbackslash{}\textbackslash{}}
\NormalTok{Emu       & stuffed     & 33.33      \textbackslash{}\textbackslash{}}
\NormalTok{Armadillo & frozen      & 8.99       \textbackslash{}\textbackslash{}}
\NormalTok{\textbackslash{}hline}
\NormalTok{\textbackslash{}end\{tabular\}}
\end{Highlighting}
\end{Shaded}
\\


\begin{minipage}{1.0\linewidth}
\begin{center}
\includegraphics[width=1.0\linewidth,height=6.5in,keepaspectratio]{../images/56.\SVGExtension}
\end{center}
\raggedright{}\myfigurewithoutcaption{56}
\end{minipage}\vspace{0.75cm}


\end{longtable}
\subsection{Using \LaTeXTT{array}}
\label{283}


\begin{longtable}{p{1.0\linewidth}}
\begin{Shaded}
\begin{Highlighting}[]

\NormalTok{\textbackslash{}usepackage\{array\} }
\CommentTok{%or \textbackslash{}usepackage\{dcolumn\}}
\NormalTok{...}
\NormalTok{\textbackslash{}begin\{tabular\}\{llr\}}
\NormalTok{\textbackslash{}firsthline}
\NormalTok{\textbackslash{}multicolumn\{2\}\{c\}\{Item\} \textbackslash{}\textbackslash{}}
\NormalTok{\textbackslash{}cline\{1-2\}}
\NormalTok{Animal    & Description & Price (\textbackslash{}$) \textbackslash{}\textbackslash{}}
\NormalTok{\textbackslash{}hline}
\NormalTok{Gnat      & per gram    & 13.65      \textbackslash{}\textbackslash{}}
          \NormalTok{& each        & 0.01       \textbackslash{}\textbackslash{}}
\NormalTok{Gnu       & stuffed     & 92.50      \textbackslash{}\textbackslash{}}
\NormalTok{Emu       & stuffed     & 33.33      \textbackslash{}\textbackslash{}}
\NormalTok{Armadillo & frozen      & 8.99       \textbackslash{}\textbackslash{}}
\NormalTok{\textbackslash{}lasthline}
\NormalTok{\textbackslash{}end\{tabular\}}
\end{Highlighting}
\end{Shaded}
\\


\begin{minipage}{1.0\linewidth}
\begin{center}
\includegraphics[width=1.0\linewidth,height=6.5in,keepaspectratio]{../images/57.\SVGExtension}
\end{center}
\raggedright{}\myfigurewithoutcaption{57}
\end{minipage}\vspace{0.75cm}


\end{longtable}
\subsection{Using \LaTeXTT{booktabs}}
\label{284}


\begin{longtable}{p{1.0\linewidth}}
\begin{Shaded}
\begin{Highlighting}[]

\NormalTok{\textbackslash{}usepackage\{booktabs\}\textbackslash{}\textbackslash{}}
\NormalTok{\textbackslash{}begin\{tabular\}\{llr\}  }
\NormalTok{\textbackslash{}toprule}
\NormalTok{\textbackslash{}multicolumn\{2\}\{c\}\{Item\} \textbackslash{}\textbackslash{}}
\NormalTok{\textbackslash{}cmidrule(r)\{1-2\}}
\NormalTok{Animal    & Description & Price (\textbackslash{}$) \textbackslash{}\textbackslash{}}
\NormalTok{\textbackslash{}midrule}
\NormalTok{Gnat      & per gram    & 13.65      \textbackslash{}\textbackslash{}}
          \NormalTok{&    each     & 0.01       \textbackslash{}\textbackslash{}}
\NormalTok{Gnu       & stuffed     & 92.50      \textbackslash{}\textbackslash{}}
\NormalTok{Emu       & stuffed     & 33.33      \textbackslash{}\textbackslash{}}
\NormalTok{Armadillo & frozen      & 8.99       \textbackslash{}\textbackslash{}}
\NormalTok{\textbackslash{}bottomrule}
\NormalTok{\textbackslash{}end\{tabular\}}
\end{Highlighting}
\end{Shaded}
\\


\begin{minipage}{1.0\linewidth}
\begin{center}
\includegraphics[width=1.0\linewidth,height=6.5in,keepaspectratio]{../images/58.\SVGExtension}
\end{center}
\raggedright{}\myfigurewithoutcaption{58}
\end{minipage}\vspace{0.75cm}


\end{longtable}

Usually the need arises for footnotes under a table (and not at the bottom of the page), with a caption properly spaced above the table. These are addressed by the \myhref{http://www.ctan.org/tex-archive/macros/latex/contrib/ctable/}{ctable} package. It provides the option of a short caption given to be inserted in the list of tables, instead of the actual caption (which may be quite long and inappropriate for the list of tables). The \LaTeXTT{ctable} uses the \LaTeXTT{booktabs} package.
\section{Sideways tables}
\label{285}

Tables can also be put on their side within a document using the \LaTeXTT{rotating} or the \LaTeXTT{rotfloat} package. See the \mylref{243}{Rotations} chapter.
\section{Table with legend}
\label{286}

To add a legend to a table the \myhref{http://www.ctan.org/tex-archive/macros/latex/contrib/caption/}{caption} package can be used. With the caption package a \LaTeXTT{\textbackslash{}caption*\{...\}} statement can be added besides the 
normal \LaTeXTT{\textbackslash{}caption\{...\}}. Example:

\begin{Shaded}
\begin{Highlighting}[]

\NormalTok{\textbackslash{}begin\{table\}}
  \NormalTok{\textbackslash{}begin\{tabular\} r  r  c  c  c \}}
 
      \NormalTok{...}
 
  \NormalTok{\textbackslash{}end\{tabular\}}
  \NormalTok{\textbackslash{}caption\{A normal caption\}}
  \NormalTok{\textbackslash{}caption*\{}
    \NormalTok{A legend, even a table can be used}
    \NormalTok{\textbackslash{}begin\{tabular\}\{l l\}}
      \NormalTok{item 1 & explanation 1 \textbackslash{}\textbackslash{}}
    \NormalTok{\textbackslash{}end\{tabular\}}
  \NormalTok{\}}
\NormalTok{\textbackslash{}end\{table\}}
\end{Highlighting}
\end{Shaded}


The normal caption is needed for labels and references.
\section{The {\itshape \setmainfont[Path=/usr/share/fonts/truetype/cmu/,UprightFont=cmunrm.ttf,BoldFont=cmunbx.ttf,ItalicFont=cmunti.ttf,BoldItalicFont=cmunbi.ttf]{cmunti.ttf}\setmonofont[Path=/usr/share/fonts/truetype/cmu/,UprightFont=cmuntt.ttf,BoldFont=cmuntb.ttf,ItalicFont=cmunit.ttf,BoldItalicFont=cmuntx.ttf]{cmunti.ttf}\itshape eqparbox}{$\text{ }$}\setmainfont[Path=/usr/share/fonts/truetype/cmu/,UprightFont=cmunrm.ttf,BoldFont=cmunbx.ttf,ItalicFont=cmunti.ttf,BoldItalicFont=cmunbi.ttf]{cmunrm.ttf}\setmonofont[Path=/usr/share/fonts/truetype/cmu/,UprightFont=cmuntt.ttf,BoldFont=cmuntb.ttf,ItalicFont=cmunit.ttf,BoldItalicFont=cmuntx.ttf]{cmunrm.ttf} package}
\label{287}

On rare occasions, it might be necessary to stretch every row in a table to the natural width of its longest line, for instance when one has the same text in two languages and wishes to present these next to each other with lines synching up. A tabular environment helps control where lines should break, but cannot justify the text, which leads to ragged right edges. The \LaTeXTT{eqparbox} package provides the command \LaTeXTT{\textbackslash{}eqmakebox} which is like \LaTeXTT{\textbackslash{}makebox} but instead of a \LaTeXTT{width} argument, it takes a tag. During compilation it bookkeeps which \LaTeXTT{\textbackslash{}eqmakebox} with a certain tag contains the widest text and can stretch all \LaTeXTT{\textbackslash{}eqmakebox}es with the same tag to that width. Combined with the \LaTeXTT{array} package, one can define a column specifier that justifies the text in all lines:

\begin{Shaded}
\begin{Highlighting}[]

\NormalTok{\textbackslash{}newsavebox\{\textbackslash{}tstretchbox\}}
\NormalTok{\textbackslash{}newcolumntype\{S\}[1]\{}\CommentTok
  \NormalTok{l}\CommentTok
  \NormalTok{\textbackslash{}eqmakebox[#1][s]\{\textbackslash{}unhcopy\textbackslash{}tstretchbox\} \}}\CommentTok{%}
\NormalTok{\}}
\end{Highlighting}
\end{Shaded}


See the documentation of the \LaTeXTT{eqparbox} package for more details.
\section{Floating with {\itshape \setmainfont[Path=/usr/share/fonts/truetype/cmu/,UprightFont=cmunrm.ttf,BoldFont=cmunbx.ttf,ItalicFont=cmunti.ttf,BoldItalicFont=cmunbi.ttf]{cmunti.ttf}\setmonofont[Path=/usr/share/fonts/truetype/cmu/,UprightFont=cmuntt.ttf,BoldFont=cmuntb.ttf,ItalicFont=cmunit.ttf,BoldItalicFont=cmuntx.ttf]{cmunti.ttf}\itshape table}{$\text{ }$}\setmainfont[Path=/usr/share/fonts/truetype/cmu/,UprightFont=cmunrm.ttf,BoldFont=cmunbx.ttf,ItalicFont=cmunti.ttf,BoldItalicFont=cmunbi.ttf]{cmunrm.ttf}\setmonofont[Path=/usr/share/fonts/truetype/cmu/,UprightFont=cmuntt.ttf,BoldFont=cmuntb.ttf,ItalicFont=cmunit.ttf,BoldItalicFont=cmuntx.ttf]{cmunrm.ttf}}
\label{288}

In WYSIWYG document processors, it is common to put tables in the middle of the text. This is what we have been doing until now.
Professional documents, however, often make it a point to print tables on a dedicated page so that they do not disrupt the flow. From the point of view of the source code, one has no idea on which page the current text is going to lie, so it is hardly possible to guess which page may be appropriate for our table. LaTeX can automate this task by abstracting objects such as tables, pictures, etc., and deciding for us where they might fit best. This abstraction is called a {\itshape \setmainfont[Path=/usr/share/fonts/truetype/cmu/,UprightFont=cmunrm.ttf,BoldFont=cmunbx.ttf,ItalicFont=cmunti.ttf,BoldItalicFont=cmunbi.ttf]{cmunti.ttf}\setmonofont[Path=/usr/share/fonts/truetype/cmu/,UprightFont=cmuntt.ttf,BoldFont=cmuntb.ttf,ItalicFont=cmunit.ttf,BoldItalicFont=cmuntx.ttf]{cmunti.ttf}\itshape float}\setmainfont[Path=/usr/share/fonts/truetype/cmu/,UprightFont=cmunrm.ttf,BoldFont=cmunbx.ttf,ItalicFont=cmunti.ttf,BoldItalicFont=cmunbi.ttf]{cmunrm.ttf}\setmonofont[Path=/usr/share/fonts/truetype/cmu/,UprightFont=cmuntt.ttf,BoldFont=cmuntb.ttf,ItalicFont=cmunit.ttf,BoldItalicFont=cmuntx.ttf]{cmunrm.ttf}. Generally, an object that is floated will appear in the vicinity of its introduction in the source file, but one can choose to control its position also.

To tell LaTeX we want to use our table as a float, we need to put a \LaTeXTT{table} environment around the \LaTeXTT{tabular} environment, which is able to float and add a label and caption.

\begin{TemplateInfo}{\danger}{Warning}
Please understand: you do not {\itshape \setmainfont[Path=/usr/share/fonts/truetype/cmu/,UprightFont=cmunrm.ttf,BoldFont=cmunbx.ttf,ItalicFont=cmunti.ttf,BoldItalicFont=cmunbi.ttf]{cmunti.ttf}\setmonofont[Path=/usr/share/fonts/truetype/cmu/,UprightFont=cmuntt.ttf,BoldFont=cmuntb.ttf,ItalicFont=cmunit.ttf,BoldItalicFont=cmuntx.ttf]{cmunti.ttf}\itshape have to}{$\text{ }$}\setmainfont[Path=/usr/share/fonts/truetype/cmu/,UprightFont=cmunrm.ttf,BoldFont=cmunbx.ttf,ItalicFont=cmunti.ttf,BoldItalicFont=cmunbi.ttf]{cmunrm.ttf}\setmonofont[Path=/usr/share/fonts/truetype/cmu/,UprightFont=cmuntt.ttf,BoldFont=cmuntb.ttf,ItalicFont=cmunit.ttf,BoldItalicFont=cmuntx.ttf]{cmunrm.ttf} use floating tables. If you want to place your tables where they lie in your source code and you do not need any label, do not use \LaTeXTT{table} at all! This is a very common misunderstanding among newcomers.
\end{TemplateInfo}

The \LaTeXTT{table} environment initiates a type of float just as the environment \LaTeXTT{figure}. In fact, the two bear a lot of similarities (positioning, captions, etc.). More information about floating environments, captions etc. can be found in \mylref{362}{Floats, Figures and Captions}.

The environment names may now seem quite confusing. Let\textquotesingle{}s sum it up:
\begin{myitemize}
\item{}  \LaTeXTT{tabular} is for the content itself (columns, lines, etc.).
\item{}  \LaTeXTT{table} is for the location of the table on the document, plus caption and label support.
\end{myitemize}


\begin{Shaded}
\begin{Highlighting}[]

\NormalTok{\textbackslash{}begin\{table\}[position specifier]}
  \NormalTok{\textbackslash{}centering}
  \NormalTok{\textbackslash{}begin\{tabular\}l\}}
    \NormalTok{... your table ...}
  \NormalTok{\textbackslash{}end\{tabular\}}
  \NormalTok{\textbackslash{}caption\{This table shows some data\}}
  \NormalTok{\textbackslash{}label\{tab:myfirsttable\}}
\NormalTok{\textbackslash{}end\{table\}}
\end{Highlighting}
\end{Shaded}


In the table, we used a label, so now we can refer to it just like any other reference:
\begin{Shaded}
\begin{Highlighting}[]

\NormalTok{\textbackslash{}ref\{tab:myfirsttable\}}
\end{Highlighting}
\end{Shaded}


The \LaTeXTT{table} environment is also useful when you want to have a
list of tables at the beginning or end of your document with the command

\begin{Shaded}
\begin{Highlighting}[]

\NormalTok{\textbackslash{}listoftables}
\end{Highlighting}
\end{Shaded}

The captions now show up in the list of tables, if displayed.

You can set the optional parameter \LaTeXTT{position specifier} to define the position of the table, where it should be placed. The following characters are all possible placements. Using sequences of it define your \symbol{34}wishlist\symbol{34} to LaTeX.

\begin{longtable}{|>{\RaggedRight}p{0.04150\linewidth}|>{\RaggedRight}p{0.90135\linewidth}|} \hline 
\hspace*{0pt}\ignorespaces{}\hspace*{0pt} \LaTeXTT{h}&\hspace*{0pt}\ignorespaces{}\hspace*{0pt} where the table is declared ({\bfseries \setmainfont[Path=/usr/share/fonts/truetype/cmu/,UprightFont=cmunrm.ttf,BoldFont=cmunbx.ttf,ItalicFont=cmunti.ttf,BoldItalicFont=cmunbi.ttf]{cmunbx.ttf}\setmonofont[Path=/usr/share/fonts/truetype/cmu/,UprightFont=cmuntt.ttf,BoldFont=cmuntb.ttf,ItalicFont=cmunit.ttf,BoldItalicFont=cmuntx.ttf]{cmunbx.ttf}\bfseries h}\setmainfont[Path=/usr/share/fonts/truetype/cmu/,UprightFont=cmunrm.ttf,BoldFont=cmunbx.ttf,ItalicFont=cmunti.ttf,BoldItalicFont=cmunbi.ttf]{cmunrm.ttf}\setmonofont[Path=/usr/share/fonts/truetype/cmu/,UprightFont=cmuntt.ttf,BoldFont=cmuntb.ttf,ItalicFont=cmunit.ttf,BoldItalicFont=cmuntx.ttf]{cmunrm.ttf}ere)\\ \hline \hspace*{0pt}\ignorespaces{}\hspace*{0pt} \LaTeXTT{t}&\hspace*{0pt}\ignorespaces{}\hspace*{0pt} at the {\bfseries \setmainfont[Path=/usr/share/fonts/truetype/cmu/,UprightFont=cmunrm.ttf,BoldFont=cmunbx.ttf,ItalicFont=cmunti.ttf,BoldItalicFont=cmunbi.ttf]{cmunbx.ttf}\setmonofont[Path=/usr/share/fonts/truetype/cmu/,UprightFont=cmuntt.ttf,BoldFont=cmuntb.ttf,ItalicFont=cmunit.ttf,BoldItalicFont=cmuntx.ttf]{cmunbx.ttf}\bfseries t}\setmainfont[Path=/usr/share/fonts/truetype/cmu/,UprightFont=cmunrm.ttf,BoldFont=cmunbx.ttf,ItalicFont=cmunti.ttf,BoldItalicFont=cmunbi.ttf]{cmunrm.ttf}\setmonofont[Path=/usr/share/fonts/truetype/cmu/,UprightFont=cmuntt.ttf,BoldFont=cmuntb.ttf,ItalicFont=cmunit.ttf,BoldItalicFont=cmuntx.ttf]{cmunrm.ttf}op of the page\\ \hline \hspace*{0pt}\ignorespaces{}\hspace*{0pt} \LaTeXTT{b}&\hspace*{0pt}\ignorespaces{}\hspace*{0pt} at the {\bfseries \setmainfont[Path=/usr/share/fonts/truetype/cmu/,UprightFont=cmunrm.ttf,BoldFont=cmunbx.ttf,ItalicFont=cmunti.ttf,BoldItalicFont=cmunbi.ttf]{cmunbx.ttf}\setmonofont[Path=/usr/share/fonts/truetype/cmu/,UprightFont=cmuntt.ttf,BoldFont=cmuntb.ttf,ItalicFont=cmunit.ttf,BoldItalicFont=cmuntx.ttf]{cmunbx.ttf}\bfseries b}\setmainfont[Path=/usr/share/fonts/truetype/cmu/,UprightFont=cmunrm.ttf,BoldFont=cmunbx.ttf,ItalicFont=cmunti.ttf,BoldItalicFont=cmunbi.ttf]{cmunrm.ttf}\setmonofont[Path=/usr/share/fonts/truetype/cmu/,UprightFont=cmuntt.ttf,BoldFont=cmuntb.ttf,ItalicFont=cmunit.ttf,BoldItalicFont=cmuntx.ttf]{cmunrm.ttf}ottom of the page\\ \hline \hspace*{0pt}\ignorespaces{}\hspace*{0pt} \LaTeXTT{p}&\hspace*{0pt}\ignorespaces{}\hspace*{0pt} on a dedicated {\bfseries \setmainfont[Path=/usr/share/fonts/truetype/cmu/,UprightFont=cmunrm.ttf,BoldFont=cmunbx.ttf,ItalicFont=cmunti.ttf,BoldItalicFont=cmunbi.ttf]{cmunbx.ttf}\setmonofont[Path=/usr/share/fonts/truetype/cmu/,UprightFont=cmuntt.ttf,BoldFont=cmuntb.ttf,ItalicFont=cmunit.ttf,BoldItalicFont=cmuntx.ttf]{cmunbx.ttf}\bfseries p}\setmainfont[Path=/usr/share/fonts/truetype/cmu/,UprightFont=cmunrm.ttf,BoldFont=cmunbx.ttf,ItalicFont=cmunti.ttf,BoldItalicFont=cmunbi.ttf]{cmunrm.ttf}\setmonofont[Path=/usr/share/fonts/truetype/cmu/,UprightFont=cmuntt.ttf,BoldFont=cmuntb.ttf,ItalicFont=cmunit.ttf,BoldItalicFont=cmuntx.ttf]{cmunrm.ttf}age of floats\\ \hline \hspace*{0pt}\ignorespaces{}\hspace*{0pt} \LaTeXTT{!}&\hspace*{0pt}\ignorespaces{}\hspace*{0pt}override the default float restrictions. E.g., the maximum size allowed of a \LaTeXTT{b} float is normally quite small; if you want a large one, you need this \LaTeXTT{!} parameter as well.\\ \hline 
\end{longtable}


Default is {\itshape \setmainfont[Path=/usr/share/fonts/truetype/cmu/,UprightFont=cmunrm.ttf,BoldFont=cmunbx.ttf,ItalicFont=cmunti.ttf,BoldItalicFont=cmunbi.ttf]{cmunti.ttf}\setmonofont[Path=/usr/share/fonts/truetype/cmu/,UprightFont=cmuntt.ttf,BoldFont=cmuntb.ttf,ItalicFont=cmunit.ttf,BoldItalicFont=cmuntx.ttf]{cmunti.ttf}\itshape tbp}\setmainfont[Path=/usr/share/fonts/truetype/cmu/,UprightFont=cmunrm.ttf,BoldFont=cmunbx.ttf,ItalicFont=cmunti.ttf,BoldItalicFont=cmunbi.ttf]{cmunrm.ttf}\setmonofont[Path=/usr/share/fonts/truetype/cmu/,UprightFont=cmuntt.ttf,BoldFont=cmuntb.ttf,ItalicFont=cmunit.ttf,BoldItalicFont=cmuntx.ttf]{cmunrm.ttf}, which means that it is by default placed on the top of the page. If that\textquotesingle{}s not possible, it\textquotesingle{}s placed at the bottom if possible, or finally with other floating environments on an extra page.

You can force LaTeX to use one given position. E.g. {\itshape \setmainfont[Path=/usr/share/fonts/truetype/cmu/,UprightFont=cmunrm.ttf,BoldFont=cmunbx.ttf,ItalicFont=cmunti.ttf,BoldItalicFont=cmunbi.ttf]{cmunti.ttf}\setmonofont[Path=/usr/share/fonts/truetype/cmu/,UprightFont=cmuntt.ttf,BoldFont=cmuntb.ttf,ItalicFont=cmunit.ttf,BoldItalicFont=cmuntx.ttf]{cmunti.ttf}\itshape {$\text{[}$}!h{$\text{]}$}}{$\text{ }$}\setmainfont[Path=/usr/share/fonts/truetype/cmu/,UprightFont=cmunrm.ttf,BoldFont=cmunbx.ttf,ItalicFont=cmunti.ttf,BoldItalicFont=cmunbi.ttf]{cmunrm.ttf}\setmonofont[Path=/usr/share/fonts/truetype/cmu/,UprightFont=cmuntt.ttf,BoldFont=cmuntb.ttf,ItalicFont=cmunit.ttf,BoldItalicFont=cmuntx.ttf]{cmunrm.ttf} forces LaTeX to place it exactly where you place it (Except when it\textquotesingle{}s really impossible, e.g you place a table {\itshape \setmainfont[Path=/usr/share/fonts/truetype/cmu/,UprightFont=cmunrm.ttf,BoldFont=cmunbx.ttf,ItalicFont=cmunti.ttf,BoldItalicFont=cmunbi.ttf]{cmunti.ttf}\setmonofont[Path=/usr/share/fonts/truetype/cmu/,UprightFont=cmuntt.ttf,BoldFont=cmuntb.ttf,ItalicFont=cmunit.ttf,BoldItalicFont=cmuntx.ttf]{cmunti.ttf}\itshape here}{$\text{ }$}\setmainfont[Path=/usr/share/fonts/truetype/cmu/,UprightFont=cmunrm.ttf,BoldFont=cmunbx.ttf,ItalicFont=cmunti.ttf,BoldItalicFont=cmunbi.ttf]{cmunrm.ttf}\setmonofont[Path=/usr/share/fonts/truetype/cmu/,UprightFont=cmuntt.ttf,BoldFont=cmuntb.ttf,ItalicFont=cmunit.ttf,BoldItalicFont=cmuntx.ttf]{cmunrm.ttf} and this place would be the last line on a page). Again, understand it correctly: it urges LaTeX to put the table at a specific place, but it will not be placed there if LaTeX thinks it will not look great. If you really want to place your table manually, do not use the \LaTeXTT{table} environment.

Centering the table horizontally works like everything else, using the \LaTeXTT{\textbackslash{}centering} command just after opening the \LaTeXTT{table} environment, or by enclosing it with a \LaTeXTT{center} environment.
\section{Using spreadsheets and data analysis tools}
\label{289}

For complex or dynamic tables, you may want to use a spreadsheet. You might save lots of time by building tables using specialized software and exporting them in LaTeX format. The following plugins and libraries are available for some popular software:
\begin{myitemize}
\item{}  \myhref{http://calc2latex.sourceforge.net/}{calc2latex}: for OpenOffice.org Calc spreadsheets,
\item{}  \myhref{http://www.ctan.org/tex-archive/support/excel2latex/}{excel2latex}: for Microsoft Office Excel,
\item{}  \myhref{http://www.mathworks.com/matlabcentral/fileexchange/4894-matrix2latex}{matrix2latex}: for MATLAB,
\item{}  \myhref{https://github.com/TheChymera/matrix2latex}{matrix2latex}: for Python and MATLAB,
\item{}  \myhref{http://pandas.pydata.org/pandas-docs/stable/generated/pandas.DataFrame.to_latex.html}{pandas}: pandas DataFrame\textquotesingle{}s have a method to convert data they contain to latex,
\item{}  \myhref{http://rubygems.org/gems/latex-tools}{latex-{}tools}: a Ruby library,
\item{}  \myhref{http://cran.r-project.org/web/packages/xtable/index.html}{xtable}: a library for R,
\item{}  \myhref{http://orgmode.org/}{org-{}mode}: for Emacs users, org-{}mode tables can be used inline in LaTeX documents, see \myplainurl{https://www.gnu.org/software/emacs/manual/html_node/org/A-LaTeX-example.html} for a tutorial.
\item{}  \myhref{http://emacswiki.org/emacs/AlignCommands}{Emacs align commands}: the align commands can clean up a messy LaTeX table.
\item{}  \myhref{http://truben.no/latex/table/}{Online Table generator for LATeX}: An online tool for creating simple tables within the browser. LaTeX format is directly generated as you type.
\item{}  \myhref{http://www.tablesgenerator.com/}{Create LaTeX tables online }: Online tool.
\end{myitemize}


However, copying the generated source code to your document is not convenient at all. For maximum flexibility, generate the source code to a separate file which you can input from your main document file with the \LaTeXTT{\textbackslash{}input} command.
If your speadsheet supports command-{}line, you can generate your complete document (table included) in one command, using a Makefile for example.

See \mylref{895}{Modular Documents} for more details.
\section{Need more complicated features?}
\label{290}

Have a look at one of the following packages:

\begin{myitemize}
\item{}  \myhref{http://www.ctan.org/pkg/hhline}{{\ttfamily \setmainfont[Path=/usr/share/fonts/truetype/cmu/,UprightFont=cmunrm.ttf,BoldFont=cmunbx.ttf,ItalicFont=cmunti.ttf,BoldItalicFont=cmunbi.ttf]{cmuntt.ttf}\setmonofont[Path=/usr/share/fonts/truetype/cmu/,UprightFont=cmuntt.ttf,BoldFont=cmuntb.ttf,ItalicFont=cmunit.ttf,BoldItalicFont=cmuntx.ttf]{cmuntt.ttf}\ttfamily hhline}}:   do whatever you want with horizontal lines
\item{}  \myhref{http://www.ctan.org/pkg/array}{{\ttfamily \setmainfont[Path=/usr/share/fonts/truetype/cmu/,UprightFont=cmunrm.ttf,BoldFont=cmunbx.ttf,ItalicFont=cmunti.ttf,BoldItalicFont=cmunbi.ttf]{cmuntt.ttf}\setmonofont[Path=/usr/share/fonts/truetype/cmu/,UprightFont=cmuntt.ttf,BoldFont=cmuntb.ttf,ItalicFont=cmunit.ttf,BoldItalicFont=cmuntx.ttf]{cmuntt.ttf}\ttfamily array}}:    gives you more freedom on how to define columns
\item{}  \myhref{http://www.ctan.org/pkg/colortbl}{{\ttfamily \setmainfont[Path=/usr/share/fonts/truetype/cmu/,UprightFont=cmunrm.ttf,BoldFont=cmunbx.ttf,ItalicFont=cmunti.ttf,BoldItalicFont=cmunbi.ttf]{cmuntt.ttf}\setmonofont[Path=/usr/share/fonts/truetype/cmu/,UprightFont=cmuntt.ttf,BoldFont=cmuntb.ttf,ItalicFont=cmunit.ttf,BoldItalicFont=cmuntx.ttf]{cmuntt.ttf}\ttfamily colortbl}}: make your table more colorful
\item{}  \myhref{http://ctan.org/tex-archive/macros/latex/contrib/threeparttable}{{\ttfamily \setmainfont[Path=/usr/share/fonts/truetype/cmu/,UprightFont=cmunrm.ttf,BoldFont=cmunbx.ttf,ItalicFont=cmunti.ttf,BoldItalicFont=cmunbi.ttf]{cmuntt.ttf}\setmonofont[Path=/usr/share/fonts/truetype/cmu/,UprightFont=cmuntt.ttf,BoldFont=cmuntb.ttf,ItalicFont=cmunit.ttf,BoldItalicFont=cmuntx.ttf]{cmuntt.ttf}\ttfamily threeparttable}} makes it possible to put footnotes both within the table and its caption
\item{}  \myhref{http://www.ctan.org/pkg/arydshln}{{\ttfamily \setmainfont[Path=/usr/share/fonts/truetype/cmu/,UprightFont=cmunrm.ttf,BoldFont=cmunbx.ttf,ItalicFont=cmunti.ttf,BoldItalicFont=cmunbi.ttf]{cmuntt.ttf}\setmonofont[Path=/usr/share/fonts/truetype/cmu/,UprightFont=cmuntt.ttf,BoldFont=cmuntb.ttf,ItalicFont=cmunit.ttf,BoldItalicFont=cmuntx.ttf]{cmuntt.ttf}\ttfamily arydshln}}: creates dashed horizontal and vertical lines
\item{}  \myhref{http://www.ctan.org/pkg/ctable}{{\ttfamily \setmainfont[Path=/usr/share/fonts/truetype/cmu/,UprightFont=cmunrm.ttf,BoldFont=cmunbx.ttf,ItalicFont=cmunti.ttf,BoldItalicFont=cmunbi.ttf]{cmuntt.ttf}\setmonofont[Path=/usr/share/fonts/truetype/cmu/,UprightFont=cmuntt.ttf,BoldFont=cmuntb.ttf,ItalicFont=cmunit.ttf,BoldItalicFont=cmuntx.ttf]{cmuntt.ttf}\ttfamily ctable}}: allows for footnotes under table and properly spaced caption above (incorporates booktabs package)
\item{}  \myhref{http://www.ctan.org/pkg/slashbox}{{\ttfamily \setmainfont[Path=/usr/share/fonts/truetype/cmu/,UprightFont=cmunrm.ttf,BoldFont=cmunbx.ttf,ItalicFont=cmunti.ttf,BoldItalicFont=cmunbi.ttf]{cmuntt.ttf}\setmonofont[Path=/usr/share/fonts/truetype/cmu/,UprightFont=cmuntt.ttf,BoldFont=cmuntb.ttf,ItalicFont=cmunit.ttf,BoldItalicFont=cmuntx.ttf]{cmuntt.ttf}\ttfamily slashbox}}: create 2D tables with the first cell containing a description for both axes. Not available in Tex Live 2011 or later. 
\item{}  \myhref{http://mirror.jmu.edu/pub/CTAN/macros/latex/contrib/diagbox/}{{\ttfamily \setmainfont[Path=/usr/share/fonts/truetype/cmu/,UprightFont=cmunrm.ttf,BoldFont=cmunbx.ttf,ItalicFont=cmunti.ttf,BoldItalicFont=cmunbi.ttf]{cmuntt.ttf}\setmonofont[Path=/usr/share/fonts/truetype/cmu/,UprightFont=cmuntt.ttf,BoldFont=cmuntb.ttf,ItalicFont=cmunit.ttf,BoldItalicFont=cmuntx.ttf]{cmuntt.ttf}\ttfamily diagbox}}: compatible to {\ttfamily \setmainfont[Path=/usr/share/fonts/truetype/cmu/,UprightFont=cmunrm.ttf,BoldFont=cmunbx.ttf,ItalicFont=cmunti.ttf,BoldItalicFont=cmunbi.ttf]{cmuntt.ttf}\setmonofont[Path=/usr/share/fonts/truetype/cmu/,UprightFont=cmuntt.ttf,BoldFont=cmuntb.ttf,ItalicFont=cmunit.ttf,BoldItalicFont=cmuntx.ttf]{cmuntt.ttf}\ttfamily slashbox}\setmainfont[Path=/usr/share/fonts/truetype/cmu/,UprightFont=cmunrm.ttf,BoldFont=cmunbx.ttf,ItalicFont=cmunti.ttf,BoldItalicFont=cmunbi.ttf]{cmunrm.ttf}\setmonofont[Path=/usr/share/fonts/truetype/cmu/,UprightFont=cmuntt.ttf,BoldFont=cmuntb.ttf,ItalicFont=cmunit.ttf,BoldItalicFont=cmuntx.ttf]{cmunrm.ttf}, come with Tex Live 2011 or later
\item{}  \myhref{http://www.ctan.org/pkg/dcolumn}{{\ttfamily \setmainfont[Path=/usr/share/fonts/truetype/cmu/,UprightFont=cmunrm.ttf,BoldFont=cmunbx.ttf,ItalicFont=cmunti.ttf,BoldItalicFont=cmunbi.ttf]{cmuntt.ttf}\setmonofont[Path=/usr/share/fonts/truetype/cmu/,UprightFont=cmuntt.ttf,BoldFont=cmuntb.ttf,ItalicFont=cmunit.ttf,BoldItalicFont=cmuntx.ttf]{cmuntt.ttf}\ttfamily dcolumn}}: decimal point alignment of numeric cells
\item{}  \myhref{http://www.ctan.org/pkg/rccol}{{\ttfamily \setmainfont[Path=/usr/share/fonts/truetype/cmu/,UprightFont=cmunrm.ttf,BoldFont=cmunbx.ttf,ItalicFont=cmunti.ttf,BoldItalicFont=cmunbi.ttf]{cmuntt.ttf}\setmonofont[Path=/usr/share/fonts/truetype/cmu/,UprightFont=cmuntt.ttf,BoldFont=cmuntb.ttf,ItalicFont=cmunit.ttf,BoldItalicFont=cmuntx.ttf]{cmuntt.ttf}\ttfamily rccol}}: advanced decimal point alignment of numeric cells with rounding
\item{}  \myhref{http://www.ctan.org/pkg/numprint}{{\ttfamily \setmainfont[Path=/usr/share/fonts/truetype/cmu/,UprightFont=cmunrm.ttf,BoldFont=cmunbx.ttf,ItalicFont=cmunti.ttf,BoldItalicFont=cmunbi.ttf]{cmuntt.ttf}\setmonofont[Path=/usr/share/fonts/truetype/cmu/,UprightFont=cmuntt.ttf,BoldFont=cmuntb.ttf,ItalicFont=cmunit.ttf,BoldItalicFont=cmuntx.ttf]{cmuntt.ttf}\ttfamily numprint}}: print numbers, in the current mode (text or math) in order to use the correct font, with separators, exponent and/or rounded to a given number of digits. tabular(*), array, tabularx, and longtable environments is supported using all features of numprint
\item{}  \myhref{http://www.ctan.org/pkg/spreadtab}{{\ttfamily \setmainfont[Path=/usr/share/fonts/truetype/cmu/,UprightFont=cmunrm.ttf,BoldFont=cmunbx.ttf,ItalicFont=cmunti.ttf,BoldItalicFont=cmunbi.ttf]{cmuntt.ttf}\setmonofont[Path=/usr/share/fonts/truetype/cmu/,UprightFont=cmuntt.ttf,BoldFont=cmuntb.ttf,ItalicFont=cmunit.ttf,BoldItalicFont=cmuntx.ttf]{cmuntt.ttf}\ttfamily spreadtab}}: spread sheets allowing the use of formulae
\item{}  \myhref{http://ctan.org/tex-archive/macros/latex/contrib/siunitx}{{\ttfamily \setmainfont[Path=/usr/share/fonts/truetype/cmu/,UprightFont=cmunrm.ttf,BoldFont=cmunbx.ttf,ItalicFont=cmunti.ttf,BoldItalicFont=cmunbi.ttf]{cmuntt.ttf}\setmonofont[Path=/usr/share/fonts/truetype/cmu/,UprightFont=cmuntt.ttf,BoldFont=cmuntb.ttf,ItalicFont=cmunit.ttf,BoldItalicFont=cmuntx.ttf]{cmuntt.ttf}\ttfamily siunitx}}: alignment of tabular entries
\item{}  \myhref{http://www.ctan.org/tex-archive/graphics/pgf/contrib/pgfplots}{{\ttfamily \setmainfont[Path=/usr/share/fonts/truetype/cmu/,UprightFont=cmunrm.ttf,BoldFont=cmunbx.ttf,ItalicFont=cmunti.ttf,BoldItalicFont=cmunbi.ttf]{cmuntt.ttf}\setmonofont[Path=/usr/share/fonts/truetype/cmu/,UprightFont=cmuntt.ttf,BoldFont=cmuntb.ttf,ItalicFont=cmunit.ttf,BoldItalicFont=cmuntx.ttf]{cmuntt.ttf}\ttfamily pgfplotstable}}: Loads, rounds, formats and postprocesses numerical tables.
\end{myitemize}

\section{References}
\label{291}

\LaTeXNullTemplate{}



\myhref{https://fr.wikibooks.org/wiki/LaTeX\%2FFaire_des_tableaux}{fr:LaTeX/Faire\_des\_tableaux}
\myhref{https://nl.wikibooks.org/wiki/LaTeX\%2FTabellen}{nl:LaTeX/Tabellen}
\myhref{https://pl.wikibooks.org/wiki/LaTeX\%2FTabele}{pl:LaTeX/Tabele}\chapter{Title creation}

\myminitoc
\label{292}

\label{293}


For documents such as basic articles, the output of
\LaTeXTT{\textbackslash{}maketitle} is often adequate, but longer
documents (such as books and reports) often require more involved
formatting. We will detail the process here.

There are several situations where you might want to create a title in a
custom format, rather than in the format natively supported by LaTeX
classes.  While it is possible to change the output of
\LaTeXTT{\textbackslash{}maketitle}, it can be complicated even with
minor changes to the title.  In such cases it is often better to create
the title from scratch, and this section will show you how to accomplish
this.
\section{Standard Titles}
\label{294}

Many document classes will form a title or a title page for you.  One
must specify what to fill it with using these commands placed in the
\mylref{95}{top matter}:

\begin{Shaded}
\begin{Highlighting}[]

\NormalTok{\textbackslash{}title\{The Triangulation of Titling Data in Non-Linear Gaussian Fashion via}
 \NormalTok{$\textbackslash{}rho$ Series\}}
\NormalTok{\textbackslash{}date\{October 31, 2014\}}
\NormalTok{\textbackslash{}author\{John Doe\textbackslash{}\textbackslash{} Magic Department, Richard Miles University \textbackslash{}and Richard Row,}
 \NormalTok{\textbackslash{}LaTeX\textbackslash{} Academy\}}
\end{Highlighting}
\end{Shaded}


Commonly the date is excluded from the title page by using
\LaTeXTT{\textbackslash{}date\{\}}. It defaults to
\LaTeXTT{\textbackslash{}today} if omitted in the source file.

To form a title, use

\begin{Shaded}
\begin{Highlighting}[]

\NormalTok{\textbackslash{}maketitle}
\end{Highlighting}
\end{Shaded}


This should go after the preceding commands. For most document classes,
this will form a separate page, while the \LaTeXTT{article}
document class will place the title on the top of the first page. 
If you want to have a separate title page for articles as well, use the
documentclass option \LaTeXTT{titlepage}.


Footnotes within the title page can be specified with the
\LaTeXTT{\textbackslash{}thanks} command. For example, one may add

\begin{Shaded}
\begin{Highlighting}[]

\NormalTok{\textbackslash{}author\{John Doe\textbackslash{}thanks\{Funded by NASA Grant \textbackslash{}#42\}\}}
\end{Highlighting}
\end{Shaded}


The \LaTeXTT{\textbackslash{}thanks} command can also be used in the
\LaTeXTT{\textbackslash{}title}.

It is dependent on the document class which commands are used in the
title generated by \LaTeXTT{\textbackslash{}maketitle}. Referring to
the documentation will lead to trusted information.
\section{Custom Title Pages}
\label{295}

Normally, the benefit of using LaTeX instead of traditional word
processing programs is that LaTeX frees you to concentrate on content by
handling margins, justification, and other typesetting concerns. On the
other hand, if you want to write your own title format, it is exactly
the opposite: you have to take care of everything — this time LaTeX will
do nothing to help you. It can be challenging to create your own title
format since LaTeX was not designed to be graphically interactive in the
adjustment of layout. The process is similar to working with raw HTML
with the added step that each time you want to see how your changes
look, you have to re-{}compile the source. While this may seem like a
major inconvenience, the benefit is that once the customized title
format has been written, it serves as a template for all other documents
that would use the title format you have just made. In other words, once
you have a layout you like, you can use it for any other documents where
you would like the same layout without any additional fiddling with
layout.

The title page of a book or a report is the first a reader will see.
Keep that in mind when preparing your title page.
\subsection{Create the title for a report or book}
\label{296}

A title page for reports to get a university degree is quite
static, it doesn\textquotesingle{}t really change over time.  You can prepare the
titlepage in its own little document and prepare a one page pdf
that you later include into your real document. This is really
useful, if the title page is required to have completely
different margins as the rest of the document.  It also saves
compile time, though it is not much. 

You need to know very basic LaTeX layout commands  in order to
get your own title page perfect. Usually a custom titlepage does
not contain any semantic markup, everything is hand crafted.
Here are some of the most often needed things:
{\bfseries
\begin{mydescription} Alignment 
\end{mydescription}
}

if you want to center some text just use
\LaTeXTT{\textbackslash{}centering}. If you want
to align it differently you can use the environment
\LaTeXTT{\textbackslash{}raggedleft} for {\itshape \setmainfont[Path=/usr/share/fonts/truetype/cmu/,UprightFont=cmunrm.ttf,BoldFont=cmunbx.ttf,ItalicFont=cmunti.ttf,BoldItalicFont=cmunbi.ttf]{cmunti.ttf}\setmonofont[Path=/usr/share/fonts/truetype/cmu/,UprightFont=cmuntt.ttf,BoldFont=cmuntb.ttf,ItalicFont=cmunit.ttf,BoldItalicFont=cmuntx.ttf]{cmunti.ttf}\itshape right}\setmainfont[Path=/usr/share/fonts/truetype/cmu/,UprightFont=cmunrm.ttf,BoldFont=cmunbx.ttf,ItalicFont=cmunti.ttf,BoldItalicFont=cmunbi.ttf]{cmunrm.ttf}\setmonofont[Path=/usr/share/fonts/truetype/cmu/,UprightFont=cmuntt.ttf,BoldFont=cmuntb.ttf,ItalicFont=cmunit.ttf,BoldItalicFont=cmuntx.ttf]{cmunrm.ttf}-{}alignment and
\LaTeXTT{\textbackslash{}raggedright} for {\itshape \setmainfont[Path=/usr/share/fonts/truetype/cmu/,UprightFont=cmunrm.ttf,BoldFont=cmunbx.ttf,ItalicFont=cmunti.ttf,BoldItalicFont=cmunbi.ttf]{cmunti.ttf}\setmonofont[Path=/usr/share/fonts/truetype/cmu/,UprightFont=cmuntt.ttf,BoldFont=cmuntb.ttf,ItalicFont=cmunit.ttf,BoldItalicFont=cmuntx.ttf]{cmunti.ttf}\itshape left}\setmainfont[Path=/usr/share/fonts/truetype/cmu/,UprightFont=cmunrm.ttf,BoldFont=cmunbx.ttf,ItalicFont=cmunti.ttf,BoldItalicFont=cmunbi.ttf]{cmunrm.ttf}\setmonofont[Path=/usr/share/fonts/truetype/cmu/,UprightFont=cmuntt.ttf,BoldFont=cmuntb.ttf,ItalicFont=cmunit.ttf,BoldItalicFont=cmuntx.ttf]{cmunrm.ttf}-{}alignment.  {\bfseries
\begin{mydescription} Images 
\end{mydescription}
}

the command for including images (a logo for example) is the following :
\LaTeXTT{\textbackslash{}includegraphics{$\text{[}$}width=0.15\textbackslash{}textwidth{$\text{]}$}\{./logo\}}.
There is no \LaTeXTT{\textbackslash{}begin\{figure\}} as you
	would usually use
since you don\textquotesingle{}t want it to be \mylref{362}{floating},
you just want it exactly where want it to be. When
handling it, remember that it is considered like a big box by the TeX
engine.  {\bfseries
\begin{mydescription} Text size 
\end{mydescription}
}
\begin{myquote}\item{} 
\end{myquote}

If you want to change the size of some text just
place it within braces, {\itshape \setmainfont[Path=/usr/share/fonts/truetype/cmu/,UprightFont=cmunrm.ttf,BoldFont=cmunbx.ttf,ItalicFont=cmunti.ttf,BoldItalicFont=cmunbi.ttf]{cmunti.ttf}\setmonofont[Path=/usr/share/fonts/truetype/cmu/,UprightFont=cmuntt.ttf,BoldFont=cmuntb.ttf,ItalicFont=cmunit.ttf,BoldItalicFont=cmuntx.ttf]{cmunti.ttf}\itshape \{like this\}}\setmainfont[Path=/usr/share/fonts/truetype/cmu/,UprightFont=cmunrm.ttf,BoldFont=cmunbx.ttf,ItalicFont=cmunti.ttf,BoldItalicFont=cmunbi.ttf]{cmunrm.ttf}\setmonofont[Path=/usr/share/fonts/truetype/cmu/,UprightFont=cmuntt.ttf,BoldFont=cmuntb.ttf,ItalicFont=cmunit.ttf,BoldItalicFont=cmuntx.ttf]{cmunrm.ttf}, and you can use the following
commands (in order of size): \LaTeXTT{\textbackslash{}Huge},
\LaTeXTT{\textbackslash{}huge}, \LaTeXTT{\textbackslash{}LARGE},
\LaTeXTT{\textbackslash{}Large}, \LaTeXTT{\textbackslash{}large},
\LaTeXTT{\textbackslash{}normalsize},
\LaTeXTT{\textbackslash{}small},
\LaTeXTT{\textbackslash{}footnotesize},
\LaTeXTT{\textbackslash{}tiny}. So for example: 

\begin{Shaded}
\begin{Highlighting}[]

\NormalTok{\{\textbackslash{}large this text is slightly bigger than normal\}, this one is not.}
\end{Highlighting}
\end{Shaded}


Remember, if you have a block of text in a different size, even if it is
a bit of text on a single line, end it with
\LaTeXTT{\textbackslash{}par}. {\bfseries
\begin{mydescription} Filling the page 
\end{mydescription}
}

the command \LaTeXTT{\textbackslash{}vfill} as
the last item of your content will add empty space until the page is
full. If you put it within the page, you will ensure that all the
following text will be placed at the bottom of the page.
\subsubsection{A practical example}
\label{297}

All these tips might have made you confused.  Here is a practical
and compilable example. The picture in use comes with
package {\ttfamily \setmainfont[Path=/usr/share/fonts/truetype/cmu/,UprightFont=cmunrm.ttf,BoldFont=cmunbx.ttf,ItalicFont=cmunti.ttf,BoldItalicFont=cmunbi.ttf]{cmuntt.ttf}\setmonofont[Path=/usr/share/fonts/truetype/cmu/,UprightFont=cmuntt.ttf,BoldFont=cmuntb.ttf,ItalicFont=cmunit.ttf,BoldItalicFont=cmuntx.ttf]{cmuntt.ttf}\ttfamily mwe}{$\text{ }$}\setmainfont[Path=/usr/share/fonts/truetype/cmu/,UprightFont=cmunrm.ttf,BoldFont=cmunbx.ttf,ItalicFont=cmunti.ttf,BoldItalicFont=cmunbi.ttf]{cmunrm.ttf}\setmonofont[Path=/usr/share/fonts/truetype/cmu/,UprightFont=cmuntt.ttf,BoldFont=cmuntb.ttf,ItalicFont=cmunit.ttf,BoldItalicFont=cmuntx.ttf]{cmunrm.ttf} and should be available with every complete LaTeX
installation. You can start testing right away. 

\begin{Shaded}
\begin{Highlighting}[]

\NormalTok{\textbackslash{}documentclass[12pt,a4paper]\{report\}}
\NormalTok{\textbackslash{}usepackage\{graphicx\}}
\NormalTok{\textbackslash{}begin\{document\}}
\NormalTok{\textbackslash{}begin\{titlepage\}}
	\NormalTok{\textbackslash{}centering}
	\NormalTok{\textbackslash{}includegraphics[width=0.15\textbackslash{}textwidth]\{example-image-1x1\}\textbackslash{}par\textbackslash{}vspace\{1cm\}}
\NormalTok{	\{\textbackslash{}scshape\textbackslash{}LARGE Columbidae University \textbackslash{}par\}}
	\NormalTok{\textbackslash{}vspace\{1cm\}}
\NormalTok{	\{\textbackslash{}scshape\textbackslash{}Large Final year project\textbackslash{}par\}}
	\NormalTok{\textbackslash{}vspace\{1.5cm\}}
\NormalTok{	\{\textbackslash{}huge\textbackslash{}bfseries Pigeons love doves\textbackslash{}par\}}
	\NormalTok{\textbackslash{}vspace\{2cm\}}
\NormalTok{	\{\textbackslash{}Large\textbackslash{}itshape John Birdwatch\textbackslash{}par\}}
	\NormalTok{\textbackslash{}vfill}
	\NormalTok{supervised by\textbackslash{}par}
	\NormalTok{Dr.~Mark \textbackslash{}textsc\{Brown\}}
 
	\NormalTok{\textbackslash{}vfill}
 
\CommentTok{% Bottom of the page}
	\NormalTok{\{\textbackslash{}large \textbackslash{}today\textbackslash{}par\}}
\NormalTok{\textbackslash{}end\{titlepage\}}
\NormalTok{\textbackslash{}end\{document\}}
\end{Highlighting}
\end{Shaded}



As you can see, the code looks \symbol{34}dirtier\symbol{34} than standard LaTeX source
because you have to take care of the output as well. If you start
changing fonts it gets even more complicated, but you can do it:
it\textquotesingle{}s only for the title and your complicated code will be isolated from
all the rest within its own file.


The result is shown below



\begin{minipage}{1.0\linewidth}
\begin{center}
\includegraphics[width=1.0\linewidth,height=6.5in,keepaspectratio]{../images/59.png}
\end{center}
\raggedright{}\myfigurewithcaption{59}{TitlepageWikibook}
\end{minipage}\vspace{0.75cm}


\subsubsection{Integrating the title page}
\label{298}

Assuming you have done the title page of your report in an extra
document, let\textquotesingle{}s pretend it is called
{\ttfamily \setmainfont[Path=/usr/share/fonts/truetype/cmu/,UprightFont=cmunrm.ttf,BoldFont=cmunbx.ttf,ItalicFont=cmunti.ttf,BoldItalicFont=cmunbi.ttf]{cmuntt.ttf}\setmonofont[Path=/usr/share/fonts/truetype/cmu/,UprightFont=cmuntt.ttf,BoldFont=cmuntb.ttf,ItalicFont=cmunit.ttf,BoldItalicFont=cmuntx.ttf]{cmuntt.ttf}\ttfamily reportTitlepage2015.pdf}\setmainfont[Path=/usr/share/fonts/truetype/cmu/,UprightFont=cmunrm.ttf,BoldFont=cmunbx.ttf,ItalicFont=cmunti.ttf,BoldItalicFont=cmunbi.ttf]{cmunrm.ttf}\setmonofont[Path=/usr/share/fonts/truetype/cmu/,UprightFont=cmuntt.ttf,BoldFont=cmuntb.ttf,ItalicFont=cmunit.ttf,BoldItalicFont=cmuntx.ttf]{cmunrm.ttf}, you can include it quite
simply. Here is a short document setup.

\begin{Shaded}
\begin{Highlighting}[]

\NormalTok{\textbackslash{}documentclass\{report\}}
\NormalTok{\textbackslash{}usepackage\{pdfpages\}}
\NormalTok{\textbackslash{}begin\{document\}}
\NormalTok{\textbackslash{}includepdf\{reportTitlepage2015\}}
\NormalTok{\textbackslash{}tableofcontents}
\NormalTok{\textbackslash{}chapter\{Introducing birds\}}
\NormalTok{\textbackslash{}end\{document\}}
\end{Highlighting}
\end{Shaded}

\subsection{A title to be re-{}used multiple times}
\label{299}

Some universities, departments and companies have strict rules
how a title page of a report should look like. To ensure the very
same output for all reports, a redefiniton of the
\LaTeXTT{\textbackslash{}maketitle} command is recommended.

This is best done by an experienced LaTeX user. A simple example
follows, as usual there is no real limit with respect to
complexity. 


As a starting point, a LaTeX package called
{\ttfamily \setmainfont[Path=/usr/share/fonts/truetype/cmu/,UprightFont=cmunrm.ttf,BoldFont=cmunbx.ttf,ItalicFont=cmunti.ttf,BoldItalicFont=cmunbi.ttf]{cmuntt.ttf}\setmonofont[Path=/usr/share/fonts/truetype/cmu/,UprightFont=cmuntt.ttf,BoldFont=cmuntb.ttf,ItalicFont=cmunit.ttf,BoldItalicFont=cmuntx.ttf]{cmuntt.ttf}\ttfamily columbidaeTitle.sty}{$\text{ }$}\setmainfont[Path=/usr/share/fonts/truetype/cmu/,UprightFont=cmunrm.ttf,BoldFont=cmunbx.ttf,ItalicFont=cmunti.ttf,BoldItalicFont=cmunbi.ttf]{cmunrm.ttf}\setmonofont[Path=/usr/share/fonts/truetype/cmu/,UprightFont=cmuntt.ttf,BoldFont=cmuntb.ttf,ItalicFont=cmunit.ttf,BoldItalicFont=cmuntx.ttf]{cmunrm.ttf} is generated that defines the
complete title matter. It will later be hidden from the end user.
Ideally, the person creating the package should maintain it for a
long time, create an accompanying documentation and ensure user
support. 

\begin{Shaded}
\begin{Highlighting}[]

\CommentTok{% Copyright note: This package defines how titles should}
\CommentTok{% be typeset at the columbidae University}
\CommentTok
\NormalTok{\}}
\NormalTok{\textbackslash{}newcommand*\{\textbackslash{}@project\}\{Final Year Project\}}
\NormalTok{\textbackslash{}newcommand*\{\textbackslash{}supervisor\}[1]\{\textbackslash{}gdef\textbackslash{}@supervisor\{#1\}}\CommentTok
\NormalTok{\textbackslash{}begin\{titlepage\}}
\NormalTok{\{\textbackslash{}raggedleft}\CommentTok{%}
	\NormalTok{\textbackslash{}includegraphics[width=3cm]\{example-image-16x9\}\textbackslash{}par}
\NormalTok{\}\textbackslash{}vspace\{1cm\}}
	\NormalTok{\textbackslash{}centering}
\NormalTok{\{\textbackslash{}scshape\textbackslash{}LARGE Columbidae University \textbackslash{}par\}}
\NormalTok{\textbackslash{}vspace\{1cm\}}
\NormalTok{\{\textbackslash{}scshape\textbackslash{}Large\textbackslash{}@project\textbackslash{}unskip\textbackslash{}strut\textbackslash{}par\}}
\NormalTok{\textbackslash{}vspace\{1.5cm\}}
\NormalTok{\{\textbackslash{}huge\textbackslash{}bfseries\textbackslash{}@title\textbackslash{}unskip\textbackslash{}strut\textbackslash{}par\}}
\NormalTok{\textbackslash{}vspace\{2cm\}}
\NormalTok{\{\textbackslash{}Large\textbackslash{}itshape\textbackslash{}@author\textbackslash{}unskip\textbackslash{}strut\textbackslash{}par\}}
\NormalTok{\textbackslash{}vfill}
\NormalTok{supervised by\textbackslash{}par}
\NormalTok{\textbackslash{}@supervisor\textbackslash{}unskip\textbackslash{}strut\textbackslash{}par}
 
\NormalTok{\textbackslash{}vfill}
 
\NormalTok{\{\textbackslash{}large \textbackslash{}@date\textbackslash{}par\}}
\NormalTok{\textbackslash{}end\{titlepage\}}
\NormalTok{\}}
\NormalTok{\textbackslash{}endinput}
\end{Highlighting}
\end{Shaded}



This package can be loaded within a usual document. The user can
set the variables for {\ttfamily \setmainfont[Path=/usr/share/fonts/truetype/cmu/,UprightFont=cmunrm.ttf,BoldFont=cmunbx.ttf,ItalicFont=cmunti.ttf,BoldItalicFont=cmunbi.ttf]{cmuntt.ttf}\setmonofont[Path=/usr/share/fonts/truetype/cmu/,UprightFont=cmuntt.ttf,BoldFont=cmuntb.ttf,ItalicFont=cmunit.ttf,BoldItalicFont=cmuntx.ttf]{cmuntt.ttf}\ttfamily title}{$\text{ }$}\setmainfont[Path=/usr/share/fonts/truetype/cmu/,UprightFont=cmunrm.ttf,BoldFont=cmunbx.ttf,ItalicFont=cmunti.ttf,BoldItalicFont=cmunbi.ttf]{cmunrm.ttf}\setmonofont[Path=/usr/share/fonts/truetype/cmu/,UprightFont=cmuntt.ttf,BoldFont=cmuntb.ttf,ItalicFont=cmunit.ttf,BoldItalicFont=cmuntx.ttf]{cmunrm.ttf} and the like. Which commands
are actually available, and which might be omissible should be
written in a documentation that is bundled with the package. 

Look around what happens if you leave one or the other command
out.


\begin{Shaded}
\begin{Highlighting}[]

\NormalTok{\textbackslash{}documentclass\{book\}}
\NormalTok{\textbackslash{}usepackage\{columbidaeTitle\}}
\CommentTok{%\textbackslash{}supervisor\{Dr. James Miller\}}
\NormalTok{\textbackslash{}project\{Bachelor Thesis\}}
\NormalTok{\textbackslash{}author\{A LaTeX enthusiast\}}
\NormalTok{\textbackslash{}title\{Why i want to be a duck\}}
\NormalTok{\textbackslash{}begin\{document\}}
\NormalTok{\textbackslash{}maketitle}
\NormalTok{\textbackslash{}tableofcontents}
\NormalTok{\textbackslash{}chapter\{Ducks are awesome\}}
\NormalTok{\textbackslash{}end\{document\}}
\end{Highlighting}
\end{Shaded}

\section{Packages for custom titles}
\label{300}
The \LaTeXTT{titling}
package\myfootnote{{$\text{[}$}\myplainurl{http://www.ctan.org/tex-archive/macros/latex/contrib/titling}
Titling package webpage in CTAN{$\text{]}$}} provides control over the
typesetting of the \LaTeXTT{\textbackslash{}maketitle} and
\LaTeXTT{\textbackslash{}thanks} commands. The
\LaTeXTT{titlepages} package presents many styles of designs for
title pages. Italian users may also want to use the
\LaTeXTT{frontespizio}
package\myfootnote{{$\text{[}$}\myplainurl{http://www.ctan.org/tex-archive/macros/latex/contrib/frontespizio}
Frontespizio package webpage in CTAN{$\text{]}$}}.
\section{Notes and References}
\label{301}
\LaTeXNullTemplate{}



\myhref{https://pt.wikibooks.org/wiki/Latex\%2FT\%C3\%ADtulo}{pt:Latex/Título}\chapter{Page Layout}

\myminitoc
\label{302}

\label{303}


LaTeX and the document class will normally take care of page layout issues for you. For submission to an academic publication, this entire topic will be out of your hands, as the publishers want to control the presentation. However, for your own documents, there are some obvious settings that you may wish to change: margins, page orientation and columns, to name but three. The purpose of this tutorial is to show you how to configure your pages.

We will often have to deal with TeX lengths in this chapter. You should have a look at \mylref{456}{Lengths} for comprehensive details on the topic.
\section{Two-{}sided documents}
\label{304}

Documents can be either one-{} or two-{}sided. Articles are by default one-{}sided, books are two-{}sided. Two-{}sided documents differentiate the left (even) and right (odd) pages, whereas one-{}sided do not. The most notable effect can be seen in page margins. If you want to make the {\itshape \setmainfont[Path=/usr/share/fonts/truetype/cmu/,UprightFont=cmunrm.ttf,BoldFont=cmunbx.ttf,ItalicFont=cmunti.ttf,BoldItalicFont=cmunbi.ttf]{cmunti.ttf}\setmonofont[Path=/usr/share/fonts/truetype/cmu/,UprightFont=cmuntt.ttf,BoldFont=cmuntb.ttf,ItalicFont=cmunit.ttf,BoldItalicFont=cmuntx.ttf]{cmunti.ttf}\itshape article class two-{}sided}\setmainfont[Path=/usr/share/fonts/truetype/cmu/,UprightFont=cmunrm.ttf,BoldFont=cmunbx.ttf,ItalicFont=cmunti.ttf,BoldItalicFont=cmunbi.ttf]{cmunrm.ttf}\setmonofont[Path=/usr/share/fonts/truetype/cmu/,UprightFont=cmuntt.ttf,BoldFont=cmuntb.ttf,ItalicFont=cmunit.ttf,BoldItalicFont=cmuntx.ttf]{cmunrm.ttf}, use \LaTeXTT{\textbackslash{}documentclass{$\text{[}$}twoside{$\text{]}$}\{article\}}.

Many commands and variables in LaTeX take this concept into account. They are referred to as {\itshape \setmainfont[Path=/usr/share/fonts/truetype/cmu/,UprightFont=cmunrm.ttf,BoldFont=cmunbx.ttf,ItalicFont=cmunti.ttf,BoldItalicFont=cmunbi.ttf]{cmunti.ttf}\setmonofont[Path=/usr/share/fonts/truetype/cmu/,UprightFont=cmuntt.ttf,BoldFont=cmuntb.ttf,ItalicFont=cmunit.ttf,BoldItalicFont=cmuntx.ttf]{cmunti.ttf}\itshape even}{$\text{ }$}\setmainfont[Path=/usr/share/fonts/truetype/cmu/,UprightFont=cmunrm.ttf,BoldFont=cmunbx.ttf,ItalicFont=cmunti.ttf,BoldItalicFont=cmunbi.ttf]{cmunrm.ttf}\setmonofont[Path=/usr/share/fonts/truetype/cmu/,UprightFont=cmuntt.ttf,BoldFont=cmuntb.ttf,ItalicFont=cmunit.ttf,BoldItalicFont=cmuntx.ttf]{cmunrm.ttf} and {\itshape \setmainfont[Path=/usr/share/fonts/truetype/cmu/,UprightFont=cmunrm.ttf,BoldFont=cmunbx.ttf,ItalicFont=cmunti.ttf,BoldItalicFont=cmunbi.ttf]{cmunti.ttf}\setmonofont[Path=/usr/share/fonts/truetype/cmu/,UprightFont=cmuntt.ttf,BoldFont=cmuntb.ttf,ItalicFont=cmunit.ttf,BoldItalicFont=cmuntx.ttf]{cmunti.ttf}\itshape odd}\setmainfont[Path=/usr/share/fonts/truetype/cmu/,UprightFont=cmunrm.ttf,BoldFont=cmunbx.ttf,ItalicFont=cmunti.ttf,BoldItalicFont=cmunbi.ttf]{cmunrm.ttf}\setmonofont[Path=/usr/share/fonts/truetype/cmu/,UprightFont=cmuntt.ttf,BoldFont=cmuntb.ttf,ItalicFont=cmunit.ttf,BoldItalicFont=cmuntx.ttf]{cmunrm.ttf}.
For one-{}sided document, only the {\itshape \setmainfont[Path=/usr/share/fonts/truetype/cmu/,UprightFont=cmunrm.ttf,BoldFont=cmunbx.ttf,ItalicFont=cmunti.ttf,BoldItalicFont=cmunbi.ttf]{cmunti.ttf}\setmonofont[Path=/usr/share/fonts/truetype/cmu/,UprightFont=cmuntt.ttf,BoldFont=cmuntb.ttf,ItalicFont=cmunit.ttf,BoldItalicFont=cmuntx.ttf]{cmunti.ttf}\itshape odd}{$\text{ }$}\setmainfont[Path=/usr/share/fonts/truetype/cmu/,UprightFont=cmunrm.ttf,BoldFont=cmunbx.ttf,ItalicFont=cmunti.ttf,BoldItalicFont=cmunbi.ttf]{cmunrm.ttf}\setmonofont[Path=/usr/share/fonts/truetype/cmu/,UprightFont=cmuntt.ttf,BoldFont=cmuntb.ttf,ItalicFont=cmunit.ttf,BoldItalicFont=cmuntx.ttf]{cmunrm.ttf} commands and variables will be in effect.
\section{Page dimensions}
\label{305}

A page in LaTeX is defined by many internal parameters. Each parameter corresponds to the length of an element of the page, for example, \LaTeXTT{\textbackslash{}paperheight} is the physical height of the page. Here you can see a diagram showing all the variables defining the page. All sizes are given in TeX points (pt), there are 72.27pt in an inch or 1pt \setmainfont[Path=/usr/share/fonts/truetype/freefont/,UprightFont=FreeSerif.ttf,BoldFont=FreeSerifBold.ttf,ItalicFont=FreeSerifItalic.ttf,BoldItalicFont=FreeSerifBoldItalic.ttf]{FreeSerif.ttf}\setmonofont[Path=/usr/share/fonts/truetype/freefont/,UprightFont=FreeMono.ttf,BoldFont=FreeMonoBold.ttf,ItalicFont=FreeMonoOblique.ttf,BoldItalicFont=FreeMonoBoldOblique.ttf]{FreeSerif.ttf}≈\setmainfont[Path=/usr/share/fonts/truetype/cmu/,UprightFont=cmunrm.ttf,BoldFont=cmunbx.ttf,ItalicFont=cmunti.ttf,BoldItalicFont=cmunbi.ttf]{cmunrm.ttf}\setmonofont[Path=/usr/share/fonts/truetype/cmu/,UprightFont=cmuntt.ttf,BoldFont=cmuntb.ttf,ItalicFont=cmunit.ttf,BoldItalicFont=cmuntx.ttf]{cmunrm.ttf} 0.3515mm.




\begin{minipage}{1.0\linewidth}
\begin{center}
\includegraphics[width=1.0\linewidth,height=6.5in,keepaspectratio]{../images/60.\SVGExtension}
\end{center}
\raggedright{}\myfigurewithoutcaption{60}
\end{minipage}\vspace{0.75cm}



\begin{myenumerate}
\item{}  one inch + \LaTeXTT{\textbackslash{}hoffset}
\item{}  one inch + \LaTeXTT{\textbackslash{}voffset}
\item{}  \LaTeXTT{\textbackslash{}oddsidemargin} = 31pt
\item{}  \LaTeXTT{\textbackslash{}topmargin} = 20pt
\item{}  \LaTeXTT{\textbackslash{}headheight} = 12pt
\item{}  \LaTeXTT{\textbackslash{}headsep} = 25pt
\item{}  \LaTeXTT{\textbackslash{}textheight} = 592pt
\item{}  \LaTeXTT{\textbackslash{}textwidth} = 390pt
\item{}  \LaTeXTT{\textbackslash{}marginparsep} = 10pt
\item{}  \LaTeXTT{\textbackslash{}marginparwidth} = 35pt
\item{}  \LaTeXTT{\textbackslash{}footskip} = 30pt
\end{myenumerate}

\begin{myitemize}
\item{}  \LaTeXTT{\textbackslash{}marginparpush} = 7pt (not shown)
\item{}  \LaTeXTT{\textbackslash{}hoffset} = 0pt
\item{}  \LaTeXTT{\textbackslash{}voffset} = 0pt
\item{}  \LaTeXTT{\textbackslash{}paperwidth} = 597pt
\item{}  \LaTeXTT{\textbackslash{}paperheight} = 845pt
\end{myitemize}




The current details plus the layout shape can be printed from a LaTeX document itself. Use the {\itshape \setmainfont[Path=/usr/share/fonts/truetype/cmu/,UprightFont=cmunrm.ttf,BoldFont=cmunbx.ttf,ItalicFont=cmunti.ttf,BoldItalicFont=cmunbi.ttf]{cmunti.ttf}\setmonofont[Path=/usr/share/fonts/truetype/cmu/,UprightFont=cmuntt.ttf,BoldFont=cmuntb.ttf,ItalicFont=cmunit.ttf,BoldItalicFont=cmuntx.ttf]{cmunti.ttf}\itshape layout}{$\text{ }$}\setmainfont[Path=/usr/share/fonts/truetype/cmu/,UprightFont=cmunrm.ttf,BoldFont=cmunbx.ttf,ItalicFont=cmunti.ttf,BoldItalicFont=cmunbi.ttf]{cmunrm.ttf}\setmonofont[Path=/usr/share/fonts/truetype/cmu/,UprightFont=cmuntt.ttf,BoldFont=cmuntb.ttf,ItalicFont=cmunit.ttf,BoldItalicFont=cmuntx.ttf]{cmunrm.ttf} package and the command of the same name: 
\begin{Shaded}
\begin{Highlighting}[]

\NormalTok{\textbackslash{}usepackage\{layout\}\ensuremath{\text{ }}...\ensuremath{\text{ }}\textbackslash{}layout\{\}}\newline
\end{Highlighting}
\end{Shaded}

To render a frame marking the margins of a document you are currently working on, add 
\begin{Shaded}
\begin{Highlighting}[]

\NormalTok{\textbackslash{}usepackage\{showframe\}}\newline
\end{Highlighting}
\end{Shaded}
 to the document.
\section{Page size}
\label{306}

It will not have been immediately obvious -{} because it doesn\textquotesingle{}t really cause any serious problems -{} that the default page size for all standard document classes is {\itshape \setmainfont[Path=/usr/share/fonts/truetype/cmu/,UprightFont=cmunrm.ttf,BoldFont=cmunbx.ttf,ItalicFont=cmunti.ttf,BoldItalicFont=cmunbi.ttf]{cmunti.ttf}\setmonofont[Path=/usr/share/fonts/truetype/cmu/,UprightFont=cmuntt.ttf,BoldFont=cmuntb.ttf,ItalicFont=cmunit.ttf,BoldItalicFont=cmuntx.ttf]{cmunti.ttf}\itshape US letter}\setmainfont[Path=/usr/share/fonts/truetype/cmu/,UprightFont=cmunrm.ttf,BoldFont=cmunbx.ttf,ItalicFont=cmunti.ttf,BoldItalicFont=cmunbi.ttf]{cmunrm.ttf}\setmonofont[Path=/usr/share/fonts/truetype/cmu/,UprightFont=cmuntt.ttf,BoldFont=cmuntb.ttf,ItalicFont=cmunit.ttf,BoldItalicFont=cmuntx.ttf]{cmunrm.ttf}. This is shorter by 18 mm (about 3/4 inch), and slightly wider by 8 mm (about 1/4 inch), compared to A4 (which is the standard in almost all the rest of the world). While this is not a serious issue (most printers will print the document without any problems), it is possible to specify alternative sizes as \mylref{92}{class option}. For A4 format:

\begin{Shaded}
\begin{Highlighting}[]

\NormalTok{\textbackslash{}documentclass[a4paper]\{article\}}
\end{Highlighting}
\end{Shaded}


\begin{TemplateInfo}{\danger}{Warning}Note that the standard LaTeX classes use {\itshape \setmainfont[Path=/usr/share/fonts/truetype/cmu/,UprightFont=cmunrm.ttf,BoldFont=cmunbx.ttf,ItalicFont=cmunti.ttf,BoldItalicFont=cmunbi.ttf]{cmunti.ttf}\setmonofont[Path=/usr/share/fonts/truetype/cmu/,UprightFont=cmuntt.ttf,BoldFont=cmuntb.ttf,ItalicFont=cmunit.ttf,BoldItalicFont=cmuntx.ttf]{cmunti.ttf}\itshape US Letter}{$\text{ }$}\setmainfont[Path=/usr/share/fonts/truetype/cmu/,UprightFont=cmunrm.ttf,BoldFont=cmunbx.ttf,ItalicFont=cmunti.ttf,BoldItalicFont=cmunbi.ttf]{cmunrm.ttf}\setmonofont[Path=/usr/share/fonts/truetype/cmu/,UprightFont=cmuntt.ttf,BoldFont=cmuntb.ttf,ItalicFont=cmunit.ttf,BoldItalicFont=cmuntx.ttf]{cmunrm.ttf} by default regardless of your TeX distribution configuration. If you have TeX Live configured to use A4 paper, it will be the default only for plainTeX and classes not specifying the paper dimension.\end{TemplateInfo}

\begin{TemplateInfo}{\danger}{Warning}The a4paper option with the article document class by itself has no effect. It will only affect the page size
in connection with some appropriate package, like the \LaTeXTT{geometry} package or the \LaTeXTT{hyperref} package.\end{TemplateInfo}\subsection{More size options with {\itshape \setmainfont[Path=/usr/share/fonts/truetype/cmu/,UprightFont=cmunrm.ttf,BoldFont=cmunbx.ttf,ItalicFont=cmunti.ttf,BoldItalicFont=cmunbi.ttf]{cmunti.ttf}\setmonofont[Path=/usr/share/fonts/truetype/cmu/,UprightFont=cmuntt.ttf,BoldFont=cmuntb.ttf,ItalicFont=cmunit.ttf,BoldItalicFont=cmuntx.ttf]{cmunti.ttf}\itshape geometry}{$\text{ }$}\setmainfont[Path=/usr/share/fonts/truetype/cmu/,UprightFont=cmunrm.ttf,BoldFont=cmunbx.ttf,ItalicFont=cmunti.ttf,BoldItalicFont=cmunbi.ttf]{cmunrm.ttf}\setmonofont[Path=/usr/share/fonts/truetype/cmu/,UprightFont=cmuntt.ttf,BoldFont=cmuntb.ttf,ItalicFont=cmunit.ttf,BoldItalicFont=cmuntx.ttf]{cmunrm.ttf}}
\label{307}

One of the most versatile packages for page layout is the \LaTeXTT{geometry} package. The immediate advantage of this package is that it lets you customize the page size even with classes that do not support the options. For instance, to set the page size, add the following to your preamble:

\begin{Shaded}
\begin{Highlighting}[]

\NormalTok{\textbackslash{}usepackage[a4paper]\{geometry\}}
\end{Highlighting}
\end{Shaded}


The \LaTeXTT{geometry} package has many pre-{}defined page sizes, like \LaTeXTT{a4paper}, built in.  Others include:
\begin{myitemize}
\item{}  \LaTeXTT{a0paper}, \LaTeXTT{a1paper}, ..., \LaTeXTT{a6paper},
\item{}  \LaTeXTT{b0paper}, \LaTeXTT{b1paper}, ..., \LaTeXTT{b6paper},
\item{}  \LaTeXTT{letterpaper},
\item{}  \LaTeXTT{legalpaper},
\item{}  \LaTeXTT{executivepaper}.
\end{myitemize}


To explicitly change the paper dimensions using the \LaTeXTT{geometry} package, the \LaTeXTT{paperwidth} and \LaTeXTT{paperheight} options can be used. For example:

\begin{Shaded}
\begin{Highlighting}[]

\NormalTok{\textbackslash{}usepackage[paperwidth=5.5in, paperheight=8.5in]\{geometry\}}
\end{Highlighting}
\end{Shaded}


\subsection{Page size issues}
\label{308}

If you intend to get a PDF in the end, there are basically three ways:
\begin{myitemize}
\item{}  TeX → PDF
\end{myitemize}
\\

\TemplateSpaceIndent{$\text{ }${}pdflatex$\text{ }${}myfile$\text{ }${}$\text{ }${}$\text{ }${}$\text{ }${}$\text{ }${}$\text{ }${}$\text{ }${}$\text{ }${}$\text{ }${}$\text{ }${}$\text{ }${}$\text{ }${}$\text{ }${}$\text{ }${}$\text{ }${}\#$\text{ }${}TeX$\text{ }${}→$\text{ }${}PDF}

\begin{myitemize}
\item{}  TeX → DVI → PDF
\end{myitemize}
\\

\TemplateSpaceIndent{$\text{ }${}latex$\text{ }${}myfile$\text{ }${}$\text{ }${}$\text{ }${}$\text{ }${}$\text{ }${}$\text{ }${}$\text{ }${}$\text{ }${}$\text{ }${}$\text{ }${}$\text{ }${}$\text{ }${}$\text{ }${}$\text{ }${}$\text{ }${}$\text{ }${}$\text{ }${}$\text{ }${}\#$\text{ }${}TeX$\text{ }${}→$\text{ }${}DVI$\text{ }$\newline{}
$\text{ }${}dvipdf$\text{ }${}myfile$\text{ }${}$\text{ }${}$\text{ }${}$\text{ }${}$\text{ }${}$\text{ }${}$\text{ }${}$\text{ }${}$\text{ }${}$\text{ }${}$\text{ }${}$\text{ }${}$\text{ }${}$\text{ }${}$\text{ }${}$\text{ }${}$\text{ }${}\#$\text{ }${}DVI$\text{ }${}→$\text{ }${}PDF}

\begin{myitemize}
\item{}  TeX → DVI → PS → PDF
\end{myitemize}
\\

\TemplateSpaceIndent{$\text{ }${}latex$\text{ }${}myfile$\text{ }${}$\text{ }${}$\text{ }${}$\text{ }${}$\text{ }${}$\text{ }${}$\text{ }${}$\text{ }${}$\text{ }${}$\text{ }${}$\text{ }${}$\text{ }${}$\text{ }${}$\text{ }${}$\text{ }${}$\text{ }${}$\text{ }${}$\text{ }${}\#$\text{ }${}TeX$\text{ }${}→$\text{ }${}DVI$\text{ }$\newline{}
$\text{ }${}dvips$\text{ }${}myfile$\text{ }${}-{}o$\text{ }${}myfile.ps$\text{ }${}$\text{ }${}$\text{ }${}$\text{ }${}$\text{ }${}\#$\text{ }${}DVI$\text{ }${}→$\text{ }${}PS$\text{ }$\newline{}
$\text{ }${}ps2pdf$\text{ }${}myfile.ps$\text{ }${}myfile.pdf$\text{ }${}$\text{ }${}$\text{ }${}\#$\text{ }${}PS$\text{ }${}$\text{ }${}→$\text{ }${}PDF}


Sadly the PDF output page size may not be completely respectful of your settings. Some of these tools do not have the same interpretation of the DVI, PS and PDF specifications, and you may end up with a PDF which has not exactly the right size. Thankfully there is a solution to that: the \LaTeXTT{\textbackslash{}special} command lets the user pass PostScript or PDF parameters, which can be used here to set the page size appropriately.

\begin{myitemize}
\item{}  For {\ttfamily \setmainfont[Path=/usr/share/fonts/truetype/cmu/,UprightFont=cmunrm.ttf,BoldFont=cmunbx.ttf,ItalicFont=cmunti.ttf,BoldItalicFont=cmunbi.ttf]{cmuntt.ttf}\setmonofont[Path=/usr/share/fonts/truetype/cmu/,UprightFont=cmuntt.ttf,BoldFont=cmuntb.ttf,ItalicFont=cmunit.ttf,BoldItalicFont=cmuntx.ttf]{cmuntt.ttf}\ttfamily pdflatex}{$\text{ }$}\setmainfont[Path=/usr/share/fonts/truetype/cmu/,UprightFont=cmunrm.ttf,BoldFont=cmunbx.ttf,ItalicFont=cmunti.ttf,BoldItalicFont=cmunbi.ttf]{cmunrm.ttf}\setmonofont[Path=/usr/share/fonts/truetype/cmu/,UprightFont=cmuntt.ttf,BoldFont=cmuntb.ttf,ItalicFont=cmunit.ttf,BoldItalicFont=cmuntx.ttf]{cmunrm.ttf} to work fine, using the package \LaTeXTT{geometry} usually works.
\item{}  For the DVI and PS ways, the safest way to always get the right paper size in the end is to add
\end{myitemize}

\begin{Shaded}
\begin{Highlighting}[]

\NormalTok{\textbackslash{}documentclass[...,a4paper,...]\{...\}}
\NormalTok{\textbackslash{}special\{papersize=210mm,297mm\}}
\end{Highlighting}
\end{Shaded}

to the tex file, and to append the appropriate parameters to the processors used during output generation:\\

\TemplateSpaceIndent{$\text{ }${}dvips$\text{ }${}-{}t$\text{ }${}a4$\text{ }${}...$\text{ }$\newline{}
$\text{ }${}ps2pdf$\text{ }${}-{}sPAPERSIZE=a4$\text{ }${}...$\text{ }${}\#$\text{ }${}On$\text{ }${}Windows:$\text{ }${}ps2pdf$\text{ }${}-{}sPAPERSIZE\#a4$\text{ }${}...$\text{ }${}\myfootnote{\myfnhref{http://ghostscript.com/doc/current/Use.htm\#MS_Windows}{How to use Ghostscript}}}


If you want US Letter instead, replace \LaTeXTT{210mm,297mm} by \LaTeXTT{8.5in,11in} and \LaTeXTT{a4paper} by \LaTeXTT{letter}. Also replace {\ttfamily \setmainfont[Path=/usr/share/fonts/truetype/cmu/,UprightFont=cmunrm.ttf,BoldFont=cmunbx.ttf,ItalicFont=cmunti.ttf,BoldItalicFont=cmunbi.ttf]{cmuntt.ttf}\setmonofont[Path=/usr/share/fonts/truetype/cmu/,UprightFont=cmuntt.ttf,BoldFont=cmuntb.ttf,ItalicFont=cmunit.ttf,BoldItalicFont=cmuntx.ttf]{cmuntt.ttf}\ttfamily a4}{$\text{ }$}\setmainfont[Path=/usr/share/fonts/truetype/cmu/,UprightFont=cmunrm.ttf,BoldFont=cmunbx.ttf,ItalicFont=cmunti.ttf,BoldItalicFont=cmunbi.ttf]{cmunrm.ttf}\setmonofont[Path=/usr/share/fonts/truetype/cmu/,UprightFont=cmuntt.ttf,BoldFont=cmuntb.ttf,ItalicFont=cmunit.ttf,BoldItalicFont=cmuntx.ttf]{cmunrm.ttf} by {\ttfamily \setmainfont[Path=/usr/share/fonts/truetype/cmu/,UprightFont=cmunrm.ttf,BoldFont=cmunbx.ttf,ItalicFont=cmunti.ttf,BoldItalicFont=cmunbi.ttf]{cmuntt.ttf}\setmonofont[Path=/usr/share/fonts/truetype/cmu/,UprightFont=cmuntt.ttf,BoldFont=cmuntb.ttf,ItalicFont=cmunit.ttf,BoldItalicFont=cmuntx.ttf]{cmuntt.ttf}\ttfamily letter}{$\text{ }$}\setmainfont[Path=/usr/share/fonts/truetype/cmu/,UprightFont=cmunrm.ttf,BoldFont=cmunbx.ttf,ItalicFont=cmunti.ttf,BoldItalicFont=cmunbi.ttf]{cmunrm.ttf}\setmonofont[Path=/usr/share/fonts/truetype/cmu/,UprightFont=cmuntt.ttf,BoldFont=cmuntb.ttf,ItalicFont=cmunit.ttf,BoldItalicFont=cmuntx.ttf]{cmunrm.ttf} in command-{}line parameters.
\subsection{Page size for tablets}
\label{309}

Those who want to read on tablets or other handheld digital devices need to create documents without the extra whitespace. In order to create PDF documents with optimal handheld viewing, not only must the text field and margins be adjusted, so must the page size. If you are looking for a sensible dimension, consider following the paper size used by the Supreme Court of the United States, 441pt by 666pt (or 6.125 inches by 9.25 inches), which looks great on tablets. You could also use the Supreme Court\textquotesingle{}s text field size of 297 pt by 513 pt, but this is too wide for fonts other than Century Schoolbook, the font required by the Supreme Court.
\section{Margins}
\label{310}

Readers used to perusing typical physical literature are probably wondering why there is so much white space surrounding the text. For example, on A4 paper a document will typically have 44 mm margin widths on the left and right of the page, leaving about 60\% of the page width for text. The reason is improved readability. Studies have shown\myfootnote{\myplainurl{http://webtypography.net/2.1.2}}\myfootnote{\myplainurl{http://baymard.com/blog/line-length-readability}} that it\textquotesingle{}s easier to read text when there are 60{\mbox{$-$}}70 characters per line{\mbox{$\text{---}$}}and it would seem that 66 is the optimal number. Therefore, the page margins are set to ensure optimal readability, and excessive margin white space is tolerated as a consequence. Sometimes, this white space is left in the inner margin with the assumption that the document will be bound.

If you wish to avoid excessive white space, rather than changing the margins, consider instead using a two-{}column (or more) layout. This approach is the one usually taken by print magazines because it provides both readable line lengths and good use of the page. Another option for reducing the amount of whitespace on the page without changing the margins is to increase the font size using the \LaTeXTT{12pt} option to the document class.

If you wish to change the margins of your document, there are many ways to do so:

\begin{myitemize}
\item{}  One older approach is to use the \LaTeXTT{fullpage} package for somewhat standardized smaller margins (around an inch), but it creates lines of more than 100 characters per line at with the 10pt default font size (and about 90 if the \LaTeXTT{12pt} documentclass option is used):
\end{myitemize}


\begin{Shaded}
\begin{Highlighting}[]

\NormalTok{\textbackslash{}usepackage\{fullpage\}}
\end{Highlighting}
\end{Shaded}


For even narrower margins, the \LaTeXTT{fullpage} package has a \LaTeXTT{cm} option (around 1.5cm), which results in about 120 characters per line at the 10pt default font size, about double what is considered readable:

\begin{Shaded}
\begin{Highlighting}[]

\NormalTok{\textbackslash{}usepackage[cm]\{fullpage\}}
\end{Highlighting}
\end{Shaded}


\begin{myitemize}
\item{}  A more modern and flexible approach is to use the \LaTeXTT{geometry} package.  This package allows you to specify the 4 margins without needing to remember the particular page dimensions commands.  You can enter the measures in centimeters and inches as well. Use \LaTeXTT{cm} for centimeters and \LaTeXTT{in} for inches after each value ({\itshape \setmainfont[Path=/usr/share/fonts/truetype/cmu/,UprightFont=cmunrm.ttf,BoldFont=cmunbx.ttf,ItalicFont=cmunti.ttf,BoldItalicFont=cmunbi.ttf]{cmunti.ttf}\setmonofont[Path=/usr/share/fonts/truetype/cmu/,UprightFont=cmuntt.ttf,BoldFont=cmuntb.ttf,ItalicFont=cmunit.ttf,BoldItalicFont=cmuntx.ttf]{cmunti.ttf}\itshape e.g.}{$\text{ }$}\setmainfont[Path=/usr/share/fonts/truetype/cmu/,UprightFont=cmunrm.ttf,BoldFont=cmunbx.ttf,ItalicFont=cmunti.ttf,BoldItalicFont=cmunbi.ttf]{cmunrm.ttf}\setmonofont[Path=/usr/share/fonts/truetype/cmu/,UprightFont=cmuntt.ttf,BoldFont=cmuntb.ttf,ItalicFont=cmunit.ttf,BoldItalicFont=cmuntx.ttf]{cmunrm.ttf} 1.0in or 2.54cm). Note that by default ({\itshape \setmainfont[Path=/usr/share/fonts/truetype/cmu/,UprightFont=cmunrm.ttf,BoldFont=cmunbx.ttf,ItalicFont=cmunti.ttf,BoldItalicFont=cmunbi.ttf]{cmunti.ttf}\setmonofont[Path=/usr/share/fonts/truetype/cmu/,UprightFont=cmuntt.ttf,BoldFont=cmuntb.ttf,ItalicFont=cmunit.ttf,BoldItalicFont=cmuntx.ttf]{cmunti.ttf}\itshape i.e.}{$\text{ }$}\setmainfont[Path=/usr/share/fonts/truetype/cmu/,UprightFont=cmunrm.ttf,BoldFont=cmunbx.ttf,ItalicFont=cmunti.ttf,BoldItalicFont=cmunbi.ttf]{cmunrm.ttf}\setmonofont[Path=/usr/share/fonts/truetype/cmu/,UprightFont=cmuntt.ttf,BoldFont=cmuntb.ttf,ItalicFont=cmunit.ttf,BoldItalicFont=cmuntx.ttf]{cmunrm.ttf} without any options) this package already reduces the margins, so for a \textquotesingle{}standard layout\textquotesingle{} you may not need to specify anything. These values are relative to the edge of paper (0in) and go inward. For example, this command provides more conventional margins, better using the vertical space of the page, without creating the dramatically long lines of the \LaTeXTT{fullpage} package (if the \LaTeXTT{11pt} documentclass option is used, the line lengths are about 88 characters for letter-{}sized paper and slightly less when using \LaTeXTT{a4paper}). 
\end{myitemize}


\begin{Shaded}
\begin{Highlighting}[]

\NormalTok{\textbackslash{}usepackage[top=1in, bottom=1.25in, left=1.25in, right=1.25in]\{geometry\}}
\end{Highlighting}
\end{Shaded}


It can also recreate the behavior of the \LaTeXTT{fullpage} package using
\begin{Shaded}
\begin{Highlighting}[]

\NormalTok{\textbackslash{}usepackage[margin=1in]\{geometry\}}
\end{Highlighting}
\end{Shaded}


You can combine the margin options with the page size options seen in \mylref{332}{this paragraph}.

\begin{myitemize}
\item{}  You should not use the \LaTeXTT{a4wide} package for a page with A4 document size with smaller margins. It is obsolete and buggy. Use geometry package instead like this:
\end{myitemize}


\begin{Shaded}
\begin{Highlighting}[]

\NormalTok{\textbackslash{}usepackage[a4paper,includeheadfoot,margin=2.54cm]\{geometry\}}
\end{Highlighting}
\end{Shaded}


\begin{myitemize}
\item{}  Edit individual page dimension variables described above, using the \LaTeXTT{\textbackslash{}addtolength} and \LaTeXTT{\textbackslash{}setlength} commands. See the \mylref{456}{Lengths} chapter. For instance, 
\end{myitemize}


\begin{Shaded}
\begin{Highlighting}[]

\NormalTok{\textbackslash{}setlength\{\textbackslash{}textwidth\}\{6.5in\}}
\NormalTok{\textbackslash{}addtolength\{\textbackslash{}voffset\}\{-5pt\}}
\end{Highlighting}
\end{Shaded}

\subsection{Odd and even margins}
\label{311}

Using the geometry package, the options left and right are used for the inside and outside margins respectively. They also have aliases inner and outer. Thus, the easiest way to handle different margins for odd and even pages is to give the twoside option in the document class command and specify the margins as usually.

\begin{Shaded}
\begin{Highlighting}[]

\NormalTok{\textbackslash{}documentclass[twoside]\{report\}}
\NormalTok{\textbackslash{}usepackage[inner=4cm,outer=2cm]\{geometry\} }\CommentTok{%left=4cm,right=2cm would be}
 \NormalTok{equivalent}
\end{Highlighting}
\end{Shaded}


This will result in a value of 4cm on all inner margins (left margin for odd number pages and right margin for even pages) and 2cm margin on outer margins. 

Setting the same value for the inner and outer for geometry will remove the difference between the margins. Another quick way to eliminate the difference in position between even and odd numbered pages would be setting the values to {\bfseries \setmainfont[Path=/usr/share/fonts/truetype/cmu/,UprightFont=cmunrm.ttf,BoldFont=cmunbx.ttf,ItalicFont=cmunti.ttf,BoldItalicFont=cmunbi.ttf]{cmunbx.ttf}\setmonofont[Path=/usr/share/fonts/truetype/cmu/,UprightFont=cmuntt.ttf,BoldFont=cmuntb.ttf,ItalicFont=cmunit.ttf,BoldItalicFont=cmuntx.ttf]{cmunbx.ttf}\bfseries evensidemargin}{$\text{ }$}\setmainfont[Path=/usr/share/fonts/truetype/cmu/,UprightFont=cmunrm.ttf,BoldFont=cmunbx.ttf,ItalicFont=cmunti.ttf,BoldItalicFont=cmunbi.ttf]{cmunrm.ttf}\setmonofont[Path=/usr/share/fonts/truetype/cmu/,UprightFont=cmuntt.ttf,BoldFont=cmuntb.ttf,ItalicFont=cmunit.ttf,BoldItalicFont=cmuntx.ttf]{cmunrm.ttf} and {\bfseries \setmainfont[Path=/usr/share/fonts/truetype/cmu/,UprightFont=cmunrm.ttf,BoldFont=cmunbx.ttf,ItalicFont=cmunti.ttf,BoldItalicFont=cmunbi.ttf]{cmunbx.ttf}\setmonofont[Path=/usr/share/fonts/truetype/cmu/,UprightFont=cmuntt.ttf,BoldFont=cmuntb.ttf,ItalicFont=cmunit.ttf,BoldItalicFont=cmuntx.ttf]{cmunbx.ttf}\bfseries oddsidemargin}{$\text{ }$}\setmainfont[Path=/usr/share/fonts/truetype/cmu/,UprightFont=cmunrm.ttf,BoldFont=cmunbx.ttf,ItalicFont=cmunti.ttf,BoldItalicFont=cmunbi.ttf]{cmunrm.ttf}\setmonofont[Path=/usr/share/fonts/truetype/cmu/,UprightFont=cmuntt.ttf,BoldFont=cmuntb.ttf,ItalicFont=cmunit.ttf,BoldItalicFont=cmuntx.ttf]{cmunrm.ttf} to the half of odd\textquotesingle{}s default:
\begin{Shaded}
\begin{Highlighting}[]

\NormalTok{\textbackslash{}setlength\{\textbackslash{}oddsidemargin\}\{15.5pt\} }
\NormalTok{\textbackslash{}setlength\{\textbackslash{}evensidemargin\}\{15.5pt\}}
\end{Highlighting}
\end{Shaded}


By default, the value of {\bfseries \setmainfont[Path=/usr/share/fonts/truetype/cmu/,UprightFont=cmunrm.ttf,BoldFont=cmunbx.ttf,ItalicFont=cmunti.ttf,BoldItalicFont=cmunbi.ttf]{cmunbx.ttf}\setmonofont[Path=/usr/share/fonts/truetype/cmu/,UprightFont=cmuntt.ttf,BoldFont=cmuntb.ttf,ItalicFont=cmunit.ttf,BoldItalicFont=cmuntx.ttf]{cmunbx.ttf}\bfseries evensidemargin}{$\text{ }$}\setmainfont[Path=/usr/share/fonts/truetype/cmu/,UprightFont=cmunrm.ttf,BoldFont=cmunbx.ttf,ItalicFont=cmunti.ttf,BoldItalicFont=cmunbi.ttf]{cmunrm.ttf}\setmonofont[Path=/usr/share/fonts/truetype/cmu/,UprightFont=cmuntt.ttf,BoldFont=cmuntb.ttf,ItalicFont=cmunit.ttf,BoldItalicFont=cmuntx.ttf]{cmunrm.ttf} is larger than {\bfseries \setmainfont[Path=/usr/share/fonts/truetype/cmu/,UprightFont=cmunrm.ttf,BoldFont=cmunbx.ttf,ItalicFont=cmunti.ttf,BoldItalicFont=cmunbi.ttf]{cmunbx.ttf}\setmonofont[Path=/usr/share/fonts/truetype/cmu/,UprightFont=cmuntt.ttf,BoldFont=cmuntb.ttf,ItalicFont=cmunit.ttf,BoldItalicFont=cmuntx.ttf]{cmunbx.ttf}\bfseries oddsidemargin}{$\text{ }$}\setmainfont[Path=/usr/share/fonts/truetype/cmu/,UprightFont=cmunrm.ttf,BoldFont=cmunbx.ttf,ItalicFont=cmunti.ttf,BoldItalicFont=cmunbi.ttf]{cmunrm.ttf}\setmonofont[Path=/usr/share/fonts/truetype/cmu/,UprightFont=cmuntt.ttf,BoldFont=cmuntb.ttf,ItalicFont=cmunit.ttf,BoldItalicFont=cmuntx.ttf]{cmunrm.ttf} in the two-{}sided layout, as one could wish to write notes on the side of the page. The side for the large margin is chosen opposite to the side where pages are joined together.

See the \mylref{456}{Lengths}.
\subsection{Top margin above Chapter}
\label{312}
The top margin above a chapter can be changed using the \LaTeXTT{titlesec} package. Example: \myplainurl{http://www.ctex.org/documents/packages/layout/titlesec.pdf}

\begin{Shaded}
\begin{Highlighting}[]

\NormalTok{\textbackslash{}usepackage\{titlesec\}}
\NormalTok{\textbackslash{}titlespacing*\{\textbackslash{}chapter\}\{0pt\}\{-50pt\}\{20pt\}}
\NormalTok{\textbackslash{}titleformat\{\textbackslash{}chapter\}[display]\{\textbackslash{}normalfont\textbackslash{}huge\textbackslash{}bfseries\}\{\textbackslash{}chaptertitlename\textbackslash{}}
 \NormalTok{\textbackslash{}thechapter\}\{20pt\}\{\textbackslash{}Huge\}}
\end{Highlighting}
\end{Shaded}


The command \LaTeXTT{\textbackslash{}titleformat} must be used when the spacing of a chapter is changed. In case of a section this command can be omitted.
\section{Page orientation}
\label{313}

When you talk about changing page orientation, it usually means changing to landscape mode, since portrait is the default. We shall introduce two slightly different styles of changing orientation.
\subsection{Change orientation of the whole document}
\label{314}

The first is for when you want all of your document to be in landscape from the very beginning. There are various packages available to achieve this, but the one we prefer is the \LaTeXTT{geometry} package. All you need to do is call the package, with {\itshape \setmainfont[Path=/usr/share/fonts/truetype/cmu/,UprightFont=cmunrm.ttf,BoldFont=cmunbx.ttf,ItalicFont=cmunti.ttf,BoldItalicFont=cmunbi.ttf]{cmunti.ttf}\setmonofont[Path=/usr/share/fonts/truetype/cmu/,UprightFont=cmuntt.ttf,BoldFont=cmuntb.ttf,ItalicFont=cmunit.ttf,BoldItalicFont=cmuntx.ttf]{cmunti.ttf}\itshape landscape}{$\text{ }$}\setmainfont[Path=/usr/share/fonts/truetype/cmu/,UprightFont=cmunrm.ttf,BoldFont=cmunbx.ttf,ItalicFont=cmunti.ttf,BoldItalicFont=cmunbi.ttf]{cmunrm.ttf}\setmonofont[Path=/usr/share/fonts/truetype/cmu/,UprightFont=cmuntt.ttf,BoldFont=cmuntb.ttf,ItalicFont=cmunit.ttf,BoldItalicFont=cmuntx.ttf]{cmunrm.ttf} as an option:

\begin{Shaded}
\begin{Highlighting}[]

\NormalTok{\textbackslash{}usepackage[landscape]\{geometry\}}
\end{Highlighting}
\end{Shaded}


Although, if you intend to use \LaTeXTT{geometry} to set your paper size, don\textquotesingle{}t add the \LaTeXTT{\textbackslash{}usepackage} commands twice, simply string all the options together, separating with a comma:

\begin{Shaded}
\begin{Highlighting}[]

\NormalTok{\textbackslash{}usepackage[a4paper,landscape]\{geometry\}}
\end{Highlighting}
\end{Shaded}


Using standard LaTeX classes, you can use the same class options:

\begin{Shaded}
\begin{Highlighting}[]

\NormalTok{\textbackslash{}documentclass[a4paper,landscape]\{article\}}
\end{Highlighting}
\end{Shaded}

\subsection{Change orientation of specific part}
\label{315}

The second method is for when you are writing a document in portrait, but you have some contents, like a large diagram or table that would be displayed better on a landscape page. However, you still want the consistency of your headers and footers appearing the same place as the other pages.

The \LaTeXTT{lscape} package is for this very purpose. It supplies a \LaTeXTT{landscape} environment, and anything inside is basically rotated. No actual page dimensions are changed. This approach is more applicable to books or reports than to typical academic publications.  Using \LaTeXTT{pdflscape} instead of \LaTeXTT{lscape} when generating a PDF document will make the page appear right side up when viewed: the single page that is in landscape format will be rotated, while the rest will be left in portrait orientation.

Also, to get a table to appear correctly centered on a landscaped page, one must place the \LaTeXTT{tabular} environment inside a \LaTeXTT{table} environment, which is itself inside the \LaTeXTT{landscape} environment. For instance it should look like this:

\begin{Shaded}
\begin{Highlighting}[]

\NormalTok{\textbackslash{}usepackage\{pdflscape\}}
\CommentTok{% ...}
 
\NormalTok{\textbackslash{}begin\{landscape\}}
\NormalTok{\textbackslash{}begin\{table\}}
\NormalTok{\textbackslash{}centering     }\CommentTok{% optional, probably makes it look better to have it centered on}
 \NormalTok{the page}
\NormalTok{\textbackslash{}begin\{tabular\}\{....\}}
\CommentTok{% ...}
\NormalTok{\textbackslash{}end\{tabular\}}
\NormalTok{\textbackslash{}end\{table\}}
\NormalTok{\textbackslash{}end\{landscape\}}
\end{Highlighting}
\end{Shaded}


For books (and in general documents using the {\ttfamily \setmainfont[Path=/usr/share/fonts/truetype/cmu/,UprightFont=cmunrm.ttf,BoldFont=cmunbx.ttf,ItalicFont=cmunti.ttf,BoldItalicFont=cmunbi.ttf]{cmuntt.ttf}\setmonofont[Path=/usr/share/fonts/truetype/cmu/,UprightFont=cmuntt.ttf,BoldFont=cmuntb.ttf,ItalicFont=cmunit.ttf,BoldItalicFont=cmuntx.ttf]{cmuntt.ttf}\ttfamily twoside}{$\text{ }$}\setmainfont[Path=/usr/share/fonts/truetype/cmu/,UprightFont=cmunrm.ttf,BoldFont=cmunbx.ttf,ItalicFont=cmunti.ttf,BoldItalicFont=cmunbi.ttf]{cmunrm.ttf}\setmonofont[Path=/usr/share/fonts/truetype/cmu/,UprightFont=cmuntt.ttf,BoldFont=cmuntb.ttf,ItalicFont=cmunit.ttf,BoldItalicFont=cmuntx.ttf]{cmunrm.ttf} option), the {\ttfamily \setmainfont[Path=/usr/share/fonts/truetype/cmu/,UprightFont=cmunrm.ttf,BoldFont=cmunbx.ttf,ItalicFont=cmunti.ttf,BoldItalicFont=cmunbi.ttf]{cmuntt.ttf}\setmonofont[Path=/usr/share/fonts/truetype/cmu/,UprightFont=cmuntt.ttf,BoldFont=cmuntb.ttf,ItalicFont=cmunit.ttf,BoldItalicFont=cmuntx.ttf]{cmuntt.ttf}\ttfamily landscape}\setmainfont[Path=/usr/share/fonts/truetype/cmu/,UprightFont=cmunrm.ttf,BoldFont=cmunbx.ttf,ItalicFont=cmunti.ttf,BoldItalicFont=cmunbi.ttf]{cmunrm.ttf}\setmonofont[Path=/usr/share/fonts/truetype/cmu/,UprightFont=cmuntt.ttf,BoldFont=cmuntb.ttf,ItalicFont=cmunit.ttf,BoldItalicFont=cmuntx.ttf]{cmunrm.ttf}-{}environment unfortunately does not pay attention to the different layout of even and odd pages. The macro can be fixed using a few lines of extra code in the preamble\myfootnote{\myplainurl{https://stackoverflow.com/questions/4982219/how-to-make-landscape-mode-rotate-properly-in-a-twoside-book/5320962\#5320962}}.
\subsection{Change orientation of floating environment}
\label{316}

If you use the above code, you will see that the table is inserted where it is in the code. It will not be floated! To fix this you need the package \LaTeXTT{rotating}. See the \mylref{243}{Rotations} chapter.
\section{Margins, page size and rotation of a specific page}
\label{317}

If you need to rotate the page so that the figure fits, the chances are good that you need to scale the margins and the font size too. Again, the \LaTeXTT{geometry} package comes in handy for specifying new margins for a single page only.

\begin{Shaded}
\begin{Highlighting}[]

\NormalTok{\textbackslash{}usepackage\{geometry\}}
\NormalTok{\textbackslash{}usepackage\{pdflscape\}}
\CommentTok{% ...}
 
\NormalTok{\textbackslash{}newgeometry\{margin=1cm\}}
\NormalTok{\textbackslash{}begin\{landscape\}}
\NormalTok{\textbackslash{}thispagestyle\{empty\} }\CommentTok{%% Remove header and footer.}
 
\NormalTok{\textbackslash{}begin\{table\}}
\NormalTok{\textbackslash{}begin\{center\}}
\NormalTok{\textbackslash{}footnotesize }\CommentTok{%% Smaller font size.}
 
\NormalTok{\textbackslash{}begin\{tabular\}\{....\}}
\CommentTok{% ...}
\NormalTok{\textbackslash{}end\{tabular\}}
 
\NormalTok{\textbackslash{}end\{center\}}
\NormalTok{\textbackslash{}end\{table\}}
 
\NormalTok{\textbackslash{}end\{landscape\}}
\NormalTok{\textbackslash{}restoregeometry}
\end{Highlighting}
\end{Shaded}


Note that order matters!
\section{Page styles}
\label{318}

Page styles in Latex terms refers not to page dimensions, but to the running headers and footers of a document. These headers typically contain document titles, chapter or section numbers/names, and page numbers.
\subsection{Standard page styles}
\label{319}

The possibilities of changing the headers in plain Latex are actually quite limited. There are two commands available: \LaTeXTT{\textbackslash{}pagestyle\{\textquotesingle{}\textquotesingle{}style\textquotesingle{}\textquotesingle{}\}} will apply the specified style to the current and all subsequent pages, and \LaTeXTT{\textbackslash{}thispagestyle\{\textquotesingle{}\textquotesingle{}style\textquotesingle{}\textquotesingle{}\}} will only affect the current page. The possible styles are:

\begin{longtable}{>{\RaggedRight}p{0.17133\linewidth}>{\RaggedRight}p{0.77153\linewidth}} 
\hspace*{0pt}\ignorespaces{}\hspace*{0pt} \LaTeXTT{empty}&\hspace*{0pt}\ignorespaces{}\hspace*{0pt} Both header and footer are cleared\\ \hspace*{0pt}\ignorespaces{}\hspace*{0pt} \LaTeXTT{plain}&\hspace*{0pt}\ignorespaces{}\hspace*{0pt} Header is clear, but the footer contains the page number in the center.\\ \hspace*{0pt}\ignorespaces{}\hspace*{0pt} \LaTeXTT{headings}&\hspace*{0pt}\ignorespaces{}\hspace*{0pt} Footer is blank, header displays information according to document class (e.g., section name) and page number top right.\\ \hspace*{0pt}\ignorespaces{}\hspace*{0pt} \LaTeXTT{myheadings}&\hspace*{0pt}\ignorespaces{}\hspace*{0pt} Page number is top right, and it is possible to control the rest of the header. 
\end{longtable}


The commands \LaTeXTT{\textbackslash{}markright}  and \LaTeXTT{\textbackslash{}markboth} can be used to set the content of the headings by hand. The following commands placed at the beginning of an article document will set the header of all pages (one-{}sided)  to contain \symbol{34}John Smith\symbol{34} top left, \symbol{34}On page styles\symbol{34} centered and the page number top right:

\begin{Shaded}
\begin{Highlighting}[]

\NormalTok{\textbackslash{}pagestyle\{headings\}}
\NormalTok{\textbackslash{}markright\{John Smith\textbackslash{}hfill On page styles\textbackslash{}hfill\}}
\end{Highlighting}
\end{Shaded}


There are special commands containing details on the running page of the document.

\begin{longtable}{>{\RaggedRight}p{0.20018\linewidth}>{\RaggedRight}p{0.74268\linewidth}} 
\hspace*{0pt}\ignorespaces{}\hspace*{0pt} \LaTeXTT{\textbackslash{}thepage}     &\hspace*{0pt}\ignorespaces{}\hspace*{0pt} number of the current page\\ \hspace*{0pt}\ignorespaces{}\hspace*{0pt} \LaTeXTT{\textbackslash{}leftmark}    &\hspace*{0pt}\ignorespaces{}\hspace*{0pt} current chapter name printed like \symbol{34}CHAPTER 3. THIS IS THE CHAPTER TITLE\symbol{34}\\ \hspace*{0pt}\ignorespaces{}\hspace*{0pt} \LaTeXTT{\textbackslash{}rightmark}   &\hspace*{0pt}\ignorespaces{}\hspace*{0pt} current section name printed like \symbol{34}1.6. THIS IS THE SECTION TITLE\symbol{34}\\ \hspace*{0pt}\ignorespaces{}\hspace*{0pt} \LaTeXTT{\textbackslash{}chaptername} &\hspace*{0pt}\ignorespaces{}\hspace*{0pt} the name {\itshape \setmainfont[Path=/usr/share/fonts/truetype/cmu/,UprightFont=cmunrm.ttf,BoldFont=cmunbx.ttf,ItalicFont=cmunti.ttf,BoldItalicFont=cmunbi.ttf]{cmunti.ttf}\setmonofont[Path=/usr/share/fonts/truetype/cmu/,UprightFont=cmuntt.ttf,BoldFont=cmuntb.ttf,ItalicFont=cmunit.ttf,BoldItalicFont=cmuntx.ttf]{cmunti.ttf}\itshape chapter}{$\text{ }$}\setmainfont[Path=/usr/share/fonts/truetype/cmu/,UprightFont=cmunrm.ttf,BoldFont=cmunbx.ttf,ItalicFont=cmunti.ttf,BoldItalicFont=cmunbi.ttf]{cmunrm.ttf}\setmonofont[Path=/usr/share/fonts/truetype/cmu/,UprightFont=cmuntt.ttf,BoldFont=cmuntb.ttf,ItalicFont=cmunit.ttf,BoldItalicFont=cmuntx.ttf]{cmunrm.ttf} in the current language. If this is English, it will display \symbol{34}Chapter\symbol{34}\\ \hspace*{0pt}\ignorespaces{}\hspace*{0pt} \LaTeXTT{\textbackslash{}thechapter}  &\hspace*{0pt}\ignorespaces{}\hspace*{0pt} current chapter number\\ \hspace*{0pt}\ignorespaces{}\hspace*{0pt} \LaTeXTT{\textbackslash{}thesection}  &\hspace*{0pt}\ignorespaces{}\hspace*{0pt} current section number 
\end{longtable}


Note that \LaTeXTT{\textbackslash{}leftmark} and \LaTeXTT{\textbackslash{}rightmark} convert the names to uppercase, whichever was the formatting of the text. If you want them to print the actual name of the chapter without converting it to uppercase use the following command:

\begin{Shaded}
\begin{Highlighting}[]

\NormalTok{\textbackslash{}renewcommand\{\textbackslash{}chaptermark\}[1]\{ \textbackslash{}markboth\{#1\}\{\} \}}
\NormalTok{\textbackslash{}renewcommand\{\textbackslash{}sectionmark\}[1]\{ \textbackslash{}markright\{#1\}\{\} \}}
\end{Highlighting}
\end{Shaded}


Now \LaTeXTT{\textbackslash{}leftmark} and \LaTeXTT{\textbackslash{}rightmark} will just print the name of the chapter and section, without number and without affecting the formatting. Note that these redefinitions must be inserted {\itshape \setmainfont[Path=/usr/share/fonts/truetype/cmu/,UprightFont=cmunrm.ttf,BoldFont=cmunbx.ttf,ItalicFont=cmunti.ttf,BoldItalicFont=cmunbi.ttf]{cmunti.ttf}\setmonofont[Path=/usr/share/fonts/truetype/cmu/,UprightFont=cmuntt.ttf,BoldFont=cmuntb.ttf,ItalicFont=cmunit.ttf,BoldItalicFont=cmuntx.ttf]{cmunti.ttf}\itshape after}{$\text{ }$}\setmainfont[Path=/usr/share/fonts/truetype/cmu/,UprightFont=cmunrm.ttf,BoldFont=cmunbx.ttf,ItalicFont=cmunti.ttf,BoldItalicFont=cmunbi.ttf]{cmunrm.ttf}\setmonofont[Path=/usr/share/fonts/truetype/cmu/,UprightFont=cmuntt.ttf,BoldFont=cmuntb.ttf,ItalicFont=cmunit.ttf,BoldItalicFont=cmuntx.ttf]{cmunrm.ttf} the first call of \LaTeXTT{\textbackslash{}pagestyle\{fancy\}}.  The standard book formatting of the \LaTeXTT{\textbackslash{}chaptermark} is:

\begin{Shaded}
\begin{Highlighting}[]

\NormalTok{\textbackslash{}renewcommand\{\textbackslash{}chaptermark\}[1]\{\textbackslash{}markboth\{\textbackslash{}MakeUppercase\{\textbackslash{}chaptername\textbackslash{}}
 \NormalTok{\textbackslash{}thechapter.\textbackslash{} #1\}\}\{\}\}}
\end{Highlighting}
\end{Shaded}


Watch out: if you provide long text in two different \symbol{34}parts\symbol{34} only in the footer or only in the header, you might see overlapping text.

Moreover, with the following commands you can define the thickness of the decorative lines on both the header and the footer:
\begin{Shaded}
\begin{Highlighting}[]

\NormalTok{\textbackslash{}renewcommand\{\textbackslash{}headrulewidth\}\{0.5pt\}}
\NormalTok{\textbackslash{}renewcommand\{\textbackslash{}footrulewidth\}\{0pt\}}
\end{Highlighting}
\end{Shaded}


The first line for the header, the second for the footer. Setting it to zero means that there will be no line.
\subsubsection{Plain pages issue}
\label{320}

An issue to look out for is that the major sectioning commands (\LaTeXTT{\textbackslash{}part}, \LaTeXTT{\textbackslash{}chapter} or \LaTeXTT{\textbackslash{}maketitle}) specify a \LaTeXTT{\textbackslash{}thispagestyle\{plain\}}. So, if you wish to suppress all styles by inserting a \LaTeXTT{\textbackslash{}pagestyle\{empty\}} at the beginning of your document, then the style command at each section will override your initial rule, for those pages only. To achieve the intended result one can follow the new section commands with \LaTeXTT{\textbackslash{}thispagestyle\{empty\}}. The \LaTeXTT{\textbackslash{}part} command, however, cannot be fixed this way, because it sets the page style, but also advances to the next page, so that \LaTeXTT{\textbackslash{}thispagestyle\{\}} cannot be applied to that page. Two solutions:
\begin{myitemize}
\item{}  simply write \LaTeXTT{\textbackslash{}usepackage\{nopageno\}} in the preamble. This package will make \LaTeXTT{\textbackslash{}pagestyle\{plain\}} have the same effect as \LaTeXTT{\textbackslash{}pagestyle\{empty\}}, effectively suppressing page numbering when it is used.
\item{}  Use \LaTeXTT{fancyhdr} as described below.
\end{myitemize}


The tricky problem when customizing headers and footers is to get things like running section and chapter names in there. Standard LaTeX accomplishes this with a two-{}stage approach. In the header and footer definition, you use the commands \LaTeXTT{\textbackslash{}rightmark} and \LaTeXTT{\textbackslash{}leftmark} to represent the current section and chapter heading, respectively. The values of these two commands are overwritten whenever a chapter or section command is processed. For ultimate flexibility, the \LaTeXTT{\textbackslash{}chapter} command and its friends do not redefine \LaTeXTT{\textbackslash{}rightmark} and \LaTeXTT{\textbackslash{}leftmark} themselves. They call yet another command (\LaTeXTT{\textbackslash{}chaptermark}, \LaTeXTT{\textbackslash{}sectionmark}, or \LaTeXTT{\textbackslash{}subsectionmark}) that is responsible for redefining \LaTeXTT{\textbackslash{}rightmark} and \LaTeXTT{\textbackslash{}leftmark}, except if they are starred -{}-{} in such a case, \LaTeXTT{\textbackslash{}markboth\{Chapter/Section name\}\{\}} must be used inside the sectioning command if header and footer lines are to be updated.

Again, several packages provide a solution:
\begin{myitemize}
\item{}  an alternative one-{}stage mechanism is provided by the package \LaTeXTT{titleps});
\item{}  \LaTeXTT{fancyhdr} will handle the process its own way.
\end{myitemize}

\subsection{Customizing with {\itshape \setmainfont[Path=/usr/share/fonts/truetype/cmu/,UprightFont=cmunrm.ttf,BoldFont=cmunbx.ttf,ItalicFont=cmunti.ttf,BoldItalicFont=cmunbi.ttf]{cmunti.ttf}\setmonofont[Path=/usr/share/fonts/truetype/cmu/,UprightFont=cmuntt.ttf,BoldFont=cmuntb.ttf,ItalicFont=cmunit.ttf,BoldItalicFont=cmuntx.ttf]{cmunti.ttf}\itshape fancyhdr}{$\text{ }$}\setmainfont[Path=/usr/share/fonts/truetype/cmu/,UprightFont=cmunrm.ttf,BoldFont=cmunbx.ttf,ItalicFont=cmunti.ttf,BoldItalicFont=cmunbi.ttf]{cmunrm.ttf}\setmonofont[Path=/usr/share/fonts/truetype/cmu/,UprightFont=cmuntt.ttf,BoldFont=cmuntb.ttf,ItalicFont=cmunit.ttf,BoldItalicFont=cmuntx.ttf]{cmunrm.ttf}}
\label{321}

To get better control over the headers, one can use the package \LaTeXTT{fancyhdr} written by
Piet van Oostrum. It provides several commands that allow you to customize the header and footer lines of
your document. For a more complete guide, the author of the package produced this \myhref{http://www.ctan.org/tex-archive/macros/latex/contrib/fancyhdr/fancyhdr.pdf}{documentation}.

To begin, add the following lines to your preamble:

\begin{Shaded}
\begin{Highlighting}[]

\NormalTok{\textbackslash{}usepackage\{fancyhdr\}}
\NormalTok{\textbackslash{}setlength\{\textbackslash{}headheight\}\{15.2pt\}}
\NormalTok{\textbackslash{}pagestyle\{fancy\}}
\end{Highlighting}
\end{Shaded}


You can now observe a new style in your document.

The \LaTeXTT{\textbackslash{}headheight} needs to be 13.6pt or more, otherwise you will get a warning and possibly formatting issues. Both the header and footer comprise three elements each according to its horizontal position (left, centre or right).

The styles supported by \LaTeXTT{fancyhdr}:
\begin{myitemize}
\item{}  the four LaTeX styles;
\item{}  \LaTeXTT{fancy} defines a new header for all pages but {\itshape \setmainfont[Path=/usr/share/fonts/truetype/cmu/,UprightFont=cmunrm.ttf,BoldFont=cmunbx.ttf,ItalicFont=cmunti.ttf,BoldItalicFont=cmunbi.ttf]{cmunti.ttf}\setmonofont[Path=/usr/share/fonts/truetype/cmu/,UprightFont=cmuntt.ttf,BoldFont=cmuntb.ttf,ItalicFont=cmunit.ttf,BoldItalicFont=cmuntx.ttf]{cmunti.ttf}\itshape plain-{}style}{$\text{ }$}\setmainfont[Path=/usr/share/fonts/truetype/cmu/,UprightFont=cmunrm.ttf,BoldFont=cmunbx.ttf,ItalicFont=cmunti.ttf,BoldItalicFont=cmunbi.ttf]{cmunrm.ttf}\setmonofont[Path=/usr/share/fonts/truetype/cmu/,UprightFont=cmuntt.ttf,BoldFont=cmuntb.ttf,ItalicFont=cmunit.ttf,BoldItalicFont=cmuntx.ttf]{cmunrm.ttf} pages such as chapters and titlepage;
\item{}  \LaTeXTT{fancyplain} is the same, but for absolutely all pages.
\end{myitemize}

\subsubsection{Style customization}
\label{322}

The styles can be customized with \LaTeXTT{fancyhdr} specific commands.
Those two styles may be configured directly, whereas for LaTeX styles you need to make a call to the \LaTeXTT{\textbackslash{}fancypagestyle} command.

To set header and footer style, \LaTeXTT{fancyhdr} provides three interfaces. They all provide the same features, you just use them differently. Choose the one you like most.

\begin{myitemize}
\item{}  You can use the following six commands.
\end{myitemize}

\begin{Shaded}
\begin{Highlighting}[]

\NormalTok{\textbackslash{}lhead[<even output>]\{<odd output>\}}
\NormalTok{\textbackslash{}chead[<even output>]\{<odd output>\}}
\NormalTok{\textbackslash{}rhead[<even output>]\{<odd output>\}}
\end{Highlighting}
\end{Shaded}

\begin{Shaded}
\begin{Highlighting}[]

\NormalTok{\textbackslash{}lfoot[<even output>]\{<odd output>\}}
\NormalTok{\textbackslash{}cfoot[<even output>]\{<odd output>\}}
\NormalTok{\textbackslash{}rfoot[<even output>]\{<odd output>\}}
\end{Highlighting}
\end{Shaded}


Hopefully, the behaviour of the above commands is fairly intuitive: if it has {\itshape \setmainfont[Path=/usr/share/fonts/truetype/cmu/,UprightFont=cmunrm.ttf,BoldFont=cmunbx.ttf,ItalicFont=cmunti.ttf,BoldItalicFont=cmunbi.ttf]{cmunti.ttf}\setmonofont[Path=/usr/share/fonts/truetype/cmu/,UprightFont=cmuntt.ttf,BoldFont=cmuntb.ttf,ItalicFont=cmunit.ttf,BoldItalicFont=cmuntx.ttf]{cmunti.ttf}\itshape head}{$\text{ }$}\setmainfont[Path=/usr/share/fonts/truetype/cmu/,UprightFont=cmunrm.ttf,BoldFont=cmunbx.ttf,ItalicFont=cmunti.ttf,BoldItalicFont=cmunbi.ttf]{cmunrm.ttf}\setmonofont[Path=/usr/share/fonts/truetype/cmu/,UprightFont=cmuntt.ttf,BoldFont=cmuntb.ttf,ItalicFont=cmunit.ttf,BoldItalicFont=cmuntx.ttf]{cmunrm.ttf} in it, it affects the head etc, and obviously, {\itshape \setmainfont[Path=/usr/share/fonts/truetype/cmu/,UprightFont=cmunrm.ttf,BoldFont=cmunbx.ttf,ItalicFont=cmunti.ttf,BoldItalicFont=cmunbi.ttf]{cmunti.ttf}\setmonofont[Path=/usr/share/fonts/truetype/cmu/,UprightFont=cmuntt.ttf,BoldFont=cmuntb.ttf,ItalicFont=cmunit.ttf,BoldItalicFont=cmuntx.ttf]{cmunti.ttf}\itshape l}\setmainfont[Path=/usr/share/fonts/truetype/cmu/,UprightFont=cmunrm.ttf,BoldFont=cmunbx.ttf,ItalicFont=cmunti.ttf,BoldItalicFont=cmunbi.ttf]{cmunrm.ttf}\setmonofont[Path=/usr/share/fonts/truetype/cmu/,UprightFont=cmuntt.ttf,BoldFont=cmuntb.ttf,ItalicFont=cmunit.ttf,BoldItalicFont=cmuntx.ttf]{cmunrm.ttf}, {\itshape \setmainfont[Path=/usr/share/fonts/truetype/cmu/,UprightFont=cmunrm.ttf,BoldFont=cmunbx.ttf,ItalicFont=cmunti.ttf,BoldItalicFont=cmunbi.ttf]{cmunti.ttf}\setmonofont[Path=/usr/share/fonts/truetype/cmu/,UprightFont=cmuntt.ttf,BoldFont=cmuntb.ttf,ItalicFont=cmunit.ttf,BoldItalicFont=cmuntx.ttf]{cmunti.ttf}\itshape c}{$\text{ }$}\setmainfont[Path=/usr/share/fonts/truetype/cmu/,UprightFont=cmunrm.ttf,BoldFont=cmunbx.ttf,ItalicFont=cmunti.ttf,BoldItalicFont=cmunbi.ttf]{cmunrm.ttf}\setmonofont[Path=/usr/share/fonts/truetype/cmu/,UprightFont=cmuntt.ttf,BoldFont=cmuntb.ttf,ItalicFont=cmunit.ttf,BoldItalicFont=cmuntx.ttf]{cmunrm.ttf} and {\itshape \setmainfont[Path=/usr/share/fonts/truetype/cmu/,UprightFont=cmunrm.ttf,BoldFont=cmunbx.ttf,ItalicFont=cmunti.ttf,BoldItalicFont=cmunbi.ttf]{cmunti.ttf}\setmonofont[Path=/usr/share/fonts/truetype/cmu/,UprightFont=cmuntt.ttf,BoldFont=cmuntb.ttf,ItalicFont=cmunit.ttf,BoldItalicFont=cmuntx.ttf]{cmunti.ttf}\itshape r}{$\text{ }$}\setmainfont[Path=/usr/share/fonts/truetype/cmu/,UprightFont=cmunrm.ttf,BoldFont=cmunbx.ttf,ItalicFont=cmunti.ttf,BoldItalicFont=cmunbi.ttf]{cmunrm.ttf}\setmonofont[Path=/usr/share/fonts/truetype/cmu/,UprightFont=cmuntt.ttf,BoldFont=cmuntb.ttf,ItalicFont=cmunit.ttf,BoldItalicFont=cmuntx.ttf]{cmunrm.ttf} means {\bfseries \setmainfont[Path=/usr/share/fonts/truetype/cmu/,UprightFont=cmunrm.ttf,BoldFont=cmunbx.ttf,ItalicFont=cmunti.ttf,BoldItalicFont=cmunbi.ttf]{cmunbx.ttf}\setmonofont[Path=/usr/share/fonts/truetype/cmu/,UprightFont=cmuntt.ttf,BoldFont=cmuntb.ttf,ItalicFont=cmunit.ttf,BoldItalicFont=cmuntx.ttf]{cmunbx.ttf}\bfseries l}\setmainfont[Path=/usr/share/fonts/truetype/cmu/,UprightFont=cmunrm.ttf,BoldFont=cmunbx.ttf,ItalicFont=cmunti.ttf,BoldItalicFont=cmunbi.ttf]{cmunrm.ttf}\setmonofont[Path=/usr/share/fonts/truetype/cmu/,UprightFont=cmuntt.ttf,BoldFont=cmuntb.ttf,ItalicFont=cmunit.ttf,BoldItalicFont=cmuntx.ttf]{cmunrm.ttf}eft, {\bfseries \setmainfont[Path=/usr/share/fonts/truetype/cmu/,UprightFont=cmunrm.ttf,BoldFont=cmunbx.ttf,ItalicFont=cmunti.ttf,BoldItalicFont=cmunbi.ttf]{cmunbx.ttf}\setmonofont[Path=/usr/share/fonts/truetype/cmu/,UprightFont=cmuntt.ttf,BoldFont=cmuntb.ttf,ItalicFont=cmunit.ttf,BoldItalicFont=cmuntx.ttf]{cmunbx.ttf}\bfseries c}\setmainfont[Path=/usr/share/fonts/truetype/cmu/,UprightFont=cmunrm.ttf,BoldFont=cmunbx.ttf,ItalicFont=cmunti.ttf,BoldItalicFont=cmunbi.ttf]{cmunrm.ttf}\setmonofont[Path=/usr/share/fonts/truetype/cmu/,UprightFont=cmuntt.ttf,BoldFont=cmuntb.ttf,ItalicFont=cmunit.ttf,BoldItalicFont=cmuntx.ttf]{cmunrm.ttf}entre and {\bfseries \setmainfont[Path=/usr/share/fonts/truetype/cmu/,UprightFont=cmunrm.ttf,BoldFont=cmunbx.ttf,ItalicFont=cmunti.ttf,BoldItalicFont=cmunbi.ttf]{cmunbx.ttf}\setmonofont[Path=/usr/share/fonts/truetype/cmu/,UprightFont=cmuntt.ttf,BoldFont=cmuntb.ttf,ItalicFont=cmunit.ttf,BoldItalicFont=cmuntx.ttf]{cmunbx.ttf}\bfseries r}\setmainfont[Path=/usr/share/fonts/truetype/cmu/,UprightFont=cmunrm.ttf,BoldFont=cmunbx.ttf,ItalicFont=cmunti.ttf,BoldItalicFont=cmunbi.ttf]{cmunrm.ttf}\setmonofont[Path=/usr/share/fonts/truetype/cmu/,UprightFont=cmuntt.ttf,BoldFont=cmuntb.ttf,ItalicFont=cmunit.ttf,BoldItalicFont=cmuntx.ttf]{cmunrm.ttf}ight respectively.

\begin{myitemize}
\item{}  You can also use the command \LaTeXTT{\textbackslash{}fancyhead} for header and \LaTeXTT{\textbackslash{}fancyfoot} for footer. They work in the same way, so we\textquotesingle{}ll explain only the first one. The syntax is:
\end{myitemize}


\begin{Shaded}
\begin{Highlighting}[]

\NormalTok{\textbackslash{}fancyhead[selectors]\{output you want\}}
\end{Highlighting}
\end{Shaded}


You can use multiple selectors optionally separated by a comma. The selectors are the following:
\begin{longtable}{>{\RaggedRight}p{0.24237\linewidth}>{\RaggedRight}p{0.70048\linewidth}} 
\hspace*{0pt}\ignorespaces{}\hspace*{0pt} \LaTeXTT{E} &\hspace*{0pt}\ignorespaces{}\hspace*{0pt} even page\\ \hspace*{0pt}\ignorespaces{}\hspace*{0pt} \LaTeXTT{O} &\hspace*{0pt}\ignorespaces{}\hspace*{0pt} odd page\\ \hspace*{0pt}\ignorespaces{}\hspace*{0pt} \LaTeXTT{L} &\hspace*{0pt}\ignorespaces{}\hspace*{0pt} left side\\ \hspace*{0pt}\ignorespaces{}\hspace*{0pt} \LaTeXTT{C} &\hspace*{0pt}\ignorespaces{}\hspace*{0pt} centered\\ \hspace*{0pt}\ignorespaces{}\hspace*{0pt} \LaTeXTT{R} &\hspace*{0pt}\ignorespaces{}\hspace*{0pt} right side 
\end{longtable}

so \LaTeXTT{CE,RO} will refer to the center of the even pages and to the right side of the odd pages. 

\begin{myitemize}
\item{}  \LaTeXTT{\textbackslash{}fancyhf} is a merge of \LaTeXTT{\textbackslash{}fancyhead} and \LaTeXTT{\textbackslash{}fancyfoot}, hence the name. There are two additional selectors \LaTeXTT{H} and \LaTeXTT{F} to specify the header or the footer, respectively. If you omit the \LaTeXTT{H} and the \LaTeXTT{F}, it will set the fields for both.
\end{myitemize}


These commands will only work for \LaTeXTT{fancy} and \LaTeXTT{fancyplain}. To customize LaTeX default style you need the \LaTeXTT{\textbackslash{}fancyplainstyle} command. See below for examples.

For a clean customization, we recommend you start from scratch. To do so you should {\itshape \setmainfont[Path=/usr/share/fonts/truetype/cmu/,UprightFont=cmunrm.ttf,BoldFont=cmunbx.ttf,ItalicFont=cmunti.ttf,BoldItalicFont=cmunbi.ttf]{cmunti.ttf}\setmonofont[Path=/usr/share/fonts/truetype/cmu/,UprightFont=cmuntt.ttf,BoldFont=cmuntb.ttf,ItalicFont=cmunit.ttf,BoldItalicFont=cmuntx.ttf]{cmunti.ttf}\itshape erase}{$\text{ }$}\setmainfont[Path=/usr/share/fonts/truetype/cmu/,UprightFont=cmunrm.ttf,BoldFont=cmunbx.ttf,ItalicFont=cmunti.ttf,BoldItalicFont=cmunbi.ttf]{cmunrm.ttf}\setmonofont[Path=/usr/share/fonts/truetype/cmu/,UprightFont=cmuntt.ttf,BoldFont=cmuntb.ttf,ItalicFont=cmunit.ttf,BoldItalicFont=cmuntx.ttf]{cmunrm.ttf} the current pagestyle.  Providing empty values will make the field blank. So
\begin{Shaded}
\begin{Highlighting}[]

\NormalTok{\textbackslash{}fancyhf\{\}}
\end{Highlighting}
\end{Shaded}

will just delete the current heading/footer configuration, so you can make your own.
\subsubsection{Plain pages}
\label{323}

There are two ways to change the style of plain pages like chapters and titlepage.

First you can use the \LaTeXTT{fancyplain} style. If you do so, you can use the command \LaTeXTT{\textbackslash{}fancyplain\{...\}\{...\}} inside \LaTeXTT{fancyhdr} commands like \LaTeXTT{\textbackslash{}lhead\{...\}}, etc.

When LaTeX wants to create a page with an empty style, it will insert the first argument of \LaTeXTT{\textbackslash{}fancyplain}, in all the other cases it will use the second argument. For instance:

\begin{Shaded}
\begin{Highlighting}[]

\NormalTok{\textbackslash{}pagestyle\{fancyplain\}}
\NormalTok{\textbackslash{}fancyhf\{\}}
\NormalTok{\textbackslash{}lhead\{ \textbackslash{}fancyplain\{\}\{Author Name\} \}}
\NormalTok{\textbackslash{}rhead\{ \textbackslash{}fancyplain\{\}\{\textbackslash{}today\} \}}
\NormalTok{\textbackslash{}rfoot\{ \textbackslash{}fancyplain\{\}\{\textbackslash{}thepage\} \}}
\end{Highlighting}
\end{Shaded}


It has the same behavior of the previous code, but you will get empty header and footer in the title and at the beginning of chapters.

Alternatively you could redefine the {\itshape \setmainfont[Path=/usr/share/fonts/truetype/cmu/,UprightFont=cmunrm.ttf,BoldFont=cmunbx.ttf,ItalicFont=cmunti.ttf,BoldItalicFont=cmunbi.ttf]{cmunti.ttf}\setmonofont[Path=/usr/share/fonts/truetype/cmu/,UprightFont=cmuntt.ttf,BoldFont=cmuntb.ttf,ItalicFont=cmunit.ttf,BoldItalicFont=cmuntx.ttf]{cmunti.ttf}\itshape plain}{$\text{ }$}\setmainfont[Path=/usr/share/fonts/truetype/cmu/,UprightFont=cmunrm.ttf,BoldFont=cmunbx.ttf,ItalicFont=cmunti.ttf,BoldItalicFont=cmunbi.ttf]{cmunrm.ttf}\setmonofont[Path=/usr/share/fonts/truetype/cmu/,UprightFont=cmuntt.ttf,BoldFont=cmuntb.ttf,ItalicFont=cmunit.ttf,BoldItalicFont=cmuntx.ttf]{cmunrm.ttf} style, for example to have a really plain page when you want. The command to use is \LaTeXTT{\textbackslash{}fancypagestyle\{plain\}\{...\}} and the argument can contain all the commands explained before. An example is the following:

\begin{Shaded}
\begin{Highlighting}[]

\NormalTok{\textbackslash{}pagestyle\{fancy\}}
 
\NormalTok{\textbackslash{}fancypagestyle\{plain\}\{ }\CommentTok{%}
  \NormalTok{\textbackslash{}fancyhf\{\} }\CommentTok{% remove everything}
  \NormalTok{\textbackslash{}renewcommand\{\textbackslash{}headrulewidth\}\{0pt\} }\CommentTok{% remove lines as well}
  \NormalTok{\textbackslash{}renewcommand\{\textbackslash{}footrulewidth\}\{0pt\}}
\NormalTok{\}}
\end{Highlighting}
\end{Shaded}


In that case you can use any style but \LaTeXTT{fancyplain} because it would override your redefinition.
\subsubsection{Examples}
\label{324}

For two-{}sided, it\textquotesingle{}s common to mirror the style of opposite pages, you tend to think in terms of {\itshape \setmainfont[Path=/usr/share/fonts/truetype/cmu/,UprightFont=cmunrm.ttf,BoldFont=cmunbx.ttf,ItalicFont=cmunti.ttf,BoldItalicFont=cmunbi.ttf]{cmunti.ttf}\setmonofont[Path=/usr/share/fonts/truetype/cmu/,UprightFont=cmuntt.ttf,BoldFont=cmuntb.ttf,ItalicFont=cmunit.ttf,BoldItalicFont=cmuntx.ttf]{cmunti.ttf}\itshape inner}{$\text{ }$}\setmainfont[Path=/usr/share/fonts/truetype/cmu/,UprightFont=cmunrm.ttf,BoldFont=cmunbx.ttf,ItalicFont=cmunti.ttf,BoldItalicFont=cmunbi.ttf]{cmunrm.ttf}\setmonofont[Path=/usr/share/fonts/truetype/cmu/,UprightFont=cmuntt.ttf,BoldFont=cmuntb.ttf,ItalicFont=cmunit.ttf,BoldItalicFont=cmuntx.ttf]{cmunrm.ttf} and {\itshape \setmainfont[Path=/usr/share/fonts/truetype/cmu/,UprightFont=cmunrm.ttf,BoldFont=cmunbx.ttf,ItalicFont=cmunti.ttf,BoldItalicFont=cmunbi.ttf]{cmunti.ttf}\setmonofont[Path=/usr/share/fonts/truetype/cmu/,UprightFont=cmuntt.ttf,BoldFont=cmuntb.ttf,ItalicFont=cmunit.ttf,BoldItalicFont=cmuntx.ttf]{cmunti.ttf}\itshape outer}\setmainfont[Path=/usr/share/fonts/truetype/cmu/,UprightFont=cmunrm.ttf,BoldFont=cmunbx.ttf,ItalicFont=cmunti.ttf,BoldItalicFont=cmunbi.ttf]{cmunrm.ttf}\setmonofont[Path=/usr/share/fonts/truetype/cmu/,UprightFont=cmuntt.ttf,BoldFont=cmuntb.ttf,ItalicFont=cmunit.ttf,BoldItalicFont=cmuntx.ttf]{cmunrm.ttf}. So, the same example as above for two-{}sided is:

\begin{Shaded}
\begin{Highlighting}[]

\NormalTok{\textbackslash{}lhead[Author Name]\{\}}
\NormalTok{\textbackslash{}rhead[]\{Author Name\}}
\NormalTok{\textbackslash{}lhead[]\{\textbackslash{}today\}}
\NormalTok{\textbackslash{}rhead[\textbackslash{}today]\{\}}
\NormalTok{\textbackslash{}lfoot[\textbackslash{}thepage]\{\}}
\NormalTok{\textbackslash{}rfoot[]\{\textbackslash{}thepage\}}
\end{Highlighting}
\end{Shaded}


This is effectively saying author name is top outer, today\textquotesingle{}s date is top inner, and current page number is bottom outer. Using \LaTeXTT{\textbackslash{}fancyhf} can make it shorter:

\begin{Shaded}
\begin{Highlighting}[]

\NormalTok{\textbackslash{}fancyhf[HLE,HRO]\{Author's Name\}}
\NormalTok{\textbackslash{}fancyhf[HRE,HLO]\{\textbackslash{}today\}}
\NormalTok{\textbackslash{}fancyhf[FLE,FRO]\{\textbackslash{}thepage\}}
\end{Highlighting}
\end{Shaded}


Here is the complete code of a possible style you could use for a two-{}sided document:

\begin{Shaded}
\begin{Highlighting}[]

\NormalTok{\textbackslash{}usepackage\{fancyhdr\}}
\NormalTok{\textbackslash{}setlength\{\textbackslash{}headheight\}\{15pt\}}
 
\NormalTok{\textbackslash{}pagestyle\{fancy\}}
\NormalTok{\textbackslash{}renewcommand\{\textbackslash{}chaptermark\}[1]\{ \textbackslash{}markboth\{#1\}\{\} \}}
\NormalTok{\textbackslash{}renewcommand\{\textbackslash{}sectionmark\}[1]\{ \textbackslash{}markright\{#1\} \}}
 
\NormalTok{\textbackslash{}fancyhf\{\}}
\NormalTok{\textbackslash{}fancyhead[LE,RO]\{\textbackslash{}thepage\}}
\NormalTok{\textbackslash{}fancyhead[RE]\{\textbackslash{}textit\{ \textbackslash{}nouppercase\{\textbackslash{}leftmark\}\} \}}
\NormalTok{\textbackslash{}fancyhead[LO]\{\textbackslash{}textit\{ \textbackslash{}nouppercase\{\textbackslash{}rightmark\}\} \}}
 
\NormalTok{\textbackslash{}fancypagestyle\{plain\}\{ }\CommentTok{%}
  \NormalTok{\textbackslash{}fancyhf\{\} }\CommentTok{% remove everything}
  \NormalTok{\textbackslash{}renewcommand\{\textbackslash{}headrulewidth\}\{0pt\} }\CommentTok
    \NormalTok{\textbackslash{}fancyhf\{\}}\CommentTok{%}
    \CommentTok{% Note the ## here. It's required because \textbackslash{}fancypagestyle is making a macro}
 \NormalTok{(\textbackslash{}ps@fancybook).}
    \CommentTok{% If we just wrote #1, TeX would think that it's the argument to}
 \NormalTok{\textbackslash{}ps@fancybook, but}
    \CommentTok{% \textbackslash{}ps@fancybook doesn't take any arguments, so TeX would complain with an}
 \NormalTok{error message.}
    \CommentTok
    \NormalTok{\textbackslash{}renewcommand*\{\textbackslash{}chaptermark\}[1]\{ \textbackslash{}markboth\{\textbackslash{}chaptername\textbackslash{} \textbackslash{}thechapter: ##1\}\{\}}
 \NormalTok{\}}\CommentTok{%}
    \CommentTok{% Increase the length of the header such that the folios }
    \CommentTok{% (typography jargon for page numbers) move into the margin}
    \NormalTok{\textbackslash{}fancyhfoffset[LE]\{6mm\}}\CommentTok
    \CommentTok
    \NormalTok{\textbackslash{}fancyhead[RO]\{\textbackslash{}rightmark\textbackslash{}hskip3mm\textbackslash{}vrule\textbackslash{}hskip3mm\textbackslash{}thepage\}}\CommentTok{%}
\NormalTok{\}}
\end{Highlighting}
\end{Shaded}

\subsection{Page {\itshape \setmainfont[Path=/usr/share/fonts/truetype/cmu/,UprightFont=cmunrm.ttf,BoldFont=cmunbx.ttf,ItalicFont=cmunti.ttf,BoldItalicFont=cmunbi.ttf]{cmunti.ttf}\setmonofont[Path=/usr/share/fonts/truetype/cmu/,UprightFont=cmuntt.ttf,BoldFont=cmuntb.ttf,ItalicFont=cmunit.ttf,BoldItalicFont=cmuntx.ttf]{cmunti.ttf}\itshape n}{$\text{ }$}\setmainfont[Path=/usr/share/fonts/truetype/cmu/,UprightFont=cmunrm.ttf,BoldFont=cmunbx.ttf,ItalicFont=cmunti.ttf,BoldItalicFont=cmunbi.ttf]{cmunrm.ttf}\setmonofont[Path=/usr/share/fonts/truetype/cmu/,UprightFont=cmuntt.ttf,BoldFont=cmuntb.ttf,ItalicFont=cmunit.ttf,BoldItalicFont=cmuntx.ttf]{cmunrm.ttf} of {\itshape \setmainfont[Path=/usr/share/fonts/truetype/cmu/,UprightFont=cmunrm.ttf,BoldFont=cmunbx.ttf,ItalicFont=cmunti.ttf,BoldItalicFont=cmunbi.ttf]{cmunti.ttf}\setmonofont[Path=/usr/share/fonts/truetype/cmu/,UprightFont=cmuntt.ttf,BoldFont=cmuntb.ttf,ItalicFont=cmunit.ttf,BoldItalicFont=cmuntx.ttf]{cmunti.ttf}\itshape m}{$\text{ }$}\setmainfont[Path=/usr/share/fonts/truetype/cmu/,UprightFont=cmunrm.ttf,BoldFont=cmunbx.ttf,ItalicFont=cmunti.ttf,BoldItalicFont=cmunbi.ttf]{cmunrm.ttf}\setmonofont[Path=/usr/share/fonts/truetype/cmu/,UprightFont=cmuntt.ttf,BoldFont=cmuntb.ttf,ItalicFont=cmunit.ttf,BoldItalicFont=cmuntx.ttf]{cmunrm.ttf}}
\label{325}

Some people like to put the current page number in context with the whole document. LaTeX only provides access to the current page number.  However, you can use the \LaTeXTT{lastpage} package to find the total number of pages, like this:

\begin{Shaded}
\begin{Highlighting}[]

\NormalTok{\textbackslash{}usepackage\{lastpage\}}
\NormalTok{...}
\NormalTok{\textbackslash{}cfoot\{\textbackslash{}thepage\textbackslash{} of \textbackslash{}pageref\{LastPage\} \}}
\end{Highlighting}
\end{Shaded}


{\itshape \setmainfont[Path=/usr/share/fonts/truetype/cmu/,UprightFont=cmunrm.ttf,BoldFont=cmunbx.ttf,ItalicFont=cmunti.ttf,BoldItalicFont=cmunbi.ttf]{cmunti.ttf}\setmonofont[Path=/usr/share/fonts/truetype/cmu/,UprightFont=cmuntt.ttf,BoldFont=cmuntb.ttf,ItalicFont=cmunit.ttf,BoldItalicFont=cmuntx.ttf]{cmunti.ttf}\itshape Note the capital letters}\setmainfont[Path=/usr/share/fonts/truetype/cmu/,UprightFont=cmunrm.ttf,BoldFont=cmunbx.ttf,ItalicFont=cmunti.ttf,BoldItalicFont=cmunbi.ttf]{cmunrm.ttf}\setmonofont[Path=/usr/share/fonts/truetype/cmu/,UprightFont=cmuntt.ttf,BoldFont=cmuntb.ttf,ItalicFont=cmunit.ttf,BoldItalicFont=cmuntx.ttf]{cmunrm.ttf}. Also, add a backslash after \LaTeXTT{\textbackslash{}thepage} to ensure adequate space between the page number and \textquotesingle{}of\textquotesingle{}. And recall, when using references, that you have to run LaTeX an extra time to resolve the cross-{}references.
\subsection{Alternative packages}
\label{326}

Other packages for page styles are \LaTeXTT{scrpage2}, very similar to \LaTeXTT{fancyhdr}, and \LaTeXTT{titleps}, which takes a one-{}stage approach, without having to use \LaTeXTT{\textbackslash{}leftmark} or \LaTeXTT{\textbackslash{}rightmark}.
\section{Page background}
\label{327}

The \LaTeXTT{eso-{}pic} package will let you print content in the background of every page or individual pages.

\begin{Shaded}
\begin{Highlighting}[]

\NormalTok{\textbackslash{}usepackage\{tikz\} }\CommentTok
  \NormalTok{\textbackslash{}begin\{tikzpicture\}}
    \NormalTok{\textbackslash{}node[left color=#1,right color=#2] \{#3\};}
  \NormalTok{\textbackslash{}end\{tikzpicture\}}\CommentTok
  \NormalTok{\textbackslash{}AtPageLowerLeft\{}\CommentTok
        \NormalTok{\textbackslash{}begin\{minipage\}\{\textbackslash{}paperheight\}}\CommentTok
      \NormalTok{\}}
    \NormalTok{\}}\CommentTok
\NormalTok{\}}
\end{Highlighting}
\end{Shaded}


The starred-{}version of the \LaTeXTT{\textbackslash{}AddToShipoutPicture} command applies to the current page only.
\section{Multi-{}column pages}
\label{328}\subsection{Using the {\itshape \setmainfont[Path=/usr/share/fonts/truetype/cmu/,UprightFont=cmunrm.ttf,BoldFont=cmunbx.ttf,ItalicFont=cmunti.ttf,BoldItalicFont=cmunbi.ttf]{cmunti.ttf}\setmonofont[Path=/usr/share/fonts/truetype/cmu/,UprightFont=cmuntt.ttf,BoldFont=cmuntb.ttf,ItalicFont=cmunit.ttf,BoldItalicFont=cmuntx.ttf]{cmunti.ttf}\itshape twocolumn}{$\text{ }$}\setmainfont[Path=/usr/share/fonts/truetype/cmu/,UprightFont=cmunrm.ttf,BoldFont=cmunbx.ttf,ItalicFont=cmunti.ttf,BoldItalicFont=cmunbi.ttf]{cmunrm.ttf}\setmonofont[Path=/usr/share/fonts/truetype/cmu/,UprightFont=cmuntt.ttf,BoldFont=cmuntb.ttf,ItalicFont=cmunit.ttf,BoldItalicFont=cmuntx.ttf]{cmunrm.ttf} optional class argument}
\label{329}
Using a standard Latex document class, like article, you can simply pass the optional argument {\itshape \setmainfont[Path=/usr/share/fonts/truetype/cmu/,UprightFont=cmunrm.ttf,BoldFont=cmunbx.ttf,ItalicFont=cmunti.ttf,BoldItalicFont=cmunbi.ttf]{cmunti.ttf}\setmonofont[Path=/usr/share/fonts/truetype/cmu/,UprightFont=cmuntt.ttf,BoldFont=cmuntb.ttf,ItalicFont=cmunit.ttf,BoldItalicFont=cmuntx.ttf]{cmunti.ttf}\itshape twocolumn}{$\text{ }$}\setmainfont[Path=/usr/share/fonts/truetype/cmu/,UprightFont=cmunrm.ttf,BoldFont=cmunbx.ttf,ItalicFont=cmunti.ttf,BoldItalicFont=cmunbi.ttf]{cmunrm.ttf}\setmonofont[Path=/usr/share/fonts/truetype/cmu/,UprightFont=cmuntt.ttf,BoldFont=cmuntb.ttf,ItalicFont=cmunit.ttf,BoldItalicFont=cmuntx.ttf]{cmunrm.ttf} to the document class: \LaTeXTT{\textbackslash{}documentclass{$\text{[}$}twocolumn{$\text{]}$}\{article\}} which will give the desired effect.

While this approach is useful, it has limitations. The \LaTeXTT{multicol} package provides the following advantages:

\begin{myitemize}
\item{}  Can support up to ten columns.
\item{}  Implements a {\itshape \setmainfont[Path=/usr/share/fonts/truetype/cmu/,UprightFont=cmunrm.ttf,BoldFont=cmunbx.ttf,ItalicFont=cmunti.ttf,BoldItalicFont=cmunbi.ttf]{cmunti.ttf}\setmonofont[Path=/usr/share/fonts/truetype/cmu/,UprightFont=cmuntt.ttf,BoldFont=cmuntb.ttf,ItalicFont=cmunit.ttf,BoldItalicFont=cmuntx.ttf]{cmunti.ttf}\itshape multicols}{$\text{ }$}\setmainfont[Path=/usr/share/fonts/truetype/cmu/,UprightFont=cmunrm.ttf,BoldFont=cmunbx.ttf,ItalicFont=cmunti.ttf,BoldItalicFont=cmunbi.ttf]{cmunrm.ttf}\setmonofont[Path=/usr/share/fonts/truetype/cmu/,UprightFont=cmuntt.ttf,BoldFont=cmuntb.ttf,ItalicFont=cmunit.ttf,BoldItalicFont=cmuntx.ttf]{cmunrm.ttf} environment, therefore, it is possible to mix the number of columns within a document.
\item{}  Additionally, the environment can be nested inside other environments, such as \LaTeXTT{figure}.
\item{}  \LaTeXTT{multicol} outputs {\itshape \setmainfont[Path=/usr/share/fonts/truetype/cmu/,UprightFont=cmunrm.ttf,BoldFont=cmunbx.ttf,ItalicFont=cmunti.ttf,BoldItalicFont=cmunbi.ttf]{cmunti.ttf}\setmonofont[Path=/usr/share/fonts/truetype/cmu/,UprightFont=cmuntt.ttf,BoldFont=cmuntb.ttf,ItalicFont=cmunit.ttf,BoldItalicFont=cmuntx.ttf]{cmunti.ttf}\itshape balanced}{$\text{ }$}\setmainfont[Path=/usr/share/fonts/truetype/cmu/,UprightFont=cmunrm.ttf,BoldFont=cmunbx.ttf,ItalicFont=cmunti.ttf,BoldItalicFont=cmunbi.ttf]{cmunrm.ttf}\setmonofont[Path=/usr/share/fonts/truetype/cmu/,UprightFont=cmuntt.ttf,BoldFont=cmuntb.ttf,ItalicFont=cmunit.ttf,BoldItalicFont=cmuntx.ttf]{cmunrm.ttf} columns, whereby the columns on the final page will be of roughly equal length.
\item{}  Vertical rules between columns can be customised.
\item{}  Column environments can be easily customised locally or globally.
\end{myitemize}

\subsection{Using \LaTeXTT{multicol} package}
\label{330}
The \LaTeXTT{multicol} package overcomes some of the shortcomings of {\itshape \setmainfont[Path=/usr/share/fonts/truetype/cmu/,UprightFont=cmunrm.ttf,BoldFont=cmunbx.ttf,ItalicFont=cmunti.ttf,BoldItalicFont=cmunbi.ttf]{cmunti.ttf}\setmonofont[Path=/usr/share/fonts/truetype/cmu/,UprightFont=cmuntt.ttf,BoldFont=cmuntb.ttf,ItalicFont=cmunit.ttf,BoldItalicFont=cmuntx.ttf]{cmunti.ttf}\itshape twocolumn}{$\text{ }$}\setmainfont[Path=/usr/share/fonts/truetype/cmu/,UprightFont=cmunrm.ttf,BoldFont=cmunbx.ttf,ItalicFont=cmunti.ttf,BoldItalicFont=cmunbi.ttf]{cmunrm.ttf}\setmonofont[Path=/usr/share/fonts/truetype/cmu/,UprightFont=cmuntt.ttf,BoldFont=cmuntb.ttf,ItalicFont=cmunit.ttf,BoldItalicFont=cmuntx.ttf]{cmunrm.ttf} and provides the \LaTeXTT{multicol} environment. To create a typical two-{}column layout:

\begin{Shaded}
\begin{Highlighting}[]

\NormalTok{\textbackslash{}begin\{multicols\}\{2\}}
  \NormalTok{lots of text}
\NormalTok{\textbackslash{}end\{multicols\}}
\end{Highlighting}
\end{Shaded}


Floats are not fully supported by this environment. It can only cope if you use the starred forms of the float commands (e.g., \LaTeXTT{\textbackslash{}begin\{figure*\}} ) which makes the float span all columns. This is not hugely problematic, since floats of the same width as a column may be too small, and you would probably want to span them anyway. See \mylref{374}{this section} for a more detailed discussion.

The \LaTeXTT{multicol} package has two important parameters which can be \mylref{459}{set} using \LaTeXTT{\textbackslash{}setlength}:

\begin{myitemize}
\item{}  \LaTeXTT{\textbackslash{}columnseprule}, sets the width of the vertical rule between columns and defaults to 0pt
\item{}  \LaTeXTT{\textbackslash{}columnsep}, sets the horizontal space between columns and the defaults to 10pt, which is quite narrow
\end{myitemize}


To force a break in a column, the command \LaTeXTT{\textbackslash{}columnbreak} is used.
\section{Manual page formatting}
\label{331}

There may be instances, especially in very long documents, such as books, that LaTeX will not get all page breaks looking as good as it could. It may, therefore, be necessary to manually tweak the page formatting. Of course, you should only do this at the very final stage of producing your document, once all the content is complete. LaTeX offers the following:

\begin{longtable}{|>{\RaggedRight}p{0.31558\linewidth}|>{\RaggedRight}p{0.62728\linewidth}|} \hline 
\hspace*{0pt}\ignorespaces{}\hspace*{0pt} \LaTeXTT{\textbackslash{}newpage}&\hspace*{0pt}\ignorespaces{}\hspace*{0pt} Ends the current page and starts a new one.\\ \hline \hspace*{0pt}\ignorespaces{}\hspace*{0pt} \LaTeXTT{\textbackslash{}pagebreak{$\text{[}$}number{$\text{]}$}}&\hspace*{0pt}\ignorespaces{}\hspace*{0pt} Breaks the current page at the point of the command. The optional {\itshape \setmainfont[Path=/usr/share/fonts/truetype/cmu/,UprightFont=cmunrm.ttf,BoldFont=cmunbx.ttf,ItalicFont=cmunti.ttf,BoldItalicFont=cmunbi.ttf]{cmunti.ttf}\setmonofont[Path=/usr/share/fonts/truetype/cmu/,UprightFont=cmuntt.ttf,BoldFont=cmuntb.ttf,ItalicFont=cmunit.ttf,BoldItalicFont=cmuntx.ttf]{cmunti.ttf}\itshape number}{$\text{ }$}\setmainfont[Path=/usr/share/fonts/truetype/cmu/,UprightFont=cmunrm.ttf,BoldFont=cmunbx.ttf,ItalicFont=cmunti.ttf,BoldItalicFont=cmunbi.ttf]{cmunrm.ttf}\setmonofont[Path=/usr/share/fonts/truetype/cmu/,UprightFont=cmuntt.ttf,BoldFont=cmuntb.ttf,ItalicFont=cmunit.ttf,BoldItalicFont=cmuntx.ttf]{cmunrm.ttf} argument sets the priority in a scale from 0 to 4.\\ \hline \hspace*{0pt}\ignorespaces{}\hspace*{0pt} \LaTeXTT{\textbackslash{}nopagebreak{$\text{[}$}number{$\text{]}$}}&\hspace*{0pt}\ignorespaces{}\hspace*{0pt} Stops the page being broken at the point of the command. The optional {\itshape \setmainfont[Path=/usr/share/fonts/truetype/cmu/,UprightFont=cmunrm.ttf,BoldFont=cmunbx.ttf,ItalicFont=cmunti.ttf,BoldItalicFont=cmunbi.ttf]{cmunti.ttf}\setmonofont[Path=/usr/share/fonts/truetype/cmu/,UprightFont=cmuntt.ttf,BoldFont=cmuntb.ttf,ItalicFont=cmunit.ttf,BoldItalicFont=cmuntx.ttf]{cmunti.ttf}\itshape number}{$\text{ }$}\setmainfont[Path=/usr/share/fonts/truetype/cmu/,UprightFont=cmunrm.ttf,BoldFont=cmunbx.ttf,ItalicFont=cmunti.ttf,BoldItalicFont=cmunbi.ttf]{cmunrm.ttf}\setmonofont[Path=/usr/share/fonts/truetype/cmu/,UprightFont=cmuntt.ttf,BoldFont=cmuntb.ttf,ItalicFont=cmunit.ttf,BoldItalicFont=cmuntx.ttf]{cmunrm.ttf} argument sets the priority in a scale from 0 to 4.\\ \hline \hspace*{0pt}\ignorespaces{}\hspace*{0pt} \LaTeXTT{\textbackslash{}clearpage}&\hspace*{0pt}\ignorespaces{}\hspace*{0pt} Ends the current page and causes any floats encountered in the input, but yet to appear, to be printed.\\ \hline 
\end{longtable}

\section{Widows and orphans}
\label{332}
\myhref{https://en.wikipedia.org/wiki/Widows\%20and\%20orphans}{w:Widows and orphans}
In professional books, it\textquotesingle{}s not desirable to have single lines at the beginning or end of a page. In typesetting such situations are called \textquotesingle{}widows\textquotesingle{} and \textquotesingle{}orphans\textquotesingle{}. Normally it is possible that widows and orphans appear in LaTeX documents. You can try to deal with them using manual page formatting, but there\textquotesingle{}s also an automatic solution.

LaTeX has a parameter for \textquotesingle{}penalty\textquotesingle{} for widows and orphans (\textquotesingle{}club lines\textquotesingle{} in LaTeX terminology). With the greater penalty LaTeX will try more to avoid widows and orphans. You can try to increase these penalties by putting following commands in your document preamble:

\begin{Shaded}
\begin{Highlighting}[]

\NormalTok{\textbackslash{}widowpenalty=300}
\NormalTok{\textbackslash{}clubpenalty=300}
\end{Highlighting}
\end{Shaded}


If this does not help, you can try increasing these values even more, to a maximum of 10000. However, it is not recommended to set this value too high, as setting it to 10000 forbids LaTeX from doing this altogether, which might result in strange behavior.

It also helps to have rubber band values for the space between paragraphs:

\begin{Shaded}
\begin{Highlighting}[]

\NormalTok{\textbackslash{}setlength\{\textbackslash{}parskip\}\{3ex plus 2ex minus 2ex\}}
\end{Highlighting}
\end{Shaded}


Alternatively, you can use the {\ttfamily \setmainfont[Path=/usr/share/fonts/truetype/cmu/,UprightFont=cmunrm.ttf,BoldFont=cmunbx.ttf,ItalicFont=cmunti.ttf,BoldItalicFont=cmunbi.ttf]{cmuntt.ttf}\setmonofont[Path=/usr/share/fonts/truetype/cmu/,UprightFont=cmuntt.ttf,BoldFont=cmuntb.ttf,ItalicFont=cmunit.ttf,BoldItalicFont=cmuntx.ttf]{cmuntt.ttf}\ttfamily needspace}{$\text{ }$}\setmainfont[Path=/usr/share/fonts/truetype/cmu/,UprightFont=cmunrm.ttf,BoldFont=cmunbx.ttf,ItalicFont=cmunti.ttf,BoldItalicFont=cmunbi.ttf]{cmunrm.ttf}\setmonofont[Path=/usr/share/fonts/truetype/cmu/,UprightFont=cmuntt.ttf,BoldFont=cmuntb.ttf,ItalicFont=cmunit.ttf,BoldItalicFont=cmuntx.ttf]{cmunrm.ttf} package to {\itshape \setmainfont[Path=/usr/share/fonts/truetype/cmu/,UprightFont=cmunrm.ttf,BoldFont=cmunbx.ttf,ItalicFont=cmunti.ttf,BoldItalicFont=cmunbi.ttf]{cmunti.ttf}\setmonofont[Path=/usr/share/fonts/truetype/cmu/,UprightFont=cmuntt.ttf,BoldFont=cmuntb.ttf,ItalicFont=cmunit.ttf,BoldItalicFont=cmuntx.ttf]{cmunti.ttf}\itshape reserve}{$\text{ }$}\setmainfont[Path=/usr/share/fonts/truetype/cmu/,UprightFont=cmunrm.ttf,BoldFont=cmunbx.ttf,ItalicFont=cmunti.ttf,BoldItalicFont=cmunbi.ttf]{cmunrm.ttf}\setmonofont[Path=/usr/share/fonts/truetype/cmu/,UprightFont=cmuntt.ttf,BoldFont=cmuntb.ttf,ItalicFont=cmunit.ttf,BoldItalicFont=cmuntx.ttf]{cmunrm.ttf} some lines and thus to prevent page breaking for those lines.

\begin{Shaded}
\begin{Highlighting}[]

\NormalTok{\textbackslash{}needspace\{5\textbackslash{}baselineskip\}}
\NormalTok{Some}
\NormalTok{text}
\NormalTok{on}
\NormalTok{5}
\NormalTok{lines.}
\end{Highlighting}
\end{Shaded}

\section{Troubleshooting}
\label{333}
A very useful troubleshooting and designing technique is to turn on the showframe option in the geometry package (which has the same effect
as the showframe package described above).  It draws bounding boxes around the major page elements, which can be helpful because the boundaries of various regions are usually invisible, and complicated by padding whitespace.


\begin{Shaded}
\begin{Highlighting}[]

\NormalTok{\textbackslash{}usepackage[showframe]\{geometry\}}
\end{Highlighting}
\end{Shaded}

\section{Notes and References}
\label{334}
\ARoberts{}
\LaTeXNullTemplate{}
\chapter{Importing Graphics}

\myminitoc
\label{335}

\label{336}


There are two possibilities to include graphics in your document. Either create them with some special code, a topic which will be discussed in the {\itshape \setmainfont[Path=/usr/share/fonts/truetype/cmu/,UprightFont=cmunrm.ttf,BoldFont=cmunbx.ttf,ItalicFont=cmunti.ttf,BoldItalicFont=cmunbi.ttf]{cmunti.ttf}\setmonofont[Path=/usr/share/fonts/truetype/cmu/,UprightFont=cmuntt.ttf,BoldFont=cmuntb.ttf,ItalicFont=cmunit.ttf,BoldItalicFont=cmuntx.ttf]{cmunti.ttf}\itshape Creating Graphics}{$\text{ }$}\setmainfont[Path=/usr/share/fonts/truetype/cmu/,UprightFont=cmunrm.ttf,BoldFont=cmunbx.ttf,ItalicFont=cmunti.ttf,BoldItalicFont=cmunbi.ttf]{cmunrm.ttf}\setmonofont[Path=/usr/share/fonts/truetype/cmu/,UprightFont=cmuntt.ttf,BoldFont=cmuntb.ttf,ItalicFont=cmunit.ttf,BoldItalicFont=cmuntx.ttf]{cmunrm.ttf} part, (see \mylref{774}{Introducing Procedural Graphics}) or import productions from \mylref{355}{third party tools}, which is what we will be discussing here.

Strictly speaking, LaTeX cannot manage pictures directly: in order to introduce graphics within documents, LaTeX just creates a box with the same size as the image you want to include and embeds the picture, without any other processing. This means you will have to take care that the images you want to include are in the right format to be included. This is not such a hard task because LaTeX supports the most common picture formats around.
\section{Raster graphics vs. vector graphics}
\label{337}

Raster graphics will highly contrast with the quality of the document if they are not in a high resolution, which is the case with most graphics. The result may be even worse once printed.

Most drawing tools (e.g. for diagrams) can export in vector format. So you should always prefer PDF or EPS to PNG or JPG.
\section{The {\itshape \setmainfont[Path=/usr/share/fonts/truetype/cmu/,UprightFont=cmunrm.ttf,BoldFont=cmunbx.ttf,ItalicFont=cmunti.ttf,BoldItalicFont=cmunbi.ttf]{cmunti.ttf}\setmonofont[Path=/usr/share/fonts/truetype/cmu/,UprightFont=cmuntt.ttf,BoldFont=cmuntb.ttf,ItalicFont=cmunit.ttf,BoldItalicFont=cmuntx.ttf]{cmunti.ttf}\itshape graphicx}{$\text{ }$}\setmainfont[Path=/usr/share/fonts/truetype/cmu/,UprightFont=cmunrm.ttf,BoldFont=cmunbx.ttf,ItalicFont=cmunti.ttf,BoldItalicFont=cmunbi.ttf]{cmunrm.ttf}\setmonofont[Path=/usr/share/fonts/truetype/cmu/,UprightFont=cmuntt.ttf,BoldFont=cmuntb.ttf,ItalicFont=cmunit.ttf,BoldItalicFont=cmuntx.ttf]{cmunrm.ttf} package}
\label{338}

As stated before, LaTeX can\textquotesingle{}t manage pictures directly, so we will need some extra help: we have to load the \LaTeXTT{graphicx} package\myplainurl{http://ctan.org/pkg/graphicx/} in the preamble of our document:

\begin{Shaded}
\begin{Highlighting}[]

\NormalTok{\textbackslash{}usepackage\{graphicx\}}
\end{Highlighting}
\end{Shaded}


This package accepts as an argument the external driver to be used to manage pictures; however, the latest version of this package takes care of everything by itself, changing the driver according to the compiler you are using, so you don\textquotesingle{}t have to worry about this. Still, just in case you want to understand better how it works, here are the possible options you can pass to the package:
\begin{myitemize}
\item{}  \LaTeXTT{dvips} (default if compiling with {\ttfamily \setmainfont[Path=/usr/share/fonts/truetype/cmu/,UprightFont=cmunrm.ttf,BoldFont=cmunbx.ttf,ItalicFont=cmunti.ttf,BoldItalicFont=cmunbi.ttf]{cmuntt.ttf}\setmonofont[Path=/usr/share/fonts/truetype/cmu/,UprightFont=cmuntt.ttf,BoldFont=cmuntb.ttf,ItalicFont=cmunit.ttf,BoldItalicFont=cmuntx.ttf]{cmuntt.ttf}\ttfamily latex}\setmainfont[Path=/usr/share/fonts/truetype/cmu/,UprightFont=cmunrm.ttf,BoldFont=cmunbx.ttf,ItalicFont=cmunti.ttf,BoldItalicFont=cmunbi.ttf]{cmunrm.ttf}\setmonofont[Path=/usr/share/fonts/truetype/cmu/,UprightFont=cmuntt.ttf,BoldFont=cmuntb.ttf,ItalicFont=cmunit.ttf,BoldItalicFont=cmuntx.ttf]{cmunrm.ttf}), if you are compiling with {\ttfamily \setmainfont[Path=/usr/share/fonts/truetype/cmu/,UprightFont=cmunrm.ttf,BoldFont=cmunbx.ttf,ItalicFont=cmunti.ttf,BoldItalicFont=cmunbi.ttf]{cmuntt.ttf}\setmonofont[Path=/usr/share/fonts/truetype/cmu/,UprightFont=cmuntt.ttf,BoldFont=cmuntb.ttf,ItalicFont=cmunit.ttf,BoldItalicFont=cmuntx.ttf]{cmuntt.ttf}\ttfamily latex}{$\text{ }$}\setmainfont[Path=/usr/share/fonts/truetype/cmu/,UprightFont=cmunrm.ttf,BoldFont=cmunbx.ttf,ItalicFont=cmunti.ttf,BoldItalicFont=cmunbi.ttf]{cmunrm.ttf}\setmonofont[Path=/usr/share/fonts/truetype/cmu/,UprightFont=cmuntt.ttf,BoldFont=cmuntb.ttf,ItalicFont=cmunit.ttf,BoldItalicFont=cmuntx.ttf]{cmunrm.ttf} to get a DVI and you want to see your document with a DVI or PS viewer.
\item{}  \LaTeXTT{dvipdfm}, if you are compiling with {\ttfamily \setmainfont[Path=/usr/share/fonts/truetype/cmu/,UprightFont=cmunrm.ttf,BoldFont=cmunbx.ttf,ItalicFont=cmunti.ttf,BoldItalicFont=cmunbi.ttf]{cmuntt.ttf}\setmonofont[Path=/usr/share/fonts/truetype/cmu/,UprightFont=cmuntt.ttf,BoldFont=cmuntb.ttf,ItalicFont=cmunit.ttf,BoldItalicFont=cmuntx.ttf]{cmuntt.ttf}\ttfamily latex}{$\text{ }$}\setmainfont[Path=/usr/share/fonts/truetype/cmu/,UprightFont=cmunrm.ttf,BoldFont=cmunbx.ttf,ItalicFont=cmunti.ttf,BoldItalicFont=cmunbi.ttf]{cmunrm.ttf}\setmonofont[Path=/usr/share/fonts/truetype/cmu/,UprightFont=cmuntt.ttf,BoldFont=cmuntb.ttf,ItalicFont=cmunit.ttf,BoldItalicFont=cmuntx.ttf]{cmunrm.ttf} to get a DVI that you want to convert to PDF using {\itshape \setmainfont[Path=/usr/share/fonts/truetype/cmu/,UprightFont=cmunrm.ttf,BoldFont=cmunbx.ttf,ItalicFont=cmunti.ttf,BoldItalicFont=cmunbi.ttf]{cmunti.ttf}\setmonofont[Path=/usr/share/fonts/truetype/cmu/,UprightFont=cmuntt.ttf,BoldFont=cmuntb.ttf,ItalicFont=cmunit.ttf,BoldItalicFont=cmuntx.ttf]{cmunti.ttf}\itshape dvipdfm}\setmainfont[Path=/usr/share/fonts/truetype/cmu/,UprightFont=cmunrm.ttf,BoldFont=cmunbx.ttf,ItalicFont=cmunti.ttf,BoldItalicFont=cmunbi.ttf]{cmunrm.ttf}\setmonofont[Path=/usr/share/fonts/truetype/cmu/,UprightFont=cmuntt.ttf,BoldFont=cmuntb.ttf,ItalicFont=cmunit.ttf,BoldItalicFont=cmuntx.ttf]{cmunrm.ttf}, to see your document with any PDF viewer.
\item{}  \LaTeXTT{pdftex} (default if compiling with {\ttfamily \setmainfont[Path=/usr/share/fonts/truetype/cmu/,UprightFont=cmunrm.ttf,BoldFont=cmunbx.ttf,ItalicFont=cmunti.ttf,BoldItalicFont=cmunbi.ttf]{cmuntt.ttf}\setmonofont[Path=/usr/share/fonts/truetype/cmu/,UprightFont=cmuntt.ttf,BoldFont=cmuntb.ttf,ItalicFont=cmunit.ttf,BoldItalicFont=cmuntx.ttf]{cmuntt.ttf}\ttfamily pdflatex}\setmainfont[Path=/usr/share/fonts/truetype/cmu/,UprightFont=cmunrm.ttf,BoldFont=cmunbx.ttf,ItalicFont=cmunti.ttf,BoldItalicFont=cmunbi.ttf]{cmunrm.ttf}\setmonofont[Path=/usr/share/fonts/truetype/cmu/,UprightFont=cmuntt.ttf,BoldFont=cmuntb.ttf,ItalicFont=cmunit.ttf,BoldItalicFont=cmuntx.ttf]{cmunrm.ttf}), if you are compiling with {\ttfamily \setmainfont[Path=/usr/share/fonts/truetype/cmu/,UprightFont=cmunrm.ttf,BoldFont=cmunbx.ttf,ItalicFont=cmunti.ttf,BoldItalicFont=cmunbi.ttf]{cmuntt.ttf}\setmonofont[Path=/usr/share/fonts/truetype/cmu/,UprightFont=cmuntt.ttf,BoldFont=cmuntb.ttf,ItalicFont=cmunit.ttf,BoldItalicFont=cmuntx.ttf]{cmuntt.ttf}\ttfamily pdftex}{$\text{ }$}\setmainfont[Path=/usr/share/fonts/truetype/cmu/,UprightFont=cmunrm.ttf,BoldFont=cmunbx.ttf,ItalicFont=cmunti.ttf,BoldItalicFont=cmunbi.ttf]{cmunrm.ttf}\setmonofont[Path=/usr/share/fonts/truetype/cmu/,UprightFont=cmuntt.ttf,BoldFont=cmuntb.ttf,ItalicFont=cmunit.ttf,BoldItalicFont=cmuntx.ttf]{cmunrm.ttf} to get a PDF that you will see with any PDF viewer.
\end{myitemize}

But, again, you don\textquotesingle{}t need to pass any option to the package because the default settings are fine in most of the cases.

In many respects, importing your images into your document using LaTeX is fairly simple... {\itshape \setmainfont[Path=/usr/share/fonts/truetype/cmu/,UprightFont=cmunrm.ttf,BoldFont=cmunbx.ttf,ItalicFont=cmunti.ttf,BoldItalicFont=cmunbi.ttf]{cmunti.ttf}\setmonofont[Path=/usr/share/fonts/truetype/cmu/,UprightFont=cmuntt.ttf,BoldFont=cmuntb.ttf,ItalicFont=cmunit.ttf,BoldItalicFont=cmuntx.ttf]{cmunti.ttf}\itshape once}{$\text{ }$}\setmainfont[Path=/usr/share/fonts/truetype/cmu/,UprightFont=cmunrm.ttf,BoldFont=cmunbx.ttf,ItalicFont=cmunti.ttf,BoldItalicFont=cmunbi.ttf]{cmunrm.ttf}\setmonofont[Path=/usr/share/fonts/truetype/cmu/,UprightFont=cmuntt.ttf,BoldFont=cmuntb.ttf,ItalicFont=cmunit.ttf,BoldItalicFont=cmuntx.ttf]{cmunrm.ttf} you have your images in the right format that is! Therefore, I fear for many people the biggest effort will be the process of converting their graphics files. Now we will see which formats we can include and then we will see how to do it.
\section{Document Options}
\label{339}

The graphics and graphicx packages recognize the \LaTeXTT{draft} and \LaTeXTT{final} options given in the \LaTeXTT{\textbackslash{}documentclass{$\text{[}$}...{$\text{]}$}\{...\}} command at the start of the file. (See \mylref{92}{Document Classes}.) Using \LaTeXTT{draft} as the option will suppress the inclusion of the image in the output file and will replace the contents with the name of the image file that would have been seen. Using \LaTeXTT{final} will result in the image being placed in the output file. The default is \LaTeXTT{final}.
\section{Supported image formats}
\label{340}

As explained before, the image formats you can use depend on the driver that \LaTeXTT{graphicx} is using but, since the driver is automatically chosen according to the compiler, then the allowed image formats will depend on the compiler you are using.

\begin{TemplateInfo}{\danger}{Warning}Using {\ttfamily \setmainfont[Path=/usr/share/fonts/truetype/cmu/,UprightFont=cmunrm.ttf,BoldFont=cmunbx.ttf,ItalicFont=cmunti.ttf,BoldItalicFont=cmunbi.ttf]{cmuntt.ttf}\setmonofont[Path=/usr/share/fonts/truetype/cmu/,UprightFont=cmuntt.ttf,BoldFont=cmuntb.ttf,ItalicFont=cmunit.ttf,BoldItalicFont=cmuntx.ttf]{cmuntt.ttf}\ttfamily pdflatex}{$\text{ }$}\setmainfont[Path=/usr/share/fonts/truetype/cmu/,UprightFont=cmunrm.ttf,BoldFont=cmunbx.ttf,ItalicFont=cmunti.ttf,BoldItalicFont=cmunbi.ttf]{cmunrm.ttf}\setmonofont[Path=/usr/share/fonts/truetype/cmu/,UprightFont=cmuntt.ttf,BoldFont=cmuntb.ttf,ItalicFont=cmunit.ttf,BoldItalicFont=cmuntx.ttf]{cmunrm.ttf} will be usually much more simple for graphics inclusion as it supports widespread formats such as PDF, PNG and JPG. Read this chapter carefully if you are using the DVI compiler ({\ttfamily \setmainfont[Path=/usr/share/fonts/truetype/cmu/,UprightFont=cmunrm.ttf,BoldFont=cmunbx.ttf,ItalicFont=cmunti.ttf,BoldItalicFont=cmunbi.ttf]{cmuntt.ttf}\setmonofont[Path=/usr/share/fonts/truetype/cmu/,UprightFont=cmuntt.ttf,BoldFont=cmuntb.ttf,ItalicFont=cmunit.ttf,BoldItalicFont=cmuntx.ttf]{cmuntt.ttf}\ttfamily latex}\setmainfont[Path=/usr/share/fonts/truetype/cmu/,UprightFont=cmunrm.ttf,BoldFont=cmunbx.ttf,ItalicFont=cmunti.ttf,BoldItalicFont=cmunbi.ttf]{cmunrm.ttf}\setmonofont[Path=/usr/share/fonts/truetype/cmu/,UprightFont=cmuntt.ttf,BoldFont=cmuntb.ttf,ItalicFont=cmunit.ttf,BoldItalicFont=cmuntx.ttf]{cmunrm.ttf}), otherwise you might encounter a lot of errors at compile time.\end{TemplateInfo}

Consider the following situation: you have added some pictures to your document in JPG and you have successfully compiled it in PDF. Now you want to compile it in DVI, you run {\ttfamily \setmainfont[Path=/usr/share/fonts/truetype/cmu/,UprightFont=cmunrm.ttf,BoldFont=cmunbx.ttf,ItalicFont=cmunti.ttf,BoldItalicFont=cmunbi.ttf]{cmuntt.ttf}\setmonofont[Path=/usr/share/fonts/truetype/cmu/,UprightFont=cmuntt.ttf,BoldFont=cmuntb.ttf,ItalicFont=cmunit.ttf,BoldItalicFont=cmuntx.ttf]{cmuntt.ttf}\ttfamily latex}{$\text{ }$}\setmainfont[Path=/usr/share/fonts/truetype/cmu/,UprightFont=cmunrm.ttf,BoldFont=cmunbx.ttf,ItalicFont=cmunti.ttf,BoldItalicFont=cmunbi.ttf]{cmunrm.ttf}\setmonofont[Path=/usr/share/fonts/truetype/cmu/,UprightFont=cmuntt.ttf,BoldFont=cmuntb.ttf,ItalicFont=cmunit.ttf,BoldItalicFont=cmuntx.ttf]{cmunrm.ttf} and you get a lot of errors... because you forgot to provide the EPS versions of the pictures you want to insert.

At the beginning of this book, we had stated that the same LaTeX source can be compiled in both DVI and PDF without any change. This is true, as long as you don\textquotesingle{}t use particular packages, and \LaTeXTT{graphicx} is one of those. In any case, you can still use both compilers with documents with pictures as well, as long as you always remember to provide the pictures in two formats (EPS and one of JPG, PNG or PDF).
\subsection{Compiling with {\itshape \setmainfont[Path=/usr/share/fonts/truetype/cmu/,UprightFont=cmunrm.ttf,BoldFont=cmunbx.ttf,ItalicFont=cmunti.ttf,BoldItalicFont=cmunbi.ttf]{cmunti.ttf}\setmonofont[Path=/usr/share/fonts/truetype/cmu/,UprightFont=cmuntt.ttf,BoldFont=cmuntb.ttf,ItalicFont=cmunit.ttf,BoldItalicFont=cmuntx.ttf]{cmunti.ttf}\itshape latex}}
\label{341}\setmainfont[Path=/usr/share/fonts/truetype/cmu/,UprightFont=cmunrm.ttf,BoldFont=cmunbx.ttf,ItalicFont=cmunti.ttf,BoldItalicFont=cmunbi.ttf]{cmunrm.ttf}\setmonofont[Path=/usr/share/fonts/truetype/cmu/,UprightFont=cmuntt.ttf,BoldFont=cmuntb.ttf,ItalicFont=cmunit.ttf,BoldItalicFont=cmuntx.ttf]{cmunrm.ttf}

The only format you can include while compiling with {\ttfamily \setmainfont[Path=/usr/share/fonts/truetype/cmu/,UprightFont=cmunrm.ttf,BoldFont=cmunbx.ttf,ItalicFont=cmunti.ttf,BoldItalicFont=cmunbi.ttf]{cmuntt.ttf}\setmonofont[Path=/usr/share/fonts/truetype/cmu/,UprightFont=cmuntt.ttf,BoldFont=cmuntb.ttf,ItalicFont=cmunit.ttf,BoldItalicFont=cmuntx.ttf]{cmuntt.ttf}\ttfamily latex}{$\text{ }$}\setmainfont[Path=/usr/share/fonts/truetype/cmu/,UprightFont=cmunrm.ttf,BoldFont=cmunbx.ttf,ItalicFont=cmunti.ttf,BoldItalicFont=cmunbi.ttf]{cmunrm.ttf}\setmonofont[Path=/usr/share/fonts/truetype/cmu/,UprightFont=cmuntt.ttf,BoldFont=cmuntb.ttf,ItalicFont=cmunit.ttf,BoldItalicFont=cmuntx.ttf]{cmunrm.ttf} is \myhref{https://en.wikipedia.org/wiki/Encapsulated\%20PostScript}{Encapsulated PostScript} ({\bfseries \setmainfont[Path=/usr/share/fonts/truetype/cmu/,UprightFont=cmunrm.ttf,BoldFont=cmunbx.ttf,ItalicFont=cmunti.ttf,BoldItalicFont=cmunbi.ttf]{cmunbx.ttf}\setmonofont[Path=/usr/share/fonts/truetype/cmu/,UprightFont=cmuntt.ttf,BoldFont=cmuntb.ttf,ItalicFont=cmunit.ttf,BoldItalicFont=cmuntx.ttf]{cmunbx.ttf}\bfseries EPS}\setmainfont[Path=/usr/share/fonts/truetype/cmu/,UprightFont=cmunrm.ttf,BoldFont=cmunbx.ttf,ItalicFont=cmunti.ttf,BoldItalicFont=cmunbi.ttf]{cmunrm.ttf}\setmonofont[Path=/usr/share/fonts/truetype/cmu/,UprightFont=cmuntt.ttf,BoldFont=cmuntb.ttf,ItalicFont=cmunit.ttf,BoldItalicFont=cmuntx.ttf]{cmunrm.ttf}).

The EPS format was defined by Adobe Systems for making it easy for applications to import postscript-{}based graphics into documents. Because an EPS file declares the size of the image, it makes it easy for systems like LaTeX to arrange the text and the graphics in the best way. EPS is a \myhref{https://en.wikipedia.org/wiki/Vector\%20graphics}{vector format}—this means that it can have very high quality if it is created properly, with programs that are able to manage vector graphics. It is also possible to store bit-{}map pictures within EPS, but they will need {\itshape \setmainfont[Path=/usr/share/fonts/truetype/cmu/,UprightFont=cmunrm.ttf,BoldFont=cmunbx.ttf,ItalicFont=cmunti.ttf,BoldItalicFont=cmunbi.ttf]{cmunti.ttf}\setmonofont[Path=/usr/share/fonts/truetype/cmu/,UprightFont=cmuntt.ttf,BoldFont=cmuntb.ttf,ItalicFont=cmunit.ttf,BoldItalicFont=cmuntx.ttf]{cmunti.ttf}\itshape a lot}{$\text{ }$}\setmainfont[Path=/usr/share/fonts/truetype/cmu/,UprightFont=cmunrm.ttf,BoldFont=cmunbx.ttf,ItalicFont=cmunti.ttf,BoldItalicFont=cmunbi.ttf]{cmunrm.ttf}\setmonofont[Path=/usr/share/fonts/truetype/cmu/,UprightFont=cmuntt.ttf,BoldFont=cmuntb.ttf,ItalicFont=cmunit.ttf,BoldItalicFont=cmuntx.ttf]{cmunrm.ttf} of disk space.
\subsection{Compiling with {\itshape \setmainfont[Path=/usr/share/fonts/truetype/cmu/,UprightFont=cmunrm.ttf,BoldFont=cmunbx.ttf,ItalicFont=cmunti.ttf,BoldItalicFont=cmunbi.ttf]{cmunti.ttf}\setmonofont[Path=/usr/share/fonts/truetype/cmu/,UprightFont=cmuntt.ttf,BoldFont=cmuntb.ttf,ItalicFont=cmunit.ttf,BoldItalicFont=cmuntx.ttf]{cmunti.ttf}\itshape pdflatex}}
\label{342}\setmainfont[Path=/usr/share/fonts/truetype/cmu/,UprightFont=cmunrm.ttf,BoldFont=cmunbx.ttf,ItalicFont=cmunti.ttf,BoldItalicFont=cmunbi.ttf]{cmunrm.ttf}\setmonofont[Path=/usr/share/fonts/truetype/cmu/,UprightFont=cmuntt.ttf,BoldFont=cmuntb.ttf,ItalicFont=cmunit.ttf,BoldItalicFont=cmuntx.ttf]{cmunrm.ttf}

If you are compiling with {\ttfamily \setmainfont[Path=/usr/share/fonts/truetype/cmu/,UprightFont=cmunrm.ttf,BoldFont=cmunbx.ttf,ItalicFont=cmunti.ttf,BoldItalicFont=cmunbi.ttf]{cmuntt.ttf}\setmonofont[Path=/usr/share/fonts/truetype/cmu/,UprightFont=cmuntt.ttf,BoldFont=cmuntb.ttf,ItalicFont=cmunit.ttf,BoldItalicFont=cmuntx.ttf]{cmuntt.ttf}\ttfamily pdflatex}{$\text{ }$}\setmainfont[Path=/usr/share/fonts/truetype/cmu/,UprightFont=cmunrm.ttf,BoldFont=cmunbx.ttf,ItalicFont=cmunti.ttf,BoldItalicFont=cmunbi.ttf]{cmunrm.ttf}\setmonofont[Path=/usr/share/fonts/truetype/cmu/,UprightFont=cmuntt.ttf,BoldFont=cmuntb.ttf,ItalicFont=cmunit.ttf,BoldItalicFont=cmuntx.ttf]{cmunrm.ttf} to produce a PDF, you have a wider choice. You can insert:
\begin{myitemize}
\item{}  {\bfseries \setmainfont[Path=/usr/share/fonts/truetype/cmu/,UprightFont=cmunrm.ttf,BoldFont=cmunbx.ttf,ItalicFont=cmunti.ttf,BoldItalicFont=cmunbi.ttf]{cmunbx.ttf}\setmonofont[Path=/usr/share/fonts/truetype/cmu/,UprightFont=cmuntt.ttf,BoldFont=cmuntb.ttf,ItalicFont=cmunit.ttf,BoldItalicFont=cmuntx.ttf]{cmunbx.ttf}\bfseries JPG}\setmainfont[Path=/usr/share/fonts/truetype/cmu/,UprightFont=cmunrm.ttf,BoldFont=cmunbx.ttf,ItalicFont=cmunti.ttf,BoldItalicFont=cmunbi.ttf]{cmunrm.ttf}\setmonofont[Path=/usr/share/fonts/truetype/cmu/,UprightFont=cmuntt.ttf,BoldFont=cmuntb.ttf,ItalicFont=cmunit.ttf,BoldItalicFont=cmuntx.ttf]{cmunrm.ttf}, widely used on Internet, digital cameras, etc. They are the best choice if you want to insert photos.
\item{}  {\bfseries \setmainfont[Path=/usr/share/fonts/truetype/cmu/,UprightFont=cmunrm.ttf,BoldFont=cmunbx.ttf,ItalicFont=cmunti.ttf,BoldItalicFont=cmunbi.ttf]{cmunbx.ttf}\setmonofont[Path=/usr/share/fonts/truetype/cmu/,UprightFont=cmuntt.ttf,BoldFont=cmuntb.ttf,ItalicFont=cmunit.ttf,BoldItalicFont=cmuntx.ttf]{cmunbx.ttf}\bfseries PNG}\setmainfont[Path=/usr/share/fonts/truetype/cmu/,UprightFont=cmunrm.ttf,BoldFont=cmunbx.ttf,ItalicFont=cmunti.ttf,BoldItalicFont=cmunbi.ttf]{cmunrm.ttf}\setmonofont[Path=/usr/share/fonts/truetype/cmu/,UprightFont=cmuntt.ttf,BoldFont=cmuntb.ttf,ItalicFont=cmunit.ttf,BoldItalicFont=cmuntx.ttf]{cmunrm.ttf}, a very common format (even if not as much as JPG); it\textquotesingle{}s a \myhref{https://en.wikipedia.org/wiki/lossless}{lossless} format and it\textquotesingle{}s the best choice for diagrams (if you were not able to generate a \myhref{https://en.wikipedia.org/wiki/Vector\%20graphics}{vector} version) and screenshots.
\item{}  {\bfseries \setmainfont[Path=/usr/share/fonts/truetype/cmu/,UprightFont=cmunrm.ttf,BoldFont=cmunbx.ttf,ItalicFont=cmunti.ttf,BoldItalicFont=cmunbi.ttf]{cmunbx.ttf}\setmonofont[Path=/usr/share/fonts/truetype/cmu/,UprightFont=cmuntt.ttf,BoldFont=cmuntb.ttf,ItalicFont=cmunit.ttf,BoldItalicFont=cmuntx.ttf]{cmunbx.ttf}\bfseries PDF}\setmainfont[Path=/usr/share/fonts/truetype/cmu/,UprightFont=cmunrm.ttf,BoldFont=cmunbx.ttf,ItalicFont=cmunti.ttf,BoldItalicFont=cmunbi.ttf]{cmunrm.ttf}\setmonofont[Path=/usr/share/fonts/truetype/cmu/,UprightFont=cmuntt.ttf,BoldFont=cmuntb.ttf,ItalicFont=cmunit.ttf,BoldItalicFont=cmuntx.ttf]{cmunrm.ttf}, it is widely used for documents but can be used to store images as well. It supports both vector and \myhref{https://en.wikipedia.org/wiki/Raster\%20graphics}{bit-{}map} images, but it\textquotesingle{}s not recommended for the latter, as JPG or PNG will provide the same result using less disk space.
\item{}  {\bfseries \setmainfont[Path=/usr/share/fonts/truetype/cmu/,UprightFont=cmunrm.ttf,BoldFont=cmunbx.ttf,ItalicFont=cmunti.ttf,BoldItalicFont=cmunbi.ttf]{cmunbx.ttf}\setmonofont[Path=/usr/share/fonts/truetype/cmu/,UprightFont=cmuntt.ttf,BoldFont=cmuntb.ttf,ItalicFont=cmunit.ttf,BoldItalicFont=cmuntx.ttf]{cmunbx.ttf}\bfseries EPS}{$\text{ }$}\setmainfont[Path=/usr/share/fonts/truetype/cmu/,UprightFont=cmunrm.ttf,BoldFont=cmunbx.ttf,ItalicFont=cmunti.ttf,BoldItalicFont=cmunbi.ttf]{cmunrm.ttf}\setmonofont[Path=/usr/share/fonts/truetype/cmu/,UprightFont=cmuntt.ttf,BoldFont=cmuntb.ttf,ItalicFont=cmunit.ttf,BoldItalicFont=cmuntx.ttf]{cmunrm.ttf} can be used with the help of the \LaTeXTT{epstopdf} package. Depending on your installation,
\begin{myitemize}
\item{}  you may just need to have it installed, there is no need to load it in your document;
\item{}  if it does not work, you need to load it just after the \LaTeXTT{graphicx} package. Additionally, since \LaTeXTT{epstopdf} will need to convert the EPS file into a PDF file and store it, you need to give \symbol{34}writing permissions\symbol{34} to your compiler. This is done by adding an option to the compiling command, {\itshape \setmainfont[Path=/usr/share/fonts/truetype/cmu/,UprightFont=cmunrm.ttf,BoldFont=cmunbx.ttf,ItalicFont=cmunti.ttf,BoldItalicFont=cmunbi.ttf]{cmunti.ttf}\setmonofont[Path=/usr/share/fonts/truetype/cmu/,UprightFont=cmuntt.ttf,BoldFont=cmuntb.ttf,ItalicFont=cmunit.ttf,BoldItalicFont=cmuntx.ttf]{cmunti.ttf}\itshape e.g.}{$\text{ }$}\setmainfont[Path=/usr/share/fonts/truetype/cmu/,UprightFont=cmunrm.ttf,BoldFont=cmunbx.ttf,ItalicFont=cmunti.ttf,BoldItalicFont=cmunbi.ttf]{cmunrm.ttf}\setmonofont[Path=/usr/share/fonts/truetype/cmu/,UprightFont=cmuntt.ttf,BoldFont=cmuntb.ttf,ItalicFont=cmunit.ttf,BoldItalicFont=cmuntx.ttf]{cmunrm.ttf} {\ttfamily \setmainfont[Path=/usr/share/fonts/truetype/cmu/,UprightFont=cmunrm.ttf,BoldFont=cmunbx.ttf,ItalicFont=cmunti.ttf,BoldItalicFont=cmunbi.ttf]{cmuntt.ttf}\setmonofont[Path=/usr/share/fonts/truetype/cmu/,UprightFont=cmuntt.ttf,BoldFont=cmuntb.ttf,ItalicFont=cmunit.ttf,BoldItalicFont=cmuntx.ttf]{cmuntt.ttf}\ttfamily pdflatex -{}shell-{}escape file.tex}{$\text{ }$}\setmainfont[Path=/usr/share/fonts/truetype/cmu/,UprightFont=cmunrm.ttf,BoldFont=cmunbx.ttf,ItalicFont=cmunti.ttf,BoldItalicFont=cmunbi.ttf]{cmunrm.ttf}\setmonofont[Path=/usr/share/fonts/truetype/cmu/,UprightFont=cmuntt.ttf,BoldFont=cmuntb.ttf,ItalicFont=cmunit.ttf,BoldItalicFont=cmuntx.ttf]{cmunrm.ttf} (if you use a LaTeX editor, they usually allow to modify the command in the configuration options). Check the epstopdf documentation for other compilers.
\end{myitemize}

\end{myitemize}

\section{Including graphics}
\label{343}

Now that we have seen which formats we can include and how we could manage those formats, it\textquotesingle{}s time to learn how to include them in our document.
After you have loaded the \LaTeXTT{graphicx} package in your preamble, you can include images with \LaTeXTT{\textbackslash{}includegraphics}, whose syntax is the following:

\begin{Shaded}
\begin{Highlighting}[]

\NormalTok{\textbackslash{}includegraphics[attr1=val1, attr2=val2, ..., attrn=valn]\{imagename\}}
\end{Highlighting}
\end{Shaded}


As usual, arguments in square brackets are optional, whereas arguments in curly braces are compulsory.

The argument in the curly braces is the name of the image. Write it {\itshape \setmainfont[Path=/usr/share/fonts/truetype/cmu/,UprightFont=cmunrm.ttf,BoldFont=cmunbx.ttf,ItalicFont=cmunti.ttf,BoldItalicFont=cmunbi.ttf]{cmunti.ttf}\setmonofont[Path=/usr/share/fonts/truetype/cmu/,UprightFont=cmuntt.ttf,BoldFont=cmuntb.ttf,ItalicFont=cmunit.ttf,BoldItalicFont=cmuntx.ttf]{cmunti.ttf}\itshape without}{$\text{ }$}\setmainfont[Path=/usr/share/fonts/truetype/cmu/,UprightFont=cmunrm.ttf,BoldFont=cmunbx.ttf,ItalicFont=cmunti.ttf,BoldItalicFont=cmunbi.ttf]{cmunrm.ttf}\setmonofont[Path=/usr/share/fonts/truetype/cmu/,UprightFont=cmuntt.ttf,BoldFont=cmuntb.ttf,ItalicFont=cmunit.ttf,BoldItalicFont=cmuntx.ttf]{cmunrm.ttf} the extension. This way the LaTeX compiler will look for any supported image format in that directory and will take the best one (EPS if the output is DVI; JPEG, PNG or PDF if the output is PDF). Images can be saved in multiple formats for different purposes. For example, a directory can have \symbol{34}{\ttfamily \setmainfont[Path=/usr/share/fonts/truetype/cmu/,UprightFont=cmunrm.ttf,BoldFont=cmunbx.ttf,ItalicFont=cmunti.ttf,BoldItalicFont=cmunbi.ttf]{cmuntt.ttf}\setmonofont[Path=/usr/share/fonts/truetype/cmu/,UprightFont=cmuntt.ttf,BoldFont=cmuntb.ttf,ItalicFont=cmunit.ttf,BoldItalicFont=cmuntx.ttf]{cmuntt.ttf}\ttfamily diagram.pdf}\setmainfont[Path=/usr/share/fonts/truetype/cmu/,UprightFont=cmunrm.ttf,BoldFont=cmunbx.ttf,ItalicFont=cmunti.ttf,BoldItalicFont=cmunbi.ttf]{cmunrm.ttf}\setmonofont[Path=/usr/share/fonts/truetype/cmu/,UprightFont=cmuntt.ttf,BoldFont=cmuntb.ttf,ItalicFont=cmunit.ttf,BoldItalicFont=cmuntx.ttf]{cmunrm.ttf}\symbol{34} for high-{}resolution printing, while \symbol{34}{\ttfamily \setmainfont[Path=/usr/share/fonts/truetype/cmu/,UprightFont=cmunrm.ttf,BoldFont=cmunbx.ttf,ItalicFont=cmunti.ttf,BoldItalicFont=cmunbi.ttf]{cmuntt.ttf}\setmonofont[Path=/usr/share/fonts/truetype/cmu/,UprightFont=cmuntt.ttf,BoldFont=cmuntb.ttf,ItalicFont=cmunit.ttf,BoldItalicFont=cmuntx.ttf]{cmuntt.ttf}\ttfamily diagram.png}\setmainfont[Path=/usr/share/fonts/truetype/cmu/,UprightFont=cmunrm.ttf,BoldFont=cmunbx.ttf,ItalicFont=cmunti.ttf,BoldItalicFont=cmunbi.ttf]{cmunrm.ttf}\setmonofont[Path=/usr/share/fonts/truetype/cmu/,UprightFont=cmuntt.ttf,BoldFont=cmuntb.ttf,ItalicFont=cmunit.ttf,BoldItalicFont=cmuntx.ttf]{cmunrm.ttf}\symbol{34} can be used for previewing on the monitor. You can specify which image file is to be used by {\ttfamily \setmainfont[Path=/usr/share/fonts/truetype/cmu/,UprightFont=cmunrm.ttf,BoldFont=cmunbx.ttf,ItalicFont=cmunti.ttf,BoldItalicFont=cmunbi.ttf]{cmuntt.ttf}\setmonofont[Path=/usr/share/fonts/truetype/cmu/,UprightFont=cmuntt.ttf,BoldFont=cmuntb.ttf,ItalicFont=cmunit.ttf,BoldItalicFont=cmuntx.ttf]{cmuntt.ttf}\ttfamily pdflatex}{$\text{ }$}\setmainfont[Path=/usr/share/fonts/truetype/cmu/,UprightFont=cmunrm.ttf,BoldFont=cmunbx.ttf,ItalicFont=cmunti.ttf,BoldItalicFont=cmunbi.ttf]{cmunrm.ttf}\setmonofont[Path=/usr/share/fonts/truetype/cmu/,UprightFont=cmuntt.ttf,BoldFont=cmuntb.ttf,ItalicFont=cmunit.ttf,BoldItalicFont=cmuntx.ttf]{cmunrm.ttf} through the preamble command:
\begin{Shaded}
\begin{Highlighting}[]

\NormalTok{\textbackslash{}DeclareGraphicsExtensions\{.pdf,.png,.jpg\}}
\end{Highlighting}
\end{Shaded}

which specifies the files to include in the document (in order of preference), if files with the same basename exist, but with different extensions.

The variety of possible attributes that can be set is fairly large, so only the most common are covered below:

\begin{longtable}{|>{\RaggedRight}p{0.24346\linewidth}|>{\RaggedRight}p{0.33542\linewidth}|>{\RaggedRight}p{0.33542\linewidth}|} \hline 
\hspace*{0pt}\ignorespaces{}\hspace*{0pt} \LaTeXTT{width=xx}&\hspace*{0pt}\ignorespaces{}\hspace*{0pt} Specify the preferred width of the imported image to {\itshape \setmainfont[Path=/usr/share/fonts/truetype/cmu/,UprightFont=cmunrm.ttf,BoldFont=cmunbx.ttf,ItalicFont=cmunti.ttf,BoldItalicFont=cmunbi.ttf]{cmunti.ttf}\setmonofont[Path=/usr/share/fonts/truetype/cmu/,UprightFont=cmuntt.ttf,BoldFont=cmuntb.ttf,ItalicFont=cmunit.ttf,BoldItalicFont=cmuntx.ttf]{cmunti.ttf}\itshape xx}\setmainfont[Path=/usr/share/fonts/truetype/cmu/,UprightFont=cmunrm.ttf,BoldFont=cmunbx.ttf,ItalicFont=cmunti.ttf,BoldItalicFont=cmunbi.ttf]{cmunrm.ttf}\setmonofont[Path=/usr/share/fonts/truetype/cmu/,UprightFont=cmuntt.ttf,BoldFont=cmuntb.ttf,ItalicFont=cmunit.ttf,BoldItalicFont=cmuntx.ttf]{cmunrm.ttf}.&\multirow{2}{\linewidth}{\hspace*{0pt}\ignorespaces{}\hspace*{0pt} {\itshape \setmainfont[Path=/usr/share/fonts/truetype/cmu/,UprightFont=cmunrm.ttf,BoldFont=cmunbx.ttf,ItalicFont=cmunti.ttf,BoldItalicFont=cmunbi.ttf]{cmunti.ttf}\setmonofont[Path=/usr/share/fonts/truetype/cmu/,UprightFont=cmuntt.ttf,BoldFont=cmuntb.ttf,ItalicFont=cmunit.ttf,BoldItalicFont=cmuntx.ttf]{cmunti.ttf}\itshape NB. Only specifying either width or height will scale the image while maintaining the aspect ratio.}}\\ \cline{1-1}\cline{2-2} \hspace*{0pt}\ignorespaces{}\hspace*{0pt}{$\text{ }$}\setmainfont[Path=/usr/share/fonts/truetype/cmu/,UprightFont=cmunrm.ttf,BoldFont=cmunbx.ttf,ItalicFont=cmunti.ttf,BoldItalicFont=cmunbi.ttf]{cmunrm.ttf}\setmonofont[Path=/usr/share/fonts/truetype/cmu/,UprightFont=cmuntt.ttf,BoldFont=cmuntb.ttf,ItalicFont=cmunit.ttf,BoldItalicFont=cmuntx.ttf]{cmunrm.ttf} \LaTeXTT{height=xx}&\hspace*{0pt}\ignorespaces{}\hspace*{0pt} Specify the preferred height of the imported image to {\itshape \setmainfont[Path=/usr/share/fonts/truetype/cmu/,UprightFont=cmunrm.ttf,BoldFont=cmunbx.ttf,ItalicFont=cmunti.ttf,BoldItalicFont=cmunbi.ttf]{cmunti.ttf}\setmonofont[Path=/usr/share/fonts/truetype/cmu/,UprightFont=cmuntt.ttf,BoldFont=cmuntb.ttf,ItalicFont=cmunit.ttf,BoldItalicFont=cmuntx.ttf]{cmunti.ttf}\itshape xx}\setmainfont[Path=/usr/share/fonts/truetype/cmu/,UprightFont=cmunrm.ttf,BoldFont=cmunbx.ttf,ItalicFont=cmunti.ttf,BoldItalicFont=cmunbi.ttf]{cmunrm.ttf}\setmonofont[Path=/usr/share/fonts/truetype/cmu/,UprightFont=cmuntt.ttf,BoldFont=cmuntb.ttf,ItalicFont=cmunit.ttf,BoldItalicFont=cmuntx.ttf]{cmunrm.ttf}.&\multicolumn{1}{|c|}{}\\ \cline{1-1}\cline{2-2} \hspace*{0pt}\ignorespaces{}\hspace*{0pt} \LaTeXTT{keepaspectratio}&\multicolumn{2}{|>{\RaggedRight}p{0.69116\linewidth}|}{\hspace*{0pt}\ignorespaces{}\hspace*{0pt} This can be set to either {\itshape \setmainfont[Path=/usr/share/fonts/truetype/cmu/,UprightFont=cmunrm.ttf,BoldFont=cmunbx.ttf,ItalicFont=cmunti.ttf,BoldItalicFont=cmunbi.ttf]{cmunti.ttf}\setmonofont[Path=/usr/share/fonts/truetype/cmu/,UprightFont=cmuntt.ttf,BoldFont=cmuntb.ttf,ItalicFont=cmunit.ttf,BoldItalicFont=cmuntx.ttf]{cmunti.ttf}\itshape true}{$\text{ }$}\setmainfont[Path=/usr/share/fonts/truetype/cmu/,UprightFont=cmunrm.ttf,BoldFont=cmunbx.ttf,ItalicFont=cmunti.ttf,BoldItalicFont=cmunbi.ttf]{cmunrm.ttf}\setmonofont[Path=/usr/share/fonts/truetype/cmu/,UprightFont=cmuntt.ttf,BoldFont=cmuntb.ttf,ItalicFont=cmunit.ttf,BoldItalicFont=cmuntx.ttf]{cmunrm.ttf} or {\itshape \setmainfont[Path=/usr/share/fonts/truetype/cmu/,UprightFont=cmunrm.ttf,BoldFont=cmunbx.ttf,ItalicFont=cmunti.ttf,BoldItalicFont=cmunbi.ttf]{cmunti.ttf}\setmonofont[Path=/usr/share/fonts/truetype/cmu/,UprightFont=cmuntt.ttf,BoldFont=cmuntb.ttf,ItalicFont=cmunit.ttf,BoldItalicFont=cmuntx.ttf]{cmunti.ttf}\itshape false}\setmainfont[Path=/usr/share/fonts/truetype/cmu/,UprightFont=cmunrm.ttf,BoldFont=cmunbx.ttf,ItalicFont=cmunti.ttf,BoldItalicFont=cmunbi.ttf]{cmunrm.ttf}\setmonofont[Path=/usr/share/fonts/truetype/cmu/,UprightFont=cmuntt.ttf,BoldFont=cmuntb.ttf,ItalicFont=cmunit.ttf,BoldItalicFont=cmuntx.ttf]{cmunrm.ttf}. When true, it will scale the image according to both height and width, but will not distort the image, so that neither width nor height are exceeded.}\\ \hline \hspace*{0pt}\ignorespaces{}\hspace*{0pt} \LaTeXTT{scale=xx}&\multicolumn{2}{|>{\RaggedRight}p{0.69116\linewidth}|}{\hspace*{0pt}\ignorespaces{}\hspace*{0pt} Scales the image by the desired scale factor. e.g, 0.5 to reduce by half, or 2 to double.}\\ \hline \hspace*{0pt}\ignorespaces{}\hspace*{0pt} \LaTeXTT{angle=xx}&\multicolumn{2}{|>{\RaggedRight}p{0.69116\linewidth}|}{\hspace*{0pt}\ignorespaces{}\hspace*{0pt} This option can rotate the image by {\itshape \setmainfont[Path=/usr/share/fonts/truetype/cmu/,UprightFont=cmunrm.ttf,BoldFont=cmunbx.ttf,ItalicFont=cmunti.ttf,BoldItalicFont=cmunbi.ttf]{cmunti.ttf}\setmonofont[Path=/usr/share/fonts/truetype/cmu/,UprightFont=cmuntt.ttf,BoldFont=cmuntb.ttf,ItalicFont=cmunit.ttf,BoldItalicFont=cmuntx.ttf]{cmunti.ttf}\itshape xx}{$\text{ }$}\setmainfont[Path=/usr/share/fonts/truetype/cmu/,UprightFont=cmunrm.ttf,BoldFont=cmunbx.ttf,ItalicFont=cmunti.ttf,BoldItalicFont=cmunbi.ttf]{cmunrm.ttf}\setmonofont[Path=/usr/share/fonts/truetype/cmu/,UprightFont=cmuntt.ttf,BoldFont=cmuntb.ttf,ItalicFont=cmunit.ttf,BoldItalicFont=cmuntx.ttf]{cmunrm.ttf} degrees (counter-{}clockwise)}\\ \hline \hspace*{0pt}\ignorespaces{}\hspace*{0pt} \LaTeXTT{trim=l b r t}&\multicolumn{2}{|>{\RaggedRight}p{0.69116\linewidth}|}{\hspace*{0pt}\ignorespaces{}\hspace*{0pt} This option will crop the imported image by {\itshape \setmainfont[Path=/usr/share/fonts/truetype/cmu/,UprightFont=cmunrm.ttf,BoldFont=cmunbx.ttf,ItalicFont=cmunti.ttf,BoldItalicFont=cmunbi.ttf]{cmunti.ttf}\setmonofont[Path=/usr/share/fonts/truetype/cmu/,UprightFont=cmuntt.ttf,BoldFont=cmuntb.ttf,ItalicFont=cmunit.ttf,BoldItalicFont=cmuntx.ttf]{cmunti.ttf}\itshape l}{$\text{ }$}\setmainfont[Path=/usr/share/fonts/truetype/cmu/,UprightFont=cmunrm.ttf,BoldFont=cmunbx.ttf,ItalicFont=cmunti.ttf,BoldItalicFont=cmunbi.ttf]{cmunrm.ttf}\setmonofont[Path=/usr/share/fonts/truetype/cmu/,UprightFont=cmuntt.ttf,BoldFont=cmuntb.ttf,ItalicFont=cmunit.ttf,BoldItalicFont=cmuntx.ttf]{cmunrm.ttf} from the left, {\itshape \setmainfont[Path=/usr/share/fonts/truetype/cmu/,UprightFont=cmunrm.ttf,BoldFont=cmunbx.ttf,ItalicFont=cmunti.ttf,BoldItalicFont=cmunbi.ttf]{cmunti.ttf}\setmonofont[Path=/usr/share/fonts/truetype/cmu/,UprightFont=cmuntt.ttf,BoldFont=cmuntb.ttf,ItalicFont=cmunit.ttf,BoldItalicFont=cmuntx.ttf]{cmunti.ttf}\itshape b}{$\text{ }$}\setmainfont[Path=/usr/share/fonts/truetype/cmu/,UprightFont=cmunrm.ttf,BoldFont=cmunbx.ttf,ItalicFont=cmunti.ttf,BoldItalicFont=cmunbi.ttf]{cmunrm.ttf}\setmonofont[Path=/usr/share/fonts/truetype/cmu/,UprightFont=cmuntt.ttf,BoldFont=cmuntb.ttf,ItalicFont=cmunit.ttf,BoldItalicFont=cmuntx.ttf]{cmunrm.ttf} from the bottom, {\itshape \setmainfont[Path=/usr/share/fonts/truetype/cmu/,UprightFont=cmunrm.ttf,BoldFont=cmunbx.ttf,ItalicFont=cmunti.ttf,BoldItalicFont=cmunbi.ttf]{cmunti.ttf}\setmonofont[Path=/usr/share/fonts/truetype/cmu/,UprightFont=cmuntt.ttf,BoldFont=cmuntb.ttf,ItalicFont=cmunit.ttf,BoldItalicFont=cmuntx.ttf]{cmunti.ttf}\itshape r}{$\text{ }$}\setmainfont[Path=/usr/share/fonts/truetype/cmu/,UprightFont=cmunrm.ttf,BoldFont=cmunbx.ttf,ItalicFont=cmunti.ttf,BoldItalicFont=cmunbi.ttf]{cmunrm.ttf}\setmonofont[Path=/usr/share/fonts/truetype/cmu/,UprightFont=cmuntt.ttf,BoldFont=cmuntb.ttf,ItalicFont=cmunit.ttf,BoldItalicFont=cmuntx.ttf]{cmunrm.ttf} from the right, and {\itshape \setmainfont[Path=/usr/share/fonts/truetype/cmu/,UprightFont=cmunrm.ttf,BoldFont=cmunbx.ttf,ItalicFont=cmunti.ttf,BoldItalicFont=cmunbi.ttf]{cmunti.ttf}\setmonofont[Path=/usr/share/fonts/truetype/cmu/,UprightFont=cmuntt.ttf,BoldFont=cmuntb.ttf,ItalicFont=cmunit.ttf,BoldItalicFont=cmuntx.ttf]{cmunti.ttf}\itshape t}{$\text{ }$}\setmainfont[Path=/usr/share/fonts/truetype/cmu/,UprightFont=cmunrm.ttf,BoldFont=cmunbx.ttf,ItalicFont=cmunti.ttf,BoldItalicFont=cmunbi.ttf]{cmunrm.ttf}\setmonofont[Path=/usr/share/fonts/truetype/cmu/,UprightFont=cmuntt.ttf,BoldFont=cmuntb.ttf,ItalicFont=cmunit.ttf,BoldItalicFont=cmuntx.ttf]{cmunrm.ttf} from the top. Where l, b, r and t are lengths.}\\ \hline \hspace*{0pt}\ignorespaces{}\hspace*{0pt} \LaTeXTT{clip}&\multicolumn{2}{|>{\RaggedRight}p{0.69116\linewidth}|}{\hspace*{0pt}\ignorespaces{}\hspace*{0pt} For the \LaTeXTT{trim} option to work, you must set \LaTeXTT{clip=true}.}\\ \hline \hspace*{0pt}\ignorespaces{}\hspace*{0pt} \LaTeXTT{page=x}&\multicolumn{2}{|>{\RaggedRight}p{0.69116\linewidth}|}{\hspace*{0pt}\ignorespaces{}\hspace*{0pt} If the image file is a pdf file with multiple pages, this parameter allows you to use a different page than the first.}\\ \hline \hspace*{0pt}\ignorespaces{}\hspace*{0pt} \LaTeXTT{resolution=x}&\multicolumn{2}{|>{\RaggedRight}p{0.69116\linewidth}|}{\hspace*{0pt}\ignorespaces{}\hspace*{0pt} Specify image resolution in dpi}\\ \hline 
\end{longtable}


In order to use more than one option at a time, simply separate each with a comma. The order you give the options matters. E.g you should first rotate your graphic (with angle) and then specify its width. 

Included graphics will be inserted just {\itshape \setmainfont[Path=/usr/share/fonts/truetype/cmu/,UprightFont=cmunrm.ttf,BoldFont=cmunbx.ttf,ItalicFont=cmunti.ttf,BoldItalicFont=cmunbi.ttf]{cmunti.ttf}\setmonofont[Path=/usr/share/fonts/truetype/cmu/,UprightFont=cmuntt.ttf,BoldFont=cmuntb.ttf,ItalicFont=cmunit.ttf,BoldItalicFont=cmuntx.ttf]{cmunti.ttf}\itshape there}\setmainfont[Path=/usr/share/fonts/truetype/cmu/,UprightFont=cmunrm.ttf,BoldFont=cmunbx.ttf,ItalicFont=cmunti.ttf,BoldItalicFont=cmunbi.ttf]{cmunrm.ttf}\setmonofont[Path=/usr/share/fonts/truetype/cmu/,UprightFont=cmuntt.ttf,BoldFont=cmuntb.ttf,ItalicFont=cmunit.ttf,BoldItalicFont=cmuntx.ttf]{cmunrm.ttf}, where you placed the code, and the compiler will handle them as \symbol{34}big boxes\symbol{34}. As we will see in the \mylref{362}{floats section}, this can disrupt the layout; you\textquotesingle{}ll probably want to place graphics inside floating objects.

Also note that the trim option does not work with XeLaTex.

Be careful using any options, if you are working with the chemnum-{}package. The labels defined by \LaTeXTT{\textbackslash{}cmpdref\{<{}label name>{}\}} might not behave as expected. Scaling the image for instance may be done by \LaTeXTT{\textbackslash{}scalebox} instead.

The {\itshape \setmainfont[Path=/usr/share/fonts/truetype/cmu/,UprightFont=cmunrm.ttf,BoldFont=cmunbx.ttf,ItalicFont=cmunti.ttf,BoldItalicFont=cmunbi.ttf]{cmunti.ttf}\setmonofont[Path=/usr/share/fonts/truetype/cmu/,UprightFont=cmuntt.ttf,BoldFont=cmuntb.ttf,ItalicFont=cmunit.ttf,BoldItalicFont=cmuntx.ttf]{cmunti.ttf}\itshape star}{$\text{ }$}\setmainfont[Path=/usr/share/fonts/truetype/cmu/,UprightFont=cmunrm.ttf,BoldFont=cmunbx.ttf,ItalicFont=cmunti.ttf,BoldItalicFont=cmunbi.ttf]{cmunrm.ttf}\setmonofont[Path=/usr/share/fonts/truetype/cmu/,UprightFont=cmuntt.ttf,BoldFont=cmuntb.ttf,ItalicFont=cmunit.ttf,BoldItalicFont=cmuntx.ttf]{cmunrm.ttf} version of the command will work for .eps files only. For a more portable solution, the standard way should take precedence. The star command will take the crop dimension as extra parameter:
\begin{Shaded}
\begin{Highlighting}[]

\NormalTok{\textbackslash{}includegraphics*[100,100][300,300]\{mypicture\}}
\end{Highlighting}
\end{Shaded}

\subsection{Examples}
\label{344}

OK, it\textquotesingle{}s time to see graphicx in action. Here are some examples.  Say you had a file \textquotesingle{}chick.jpg\textquotesingle{} you would include it like:

\begin{Shaded}
\begin{Highlighting}[]

\NormalTok{\textbackslash{}includegraphics\{chick\}}
\end{Highlighting}
\end{Shaded}


This simply imports the image, without any other processing. However, it is very large (so we won\textquotesingle{}t give an example of how it would look here!) So, let\textquotesingle{}s scale it down:

\begin{longtable}{p{1.0\linewidth}}
\begin{Shaded}
\begin{Highlighting}[]

\NormalTok{\textbackslash{}includegraphics[scale=0.5]\{chick\}}
\end{Highlighting}
\end{Shaded}
\\



\begin{minipage}{1.0\linewidth}
\begin{center}
\includegraphics[width=1.0\linewidth,height=6.5in,keepaspectratio]{../images/61.png}
\end{center}
\raggedright{}\myfigurewithoutcaption{61}
\end{minipage}\vspace{0.75cm}



\end{longtable}

This has now scaled it by half. If you wish to be more specific and give actual lengths of the image dimensions, this is how to go about it:

\begin{longtable}{p{1.0\linewidth}}
\begin{Shaded}
\begin{Highlighting}[]

\NormalTok{\textbackslash{}includegraphics[width=2.5cm]\{chick\}}
\end{Highlighting}
\end{Shaded}
\\



\end{longtable}

One can also specify the scale with respect to the width of a line in the local environment (\LaTeXTT{\textbackslash{}linewidth}), the width of the text on a page (\LaTeXTT{\textbackslash{}textwidth}) or the height of the text on a page (\LaTeXTT{\textbackslash{}textheight}) (pictures not shown):

\begin{Shaded}
\begin{Highlighting}[]

\NormalTok{\textbackslash{}includegraphics[width=\textbackslash{}linewidth]\{chick\}}
\NormalTok{\textbackslash{}includegraphics[width=\textbackslash{}textwidth]\{chick\}}
\NormalTok{\textbackslash{}includegraphics[height=\textbackslash{}textheight]\{chick\}}
\end{Highlighting}
\end{Shaded}


To rotate (I also scaled the image down):

\begin{longtable}{p{1.0\linewidth}}
\begin{Shaded}
\begin{Highlighting}[]

\NormalTok{\textbackslash{}includegraphics[scale=0.5, angle=180]\{chick\}}
\end{Highlighting}
\end{Shaded}
\\



\end{longtable}

And finally, an example of how to crop an image should you wish to focus on one particular area of interest:

\begin{longtable}{p{1.0\linewidth}}
\begin{Shaded}
\begin{Highlighting}[]

\CommentTok{%trim option's parameter order: left bottom right top}
\NormalTok{\textbackslash{}includegraphics[trim = 10mm 80mm 20mm 5mm, clip, width=3cm]\{chick\}}
\end{Highlighting}
\end{Shaded}
\\



\end{longtable}

{\bfseries \setmainfont[Path=/usr/share/fonts/truetype/cmu/,UprightFont=cmunrm.ttf,BoldFont=cmunbx.ttf,ItalicFont=cmunti.ttf,BoldItalicFont=cmunbi.ttf]{cmunbx.ttf}\setmonofont[Path=/usr/share/fonts/truetype/cmu/,UprightFont=cmuntt.ttf,BoldFont=cmuntb.ttf,ItalicFont=cmunit.ttf,BoldItalicFont=cmuntx.ttf]{cmunbx.ttf}\bfseries Note:}{$\text{ }$}\setmainfont[Path=/usr/share/fonts/truetype/cmu/,UprightFont=cmunrm.ttf,BoldFont=cmunbx.ttf,ItalicFont=cmunti.ttf,BoldItalicFont=cmunbi.ttf]{cmunrm.ttf}\setmonofont[Path=/usr/share/fonts/truetype/cmu/,UprightFont=cmuntt.ttf,BoldFont=cmuntb.ttf,ItalicFont=cmunit.ttf,BoldItalicFont=cmuntx.ttf]{cmunrm.ttf} the presence of \LaTeXTT{clip}, as the trim operation will not work without it.

{\bfseries \setmainfont[Path=/usr/share/fonts/truetype/cmu/,UprightFont=cmunrm.ttf,BoldFont=cmunbx.ttf,ItalicFont=cmunti.ttf,BoldItalicFont=cmunbi.ttf]{cmunbx.ttf}\setmonofont[Path=/usr/share/fonts/truetype/cmu/,UprightFont=cmuntt.ttf,BoldFont=cmuntb.ttf,ItalicFont=cmunit.ttf,BoldItalicFont=cmuntx.ttf]{cmunbx.ttf}\bfseries Trick:}{$\text{ }$}\setmainfont[Path=/usr/share/fonts/truetype/cmu/,UprightFont=cmunrm.ttf,BoldFont=cmunbx.ttf,ItalicFont=cmunti.ttf,BoldItalicFont=cmunbi.ttf]{cmunrm.ttf}\setmonofont[Path=/usr/share/fonts/truetype/cmu/,UprightFont=cmuntt.ttf,BoldFont=cmuntb.ttf,ItalicFont=cmunit.ttf,BoldItalicFont=cmuntx.ttf]{cmunrm.ttf} You can also use negative trim values to add blank space to your graphics, in cases where you need some manual alignment.
\subsection{Spaces in names}
\label{345}

If the image file were called \symbol{34}chick picture.png\symbol{34}, then you need to include the full filename when importing the image:

\begin{longtable}{p{1.0\linewidth}}
\begin{Shaded}
\begin{Highlighting}[]

\NormalTok{\textbackslash{}includegraphics[scale=0.5]\{chick picture.png\}}
\end{Highlighting}
\end{Shaded}
\\


\begin{minipage}{1.0\linewidth}
\begin{center}
\includegraphics[width=1.0\linewidth,height=6.5in,keepaspectratio]{../images/62.png}
\end{center}
\raggedright{}\myfigurewithoutcaption{62}
\end{minipage}\vspace{0.75cm}



\end{longtable}

One option is to not use spaces in file names (if possible), or to simply replace spaces with underscores (\symbol{34}chick picture.png\symbol{34} to \symbol{34}chick\_picture.png\symbol{34})

\begin{longtable}{p{1.0\linewidth}}
\begin{Shaded}
\begin{Highlighting}[]

\NormalTok{\textbackslash{}includegraphics[scale=0.5]\{chick_picture\}}
\end{Highlighting}
\end{Shaded}
\\


\begin{minipage}{1.0\linewidth}
\begin{center}
\includegraphics[width=1.0\linewidth,height=6.5in,keepaspectratio]{../images/63.png}
\end{center}
\raggedright{}\myfigurewithoutcaption{63}
\end{minipage}\vspace{0.75cm}



\end{longtable}
\subsection{Borders}
\label{346}
It is possible to have LaTeX create a border around your image by using \LaTeXTT{\textbackslash{}fbox}:

\begin{Shaded}
\begin{Highlighting}[]

\NormalTok{\textbackslash{}setlength\textbackslash{}fboxsep\{0pt\}}
\NormalTok{\textbackslash{}setlength\textbackslash{}fboxrule\{0.5pt\}}
\NormalTok{\textbackslash{}fbox\{\textbackslash{}includegraphics\{chick\}\}}
\end{Highlighting}
\end{Shaded}


You can control the border padding with the \LaTeXTT{\textbackslash{}setlength\textbackslash{}fboxsep\{0pt\}} command, in this case I set it to 0pt to avoid any padding, so the border will be placed tightly around the image. You can control the thickness of the border by adjusting the \LaTeXTT{\textbackslash{}setlength\textbackslash{}fboxrule\{0.5pt\}} command.

See \mylref{478}{Boxes} for more details on \LaTeXTT{\textbackslash{}framebox} and \LaTeXTT{\textbackslash{}fbox}.
\section{Graphics storage}
\label{347}

The command \LaTeXTT{\textbackslash{}graphicspath} tells LaTeX where to look for images, which can be useful if you store images centrally for use in many different documents. The \LaTeXTT{\textbackslash{}graphicspath} command takes one argument, which specifies the additional paths you want to be searched when the \LaTeXTT{\textbackslash{}includegraphics} command is used. Here are some examples (trailing / is required):

\begin{Shaded}
\begin{Highlighting}[]

\NormalTok{\textbackslash{}graphicspath\{ \{/var/lib/images/\} \}}
\NormalTok{\textbackslash{}graphicspath\{ \{images_folder/\}\{other_folder/\}\{third_folder/\} \}}
\NormalTok{\textbackslash{}graphicspath\{ \{./images/\} \}}
\NormalTok{\textbackslash{}graphicspath\{ \{c:\textbackslash{}mypict~1\textbackslash{}camera\} \}}
\NormalTok{\textbackslash{}graphicspath\{ \{c:/mypict~1/camera/\} \} }\CommentTok{% works well in Win XP}
\end{Highlighting}
\end{Shaded}


Notice that, even if there is only one path given, there are two curly brackets around the path name.

Please see \myplainurl{http://www.ctan.org/tex-archive/macros/latex/required/graphics/grfguide.pdf.} In the third example shown there should be a directory named \symbol{34}images\symbol{34} in the same directory as your main tex file, i.e. this is RELATIVE addressing. 

Using absolute paths, \LaTeXTT{\textbackslash{}graphicspath} makes your file less portable, while using relative paths (like the third example), there should not be any problem with portability.  The fourth example uses the \symbol{34}safe\symbol{34} (MS-{}DOS) form of the Windows {\itshape \setmainfont[Path=/usr/share/fonts/truetype/cmu/,UprightFont=cmunrm.ttf,BoldFont=cmunbx.ttf,ItalicFont=cmunti.ttf,BoldItalicFont=cmunbi.ttf]{cmunti.ttf}\setmonofont[Path=/usr/share/fonts/truetype/cmu/,UprightFont=cmuntt.ttf,BoldFont=cmuntb.ttf,ItalicFont=cmunit.ttf,BoldItalicFont=cmuntx.ttf]{cmunti.ttf}\itshape MyPictures}{$\text{ }$}\setmainfont[Path=/usr/share/fonts/truetype/cmu/,UprightFont=cmunrm.ttf,BoldFont=cmunbx.ttf,ItalicFont=cmunti.ttf,BoldItalicFont=cmunbi.ttf]{cmunrm.ttf}\setmonofont[Path=/usr/share/fonts/truetype/cmu/,UprightFont=cmuntt.ttf,BoldFont=cmuntb.ttf,ItalicFont=cmunit.ttf,BoldItalicFont=cmuntx.ttf]{cmunrm.ttf} folder because it\textquotesingle{}s a bad idea to use directory names containing spaces.  Again, ensure file names do not contain spaces or alternatively if you are using PDFLaTeX, you can use the package \LaTeXTT{grffile} which will allow you to use spaces in file names.

Note that you cannot make the graphicx package search directories recursively.
Under Linux/Unix, you can achieve a recursive search using the environment variable {\ttfamily \setmainfont[Path=/usr/share/fonts/truetype/cmu/,UprightFont=cmunrm.ttf,BoldFont=cmunbx.ttf,ItalicFont=cmunti.ttf,BoldItalicFont=cmunbi.ttf]{cmuntt.ttf}\setmonofont[Path=/usr/share/fonts/truetype/cmu/,UprightFont=cmuntt.ttf,BoldFont=cmuntb.ttf,ItalicFont=cmunit.ttf,BoldItalicFont=cmuntx.ttf]{cmuntt.ttf}\ttfamily TEXINPUTS}\setmainfont[Path=/usr/share/fonts/truetype/cmu/,UprightFont=cmunrm.ttf,BoldFont=cmunbx.ttf,ItalicFont=cmunti.ttf,BoldItalicFont=cmunbi.ttf]{cmunrm.ttf}\setmonofont[Path=/usr/share/fonts/truetype/cmu/,UprightFont=cmuntt.ttf,BoldFont=cmuntb.ttf,ItalicFont=cmunit.ttf,BoldItalicFont=cmuntx.ttf]{cmunrm.ttf}, e.g., by setting it to 

\begin{Shaded}
\begin{Highlighting}[]

\KeywordTok{export}\ensuremath{\text{ }}\OtherTok{TEXINPUTS=}\NormalTok{./images//:./Snapshots//}\newline
\end{Highlighting}
\end{Shaded}


before running latex/pdflatex or your TeX-{}IDE.
(But this, of course, is not a portable method.)
\section{Images as figures}
\label{348}

The figure environment is not exclusively used for images. We will only give a short preview of figures here. More information on the figure environment and how to use it can be found in \mylref{362}{Floats, Figures and Captions}.

There are many scenarios where you might want to accompany an image with a caption and possibly a cross-{}reference. This is done using the \LaTeXTT{figure} environment. The following code sample shows the bare minimum required to use an image as a figure.

\begin{Shaded}
\begin{Highlighting}[]

\NormalTok{\textbackslash{}begin\{figure\}[p]}
    \NormalTok{\textbackslash{}includegraphics\{image.png\}}
\NormalTok{\textbackslash{}end\{figure\}}
\end{Highlighting}
\end{Shaded}


The above code extract is relatively trivial, and doesn\textquotesingle{}t offer much functionality. The following code sample shows an extended use of the figure environment which is almost universally useful, offering a caption and label, centering the image and scaling it to 80\% of the width of the text.

\begin{Shaded}
\begin{Highlighting}[]

\NormalTok{\textbackslash{}begin\{figure\}[p]}
    \NormalTok{\textbackslash{}centering}
    \NormalTok{\textbackslash{}includegraphics[width=0.8\textbackslash{}textwidth]\{image.png\}}
    \NormalTok{\textbackslash{}caption\{Awesome Image\}}
    \NormalTok{\textbackslash{}label\{fig:awesome_image\}}
\NormalTok{\textbackslash{}end\{figure\}}
\end{Highlighting}
\end{Shaded}

\section{Text wrapping around pictures}
\label{349}

See \mylref{362}{Floats, Figures and Captions}.
\section{Seamless text integration}
\label{350}

The drawback of importing graphics that were generated with a third-{}party tool is that font and size will not match with the rest of the document. There are still some workarounds though.

The easiest solution is to use the picture environment and then simply use the \symbol{34}put\symbol{34} command to put a graphics file inside the picture, along with any other desired LaTeX element. For example:




\begin{longtable}{p{1.0\linewidth}}
\begin{Shaded}
\begin{Highlighting}[]
\NormalTok{\textbackslash{}setlength\{\textbackslash{}unitlength\}\{0.8cm\}}
\NormalTok{\textbackslash{}begin\{picture\}(6,5)}
\NormalTok{\textbackslash{}put(3.5,0.4)\{$\textbackslash{}displaystyle}
\NormalTok{s:=\textbackslash{}frac\{a+b+c\}\{2\}$\}}
\NormalTok{\textbackslash{}put(1,1)\{\textbackslash{}includegraphics[}
  \NormalTok{width=2cm,height=2cm]\{picture.eps\} \}}
\NormalTok{\textbackslash{}end\{picture\}}
\end{Highlighting}
\end{Shaded}
\\


\begin{minipage}{0.37500\textwidth}
\begin{center}
\includegraphics[width=1.0\textwidth,height=6.5in,keepaspectratio]{../images/64.png}
\end{center}
\raggedright{}\myfigurewithoutcaption{64}
\end{minipage}\vspace{0.75cm}



\end{longtable}

Note that the border around the picture in the above example was added by using \mylref{346}{{\ttfamily \setmainfont[Path=/usr/share/fonts/truetype/cmu/,UprightFont=cmunrm.ttf,BoldFont=cmunbx.ttf,ItalicFont=cmunti.ttf,BoldItalicFont=cmunbi.ttf]{cmuntt.ttf}\setmonofont[Path=/usr/share/fonts/truetype/cmu/,UprightFont=cmuntt.ttf,BoldFont=cmuntb.ttf,ItalicFont=cmunit.ttf,BoldItalicFont=cmuntx.ttf]{cmuntt.ttf}\ttfamily \textbackslash{}fbox}}, so the contents of the border is the picture as generated by the above code.

Tools like Inkscape or Xfig have a dedicated LaTeX export feature that will let you use correct font and size for text in vector graphics. See \mylref{355}{\#Third-{}party graphics tools}.

For a perfect integration of graphics, you might consider \mylref{774}{procedural graphics} capabilities of some LaTeX packages like TikZ or PSTricks. It lets you {\itshape \setmainfont[Path=/usr/share/fonts/truetype/cmu/,UprightFont=cmunrm.ttf,BoldFont=cmunbx.ttf,ItalicFont=cmunti.ttf,BoldItalicFont=cmunbi.ttf]{cmunti.ttf}\setmonofont[Path=/usr/share/fonts/truetype/cmu/,UprightFont=cmuntt.ttf,BoldFont=cmuntb.ttf,ItalicFont=cmunit.ttf,BoldItalicFont=cmuntx.ttf]{cmunti.ttf}\itshape draw}{$\text{ }$}\setmainfont[Path=/usr/share/fonts/truetype/cmu/,UprightFont=cmunrm.ttf,BoldFont=cmunbx.ttf,ItalicFont=cmunti.ttf,BoldItalicFont=cmunbi.ttf]{cmunrm.ttf}\setmonofont[Path=/usr/share/fonts/truetype/cmu/,UprightFont=cmuntt.ttf,BoldFont=cmuntb.ttf,ItalicFont=cmunit.ttf,BoldItalicFont=cmuntx.ttf]{cmunrm.ttf} from within a document source. While the learning curve is steeper, it is worth it most of the time.
\section{Including full PDF pages}
\label{351}

There is a great package for including full pages of PDF files: \myhref{http://www.ctan.org/tex-archive/macros/latex/contrib/pdfpages}{pdfpages}. It is capable of inserting entire pages as is and more pages per one page in any layout (e.g. 2x3).

The package has several options:
\begin{Shaded}
\begin{Highlighting}[]

\NormalTok{\textbackslash{}usepackage[ options ]\{pdfpages\}}
\end{Highlighting}
\end{Shaded}

Options:
\begin{myitemize}
\item{}  {\bfseries \setmainfont[Path=/usr/share/fonts/truetype/cmu/,UprightFont=cmunrm.ttf,BoldFont=cmunbx.ttf,ItalicFont=cmunti.ttf,BoldItalicFont=cmunbi.ttf]{cmunbx.ttf}\setmonofont[Path=/usr/share/fonts/truetype/cmu/,UprightFont=cmuntt.ttf,BoldFont=cmuntb.ttf,ItalicFont=cmunit.ttf,BoldItalicFont=cmuntx.ttf]{cmunbx.ttf}\bfseries final}\setmainfont[Path=/usr/share/fonts/truetype/cmu/,UprightFont=cmunrm.ttf,BoldFont=cmunbx.ttf,ItalicFont=cmunti.ttf,BoldItalicFont=cmunbi.ttf]{cmunrm.ttf}\setmonofont[Path=/usr/share/fonts/truetype/cmu/,UprightFont=cmuntt.ttf,BoldFont=cmuntb.ttf,ItalicFont=cmunit.ttf,BoldItalicFont=cmuntx.ttf]{cmunrm.ttf}: Inserts pages. This is the default.
\item{}  {\bfseries \setmainfont[Path=/usr/share/fonts/truetype/cmu/,UprightFont=cmunrm.ttf,BoldFont=cmunbx.ttf,ItalicFont=cmunti.ttf,BoldItalicFont=cmunbi.ttf]{cmunbx.ttf}\setmonofont[Path=/usr/share/fonts/truetype/cmu/,UprightFont=cmuntt.ttf,BoldFont=cmuntb.ttf,ItalicFont=cmunit.ttf,BoldItalicFont=cmuntx.ttf]{cmunbx.ttf}\bfseries draft}\setmainfont[Path=/usr/share/fonts/truetype/cmu/,UprightFont=cmunrm.ttf,BoldFont=cmunbx.ttf,ItalicFont=cmunti.ttf,BoldItalicFont=cmunbi.ttf]{cmunrm.ttf}\setmonofont[Path=/usr/share/fonts/truetype/cmu/,UprightFont=cmuntt.ttf,BoldFont=cmuntb.ttf,ItalicFont=cmunit.ttf,BoldItalicFont=cmuntx.ttf]{cmunrm.ttf}: Does not insert pages, but prints a box and the filename instead.
\item{}  {\bfseries \setmainfont[Path=/usr/share/fonts/truetype/cmu/,UprightFont=cmunrm.ttf,BoldFont=cmunbx.ttf,ItalicFont=cmunti.ttf,BoldItalicFont=cmunbi.ttf]{cmunbx.ttf}\setmonofont[Path=/usr/share/fonts/truetype/cmu/,UprightFont=cmuntt.ttf,BoldFont=cmuntb.ttf,ItalicFont=cmunit.ttf,BoldItalicFont=cmuntx.ttf]{cmunbx.ttf}\bfseries enable-{}survey}\setmainfont[Path=/usr/share/fonts/truetype/cmu/,UprightFont=cmunrm.ttf,BoldFont=cmunbx.ttf,ItalicFont=cmunti.ttf,BoldItalicFont=cmunbi.ttf]{cmunrm.ttf}\setmonofont[Path=/usr/share/fonts/truetype/cmu/,UprightFont=cmuntt.ttf,BoldFont=cmuntb.ttf,ItalicFont=cmunit.ttf,BoldItalicFont=cmuntx.ttf]{cmunrm.ttf}: Activates survey functionalities. (Experimental, subject to change.)
\end{myitemize}


The first command is
\begin{Shaded}
\begin{Highlighting}[]

\NormalTok{\textbackslash{}includepdf[ key=val ]\{ filename \}}
\end{Highlighting}
\end{Shaded}

Options for {\bfseries \setmainfont[Path=/usr/share/fonts/truetype/cmu/,UprightFont=cmunrm.ttf,BoldFont=cmunbx.ttf,ItalicFont=cmunti.ttf,BoldItalicFont=cmunbi.ttf]{cmunbx.ttf}\setmonofont[Path=/usr/share/fonts/truetype/cmu/,UprightFont=cmuntt.ttf,BoldFont=cmuntb.ttf,ItalicFont=cmunit.ttf,BoldItalicFont=cmuntx.ttf]{cmunbx.ttf}\bfseries key=val}{$\text{ }$}\setmainfont[Path=/usr/share/fonts/truetype/cmu/,UprightFont=cmunrm.ttf,BoldFont=cmunbx.ttf,ItalicFont=cmunti.ttf,BoldItalicFont=cmunbi.ttf]{cmunrm.ttf}\setmonofont[Path=/usr/share/fonts/truetype/cmu/,UprightFont=cmuntt.ttf,BoldFont=cmuntb.ttf,ItalicFont=cmunit.ttf,BoldItalicFont=cmuntx.ttf]{cmunrm.ttf} (A comma separated list of options using the key = value syntax)

\begin{longtable}{|>{\RaggedRight}p{0.20465\linewidth}|>{\RaggedRight}p{0.73820\linewidth}|} \hline 
\hspace*{0pt}\ignorespaces{}\hspace*{0pt} {\bfseries \setmainfont[Path=/usr/share/fonts/truetype/cmu/,UprightFont=cmunrm.ttf,BoldFont=cmunbx.ttf,ItalicFont=cmunti.ttf,BoldItalicFont=cmunbi.ttf]{cmunbx.ttf}\setmonofont[Path=/usr/share/fonts/truetype/cmu/,UprightFont=cmuntt.ttf,BoldFont=cmuntb.ttf,ItalicFont=cmunit.ttf,BoldItalicFont=cmuntx.ttf]{cmunbx.ttf}\bfseries pages}&\hspace*{0pt}\ignorespaces{}\hspace*{0pt}{$\text{ }$}\setmainfont[Path=/usr/share/fonts/truetype/cmu/,UprightFont=cmunrm.ttf,BoldFont=cmunbx.ttf,ItalicFont=cmunti.ttf,BoldItalicFont=cmunbi.ttf]{cmunrm.ttf}\setmonofont[Path=/usr/share/fonts/truetype/cmu/,UprightFont=cmuntt.ttf,BoldFont=cmuntb.ttf,ItalicFont=cmunit.ttf,BoldItalicFont=cmuntx.ttf]{cmunrm.ttf} Selects pages to insert. The argument is a comma separated list, containing page numbers (pages=\{3,5,6,8\}), ranges of page numbers (pages=\{4-{}9\}) or any combination. To insert empty pages use \{\}. For instance \LaTeXTT{pages=\{3,\{\},8-{}11,15\}} will insert page 3, an empty page, and pages 8, 9, 10, 11, and 15.Actually not only links but all kinds of PDF annotations will get lost. Page ranges are specified by the following syntax: {\itshape \setmainfont[Path=/usr/share/fonts/truetype/cmu/,UprightFont=cmunrm.ttf,BoldFont=cmunbx.ttf,ItalicFont=cmunti.ttf,BoldItalicFont=cmunbi.ttf]{cmunti.ttf}\setmonofont[Path=/usr/share/fonts/truetype/cmu/,UprightFont=cmuntt.ttf,BoldFont=cmuntb.ttf,ItalicFont=cmunit.ttf,BoldItalicFont=cmuntx.ttf]{cmunti.ttf}\itshape m -{} n}\setmainfont[Path=/usr/share/fonts/truetype/cmu/,UprightFont=cmunrm.ttf,BoldFont=cmunbx.ttf,ItalicFont=cmunti.ttf,BoldItalicFont=cmunbi.ttf]{cmunrm.ttf}\setmonofont[Path=/usr/share/fonts/truetype/cmu/,UprightFont=cmuntt.ttf,BoldFont=cmuntb.ttf,ItalicFont=cmunit.ttf,BoldItalicFont=cmuntx.ttf]{cmunrm.ttf}. This selects all pages from {\itshape \setmainfont[Path=/usr/share/fonts/truetype/cmu/,UprightFont=cmunrm.ttf,BoldFont=cmunbx.ttf,ItalicFont=cmunti.ttf,BoldItalicFont=cmunbi.ttf]{cmunti.ttf}\setmonofont[Path=/usr/share/fonts/truetype/cmu/,UprightFont=cmuntt.ttf,BoldFont=cmuntb.ttf,ItalicFont=cmunit.ttf,BoldItalicFont=cmuntx.ttf]{cmunti.ttf}\itshape m to n}\setmainfont[Path=/usr/share/fonts/truetype/cmu/,UprightFont=cmunrm.ttf,BoldFont=cmunbx.ttf,ItalicFont=cmunti.ttf,BoldItalicFont=cmunbi.ttf]{cmunrm.ttf}\setmonofont[Path=/usr/share/fonts/truetype/cmu/,UprightFont=cmuntt.ttf,BoldFont=cmuntb.ttf,ItalicFont=cmunit.ttf,BoldItalicFont=cmuntx.ttf]{cmunrm.ttf}. Omitting m defaults to the first page; omitting n defaults to the last page of the document. Another way to select the last page of the document, is to use the keyword last. (This is only permitted in a page range.)E.g.: {\itshape \setmainfont[Path=/usr/share/fonts/truetype/cmu/,UprightFont=cmunrm.ttf,BoldFont=cmunbx.ttf,ItalicFont=cmunti.ttf,BoldItalicFont=cmunbi.ttf]{cmunti.ttf}\setmonofont[Path=/usr/share/fonts/truetype/cmu/,UprightFont=cmuntt.ttf,BoldFont=cmuntb.ttf,ItalicFont=cmunit.ttf,BoldItalicFont=cmuntx.ttf]{cmunti.ttf}\itshape pages=-{}}{$\text{ }$}\setmainfont[Path=/usr/share/fonts/truetype/cmu/,UprightFont=cmunrm.ttf,BoldFont=cmunbx.ttf,ItalicFont=cmunti.ttf,BoldItalicFont=cmunbi.ttf]{cmunrm.ttf}\setmonofont[Path=/usr/share/fonts/truetype/cmu/,UprightFont=cmuntt.ttf,BoldFont=cmuntb.ttf,ItalicFont=cmunit.ttf,BoldItalicFont=cmuntx.ttf]{cmunrm.ttf} will insert all pages of the document, and {\itshape \setmainfont[Path=/usr/share/fonts/truetype/cmu/,UprightFont=cmunrm.ttf,BoldFont=cmunbx.ttf,ItalicFont=cmunti.ttf,BoldItalicFont=cmunbi.ttf]{cmunti.ttf}\setmonofont[Path=/usr/share/fonts/truetype/cmu/,UprightFont=cmuntt.ttf,BoldFont=cmuntb.ttf,ItalicFont=cmunit.ttf,BoldItalicFont=cmuntx.ttf]{cmunti.ttf}\itshape pages=last-{}1}{$\text{ }$}\setmainfont[Path=/usr/share/fonts/truetype/cmu/,UprightFont=cmunrm.ttf,BoldFont=cmunbx.ttf,ItalicFont=cmunti.ttf,BoldItalicFont=cmunbi.ttf]{cmunrm.ttf}\setmonofont[Path=/usr/share/fonts/truetype/cmu/,UprightFont=cmuntt.ttf,BoldFont=cmuntb.ttf,ItalicFont=cmunit.ttf,BoldItalicFont=cmuntx.ttf]{cmunrm.ttf} will insert all pages in reverse order.(Default: pages=1)\\ \hline \hspace*{0pt}\ignorespaces{}\hspace*{0pt} {\bfseries \setmainfont[Path=/usr/share/fonts/truetype/cmu/,UprightFont=cmunrm.ttf,BoldFont=cmunbx.ttf,ItalicFont=cmunti.ttf,BoldItalicFont=cmunbi.ttf]{cmunbx.ttf}\setmonofont[Path=/usr/share/fonts/truetype/cmu/,UprightFont=cmuntt.ttf,BoldFont=cmuntb.ttf,ItalicFont=cmunit.ttf,BoldItalicFont=cmuntx.ttf]{cmunbx.ttf}\bfseries angle}&\hspace*{0pt}\ignorespaces{}\hspace*{0pt}{$\text{ }$}\setmainfont[Path=/usr/share/fonts/truetype/cmu/,UprightFont=cmunrm.ttf,BoldFont=cmunbx.ttf,ItalicFont=cmunti.ttf,BoldItalicFont=cmunbi.ttf]{cmunrm.ttf}\setmonofont[Path=/usr/share/fonts/truetype/cmu/,UprightFont=cmuntt.ttf,BoldFont=cmuntb.ttf,ItalicFont=cmunit.ttf,BoldItalicFont=cmuntx.ttf]{cmunrm.ttf} You can use the angle-{}option for turning the included page, for exampe for turning a landscape document when the latex-{}document is portrait. Example: \LaTeXTT{angle=90}\\ \hline \hspace*{0pt}\ignorespaces{}\hspace*{0pt} {\bfseries \setmainfont[Path=/usr/share/fonts/truetype/cmu/,UprightFont=cmunrm.ttf,BoldFont=cmunbx.ttf,ItalicFont=cmunti.ttf,BoldItalicFont=cmunbi.ttf]{cmunbx.ttf}\setmonofont[Path=/usr/share/fonts/truetype/cmu/,UprightFont=cmuntt.ttf,BoldFont=cmuntb.ttf,ItalicFont=cmunit.ttf,BoldItalicFont=cmuntx.ttf]{cmunbx.ttf}\bfseries addtolist}&\hspace*{0pt}\ignorespaces{}\hspace*{0pt}{$\text{ }$}\setmainfont[Path=/usr/share/fonts/truetype/cmu/,UprightFont=cmunrm.ttf,BoldFont=cmunbx.ttf,ItalicFont=cmunti.ttf,BoldItalicFont=cmunbi.ttf]{cmunrm.ttf}\setmonofont[Path=/usr/share/fonts/truetype/cmu/,UprightFont=cmuntt.ttf,BoldFont=cmuntb.ttf,ItalicFont=cmunit.ttf,BoldItalicFont=cmuntx.ttf]{cmunrm.ttf} Adds an entry to the list of figures, the list of tables, or any other list (e.g. from float.sty). This option requires four arguments, separated by commas:\LaTeXTT{addtolist=\{ page number , type , heading , label \}}\begin{myitemize}\item{}  \LaTeXTT{page number} : Page number of the inserted page.\item{}  \LaTeXTT{type}: Name of a floating environment. (figure, table, etc.)\item{}  \LaTeXTT{heading}: Title inserted into LoF, LoT, etc.\item{}  \LaTeXTT{label}: Name of the label. This label can be referred to with \textbackslash{}ref and \textbackslash{}pageref.\end{myitemize}Like addtotoc, addtolist accepts multiple sets of the above mentioned four arguments, all separated by commas. The proper recursive definition is:\LaTeXTT{addtolist=\{ page number , type , heading , label {$\text{[}$}, lof-{}list {$\text{]}$} \}}\\ \hline \hspace*{0pt}\ignorespaces{}\hspace*{0pt} {\bfseries \setmainfont[Path=/usr/share/fonts/truetype/cmu/,UprightFont=cmunrm.ttf,BoldFont=cmunbx.ttf,ItalicFont=cmunti.ttf,BoldItalicFont=cmunbi.ttf]{cmunbx.ttf}\setmonofont[Path=/usr/share/fonts/truetype/cmu/,UprightFont=cmuntt.ttf,BoldFont=cmuntb.ttf,ItalicFont=cmunit.ttf,BoldItalicFont=cmuntx.ttf]{cmunbx.ttf}\bfseries pagecommand}&\hspace*{0pt}\ignorespaces{}\hspace*{0pt}{$\text{ }$}\setmainfont[Path=/usr/share/fonts/truetype/cmu/,UprightFont=cmunrm.ttf,BoldFont=cmunbx.ttf,ItalicFont=cmunti.ttf,BoldItalicFont=cmunbi.ttf]{cmunrm.ttf}\setmonofont[Path=/usr/share/fonts/truetype/cmu/,UprightFont=cmuntt.ttf,BoldFont=cmuntb.ttf,ItalicFont=cmunit.ttf,BoldItalicFont=cmuntx.ttf]{cmunrm.ttf} Declares LaTeX-{}commands, which are executed on each sheet of paper. (Default: {\itshape \setmainfont[Path=/usr/share/fonts/truetype/cmu/,UprightFont=cmunrm.ttf,BoldFont=cmunbx.ttf,ItalicFont=cmunti.ttf,BoldItalicFont=cmunbi.ttf]{cmunti.ttf}\setmonofont[Path=/usr/share/fonts/truetype/cmu/,UprightFont=cmuntt.ttf,BoldFont=cmuntb.ttf,ItalicFont=cmunit.ttf,BoldItalicFont=cmuntx.ttf]{cmunti.ttf}\itshape pagecommand=\{\textbackslash{}thispagestyle\{empty\}\}}\setmainfont[Path=/usr/share/fonts/truetype/cmu/,UprightFont=cmunrm.ttf,BoldFont=cmunbx.ttf,ItalicFont=cmunti.ttf,BoldItalicFont=cmunbi.ttf]{cmunrm.ttf}\setmonofont[Path=/usr/share/fonts/truetype/cmu/,UprightFont=cmuntt.ttf,BoldFont=cmuntb.ttf,ItalicFont=cmunit.ttf,BoldItalicFont=cmuntx.ttf]{cmunrm.ttf}\LaTeXTT{pagecommand=\{\textbackslash{}label\{fig:mylabel\}\}}\\ \hline 
\end{longtable}


You can also insert pages of several external PDF documents.
\begin{Shaded}
\begin{Highlighting}[]

\NormalTok{\textbackslash{}includepdfmerge[ key=val ]\{ file-page-list \}}
\end{Highlighting}
\end{Shaded}


Several PDFs can be placed table-{}like on one page.
See more information in its \myhref{http://www.ctan.org/tex-archive/macros/latex/contrib/pdfpages/pdfpages.pdf}{documentation}.
\section{Converting graphics}
\label{352}
{\bfseries
\begin{mydescription}Note
\end{mydescription}
}

You should also take a look at \mylref{925}{Export To Other Formats} for other possibilities.
{\bfseries
\begin{mydescription}epstopdf
\end{mydescription}
}

You can convert EPS to PDF with the \myhref{http://www.ctan.org/tex-archive/support/epstopdf/}{epstopdf utility}, included in package of the same name. This tool is actually called by {\ttfamily \setmainfont[Path=/usr/share/fonts/truetype/cmu/,UprightFont=cmunrm.ttf,BoldFont=cmunbx.ttf,ItalicFont=cmunti.ttf,BoldItalicFont=cmunbi.ttf]{cmuntt.ttf}\setmonofont[Path=/usr/share/fonts/truetype/cmu/,UprightFont=cmuntt.ttf,BoldFont=cmuntb.ttf,ItalicFont=cmunit.ttf,BoldItalicFont=cmuntx.ttf]{cmuntt.ttf}\ttfamily pdflatex}{$\text{ }$}\setmainfont[Path=/usr/share/fonts/truetype/cmu/,UprightFont=cmunrm.ttf,BoldFont=cmunbx.ttf,ItalicFont=cmunti.ttf,BoldItalicFont=cmunbi.ttf]{cmunrm.ttf}\setmonofont[Path=/usr/share/fonts/truetype/cmu/,UprightFont=cmuntt.ttf,BoldFont=cmuntb.ttf,ItalicFont=cmunit.ttf,BoldItalicFont=cmuntx.ttf]{cmunrm.ttf} to convert EPS files to PDF in the background when the \LaTeXTT{graphicx} package is loaded. This process is completely invisible to the user.

You can batch convert files using the command-{}line.
In Bourne Shell (Unix) this can be done by:

\begin{Shaded}
\begin{Highlighting}[]

\NormalTok{\$\ensuremath{\text{ }}}\KeywordTok{for}\ensuremath{\text{ }}\KeywordTok{i}\ensuremath{\text{ }}\NormalTok{in\ensuremath{\text{ }}*.eps}\KeywordTok{;}\ensuremath{\text{ }}\KeywordTok{do}\ensuremath{\text{ }}\KeywordTok{epstopdf}\ensuremath{\text{ }}\StringTok{"}\OtherTok{\$i}\StringTok{"}\KeywordTok{;}\ensuremath{\text{ }}\KeywordTok{done}\newline
\end{Highlighting}
\end{Shaded}


In Windows, multiple files can be converted by placing the following line in a \myhref{https://en.wikipedia.org/wiki/Batch\%20file}{batch file} (a text file with a {\ttfamily \setmainfont[Path=/usr/share/fonts/truetype/cmu/,UprightFont=cmunrm.ttf,BoldFont=cmunbx.ttf,ItalicFont=cmunti.ttf,BoldItalicFont=cmunbi.ttf]{cmuntt.ttf}\setmonofont[Path=/usr/share/fonts/truetype/cmu/,UprightFont=cmuntt.ttf,BoldFont=cmuntb.ttf,ItalicFont=cmunit.ttf,BoldItalicFont=cmuntx.ttf]{cmuntt.ttf}\ttfamily .bat}{$\text{ }$}\setmainfont[Path=/usr/share/fonts/truetype/cmu/,UprightFont=cmunrm.ttf,BoldFont=cmunbx.ttf,ItalicFont=cmunti.ttf,BoldItalicFont=cmunbi.ttf]{cmunrm.ttf}\setmonofont[Path=/usr/share/fonts/truetype/cmu/,UprightFont=cmuntt.ttf,BoldFont=cmuntb.ttf,ItalicFont=cmunit.ttf,BoldItalicFont=cmuntx.ttf]{cmunrm.ttf} extension) in the same directory as the images:

\begin{Shaded}
\begin{Highlighting}[]

\KeywordTok{for}\ensuremath{\text{ }}\KeywordTok{\%\%f}\ensuremath{\text{ }}\NormalTok{in\ensuremath{\text{ }}(*.eps)\ensuremath{\text{ }}}\KeywordTok{do}\ensuremath{\text{ }}\KeywordTok{epstopdf}\ensuremath{\text{ }}\NormalTok{\%\%f}\newline
\end{Highlighting}
\end{Shaded}

which can then be run from the command line.

If \LaTeXTT{epstopdf} produces whole page with your small graphics somewhere on it, use 

\begin{Shaded}
\begin{Highlighting}[]

\NormalTok{\$\ensuremath{\text{ }}}\KeywordTok{epstopdf}\ensuremath{\text{ }}\NormalTok{--gsopt=-dEPSCrop\ensuremath{\text{ }}foo.eps}\newline
\end{Highlighting}
\end{Shaded}


or try using {\ttfamily \setmainfont[Path=/usr/share/fonts/truetype/cmu/,UprightFont=cmunrm.ttf,BoldFont=cmunbx.ttf,ItalicFont=cmunti.ttf,BoldItalicFont=cmunbi.ttf]{cmuntt.ttf}\setmonofont[Path=/usr/share/fonts/truetype/cmu/,UprightFont=cmuntt.ttf,BoldFont=cmuntb.ttf,ItalicFont=cmunit.ttf,BoldItalicFont=cmuntx.ttf]{cmuntt.ttf}\ttfamily ps2pdf}{$\text{ }$}\setmainfont[Path=/usr/share/fonts/truetype/cmu/,UprightFont=cmunrm.ttf,BoldFont=cmunbx.ttf,ItalicFont=cmunti.ttf,BoldItalicFont=cmunbi.ttf]{cmunrm.ttf}\setmonofont[Path=/usr/share/fonts/truetype/cmu/,UprightFont=cmuntt.ttf,BoldFont=cmuntb.ttf,ItalicFont=cmunit.ttf,BoldItalicFont=cmuntx.ttf]{cmunrm.ttf} utility which should be installed with Ghostscript (required for any TeX distribution).

\begin{Shaded}
\begin{Highlighting}[]

\NormalTok{\$\ensuremath{\text{ }}}\KeywordTok{ps2pdf}\ensuremath{\text{ }}\NormalTok{-dEPSCrop\ensuremath{\text{ }}foo.eps}\newline
\end{Highlighting}
\end{Shaded}
 
to crop final PDF.
{\bfseries
\begin{mydescription}eps2eps
\end{mydescription}
}

When all of the above fails, one can simplify the EPS file before attempting other conversions, by using the \myhref{http://linuxcommand.org/man_pages/eps2eps1.html}{eps2eps} tool (also see next section):

\begin{Shaded}
\begin{Highlighting}[]

\NormalTok{\$\ensuremath{\text{ }}}\KeywordTok{eps2eps}\ensuremath{\text{ }}\NormalTok{input.eps\ensuremath{\text{ }}input-e2.eps}\newline
\end{Highlighting}
\end{Shaded}

This will convert all the fonts to pre-{}drawn images, which is sometimes desirable when submitting manuscripts for publication. However, on the downside, the fonts are NOT converted to lines, but instead to bitmaps, which reduces the quality of the fonts.
{\bfseries
\begin{mydescription}imgtops
\end{mydescription}
}

\myhref{http://imgtops.sourceforge.net/}{imgtops} is  a lightweight graphics utility for conversions between raster graphics (JPG, PNG, ...) and EPS/PS files.
{\bfseries
\begin{mydescription}Inkscape
\end{mydescription}
}

Inkscape can also convert files from and to several formats, either from the GUI or from the command-{}line. For instance, to obtain a PDF from a SVG image you can do:

\begin{Shaded}
\begin{Highlighting}[]

\NormalTok{\$\ensuremath{\text{ }}}\KeywordTok{inkscape}\ensuremath{\text{ }}\NormalTok{-z\ensuremath{\text{ }}-D\ensuremath{\text{ }}--file=input.svg\ensuremath{\text{ }}--export-pdf=output.pdf}\newline
\end{Highlighting}
\end{Shaded}

It is possible to run this from within a LaTeX file, the 

UNKNOWN TEMPLATE  
LaTeX/package

{svg}

 package (when running (pdf)latex with the -{}-{}shell-{}escape option) can do this using Inkscape\textquotesingle{}s pdf+tex export option, or a simple macro can be used.  See \myhref{http://tex.stackexchange.com/q/2099/28808}{How to include SVG diagrams in LaTeX? -{}-{} Stackexchange}
See \mylref{925}{Export To Other Formats} for more details.
{\bfseries
\begin{mydescription}pstoedit
\end{mydescription}
}

To properly edit an EPS file, you can convert it to an {\itshape \setmainfont[Path=/usr/share/fonts/truetype/cmu/,UprightFont=cmunrm.ttf,BoldFont=cmunbx.ttf,ItalicFont=cmunti.ttf,BoldItalicFont=cmunbi.ttf]{cmunti.ttf}\setmonofont[Path=/usr/share/fonts/truetype/cmu/,UprightFont=cmuntt.ttf,BoldFont=cmuntb.ttf,ItalicFont=cmunit.ttf,BoldItalicFont=cmuntx.ttf]{cmunti.ttf}\itshape editable}{$\text{ }$}\setmainfont[Path=/usr/share/fonts/truetype/cmu/,UprightFont=cmunrm.ttf,BoldFont=cmunbx.ttf,ItalicFont=cmunti.ttf,BoldItalicFont=cmunbi.ttf]{cmunrm.ttf}\setmonofont[Path=/usr/share/fonts/truetype/cmu/,UprightFont=cmuntt.ttf,BoldFont=cmuntb.ttf,ItalicFont=cmunit.ttf,BoldItalicFont=cmuntx.ttf]{cmunrm.ttf} format using \myhref{http://www.pstoedit.net/}{pstoedit}. For instance, to get an Xfig-{}editable file, do:

\begin{Shaded}
\begin{Highlighting}[]

\NormalTok{\$\ensuremath{\text{ }}}\KeywordTok{pstoedit}\ensuremath{\text{ }}\NormalTok{-f\ensuremath{\text{ }}fig\ensuremath{\text{ }}input.eps\ensuremath{\text{ }}output.fig}\newline
\end{Highlighting}
\end{Shaded}

And to get an SVG file (editable with any vector graphics tool like Inkscape) you can do:

\begin{Shaded}
\begin{Highlighting}[]

\NormalTok{\$\ensuremath{\text{ }}}\KeywordTok{pstoedit}\ensuremath{\text{ }}\NormalTok{-f\ensuremath{\text{ }}plot-svg\ensuremath{\text{ }}input.eps\ensuremath{\text{ }}output.svg}\newline
\end{Highlighting}
\end{Shaded}

Sometimes pstoedit fails to create the target format (for example when the EPS file contains clipping information).
{\bfseries
\begin{mydescription}PDFCreator
\end{mydescription}
}

Under Windows, \myhref{http://sourceforge.net/projects/pdfcreator/}{PDFCreator} is an open source software that can create PDF as well as EPS files. It installs a virtual printer that can be accessed from other software having a \symbol{34}print...\symbol{34} entry in their menu (virtually any program).
{\bfseries
\begin{mydescription}Raster graphics converters
\end{mydescription}
}

\begin{myitemize}
\item{} \myhref{http://pts.szit.bme.hu/sam2p/}{Sam2p} ({\ttfamily \setmainfont[Path=/usr/share/fonts/truetype/cmu/,UprightFont=cmunrm.ttf,BoldFont=cmunbx.ttf,ItalicFont=cmunti.ttf,BoldItalicFont=cmunbi.ttf]{cmuntt.ttf}\setmonofont[Path=/usr/share/fonts/truetype/cmu/,UprightFont=cmuntt.ttf,BoldFont=cmuntb.ttf,ItalicFont=cmunit.ttf,BoldItalicFont=cmuntx.ttf]{cmuntt.ttf}\ttfamily convert}\setmainfont[Path=/usr/share/fonts/truetype/cmu/,UprightFont=cmunrm.ttf,BoldFont=cmunbx.ttf,ItalicFont=cmunti.ttf,BoldItalicFont=cmunbi.ttf]{cmunrm.ttf}\setmonofont[Path=/usr/share/fonts/truetype/cmu/,UprightFont=cmuntt.ttf,BoldFont=cmuntb.ttf,ItalicFont=cmunit.ttf,BoldItalicFont=cmuntx.ttf]{cmunrm.ttf}) or 
\item{} \myhref{http://www.imagemagick.org/}{ImageMagick} ({\ttfamily \setmainfont[Path=/usr/share/fonts/truetype/cmu/,UprightFont=cmunrm.ttf,BoldFont=cmunbx.ttf,ItalicFont=cmunti.ttf,BoldItalicFont=cmunbi.ttf]{cmuntt.ttf}\setmonofont[Path=/usr/share/fonts/truetype/cmu/,UprightFont=cmuntt.ttf,BoldFont=cmuntb.ttf,ItalicFont=cmunit.ttf,BoldItalicFont=cmuntx.ttf]{cmuntt.ttf}\ttfamily convert}\setmainfont[Path=/usr/share/fonts/truetype/cmu/,UprightFont=cmunrm.ttf,BoldFont=cmunbx.ttf,ItalicFont=cmunti.ttf,BoldItalicFont=cmunbi.ttf]{cmunrm.ttf}\setmonofont[Path=/usr/share/fonts/truetype/cmu/,UprightFont=cmuntt.ttf,BoldFont=cmuntb.ttf,ItalicFont=cmunit.ttf,BoldItalicFont=cmuntx.ttf]{cmunrm.ttf}) or
\item{} \myhref{http://www.graphicsmagick.org/}{GraphicsMagick} ({\ttfamily \setmainfont[Path=/usr/share/fonts/truetype/cmu/,UprightFont=cmunrm.ttf,BoldFont=cmunbx.ttf,ItalicFont=cmunti.ttf,BoldItalicFont=cmunbi.ttf]{cmuntt.ttf}\setmonofont[Path=/usr/share/fonts/truetype/cmu/,UprightFont=cmuntt.ttf,BoldFont=cmuntb.ttf,ItalicFont=cmunit.ttf,BoldItalicFont=cmuntx.ttf]{cmuntt.ttf}\ttfamily gm convert}\setmainfont[Path=/usr/share/fonts/truetype/cmu/,UprightFont=cmunrm.ttf,BoldFont=cmunbx.ttf,ItalicFont=cmunti.ttf,BoldItalicFont=cmunbi.ttf]{cmunrm.ttf}\setmonofont[Path=/usr/share/fonts/truetype/cmu/,UprightFont=cmuntt.ttf,BoldFont=cmuntb.ttf,ItalicFont=cmunit.ttf,BoldItalicFont=cmuntx.ttf]{cmunrm.ttf}).                
\end{myitemize}


These three programs operate much the same way, and can convert between most graphics formats. Sam2p however is the most recent of the three and seems to offer both the best quality and to result in the smallest files.
\subsection{PNG alpha channel}
\label{353}
Acrobat Reader sometimes has problems with displaying colors correctly if you include graphics in PNG format with alpha channel. You can solve this problem by dropping the alpha channel. On Linux it can be achieved with {\ttfamily \setmainfont[Path=/usr/share/fonts/truetype/cmu/,UprightFont=cmunrm.ttf,BoldFont=cmunbx.ttf,ItalicFont=cmunti.ttf,BoldItalicFont=cmunbi.ttf]{cmuntt.ttf}\setmonofont[Path=/usr/share/fonts/truetype/cmu/,UprightFont=cmuntt.ttf,BoldFont=cmuntb.ttf,ItalicFont=cmunit.ttf,BoldItalicFont=cmuntx.ttf]{cmuntt.ttf}\ttfamily convert}{$\text{ }$}\setmainfont[Path=/usr/share/fonts/truetype/cmu/,UprightFont=cmunrm.ttf,BoldFont=cmunbx.ttf,ItalicFont=cmunti.ttf,BoldItalicFont=cmunbi.ttf]{cmunrm.ttf}\setmonofont[Path=/usr/share/fonts/truetype/cmu/,UprightFont=cmuntt.ttf,BoldFont=cmuntb.ttf,ItalicFont=cmunit.ttf,BoldItalicFont=cmuntx.ttf]{cmunrm.ttf} from the \myhref{https://en.wikipedia.org/wiki/ImageMagick}{ImageMagick} program:

\begin{Shaded}
\begin{Highlighting}[]

\KeywordTok{convert}\ensuremath{\text{ }}\NormalTok{-alpha\ensuremath{\text{ }}off\ensuremath{\text{ }}input.png\ensuremath{\text{ }}output.png}\newline
\end{Highlighting}
\end{Shaded}

\subsection{Converting a color EPS to grayscale}
\label{354}

Sometimes color EPS figures need to be converted to black-{}and-{}white or grayscale to meet publication requirements. This can be achieved with the \myhref{http://linuxcommand.org/man_pages/eps2eps1.html}{eps2eps} of the  \myhref{http://ghostscript.com/}{Ghostscript} package and \myplainurl{http://www.pa.op.dlr.de/~PatrickJoeckel/pscol/index.html} programs:

\begin{Shaded}
\begin{Highlighting}[]

\NormalTok{\$\ensuremath{\text{ }}}\KeywordTok{eps2eps}\ensuremath{\text{ }}\NormalTok{input.eps\ensuremath{\text{ }}input-e2.eps}\newline
\NormalTok{\$\ensuremath{\text{ }}}\KeywordTok{pscol}\ensuremath{\text{ }}\NormalTok{-0gray\ensuremath{\text{ }}input-e2.eps\ensuremath{\text{ }}input-gray.eps}\newline
\end{Highlighting}
\end{Shaded}

\section{Third-{}party graphics tools}
\label{355}

We will not tackle the topic of procedural graphics created from within LaTeX code here (TikZ, PSTricks, MetaPost and friends). See \mylref{774}{Introducing Procedural Graphics} for that.

You should prefer vector graphics over raster graphics for their quality. Raster graphics should only be used in case of photos. Diagrams of any sort should be vectors.

As we have seen before, LaTeX handles
\begin{myitemize}
\item{}  EPS and PDF for vector graphics;
\item{}  PNG and JPG for raster graphics.
\end{myitemize}


If some tools cannot save in those formats, you may want to \mylref{352}{convert} them before importing them.
\subsection{Vector graphics}
\label{356}

{\bfseries
\begin{mydescription}Dia
\end{mydescription}
}

\myhref{http://live.gnome.org/Dia}{Dia} is a cross platform diagramming utility which can export eps images, or generate tex drawn using the \LaTeXTT{tikz} package.
{\bfseries
\begin{mydescription}Inkscape
\end{mydescription}
}


Another program for creating vector graphics is \myhref{http://www.inkscape.org/}{Inkscape}. It can run natively under Windows, Linux or Mac OS X (with X11). It works with \myhref{http://www.w3.org/Graphics/SVG/}{Scalable Vector Graphics (SVG)} files, although it can export to many formats that can be included in \myhref{https://en.wikibooks.org/wiki/LaTeX}{LaTeX} files, such as EPS and PDF.
From version 0.48, there is a combined PDF/EPS/PS+LaTeX output option, similar to that offered by \myhref{https://en.wikipedia.org/wiki/Xfig}{Xfig}.
\myhref{http://mirrors.ctan.org/info/svg-inkscape/InkscapePDFLaTeX.pdf}{There are instructions} on how to save your vector images in a PDF format understood by LaTeX and have LaTeX manage the text styles and sizes in the image automatically.\myfootnote{\myhref{}{How to include an SVG image in LATEX}. mirrorcatalogs.com. Retrieved  }. Today there is the \LaTeXTT{svg package}\myfootnote{\myhref{}{The svg package on CTAN}. ctan.org. Retrieved  } which provides an \LaTeXTT{\textbackslash{}includesvg} command to convert and include svg-{}graphics directly in your LaTeX document using Inkscape. You may have a look at this \myhref{http://laclaro.wordpress.com/2013/09/09/updated-includesvg-example/}{extended example} too.

An extremely useful plug-{}in is \myhref{http://pav.iki.fi/software/textext/}{textext}, which can import LaTeX objects. This can be used for inserting mathematical notation or LaTeX fonts into graphics (which may then be imported into LaTeX documents).
{\bfseries
\begin{mydescription}Ipe
\end{mydescription}
}


The \myhref{https://en.wikipedia.org/wiki/Ipe\%20\%28program\%29}{Ipe} extensible drawing editor is a free vector graphics editor for creating figures in PDF or EPS format.
Unlike Xfig, Ipe represents \myhref{https://en.wikibooks.org/wiki/LaTeX}{LaTeX} fonts in their correct size on the screen which makes it easier to place text labels at the right spot.
Ipe also has various snapping modes (for example, snapping to points, lines, or intersections) that can be used for geometric constructions.
{\bfseries
\begin{mydescription}lpic
\end{mydescription}
}

Yet another solution is provided by the \LaTeXTT{lpic} packages \myplainurl{http://www.math.uni-leipzig.de/~matveyev/lpic/}, which allows TeX annotations to imported graphics. See \mylref{380}{Labels in the figures}.
{\bfseries
\begin{mydescription}OpenOffice.org
\end{mydescription}
}

It is also possible to export vector graphics to EPS format using \myhref{https://en.wikipedia.org/wiki/OpenOffice.org}{OpenOffice.org} Draw, which is an open source office suite available for Windows, Linux and Mac.
{\bfseries
\begin{mydescription}TpX
\end{mydescription}
}

Vector editor \myhref{http://tpx.sourceforge.net/}{TpX} separates geometric objects from text objects. Geometric objects are saved into .PDF file, the rest is saved in .TpX file to be processed by LaTeX. User just create the graphics in TpX editor and calls the .TpX file from latex file by command   \textbackslash{}input\{...TpX\}.
{\bfseries
\begin{mydescription}Xfig
\end{mydescription}
}

\myhref{https://en.wikipedia.org/wiki/Xfig}{Xfig} is a basic program that can produce vector graphics, which can be exported to LaTeX. It can be installed on Unix platforms.

On Microsoft Windows systems, Xfig can only be installed using \myhref{http://www.cygwin.com/}{Cygwin-{}X}; however, this will require a fast internet connection and about 2 gigabytes of space on your computer. With Cygwin, to run Xfig, you need to first start the \symbol{34}Start X -{} Server\symbol{34}, then launch \symbol{34}xterm\symbol{34} to bring up a terminal. In this terminal type \symbol{34}xfig\symbol{34} (without the quotation marks) and press return.

Alternatively, \myhref{http://www.schmidt-web-berlin.de/winfig/index.shtml}{WinFIG} is an attempt to achieve the functionality of xfig on Windows computers.

There are many ways to use xfig to create graphics for LaTeX documents. One method is to export the drawing as a LaTeX document. This method, however, suffers from various drawbacks: lines can be drawn only at angles that are multiples of 30 and 45 degrees, lines with arrows can only be drawn at angles that are multiples of 45 degrees, several curves are not supported, etc.

Exporting a file as PDF/LaTeX or PS/LaTeX, on the other hand, offers a good deal more flexibility in drawing. Here\textquotesingle{}s how it\textquotesingle{}s done:

\begin{myenumerate}
\item{} Create the drawing in xfig. Wherever you need LaTeX text, such as a mathematical formula, enter a LaTeX string in a textbox.
\item{} Use the Edit tool to open the properties of each of those textboxes, and change the option on the \symbol{34}Special Flag\symbol{34} field to Special. This tells LaTeX to interpret these textboxes when it opens the figure.
\item{} Go to File -{}>{} Export and export the file as PDF/LaTeX (both parts) or PS/LaTeX (both parts), depending on whether you are using pdflatex or pslatex to compile your file.
\item{} In your LaTeX document, where the picture should be, use the following, where \symbol{34}test\symbol{34} is replaced by the name of the image:

\begin{Shaded}
\begin{Highlighting}[]

\NormalTok{\textbackslash{}begin\{figure\}[htbp]}
 \NormalTok{\textbackslash{}centering}
 \NormalTok{\textbackslash{}input\{test.pdf_t\}}
 \NormalTok{\textbackslash{}caption\{Your figure\}}
 \NormalTok{\textbackslash{}label\{figure:example\}}
\NormalTok{\textbackslash{}end\{figure\}}
\end{Highlighting}
\end{Shaded}


Observe that this is just like including a picture, except that rather than using {\ttfamily \setmainfont[Path=/usr/share/fonts/truetype/cmu/,UprightFont=cmunrm.ttf,BoldFont=cmunbx.ttf,ItalicFont=cmunti.ttf,BoldItalicFont=cmunbi.ttf]{cmuntt.ttf}\setmonofont[Path=/usr/share/fonts/truetype/cmu/,UprightFont=cmuntt.ttf,BoldFont=cmuntb.ttf,ItalicFont=cmunit.ttf,BoldItalicFont=cmuntx.ttf]{cmuntt.ttf}\ttfamily \textbackslash{}includegraphics}\setmainfont[Path=/usr/share/fonts/truetype/cmu/,UprightFont=cmunrm.ttf,BoldFont=cmunbx.ttf,ItalicFont=cmunti.ttf,BoldItalicFont=cmunbi.ttf]{cmunrm.ttf}\setmonofont[Path=/usr/share/fonts/truetype/cmu/,UprightFont=cmuntt.ttf,BoldFont=cmuntb.ttf,ItalicFont=cmunit.ttf,BoldItalicFont=cmuntx.ttf]{cmunrm.ttf}, we use {\ttfamily \setmainfont[Path=/usr/share/fonts/truetype/cmu/,UprightFont=cmunrm.ttf,BoldFont=cmunbx.ttf,ItalicFont=cmunti.ttf,BoldItalicFont=cmunbi.ttf]{cmuntt.ttf}\setmonofont[Path=/usr/share/fonts/truetype/cmu/,UprightFont=cmuntt.ttf,BoldFont=cmuntb.ttf,ItalicFont=cmunit.ttf,BoldItalicFont=cmuntx.ttf]{cmuntt.ttf}\ttfamily \textbackslash{}input}\setmainfont[Path=/usr/share/fonts/truetype/cmu/,UprightFont=cmunrm.ttf,BoldFont=cmunbx.ttf,ItalicFont=cmunti.ttf,BoldItalicFont=cmunbi.ttf]{cmunrm.ttf}\setmonofont[Path=/usr/share/fonts/truetype/cmu/,UprightFont=cmuntt.ttf,BoldFont=cmuntb.ttf,ItalicFont=cmunit.ttf,BoldItalicFont=cmuntx.ttf]{cmunrm.ttf}. If the export was into PS/LaTeX, the file extension to include would be .pstex\_t instead of .pdf\_t.
\item{} Make sure to include packages \LaTeXTT{graphicx} and \LaTeXTT{color} in the file, with the \LaTeXTT{\textbackslash{}usepackage} command right below the \LaTeXTT{\textbackslash{}documentclass} command, like this:
\begin{Shaded}
\begin{Highlighting}[]

\NormalTok{\textbackslash{}usepackage\{graphicx\}}
\NormalTok{\textbackslash{}usepackage\{color\}}
\end{Highlighting}
\end{Shaded}



\end{myenumerate}
\item{}\item{}\item{}\item{}\item{}
And you\textquotesingle{}re done!

For more details on using xfig with LaTeX, \myhref{http://www-epb.lbl.gov/xfig/latex_and_xfig.html}{this chapter} of the \myhref{http://www-epb.lbl.gov/xfig/contents.html}{xfig User Manual} may prove helpful.
{\bfseries
\begin{mydescription}Other tools
\end{mydescription}
}

Commercial vector graphics software, such as Adobe Illustrator, CorelDRAW, and FreeHand are commonly used and can {\itshape \setmainfont[Path=/usr/share/fonts/truetype/cmu/,UprightFont=cmunrm.ttf,BoldFont=cmunbx.ttf,ItalicFont=cmunti.ttf,BoldItalicFont=cmunbi.ttf]{cmunti.ttf}\setmonofont[Path=/usr/share/fonts/truetype/cmu/,UprightFont=cmuntt.ttf,BoldFont=cmuntb.ttf,ItalicFont=cmunit.ttf,BoldItalicFont=cmuntx.ttf]{cmunti.ttf}\itshape read}{$\text{ }$}\setmainfont[Path=/usr/share/fonts/truetype/cmu/,UprightFont=cmunrm.ttf,BoldFont=cmunbx.ttf,ItalicFont=cmunti.ttf,BoldItalicFont=cmunbi.ttf]{cmunrm.ttf}\setmonofont[Path=/usr/share/fonts/truetype/cmu/,UprightFont=cmuntt.ttf,BoldFont=cmuntb.ttf,ItalicFont=cmunit.ttf,BoldItalicFont=cmuntx.ttf]{cmunrm.ttf} and {\itshape \setmainfont[Path=/usr/share/fonts/truetype/cmu/,UprightFont=cmunrm.ttf,BoldFont=cmunbx.ttf,ItalicFont=cmunti.ttf,BoldItalicFont=cmunbi.ttf]{cmunti.ttf}\setmonofont[Path=/usr/share/fonts/truetype/cmu/,UprightFont=cmuntt.ttf,BoldFont=cmuntb.ttf,ItalicFont=cmunit.ttf,BoldItalicFont=cmuntx.ttf]{cmunti.ttf}\itshape write}{$\text{ }$}\setmainfont[Path=/usr/share/fonts/truetype/cmu/,UprightFont=cmunrm.ttf,BoldFont=cmunbx.ttf,ItalicFont=cmunti.ttf,BoldItalicFont=cmunbi.ttf]{cmunrm.ttf}\setmonofont[Path=/usr/share/fonts/truetype/cmu/,UprightFont=cmuntt.ttf,BoldFont=cmuntb.ttf,ItalicFont=cmunit.ttf,BoldItalicFont=cmuntx.ttf]{cmunrm.ttf} EPS figures. However, these products are limited to Windows and Mac OS X platforms.
\subsection{Raster graphics}
\label{357}{\bfseries
\begin{mydescription}Adobe Photoshop
\end{mydescription}
}

It can save to EPS.
{\bfseries
\begin{mydescription}GIMP
\end{mydescription}
}

\myhref{http://www.gimp.org}{GIMP}, has a graphical user interface, and it is multi-{}platform. It can save to EPS and PDF.
\subsection{Plots and Charts}
\label{358}
{\bfseries
\begin{mydescription}Generic Mapping Tools (GMT)
\end{mydescription}
}

\myhref{http://gmt.soest.hawaii.edu/}{Generic Mapping Tools (GMT)}, maps and a wide range of highly customisable plots.
{\bfseries
\begin{mydescription}Gnumeric
\end{mydescription}
}

\myhref{http://projects.gnome.org/gnumeric/}{Gnumeric}, spreadsheets has SVG, EPS, PDF export
{\bfseries
\begin{mydescription}Gnuplot
\end{mydescription}
}

\myhref{http://www.gnuplot.info}{Gnuplot}, producing scientific graphics since 1986. If you want to make mathematical plots, then \myhref{http://www.gnuplot.info}{Gnuplot} can save in any format. You can get best results when used along \mylref{793}{PGF/TikZ}.
{\bfseries
\begin{mydescription}matplotlib
\end{mydescription}
}

\myhref{http://matplotlib.sourceforge.net/}{matplotlib}, plotting library written in python, with PDF and EPS export. On the other hand there is a PGF export also.
There are some tricks to be able to import formats other than EPS into your DVI document, but they\textquotesingle{}re very complicated. On the other hand, converting any image to EPS is very simple, so it\textquotesingle{}s not worth considering them.
{\bfseries
\begin{mydescription}R
\end{mydescription}
}

\myhref{http://www.r-project.org/}{R}, statistical and scientific figures.
\subsection{Editing EPS graphics}
\label{359}

As described above, graphics content can be imported into \myhref{https://en.wikibooks.org/wiki/LaTeX}{LaTeX} from outside programs as EPS files. But sometimes you want to edit or retouch these graphics files. An EPS file can be edited with any text editor since it is formatted as ASCII. In a text editor, you can achieve simple operations like replacing strings or moving items slightly, but anything further becomes cumbersome.  Vector graphics editors, like Inkscape, may also be able to import EPS files for subsequent editing.  This approach also for easier editing.  However, the importing process may occassionally modify the original EPS image.
\section{Notes and References}
\label{360}
\LaTeXNullTemplate{}

\chapter{Floats, Figures and Captions}

\myminitoc
\label{361}

\label{362}


The \mylref{336}{previous chapter} introduced importing graphics. However, just having a picture stuck in between paragraphs does not look professional. To start with, we want a way of adding captions, and to be able to cross-{}reference. What we need is a way of defining {\itshape \setmainfont[Path=/usr/share/fonts/truetype/cmu/,UprightFont=cmunrm.ttf,BoldFont=cmunbx.ttf,ItalicFont=cmunti.ttf,BoldItalicFont=cmunbi.ttf]{cmunti.ttf}\setmonofont[Path=/usr/share/fonts/truetype/cmu/,UprightFont=cmuntt.ttf,BoldFont=cmuntb.ttf,ItalicFont=cmunit.ttf,BoldItalicFont=cmuntx.ttf]{cmunti.ttf}\itshape figures}\setmainfont[Path=/usr/share/fonts/truetype/cmu/,UprightFont=cmunrm.ttf,BoldFont=cmunbx.ttf,ItalicFont=cmunti.ttf,BoldItalicFont=cmunbi.ttf]{cmunrm.ttf}\setmonofont[Path=/usr/share/fonts/truetype/cmu/,UprightFont=cmuntt.ttf,BoldFont=cmuntb.ttf,ItalicFont=cmunit.ttf,BoldItalicFont=cmuntx.ttf]{cmunrm.ttf}. It would also be good if LaTeX could apply principles similar to when it arranges text to look its best to arranging pictures as well. This is where {\itshape \setmainfont[Path=/usr/share/fonts/truetype/cmu/,UprightFont=cmunrm.ttf,BoldFont=cmunbx.ttf,ItalicFont=cmunti.ttf,BoldItalicFont=cmunbi.ttf]{cmunti.ttf}\setmonofont[Path=/usr/share/fonts/truetype/cmu/,UprightFont=cmuntt.ttf,BoldFont=cmuntb.ttf,ItalicFont=cmunit.ttf,BoldItalicFont=cmuntx.ttf]{cmunti.ttf}\itshape floats}{$\text{ }$}\setmainfont[Path=/usr/share/fonts/truetype/cmu/,UprightFont=cmunrm.ttf,BoldFont=cmunbx.ttf,ItalicFont=cmunti.ttf,BoldItalicFont=cmunbi.ttf]{cmunrm.ttf}\setmonofont[Path=/usr/share/fonts/truetype/cmu/,UprightFont=cmuntt.ttf,BoldFont=cmuntb.ttf,ItalicFont=cmunit.ttf,BoldItalicFont=cmuntx.ttf]{cmunrm.ttf} come into play.
\section{Floats}
\label{363}

Floats are containers for things in a document that cannot be broken over a page. LaTeX by default recognizes \symbol{34}table\symbol{34} and \symbol{34}figure\symbol{34} floats, but you can define new ones of your own (see \mylref{378}{Custom floats} below).  Floats are there to deal with the problem of the object that won\textquotesingle{}t fit on the present page, and to help when you really don\textquotesingle{}t want the object here just now.

Floats are not part of the normal stream of text, but separate entities, positioned in a part of the page to themselves (top, middle, bottom, left, right, or wherever the designer specifies). They always have a caption describing them and they are always numbered so they can be referred to from elsewhere in the text. LaTeX automatically floats Tables and Figures, depending on how much space is left on the page at the point that they are processed. If there is not enough room on the current page, the float is moved to the top of the next page. This can be changed by moving the Table or Figure definition to an earlier or later point in the text, or by adjusting some of the parameters which control automatic floating.

Authors sometimes have many floats occurring in rapid succession, which raises the problem of how they are supposed to fit on the page and still leave room for text. In this case, LaTeX stacks them all up and prints them together if possible, or leaves them to the end of the chapter in protest. The skill is to space them out within your text so that they intrude neither on the thread of your argument or discussion, nor on the visual balance of the typeset pages.

As with various other entities, there exist limitations on the number of floats. LaTeX by default can cope with maximum 18 floats and a symptomatic error is:

\TemplatePreformat{$\text{ }$\newline{}
!$\text{ }${}LaTeX$\text{ }${}Error:$\text{ }${}Too$\text{ }${}many$\text{ }${}unprocessed$\text{ }${}floats.$\text{ }$\newline{}
}

The \LaTeXTT{morefloats} package lifts such limit.
\subsection{Figures}
\label{364}

To create a figure that floats, use the \LaTeXTT{figure} environment.

\begin{Shaded}
\begin{Highlighting}[]

\NormalTok{\textbackslash{}begin\{figure\}[placement specifier]}
\NormalTok{... figure contents ...}
\NormalTok{\textbackslash{}end\{figure\}}
\end{Highlighting}
\end{Shaded}


The previous section mentioned how floats are used to allow LaTeX to handle figures, while maintaining the best possible presentation. However, there may be times when you disagree, and a typical example is with its positioning of figures. The {\itshape \setmainfont[Path=/usr/share/fonts/truetype/cmu/,UprightFont=cmunrm.ttf,BoldFont=cmunbx.ttf,ItalicFont=cmunti.ttf,BoldItalicFont=cmunbi.ttf]{cmunti.ttf}\setmonofont[Path=/usr/share/fonts/truetype/cmu/,UprightFont=cmuntt.ttf,BoldFont=cmuntb.ttf,ItalicFont=cmunit.ttf,BoldItalicFont=cmuntx.ttf]{cmunti.ttf}\itshape placement specifier}{$\text{ }$}\setmainfont[Path=/usr/share/fonts/truetype/cmu/,UprightFont=cmunrm.ttf,BoldFont=cmunbx.ttf,ItalicFont=cmunti.ttf,BoldItalicFont=cmunbi.ttf]{cmunrm.ttf}\setmonofont[Path=/usr/share/fonts/truetype/cmu/,UprightFont=cmuntt.ttf,BoldFont=cmuntb.ttf,ItalicFont=cmunit.ttf,BoldItalicFont=cmuntx.ttf]{cmunrm.ttf} parameter exists as a compromise, and its purpose is to give the author a greater degree of control over where certain floats are placed.

\LaTeXNullTemplate{}
\begin{longtable}{|>{\RaggedRight}p{0.11572\linewidth}|>{\RaggedRight}p{0.82714\linewidth}|} \hline 
{\bfseries \hspace*{0pt}\ignorespaces{}\hspace*{0pt} Specifier}&{\bfseries \hspace*{0pt}\ignorespaces{}\hspace*{0pt} Permission}\endhead  \hline \hspace*{0pt}\ignorespaces{}\hspace*{0pt} {\ttfamily \setmainfont[Path=/usr/share/fonts/truetype/cmu/,UprightFont=cmunrm.ttf,BoldFont=cmunbx.ttf,ItalicFont=cmunti.ttf,BoldItalicFont=cmunbi.ttf]{cmuntt.ttf}\setmonofont[Path=/usr/share/fonts/truetype/cmu/,UprightFont=cmuntt.ttf,BoldFont=cmuntb.ttf,ItalicFont=cmunit.ttf,BoldItalicFont=cmuntx.ttf]{cmuntt.ttf}\ttfamily h}&\hspace*{0pt}\ignorespaces{}\hspace*{0pt}{$\text{ }$}\setmainfont[Path=/usr/share/fonts/truetype/cmu/,UprightFont=cmunrm.ttf,BoldFont=cmunbx.ttf,ItalicFont=cmunti.ttf,BoldItalicFont=cmunbi.ttf]{cmunrm.ttf}\setmonofont[Path=/usr/share/fonts/truetype/cmu/,UprightFont=cmuntt.ttf,BoldFont=cmuntb.ttf,ItalicFont=cmunit.ttf,BoldItalicFont=cmuntx.ttf]{cmunrm.ttf} Place the float {\itshape \setmainfont[Path=/usr/share/fonts/truetype/cmu/,UprightFont=cmunrm.ttf,BoldFont=cmunbx.ttf,ItalicFont=cmunti.ttf,BoldItalicFont=cmunbi.ttf]{cmunti.ttf}\setmonofont[Path=/usr/share/fonts/truetype/cmu/,UprightFont=cmuntt.ttf,BoldFont=cmuntb.ttf,ItalicFont=cmunit.ttf,BoldItalicFont=cmuntx.ttf]{cmunti.ttf}\itshape here}\setmainfont[Path=/usr/share/fonts/truetype/cmu/,UprightFont=cmunrm.ttf,BoldFont=cmunbx.ttf,ItalicFont=cmunti.ttf,BoldItalicFont=cmunbi.ttf]{cmunrm.ttf}\setmonofont[Path=/usr/share/fonts/truetype/cmu/,UprightFont=cmuntt.ttf,BoldFont=cmuntb.ttf,ItalicFont=cmunit.ttf,BoldItalicFont=cmuntx.ttf]{cmunrm.ttf}, i.e., {\itshape \setmainfont[Path=/usr/share/fonts/truetype/cmu/,UprightFont=cmunrm.ttf,BoldFont=cmunbx.ttf,ItalicFont=cmunti.ttf,BoldItalicFont=cmunbi.ttf]{cmunti.ttf}\setmonofont[Path=/usr/share/fonts/truetype/cmu/,UprightFont=cmuntt.ttf,BoldFont=cmuntb.ttf,ItalicFont=cmunit.ttf,BoldItalicFont=cmuntx.ttf]{cmunti.ttf}\itshape approximately}{$\text{ }$}\setmainfont[Path=/usr/share/fonts/truetype/cmu/,UprightFont=cmunrm.ttf,BoldFont=cmunbx.ttf,ItalicFont=cmunti.ttf,BoldItalicFont=cmunbi.ttf]{cmunrm.ttf}\setmonofont[Path=/usr/share/fonts/truetype/cmu/,UprightFont=cmuntt.ttf,BoldFont=cmuntb.ttf,ItalicFont=cmunit.ttf,BoldItalicFont=cmuntx.ttf]{cmunrm.ttf} at the same point it occurs in the source text (however, not {\itshape \setmainfont[Path=/usr/share/fonts/truetype/cmu/,UprightFont=cmunrm.ttf,BoldFont=cmunbx.ttf,ItalicFont=cmunti.ttf,BoldItalicFont=cmunbi.ttf]{cmunti.ttf}\setmonofont[Path=/usr/share/fonts/truetype/cmu/,UprightFont=cmuntt.ttf,BoldFont=cmuntb.ttf,ItalicFont=cmunit.ttf,BoldItalicFont=cmuntx.ttf]{cmunti.ttf}\itshape exactly}{$\text{ }$}\setmainfont[Path=/usr/share/fonts/truetype/cmu/,UprightFont=cmunrm.ttf,BoldFont=cmunbx.ttf,ItalicFont=cmunti.ttf,BoldItalicFont=cmunbi.ttf]{cmunrm.ttf}\setmonofont[Path=/usr/share/fonts/truetype/cmu/,UprightFont=cmuntt.ttf,BoldFont=cmuntb.ttf,ItalicFont=cmunit.ttf,BoldItalicFont=cmuntx.ttf]{cmunrm.ttf} at the spot)\\ \hline \hspace*{0pt}\ignorespaces{}\hspace*{0pt} {\ttfamily \setmainfont[Path=/usr/share/fonts/truetype/cmu/,UprightFont=cmunrm.ttf,BoldFont=cmunbx.ttf,ItalicFont=cmunti.ttf,BoldItalicFont=cmunbi.ttf]{cmuntt.ttf}\setmonofont[Path=/usr/share/fonts/truetype/cmu/,UprightFont=cmuntt.ttf,BoldFont=cmuntb.ttf,ItalicFont=cmunit.ttf,BoldItalicFont=cmuntx.ttf]{cmuntt.ttf}\ttfamily t}&\hspace*{0pt}\ignorespaces{}\hspace*{0pt}{$\text{ }$}\setmainfont[Path=/usr/share/fonts/truetype/cmu/,UprightFont=cmunrm.ttf,BoldFont=cmunbx.ttf,ItalicFont=cmunti.ttf,BoldItalicFont=cmunbi.ttf]{cmunrm.ttf}\setmonofont[Path=/usr/share/fonts/truetype/cmu/,UprightFont=cmuntt.ttf,BoldFont=cmuntb.ttf,ItalicFont=cmunit.ttf,BoldItalicFont=cmuntx.ttf]{cmunrm.ttf} Position at the {\itshape \setmainfont[Path=/usr/share/fonts/truetype/cmu/,UprightFont=cmunrm.ttf,BoldFont=cmunbx.ttf,ItalicFont=cmunti.ttf,BoldItalicFont=cmunbi.ttf]{cmunti.ttf}\setmonofont[Path=/usr/share/fonts/truetype/cmu/,UprightFont=cmuntt.ttf,BoldFont=cmuntb.ttf,ItalicFont=cmunit.ttf,BoldItalicFont=cmuntx.ttf]{cmunti.ttf}\itshape top}{$\text{ }$}\setmainfont[Path=/usr/share/fonts/truetype/cmu/,UprightFont=cmunrm.ttf,BoldFont=cmunbx.ttf,ItalicFont=cmunti.ttf,BoldItalicFont=cmunbi.ttf]{cmunrm.ttf}\setmonofont[Path=/usr/share/fonts/truetype/cmu/,UprightFont=cmuntt.ttf,BoldFont=cmuntb.ttf,ItalicFont=cmunit.ttf,BoldItalicFont=cmuntx.ttf]{cmunrm.ttf} of the page.\\ \hline \hspace*{0pt}\ignorespaces{}\hspace*{0pt} {\ttfamily \setmainfont[Path=/usr/share/fonts/truetype/cmu/,UprightFont=cmunrm.ttf,BoldFont=cmunbx.ttf,ItalicFont=cmunti.ttf,BoldItalicFont=cmunbi.ttf]{cmuntt.ttf}\setmonofont[Path=/usr/share/fonts/truetype/cmu/,UprightFont=cmuntt.ttf,BoldFont=cmuntb.ttf,ItalicFont=cmunit.ttf,BoldItalicFont=cmuntx.ttf]{cmuntt.ttf}\ttfamily b}&\hspace*{0pt}\ignorespaces{}\hspace*{0pt}{$\text{ }$}\setmainfont[Path=/usr/share/fonts/truetype/cmu/,UprightFont=cmunrm.ttf,BoldFont=cmunbx.ttf,ItalicFont=cmunti.ttf,BoldItalicFont=cmunbi.ttf]{cmunrm.ttf}\setmonofont[Path=/usr/share/fonts/truetype/cmu/,UprightFont=cmuntt.ttf,BoldFont=cmuntb.ttf,ItalicFont=cmunit.ttf,BoldItalicFont=cmuntx.ttf]{cmunrm.ttf} Position at the {\itshape \setmainfont[Path=/usr/share/fonts/truetype/cmu/,UprightFont=cmunrm.ttf,BoldFont=cmunbx.ttf,ItalicFont=cmunti.ttf,BoldItalicFont=cmunbi.ttf]{cmunti.ttf}\setmonofont[Path=/usr/share/fonts/truetype/cmu/,UprightFont=cmuntt.ttf,BoldFont=cmuntb.ttf,ItalicFont=cmunit.ttf,BoldItalicFont=cmuntx.ttf]{cmunti.ttf}\itshape bottom}{$\text{ }$}\setmainfont[Path=/usr/share/fonts/truetype/cmu/,UprightFont=cmunrm.ttf,BoldFont=cmunbx.ttf,ItalicFont=cmunti.ttf,BoldItalicFont=cmunbi.ttf]{cmunrm.ttf}\setmonofont[Path=/usr/share/fonts/truetype/cmu/,UprightFont=cmuntt.ttf,BoldFont=cmuntb.ttf,ItalicFont=cmunit.ttf,BoldItalicFont=cmuntx.ttf]{cmunrm.ttf} of the page.\\ \hline \hspace*{0pt}\ignorespaces{}\hspace*{0pt} {\ttfamily \setmainfont[Path=/usr/share/fonts/truetype/cmu/,UprightFont=cmunrm.ttf,BoldFont=cmunbx.ttf,ItalicFont=cmunti.ttf,BoldItalicFont=cmunbi.ttf]{cmuntt.ttf}\setmonofont[Path=/usr/share/fonts/truetype/cmu/,UprightFont=cmuntt.ttf,BoldFont=cmuntb.ttf,ItalicFont=cmunit.ttf,BoldItalicFont=cmuntx.ttf]{cmuntt.ttf}\ttfamily p}&\hspace*{0pt}\ignorespaces{}\hspace*{0pt}{$\text{ }$}\setmainfont[Path=/usr/share/fonts/truetype/cmu/,UprightFont=cmunrm.ttf,BoldFont=cmunbx.ttf,ItalicFont=cmunti.ttf,BoldItalicFont=cmunbi.ttf]{cmunrm.ttf}\setmonofont[Path=/usr/share/fonts/truetype/cmu/,UprightFont=cmuntt.ttf,BoldFont=cmuntb.ttf,ItalicFont=cmunit.ttf,BoldItalicFont=cmuntx.ttf]{cmunrm.ttf} Put on a special {\itshape \setmainfont[Path=/usr/share/fonts/truetype/cmu/,UprightFont=cmunrm.ttf,BoldFont=cmunbx.ttf,ItalicFont=cmunti.ttf,BoldItalicFont=cmunbi.ttf]{cmunti.ttf}\setmonofont[Path=/usr/share/fonts/truetype/cmu/,UprightFont=cmuntt.ttf,BoldFont=cmuntb.ttf,ItalicFont=cmunit.ttf,BoldItalicFont=cmuntx.ttf]{cmunti.ttf}\itshape page}{$\text{ }$}\setmainfont[Path=/usr/share/fonts/truetype/cmu/,UprightFont=cmunrm.ttf,BoldFont=cmunbx.ttf,ItalicFont=cmunti.ttf,BoldItalicFont=cmunbi.ttf]{cmunrm.ttf}\setmonofont[Path=/usr/share/fonts/truetype/cmu/,UprightFont=cmuntt.ttf,BoldFont=cmuntb.ttf,ItalicFont=cmunit.ttf,BoldItalicFont=cmuntx.ttf]{cmunrm.ttf} for floats only.\\ \hline \hspace*{0pt}\ignorespaces{}\hspace*{0pt} {\ttfamily \setmainfont[Path=/usr/share/fonts/truetype/cmu/,UprightFont=cmunrm.ttf,BoldFont=cmunbx.ttf,ItalicFont=cmunti.ttf,BoldItalicFont=cmunbi.ttf]{cmuntt.ttf}\setmonofont[Path=/usr/share/fonts/truetype/cmu/,UprightFont=cmuntt.ttf,BoldFont=cmuntb.ttf,ItalicFont=cmunit.ttf,BoldItalicFont=cmuntx.ttf]{cmuntt.ttf}\ttfamily !}&\hspace*{0pt}\ignorespaces{}\hspace*{0pt}{$\text{ }$}\setmainfont[Path=/usr/share/fonts/truetype/cmu/,UprightFont=cmunrm.ttf,BoldFont=cmunbx.ttf,ItalicFont=cmunti.ttf,BoldItalicFont=cmunbi.ttf]{cmunrm.ttf}\setmonofont[Path=/usr/share/fonts/truetype/cmu/,UprightFont=cmuntt.ttf,BoldFont=cmuntb.ttf,ItalicFont=cmunit.ttf,BoldItalicFont=cmuntx.ttf]{cmunrm.ttf} Override internal parameters LaTeX uses for determining \symbol{34}good\symbol{34} float positions.\\ \hline \hspace*{0pt}\ignorespaces{}\hspace*{0pt} {\ttfamily \setmainfont[Path=/usr/share/fonts/truetype/cmu/,UprightFont=cmunrm.ttf,BoldFont=cmunbx.ttf,ItalicFont=cmunti.ttf,BoldItalicFont=cmunbi.ttf]{cmuntt.ttf}\setmonofont[Path=/usr/share/fonts/truetype/cmu/,UprightFont=cmuntt.ttf,BoldFont=cmuntb.ttf,ItalicFont=cmunit.ttf,BoldItalicFont=cmuntx.ttf]{cmuntt.ttf}\ttfamily H}&\hspace*{0pt}\ignorespaces{}\hspace*{0pt}{$\text{ }$}\setmainfont[Path=/usr/share/fonts/truetype/cmu/,UprightFont=cmunrm.ttf,BoldFont=cmunbx.ttf,ItalicFont=cmunti.ttf,BoldItalicFont=cmunbi.ttf]{cmunrm.ttf}\setmonofont[Path=/usr/share/fonts/truetype/cmu/,UprightFont=cmuntt.ttf,BoldFont=cmuntb.ttf,ItalicFont=cmunit.ttf,BoldItalicFont=cmuntx.ttf]{cmunrm.ttf} Places the float at precisely the location in the LaTeX code. Requires the \LaTeXTT{float} package,\myfootnote{\myplainurl{http://www.ctan.org/tex-archive/macros/latex/contrib/float/}} i.e., \LaTeXTT{\textbackslash{}usepackage\{float\}}. This is somewhat equivalent to \LaTeXTT{!ht}.\\ \hline 
\end{longtable}


What you do with these {\itshape \setmainfont[Path=/usr/share/fonts/truetype/cmu/,UprightFont=cmunrm.ttf,BoldFont=cmunbx.ttf,ItalicFont=cmunti.ttf,BoldItalicFont=cmunbi.ttf]{cmunti.ttf}\setmonofont[Path=/usr/share/fonts/truetype/cmu/,UprightFont=cmuntt.ttf,BoldFont=cmuntb.ttf,ItalicFont=cmunit.ttf,BoldItalicFont=cmuntx.ttf]{cmunti.ttf}\itshape placement permissions}{$\text{ }$}\setmainfont[Path=/usr/share/fonts/truetype/cmu/,UprightFont=cmunrm.ttf,BoldFont=cmunbx.ttf,ItalicFont=cmunti.ttf,BoldItalicFont=cmunbi.ttf]{cmunrm.ttf}\setmonofont[Path=/usr/share/fonts/truetype/cmu/,UprightFont=cmuntt.ttf,BoldFont=cmuntb.ttf,ItalicFont=cmunit.ttf,BoldItalicFont=cmuntx.ttf]{cmunrm.ttf} is to list which of the options you wish to make available to LaTeX. These are simply possibilities, and LaTeX will decide when typesetting your document which of your supplied specifiers it thinks is best. Frank Mittelbach describes the algorithm\myfootnote{\myfnhref{http://tex.stackexchange.com/a/39020}{Float environment positioning, by Frank Mittelbach}}:

\begin{myitemize}
\item{} If a float is encountered, LaTeX attempts to place it immediately according to its rules (detailed later)
\begin{myitemize}
\item{} if this succeeds, the float is placed and that decision is never changed;
\item{} if this does not succeed, then LaTeX places the float into a holding queue to be reconsidered when the next page is started (but not earlier).
\end{myitemize}

\item{} Once a page has finished, LaTeX examines this holding queue and tries to empty it as best as possible. For this it will first try to generate as many float pages as possible (in the hope of getting floats off the queue). Once this possibility is exhausted, it will next try to place the remaining floats into top and bottom areas. It looks at all the remaining floats and either places them or defers them to a later page (i.e., re-{}adding them to the holding queue once more).
\item{} After that, it starts processing document material for this page. In the process, it may encounter further floats.
\item{} If the end of the document has been reached or if a \textbackslash{}clearpage is encountered, LaTeX starts a new page, relaxes all restrictive float conditions, and outputs all floats in the holding queue by placing them on float page(s).
\end{myitemize}


In some special cases LaTeX won\textquotesingle{}t follow these positioning parameters and additional commands will be necessary, for example, if one needs to specify an alignment other than centered for a float that sits alone in one page\myfootnote{\myplainurl{http://tex.stackexchange.com/questions/28556/how-to-place-a-float-at-the-top-of-a-floats-only-page}}.

Use \LaTeXTT{\textbackslash{}listoffigures} to add a list of the figures in the beginning of the document.
To change the name used in the caption from {\bfseries \setmainfont[Path=/usr/share/fonts/truetype/cmu/,UprightFont=cmunrm.ttf,BoldFont=cmunbx.ttf,ItalicFont=cmunti.ttf,BoldItalicFont=cmunbi.ttf]{cmunbx.ttf}\setmonofont[Path=/usr/share/fonts/truetype/cmu/,UprightFont=cmuntt.ttf,BoldFont=cmuntb.ttf,ItalicFont=cmunit.ttf,BoldItalicFont=cmuntx.ttf]{cmunbx.ttf}\bfseries Figure}{$\text{ }$}\setmainfont[Path=/usr/share/fonts/truetype/cmu/,UprightFont=cmunrm.ttf,BoldFont=cmunbx.ttf,ItalicFont=cmunti.ttf,BoldItalicFont=cmunbi.ttf]{cmunrm.ttf}\setmonofont[Path=/usr/share/fonts/truetype/cmu/,UprightFont=cmuntt.ttf,BoldFont=cmuntb.ttf,ItalicFont=cmunit.ttf,BoldItalicFont=cmuntx.ttf]{cmunrm.ttf} to {\bfseries \setmainfont[Path=/usr/share/fonts/truetype/cmu/,UprightFont=cmunrm.ttf,BoldFont=cmunbx.ttf,ItalicFont=cmunti.ttf,BoldItalicFont=cmunbi.ttf]{cmunbx.ttf}\setmonofont[Path=/usr/share/fonts/truetype/cmu/,UprightFont=cmuntt.ttf,BoldFont=cmuntb.ttf,ItalicFont=cmunit.ttf,BoldItalicFont=cmuntx.ttf]{cmunbx.ttf}\bfseries Example}\setmainfont[Path=/usr/share/fonts/truetype/cmu/,UprightFont=cmunrm.ttf,BoldFont=cmunbx.ttf,ItalicFont=cmunti.ttf,BoldItalicFont=cmunbi.ttf]{cmunrm.ttf}\setmonofont[Path=/usr/share/fonts/truetype/cmu/,UprightFont=cmuntt.ttf,BoldFont=cmuntb.ttf,ItalicFont=cmunit.ttf,BoldItalicFont=cmuntx.ttf]{cmunrm.ttf},
use \LaTeXTT{\textbackslash{}renewcommand\{\textbackslash{}figurename\}\{Example\}} in the figure contents.
\subsubsection{Figures with borders}
\label{365}

It\textquotesingle{}s possible to get a thin border around all figures. You have to write the following once at the beginning of
the document:
\begin{Shaded}
\begin{Highlighting}[]

\NormalTok{\textbackslash{}usepackage\{float\}}
\NormalTok{\textbackslash{}floatstyle\{boxed\} }
\NormalTok{\textbackslash{}restylefloat\{figure\}}
\end{Highlighting}
\end{Shaded}

The border will not include the caption.
\subsection{Tables}
\label{366}

Floating tables are covered in a \mylref{272}{separate chapter}.
Let\textquotesingle{}s give a quick reminder here. The \LaTeXTT{tabular} environment that was used to construct the tables is not a float by default. Therefore, for tables you wish to float, wrap the \LaTeXTT{tabular} environment within a \LaTeXTT{table} environment, like this:

\begin{Shaded}
\begin{Highlighting}[]

\NormalTok{\textbackslash{}begin\{table\}}
  \NormalTok{\textbackslash{}begin\{tabular\}\{...\}}
  \NormalTok{... table data ...}
  \NormalTok{\textbackslash{}end\{tabular\}}
\NormalTok{\textbackslash{}end\{table\}}
\end{Highlighting}
\end{Shaded}


You may feel that it is a bit long winded, but such distinctions are necessary, because you may not want all tables to be treated as a float.

Use \LaTeXTT{\textbackslash{}listoftables} to add a list of the tables in the beginning of the document.
\section{Keeping floats in their place}
\label{367}

The \LaTeXTT{placeins}\myplainurl{http://www.ctan.org/pkg/placeins} package provides the command \LaTeXTT{\textbackslash{}FloatBarrier}, which can be used to prevent floats from being moved over it. This can, e.g., be useful at the beginning of each section. The package even provides an option to change the definition of \LaTeXTT{\textbackslash{}section} to automatically include a \LaTeXTT{\textbackslash{}FloatBarrier}. This can be set by loading the package with the option \LaTeXTT{{$\text{[}$}section{$\text{]}$}} (\LaTeXTT{\textbackslash{}usepackage{$\text{[}$}section{$\text{]}$}\{placeins\}}).  \LaTeXTT{\textbackslash{}FloatBarrier} may also be useful to prevent floats intruding on lists created using \LaTeXTT{itemize} or \LaTeXTT{enumerate}.  The \LaTeXTT{flafter} package can be used to force floats to appear after they are defined, and the \LaTeXTT{endfloat}\myplainurl{http://www.ctan.org/pkg/endfloat} package can be used to place all floats at the end of a document. The \LaTeXTT{float}\myplainurl{http://www.ctan.org/pkg/float} package provides the {\ttfamily \setmainfont[Path=/usr/share/fonts/truetype/cmu/,UprightFont=cmunrm.ttf,BoldFont=cmunbx.ttf,ItalicFont=cmunti.ttf,BoldItalicFont=cmunbi.ttf]{cmuntt.ttf}\setmonofont[Path=/usr/share/fonts/truetype/cmu/,UprightFont=cmuntt.ttf,BoldFont=cmuntb.ttf,ItalicFont=cmunit.ttf,BoldItalicFont=cmuntx.ttf]{cmuntt.ttf}\ttfamily H}{$\text{ }$}\setmainfont[Path=/usr/share/fonts/truetype/cmu/,UprightFont=cmunrm.ttf,BoldFont=cmunbx.ttf,ItalicFont=cmunti.ttf,BoldItalicFont=cmunbi.ttf]{cmunrm.ttf}\setmonofont[Path=/usr/share/fonts/truetype/cmu/,UprightFont=cmuntt.ttf,BoldFont=cmuntb.ttf,ItalicFont=cmunit.ttf,BoldItalicFont=cmuntx.ttf]{cmunrm.ttf} option to floating environments, which stops them from floating.
\section{Captions}
\label{368}

It is always good practice to add a caption to any figure or table. Fortunately, this is very simple in LaTeX. All you need to do is use the \LaTeXTT{\textbackslash{}caption\{\textquotesingle{}\textquotesingle{}text\textquotesingle{}\textquotesingle{}\}} command within the float environment. LaTeX will automatically keep track of the numbering of figures, so you do not need to include this within the caption text.

The location of the caption is traditionally underneath the float. However, it is up to you to therefore insert the caption command after the actual contents of the float (but still within the environment). If you place it before, then the caption will appear above the float. Try out the following example to demonstrate this effect:

\begin{longtable}{p{1.0\linewidth}}
\begin{Shaded}
\begin{Highlighting}[]

\NormalTok{\textbackslash{}documentclass[a4paper,12pt]\{article\}}
 
\NormalTok{\textbackslash{}usepackage[english]\{babel\}}
\NormalTok{\textbackslash{}usepackage\{graphicx\}}
 
\NormalTok{\textbackslash{}begin\{document\}}
 
\NormalTok{\textbackslash{}begin\{figure\}[!ht]}
  \NormalTok{\textbackslash{}caption\{A picture of a gull.\}}
  \NormalTok{\textbackslash{}centering}
    \NormalTok{\textbackslash{}includegraphics[width=0.5\textbackslash{}textwidth]\{gull\}}
\NormalTok{\textbackslash{}end\{figure\}}
 
\NormalTok{\textbackslash{}begin\{figure\}[!ht]}
  \NormalTok{\textbackslash{}centering}
    \NormalTok{\textbackslash{}reflectbox\{}\CommentTok{%}
      \NormalTok{\textbackslash{}includegraphics[width=0.5\textbackslash{}textwidth]\{gull\}<!---->\}}
  \NormalTok{\textbackslash{}caption\{A picture of the same gull}
           \NormalTok{looking the other way!\}}
\NormalTok{\textbackslash{}end\{figure\}}
 
\NormalTok{\textbackslash{}begin\{table\}[!ht]}
  \NormalTok{\textbackslash{}begin\{center\}}
    \NormalTok{\textbackslash{}begin\{tabular\}\{ l c r \}}
    \NormalTok{\textbackslash{}hline}
    \NormalTok{1 & 2 & 3 \textbackslash{}\textbackslash{}}
    \NormalTok{4 & 5 & 6 \textbackslash{}\textbackslash{}}
    \NormalTok{7 & 8 & 9 \textbackslash{}\textbackslash{}}
    \NormalTok{\textbackslash{}hline}
    \NormalTok{\textbackslash{}end\{tabular\}}
  \NormalTok{\textbackslash{}end\{center\}}
  \NormalTok{\textbackslash{}caption\{A simple table\}}
\NormalTok{\textbackslash{}end\{table\}}
 
\NormalTok{Notice how the tables and figures}
\NormalTok{have independent counters.}
 
\NormalTok{\textbackslash{}end\{document\}}
\end{Highlighting}
\end{Shaded}
\\



\begin{minipage}{0.86750\textwidth}
\begin{center}
\includegraphics[width=1.0\textwidth,height=6.5in,keepaspectratio]{../images/65.png}
\end{center}
\raggedright{}\myfigurewithoutcaption{65}
\end{minipage}\vspace{0.75cm}



\end{longtable}

Note that the command \LaTeXTT{\textbackslash{}reflectbox\{...\}} flips its content horizontally.
\subsection{Side captions}
\label{369}

It is sometimes desirable to have a caption appear on the side of a float, rather than above or below. The \LaTeXTT{sidecap} package can be used to place a caption beside a figure or table. The following example demonstrates this for a figure by using a \LaTeXTT{SCfigure} environment in place of the \LaTeXTT{figure} environment. The \LaTeXTT{floatrow} package is newer and has more capabilities.

\begin{longtable}{p{1.0\linewidth}}
\begin{Shaded}
\begin{Highlighting}[]

\NormalTok{\textbackslash{}documentclass\{article\}}
 
\NormalTok{\textbackslash{}usepackage\{graphicx\}}
\NormalTok{\textbackslash{}usepackage\{sidecap\}}
 
\NormalTok{\textbackslash{}begin\{document\}}
 
\NormalTok{\textbackslash{}begin\{SCfigure\}}
  \NormalTok{\textbackslash{}centering}
  \NormalTok{\textbackslash{}caption\{ ... caption text ... \}}
  \NormalTok{\textbackslash{}includegraphics[width=0.3\textbackslash{}textwidth]}\CommentTok{%}
    \NormalTok{\{Giraffe_picture\}}\CommentTok{% picture filename}
\NormalTok{\textbackslash{}end\{SCfigure\}}
 
\NormalTok{\textbackslash{}end\{document\}}
\end{Highlighting}
\end{Shaded}
\\



\begin{minipage}{1.0\linewidth}
\begin{center}
\includegraphics[width=1.0\linewidth,height=6.5in,keepaspectratio]{../images/66.png}
\end{center}
\raggedright{}\myfigurewithoutcaption{66}
\end{minipage}\vspace{0.75cm}



\end{longtable}
\subsection{Unnumbered captions}
\label{370}
In some types of document (such as presentations), it may not be desirable for figure captions to start {\itshape \setmainfont[Path=/usr/share/fonts/truetype/cmu/,UprightFont=cmunrm.ttf,BoldFont=cmunbx.ttf,ItalicFont=cmunti.ttf,BoldItalicFont=cmunbi.ttf]{cmunti.ttf}\setmonofont[Path=/usr/share/fonts/truetype/cmu/,UprightFont=cmuntt.ttf,BoldFont=cmuntb.ttf,ItalicFont=cmunit.ttf,BoldItalicFont=cmuntx.ttf]{cmunti.ttf}\itshape Figure:}\setmainfont[Path=/usr/share/fonts/truetype/cmu/,UprightFont=cmunrm.ttf,BoldFont=cmunbx.ttf,ItalicFont=cmunti.ttf,BoldItalicFont=cmunbi.ttf]{cmunrm.ttf}\setmonofont[Path=/usr/share/fonts/truetype/cmu/,UprightFont=cmuntt.ttf,BoldFont=cmuntb.ttf,ItalicFont=cmunit.ttf,BoldItalicFont=cmuntx.ttf]{cmunrm.ttf}.  This is easy to suppress by just placing the caption text in the  \LaTeXTT{Figure} environment, without enclosing it in a \LaTeXTT{Caption}.  This however means that there is no caption available for inclusion in a list of figures.
\subsection{Renaming table caption prefix}
\label{371}

In case you want to rename your table caption from \symbol{34}Table\symbol{34} to something else, you can use the \LaTeXTT{\textbackslash{}captionsetup} command. For example,
\begin{Shaded}
\begin{Highlighting}[]

\NormalTok{\textbackslash{}usepackage\{caption\}}
\NormalTok{\textbackslash{}captionsetup[table]\{name=New Table Name\}}
\end{Highlighting}
\end{Shaded}

\section{Lists of figures and tables}
\label{372}

Captions can be listed at the beginning of a paper or report in a \symbol{34}List of Tables\symbol{34} or a \symbol{34}List of Figures\symbol{34} section by using the \LaTeXTT{\textbackslash{}listoftables} or \LaTeXTT{\textbackslash{}listoffigures} commands, respectively. The caption used for each figure will appear in these lists, along with the figure numbers, and page numbers that they appear on.

The \LaTeXTT{\textbackslash{}caption} command also has an optional parameter, \LaTeXTT{\textbackslash{}caption{$\text{[}$}\textquotesingle{}\textquotesingle{}short\textquotesingle{}\textquotesingle{}{$\text{]}$}\{\textquotesingle{}\textquotesingle{}long\textquotesingle{}\textquotesingle{}\}} which is used for the {\itshape \setmainfont[Path=/usr/share/fonts/truetype/cmu/,UprightFont=cmunrm.ttf,BoldFont=cmunbx.ttf,ItalicFont=cmunti.ttf,BoldItalicFont=cmunbi.ttf]{cmunti.ttf}\setmonofont[Path=/usr/share/fonts/truetype/cmu/,UprightFont=cmuntt.ttf,BoldFont=cmuntb.ttf,ItalicFont=cmunit.ttf,BoldItalicFont=cmuntx.ttf]{cmunti.ttf}\itshape List of Tables}{$\text{ }$}\setmainfont[Path=/usr/share/fonts/truetype/cmu/,UprightFont=cmunrm.ttf,BoldFont=cmunbx.ttf,ItalicFont=cmunti.ttf,BoldItalicFont=cmunbi.ttf]{cmunrm.ttf}\setmonofont[Path=/usr/share/fonts/truetype/cmu/,UprightFont=cmuntt.ttf,BoldFont=cmuntb.ttf,ItalicFont=cmunit.ttf,BoldItalicFont=cmuntx.ttf]{cmunrm.ttf} or {\itshape \setmainfont[Path=/usr/share/fonts/truetype/cmu/,UprightFont=cmunrm.ttf,BoldFont=cmunbx.ttf,ItalicFont=cmunti.ttf,BoldItalicFont=cmunbi.ttf]{cmunti.ttf}\setmonofont[Path=/usr/share/fonts/truetype/cmu/,UprightFont=cmuntt.ttf,BoldFont=cmuntb.ttf,ItalicFont=cmunit.ttf,BoldItalicFont=cmuntx.ttf]{cmunti.ttf}\itshape List of Figures}\setmainfont[Path=/usr/share/fonts/truetype/cmu/,UprightFont=cmunrm.ttf,BoldFont=cmunbx.ttf,ItalicFont=cmunti.ttf,BoldItalicFont=cmunbi.ttf]{cmunrm.ttf}\setmonofont[Path=/usr/share/fonts/truetype/cmu/,UprightFont=cmuntt.ttf,BoldFont=cmuntb.ttf,ItalicFont=cmunit.ttf,BoldItalicFont=cmuntx.ttf]{cmunrm.ttf}. Typically the \LaTeXTT{short} description is for the caption listing, and the \LaTeXTT{long} description will be placed beside the figure or table. This is particularly useful if the caption is long, and only a \symbol{34}one-{}liner\symbol{34} is desired in the figure/table listing. Here is an example of this usage:

\begin{longtable}{p{1.0\linewidth}}
\begin{Shaded}
\begin{Highlighting}[]

\NormalTok{\textbackslash{}documentclass[12pt]\{article\}}
\NormalTok{\textbackslash{}usepackage\{graphicx\}}
 
\NormalTok{\textbackslash{}begin\{document\}}
 
\NormalTok{\textbackslash{}listoffigures}
 
\NormalTok{\textbackslash{}section\{Introduction\}}
 
\NormalTok{\textbackslash{}begin\{figure\}[hb]}
  \NormalTok{\textbackslash{}centering}
  \NormalTok{\textbackslash{}includegraphics[width=4in]\{gecko\}}
  \NormalTok{\textbackslash{}caption[Close up of \textbackslash{}textit\{Hemidactylus\} sp.]}
   \NormalTok{\{Close up of \textbackslash{}textit\{Hemidactylus\} sp., which is}
   \NormalTok{part the genus of the gecko family. It is the}
   \NormalTok{second most speciose genus in the family.\}}
\NormalTok{\textbackslash{}end\{figure\}}
 
\NormalTok{\textbackslash{}end\{document\}}
\end{Highlighting}
\end{Shaded}
\\



\begin{minipage}{1.0\linewidth}
\begin{center}
\includegraphics[width=1.0\linewidth,height=6.5in,keepaspectratio]{../images/67.png}
\end{center}
\raggedright{}\myfigurewithoutcaption{67}
\end{minipage}\vspace{0.75cm}



\end{longtable}
\section{Labels and cross-{}referencing}
\label{373}

Labels and cross-{}references work fairly similarly to the general case -{} see the \mylref{417}{Labels and Cross-{}referencing} section for more information.

\begin{TemplateInfo}{\danger}{Warning}If you want to label a figure so that you can reference it later, you have to add the label {\bfseries \setmainfont[Path=/usr/share/fonts/truetype/cmu/,UprightFont=cmunrm.ttf,BoldFont=cmunbx.ttf,ItalicFont=cmunti.ttf,BoldItalicFont=cmunbi.ttf]{cmunbx.ttf}\setmonofont[Path=/usr/share/fonts/truetype/cmu/,UprightFont=cmuntt.ttf,BoldFont=cmuntb.ttf,ItalicFont=cmunit.ttf,BoldItalicFont=cmuntx.ttf]{cmunbx.ttf}\bfseries after the caption}{$\text{ }$}\setmainfont[Path=/usr/share/fonts/truetype/cmu/,UprightFont=cmunrm.ttf,BoldFont=cmunbx.ttf,ItalicFont=cmunti.ttf,BoldItalicFont=cmunbi.ttf]{cmunrm.ttf}\setmonofont[Path=/usr/share/fonts/truetype/cmu/,UprightFont=cmuntt.ttf,BoldFont=cmuntb.ttf,ItalicFont=cmunit.ttf,BoldItalicFont=cmuntx.ttf]{cmunrm.ttf} (inside seems to work in LaTeX 2e) but {\bfseries \setmainfont[Path=/usr/share/fonts/truetype/cmu/,UprightFont=cmunrm.ttf,BoldFont=cmunbx.ttf,ItalicFont=cmunti.ttf,BoldItalicFont=cmunbi.ttf]{cmunbx.ttf}\setmonofont[Path=/usr/share/fonts/truetype/cmu/,UprightFont=cmuntt.ttf,BoldFont=cmuntb.ttf,ItalicFont=cmunit.ttf,BoldItalicFont=cmuntx.ttf]{cmunbx.ttf}\bfseries inside the floating environment}\setmainfont[Path=/usr/share/fonts/truetype/cmu/,UprightFont=cmunrm.ttf,BoldFont=cmunbx.ttf,ItalicFont=cmunti.ttf,BoldItalicFont=cmunbi.ttf]{cmunrm.ttf}\setmonofont[Path=/usr/share/fonts/truetype/cmu/,UprightFont=cmuntt.ttf,BoldFont=cmuntb.ttf,ItalicFont=cmunit.ttf,BoldItalicFont=cmuntx.ttf]{cmunrm.ttf}. If it is declared outside, it will give the section number.\end{TemplateInfo}

If the label picks up the section or list number instead of the figure number, put the label inside the caption to ensure correct numbering.  If you get an error when the label is inside the caption, use \LaTeXTT{\textbackslash{}protect} in front of the \LaTeXTT{\textbackslash{}label} command.
\section{Wrapping text around figures}
\label{374}

An author may prefer that some floats do not break the flow of text, but instead allow text to wrap around it. (Obviously, this effect only looks decent when the figure in question is significantly narrower than the text width.) 

A word of warning: Wrapping figures in LaTex will require a lot of manual adjustment of your document. There are several packages available for the task, but none of them works perfectly. Before you make the choice of including figures with text wrapping in your document, make sure you have considered all the options. For example, you could use a layout with two columns for your documents and have no text-{}wrapping at all.

Anyway, we will look at the package \LaTeXTT{wrapfig}. Note that \LaTeXTT{wrapfig} may not come with the default installation of LaTeX; you might need to \mylref{53}{install additional packages}. Noted also, wrapfig is incompatible with the enumerate and itemize environments

To use \LaTeXTT{wrapfig}, you must first add this to the preamble:

\begin{Shaded}
\begin{Highlighting}[]

\NormalTok{\textbackslash{}usepackage\{wrapfig\}}
\end{Highlighting}
\end{Shaded}

This then gives you access to:

\begin{Shaded}
\begin{Highlighting}[]

\NormalTok{\textbackslash{}begin\{wrapfigure\}[lineheight]\{position\}[overhang]\{width\}}
\end{Highlighting}
\end{Shaded}


The lineheight is expressed as the number of lines of text the figure spans. LaTeX will automatically calculate the value if this option is left blank but this can result in figures that look ugly (with too much spacing). The LaTeX calculation is manually overridden by entering the number of lines you would like the figure to span. This option can\textquotesingle{}t be entered in pt, mm etc...

There are overall eight possible positioning targets:

\begin{longtable}{|>{\RaggedRight}p{0.05526\linewidth}|>{\RaggedRight}p{0.06322\linewidth}|>{\RaggedRight}p{0.79580\linewidth}|} \hline 
\hspace*{0pt}\ignorespaces{}\hspace*{0pt} r&\hspace*{0pt}\ignorespaces{}\hspace*{0pt} R&\hspace*{0pt}\ignorespaces{}\hspace*{0pt} right side of the text\\ \hline \hspace*{0pt}\ignorespaces{}\hspace*{0pt} l&\hspace*{0pt}\ignorespaces{}\hspace*{0pt} L&\hspace*{0pt}\ignorespaces{}\hspace*{0pt} left side of the text\\ \hline \hspace*{0pt}\ignorespaces{}\hspace*{0pt} i&\hspace*{0pt}\ignorespaces{}\hspace*{0pt} I&\hspace*{0pt}\ignorespaces{}\hspace*{0pt} inside edge–near the binding (in a \LaTeXTT{twoside} document)\\ \hline \hspace*{0pt}\ignorespaces{}\hspace*{0pt} o&\hspace*{0pt}\ignorespaces{}\hspace*{0pt} O&\hspace*{0pt}\ignorespaces{}\hspace*{0pt} outside edge–far from the binding\\ \hline 
\end{longtable}


The uppercase-{}character allows the figure to float, while the lowercase version means \symbol{34}exactly here\symbol{34}. \myfootnote{\myplainurl{http://ftp.univie.ac.at/packages/tex/macros/latex/contrib/wrapfig/wrapfig-doc.pdf}}

The overhang of the figure can be manually set using the overhang option in pt, cm etc... 

The {\itshape \setmainfont[Path=/usr/share/fonts/truetype/cmu/,UprightFont=cmunrm.ttf,BoldFont=cmunbx.ttf,ItalicFont=cmunti.ttf,BoldItalicFont=cmunbi.ttf]{cmunti.ttf}\setmonofont[Path=/usr/share/fonts/truetype/cmu/,UprightFont=cmuntt.ttf,BoldFont=cmuntb.ttf,ItalicFont=cmunit.ttf,BoldItalicFont=cmuntx.ttf]{cmunti.ttf}\itshape width}{$\text{ }$}\setmainfont[Path=/usr/share/fonts/truetype/cmu/,UprightFont=cmunrm.ttf,BoldFont=cmunbx.ttf,ItalicFont=cmunti.ttf,BoldItalicFont=cmunbi.ttf]{cmunrm.ttf}\setmonofont[Path=/usr/share/fonts/truetype/cmu/,UprightFont=cmuntt.ttf,BoldFont=cmuntb.ttf,ItalicFont=cmunit.ttf,BoldItalicFont=cmuntx.ttf]{cmunrm.ttf} is, of course, the width of the figure. An example:

\begin{longtable}{p{1.0\linewidth}}
\begin{Shaded}
\begin{Highlighting}[]

\NormalTok{\textbackslash{}begin\{wrapfigure\}\{r\}\{0.5\textbackslash{}textwidth\}}
  \NormalTok{\textbackslash{}begin\{center\}}
    \NormalTok{\textbackslash{}includegraphics[width=0.48\textbackslash{}textwidth]\{gull\}}
  \NormalTok{\textbackslash{}end\{center\}}
  \NormalTok{\textbackslash{}caption\{A gull\}}
\NormalTok{\textbackslash{}end\{wrapfigure\}}
\end{Highlighting}
\end{Shaded}
\\



\begin{minipage}{0.84250\textwidth}
\begin{center}
\includegraphics[width=1.0\textwidth,height=6.5in,keepaspectratio]{../images/68.png}
\end{center}
\raggedright{}\myfigurewithoutcaption{68}
\end{minipage}\vspace{0.75cm}



\end{longtable}

You can also allow LaTeX to assign a width to the wrap by setting the width to 0pt. \textbackslash{}begin\{wrapfigure\}\{l\}\{0pt\}

Note that we have specified a size for both the \LaTeXTT{wrapfigure} environment and the image we have included. We did it in terms of the text width: it is always better to use relative sizes in LaTeX, let LaTeX do the work for you! The \symbol{34}wrap\symbol{34} is slightly bigger than the picture, so the compiler will not return any strange warning and you will have a small white frame between the image and the surrounding text. You can change it to get a better result, but if you don\textquotesingle{}t keep the image smaller than the \symbol{34}wrap\symbol{34}, you will see the image {\itshape \setmainfont[Path=/usr/share/fonts/truetype/cmu/,UprightFont=cmunrm.ttf,BoldFont=cmunbx.ttf,ItalicFont=cmunti.ttf,BoldItalicFont=cmunbi.ttf]{cmunti.ttf}\setmonofont[Path=/usr/share/fonts/truetype/cmu/,UprightFont=cmuntt.ttf,BoldFont=cmuntb.ttf,ItalicFont=cmunit.ttf,BoldItalicFont=cmuntx.ttf]{cmunti.ttf}\itshape over}{$\text{ }$}\setmainfont[Path=/usr/share/fonts/truetype/cmu/,UprightFont=cmunrm.ttf,BoldFont=cmunbx.ttf,ItalicFont=cmunti.ttf,BoldItalicFont=cmunbi.ttf]{cmunrm.ttf}\setmonofont[Path=/usr/share/fonts/truetype/cmu/,UprightFont=cmuntt.ttf,BoldFont=cmuntb.ttf,ItalicFont=cmunit.ttf,BoldItalicFont=cmuntx.ttf]{cmunrm.ttf} the text.

The wrapfig package can also be used with user-{}defined floats with float package. See below in the \mylref{378}{section on custom floats}.
\subsection{Tip for figures with too much white space}
\label{375}

It happens that you\textquotesingle{}ll generate figures with too much (or too little) white space on the top or bottom. In such a case, you can simply make use of the optional argument \LaTeXTT{{$\text{[}$}lineheight{$\text{]}$}}. It specifies the height of the figure in number of lines of text. Also remember that the environment \LaTeXTT{center} adds some extra white space at its top and bottom; consider using the command \LaTeXTT{\textbackslash{}centering} instead.

Another possibility is adding space within the float using the \LaTeXTT{\textbackslash{}vspace\{...\}} command. The argument is the size of the space you want to add, you can use any unit you want, including pt, mm, in, etc. If you provide a negative argument, it will add a {\itshape \setmainfont[Path=/usr/share/fonts/truetype/cmu/,UprightFont=cmunrm.ttf,BoldFont=cmunbx.ttf,ItalicFont=cmunti.ttf,BoldItalicFont=cmunbi.ttf]{cmunti.ttf}\setmonofont[Path=/usr/share/fonts/truetype/cmu/,UprightFont=cmuntt.ttf,BoldFont=cmuntb.ttf,ItalicFont=cmunit.ttf,BoldItalicFont=cmuntx.ttf]{cmunti.ttf}\itshape negative}{$\text{ }$}\setmainfont[Path=/usr/share/fonts/truetype/cmu/,UprightFont=cmunrm.ttf,BoldFont=cmunbx.ttf,ItalicFont=cmunti.ttf,BoldItalicFont=cmunbi.ttf]{cmunrm.ttf}\setmonofont[Path=/usr/share/fonts/truetype/cmu/,UprightFont=cmuntt.ttf,BoldFont=cmuntb.ttf,ItalicFont=cmunit.ttf,BoldItalicFont=cmuntx.ttf]{cmunrm.ttf} space, thus removing some white space. Using \LaTeXTT{\textbackslash{}vspace} tends to move the caption relative to the float while the \LaTeXTT{{$\text{[}$}lineheight{$\text{]}$}} argument does not. Here is an example using the \LaTeXTT{\textbackslash{}vspace} command, the code is exactly the one of the previous case, we just added some negative vertical spaces to shrink everything up:

\begin{longtable}{p{1.0\linewidth}}
\begin{Shaded}
\begin{Highlighting}[]

\NormalTok{\textbackslash{}begin\{wrapfigure\}\{r\}\{0.5\textbackslash{}textwidth\}}
  \NormalTok{\textbackslash{}vspace\{-20pt\}}
  \NormalTok{\textbackslash{}begin\{center\}}
    \NormalTok{\textbackslash{}includegraphics[width=0.48\textbackslash{}textwidth]\{gull\}}
  \NormalTok{\textbackslash{}end\{center\}}
  \NormalTok{\textbackslash{}vspace\{-20pt\}}
  \NormalTok{\textbackslash{}caption\{A gull\}}
  \NormalTok{\textbackslash{}vspace\{-10pt\}}
\NormalTok{\textbackslash{}end\{wrapfigure\}}
\end{Highlighting}
\end{Shaded}
\\



\begin{minipage}{1.0\linewidth}
\begin{center}
\includegraphics[width=1.0\linewidth,height=6.5in,keepaspectratio]{../images/69.png}
\end{center}
\raggedright{}\myfigurewithcaption{69}{336}
\end{minipage}\vspace{0.75cm}



\end{longtable}

In this case it may look too shrunk, but you can manage spaces the way you like. In general, it is best not to add any space at all: let LaTeX do the formatting work!

(In this case, the problem is the use of \LaTeXTT{\textbackslash{}begin\{center\}} to center the image. The \LaTeXTT{center} environment adds extra space that can be avoided if \LaTeXTT{\textbackslash{}centering} is used instead.)

You can use \LaTeXTT{intextsep} parameter to control additional space above and below the figure: \LaTeXTT{\textbackslash{}setlength\textbackslash{}intextsep\{0pt\}}

Alternatively you might use the \LaTeXTT{picins} package instead of the wrapfig package which produces a correct version without the excess white space out of the box without any hand tuning.

There is also an alternative to \LaTeXTT{wrapfig}: the package \LaTeXTT{floatflt} \myplainurl{http://www.ctan.org/pkg/floatflt}.

To remove the white space from a figure once for all, one should refer to the program pdfcrop, included in most TeX installations.
\section{Subfloats}
\label{376}

A useful extension is the \LaTeXTT{subcaption}\myplainurl{http://www.ctan.org/pkg/subcaption} package, which uses subfloats within a single float. The \LaTeXTT{subfigure} and \LaTeXTT{subfig} packages are deprecated; however they are useful alternatives when used in-{}conjunction with latex templates (i.e templates for journals from Springer and IOP, IEEETran and ACM SIG) that are not compatible with \LaTeXTT{subcaption}. These packages give the author the ability to have subfigures within figures, or subtables within table floats. Subfloats have their own caption, and an optional global caption.  An example will best illustrate the usage of the \LaTeXTT{subcaption} package:

\begin{Shaded}
\begin{Highlighting}[]

\NormalTok{\textbackslash{}usepackage\{graphicx\}}
\NormalTok{\textbackslash{}usepackage\{caption\}}
\NormalTok{\textbackslash{}usepackage\{subcaption\}}
 
\NormalTok{\textbackslash{}begin\{figure\}}
    \NormalTok{\textbackslash{}centering}
    \NormalTok{\textbackslash{}begin\{subfigure\}[b]\{0.3\textbackslash{}textwidth\}}
        \NormalTok{\textbackslash{}includegraphics[width=\textbackslash{}textwidth]\{gull\}}
        \NormalTok{\textbackslash{}caption\{A gull\}}
        \NormalTok{\textbackslash{}label\{fig:gull\}}
    \NormalTok{\textbackslash{}end\{subfigure\}}
    \NormalTok{~ }\CommentTok{%add desired spacing between images, e. g. ~, \textbackslash{}quad, \textbackslash{}qquad, \textbackslash{}hfill etc. }
      \CommentTok{%(or a blank line to force the subfigure onto a new line)}
    \NormalTok{\textbackslash{}begin\{subfigure\}[b]\{0.3\textbackslash{}textwidth\}}
        \NormalTok{\textbackslash{}includegraphics[width=\textbackslash{}textwidth]\{tiger\}}
        \NormalTok{\textbackslash{}caption\{A tiger\}}
        \NormalTok{\textbackslash{}label\{fig:tiger\}}
    \NormalTok{\textbackslash{}end\{subfigure\}}
    \NormalTok{~ }\CommentTok{%add desired spacing between images, e. g. ~, \textbackslash{}quad, \textbackslash{}qquad, \textbackslash{}hfill etc. }
    \CommentTok{%(or a blank line to force the subfigure onto a new line)}
    \NormalTok{\textbackslash{}begin\{subfigure\}[b]\{0.3\textbackslash{}textwidth\}}
        \NormalTok{\textbackslash{}includegraphics[width=\textbackslash{}textwidth]\{mouse\}}
        \NormalTok{\textbackslash{}caption\{A mouse\}}
        \NormalTok{\textbackslash{}label\{fig:mouse\}}
    \NormalTok{\textbackslash{}end\{subfigure\}}
    \NormalTok{\textbackslash{}caption\{Pictures of animals\}\textbackslash{}label\{fig:animals\}}
\NormalTok{\textbackslash{}end\{figure\}}
\end{Highlighting}
\end{Shaded}




\begin{minipage}{1.0\linewidth}
\begin{center}
\includegraphics[width=1.0\linewidth,height=6.5in,keepaspectratio]{../images/70.png}
\end{center}
\raggedright{}\myfigurewithoutcaption{70}
\end{minipage}\vspace{0.75cm}



You will notice that the figure environment is set up as usual. You may also use a table environment for subtables. For each subfloat, you need to use: 

\begin{Shaded}
\begin{Highlighting}[]

\NormalTok{\textbackslash{}begin\{table\}[<placement specifier>]}
    \NormalTok{\textbackslash{}begin\{subtable\}[<placement specifier>]\{<width>\}}
        \NormalTok{\textbackslash{}centering}
        \NormalTok{... table 1 ...}
    \NormalTok{\textbackslash{}caption\{<sub caption>\}}
    \NormalTok{\textbackslash{}end\{subtable\}}
    \NormalTok{~}
    \NormalTok{\textbackslash{}begin\{subtable\}[<placement specifier>]\{<width>\}}
        \NormalTok{\textbackslash{}centering}
        \NormalTok{... table 2 ...}
        \NormalTok{\textbackslash{}caption\{<sub caption>\}}
    \NormalTok{\textbackslash{}end\{subtable\}}
\NormalTok{\textbackslash{}end\{table\}}
\end{Highlighting}
\end{Shaded}


If you intend to cross-{}reference any of the subfloats, see where the label is inserted; \LaTeXTT{\textbackslash{}caption} outside the \LaTeXTT{subfigure}-{}environment will provide the global caption. 

\LaTeXTT{subcaption} will arrange the figures or tables side-{}by-{}side providing they can fit, otherwise, it will automatically shift subfloats below. This effect can be added manually, by putting the newline command (\LaTeXTT{\textbackslash{}\textbackslash{}}) before the figure you wish to move to a newline.

Horizontal spaces between figures are controlled by one of several commands, which are placed in between \LaTeXTT{\textbackslash{}begin\{subfigure\}} and \LaTeXTT{\textbackslash{}end\{subfigure\}}:
\begin{myitemize}
\item{} A non-{}breaking space (specified by \~{} as in the example above) can be used to insert a space in between the subfigs.
\item{} \mylref{521}{Math spaces}: \LaTeXTT{\textbackslash{}qquad}, \LaTeXTT{\textbackslash{}quad}, \LaTeXTT{\textbackslash{};}, and \LaTeXTT{\textbackslash{},}
\item{} Generic space: \LaTeXTT{\textbackslash{}hspace\{\textquotesingle{}\textquotesingle{}length\textquotesingle{}\textquotesingle{}\}}
\item{} Automatically expanding/contracting space: \LaTeXTT{\textbackslash{}hfill}
\end{myitemize}

\section{Wide figures in two-{}column documents}
\label{377}

If you are writing a document using two columns (i.e. you started your document with something like \LaTeXTT{\textbackslash{}documentclass{$\text{[}$}twocolumn{$\text{]}$}\{article\}}), you might have noticed that you can\textquotesingle{}t use floating elements that are wider than the width of a column (using a LaTeX notation, wider than \LaTeXTT{0.5\textbackslash{}textwidth}), otherwise you will see the image overlapping with text. If you really have to use such wide elements, the only solution is to use the \symbol{34}starred\symbol{34} variants of the floating environments, that are \LaTeXTT{\{figure*\}} and \LaTeXTT{\{table*\}}. Those \symbol{34}starred\symbol{34} versions work like the standard ones, but they will be as wide as the page, so you will get no overlapping.

A bad point of those environments is that they can be placed only at the top of the page or on their own page. If you try to specify their position using modifiers like {\itshape \setmainfont[Path=/usr/share/fonts/truetype/cmu/,UprightFont=cmunrm.ttf,BoldFont=cmunbx.ttf,ItalicFont=cmunti.ttf,BoldItalicFont=cmunbi.ttf]{cmunti.ttf}\setmonofont[Path=/usr/share/fonts/truetype/cmu/,UprightFont=cmuntt.ttf,BoldFont=cmuntb.ttf,ItalicFont=cmunit.ttf,BoldItalicFont=cmuntx.ttf]{cmunti.ttf}\itshape b}{$\text{ }$}\setmainfont[Path=/usr/share/fonts/truetype/cmu/,UprightFont=cmunrm.ttf,BoldFont=cmunbx.ttf,ItalicFont=cmunti.ttf,BoldItalicFont=cmunbi.ttf]{cmunrm.ttf}\setmonofont[Path=/usr/share/fonts/truetype/cmu/,UprightFont=cmuntt.ttf,BoldFont=cmuntb.ttf,ItalicFont=cmunit.ttf,BoldItalicFont=cmuntx.ttf]{cmunrm.ttf} or {\itshape \setmainfont[Path=/usr/share/fonts/truetype/cmu/,UprightFont=cmunrm.ttf,BoldFont=cmunbx.ttf,ItalicFont=cmunti.ttf,BoldItalicFont=cmunbi.ttf]{cmunti.ttf}\setmonofont[Path=/usr/share/fonts/truetype/cmu/,UprightFont=cmuntt.ttf,BoldFont=cmuntb.ttf,ItalicFont=cmunit.ttf,BoldItalicFont=cmuntx.ttf]{cmunti.ttf}\itshape h}\setmainfont[Path=/usr/share/fonts/truetype/cmu/,UprightFont=cmunrm.ttf,BoldFont=cmunbx.ttf,ItalicFont=cmunti.ttf,BoldItalicFont=cmunbi.ttf]{cmunrm.ttf}\setmonofont[Path=/usr/share/fonts/truetype/cmu/,UprightFont=cmuntt.ttf,BoldFont=cmuntb.ttf,ItalicFont=cmunit.ttf,BoldItalicFont=cmuntx.ttf]{cmunrm.ttf}, they will be ignored. Add \LaTeXTT{\textbackslash{}usepackage\{dblfloatfix\}} to the preamble in order to alleviate this problem with regard to placing these floats at the bottom of a page, using the optional specifier \LaTeXTT{{$\text{[}$}b{$\text{]}$}}. Default is \LaTeXTT{{$\text{[}$}tbp{$\text{]}$}}. However, {\itshape \setmainfont[Path=/usr/share/fonts/truetype/cmu/,UprightFont=cmunrm.ttf,BoldFont=cmunbx.ttf,ItalicFont=cmunti.ttf,BoldItalicFont=cmunbi.ttf]{cmunti.ttf}\setmonofont[Path=/usr/share/fonts/truetype/cmu/,UprightFont=cmuntt.ttf,BoldFont=cmuntb.ttf,ItalicFont=cmunit.ttf,BoldItalicFont=cmuntx.ttf]{cmunti.ttf}\itshape h}{$\text{ }$}\setmainfont[Path=/usr/share/fonts/truetype/cmu/,UprightFont=cmunrm.ttf,BoldFont=cmunbx.ttf,ItalicFont=cmunti.ttf,BoldItalicFont=cmunbi.ttf]{cmunrm.ttf}\setmonofont[Path=/usr/share/fonts/truetype/cmu/,UprightFont=cmuntt.ttf,BoldFont=cmuntb.ttf,ItalicFont=cmunit.ttf,BoldItalicFont=cmuntx.ttf]{cmunrm.ttf} still does not work.

To prevent the figures from being placed out-{}of-{}order with respect to their \symbol{34}non-{}starred\symbol{34} counterparts, the package \LaTeXTT{fixltx2e}  \myfootnote{\myplainurl{http://www.tex.ac.uk/cgi-bin/texfaq2html?label=2colfltorder}} should be used (e.g. \LaTeXTT{\textbackslash{}usepackage\{fixltx2e\}}).
\section{Custom floats}
\label{378}

If tables and figures are not adequate for your needs, then you always have the option to create your own! Examples of such instances could be source code examples, or maps. For a program float example, one might therefore wish to create a float named \LaTeXTT{program}. The package \LaTeXTT{float} is your friend for this task. {\itshape \setmainfont[Path=/usr/share/fonts/truetype/cmu/,UprightFont=cmunrm.ttf,BoldFont=cmunbx.ttf,ItalicFont=cmunti.ttf,BoldItalicFont=cmunbi.ttf]{cmunti.ttf}\setmonofont[Path=/usr/share/fonts/truetype/cmu/,UprightFont=cmuntt.ttf,BoldFont=cmuntb.ttf,ItalicFont=cmunit.ttf,BoldItalicFont=cmuntx.ttf]{cmunti.ttf}\itshape All commands to set up the new float must be placed in the preamble, and not within the document.}\setmainfont[Path=/usr/share/fonts/truetype/cmu/,UprightFont=cmunrm.ttf,BoldFont=cmunbx.ttf,ItalicFont=cmunti.ttf,BoldItalicFont=cmunbi.ttf]{cmunrm.ttf}\setmonofont[Path=/usr/share/fonts/truetype/cmu/,UprightFont=cmuntt.ttf,BoldFont=cmuntb.ttf,ItalicFont=cmunit.ttf,BoldItalicFont=cmuntx.ttf]{cmunrm.ttf}

\begin{myenumerate}
\item{}  Add \LaTeXTT{\textbackslash{}usepackage\{float\}} to the preamble of your document
\item{}  Declare your new float using: \LaTeXTT{\textbackslash{}newfloat\{type\}\{placement\}\{ext\}{$\text{[}$}outer counter{$\text{]}$}}, where:
\begin{myitemize}
\item{}  {\itshape \setmainfont[Path=/usr/share/fonts/truetype/cmu/,UprightFont=cmunrm.ttf,BoldFont=cmunbx.ttf,ItalicFont=cmunti.ttf,BoldItalicFont=cmunbi.ttf]{cmunti.ttf}\setmonofont[Path=/usr/share/fonts/truetype/cmu/,UprightFont=cmuntt.ttf,BoldFont=cmuntb.ttf,ItalicFont=cmunit.ttf,BoldItalicFont=cmuntx.ttf]{cmunti.ttf}\itshape type}{$\text{ }$}\setmainfont[Path=/usr/share/fonts/truetype/cmu/,UprightFont=cmunrm.ttf,BoldFont=cmunbx.ttf,ItalicFont=cmunti.ttf,BoldItalicFont=cmunbi.ttf]{cmunrm.ttf}\setmonofont[Path=/usr/share/fonts/truetype/cmu/,UprightFont=cmuntt.ttf,BoldFont=cmuntb.ttf,ItalicFont=cmunit.ttf,BoldItalicFont=cmuntx.ttf]{cmunrm.ttf} -{} the new name you wish to call your float, in this instance, \textquotesingle{}program\textquotesingle{}.
\item{}  {\itshape \setmainfont[Path=/usr/share/fonts/truetype/cmu/,UprightFont=cmunrm.ttf,BoldFont=cmunbx.ttf,ItalicFont=cmunti.ttf,BoldItalicFont=cmunbi.ttf]{cmunti.ttf}\setmonofont[Path=/usr/share/fonts/truetype/cmu/,UprightFont=cmuntt.ttf,BoldFont=cmuntb.ttf,ItalicFont=cmunit.ttf,BoldItalicFont=cmuntx.ttf]{cmunti.ttf}\itshape placement}{$\text{ }$}\setmainfont[Path=/usr/share/fonts/truetype/cmu/,UprightFont=cmunrm.ttf,BoldFont=cmunbx.ttf,ItalicFont=cmunti.ttf,BoldItalicFont=cmunbi.ttf]{cmunrm.ttf}\setmonofont[Path=/usr/share/fonts/truetype/cmu/,UprightFont=cmuntt.ttf,BoldFont=cmuntb.ttf,ItalicFont=cmunit.ttf,BoldItalicFont=cmuntx.ttf]{cmunrm.ttf} -{} t, b, p, or h (as previously described in \mylref{374}{Placement}), where letters enumerate permitted placements.
\item{}  {\itshape \setmainfont[Path=/usr/share/fonts/truetype/cmu/,UprightFont=cmunrm.ttf,BoldFont=cmunbx.ttf,ItalicFont=cmunti.ttf,BoldItalicFont=cmunbi.ttf]{cmunti.ttf}\setmonofont[Path=/usr/share/fonts/truetype/cmu/,UprightFont=cmuntt.ttf,BoldFont=cmuntb.ttf,ItalicFont=cmunit.ttf,BoldItalicFont=cmuntx.ttf]{cmunti.ttf}\itshape ext}{$\text{ }$}\setmainfont[Path=/usr/share/fonts/truetype/cmu/,UprightFont=cmunrm.ttf,BoldFont=cmunbx.ttf,ItalicFont=cmunti.ttf,BoldItalicFont=cmunbi.ttf]{cmunrm.ttf}\setmonofont[Path=/usr/share/fonts/truetype/cmu/,UprightFont=cmuntt.ttf,BoldFont=cmuntb.ttf,ItalicFont=cmunit.ttf,BoldItalicFont=cmuntx.ttf]{cmunrm.ttf} -{} the file name extension of an auxiliary file for the list of figures (or whatever). Latex writes the captions to this file. 
\item{}  {\itshape \setmainfont[Path=/usr/share/fonts/truetype/cmu/,UprightFont=cmunrm.ttf,BoldFont=cmunbx.ttf,ItalicFont=cmunti.ttf,BoldItalicFont=cmunbi.ttf]{cmunti.ttf}\setmonofont[Path=/usr/share/fonts/truetype/cmu/,UprightFont=cmuntt.ttf,BoldFont=cmuntb.ttf,ItalicFont=cmunit.ttf,BoldItalicFont=cmuntx.ttf]{cmunti.ttf}\itshape outer counter}{$\text{ }$}\setmainfont[Path=/usr/share/fonts/truetype/cmu/,UprightFont=cmunrm.ttf,BoldFont=cmunbx.ttf,ItalicFont=cmunti.ttf,BoldItalicFont=cmunbi.ttf]{cmunrm.ttf}\setmonofont[Path=/usr/share/fonts/truetype/cmu/,UprightFont=cmuntt.ttf,BoldFont=cmuntb.ttf,ItalicFont=cmunit.ttf,BoldItalicFont=cmuntx.ttf]{cmunrm.ttf} -{} the presence of this parameter indicates that the counter associated with this new float should depend on outer counter, for example \textquotesingle{}chapter\textquotesingle{}.
\end{myitemize}

\item{}  The default name that appears at the start of the caption is the type. If you wish to alter this, use \LaTeXTT{\textbackslash{}floatname\{type\}\{floatname\}}
\item{}  Changing float style can be issued with \LaTeXTT{\textbackslash{}floatstyle\{style\}} (Works on all subsequent \LaTeXTT{\textbackslash{}newfloat} commands, therefore, must be inserted before \LaTeXTT{\textbackslash{}newfloat} to be effective). 
\begin{myitemize}
\item{}  \LaTeXTT{plain} -{} the normal style for Latex floats, but the caption is always below the content.
\item{}  \LaTeXTT{plaintop} -{} the normal style for Latex floats, but the caption is always above the content.
\item{}  \LaTeXTT{boxed} -{} a box is drawn that surrounds the float, and the caption is printed below.
\item{}  \LaTeXTT{ruled} -{} the caption appears above the float, with rules immediately above and below. Then the float contents, followed by a final horizontal rule.
\end{myitemize}

\end{myenumerate}


Float styles can also be customized as the second example below illustrates.

An example document using a new \LaTeXTT{program} float type:

\begin{Shaded}
\begin{Highlighting}[]

\NormalTok{\textbackslash{}documentclass\{article\}}
 
\NormalTok{\textbackslash{}usepackage\{float\}}
 
\NormalTok{\textbackslash{}floatstyle\{ruled\}}
\NormalTok{\textbackslash{}newfloat\{program\}\{thp\}\{lop\}}
\NormalTok{\textbackslash{}floatname\{program\}\{Program\}}
 
\NormalTok{\textbackslash{}begin\{document\}}
 
\NormalTok{\textbackslash{}begin\{program\}}
  \NormalTok{\textbackslash{}begin\{verbatim\}}
 
\NormalTok{class HelloWorldApp \{}
  \NormalTok{public static void main(String[] args) \{}
    \NormalTok{//Display the string}
    \NormalTok{System.out.println("Hello World!");}
  \NormalTok{\}}
\NormalTok{\}}
\NormalTok{\textbackslash{}end\{verbatim\}}
  \NormalTok{\textbackslash{}caption\{The Hello World! program in Java.\}}
\NormalTok{\textbackslash{}end\{program\}}
 
\NormalTok{\textbackslash{}end\{document\}}
\end{Highlighting}
\end{Shaded}


The \LaTeXTT{verbatim} environment is an environment that is already part of LaTeX. Although not introduced so far, its name is fairly intuitive! LaTeX will reproduce everything you give it, including new lines, spaces, etc. It is good for source code, but if you want to introduce a lot of code you might consider using the \LaTeXTT{listings} package, that was made just for it.

While this is useful, one should be careful when embedding the float within another float. In particular, the error\\

\TemplateSpaceIndent{$\text{ }${}not$\text{ }${}in$\text{ }${}outer$\text{ }${}par$\text{ }${}mode}

may occur. One solution might be to use the {$\text{[}$}H{$\text{]}$} option (not any other) on the inner float, as this option \symbol{34}pins\symbol{34} the inner float to the outer one.

Newly created floats with \LaTeXTT{\textbackslash{}newfloat} can also be used in combination with the \LaTeXTT{wrapfig} package from above. E.g. the following code creates a floating text box, which floats in the text on the right side of the page and is complete with caption, numbering, an index file with the extension .lob and a customization of the float\textquotesingle{}s visual layout:

\begin{Shaded}
\begin{Highlighting}[]

\NormalTok{\textbackslash{}documentclass\{article\}}
 
\CommentTok{% have hyperref package before float in order to get strange errors with}
 \NormalTok{.\textbackslash{}theHfloatbox}
\NormalTok{\textbackslash{}usepackage\{hyperref\}}
 
\NormalTok{\textbackslash{}usepackage\{float\}}
 
\CommentTok{% allows use of "@" in control sequence names}
\NormalTok{\textbackslash{}makeatletter}
 
\CommentTok{% this creates a custom and simpler ruled box style}
\NormalTok{\textbackslash{}newcommand\textbackslash{}floatc@simplerule[2]\{\{\textbackslash{}@fs@cfont #1 #2\}\textbackslash{}par\}}
\NormalTok{\textbackslash{}n}
\NormalTok{ewcommand\textbackslash{}fs@simplerule\{\textbackslash{}def\textbackslash{}@fs@cfont\{\textbackslash{}bfseries\}\textbackslash{}let\textbackslash{}@fs@capt\textbackslash{}floatc@simplerule}
  \NormalTok{\textbackslash{}def\textbackslash{}@fs@pre\{\textbackslash{}hrule height.8pt depth0pt \textbackslash{}kern4pt\}}\CommentTok
  \NormalTok{\textbackslash{}def\textbackslash{}@fs@mid\{\textbackslash{}kern8pt\}}\CommentTok{%}
  \NormalTok{\textbackslash{}let\textbackslash{}@fs@iftopcapt\textbackslash{}iftrue\}}
 
\CommentTok{% this code block defines the new and custom floatbox float environment}
\NormalTok{\textbackslash{}floatstyle\{simplerule\}}
\NormalTok{\textbackslash{}newfloat\{floatbox\}\{thp\}\{lob\}[section]}
\NormalTok{\textbackslash{}floatname\{floatbox\}\{Text Box\}}
 
\NormalTok{\textbackslash{}begin\{document\}}
 
\NormalTok{\textbackslash{}begin\{floatbox\}\{r\}\{\}}
  \NormalTok{\textbackslash{}textit\{Bootstrapping\} is a resampling technique used }
  \NormalTok{for robustly estimating statistical quantities, such as }
  \NormalTok{the model fit $R^2$. It offers some protection against }
  \NormalTok{the sampling bias.}
  \NormalTok{\textbackslash{}caption\{Bootstrapping\}}
\NormalTok{\textbackslash{}end\{floatbox\}}
 
\NormalTok{\textbackslash{}end\{document\}}
\end{Highlighting}
\end{Shaded}

\subsection{Caption styles}
\label{379}

To change the appearance of captions, use the \LaTeXTT{caption} \myplainurl{http://www.ctan.org/pkg/caption} package. For example, to make all caption labels small and bold:

\begin{Shaded}
\begin{Highlighting}[]

\NormalTok{\textbackslash{}usepackage[font=small,labelfont=bf]\{caption\}}
\end{Highlighting}
\end{Shaded}


The KOMA script packages \myplainurl{http://www.komascript.de/} have their own caption customizing features with e.g. \LaTeXTT{\textbackslash{}captionabove}, \LaTeXTT{\textbackslash{}captionformat} and \LaTeXTT{\textbackslash{}setcapwidth}. However these definitions have limited effect on newly created float environments with the \LaTeXTT{wrapfig} package.

Alternatively, you can redefine the \LaTeXTT{\textbackslash{}thefigure} command:
\begin{Shaded}
\begin{Highlighting}[]

\NormalTok{\textbackslash{}renewcommand\{\textbackslash{}thefigure\}\{\textbackslash{}arabic\{section\}.\textbackslash{}arabic\{figure\}\}}
\end{Highlighting}
\end{Shaded}


See \mylref{469}{this page} for more information on counters.
Finally, note that the \LaTeXTT{caption2} package has long been deprecated.
\section{Labels in the figures}
\label{380}

There is a LaTeX package \LaTeXTT{lpic} \myplainurl{http://www.ctan.org/pkg/lpic} to put LaTeX on top of included graphics, thus allowing to add TeX annotations to imported graphics. 
It defines a convenient interface to put TeX over included graphics, and allows for drawing a white background under the typeset material to overshadow the graphics.
It is a better alternative for labels inside of graphics; you do not have to change text size when rescaling pictures, and all LaTeX power is available for labels.

A very similar package, with somewhat different syntax, is \LaTeXTT{pinlabel} \myplainurl{http://www.ctan.org/pkg/pinlabel}.  The link given also points to the packages \LaTeXTT{psfrag} and \LaTeXTT{overpic}.

A much more complicated package which can be used in the same way is \mylref{793}{TikZ}. TikZ is a front-{}end to a drawing library called pgf (which is used to make beamer for instance). It can be used to label figures by adding text nodes on top of an image node.
\section{Summary}
\label{381}

That concludes all the fundamentals of floats. You will hopefully see how much easier it is to let LaTeX do all the hard work and tweak the page layouts in order to get your figures in the best place. As always, the fact that LaTeX takes care of all caption and reference numbering is a great time saver.
\section{Notes and references}
\label{382} 
\LaTeXNullTemplate{}
\ARoberts{}

\chapter{Footnotes and Margin Notes}

\myminitoc
\label{383}

\label{384}

\section{Footnotes}
\label{385}

Footnotes are a very useful way of providing extra information to the reader. Usually, it is non-{}essential information which can be placed at the bottom of the page. This keeps the main body of text concise.

The footnote facility is easy to use. The command you need is: 
\begin{Shaded}
\begin{Highlighting}[]

\NormalTok{\textbackslash{}footnote\{text\}}\newline
\end{Highlighting}
\end{Shaded}
. Do not leave a space between the command and the word where you wish the footnote marker to appear, otherwise LaTeX will process that space and will leave the output not looking as intended.

\begin{longtable}{p{1.0\linewidth}}
\begin{Shaded}
\begin{Highlighting}[]

\NormalTok{Creating a footnote is easy.\textbackslash{}footnote\{An example footnote.\}}
\end{Highlighting}
\end{Shaded}
\\



\begin{minipage}{1.0\linewidth}
\begin{center}
\includegraphics[width=1.0\linewidth,height=6.5in,keepaspectratio]{../images/71.png}
\end{center}
\raggedright{}\myfigurewithoutcaption{71}
\end{minipage}\vspace{0.75cm}



\end{longtable}

LaTeX will obviously take care of typesetting the footnote at the bottom of the page. Each footnote is numbered sequentially -{} a process that, as you should have guessed by now, is automatically done for you.

You can also choose to place the footnote text manually. In this case we use the \LaTeXTT{\textbackslash{}footnotemark}-{}\LaTeXTT{\textbackslash{}footnotetext} duo:
\begin{Shaded}
\begin{Highlighting}[]

\NormalTok{\textbackslash{}footnotemark}
\CommentTok{% ...}
\NormalTok{Somewhere else\textbackslash{}footnotetext\{This is my footnote!\}}
\end{Highlighting}
\end{Shaded}

The footnote number can also be explicitly specified.
\begin{Shaded}
\begin{Highlighting}[]

\NormalTok{\textbackslash{}footnotemark[17]}
\CommentTok{% ...}
\NormalTok{Somewhere else\textbackslash{}footnotetext[17]\{This is my footnote!\}}
\end{Highlighting}
\end{Shaded}

\subsection{Customization}
\label{386}

It is possible to customize the footnote marking. By default, they are numbered sequentially (Arabic). However, without going too much into the mechanics of LaTeX at this point, it is possible to change this using the following command (which needs to be placed at the beginning of the document, or at least before the first footnote command is issued).

\begin{longtable}{>{\RaggedRight}p{0.49226\linewidth}>{\RaggedRight}p{0.45060\linewidth}} 
\hspace*{0pt}\ignorespaces{}\hspace*{0pt} \LaTeXTT{\textbackslash{}renewcommand\{\textbackslash{}thefootnote\}\{\textbackslash{}arabic\{footnote\}\}}&\hspace*{0pt}\ignorespaces{}\hspace*{0pt} Arabic numerals, e.g., 1, 2, 3...\\ \hspace*{0pt}\ignorespaces{}\hspace*{0pt} \LaTeXTT{\textbackslash{}renewcommand\{\textbackslash{}thefootnote\}\{\textbackslash{}roman\{footnote\}\}}&\hspace*{0pt}\ignorespaces{}\hspace*{0pt} Roman numerals (lowercase), e.g., i, ii, iii...\\ \hspace*{0pt}\ignorespaces{}\hspace*{0pt} \LaTeXTT{\textbackslash{}renewcommand\{\textbackslash{}thefootnote\}\{\textbackslash{}Roman\{footnote\}\}}&\hspace*{0pt}\ignorespaces{}\hspace*{0pt} Roman numerals (uppercase), e.g., I, II, III...\\ \hspace*{0pt}\ignorespaces{}\hspace*{0pt} \LaTeXTT{\textbackslash{}renewcommand\{\textbackslash{}thefootnote\}\{\textbackslash{}alph\{footnote\}\}}&\hspace*{0pt}\ignorespaces{}\hspace*{0pt} Alphabetic (lowercase), e.g., a, b, c...\\ \hspace*{0pt}\ignorespaces{}\hspace*{0pt} \LaTeXTT{\textbackslash{}renewcommand\{\textbackslash{}thefootnote\}\{\textbackslash{}Alph\{footnote\}\}}&\hspace*{0pt}\ignorespaces{}\hspace*{0pt} Alphabetic (uppercase), e.g., A, B, C...\\ \hspace*{0pt}\ignorespaces{}\hspace*{0pt} \LaTeXTT{\textbackslash{}renewcommand\{\textbackslash{}thefootnote\}\{\textbackslash{}fnsymbol\{footnote\}\}}&\hspace*{0pt}\ignorespaces{}\hspace*{0pt} A sequence of nine symbols, try it and see! 
\end{longtable}


To make a footnote without number mark use this declaration:
\begin{Shaded}
\begin{Highlighting}[]
\NormalTok{\textbackslash{}let\textbackslash{}thefootnote\textbackslash{}relax\textbackslash{}footnote\{There is no number in this footnote\} }
\end{Highlighting}
\end{Shaded}

In this way, the numbering is switched off globally. To have only one footnote without number mark, the above command has to be placed between \{ \}.
Nevertheless, in that case, the current footnote counter is still incremented, so for instance you\textquotesingle{}d get footnote 1, unnumbered, and 3.  A better solution\myfootnote{\myhref{ 
}{ LaTeX footnotes
}. . Retrieved  2016-{}01-{}14 } consists in defining the following macro in the preamble, and to use it:

\begin{Shaded}
\begin{Highlighting}[]

\NormalTok{\textbackslash{}makeatletter}
\NormalTok{\textbackslash{}def\textbackslash{}blfootnote\{\textbackslash{}xdef\textbackslash{}@thefnmark\{\}\textbackslash{}@footnotetext\}}
\NormalTok{\textbackslash{}makeatother}
\end{Highlighting}
\end{Shaded}


The package \myhref{http://www.ctan.org/pkg/footmisc}{footmisc} offers many possibilities for customizing the appearance of footnotes. It can be used, for example, to use a different font within footnotes.
\subsection{Reset counter}
\label{387}
{\bfseries
\begin{mydescription}every section
\end{mydescription}
}

\begin{Shaded}
\begin{Highlighting}[]

\NormalTok{\textbackslash{}makeatletter}
\NormalTok{\textbackslash{}@addtoreset\{footnote\}\{section\}}
\NormalTok{\textbackslash{}makeatother}
\end{Highlighting}
\end{Shaded}

{\bfseries
\begin{mydescription}every page 
\end{mydescription}
}

(This may require running LaTeX twice)
\begin{Shaded}
\begin{Highlighting}[]

\NormalTok{\textbackslash{}usepackage\{perpage\} }\CommentTok{%the perpage package}
\NormalTok{\textbackslash{}MakePerPage\{footnote\} }\CommentTok{%the perpage package command}
\end{Highlighting}
\end{Shaded}

\subsection{Common problems and workarounds}
\label{388}
\begin{myitemize}
\item{}  Footnotes unfortunately don\textquotesingle{}t work with tables, as it is considered a bad practice.  You can overcome this limitation with several techniques: you can use \LaTeXTT{\textbackslash{}footnotemark{$\text{[}$}123{$\text{]}$}} in the table, and \LaTeXTT{\textbackslash{}footnotetext{$\text{[}$}123{$\text{]}$}\{HelloWorld!\}} somewhere on the page. The same with references: use \LaTeXTT{\textbackslash{}footnote\{HelloWorld!\textbackslash{}label\{fnote}\}\} somewhere on the page and \LaTeXTT{\textbackslash{}textsuperscript\{\textbackslash{}ref\{fnote}\}\} in the table.  Or, you can add \LaTeXTT{\textbackslash{}usepackage\{footnote\}} and \LaTeXTT{\textbackslash{}makesavenoteenv\{tabular\}} to the preamble, and put your \LaTeXTT{table} environment in a \LaTeXTT{\textbackslash{}begin\{savenotes\}} environment.  Note that the latter does not work with the packages \LaTeXTT{color} or \LaTeXTT{colortbl}. See \myhref{http://www.tex.ac.uk/cgi-bin/texfaq2html?label=footintab}{this FAQ page} for other approaches (such as the use of tablenotes with \LaTeXTT{threeparttable}).
\end{myitemize}


\begin{myitemize}
\item{}  Footnotes also don\textquotesingle{}t work inside minipage environment (In fact, several environments break footnote support. the \LaTeXTT{\textbackslash{}makesavenoteenv\{environmentname\}} command of the footnote package might fix most). The minipage includes its own footnotes, independent of the document\textquotesingle{}s. The package \myhref{http://www.cs.brown.edu/system/software/latex/doc/mpfnmark.pdf}{mpfnmark} allows greater flexibility in managing these two sets of footnotes.
\end{myitemize}


\begin{myitemize}
\item{}  If the text within the footnote is a URL (using \LaTeXTT{\textbackslash{}url} or \LaTeXTT{\textbackslash{}href} commands) with special characters, it will not compile. You must either escape the characters with a leading backslash, or use another command.
\end{myitemize}


\begin{myitemize}
\item{}  If the text within the footnote is very long, LaTeX may split the footnote over several pages. You can prevent LaTeX from doing so by increasing the penalty for such an operation. To do this, insert the following line into the preamble of your document:
\end{myitemize}

\begin{Shaded}
\begin{Highlighting}[]

\NormalTok{\textbackslash{}interfootnotelinepenalty=10000}
\end{Highlighting}
\end{Shaded}


\begin{myitemize}
\item{}  To make multiple references to the same footnote, you can use the following syntax:
\end{myitemize}

\begin{Shaded}
\begin{Highlighting}[]

\NormalTok{Text that has a footnote\textbackslash{}footnote\{This is the footnote\} looks like this. Later}
 \NormalTok{text referring to same footnote\textbackslash{}footnotemark[\textbackslash{}value\{footnote\}] uses the other}
 \NormalTok{command.}
\end{Highlighting}
\end{Shaded}


If you need hyperref support, use instead:
\begin{Shaded}
\begin{Highlighting}[]

\NormalTok{Text that has a footnote\textbackslash{}footnote\{This is the}
 \NormalTok{footnote\}\textbackslash{}addtocounter\{footnote\}\{-1\}\textbackslash{}addtocounter\{Hfootnote\}\{-1\} looks like}
 \NormalTok{this. Later text referring to same footnote\textbackslash{}footnotemark uses the other command.}
\end{Highlighting}
\end{Shaded}


Note that these approaches will not work if there are other footnotes between the first reference and the subsequent \symbol{34}duplicate\symbol{34} references.  For more general solutions, see \myhref{http://tex.stackexchange.com/questions/35043}{here} and \myhref{http://tex.stackexchange.com/questions/10102/multiple-references-to-the-same-footnote-with-hyperref-support-is-there-a-bett}{here}.

\begin{myitemize}
\item{}  If the footnote is intended to be added to the title of a chapter, a section, or similar, two methods can be used:
\end{myitemize}

\begin{myenumerate}
\item{}  Write \LaTeXTT{\textbackslash{}section{$\text{[}$}title{$\text{]}$} \{title\textbackslash{}footnote\{I\textquotesingle{}m a footnote referred to the section\} \}} where \LaTeXTT{title} is the title of the section.
\item{}  Use the \LaTeXTT{footmisc} package, with package option \LaTeXTT{stable}, and simply add the footnote to the section title.
\end{myenumerate}

\section{Margin Notes}
\label{389}



\begin{minipage}{0.62500\textwidth}
\begin{center}
\includegraphics[width=1.0\textwidth,height=6.5in,keepaspectratio]{../images/72.png}
\end{center}
\raggedright{}\myfigurewithcaption{72}{A margin note.}
\end{minipage}\vspace{0.75cm}


Margin Notes are useful during the editorial process, to exchange comments among authors. To insert a margin note use \LaTeXTT{\textbackslash{}marginpar\{margin text\}}. For one-{}sided layout (simplex), the text will be placed in the right margin, starting from the line where it is defined. For two-{}sided layout (duplex), it will be placed in the outside margin and for two-{}column layout it will be placed in the nearest margin.

To swap the default side, use \LaTeXTT{\textbackslash{}reversemarginpar} and margin notes will then be placed on the opposite side, which would be the inside margin for two-{}sided layout.

If the text of your marginpar depends on which margin it is put in (say it includes an arrow pointing at the text or refers to a direction as in \symbol{34}as seen to the left...\symbol{34}), you can use \LaTeXTT{\textbackslash{}marginpar{$\text{[}$}left text{$\text{]}$}\{right text\}} to specify the variants. 

To insert a margin note in an area that \LaTeXTT{\textbackslash{}marginpar} can\textquotesingle{}t handle, such as footnotes or equation environments, use the package \LaTeXTT{marginnote}.

Another option for adding colored margin notes in a fancy way provides the package \LaTeXTT{todonotes} by using \LaTeXTT{\textbackslash{}todo\{todo note\}}. It makes use of the package \LaTeXTT{pgf} used for designing and drawing with a huge tool database.

The packages \LaTeXTT{mparhack} and \LaTeXTT{marginnote} can be used if the native \LaTeXTT{\textbackslash{}marginpar} command does not meet your needs.



\begin{minipage}{1.0\linewidth}
\begin{center}
\includegraphics[width=1.0\linewidth,height=6.5in,keepaspectratio]{../images/73.png}
\end{center}
\raggedright{}\myfigurewithcaption{73}{Margin geometry (bottom margin H not shown).}
\end{minipage}\vspace{0.75cm}



The marginnote and geometry package can set the widths of the margins and marginnotes as follows.

In the preamble, insert 

\begin{Shaded}
\begin{Highlighting}[]

\NormalTok{\textbackslash{}usepackage\{marginnote\}}
\end{Highlighting}
\end{Shaded}


and use the geometry package with custom sizes:

\begin{Shaded}
\begin{Highlighting}[]

\NormalTok{\textbackslash{}usepackage[top=Bcm, bottom=Hcm, outer=Ccm, inner=Acm, heightrounded,}
 \NormalTok{marginparwidth=Ecm, marginparsep=Dcm]\{geometry\}}
\end{Highlighting}
\end{Shaded}


where A, B, C, D, E, F, G, X are all numbers in cm (of course other units than cm can be used). 

In the main text, employ the marginnote package according to:

\begin{Shaded}
\begin{Highlighting}[]

\NormalTok{\textbackslash{}marginnote\{typeset text here...\}[Fcm]}
\end{Highlighting}
\end{Shaded}


Specifically, 
\begin{myitemize}
\item{}  \LaTeXTT{marginparwidth} (E) is the width of the margin note, 
\item{} \LaTeXTT{marginparsep} (D) is the separation between the paragraph and the margin note, 
\item{}  F is the downwards vertical offset from the first line the margin note was written (negative values of F shift the margin note upwards), and
\item{}  the value {\itshape \setmainfont[Path=/usr/share/fonts/truetype/cmu/,UprightFont=cmunrm.ttf,BoldFont=cmunbx.ttf,ItalicFont=cmunti.ttf,BoldItalicFont=cmunbi.ttf]{cmunti.ttf}\setmonofont[Path=/usr/share/fonts/truetype/cmu/,UprightFont=cmuntt.ttf,BoldFont=cmuntb.ttf,ItalicFont=cmunit.ttf,BoldItalicFont=cmuntx.ttf]{cmunti.ttf}\itshape G = C − (D + E)}{$\text{ }$}\setmainfont[Path=/usr/share/fonts/truetype/cmu/,UprightFont=cmunrm.ttf,BoldFont=cmunbx.ttf,ItalicFont=cmunti.ttf,BoldItalicFont=cmunbi.ttf]{cmunrm.ttf}\setmonofont[Path=/usr/share/fonts/truetype/cmu/,UprightFont=cmuntt.ttf,BoldFont=cmuntb.ttf,ItalicFont=cmunit.ttf,BoldItalicFont=cmuntx.ttf]{cmunrm.ttf} is the separation between the edge of the margin note and the edge.
\end{myitemize}


The example on the right was typeset by the following:

\begin{Shaded}
\begin{Highlighting}[]

\NormalTok{\textbackslash{}documentclass[a4paper,twoside,english]\{article\}}
\NormalTok{\textbackslash{}usepackage\{lmodern\}}
\NormalTok{\textbackslash{}renewcommand\{\textbackslash{}sfdefault\}\{lmss\}}
\NormalTok{\textbackslash{}usepackage[T1]\{fontenc\}}
 
\NormalTok{\textbackslash{}makeatletter}
\NormalTok{\textbackslash{}special\{papersize=\textbackslash{}the\textbackslash{}paperwidth,\textbackslash{}the\textbackslash{}paperheight\}}
 
\NormalTok{\textbackslash{}usepackage\{lipsum\}}
\NormalTok{\textbackslash{}usepackage\{marginnote\}}
\NormalTok{\textbackslash{}usepackage[top=1.5cm, bottom=1.5cm, outer=5cm, inner=2cm, heightrounded,}
 \NormalTok{marginparwidth=2.5cm, marginparsep=2cm]\{geometry\}}
 
\NormalTok{\textbackslash{}makeatother}
 
\NormalTok{\textbackslash{}usepackage\{babel\}}
\NormalTok{\textbackslash{}begin\{document\}}
 
\NormalTok{\textbackslash{}section\{Margin notes\}}
 
\NormalTok{\textbackslash{}marginnote\{This is a margin note using the geometry package, set at 0cm}
 \NormalTok{vertical offset to the first line it is typeset.\}[0cm]}
\NormalTok{\textbackslash{}marginnote\{This is a margin note using the geometry package, set at 5cm}
 \NormalTok{vertical offset to the first line it is typeset.\}[5cm]}
\NormalTok{\textbackslash{}lipsum[1-10]}
\NormalTok{\textbackslash{}end\{document\}}
\end{Highlighting}
\end{Shaded}


Additionally, the minimum vertical gap between margin notes can be adjusted with \LaTeXTT{\textbackslash{}marginparpush}, such as with \LaTeXTT{\textbackslash{}setlength\{\textbackslash{}marginparpush\}\{0pt\}}.
\section{Notes and References}
\label{390}

\LaTeXNullTemplate{}
\ARoberts{}



\myhref{https://ru.wikibooks.org/wiki/LaTeX\%2F\%D0\%9F\%D0\%BE\%D0\%B4\%D1\%81\%D1\%82\%D1\%80\%D0\%BE\%D1\%87\%D0\%BD\%D1\%8B\%D0\%B5\%20\%D0\%BF\%D1\%80\%D0\%B8\%D0\%BC\%D0\%B5\%D1\%87\%D0\%B0\%D0\%BD\%D0\%B8\%D1\%8F\%20\%D0\%B8\%20\%D0\%B7\%D0\%B0\%D0\%BC\%D0\%B5\%D1\%82\%D0\%BA\%D0\%B8\%20\%D0\%BD\%D0\%B0\%20\%D0\%BF\%D0\%BE\%D0\%BB\%D1\%8F\%D1\%85}{ru:LaTeX/Подстрочные примечания и заметки на полях}\chapter{Hyperlinks}

\myminitoc
\label{391}

\label{392}

LaTeX enables typesetting of hyperlinks, useful when the resulting format is PDF, and the hyperlinks can be followed. It does so using the package \LaTeXTT{hyperref}.
\section{Hyperref}
\label{393}

The package \LaTeXTT{hyperref}\myfootnote{\myfnhref{http://www.ctan.org/tex-archive/macros/latex/contrib/hyperref}{Hyperref package webpage in CTAN}} provides LaTeX the ability to create hyperlinks within the document. It works with {\itshape \setmainfont[Path=/usr/share/fonts/truetype/cmu/,UprightFont=cmunrm.ttf,BoldFont=cmunbx.ttf,ItalicFont=cmunti.ttf,BoldItalicFont=cmunbi.ttf]{cmunti.ttf}\setmonofont[Path=/usr/share/fonts/truetype/cmu/,UprightFont=cmuntt.ttf,BoldFont=cmuntb.ttf,ItalicFont=cmunit.ttf,BoldItalicFont=cmuntx.ttf]{cmunti.ttf}\itshape pdflatex}{$\text{ }$}\setmainfont[Path=/usr/share/fonts/truetype/cmu/,UprightFont=cmunrm.ttf,BoldFont=cmunbx.ttf,ItalicFont=cmunti.ttf,BoldItalicFont=cmunbi.ttf]{cmunrm.ttf}\setmonofont[Path=/usr/share/fonts/truetype/cmu/,UprightFont=cmuntt.ttf,BoldFont=cmuntb.ttf,ItalicFont=cmunit.ttf,BoldItalicFont=cmuntx.ttf]{cmunrm.ttf} and also with standard \symbol{34}latex\symbol{34} used with dvips and ghostscript or dvipdfm to build a PDF file. If you load it, you will have the possibility to include interactive external links and all your internal references will be turned to hyperlinks. The compiler {\itshape \setmainfont[Path=/usr/share/fonts/truetype/cmu/,UprightFont=cmunrm.ttf,BoldFont=cmunbx.ttf,ItalicFont=cmunti.ttf,BoldItalicFont=cmunbi.ttf]{cmunti.ttf}\setmonofont[Path=/usr/share/fonts/truetype/cmu/,UprightFont=cmuntt.ttf,BoldFont=cmuntb.ttf,ItalicFont=cmunit.ttf,BoldItalicFont=cmuntx.ttf]{cmunti.ttf}\itshape pdflatex}{$\text{ }$}\setmainfont[Path=/usr/share/fonts/truetype/cmu/,UprightFont=cmunrm.ttf,BoldFont=cmunbx.ttf,ItalicFont=cmunti.ttf,BoldItalicFont=cmunbi.ttf]{cmunrm.ttf}\setmonofont[Path=/usr/share/fonts/truetype/cmu/,UprightFont=cmuntt.ttf,BoldFont=cmuntb.ttf,ItalicFont=cmunit.ttf,BoldItalicFont=cmuntx.ttf]{cmunrm.ttf} makes it possible to create PDF files directly from the LaTeX source, and PDF supports more features than DVI. In particular PDF supports hyperlinks, and the only way to introduce them in LaTeX is using \LaTeXTT{hyperref}. Moreover, PDF can contain other information about a document such as the title, the author, etc., which can be  edited using this same package.
\section{Usage}
\label{394}

The basic usage with the standard settings is straightforward. Just load the package in the preamble:

\begin{Shaded}
\begin{Highlighting}[]

\NormalTok{\textbackslash{}usepackage\{hyperref\}}
\end{Highlighting}
\end{Shaded}


This will automatically turn all your internal references into hyperlinks. It won\textquotesingle{}t affect the way to write your documents: just keep on using the standard \LaTeXTT{\textbackslash{}label}-{}\LaTeXTT{\textbackslash{}ref} system (discussed in the chapter on \mylref{417}{Labels and Cross-{}referencing}); with \LaTeXTT{hyperref} those \symbol{34}connections\symbol{34} will become links and you will be able to click on them to be redirected to the right page. Moreover the table of contents, list of figures/tables and index will be made of hyperlinks, too. The hyperlinks will not show-{}up if you are working in draft mode.
\subsection{Commands}
\label{395}
The package provides some useful commands for inserting links pointing outside the document.
\subsubsection{\textbackslash{}hyperref}
\label{396}
Usage:

\begin{Shaded}
\begin{Highlighting}[]

\NormalTok{\textbackslash{}hyperref[label_name]\{''link text''\}}
\end{Highlighting}
\end{Shaded}


This will have the same effect as \LaTeXTT{\textbackslash{}ref\{label\_name\}} but will make the text {\itshape \setmainfont[Path=/usr/share/fonts/truetype/cmu/,UprightFont=cmunrm.ttf,BoldFont=cmunbx.ttf,ItalicFont=cmunti.ttf,BoldItalicFont=cmunbi.ttf]{cmunti.ttf}\setmonofont[Path=/usr/share/fonts/truetype/cmu/,UprightFont=cmuntt.ttf,BoldFont=cmuntb.ttf,ItalicFont=cmunit.ttf,BoldItalicFont=cmuntx.ttf]{cmunti.ttf}\itshape link text}{$\text{ }$}\setmainfont[Path=/usr/share/fonts/truetype/cmu/,UprightFont=cmunrm.ttf,BoldFont=cmunbx.ttf,ItalicFont=cmunti.ttf,BoldItalicFont=cmunbi.ttf]{cmunrm.ttf}\setmonofont[Path=/usr/share/fonts/truetype/cmu/,UprightFont=cmuntt.ttf,BoldFont=cmuntb.ttf,ItalicFont=cmunit.ttf,BoldItalicFont=cmuntx.ttf]{cmunrm.ttf} a full link, instead. The two can be combined. If the lemma labelled as \LaTeXTT{mainlemma} was number 4.1.1 the following example would result in

\begin{longtable}{p{1.0\linewidth}}
\begin{Shaded}
\begin{Highlighting}[]

\NormalTok{We use \textbackslash{}hyperref[mainlemma]\{lemma \textbackslash{}ref}\AlertTok{*}\NormalTok{\{mainlemma\} \}.}
\end{Highlighting}
\end{Shaded}
\\

We use lemma 4.1.1.

\end{longtable}
with the hyperlink as expected. Note the \symbol{34}*\symbol{34} after \LaTeXTT{\textbackslash{}ref} for avoiding nested hyperlinks.
\subsubsection{\textbackslash{}url}
\label{397}
Usage:
\begin{Shaded}
\begin{Highlighting}[]

\NormalTok{\textbackslash{}url\{<my_url>\}}
\end{Highlighting}
\end{Shaded}


It will show the URL using a mono-{}spaced font and, if you click on it, your browser will be opened pointing at it.
\subsubsection{\textbackslash{}href}
\label{398}
Usage:
\begin{Shaded}
\begin{Highlighting}[]

\NormalTok{\textbackslash{}href\{<my_url>\}\{<description>\}}
\end{Highlighting}
\end{Shaded}


It will show the string \LaTeXTT{description} using standard document font but, if you click on it, your browser will be opened pointing at \LaTeXTT{my\_url}. Here is an example:

\begin{Shaded}
\begin{Highlighting}[]

\NormalTok{\textbackslash{}url\{http://www.wikibooks.org\}}
\NormalTok{\textbackslash{}href\{http://www.wikibooks.org\}\{Wikibooks home\}}
\end{Highlighting}
\end{Shaded}


Both point at the same page, but in the first case the URL will be shown, while in the second case the URL will be hidden. Note that, if you print your document, the link stored using \LaTeXTT{\textbackslash{}href} will not be shown anywhere in the document.
\subsection{Other possibilities}
\label{399}
Apart from linking to websites discussed above, \LaTeXTT{hyperref} can be used to provide {\itshape \setmainfont[Path=/usr/share/fonts/truetype/cmu/,UprightFont=cmunrm.ttf,BoldFont=cmunbx.ttf,ItalicFont=cmunti.ttf,BoldItalicFont=cmunbi.ttf]{cmunti.ttf}\setmonofont[Path=/usr/share/fonts/truetype/cmu/,UprightFont=cmuntt.ttf,BoldFont=cmuntb.ttf,ItalicFont=cmunit.ttf,BoldItalicFont=cmuntx.ttf]{cmunti.ttf}\itshape mailto}{$\text{ }$}\setmainfont[Path=/usr/share/fonts/truetype/cmu/,UprightFont=cmunrm.ttf,BoldFont=cmunbx.ttf,ItalicFont=cmunti.ttf,BoldItalicFont=cmunbi.ttf]{cmunrm.ttf}\setmonofont[Path=/usr/share/fonts/truetype/cmu/,UprightFont=cmuntt.ttf,BoldFont=cmuntb.ttf,ItalicFont=cmunit.ttf,BoldItalicFont=cmuntx.ttf]{cmunrm.ttf} links, links to local files, and links to anywhere within the PDF output file.
\subsubsection{E-{}mail address}
\label{400}
A possible way to insert email links is by

\begin{Shaded}
\begin{Highlighting}[]

\NormalTok{\textbackslash{}href\{mailto:my_address@wikibooks.org\}\{my\textbackslash{}_address@wikibooks.org\}}
\end{Highlighting}
\end{Shaded}


It just shows your email address (so people can know it even if the document is printed on paper) but, if the reader clicks on it, (s)he can easily send you an email. Or, to incorporate the \LaTeXTT{url} package\textquotesingle{}s formatting and line breaking abilities into the displayed text, use\myfootnote{\myhref{ }{Email link with hyperref, url packages }. . Retrieved  }

\begin{Shaded}
\begin{Highlighting}[]

\NormalTok{\textbackslash{}href\{mailto:my_address@wikibooks.org\}\{\textbackslash{}nolinkurl\{my_address@wikibooks.org\} \}}
\end{Highlighting}
\end{Shaded}


When using this form, note that the \LaTeXTT{\textbackslash{}nolinkurl} command is fragile and if the hyperlink is inside of a moving argument, it must be preceeded by a \LaTeXTT{\textbackslash{}protect} command.
\subsubsection{Local file}
\label{401}
Files can also be linked using the url or the href commands. You simply have to add the string \LaTeXTT{run:} at the beginning of the link string:

\begin{Shaded}
\begin{Highlighting}[]

\NormalTok{\textbackslash{}url\{run:/path/to/my/file.ext\}}
\NormalTok{\textbackslash{}href\{run:/path/to/my/file.ext\}\{text displayed\}}
\end{Highlighting}
\end{Shaded}

Following \myplainurl{http://tex.stackexchange.com/questions/46488/link-to-local-pdf-file} the version with {\ttfamily \setmainfont[Path=/usr/share/fonts/truetype/cmu/,UprightFont=cmunrm.ttf,BoldFont=cmunbx.ttf,ItalicFont=cmunti.ttf,BoldItalicFont=cmunbi.ttf]{cmuntt.ttf}\setmonofont[Path=/usr/share/fonts/truetype/cmu/,UprightFont=cmuntt.ttf,BoldFont=cmuntb.ttf,ItalicFont=cmunit.ttf,BoldItalicFont=cmuntx.ttf]{cmuntt.ttf}\ttfamily url}{$\text{ }$}\setmainfont[Path=/usr/share/fonts/truetype/cmu/,UprightFont=cmunrm.ttf,BoldFont=cmunbx.ttf,ItalicFont=cmunti.ttf,BoldItalicFont=cmunbi.ttf]{cmunrm.ttf}\setmonofont[Path=/usr/share/fonts/truetype/cmu/,UprightFont=cmuntt.ttf,BoldFont=cmuntb.ttf,ItalicFont=cmunit.ttf,BoldItalicFont=cmuntx.ttf]{cmunrm.ttf} does not always work, but {\ttfamily \setmainfont[Path=/usr/share/fonts/truetype/cmu/,UprightFont=cmunrm.ttf,BoldFont=cmunbx.ttf,ItalicFont=cmunti.ttf,BoldItalicFont=cmunbi.ttf]{cmuntt.ttf}\setmonofont[Path=/usr/share/fonts/truetype/cmu/,UprightFont=cmuntt.ttf,BoldFont=cmuntb.ttf,ItalicFont=cmunit.ttf,BoldItalicFont=cmuntx.ttf]{cmuntt.ttf}\ttfamily href}{$\text{ }$}\setmainfont[Path=/usr/share/fonts/truetype/cmu/,UprightFont=cmunrm.ttf,BoldFont=cmunbx.ttf,ItalicFont=cmunti.ttf,BoldItalicFont=cmunbi.ttf]{cmunrm.ttf}\setmonofont[Path=/usr/share/fonts/truetype/cmu/,UprightFont=cmuntt.ttf,BoldFont=cmuntb.ttf,ItalicFont=cmunit.ttf,BoldItalicFont=cmuntx.ttf]{cmunrm.ttf} does.

It is possible to use relative paths to link documents near the location of your current document; in order to do so, use the standard Unix-{}like notation ({\ttfamily \setmainfont[Path=/usr/share/fonts/truetype/cmu/,UprightFont=cmunrm.ttf,BoldFont=cmunbx.ttf,ItalicFont=cmunti.ttf,BoldItalicFont=cmunbi.ttf]{cmuntt.ttf}\setmonofont[Path=/usr/share/fonts/truetype/cmu/,UprightFont=cmuntt.ttf,BoldFont=cmuntb.ttf,ItalicFont=cmunit.ttf,BoldItalicFont=cmuntx.ttf]{cmuntt.ttf}\ttfamily ./}{$\text{ }$}\setmainfont[Path=/usr/share/fonts/truetype/cmu/,UprightFont=cmunrm.ttf,BoldFont=cmunbx.ttf,ItalicFont=cmunti.ttf,BoldItalicFont=cmunbi.ttf]{cmunrm.ttf}\setmonofont[Path=/usr/share/fonts/truetype/cmu/,UprightFont=cmuntt.ttf,BoldFont=cmuntb.ttf,ItalicFont=cmunit.ttf,BoldItalicFont=cmuntx.ttf]{cmunrm.ttf} is the current directory, {\ttfamily \setmainfont[Path=/usr/share/fonts/truetype/cmu/,UprightFont=cmunrm.ttf,BoldFont=cmunbx.ttf,ItalicFont=cmunti.ttf,BoldItalicFont=cmunbi.ttf]{cmuntt.ttf}\setmonofont[Path=/usr/share/fonts/truetype/cmu/,UprightFont=cmuntt.ttf,BoldFont=cmuntb.ttf,ItalicFont=cmunit.ttf,BoldItalicFont=cmuntx.ttf]{cmuntt.ttf}\ttfamily ../}{$\text{ }$}\setmainfont[Path=/usr/share/fonts/truetype/cmu/,UprightFont=cmunrm.ttf,BoldFont=cmunbx.ttf,ItalicFont=cmunti.ttf,BoldItalicFont=cmunbi.ttf]{cmunrm.ttf}\setmonofont[Path=/usr/share/fonts/truetype/cmu/,UprightFont=cmuntt.ttf,BoldFont=cmuntb.ttf,ItalicFont=cmunit.ttf,BoldItalicFont=cmuntx.ttf]{cmunrm.ttf} is the previous directory, etc.)
\subsubsection{Hyperlink and Hypertarget}
\label{402}

It is also possible to create an anchor anywhere in the document (with or without caption) and to link to it. To create an anchor, use:

\begin{Shaded}
\begin{Highlighting}[]

\NormalTok{\textbackslash{}hypertarget\{label\}\{target caption\}}
\end{Highlighting}
\end{Shaded}


and to link to it, use:

\begin{Shaded}
\begin{Highlighting}[]

\NormalTok{\textbackslash{}hyperlink\{label\}\{link caption\}}
\end{Highlighting}
\end{Shaded}


where the \LaTeXTT{target caption} and \LaTeXTT{link caption} are the text that is displayed at the target location and link location respectively.
\section{Customization}
\label{403}

The standard settings should be fine for most users, but if you want to change something, that is also possible. There are several variables and  two methods to pass those to the package. Options can be passed as an argument of the package when it is loaded (the standard way packages work), or the \LaTeXTT{\textbackslash{}hypersetup} command can be used as follows:

\begin{Shaded}
\begin{Highlighting}[]

\NormalTok{\textbackslash{}hypersetup\{<option1> [, ...]\}}
\end{Highlighting}
\end{Shaded}


you can pass as many options as you want; separate them with a comma. Options have to be in the form:

\begin{Shaded}
\begin{Highlighting}[]

\NormalTok{variable_name=new_value}
\end{Highlighting}
\end{Shaded}


exactly the same format has to be used if you pass those options to the package while loading it, like this:

\begin{Shaded}
\begin{Highlighting}[]

\NormalTok{\textbackslash{}usepackage[<option1, option2>]\{hyperref\}}
\end{Highlighting}
\end{Shaded}


Here is a list of the possible variables you can change (for the complete list, see the official documentation). The default values are written in an upright font:

\myhref{http://www.tug.org/applications/hyperref/manual.html\#x1-120003.8}{Checkout 3.8 Big list at hyperref-{}manual at tug.org }
\begin{longtable}{|>{\RaggedRight}p{0.24346\linewidth}|>{\RaggedRight}p{0.33542\linewidth}|>{\RaggedRight}p{0.33542\linewidth}|} \hline 
{\bfseries \hspace*{0pt}\ignorespaces{}\hspace*{0pt}variable}&{\bfseries \hspace*{0pt}\ignorespaces{}\hspace*{0pt}values}&{\bfseries \hspace*{0pt}\ignorespaces{}\hspace*{0pt}comment}\endhead  \hline \hspace*{0pt}\ignorespaces{}\hspace*{0pt}{\ttfamily \setmainfont[Path=/usr/share/fonts/truetype/cmu/,UprightFont=cmunrm.ttf,BoldFont=cmunbx.ttf,ItalicFont=cmunti.ttf,BoldItalicFont=cmunbi.ttf]{cmuntt.ttf}\setmonofont[Path=/usr/share/fonts/truetype/cmu/,UprightFont=cmuntt.ttf,BoldFont=cmuntb.ttf,ItalicFont=cmunit.ttf,BoldItalicFont=cmuntx.ttf]{cmuntt.ttf}\ttfamily bookmarks}&\hspace*{0pt}\ignorespaces{}\hspace*{0pt}{\ttfamily =true{\itshape \setmainfont[Path=/usr/share/fonts/truetype/cmu/,UprightFont=cmunrm.ttf,BoldFont=cmunbx.ttf,ItalicFont=cmunti.ttf,BoldItalicFont=cmunbi.ttf]{cmunit.ttf}\setmonofont[Path=/usr/share/fonts/truetype/cmu/,UprightFont=cmuntt.ttf,BoldFont=cmuntb.ttf,ItalicFont=cmunit.ttf,BoldItalicFont=cmuntx.ttf]{cmunit.ttf}\ttfamily \itshape ,false}}&\hspace*{0pt}\ignorespaces{}\hspace*{0pt}\setmainfont[Path=/usr/share/fonts/truetype/cmu/,UprightFont=cmunrm.ttf,BoldFont=cmunbx.ttf,ItalicFont=cmunti.ttf,BoldItalicFont=cmunbi.ttf]{cmunrm.ttf}\setmonofont[Path=/usr/share/fonts/truetype/cmu/,UprightFont=cmuntt.ttf,BoldFont=cmuntb.ttf,ItalicFont=cmunit.ttf,BoldItalicFont=cmuntx.ttf]{cmunrm.ttf}show or hide the bookmarks bar when displaying the document\\ \hline \hspace*{0pt}\ignorespaces{}\hspace*{0pt}{\ttfamily \setmainfont[Path=/usr/share/fonts/truetype/cmu/,UprightFont=cmunrm.ttf,BoldFont=cmunbx.ttf,ItalicFont=cmunti.ttf,BoldItalicFont=cmunbi.ttf]{cmuntt.ttf}\setmonofont[Path=/usr/share/fonts/truetype/cmu/,UprightFont=cmuntt.ttf,BoldFont=cmuntb.ttf,ItalicFont=cmunit.ttf,BoldItalicFont=cmuntx.ttf]{cmuntt.ttf}\ttfamily unicode}&\hspace*{0pt}\ignorespaces{}\hspace*{0pt}{\ttfamily =false{\itshape \setmainfont[Path=/usr/share/fonts/truetype/cmu/,UprightFont=cmunrm.ttf,BoldFont=cmunbx.ttf,ItalicFont=cmunti.ttf,BoldItalicFont=cmunbi.ttf]{cmunit.ttf}\setmonofont[Path=/usr/share/fonts/truetype/cmu/,UprightFont=cmuntt.ttf,BoldFont=cmuntb.ttf,ItalicFont=cmunit.ttf,BoldItalicFont=cmuntx.ttf]{cmunit.ttf}\ttfamily \itshape ,true}}&\hspace*{0pt}\ignorespaces{}\hspace*{0pt}\setmainfont[Path=/usr/share/fonts/truetype/cmu/,UprightFont=cmunrm.ttf,BoldFont=cmunbx.ttf,ItalicFont=cmunti.ttf,BoldItalicFont=cmunbi.ttf]{cmunrm.ttf}\setmonofont[Path=/usr/share/fonts/truetype/cmu/,UprightFont=cmuntt.ttf,BoldFont=cmuntb.ttf,ItalicFont=cmunit.ttf,BoldItalicFont=cmuntx.ttf]{cmunrm.ttf}allows to use characters of non-{}Latin based languages in Acrobat’s bookmarks\\ \hline \hspace*{0pt}\ignorespaces{}\hspace*{0pt}{\ttfamily \setmainfont[Path=/usr/share/fonts/truetype/cmu/,UprightFont=cmunrm.ttf,BoldFont=cmunbx.ttf,ItalicFont=cmunti.ttf,BoldItalicFont=cmunbi.ttf]{cmuntt.ttf}\setmonofont[Path=/usr/share/fonts/truetype/cmu/,UprightFont=cmuntt.ttf,BoldFont=cmuntb.ttf,ItalicFont=cmunit.ttf,BoldItalicFont=cmuntx.ttf]{cmuntt.ttf}\ttfamily pdfborder}&\hspace*{0pt}\ignorespaces{}\hspace*{0pt}{\ttfamily =\{{\itshape \setmainfont[Path=/usr/share/fonts/truetype/cmu/,UprightFont=cmunrm.ttf,BoldFont=cmunbx.ttf,ItalicFont=cmunti.ttf,BoldItalicFont=cmunbi.ttf]{cmunit.ttf}\setmonofont[Path=/usr/share/fonts/truetype/cmu/,UprightFont=cmuntt.ttf,BoldFont=cmuntb.ttf,ItalicFont=cmunit.ttf,BoldItalicFont=cmuntx.ttf]{cmunit.ttf}\ttfamily \itshape RadiusH}{$\text{ }$}\setmainfont[Path=/usr/share/fonts/truetype/cmu/,UprightFont=cmunrm.ttf,BoldFont=cmunbx.ttf,ItalicFont=cmunti.ttf,BoldItalicFont=cmunbi.ttf]{cmuntt.ttf}\setmonofont[Path=/usr/share/fonts/truetype/cmu/,UprightFont=cmuntt.ttf,BoldFont=cmuntb.ttf,ItalicFont=cmunit.ttf,BoldItalicFont=cmuntx.ttf]{cmuntt.ttf}\ttfamily  {\itshape \setmainfont[Path=/usr/share/fonts/truetype/cmu/,UprightFont=cmunrm.ttf,BoldFont=cmunbx.ttf,ItalicFont=cmunti.ttf,BoldItalicFont=cmunbi.ttf]{cmunit.ttf}\setmonofont[Path=/usr/share/fonts/truetype/cmu/,UprightFont=cmuntt.ttf,BoldFont=cmuntb.ttf,ItalicFont=cmunit.ttf,BoldItalicFont=cmuntx.ttf]{cmunit.ttf}\ttfamily \itshape RadiusV}{$\text{ }$}\setmainfont[Path=/usr/share/fonts/truetype/cmu/,UprightFont=cmunrm.ttf,BoldFont=cmunbx.ttf,ItalicFont=cmunti.ttf,BoldItalicFont=cmunbi.ttf]{cmuntt.ttf}\setmonofont[Path=/usr/share/fonts/truetype/cmu/,UprightFont=cmuntt.ttf,BoldFont=cmuntb.ttf,ItalicFont=cmunit.ttf,BoldItalicFont=cmuntx.ttf]{cmuntt.ttf}\ttfamily  {\itshape \setmainfont[Path=/usr/share/fonts/truetype/cmu/,UprightFont=cmunrm.ttf,BoldFont=cmunbx.ttf,ItalicFont=cmunti.ttf,BoldItalicFont=cmunbi.ttf]{cmunit.ttf}\setmonofont[Path=/usr/share/fonts/truetype/cmu/,UprightFont=cmuntt.ttf,BoldFont=cmuntb.ttf,ItalicFont=cmunit.ttf,BoldItalicFont=cmuntx.ttf]{cmunit.ttf}\ttfamily \itshape Width}{$\text{ }$}\setmainfont[Path=/usr/share/fonts/truetype/cmu/,UprightFont=cmunrm.ttf,BoldFont=cmunbx.ttf,ItalicFont=cmunti.ttf,BoldItalicFont=cmunbi.ttf]{cmuntt.ttf}\setmonofont[Path=/usr/share/fonts/truetype/cmu/,UprightFont=cmuntt.ttf,BoldFont=cmuntb.ttf,ItalicFont=cmunit.ttf,BoldItalicFont=cmuntx.ttf]{cmuntt.ttf}\ttfamily  {$\text{[}$}{\itshape \setmainfont[Path=/usr/share/fonts/truetype/cmu/,UprightFont=cmunrm.ttf,BoldFont=cmunbx.ttf,ItalicFont=cmunti.ttf,BoldItalicFont=cmunbi.ttf]{cmunit.ttf}\setmonofont[Path=/usr/share/fonts/truetype/cmu/,UprightFont=cmuntt.ttf,BoldFont=cmuntb.ttf,ItalicFont=cmunit.ttf,BoldItalicFont=cmuntx.ttf]{cmunit.ttf}\ttfamily \itshape Dash-{}Pattern}\setmainfont[Path=/usr/share/fonts/truetype/cmu/,UprightFont=cmunrm.ttf,BoldFont=cmunbx.ttf,ItalicFont=cmunti.ttf,BoldItalicFont=cmunbi.ttf]{cmuntt.ttf}\setmonofont[Path=/usr/share/fonts/truetype/cmu/,UprightFont=cmuntt.ttf,BoldFont=cmuntb.ttf,ItalicFont=cmunit.ttf,BoldItalicFont=cmuntx.ttf]{cmuntt.ttf}\ttfamily {$\text{]}$}\}}&\hspace*{0pt}\ignorespaces{}\hspace*{0pt}\setmainfont[Path=/usr/share/fonts/truetype/cmu/,UprightFont=cmunrm.ttf,BoldFont=cmunbx.ttf,ItalicFont=cmunti.ttf,BoldItalicFont=cmunbi.ttf]{cmunrm.ttf}\setmonofont[Path=/usr/share/fonts/truetype/cmu/,UprightFont=cmuntt.ttf,BoldFont=cmuntb.ttf,ItalicFont=cmunit.ttf,BoldItalicFont=cmuntx.ttf]{cmunrm.ttf}set the style of the border around a link. The first two parameters (RadiusH, RadiusV) have no effect in most pdf viewers. {\itshape \setmainfont[Path=/usr/share/fonts/truetype/cmu/,UprightFont=cmunrm.ttf,BoldFont=cmunbx.ttf,ItalicFont=cmunti.ttf,BoldItalicFont=cmunbi.ttf]{cmunti.ttf}\setmonofont[Path=/usr/share/fonts/truetype/cmu/,UprightFont=cmuntt.ttf,BoldFont=cmuntb.ttf,ItalicFont=cmunit.ttf,BoldItalicFont=cmuntx.ttf]{cmunti.ttf}\itshape Width}{$\text{ }$}\setmainfont[Path=/usr/share/fonts/truetype/cmu/,UprightFont=cmunrm.ttf,BoldFont=cmunbx.ttf,ItalicFont=cmunti.ttf,BoldItalicFont=cmunbi.ttf]{cmunrm.ttf}\setmonofont[Path=/usr/share/fonts/truetype/cmu/,UprightFont=cmuntt.ttf,BoldFont=cmuntb.ttf,ItalicFont=cmunit.ttf,BoldItalicFont=cmuntx.ttf]{cmunrm.ttf} defines the thickness of the border. {\itshape \setmainfont[Path=/usr/share/fonts/truetype/cmu/,UprightFont=cmunrm.ttf,BoldFont=cmunbx.ttf,ItalicFont=cmunti.ttf,BoldItalicFont=cmunbi.ttf]{cmunti.ttf}\setmonofont[Path=/usr/share/fonts/truetype/cmu/,UprightFont=cmuntt.ttf,BoldFont=cmuntb.ttf,ItalicFont=cmunit.ttf,BoldItalicFont=cmuntx.ttf]{cmunti.ttf}\itshape Dash-{}Pattern}{$\text{ }$}\setmainfont[Path=/usr/share/fonts/truetype/cmu/,UprightFont=cmunrm.ttf,BoldFont=cmunbx.ttf,ItalicFont=cmunti.ttf,BoldItalicFont=cmunbi.ttf]{cmunrm.ttf}\setmonofont[Path=/usr/share/fonts/truetype/cmu/,UprightFont=cmuntt.ttf,BoldFont=cmuntb.ttf,ItalicFont=cmunit.ttf,BoldItalicFont=cmuntx.ttf]{cmunrm.ttf} is a series of numbers separated by space and enclosed by box-{}brackets. It is an optional parameter to specify the length of each line \& gap in the dash pattern. For example, \{0 0 0.5 {$\text{[}$}3 3{$\text{]}$}\} is supposed to draw a square box (no rounded corners) of width 0.5 and a dash pattern with a dash of length 3 followed by a gap of length 3. There is no uniformity in whether/how different pdf viewers render the dash pattern.\\ \hline \hspace*{0pt}\ignorespaces{}\hspace*{0pt}{\ttfamily \setmainfont[Path=/usr/share/fonts/truetype/cmu/,UprightFont=cmunrm.ttf,BoldFont=cmunbx.ttf,ItalicFont=cmunti.ttf,BoldItalicFont=cmunbi.ttf]{cmuntt.ttf}\setmonofont[Path=/usr/share/fonts/truetype/cmu/,UprightFont=cmuntt.ttf,BoldFont=cmuntb.ttf,ItalicFont=cmunit.ttf,BoldItalicFont=cmuntx.ttf]{cmuntt.ttf}\ttfamily pdftoolbar}&\hspace*{0pt}\ignorespaces{}\hspace*{0pt}{\ttfamily =true{\itshape \setmainfont[Path=/usr/share/fonts/truetype/cmu/,UprightFont=cmunrm.ttf,BoldFont=cmunbx.ttf,ItalicFont=cmunti.ttf,BoldItalicFont=cmunbi.ttf]{cmunit.ttf}\setmonofont[Path=/usr/share/fonts/truetype/cmu/,UprightFont=cmuntt.ttf,BoldFont=cmuntb.ttf,ItalicFont=cmunit.ttf,BoldItalicFont=cmuntx.ttf]{cmunit.ttf}\ttfamily \itshape ,false}}&\hspace*{0pt}\ignorespaces{}\hspace*{0pt}\setmainfont[Path=/usr/share/fonts/truetype/cmu/,UprightFont=cmunrm.ttf,BoldFont=cmunbx.ttf,ItalicFont=cmunti.ttf,BoldItalicFont=cmunbi.ttf]{cmunrm.ttf}\setmonofont[Path=/usr/share/fonts/truetype/cmu/,UprightFont=cmuntt.ttf,BoldFont=cmuntb.ttf,ItalicFont=cmunit.ttf,BoldItalicFont=cmuntx.ttf]{cmunrm.ttf}show or hide Acrobat’s toolbar\\ \hline \hspace*{0pt}\ignorespaces{}\hspace*{0pt}{\ttfamily \setmainfont[Path=/usr/share/fonts/truetype/cmu/,UprightFont=cmunrm.ttf,BoldFont=cmunbx.ttf,ItalicFont=cmunti.ttf,BoldItalicFont=cmunbi.ttf]{cmuntt.ttf}\setmonofont[Path=/usr/share/fonts/truetype/cmu/,UprightFont=cmuntt.ttf,BoldFont=cmuntb.ttf,ItalicFont=cmunit.ttf,BoldItalicFont=cmuntx.ttf]{cmuntt.ttf}\ttfamily pdfmenubar}&\hspace*{0pt}\ignorespaces{}\hspace*{0pt}{\ttfamily =true{\itshape \setmainfont[Path=/usr/share/fonts/truetype/cmu/,UprightFont=cmunrm.ttf,BoldFont=cmunbx.ttf,ItalicFont=cmunti.ttf,BoldItalicFont=cmunbi.ttf]{cmunit.ttf}\setmonofont[Path=/usr/share/fonts/truetype/cmu/,UprightFont=cmuntt.ttf,BoldFont=cmuntb.ttf,ItalicFont=cmunit.ttf,BoldItalicFont=cmuntx.ttf]{cmunit.ttf}\ttfamily \itshape ,false}}&\hspace*{0pt}\ignorespaces{}\hspace*{0pt}\setmainfont[Path=/usr/share/fonts/truetype/cmu/,UprightFont=cmunrm.ttf,BoldFont=cmunbx.ttf,ItalicFont=cmunti.ttf,BoldItalicFont=cmunbi.ttf]{cmunrm.ttf}\setmonofont[Path=/usr/share/fonts/truetype/cmu/,UprightFont=cmuntt.ttf,BoldFont=cmuntb.ttf,ItalicFont=cmunit.ttf,BoldItalicFont=cmuntx.ttf]{cmunrm.ttf}show or hide Acrobat’s menu\\ \hline \hspace*{0pt}\ignorespaces{}\hspace*{0pt}{\ttfamily \setmainfont[Path=/usr/share/fonts/truetype/cmu/,UprightFont=cmunrm.ttf,BoldFont=cmunbx.ttf,ItalicFont=cmunti.ttf,BoldItalicFont=cmunbi.ttf]{cmuntt.ttf}\setmonofont[Path=/usr/share/fonts/truetype/cmu/,UprightFont=cmuntt.ttf,BoldFont=cmuntb.ttf,ItalicFont=cmunit.ttf,BoldItalicFont=cmuntx.ttf]{cmuntt.ttf}\ttfamily pdffitwindow}&\hspace*{0pt}\ignorespaces{}\hspace*{0pt}{\ttfamily =true{\itshape \setmainfont[Path=/usr/share/fonts/truetype/cmu/,UprightFont=cmunrm.ttf,BoldFont=cmunbx.ttf,ItalicFont=cmunti.ttf,BoldItalicFont=cmunbi.ttf]{cmunit.ttf}\setmonofont[Path=/usr/share/fonts/truetype/cmu/,UprightFont=cmuntt.ttf,BoldFont=cmuntb.ttf,ItalicFont=cmunit.ttf,BoldItalicFont=cmuntx.ttf]{cmunit.ttf}\ttfamily \itshape ,false}}&\hspace*{0pt}\ignorespaces{}\hspace*{0pt}\setmainfont[Path=/usr/share/fonts/truetype/cmu/,UprightFont=cmunrm.ttf,BoldFont=cmunbx.ttf,ItalicFont=cmunti.ttf,BoldItalicFont=cmunbi.ttf]{cmunrm.ttf}\setmonofont[Path=/usr/share/fonts/truetype/cmu/,UprightFont=cmuntt.ttf,BoldFont=cmuntb.ttf,ItalicFont=cmunit.ttf,BoldItalicFont=cmuntx.ttf]{cmunrm.ttf}resize document window to fit document size\\ \hline \hspace*{0pt}\ignorespaces{}\hspace*{0pt}{\ttfamily \setmainfont[Path=/usr/share/fonts/truetype/cmu/,UprightFont=cmunrm.ttf,BoldFont=cmunbx.ttf,ItalicFont=cmunti.ttf,BoldItalicFont=cmunbi.ttf]{cmuntt.ttf}\setmonofont[Path=/usr/share/fonts/truetype/cmu/,UprightFont=cmuntt.ttf,BoldFont=cmuntb.ttf,ItalicFont=cmunit.ttf,BoldItalicFont=cmuntx.ttf]{cmuntt.ttf}\ttfamily pdfstartview}&\hspace*{0pt}\ignorespaces{}\hspace*{0pt}{\ttfamily =\{FitH\},{\itshape \setmainfont[Path=/usr/share/fonts/truetype/cmu/,UprightFont=cmunrm.ttf,BoldFont=cmunbx.ttf,ItalicFont=cmunti.ttf,BoldItalicFont=cmunbi.ttf]{cmunit.ttf}\setmonofont[Path=/usr/share/fonts/truetype/cmu/,UprightFont=cmuntt.ttf,BoldFont=cmuntb.ttf,ItalicFont=cmunit.ttf,BoldItalicFont=cmuntx.ttf]{cmunit.ttf}\ttfamily \itshape \{FitV\}}\setmainfont[Path=/usr/share/fonts/truetype/cmu/,UprightFont=cmunrm.ttf,BoldFont=cmunbx.ttf,ItalicFont=cmunti.ttf,BoldItalicFont=cmunbi.ttf]{cmuntt.ttf}\setmonofont[Path=/usr/share/fonts/truetype/cmu/,UprightFont=cmuntt.ttf,BoldFont=cmuntb.ttf,ItalicFont=cmunit.ttf,BoldItalicFont=cmuntx.ttf]{cmuntt.ttf}\ttfamily ,etc\myfootnote{Other possible values are defined in the \myfnhref{http://mirror.switch.ch/ftp/mirror/tex/macros/latex/contrib/hyperref/doc/manual.html\#TBL-7-40-1}{hyperref manual}}.}&\hspace*{0pt}\ignorespaces{}\hspace*{0pt}\setmainfont[Path=/usr/share/fonts/truetype/cmu/,UprightFont=cmunrm.ttf,BoldFont=cmunbx.ttf,ItalicFont=cmunti.ttf,BoldItalicFont=cmunbi.ttf]{cmunrm.ttf}\setmonofont[Path=/usr/share/fonts/truetype/cmu/,UprightFont=cmuntt.ttf,BoldFont=cmuntb.ttf,ItalicFont=cmunit.ttf,BoldItalicFont=cmuntx.ttf]{cmunrm.ttf}fit the width of the page to the window\\ \hline \hspace*{0pt}\ignorespaces{}\hspace*{0pt}{\ttfamily \setmainfont[Path=/usr/share/fonts/truetype/cmu/,UprightFont=cmunrm.ttf,BoldFont=cmunbx.ttf,ItalicFont=cmunti.ttf,BoldItalicFont=cmunbi.ttf]{cmuntt.ttf}\setmonofont[Path=/usr/share/fonts/truetype/cmu/,UprightFont=cmuntt.ttf,BoldFont=cmuntb.ttf,ItalicFont=cmunit.ttf,BoldItalicFont=cmuntx.ttf]{cmuntt.ttf}\ttfamily pdftitle}&\hspace*{0pt}\ignorespaces{}\hspace*{0pt}{\ttfamily =\{text\}}&\hspace*{0pt}\ignorespaces{}\hspace*{0pt}\setmainfont[Path=/usr/share/fonts/truetype/cmu/,UprightFont=cmunrm.ttf,BoldFont=cmunbx.ttf,ItalicFont=cmunti.ttf,BoldItalicFont=cmunbi.ttf]{cmunrm.ttf}\setmonofont[Path=/usr/share/fonts/truetype/cmu/,UprightFont=cmuntt.ttf,BoldFont=cmuntb.ttf,ItalicFont=cmunit.ttf,BoldItalicFont=cmuntx.ttf]{cmunrm.ttf}define the title that gets displayed in the \symbol{34}Document Info\symbol{34} window of Acrobat\\ \hline \hspace*{0pt}\ignorespaces{}\hspace*{0pt}{\ttfamily \setmainfont[Path=/usr/share/fonts/truetype/cmu/,UprightFont=cmunrm.ttf,BoldFont=cmunbx.ttf,ItalicFont=cmunti.ttf,BoldItalicFont=cmunbi.ttf]{cmuntt.ttf}\setmonofont[Path=/usr/share/fonts/truetype/cmu/,UprightFont=cmuntt.ttf,BoldFont=cmuntb.ttf,ItalicFont=cmunit.ttf,BoldItalicFont=cmuntx.ttf]{cmuntt.ttf}\ttfamily pdfauthor}&\hspace*{0pt}\ignorespaces{}\hspace*{0pt}{\ttfamily =\{text\}}&\hspace*{0pt}\ignorespaces{}\hspace*{0pt}\setmainfont[Path=/usr/share/fonts/truetype/cmu/,UprightFont=cmunrm.ttf,BoldFont=cmunbx.ttf,ItalicFont=cmunti.ttf,BoldItalicFont=cmunbi.ttf]{cmunrm.ttf}\setmonofont[Path=/usr/share/fonts/truetype/cmu/,UprightFont=cmuntt.ttf,BoldFont=cmuntb.ttf,ItalicFont=cmunit.ttf,BoldItalicFont=cmuntx.ttf]{cmunrm.ttf}the name of the PDF’s author, it works like the one above\\ \hline \hspace*{0pt}\ignorespaces{}\hspace*{0pt}{\ttfamily \setmainfont[Path=/usr/share/fonts/truetype/cmu/,UprightFont=cmunrm.ttf,BoldFont=cmunbx.ttf,ItalicFont=cmunti.ttf,BoldItalicFont=cmunbi.ttf]{cmuntt.ttf}\setmonofont[Path=/usr/share/fonts/truetype/cmu/,UprightFont=cmuntt.ttf,BoldFont=cmuntb.ttf,ItalicFont=cmunit.ttf,BoldItalicFont=cmuntx.ttf]{cmuntt.ttf}\ttfamily pdfsubject}&\hspace*{0pt}\ignorespaces{}\hspace*{0pt}{\ttfamily =\{text\}}&\hspace*{0pt}\ignorespaces{}\hspace*{0pt}\setmainfont[Path=/usr/share/fonts/truetype/cmu/,UprightFont=cmunrm.ttf,BoldFont=cmunbx.ttf,ItalicFont=cmunti.ttf,BoldItalicFont=cmunbi.ttf]{cmunrm.ttf}\setmonofont[Path=/usr/share/fonts/truetype/cmu/,UprightFont=cmuntt.ttf,BoldFont=cmuntb.ttf,ItalicFont=cmunit.ttf,BoldItalicFont=cmuntx.ttf]{cmunrm.ttf}subject of the document, it works like the one above\\ \hline \hspace*{0pt}\ignorespaces{}\hspace*{0pt}{\ttfamily \setmainfont[Path=/usr/share/fonts/truetype/cmu/,UprightFont=cmunrm.ttf,BoldFont=cmunbx.ttf,ItalicFont=cmunti.ttf,BoldItalicFont=cmunbi.ttf]{cmuntt.ttf}\setmonofont[Path=/usr/share/fonts/truetype/cmu/,UprightFont=cmuntt.ttf,BoldFont=cmuntb.ttf,ItalicFont=cmunit.ttf,BoldItalicFont=cmuntx.ttf]{cmuntt.ttf}\ttfamily pdfcreator}&\hspace*{0pt}\ignorespaces{}\hspace*{0pt}{\ttfamily =\{text\}}&\hspace*{0pt}\ignorespaces{}\hspace*{0pt}\setmainfont[Path=/usr/share/fonts/truetype/cmu/,UprightFont=cmunrm.ttf,BoldFont=cmunbx.ttf,ItalicFont=cmunti.ttf,BoldItalicFont=cmunbi.ttf]{cmunrm.ttf}\setmonofont[Path=/usr/share/fonts/truetype/cmu/,UprightFont=cmuntt.ttf,BoldFont=cmuntb.ttf,ItalicFont=cmunit.ttf,BoldItalicFont=cmuntx.ttf]{cmunrm.ttf}creator of the document, it works like the one above\\ \hline \hspace*{0pt}\ignorespaces{}\hspace*{0pt}{\ttfamily \setmainfont[Path=/usr/share/fonts/truetype/cmu/,UprightFont=cmunrm.ttf,BoldFont=cmunbx.ttf,ItalicFont=cmunti.ttf,BoldItalicFont=cmunbi.ttf]{cmuntt.ttf}\setmonofont[Path=/usr/share/fonts/truetype/cmu/,UprightFont=cmuntt.ttf,BoldFont=cmuntb.ttf,ItalicFont=cmunit.ttf,BoldItalicFont=cmuntx.ttf]{cmuntt.ttf}\ttfamily pdfproducer}&\hspace*{0pt}\ignorespaces{}\hspace*{0pt}{\ttfamily =\{text\}}&\hspace*{0pt}\ignorespaces{}\hspace*{0pt}\setmainfont[Path=/usr/share/fonts/truetype/cmu/,UprightFont=cmunrm.ttf,BoldFont=cmunbx.ttf,ItalicFont=cmunti.ttf,BoldItalicFont=cmunbi.ttf]{cmunrm.ttf}\setmonofont[Path=/usr/share/fonts/truetype/cmu/,UprightFont=cmuntt.ttf,BoldFont=cmuntb.ttf,ItalicFont=cmunit.ttf,BoldItalicFont=cmuntx.ttf]{cmunrm.ttf}producer of the document, it works like the one above\\ \hline \hspace*{0pt}\ignorespaces{}\hspace*{0pt}{\ttfamily \setmainfont[Path=/usr/share/fonts/truetype/cmu/,UprightFont=cmunrm.ttf,BoldFont=cmunbx.ttf,ItalicFont=cmunti.ttf,BoldItalicFont=cmunbi.ttf]{cmuntt.ttf}\setmonofont[Path=/usr/share/fonts/truetype/cmu/,UprightFont=cmuntt.ttf,BoldFont=cmuntb.ttf,ItalicFont=cmunit.ttf,BoldItalicFont=cmuntx.ttf]{cmuntt.ttf}\ttfamily pdfkeywords}&\hspace*{0pt}\ignorespaces{}\hspace*{0pt}{\ttfamily =\{text\}}&\hspace*{0pt}\ignorespaces{}\hspace*{0pt}\setmainfont[Path=/usr/share/fonts/truetype/cmu/,UprightFont=cmunrm.ttf,BoldFont=cmunbx.ttf,ItalicFont=cmunti.ttf,BoldItalicFont=cmunbi.ttf]{cmunrm.ttf}\setmonofont[Path=/usr/share/fonts/truetype/cmu/,UprightFont=cmuntt.ttf,BoldFont=cmuntb.ttf,ItalicFont=cmunit.ttf,BoldItalicFont=cmuntx.ttf]{cmunrm.ttf}list of keywords, separated by commas, example below\\ \hline \hspace*{0pt}\ignorespaces{}\hspace*{0pt}{\ttfamily \setmainfont[Path=/usr/share/fonts/truetype/cmu/,UprightFont=cmunrm.ttf,BoldFont=cmunbx.ttf,ItalicFont=cmunti.ttf,BoldItalicFont=cmunbi.ttf]{cmuntt.ttf}\setmonofont[Path=/usr/share/fonts/truetype/cmu/,UprightFont=cmuntt.ttf,BoldFont=cmuntb.ttf,ItalicFont=cmunit.ttf,BoldItalicFont=cmuntx.ttf]{cmuntt.ttf}\ttfamily pdfnewwindow}&\hspace*{0pt}\ignorespaces{}\hspace*{0pt}{\ttfamily (=true{\itshape \setmainfont[Path=/usr/share/fonts/truetype/cmu/,UprightFont=cmunrm.ttf,BoldFont=cmunbx.ttf,ItalicFont=cmunti.ttf,BoldItalicFont=cmunbi.ttf]{cmunit.ttf}\setmonofont[Path=/usr/share/fonts/truetype/cmu/,UprightFont=cmuntt.ttf,BoldFont=cmuntb.ttf,ItalicFont=cmunit.ttf,BoldItalicFont=cmuntx.ttf]{cmunit.ttf}\ttfamily \itshape ,false)}}&\hspace*{0pt}\ignorespaces{}\hspace*{0pt}\setmainfont[Path=/usr/share/fonts/truetype/cmu/,UprightFont=cmunrm.ttf,BoldFont=cmunbx.ttf,ItalicFont=cmunti.ttf,BoldItalicFont=cmunbi.ttf]{cmunrm.ttf}\setmonofont[Path=/usr/share/fonts/truetype/cmu/,UprightFont=cmuntt.ttf,BoldFont=cmuntb.ttf,ItalicFont=cmunit.ttf,BoldItalicFont=cmuntx.ttf]{cmunrm.ttf}define if a new PDF window should get opened when a link leads out of the current document. NB: This option is ignored if the link leads to an http/https address. \\ \hline \hspace*{0pt}\ignorespaces{}\hspace*{0pt}{\ttfamily \setmainfont[Path=/usr/share/fonts/truetype/cmu/,UprightFont=cmunrm.ttf,BoldFont=cmunbx.ttf,ItalicFont=cmunti.ttf,BoldItalicFont=cmunbi.ttf]{cmuntt.ttf}\setmonofont[Path=/usr/share/fonts/truetype/cmu/,UprightFont=cmuntt.ttf,BoldFont=cmuntb.ttf,ItalicFont=cmunit.ttf,BoldItalicFont=cmuntx.ttf]{cmuntt.ttf}\ttfamily pagebackref}&\hspace*{0pt}\ignorespaces{}\hspace*{0pt}{\ttfamily (=false{\itshape \setmainfont[Path=/usr/share/fonts/truetype/cmu/,UprightFont=cmunrm.ttf,BoldFont=cmunbx.ttf,ItalicFont=cmunti.ttf,BoldItalicFont=cmunbi.ttf]{cmunit.ttf}\setmonofont[Path=/usr/share/fonts/truetype/cmu/,UprightFont=cmuntt.ttf,BoldFont=cmuntb.ttf,ItalicFont=cmunit.ttf,BoldItalicFont=cmuntx.ttf]{cmunit.ttf}\ttfamily \itshape ,true)}}&\hspace*{0pt}\ignorespaces{}\hspace*{0pt}\setmainfont[Path=/usr/share/fonts/truetype/cmu/,UprightFont=cmunrm.ttf,BoldFont=cmunbx.ttf,ItalicFont=cmunti.ttf,BoldItalicFont=cmunbi.ttf]{cmunrm.ttf}\setmonofont[Path=/usr/share/fonts/truetype/cmu/,UprightFont=cmuntt.ttf,BoldFont=cmuntb.ttf,ItalicFont=cmunit.ttf,BoldItalicFont=cmuntx.ttf]{cmunrm.ttf}activate back references inside bibliography. Must be specified as part of the {\itshape \setmainfont[Path=/usr/share/fonts/truetype/cmu/,UprightFont=cmunrm.ttf,BoldFont=cmunbx.ttf,ItalicFont=cmunti.ttf,BoldItalicFont=cmunbi.ttf]{cmunti.ttf}\setmonofont[Path=/usr/share/fonts/truetype/cmu/,UprightFont=cmuntt.ttf,BoldFont=cmuntb.ttf,ItalicFont=cmunit.ttf,BoldItalicFont=cmuntx.ttf]{cmunti.ttf}\itshape \textbackslash{}usepackage\{\}}{$\text{ }$}\setmainfont[Path=/usr/share/fonts/truetype/cmu/,UprightFont=cmunrm.ttf,BoldFont=cmunbx.ttf,ItalicFont=cmunti.ttf,BoldItalicFont=cmunbi.ttf]{cmunrm.ttf}\setmonofont[Path=/usr/share/fonts/truetype/cmu/,UprightFont=cmuntt.ttf,BoldFont=cmuntb.ttf,ItalicFont=cmunit.ttf,BoldItalicFont=cmuntx.ttf]{cmunrm.ttf} statement.\\ \hline \hspace*{0pt}\ignorespaces{}\hspace*{0pt}{\ttfamily \setmainfont[Path=/usr/share/fonts/truetype/cmu/,UprightFont=cmunrm.ttf,BoldFont=cmunbx.ttf,ItalicFont=cmunti.ttf,BoldItalicFont=cmunbi.ttf]{cmuntt.ttf}\setmonofont[Path=/usr/share/fonts/truetype/cmu/,UprightFont=cmuntt.ttf,BoldFont=cmuntb.ttf,ItalicFont=cmunit.ttf,BoldItalicFont=cmuntx.ttf]{cmuntt.ttf}\ttfamily colorlinks}&\hspace*{0pt}\ignorespaces{}\hspace*{0pt}{\ttfamily (=false{\itshape \setmainfont[Path=/usr/share/fonts/truetype/cmu/,UprightFont=cmunrm.ttf,BoldFont=cmunbx.ttf,ItalicFont=cmunti.ttf,BoldItalicFont=cmunbi.ttf]{cmunit.ttf}\setmonofont[Path=/usr/share/fonts/truetype/cmu/,UprightFont=cmuntt.ttf,BoldFont=cmuntb.ttf,ItalicFont=cmunit.ttf,BoldItalicFont=cmuntx.ttf]{cmunit.ttf}\ttfamily \itshape ,true)}}&\hspace*{0pt}\ignorespaces{}\hspace*{0pt}\setmainfont[Path=/usr/share/fonts/truetype/cmu/,UprightFont=cmunrm.ttf,BoldFont=cmunbx.ttf,ItalicFont=cmunti.ttf,BoldItalicFont=cmunbi.ttf]{cmunrm.ttf}\setmonofont[Path=/usr/share/fonts/truetype/cmu/,UprightFont=cmuntt.ttf,BoldFont=cmuntb.ttf,ItalicFont=cmunit.ttf,BoldItalicFont=cmuntx.ttf]{cmunrm.ttf}surround the links by color frames ({\ttfamily \setmainfont[Path=/usr/share/fonts/truetype/cmu/,UprightFont=cmunrm.ttf,BoldFont=cmunbx.ttf,ItalicFont=cmunti.ttf,BoldItalicFont=cmunbi.ttf]{cmuntt.ttf}\setmonofont[Path=/usr/share/fonts/truetype/cmu/,UprightFont=cmuntt.ttf,BoldFont=cmuntb.ttf,ItalicFont=cmunit.ttf,BoldItalicFont=cmuntx.ttf]{cmuntt.ttf}\ttfamily false}\setmainfont[Path=/usr/share/fonts/truetype/cmu/,UprightFont=cmunrm.ttf,BoldFont=cmunbx.ttf,ItalicFont=cmunti.ttf,BoldItalicFont=cmunbi.ttf]{cmunrm.ttf}\setmonofont[Path=/usr/share/fonts/truetype/cmu/,UprightFont=cmuntt.ttf,BoldFont=cmuntb.ttf,ItalicFont=cmunit.ttf,BoldItalicFont=cmuntx.ttf]{cmunrm.ttf}) or colors the text of the links ({\ttfamily \setmainfont[Path=/usr/share/fonts/truetype/cmu/,UprightFont=cmunrm.ttf,BoldFont=cmunbx.ttf,ItalicFont=cmunti.ttf,BoldItalicFont=cmunbi.ttf]{cmuntt.ttf}\setmonofont[Path=/usr/share/fonts/truetype/cmu/,UprightFont=cmuntt.ttf,BoldFont=cmuntb.ttf,ItalicFont=cmunit.ttf,BoldItalicFont=cmuntx.ttf]{cmuntt.ttf}\ttfamily true}\setmainfont[Path=/usr/share/fonts/truetype/cmu/,UprightFont=cmunrm.ttf,BoldFont=cmunbx.ttf,ItalicFont=cmunti.ttf,BoldItalicFont=cmunbi.ttf]{cmunrm.ttf}\setmonofont[Path=/usr/share/fonts/truetype/cmu/,UprightFont=cmuntt.ttf,BoldFont=cmuntb.ttf,ItalicFont=cmunit.ttf,BoldItalicFont=cmuntx.ttf]{cmunrm.ttf}). The color of these links can be configured using the following options (default colors are shown):\\ \hline \hspace*{0pt}\ignorespaces{}\hspace*{0pt}{\ttfamily \setmainfont[Path=/usr/share/fonts/truetype/cmu/,UprightFont=cmunrm.ttf,BoldFont=cmunbx.ttf,ItalicFont=cmunti.ttf,BoldItalicFont=cmunbi.ttf]{cmuntt.ttf}\setmonofont[Path=/usr/share/fonts/truetype/cmu/,UprightFont=cmuntt.ttf,BoldFont=cmuntb.ttf,ItalicFont=cmunit.ttf,BoldItalicFont=cmuntx.ttf]{cmuntt.ttf}\ttfamily hidelinks}&\hspace*{0pt}\ignorespaces{}\hspace*{0pt}{\ttfamily }&\hspace*{0pt}\ignorespaces{}\hspace*{0pt}\setmainfont[Path=/usr/share/fonts/truetype/cmu/,UprightFont=cmunrm.ttf,BoldFont=cmunbx.ttf,ItalicFont=cmunti.ttf,BoldItalicFont=cmunbi.ttf]{cmunrm.ttf}\setmonofont[Path=/usr/share/fonts/truetype/cmu/,UprightFont=cmuntt.ttf,BoldFont=cmuntb.ttf,ItalicFont=cmunit.ttf,BoldItalicFont=cmuntx.ttf]{cmunrm.ttf}hide links (removing color and border)\\ \hline \hspace*{0pt}\ignorespaces{}\hspace*{0pt}{\ttfamily \setmainfont[Path=/usr/share/fonts/truetype/cmu/,UprightFont=cmunrm.ttf,BoldFont=cmunbx.ttf,ItalicFont=cmunti.ttf,BoldItalicFont=cmunbi.ttf]{cmuntt.ttf}\setmonofont[Path=/usr/share/fonts/truetype/cmu/,UprightFont=cmuntt.ttf,BoldFont=cmuntb.ttf,ItalicFont=cmunit.ttf,BoldItalicFont=cmuntx.ttf]{cmuntt.ttf}\ttfamily linkcolor}&\hspace*{0pt}\ignorespaces{}\hspace*{0pt}{\ttfamily =red}&\hspace*{0pt}\ignorespaces{}\hspace*{0pt}\setmainfont[Path=/usr/share/fonts/truetype/cmu/,UprightFont=cmunrm.ttf,BoldFont=cmunbx.ttf,ItalicFont=cmunti.ttf,BoldItalicFont=cmunbi.ttf]{cmunrm.ttf}\setmonofont[Path=/usr/share/fonts/truetype/cmu/,UprightFont=cmuntt.ttf,BoldFont=cmuntb.ttf,ItalicFont=cmunit.ttf,BoldItalicFont=cmuntx.ttf]{cmunrm.ttf}color of internal links (sections, pages, etc.)\\ \hline \hspace*{0pt}\ignorespaces{}\hspace*{0pt}{\ttfamily \setmainfont[Path=/usr/share/fonts/truetype/cmu/,UprightFont=cmunrm.ttf,BoldFont=cmunbx.ttf,ItalicFont=cmunti.ttf,BoldItalicFont=cmunbi.ttf]{cmuntt.ttf}\setmonofont[Path=/usr/share/fonts/truetype/cmu/,UprightFont=cmuntt.ttf,BoldFont=cmuntb.ttf,ItalicFont=cmunit.ttf,BoldItalicFont=cmuntx.ttf]{cmuntt.ttf}\ttfamily linktoc}&\hspace*{0pt}\ignorespaces{}\hspace*{0pt}{\ttfamily ={\itshape \setmainfont[Path=/usr/share/fonts/truetype/cmu/,UprightFont=cmunrm.ttf,BoldFont=cmunbx.ttf,ItalicFont=cmunti.ttf,BoldItalicFont=cmunbi.ttf]{cmunit.ttf}\setmonofont[Path=/usr/share/fonts/truetype/cmu/,UprightFont=cmuntt.ttf,BoldFont=cmuntb.ttf,ItalicFont=cmunit.ttf,BoldItalicFont=cmuntx.ttf]{cmunit.ttf}\ttfamily \itshape none}\setmainfont[Path=/usr/share/fonts/truetype/cmu/,UprightFont=cmunrm.ttf,BoldFont=cmunbx.ttf,ItalicFont=cmunti.ttf,BoldItalicFont=cmunbi.ttf]{cmuntt.ttf}\setmonofont[Path=/usr/share/fonts/truetype/cmu/,UprightFont=cmuntt.ttf,BoldFont=cmuntb.ttf,ItalicFont=cmunit.ttf,BoldItalicFont=cmuntx.ttf]{cmuntt.ttf}\ttfamily ,section,{\itshape \setmainfont[Path=/usr/share/fonts/truetype/cmu/,UprightFont=cmunrm.ttf,BoldFont=cmunbx.ttf,ItalicFont=cmunti.ttf,BoldItalicFont=cmunbi.ttf]{cmunit.ttf}\setmonofont[Path=/usr/share/fonts/truetype/cmu/,UprightFont=cmuntt.ttf,BoldFont=cmuntb.ttf,ItalicFont=cmunit.ttf,BoldItalicFont=cmuntx.ttf]{cmunit.ttf}\ttfamily \itshape page}\setmainfont[Path=/usr/share/fonts/truetype/cmu/,UprightFont=cmunrm.ttf,BoldFont=cmunbx.ttf,ItalicFont=cmunti.ttf,BoldItalicFont=cmunbi.ttf]{cmuntt.ttf}\setmonofont[Path=/usr/share/fonts/truetype/cmu/,UprightFont=cmuntt.ttf,BoldFont=cmuntb.ttf,ItalicFont=cmunit.ttf,BoldItalicFont=cmuntx.ttf]{cmuntt.ttf}\ttfamily ,{\itshape \setmainfont[Path=/usr/share/fonts/truetype/cmu/,UprightFont=cmunrm.ttf,BoldFont=cmunbx.ttf,ItalicFont=cmunti.ttf,BoldItalicFont=cmunbi.ttf]{cmunit.ttf}\setmonofont[Path=/usr/share/fonts/truetype/cmu/,UprightFont=cmuntt.ttf,BoldFont=cmuntb.ttf,ItalicFont=cmunit.ttf,BoldItalicFont=cmuntx.ttf]{cmunit.ttf}\ttfamily \itshape all}}&\hspace*{0pt}\ignorespaces{}\hspace*{0pt}\setmainfont[Path=/usr/share/fonts/truetype/cmu/,UprightFont=cmunrm.ttf,BoldFont=cmunbx.ttf,ItalicFont=cmunti.ttf,BoldItalicFont=cmunbi.ttf]{cmunrm.ttf}\setmonofont[Path=/usr/share/fonts/truetype/cmu/,UprightFont=cmuntt.ttf,BoldFont=cmuntb.ttf,ItalicFont=cmunit.ttf,BoldItalicFont=cmuntx.ttf]{cmunrm.ttf}defines which part of an entry in the table of contents is made into a link\\ \hline \hspace*{0pt}\ignorespaces{}\hspace*{0pt}{\ttfamily \setmainfont[Path=/usr/share/fonts/truetype/cmu/,UprightFont=cmunrm.ttf,BoldFont=cmunbx.ttf,ItalicFont=cmunti.ttf,BoldItalicFont=cmunbi.ttf]{cmuntt.ttf}\setmonofont[Path=/usr/share/fonts/truetype/cmu/,UprightFont=cmuntt.ttf,BoldFont=cmuntb.ttf,ItalicFont=cmunit.ttf,BoldItalicFont=cmuntx.ttf]{cmuntt.ttf}\ttfamily citecolor}&\hspace*{0pt}\ignorespaces{}\hspace*{0pt}{\ttfamily =green}&\hspace*{0pt}\ignorespaces{}\hspace*{0pt}\setmainfont[Path=/usr/share/fonts/truetype/cmu/,UprightFont=cmunrm.ttf,BoldFont=cmunbx.ttf,ItalicFont=cmunti.ttf,BoldItalicFont=cmunbi.ttf]{cmunrm.ttf}\setmonofont[Path=/usr/share/fonts/truetype/cmu/,UprightFont=cmuntt.ttf,BoldFont=cmuntb.ttf,ItalicFont=cmunit.ttf,BoldItalicFont=cmuntx.ttf]{cmunrm.ttf}color of citation links (bibliography)\\ \hline \hspace*{0pt}\ignorespaces{}\hspace*{0pt}{\ttfamily \setmainfont[Path=/usr/share/fonts/truetype/cmu/,UprightFont=cmunrm.ttf,BoldFont=cmunbx.ttf,ItalicFont=cmunti.ttf,BoldItalicFont=cmunbi.ttf]{cmuntt.ttf}\setmonofont[Path=/usr/share/fonts/truetype/cmu/,UprightFont=cmuntt.ttf,BoldFont=cmuntb.ttf,ItalicFont=cmunit.ttf,BoldItalicFont=cmuntx.ttf]{cmuntt.ttf}\ttfamily filecolor}&\hspace*{0pt}\ignorespaces{}\hspace*{0pt}{\ttfamily =cyan}&\hspace*{0pt}\ignorespaces{}\hspace*{0pt}\setmainfont[Path=/usr/share/fonts/truetype/cmu/,UprightFont=cmunrm.ttf,BoldFont=cmunbx.ttf,ItalicFont=cmunti.ttf,BoldItalicFont=cmunbi.ttf]{cmunrm.ttf}\setmonofont[Path=/usr/share/fonts/truetype/cmu/,UprightFont=cmuntt.ttf,BoldFont=cmuntb.ttf,ItalicFont=cmunit.ttf,BoldItalicFont=cmuntx.ttf]{cmunrm.ttf}color of file links\\ \hline \hspace*{0pt}\ignorespaces{}\hspace*{0pt}{\ttfamily \setmainfont[Path=/usr/share/fonts/truetype/cmu/,UprightFont=cmunrm.ttf,BoldFont=cmunbx.ttf,ItalicFont=cmunti.ttf,BoldItalicFont=cmunbi.ttf]{cmuntt.ttf}\setmonofont[Path=/usr/share/fonts/truetype/cmu/,UprightFont=cmuntt.ttf,BoldFont=cmuntb.ttf,ItalicFont=cmunit.ttf,BoldItalicFont=cmuntx.ttf]{cmuntt.ttf}\ttfamily urlcolor}&\hspace*{0pt}\ignorespaces{}\hspace*{0pt}{\ttfamily =magenta}&\hspace*{0pt}\ignorespaces{}\hspace*{0pt}\setmainfont[Path=/usr/share/fonts/truetype/cmu/,UprightFont=cmunrm.ttf,BoldFont=cmunbx.ttf,ItalicFont=cmunti.ttf,BoldItalicFont=cmunbi.ttf]{cmunrm.ttf}\setmonofont[Path=/usr/share/fonts/truetype/cmu/,UprightFont=cmuntt.ttf,BoldFont=cmuntb.ttf,ItalicFont=cmunit.ttf,BoldItalicFont=cmuntx.ttf]{cmunrm.ttf}color of URL links (mail, web)\\ \hline \hspace*{0pt}\ignorespaces{}\hspace*{0pt}{\ttfamily \setmainfont[Path=/usr/share/fonts/truetype/cmu/,UprightFont=cmunrm.ttf,BoldFont=cmunbx.ttf,ItalicFont=cmunti.ttf,BoldItalicFont=cmunbi.ttf]{cmuntt.ttf}\setmonofont[Path=/usr/share/fonts/truetype/cmu/,UprightFont=cmuntt.ttf,BoldFont=cmuntb.ttf,ItalicFont=cmunit.ttf,BoldItalicFont=cmuntx.ttf]{cmuntt.ttf}\ttfamily linkbordercolor}&\hspace*{0pt}\ignorespaces{}\hspace*{0pt}{\ttfamily =\{1 0 0\}}&\hspace*{0pt}\ignorespaces{}\hspace*{0pt}\setmainfont[Path=/usr/share/fonts/truetype/cmu/,UprightFont=cmunrm.ttf,BoldFont=cmunbx.ttf,ItalicFont=cmunti.ttf,BoldItalicFont=cmunbi.ttf]{cmunrm.ttf}\setmonofont[Path=/usr/share/fonts/truetype/cmu/,UprightFont=cmuntt.ttf,BoldFont=cmuntb.ttf,ItalicFont=cmunit.ttf,BoldItalicFont=cmuntx.ttf]{cmunrm.ttf}color of frame around internal links (if {\ttfamily \setmainfont[Path=/usr/share/fonts/truetype/cmu/,UprightFont=cmunrm.ttf,BoldFont=cmunbx.ttf,ItalicFont=cmunti.ttf,BoldItalicFont=cmunbi.ttf]{cmuntt.ttf}\setmonofont[Path=/usr/share/fonts/truetype/cmu/,UprightFont=cmuntt.ttf,BoldFont=cmuntb.ttf,ItalicFont=cmunit.ttf,BoldItalicFont=cmuntx.ttf]{cmuntt.ttf}\ttfamily colorlinks=false}\setmainfont[Path=/usr/share/fonts/truetype/cmu/,UprightFont=cmunrm.ttf,BoldFont=cmunbx.ttf,ItalicFont=cmunti.ttf,BoldItalicFont=cmunbi.ttf]{cmunrm.ttf}\setmonofont[Path=/usr/share/fonts/truetype/cmu/,UprightFont=cmuntt.ttf,BoldFont=cmuntb.ttf,ItalicFont=cmunit.ttf,BoldItalicFont=cmuntx.ttf]{cmunrm.ttf})\\ \hline \hspace*{0pt}\ignorespaces{}\hspace*{0pt}{\ttfamily \setmainfont[Path=/usr/share/fonts/truetype/cmu/,UprightFont=cmunrm.ttf,BoldFont=cmunbx.ttf,ItalicFont=cmunti.ttf,BoldItalicFont=cmunbi.ttf]{cmuntt.ttf}\setmonofont[Path=/usr/share/fonts/truetype/cmu/,UprightFont=cmuntt.ttf,BoldFont=cmuntb.ttf,ItalicFont=cmunit.ttf,BoldItalicFont=cmuntx.ttf]{cmuntt.ttf}\ttfamily citebordercolor}&\hspace*{0pt}\ignorespaces{}\hspace*{0pt}{\ttfamily =\{0 1 0\}}&\hspace*{0pt}\ignorespaces{}\hspace*{0pt}\setmainfont[Path=/usr/share/fonts/truetype/cmu/,UprightFont=cmunrm.ttf,BoldFont=cmunbx.ttf,ItalicFont=cmunti.ttf,BoldItalicFont=cmunbi.ttf]{cmunrm.ttf}\setmonofont[Path=/usr/share/fonts/truetype/cmu/,UprightFont=cmuntt.ttf,BoldFont=cmuntb.ttf,ItalicFont=cmunit.ttf,BoldItalicFont=cmuntx.ttf]{cmunrm.ttf}color of frame around citations\\ \hline \hspace*{0pt}\ignorespaces{}\hspace*{0pt}{\ttfamily \setmainfont[Path=/usr/share/fonts/truetype/cmu/,UprightFont=cmunrm.ttf,BoldFont=cmunbx.ttf,ItalicFont=cmunti.ttf,BoldItalicFont=cmunbi.ttf]{cmuntt.ttf}\setmonofont[Path=/usr/share/fonts/truetype/cmu/,UprightFont=cmuntt.ttf,BoldFont=cmuntb.ttf,ItalicFont=cmunit.ttf,BoldItalicFont=cmuntx.ttf]{cmuntt.ttf}\ttfamily urlbordercolor}&\hspace*{0pt}\ignorespaces{}\hspace*{0pt}{\ttfamily =\{0 1 1\}}&\hspace*{0pt}\ignorespaces{}\hspace*{0pt}\setmainfont[Path=/usr/share/fonts/truetype/cmu/,UprightFont=cmunrm.ttf,BoldFont=cmunbx.ttf,ItalicFont=cmunti.ttf,BoldItalicFont=cmunbi.ttf]{cmunrm.ttf}\setmonofont[Path=/usr/share/fonts/truetype/cmu/,UprightFont=cmuntt.ttf,BoldFont=cmuntb.ttf,ItalicFont=cmunit.ttf,BoldItalicFont=cmuntx.ttf]{cmunrm.ttf}color of frame around URL links\\ \hline 
\end{longtable}


Please note, that explicit RGB specification is only allowed for the border colors (like linkbordercolor etc.), while the others may only assigned to named colors (which you can define your own, see \mylref{147}{Colors}). In order to speed up your customization process, here is a list with the variables with their default value. Copy it in your document and make the changes you want. Next to the variables, there is a short explanations of their meaning:

\begin{Shaded}
\begin{Highlighting}[]

\NormalTok{\textbackslash{}hypersetup\{}
    \NormalTok{bookmarks=true,         }\CommentTok{% show bookmarks bar?}
    \NormalTok{unicode=false,          }\CommentTok{% non-Latin characters in Acrobat’s bookmarks}
    \NormalTok{pdftoolbar=true,        }\CommentTok{% show Acrobat’s toolbar?}
    \NormalTok{pdfmenubar=true,        }\CommentTok{% show Acrobat’s menu?}
    \NormalTok{pdffitwindow=false,     }\CommentTok{% window fit to page when opened}
    \NormalTok{pdfstartview=\{FitH\},    }\CommentTok{% fits the width of the page to the window}
    \NormalTok{pdftitle=\{My title\},    }\CommentTok{% title}
    \NormalTok{pdfauthor=\{Author\},     }\CommentTok{% author}
    \NormalTok{pdfsubject=\{Subject\},   }\CommentTok{% subject of the document}
    \NormalTok{pdfcreator=\{Creator\},   }\CommentTok{% creator of the document}
    \NormalTok{pdfproducer=\{Producer\}, }\CommentTok{% producer of the document}
    \NormalTok{pdfkeywords=\{keyword1, key2, key3\}, }\CommentTok{% list of keywords}
    \NormalTok{pdfnewwindow=true,      }\CommentTok{% links in new PDF window}
    \NormalTok{colorlinks=false,       }\CommentTok{% false: boxed links; true: colored links}
    \NormalTok{linkcolor=red,          }\CommentTok{% color of internal links (change box color with}
 \NormalTok{linkbordercolor)}
    \NormalTok{citecolor=green,        }\CommentTok{% color of links to bibliography}
    \NormalTok{filecolor=magenta,      }\CommentTok{% color of file links}
    \NormalTok{urlcolor=cyan           }\CommentTok{% color of external links}
\NormalTok{\}}
\end{Highlighting}
\end{Shaded}


If you don\textquotesingle{}t need such a high customization, here are some smaller but useful examples. When creating PDFs destined for printing, colored links are not a good thing as they end up in gray in the final output, making it difficult to read. You can use color frames, which are not printed:

\begin{Shaded}
\begin{Highlighting}[]

\NormalTok{\textbackslash{}usepackage\{hyperref\}}
\NormalTok{\textbackslash{}hypersetup\{colorlinks=false\}}
\end{Highlighting}
\end{Shaded}


or make links black:
\begin{Shaded}
\begin{Highlighting}[]

\NormalTok{\textbackslash{}usepackage[hidelinks]\{hyperref\}}
\end{Highlighting}
\end{Shaded}


When you just want to provide information for the Document Info section of the PDF file, as well as enabling back references inside bibliography:
\begin{Shaded}
\begin{Highlighting}[]

\NormalTok{\textbackslash{}usepackage[pdfauthor=\{Author's name\},}\CommentTok
\NormalTok{pagebackref=true,}\CommentTok{%}
\NormalTok{pdftex]\{hyperref\}}
\end{Highlighting}
\end{Shaded}


By default, URLs are printed using mono-{}spaced fonts. If you don\textquotesingle{}t like it and you want them to be printed with the same style of the rest of the text, you can use this:
\begin{Shaded}
\begin{Highlighting}[]

\NormalTok{\textbackslash{}urlstyle\{same\}}
\end{Highlighting}
\end{Shaded}

\section{Troubleshooting}
\label{404}
\subsection{Problems with Links and Equations 1}
\label{405}
Messages like the following\\

\TemplateSpaceIndent{$\text{ }${}!$\text{ }${}pdfTeX$\text{ }${}warning$\text{ }${}(ext4):$\text{ }${}destination$\text{ }${}with$\text{ }${}the$\text{ }${}same$\text{ }${}identifier$\text{ }${}(name\{$\text{ }$\newline{}
$\text{ }${}equation.1.7.7.30\})$\text{ }${}has$\text{ }${}been$\text{ }${}already$\text{ }${}used,$\text{ }${}duplicate$\text{ }${}ignored}


appear, when you have made something like

\begin{Shaded}
\begin{Highlighting}[]

\NormalTok{\textbackslash{}begin\{eqnarray\}a=b\textbackslash{}nonumber\textbackslash{}end\{eqnarray\}}
\end{Highlighting}
\end{Shaded}


The error disappears, if you use instead this form:

\begin{Shaded}
\begin{Highlighting}[]

\NormalTok{\textbackslash{}begin\{eqnarray*\}a=b\textbackslash{}end\{eqnarray*\}}
\end{Highlighting}
\end{Shaded}


Beware that the shown line number is often completely different from the erroneous line. 

Possible solution: Place the \LaTeXTT{amsmath} package before the \LaTeXTT{hyperref} package.
\subsection{Problems with Links and Equations 2}
\label{406}

Messages like the following\\

\TemplateSpaceIndent{$\text{ }${}!$\text{ }${}Runaway$\text{ }${}argument?$\text{ }$\newline{}
$\text{ }${}\{\textbackslash{}@firstoffive$\text{ }${}\}\textbackslash{}fi$\text{ }${}),$\text{ }${}Some$\text{ }${}text$\text{ }${}from$\text{ }${}your$\text{ }${}document$\text{ }${}here$\text{ }${}(\textbackslash{}ref$\text{ }${}\{re\textbackslash{}ETC.$\text{ }$\newline{}
$\text{ }${}Latex$\text{ }${}Error:$\text{ }${}Paragraph$\text{ }${}ended$\text{ }${}before$\text{ }${}\textbackslash{}Hy@setref@link$\text{ }${}was$\text{ }${}complete.}


appear when you use {\ttfamily \setmainfont[Path=/usr/share/fonts/truetype/cmu/,UprightFont=cmunrm.ttf,BoldFont=cmunbx.ttf,ItalicFont=cmunti.ttf,BoldItalicFont=cmunbi.ttf]{cmuntt.ttf}\setmonofont[Path=/usr/share/fonts/truetype/cmu/,UprightFont=cmuntt.ttf,BoldFont=cmuntb.ttf,ItalicFont=cmunit.ttf,BoldItalicFont=cmuntx.ttf]{cmuntt.ttf}\ttfamily \textbackslash{}label}{$\text{ }$}\setmainfont[Path=/usr/share/fonts/truetype/cmu/,UprightFont=cmunrm.ttf,BoldFont=cmunbx.ttf,ItalicFont=cmunti.ttf,BoldItalicFont=cmunbi.ttf]{cmunrm.ttf}\setmonofont[Path=/usr/share/fonts/truetype/cmu/,UprightFont=cmuntt.ttf,BoldFont=cmuntb.ttf,ItalicFont=cmunit.ttf,BoldItalicFont=cmuntx.ttf]{cmunrm.ttf} inside an {\ttfamily \setmainfont[Path=/usr/share/fonts/truetype/cmu/,UprightFont=cmunrm.ttf,BoldFont=cmunbx.ttf,ItalicFont=cmunti.ttf,BoldItalicFont=cmunbi.ttf]{cmuntt.ttf}\setmonofont[Path=/usr/share/fonts/truetype/cmu/,UprightFont=cmuntt.ttf,BoldFont=cmuntb.ttf,ItalicFont=cmunit.ttf,BoldItalicFont=cmuntx.ttf]{cmuntt.ttf}\ttfamily align}{$\text{ }$}\setmainfont[Path=/usr/share/fonts/truetype/cmu/,UprightFont=cmunrm.ttf,BoldFont=cmunbx.ttf,ItalicFont=cmunti.ttf,BoldItalicFont=cmunbi.ttf]{cmunrm.ttf}\setmonofont[Path=/usr/share/fonts/truetype/cmu/,UprightFont=cmuntt.ttf,BoldFont=cmuntb.ttf,ItalicFont=cmunit.ttf,BoldItalicFont=cmuntx.ttf]{cmunrm.ttf} environment. 

Possible solution: Add the following to your preamble:

\begin{Shaded}
\begin{Highlighting}[]

\NormalTok{\textbackslash{}AtBeginDocument\{\textbackslash{}let\textbackslash{}textlabel\textbackslash{}label}\AlertTok{\}}
\end{Highlighting}
\end{Shaded}

\subsection{Problems with Links and Pages}
\label{407}

Messages like the following:\\

\TemplateSpaceIndent{$\text{ }${}!$\text{ }${}pdfTeX$\text{ }${}warning$\text{ }${}(ext4):$\text{ }${}destination$\text{ }${}with$\text{ }${}the$\text{ }${}same$\text{ }$\newline{}
$\text{ }${}identifier$\text{ }${}(name\{page.1\})$\text{ }${}has$\text{ }${}been$\text{ }${}already$\text{ }${}used,$\text{ }$\newline{}
$\text{ }${}duplicate$\text{ }${}ignored}


appear when a counter gets reinitialized, for example by using the command \LaTeXTT{\textbackslash{}mainmatter} provided by the book document class. It resets the page number counter to 1 prior to the first chapter of the book. But as the preface of the book also has a page number 1 all links to \symbol{34}page 1\symbol{34} would not be unique anymore, hence the notice that \symbol{34}duplicate has been ignored.\symbol{34} The counter measure consists of putting \LaTeXTT{plainpages=false} into the \LaTeXTT{hyperref} options. This unfortunately only helps with the page counter. An even more radical solution is to use the option \LaTeXTT{hypertexnames=false}, but this will cause the page links in the index to stop working.

The best solution is to give each page a unique name by using the \LaTeXTT{\textbackslash{}pagenumbering} command:

\begin{Shaded}
\begin{Highlighting}[]

\NormalTok{\textbackslash{}pagenumbering\{alph\}    }\CommentTok{% a, b, c, ...}
\NormalTok{... titlepage, other front matter ...}
\NormalTok{\textbackslash{}pagenumbering\{roman\}   }\CommentTok{% i, ii, iii, iv, ...}
\NormalTok{... table of contents, table of figures, ...}
\NormalTok{\textbackslash{}pagenumbering\{arabic\}  }\CommentTok{% 1, 2, 3, 4, ...}
\NormalTok{... beginning of the main matter (chapter 1) ...}
\end{Highlighting}
\end{Shaded}


Another solution is to use \LaTeXTT{\textbackslash{}pagenumbering\{alph\}} before the command \LaTeXTT{\textbackslash{}maketitle}, which will give the title page the label page.a. Since the page number is suppressed, it won\textquotesingle{}t make a difference to the output.

By changing the page numbering every time before the counter is reset, each page gets a unique name. In this case, the pages would be numbered a, b, c, i, ii, iii, iv, v, 1, 2, 3, 4, 5, ...

If you don\textquotesingle{}t want the page numbers to be visible (for example, during the front matter part), use \LaTeXTT{\textbackslash{}pagestyle\{empty\} ... \textbackslash{}pagestyle\{plain\}}. The important point is that although the numbers are not visible, each page will have a unique name.

Another more flexible approach is to set the counter to something negative:

\begin{Shaded}
\begin{Highlighting}[]

\NormalTok{\textbackslash{}setcounter\{page\}\{-100\}}
\NormalTok{... titlepage, other front matter ...}
\NormalTok{\textbackslash{}pagenumbering\{roman\}   }\CommentTok{% i, ii, iii, iv, ...}
\NormalTok{... table of contents, table of figures, ...}
\NormalTok{\textbackslash{}pagenumbering\{arabic\}  }\CommentTok{% 1, 2, 3, 4, ...}
\NormalTok{... beginning of the main matter (chapter 1) ...}
\end{Highlighting}
\end{Shaded}


which will give the first pages a unique negative number.

The problem can also occur with the {\ttfamily \setmainfont[Path=/usr/share/fonts/truetype/cmu/,UprightFont=cmunrm.ttf,BoldFont=cmunbx.ttf,ItalicFont=cmunti.ttf,BoldItalicFont=cmunbi.ttf]{cmuntt.ttf}\setmonofont[Path=/usr/share/fonts/truetype/cmu/,UprightFont=cmuntt.ttf,BoldFont=cmuntb.ttf,ItalicFont=cmunit.ttf,BoldItalicFont=cmuntx.ttf]{cmuntt.ttf}\ttfamily algorithms}{$\text{ }$}\setmainfont[Path=/usr/share/fonts/truetype/cmu/,UprightFont=cmunrm.ttf,BoldFont=cmunbx.ttf,ItalicFont=cmunti.ttf,BoldItalicFont=cmunbi.ttf]{cmunrm.ttf}\setmonofont[Path=/usr/share/fonts/truetype/cmu/,UprightFont=cmuntt.ttf,BoldFont=cmuntb.ttf,ItalicFont=cmunit.ttf,BoldItalicFont=cmuntx.ttf]{cmunrm.ttf} package: because each algorithm uses the same line-{}numbering scheme, the line identifiers for the second and follow-{}on algorithms will be duplicates of the first.

The problem occurs with equation identifiers if you use \LaTeXTT{\textbackslash{}nonumber} on every line of an eqnarray environment.  In this case, use the *\textquotesingle{}ed form instead, e.g. \LaTeXTT{\textbackslash{}begin\{eqnarray*\} ... \textbackslash{}end\{eqnarray*\}} (which is an unnumbered equation array), and remove the now unnecessary \LaTeXTT{\textbackslash{}nonumber} commands.

If your url\textquotesingle{}s are too long and running off of the page, try using the {\ttfamily \setmainfont[Path=/usr/share/fonts/truetype/cmu/,UprightFont=cmunrm.ttf,BoldFont=cmunbx.ttf,ItalicFont=cmunti.ttf,BoldItalicFont=cmunbi.ttf]{cmuntt.ttf}\setmonofont[Path=/usr/share/fonts/truetype/cmu/,UprightFont=cmuntt.ttf,BoldFont=cmuntb.ttf,ItalicFont=cmunit.ttf,BoldItalicFont=cmuntx.ttf]{cmuntt.ttf}\ttfamily breakurl}{$\text{ }$}\setmainfont[Path=/usr/share/fonts/truetype/cmu/,UprightFont=cmunrm.ttf,BoldFont=cmunbx.ttf,ItalicFont=cmunti.ttf,BoldItalicFont=cmunbi.ttf]{cmunrm.ttf}\setmonofont[Path=/usr/share/fonts/truetype/cmu/,UprightFont=cmuntt.ttf,BoldFont=cmuntb.ttf,ItalicFont=cmunit.ttf,BoldItalicFont=cmuntx.ttf]{cmunrm.ttf} package to split the url over multiple lines.  This is especially important in a multicolumn environment where the line width is greatly shortened.
\subsection{Problems with bookmarks}
\label{408}
The text displayed by bookmarks does not always look like you expect it
to look. Because bookmarks are \symbol{34}just text\symbol{34}, much fewer characters are
available for bookmarks than for normal LaTeX text. Hyperref will normally
notice such problems and put up a warning:
\\

\TemplateSpaceIndent{$\text{ }${}Package$\text{ }${}hyperref$\text{ }${}Warning:$\text{ }$\newline{}
$\text{ }${}Token$\text{ }${}not$\text{ }${}allowed$\text{ }${}in$\text{ }${}a$\text{ }${}PDFDocEncoded$\text{ }${}string:}


You can now work around this problem by providing a text string for the bookmarks, which replaces the offending text:

\begin{Shaded}
\begin{Highlighting}[]

\NormalTok{\textbackslash{}texorpdfstring\{''TEX text''\}\{''Bookmark Text''\}}
\end{Highlighting}
\end{Shaded}


Math expressions are a prime candidate for this kind of problem:

\begin{Shaded}
\begin{Highlighting}[]

\NormalTok{\textbackslash{}section\{ \textbackslash{}texorpdfstring\{$E=mc^2$\}\{E=mc2\} \}}
\end{Highlighting}
\end{Shaded}


which turns \LaTeXTT{\textbackslash{}section\{\${}E=mc\^{}2\${}\}} to {\ttfamily \setmainfont[Path=/usr/share/fonts/truetype/cmu/,UprightFont=cmunrm.ttf,BoldFont=cmunbx.ttf,ItalicFont=cmunti.ttf,BoldItalicFont=cmunbi.ttf]{cmuntt.ttf}\setmonofont[Path=/usr/share/fonts/truetype/cmu/,UprightFont=cmuntt.ttf,BoldFont=cmuntb.ttf,ItalicFont=cmunit.ttf,BoldItalicFont=cmuntx.ttf]{cmuntt.ttf}\ttfamily E=mc2}{$\text{ }$}\setmainfont[Path=/usr/share/fonts/truetype/cmu/,UprightFont=cmunrm.ttf,BoldFont=cmunbx.ttf,ItalicFont=cmunti.ttf,BoldItalicFont=cmunbi.ttf]{cmunrm.ttf}\setmonofont[Path=/usr/share/fonts/truetype/cmu/,UprightFont=cmuntt.ttf,BoldFont=cmuntb.ttf,ItalicFont=cmunit.ttf,BoldItalicFont=cmuntx.ttf]{cmunrm.ttf} in the bookmark area. Color changes also do not travel well into bookmarks:

\begin{Shaded}
\begin{Highlighting}[]

\NormalTok{\textbackslash{}section\{ \textbackslash{}textcolor\{red\}\{Red !\} \}}
\end{Highlighting}
\end{Shaded}


produces the string \symbol{34}redRed!\symbol{34}. The command \LaTeXTT{\textbackslash{}textcolor} gets ignored but its argument (red) gets printed.
If you use:

\begin{Shaded}
\begin{Highlighting}[]

\NormalTok{\textbackslash{}section\{ \textbackslash{}texorpdfstring\{\textbackslash{}textcolor\{red\}\{Red !\}\}\{Red\textbackslash{} !\} \}}
\end{Highlighting}
\end{Shaded}


the result will be much more legible.

If you write your document in unicode and use the \LaTeXTT{unicode} option for the \LaTeXTT{hyperref} package you can use unicode characters in bookmarks. This will give you a much larger selection of characters to pick from when using \LaTeXTT{\textbackslash{}texorpdfstring}.
\subsection{Problems with tables and figures}
\label{409}

The links created by \LaTeXTT{hyperref} point to the label created within the float environment, which, as \mylref{373}{previously described}, must always be set after the caption. Since the caption is usually below a figure or table, the figure or table itself will not be visible upon clicking the link\myfootnote{\myplainurl{http://www.ctan.org/tex-archive/macros/latex/contrib/hyperref/README}}. A workaround exists by using the package \LaTeXTT{hypcap} \myplainurl{http://www.ctan.org/tex-archive/macros/latex/contrib/oberdiek/hypcap.pdf} with:

\begin{Shaded}
\begin{Highlighting}[]

\NormalTok{\textbackslash{}usepackage[all]\{hypcap\}}
\end{Highlighting}
\end{Shaded}


Be sure to call this package {\itshape \setmainfont[Path=/usr/share/fonts/truetype/cmu/,UprightFont=cmunrm.ttf,BoldFont=cmunbx.ttf,ItalicFont=cmunti.ttf,BoldItalicFont=cmunbi.ttf]{cmunti.ttf}\setmonofont[Path=/usr/share/fonts/truetype/cmu/,UprightFont=cmuntt.ttf,BoldFont=cmuntb.ttf,ItalicFont=cmunit.ttf,BoldItalicFont=cmuntx.ttf]{cmunti.ttf}\itshape after}{$\text{ }$}\setmainfont[Path=/usr/share/fonts/truetype/cmu/,UprightFont=cmunrm.ttf,BoldFont=cmunbx.ttf,ItalicFont=cmunti.ttf,BoldItalicFont=cmunbi.ttf]{cmunrm.ttf}\setmonofont[Path=/usr/share/fonts/truetype/cmu/,UprightFont=cmuntt.ttf,BoldFont=cmuntb.ttf,ItalicFont=cmunit.ttf,BoldItalicFont=cmuntx.ttf]{cmunrm.ttf} loading \LaTeXTT{hyperref}.

If you use the \LaTeXTT{wrapfig} package\myfootnote{\myfnhref{http://www.ctan.org/tex-archive/macros/latex/contrib/wrapfig}{Wrapfig package webpage in CTAN}} mentioned in the \symbol{34}\mylref{374}{Wrapping text around figures}\symbol{34} section of the \symbol{34}Floats, Figures and Captions\symbol{34} chapter, or other similar packages that define their own environments, you will need to manually include \LaTeXTT{\textbackslash{}capstart} in those environments, {\itshape \setmainfont[Path=/usr/share/fonts/truetype/cmu/,UprightFont=cmunrm.ttf,BoldFont=cmunbx.ttf,ItalicFont=cmunti.ttf,BoldItalicFont=cmunbi.ttf]{cmunti.ttf}\setmonofont[Path=/usr/share/fonts/truetype/cmu/,UprightFont=cmuntt.ttf,BoldFont=cmuntb.ttf,ItalicFont=cmunit.ttf,BoldItalicFont=cmuntx.ttf]{cmunti.ttf}\itshape e.g.}\setmainfont[Path=/usr/share/fonts/truetype/cmu/,UprightFont=cmunrm.ttf,BoldFont=cmunbx.ttf,ItalicFont=cmunti.ttf,BoldItalicFont=cmunbi.ttf]{cmunrm.ttf}\setmonofont[Path=/usr/share/fonts/truetype/cmu/,UprightFont=cmuntt.ttf,BoldFont=cmuntb.ttf,ItalicFont=cmunit.ttf,BoldItalicFont=cmuntx.ttf]{cmunrm.ttf}:

\begin{Shaded}
\begin{Highlighting}[]

\NormalTok{\textbackslash{}begin\{wrapfigure\}\{R\}\{0.5\textbackslash{}textwidth\}}
  \NormalTok{\textbackslash{}capstart}
  \NormalTok{\textbackslash{}begin\{center\}}
    \NormalTok{\textbackslash{}includegraphics[width=0.48\textbackslash{}textwidth]\{filename\}}
  \NormalTok{\textbackslash{}end\{center\}  }
  \NormalTok{\textbackslash{}caption\{\textbackslash{}label\{labelname\}a figure\}}
\NormalTok{\textbackslash{}end\{wrapfigure\}}
\end{Highlighting}
\end{Shaded}

\subsection{Problems with long caption and \textbackslash{}listoffigures or long title}
\label{410}
There is an issue when using \LaTeXTT{\textbackslash{}listoffigures} with \LaTeXTT{hyperref} for long captions or long titles. This happens when the captions (or the titles) are longer than the page width (about 7-{}9 words depending on your settings). To fix this, you need to use the option breaklinks when first declaring:

\begin{Shaded}
\begin{Highlighting}[]

\NormalTok{\textbackslash{}usepackage[breaklinks]\{hyperref\}}
\end{Highlighting}
\end{Shaded}


This will then cause the links in the \LaTeXTT{\textbackslash{}listoffigures} to word wrap properly.
\subsection{Problems with already existing .toc, .lof and similar files}
\label{411}
The format of some of the auxilliary files generated by latex changes when you include the \LaTeXTT{hyperref} package. One can therefore encounter errors like\\

\TemplateSpaceIndent{$\text{ }${}!$\text{ }${}Argument$\text{ }${}of$\text{ }${}\textbackslash{}Hy@setref@link$\text{ }${}has$\text{ }${}an$\text{ }${}extra$\text{ }${}\}.}

when the document is typeset with \LaTeXTT{hyperref} for the first time and these files already exist. The solution to the problem is to delete all the files that latex uses to get references right and typeset again.
\subsection{Problems with footnotes and special characters}
\label{412}
See the \mylref{388}{relevant section}.
\subsection{Problems with Beamer}
\label{413}
Using the command

\begin{Shaded}
\begin{Highlighting}[]

\NormalTok{\textbackslash{}hyperref[some_label]\{some text\}}
\end{Highlighting}
\end{Shaded}


is broken when pointed at a label. Instead of sending the user to the desired label, upon clicking the user will be sent to the first frame. A simple work around exists; instead of using

\begin{Shaded}
\begin{Highlighting}[]

\NormalTok{\textbackslash{}phantomsection\textbackslash{}label\{some_label\}}
\end{Highlighting}
\end{Shaded}


to label your frames, use

\begin{Shaded}
\begin{Highlighting}[]

\NormalTok{\textbackslash{}hypertarget\{some_label\}\{\}}
\end{Highlighting}
\end{Shaded}


and reference it with

\begin{Shaded}
\begin{Highlighting}[]

\NormalTok{\textbackslash{}hyperlink\{some_label\}\{some text\}}
\end{Highlighting}
\end{Shaded}

\subsection{Problems with draft mode}
\label{414}

{\bfseries \setmainfont[Path=/usr/share/fonts/truetype/cmu/,UprightFont=cmunrm.ttf,BoldFont=cmunbx.ttf,ItalicFont=cmunti.ttf,BoldItalicFont=cmunbi.ttf]{cmunbx.ttf}\setmonofont[Path=/usr/share/fonts/truetype/cmu/,UprightFont=cmuntt.ttf,BoldFont=cmuntb.ttf,ItalicFont=cmunit.ttf,BoldItalicFont=cmuntx.ttf]{cmunbx.ttf}\bfseries WARNING!}{$\text{ }$}\setmainfont[Path=/usr/share/fonts/truetype/cmu/,UprightFont=cmunrm.ttf,BoldFont=cmunbx.ttf,ItalicFont=cmunti.ttf,BoldItalicFont=cmunbi.ttf]{cmunrm.ttf}\setmonofont[Path=/usr/share/fonts/truetype/cmu/,UprightFont=cmuntt.ttf,BoldFont=cmuntb.ttf,ItalicFont=cmunit.ttf,BoldItalicFont=cmuntx.ttf]{cmunrm.ttf} Please note that if you have activated the \symbol{34}draft\symbol{34}-{}option in your \textbackslash{}documentclass declaration the hyperlinks will not show up in the table of contents, or anywhere else for that matter!!!

The hyperlinks can be re-{}enabled by using the \symbol{34}final=true\symbol{34} option in the following initialization of the hyperref package, just after the package was included:
\begin{Shaded}
\begin{Highlighting}[]

\NormalTok{\textbackslash{}usepackage\{hyperref\}}
\NormalTok{\textbackslash{}hypersetup\{final=true\}}
\end{Highlighting}
\end{Shaded}


A good source of further options for the hyperref package can be found here \myfootnote{\myplainurl{http://hyperref.de/?page_id=44}Hyperref -{} Hyperlinks With LaTeX Page}.
\section{Notes and References}
\label{415}
\LaTeXNullTemplate{}

\chapter{Labels and Cross-{}referencing}

\myminitoc
\label{416}

\label{417}

\section{Introduction}
\label{418}
In LaTeX you can easily reference almost anything that is numbered (sections, figures, formulas), and LaTeX will take care of numbering, updating it whenever necessary. The commands to be used do not depend on what you are referencing, and they are:{\bfseries
\begin{mydescription}{\ttfamily \setmainfont[Path=/usr/share/fonts/truetype/cmu/,UprightFont=cmunrm.ttf,BoldFont=cmunbx.ttf,ItalicFont=cmunti.ttf,BoldItalicFont=cmunbi.ttf]{cmuntt.ttf}\setmonofont[Path=/usr/share/fonts/truetype/cmu/,UprightFont=cmuntt.ttf,BoldFont=cmuntb.ttf,ItalicFont=cmunit.ttf,BoldItalicFont=cmuntx.ttf]{cmuntt.ttf}\ttfamily \textbackslash{}label\{{\itshape \setmainfont[Path=/usr/share/fonts/truetype/cmu/,UprightFont=cmunrm.ttf,BoldFont=cmunbx.ttf,ItalicFont=cmunti.ttf,BoldItalicFont=cmunbi.ttf]{cmunit.ttf}\setmonofont[Path=/usr/share/fonts/truetype/cmu/,UprightFont=cmuntt.ttf,BoldFont=cmuntb.ttf,ItalicFont=cmunit.ttf,BoldItalicFont=cmuntx.ttf]{cmunit.ttf}\ttfamily \itshape marker}\setmainfont[Path=/usr/share/fonts/truetype/cmu/,UprightFont=cmunrm.ttf,BoldFont=cmunbx.ttf,ItalicFont=cmunti.ttf,BoldItalicFont=cmunbi.ttf]{cmuntt.ttf}\setmonofont[Path=/usr/share/fonts/truetype/cmu/,UprightFont=cmuntt.ttf,BoldFont=cmuntb.ttf,ItalicFont=cmunit.ttf,BoldItalicFont=cmuntx.ttf]{cmuntt.ttf}\ttfamily \}}
\end{mydescription}
}

\begin{myquote}
\item{} \setmainfont[Path=/usr/share/fonts/truetype/cmu/,UprightFont=cmunrm.ttf,BoldFont=cmunbx.ttf,ItalicFont=cmunti.ttf,BoldItalicFont=cmunbi.ttf]{cmunrm.ttf}\setmonofont[Path=/usr/share/fonts/truetype/cmu/,UprightFont=cmuntt.ttf,BoldFont=cmuntb.ttf,ItalicFont=cmunit.ttf,BoldItalicFont=cmuntx.ttf]{cmunrm.ttf}you give the object you want to reference a {\itshape \setmainfont[Path=/usr/share/fonts/truetype/cmu/,UprightFont=cmunrm.ttf,BoldFont=cmunbx.ttf,ItalicFont=cmunti.ttf,BoldItalicFont=cmunbi.ttf]{cmunti.ttf}\setmonofont[Path=/usr/share/fonts/truetype/cmu/,UprightFont=cmuntt.ttf,BoldFont=cmuntb.ttf,ItalicFont=cmunit.ttf,BoldItalicFont=cmuntx.ttf]{cmunti.ttf}\itshape marker}\setmainfont[Path=/usr/share/fonts/truetype/cmu/,UprightFont=cmunrm.ttf,BoldFont=cmunbx.ttf,ItalicFont=cmunti.ttf,BoldItalicFont=cmunbi.ttf]{cmunrm.ttf}\setmonofont[Path=/usr/share/fonts/truetype/cmu/,UprightFont=cmuntt.ttf,BoldFont=cmuntb.ttf,ItalicFont=cmunit.ttf,BoldItalicFont=cmuntx.ttf]{cmunrm.ttf}, you can see it like a name.
\end{myquote}
{\bfseries
\begin{mydescription}{\ttfamily \setmainfont[Path=/usr/share/fonts/truetype/cmu/,UprightFont=cmunrm.ttf,BoldFont=cmunbx.ttf,ItalicFont=cmunti.ttf,BoldItalicFont=cmunbi.ttf]{cmuntt.ttf}\setmonofont[Path=/usr/share/fonts/truetype/cmu/,UprightFont=cmuntt.ttf,BoldFont=cmuntb.ttf,ItalicFont=cmunit.ttf,BoldItalicFont=cmuntx.ttf]{cmuntt.ttf}\ttfamily \textbackslash{}ref\{{\itshape \setmainfont[Path=/usr/share/fonts/truetype/cmu/,UprightFont=cmunrm.ttf,BoldFont=cmunbx.ttf,ItalicFont=cmunti.ttf,BoldItalicFont=cmunbi.ttf]{cmunit.ttf}\setmonofont[Path=/usr/share/fonts/truetype/cmu/,UprightFont=cmuntt.ttf,BoldFont=cmuntb.ttf,ItalicFont=cmunit.ttf,BoldItalicFont=cmuntx.ttf]{cmunit.ttf}\ttfamily \itshape marker}\setmainfont[Path=/usr/share/fonts/truetype/cmu/,UprightFont=cmunrm.ttf,BoldFont=cmunbx.ttf,ItalicFont=cmunti.ttf,BoldItalicFont=cmunbi.ttf]{cmuntt.ttf}\setmonofont[Path=/usr/share/fonts/truetype/cmu/,UprightFont=cmuntt.ttf,BoldFont=cmuntb.ttf,ItalicFont=cmunit.ttf,BoldItalicFont=cmuntx.ttf]{cmuntt.ttf}\ttfamily \}}
\end{mydescription}
}

\begin{myquote}
\item{} \setmainfont[Path=/usr/share/fonts/truetype/cmu/,UprightFont=cmunrm.ttf,BoldFont=cmunbx.ttf,ItalicFont=cmunti.ttf,BoldItalicFont=cmunbi.ttf]{cmunrm.ttf}\setmonofont[Path=/usr/share/fonts/truetype/cmu/,UprightFont=cmuntt.ttf,BoldFont=cmuntb.ttf,ItalicFont=cmunit.ttf,BoldItalicFont=cmuntx.ttf]{cmunrm.ttf}you can reference the object you have {\itshape \setmainfont[Path=/usr/share/fonts/truetype/cmu/,UprightFont=cmunrm.ttf,BoldFont=cmunbx.ttf,ItalicFont=cmunti.ttf,BoldItalicFont=cmunbi.ttf]{cmunti.ttf}\setmonofont[Path=/usr/share/fonts/truetype/cmu/,UprightFont=cmuntt.ttf,BoldFont=cmuntb.ttf,ItalicFont=cmunit.ttf,BoldItalicFont=cmuntx.ttf]{cmunti.ttf}\itshape marked}{$\text{ }$}\setmainfont[Path=/usr/share/fonts/truetype/cmu/,UprightFont=cmunrm.ttf,BoldFont=cmunbx.ttf,ItalicFont=cmunti.ttf,BoldItalicFont=cmunbi.ttf]{cmunrm.ttf}\setmonofont[Path=/usr/share/fonts/truetype/cmu/,UprightFont=cmuntt.ttf,BoldFont=cmuntb.ttf,ItalicFont=cmunit.ttf,BoldItalicFont=cmuntx.ttf]{cmunrm.ttf} before. This prints the number that was assigned to the object.
\end{myquote}
{\bfseries
\begin{mydescription}{\ttfamily \setmainfont[Path=/usr/share/fonts/truetype/cmu/,UprightFont=cmunrm.ttf,BoldFont=cmunbx.ttf,ItalicFont=cmunti.ttf,BoldItalicFont=cmunbi.ttf]{cmuntt.ttf}\setmonofont[Path=/usr/share/fonts/truetype/cmu/,UprightFont=cmuntt.ttf,BoldFont=cmuntb.ttf,ItalicFont=cmunit.ttf,BoldItalicFont=cmuntx.ttf]{cmuntt.ttf}\ttfamily \textbackslash{}pageref\{{\itshape \setmainfont[Path=/usr/share/fonts/truetype/cmu/,UprightFont=cmunrm.ttf,BoldFont=cmunbx.ttf,ItalicFont=cmunti.ttf,BoldItalicFont=cmunbi.ttf]{cmunit.ttf}\setmonofont[Path=/usr/share/fonts/truetype/cmu/,UprightFont=cmuntt.ttf,BoldFont=cmuntb.ttf,ItalicFont=cmunit.ttf,BoldItalicFont=cmuntx.ttf]{cmunit.ttf}\ttfamily \itshape marker}\setmainfont[Path=/usr/share/fonts/truetype/cmu/,UprightFont=cmunrm.ttf,BoldFont=cmunbx.ttf,ItalicFont=cmunti.ttf,BoldItalicFont=cmunbi.ttf]{cmuntt.ttf}\setmonofont[Path=/usr/share/fonts/truetype/cmu/,UprightFont=cmuntt.ttf,BoldFont=cmuntb.ttf,ItalicFont=cmunit.ttf,BoldItalicFont=cmuntx.ttf]{cmuntt.ttf}\ttfamily \}}
\end{mydescription}
}

\begin{myquote}
\item{} \setmainfont[Path=/usr/share/fonts/truetype/cmu/,UprightFont=cmunrm.ttf,BoldFont=cmunbx.ttf,ItalicFont=cmunti.ttf,BoldItalicFont=cmunbi.ttf]{cmunrm.ttf}\setmonofont[Path=/usr/share/fonts/truetype/cmu/,UprightFont=cmuntt.ttf,BoldFont=cmuntb.ttf,ItalicFont=cmunit.ttf,BoldItalicFont=cmuntx.ttf]{cmunrm.ttf}It will print the number of the page where the object is.
\end{myquote}


LaTeX will calculate the right numbering for the objects in the document; the {\itshape \setmainfont[Path=/usr/share/fonts/truetype/cmu/,UprightFont=cmunrm.ttf,BoldFont=cmunbx.ttf,ItalicFont=cmunti.ttf,BoldItalicFont=cmunbi.ttf]{cmunti.ttf}\setmonofont[Path=/usr/share/fonts/truetype/cmu/,UprightFont=cmuntt.ttf,BoldFont=cmuntb.ttf,ItalicFont=cmunit.ttf,BoldItalicFont=cmuntx.ttf]{cmunti.ttf}\itshape marker}{$\text{ }$}\setmainfont[Path=/usr/share/fonts/truetype/cmu/,UprightFont=cmunrm.ttf,BoldFont=cmunbx.ttf,ItalicFont=cmunti.ttf,BoldItalicFont=cmunbi.ttf]{cmunrm.ttf}\setmonofont[Path=/usr/share/fonts/truetype/cmu/,UprightFont=cmuntt.ttf,BoldFont=cmuntb.ttf,ItalicFont=cmunit.ttf,BoldItalicFont=cmuntx.ttf]{cmunrm.ttf} you have used to label the object will not be shown anywhere in the document. Then LaTeX will replace the string \symbol{34}{\ttfamily \setmainfont[Path=/usr/share/fonts/truetype/cmu/,UprightFont=cmunrm.ttf,BoldFont=cmunbx.ttf,ItalicFont=cmunti.ttf,BoldItalicFont=cmunbi.ttf]{cmuntt.ttf}\setmonofont[Path=/usr/share/fonts/truetype/cmu/,UprightFont=cmuntt.ttf,BoldFont=cmuntb.ttf,ItalicFont=cmunit.ttf,BoldItalicFont=cmuntx.ttf]{cmuntt.ttf}\ttfamily \textbackslash{}ref\{{\itshape \setmainfont[Path=/usr/share/fonts/truetype/cmu/,UprightFont=cmunrm.ttf,BoldFont=cmunbx.ttf,ItalicFont=cmunti.ttf,BoldItalicFont=cmunbi.ttf]{cmunit.ttf}\setmonofont[Path=/usr/share/fonts/truetype/cmu/,UprightFont=cmuntt.ttf,BoldFont=cmuntb.ttf,ItalicFont=cmunit.ttf,BoldItalicFont=cmuntx.ttf]{cmunit.ttf}\ttfamily \itshape marker}\setmainfont[Path=/usr/share/fonts/truetype/cmu/,UprightFont=cmunrm.ttf,BoldFont=cmunbx.ttf,ItalicFont=cmunti.ttf,BoldItalicFont=cmunbi.ttf]{cmuntt.ttf}\setmonofont[Path=/usr/share/fonts/truetype/cmu/,UprightFont=cmuntt.ttf,BoldFont=cmuntb.ttf,ItalicFont=cmunit.ttf,BoldItalicFont=cmuntx.ttf]{cmuntt.ttf}\ttfamily \}}\setmainfont[Path=/usr/share/fonts/truetype/cmu/,UprightFont=cmunrm.ttf,BoldFont=cmunbx.ttf,ItalicFont=cmunti.ttf,BoldItalicFont=cmunbi.ttf]{cmunrm.ttf}\setmonofont[Path=/usr/share/fonts/truetype/cmu/,UprightFont=cmuntt.ttf,BoldFont=cmuntb.ttf,ItalicFont=cmunit.ttf,BoldItalicFont=cmuntx.ttf]{cmunrm.ttf}\symbol{34} with the right number that was assigned to the object. If you reference a {\itshape \setmainfont[Path=/usr/share/fonts/truetype/cmu/,UprightFont=cmunrm.ttf,BoldFont=cmunbx.ttf,ItalicFont=cmunti.ttf,BoldItalicFont=cmunbi.ttf]{cmunti.ttf}\setmonofont[Path=/usr/share/fonts/truetype/cmu/,UprightFont=cmuntt.ttf,BoldFont=cmuntb.ttf,ItalicFont=cmunit.ttf,BoldItalicFont=cmuntx.ttf]{cmunti.ttf}\itshape marker}{$\text{ }$}\setmainfont[Path=/usr/share/fonts/truetype/cmu/,UprightFont=cmunrm.ttf,BoldFont=cmunbx.ttf,ItalicFont=cmunti.ttf,BoldItalicFont=cmunbi.ttf]{cmunrm.ttf}\setmonofont[Path=/usr/share/fonts/truetype/cmu/,UprightFont=cmuntt.ttf,BoldFont=cmuntb.ttf,ItalicFont=cmunit.ttf,BoldItalicFont=cmuntx.ttf]{cmunrm.ttf} that does not exist, the compilation of the document will be successful but LaTeX will return a warning:\\

\TemplateSpaceIndent{$\text{ }${}LaTeX$\text{ }${}Warning:$\text{ }${}There$\text{ }${}were$\text{ }${}undefined$\text{ }${}references.}

and it will replace \symbol{34}{\ttfamily \setmainfont[Path=/usr/share/fonts/truetype/cmu/,UprightFont=cmunrm.ttf,BoldFont=cmunbx.ttf,ItalicFont=cmunti.ttf,BoldItalicFont=cmunbi.ttf]{cmuntt.ttf}\setmonofont[Path=/usr/share/fonts/truetype/cmu/,UprightFont=cmuntt.ttf,BoldFont=cmuntb.ttf,ItalicFont=cmunit.ttf,BoldItalicFont=cmuntx.ttf]{cmuntt.ttf}\ttfamily \textbackslash{}ref\{{\itshape \setmainfont[Path=/usr/share/fonts/truetype/cmu/,UprightFont=cmunrm.ttf,BoldFont=cmunbx.ttf,ItalicFont=cmunti.ttf,BoldItalicFont=cmunbi.ttf]{cmunit.ttf}\setmonofont[Path=/usr/share/fonts/truetype/cmu/,UprightFont=cmuntt.ttf,BoldFont=cmuntb.ttf,ItalicFont=cmunit.ttf,BoldItalicFont=cmuntx.ttf]{cmunit.ttf}\ttfamily \itshape unknown-{}marker}\setmainfont[Path=/usr/share/fonts/truetype/cmu/,UprightFont=cmunrm.ttf,BoldFont=cmunbx.ttf,ItalicFont=cmunti.ttf,BoldItalicFont=cmunbi.ttf]{cmuntt.ttf}\setmonofont[Path=/usr/share/fonts/truetype/cmu/,UprightFont=cmuntt.ttf,BoldFont=cmuntb.ttf,ItalicFont=cmunit.ttf,BoldItalicFont=cmuntx.ttf]{cmuntt.ttf}\ttfamily \}}\setmainfont[Path=/usr/share/fonts/truetype/cmu/,UprightFont=cmunrm.ttf,BoldFont=cmunbx.ttf,ItalicFont=cmunti.ttf,BoldItalicFont=cmunbi.ttf]{cmunrm.ttf}\setmonofont[Path=/usr/share/fonts/truetype/cmu/,UprightFont=cmuntt.ttf,BoldFont=cmuntb.ttf,ItalicFont=cmunit.ttf,BoldItalicFont=cmuntx.ttf]{cmunrm.ttf}\symbol{34} with \symbol{34}??\symbol{34} (so it will be easy to find in the document).

As you may have noticed reading how it works, it is a two-{}step process: first the compiler has to store the labels with the right number to be used for referencing, then it has to replace the {\ttfamily \setmainfont[Path=/usr/share/fonts/truetype/cmu/,UprightFont=cmunrm.ttf,BoldFont=cmunbx.ttf,ItalicFont=cmunti.ttf,BoldItalicFont=cmunbi.ttf]{cmuntt.ttf}\setmonofont[Path=/usr/share/fonts/truetype/cmu/,UprightFont=cmuntt.ttf,BoldFont=cmuntb.ttf,ItalicFont=cmunit.ttf,BoldItalicFont=cmuntx.ttf]{cmuntt.ttf}\ttfamily \textbackslash{}ref}{$\text{ }$}\setmainfont[Path=/usr/share/fonts/truetype/cmu/,UprightFont=cmunrm.ttf,BoldFont=cmunbx.ttf,ItalicFont=cmunti.ttf,BoldItalicFont=cmunbi.ttf]{cmunrm.ttf}\setmonofont[Path=/usr/share/fonts/truetype/cmu/,UprightFont=cmuntt.ttf,BoldFont=cmuntb.ttf,ItalicFont=cmunit.ttf,BoldItalicFont=cmuntx.ttf]{cmunrm.ttf} with the right number. That is why, when you use references, you have to compile your document twice to see the proper output. If you compile it only once, LaTeX will use the older information it collected in previous compilations (that might be outdated), but the compiler will inform you printing on the screen at the end of the compilation:
\begin{myquote}
\item{} {\ttfamily \setmainfont[Path=/usr/share/fonts/truetype/cmu/,UprightFont=cmunrm.ttf,BoldFont=cmunbx.ttf,ItalicFont=cmunti.ttf,BoldItalicFont=cmunbi.ttf]{cmuntt.ttf}\setmonofont[Path=/usr/share/fonts/truetype/cmu/,UprightFont=cmuntt.ttf,BoldFont=cmuntb.ttf,ItalicFont=cmunit.ttf,BoldItalicFont=cmuntx.ttf]{cmuntt.ttf}\ttfamily LaTeX Warning: Label(s) may have changed. Rerun to get cross-{}references right.}
\end{myquote}
\setmainfont[Path=/usr/share/fonts/truetype/cmu/,UprightFont=cmunrm.ttf,BoldFont=cmunbx.ttf,ItalicFont=cmunti.ttf,BoldItalicFont=cmunbi.ttf]{cmunrm.ttf}\setmonofont[Path=/usr/share/fonts/truetype/cmu/,UprightFont=cmuntt.ttf,BoldFont=cmuntb.ttf,ItalicFont=cmunit.ttf,BoldItalicFont=cmuntx.ttf]{cmunrm.ttf}
Using the command {\ttfamily \setmainfont[Path=/usr/share/fonts/truetype/cmu/,UprightFont=cmunrm.ttf,BoldFont=cmunbx.ttf,ItalicFont=cmunti.ttf,BoldItalicFont=cmunbi.ttf]{cmuntt.ttf}\setmonofont[Path=/usr/share/fonts/truetype/cmu/,UprightFont=cmuntt.ttf,BoldFont=cmuntb.ttf,ItalicFont=cmunit.ttf,BoldItalicFont=cmuntx.ttf]{cmuntt.ttf}\ttfamily \textbackslash{}pageref\{\}}{$\text{ }$}\setmainfont[Path=/usr/share/fonts/truetype/cmu/,UprightFont=cmunrm.ttf,BoldFont=cmunbx.ttf,ItalicFont=cmunti.ttf,BoldItalicFont=cmunbi.ttf]{cmunrm.ttf}\setmonofont[Path=/usr/share/fonts/truetype/cmu/,UprightFont=cmuntt.ttf,BoldFont=cmuntb.ttf,ItalicFont=cmunit.ttf,BoldItalicFont=cmuntx.ttf]{cmunrm.ttf} you can help the reader to find the referenced object by providing also the page number where it can be found. You could write something like:

\begin{Shaded}
\begin{Highlighting}[]

\NormalTok{See\ensuremath{\text{ }}figure~\textbackslash{}ref\{fig:test\}\ensuremath{\text{ }}on\ensuremath{\text{ }}page~\textbackslash{}pageref\{fig:test\}.}\newline
\end{Highlighting}
\end{Shaded}


Since you can use exactly the same commands to reference almost anything, you might get a bit confused after you have introduced a lot of references. It is common practice among LaTeX users to add a few letters to the label to describe {\itshape \setmainfont[Path=/usr/share/fonts/truetype/cmu/,UprightFont=cmunrm.ttf,BoldFont=cmunbx.ttf,ItalicFont=cmunti.ttf,BoldItalicFont=cmunbi.ttf]{cmunti.ttf}\setmonofont[Path=/usr/share/fonts/truetype/cmu/,UprightFont=cmuntt.ttf,BoldFont=cmuntb.ttf,ItalicFont=cmunit.ttf,BoldItalicFont=cmuntx.ttf]{cmunti.ttf}\itshape what}{$\text{ }$}\setmainfont[Path=/usr/share/fonts/truetype/cmu/,UprightFont=cmunrm.ttf,BoldFont=cmunbx.ttf,ItalicFont=cmunti.ttf,BoldItalicFont=cmunbi.ttf]{cmunrm.ttf}\setmonofont[Path=/usr/share/fonts/truetype/cmu/,UprightFont=cmuntt.ttf,BoldFont=cmuntb.ttf,ItalicFont=cmunit.ttf,BoldItalicFont=cmuntx.ttf]{cmunrm.ttf} you are referencing. Some packages, such as {\ttfamily \setmainfont[Path=/usr/share/fonts/truetype/cmu/,UprightFont=cmunrm.ttf,BoldFont=cmunbx.ttf,ItalicFont=cmunti.ttf,BoldItalicFont=cmunbi.ttf]{cmuntt.ttf}\setmonofont[Path=/usr/share/fonts/truetype/cmu/,UprightFont=cmuntt.ttf,BoldFont=cmuntb.ttf,ItalicFont=cmunit.ttf,BoldItalicFont=cmuntx.ttf]{cmuntt.ttf}\ttfamily fancyref}\setmainfont[Path=/usr/share/fonts/truetype/cmu/,UprightFont=cmunrm.ttf,BoldFont=cmunbx.ttf,ItalicFont=cmunti.ttf,BoldItalicFont=cmunbi.ttf]{cmunrm.ttf}\setmonofont[Path=/usr/share/fonts/truetype/cmu/,UprightFont=cmuntt.ttf,BoldFont=cmuntb.ttf,ItalicFont=cmunit.ttf,BoldItalicFont=cmuntx.ttf]{cmunrm.ttf}, rely on this meta information. Here is an example:

\begin{longtable}{|>{\RaggedRight}p{0.34695\linewidth}|>{\RaggedRight}p{0.59591\linewidth}|} \hline 
\hspace*{0pt}\ignorespaces{}\hspace*{0pt}{\bfseries {\ttfamily \setmainfont[Path=/usr/share/fonts/truetype/cmu/,UprightFont=cmunrm.ttf,BoldFont=cmunbx.ttf,ItalicFont=cmunti.ttf,BoldItalicFont=cmunbi.ttf]{cmuntb.ttf}\setmonofont[Path=/usr/share/fonts/truetype/cmu/,UprightFont=cmuntt.ttf,BoldFont=cmuntb.ttf,ItalicFont=cmunit.ttf,BoldItalicFont=cmuntx.ttf]{cmuntb.ttf}\ttfamily \bfseries ch:}}&\hspace*{0pt}\ignorespaces{}\hspace*{0pt}\setmainfont[Path=/usr/share/fonts/truetype/cmu/,UprightFont=cmunrm.ttf,BoldFont=cmunbx.ttf,ItalicFont=cmunti.ttf,BoldItalicFont=cmunbi.ttf]{cmunrm.ttf}\setmonofont[Path=/usr/share/fonts/truetype/cmu/,UprightFont=cmuntt.ttf,BoldFont=cmuntb.ttf,ItalicFont=cmunit.ttf,BoldItalicFont=cmuntx.ttf]{cmunrm.ttf}chapter\\ \hline \hspace*{0pt}\ignorespaces{}\hspace*{0pt}{\bfseries {\ttfamily \setmainfont[Path=/usr/share/fonts/truetype/cmu/,UprightFont=cmunrm.ttf,BoldFont=cmunbx.ttf,ItalicFont=cmunti.ttf,BoldItalicFont=cmunbi.ttf]{cmuntb.ttf}\setmonofont[Path=/usr/share/fonts/truetype/cmu/,UprightFont=cmuntt.ttf,BoldFont=cmuntb.ttf,ItalicFont=cmunit.ttf,BoldItalicFont=cmuntx.ttf]{cmuntb.ttf}\ttfamily \bfseries sec:}}&\hspace*{0pt}\ignorespaces{}\hspace*{0pt}\setmainfont[Path=/usr/share/fonts/truetype/cmu/,UprightFont=cmunrm.ttf,BoldFont=cmunbx.ttf,ItalicFont=cmunti.ttf,BoldItalicFont=cmunbi.ttf]{cmunrm.ttf}\setmonofont[Path=/usr/share/fonts/truetype/cmu/,UprightFont=cmuntt.ttf,BoldFont=cmuntb.ttf,ItalicFont=cmunit.ttf,BoldItalicFont=cmuntx.ttf]{cmunrm.ttf}section\\ \hline \hspace*{0pt}\ignorespaces{}\hspace*{0pt}{\bfseries {\ttfamily \setmainfont[Path=/usr/share/fonts/truetype/cmu/,UprightFont=cmunrm.ttf,BoldFont=cmunbx.ttf,ItalicFont=cmunti.ttf,BoldItalicFont=cmunbi.ttf]{cmuntb.ttf}\setmonofont[Path=/usr/share/fonts/truetype/cmu/,UprightFont=cmuntt.ttf,BoldFont=cmuntb.ttf,ItalicFont=cmunit.ttf,BoldItalicFont=cmuntx.ttf]{cmuntb.ttf}\ttfamily \bfseries subsec:}}&\hspace*{0pt}\ignorespaces{}\hspace*{0pt}\setmainfont[Path=/usr/share/fonts/truetype/cmu/,UprightFont=cmunrm.ttf,BoldFont=cmunbx.ttf,ItalicFont=cmunti.ttf,BoldItalicFont=cmunbi.ttf]{cmunrm.ttf}\setmonofont[Path=/usr/share/fonts/truetype/cmu/,UprightFont=cmuntt.ttf,BoldFont=cmuntb.ttf,ItalicFont=cmunit.ttf,BoldItalicFont=cmuntx.ttf]{cmunrm.ttf}subsection\\ \hline \hspace*{0pt}\ignorespaces{}\hspace*{0pt}{\bfseries {\ttfamily \setmainfont[Path=/usr/share/fonts/truetype/cmu/,UprightFont=cmunrm.ttf,BoldFont=cmunbx.ttf,ItalicFont=cmunti.ttf,BoldItalicFont=cmunbi.ttf]{cmuntb.ttf}\setmonofont[Path=/usr/share/fonts/truetype/cmu/,UprightFont=cmuntt.ttf,BoldFont=cmuntb.ttf,ItalicFont=cmunit.ttf,BoldItalicFont=cmuntx.ttf]{cmuntb.ttf}\ttfamily \bfseries fig:}}&\hspace*{0pt}\ignorespaces{}\hspace*{0pt}\setmainfont[Path=/usr/share/fonts/truetype/cmu/,UprightFont=cmunrm.ttf,BoldFont=cmunbx.ttf,ItalicFont=cmunti.ttf,BoldItalicFont=cmunbi.ttf]{cmunrm.ttf}\setmonofont[Path=/usr/share/fonts/truetype/cmu/,UprightFont=cmuntt.ttf,BoldFont=cmuntb.ttf,ItalicFont=cmunit.ttf,BoldItalicFont=cmuntx.ttf]{cmunrm.ttf}figure\\ \hline \hspace*{0pt}\ignorespaces{}\hspace*{0pt}{\bfseries {\ttfamily \setmainfont[Path=/usr/share/fonts/truetype/cmu/,UprightFont=cmunrm.ttf,BoldFont=cmunbx.ttf,ItalicFont=cmunti.ttf,BoldItalicFont=cmunbi.ttf]{cmuntb.ttf}\setmonofont[Path=/usr/share/fonts/truetype/cmu/,UprightFont=cmuntt.ttf,BoldFont=cmuntb.ttf,ItalicFont=cmunit.ttf,BoldItalicFont=cmuntx.ttf]{cmuntb.ttf}\ttfamily \bfseries tab:}}&\hspace*{0pt}\ignorespaces{}\hspace*{0pt}\setmainfont[Path=/usr/share/fonts/truetype/cmu/,UprightFont=cmunrm.ttf,BoldFont=cmunbx.ttf,ItalicFont=cmunti.ttf,BoldItalicFont=cmunbi.ttf]{cmunrm.ttf}\setmonofont[Path=/usr/share/fonts/truetype/cmu/,UprightFont=cmuntt.ttf,BoldFont=cmuntb.ttf,ItalicFont=cmunit.ttf,BoldItalicFont=cmuntx.ttf]{cmunrm.ttf}table\\ \hline \hspace*{0pt}\ignorespaces{}\hspace*{0pt}{\bfseries {\ttfamily \setmainfont[Path=/usr/share/fonts/truetype/cmu/,UprightFont=cmunrm.ttf,BoldFont=cmunbx.ttf,ItalicFont=cmunti.ttf,BoldItalicFont=cmunbi.ttf]{cmuntb.ttf}\setmonofont[Path=/usr/share/fonts/truetype/cmu/,UprightFont=cmuntt.ttf,BoldFont=cmuntb.ttf,ItalicFont=cmunit.ttf,BoldItalicFont=cmuntx.ttf]{cmuntb.ttf}\ttfamily \bfseries eq:}}&\hspace*{0pt}\ignorespaces{}\hspace*{0pt}\setmainfont[Path=/usr/share/fonts/truetype/cmu/,UprightFont=cmunrm.ttf,BoldFont=cmunbx.ttf,ItalicFont=cmunti.ttf,BoldItalicFont=cmunbi.ttf]{cmunrm.ttf}\setmonofont[Path=/usr/share/fonts/truetype/cmu/,UprightFont=cmuntt.ttf,BoldFont=cmuntb.ttf,ItalicFont=cmunit.ttf,BoldItalicFont=cmuntx.ttf]{cmunrm.ttf}equation\\ \hline \hspace*{0pt}\ignorespaces{}\hspace*{0pt}{\bfseries {\ttfamily \setmainfont[Path=/usr/share/fonts/truetype/cmu/,UprightFont=cmunrm.ttf,BoldFont=cmunbx.ttf,ItalicFont=cmunti.ttf,BoldItalicFont=cmunbi.ttf]{cmuntb.ttf}\setmonofont[Path=/usr/share/fonts/truetype/cmu/,UprightFont=cmuntt.ttf,BoldFont=cmuntb.ttf,ItalicFont=cmunit.ttf,BoldItalicFont=cmuntx.ttf]{cmuntb.ttf}\ttfamily \bfseries lst:}}&\hspace*{0pt}\ignorespaces{}\hspace*{0pt}\setmainfont[Path=/usr/share/fonts/truetype/cmu/,UprightFont=cmunrm.ttf,BoldFont=cmunbx.ttf,ItalicFont=cmunti.ttf,BoldItalicFont=cmunbi.ttf]{cmunrm.ttf}\setmonofont[Path=/usr/share/fonts/truetype/cmu/,UprightFont=cmuntt.ttf,BoldFont=cmuntb.ttf,ItalicFont=cmunit.ttf,BoldItalicFont=cmuntx.ttf]{cmunrm.ttf}code listing\\ \hline \hspace*{0pt}\ignorespaces{}\hspace*{0pt}{\bfseries {\ttfamily \setmainfont[Path=/usr/share/fonts/truetype/cmu/,UprightFont=cmunrm.ttf,BoldFont=cmunbx.ttf,ItalicFont=cmunti.ttf,BoldItalicFont=cmunbi.ttf]{cmuntb.ttf}\setmonofont[Path=/usr/share/fonts/truetype/cmu/,UprightFont=cmuntt.ttf,BoldFont=cmuntb.ttf,ItalicFont=cmunit.ttf,BoldItalicFont=cmuntx.ttf]{cmuntb.ttf}\ttfamily \bfseries itm:}}&\hspace*{0pt}\ignorespaces{}\hspace*{0pt}\setmainfont[Path=/usr/share/fonts/truetype/cmu/,UprightFont=cmunrm.ttf,BoldFont=cmunbx.ttf,ItalicFont=cmunti.ttf,BoldItalicFont=cmunbi.ttf]{cmunrm.ttf}\setmonofont[Path=/usr/share/fonts/truetype/cmu/,UprightFont=cmuntt.ttf,BoldFont=cmuntb.ttf,ItalicFont=cmunit.ttf,BoldItalicFont=cmuntx.ttf]{cmunrm.ttf}enumerated list item\\ \hline \hspace*{0pt}\ignorespaces{}\hspace*{0pt}{\bfseries {\ttfamily \setmainfont[Path=/usr/share/fonts/truetype/cmu/,UprightFont=cmunrm.ttf,BoldFont=cmunbx.ttf,ItalicFont=cmunti.ttf,BoldItalicFont=cmunbi.ttf]{cmuntb.ttf}\setmonofont[Path=/usr/share/fonts/truetype/cmu/,UprightFont=cmuntt.ttf,BoldFont=cmuntb.ttf,ItalicFont=cmunit.ttf,BoldItalicFont=cmuntx.ttf]{cmuntb.ttf}\ttfamily \bfseries alg:}}&\hspace*{0pt}\ignorespaces{}\hspace*{0pt}\setmainfont[Path=/usr/share/fonts/truetype/cmu/,UprightFont=cmunrm.ttf,BoldFont=cmunbx.ttf,ItalicFont=cmunti.ttf,BoldItalicFont=cmunbi.ttf]{cmunrm.ttf}\setmonofont[Path=/usr/share/fonts/truetype/cmu/,UprightFont=cmuntt.ttf,BoldFont=cmuntb.ttf,ItalicFont=cmunit.ttf,BoldItalicFont=cmuntx.ttf]{cmunrm.ttf}algorithm\\ \hline \hspace*{0pt}\ignorespaces{}\hspace*{0pt}{\bfseries {\ttfamily \setmainfont[Path=/usr/share/fonts/truetype/cmu/,UprightFont=cmunrm.ttf,BoldFont=cmunbx.ttf,ItalicFont=cmunti.ttf,BoldItalicFont=cmunbi.ttf]{cmuntb.ttf}\setmonofont[Path=/usr/share/fonts/truetype/cmu/,UprightFont=cmuntt.ttf,BoldFont=cmuntb.ttf,ItalicFont=cmunit.ttf,BoldItalicFont=cmuntx.ttf]{cmuntb.ttf}\ttfamily \bfseries app:}}&\hspace*{0pt}\ignorespaces{}\hspace*{0pt}\setmainfont[Path=/usr/share/fonts/truetype/cmu/,UprightFont=cmunrm.ttf,BoldFont=cmunbx.ttf,ItalicFont=cmunti.ttf,BoldItalicFont=cmunbi.ttf]{cmunrm.ttf}\setmonofont[Path=/usr/share/fonts/truetype/cmu/,UprightFont=cmuntt.ttf,BoldFont=cmuntb.ttf,ItalicFont=cmunit.ttf,BoldItalicFont=cmuntx.ttf]{cmunrm.ttf}appendix subsection\\ \hline 
\end{longtable}


Following this convention, the label of a figure will look like {\ttfamily \setmainfont[Path=/usr/share/fonts/truetype/cmu/,UprightFont=cmunrm.ttf,BoldFont=cmunbx.ttf,ItalicFont=cmunti.ttf,BoldItalicFont=cmunbi.ttf]{cmuntt.ttf}\setmonofont[Path=/usr/share/fonts/truetype/cmu/,UprightFont=cmuntt.ttf,BoldFont=cmuntb.ttf,ItalicFont=cmunit.ttf,BoldItalicFont=cmuntx.ttf]{cmuntt.ttf}\ttfamily \textbackslash{}label\{fig:{\itshape \setmainfont[Path=/usr/share/fonts/truetype/cmu/,UprightFont=cmunrm.ttf,BoldFont=cmunbx.ttf,ItalicFont=cmunti.ttf,BoldItalicFont=cmunbi.ttf]{cmunit.ttf}\setmonofont[Path=/usr/share/fonts/truetype/cmu/,UprightFont=cmuntt.ttf,BoldFont=cmuntb.ttf,ItalicFont=cmunit.ttf,BoldItalicFont=cmuntx.ttf]{cmunit.ttf}\ttfamily \itshape my\_figure}\setmainfont[Path=/usr/share/fonts/truetype/cmu/,UprightFont=cmunrm.ttf,BoldFont=cmunbx.ttf,ItalicFont=cmunti.ttf,BoldItalicFont=cmunbi.ttf]{cmuntt.ttf}\setmonofont[Path=/usr/share/fonts/truetype/cmu/,UprightFont=cmuntt.ttf,BoldFont=cmuntb.ttf,ItalicFont=cmunit.ttf,BoldItalicFont=cmuntx.ttf]{cmuntt.ttf}\ttfamily \}}\setmainfont[Path=/usr/share/fonts/truetype/cmu/,UprightFont=cmunrm.ttf,BoldFont=cmunbx.ttf,ItalicFont=cmunti.ttf,BoldItalicFont=cmunbi.ttf]{cmunrm.ttf}\setmonofont[Path=/usr/share/fonts/truetype/cmu/,UprightFont=cmuntt.ttf,BoldFont=cmuntb.ttf,ItalicFont=cmunit.ttf,BoldItalicFont=cmuntx.ttf]{cmunrm.ttf}, etc. You are not obligated to use these prefixes. You can use any string as argument of {\ttfamily \setmainfont[Path=/usr/share/fonts/truetype/cmu/,UprightFont=cmunrm.ttf,BoldFont=cmunbx.ttf,ItalicFont=cmunti.ttf,BoldItalicFont=cmunbi.ttf]{cmuntt.ttf}\setmonofont[Path=/usr/share/fonts/truetype/cmu/,UprightFont=cmuntt.ttf,BoldFont=cmuntb.ttf,ItalicFont=cmunit.ttf,BoldItalicFont=cmuntx.ttf]{cmuntt.ttf}\ttfamily \textbackslash{}label\{...\}}\setmainfont[Path=/usr/share/fonts/truetype/cmu/,UprightFont=cmunrm.ttf,BoldFont=cmunbx.ttf,ItalicFont=cmunti.ttf,BoldItalicFont=cmunbi.ttf]{cmunrm.ttf}\setmonofont[Path=/usr/share/fonts/truetype/cmu/,UprightFont=cmuntt.ttf,BoldFont=cmuntb.ttf,ItalicFont=cmunit.ttf,BoldItalicFont=cmuntx.ttf]{cmunrm.ttf}, but these prefixes become increasingly useful as your document grows in size.

Another suggestion: try to avoid using numbers within labels. You are better off describing {\itshape \setmainfont[Path=/usr/share/fonts/truetype/cmu/,UprightFont=cmunrm.ttf,BoldFont=cmunbx.ttf,ItalicFont=cmunti.ttf,BoldItalicFont=cmunbi.ttf]{cmunti.ttf}\setmonofont[Path=/usr/share/fonts/truetype/cmu/,UprightFont=cmuntt.ttf,BoldFont=cmuntb.ttf,ItalicFont=cmunit.ttf,BoldItalicFont=cmuntx.ttf]{cmunti.ttf}\itshape what}{$\text{ }$}\setmainfont[Path=/usr/share/fonts/truetype/cmu/,UprightFont=cmunrm.ttf,BoldFont=cmunbx.ttf,ItalicFont=cmunti.ttf,BoldItalicFont=cmunbi.ttf]{cmunrm.ttf}\setmonofont[Path=/usr/share/fonts/truetype/cmu/,UprightFont=cmuntt.ttf,BoldFont=cmuntb.ttf,ItalicFont=cmunit.ttf,BoldItalicFont=cmuntx.ttf]{cmunrm.ttf} the object is about. This way, if you change the order of the objects, you will not have to rename all your labels and their references.

If you want to be able to see the markers you are using in the output document as well, you can use the {\ttfamily \setmainfont[Path=/usr/share/fonts/truetype/cmu/,UprightFont=cmunrm.ttf,BoldFont=cmunbx.ttf,ItalicFont=cmunti.ttf,BoldItalicFont=cmunbi.ttf]{cmuntt.ttf}\setmonofont[Path=/usr/share/fonts/truetype/cmu/,UprightFont=cmuntt.ttf,BoldFont=cmuntb.ttf,ItalicFont=cmunit.ttf,BoldItalicFont=cmuntx.ttf]{cmuntt.ttf}\ttfamily showkeys}{$\text{ }$}\setmainfont[Path=/usr/share/fonts/truetype/cmu/,UprightFont=cmunrm.ttf,BoldFont=cmunbx.ttf,ItalicFont=cmunti.ttf,BoldItalicFont=cmunbi.ttf]{cmunrm.ttf}\setmonofont[Path=/usr/share/fonts/truetype/cmu/,UprightFont=cmuntt.ttf,BoldFont=cmuntb.ttf,ItalicFont=cmunit.ttf,BoldItalicFont=cmuntx.ttf]{cmunrm.ttf} package; this can be very useful while developing your document. For more information see the \myhref{https://en.wikibooks.org/wiki/LaTeX\%2FPackages}{Packages} section.
\section{Examples}
\label{419}

Here are some practical examples, but you will notice that they are all the same because they all use the same commands.
\subsection{Sections}
\label{420}


\begin{Shaded}
\begin{Highlighting}[]

\NormalTok{\textbackslash{}section\{Greetings\}}\newline
\NormalTok{\textbackslash{}label\{sec:greetings\}}\newline
\ensuremath{\text{ }}\newline
\NormalTok{Hello!}\newline
\ensuremath{\text{ }}\newline
\NormalTok{\textbackslash{}section\{Referencing\}}\newline
\ensuremath{\text{ }}\newline
\NormalTok{I\ensuremath{\text{ }}greeted\ensuremath{\text{ }}in\ensuremath{\text{ }}section~\textbackslash{}ref\{sec:greetings\}.}\newline
\end{Highlighting}
\end{Shaded}




\begin{minipage}{0.37500\textwidth}
\begin{center}
\includegraphics[width=1.0\textwidth,height=6.5in,keepaspectratio]{../images/74.png}
\end{center}
\raggedright{}\myfigurewithoutcaption{74}
\end{minipage}\vspace{0.75cm}



You could place the label anywhere in the section; however, in order to avoid confusion, it is better to place it immediately after the beginning of the section. Note how the marker starts with {\itshape \setmainfont[Path=/usr/share/fonts/truetype/cmu/,UprightFont=cmunrm.ttf,BoldFont=cmunbx.ttf,ItalicFont=cmunti.ttf,BoldItalicFont=cmunbi.ttf]{cmunti.ttf}\setmonofont[Path=/usr/share/fonts/truetype/cmu/,UprightFont=cmuntt.ttf,BoldFont=cmuntb.ttf,ItalicFont=cmunit.ttf,BoldItalicFont=cmuntx.ttf]{cmunti.ttf}\itshape sec:}\setmainfont[Path=/usr/share/fonts/truetype/cmu/,UprightFont=cmunrm.ttf,BoldFont=cmunbx.ttf,ItalicFont=cmunti.ttf,BoldItalicFont=cmunbi.ttf]{cmunrm.ttf}\setmonofont[Path=/usr/share/fonts/truetype/cmu/,UprightFont=cmuntt.ttf,BoldFont=cmuntb.ttf,ItalicFont=cmunit.ttf,BoldItalicFont=cmuntx.ttf]{cmunrm.ttf}, as suggested before. The label is then referenced in a different section. The tilde (\~{}) indicates a \myhref{https://en.wikipedia.org/wiki/non-breaking\%20space}{non-{}breaking space}.
\subsection{Pictures}
\label{421}

You can reference a picture by inserting it in the {\ttfamily \setmainfont[Path=/usr/share/fonts/truetype/cmu/,UprightFont=cmunrm.ttf,BoldFont=cmunbx.ttf,ItalicFont=cmunti.ttf,BoldItalicFont=cmunbi.ttf]{cmuntt.ttf}\setmonofont[Path=/usr/share/fonts/truetype/cmu/,UprightFont=cmuntt.ttf,BoldFont=cmuntb.ttf,ItalicFont=cmunit.ttf,BoldItalicFont=cmuntx.ttf]{cmuntt.ttf}\ttfamily figure}{$\text{ }$}\setmainfont[Path=/usr/share/fonts/truetype/cmu/,UprightFont=cmunrm.ttf,BoldFont=cmunbx.ttf,ItalicFont=cmunti.ttf,BoldItalicFont=cmunbi.ttf]{cmunrm.ttf}\setmonofont[Path=/usr/share/fonts/truetype/cmu/,UprightFont=cmuntt.ttf,BoldFont=cmuntb.ttf,ItalicFont=cmunit.ttf,BoldItalicFont=cmuntx.ttf]{cmunrm.ttf} floating environment.


\begin{Shaded}
\begin{Highlighting}[]

\NormalTok{\textbackslash{}begin\{figure\}}\newline
\ensuremath{\text{ }}\ensuremath{\text{ }}\NormalTok{\textbackslash{}centering}\newline
\ensuremath{\text{ }}\ensuremath{\text{ }}\ensuremath{\text{ }}\ensuremath{\text{ }}\NormalTok{\textbackslash{}includegraphics[width=0.5\textbackslash{}textwidth]\{gull\}}\newline
\ensuremath{\text{ }}\ensuremath{\text{ }}\NormalTok{\textbackslash{}caption\{Close-up\ensuremath{\text{ }}of\ensuremath{\text{ }}a\ensuremath{\text{ }}gull\}}\newline
\ensuremath{\text{ }}\ensuremath{\text{ }}\NormalTok{\textbackslash{}label\{fig:gull\}}\newline
\NormalTok{\textbackslash{}end\{figure\}}\newline
\NormalTok{Figure\ensuremath{\text{ }}\textbackslash{}ref\{fig:gull\}\ensuremath{\text{ }}shows\ensuremath{\text{ }}a\ensuremath{\text{ }}photograph\ensuremath{\text{ }}of\ensuremath{\text{ }}a\ensuremath{\text{ }}gull.}\newline
\end{Highlighting}
\end{Shaded}




\begin{minipage}{0.75000\textwidth}
\begin{center}
\includegraphics[width=1.0\textwidth,height=6.5in,keepaspectratio]{../images/75.png}
\end{center}
\raggedright{}\myfigurewithoutcaption{75}
\end{minipage}\vspace{0.75cm}



When a label is declared within a float environment, the {\ttfamily \setmainfont[Path=/usr/share/fonts/truetype/cmu/,UprightFont=cmunrm.ttf,BoldFont=cmunbx.ttf,ItalicFont=cmunti.ttf,BoldItalicFont=cmunbi.ttf]{cmuntt.ttf}\setmonofont[Path=/usr/share/fonts/truetype/cmu/,UprightFont=cmuntt.ttf,BoldFont=cmuntb.ttf,ItalicFont=cmunit.ttf,BoldItalicFont=cmuntx.ttf]{cmuntt.ttf}\ttfamily \textbackslash{}ref\{...\}}{$\text{ }$}\setmainfont[Path=/usr/share/fonts/truetype/cmu/,UprightFont=cmunrm.ttf,BoldFont=cmunbx.ttf,ItalicFont=cmunti.ttf,BoldItalicFont=cmunbi.ttf]{cmunrm.ttf}\setmonofont[Path=/usr/share/fonts/truetype/cmu/,UprightFont=cmuntt.ttf,BoldFont=cmuntb.ttf,ItalicFont=cmunit.ttf,BoldItalicFont=cmuntx.ttf]{cmunrm.ttf} will return the respective fig/table number, but it must occur {\bfseries \setmainfont[Path=/usr/share/fonts/truetype/cmu/,UprightFont=cmunrm.ttf,BoldFont=cmunbx.ttf,ItalicFont=cmunti.ttf,BoldItalicFont=cmunbi.ttf]{cmunbx.ttf}\setmonofont[Path=/usr/share/fonts/truetype/cmu/,UprightFont=cmuntt.ttf,BoldFont=cmuntb.ttf,ItalicFont=cmunit.ttf,BoldItalicFont=cmuntx.ttf]{cmunbx.ttf}\bfseries after}{$\text{ }$}\setmainfont[Path=/usr/share/fonts/truetype/cmu/,UprightFont=cmunrm.ttf,BoldFont=cmunbx.ttf,ItalicFont=cmunti.ttf,BoldItalicFont=cmunbi.ttf]{cmunrm.ttf}\setmonofont[Path=/usr/share/fonts/truetype/cmu/,UprightFont=cmuntt.ttf,BoldFont=cmuntb.ttf,ItalicFont=cmunit.ttf,BoldItalicFont=cmuntx.ttf]{cmunrm.ttf} the caption. When declared outside, it will give the section number. To be completely safe, the label for any picture or table can go within the {\ttfamily \setmainfont[Path=/usr/share/fonts/truetype/cmu/,UprightFont=cmunrm.ttf,BoldFont=cmunbx.ttf,ItalicFont=cmunti.ttf,BoldItalicFont=cmunbi.ttf]{cmuntt.ttf}\setmonofont[Path=/usr/share/fonts/truetype/cmu/,UprightFont=cmuntt.ttf,BoldFont=cmuntb.ttf,ItalicFont=cmunit.ttf,BoldItalicFont=cmuntx.ttf]{cmuntt.ttf}\ttfamily \textbackslash{}caption\{\}}{$\text{ }$}\setmainfont[Path=/usr/share/fonts/truetype/cmu/,UprightFont=cmunrm.ttf,BoldFont=cmunbx.ttf,ItalicFont=cmunti.ttf,BoldItalicFont=cmunbi.ttf]{cmunrm.ttf}\setmonofont[Path=/usr/share/fonts/truetype/cmu/,UprightFont=cmuntt.ttf,BoldFont=cmuntb.ttf,ItalicFont=cmunit.ttf,BoldItalicFont=cmuntx.ttf]{cmunrm.ttf} command, as in

\begin{Shaded}
\begin{Highlighting}[]

\NormalTok{\textbackslash{}caption\{Close-up\ensuremath{\text{ }}of\ensuremath{\text{ }}a\ensuremath{\text{ }}gull\textbackslash{}label\{fig:gull\}\}}\newline
\end{Highlighting}
\end{Shaded}


See the \mylref{362}{Floats, Figures and Captions} section for more about the {\ttfamily \setmainfont[Path=/usr/share/fonts/truetype/cmu/,UprightFont=cmunrm.ttf,BoldFont=cmunbx.ttf,ItalicFont=cmunti.ttf,BoldItalicFont=cmunbi.ttf]{cmuntt.ttf}\setmonofont[Path=/usr/share/fonts/truetype/cmu/,UprightFont=cmuntt.ttf,BoldFont=cmuntb.ttf,ItalicFont=cmunit.ttf,BoldItalicFont=cmuntx.ttf]{cmuntt.ttf}\ttfamily figure}{$\text{ }$}\setmainfont[Path=/usr/share/fonts/truetype/cmu/,UprightFont=cmunrm.ttf,BoldFont=cmunbx.ttf,ItalicFont=cmunti.ttf,BoldItalicFont=cmunbi.ttf]{cmunrm.ttf}\setmonofont[Path=/usr/share/fonts/truetype/cmu/,UprightFont=cmuntt.ttf,BoldFont=cmuntb.ttf,ItalicFont=cmunit.ttf,BoldItalicFont=cmuntx.ttf]{cmunrm.ttf} and related environments.
\subsubsection{Fixing wrong labels}
\label{422}

The command {\ttfamily \setmainfont[Path=/usr/share/fonts/truetype/cmu/,UprightFont=cmunrm.ttf,BoldFont=cmunbx.ttf,ItalicFont=cmunti.ttf,BoldItalicFont=cmunbi.ttf]{cmuntt.ttf}\setmonofont[Path=/usr/share/fonts/truetype/cmu/,UprightFont=cmuntt.ttf,BoldFont=cmuntb.ttf,ItalicFont=cmunit.ttf,BoldItalicFont=cmuntx.ttf]{cmuntt.ttf}\ttfamily \textbackslash{}label}{$\text{ }$}\setmainfont[Path=/usr/share/fonts/truetype/cmu/,UprightFont=cmunrm.ttf,BoldFont=cmunbx.ttf,ItalicFont=cmunti.ttf,BoldItalicFont=cmunbi.ttf]{cmunrm.ttf}\setmonofont[Path=/usr/share/fonts/truetype/cmu/,UprightFont=cmuntt.ttf,BoldFont=cmuntb.ttf,ItalicFont=cmunit.ttf,BoldItalicFont=cmuntx.ttf]{cmunrm.ttf} must appear after (or inside) {\ttfamily \setmainfont[Path=/usr/share/fonts/truetype/cmu/,UprightFont=cmunrm.ttf,BoldFont=cmunbx.ttf,ItalicFont=cmunti.ttf,BoldItalicFont=cmunbi.ttf]{cmuntt.ttf}\setmonofont[Path=/usr/share/fonts/truetype/cmu/,UprightFont=cmuntt.ttf,BoldFont=cmuntb.ttf,ItalicFont=cmunit.ttf,BoldItalicFont=cmuntx.ttf]{cmuntt.ttf}\ttfamily \textbackslash{}caption}\setmainfont[Path=/usr/share/fonts/truetype/cmu/,UprightFont=cmunrm.ttf,BoldFont=cmunbx.ttf,ItalicFont=cmunti.ttf,BoldItalicFont=cmunbi.ttf]{cmunrm.ttf}\setmonofont[Path=/usr/share/fonts/truetype/cmu/,UprightFont=cmuntt.ttf,BoldFont=cmuntb.ttf,ItalicFont=cmunit.ttf,BoldItalicFont=cmuntx.ttf]{cmunrm.ttf}. Otherwise, it will pick up the current section or list number instead of what you intended.


\begin{Shaded}
\begin{Highlighting}[]

\NormalTok{\textbackslash{}begin\{figure\}}\newline
\ensuremath{\text{ }}\ensuremath{\text{ }}\NormalTok{\textbackslash{}centering}\newline
\ensuremath{\text{ }}\ensuremath{\text{ }}\NormalTok{\textbackslash{}includegraphics[width=0.5\textbackslash{}textwidth]\{gull\}}\newline
\ensuremath{\text{ }}\ensuremath{\text{ }}\NormalTok{\textbackslash{}caption\{Close-up\ensuremath{\text{ }}of\ensuremath{\text{ }}a\ensuremath{\text{ }}gull\}\ensuremath{\text{ }}\textbackslash{}label\{fig:gull\}\ensuremath{\text{ }}}\newline
\NormalTok{\textbackslash{}end\{figure\}}\newline
\end{Highlighting}
\end{Shaded}

\subsubsection{Issues with links to tables and figures handled by hyperref}
\label{423}

In case you use the package {\ttfamily \setmainfont[Path=/usr/share/fonts/truetype/cmu/,UprightFont=cmunrm.ttf,BoldFont=cmunbx.ttf,ItalicFont=cmunti.ttf,BoldItalicFont=cmunbi.ttf]{cmuntt.ttf}\setmonofont[Path=/usr/share/fonts/truetype/cmu/,UprightFont=cmuntt.ttf,BoldFont=cmuntb.ttf,ItalicFont=cmunit.ttf,BoldItalicFont=cmuntx.ttf]{cmuntt.ttf}\ttfamily hyperref}{$\text{ }$}\setmainfont[Path=/usr/share/fonts/truetype/cmu/,UprightFont=cmunrm.ttf,BoldFont=cmunbx.ttf,ItalicFont=cmunti.ttf,BoldItalicFont=cmunbi.ttf]{cmunrm.ttf}\setmonofont[Path=/usr/share/fonts/truetype/cmu/,UprightFont=cmuntt.ttf,BoldFont=cmuntb.ttf,ItalicFont=cmunit.ttf,BoldItalicFont=cmuntx.ttf]{cmunrm.ttf} to create a PDF, the links to tables or figures will point to the caption of the table or figure, which is always below the table or figure itself\myfootnote{\myplainurl{http://www.ctan.org/tex-archive/macros/latex/contrib/hyperref/README}}. Therefore the table or figure will not be visible, if it is above the pointer and one has to scroll up in order to see it. If you want the link point to the top of the image you can give the option {\ttfamily \setmainfont[Path=/usr/share/fonts/truetype/cmu/,UprightFont=cmunrm.ttf,BoldFont=cmunbx.ttf,ItalicFont=cmunti.ttf,BoldItalicFont=cmunbi.ttf]{cmuntt.ttf}\setmonofont[Path=/usr/share/fonts/truetype/cmu/,UprightFont=cmuntt.ttf,BoldFont=cmuntb.ttf,ItalicFont=cmunit.ttf,BoldItalicFont=cmuntx.ttf]{cmuntt.ttf}\ttfamily hypcap}{$\text{ }$}\setmainfont[Path=/usr/share/fonts/truetype/cmu/,UprightFont=cmunrm.ttf,BoldFont=cmunbx.ttf,ItalicFont=cmunti.ttf,BoldItalicFont=cmunbi.ttf]{cmunrm.ttf}\setmonofont[Path=/usr/share/fonts/truetype/cmu/,UprightFont=cmuntt.ttf,BoldFont=cmuntb.ttf,ItalicFont=cmunit.ttf,BoldItalicFont=cmuntx.ttf]{cmunrm.ttf} to the {\ttfamily \setmainfont[Path=/usr/share/fonts/truetype/cmu/,UprightFont=cmunrm.ttf,BoldFont=cmunbx.ttf,ItalicFont=cmunti.ttf,BoldItalicFont=cmunbi.ttf]{cmuntt.ttf}\setmonofont[Path=/usr/share/fonts/truetype/cmu/,UprightFont=cmuntt.ttf,BoldFont=cmuntb.ttf,ItalicFont=cmunit.ttf,BoldItalicFont=cmuntx.ttf]{cmuntt.ttf}\ttfamily caption}{$\text{ }$}\setmainfont[Path=/usr/share/fonts/truetype/cmu/,UprightFont=cmunrm.ttf,BoldFont=cmunbx.ttf,ItalicFont=cmunti.ttf,BoldItalicFont=cmunbi.ttf]{cmunrm.ttf}\setmonofont[Path=/usr/share/fonts/truetype/cmu/,UprightFont=cmuntt.ttf,BoldFont=cmuntb.ttf,ItalicFont=cmunit.ttf,BoldItalicFont=cmuntx.ttf]{cmunrm.ttf} package:

\begin{Shaded}
\begin{Highlighting}[]

\NormalTok{\textbackslash{}usepackage\{caption\}\ensuremath{\text{ }}}\CommentTok{\%\ensuremath{\text{ }}hypcap\ensuremath{\text{ }}is\ensuremath{\text{ }}true\ensuremath{\text{ }}by\ensuremath{\text{ }}default\ensuremath{\text{ }}so\ensuremath{\text{ }}[hypcap=true]\ensuremath{\text{ }}is\ensuremath{\text{ }}optionnal}\newline
\ensuremath{\text{ }}\NormalTok{in\ensuremath{\text{ }}\textbackslash{}usepackage[hypcap=true]\{caption\}\ensuremath{\text{ }}}\newline
\end{Highlighting}
\end{Shaded}

\subsection{Formulae}
\label{424}

Here is an example showing how to reference formulae:


\begin{Shaded}
\begin{Highlighting}[]

\NormalTok{\textbackslash{}begin\{equation\}\ensuremath{\text{ }}\textbackslash{}label\{eq:solve\}}\newline
\NormalTok{x^2\ensuremath{\text{ }}-\ensuremath{\text{ }}5\ensuremath{\text{ }}x\ensuremath{\text{ }}+\ensuremath{\text{ }}6\ensuremath{\text{ }}=\ensuremath{\text{ }}0}\newline
\NormalTok{\textbackslash{}end\{equation\}}\newline
\ensuremath{\text{ }}\newline
\NormalTok{\textbackslash{}begin\{equation\}}\newline
\NormalTok{x_1\ensuremath{\text{ }}=\ensuremath{\text{ }}\textbackslash{}frac\{5\ensuremath{\text{ }}+\ensuremath{\text{ }}\textbackslash{}sqrt\{25\ensuremath{\text{ }}-\ensuremath{\text{ }}4\ensuremath{\text{ }}\textbackslash{}times\ensuremath{\text{ }}6\}\}\{2\}\ensuremath{\text{ }}=\ensuremath{\text{ }}3}\newline
\NormalTok{\textbackslash{}end\{equation\}}\newline
\ensuremath{\text{ }}\newline
\NormalTok{\textbackslash{}begin\{equation\}}\newline
\NormalTok{x_2\ensuremath{\text{ }}=\ensuremath{\text{ }}\textbackslash{}frac\{5\ensuremath{\text{ }}-\ensuremath{\text{ }}\textbackslash{}sqrt\{25\ensuremath{\text{ }}-\ensuremath{\text{ }}4\ensuremath{\text{ }}\textbackslash{}times\ensuremath{\text{ }}6\}\}\{2\}\ensuremath{\text{ }}=\ensuremath{\text{ }}2}\newline
\NormalTok{\textbackslash{}end\{equation\}}\newline
\ensuremath{\text{ }}\newline
\NormalTok{and\ensuremath{\text{ }}so\ensuremath{\text{ }}we\ensuremath{\text{ }}have\ensuremath{\text{ }}solved\ensuremath{\text{ }}equation~\textbackslash{}ref\{eq:solve\}}\newline
\end{Highlighting}
\end{Shaded}




\begin{minipage}{0.87500\textwidth}
\begin{center}
\includegraphics[width=1.0\textwidth,height=6.5in,keepaspectratio]{../images/76.png}
\end{center}
\raggedright{}\myfigurewithoutcaption{76}
\end{minipage}\vspace{0.75cm}



As you can see, the label is placed soon after the beginning of the math mode. In order to reference a formula, you have to use an environment that adds numbers. Most of the times you will be using the {\ttfamily \setmainfont[Path=/usr/share/fonts/truetype/cmu/,UprightFont=cmunrm.ttf,BoldFont=cmunbx.ttf,ItalicFont=cmunti.ttf,BoldItalicFont=cmunbi.ttf]{cmuntt.ttf}\setmonofont[Path=/usr/share/fonts/truetype/cmu/,UprightFont=cmuntt.ttf,BoldFont=cmuntb.ttf,ItalicFont=cmunit.ttf,BoldItalicFont=cmuntx.ttf]{cmuntt.ttf}\ttfamily equation}{$\text{ }$}\setmainfont[Path=/usr/share/fonts/truetype/cmu/,UprightFont=cmunrm.ttf,BoldFont=cmunbx.ttf,ItalicFont=cmunti.ttf,BoldItalicFont=cmunbi.ttf]{cmunrm.ttf}\setmonofont[Path=/usr/share/fonts/truetype/cmu/,UprightFont=cmuntt.ttf,BoldFont=cmuntb.ttf,ItalicFont=cmunit.ttf,BoldItalicFont=cmuntx.ttf]{cmunrm.ttf} environment; that is the best choice for one-{}line formulae, whether you are using {\ttfamily \setmainfont[Path=/usr/share/fonts/truetype/cmu/,UprightFont=cmunrm.ttf,BoldFont=cmunbx.ttf,ItalicFont=cmunti.ttf,BoldItalicFont=cmunbi.ttf]{cmuntt.ttf}\setmonofont[Path=/usr/share/fonts/truetype/cmu/,UprightFont=cmuntt.ttf,BoldFont=cmuntb.ttf,ItalicFont=cmunit.ttf,BoldItalicFont=cmuntx.ttf]{cmuntt.ttf}\ttfamily amsmath}{$\text{ }$}\setmainfont[Path=/usr/share/fonts/truetype/cmu/,UprightFont=cmunrm.ttf,BoldFont=cmunbx.ttf,ItalicFont=cmunti.ttf,BoldItalicFont=cmunbi.ttf]{cmunrm.ttf}\setmonofont[Path=/usr/share/fonts/truetype/cmu/,UprightFont=cmuntt.ttf,BoldFont=cmuntb.ttf,ItalicFont=cmunit.ttf,BoldItalicFont=cmuntx.ttf]{cmunrm.ttf} or not. Note also the {\itshape \setmainfont[Path=/usr/share/fonts/truetype/cmu/,UprightFont=cmunrm.ttf,BoldFont=cmunbx.ttf,ItalicFont=cmunti.ttf,BoldItalicFont=cmunbi.ttf]{cmunti.ttf}\setmonofont[Path=/usr/share/fonts/truetype/cmu/,UprightFont=cmuntt.ttf,BoldFont=cmuntb.ttf,ItalicFont=cmunit.ttf,BoldItalicFont=cmuntx.ttf]{cmunti.ttf}\itshape eq:}{$\text{ }$}\setmainfont[Path=/usr/share/fonts/truetype/cmu/,UprightFont=cmunrm.ttf,BoldFont=cmunbx.ttf,ItalicFont=cmunti.ttf,BoldItalicFont=cmunbi.ttf]{cmunrm.ttf}\setmonofont[Path=/usr/share/fonts/truetype/cmu/,UprightFont=cmuntt.ttf,BoldFont=cmuntb.ttf,ItalicFont=cmunit.ttf,BoldItalicFont=cmuntx.ttf]{cmunrm.ttf} prefix in the label.
\subsubsection{{\ttfamily \setmainfont[Path=/usr/share/fonts/truetype/cmu/,UprightFont=cmunrm.ttf,BoldFont=cmunbx.ttf,ItalicFont=cmunti.ttf,BoldItalicFont=cmunbi.ttf]{cmuntt.ttf}\setmonofont[Path=/usr/share/fonts/truetype/cmu/,UprightFont=cmuntt.ttf,BoldFont=cmuntb.ttf,ItalicFont=cmunit.ttf,BoldItalicFont=cmuntx.ttf]{cmuntt.ttf}\ttfamily eqref}}
\label{425}\setmainfont[Path=/usr/share/fonts/truetype/cmu/,UprightFont=cmunrm.ttf,BoldFont=cmunbx.ttf,ItalicFont=cmunti.ttf,BoldItalicFont=cmunbi.ttf]{cmunrm.ttf}\setmonofont[Path=/usr/share/fonts/truetype/cmu/,UprightFont=cmuntt.ttf,BoldFont=cmuntb.ttf,ItalicFont=cmunit.ttf,BoldItalicFont=cmuntx.ttf]{cmunrm.ttf}

The {\ttfamily \setmainfont[Path=/usr/share/fonts/truetype/cmu/,UprightFont=cmunrm.ttf,BoldFont=cmunbx.ttf,ItalicFont=cmunti.ttf,BoldItalicFont=cmunbi.ttf]{cmuntt.ttf}\setmonofont[Path=/usr/share/fonts/truetype/cmu/,UprightFont=cmuntt.ttf,BoldFont=cmuntb.ttf,ItalicFont=cmunit.ttf,BoldItalicFont=cmuntx.ttf]{cmuntt.ttf}\ttfamily amsmath}{$\text{ }$}\setmainfont[Path=/usr/share/fonts/truetype/cmu/,UprightFont=cmunrm.ttf,BoldFont=cmunbx.ttf,ItalicFont=cmunti.ttf,BoldItalicFont=cmunbi.ttf]{cmunrm.ttf}\setmonofont[Path=/usr/share/fonts/truetype/cmu/,UprightFont=cmuntt.ttf,BoldFont=cmuntb.ttf,ItalicFont=cmunit.ttf,BoldItalicFont=cmuntx.ttf]{cmunrm.ttf} package adds a new command for referencing formulae; it is {\ttfamily \setmainfont[Path=/usr/share/fonts/truetype/cmu/,UprightFont=cmunrm.ttf,BoldFont=cmunbx.ttf,ItalicFont=cmunti.ttf,BoldItalicFont=cmunbi.ttf]{cmuntt.ttf}\setmonofont[Path=/usr/share/fonts/truetype/cmu/,UprightFont=cmuntt.ttf,BoldFont=cmuntb.ttf,ItalicFont=cmunit.ttf,BoldItalicFont=cmuntx.ttf]{cmuntt.ttf}\ttfamily \textbackslash{}eqref\{\}}\setmainfont[Path=/usr/share/fonts/truetype/cmu/,UprightFont=cmunrm.ttf,BoldFont=cmunbx.ttf,ItalicFont=cmunti.ttf,BoldItalicFont=cmunbi.ttf]{cmunrm.ttf}\setmonofont[Path=/usr/share/fonts/truetype/cmu/,UprightFont=cmuntt.ttf,BoldFont=cmuntb.ttf,ItalicFont=cmunit.ttf,BoldItalicFont=cmuntx.ttf]{cmunrm.ttf}. It works exactly like {\ttfamily \setmainfont[Path=/usr/share/fonts/truetype/cmu/,UprightFont=cmunrm.ttf,BoldFont=cmunbx.ttf,ItalicFont=cmunti.ttf,BoldItalicFont=cmunbi.ttf]{cmuntt.ttf}\setmonofont[Path=/usr/share/fonts/truetype/cmu/,UprightFont=cmuntt.ttf,BoldFont=cmuntb.ttf,ItalicFont=cmunit.ttf,BoldItalicFont=cmuntx.ttf]{cmuntt.ttf}\ttfamily \textbackslash{}ref\{\}}\setmainfont[Path=/usr/share/fonts/truetype/cmu/,UprightFont=cmunrm.ttf,BoldFont=cmunbx.ttf,ItalicFont=cmunti.ttf,BoldItalicFont=cmunbi.ttf]{cmunrm.ttf}\setmonofont[Path=/usr/share/fonts/truetype/cmu/,UprightFont=cmuntt.ttf,BoldFont=cmuntb.ttf,ItalicFont=cmunit.ttf,BoldItalicFont=cmuntx.ttf]{cmunrm.ttf}, but it adds parentheses so that, instead of printing a plain number as {\itshape \setmainfont[Path=/usr/share/fonts/truetype/cmu/,UprightFont=cmunrm.ttf,BoldFont=cmunbx.ttf,ItalicFont=cmunti.ttf,BoldItalicFont=cmunbi.ttf]{cmunti.ttf}\setmonofont[Path=/usr/share/fonts/truetype/cmu/,UprightFont=cmuntt.ttf,BoldFont=cmuntb.ttf,ItalicFont=cmunit.ttf,BoldItalicFont=cmuntx.ttf]{cmunti.ttf}\itshape 5}\setmainfont[Path=/usr/share/fonts/truetype/cmu/,UprightFont=cmunrm.ttf,BoldFont=cmunbx.ttf,ItalicFont=cmunti.ttf,BoldItalicFont=cmunbi.ttf]{cmunrm.ttf}\setmonofont[Path=/usr/share/fonts/truetype/cmu/,UprightFont=cmuntt.ttf,BoldFont=cmuntb.ttf,ItalicFont=cmunit.ttf,BoldItalicFont=cmuntx.ttf]{cmunrm.ttf}, it will print {\itshape \setmainfont[Path=/usr/share/fonts/truetype/cmu/,UprightFont=cmunrm.ttf,BoldFont=cmunbx.ttf,ItalicFont=cmunti.ttf,BoldItalicFont=cmunbi.ttf]{cmunti.ttf}\setmonofont[Path=/usr/share/fonts/truetype/cmu/,UprightFont=cmuntt.ttf,BoldFont=cmuntb.ttf,ItalicFont=cmunit.ttf,BoldItalicFont=cmuntx.ttf]{cmunti.ttf}\itshape (5)}\setmainfont[Path=/usr/share/fonts/truetype/cmu/,UprightFont=cmunrm.ttf,BoldFont=cmunbx.ttf,ItalicFont=cmunti.ttf,BoldItalicFont=cmunbi.ttf]{cmunrm.ttf}\setmonofont[Path=/usr/share/fonts/truetype/cmu/,UprightFont=cmuntt.ttf,BoldFont=cmuntb.ttf,ItalicFont=cmunit.ttf,BoldItalicFont=cmuntx.ttf]{cmunrm.ttf}. This can be useful to help the reader distinguish between formulae and other things, without the need to repeat the word \symbol{34}formula\symbol{34} before any reference. Its output can be changed as desired; for more information see the {\ttfamily \setmainfont[Path=/usr/share/fonts/truetype/cmu/,UprightFont=cmunrm.ttf,BoldFont=cmunbx.ttf,ItalicFont=cmunti.ttf,BoldItalicFont=cmunbi.ttf]{cmuntt.ttf}\setmonofont[Path=/usr/share/fonts/truetype/cmu/,UprightFont=cmuntt.ttf,BoldFont=cmuntb.ttf,ItalicFont=cmunit.ttf,BoldItalicFont=cmuntx.ttf]{cmuntt.ttf}\ttfamily amsmath}{$\text{ }$}\setmainfont[Path=/usr/share/fonts/truetype/cmu/,UprightFont=cmunrm.ttf,BoldFont=cmunbx.ttf,ItalicFont=cmunti.ttf,BoldItalicFont=cmunbi.ttf]{cmunrm.ttf}\setmonofont[Path=/usr/share/fonts/truetype/cmu/,UprightFont=cmuntt.ttf,BoldFont=cmuntb.ttf,ItalicFont=cmunit.ttf,BoldItalicFont=cmuntx.ttf]{cmunrm.ttf} documentation.
\subsubsection{{\ttfamily \setmainfont[Path=/usr/share/fonts/truetype/cmu/,UprightFont=cmunrm.ttf,BoldFont=cmunbx.ttf,ItalicFont=cmunti.ttf,BoldItalicFont=cmunbi.ttf]{cmuntt.ttf}\setmonofont[Path=/usr/share/fonts/truetype/cmu/,UprightFont=cmuntt.ttf,BoldFont=cmuntb.ttf,ItalicFont=cmunit.ttf,BoldItalicFont=cmuntx.ttf]{cmuntt.ttf}\ttfamily tag}}
\label{426}\setmainfont[Path=/usr/share/fonts/truetype/cmu/,UprightFont=cmunrm.ttf,BoldFont=cmunbx.ttf,ItalicFont=cmunti.ttf,BoldItalicFont=cmunbi.ttf]{cmunrm.ttf}\setmonofont[Path=/usr/share/fonts/truetype/cmu/,UprightFont=cmuntt.ttf,BoldFont=cmuntb.ttf,ItalicFont=cmunit.ttf,BoldItalicFont=cmuntx.ttf]{cmunrm.ttf}
The {\ttfamily \setmainfont[Path=/usr/share/fonts/truetype/cmu/,UprightFont=cmunrm.ttf,BoldFont=cmunbx.ttf,ItalicFont=cmunti.ttf,BoldItalicFont=cmunbi.ttf]{cmuntt.ttf}\setmonofont[Path=/usr/share/fonts/truetype/cmu/,UprightFont=cmuntt.ttf,BoldFont=cmuntb.ttf,ItalicFont=cmunit.ttf,BoldItalicFont=cmuntx.ttf]{cmuntt.ttf}\ttfamily \textbackslash{}tag\{eqnno\}}{$\text{ }$}\setmainfont[Path=/usr/share/fonts/truetype/cmu/,UprightFont=cmunrm.ttf,BoldFont=cmunbx.ttf,ItalicFont=cmunti.ttf,BoldItalicFont=cmunbi.ttf]{cmunrm.ttf}\setmonofont[Path=/usr/share/fonts/truetype/cmu/,UprightFont=cmuntt.ttf,BoldFont=cmuntb.ttf,ItalicFont=cmunit.ttf,BoldItalicFont=cmuntx.ttf]{cmunrm.ttf} command is used to manually set equation numbers where {\itshape \setmainfont[Path=/usr/share/fonts/truetype/cmu/,UprightFont=cmunrm.ttf,BoldFont=cmunbx.ttf,ItalicFont=cmunti.ttf,BoldItalicFont=cmunbi.ttf]{cmunti.ttf}\setmonofont[Path=/usr/share/fonts/truetype/cmu/,UprightFont=cmuntt.ttf,BoldFont=cmuntb.ttf,ItalicFont=cmunit.ttf,BoldItalicFont=cmuntx.ttf]{cmunti.ttf}\itshape eqnno}{$\text{ }$}\setmainfont[Path=/usr/share/fonts/truetype/cmu/,UprightFont=cmunrm.ttf,BoldFont=cmunbx.ttf,ItalicFont=cmunti.ttf,BoldItalicFont=cmunbi.ttf]{cmunrm.ttf}\setmonofont[Path=/usr/share/fonts/truetype/cmu/,UprightFont=cmuntt.ttf,BoldFont=cmuntb.ttf,ItalicFont=cmunit.ttf,BoldItalicFont=cmuntx.ttf]{cmunrm.ttf} is the arbitrary text string you want to appear in the document.  It is normally better to use labels, but sometimes hard-{}coded equation numbers might offer a useful work-{}around. This may for instance be useful if you want to repeat an equation that is used before, e.g. {\ttfamily \setmainfont[Path=/usr/share/fonts/truetype/cmu/,UprightFont=cmunrm.ttf,BoldFont=cmunbx.ttf,ItalicFont=cmunti.ttf,BoldItalicFont=cmunbi.ttf]{cmuntt.ttf}\setmonofont[Path=/usr/share/fonts/truetype/cmu/,UprightFont=cmuntt.ttf,BoldFont=cmuntb.ttf,ItalicFont=cmunit.ttf,BoldItalicFont=cmuntx.ttf]{cmuntt.ttf}\ttfamily \textbackslash{}tag\{\textbackslash{}ref\{eqn:before\}\}}\setmainfont[Path=/usr/share/fonts/truetype/cmu/,UprightFont=cmunrm.ttf,BoldFont=cmunbx.ttf,ItalicFont=cmunti.ttf,BoldItalicFont=cmunbi.ttf]{cmunrm.ttf}\setmonofont[Path=/usr/share/fonts/truetype/cmu/,UprightFont=cmuntt.ttf,BoldFont=cmuntb.ttf,ItalicFont=cmunit.ttf,BoldItalicFont=cmuntx.ttf]{cmunrm.ttf}.
\subsubsection{{\ttfamily \setmainfont[Path=/usr/share/fonts/truetype/cmu/,UprightFont=cmunrm.ttf,BoldFont=cmunbx.ttf,ItalicFont=cmunti.ttf,BoldItalicFont=cmunbi.ttf]{cmuntt.ttf}\setmonofont[Path=/usr/share/fonts/truetype/cmu/,UprightFont=cmuntt.ttf,BoldFont=cmuntb.ttf,ItalicFont=cmunit.ttf,BoldItalicFont=cmuntx.ttf]{cmuntt.ttf}\ttfamily numberwithin}}
\label{427}\setmainfont[Path=/usr/share/fonts/truetype/cmu/,UprightFont=cmunrm.ttf,BoldFont=cmunbx.ttf,ItalicFont=cmunti.ttf,BoldItalicFont=cmunbi.ttf]{cmunrm.ttf}\setmonofont[Path=/usr/share/fonts/truetype/cmu/,UprightFont=cmuntt.ttf,BoldFont=cmuntb.ttf,ItalicFont=cmunit.ttf,BoldItalicFont=cmuntx.ttf]{cmunrm.ttf}
The {\ttfamily \setmainfont[Path=/usr/share/fonts/truetype/cmu/,UprightFont=cmunrm.ttf,BoldFont=cmunbx.ttf,ItalicFont=cmunti.ttf,BoldItalicFont=cmunbi.ttf]{cmuntt.ttf}\setmonofont[Path=/usr/share/fonts/truetype/cmu/,UprightFont=cmuntt.ttf,BoldFont=cmuntb.ttf,ItalicFont=cmunit.ttf,BoldItalicFont=cmuntx.ttf]{cmuntt.ttf}\ttfamily amsmath}{$\text{ }$}\setmainfont[Path=/usr/share/fonts/truetype/cmu/,UprightFont=cmunrm.ttf,BoldFont=cmunbx.ttf,ItalicFont=cmunti.ttf,BoldItalicFont=cmunbi.ttf]{cmunrm.ttf}\setmonofont[Path=/usr/share/fonts/truetype/cmu/,UprightFont=cmuntt.ttf,BoldFont=cmuntb.ttf,ItalicFont=cmunit.ttf,BoldItalicFont=cmuntx.ttf]{cmunrm.ttf} package adds the {\ttfamily \setmainfont[Path=/usr/share/fonts/truetype/cmu/,UprightFont=cmunrm.ttf,BoldFont=cmunbx.ttf,ItalicFont=cmunti.ttf,BoldItalicFont=cmunbi.ttf]{cmuntt.ttf}\setmonofont[Path=/usr/share/fonts/truetype/cmu/,UprightFont=cmuntt.ttf,BoldFont=cmuntb.ttf,ItalicFont=cmunit.ttf,BoldItalicFont=cmuntx.ttf]{cmuntt.ttf}\ttfamily \textbackslash{}numberwithin\{countera\}\{counterb\}}{$\text{ }$}\setmainfont[Path=/usr/share/fonts/truetype/cmu/,UprightFont=cmunrm.ttf,BoldFont=cmunbx.ttf,ItalicFont=cmunti.ttf,BoldItalicFont=cmunbi.ttf]{cmunrm.ttf}\setmonofont[Path=/usr/share/fonts/truetype/cmu/,UprightFont=cmuntt.ttf,BoldFont=cmuntb.ttf,ItalicFont=cmunit.ttf,BoldItalicFont=cmuntx.ttf]{cmunrm.ttf} command which replaces the simple {\ttfamily \setmainfont[Path=/usr/share/fonts/truetype/cmu/,UprightFont=cmunrm.ttf,BoldFont=cmunbx.ttf,ItalicFont=cmunti.ttf,BoldItalicFont=cmunbi.ttf]{cmuntt.ttf}\setmonofont[Path=/usr/share/fonts/truetype/cmu/,UprightFont=cmuntt.ttf,BoldFont=cmuntb.ttf,ItalicFont=cmunit.ttf,BoldItalicFont=cmuntx.ttf]{cmuntt.ttf}\ttfamily countera}{$\text{ }$}\setmainfont[Path=/usr/share/fonts/truetype/cmu/,UprightFont=cmunrm.ttf,BoldFont=cmunbx.ttf,ItalicFont=cmunti.ttf,BoldItalicFont=cmunbi.ttf]{cmunrm.ttf}\setmonofont[Path=/usr/share/fonts/truetype/cmu/,UprightFont=cmuntt.ttf,BoldFont=cmuntb.ttf,ItalicFont=cmunit.ttf,BoldItalicFont=cmuntx.ttf]{cmunrm.ttf} by a more sophisticated
{\ttfamily \setmainfont[Path=/usr/share/fonts/truetype/cmu/,UprightFont=cmunrm.ttf,BoldFont=cmunbx.ttf,ItalicFont=cmunti.ttf,BoldItalicFont=cmunbi.ttf]{cmuntt.ttf}\setmonofont[Path=/usr/share/fonts/truetype/cmu/,UprightFont=cmuntt.ttf,BoldFont=cmuntb.ttf,ItalicFont=cmunit.ttf,BoldItalicFont=cmuntx.ttf]{cmuntt.ttf}\ttfamily counterb.countera}\setmainfont[Path=/usr/share/fonts/truetype/cmu/,UprightFont=cmunrm.ttf,BoldFont=cmunbx.ttf,ItalicFont=cmunti.ttf,BoldItalicFont=cmunbi.ttf]{cmunrm.ttf}\setmonofont[Path=/usr/share/fonts/truetype/cmu/,UprightFont=cmuntt.ttf,BoldFont=cmuntb.ttf,ItalicFont=cmunit.ttf,BoldItalicFont=cmuntx.ttf]{cmunrm.ttf}.  For example {\ttfamily \setmainfont[Path=/usr/share/fonts/truetype/cmu/,UprightFont=cmunrm.ttf,BoldFont=cmunbx.ttf,ItalicFont=cmunti.ttf,BoldItalicFont=cmunbi.ttf]{cmuntt.ttf}\setmonofont[Path=/usr/share/fonts/truetype/cmu/,UprightFont=cmuntt.ttf,BoldFont=cmuntb.ttf,ItalicFont=cmunit.ttf,BoldItalicFont=cmuntx.ttf]{cmuntt.ttf}\ttfamily \textbackslash{}numberwithin\{equation\}\{section\}}{$\text{ }$}\setmainfont[Path=/usr/share/fonts/truetype/cmu/,UprightFont=cmunrm.ttf,BoldFont=cmunbx.ttf,ItalicFont=cmunti.ttf,BoldItalicFont=cmunbi.ttf]{cmunrm.ttf}\setmonofont[Path=/usr/share/fonts/truetype/cmu/,UprightFont=cmuntt.ttf,BoldFont=cmuntb.ttf,ItalicFont=cmunit.ttf,BoldItalicFont=cmuntx.ttf]{cmunrm.ttf} in the preamble will prepend the section number to all equation numbers.
\subsubsection{{\ttfamily \setmainfont[Path=/usr/share/fonts/truetype/cmu/,UprightFont=cmunrm.ttf,BoldFont=cmunbx.ttf,ItalicFont=cmunti.ttf,BoldItalicFont=cmunbi.ttf]{cmuntt.ttf}\setmonofont[Path=/usr/share/fonts/truetype/cmu/,UprightFont=cmuntt.ttf,BoldFont=cmuntb.ttf,ItalicFont=cmunit.ttf,BoldItalicFont=cmuntx.ttf]{cmuntt.ttf}\ttfamily cases}}
\label{428}\setmainfont[Path=/usr/share/fonts/truetype/cmu/,UprightFont=cmunrm.ttf,BoldFont=cmunbx.ttf,ItalicFont=cmunti.ttf,BoldItalicFont=cmunbi.ttf]{cmunrm.ttf}\setmonofont[Path=/usr/share/fonts/truetype/cmu/,UprightFont=cmuntt.ttf,BoldFont=cmuntb.ttf,ItalicFont=cmunit.ttf,BoldItalicFont=cmuntx.ttf]{cmunrm.ttf}
The {\ttfamily \setmainfont[Path=/usr/share/fonts/truetype/cmu/,UprightFont=cmunrm.ttf,BoldFont=cmunbx.ttf,ItalicFont=cmunti.ttf,BoldItalicFont=cmunbi.ttf]{cmuntt.ttf}\setmonofont[Path=/usr/share/fonts/truetype/cmu/,UprightFont=cmuntt.ttf,BoldFont=cmuntb.ttf,ItalicFont=cmunit.ttf,BoldItalicFont=cmuntx.ttf]{cmuntt.ttf}\ttfamily cases}{$\text{ }$}\setmainfont[Path=/usr/share/fonts/truetype/cmu/,UprightFont=cmunrm.ttf,BoldFont=cmunbx.ttf,ItalicFont=cmunti.ttf,BoldItalicFont=cmunbi.ttf]{cmunrm.ttf}\setmonofont[Path=/usr/share/fonts/truetype/cmu/,UprightFont=cmuntt.ttf,BoldFont=cmuntb.ttf,ItalicFont=cmunit.ttf,BoldItalicFont=cmuntx.ttf]{cmunrm.ttf} package adds the {\ttfamily \setmainfont[Path=/usr/share/fonts/truetype/cmu/,UprightFont=cmunrm.ttf,BoldFont=cmunbx.ttf,ItalicFont=cmunti.ttf,BoldItalicFont=cmunbi.ttf]{cmuntt.ttf}\setmonofont[Path=/usr/share/fonts/truetype/cmu/,UprightFont=cmuntt.ttf,BoldFont=cmuntb.ttf,ItalicFont=cmunit.ttf,BoldItalicFont=cmuntx.ttf]{cmuntt.ttf}\ttfamily \textbackslash{}numcases}{$\text{ }$}\setmainfont[Path=/usr/share/fonts/truetype/cmu/,UprightFont=cmunrm.ttf,BoldFont=cmunbx.ttf,ItalicFont=cmunti.ttf,BoldItalicFont=cmunbi.ttf]{cmunrm.ttf}\setmonofont[Path=/usr/share/fonts/truetype/cmu/,UprightFont=cmuntt.ttf,BoldFont=cmuntb.ttf,ItalicFont=cmunit.ttf,BoldItalicFont=cmuntx.ttf]{cmunrm.ttf} and the {\ttfamily \setmainfont[Path=/usr/share/fonts/truetype/cmu/,UprightFont=cmunrm.ttf,BoldFont=cmunbx.ttf,ItalicFont=cmunti.ttf,BoldItalicFont=cmunbi.ttf]{cmuntt.ttf}\setmonofont[Path=/usr/share/fonts/truetype/cmu/,UprightFont=cmuntt.ttf,BoldFont=cmuntb.ttf,ItalicFont=cmunit.ttf,BoldItalicFont=cmuntx.ttf]{cmuntt.ttf}\ttfamily \textbackslash{}subnumcases}{$\text{ }$}\setmainfont[Path=/usr/share/fonts/truetype/cmu/,UprightFont=cmunrm.ttf,BoldFont=cmunbx.ttf,ItalicFont=cmunti.ttf,BoldItalicFont=cmunbi.ttf]{cmunrm.ttf}\setmonofont[Path=/usr/share/fonts/truetype/cmu/,UprightFont=cmuntt.ttf,BoldFont=cmuntb.ttf,ItalicFont=cmunit.ttf,BoldItalicFont=cmuntx.ttf]{cmunrm.ttf} commands, which produce multi-{}case equations with a separate equation number and a separate equation number plus a letter, respectively, for each case.
\section{The {\ttfamily \setmainfont[Path=/usr/share/fonts/truetype/cmu/,UprightFont=cmunrm.ttf,BoldFont=cmunbx.ttf,ItalicFont=cmunti.ttf,BoldItalicFont=cmunbi.ttf]{cmuntt.ttf}\setmonofont[Path=/usr/share/fonts/truetype/cmu/,UprightFont=cmuntt.ttf,BoldFont=cmuntb.ttf,ItalicFont=cmunit.ttf,BoldItalicFont=cmuntx.ttf]{cmuntt.ttf}\ttfamily varioref}{$\text{ }$}\setmainfont[Path=/usr/share/fonts/truetype/cmu/,UprightFont=cmunrm.ttf,BoldFont=cmunbx.ttf,ItalicFont=cmunti.ttf,BoldItalicFont=cmunbi.ttf]{cmunrm.ttf}\setmonofont[Path=/usr/share/fonts/truetype/cmu/,UprightFont=cmuntt.ttf,BoldFont=cmuntb.ttf,ItalicFont=cmunit.ttf,BoldItalicFont=cmuntx.ttf]{cmunrm.ttf} package}
\label{429}

The {\ttfamily \setmainfont[Path=/usr/share/fonts/truetype/cmu/,UprightFont=cmunrm.ttf,BoldFont=cmunbx.ttf,ItalicFont=cmunti.ttf,BoldItalicFont=cmunbi.ttf]{cmuntt.ttf}\setmonofont[Path=/usr/share/fonts/truetype/cmu/,UprightFont=cmuntt.ttf,BoldFont=cmuntb.ttf,ItalicFont=cmunit.ttf,BoldItalicFont=cmuntx.ttf]{cmuntt.ttf}\ttfamily varioref}{$\text{ }$}\setmainfont[Path=/usr/share/fonts/truetype/cmu/,UprightFont=cmunrm.ttf,BoldFont=cmunbx.ttf,ItalicFont=cmunti.ttf,BoldItalicFont=cmunbi.ttf]{cmunrm.ttf}\setmonofont[Path=/usr/share/fonts/truetype/cmu/,UprightFont=cmuntt.ttf,BoldFont=cmuntb.ttf,ItalicFont=cmunit.ttf,BoldItalicFont=cmuntx.ttf]{cmunrm.ttf} package introduces a new command called {\ttfamily \setmainfont[Path=/usr/share/fonts/truetype/cmu/,UprightFont=cmunrm.ttf,BoldFont=cmunbx.ttf,ItalicFont=cmunti.ttf,BoldItalicFont=cmunbi.ttf]{cmuntt.ttf}\setmonofont[Path=/usr/share/fonts/truetype/cmu/,UprightFont=cmuntt.ttf,BoldFont=cmuntb.ttf,ItalicFont=cmunit.ttf,BoldItalicFont=cmuntx.ttf]{cmuntt.ttf}\ttfamily \textbackslash{}vref\{\}}\setmainfont[Path=/usr/share/fonts/truetype/cmu/,UprightFont=cmunrm.ttf,BoldFont=cmunbx.ttf,ItalicFont=cmunti.ttf,BoldItalicFont=cmunbi.ttf]{cmunrm.ttf}\setmonofont[Path=/usr/share/fonts/truetype/cmu/,UprightFont=cmuntt.ttf,BoldFont=cmuntb.ttf,ItalicFont=cmunit.ttf,BoldItalicFont=cmuntx.ttf]{cmunrm.ttf}. This command is used exactly like the basic {\ttfamily \setmainfont[Path=/usr/share/fonts/truetype/cmu/,UprightFont=cmunrm.ttf,BoldFont=cmunbx.ttf,ItalicFont=cmunti.ttf,BoldItalicFont=cmunbi.ttf]{cmuntt.ttf}\setmonofont[Path=/usr/share/fonts/truetype/cmu/,UprightFont=cmuntt.ttf,BoldFont=cmuntb.ttf,ItalicFont=cmunit.ttf,BoldItalicFont=cmuntx.ttf]{cmuntt.ttf}\ttfamily \textbackslash{}ref}\setmainfont[Path=/usr/share/fonts/truetype/cmu/,UprightFont=cmunrm.ttf,BoldFont=cmunbx.ttf,ItalicFont=cmunti.ttf,BoldItalicFont=cmunbi.ttf]{cmunrm.ttf}\setmonofont[Path=/usr/share/fonts/truetype/cmu/,UprightFont=cmuntt.ttf,BoldFont=cmuntb.ttf,ItalicFont=cmunit.ttf,BoldItalicFont=cmuntx.ttf]{cmunrm.ttf}, but it has a different output according to the context. If the object to be referenced is in the same page, it works just like {\ttfamily \setmainfont[Path=/usr/share/fonts/truetype/cmu/,UprightFont=cmunrm.ttf,BoldFont=cmunbx.ttf,ItalicFont=cmunti.ttf,BoldItalicFont=cmunbi.ttf]{cmuntt.ttf}\setmonofont[Path=/usr/share/fonts/truetype/cmu/,UprightFont=cmuntt.ttf,BoldFont=cmuntb.ttf,ItalicFont=cmunit.ttf,BoldItalicFont=cmuntx.ttf]{cmuntt.ttf}\ttfamily \textbackslash{}ref}\setmainfont[Path=/usr/share/fonts/truetype/cmu/,UprightFont=cmunrm.ttf,BoldFont=cmunbx.ttf,ItalicFont=cmunti.ttf,BoldItalicFont=cmunbi.ttf]{cmunrm.ttf}\setmonofont[Path=/usr/share/fonts/truetype/cmu/,UprightFont=cmuntt.ttf,BoldFont=cmuntb.ttf,ItalicFont=cmunit.ttf,BoldItalicFont=cmuntx.ttf]{cmunrm.ttf}; if the object is far away it will print something like \symbol{34}5 on page 25\symbol{34}, i.e. it adds the page number automatically. If the object is close, it can use more refined sentences like \symbol{34}on the next page\symbol{34} or \symbol{34}on the facing page\symbol{34} automatically, according to the context and the document class.

This command has to be used very carefully. It outputs more than one word, so it may happen its output falls on two different pages. In this case, the algorithm can get confused and cause a loop. Let\textquotesingle{}s make an example. You label an object on page 23 and the {\ttfamily \setmainfont[Path=/usr/share/fonts/truetype/cmu/,UprightFont=cmunrm.ttf,BoldFont=cmunbx.ttf,ItalicFont=cmunti.ttf,BoldItalicFont=cmunbi.ttf]{cmuntt.ttf}\setmonofont[Path=/usr/share/fonts/truetype/cmu/,UprightFont=cmuntt.ttf,BoldFont=cmuntb.ttf,ItalicFont=cmunit.ttf,BoldItalicFont=cmuntx.ttf]{cmuntt.ttf}\ttfamily \textbackslash{}vref}{$\text{ }$}\setmainfont[Path=/usr/share/fonts/truetype/cmu/,UprightFont=cmunrm.ttf,BoldFont=cmunbx.ttf,ItalicFont=cmunti.ttf,BoldItalicFont=cmunbi.ttf]{cmunrm.ttf}\setmonofont[Path=/usr/share/fonts/truetype/cmu/,UprightFont=cmuntt.ttf,BoldFont=cmuntb.ttf,ItalicFont=cmunit.ttf,BoldItalicFont=cmuntx.ttf]{cmunrm.ttf} output happens to stay between page 23 and 24. If it were on page 23, it would print like the basic {\ttfamily \setmainfont[Path=/usr/share/fonts/truetype/cmu/,UprightFont=cmunrm.ttf,BoldFont=cmunbx.ttf,ItalicFont=cmunti.ttf,BoldItalicFont=cmunbi.ttf]{cmuntt.ttf}\setmonofont[Path=/usr/share/fonts/truetype/cmu/,UprightFont=cmuntt.ttf,BoldFont=cmuntb.ttf,ItalicFont=cmunit.ttf,BoldItalicFont=cmuntx.ttf]{cmuntt.ttf}\ttfamily ref}\setmainfont[Path=/usr/share/fonts/truetype/cmu/,UprightFont=cmunrm.ttf,BoldFont=cmunbx.ttf,ItalicFont=cmunti.ttf,BoldItalicFont=cmunbi.ttf]{cmunrm.ttf}\setmonofont[Path=/usr/share/fonts/truetype/cmu/,UprightFont=cmuntt.ttf,BoldFont=cmuntb.ttf,ItalicFont=cmunit.ttf,BoldItalicFont=cmuntx.ttf]{cmunrm.ttf}, if it were on page 24, it would print \symbol{34}on the previous page\symbol{34}, but it is on both, and this may cause some strange errors at compiling time that are very hard to be fixed. You could think that this happens very rarely; unfortunately, if you write a long document it is not uncommon to have hundreds of references, so situations like these are likely to happen. One way to avoid problems during development is to use the standard {\ttfamily \setmainfont[Path=/usr/share/fonts/truetype/cmu/,UprightFont=cmunrm.ttf,BoldFont=cmunbx.ttf,ItalicFont=cmunti.ttf,BoldItalicFont=cmunbi.ttf]{cmuntt.ttf}\setmonofont[Path=/usr/share/fonts/truetype/cmu/,UprightFont=cmuntt.ttf,BoldFont=cmuntb.ttf,ItalicFont=cmunit.ttf,BoldItalicFont=cmuntx.ttf]{cmuntt.ttf}\ttfamily ref}{$\text{ }$}\setmainfont[Path=/usr/share/fonts/truetype/cmu/,UprightFont=cmunrm.ttf,BoldFont=cmunbx.ttf,ItalicFont=cmunti.ttf,BoldItalicFont=cmunbi.ttf]{cmunrm.ttf}\setmonofont[Path=/usr/share/fonts/truetype/cmu/,UprightFont=cmuntt.ttf,BoldFont=cmuntb.ttf,ItalicFont=cmunit.ttf,BoldItalicFont=cmuntx.ttf]{cmunrm.ttf} all the time, and convert it to {\ttfamily \setmainfont[Path=/usr/share/fonts/truetype/cmu/,UprightFont=cmunrm.ttf,BoldFont=cmunbx.ttf,ItalicFont=cmunti.ttf,BoldItalicFont=cmunbi.ttf]{cmuntt.ttf}\setmonofont[Path=/usr/share/fonts/truetype/cmu/,UprightFont=cmuntt.ttf,BoldFont=cmuntb.ttf,ItalicFont=cmunit.ttf,BoldItalicFont=cmuntx.ttf]{cmuntt.ttf}\ttfamily vref}{$\text{ }$}\setmainfont[Path=/usr/share/fonts/truetype/cmu/,UprightFont=cmunrm.ttf,BoldFont=cmunbx.ttf,ItalicFont=cmunti.ttf,BoldItalicFont=cmunbi.ttf]{cmunrm.ttf}\setmonofont[Path=/usr/share/fonts/truetype/cmu/,UprightFont=cmuntt.ttf,BoldFont=cmuntb.ttf,ItalicFont=cmunit.ttf,BoldItalicFont=cmuntx.ttf]{cmunrm.ttf} when the document is close to its final version, and then making adjustments to fix possible problems.
\section{The {\ttfamily \setmainfont[Path=/usr/share/fonts/truetype/cmu/,UprightFont=cmunrm.ttf,BoldFont=cmunbx.ttf,ItalicFont=cmunti.ttf,BoldItalicFont=cmunbi.ttf]{cmuntt.ttf}\setmonofont[Path=/usr/share/fonts/truetype/cmu/,UprightFont=cmuntt.ttf,BoldFont=cmuntb.ttf,ItalicFont=cmunit.ttf,BoldItalicFont=cmuntx.ttf]{cmuntt.ttf}\ttfamily hyperref}{$\text{ }$}\setmainfont[Path=/usr/share/fonts/truetype/cmu/,UprightFont=cmunrm.ttf,BoldFont=cmunbx.ttf,ItalicFont=cmunti.ttf,BoldItalicFont=cmunbi.ttf]{cmunrm.ttf}\setmonofont[Path=/usr/share/fonts/truetype/cmu/,UprightFont=cmuntt.ttf,BoldFont=cmuntb.ttf,ItalicFont=cmunit.ttf,BoldItalicFont=cmuntx.ttf]{cmunrm.ttf} package}
\label{430}
\subsection{{\ttfamily \setmainfont[Path=/usr/share/fonts/truetype/cmu/,UprightFont=cmunrm.ttf,BoldFont=cmunbx.ttf,ItalicFont=cmunti.ttf,BoldItalicFont=cmunbi.ttf]{cmuntt.ttf}\setmonofont[Path=/usr/share/fonts/truetype/cmu/,UprightFont=cmuntt.ttf,BoldFont=cmuntb.ttf,ItalicFont=cmunit.ttf,BoldItalicFont=cmuntx.ttf]{cmuntt.ttf}\ttfamily autoref}{$\text{ }$}\setmainfont[Path=/usr/share/fonts/truetype/cmu/,UprightFont=cmunrm.ttf,BoldFont=cmunbx.ttf,ItalicFont=cmunti.ttf,BoldItalicFont=cmunbi.ttf]{cmunrm.ttf}\setmonofont[Path=/usr/share/fonts/truetype/cmu/,UprightFont=cmuntt.ttf,BoldFont=cmuntb.ttf,ItalicFont=cmunit.ttf,BoldItalicFont=cmuntx.ttf]{cmunrm.ttf}}
\label{431}

The {\ttfamily \myhref{https://en.wikibooks.org/wiki/LaTeX\%2FPackages\%2FHyperref}{\setmainfont[Path=/usr/share/fonts/truetype/cmu/,UprightFont=cmunrm.ttf,BoldFont=cmunbx.ttf,ItalicFont=cmunti.ttf,BoldItalicFont=cmunbi.ttf]{cmuntt.ttf}\setmonofont[Path=/usr/share/fonts/truetype/cmu/,UprightFont=cmuntt.ttf,BoldFont=cmuntb.ttf,ItalicFont=cmunit.ttf,BoldItalicFont=cmuntx.ttf]{cmuntt.ttf}\ttfamily hyperref}} package introduces another useful command; {\ttfamily \setmainfont[Path=/usr/share/fonts/truetype/cmu/,UprightFont=cmunrm.ttf,BoldFont=cmunbx.ttf,ItalicFont=cmunti.ttf,BoldItalicFont=cmunbi.ttf]{cmuntt.ttf}\setmonofont[Path=/usr/share/fonts/truetype/cmu/,UprightFont=cmuntt.ttf,BoldFont=cmuntb.ttf,ItalicFont=cmunit.ttf,BoldItalicFont=cmuntx.ttf]{cmuntt.ttf}\ttfamily \textbackslash{}autoref\{\}}\setmainfont[Path=/usr/share/fonts/truetype/cmu/,UprightFont=cmunrm.ttf,BoldFont=cmunbx.ttf,ItalicFont=cmunti.ttf,BoldItalicFont=cmunbi.ttf]{cmunrm.ttf}\setmonofont[Path=/usr/share/fonts/truetype/cmu/,UprightFont=cmuntt.ttf,BoldFont=cmuntb.ttf,ItalicFont=cmunit.ttf,BoldItalicFont=cmuntx.ttf]{cmunrm.ttf}. This command creates a reference with additional text corresponding to the target\textquotesingle{}s type, all of which will be a hyperlink.  For example, the command {\ttfamily \setmainfont[Path=/usr/share/fonts/truetype/cmu/,UprightFont=cmunrm.ttf,BoldFont=cmunbx.ttf,ItalicFont=cmunti.ttf,BoldItalicFont=cmunbi.ttf]{cmuntt.ttf}\setmonofont[Path=/usr/share/fonts/truetype/cmu/,UprightFont=cmuntt.ttf,BoldFont=cmuntb.ttf,ItalicFont=cmunit.ttf,BoldItalicFont=cmuntx.ttf]{cmuntt.ttf}\ttfamily \textbackslash{}autoref\{sec:intro\}}{$\text{ }$}\setmainfont[Path=/usr/share/fonts/truetype/cmu/,UprightFont=cmunrm.ttf,BoldFont=cmunbx.ttf,ItalicFont=cmunti.ttf,BoldItalicFont=cmunbi.ttf]{cmunrm.ttf}\setmonofont[Path=/usr/share/fonts/truetype/cmu/,UprightFont=cmuntt.ttf,BoldFont=cmuntb.ttf,ItalicFont=cmunit.ttf,BoldItalicFont=cmuntx.ttf]{cmunrm.ttf} would create a hyperlink to the {\ttfamily \setmainfont[Path=/usr/share/fonts/truetype/cmu/,UprightFont=cmunrm.ttf,BoldFont=cmunbx.ttf,ItalicFont=cmunti.ttf,BoldItalicFont=cmunbi.ttf]{cmuntt.ttf}\setmonofont[Path=/usr/share/fonts/truetype/cmu/,UprightFont=cmuntt.ttf,BoldFont=cmuntb.ttf,ItalicFont=cmunit.ttf,BoldItalicFont=cmuntx.ttf]{cmuntt.ttf}\ttfamily \textbackslash{}label\{sec:intro\}}{$\text{ }$}\setmainfont[Path=/usr/share/fonts/truetype/cmu/,UprightFont=cmunrm.ttf,BoldFont=cmunbx.ttf,ItalicFont=cmunti.ttf,BoldItalicFont=cmunbi.ttf]{cmunrm.ttf}\setmonofont[Path=/usr/share/fonts/truetype/cmu/,UprightFont=cmuntt.ttf,BoldFont=cmuntb.ttf,ItalicFont=cmunit.ttf,BoldItalicFont=cmuntx.ttf]{cmunrm.ttf} command, wherever it is. Assuming that this label is pointing to a section, the hyperlink would contain the text \symbol{34}section 3.4\symbol{34}, or similar (the full list of default names can be found \myhref{http://www.tug.org/applications/hyperref/manual.html\#TBL-24}{here}). Note that, while there\textquotesingle{}s an 
\begin{Shaded}
\begin{Highlighting}[]

\NormalTok{\textbackslash{}autoref}\AlertTok{*}\newline
\end{Highlighting}
\end{Shaded}
 command that produces an unlinked prefix (useful if the label is on the same page as the reference), no alternative 
\begin{Shaded}
\begin{Highlighting}[]

\NormalTok{\textbackslash{}Autoref}\newline
\end{Highlighting}
\end{Shaded}
 command is defined to produce capitalized versions (useful, for instance, when starting sentences); but since the capitalization of autoref names was chosen by the package author, you can customize the prefixed text by redefining {\ttfamily \setmainfont[Path=/usr/share/fonts/truetype/cmu/,UprightFont=cmunrm.ttf,BoldFont=cmunbx.ttf,ItalicFont=cmunti.ttf,BoldItalicFont=cmunbi.ttf]{cmuntt.ttf}\setmonofont[Path=/usr/share/fonts/truetype/cmu/,UprightFont=cmuntt.ttf,BoldFont=cmuntb.ttf,ItalicFont=cmunit.ttf,BoldItalicFont=cmuntx.ttf]{cmuntt.ttf}\ttfamily \textbackslash{}{\itshape \setmainfont[Path=/usr/share/fonts/truetype/cmu/,UprightFont=cmunrm.ttf,BoldFont=cmunbx.ttf,ItalicFont=cmunti.ttf,BoldItalicFont=cmunbi.ttf]{cmunit.ttf}\setmonofont[Path=/usr/share/fonts/truetype/cmu/,UprightFont=cmuntt.ttf,BoldFont=cmuntb.ttf,ItalicFont=cmunit.ttf,BoldItalicFont=cmuntx.ttf]{cmunit.ttf}\ttfamily \itshape type}\setmainfont[Path=/usr/share/fonts/truetype/cmu/,UprightFont=cmunrm.ttf,BoldFont=cmunbx.ttf,ItalicFont=cmunti.ttf,BoldItalicFont=cmunbi.ttf]{cmuntt.ttf}\setmonofont[Path=/usr/share/fonts/truetype/cmu/,UprightFont=cmuntt.ttf,BoldFont=cmuntb.ttf,ItalicFont=cmunit.ttf,BoldItalicFont=cmuntx.ttf]{cmuntt.ttf}\ttfamily autorefname}{$\text{ }$}\setmainfont[Path=/usr/share/fonts/truetype/cmu/,UprightFont=cmunrm.ttf,BoldFont=cmunbx.ttf,ItalicFont=cmunti.ttf,BoldItalicFont=cmunbi.ttf]{cmunrm.ttf}\setmonofont[Path=/usr/share/fonts/truetype/cmu/,UprightFont=cmuntt.ttf,BoldFont=cmuntb.ttf,ItalicFont=cmunit.ttf,BoldItalicFont=cmuntx.ttf]{cmunrm.ttf} to the prefix you want, as in:

\begin{Shaded}
\begin{Highlighting}[]

\NormalTok{\textbackslash{}def\textbackslash{}sectionautorefname\{Section\}}\newline
\end{Highlighting}
\end{Shaded}

This renaming trick can, of course, be used for other purposes as well.

\begin{myitemize}
\item{}  If you would like a hyperlink reference, but do not want the predefined text that {\ttfamily \setmainfont[Path=/usr/share/fonts/truetype/cmu/,UprightFont=cmunrm.ttf,BoldFont=cmunbx.ttf,ItalicFont=cmunti.ttf,BoldItalicFont=cmunbi.ttf]{cmuntt.ttf}\setmonofont[Path=/usr/share/fonts/truetype/cmu/,UprightFont=cmuntt.ttf,BoldFont=cmuntb.ttf,ItalicFont=cmunit.ttf,BoldItalicFont=cmuntx.ttf]{cmuntt.ttf}\ttfamily \textbackslash{}autoref\{\}}{$\text{ }$}\setmainfont[Path=/usr/share/fonts/truetype/cmu/,UprightFont=cmunrm.ttf,BoldFont=cmunbx.ttf,ItalicFont=cmunti.ttf,BoldItalicFont=cmunbi.ttf]{cmunrm.ttf}\setmonofont[Path=/usr/share/fonts/truetype/cmu/,UprightFont=cmuntt.ttf,BoldFont=cmuntb.ttf,ItalicFont=cmunit.ttf,BoldItalicFont=cmuntx.ttf]{cmunrm.ttf} provides, you can do this with a command such as {\ttfamily \setmainfont[Path=/usr/share/fonts/truetype/cmu/,UprightFont=cmunrm.ttf,BoldFont=cmunbx.ttf,ItalicFont=cmunti.ttf,BoldItalicFont=cmunbi.ttf]{cmuntt.ttf}\setmonofont[Path=/usr/share/fonts/truetype/cmu/,UprightFont=cmuntt.ttf,BoldFont=cmuntb.ttf,ItalicFont=cmunit.ttf,BoldItalicFont=cmuntx.ttf]{cmuntt.ttf}\ttfamily \textbackslash{}hyperref{$\text{[}$}sec:intro{$\text{]}$}\{Appendix\~{}\textbackslash{}ref*\{sec:intro\}\}}\setmainfont[Path=/usr/share/fonts/truetype/cmu/,UprightFont=cmunrm.ttf,BoldFont=cmunbx.ttf,ItalicFont=cmunti.ttf,BoldItalicFont=cmunbi.ttf]{cmunrm.ttf}\setmonofont[Path=/usr/share/fonts/truetype/cmu/,UprightFont=cmuntt.ttf,BoldFont=cmuntb.ttf,ItalicFont=cmunit.ttf,BoldItalicFont=cmuntx.ttf]{cmunrm.ttf}. Note that you can disable the creation of hyperlinks in {\ttfamily \setmainfont[Path=/usr/share/fonts/truetype/cmu/,UprightFont=cmunrm.ttf,BoldFont=cmunbx.ttf,ItalicFont=cmunti.ttf,BoldItalicFont=cmunbi.ttf]{cmuntt.ttf}\setmonofont[Path=/usr/share/fonts/truetype/cmu/,UprightFont=cmuntt.ttf,BoldFont=cmuntb.ttf,ItalicFont=cmunit.ttf,BoldItalicFont=cmuntx.ttf]{cmuntt.ttf}\ttfamily hyperref}\setmainfont[Path=/usr/share/fonts/truetype/cmu/,UprightFont=cmunrm.ttf,BoldFont=cmunbx.ttf,ItalicFont=cmunti.ttf,BoldItalicFont=cmunbi.ttf]{cmunrm.ttf}\setmonofont[Path=/usr/share/fonts/truetype/cmu/,UprightFont=cmuntt.ttf,BoldFont=cmuntb.ttf,ItalicFont=cmunit.ttf,BoldItalicFont=cmuntx.ttf]{cmunrm.ttf}, and just use these commands for automatic text.
\end{myitemize}


\begin{myitemize}
\item{}  Keep in mind that the \textbackslash{}label {\bfseries \setmainfont[Path=/usr/share/fonts/truetype/cmu/,UprightFont=cmunrm.ttf,BoldFont=cmunbx.ttf,ItalicFont=cmunti.ttf,BoldItalicFont=cmunbi.ttf]{cmunbx.ttf}\setmonofont[Path=/usr/share/fonts/truetype/cmu/,UprightFont=cmuntt.ttf,BoldFont=cmuntb.ttf,ItalicFont=cmunit.ttf,BoldItalicFont=cmuntx.ttf]{cmunbx.ttf}\bfseries must}{$\text{ }$}\setmainfont[Path=/usr/share/fonts/truetype/cmu/,UprightFont=cmunrm.ttf,BoldFont=cmunbx.ttf,ItalicFont=cmunti.ttf,BoldItalicFont=cmunbi.ttf]{cmunrm.ttf}\setmonofont[Path=/usr/share/fonts/truetype/cmu/,UprightFont=cmuntt.ttf,BoldFont=cmuntb.ttf,ItalicFont=cmunit.ttf,BoldItalicFont=cmuntx.ttf]{cmunrm.ttf} be placed inside an environment with a counter, such as a table or a figure. Otherwise, not only the number will refer to the current section, as mentioned \mylref{422}{above}, but the name will refer to the previous environment with a counter. For example, if you put a label after closing a figure, the label will still say \symbol{34}figure n\symbol{34}, on which n is the current section number.
\end{myitemize}

\subsection{{\ttfamily \setmainfont[Path=/usr/share/fonts/truetype/cmu/,UprightFont=cmunrm.ttf,BoldFont=cmunbx.ttf,ItalicFont=cmunti.ttf,BoldItalicFont=cmunbi.ttf]{cmuntt.ttf}\setmonofont[Path=/usr/share/fonts/truetype/cmu/,UprightFont=cmuntt.ttf,BoldFont=cmuntb.ttf,ItalicFont=cmunit.ttf,BoldItalicFont=cmuntx.ttf]{cmuntt.ttf}\ttfamily nameref}{$\text{ }$}\setmainfont[Path=/usr/share/fonts/truetype/cmu/,UprightFont=cmunrm.ttf,BoldFont=cmunbx.ttf,ItalicFont=cmunti.ttf,BoldItalicFont=cmunbi.ttf]{cmunrm.ttf}\setmonofont[Path=/usr/share/fonts/truetype/cmu/,UprightFont=cmuntt.ttf,BoldFont=cmuntb.ttf,ItalicFont=cmunit.ttf,BoldItalicFont=cmuntx.ttf]{cmunrm.ttf}}
\label{432}

The {\ttfamily \setmainfont[Path=/usr/share/fonts/truetype/cmu/,UprightFont=cmunrm.ttf,BoldFont=cmunbx.ttf,ItalicFont=cmunti.ttf,BoldItalicFont=cmunbi.ttf]{cmuntt.ttf}\setmonofont[Path=/usr/share/fonts/truetype/cmu/,UprightFont=cmuntt.ttf,BoldFont=cmuntb.ttf,ItalicFont=cmunit.ttf,BoldItalicFont=cmuntx.ttf]{cmuntt.ttf}\ttfamily hyperref}{$\text{ }$}\setmainfont[Path=/usr/share/fonts/truetype/cmu/,UprightFont=cmunrm.ttf,BoldFont=cmunbx.ttf,ItalicFont=cmunti.ttf,BoldItalicFont=cmunbi.ttf]{cmunrm.ttf}\setmonofont[Path=/usr/share/fonts/truetype/cmu/,UprightFont=cmuntt.ttf,BoldFont=cmuntb.ttf,ItalicFont=cmunit.ttf,BoldItalicFont=cmuntx.ttf]{cmunrm.ttf} package also automatically includes the {\ttfamily \setmainfont[Path=/usr/share/fonts/truetype/cmu/,UprightFont=cmunrm.ttf,BoldFont=cmunbx.ttf,ItalicFont=cmunti.ttf,BoldItalicFont=cmunbi.ttf]{cmuntt.ttf}\setmonofont[Path=/usr/share/fonts/truetype/cmu/,UprightFont=cmuntt.ttf,BoldFont=cmuntb.ttf,ItalicFont=cmunit.ttf,BoldItalicFont=cmuntx.ttf]{cmuntt.ttf}\ttfamily nameref}{$\text{ }$}\setmainfont[Path=/usr/share/fonts/truetype/cmu/,UprightFont=cmunrm.ttf,BoldFont=cmunbx.ttf,ItalicFont=cmunti.ttf,BoldItalicFont=cmunbi.ttf]{cmunrm.ttf}\setmonofont[Path=/usr/share/fonts/truetype/cmu/,UprightFont=cmuntt.ttf,BoldFont=cmuntb.ttf,ItalicFont=cmunit.ttf,BoldItalicFont=cmuntx.ttf]{cmunrm.ttf} package, and a similarly named command.  It is similar to {\ttfamily \setmainfont[Path=/usr/share/fonts/truetype/cmu/,UprightFont=cmunrm.ttf,BoldFont=cmunbx.ttf,ItalicFont=cmunti.ttf,BoldItalicFont=cmunbi.ttf]{cmuntt.ttf}\setmonofont[Path=/usr/share/fonts/truetype/cmu/,UprightFont=cmuntt.ttf,BoldFont=cmuntb.ttf,ItalicFont=cmunit.ttf,BoldItalicFont=cmuntx.ttf]{cmuntt.ttf}\ttfamily \textbackslash{}autoref\{\}}\setmainfont[Path=/usr/share/fonts/truetype/cmu/,UprightFont=cmunrm.ttf,BoldFont=cmunbx.ttf,ItalicFont=cmunti.ttf,BoldItalicFont=cmunbi.ttf]{cmunrm.ttf}\setmonofont[Path=/usr/share/fonts/truetype/cmu/,UprightFont=cmuntt.ttf,BoldFont=cmuntb.ttf,ItalicFont=cmunit.ttf,BoldItalicFont=cmuntx.ttf]{cmunrm.ttf}, but inserts text corresponding to the section name, for example.

Input:

\begin{Shaded}
\begin{Highlighting}[]

\NormalTok{\textbackslash{}section\{MyFirstSection\}\ensuremath{\text{ }}\textbackslash{}label\{sec:marker\}}\newline
\NormalTok{\textbackslash{}section\{MySecondSection\}}\newline
\NormalTok{In\ensuremath{\text{ }}section~\textbackslash{}nameref\{sec:marker\}\ensuremath{\text{ }}we\ensuremath{\text{ }}defined...}\newline
\end{Highlighting}
\end{Shaded}


Output:

In section MyFirstSection we defined...
\subsection{Anchor manual positioning}
\label{433}

When you define a {\ttfamily \setmainfont[Path=/usr/share/fonts/truetype/cmu/,UprightFont=cmunrm.ttf,BoldFont=cmunbx.ttf,ItalicFont=cmunti.ttf,BoldItalicFont=cmunbi.ttf]{cmuntt.ttf}\setmonofont[Path=/usr/share/fonts/truetype/cmu/,UprightFont=cmuntt.ttf,BoldFont=cmuntb.ttf,ItalicFont=cmunit.ttf,BoldItalicFont=cmuntx.ttf]{cmuntt.ttf}\ttfamily \textbackslash{}label}{$\text{ }$}\setmainfont[Path=/usr/share/fonts/truetype/cmu/,UprightFont=cmunrm.ttf,BoldFont=cmunbx.ttf,ItalicFont=cmunti.ttf,BoldItalicFont=cmunbi.ttf]{cmunrm.ttf}\setmonofont[Path=/usr/share/fonts/truetype/cmu/,UprightFont=cmuntt.ttf,BoldFont=cmuntb.ttf,ItalicFont=cmunit.ttf,BoldItalicFont=cmuntx.ttf]{cmunrm.ttf} outside a figure, a table, or other floating objects, the label points to the current section. In some cases, this behavior is
not what you\textquotesingle{}d like and you\textquotesingle{}d prefer the generated link to point to the line where the {\ttfamily \setmainfont[Path=/usr/share/fonts/truetype/cmu/,UprightFont=cmunrm.ttf,BoldFont=cmunbx.ttf,ItalicFont=cmunti.ttf,BoldItalicFont=cmunbi.ttf]{cmuntt.ttf}\setmonofont[Path=/usr/share/fonts/truetype/cmu/,UprightFont=cmuntt.ttf,BoldFont=cmuntb.ttf,ItalicFont=cmunit.ttf,BoldItalicFont=cmuntx.ttf]{cmuntt.ttf}\ttfamily \textbackslash{}label}{$\text{ }$}\setmainfont[Path=/usr/share/fonts/truetype/cmu/,UprightFont=cmunrm.ttf,BoldFont=cmunbx.ttf,ItalicFont=cmunti.ttf,BoldItalicFont=cmunbi.ttf]{cmunrm.ttf}\setmonofont[Path=/usr/share/fonts/truetype/cmu/,UprightFont=cmuntt.ttf,BoldFont=cmuntb.ttf,ItalicFont=cmunit.ttf,BoldItalicFont=cmuntx.ttf]{cmunrm.ttf} is defined. This can be achieved with the command
{\ttfamily \setmainfont[Path=/usr/share/fonts/truetype/cmu/,UprightFont=cmunrm.ttf,BoldFont=cmunbx.ttf,ItalicFont=cmunti.ttf,BoldItalicFont=cmunbi.ttf]{cmuntt.ttf}\setmonofont[Path=/usr/share/fonts/truetype/cmu/,UprightFont=cmuntt.ttf,BoldFont=cmuntb.ttf,ItalicFont=cmunit.ttf,BoldItalicFont=cmuntx.ttf]{cmuntt.ttf}\ttfamily \textbackslash{}phantomsection}{$\text{ }$}\setmainfont[Path=/usr/share/fonts/truetype/cmu/,UprightFont=cmunrm.ttf,BoldFont=cmunbx.ttf,ItalicFont=cmunti.ttf,BoldItalicFont=cmunbi.ttf]{cmunrm.ttf}\setmonofont[Path=/usr/share/fonts/truetype/cmu/,UprightFont=cmuntt.ttf,BoldFont=cmuntb.ttf,ItalicFont=cmunit.ttf,BoldItalicFont=cmuntx.ttf]{cmunrm.ttf} as in this example:


\begin{Shaded}
\begin{Highlighting}[]

\CommentTok{\%The\ensuremath{\text{ }}link\ensuremath{\text{ }}location\ensuremath{\text{ }}will\ensuremath{\text{ }}be\ensuremath{\text{ }}placed\ensuremath{\text{ }}on\ensuremath{\text{ }}the\ensuremath{\text{ }}line\ensuremath{\text{ }}below.}\newline
\NormalTok{\textbackslash{}phantomsection}\newline
\NormalTok{\textbackslash{}label\{the_label\}}\newline
\end{Highlighting}
\end{Shaded}

\section{The {\ttfamily \setmainfont[Path=/usr/share/fonts/truetype/cmu/,UprightFont=cmunrm.ttf,BoldFont=cmunbx.ttf,ItalicFont=cmunti.ttf,BoldItalicFont=cmunbi.ttf]{cmuntt.ttf}\setmonofont[Path=/usr/share/fonts/truetype/cmu/,UprightFont=cmuntt.ttf,BoldFont=cmuntb.ttf,ItalicFont=cmunit.ttf,BoldItalicFont=cmuntx.ttf]{cmuntt.ttf}\ttfamily cleveref}{$\text{ }$}\setmainfont[Path=/usr/share/fonts/truetype/cmu/,UprightFont=cmunrm.ttf,BoldFont=cmunbx.ttf,ItalicFont=cmunti.ttf,BoldItalicFont=cmunbi.ttf]{cmunrm.ttf}\setmonofont[Path=/usr/share/fonts/truetype/cmu/,UprightFont=cmuntt.ttf,BoldFont=cmuntb.ttf,ItalicFont=cmunit.ttf,BoldItalicFont=cmuntx.ttf]{cmunrm.ttf} package}
\label{434}

The {\ttfamily \setmainfont[Path=/usr/share/fonts/truetype/cmu/,UprightFont=cmunrm.ttf,BoldFont=cmunbx.ttf,ItalicFont=cmunti.ttf,BoldItalicFont=cmunbi.ttf]{cmuntt.ttf}\setmonofont[Path=/usr/share/fonts/truetype/cmu/,UprightFont=cmuntt.ttf,BoldFont=cmuntb.ttf,ItalicFont=cmunit.ttf,BoldItalicFont=cmuntx.ttf]{cmuntt.ttf}\ttfamily cleveref}{$\text{ }$}\setmainfont[Path=/usr/share/fonts/truetype/cmu/,UprightFont=cmunrm.ttf,BoldFont=cmunbx.ttf,ItalicFont=cmunti.ttf,BoldItalicFont=cmunbi.ttf]{cmunrm.ttf}\setmonofont[Path=/usr/share/fonts/truetype/cmu/,UprightFont=cmuntt.ttf,BoldFont=cmuntb.ttf,ItalicFont=cmunit.ttf,BoldItalicFont=cmuntx.ttf]{cmunrm.ttf} package introduces the new command {\ttfamily \setmainfont[Path=/usr/share/fonts/truetype/cmu/,UprightFont=cmunrm.ttf,BoldFont=cmunbx.ttf,ItalicFont=cmunti.ttf,BoldItalicFont=cmunbi.ttf]{cmuntt.ttf}\setmonofont[Path=/usr/share/fonts/truetype/cmu/,UprightFont=cmuntt.ttf,BoldFont=cmuntb.ttf,ItalicFont=cmunit.ttf,BoldItalicFont=cmuntx.ttf]{cmuntt.ttf}\ttfamily \textbackslash{}cref\{\}}{$\text{ }$}\setmainfont[Path=/usr/share/fonts/truetype/cmu/,UprightFont=cmunrm.ttf,BoldFont=cmunbx.ttf,ItalicFont=cmunti.ttf,BoldItalicFont=cmunbi.ttf]{cmunrm.ttf}\setmonofont[Path=/usr/share/fonts/truetype/cmu/,UprightFont=cmuntt.ttf,BoldFont=cmuntb.ttf,ItalicFont=cmunit.ttf,BoldItalicFont=cmuntx.ttf]{cmunrm.ttf} which includes the type of referenced object like {\ttfamily \setmainfont[Path=/usr/share/fonts/truetype/cmu/,UprightFont=cmunrm.ttf,BoldFont=cmunbx.ttf,ItalicFont=cmunti.ttf,BoldItalicFont=cmunbi.ttf]{cmuntt.ttf}\setmonofont[Path=/usr/share/fonts/truetype/cmu/,UprightFont=cmuntt.ttf,BoldFont=cmuntb.ttf,ItalicFont=cmunit.ttf,BoldItalicFont=cmuntx.ttf]{cmuntt.ttf}\ttfamily \textbackslash{}autoref\{\}}{$\text{ }$}\setmainfont[Path=/usr/share/fonts/truetype/cmu/,UprightFont=cmunrm.ttf,BoldFont=cmunbx.ttf,ItalicFont=cmunti.ttf,BoldItalicFont=cmunbi.ttf]{cmunrm.ttf}\setmonofont[Path=/usr/share/fonts/truetype/cmu/,UprightFont=cmuntt.ttf,BoldFont=cmuntb.ttf,ItalicFont=cmunit.ttf,BoldItalicFont=cmuntx.ttf]{cmunrm.ttf} does. The alternate {\ttfamily \setmainfont[Path=/usr/share/fonts/truetype/cmu/,UprightFont=cmunrm.ttf,BoldFont=cmunbx.ttf,ItalicFont=cmunti.ttf,BoldItalicFont=cmunbi.ttf]{cmuntt.ttf}\setmonofont[Path=/usr/share/fonts/truetype/cmu/,UprightFont=cmuntt.ttf,BoldFont=cmuntb.ttf,ItalicFont=cmunit.ttf,BoldItalicFont=cmuntx.ttf]{cmuntt.ttf}\ttfamily \textbackslash{}labelcref\{\}}{$\text{ }$}\setmainfont[Path=/usr/share/fonts/truetype/cmu/,UprightFont=cmunrm.ttf,BoldFont=cmunbx.ttf,ItalicFont=cmunti.ttf,BoldItalicFont=cmunbi.ttf]{cmunrm.ttf}\setmonofont[Path=/usr/share/fonts/truetype/cmu/,UprightFont=cmuntt.ttf,BoldFont=cmuntb.ttf,ItalicFont=cmunit.ttf,BoldItalicFont=cmuntx.ttf]{cmunrm.ttf} command works more like standard {\ttfamily \setmainfont[Path=/usr/share/fonts/truetype/cmu/,UprightFont=cmunrm.ttf,BoldFont=cmunbx.ttf,ItalicFont=cmunti.ttf,BoldItalicFont=cmunbi.ttf]{cmuntt.ttf}\setmonofont[Path=/usr/share/fonts/truetype/cmu/,UprightFont=cmuntt.ttf,BoldFont=cmuntb.ttf,ItalicFont=cmunit.ttf,BoldItalicFont=cmuntx.ttf]{cmuntt.ttf}\ttfamily \textbackslash{}ref\{\}}\setmainfont[Path=/usr/share/fonts/truetype/cmu/,UprightFont=cmunrm.ttf,BoldFont=cmunbx.ttf,ItalicFont=cmunti.ttf,BoldItalicFont=cmunbi.ttf]{cmunrm.ttf}\setmonofont[Path=/usr/share/fonts/truetype/cmu/,UprightFont=cmuntt.ttf,BoldFont=cmuntb.ttf,ItalicFont=cmunit.ttf,BoldItalicFont=cmuntx.ttf]{cmunrm.ttf}. References to pages are handled by the {\ttfamily \setmainfont[Path=/usr/share/fonts/truetype/cmu/,UprightFont=cmunrm.ttf,BoldFont=cmunbx.ttf,ItalicFont=cmunti.ttf,BoldItalicFont=cmunbi.ttf]{cmuntt.ttf}\setmonofont[Path=/usr/share/fonts/truetype/cmu/,UprightFont=cmuntt.ttf,BoldFont=cmuntb.ttf,ItalicFont=cmunit.ttf,BoldItalicFont=cmuntx.ttf]{cmuntt.ttf}\ttfamily \textbackslash{}cpageref\{\}}{$\text{ }$}\setmainfont[Path=/usr/share/fonts/truetype/cmu/,UprightFont=cmunrm.ttf,BoldFont=cmunbx.ttf,ItalicFont=cmunti.ttf,BoldItalicFont=cmunbi.ttf]{cmunrm.ttf}\setmonofont[Path=/usr/share/fonts/truetype/cmu/,UprightFont=cmuntt.ttf,BoldFont=cmuntb.ttf,ItalicFont=cmunit.ttf,BoldItalicFont=cmuntx.ttf]{cmunrm.ttf} command.

The {\ttfamily \setmainfont[Path=/usr/share/fonts/truetype/cmu/,UprightFont=cmunrm.ttf,BoldFont=cmunbx.ttf,ItalicFont=cmunti.ttf,BoldItalicFont=cmunbi.ttf]{cmuntt.ttf}\setmonofont[Path=/usr/share/fonts/truetype/cmu/,UprightFont=cmuntt.ttf,BoldFont=cmuntb.ttf,ItalicFont=cmunit.ttf,BoldItalicFont=cmuntx.ttf]{cmuntt.ttf}\ttfamily \textbackslash{}crefrange\{\}\{\}}{$\text{ }$}\setmainfont[Path=/usr/share/fonts/truetype/cmu/,UprightFont=cmunrm.ttf,BoldFont=cmunbx.ttf,ItalicFont=cmunti.ttf,BoldItalicFont=cmunbi.ttf]{cmunrm.ttf}\setmonofont[Path=/usr/share/fonts/truetype/cmu/,UprightFont=cmuntt.ttf,BoldFont=cmuntb.ttf,ItalicFont=cmunit.ttf,BoldItalicFont=cmuntx.ttf]{cmunrm.ttf} and {\ttfamily \setmainfont[Path=/usr/share/fonts/truetype/cmu/,UprightFont=cmunrm.ttf,BoldFont=cmunbx.ttf,ItalicFont=cmunti.ttf,BoldItalicFont=cmunbi.ttf]{cmuntt.ttf}\setmonofont[Path=/usr/share/fonts/truetype/cmu/,UprightFont=cmuntt.ttf,BoldFont=cmuntb.ttf,ItalicFont=cmunit.ttf,BoldItalicFont=cmuntx.ttf]{cmuntt.ttf}\ttfamily \textbackslash{}cpagerefrange\{\}}{$\text{ }$}\setmainfont[Path=/usr/share/fonts/truetype/cmu/,UprightFont=cmunrm.ttf,BoldFont=cmunbx.ttf,ItalicFont=cmunti.ttf,BoldItalicFont=cmunbi.ttf]{cmunrm.ttf}\setmonofont[Path=/usr/share/fonts/truetype/cmu/,UprightFont=cmuntt.ttf,BoldFont=cmuntb.ttf,ItalicFont=cmunit.ttf,BoldItalicFont=cmuntx.ttf]{cmunrm.ttf} commands expect a start and end label in either order and provide a natural language ({\ttfamily \setmainfont[Path=/usr/share/fonts/truetype/cmu/,UprightFont=cmunrm.ttf,BoldFont=cmunbx.ttf,ItalicFont=cmunti.ttf,BoldItalicFont=cmunbi.ttf]{cmuntt.ttf}\setmonofont[Path=/usr/share/fonts/truetype/cmu/,UprightFont=cmuntt.ttf,BoldFont=cmuntb.ttf,ItalicFont=cmunit.ttf,BoldItalicFont=cmuntx.ttf]{cmuntt.ttf}\ttfamily babel}{$\text{ }$}\setmainfont[Path=/usr/share/fonts/truetype/cmu/,UprightFont=cmunrm.ttf,BoldFont=cmunbx.ttf,ItalicFont=cmunti.ttf,BoldItalicFont=cmunbi.ttf]{cmunrm.ttf}\setmonofont[Path=/usr/share/fonts/truetype/cmu/,UprightFont=cmuntt.ttf,BoldFont=cmuntb.ttf,ItalicFont=cmunit.ttf,BoldItalicFont=cmuntx.ttf]{cmunrm.ttf} enabled) range. If labels are enumerated as a comma-{}separated list with the usual {\ttfamily \setmainfont[Path=/usr/share/fonts/truetype/cmu/,UprightFont=cmunrm.ttf,BoldFont=cmunbx.ttf,ItalicFont=cmunti.ttf,BoldItalicFont=cmunbi.ttf]{cmuntt.ttf}\setmonofont[Path=/usr/share/fonts/truetype/cmu/,UprightFont=cmuntt.ttf,BoldFont=cmuntb.ttf,ItalicFont=cmunit.ttf,BoldItalicFont=cmuntx.ttf]{cmuntt.ttf}\ttfamily \textbackslash{}cref\{\}}{$\text{ }$}\setmainfont[Path=/usr/share/fonts/truetype/cmu/,UprightFont=cmunrm.ttf,BoldFont=cmunbx.ttf,ItalicFont=cmunti.ttf,BoldItalicFont=cmunbi.ttf]{cmunrm.ttf}\setmonofont[Path=/usr/share/fonts/truetype/cmu/,UprightFont=cmuntt.ttf,BoldFont=cmuntb.ttf,ItalicFont=cmunit.ttf,BoldItalicFont=cmuntx.ttf]{cmunrm.ttf} command, it will sort them and group into ranges automatically.

The format can be specified in the preamble.
\section{Interpackage interactions for {\ttfamily \setmainfont[Path=/usr/share/fonts/truetype/cmu/,UprightFont=cmunrm.ttf,BoldFont=cmunbx.ttf,ItalicFont=cmunti.ttf,BoldItalicFont=cmunbi.ttf]{cmuntt.ttf}\setmonofont[Path=/usr/share/fonts/truetype/cmu/,UprightFont=cmuntt.ttf,BoldFont=cmuntb.ttf,ItalicFont=cmunit.ttf,BoldItalicFont=cmuntx.ttf]{cmuntt.ttf}\ttfamily varioref}{$\text{ }$}\setmainfont[Path=/usr/share/fonts/truetype/cmu/,UprightFont=cmunrm.ttf,BoldFont=cmunbx.ttf,ItalicFont=cmunti.ttf,BoldItalicFont=cmunbi.ttf]{cmunrm.ttf}\setmonofont[Path=/usr/share/fonts/truetype/cmu/,UprightFont=cmuntt.ttf,BoldFont=cmuntb.ttf,ItalicFont=cmunit.ttf,BoldItalicFont=cmuntx.ttf]{cmunrm.ttf} , {\ttfamily \setmainfont[Path=/usr/share/fonts/truetype/cmu/,UprightFont=cmunrm.ttf,BoldFont=cmunbx.ttf,ItalicFont=cmunti.ttf,BoldItalicFont=cmunbi.ttf]{cmuntt.ttf}\setmonofont[Path=/usr/share/fonts/truetype/cmu/,UprightFont=cmuntt.ttf,BoldFont=cmuntb.ttf,ItalicFont=cmunit.ttf,BoldItalicFont=cmuntx.ttf]{cmuntt.ttf}\ttfamily hyperref}{$\text{ }$}\setmainfont[Path=/usr/share/fonts/truetype/cmu/,UprightFont=cmunrm.ttf,BoldFont=cmunbx.ttf,ItalicFont=cmunti.ttf,BoldItalicFont=cmunbi.ttf]{cmunrm.ttf}\setmonofont[Path=/usr/share/fonts/truetype/cmu/,UprightFont=cmuntt.ttf,BoldFont=cmuntb.ttf,ItalicFont=cmunit.ttf,BoldItalicFont=cmuntx.ttf]{cmunrm.ttf} , and {\ttfamily \setmainfont[Path=/usr/share/fonts/truetype/cmu/,UprightFont=cmunrm.ttf,BoldFont=cmunbx.ttf,ItalicFont=cmunti.ttf,BoldItalicFont=cmunbi.ttf]{cmuntt.ttf}\setmonofont[Path=/usr/share/fonts/truetype/cmu/,UprightFont=cmuntt.ttf,BoldFont=cmuntb.ttf,ItalicFont=cmunit.ttf,BoldItalicFont=cmuntx.ttf]{cmuntt.ttf}\ttfamily cleveref}{$\text{ }$}\setmainfont[Path=/usr/share/fonts/truetype/cmu/,UprightFont=cmunrm.ttf,BoldFont=cmunbx.ttf,ItalicFont=cmunti.ttf,BoldItalicFont=cmunbi.ttf]{cmunrm.ttf}\setmonofont[Path=/usr/share/fonts/truetype/cmu/,UprightFont=cmuntt.ttf,BoldFont=cmuntb.ttf,ItalicFont=cmunit.ttf,BoldItalicFont=cmuntx.ttf]{cmunrm.ttf}}
\label{435}
Because {\ttfamily \setmainfont[Path=/usr/share/fonts/truetype/cmu/,UprightFont=cmunrm.ttf,BoldFont=cmunbx.ttf,ItalicFont=cmunti.ttf,BoldItalicFont=cmunbi.ttf]{cmuntt.ttf}\setmonofont[Path=/usr/share/fonts/truetype/cmu/,UprightFont=cmuntt.ttf,BoldFont=cmuntb.ttf,ItalicFont=cmunit.ttf,BoldItalicFont=cmuntx.ttf]{cmuntt.ttf}\ttfamily varioref}\setmainfont[Path=/usr/share/fonts/truetype/cmu/,UprightFont=cmunrm.ttf,BoldFont=cmunbx.ttf,ItalicFont=cmunti.ttf,BoldItalicFont=cmunbi.ttf]{cmunrm.ttf}\setmonofont[Path=/usr/share/fonts/truetype/cmu/,UprightFont=cmuntt.ttf,BoldFont=cmuntb.ttf,ItalicFont=cmunit.ttf,BoldItalicFont=cmuntx.ttf]{cmunrm.ttf},{\ttfamily \setmainfont[Path=/usr/share/fonts/truetype/cmu/,UprightFont=cmunrm.ttf,BoldFont=cmunbx.ttf,ItalicFont=cmunti.ttf,BoldItalicFont=cmunbi.ttf]{cmuntt.ttf}\setmonofont[Path=/usr/share/fonts/truetype/cmu/,UprightFont=cmuntt.ttf,BoldFont=cmuntb.ttf,ItalicFont=cmunit.ttf,BoldItalicFont=cmuntx.ttf]{cmuntt.ttf}\ttfamily hyperref}\setmainfont[Path=/usr/share/fonts/truetype/cmu/,UprightFont=cmunrm.ttf,BoldFont=cmunbx.ttf,ItalicFont=cmunti.ttf,BoldItalicFont=cmunbi.ttf]{cmunrm.ttf}\setmonofont[Path=/usr/share/fonts/truetype/cmu/,UprightFont=cmuntt.ttf,BoldFont=cmuntb.ttf,ItalicFont=cmunit.ttf,BoldItalicFont=cmuntx.ttf]{cmunrm.ttf}, and {\ttfamily \setmainfont[Path=/usr/share/fonts/truetype/cmu/,UprightFont=cmunrm.ttf,BoldFont=cmunbx.ttf,ItalicFont=cmunti.ttf,BoldItalicFont=cmunbi.ttf]{cmuntt.ttf}\setmonofont[Path=/usr/share/fonts/truetype/cmu/,UprightFont=cmuntt.ttf,BoldFont=cmuntb.ttf,ItalicFont=cmunit.ttf,BoldItalicFont=cmuntx.ttf]{cmuntt.ttf}\ttfamily cleveref}{$\text{ }$}\setmainfont[Path=/usr/share/fonts/truetype/cmu/,UprightFont=cmunrm.ttf,BoldFont=cmunbx.ttf,ItalicFont=cmunti.ttf,BoldItalicFont=cmunbi.ttf]{cmunrm.ttf}\setmonofont[Path=/usr/share/fonts/truetype/cmu/,UprightFont=cmuntt.ttf,BoldFont=cmuntb.ttf,ItalicFont=cmunit.ttf,BoldItalicFont=cmuntx.ttf]{cmunrm.ttf} redefine the same commands, they can produce unexpected results when their {\ttfamily \setmainfont[Path=/usr/share/fonts/truetype/cmu/,UprightFont=cmunrm.ttf,BoldFont=cmunbx.ttf,ItalicFont=cmunti.ttf,BoldItalicFont=cmunbi.ttf]{cmuntt.ttf}\setmonofont[Path=/usr/share/fonts/truetype/cmu/,UprightFont=cmuntt.ttf,BoldFont=cmuntb.ttf,ItalicFont=cmunit.ttf,BoldItalicFont=cmuntx.ttf]{cmuntt.ttf}\ttfamily \textbackslash{}usepackage}{$\text{ }$}\setmainfont[Path=/usr/share/fonts/truetype/cmu/,UprightFont=cmunrm.ttf,BoldFont=cmunbx.ttf,ItalicFont=cmunti.ttf,BoldItalicFont=cmunbi.ttf]{cmunrm.ttf}\setmonofont[Path=/usr/share/fonts/truetype/cmu/,UprightFont=cmuntt.ttf,BoldFont=cmuntb.ttf,ItalicFont=cmunit.ttf,BoldItalicFont=cmuntx.ttf]{cmunrm.ttf} commands appear in the preamble in the wrong order.  For example, using {\ttfamily \setmainfont[Path=/usr/share/fonts/truetype/cmu/,UprightFont=cmunrm.ttf,BoldFont=cmunbx.ttf,ItalicFont=cmunti.ttf,BoldItalicFont=cmunbi.ttf]{cmuntt.ttf}\setmonofont[Path=/usr/share/fonts/truetype/cmu/,UprightFont=cmuntt.ttf,BoldFont=cmuntb.ttf,ItalicFont=cmunit.ttf,BoldItalicFont=cmuntx.ttf]{cmuntt.ttf}\ttfamily hyperref}\setmainfont[Path=/usr/share/fonts/truetype/cmu/,UprightFont=cmunrm.ttf,BoldFont=cmunbx.ttf,ItalicFont=cmunti.ttf,BoldItalicFont=cmunbi.ttf]{cmunrm.ttf}\setmonofont[Path=/usr/share/fonts/truetype/cmu/,UprightFont=cmuntt.ttf,BoldFont=cmuntb.ttf,ItalicFont=cmunit.ttf,BoldItalicFont=cmuntx.ttf]{cmunrm.ttf},{\ttfamily \setmainfont[Path=/usr/share/fonts/truetype/cmu/,UprightFont=cmunrm.ttf,BoldFont=cmunbx.ttf,ItalicFont=cmunti.ttf,BoldItalicFont=cmunbi.ttf]{cmuntt.ttf}\setmonofont[Path=/usr/share/fonts/truetype/cmu/,UprightFont=cmuntt.ttf,BoldFont=cmuntb.ttf,ItalicFont=cmunit.ttf,BoldItalicFont=cmuntx.ttf]{cmuntt.ttf}\ttfamily varioref}\setmainfont[Path=/usr/share/fonts/truetype/cmu/,UprightFont=cmunrm.ttf,BoldFont=cmunbx.ttf,ItalicFont=cmunti.ttf,BoldItalicFont=cmunbi.ttf]{cmunrm.ttf}\setmonofont[Path=/usr/share/fonts/truetype/cmu/,UprightFont=cmuntt.ttf,BoldFont=cmuntb.ttf,ItalicFont=cmunit.ttf,BoldItalicFont=cmuntx.ttf]{cmunrm.ttf}, then {\ttfamily \setmainfont[Path=/usr/share/fonts/truetype/cmu/,UprightFont=cmunrm.ttf,BoldFont=cmunbx.ttf,ItalicFont=cmunti.ttf,BoldItalicFont=cmunbi.ttf]{cmuntt.ttf}\setmonofont[Path=/usr/share/fonts/truetype/cmu/,UprightFont=cmuntt.ttf,BoldFont=cmuntb.ttf,ItalicFont=cmunit.ttf,BoldItalicFont=cmuntx.ttf]{cmuntt.ttf}\ttfamily cleveref}{$\text{ }$}\setmainfont[Path=/usr/share/fonts/truetype/cmu/,UprightFont=cmunrm.ttf,BoldFont=cmunbx.ttf,ItalicFont=cmunti.ttf,BoldItalicFont=cmunbi.ttf]{cmunrm.ttf}\setmonofont[Path=/usr/share/fonts/truetype/cmu/,UprightFont=cmuntt.ttf,BoldFont=cmuntb.ttf,ItalicFont=cmunit.ttf,BoldItalicFont=cmuntx.ttf]{cmunrm.ttf} can cause {\ttfamily \setmainfont[Path=/usr/share/fonts/truetype/cmu/,UprightFont=cmunrm.ttf,BoldFont=cmunbx.ttf,ItalicFont=cmunti.ttf,BoldItalicFont=cmunbi.ttf]{cmuntt.ttf}\setmonofont[Path=/usr/share/fonts/truetype/cmu/,UprightFont=cmuntt.ttf,BoldFont=cmuntb.ttf,ItalicFont=cmunit.ttf,BoldItalicFont=cmuntx.ttf]{cmuntt.ttf}\ttfamily \textbackslash{}vref\{\}}{$\text{ }$}\setmainfont[Path=/usr/share/fonts/truetype/cmu/,UprightFont=cmunrm.ttf,BoldFont=cmunbx.ttf,ItalicFont=cmunti.ttf,BoldItalicFont=cmunbi.ttf]{cmunrm.ttf}\setmonofont[Path=/usr/share/fonts/truetype/cmu/,UprightFont=cmuntt.ttf,BoldFont=cmuntb.ttf,ItalicFont=cmunit.ttf,BoldItalicFont=cmuntx.ttf]{cmunrm.ttf} to fail as though the marker were undefined.\myfootnote{Tests done under report class \myplainurl{http://tex.stackexchange.com/questions/139459/vref-and-input-command}}  The following order generally seems to work:
\begin{myenumerate}
\item{}  {\ttfamily \setmainfont[Path=/usr/share/fonts/truetype/cmu/,UprightFont=cmunrm.ttf,BoldFont=cmunbx.ttf,ItalicFont=cmunti.ttf,BoldItalicFont=cmunbi.ttf]{cmuntt.ttf}\setmonofont[Path=/usr/share/fonts/truetype/cmu/,UprightFont=cmuntt.ttf,BoldFont=cmuntb.ttf,ItalicFont=cmunit.ttf,BoldItalicFont=cmuntx.ttf]{cmuntt.ttf}\ttfamily varioref}
\item{} {$\text{ }$}\setmainfont[Path=/usr/share/fonts/truetype/cmu/,UprightFont=cmunrm.ttf,BoldFont=cmunbx.ttf,ItalicFont=cmunti.ttf,BoldItalicFont=cmunbi.ttf]{cmunrm.ttf}\setmonofont[Path=/usr/share/fonts/truetype/cmu/,UprightFont=cmuntt.ttf,BoldFont=cmuntb.ttf,ItalicFont=cmunit.ttf,BoldItalicFont=cmuntx.ttf]{cmunrm.ttf} {\ttfamily \setmainfont[Path=/usr/share/fonts/truetype/cmu/,UprightFont=cmunrm.ttf,BoldFont=cmunbx.ttf,ItalicFont=cmunti.ttf,BoldItalicFont=cmunbi.ttf]{cmuntt.ttf}\setmonofont[Path=/usr/share/fonts/truetype/cmu/,UprightFont=cmuntt.ttf,BoldFont=cmuntb.ttf,ItalicFont=cmunit.ttf,BoldItalicFont=cmuntx.ttf]{cmuntt.ttf}\ttfamily hyperref}
\item{} {$\text{ }$}\setmainfont[Path=/usr/share/fonts/truetype/cmu/,UprightFont=cmunrm.ttf,BoldFont=cmunbx.ttf,ItalicFont=cmunti.ttf,BoldItalicFont=cmunbi.ttf]{cmunrm.ttf}\setmonofont[Path=/usr/share/fonts/truetype/cmu/,UprightFont=cmuntt.ttf,BoldFont=cmuntb.ttf,ItalicFont=cmunit.ttf,BoldItalicFont=cmuntx.ttf]{cmunrm.ttf} {\ttfamily \setmainfont[Path=/usr/share/fonts/truetype/cmu/,UprightFont=cmunrm.ttf,BoldFont=cmunbx.ttf,ItalicFont=cmunti.ttf,BoldItalicFont=cmunbi.ttf]{cmuntt.ttf}\setmonofont[Path=/usr/share/fonts/truetype/cmu/,UprightFont=cmuntt.ttf,BoldFont=cmuntb.ttf,ItalicFont=cmunit.ttf,BoldItalicFont=cmuntx.ttf]{cmuntt.ttf}\ttfamily cleveref}\myfootnote{\setmainfont[Path=/usr/share/fonts/truetype/cmu/,UprightFont=cmunrm.ttf,BoldFont=cmunbx.ttf,ItalicFont=cmunti.ttf,BoldItalicFont=cmunbi.ttf]{cmunrm.ttf}\setmonofont[Path=/usr/share/fonts/truetype/cmu/,UprightFont=cmuntt.ttf,BoldFont=cmuntb.ttf,ItalicFont=cmunit.ttf,BoldItalicFont=cmuntx.ttf]{cmunrm.ttf}Tests done under report class \myplainurl{http://tex.stackexchange.com/questions/139459/vref-and-input-command}}
\end{myenumerate}

\section{See also}
\label{436}

\begin{myitemize}
\item{}  \mylref{643}{LaTeX/Glossary}
\end{myitemize}

\section{Notes and References}
\label{437}
\LaTeXNullTemplate{}


\mypart{Mechanics}\chapter{Errors and Warnings}

\myminitoc
\label{438}

\label{439}


LaTeX describes what it is typesetting while it does it. If it encounters something it doesn\textquotesingle{}t understand or can\textquotesingle{}t do, it will display a message saying what is wrong. It may also display warnings for less serious conditions.

{\itshape \setmainfont[Path=/usr/share/fonts/truetype/cmu/,UprightFont=cmunrm.ttf,BoldFont=cmunbx.ttf,ItalicFont=cmunti.ttf,BoldItalicFont=cmunbi.ttf]{cmunti.ttf}\setmonofont[Path=/usr/share/fonts/truetype/cmu/,UprightFont=cmuntt.ttf,BoldFont=cmuntb.ttf,ItalicFont=cmunit.ttf,BoldItalicFont=cmuntx.ttf]{cmunti.ttf}\itshape Don\textquotesingle{}t panic if you see error messages}\setmainfont[Path=/usr/share/fonts/truetype/cmu/,UprightFont=cmunrm.ttf,BoldFont=cmunbx.ttf,ItalicFont=cmunti.ttf,BoldItalicFont=cmunbi.ttf]{cmunrm.ttf}\setmonofont[Path=/usr/share/fonts/truetype/cmu/,UprightFont=cmuntt.ttf,BoldFont=cmuntb.ttf,ItalicFont=cmunit.ttf,BoldItalicFont=cmuntx.ttf]{cmunrm.ttf}: it is very common to mistype or misspell commands, forget curly braces, type a forward slash instead of a backslash, or use a special character by mistake. Errors are easily spotted and easily corrected in your editor, and you can then run LaTeX again to check you have fixed everything. Some of the most common errors are described in next sections.
\section{Error messages}
\label{440}
The format of an error message is always the same. Error messages begin with an exclamation mark at the start of the line, and give a description of the error, followed by another line starting with the number, which refers to the line-{}number in your document file which LaTeX was processing when the error was spotted. Here\textquotesingle{}s an example, showing that the user mistyped the {\ttfamily \setmainfont[Path=/usr/share/fonts/truetype/cmu/,UprightFont=cmunrm.ttf,BoldFont=cmunbx.ttf,ItalicFont=cmunti.ttf,BoldItalicFont=cmunbi.ttf]{cmuntt.ttf}\setmonofont[Path=/usr/share/fonts/truetype/cmu/,UprightFont=cmuntt.ttf,BoldFont=cmuntb.ttf,ItalicFont=cmunit.ttf,BoldItalicFont=cmuntx.ttf]{cmuntt.ttf}\ttfamily \textbackslash{}tableofcontents}\setmainfont[Path=/usr/share/fonts/truetype/cmu/,UprightFont=cmunrm.ttf,BoldFont=cmunbx.ttf,ItalicFont=cmunti.ttf,BoldItalicFont=cmunbi.ttf]{cmunrm.ttf}\setmonofont[Path=/usr/share/fonts/truetype/cmu/,UprightFont=cmuntt.ttf,BoldFont=cmuntb.ttf,ItalicFont=cmunit.ttf,BoldItalicFont=cmuntx.ttf]{cmunrm.ttf}
command:

\TemplatePreformat{$\text{ }$\newline{}
!$\text{ }${}Undefined$\text{ }${}control$\text{ }${}sequence.$\text{ }$\newline{}
l.6$\text{ }${}\textbackslash{}tableofcotnetns$\text{ }$\newline{}
}

When LaTeX finds an error like this, it displays the error message and pauses. You must type one of the following letters to
continue:

\begin{longtable}{|>{\RaggedRight}p{0.06675\linewidth}|>{\RaggedRight}p{0.87611\linewidth}|} \hline 
{\bfseries \hspace*{0pt}\ignorespaces{}\hspace*{0pt}Key }&{\bfseries \hspace*{0pt}\ignorespaces{}\hspace*{0pt}Meaning}\endhead  \hline \hspace*{0pt}\ignorespaces{}\hspace*{0pt}x&\hspace*{0pt}\ignorespaces{}\hspace*{0pt}Stop immediately and e{\bfseries \setmainfont[Path=/usr/share/fonts/truetype/cmu/,UprightFont=cmunrm.ttf,BoldFont=cmunbx.ttf,ItalicFont=cmunti.ttf,BoldItalicFont=cmunbi.ttf]{cmunbx.ttf}\setmonofont[Path=/usr/share/fonts/truetype/cmu/,UprightFont=cmuntt.ttf,BoldFont=cmuntb.ttf,ItalicFont=cmunit.ttf,BoldItalicFont=cmuntx.ttf]{cmunbx.ttf}\bfseries x}\setmainfont[Path=/usr/share/fonts/truetype/cmu/,UprightFont=cmunrm.ttf,BoldFont=cmunbx.ttf,ItalicFont=cmunti.ttf,BoldItalicFont=cmunbi.ttf]{cmunrm.ttf}\setmonofont[Path=/usr/share/fonts/truetype/cmu/,UprightFont=cmuntt.ttf,BoldFont=cmuntb.ttf,ItalicFont=cmunit.ttf,BoldItalicFont=cmuntx.ttf]{cmunrm.ttf}it the program.\\ \hline \hspace*{0pt}\ignorespaces{}\hspace*{0pt}q&\hspace*{0pt}\ignorespaces{}\hspace*{0pt}Carry on {\bfseries \setmainfont[Path=/usr/share/fonts/truetype/cmu/,UprightFont=cmunrm.ttf,BoldFont=cmunbx.ttf,ItalicFont=cmunti.ttf,BoldItalicFont=cmunbi.ttf]{cmunbx.ttf}\setmonofont[Path=/usr/share/fonts/truetype/cmu/,UprightFont=cmuntt.ttf,BoldFont=cmuntb.ttf,ItalicFont=cmunit.ttf,BoldItalicFont=cmuntx.ttf]{cmunbx.ttf}\bfseries q}\setmainfont[Path=/usr/share/fonts/truetype/cmu/,UprightFont=cmunrm.ttf,BoldFont=cmunbx.ttf,ItalicFont=cmunti.ttf,BoldItalicFont=cmunbi.ttf]{cmunrm.ttf}\setmonofont[Path=/usr/share/fonts/truetype/cmu/,UprightFont=cmuntt.ttf,BoldFont=cmuntb.ttf,ItalicFont=cmunit.ttf,BoldItalicFont=cmuntx.ttf]{cmunrm.ttf}uietly as best you can and don\textquotesingle{}t bother me with any more error messages.\\ \hline \hspace*{0pt}\ignorespaces{}\hspace*{0pt}e&\hspace*{0pt}\ignorespaces{}\hspace*{0pt}Stop the program but re-{}position the text in my {\bfseries \setmainfont[Path=/usr/share/fonts/truetype/cmu/,UprightFont=cmunrm.ttf,BoldFont=cmunbx.ttf,ItalicFont=cmunti.ttf,BoldItalicFont=cmunbi.ttf]{cmunbx.ttf}\setmonofont[Path=/usr/share/fonts/truetype/cmu/,UprightFont=cmuntt.ttf,BoldFont=cmuntb.ttf,ItalicFont=cmunit.ttf,BoldItalicFont=cmuntx.ttf]{cmunbx.ttf}\bfseries e}\setmainfont[Path=/usr/share/fonts/truetype/cmu/,UprightFont=cmunrm.ttf,BoldFont=cmunbx.ttf,ItalicFont=cmunti.ttf,BoldItalicFont=cmunbi.ttf]{cmunrm.ttf}\setmonofont[Path=/usr/share/fonts/truetype/cmu/,UprightFont=cmuntt.ttf,BoldFont=cmuntb.ttf,ItalicFont=cmunit.ttf,BoldItalicFont=cmuntx.ttf]{cmunrm.ttf}ditor at the point where you found the error (This only works if you\textquotesingle{}re using an editor which LaTeX can communicate with).\\ \hline \hspace*{0pt}\ignorespaces{}\hspace*{0pt}h&\hspace*{0pt}\ignorespaces{}\hspace*{0pt}Try to give me more {\bfseries \setmainfont[Path=/usr/share/fonts/truetype/cmu/,UprightFont=cmunrm.ttf,BoldFont=cmunbx.ttf,ItalicFont=cmunti.ttf,BoldItalicFont=cmunbi.ttf]{cmunbx.ttf}\setmonofont[Path=/usr/share/fonts/truetype/cmu/,UprightFont=cmuntt.ttf,BoldFont=cmuntb.ttf,ItalicFont=cmunit.ttf,BoldItalicFont=cmuntx.ttf]{cmunbx.ttf}\bfseries h}\setmainfont[Path=/usr/share/fonts/truetype/cmu/,UprightFont=cmunrm.ttf,BoldFont=cmunbx.ttf,ItalicFont=cmunti.ttf,BoldItalicFont=cmunbi.ttf]{cmunrm.ttf}\setmonofont[Path=/usr/share/fonts/truetype/cmu/,UprightFont=cmuntt.ttf,BoldFont=cmuntb.ttf,ItalicFont=cmunit.ttf,BoldItalicFont=cmuntx.ttf]{cmunrm.ttf}elp.\\ \hline \hspace*{0pt}\ignorespaces{}\hspace*{0pt}i&\hspace*{0pt}\ignorespaces{}\hspace*{0pt}(followed by a correction) means {\bfseries \setmainfont[Path=/usr/share/fonts/truetype/cmu/,UprightFont=cmunrm.ttf,BoldFont=cmunbx.ttf,ItalicFont=cmunti.ttf,BoldItalicFont=cmunbi.ttf]{cmunbx.ttf}\setmonofont[Path=/usr/share/fonts/truetype/cmu/,UprightFont=cmuntt.ttf,BoldFont=cmuntb.ttf,ItalicFont=cmunit.ttf,BoldItalicFont=cmuntx.ttf]{cmunbx.ttf}\bfseries i}\setmainfont[Path=/usr/share/fonts/truetype/cmu/,UprightFont=cmunrm.ttf,BoldFont=cmunbx.ttf,ItalicFont=cmunti.ttf,BoldItalicFont=cmunbi.ttf]{cmunrm.ttf}\setmonofont[Path=/usr/share/fonts/truetype/cmu/,UprightFont=cmuntt.ttf,BoldFont=cmuntb.ttf,ItalicFont=cmunit.ttf,BoldItalicFont=cmuntx.ttf]{cmunrm.ttf}nput the correction in place of the error and carry on (This is only a temporary fix to get the file processed. You still have to make that correction in the editor).\\ \hline \hspace*{0pt}\ignorespaces{}\hspace*{0pt}r&\hspace*{0pt}\ignorespaces{}\hspace*{0pt}{\bfseries \setmainfont[Path=/usr/share/fonts/truetype/cmu/,UprightFont=cmunrm.ttf,BoldFont=cmunbx.ttf,ItalicFont=cmunti.ttf,BoldItalicFont=cmunbi.ttf]{cmunbx.ttf}\setmonofont[Path=/usr/share/fonts/truetype/cmu/,UprightFont=cmuntt.ttf,BoldFont=cmuntb.ttf,ItalicFont=cmunit.ttf,BoldItalicFont=cmuntx.ttf]{cmunbx.ttf}\bfseries r}\setmainfont[Path=/usr/share/fonts/truetype/cmu/,UprightFont=cmunrm.ttf,BoldFont=cmunbx.ttf,ItalicFont=cmunti.ttf,BoldItalicFont=cmunbi.ttf]{cmunrm.ttf}\setmonofont[Path=/usr/share/fonts/truetype/cmu/,UprightFont=cmuntt.ttf,BoldFont=cmuntb.ttf,ItalicFont=cmunit.ttf,BoldItalicFont=cmuntx.ttf]{cmunrm.ttf}un in non-{}stop mode. Plow through any errors, unless too many pile up and it fails (100 errors).\\ \hline 
\end{longtable}


Some systems (Emacs is one example) run LaTeX with a \symbol{34}nonstop\symbol{34} switch turned on, so it will always process through to the end of the file, regardless of errors, or until a limit is reached.
\section{Warnings}
\label{441}
Warnings don\textquotesingle{}t begin with an exclamation mark: they are just comments by LaTeX about things you might want to look into, such as overlong or underrun lines (often caused by unusual hyphenations, for example), pages running short or long, and other typographical niceties (most of which you can ignore until later).
Unlike other systems, which try to hide unevennesses in the text (usually unsuccessfully) by interfering with the letter spacing, LaTeX takes the view that the author or editor should be able to contribute. While it is certainly possible to set LaTeX\textquotesingle{}s parameters so that the spacing is sufficiently sloppy that you will almost never get a warning about badly-{}fitting lines or pages, you will almost certainly just be delaying matters until you start to get complaints from your readers or publishers.
\section{Examples}
\label{442}

Only a few common error messages are given here: those most likely to be encountered by beginners. If you find another error message not shown here, and it\textquotesingle{}s not clear what you should do, ask for help.

Most error messages are self-{}explanatory, but be aware that the place where LaTeX spots and reports an error may be later in the file than the place where it actually occurred. For example if you forget to close a curly brace which encloses, say, italics, LaTeX won\textquotesingle{}t report this until something else occurs which can\textquotesingle{}t happen until the curly brace is encountered (e.g. the end of the document!) Some errors can only be righted by humans who can read and understand what the document is supposed to mean or look like.

Newcomers should remember to check the list of special characters: a very large number of errors when you are learning LaTeX are due to accidentally typing a special character when you didn\textquotesingle{}t mean to. This disappears after a few days as you get used to them.
\subsection{Too many \}\textquotesingle{}s}
\label{443}

\TemplatePreformat{${\text{ }}${}${\text{ }}${}${\text{ }}${}${\text{ }}${}$\text{ }$\newline{}
!$\text{ }${}Too$\text{ }${}many$\text{ }${}\}\textquotesingle{}s.$\text{ }$\newline{}
l.6$\text{ }${}\textbackslash{}date$\text{ }${}December$\text{ }${}2004\}$\text{ }$\newline{}
}

The reason LaTeX thinks there are too many \}\textquotesingle{}s here is that the opening curly brace is missing after the {\ttfamily \setmainfont[Path=/usr/share/fonts/truetype/cmu/,UprightFont=cmunrm.ttf,BoldFont=cmunbx.ttf,ItalicFont=cmunti.ttf,BoldItalicFont=cmunbi.ttf]{cmuntt.ttf}\setmonofont[Path=/usr/share/fonts/truetype/cmu/,UprightFont=cmuntt.ttf,BoldFont=cmuntb.ttf,ItalicFont=cmunit.ttf,BoldItalicFont=cmuntx.ttf]{cmuntt.ttf}\ttfamily \textbackslash{}date}{$\text{ }$}\setmainfont[Path=/usr/share/fonts/truetype/cmu/,UprightFont=cmunrm.ttf,BoldFont=cmunbx.ttf,ItalicFont=cmunti.ttf,BoldItalicFont=cmunbi.ttf]{cmunrm.ttf}\setmonofont[Path=/usr/share/fonts/truetype/cmu/,UprightFont=cmuntt.ttf,BoldFont=cmuntb.ttf,ItalicFont=cmunit.ttf,BoldItalicFont=cmuntx.ttf]{cmunrm.ttf} control sequence and before the word December, so the closing curly brace is seen as one too many (which it is!). In fact, there are other things which can follow the {\ttfamily \setmainfont[Path=/usr/share/fonts/truetype/cmu/,UprightFont=cmunrm.ttf,BoldFont=cmunbx.ttf,ItalicFont=cmunti.ttf,BoldItalicFont=cmunbi.ttf]{cmuntt.ttf}\setmonofont[Path=/usr/share/fonts/truetype/cmu/,UprightFont=cmuntt.ttf,BoldFont=cmuntb.ttf,ItalicFont=cmunit.ttf,BoldItalicFont=cmuntx.ttf]{cmuntt.ttf}\ttfamily \textbackslash{}date}{$\text{ }$}\setmainfont[Path=/usr/share/fonts/truetype/cmu/,UprightFont=cmunrm.ttf,BoldFont=cmunbx.ttf,ItalicFont=cmunti.ttf,BoldItalicFont=cmunbi.ttf]{cmunrm.ttf}\setmonofont[Path=/usr/share/fonts/truetype/cmu/,UprightFont=cmuntt.ttf,BoldFont=cmuntb.ttf,ItalicFont=cmunit.ttf,BoldItalicFont=cmuntx.ttf]{cmunrm.ttf} command apart from a date in curly braces, so LaTeX cannot possibly guess that you\textquotesingle{}ve missed out the opening curly brace until it finds a closing one!
\subsection{Undefined control sequence}
\label{444}

\TemplatePreformat{$\text{ }$\newline{}
!$\text{ }${}Undefined$\text{ }${}control$\text{ }${}sequence.$\text{ }$\newline{}
l.6$\text{ }${}\textbackslash{}dtae$\text{ }$\newline{}
\{December$\text{ }${}2004\}$\text{ }$\newline{}
}

In this example, LaTeX is complaining that it has no such command (\symbol{34}control sequence\symbol{34}) as {\ttfamily \setmainfont[Path=/usr/share/fonts/truetype/cmu/,UprightFont=cmunrm.ttf,BoldFont=cmunbx.ttf,ItalicFont=cmunti.ttf,BoldItalicFont=cmunbi.ttf]{cmuntt.ttf}\setmonofont[Path=/usr/share/fonts/truetype/cmu/,UprightFont=cmuntt.ttf,BoldFont=cmuntb.ttf,ItalicFont=cmunit.ttf,BoldItalicFont=cmuntx.ttf]{cmuntt.ttf}\ttfamily \textbackslash{}dtae}\setmainfont[Path=/usr/share/fonts/truetype/cmu/,UprightFont=cmunrm.ttf,BoldFont=cmunbx.ttf,ItalicFont=cmunti.ttf,BoldItalicFont=cmunbi.ttf]{cmunrm.ttf}\setmonofont[Path=/usr/share/fonts/truetype/cmu/,UprightFont=cmuntt.ttf,BoldFont=cmuntb.ttf,ItalicFont=cmunit.ttf,BoldItalicFont=cmuntx.ttf]{cmunrm.ttf}. Obviously it\textquotesingle{}s been mistyped, but only a human can detect that fact: all LaTeX knows is that {\ttfamily \setmainfont[Path=/usr/share/fonts/truetype/cmu/,UprightFont=cmunrm.ttf,BoldFont=cmunbx.ttf,ItalicFont=cmunti.ttf,BoldItalicFont=cmunbi.ttf]{cmuntt.ttf}\setmonofont[Path=/usr/share/fonts/truetype/cmu/,UprightFont=cmuntt.ttf,BoldFont=cmuntb.ttf,ItalicFont=cmunit.ttf,BoldItalicFont=cmuntx.ttf]{cmuntt.ttf}\ttfamily \textbackslash{}dtae}{$\text{ }$}\setmainfont[Path=/usr/share/fonts/truetype/cmu/,UprightFont=cmunrm.ttf,BoldFont=cmunbx.ttf,ItalicFont=cmunti.ttf,BoldItalicFont=cmunbi.ttf]{cmunrm.ttf}\setmonofont[Path=/usr/share/fonts/truetype/cmu/,UprightFont=cmuntt.ttf,BoldFont=cmuntb.ttf,ItalicFont=cmunit.ttf,BoldItalicFont=cmuntx.ttf]{cmunrm.ttf} is not a command it knows about: it\textquotesingle{}s undefined. Mistypings are the most common source of errors. Some editors allow common commands and environments to be inserted using drop-{}down menus or icons, which may be used to avoid these errors.
\subsection{Not in Mathematics Mode}
\label{445}

\TemplatePreformat{$\text{ }$\newline{}
!$\text{ }${}Missing$\text{ }${}\${}$\text{ }${}inserted$\text{ }$\newline{}
}

A character that can only be used in the mathematics was inserted in normal text.
If you intended to use mathematics mode, then use {\ttfamily \setmainfont[Path=/usr/share/fonts/truetype/cmu/,UprightFont=cmunrm.ttf,BoldFont=cmunbx.ttf,ItalicFont=cmunti.ttf,BoldItalicFont=cmunbi.ttf]{cmuntt.ttf}\setmonofont[Path=/usr/share/fonts/truetype/cmu/,UprightFont=cmuntt.ttf,BoldFont=cmuntb.ttf,ItalicFont=cmunit.ttf,BoldItalicFont=cmuntx.ttf]{cmuntt.ttf}\ttfamily \${}...\${}}{$\text{ }$}\setmainfont[Path=/usr/share/fonts/truetype/cmu/,UprightFont=cmunrm.ttf,BoldFont=cmunbx.ttf,ItalicFont=cmunti.ttf,BoldItalicFont=cmunbi.ttf]{cmunrm.ttf}\setmonofont[Path=/usr/share/fonts/truetype/cmu/,UprightFont=cmuntt.ttf,BoldFont=cmuntb.ttf,ItalicFont=cmunit.ttf,BoldItalicFont=cmuntx.ttf]{cmunrm.ttf} or {\ttfamily \setmainfont[Path=/usr/share/fonts/truetype/cmu/,UprightFont=cmunrm.ttf,BoldFont=cmunbx.ttf,ItalicFont=cmunti.ttf,BoldItalicFont=cmunbi.ttf]{cmuntt.ttf}\setmonofont[Path=/usr/share/fonts/truetype/cmu/,UprightFont=cmuntt.ttf,BoldFont=cmuntb.ttf,ItalicFont=cmunit.ttf,BoldItalicFont=cmuntx.ttf]{cmuntt.ttf}\ttfamily \textbackslash{}begin\{math\}...\textbackslash{}end\{math\}}{$\text{ }$}\setmainfont[Path=/usr/share/fonts/truetype/cmu/,UprightFont=cmunrm.ttf,BoldFont=cmunbx.ttf,ItalicFont=cmunti.ttf,BoldItalicFont=cmunbi.ttf]{cmunrm.ttf}\setmonofont[Path=/usr/share/fonts/truetype/cmu/,UprightFont=cmuntt.ttf,BoldFont=cmuntb.ttf,ItalicFont=cmunit.ttf,BoldItalicFont=cmuntx.ttf]{cmunrm.ttf} or use the \textquotesingle{}quick math mode\textquotesingle{}: {\ttfamily \setmainfont[Path=/usr/share/fonts/truetype/cmu/,UprightFont=cmunrm.ttf,BoldFont=cmunbx.ttf,ItalicFont=cmunti.ttf,BoldItalicFont=cmunbi.ttf]{cmuntt.ttf}\setmonofont[Path=/usr/share/fonts/truetype/cmu/,UprightFont=cmuntt.ttf,BoldFont=cmuntb.ttf,ItalicFont=cmunit.ttf,BoldItalicFont=cmuntx.ttf]{cmuntt.ttf}\ttfamily \textbackslash{}ensuremath\{...\}}\setmainfont[Path=/usr/share/fonts/truetype/cmu/,UprightFont=cmunrm.ttf,BoldFont=cmunbx.ttf,ItalicFont=cmunti.ttf,BoldItalicFont=cmunbi.ttf]{cmunrm.ttf}\setmonofont[Path=/usr/share/fonts/truetype/cmu/,UprightFont=cmuntt.ttf,BoldFont=cmuntb.ttf,ItalicFont=cmunit.ttf,BoldItalicFont=cmuntx.ttf]{cmunrm.ttf}.
If you did not intend to use mathematics mode, then perhaps you are trying to use a \mylref{87}{special character} that needs to be entered in a different way; for example {\ttfamily \setmainfont[Path=/usr/share/fonts/truetype/cmu/,UprightFont=cmunrm.ttf,BoldFont=cmunbx.ttf,ItalicFont=cmunti.ttf,BoldItalicFont=cmunbi.ttf]{cmuntt.ttf}\setmonofont[Path=/usr/share/fonts/truetype/cmu/,UprightFont=cmuntt.ttf,BoldFont=cmuntb.ttf,ItalicFont=cmunit.ttf,BoldItalicFont=cmuntx.ttf]{cmuntt.ttf}\ttfamily \_}{$\text{ }$}\setmainfont[Path=/usr/share/fonts/truetype/cmu/,UprightFont=cmunrm.ttf,BoldFont=cmunbx.ttf,ItalicFont=cmunti.ttf,BoldItalicFont=cmunbi.ttf]{cmunrm.ttf}\setmonofont[Path=/usr/share/fonts/truetype/cmu/,UprightFont=cmuntt.ttf,BoldFont=cmuntb.ttf,ItalicFont=cmunit.ttf,BoldItalicFont=cmuntx.ttf]{cmunrm.ttf} will be interpreted as a subscript operator in mathematics mode, and you need {\ttfamily \setmainfont[Path=/usr/share/fonts/truetype/cmu/,UprightFont=cmunrm.ttf,BoldFont=cmunbx.ttf,ItalicFont=cmunti.ttf,BoldItalicFont=cmunbi.ttf]{cmuntt.ttf}\setmonofont[Path=/usr/share/fonts/truetype/cmu/,UprightFont=cmuntt.ttf,BoldFont=cmuntb.ttf,ItalicFont=cmunit.ttf,BoldItalicFont=cmuntx.ttf]{cmuntt.ttf}\ttfamily \textbackslash{}\_}{$\text{ }$}\setmainfont[Path=/usr/share/fonts/truetype/cmu/,UprightFont=cmunrm.ttf,BoldFont=cmunbx.ttf,ItalicFont=cmunti.ttf,BoldItalicFont=cmunbi.ttf]{cmunrm.ttf}\setmonofont[Path=/usr/share/fonts/truetype/cmu/,UprightFont=cmuntt.ttf,BoldFont=cmuntb.ttf,ItalicFont=cmunit.ttf,BoldItalicFont=cmuntx.ttf]{cmunrm.ttf} to get an underscore character.

This can also happen if you use the wrong character encoding, for example using utf8 without \symbol{34}\textbackslash{}usepackage{$\text{[}$}utf8{$\text{]}$}\{inputenc\}\symbol{34} or using iso8859-{}1 without \symbol{34}\textbackslash{}usepackage{$\text{[}$}latin1{$\text{]}$}\{inputenc\}\symbol{34}, there are several character encoding formats, make sure to pick the right one.
\subsection{Runaway argument}
\label{446}

\TemplatePreformat{$\text{ }$\newline{}
Runaway$\text{ }${}argument?$\text{ }$\newline{}
\{December$\text{ }${}2004$\text{ }${}\textbackslash{}maketitle$\text{ }$\newline{}
!$\text{ }${}Paragraph$\text{ }${}ended$\text{ }${}before$\text{ }${}\textbackslash{}date$\text{ }${}was$\text{ }${}complete.$\text{ }$\newline{}
<{}to$\text{ }${}be$\text{ }${}read$\text{ }${}again>{}$\text{ }$\newline{}
\textbackslash{}par$\text{ }$\newline{}
l.8$\text{ }$\newline{}
}

In this error, the closing curly brace has been omitted from the date. It\textquotesingle{}s the opposite of the error of too many \}\textquotesingle{}s, and it results in {\ttfamily \setmainfont[Path=/usr/share/fonts/truetype/cmu/,UprightFont=cmunrm.ttf,BoldFont=cmunbx.ttf,ItalicFont=cmunti.ttf,BoldItalicFont=cmunbi.ttf]{cmuntt.ttf}\setmonofont[Path=/usr/share/fonts/truetype/cmu/,UprightFont=cmuntt.ttf,BoldFont=cmuntb.ttf,ItalicFont=cmunit.ttf,BoldItalicFont=cmuntx.ttf]{cmuntt.ttf}\ttfamily \textbackslash{}maketitle}{$\text{ }$}\setmainfont[Path=/usr/share/fonts/truetype/cmu/,UprightFont=cmunrm.ttf,BoldFont=cmunbx.ttf,ItalicFont=cmunti.ttf,BoldItalicFont=cmunbi.ttf]{cmunrm.ttf}\setmonofont[Path=/usr/share/fonts/truetype/cmu/,UprightFont=cmuntt.ttf,BoldFont=cmuntb.ttf,ItalicFont=cmunit.ttf,BoldItalicFont=cmuntx.ttf]{cmunrm.ttf} trying to format the title page while LaTeX is still expecting more text for the date! As \textbackslash{}maketitle creates new paragraphs on the title page, this is detected and LaTeX complains that the previous paragraph has ended but \textbackslash{}date is not yet finished.
\subsection{Underfull hbox}
\label{447}

\TemplatePreformat{${\text{ }}${}${\text{ }}${}${\text{ }}${}${\text{ }}${}$\text{ }$\newline{}
Underfull$\text{ }${}\textbackslash{}hbox$\text{ }${}(badness$\text{ }${}1394)$\text{ }${}in$\text{ }${}paragraph$\text{ }$\newline{}
at$\text{ }${}lines$\text{ }${}28-{}-{}30$\text{ }$\newline{}
{$\text{[}$}{$\text{]}$}{$\text{[}$}{$\text{]}$}\textbackslash{}LY1/brm/b/n/10$\text{ }${}Bull,$\text{ }${}RJ:$\text{ }${}\textbackslash{}LY1/brm/m/n/10$\text{ }$\newline{}
Ac-{}count-{}ing$\text{ }${}in$\text{ }${}Busi-{}$\text{ }$\newline{}
{$\text{[}$}94{$\text{]}$}$\text{ }$\newline{}
}

This is a warning that LaTeX cannot stretch the line wide enough to fit, without making the spacing bigger than its currently permitted maximum. The badness (0-{}10,000) indicates how severe this is (here you can probably ignore a badness of 1394). It says what lines of your file it was typesetting when it found this, and the number in square brackets is the number of the page onto which the offending line was printed. The codes separated by slashes are the typeface and font style and size used in the line. Ignore them for the moment. 

This comes up if you force a linebreak, e.g., \textbackslash{}\textbackslash{}, and have a return before it. Normally TeX ignores linebreaks, providing full paragraphs to ragged text. In this case it is necessary to pull the linebreak up one line to the end of the previous sentence.

This warning may also appear when inserting images. It can be avoided by using the \textbackslash{}textwidth or possibly \textbackslash{}linewidth options, e.g. \textbackslash{}includegraphics{$\text{[}$}width=\textbackslash{}textwidth{$\text{]}$}\{image\_name\}
\subsection{Overfull hbox}
\label{448}

\TemplatePreformat{$\text{ }$\newline{}
{$\text{[}$}101{$\text{]}$}$\text{ }$\newline{}
Overfull$\text{ }${}\textbackslash{}hbox$\text{ }${}(9.11617pt$\text{ }${}too$\text{ }${}wide)$\text{ }${}in$\text{ }${}paragraph$\text{ }$\newline{}
at$\text{ }${}lines$\text{ }${}860-{}-{}861$\text{ }$\newline{}
{$\text{[}$}{$\text{]}$}\textbackslash{}LY1/brm/m/n/10$\text{ }${}Windows,$\text{ }${}\textbackslash{}LY1/brm/m/it/10$\text{ }${}see$\text{ }$\newline{}
\textbackslash{}LY1/brm/m/n/10$\text{ }${}X$\text{ }${}Win-{}$\text{ }$\newline{}
}

An overfull \textbackslash{}hbox means that there is a hyphenation or justification problem: moving the last word on the line to the next line would make the spaces in the line wider than the current limit; keeping the word on the line would make the spaces smaller than the current limit, so the word is left on the line, but with the minimum allowed space between words, and which makes the line go over the edge.

The warning is given so that you can find the line in the code that originates the problem (in this case: 860-{}861) and fix it. The line on this example is too long by a shade over 9pt. The chosen hyphenation point which minimizes the error is shown at the end of the line (Win-{}). Line numbers and page numbers are given as before. In this case, 9pt is too much to ignore (over 3mm), and a manual correction needs making (such as a change to the hyphenation), or the flexibility settings need changing.

If the \symbol{34}overfull\symbol{34} word includes a forward slash, such as \symbol{34}{\ttfamily \setmainfont[Path=/usr/share/fonts/truetype/cmu/,UprightFont=cmunrm.ttf,BoldFont=cmunbx.ttf,ItalicFont=cmunti.ttf,BoldItalicFont=cmunbi.ttf]{cmuntt.ttf}\setmonofont[Path=/usr/share/fonts/truetype/cmu/,UprightFont=cmuntt.ttf,BoldFont=cmuntb.ttf,ItalicFont=cmunit.ttf,BoldItalicFont=cmuntx.ttf]{cmuntt.ttf}\ttfamily input/output}\setmainfont[Path=/usr/share/fonts/truetype/cmu/,UprightFont=cmunrm.ttf,BoldFont=cmunbx.ttf,ItalicFont=cmunti.ttf,BoldItalicFont=cmunbi.ttf]{cmunrm.ttf}\setmonofont[Path=/usr/share/fonts/truetype/cmu/,UprightFont=cmuntt.ttf,BoldFont=cmuntb.ttf,ItalicFont=cmunit.ttf,BoldItalicFont=cmuntx.ttf]{cmunrm.ttf}\symbol{34}, this should be properly typeset as \symbol{34}{\ttfamily \setmainfont[Path=/usr/share/fonts/truetype/cmu/,UprightFont=cmunrm.ttf,BoldFont=cmunbx.ttf,ItalicFont=cmunti.ttf,BoldItalicFont=cmunbi.ttf]{cmuntt.ttf}\setmonofont[Path=/usr/share/fonts/truetype/cmu/,UprightFont=cmuntt.ttf,BoldFont=cmuntb.ttf,ItalicFont=cmunit.ttf,BoldItalicFont=cmuntx.ttf]{cmuntt.ttf}\ttfamily input\textbackslash{}slash output}\setmainfont[Path=/usr/share/fonts/truetype/cmu/,UprightFont=cmunrm.ttf,BoldFont=cmunbx.ttf,ItalicFont=cmunti.ttf,BoldItalicFont=cmunbi.ttf]{cmunrm.ttf}\setmonofont[Path=/usr/share/fonts/truetype/cmu/,UprightFont=cmuntt.ttf,BoldFont=cmuntb.ttf,ItalicFont=cmunit.ttf,BoldItalicFont=cmuntx.ttf]{cmunrm.ttf}\symbol{34}. The use of {\ttfamily \setmainfont[Path=/usr/share/fonts/truetype/cmu/,UprightFont=cmunrm.ttf,BoldFont=cmunbx.ttf,ItalicFont=cmunti.ttf,BoldItalicFont=cmunbi.ttf]{cmuntt.ttf}\setmonofont[Path=/usr/share/fonts/truetype/cmu/,UprightFont=cmuntt.ttf,BoldFont=cmuntb.ttf,ItalicFont=cmunit.ttf,BoldItalicFont=cmuntx.ttf]{cmuntt.ttf}\ttfamily \textbackslash{}slash}{$\text{ }$}\setmainfont[Path=/usr/share/fonts/truetype/cmu/,UprightFont=cmunrm.ttf,BoldFont=cmunbx.ttf,ItalicFont=cmunti.ttf,BoldItalicFont=cmunbi.ttf]{cmunrm.ttf}\setmonofont[Path=/usr/share/fonts/truetype/cmu/,UprightFont=cmuntt.ttf,BoldFont=cmuntb.ttf,ItalicFont=cmunit.ttf,BoldItalicFont=cmuntx.ttf]{cmunrm.ttf} has the same effect as using the \symbol{34}{\ttfamily \setmainfont[Path=/usr/share/fonts/truetype/cmu/,UprightFont=cmunrm.ttf,BoldFont=cmunbx.ttf,ItalicFont=cmunti.ttf,BoldItalicFont=cmunbi.ttf]{cmuntt.ttf}\setmonofont[Path=/usr/share/fonts/truetype/cmu/,UprightFont=cmuntt.ttf,BoldFont=cmuntb.ttf,ItalicFont=cmunit.ttf,BoldItalicFont=cmuntx.ttf]{cmuntt.ttf}\ttfamily /}\setmainfont[Path=/usr/share/fonts/truetype/cmu/,UprightFont=cmunrm.ttf,BoldFont=cmunbx.ttf,ItalicFont=cmunti.ttf,BoldItalicFont=cmunbi.ttf]{cmunrm.ttf}\setmonofont[Path=/usr/share/fonts/truetype/cmu/,UprightFont=cmuntt.ttf,BoldFont=cmuntb.ttf,ItalicFont=cmunit.ttf,BoldItalicFont=cmuntx.ttf]{cmunrm.ttf}\symbol{34} character, except that it can form the end of a line (with the following words appearing at the start of the next line). The \symbol{34}{\ttfamily \setmainfont[Path=/usr/share/fonts/truetype/cmu/,UprightFont=cmunrm.ttf,BoldFont=cmunbx.ttf,ItalicFont=cmunti.ttf,BoldItalicFont=cmunbi.ttf]{cmuntt.ttf}\setmonofont[Path=/usr/share/fonts/truetype/cmu/,UprightFont=cmuntt.ttf,BoldFont=cmuntb.ttf,ItalicFont=cmunit.ttf,BoldItalicFont=cmuntx.ttf]{cmuntt.ttf}\ttfamily /}\setmainfont[Path=/usr/share/fonts/truetype/cmu/,UprightFont=cmunrm.ttf,BoldFont=cmunbx.ttf,ItalicFont=cmunti.ttf,BoldItalicFont=cmunbi.ttf]{cmunrm.ttf}\setmonofont[Path=/usr/share/fonts/truetype/cmu/,UprightFont=cmuntt.ttf,BoldFont=cmuntb.ttf,ItalicFont=cmunit.ttf,BoldItalicFont=cmuntx.ttf]{cmunrm.ttf}\symbol{34} character is typically used in units, such as \symbol{34}{\ttfamily \setmainfont[Path=/usr/share/fonts/truetype/cmu/,UprightFont=cmunrm.ttf,BoldFont=cmunbx.ttf,ItalicFont=cmunti.ttf,BoldItalicFont=cmunbi.ttf]{cmuntt.ttf}\setmonofont[Path=/usr/share/fonts/truetype/cmu/,UprightFont=cmuntt.ttf,BoldFont=cmuntb.ttf,ItalicFont=cmunit.ttf,BoldItalicFont=cmuntx.ttf]{cmuntt.ttf}\ttfamily mm/year}\setmainfont[Path=/usr/share/fonts/truetype/cmu/,UprightFont=cmunrm.ttf,BoldFont=cmunbx.ttf,ItalicFont=cmunti.ttf,BoldItalicFont=cmunbi.ttf]{cmunrm.ttf}\setmonofont[Path=/usr/share/fonts/truetype/cmu/,UprightFont=cmuntt.ttf,BoldFont=cmuntb.ttf,ItalicFont=cmunit.ttf,BoldItalicFont=cmuntx.ttf]{cmunrm.ttf}\symbol{34} character, which should not be broken over multiple lines.

The warning can also be issued when the \textbackslash{}end\{document\} tag was not included or was deleted.
\subsubsection{Easily spotting overfull hboxes in the document}
\label{449}
To easily find the location of overfull hbox in your document, you can make latex add a black bar where a line is too wide:
\begin{Shaded}
\begin{Highlighting}[]

\NormalTok{\textbackslash{}overfullrule=2cm}
\end{Highlighting}
\end{Shaded}

\subsection{Missing package}
\label{450}

\TemplatePreformat{$\text{ }$\newline{}
!$\text{ }${}LaTeX$\text{ }${}Error:$\text{ }${}File$\text{ }${}`paralisy.sty\textquotesingle{}$\text{ }${}not$\text{ }${}found.$\text{ }$\newline{}
Type$\text{ }${}X$\text{ }${}to$\text{ }${}quit$\text{ }${}or$\text{ }${}<{}RETURN>{}$\text{ }${}to$\text{ }${}proceed,$\text{ }$\newline{}
or$\text{ }${}enter$\text{ }${}new$\text{ }${}name.$\text{ }${}(Default$\text{ }${}extension:$\text{ }${}sty)$\text{ }$\newline{}
Enter$\text{ }${}file$\text{ }${}name:$\text{ }$\newline{}
}

When you use the {\ttfamily \setmainfont[Path=/usr/share/fonts/truetype/cmu/,UprightFont=cmunrm.ttf,BoldFont=cmunbx.ttf,ItalicFont=cmunti.ttf,BoldItalicFont=cmunbi.ttf]{cmuntt.ttf}\setmonofont[Path=/usr/share/fonts/truetype/cmu/,UprightFont=cmuntt.ttf,BoldFont=cmuntb.ttf,ItalicFont=cmunit.ttf,BoldItalicFont=cmuntx.ttf]{cmuntt.ttf}\ttfamily \textbackslash{}usepackage}{$\text{ }$}\setmainfont[Path=/usr/share/fonts/truetype/cmu/,UprightFont=cmunrm.ttf,BoldFont=cmunbx.ttf,ItalicFont=cmunti.ttf,BoldItalicFont=cmunbi.ttf]{cmunrm.ttf}\setmonofont[Path=/usr/share/fonts/truetype/cmu/,UprightFont=cmuntt.ttf,BoldFont=cmuntb.ttf,ItalicFont=cmunit.ttf,BoldItalicFont=cmuntx.ttf]{cmunrm.ttf} command to request LaTeX to use a certain package, it will look for a file with the specified name and the filetype {\ttfamily \setmainfont[Path=/usr/share/fonts/truetype/cmu/,UprightFont=cmunrm.ttf,BoldFont=cmunbx.ttf,ItalicFont=cmunti.ttf,BoldItalicFont=cmunbi.ttf]{cmuntt.ttf}\setmonofont[Path=/usr/share/fonts/truetype/cmu/,UprightFont=cmuntt.ttf,BoldFont=cmuntb.ttf,ItalicFont=cmunit.ttf,BoldItalicFont=cmuntx.ttf]{cmuntt.ttf}\ttfamily .sty}\setmainfont[Path=/usr/share/fonts/truetype/cmu/,UprightFont=cmunrm.ttf,BoldFont=cmunbx.ttf,ItalicFont=cmunti.ttf,BoldItalicFont=cmunbi.ttf]{cmunrm.ttf}\setmonofont[Path=/usr/share/fonts/truetype/cmu/,UprightFont=cmuntt.ttf,BoldFont=cmuntb.ttf,ItalicFont=cmunit.ttf,BoldItalicFont=cmuntx.ttf]{cmunrm.ttf}. In this case the user has mistyped the name of the paralist package, so it\textquotesingle{}s easy to fix. However, if you get the name right, but the package is not installed on your machine, you will need to download and install it before continuing. If you don\textquotesingle{}t want to affect the global installation of the machine, you can simply download from Internet the necessary {\ttfamily \setmainfont[Path=/usr/share/fonts/truetype/cmu/,UprightFont=cmunrm.ttf,BoldFont=cmunbx.ttf,ItalicFont=cmunti.ttf,BoldItalicFont=cmunbi.ttf]{cmuntt.ttf}\setmonofont[Path=/usr/share/fonts/truetype/cmu/,UprightFont=cmuntt.ttf,BoldFont=cmuntb.ttf,ItalicFont=cmunit.ttf,BoldItalicFont=cmuntx.ttf]{cmuntt.ttf}\ttfamily .sty}{$\text{ }$}\setmainfont[Path=/usr/share/fonts/truetype/cmu/,UprightFont=cmunrm.ttf,BoldFont=cmunbx.ttf,ItalicFont=cmunti.ttf,BoldItalicFont=cmunbi.ttf]{cmunrm.ttf}\setmonofont[Path=/usr/share/fonts/truetype/cmu/,UprightFont=cmuntt.ttf,BoldFont=cmuntb.ttf,ItalicFont=cmunit.ttf,BoldItalicFont=cmuntx.ttf]{cmunrm.ttf} file and put it in the same folder of the document you are compiling.
\subsection{Package babel Warning: No hyphenation patterns were loaded for the language X}
\label{451}
Although this is a warning from the Babel package and not from LaTeX, this error is very common and (can) give some strange hyphenation (word breaking) problems in your document. Wrong hyphenation rules can decrease the neatness of your document.
\TemplatePreformat{$\text{ }$\newline{}
Package$\text{ }${}babel$\text{ }${}Warning:$\text{ }${}No$\text{ }${}hyphenation$\text{ }${}patterns$\text{ }${}were$\text{ }${}loaded$\text{ }${}for$\text{ }$\newline{}
(babel)$\text{ }${}$\text{ }${}$\text{ }${}$\text{ }${}$\text{ }${}$\text{ }${}$\text{ }${}$\text{ }${}$\text{ }${}$\text{ }${}$\text{ }${}$\text{ }${}$\text{ }${}$\text{ }${}$\text{ }${}$\text{ }${}the$\text{ }${}language$\text{ }${}`Latin\textquotesingle{}$\text{ }$\newline{}
(babel)$\text{ }${}$\text{ }${}$\text{ }${}$\text{ }${}$\text{ }${}$\text{ }${}$\text{ }${}$\text{ }${}$\text{ }${}$\text{ }${}$\text{ }${}$\text{ }${}$\text{ }${}$\text{ }${}$\text{ }${}$\text{ }${}I$\text{ }${}will$\text{ }${}use$\text{ }${}the$\text{ }${}patterns$\text{ }${}loaded$\text{ }${}for$\text{ }${}\textbackslash{}language=0$\text{ }${}instead.$\text{ }$\newline{}
}

This can happen after the usage of: (see \mylref{209}{LaTeX/Internationalization})


\begin{Shaded}
\begin{Highlighting}[]

\NormalTok{\textbackslash{}usepackage[latin]\{babel\}}\newline
\end{Highlighting}
\end{Shaded}


The solution is not difficult, just install the used language in your \mylref{26}{LaTeX distribution}.
\subsection{Package babel Error: You haven\textquotesingle{}t loaded the option X yet.}
\label{452}
If you previously set the X language, and then decided to switch to Y, you will get this error.
This may seem awkward, as there is obviously no error in your code if you did not change anything.
The answer lies in the {\ttfamily \setmainfont[Path=/usr/share/fonts/truetype/cmu/,UprightFont=cmunrm.ttf,BoldFont=cmunbx.ttf,ItalicFont=cmunti.ttf,BoldItalicFont=cmunbi.ttf]{cmuntt.ttf}\setmonofont[Path=/usr/share/fonts/truetype/cmu/,UprightFont=cmuntt.ttf,BoldFont=cmuntb.ttf,ItalicFont=cmunit.ttf,BoldItalicFont=cmuntx.ttf]{cmuntt.ttf}\ttfamily .aux}{$\text{ }$}\setmainfont[Path=/usr/share/fonts/truetype/cmu/,UprightFont=cmunrm.ttf,BoldFont=cmunbx.ttf,ItalicFont=cmunti.ttf,BoldItalicFont=cmunbi.ttf]{cmunrm.ttf}\setmonofont[Path=/usr/share/fonts/truetype/cmu/,UprightFont=cmuntt.ttf,BoldFont=cmuntb.ttf,ItalicFont=cmunit.ttf,BoldItalicFont=cmuntx.ttf]{cmunrm.ttf} file, where babel defined your language.
If you try the compilation a second time, it should work.
If not, delete the {\ttfamily \setmainfont[Path=/usr/share/fonts/truetype/cmu/,UprightFont=cmunrm.ttf,BoldFont=cmunbx.ttf,ItalicFont=cmunti.ttf,BoldItalicFont=cmunbi.ttf]{cmuntt.ttf}\setmonofont[Path=/usr/share/fonts/truetype/cmu/,UprightFont=cmuntt.ttf,BoldFont=cmuntb.ttf,ItalicFont=cmunit.ttf,BoldItalicFont=cmuntx.ttf]{cmuntt.ttf}\ttfamily .aux}{$\text{ }$}\setmainfont[Path=/usr/share/fonts/truetype/cmu/,UprightFont=cmunrm.ttf,BoldFont=cmunbx.ttf,ItalicFont=cmunti.ttf,BoldItalicFont=cmunbi.ttf]{cmunrm.ttf}\setmonofont[Path=/usr/share/fonts/truetype/cmu/,UprightFont=cmuntt.ttf,BoldFont=cmuntb.ttf,ItalicFont=cmunit.ttf,BoldItalicFont=cmuntx.ttf]{cmunrm.ttf} file, then everything will work as usual.
\subsection{No error message, but won\textquotesingle{}t compile}
\label{453}
\LaTeXNullTemplate{}
One common cause of (pdf)LaTeX getting stuck is forgetting to include 
\begin{Shaded}
\begin{Highlighting}[]

\NormalTok{\textbackslash{}end\{document\}}\newline
\end{Highlighting}
\end{Shaded}

\section{Software that can check your .tex Code}
\label{454}
There are several programs capable of checking LaTeX source, with the aim of finding errors or highlighting bad practice, and providing more help to (particularly novice) users than the built-{}in error messages.
\begin{myitemize}
\item{}  nag (\myhref{http://www.ctan.org/tex-archive/macros/latex/contrib/nag}{www.ctan.org/tex-{}archive/macros/latex/contrib/nag}) is a LaTeX package designed to indicate the use of obsolete commands.
\item{}  lacheck (\myhref{http://www.ctan.org/tex-archive/support/lacheck}{www.ctan.org/tex-{}archive/support/lacheck}) is a consistency checker intended to spot mistakes in code.  It is available as source code or compiled for Windows and OS/2
\item{}  chktex (\myhref{http://baruch.ev-en.org/proj/chktex/}{baruch.ev-{}en.org/proj/chktex/}) is a LaTeX semantic checker available as source code for Unix-{}like systems. 
\end{myitemize}


\chapter{Lengths}

\myminitoc
\label{455}

\label{456}


In TeX, a length is
\begin{myitemize}
\item{}  a floating point number followed by a unit, optionally followed by a stretching value;
\end{myitemize}

\begin{Shaded}
\begin{Highlighting}[]

\NormalTok{3.5pt plus 1pt minus 2pt}
\end{Highlighting}
\end{Shaded}

\begin{myitemize}
\item{}  a floating point factor followed by a macro that expands to a length.
\end{myitemize}

\begin{Shaded}
\begin{Highlighting}[]

\NormalTok{1.7\textbackslash{}textwidth}
\end{Highlighting}
\end{Shaded}

\section{Units}
\label{457}
First, we introduce the LaTeX measurement units. All LaTeX units are two-{}letter abbreviations. You can choose from a variety of units. Here are the most common ones.\myfootnote{\myplainurl{http://www.giss.nasa.gov/tools/latex/ltx-86.html}}

\begin{longtable}{|>{\RaggedRight}p{0.15936\linewidth}|>{\RaggedRight}p{0.34842\linewidth}|>{\RaggedRight}p{0.40651\linewidth}|} \hline 
{\bfseries \hspace*{0pt}\ignorespaces{}\hspace*{0pt} Abbreviation}&{\bfseries \hspace*{0pt}\ignorespaces{}\hspace*{0pt} Definition}&{\bfseries \hspace*{0pt}\ignorespaces{}\hspace*{0pt} Value in points (pt)}\endhead  \hline \hspace*{0pt}\ignorespaces{}\hspace*{0pt}{\bfseries \setmainfont[Path=/usr/share/fonts/truetype/cmu/,UprightFont=cmunrm.ttf,BoldFont=cmunbx.ttf,ItalicFont=cmunti.ttf,BoldItalicFont=cmunbi.ttf]{cmunbx.ttf}\setmonofont[Path=/usr/share/fonts/truetype/cmu/,UprightFont=cmuntt.ttf,BoldFont=cmuntb.ttf,ItalicFont=cmunit.ttf,BoldItalicFont=cmuntx.ttf]{cmunbx.ttf}\bfseries pt}&\hspace*{0pt}\ignorespaces{}\hspace*{0pt}\setmainfont[Path=/usr/share/fonts/truetype/cmu/,UprightFont=cmunrm.ttf,BoldFont=cmunbx.ttf,ItalicFont=cmunti.ttf,BoldItalicFont=cmunbi.ttf]{cmunrm.ttf}\setmonofont[Path=/usr/share/fonts/truetype/cmu/,UprightFont=cmuntt.ttf,BoldFont=cmuntb.ttf,ItalicFont=cmunit.ttf,BoldItalicFont=cmuntx.ttf]{cmunrm.ttf}a point is 1/72.27 inch, that means about 0.0138 inch or 0.3515 mm. 1pt is the default length.&\hspace*{0pt}\ignorespaces{}\hspace*{0pt} 1\\ \hline \hspace*{0pt}\ignorespaces{}\hspace*{0pt}{\bfseries \setmainfont[Path=/usr/share/fonts/truetype/cmu/,UprightFont=cmunrm.ttf,BoldFont=cmunbx.ttf,ItalicFont=cmunti.ttf,BoldItalicFont=cmunbi.ttf]{cmunbx.ttf}\setmonofont[Path=/usr/share/fonts/truetype/cmu/,UprightFont=cmuntt.ttf,BoldFont=cmuntb.ttf,ItalicFont=cmunit.ttf,BoldItalicFont=cmuntx.ttf]{cmunbx.ttf}\bfseries mm}&\hspace*{0pt}\ignorespaces{}\hspace*{0pt}\setmainfont[Path=/usr/share/fonts/truetype/cmu/,UprightFont=cmunrm.ttf,BoldFont=cmunbx.ttf,ItalicFont=cmunti.ttf,BoldItalicFont=cmunbi.ttf]{cmunrm.ttf}\setmonofont[Path=/usr/share/fonts/truetype/cmu/,UprightFont=cmuntt.ttf,BoldFont=cmuntb.ttf,ItalicFont=cmunit.ttf,BoldItalicFont=cmuntx.ttf]{cmunrm.ttf}a millimeter&\hspace*{0pt}\ignorespaces{}\hspace*{0pt} 2.84\\ \hline \hspace*{0pt}\ignorespaces{}\hspace*{0pt}{\bfseries \setmainfont[Path=/usr/share/fonts/truetype/cmu/,UprightFont=cmunrm.ttf,BoldFont=cmunbx.ttf,ItalicFont=cmunti.ttf,BoldItalicFont=cmunbi.ttf]{cmunbx.ttf}\setmonofont[Path=/usr/share/fonts/truetype/cmu/,UprightFont=cmuntt.ttf,BoldFont=cmuntb.ttf,ItalicFont=cmunit.ttf,BoldItalicFont=cmuntx.ttf]{cmunbx.ttf}\bfseries cm}&\hspace*{0pt}\ignorespaces{}\hspace*{0pt}\setmainfont[Path=/usr/share/fonts/truetype/cmu/,UprightFont=cmunrm.ttf,BoldFont=cmunbx.ttf,ItalicFont=cmunti.ttf,BoldItalicFont=cmunbi.ttf]{cmunrm.ttf}\setmonofont[Path=/usr/share/fonts/truetype/cmu/,UprightFont=cmuntt.ttf,BoldFont=cmuntb.ttf,ItalicFont=cmunit.ttf,BoldItalicFont=cmuntx.ttf]{cmunrm.ttf}a centimeter&\hspace*{0pt}\ignorespaces{}\hspace*{0pt} 28.4\\ \hline \hspace*{0pt}\ignorespaces{}\hspace*{0pt}{\bfseries \setmainfont[Path=/usr/share/fonts/truetype/cmu/,UprightFont=cmunrm.ttf,BoldFont=cmunbx.ttf,ItalicFont=cmunti.ttf,BoldItalicFont=cmunbi.ttf]{cmunbx.ttf}\setmonofont[Path=/usr/share/fonts/truetype/cmu/,UprightFont=cmuntt.ttf,BoldFont=cmuntb.ttf,ItalicFont=cmunit.ttf,BoldItalicFont=cmuntx.ttf]{cmunbx.ttf}\bfseries in}&\hspace*{0pt}\ignorespaces{}\hspace*{0pt}\setmainfont[Path=/usr/share/fonts/truetype/cmu/,UprightFont=cmunrm.ttf,BoldFont=cmunbx.ttf,ItalicFont=cmunti.ttf,BoldItalicFont=cmunbi.ttf]{cmunrm.ttf}\setmonofont[Path=/usr/share/fonts/truetype/cmu/,UprightFont=cmuntt.ttf,BoldFont=cmuntb.ttf,ItalicFont=cmunit.ttf,BoldItalicFont=cmuntx.ttf]{cmunrm.ttf}inch&\hspace*{0pt}\ignorespaces{}\hspace*{0pt} 72.27\\ \hline \hspace*{0pt}\ignorespaces{}\hspace*{0pt}{\bfseries \setmainfont[Path=/usr/share/fonts/truetype/cmu/,UprightFont=cmunrm.ttf,BoldFont=cmunbx.ttf,ItalicFont=cmunti.ttf,BoldItalicFont=cmunbi.ttf]{cmunbx.ttf}\setmonofont[Path=/usr/share/fonts/truetype/cmu/,UprightFont=cmuntt.ttf,BoldFont=cmuntb.ttf,ItalicFont=cmunit.ttf,BoldItalicFont=cmuntx.ttf]{cmunbx.ttf}\bfseries ex}&\hspace*{0pt}\ignorespaces{}\hspace*{0pt}\setmainfont[Path=/usr/share/fonts/truetype/cmu/,UprightFont=cmunrm.ttf,BoldFont=cmunbx.ttf,ItalicFont=cmunti.ttf,BoldItalicFont=cmunbi.ttf]{cmunrm.ttf}\setmonofont[Path=/usr/share/fonts/truetype/cmu/,UprightFont=cmuntt.ttf,BoldFont=cmuntb.ttf,ItalicFont=cmunit.ttf,BoldItalicFont=cmuntx.ttf]{cmunrm.ttf}roughly the height of an \textquotesingle{}x\textquotesingle{} {\bfseries \setmainfont[Path=/usr/share/fonts/truetype/cmu/,UprightFont=cmunrm.ttf,BoldFont=cmunbx.ttf,ItalicFont=cmunti.ttf,BoldItalicFont=cmunbi.ttf]{cmunbx.ttf}\setmonofont[Path=/usr/share/fonts/truetype/cmu/,UprightFont=cmuntt.ttf,BoldFont=cmuntb.ttf,ItalicFont=cmunit.ttf,BoldItalicFont=cmuntx.ttf]{cmunbx.ttf}\bfseries in the current font}&\hspace*{0pt}\ignorespaces{}\hspace*{0pt}{$\text{ }$}\setmainfont[Path=/usr/share/fonts/truetype/cmu/,UprightFont=cmunrm.ttf,BoldFont=cmunbx.ttf,ItalicFont=cmunti.ttf,BoldItalicFont=cmunbi.ttf]{cmunrm.ttf}\setmonofont[Path=/usr/share/fonts/truetype/cmu/,UprightFont=cmuntt.ttf,BoldFont=cmuntb.ttf,ItalicFont=cmunit.ttf,BoldItalicFont=cmuntx.ttf]{cmunrm.ttf} {\itshape \setmainfont[Path=/usr/share/fonts/truetype/cmu/,UprightFont=cmunrm.ttf,BoldFont=cmunbx.ttf,ItalicFont=cmunti.ttf,BoldItalicFont=cmunbi.ttf]{cmunti.ttf}\setmonofont[Path=/usr/share/fonts/truetype/cmu/,UprightFont=cmuntt.ttf,BoldFont=cmuntb.ttf,ItalicFont=cmunit.ttf,BoldItalicFont=cmuntx.ttf]{cmunti.ttf}\itshape undefined, depends on the font used}\\ \hline \hspace*{0pt}\ignorespaces{}\hspace*{0pt}{\bfseries \setmainfont[Path=/usr/share/fonts/truetype/cmu/,UprightFont=cmunrm.ttf,BoldFont=cmunbx.ttf,ItalicFont=cmunti.ttf,BoldItalicFont=cmunbi.ttf]{cmunbx.ttf}\setmonofont[Path=/usr/share/fonts/truetype/cmu/,UprightFont=cmuntt.ttf,BoldFont=cmuntb.ttf,ItalicFont=cmunit.ttf,BoldItalicFont=cmuntx.ttf]{cmunbx.ttf}\bfseries em}&\hspace*{0pt}\ignorespaces{}\hspace*{0pt}\setmainfont[Path=/usr/share/fonts/truetype/cmu/,UprightFont=cmunrm.ttf,BoldFont=cmunbx.ttf,ItalicFont=cmunti.ttf,BoldItalicFont=cmunbi.ttf]{cmunrm.ttf}\setmonofont[Path=/usr/share/fonts/truetype/cmu/,UprightFont=cmuntt.ttf,BoldFont=cmuntb.ttf,ItalicFont=cmunit.ttf,BoldItalicFont=cmuntx.ttf]{cmunrm.ttf}roughly the width of an \textquotesingle{}M\textquotesingle{} (uppercase) {\bfseries \setmainfont[Path=/usr/share/fonts/truetype/cmu/,UprightFont=cmunrm.ttf,BoldFont=cmunbx.ttf,ItalicFont=cmunti.ttf,BoldItalicFont=cmunbi.ttf]{cmunbx.ttf}\setmonofont[Path=/usr/share/fonts/truetype/cmu/,UprightFont=cmuntt.ttf,BoldFont=cmuntb.ttf,ItalicFont=cmunit.ttf,BoldItalicFont=cmuntx.ttf]{cmunbx.ttf}\bfseries in the current font}&\hspace*{0pt}\ignorespaces{}\hspace*{0pt}{$\text{ }$}\setmainfont[Path=/usr/share/fonts/truetype/cmu/,UprightFont=cmunrm.ttf,BoldFont=cmunbx.ttf,ItalicFont=cmunti.ttf,BoldItalicFont=cmunbi.ttf]{cmunrm.ttf}\setmonofont[Path=/usr/share/fonts/truetype/cmu/,UprightFont=cmuntt.ttf,BoldFont=cmuntb.ttf,ItalicFont=cmunit.ttf,BoldItalicFont=cmuntx.ttf]{cmunrm.ttf} {\itshape \setmainfont[Path=/usr/share/fonts/truetype/cmu/,UprightFont=cmunrm.ttf,BoldFont=cmunbx.ttf,ItalicFont=cmunti.ttf,BoldItalicFont=cmunbi.ttf]{cmunti.ttf}\setmonofont[Path=/usr/share/fonts/truetype/cmu/,UprightFont=cmuntt.ttf,BoldFont=cmuntb.ttf,ItalicFont=cmunit.ttf,BoldItalicFont=cmuntx.ttf]{cmunti.ttf}\itshape undefined, depends on the font used}\\ \hline 
\end{longtable}
\setmainfont[Path=/usr/share/fonts/truetype/cmu/,UprightFont=cmunrm.ttf,BoldFont=cmunbx.ttf,ItalicFont=cmunti.ttf,BoldItalicFont=cmunbi.ttf]{cmunrm.ttf}\setmonofont[Path=/usr/share/fonts/truetype/cmu/,UprightFont=cmuntt.ttf,BoldFont=cmuntb.ttf,ItalicFont=cmunit.ttf,BoldItalicFont=cmuntx.ttf]{cmunrm.ttf}

And here are some less common units.\myfootnote{\myplainurl{http://www.giss.nasa.gov/tools/latex/ltx-86.html}}
\begin{longtable}{|>{\RaggedRight}p{0.15936\linewidth}|>{\RaggedRight}p{0.53133\linewidth}|>{\RaggedRight}p{0.22360\linewidth}|} \hline 
{\bfseries \hspace*{0pt}\ignorespaces{}\hspace*{0pt} Abbreviation}&{\bfseries \hspace*{0pt}\ignorespaces{}\hspace*{0pt} Definition}&{\bfseries \hspace*{0pt}\ignorespaces{}\hspace*{0pt} Value in points (pt)}\endhead  \hline \hspace*{0pt}\ignorespaces{}\hspace*{0pt}{\bfseries \setmainfont[Path=/usr/share/fonts/truetype/cmu/,UprightFont=cmunrm.ttf,BoldFont=cmunbx.ttf,ItalicFont=cmunti.ttf,BoldItalicFont=cmunbi.ttf]{cmunbx.ttf}\setmonofont[Path=/usr/share/fonts/truetype/cmu/,UprightFont=cmuntt.ttf,BoldFont=cmuntb.ttf,ItalicFont=cmunit.ttf,BoldItalicFont=cmuntx.ttf]{cmunbx.ttf}\bfseries bp}&\hspace*{0pt}\ignorespaces{}\hspace*{0pt}\setmainfont[Path=/usr/share/fonts/truetype/cmu/,UprightFont=cmunrm.ttf,BoldFont=cmunbx.ttf,ItalicFont=cmunti.ttf,BoldItalicFont=cmunbi.ttf]{cmunrm.ttf}\setmonofont[Path=/usr/share/fonts/truetype/cmu/,UprightFont=cmuntt.ttf,BoldFont=cmuntb.ttf,ItalicFont=cmunit.ttf,BoldItalicFont=cmuntx.ttf]{cmunrm.ttf}a big point is 1/72 inch, that means about 0.0139 inch or 0.3527 mm.&\hspace*{0pt}\ignorespaces{}\hspace*{0pt} 1.00375\\ \hline \hspace*{0pt}\ignorespaces{}\hspace*{0pt} {\bfseries \setmainfont[Path=/usr/share/fonts/truetype/cmu/,UprightFont=cmunrm.ttf,BoldFont=cmunbx.ttf,ItalicFont=cmunti.ttf,BoldItalicFont=cmunbi.ttf]{cmunbx.ttf}\setmonofont[Path=/usr/share/fonts/truetype/cmu/,UprightFont=cmuntt.ttf,BoldFont=cmuntb.ttf,ItalicFont=cmunit.ttf,BoldItalicFont=cmuntx.ttf]{cmunbx.ttf}\bfseries pc}&\hspace*{0pt}\ignorespaces{}\hspace*{0pt}{$\text{ }$}\setmainfont[Path=/usr/share/fonts/truetype/cmu/,UprightFont=cmunrm.ttf,BoldFont=cmunbx.ttf,ItalicFont=cmunti.ttf,BoldItalicFont=cmunbi.ttf]{cmunrm.ttf}\setmonofont[Path=/usr/share/fonts/truetype/cmu/,UprightFont=cmuntt.ttf,BoldFont=cmuntb.ttf,ItalicFont=cmunit.ttf,BoldItalicFont=cmuntx.ttf]{cmunrm.ttf} pica&\hspace*{0pt}\ignorespaces{}\hspace*{0pt} 12\\ \hline \hspace*{0pt}\ignorespaces{}\hspace*{0pt} {\bfseries \setmainfont[Path=/usr/share/fonts/truetype/cmu/,UprightFont=cmunrm.ttf,BoldFont=cmunbx.ttf,ItalicFont=cmunti.ttf,BoldItalicFont=cmunbi.ttf]{cmunbx.ttf}\setmonofont[Path=/usr/share/fonts/truetype/cmu/,UprightFont=cmuntt.ttf,BoldFont=cmuntb.ttf,ItalicFont=cmunit.ttf,BoldItalicFont=cmuntx.ttf]{cmunbx.ttf}\bfseries dd}&\hspace*{0pt}\ignorespaces{}\hspace*{0pt}{$\text{ }$}\setmainfont[Path=/usr/share/fonts/truetype/cmu/,UprightFont=cmunrm.ttf,BoldFont=cmunbx.ttf,ItalicFont=cmunti.ttf,BoldItalicFont=cmunbi.ttf]{cmunrm.ttf}\setmonofont[Path=/usr/share/fonts/truetype/cmu/,UprightFont=cmuntt.ttf,BoldFont=cmuntb.ttf,ItalicFont=cmunit.ttf,BoldItalicFont=cmuntx.ttf]{cmunrm.ttf} didôt (1157 didôt = 1238 points)&\hspace*{0pt}\ignorespaces{}\hspace*{0pt} 1.07\\ \hline \hspace*{0pt}\ignorespaces{}\hspace*{0pt} {\bfseries \setmainfont[Path=/usr/share/fonts/truetype/cmu/,UprightFont=cmunrm.ttf,BoldFont=cmunbx.ttf,ItalicFont=cmunti.ttf,BoldItalicFont=cmunbi.ttf]{cmunbx.ttf}\setmonofont[Path=/usr/share/fonts/truetype/cmu/,UprightFont=cmuntt.ttf,BoldFont=cmuntb.ttf,ItalicFont=cmunit.ttf,BoldItalicFont=cmuntx.ttf]{cmunbx.ttf}\bfseries cc}&\hspace*{0pt}\ignorespaces{}\hspace*{0pt}{$\text{ }$}\setmainfont[Path=/usr/share/fonts/truetype/cmu/,UprightFont=cmunrm.ttf,BoldFont=cmunbx.ttf,ItalicFont=cmunti.ttf,BoldItalicFont=cmunbi.ttf]{cmunrm.ttf}\setmonofont[Path=/usr/share/fonts/truetype/cmu/,UprightFont=cmuntt.ttf,BoldFont=cmuntb.ttf,ItalicFont=cmunit.ttf,BoldItalicFont=cmuntx.ttf]{cmunrm.ttf} cîcero (12 didôt)&\hspace*{0pt}\ignorespaces{}\hspace*{0pt} 12.84\\ \hline \hspace*{0pt}\ignorespaces{}\hspace*{0pt} {\bfseries \setmainfont[Path=/usr/share/fonts/truetype/cmu/,UprightFont=cmunrm.ttf,BoldFont=cmunbx.ttf,ItalicFont=cmunti.ttf,BoldItalicFont=cmunbi.ttf]{cmunbx.ttf}\setmonofont[Path=/usr/share/fonts/truetype/cmu/,UprightFont=cmuntt.ttf,BoldFont=cmuntb.ttf,ItalicFont=cmunit.ttf,BoldItalicFont=cmuntx.ttf]{cmunbx.ttf}\bfseries sp}&\hspace*{0pt}\ignorespaces{}\hspace*{0pt}{$\text{ }$}\setmainfont[Path=/usr/share/fonts/truetype/cmu/,UprightFont=cmunrm.ttf,BoldFont=cmunbx.ttf,ItalicFont=cmunti.ttf,BoldItalicFont=cmunbi.ttf]{cmunrm.ttf}\setmonofont[Path=/usr/share/fonts/truetype/cmu/,UprightFont=cmuntt.ttf,BoldFont=cmuntb.ttf,ItalicFont=cmunit.ttf,BoldItalicFont=cmuntx.ttf]{cmunrm.ttf} scaled point (65536sp per point)&\hspace*{0pt}\ignorespaces{}\hspace*{0pt} 0.000015\\ \hline 
\end{longtable}

\section{Box lengths}
\label{458}

A box in TeX is characterized by three lengths:
\begin{myitemize}
\item{}  {\itshape \setmainfont[Path=/usr/share/fonts/truetype/cmu/,UprightFont=cmunrm.ttf,BoldFont=cmunbx.ttf,ItalicFont=cmunti.ttf,BoldItalicFont=cmunbi.ttf]{cmunti.ttf}\setmonofont[Path=/usr/share/fonts/truetype/cmu/,UprightFont=cmuntt.ttf,BoldFont=cmuntb.ttf,ItalicFont=cmunit.ttf,BoldItalicFont=cmuntx.ttf]{cmunti.ttf}\itshape depth}
\item{} {$\text{ }$}\setmainfont[Path=/usr/share/fonts/truetype/cmu/,UprightFont=cmunrm.ttf,BoldFont=cmunbx.ttf,ItalicFont=cmunti.ttf,BoldItalicFont=cmunbi.ttf]{cmunrm.ttf}\setmonofont[Path=/usr/share/fonts/truetype/cmu/,UprightFont=cmuntt.ttf,BoldFont=cmuntb.ttf,ItalicFont=cmunit.ttf,BoldItalicFont=cmuntx.ttf]{cmunrm.ttf} {\itshape \setmainfont[Path=/usr/share/fonts/truetype/cmu/,UprightFont=cmunrm.ttf,BoldFont=cmunbx.ttf,ItalicFont=cmunti.ttf,BoldItalicFont=cmunbi.ttf]{cmunti.ttf}\setmonofont[Path=/usr/share/fonts/truetype/cmu/,UprightFont=cmuntt.ttf,BoldFont=cmuntb.ttf,ItalicFont=cmunit.ttf,BoldItalicFont=cmuntx.ttf]{cmunti.ttf}\itshape height}
\item{} {$\text{ }$}\setmainfont[Path=/usr/share/fonts/truetype/cmu/,UprightFont=cmunrm.ttf,BoldFont=cmunbx.ttf,ItalicFont=cmunti.ttf,BoldItalicFont=cmunbi.ttf]{cmunrm.ttf}\setmonofont[Path=/usr/share/fonts/truetype/cmu/,UprightFont=cmuntt.ttf,BoldFont=cmuntb.ttf,ItalicFont=cmunit.ttf,BoldItalicFont=cmuntx.ttf]{cmunrm.ttf} {\itshape \setmainfont[Path=/usr/share/fonts/truetype/cmu/,UprightFont=cmunrm.ttf,BoldFont=cmunbx.ttf,ItalicFont=cmunti.ttf,BoldItalicFont=cmunbi.ttf]{cmunti.ttf}\setmonofont[Path=/usr/share/fonts/truetype/cmu/,UprightFont=cmuntt.ttf,BoldFont=cmuntb.ttf,ItalicFont=cmunit.ttf,BoldItalicFont=cmuntx.ttf]{cmunti.ttf}\itshape width}
\end{myitemize}
\setmainfont[Path=/usr/share/fonts/truetype/cmu/,UprightFont=cmunrm.ttf,BoldFont=cmunbx.ttf,ItalicFont=cmunti.ttf,BoldItalicFont=cmunbi.ttf]{cmunrm.ttf}\setmonofont[Path=/usr/share/fonts/truetype/cmu/,UprightFont=cmuntt.ttf,BoldFont=cmuntb.ttf,ItalicFont=cmunit.ttf,BoldItalicFont=cmuntx.ttf]{cmunrm.ttf}

See \mylref{478}{Boxes}.
\section{Length manipulation}
\label{459}

You can change the values of the variables defining the page layout with two commands. With this one you can set a new value for an existing length variable:

\begin{Shaded}
\begin{Highlighting}[]

\NormalTok{\textbackslash{}setlength\{\textbackslash{}mylength\}\{length\}}
\end{Highlighting}
\end{Shaded}


with this other one, you can add a value to the existing one:

\begin{Shaded}
\begin{Highlighting}[]

\NormalTok{\textbackslash{}addtolength\{\textbackslash{}mylength\}\{length\}}
\end{Highlighting}
\end{Shaded}


You can create your own length with the command, and you must create a new length before you attempt to set it:

\begin{Shaded}
\begin{Highlighting}[]

\NormalTok{\textbackslash{}newlength\{\textbackslash{}mylength\}}
\end{Highlighting}
\end{Shaded}


You may also set a length from the size of a text with one of these commands:

\begin{Shaded}
\begin{Highlighting}[]

\NormalTok{\textbackslash{}settowidth\{\textbackslash{}mylength\}\{some text\}}
\NormalTok{\textbackslash{}settoheight\{\textbackslash{}mylength\}\{some text\}}
\NormalTok{\textbackslash{}settodepth\{\textbackslash{}mylength\}\{some text\}}
\end{Highlighting}
\end{Shaded}


When using these commands, you may duplicate the text that you want to use as reference if you plan to also display it. But LaTeX also provides \LaTeXTT{\textbackslash{}savebox} to avoid this duplication.
You may wish to look at the example below to see how you can use these. See \mylref{478}{Boxes} for more details.

You can also define stretched values. A stretching value is a length preceded by \LaTeXTT{plus} or \LaTeXTT{minus} to specify to what extent {\ttfamily \setmainfont[Path=/usr/share/fonts/truetype/cmu/,UprightFont=cmunrm.ttf,BoldFont=cmunbx.ttf,ItalicFont=cmunti.ttf,BoldItalicFont=cmunbi.ttf]{cmuntt.ttf}\setmonofont[Path=/usr/share/fonts/truetype/cmu/,UprightFont=cmuntt.ttf,BoldFont=cmuntb.ttf,ItalicFont=cmunit.ttf,BoldItalicFont=cmuntx.ttf]{cmuntt.ttf}\ttfamily tex}{$\text{ }$}\setmainfont[Path=/usr/share/fonts/truetype/cmu/,UprightFont=cmunrm.ttf,BoldFont=cmunbx.ttf,ItalicFont=cmunti.ttf,BoldItalicFont=cmunbi.ttf]{cmunrm.ttf}\setmonofont[Path=/usr/share/fonts/truetype/cmu/,UprightFont=cmuntt.ttf,BoldFont=cmuntb.ttf,ItalicFont=cmunit.ttf,BoldItalicFont=cmuntx.ttf]{cmunrm.ttf} is authorized to change the length. Example:

\begin{Shaded}
\begin{Highlighting}[]

\NormalTok{\textbackslash{}setlength\{\textbackslash{}parskip\}\{10pt plus 5pt minus 3pt\}}
\end{Highlighting}
\end{Shaded}


It means that {\ttfamily \setmainfont[Path=/usr/share/fonts/truetype/cmu/,UprightFont=cmunrm.ttf,BoldFont=cmunbx.ttf,ItalicFont=cmunti.ttf,BoldItalicFont=cmunbi.ttf]{cmuntt.ttf}\setmonofont[Path=/usr/share/fonts/truetype/cmu/,UprightFont=cmuntt.ttf,BoldFont=cmuntb.ttf,ItalicFont=cmunit.ttf,BoldItalicFont=cmuntx.ttf]{cmuntt.ttf}\ttfamily tex}{$\text{ }$}\setmainfont[Path=/usr/share/fonts/truetype/cmu/,UprightFont=cmunrm.ttf,BoldFont=cmunbx.ttf,ItalicFont=cmunti.ttf,BoldItalicFont=cmunbi.ttf]{cmunrm.ttf}\setmonofont[Path=/usr/share/fonts/truetype/cmu/,UprightFont=cmuntt.ttf,BoldFont=cmuntb.ttf,ItalicFont=cmunit.ttf,BoldItalicFont=cmuntx.ttf]{cmunrm.ttf}  will try to use a length of 10pt; if it is underfull, it will raise the length up to a maximum of 15pt; if it is overfull, it will lower the length up to a minimum of 7pt.

Note that it is not mandatory to specify both the \LaTeXTT{plus} and the \LaTeXTT{minus} values, but if you do, \LaTeXTT{plus} must be placed before \LaTeXTT{minus}.

To print a length, you can use the \LaTeXTT{\textbackslash{}the} command:
\begin{Shaded}
\begin{Highlighting}[]

\NormalTok{\textbackslash{}the\textbackslash{}textwidth}
\end{Highlighting}
\end{Shaded}

\subsection{Plain TeX}
\label{460}
To create a new length:
\begin{Shaded}
\begin{Highlighting}[]

\NormalTok{\textbackslash{}newdimen\textbackslash{}mylength}
\end{Highlighting}
\end{Shaded}


To set a length:
\begin{Shaded}
\begin{Highlighting}[]

\NormalTok{\textbackslash{}mylength=1.5in}
\end{Highlighting}
\end{Shaded}


To view, it is the same as with LaTeX, using the command \LaTeXTT{\textbackslash{}the}.
\section{LaTeX default lengths}
\label{461}
Common length macros are:
{\bfseries
\begin{mydescription} \textbackslash{}baselineskip 
\end{mydescription}
}
\begin{myquote}\item{} The normal vertical distance between lines in a paragraph.
\end{myquote}
{\bfseries
\begin{mydescription} \textbackslash{}baselinestretch 
\end{mydescription}
}
\begin{myquote}\item{} Multiplies \textbackslash{}baselineskip.
\end{myquote}
{\bfseries
\begin{mydescription} \textbackslash{}columnsep 
\end{mydescription}
}
\begin{myquote}\item{} The distance between columns.
\end{myquote}
{\bfseries
\begin{mydescription} \textbackslash{}columnwidth 
\end{mydescription}
}
\begin{myquote}\item{} The width of the column.
\end{myquote}
{\bfseries
\begin{mydescription} \textbackslash{}evensidemargin 
\end{mydescription}
}
\begin{myquote}\item{} The margin for \textquotesingle{}even\textquotesingle{} pages (think of a printed booklet).
\end{myquote}
{\bfseries
\begin{mydescription} \textbackslash{}linewidth 
\end{mydescription}
}
\begin{myquote}\item{} The width of a line in the local environment.
\end{myquote}
{\bfseries
\begin{mydescription} \textbackslash{}oddsidemargin 
\end{mydescription}
}
\begin{myquote}\item{} The margin for \textquotesingle{}odd\textquotesingle{} pages (think of a printed booklet).
\end{myquote}
{\bfseries
\begin{mydescription} \textbackslash{}paperwidth 
\end{mydescription}
}
\begin{myquote}\item{} The width of the page.
\end{myquote}
{\bfseries
\begin{mydescription} \textbackslash{}paperheight 
\end{mydescription}
}
\begin{myquote}\item{} The height of the page.
\end{myquote}
{\bfseries
\begin{mydescription} \textbackslash{}parindent 
\end{mydescription}
}
\begin{myquote}\item{} The normal paragraph indentation.
\end{myquote}
{\bfseries
\begin{mydescription} \textbackslash{}parskip 
\end{mydescription}
}
\begin{myquote}\item{} The extra vertical space between paragraphs.
\end{myquote}
{\bfseries
\begin{mydescription} \textbackslash{}tabcolsep 
\end{mydescription}
}
\begin{myquote}\item{} The default separation between columns in a tabular environment.
\end{myquote}
{\bfseries
\begin{mydescription} \textbackslash{}textheight 
\end{mydescription}
}
\begin{myquote}\item{} The height of text on the page.
\end{myquote}
{\bfseries
\begin{mydescription} \textbackslash{}textwidth 
\end{mydescription}
}
\begin{myquote}\item{} The width of the text on the page.
\end{myquote}
{\bfseries
\begin{mydescription} \textbackslash{}topmargin 
\end{mydescription}
}
\begin{myquote}\item{} The size of the top margin.
\end{myquote}
{\bfseries
\begin{mydescription} \textbackslash{}unitlength 
\end{mydescription}
}
\begin{myquote}\item{} Units of length in \LaTeXTT{picture} environment.
\end{myquote}

\section{Fixed-{}length spaces}
\label{462}

To insert a fixed-{}length space, use:
\begin{Shaded}
\begin{Highlighting}[]

\NormalTok{\textbackslash{}hspace\{length\}}
\NormalTok{\textbackslash{}vspace\{length\}}
\end{Highlighting}
\end{Shaded}


\LaTeXTT{\textbackslash{}hspace} stands for horizontal space, \LaTeXTT{\textbackslash{}vspace} for vertical space.

If such a space should be kept even if it falls at the end or the start of a line, use \LaTeXTT{\textbackslash{}hspace*} instead.

If the space should be preserved at the top or at the bottom of a page, use the starred version of the command, \LaTeXTT{\textbackslash{}vspace*}, instead of \LaTeXTT{\textbackslash{}vspace}.
If you want to add space at the beginning of the document, without anything else written before, then you may use
\begin{Shaded}
\begin{Highlighting}[]

\NormalTok{\{ \textbackslash{}vspace*\{length\} \}}
\end{Highlighting}
\end{Shaded}

It\textquotesingle{}s important you use the \LaTeXTT{\textbackslash{}vspace*} command instead of \LaTeXTT{\textbackslash{}vspace}, otherwise LaTeX can silently ignore the extra space.

TeX features some macros for fixed-{}length spacing.{\bfseries
\begin{mydescription} \LaTeXTT{\textbackslash{}smallskip}
\end{mydescription}
}
\begin{myquote}\item{} Inserts a small space in vertical mode (between two paragraphs).
\end{myquote}
{\bfseries
\begin{mydescription} \LaTeXTT{\textbackslash{}medskip}
\end{mydescription}
}
\begin{myquote}\item{} Inserts a medium space in vertical mode (between two paragraphs).
\end{myquote}
{\bfseries
\begin{mydescription} \LaTeXTT{\textbackslash{}bigskip}
\end{mydescription}
}
\begin{myquote}\item{} Inserts a big space in vertical mode (between two paragraphs).
\end{myquote}


The vertical mode is during the process of assembling boxes \symbol{34}vertically\symbol{34}, like paragraphs to build a page.
The horizontal mode is during the process of assembling boxes \symbol{34}horizontally\symbol{34}, like letters to build a word or words to build a paragraph.

The fact they are vertical mode commands mean they will be ignored (or fail) in horizontal mode such as in the middle of a paragraph. The first token next the a double linebreak is still in vertical mode if it does not expand to characters.

\begin{Shaded}
\begin{Highlighting}[]

\CommentTok{% WRONG!}
\NormalTok{Some words.}
\NormalTok{\textbackslash{}bigskip}
\NormalTok{Let's continue.}
 
\CommentTok{%% CORRECT!}
\NormalTok{Some words.}
 
\NormalTok{\textbackslash{}bigskip}
\NormalTok{Let's continue.}
\end{Highlighting}
\end{Shaded}


\begin{TemplateInfo}{\danger}{Warning}This is a common error! Anyway, these commands should never be used in regular documents.\LaTeXNullTemplate{}\end{TemplateInfo}
\section{Rubber/Stretching lengths}
\label{463}

The command:
\begin{Shaded}
\begin{Highlighting}[]

\NormalTok{\textbackslash{}stretch\{factor\}}
\end{Highlighting}
\end{Shaded}

generates a special rubber space where \LaTeXTT{factor} is a number, possibly a float. It stretches until all the remaining space on a line is filled up. If two \LaTeXTT{\textbackslash{}hspace\{\textbackslash{}stretch\{factor}\}\} commands are issued on the same line, they grow according to the stretch factor.

\begin{longtable}{p{1.0\linewidth}}
\begin{Shaded}
\begin{Highlighting}[]

\NormalTok{x \textbackslash{}hspace\{ \textbackslash{}stretch\{1\} \} x \textbackslash{}hspace\{ \textbackslash{}stretch\{3\} \} x}
\end{Highlighting}
\end{Shaded}
\\

\TemplatePreformat{$\text{ }$\newline{}
x$\text{ }${}$\text{ }${}$\text{ }${}$\text{ }${}$\text{ }${}$\text{ }${}x$\text{ }${}$\text{ }${}$\text{ }${}$\text{ }${}$\text{ }${}$\text{ }${}$\text{ }${}$\text{ }${}$\text{ }${}$\text{ }${}$\text{ }${}$\text{ }${}$\text{ }${}$\text{ }${}$\text{ }${}$\text{ }${}$\text{ }${}$\text{ }${}x$\text{ }$\newline{}
}

\end{longtable}

The same way, you can stretch vertically:
\begin{Shaded}
\begin{Highlighting}[]

\NormalTok{\textbackslash{}maketitle}
\NormalTok{\textbackslash{}vspace\{ \textbackslash{}stretch\{1\} \}}
\NormalTok{Some comments.}
\NormalTok{\textbackslash{}vspace\{ \textbackslash{}stretch\{1\} \}}
\NormalTok{\textbackslash{}tableofcontents}
\end{Highlighting}
\end{Shaded}


You can also use \LaTeXTT{\textbackslash{}fill} instead of \LaTeXTT{\textbackslash{}stretch\{1\}}.

The \LaTeXTT{\textbackslash{}stretch} command, in connection with \LaTeXTT{\textbackslash{}pagebreak}, can be used to typeset text on the last line of a page, or to center text vertically on a page.

There are \textquotesingle{}shortcut commands\textquotesingle{} for stretching with factor 1 ({\itshape \setmainfont[Path=/usr/share/fonts/truetype/cmu/,UprightFont=cmunrm.ttf,BoldFont=cmunbx.ttf,ItalicFont=cmunti.ttf,BoldItalicFont=cmunbi.ttf]{cmunti.ttf}\setmonofont[Path=/usr/share/fonts/truetype/cmu/,UprightFont=cmuntt.ttf,BoldFont=cmuntb.ttf,ItalicFont=cmunit.ttf,BoldItalicFont=cmuntx.ttf]{cmunti.ttf}\itshape i.e.}{$\text{ }$}\setmainfont[Path=/usr/share/fonts/truetype/cmu/,UprightFont=cmunrm.ttf,BoldFont=cmunbx.ttf,ItalicFont=cmunti.ttf,BoldItalicFont=cmunbi.ttf]{cmunrm.ttf}\setmonofont[Path=/usr/share/fonts/truetype/cmu/,UprightFont=cmuntt.ttf,BoldFont=cmuntb.ttf,ItalicFont=cmunit.ttf,BoldItalicFont=cmuntx.ttf]{cmunrm.ttf} with \LaTeXTT{\textbackslash{}stretch\{1\}} or \LaTeXTT{\textbackslash{}fill}): \LaTeXTT{\textbackslash{}hfill} and \LaTeXTT{\textbackslash{}vfill}.

Example:
\begin{Shaded}
\begin{Highlighting}[]

\NormalTok{\textbackslash{}maketitle}
\NormalTok{\textbackslash{}vfill}
\NormalTok{Some comments.}
\NormalTok{\textbackslash{}vfill}
\NormalTok{\textbackslash{}tableofcontents}
\end{Highlighting}
\end{Shaded}

\subsection{Fill the rest of the line}
\label{464}
Several macros allow filling the rest of the line -{}-{} or stretching parts of the line -{}-{} in different manners.
\begin{myitemize}
\item{}  \LaTeXTT{\textbackslash{}hfill} will produce empty space.
\item{}  \LaTeXTT{\textbackslash{}dotfill} will produce dots.
\item{}  \LaTeXTT{\textbackslash{}hrulefill} will produce a rule.
\end{myitemize}

\section{Examples}
\label{465}
Resize an image to take exactly half the text width :
\begin{Shaded}
\begin{Highlighting}[]

\NormalTok{\textbackslash{}includegraphics[width=0.5\textbackslash{}textwidth]\{mygraphic\}}
\end{Highlighting}
\end{Shaded}


Make distance between items larger (inside an itemize environment) :
\begin{Shaded}
\begin{Highlighting}[]

\NormalTok{\textbackslash{}addtolength\{\textbackslash{}itemsep\}\{0.5\textbackslash{}baselineskip\}}
\end{Highlighting}
\end{Shaded}


Use of \LaTeXTT{\textbackslash{}savebox} to resize an image to the height of the text:
\begin{Shaded}
\begin{Highlighting}[]

\CommentTok{% Create the holders we will need for our work}
\NormalTok{\textbackslash{}newlength\{\textbackslash{}mytitleheight\}}
\NormalTok{\textbackslash{}newsavebox\{\textbackslash{}mytitletext\}}
\CommentTok
  \NormalTok{\textbackslash{}Large\textbackslash{}bfseries This is our title}\CommentTok{%}
\NormalTok{\}}
\NormalTok{\textbackslash{}settoheight\{\textbackslash{}mytitleheight\}\{ \textbackslash{}usebox\{\textbackslash{}mytitletext\} \}}
\CommentTok
  \NormalTok{\textbackslash{}includegraphics[height=\textbackslash{}mytitleheight]\{my_image\}}\CommentTok
  \NormalTok{\textbackslash{}usebox\{\textbackslash{}mytitletext\}}\CommentTok{%}
\NormalTok{\}}
\end{Highlighting}
\end{Shaded}

\section{References}
\label{466}

\section{See also}
\label{467}
\begin{myitemize}
\item{}  \myhref{http://www-h.eng.cam.ac.uk/help/tpl/textprocessing/squeeze.html}{University of Cambridge >{}  Engineering Department >{}  computing help >{}  LaTeX >{} Squeezing Space in LaTeX}
\end{myitemize}


\chapter{Counters}

\myminitoc
\label{468}

\label{469}


Counters are an essential part of LaTeX: they allow you to control the numbering mechanism of everything (sections, lists, captions, etc.).
To that end each counter stores an integer value in the range of \myhref{https://en.wikipedia.org/wiki/long\%20integer}{long integer}, i.e., from {$-2^{31}$} to {$2^{31}-1$}. \myplainurl{http://tex.stackexchange.com/questions/270535/what-is-the-maximum-integer-that-can-be-saved-in-a-latex-counter}
\section{Counter manipulation}
\label{470}

In LaTeX it is fairly easy to create new counters and even counters that reset automatically when another counter is increased (think subsection in a section for example). With the command
\begin{Shaded}
\begin{Highlighting}[]

\NormalTok{\textbackslash{}newcounter\{NameOfTheNewCounter\}}
\end{Highlighting}
\end{Shaded}

you create a new counter that is automatically set to zero.
If you want the counter to be reset to zero every time another counter is increased, use:
\begin{Shaded}
\begin{Highlighting}[]

\NormalTok{\textbackslash{}newcounter\{NameOfTheNewCounter\}[NameOfTheOtherCounter]}
\end{Highlighting}
\end{Shaded}


To increase the counter, either use 
\begin{Shaded}
\begin{Highlighting}[]

\NormalTok{\textbackslash{}stepcounter\{NameOfTheNewCounter\}}
\end{Highlighting}
\end{Shaded}

or
\begin{Shaded}
\begin{Highlighting}[]

\NormalTok{\textbackslash{}refstepcounter\{NameOfTheNewCounter\} }\CommentTok{% used for labels and cross referencing}
\end{Highlighting}
\end{Shaded}

or
\begin{Shaded}
\begin{Highlighting}[]

\NormalTok{\textbackslash{}addtocounter\{NameOfTheNewCounter\}\{number\}}
\end{Highlighting}
\end{Shaded}

here the number can also be negative. For automatic resetting you need to use {\ttfamily \setmainfont[Path=/usr/share/fonts/truetype/cmu/,UprightFont=cmunrm.ttf,BoldFont=cmunbx.ttf,ItalicFont=cmunti.ttf,BoldItalicFont=cmunbi.ttf]{cmuntt.ttf}\setmonofont[Path=/usr/share/fonts/truetype/cmu/,UprightFont=cmuntt.ttf,BoldFont=cmuntb.ttf,ItalicFont=cmunit.ttf,BoldItalicFont=cmuntx.ttf]{cmuntt.ttf}\ttfamily \textbackslash{}stepcounter}\setmainfont[Path=/usr/share/fonts/truetype/cmu/,UprightFont=cmunrm.ttf,BoldFont=cmunbx.ttf,ItalicFont=cmunti.ttf,BoldItalicFont=cmunbi.ttf]{cmunrm.ttf}\setmonofont[Path=/usr/share/fonts/truetype/cmu/,UprightFont=cmuntt.ttf,BoldFont=cmuntb.ttf,ItalicFont=cmunit.ttf,BoldItalicFont=cmuntx.ttf]{cmunrm.ttf}.

To set the counter value explicitly, use 
\begin{Shaded}
\begin{Highlighting}[]

\NormalTok{\textbackslash{}setcounter\{NameOfTheNewCounter\}\{number\}}
\end{Highlighting}
\end{Shaded}

\section{Counter access}
\label{471}

There are several ways to get access to a counter.
\begin{myitemize}
\item{}  \LaTeXTT{\textbackslash{}theNameOfTheNewCounter} will print the formatted string related to the counter (note the \symbol{34}the\symbol{34} before the actual name of the counter).
\item{}  \LaTeXTT{\textbackslash{}value\{NameOfTheNewCounter\}} will return the counter value which can be used by other counters or for calculations. It is not a formatted string, so it cannot be used in text.
\item{}  \LaTeXTT{\textbackslash{}arabic\{NameOfTheNewCounter\}} will print the formatted counter using arabic numbers.
\end{myitemize}


Note that \LaTeXTT{\textbackslash{}arabic\{NameOfTheNewCounter\}} may be used as a value too, but not the others.

Strangely enough, LaTeX counters are {\itshape \setmainfont[Path=/usr/share/fonts/truetype/cmu/,UprightFont=cmunrm.ttf,BoldFont=cmunbx.ttf,ItalicFont=cmunti.ttf,BoldItalicFont=cmunbi.ttf]{cmunti.ttf}\setmonofont[Path=/usr/share/fonts/truetype/cmu/,UprightFont=cmuntt.ttf,BoldFont=cmuntb.ttf,ItalicFont=cmunit.ttf,BoldItalicFont=cmuntx.ttf]{cmunti.ttf}\itshape not}{$\text{ }$}\setmainfont[Path=/usr/share/fonts/truetype/cmu/,UprightFont=cmunrm.ttf,BoldFont=cmunbx.ttf,ItalicFont=cmunti.ttf,BoldItalicFont=cmunbi.ttf]{cmunrm.ttf}\setmonofont[Path=/usr/share/fonts/truetype/cmu/,UprightFont=cmuntt.ttf,BoldFont=cmuntb.ttf,ItalicFont=cmunit.ttf,BoldItalicFont=cmuntx.ttf]{cmunrm.ttf} introduced by a backslash in any case, even with the \LaTeXTT{\textbackslash{}the} command. plainTeX equivalents \LaTeXTT{\textbackslash{}count} and \LaTeXTT{\textbackslash{}newcounter\textbackslash{}mycounter} do abide by the backslash rule.
\section{Counter style}
\label{472}

Each counter also has a default format that dictates how it is displayed whenever LaTeX needs to print it. Such formats are specified using internal LaTeX commands:

\begin{longtable}{|>{\RaggedRight}p{0.22611\linewidth}|>{\RaggedRight}p{0.71674\linewidth}|} \hline 
{\bfseries \hspace*{0pt}\ignorespaces{}\hspace*{0pt} Command}&{\bfseries \hspace*{0pt}\ignorespaces{}\hspace*{0pt} Example}\endhead  \hline \hspace*{0pt}\ignorespaces{}\hspace*{0pt} \LaTeXTT{\textbackslash{}arabic}&\hspace*{0pt}\ignorespaces{}\hspace*{0pt} 1, 2, 3 ...\\ \hline \hspace*{0pt}\ignorespaces{}\hspace*{0pt} \LaTeXTT{\textbackslash{}alph}&\hspace*{0pt}\ignorespaces{}\hspace*{0pt} a, b, c ...\\ \hline \hspace*{0pt}\ignorespaces{}\hspace*{0pt} \LaTeXTT{\textbackslash{}Alph}&\hspace*{0pt}\ignorespaces{}\hspace*{0pt} A, B, C ...\\ \hline \hspace*{0pt}\ignorespaces{}\hspace*{0pt} \LaTeXTT{\textbackslash{}roman}&\hspace*{0pt}\ignorespaces{}\hspace*{0pt} i, ii, iii ...\\ \hline \hspace*{0pt}\ignorespaces{}\hspace*{0pt} \LaTeXTT{\textbackslash{}Roman}&\hspace*{0pt}\ignorespaces{}\hspace*{0pt} I, II, III ...\\ \hline \hspace*{0pt}\ignorespaces{}\hspace*{0pt} \LaTeXTT{\textbackslash{}fnsymbol}&\hspace*{0pt}\ignorespaces{}\hspace*{0pt} Aimed at footnotes; prints a sequence of symbols.\\ \hline 
\end{longtable}

\section{LaTeX default counters}
\label{473}

\begin{myitemize}
\item{}  part
\item{}  chapter
\item{}  section
\item{}  subsection
\item{}  subsubsection
\item{}  paragraph
\item{}  subparagraph
\item{}  page
\item{}  figure
\item{}  table
\item{}  footnote
\item{}  mpfootnote
\end{myitemize}


For the \LaTeXTT{enumerate} environment:
\begin{myitemize}
\item{}  enumi
\item{}  enumii
\item{}  enumiii
\item{}  enumiv
\end{myitemize}


For the \LaTeXTT{eqnarray} environment:
\begin{myitemize}
\item{}  equation
\end{myitemize}

\section{Book with parts, sections, but no chapters}
\label{474}

Here follows an example where we want to use parts and sections, but no chapters in the book class :
\begin{Shaded}
\begin{Highlighting}[]

\NormalTok{\textbackslash{}renewcommand\{\textbackslash{}thesection\}\{\textbackslash{}thepart .\textbackslash{}arabic\{section\}\}}
 
\NormalTok{\textbackslash{}part\{My Part\}                                                                }
\NormalTok{\textbackslash{}section\{My Section\}}
\NormalTok{\textbackslash{}subsection\{My Subsection\}}
\end{Highlighting}
\end{Shaded}

\section{Custom {\itshape \setmainfont[Path=/usr/share/fonts/truetype/cmu/,UprightFont=cmunrm.ttf,BoldFont=cmunbx.ttf,ItalicFont=cmunti.ttf,BoldItalicFont=cmunbi.ttf]{cmunti.ttf}\setmonofont[Path=/usr/share/fonts/truetype/cmu/,UprightFont=cmuntt.ttf,BoldFont=cmuntb.ttf,ItalicFont=cmunit.ttf,BoldItalicFont=cmuntx.ttf]{cmunti.ttf}\itshape enumerate}{$\text{ }$}\setmainfont[Path=/usr/share/fonts/truetype/cmu/,UprightFont=cmunrm.ttf,BoldFont=cmunbx.ttf,ItalicFont=cmunti.ttf,BoldItalicFont=cmunbi.ttf]{cmunrm.ttf}\setmonofont[Path=/usr/share/fonts/truetype/cmu/,UprightFont=cmuntt.ttf,BoldFont=cmuntb.ttf,ItalicFont=cmunit.ttf,BoldItalicFont=cmuntx.ttf]{cmunrm.ttf}}
\label{475}

See the \mylref{186}{List Structures} chapter.
\section{Custom sectioning}
\label{476}

Here is an example for recreating something similar to a section and subsection counter that already exist in LaTeX:
\begin{Shaded}
\begin{Highlighting}[]

\NormalTok{\textbackslash{}newcounter\{mysection\}}
\NormalTok{\textbackslash{}newcounter\{mysubsection\}[mysection]}
\NormalTok{\textbackslash{}addtocounter\{mysection\}\{2\} }\CommentTok{% set them to some other numbers than 0}
\NormalTok{\textbackslash{}addtocounter\{mysubsection\}\{10\} }\CommentTok
\NormalTok{\textbackslash{}arabic\{mysection\}.\textbackslash{}arabic\{mysubsection\}}
\NormalTok{Blah blah}
 
\NormalTok{\textbackslash{}stepcounter\{mysection\}}
\NormalTok{\textbackslash{}arabic\{mysection\}.\textbackslash{}arabic\{mysubsection\}}
\NormalTok{Blah blah}
 
\NormalTok{\textbackslash{}stepcounter\{mysubsection\}}
\NormalTok{\textbackslash{}arabic\{mysection\}.\textbackslash{}arabic\{mysubsection\}}
\NormalTok{Blah blah}
 
\NormalTok{\textbackslash{}addtocounter\{mysubsection\}\{25\}}
\NormalTok{\textbackslash{}arabic\{mysection\}.\textbackslash{}arabic\{mysubsection\}}
\NormalTok{Blah blah and more blah blah}
\end{Highlighting}
\end{Shaded}


\chapter{Boxes}

\myminitoc
\label{477}

\label{478}


LaTeX builds up its pages by pushing around boxes. At first, each letter is a little box, which is then glued to other letters to form words. These are again glued to other words, but with special glue, which is elastic so that a series of words can be squeezed or stretched as to exactly fill a line on the page.

Admittedly, this is a very simplistic description of what really happens, but the point is that TeX operates with glue and boxes. Letters are not the only things that can be boxes. One can put virtually everything into a box, including other boxes. Each box will then be handled by LaTeX as if it were a single letter.

The past chapters have already dealt with some boxes, although they weren\textquotesingle{}t described as such. The tabular environment and the \LaTeXTT{\textbackslash{}includegraphics}, for example, both produce a box. This means that one can easily arrange two tables or images side by side. You just have to make sure that their combined width is not larger than the \LaTeXTT{\textbackslash{}textwidth}. An General overview about different box commands can be found here: \myplainurl{http://www.personal.ceu.hu/tex/spacebox.htm} .
\section{TeX character boxes}
\label{479}

TeX characters are stored in boxes like every printed element. Boxes have three dimensional properties:
\begin{myitemize}
\item{}  The {\itshape \setmainfont[Path=/usr/share/fonts/truetype/cmu/,UprightFont=cmunrm.ttf,BoldFont=cmunbx.ttf,ItalicFont=cmunti.ttf,BoldItalicFont=cmunbi.ttf]{cmunti.ttf}\setmonofont[Path=/usr/share/fonts/truetype/cmu/,UprightFont=cmuntt.ttf,BoldFont=cmuntb.ttf,ItalicFont=cmunit.ttf,BoldItalicFont=cmuntx.ttf]{cmunti.ttf}\itshape height}{$\text{ }$}\setmainfont[Path=/usr/share/fonts/truetype/cmu/,UprightFont=cmunrm.ttf,BoldFont=cmunbx.ttf,ItalicFont=cmunti.ttf,BoldItalicFont=cmunbi.ttf]{cmunrm.ttf}\setmonofont[Path=/usr/share/fonts/truetype/cmu/,UprightFont=cmuntt.ttf,BoldFont=cmuntb.ttf,ItalicFont=cmunit.ttf,BoldItalicFont=cmuntx.ttf]{cmunrm.ttf} is the length between the baseline and the top of the box.
\item{}  The {\itshape \setmainfont[Path=/usr/share/fonts/truetype/cmu/,UprightFont=cmunrm.ttf,BoldFont=cmunbx.ttf,ItalicFont=cmunti.ttf,BoldItalicFont=cmunbi.ttf]{cmunti.ttf}\setmonofont[Path=/usr/share/fonts/truetype/cmu/,UprightFont=cmuntt.ttf,BoldFont=cmuntb.ttf,ItalicFont=cmunit.ttf,BoldItalicFont=cmuntx.ttf]{cmunti.ttf}\itshape depth}{$\text{ }$}\setmainfont[Path=/usr/share/fonts/truetype/cmu/,UprightFont=cmunrm.ttf,BoldFont=cmunbx.ttf,ItalicFont=cmunti.ttf,BoldItalicFont=cmunbi.ttf]{cmunrm.ttf}\setmonofont[Path=/usr/share/fonts/truetype/cmu/,UprightFont=cmuntt.ttf,BoldFont=cmuntb.ttf,ItalicFont=cmunit.ttf,BoldItalicFont=cmuntx.ttf]{cmunrm.ttf}  is the length between the baseline and the bottom of the box.
\item{}  The {\itshape \setmainfont[Path=/usr/share/fonts/truetype/cmu/,UprightFont=cmunrm.ttf,BoldFont=cmunbx.ttf,ItalicFont=cmunti.ttf,BoldItalicFont=cmunbi.ttf]{cmunti.ttf}\setmonofont[Path=/usr/share/fonts/truetype/cmu/,UprightFont=cmuntt.ttf,BoldFont=cmuntb.ttf,ItalicFont=cmunit.ttf,BoldItalicFont=cmuntx.ttf]{cmunti.ttf}\itshape width}{$\text{ }$}\setmainfont[Path=/usr/share/fonts/truetype/cmu/,UprightFont=cmunrm.ttf,BoldFont=cmunbx.ttf,ItalicFont=cmunti.ttf,BoldItalicFont=cmunbi.ttf]{cmunrm.ttf}\setmonofont[Path=/usr/share/fonts/truetype/cmu/,UprightFont=cmuntt.ttf,BoldFont=cmuntb.ttf,ItalicFont=cmunit.ttf,BoldItalicFont=cmuntx.ttf]{cmunrm.ttf}  is the width of the box.
\end{myitemize}




\begin{minipage}{1.0\linewidth}
\begin{center}
\includegraphics[width=1.0\linewidth,height=6.5in,keepaspectratio]{../images/77.\SVGExtension}
\end{center}
\raggedright{}\myfigurewithoutcaption{77}
\end{minipage}\vspace{0.75cm}


\section{makebox and mbox}
\label{480}

While \LaTeXTT{\textbackslash{}parbox} packs up a whole paragraph doing line breaking and everything, there is also a class of boxing commands that operates only on horizontally aligned material. We already know one of them; it’s called \LaTeXTT{\textbackslash{}mbox}. It simply packs up a series of boxes into another one, and can be used to prevent LaTeX from breaking two words. (See \mylref{116}{Hyphenation}.) As you can put boxes inside boxes, these horizontal box packers give you ultimate flexibility.

\begin{Shaded}
\begin{Highlighting}[]

\NormalTok{\textbackslash{}mbox\{text\}}
\NormalTok{\textbackslash{}makebox[width][pos]\{text\}}
\end{Highlighting}
\end{Shaded}


\LaTeXTT{width} defines the width of the resulting box as seen from the outside. This means it can be smaller than the material inside the box. You can even set the width to 0pt so that the text inside the box will be typeset without influencing the surrounding boxes. Besides the \mylref{456}{length} expressions, you can also use \LaTeXTT{\textbackslash{}width}, \LaTeXTT{\textbackslash{}height}, \LaTeXTT{\textbackslash{}depth} and \LaTeXTT{\textbackslash{}totalheight} in the \LaTeXTT{width} parameter. They are set from values obtained by measuring the typeset text.

The \LaTeXTT{pos} parameter takes a one letter value: {\bfseries \setmainfont[Path=/usr/share/fonts/truetype/cmu/,UprightFont=cmunrm.ttf,BoldFont=cmunbx.ttf,ItalicFont=cmunti.ttf,BoldItalicFont=cmunbi.ttf]{cmunbx.ttf}\setmonofont[Path=/usr/share/fonts/truetype/cmu/,UprightFont=cmuntt.ttf,BoldFont=cmuntb.ttf,ItalicFont=cmunit.ttf,BoldItalicFont=cmuntx.ttf]{cmunbx.ttf}\bfseries c}\setmainfont[Path=/usr/share/fonts/truetype/cmu/,UprightFont=cmunrm.ttf,BoldFont=cmunbx.ttf,ItalicFont=cmunti.ttf,BoldItalicFont=cmunbi.ttf]{cmunrm.ttf}\setmonofont[Path=/usr/share/fonts/truetype/cmu/,UprightFont=cmuntt.ttf,BoldFont=cmuntb.ttf,ItalicFont=cmunit.ttf,BoldItalicFont=cmuntx.ttf]{cmunrm.ttf}enter, flush{\bfseries \setmainfont[Path=/usr/share/fonts/truetype/cmu/,UprightFont=cmunrm.ttf,BoldFont=cmunbx.ttf,ItalicFont=cmunti.ttf,BoldItalicFont=cmunbi.ttf]{cmunbx.ttf}\setmonofont[Path=/usr/share/fonts/truetype/cmu/,UprightFont=cmuntt.ttf,BoldFont=cmuntb.ttf,ItalicFont=cmunit.ttf,BoldItalicFont=cmuntx.ttf]{cmunbx.ttf}\bfseries l}\setmainfont[Path=/usr/share/fonts/truetype/cmu/,UprightFont=cmunrm.ttf,BoldFont=cmunbx.ttf,ItalicFont=cmunti.ttf,BoldItalicFont=cmunbi.ttf]{cmunrm.ttf}\setmonofont[Path=/usr/share/fonts/truetype/cmu/,UprightFont=cmuntt.ttf,BoldFont=cmuntb.ttf,ItalicFont=cmunit.ttf,BoldItalicFont=cmuntx.ttf]{cmunrm.ttf}eft, flush{\bfseries \setmainfont[Path=/usr/share/fonts/truetype/cmu/,UprightFont=cmunrm.ttf,BoldFont=cmunbx.ttf,ItalicFont=cmunti.ttf,BoldItalicFont=cmunbi.ttf]{cmunbx.ttf}\setmonofont[Path=/usr/share/fonts/truetype/cmu/,UprightFont=cmuntt.ttf,BoldFont=cmuntb.ttf,ItalicFont=cmunit.ttf,BoldItalicFont=cmuntx.ttf]{cmunbx.ttf}\bfseries r}\setmainfont[Path=/usr/share/fonts/truetype/cmu/,UprightFont=cmunrm.ttf,BoldFont=cmunbx.ttf,ItalicFont=cmunti.ttf,BoldItalicFont=cmunbi.ttf]{cmunrm.ttf}\setmonofont[Path=/usr/share/fonts/truetype/cmu/,UprightFont=cmuntt.ttf,BoldFont=cmuntb.ttf,ItalicFont=cmunit.ttf,BoldItalicFont=cmuntx.ttf]{cmunrm.ttf}ight, or {\bfseries \setmainfont[Path=/usr/share/fonts/truetype/cmu/,UprightFont=cmunrm.ttf,BoldFont=cmunbx.ttf,ItalicFont=cmunti.ttf,BoldItalicFont=cmunbi.ttf]{cmunbx.ttf}\setmonofont[Path=/usr/share/fonts/truetype/cmu/,UprightFont=cmuntt.ttf,BoldFont=cmuntb.ttf,ItalicFont=cmunit.ttf,BoldItalicFont=cmuntx.ttf]{cmunbx.ttf}\bfseries s}\setmainfont[Path=/usr/share/fonts/truetype/cmu/,UprightFont=cmunrm.ttf,BoldFont=cmunbx.ttf,ItalicFont=cmunti.ttf,BoldItalicFont=cmunbi.ttf]{cmunrm.ttf}\setmonofont[Path=/usr/share/fonts/truetype/cmu/,UprightFont=cmuntt.ttf,BoldFont=cmuntb.ttf,ItalicFont=cmunit.ttf,BoldItalicFont=cmuntx.ttf]{cmunrm.ttf}pread the text to fill the box.

\begin{Shaded}
\begin{Highlighting}[]

\NormalTok{\textbackslash{}makebox[0pt]\{Some text\} over this text}
 
\NormalTok{\textbackslash{}makebox[15ex][s]\{Censored text\}\textbackslash{}hspace\{-15ex\}\textbackslash{}makebox[15ex][s]\{X X X X X\}}
 
\NormalTok{Text \textbackslash{}makebox[2\textbackslash{}width][r]\{running away\}}
\end{Highlighting}
\end{Shaded}

\section{framebox}
\label{481}

The command \LaTeXTT{\textbackslash{}framebox} works exactly the same as \LaTeXTT{\textbackslash{}makebox}, but it draws a box around the text.

\begin{Shaded}
\begin{Highlighting}[]

\NormalTok{\textbackslash{}fbox\{text\}}
\NormalTok{\textbackslash{}framebox[width][pos]\{text\}}
\end{Highlighting}
\end{Shaded}


The following example shows you some things you could do with the \LaTeXTT{\textbackslash{}makebox} and \LaTeXTT{\textbackslash{}framebox} commands:

\begin{longtable}{p{1.0\linewidth}}
\begin{Shaded}
\begin{Highlighting}[]

\NormalTok{\textbackslash{}makebox[\textbackslash{}textwidth]\{c e n t r a l\} \textbackslash{}par}
\NormalTok{\textbackslash{}makebox[\textbackslash{}textwidth][s]\{s p r e a d\} \textbackslash{}par}
\NormalTok{\textbackslash{}framebox[1.1\textbackslash{}width]\{Guess I’m framed now!\} \textbackslash{}par}
\NormalTok{\textbackslash{}framebox[0.8\textbackslash{}width][r]\{Bummer, I am too wide\} \textbackslash{}par}
\NormalTok{\textbackslash{}framebox[1cm][l]\{never mind, so am I\}}
\NormalTok{Can you read this?}
\end{Highlighting}
\end{Shaded}
\\



\begin{minipage}{0.75000\textwidth}
\begin{center}
\includegraphics[width=1.0\textwidth,height=6.5in,keepaspectratio]{../images/78.png}
\end{center}
\raggedright{}\myfigurewithoutcaption{78}
\end{minipage}\vspace{0.75cm}



\end{longtable}

You can tweak the following frame lengths.
\begin{myitemize}
\item{}  \LaTeXTT{\textbackslash{}fboxsep}: the distance between the frame and the content.
\item{}  \LaTeXTT{\textbackslash{}fboxrule}: the thickness of the rule.
\end{myitemize}


This prints a thick and more distant frame:
\begin{Shaded}
\begin{Highlighting}[]

\NormalTok{\textbackslash{}setlength\{\textbackslash{}fboxsep\}\{10pt\}}
\NormalTok{\textbackslash{}setlength\{\textbackslash{}fboxrule\}\{5pt\}}
\NormalTok{\textbackslash{}fbox\{A frame.\}}
\end{Highlighting}
\end{Shaded}


This shows the box frame of a letter.
\begin{Shaded}
\begin{Highlighting}[]

\NormalTok{\textbackslash{}setlength\{\textbackslash{}fboxsep\}\{0pt\}}
\NormalTok{\textbackslash{}fbox\{A\}}
\end{Highlighting}
\end{Shaded}

\section{framed}
\label{482}

An alternative to these approaches is the usage of the \LaTeXTT{framed} environment (you will need to include the \LaTeXTT{framed} package to use it). This provides an easy way to box a paragraph within a document:

\begin{Shaded}
\begin{Highlighting}[]

\NormalTok{\textbackslash{}usepackage\{framed\}}
\CommentTok{% ...}
 
\NormalTok{\textbackslash{}begin\{framed\}}
\NormalTok{This is an easy way to box text within a document!}
\NormalTok{\textbackslash{}end\{framed\}}
\end{Highlighting}
\end{Shaded}


You can do it manually with a \mylref{484}{parbox}.
\section{raisebox}
\label{483}

Now that we control the horizontal, the obvious next step is to go for the vertical. No problem for LaTeX. The

\begin{Shaded}
\begin{Highlighting}[]

\NormalTok{\textbackslash{}raisebox\{lift\}[height][depth]\{text\}}
\end{Highlighting}
\end{Shaded}


command lets you define the vertical properties of a box. You can use \LaTeXTT{\textbackslash{}width}, \LaTeXTT{\textbackslash{}height}, \LaTeXTT{\textbackslash{}depth} and \LaTeXTT{\textbackslash{}totalheight} in the first three parameters, in order to act upon the size of the box inside the text argument. The two optional parameters set for the height and depth of the raisebox. For instance you can observe the difference when embedded in a framebox.

\begin{longtable}{p{1.0\linewidth}}
\begin{Shaded}
\begin{Highlighting}[]

\NormalTok{\textbackslash{}raisebox\{0pt\}[0pt][0pt]\{\textbackslash{}Large}\CommentTok
    \NormalTok{\textbackslash{}raisebox\{-0.7ex\}\{aa\}}\CommentTok
    \NormalTok{\textbackslash{}raisebox\{-2.2ex\}\{g\}}\CommentTok{%}
    \NormalTok{\textbackslash{}raisebox\{-4.5ex\}\{h\}}
  \NormalTok{\}}
\NormalTok{\}}
\NormalTok{he shouted but not even the next}
\NormalTok{one in line noticed that something}
\NormalTok{terrible had happened to him.}
\end{Highlighting}
\end{Shaded}
\\



\begin{minipage}{0.75000\textwidth}
\begin{center}
\includegraphics[width=1.0\textwidth,height=6.5in,keepaspectratio]{../images/79.png}
\end{center}
\raggedright{}\myfigurewithoutcaption{79}
\end{minipage}\vspace{0.75cm}



\end{longtable}
\section{minipage and parbox}
\label{484}

Most standard LaTeX boxes are not {\itshape \setmainfont[Path=/usr/share/fonts/truetype/cmu/,UprightFont=cmunrm.ttf,BoldFont=cmunbx.ttf,ItalicFont=cmunti.ttf,BoldItalicFont=cmunbi.ttf]{cmunti.ttf}\setmonofont[Path=/usr/share/fonts/truetype/cmu/,UprightFont=cmuntt.ttf,BoldFont=cmuntb.ttf,ItalicFont=cmunit.ttf,BoldItalicFont=cmuntx.ttf]{cmunti.ttf}\itshape long}{$\text{ }$}\setmainfont[Path=/usr/share/fonts/truetype/cmu/,UprightFont=cmunrm.ttf,BoldFont=cmunbx.ttf,ItalicFont=cmunti.ttf,BoldItalicFont=cmunbi.ttf]{cmunrm.ttf}\setmonofont[Path=/usr/share/fonts/truetype/cmu/,UprightFont=cmuntt.ttf,BoldFont=cmuntb.ttf,ItalicFont=cmunit.ttf,BoldItalicFont=cmuntx.ttf]{cmunrm.ttf} commands, {\itshape \setmainfont[Path=/usr/share/fonts/truetype/cmu/,UprightFont=cmunrm.ttf,BoldFont=cmunbx.ttf,ItalicFont=cmunti.ttf,BoldItalicFont=cmunbi.ttf]{cmunti.ttf}\setmonofont[Path=/usr/share/fonts/truetype/cmu/,UprightFont=cmuntt.ttf,BoldFont=cmuntb.ttf,ItalicFont=cmunit.ttf,BoldItalicFont=cmuntx.ttf]{cmunti.ttf}\itshape i.e.}{$\text{ }$}\setmainfont[Path=/usr/share/fonts/truetype/cmu/,UprightFont=cmunrm.ttf,BoldFont=cmunbx.ttf,ItalicFont=cmunti.ttf,BoldItalicFont=cmunbi.ttf]{cmunrm.ttf}\setmonofont[Path=/usr/share/fonts/truetype/cmu/,UprightFont=cmuntt.ttf,BoldFont=cmuntb.ttf,ItalicFont=cmunit.ttf,BoldItalicFont=cmuntx.ttf]{cmunrm.ttf} they do not support breaks nor paragraphs.
However you can pack a paragraph of your choice into a box with either the \LaTeXTT{\textbackslash{}parbox{$\text{[}$}pos{$\text{]}$}{$\text{[}$}height{$\text{]}$}{$\text{[}$}contentpos{$\text{]}$}\{width\}\{text\}} command or the \LaTeXTT{\textbackslash{}begin\{minipage\}{$\text{[}$}pos{$\text{]}$}{$\text{[}$}height{$\text{]}$}{$\text{[}$}contentpos{$\text{]}$}\{width\} text \textbackslash{}end\{minipage\}} environment.

The \LaTeXTT{pos} parameter can take one of the letters {\bfseries \setmainfont[Path=/usr/share/fonts/truetype/cmu/,UprightFont=cmunrm.ttf,BoldFont=cmunbx.ttf,ItalicFont=cmunti.ttf,BoldItalicFont=cmunbi.ttf]{cmunbx.ttf}\setmonofont[Path=/usr/share/fonts/truetype/cmu/,UprightFont=cmuntt.ttf,BoldFont=cmuntb.ttf,ItalicFont=cmunit.ttf,BoldItalicFont=cmuntx.ttf]{cmunbx.ttf}\bfseries c}\setmainfont[Path=/usr/share/fonts/truetype/cmu/,UprightFont=cmunrm.ttf,BoldFont=cmunbx.ttf,ItalicFont=cmunti.ttf,BoldItalicFont=cmunbi.ttf]{cmunrm.ttf}\setmonofont[Path=/usr/share/fonts/truetype/cmu/,UprightFont=cmuntt.ttf,BoldFont=cmuntb.ttf,ItalicFont=cmunit.ttf,BoldItalicFont=cmuntx.ttf]{cmunrm.ttf}enter, {\bfseries \setmainfont[Path=/usr/share/fonts/truetype/cmu/,UprightFont=cmunrm.ttf,BoldFont=cmunbx.ttf,ItalicFont=cmunti.ttf,BoldItalicFont=cmunbi.ttf]{cmunbx.ttf}\setmonofont[Path=/usr/share/fonts/truetype/cmu/,UprightFont=cmuntt.ttf,BoldFont=cmuntb.ttf,ItalicFont=cmunit.ttf,BoldItalicFont=cmuntx.ttf]{cmunbx.ttf}\bfseries t}\setmainfont[Path=/usr/share/fonts/truetype/cmu/,UprightFont=cmunrm.ttf,BoldFont=cmunbx.ttf,ItalicFont=cmunti.ttf,BoldItalicFont=cmunbi.ttf]{cmunrm.ttf}\setmonofont[Path=/usr/share/fonts/truetype/cmu/,UprightFont=cmuntt.ttf,BoldFont=cmuntb.ttf,ItalicFont=cmunit.ttf,BoldItalicFont=cmuntx.ttf]{cmunrm.ttf}op or {\bfseries \setmainfont[Path=/usr/share/fonts/truetype/cmu/,UprightFont=cmunrm.ttf,BoldFont=cmunbx.ttf,ItalicFont=cmunti.ttf,BoldItalicFont=cmunbi.ttf]{cmunbx.ttf}\setmonofont[Path=/usr/share/fonts/truetype/cmu/,UprightFont=cmuntt.ttf,BoldFont=cmuntb.ttf,ItalicFont=cmunit.ttf,BoldItalicFont=cmuntx.ttf]{cmunbx.ttf}\bfseries b}\setmainfont[Path=/usr/share/fonts/truetype/cmu/,UprightFont=cmunrm.ttf,BoldFont=cmunbx.ttf,ItalicFont=cmunti.ttf,BoldItalicFont=cmunbi.ttf]{cmunrm.ttf}\setmonofont[Path=/usr/share/fonts/truetype/cmu/,UprightFont=cmuntt.ttf,BoldFont=cmuntb.ttf,ItalicFont=cmunit.ttf,BoldItalicFont=cmuntx.ttf]{cmunrm.ttf}ottom to control the vertical alignment of the box, relative to the baseline of the surrounding text.
The \LaTeXTT{height} parameter is the height of the parbox or minipage.
The \LaTeXTT{contentpos} parameter is the position of the content and can be one of {\bfseries \setmainfont[Path=/usr/share/fonts/truetype/cmu/,UprightFont=cmunrm.ttf,BoldFont=cmunbx.ttf,ItalicFont=cmunti.ttf,BoldItalicFont=cmunbi.ttf]{cmunbx.ttf}\setmonofont[Path=/usr/share/fonts/truetype/cmu/,UprightFont=cmuntt.ttf,BoldFont=cmuntb.ttf,ItalicFont=cmunit.ttf,BoldItalicFont=cmuntx.ttf]{cmunbx.ttf}\bfseries c}\setmainfont[Path=/usr/share/fonts/truetype/cmu/,UprightFont=cmunrm.ttf,BoldFont=cmunbx.ttf,ItalicFont=cmunti.ttf,BoldItalicFont=cmunbi.ttf]{cmunrm.ttf}\setmonofont[Path=/usr/share/fonts/truetype/cmu/,UprightFont=cmuntt.ttf,BoldFont=cmuntb.ttf,ItalicFont=cmunit.ttf,BoldItalicFont=cmuntx.ttf]{cmunrm.ttf}enter, {\bfseries \setmainfont[Path=/usr/share/fonts/truetype/cmu/,UprightFont=cmunrm.ttf,BoldFont=cmunbx.ttf,ItalicFont=cmunti.ttf,BoldItalicFont=cmunbi.ttf]{cmunbx.ttf}\setmonofont[Path=/usr/share/fonts/truetype/cmu/,UprightFont=cmuntt.ttf,BoldFont=cmuntb.ttf,ItalicFont=cmunit.ttf,BoldItalicFont=cmuntx.ttf]{cmunbx.ttf}\bfseries t}\setmainfont[Path=/usr/share/fonts/truetype/cmu/,UprightFont=cmunrm.ttf,BoldFont=cmunbx.ttf,ItalicFont=cmunti.ttf,BoldItalicFont=cmunbi.ttf]{cmunrm.ttf}\setmonofont[Path=/usr/share/fonts/truetype/cmu/,UprightFont=cmuntt.ttf,BoldFont=cmuntb.ttf,ItalicFont=cmunit.ttf,BoldItalicFont=cmuntx.ttf]{cmunrm.ttf}op,  {\bfseries \setmainfont[Path=/usr/share/fonts/truetype/cmu/,UprightFont=cmunrm.ttf,BoldFont=cmunbx.ttf,ItalicFont=cmunti.ttf,BoldItalicFont=cmunbi.ttf]{cmunbx.ttf}\setmonofont[Path=/usr/share/fonts/truetype/cmu/,UprightFont=cmuntt.ttf,BoldFont=cmuntb.ttf,ItalicFont=cmunit.ttf,BoldItalicFont=cmuntx.ttf]{cmunbx.ttf}\bfseries b}\setmainfont[Path=/usr/share/fonts/truetype/cmu/,UprightFont=cmunrm.ttf,BoldFont=cmunbx.ttf,ItalicFont=cmunti.ttf,BoldItalicFont=cmunbi.ttf]{cmunrm.ttf}\setmonofont[Path=/usr/share/fonts/truetype/cmu/,UprightFont=cmuntt.ttf,BoldFont=cmuntb.ttf,ItalicFont=cmunit.ttf,BoldItalicFont=cmuntx.ttf]{cmunrm.ttf}ottom or {\bfseries \setmainfont[Path=/usr/share/fonts/truetype/cmu/,UprightFont=cmunrm.ttf,BoldFont=cmunbx.ttf,ItalicFont=cmunti.ttf,BoldItalicFont=cmunbi.ttf]{cmunbx.ttf}\setmonofont[Path=/usr/share/fonts/truetype/cmu/,UprightFont=cmuntt.ttf,BoldFont=cmuntb.ttf,ItalicFont=cmunit.ttf,BoldItalicFont=cmuntx.ttf]{cmunbx.ttf}\bfseries s}\setmainfont[Path=/usr/share/fonts/truetype/cmu/,UprightFont=cmunrm.ttf,BoldFont=cmunbx.ttf,ItalicFont=cmunti.ttf,BoldItalicFont=cmunbi.ttf]{cmunrm.ttf}\setmonofont[Path=/usr/share/fonts/truetype/cmu/,UprightFont=cmuntt.ttf,BoldFont=cmuntb.ttf,ItalicFont=cmunit.ttf,BoldItalicFont=cmuntx.ttf]{cmunrm.ttf}pread.
\LaTeXTT{width} takes a length argument specifying the width of the box. The main difference between a \LaTeXTT{minipage} and a \LaTeXTT{\textbackslash{}parbox} is that you cannot use all commands and environments inside a parbox, while almost anything is possible in a minipage.

\begin{Shaded}
\begin{Highlighting}[]

\NormalTok{\textbackslash{}noindent}
\NormalTok{\textbackslash{}fbox\{\textbackslash{}parbox[b][4em][t]\{0.33\textbackslash{}textwidth\}\{Some \textbackslash{}\textbackslash{} text\} \}}
\NormalTok{\textbackslash{}fbox\{\textbackslash{}parbox[c][4em][s]\{0.33\textbackslash{}textwidth\}\{Some \textbackslash{}vfill text\} \}}
\NormalTok{\textbackslash{}fbox\{\textbackslash{}parbox[t][4em][c]\{0.33\textbackslash{}textwidth\}\{Some \textbackslash{}\textbackslash{} text\} \}}
\end{Highlighting}
\end{Shaded}


This should print 3 boxes on the same line. Do not put another linebreak between the \LaTeXTT{\textbackslash{}fbox}, otherwise you will put the following \LaTeXTT{\textbackslash{}fbox} in another paragraph on another line.
\subsection{Paragraphs in all boxes}
\label{485}
You can make use of the {\itshape \setmainfont[Path=/usr/share/fonts/truetype/cmu/,UprightFont=cmunrm.ttf,BoldFont=cmunbx.ttf,ItalicFont=cmunti.ttf,BoldItalicFont=cmunbi.ttf]{cmunti.ttf}\setmonofont[Path=/usr/share/fonts/truetype/cmu/,UprightFont=cmuntt.ttf,BoldFont=cmuntb.ttf,ItalicFont=cmunit.ttf,BoldItalicFont=cmuntx.ttf]{cmunti.ttf}\itshape long}{$\text{ }$}\setmainfont[Path=/usr/share/fonts/truetype/cmu/,UprightFont=cmunrm.ttf,BoldFont=cmunbx.ttf,ItalicFont=cmunti.ttf,BoldItalicFont=cmunbi.ttf]{cmunrm.ttf}\setmonofont[Path=/usr/share/fonts/truetype/cmu/,UprightFont=cmuntt.ttf,BoldFont=cmuntb.ttf,ItalicFont=cmunit.ttf,BoldItalicFont=cmuntx.ttf]{cmunrm.ttf} capabilities of minipage and parbox to embed paragraphs in non-{}long boxes. For instance:
\begin{Shaded}
\begin{Highlighting}[]

\NormalTok{\textbackslash{}fbox\{}
  \NormalTok{\textbackslash{}parbox\{\textbackslash{}textwidth\}\{}
    \NormalTok{Some very long text...}
  \NormalTok{\}}
\NormalTok{\}}
\end{Highlighting}
\end{Shaded}


This prevents the overfull badness.

You can also use \begin{Shaded}
\begin{Highlighting}[]
\NormalTok{\textbackslash{}pbox\{\textbackslash{}textwidth\}\{my text\}}
\end{Highlighting}
\end{Shaded}
 from the pbox package which will create a box of minimal size around the text. Note that the \textbackslash{}pbox command takes an optional argument that specifies the vertical position of the text:
\begin{Shaded}
\begin{Highlighting}[]
\NormalTok{\textbackslash{}pbox[b]\{\textbackslash{}textwidth\}\{my text\}}
\end{Highlighting}
\end{Shaded}

The valid values are b (bottom), t (top), and c (center). If you specify a length in the first (required) argument, the text will be wrapped:
\begin{Shaded}
\begin{Highlighting}[]
\NormalTok{\textbackslash{}pbox[b]\{5cm\}\{This is long text that will be wrapped once it reaches five}
 \NormalTok{centimeters.\}}
\end{Highlighting}
\end{Shaded}

\section{savebox}
\label{486}

A \LaTeXTT{\textbackslash{}savebox} is a reference to a box filled with contents. You can use it as a way to print or manipulate something repeatedly.

\begin{Shaded}
\begin{Highlighting}[]

\NormalTok{\textbackslash{}newsavebox\{\textbackslash{}boxname\}}
\NormalTok{\textbackslash{}savebox\{\textbackslash{}boxname\}\{some content\}}
\NormalTok{\textbackslash{}usebox\{\textbackslash{}boxname\}}
\end{Highlighting}
\end{Shaded}


The command \LaTeXTT{\textbackslash{}newsavebox} creates a placeholder for storing a text;
the command \LaTeXTT{\textbackslash{}savebox} stores the specified text in this placeholder, and does not display anything in the document; and \LaTeXTT{\textbackslash{}usebox} recalls the content of the placeholder into the document.
\section{rotatebox}
\label{487}

See \mylref{243}{Rotations}.
\section{colorbox and fcolorbox}
\label{488}

See \mylref{147}{Colors}.
\LaTeXTT{\textbackslash{}fcolorbox} can also be tweaked with \LaTeXTT{\textbackslash{}fboxsep} and \LaTeXTT{\textbackslash{}fboxrule}.
\section{resizebox and scalebox}
\label{489}
The \LaTeXTT{graphicx} package features additional boxes.

\begin{Shaded}
\begin{Highlighting}[]

\NormalTok{\textbackslash{}resizebox\{10ex\}\{2\textbackslash{}baselineskip\}\{Dunhill style\}}
\NormalTok{\textbackslash{}scalebox\{10\}\{Giant\}}
\end{Highlighting}
\end{Shaded}

\section{fancybox}
\label{490}

the \LaTeXTT{fancybox} package provides additional boxes.
\begin{myitemize}
\item{}  \LaTeXTT{\textbackslash{}doublebox}
\item{}  \LaTeXTT{\textbackslash{}ovalbox}
\item{}  \LaTeXTT{\textbackslash{}shadowbox}
\end{myitemize}


\chapter{Rules and Struts}

\myminitoc
\label{491}

\label{492}

\section{Rules}
\label{493}

The \LaTeXTT{\textbackslash{}rule} command in normal use produces a simple black box:
\begin{Shaded}
\begin{Highlighting}[]

\NormalTok{\textbackslash{}rule[raise]\{width\}\{thickness\}}
\end{Highlighting}
\end{Shaded}



The parameter \LaTeXTT{thickness} determines the height, whereas \LaTeXTT{width} determines the width  of the produced rule. With the optional parameter \LaTeXTT{raise}, you can optionally raise or lower the produced rule above or below the baseline. 

Here is an example (the thin lines are located at the baseline):
\begin{longtable}{p{1.0\linewidth}}
\begin{Shaded}
\begin{Highlighting}[]

\NormalTok{\textbackslash{}rule\{3mm\}\{.1pt\}}\CommentTok
\NormalTok{\textbackslash{}rule\{3mm\}\{.1pt\}}\CommentTok
\NormalTok{\textbackslash{}rule\{3mm\}\{.1pt\}}
\end{Highlighting}
\end{Shaded}
\\



\begin{minipage}{0.75000\textwidth}
\begin{center}
\includegraphics[width=1.0\textwidth,height=6.5in,keepaspectratio]{../images/80.png}
\end{center}
\raggedright{}\myfigurewithoutcaption{80}
\end{minipage}\vspace{0.75cm}



\end{longtable}

This is useful for drawing vertical and horizontal lines.
\section{Struts}
\label{494}

A special case is a rule with no width but a certain height. In professional typesetting, this is called a {\itshape \setmainfont[Path=/usr/share/fonts/truetype/cmu/,UprightFont=cmunrm.ttf,BoldFont=cmunbx.ttf,ItalicFont=cmunti.ttf,BoldItalicFont=cmunbi.ttf]{cmunti.ttf}\setmonofont[Path=/usr/share/fonts/truetype/cmu/,UprightFont=cmuntt.ttf,BoldFont=cmuntb.ttf,ItalicFont=cmunit.ttf,BoldItalicFont=cmuntx.ttf]{cmunti.ttf}\itshape strut}\setmainfont[Path=/usr/share/fonts/truetype/cmu/,UprightFont=cmunrm.ttf,BoldFont=cmunbx.ttf,ItalicFont=cmunti.ttf,BoldItalicFont=cmunbi.ttf]{cmunrm.ttf}\setmonofont[Path=/usr/share/fonts/truetype/cmu/,UprightFont=cmuntt.ttf,BoldFont=cmuntb.ttf,ItalicFont=cmunit.ttf,BoldItalicFont=cmuntx.ttf]{cmunrm.ttf}. It is used to guarantee that an element on a page has a certain minimal height. You could use it in a tabular environment or in boxes to make sure a row has a certain minimum height.

In LaTeX a strut is defined as
\begin{Shaded}
\begin{Highlighting}[]

\NormalTok{\textbackslash{}rule[-.3\textbackslash{}baselineskip]\{0pt\}\{\textbackslash{}baselineskip\}}
\end{Highlighting}
\end{Shaded}

\section{Stretched rules}
\label{495}

LaTeX provides the \LaTeXTT{\textbackslash{}hrulefill} command, which work like a stretched horizontal space.
See the \mylref{456}{Lengths} chapter.


\mypart{Technical Texts}\chapter{Mathematics}

\myminitoc
\label{496}

\label{497}


One of the greatest motivating forces for Donald Knuth when he began developing the original TeX system was to create something that allowed simple construction of mathematical formulae, while looking professional when printed. The fact that he succeeded was most probably why TeX (and later on, LaTeX) became so popular within the scientific community. Typesetting mathematics is one of LaTeX\textquotesingle{}s greatest strengths. It is also a large topic due to the existence of so much mathematical notation.

If your document requires only a few simple mathematical formulas, plain LaTeX has most of the tools that you will need. If you are writing a scientific document that contains numerous complicated formulas, the \LaTeXTT{amsmath} package\myfootnote{\myplainurl{http://www.ams.org/publications/authors/tex/amslatex}} introduces several new commands that are more powerful and flexible than the ones provided by LaTeX. The \LaTeXTT{mathtools} package fixes some \LaTeXTT{amsmath} quirks and adds some useful settings, symbols, and environments to amsmath.\myfootnote{\myplainurl{http://www.ctan.org/tex-archive/macros/latex/contrib/mathtools/mathtools.pdf}} To use either package, include:

\begin{Shaded}
\begin{Highlighting}[]

\NormalTok{\textbackslash{}usepackage\{amsmath\}}\newline
\end{Highlighting}
\end{Shaded}

or

\begin{Shaded}
\begin{Highlighting}[]

\NormalTok{\textbackslash{}usepackage\{mathtools\}}\newline
\end{Highlighting}
\end{Shaded}

in the preamble of the document. The \LaTeXTT{mathtools} package loads the \LaTeXTT{amsmath} package and hence there is no need to {\ttfamily \setmainfont[Path=/usr/share/fonts/truetype/cmu/,UprightFont=cmunrm.ttf,BoldFont=cmunbx.ttf,ItalicFont=cmunti.ttf,BoldItalicFont=cmunbi.ttf]{cmuntt.ttf}\setmonofont[Path=/usr/share/fonts/truetype/cmu/,UprightFont=cmuntt.ttf,BoldFont=cmuntb.ttf,ItalicFont=cmunit.ttf,BoldItalicFont=cmuntx.ttf]{cmuntt.ttf}\ttfamily \textbackslash{}usepackage\{amsmath\}}{$\text{ }$}\setmainfont[Path=/usr/share/fonts/truetype/cmu/,UprightFont=cmunrm.ttf,BoldFont=cmunbx.ttf,ItalicFont=cmunti.ttf,BoldItalicFont=cmunbi.ttf]{cmunrm.ttf}\setmonofont[Path=/usr/share/fonts/truetype/cmu/,UprightFont=cmuntt.ttf,BoldFont=cmuntb.ttf,ItalicFont=cmunit.ttf,BoldItalicFont=cmuntx.ttf]{cmunrm.ttf} in the preamble if \LaTeXTT{mathtools} is used.
\section{Mathematics environments}
\label{498}

LaTeX needs to know beforehand that the subsequent text does indeed contain mathematical elements. This is because LaTeX typesets maths notation differently from normal text. Therefore, special environments have been declared for this purpose. They can be distinguished into two categories depending on how they are presented:

\begin{myitemize}
\item{}  {\itshape \setmainfont[Path=/usr/share/fonts/truetype/cmu/,UprightFont=cmunrm.ttf,BoldFont=cmunbx.ttf,ItalicFont=cmunti.ttf,BoldItalicFont=cmunbi.ttf]{cmunti.ttf}\setmonofont[Path=/usr/share/fonts/truetype/cmu/,UprightFont=cmuntt.ttf,BoldFont=cmuntb.ttf,ItalicFont=cmunit.ttf,BoldItalicFont=cmuntx.ttf]{cmunti.ttf}\itshape text}{$\text{ }$}\setmainfont[Path=/usr/share/fonts/truetype/cmu/,UprightFont=cmunrm.ttf,BoldFont=cmunbx.ttf,ItalicFont=cmunti.ttf,BoldItalicFont=cmunbi.ttf]{cmunrm.ttf}\setmonofont[Path=/usr/share/fonts/truetype/cmu/,UprightFont=cmuntt.ttf,BoldFont=cmuntb.ttf,ItalicFont=cmunit.ttf,BoldItalicFont=cmuntx.ttf]{cmunrm.ttf} {\mbox{$\text{---}$}} text formulas are displayed inline, that is, within the body of text where it is declared, for example, I can say that {\itshape \setmainfont[Path=/usr/share/fonts/truetype/cmu/,UprightFont=cmunrm.ttf,BoldFont=cmunbx.ttf,ItalicFont=cmunti.ttf,BoldItalicFont=cmunbi.ttf]{cmunti.ttf}\setmonofont[Path=/usr/share/fonts/truetype/cmu/,UprightFont=cmuntt.ttf,BoldFont=cmuntb.ttf,ItalicFont=cmunit.ttf,BoldItalicFont=cmuntx.ttf]{cmunti.ttf}\itshape a}{$\text{ }$}\setmainfont[Path=/usr/share/fonts/truetype/cmu/,UprightFont=cmunrm.ttf,BoldFont=cmunbx.ttf,ItalicFont=cmunti.ttf,BoldItalicFont=cmunbi.ttf]{cmunrm.ttf}\setmonofont[Path=/usr/share/fonts/truetype/cmu/,UprightFont=cmuntt.ttf,BoldFont=cmuntb.ttf,ItalicFont=cmunit.ttf,BoldItalicFont=cmuntx.ttf]{cmunrm.ttf} + {\itshape \setmainfont[Path=/usr/share/fonts/truetype/cmu/,UprightFont=cmunrm.ttf,BoldFont=cmunbx.ttf,ItalicFont=cmunti.ttf,BoldItalicFont=cmunbi.ttf]{cmunti.ttf}\setmonofont[Path=/usr/share/fonts/truetype/cmu/,UprightFont=cmuntt.ttf,BoldFont=cmuntb.ttf,ItalicFont=cmunit.ttf,BoldItalicFont=cmuntx.ttf]{cmunti.ttf}\itshape a}{$\text{ }$}\setmainfont[Path=/usr/share/fonts/truetype/cmu/,UprightFont=cmunrm.ttf,BoldFont=cmunbx.ttf,ItalicFont=cmunti.ttf,BoldItalicFont=cmunbi.ttf]{cmunrm.ttf}\setmonofont[Path=/usr/share/fonts/truetype/cmu/,UprightFont=cmuntt.ttf,BoldFont=cmuntb.ttf,ItalicFont=cmunit.ttf,BoldItalicFont=cmuntx.ttf]{cmunrm.ttf} = 2{\itshape \setmainfont[Path=/usr/share/fonts/truetype/cmu/,UprightFont=cmunrm.ttf,BoldFont=cmunbx.ttf,ItalicFont=cmunti.ttf,BoldItalicFont=cmunbi.ttf]{cmunti.ttf}\setmonofont[Path=/usr/share/fonts/truetype/cmu/,UprightFont=cmuntt.ttf,BoldFont=cmuntb.ttf,ItalicFont=cmunit.ttf,BoldItalicFont=cmuntx.ttf]{cmunti.ttf}\itshape a}{$\text{ }$}\setmainfont[Path=/usr/share/fonts/truetype/cmu/,UprightFont=cmunrm.ttf,BoldFont=cmunbx.ttf,ItalicFont=cmunti.ttf,BoldItalicFont=cmunbi.ttf]{cmunrm.ttf}\setmonofont[Path=/usr/share/fonts/truetype/cmu/,UprightFont=cmuntt.ttf,BoldFont=cmuntb.ttf,ItalicFont=cmunit.ttf,BoldItalicFont=cmuntx.ttf]{cmunrm.ttf} within this sentence.
\item{}  {\itshape \setmainfont[Path=/usr/share/fonts/truetype/cmu/,UprightFont=cmunrm.ttf,BoldFont=cmunbx.ttf,ItalicFont=cmunti.ttf,BoldItalicFont=cmunbi.ttf]{cmunti.ttf}\setmonofont[Path=/usr/share/fonts/truetype/cmu/,UprightFont=cmuntt.ttf,BoldFont=cmuntb.ttf,ItalicFont=cmunit.ttf,BoldItalicFont=cmuntx.ttf]{cmunti.ttf}\itshape displayed}{$\text{ }$}\setmainfont[Path=/usr/share/fonts/truetype/cmu/,UprightFont=cmunrm.ttf,BoldFont=cmunbx.ttf,ItalicFont=cmunti.ttf,BoldItalicFont=cmunbi.ttf]{cmunrm.ttf}\setmonofont[Path=/usr/share/fonts/truetype/cmu/,UprightFont=cmuntt.ttf,BoldFont=cmuntb.ttf,ItalicFont=cmunit.ttf,BoldItalicFont=cmuntx.ttf]{cmunrm.ttf} {\mbox{$\text{---}$}} displayed formulas are separate from the main text.
\end{myitemize}


As math requires special environments, there are naturally the appropriate environment names you can use in the standard way. Unlike most other environments, however, there are some handy shorthands to declaring your formulas. The following table summarizes them:
{\scriptsize{}
\begin{longtable}{|>{\RaggedRight}p{0.20783\linewidth}|>{\RaggedRight}p{0.22366\linewidth}|>{\RaggedRight}p{0.23057\linewidth}|>{\RaggedRight}p{0.22366\linewidth}|} \hline 
{\bfseries \hspace*{0pt}\ignorespaces{}\hspace*{0pt}Type}&{\bfseries \hspace*{0pt}\ignorespaces{}\hspace*{0pt}Inline (within text) formulas}&{\bfseries \hspace*{0pt}\ignorespaces{}\hspace*{0pt}Displayed equations}&{\bfseries \hspace*{0pt}\ignorespaces{}\hspace*{0pt}Displayed and automatically numbered equations}\\ \hline {\bfseries \hspace*{0pt}\ignorespaces{}\hspace*{0pt}Environment}&\hspace*{0pt}\ignorespaces{}\hspace*{0pt}\LaTeXTT{math}&\hspace*{0pt}\ignorespaces{}\hspace*{0pt}\LaTeXTT{displaymath}&\hspace*{0pt}\ignorespaces{}\hspace*{0pt}\LaTeXTT{equation}\\ \hline {\bfseries \hspace*{0pt}\ignorespaces{}\hspace*{0pt}LaTeX shorthand}&\hspace*{0pt}\ignorespaces{}\hspace*{0pt}\LaTeXTT{\textbackslash{}(...\textbackslash{})}&\hspace*{0pt}\ignorespaces{}\hspace*{0pt}\LaTeXTT{\textbackslash{}{$\text{[}$}...\textbackslash{}{$\text{]}$}}&\hspace*{0pt}\ignorespaces{}\hspace*{0pt}\\ \hline {\bfseries \hspace*{0pt}\ignorespaces{}\hspace*{0pt}TeX shorthand}&\hspace*{0pt}\ignorespaces{}\hspace*{0pt}\LaTeXTT{\${}...\${}}&\hspace*{0pt}\ignorespaces{}\hspace*{0pt}\LaTeXTT{\${}\${}...\${}\${}}&\hspace*{0pt}\ignorespaces{}\hspace*{0pt}\\ \hline {\bfseries \hspace*{0pt}\ignorespaces{}\hspace*{0pt}Comment}&\hspace*{0pt}\ignorespaces{}\hspace*{0pt}&\hspace*{0pt}\ignorespaces{}\hspace*{0pt}&\hspace*{0pt}\ignorespaces{}\hspace*{0pt}\LaTeXTT{equation*} (starred version) suppresses numbering, but requires amsmath\\ \hline 
\end{longtable}
}
{\bfseries \setmainfont[Path=/usr/share/fonts/truetype/cmu/,UprightFont=cmunrm.ttf,BoldFont=cmunbx.ttf,ItalicFont=cmunti.ttf,BoldItalicFont=cmunbi.ttf]{cmunbx.ttf}\setmonofont[Path=/usr/share/fonts/truetype/cmu/,UprightFont=cmuntt.ttf,BoldFont=cmuntb.ttf,ItalicFont=cmunit.ttf,BoldItalicFont=cmuntx.ttf]{cmunbx.ttf}\bfseries Suggestion}\setmainfont[Path=/usr/share/fonts/truetype/cmu/,UprightFont=cmunrm.ttf,BoldFont=cmunbx.ttf,ItalicFont=cmunti.ttf,BoldItalicFont=cmunbi.ttf]{cmunrm.ttf}\setmonofont[Path=/usr/share/fonts/truetype/cmu/,UprightFont=cmuntt.ttf,BoldFont=cmuntb.ttf,ItalicFont=cmunit.ttf,BoldItalicFont=cmuntx.ttf]{cmunrm.ttf}: Using the \LaTeXTT{\${}\${}...\${}\${}} should be avoided, as it may cause problems, particularly with the AMS-{}LaTeX macros. Furthermore, should a problem occur, the error messages may not be helpful.

The \LaTeXTT{equation*} and \LaTeXTT{displaymath} environments are functionally equivalent.

If you are typing text normally, you are said to be in {\itshape \setmainfont[Path=/usr/share/fonts/truetype/cmu/,UprightFont=cmunrm.ttf,BoldFont=cmunbx.ttf,ItalicFont=cmunti.ttf,BoldItalicFont=cmunbi.ttf]{cmunti.ttf}\setmonofont[Path=/usr/share/fonts/truetype/cmu/,UprightFont=cmuntt.ttf,BoldFont=cmuntb.ttf,ItalicFont=cmunit.ttf,BoldItalicFont=cmuntx.ttf]{cmunti.ttf}\itshape text mode}\setmainfont[Path=/usr/share/fonts/truetype/cmu/,UprightFont=cmunrm.ttf,BoldFont=cmunbx.ttf,ItalicFont=cmunti.ttf,BoldItalicFont=cmunbi.ttf]{cmunrm.ttf}\setmonofont[Path=/usr/share/fonts/truetype/cmu/,UprightFont=cmuntt.ttf,BoldFont=cmuntb.ttf,ItalicFont=cmunit.ttf,BoldItalicFont=cmuntx.ttf]{cmunrm.ttf}, but while you are typing within one of those mathematical environments, you are said to be in {\itshape \setmainfont[Path=/usr/share/fonts/truetype/cmu/,UprightFont=cmunrm.ttf,BoldFont=cmunbx.ttf,ItalicFont=cmunti.ttf,BoldItalicFont=cmunbi.ttf]{cmunti.ttf}\setmonofont[Path=/usr/share/fonts/truetype/cmu/,UprightFont=cmuntt.ttf,BoldFont=cmuntb.ttf,ItalicFont=cmunit.ttf,BoldItalicFont=cmuntx.ttf]{cmunti.ttf}\itshape math mode}\setmainfont[Path=/usr/share/fonts/truetype/cmu/,UprightFont=cmunrm.ttf,BoldFont=cmunbx.ttf,ItalicFont=cmunti.ttf,BoldItalicFont=cmunbi.ttf]{cmunrm.ttf}\setmonofont[Path=/usr/share/fonts/truetype/cmu/,UprightFont=cmuntt.ttf,BoldFont=cmuntb.ttf,ItalicFont=cmunit.ttf,BoldItalicFont=cmuntx.ttf]{cmunrm.ttf}, that has some differences compared to the {\itshape \setmainfont[Path=/usr/share/fonts/truetype/cmu/,UprightFont=cmunrm.ttf,BoldFont=cmunbx.ttf,ItalicFont=cmunti.ttf,BoldItalicFont=cmunbi.ttf]{cmunti.ttf}\setmonofont[Path=/usr/share/fonts/truetype/cmu/,UprightFont=cmuntt.ttf,BoldFont=cmuntb.ttf,ItalicFont=cmunit.ttf,BoldItalicFont=cmuntx.ttf]{cmunti.ttf}\itshape text mode}\setmainfont[Path=/usr/share/fonts/truetype/cmu/,UprightFont=cmunrm.ttf,BoldFont=cmunbx.ttf,ItalicFont=cmunti.ttf,BoldItalicFont=cmunbi.ttf]{cmunrm.ttf}\setmonofont[Path=/usr/share/fonts/truetype/cmu/,UprightFont=cmuntt.ttf,BoldFont=cmuntb.ttf,ItalicFont=cmunit.ttf,BoldItalicFont=cmuntx.ttf]{cmunrm.ttf}:
\begin{myenumerate}
\item{}  Most spaces and line breaks do not have any significance, as all spaces are either derived logically from the mathematical expressions, or have to be specified with special commands such as \LaTeXTT{\textbackslash{}quad}
\item{}  Empty lines are not allowed. Only one paragraph per formula.
\item{}  Each letter is considered to be the name of a variable and will be typeset as such. If you want to typeset normal text within a formula (normal upright font and normal spacing) then you have to enter the text using \mylref{515}{dedicated commands.}
\end{myenumerate}

\subsection{Inserting \symbol{34}Displayed\symbol{34} maths inside blocks of text}
\label{499} 
In order for some operators, such as \LaTeXTT{\textbackslash{}lim} or \LaTeXTT{\textbackslash{}sum} to be displayed correctly inside some math environments (read \LaTeXTT{\${}......\${}}), it might be convenient to write the \LaTeXTT{\textbackslash{}displaystyle} class inside the environment. Doing so might cause the line to be taller, but will cause exponents and indices to be displayed correctly for some math operators. For example, the \LaTeXTT{\${}\textbackslash{}sum\${}} will print a smaller Σ and \LaTeXTT{\${}\textbackslash{}displaystyle \textbackslash{}sum\${}} will print a bigger one {$\displaystyle \sum$}, like in equations (This only works with AMSMATH package). It is also possible to force this behaviour for all math environments by declaring \LaTeXTT{\textbackslash{}everymath\{\textbackslash{}displaystyle\}} at the very beginning (i.e. before \LaTeXTT{\textbackslash{}begin\{document\}}), which is useful in longer documents.
\section{Symbols}
\label{500}

Mathematics has many symbols! One of the most difficult aspects of learning LaTeX is remembering how to produce symbols. There is of course a set of symbols that can be accessed directly from the keyboard:\\

\TemplateSpaceIndent{$\text{ }${}+$\text{ }${}-{}$\text{ }${}=$\text{ }${}!$\text{ }${}/$\text{ }${}($\text{ }${})$\text{ }${}{$\text{[}$}$\text{ }${}{$\text{]}$}$\text{ }${}<{}$\text{ }${}>{}$\text{ }${}|$\text{ }${}\textquotesingle{}$\text{ }${}:}


Beyond those listed above, distinct commands must be issued in order to display the desired symbols. There are many examples such as Greek letters, set and relations symbols, arrows, binary operators, etc.

For example: 

\begin{longtable}{p{1.0\linewidth}}
\begin{Shaded}
\begin{Highlighting}[]

 \NormalTok{\textbackslash{}forall x \textbackslash{}in X, \textbackslash{}quad \textbackslash{}exists y \textbackslash{}leq \textbackslash{}epsilon}
 
\end{Highlighting}
\end{Shaded}
\\
{$ \forall x \in X, \quad \exists y \leq \epsilon  \,$}
\end{longtable}

Fortunately, there\textquotesingle{}s a tool that can greatly simplify the search for the command for a specific symbol. Look for \symbol{34}Detexify\symbol{34} in the \mylref{531}{external links} section below. Another option would be to look in the \symbol{34}The Comprehensive LaTeX Symbol List\symbol{34} in the \mylref{531}{external links} section below.
\section{Greek letters}
\label{501}

Greek letters are commonly used in mathematics, and they are very easy to type in {\itshape \setmainfont[Path=/usr/share/fonts/truetype/cmu/,UprightFont=cmunrm.ttf,BoldFont=cmunbx.ttf,ItalicFont=cmunti.ttf,BoldItalicFont=cmunbi.ttf]{cmunti.ttf}\setmonofont[Path=/usr/share/fonts/truetype/cmu/,UprightFont=cmuntt.ttf,BoldFont=cmuntb.ttf,ItalicFont=cmunit.ttf,BoldItalicFont=cmuntx.ttf]{cmunti.ttf}\itshape math mode}\setmainfont[Path=/usr/share/fonts/truetype/cmu/,UprightFont=cmunrm.ttf,BoldFont=cmunbx.ttf,ItalicFont=cmunti.ttf,BoldItalicFont=cmunbi.ttf]{cmunrm.ttf}\setmonofont[Path=/usr/share/fonts/truetype/cmu/,UprightFont=cmuntt.ttf,BoldFont=cmuntb.ttf,ItalicFont=cmunit.ttf,BoldItalicFont=cmuntx.ttf]{cmunrm.ttf}. You just have to type the name of the letter after a backslash: if the first letter is lowercase, you will get a lowercase Greek letter, if the first letter is uppercase (and only the first letter), then you will get an uppercase letter. Note that some uppercase Greek letters look like Latin ones, so they are not provided by LaTeX (e.g. uppercase {\itshape \setmainfont[Path=/usr/share/fonts/truetype/cmu/,UprightFont=cmunrm.ttf,BoldFont=cmunbx.ttf,ItalicFont=cmunti.ttf,BoldItalicFont=cmunbi.ttf]{cmunti.ttf}\setmonofont[Path=/usr/share/fonts/truetype/cmu/,UprightFont=cmuntt.ttf,BoldFont=cmuntb.ttf,ItalicFont=cmunit.ttf,BoldItalicFont=cmuntx.ttf]{cmunti.ttf}\itshape Alpha}{$\text{ }$}\setmainfont[Path=/usr/share/fonts/truetype/cmu/,UprightFont=cmunrm.ttf,BoldFont=cmunbx.ttf,ItalicFont=cmunti.ttf,BoldItalicFont=cmunbi.ttf]{cmunrm.ttf}\setmonofont[Path=/usr/share/fonts/truetype/cmu/,UprightFont=cmuntt.ttf,BoldFont=cmuntb.ttf,ItalicFont=cmunit.ttf,BoldItalicFont=cmuntx.ttf]{cmunrm.ttf} and {\itshape \setmainfont[Path=/usr/share/fonts/truetype/cmu/,UprightFont=cmunrm.ttf,BoldFont=cmunbx.ttf,ItalicFont=cmunti.ttf,BoldItalicFont=cmunbi.ttf]{cmunti.ttf}\setmonofont[Path=/usr/share/fonts/truetype/cmu/,UprightFont=cmuntt.ttf,BoldFont=cmuntb.ttf,ItalicFont=cmunit.ttf,BoldItalicFont=cmuntx.ttf]{cmunti.ttf}\itshape Beta}{$\text{ }$}\setmainfont[Path=/usr/share/fonts/truetype/cmu/,UprightFont=cmunrm.ttf,BoldFont=cmunbx.ttf,ItalicFont=cmunti.ttf,BoldItalicFont=cmunbi.ttf]{cmunrm.ttf}\setmonofont[Path=/usr/share/fonts/truetype/cmu/,UprightFont=cmuntt.ttf,BoldFont=cmuntb.ttf,ItalicFont=cmunit.ttf,BoldItalicFont=cmuntx.ttf]{cmunrm.ttf} are just \symbol{34}A\symbol{34} and \symbol{34}B\symbol{34} respectively).
Lowercase epsilon, theta, kappa, phi, pi, rho, and sigma are provided in two different versions.  The alternate, or {\itshape \setmainfont[Path=/usr/share/fonts/truetype/cmu/,UprightFont=cmunrm.ttf,BoldFont=cmunbx.ttf,ItalicFont=cmunti.ttf,BoldItalicFont=cmunbi.ttf]{cmunti.ttf}\setmonofont[Path=/usr/share/fonts/truetype/cmu/,UprightFont=cmuntt.ttf,BoldFont=cmuntb.ttf,ItalicFont=cmunit.ttf,BoldItalicFont=cmuntx.ttf]{cmunti.ttf}\itshape var}\setmainfont[Path=/usr/share/fonts/truetype/cmu/,UprightFont=cmunrm.ttf,BoldFont=cmunbx.ttf,ItalicFont=cmunti.ttf,BoldItalicFont=cmunbi.ttf]{cmunrm.ttf}\setmonofont[Path=/usr/share/fonts/truetype/cmu/,UprightFont=cmuntt.ttf,BoldFont=cmuntb.ttf,ItalicFont=cmunit.ttf,BoldItalicFont=cmuntx.ttf]{cmunrm.ttf}iant, version is created by adding \symbol{34}var\symbol{34} before the name of the letter:

\begin{longtable}{p{1.0\linewidth}}
\begin{Shaded}
\begin{Highlighting}[]

\NormalTok{\textbackslash{}alpha, \textbackslash{}Alpha, \textbackslash{}beta, \textbackslash{}Beta, \textbackslash{}gamma, \textbackslash{}Gamma, \textbackslash{}pi, \textbackslash{}Pi, \textbackslash{}phi, \textbackslash{}varphi, \textbackslash{}mu, \textbackslash{}Phi}
   
 
 
\end{Highlighting}
\end{Shaded}
\\
{$\alpha, \Alpha, \beta, \Beta, \gamma, \Gamma, \pi, \Pi, \phi, \varphi, \mu, \Phi$}
\end{longtable}

Scroll down to \mylref{527}{\#List of Mathematical Symbols} for a complete list of Greek symbols.
\section{Operators}
\label{502}
An operator is a function that is written as a word: e.g. trigonometric functions (sin, cos, tan), logarithms and exponentials (log, exp), limits (lim), as well as trace and determinant (tr, det). LaTeX has many of these defined as commands:
\begin{longtable}{p{1.0\linewidth}}
\begin{Shaded}
\begin{Highlighting}[]

 \NormalTok{\textbackslash{}cos (2\textbackslash{}theta) = \textbackslash{}cos^2 \textbackslash{}theta - \textbackslash{}sin^2 \textbackslash{}theta}
 
\end{Highlighting}
\end{Shaded}
\\
{$\cos (2\theta) = \cos^2 \theta - \sin^2 \theta \,$}

\end{longtable}

For certain operators such as \myhref{https://en.wikipedia.org/wiki/Limit\%20\%28mathematics\%29}{limits}, the subscript is placed underneath the operator:
\begin{longtable}{p{1.0\linewidth}}
\begin{Shaded}
\begin{Highlighting}[]

 \NormalTok{\textbackslash{}lim_\{x \textbackslash{}to \textbackslash{}infty\} \textbackslash{}exp(-x) = 0}
 
\end{Highlighting}
\end{Shaded}
\\
{$\lim_{x \to \infty} \exp(-x) = 0$}
\end{longtable}

For the \myhref{https://en.wikipedia.org/wiki/Modular\%20arithmetic}{modular operator} there are two commands: \LaTeXTT{\textbackslash{}bmod} and \LaTeXTT{\textbackslash{}pmod}:
\begin{longtable}{p{1.0\linewidth}}
\begin{Shaded}
\begin{Highlighting}[]

 \NormalTok{a \textbackslash{}bmod b}
 
\end{Highlighting}
\end{Shaded}
\\
{$  a \, \bmod \, b \,$}
\end{longtable}
\begin{longtable}{p{1.0\linewidth}}
\begin{Shaded}
\begin{Highlighting}[]

 \NormalTok{x \textbackslash{}equiv a \textbackslash{}pmod b}
 
\end{Highlighting}
\end{Shaded}
\\
{$  x \equiv a \pmod b \,$}
\end{longtable}

To use operators that are not pre-{}defined, such as \myhref{https://en.wikipedia.org/wiki/argmax}{argmax}, see \mylref{546}{custom operators}
\section{Powers and indices}
\label{503}
Powers and indices are equivalent to superscripts and subscripts in normal text mode. The caret ({\ttfamily \setmainfont[Path=/usr/share/fonts/truetype/cmu/,UprightFont=cmunrm.ttf,BoldFont=cmunbx.ttf,ItalicFont=cmunti.ttf,BoldItalicFont=cmunbi.ttf]{cmuntt.ttf}\setmonofont[Path=/usr/share/fonts/truetype/cmu/,UprightFont=cmuntt.ttf,BoldFont=cmuntb.ttf,ItalicFont=cmunit.ttf,BoldItalicFont=cmuntx.ttf]{cmuntt.ttf}\ttfamily \^{}}\setmainfont[Path=/usr/share/fonts/truetype/cmu/,UprightFont=cmunrm.ttf,BoldFont=cmunbx.ttf,ItalicFont=cmunti.ttf,BoldItalicFont=cmunbi.ttf]{cmunrm.ttf}\setmonofont[Path=/usr/share/fonts/truetype/cmu/,UprightFont=cmuntt.ttf,BoldFont=cmuntb.ttf,ItalicFont=cmunit.ttf,BoldItalicFont=cmuntx.ttf]{cmunrm.ttf}; \myhref{https://en.wikipedia.org/wiki/Caret}{also known as the circumflex accent}) character is used to raise something, and the underscore ({\ttfamily \setmainfont[Path=/usr/share/fonts/truetype/cmu/,UprightFont=cmunrm.ttf,BoldFont=cmunbx.ttf,ItalicFont=cmunti.ttf,BoldItalicFont=cmunbi.ttf]{cmuntt.ttf}\setmonofont[Path=/usr/share/fonts/truetype/cmu/,UprightFont=cmuntt.ttf,BoldFont=cmuntb.ttf,ItalicFont=cmunit.ttf,BoldItalicFont=cmuntx.ttf]{cmuntt.ttf}\ttfamily \_}\setmainfont[Path=/usr/share/fonts/truetype/cmu/,UprightFont=cmunrm.ttf,BoldFont=cmunbx.ttf,ItalicFont=cmunti.ttf,BoldItalicFont=cmunbi.ttf]{cmunrm.ttf}\setmonofont[Path=/usr/share/fonts/truetype/cmu/,UprightFont=cmuntt.ttf,BoldFont=cmuntb.ttf,ItalicFont=cmunit.ttf,BoldItalicFont=cmuntx.ttf]{cmunrm.ttf}) is for lowering. If more than one expression is raised or lowered, they should be grouped using curly braces ({\ttfamily \setmainfont[Path=/usr/share/fonts/truetype/cmu/,UprightFont=cmunrm.ttf,BoldFont=cmunbx.ttf,ItalicFont=cmunti.ttf,BoldItalicFont=cmunbi.ttf]{cmuntt.ttf}\setmonofont[Path=/usr/share/fonts/truetype/cmu/,UprightFont=cmuntt.ttf,BoldFont=cmuntb.ttf,ItalicFont=cmunit.ttf,BoldItalicFont=cmuntx.ttf]{cmuntt.ttf}\ttfamily \{}{$\text{ }$}\setmainfont[Path=/usr/share/fonts/truetype/cmu/,UprightFont=cmunrm.ttf,BoldFont=cmunbx.ttf,ItalicFont=cmunti.ttf,BoldItalicFont=cmunbi.ttf]{cmunrm.ttf}\setmonofont[Path=/usr/share/fonts/truetype/cmu/,UprightFont=cmuntt.ttf,BoldFont=cmuntb.ttf,ItalicFont=cmunit.ttf,BoldItalicFont=cmuntx.ttf]{cmunrm.ttf} and {\ttfamily \setmainfont[Path=/usr/share/fonts/truetype/cmu/,UprightFont=cmunrm.ttf,BoldFont=cmunbx.ttf,ItalicFont=cmunti.ttf,BoldItalicFont=cmunbi.ttf]{cmuntt.ttf}\setmonofont[Path=/usr/share/fonts/truetype/cmu/,UprightFont=cmuntt.ttf,BoldFont=cmuntb.ttf,ItalicFont=cmunit.ttf,BoldItalicFont=cmuntx.ttf]{cmuntt.ttf}\ttfamily \}}\setmainfont[Path=/usr/share/fonts/truetype/cmu/,UprightFont=cmunrm.ttf,BoldFont=cmunbx.ttf,ItalicFont=cmunti.ttf,BoldItalicFont=cmunbi.ttf]{cmunrm.ttf}\setmonofont[Path=/usr/share/fonts/truetype/cmu/,UprightFont=cmuntt.ttf,BoldFont=cmuntb.ttf,ItalicFont=cmunit.ttf,BoldItalicFont=cmuntx.ttf]{cmunrm.ttf}).
\begin{longtable}{p{1.0\linewidth}}
\begin{Shaded}
\begin{Highlighting}[]

 \NormalTok{k_\{n+1\} = n^2 + k_n^2 - k_\{n-1\}}
 
\end{Highlighting}
\end{Shaded}
\\
{$ k_{n+1} = n^2 + k_n^2 - k_{n-1}  \,$}
\end{longtable}

For powers with more than one digit, surround the power with \{\}.
\begin{longtable}{p{1.0\linewidth}}
\begin{Shaded}
\begin{Highlighting}[]

 \NormalTok{n^\{22\}}
\end{Highlighting}
\end{Shaded}
\\
{$ n^{22}  \,$}
\end{longtable}

An underscore ({\ttfamily \setmainfont[Path=/usr/share/fonts/truetype/cmu/,UprightFont=cmunrm.ttf,BoldFont=cmunbx.ttf,ItalicFont=cmunti.ttf,BoldItalicFont=cmunbi.ttf]{cmuntt.ttf}\setmonofont[Path=/usr/share/fonts/truetype/cmu/,UprightFont=cmuntt.ttf,BoldFont=cmuntb.ttf,ItalicFont=cmunit.ttf,BoldItalicFont=cmuntx.ttf]{cmuntt.ttf}\ttfamily \_}\setmainfont[Path=/usr/share/fonts/truetype/cmu/,UprightFont=cmunrm.ttf,BoldFont=cmunbx.ttf,ItalicFont=cmunti.ttf,BoldItalicFont=cmunbi.ttf]{cmunrm.ttf}\setmonofont[Path=/usr/share/fonts/truetype/cmu/,UprightFont=cmuntt.ttf,BoldFont=cmuntb.ttf,ItalicFont=cmunit.ttf,BoldItalicFont=cmuntx.ttf]{cmunrm.ttf}) can be used with a vertical bar ({$ | $}) to denote evaluation using subscript notation in mathematics:

\begin{longtable}{p{1.0\linewidth}}
\begin{Shaded}
\begin{Highlighting}[]

 \NormalTok{f(n) = n^5 + 4n^2 + 2 _\{n=17\}}
 
\end{Highlighting}
\end{Shaded}
\\
{$  f(n) = n^5 + 4n^2 + 2 |_{n=17}  \,$}
\end{longtable}
\section{Fractions and Binomials}
\label{504}
A fraction is created using the \LaTeXTT{\textbackslash{}frac\{numerator\}\{denominator\}} command. (for those who need their memories refreshed, that\textquotesingle{}s the {\itshape \setmainfont[Path=/usr/share/fonts/truetype/cmu/,UprightFont=cmunrm.ttf,BoldFont=cmunbx.ttf,ItalicFont=cmunti.ttf,BoldItalicFont=cmunbi.ttf]{cmunti.ttf}\setmonofont[Path=/usr/share/fonts/truetype/cmu/,UprightFont=cmuntt.ttf,BoldFont=cmuntb.ttf,ItalicFont=cmunit.ttf,BoldItalicFont=cmuntx.ttf]{cmunti.ttf}\itshape top}{$\text{ }$}\setmainfont[Path=/usr/share/fonts/truetype/cmu/,UprightFont=cmunrm.ttf,BoldFont=cmunbx.ttf,ItalicFont=cmunti.ttf,BoldItalicFont=cmunbi.ttf]{cmunrm.ttf}\setmonofont[Path=/usr/share/fonts/truetype/cmu/,UprightFont=cmuntt.ttf,BoldFont=cmuntb.ttf,ItalicFont=cmunit.ttf,BoldItalicFont=cmuntx.ttf]{cmunrm.ttf} and {\itshape \setmainfont[Path=/usr/share/fonts/truetype/cmu/,UprightFont=cmunrm.ttf,BoldFont=cmunbx.ttf,ItalicFont=cmunti.ttf,BoldItalicFont=cmunbi.ttf]{cmunti.ttf}\setmonofont[Path=/usr/share/fonts/truetype/cmu/,UprightFont=cmuntt.ttf,BoldFont=cmuntb.ttf,ItalicFont=cmunit.ttf,BoldItalicFont=cmuntx.ttf]{cmunti.ttf}\itshape bottom}{$\text{ }$}\setmainfont[Path=/usr/share/fonts/truetype/cmu/,UprightFont=cmunrm.ttf,BoldFont=cmunbx.ttf,ItalicFont=cmunti.ttf,BoldItalicFont=cmunbi.ttf]{cmunrm.ttf}\setmonofont[Path=/usr/share/fonts/truetype/cmu/,UprightFont=cmuntt.ttf,BoldFont=cmuntb.ttf,ItalicFont=cmunit.ttf,BoldItalicFont=cmuntx.ttf]{cmunrm.ttf} respectively!). Likewise, the \myhref{https://en.wikipedia.org/wiki/Binomial\%20coefficient}{binomial coefficient} (aka the Choose function) may be written using the \LaTeXTT{\textbackslash{}binom} command\myfootnote{requires the \LaTeXTT{amsmath} package}:
\begin{longtable}{p{1.0\linewidth}}
\begin{Shaded}
\begin{Highlighting}[]


 \NormalTok{\textbackslash{}frac\{n!\}\{k!(n-k)!\} = \textbackslash{}binom\{n\}\{k\}}
 
\end{Highlighting}
\end{Shaded}
\\

{$ \frac{n!}{k!(n-k)!} = \binom{n}{k} $}

\end{longtable}

It is also possible to use the \LaTeXTT{\textbackslash{}choose} command without the \LaTeXTT{amsmath} package:
\begin{longtable}{p{1.0\linewidth}}
\begin{Shaded}
\begin{Highlighting}[]


 \NormalTok{\textbackslash{}frac\{n!\}\{k!(n-k)!\} = \{n \textbackslash{}choose k\}}
 
\end{Highlighting}
\end{Shaded}
\\

{$  \frac{n!}{k!(n-k)!} = {n \choose k} $}

\end{longtable}

You can embed fractions within fractions:
\begin{longtable}{p{1.0\linewidth}}
\begin{Shaded}
\begin{Highlighting}[]


 \NormalTok{\textbackslash{}frac\{\textbackslash{}frac\{1\}\{x\}+\textbackslash{}frac\{1\}\{y\}<!---->\}\{y-z\}}
 
\end{Highlighting}
\end{Shaded}
\\

{$ \frac{\frac{1}{x}+\frac{1}{y}}{y-z} $}

\end{longtable}

Note that when appearing inside another fraction, or in inline text {$\tfrac{a}{b}$}, a fraction is noticeably smaller than in displayed mathematics. The \LaTeXTT{\textbackslash{}tfrac} and \LaTeXTT{\textbackslash{}dfrac} commands\myfootnote{requires the \LaTeXTT{amsmath} package} force the use of the respective styles, \LaTeXTT{\textbackslash{}textstyle} and \LaTeXTT{\textbackslash{}displaystyle}. Similarly, the \LaTeXTT{\textbackslash{}tbinom} and \LaTeXTT{\textbackslash{}dbinom} commands typeset the binomial coefficient.

Another way to write fractions is to use the \LaTeXTT{\textbackslash{}over} command without the \LaTeXTT{amsmath} package:
\begin{longtable}{p{1.0\linewidth}}
\begin{Shaded}
\begin{Highlighting}[]


 \NormalTok{\{n! \textbackslash{}over k!(n-k)!\} = \{n \textbackslash{}choose k\}}
 
\end{Highlighting}
\end{Shaded}
\\

{$  {n! \over k!(n-k)!} = {n \choose k} $}

\end{longtable}

For relatively simple fractions, especially within the text, it may be more aesthetically pleasing to use \mylref{503}{powers and indices}:
\begin{longtable}{p{1.0\linewidth}}
\begin{Shaded}
\begin{Highlighting}[]


 \NormalTok{^3/_7}
 
\end{Highlighting}
\end{Shaded}
\\
{$ ^3/_7  \,$}

\end{longtable}

If this looks a little \symbol{34}loose\symbol{34} (overspaced), a tightened version can be defined by inserting some negative space
\begin{longtable}{p{1.0\linewidth}}
\begin{Shaded}
\begin{Highlighting}[]

 
\CommentTok{%running fraction with slash - requires math mode.}
\NormalTok{\textbackslash{}newcommand*\textbackslash{}rfrac[2]}
 
\NormalTok{\textbackslash{}rfrac\{3\}\{7\}}
\end{Highlighting}
\end{Shaded}
\\
{$ {{}^{3}/_{7}} $}

\end{longtable}

If you use them throughout the document, usage of \LaTeXTT{xfrac} package is recommended.
This package provides \LaTeXTT{\textbackslash{}sfrac} command to create slanted fractions. Usage:
\begin{longtable}{p{1.0\linewidth}}
\begin{Shaded}
\begin{Highlighting}[]

\NormalTok{Take $\textbackslash{}sfrac\{1\}\{2\}$ cup of sugar, \textbackslash{}dots}
 
  \NormalTok{3\textbackslash{}times\textbackslash{}sfrac\{1\}\{2\}=1\textbackslash{}sfrac\{1\}\{2\}}
 
 
\NormalTok{Take $\{\}^1/_2$ cup of sugar, \textbackslash{}dots}
 
  \NormalTok{3\textbackslash{}times\{\}^1/_2=1\{\}^1/_2}
 
\end{Highlighting}
\end{Shaded}
\\


\begin{minipage}{1.0\linewidth}
\begin{center}
\includegraphics[width=1.0\linewidth,height=6.5in,keepaspectratio]{../images/81.png}
\end{center}
\raggedright{}\myfigurewithoutcaption{81}
\end{minipage}\vspace{0.75cm}


\end{longtable}

If fractions are used as an exponent curly braces have to be used around the \LaTeXTT{\textbackslash{}sfrac} command:

  \${}x\^{}\textbackslash{}frac\{1\}\{2\}\${} \% no error
  \${}x\^{}\textbackslash{}sfrac\{1\}\{2\}\${} \% error
  \${}x\^{}\{\textbackslash{}sfrac\{1\}\{2\}\}\${} \% no error

\begin{longtable}{p{1.0\linewidth}}
\begin{Shaded}
\begin{Highlighting}[]

  \NormalTok{$x^\textbackslash{}frac\{1\}\{2\}$ }\CommentTok{% no error}
\end{Highlighting}
\end{Shaded}
\\

{$ x^\frac{1}{2} $}

\end{longtable}

In some cases, using the package alone will result in errors about certain font shapes not being available. In that case, the \LaTeXTT{lmodern} and \LaTeXTT{fix-{}cm} packages need to be added as well.



Alternatively, the \LaTeXTT{nicefrac} package provides the \LaTeXTT{\textbackslash{}nicefrac} command, whose usage is similar to \LaTeXTT{\textbackslash{}sfrac}.
\subsection{Continued fractions}
\label{505}
Continued fractions should be written using \LaTeXTT{\textbackslash{}cfrac} command\myfootnote{requires the \LaTeXTT{amsmath} package}:
\begin{longtable}{p{1.0\linewidth}}
\begin{Shaded}
\begin{Highlighting}[]

\NormalTok{\textbackslash{}begin\{equation\}}
  \NormalTok{x = a_0 + \textbackslash{}cfrac\{1\}\{a_1 }
          \NormalTok{+ \textbackslash{}cfrac\{1\}\{a_2 }
          \NormalTok{+ \textbackslash{}cfrac\{1\}\{a_3 + \textbackslash{}cfrac\{1\}\{a_4\} \} \} \}}
\NormalTok{\textbackslash{}end\{equation\}}
\end{Highlighting}
\end{Shaded}
\\

{$   x = a_0 + \cfrac{1}{a_1             + \cfrac{1}{a_2             + \cfrac{1}{a_3 + \cfrac{1}{a_4}}}} $}

\end{longtable}
\subsection{Multiplication of two numbers}
\label{506}
To make multiplication visually similar to a fraction, a nested array can be used, for example multiplication of numbers written one below the other.
\begin{longtable}{p{1.0\linewidth}}
\begin{Shaded}
\begin{Highlighting}[]

\NormalTok{\textbackslash{}begin\{equation\}}
\NormalTok{\textbackslash{}frac\{}
    \NormalTok{\textbackslash{}begin\{array\}[b]\{r\}}
      \NormalTok{\textbackslash{}left( x_1 x_2 \textbackslash{}right)\textbackslash{}\textbackslash{}}
      \NormalTok{\textbackslash{}times \textbackslash{}left( x'_1 x'_2 \textbackslash{}right)}
    \NormalTok{\textbackslash{}end\{array\}}
  \NormalTok{\}\{}
    \NormalTok{\textbackslash{}left( y_1y_2y_3y_4 \textbackslash{}right)}
  \NormalTok{\}}
\NormalTok{\textbackslash{}end\{equation\}}
\end{Highlighting}
\end{Shaded}
\\

{$\frac{     \begin{array}[b]{r}       \left( x_1 x_2 \right)\\       \times \left( x'_1 x'_2 \right)     \end{array}   }{     \left( y_1y_2y_3y_4 \right)   } $}

\end{longtable}
\section{Roots}
\label{507}
The \LaTeXTT{\textbackslash{}sqrt} command creates a square root surrounding an expression. It accepts an optional argument specified in square brackets ({\ttfamily \setmainfont[Path=/usr/share/fonts/truetype/cmu/,UprightFont=cmunrm.ttf,BoldFont=cmunbx.ttf,ItalicFont=cmunti.ttf,BoldItalicFont=cmunbi.ttf]{cmuntt.ttf}\setmonofont[Path=/usr/share/fonts/truetype/cmu/,UprightFont=cmuntt.ttf,BoldFont=cmuntb.ttf,ItalicFont=cmunit.ttf,BoldItalicFont=cmuntx.ttf]{cmuntt.ttf}\ttfamily {$\text{[}$}}{$\text{ }$}\setmainfont[Path=/usr/share/fonts/truetype/cmu/,UprightFont=cmunrm.ttf,BoldFont=cmunbx.ttf,ItalicFont=cmunti.ttf,BoldItalicFont=cmunbi.ttf]{cmunrm.ttf}\setmonofont[Path=/usr/share/fonts/truetype/cmu/,UprightFont=cmuntt.ttf,BoldFont=cmuntb.ttf,ItalicFont=cmunit.ttf,BoldItalicFont=cmuntx.ttf]{cmunrm.ttf} and {\ttfamily \setmainfont[Path=/usr/share/fonts/truetype/cmu/,UprightFont=cmunrm.ttf,BoldFont=cmunbx.ttf,ItalicFont=cmunti.ttf,BoldItalicFont=cmunbi.ttf]{cmuntt.ttf}\setmonofont[Path=/usr/share/fonts/truetype/cmu/,UprightFont=cmuntt.ttf,BoldFont=cmuntb.ttf,ItalicFont=cmunit.ttf,BoldItalicFont=cmuntx.ttf]{cmuntt.ttf}\ttfamily {$\text{]}$}}\setmainfont[Path=/usr/share/fonts/truetype/cmu/,UprightFont=cmunrm.ttf,BoldFont=cmunbx.ttf,ItalicFont=cmunti.ttf,BoldItalicFont=cmunbi.ttf]{cmunrm.ttf}\setmonofont[Path=/usr/share/fonts/truetype/cmu/,UprightFont=cmuntt.ttf,BoldFont=cmuntb.ttf,ItalicFont=cmunit.ttf,BoldItalicFont=cmuntx.ttf]{cmunrm.ttf}) to change magnitude:
\begin{longtable}{p{1.0\linewidth}}
\begin{Shaded}
\begin{Highlighting}[]

 
\NormalTok{\textbackslash{}sqrt\{\textbackslash{}frac\{a\}\{b\}\}}
 
\end{Highlighting}
\end{Shaded}
\\

{$ \sqrt{\frac{a}{b}} $}

\end{longtable}
\begin{longtable}{p{1.0\linewidth}}
\begin{Shaded}
\begin{Highlighting}[]

 
\NormalTok{\textbackslash{}sqrt[n]\{1+x+x^2+x^3+\textbackslash{}ldots\}}
 
\end{Highlighting}
\end{Shaded}
\\
{$ \sqrt[n]{1+x+x^2+x^3+\ldots} $}

\end{longtable}


Some people prefer writing the square root \symbol{34}closing\symbol{34} it over its content. This method arguably makes it more clear what is in the scope of the root sign. This habit is not normally used while writing with the computer, but if you still want to change the output of the square root, LaTeX gives you this possibility. Just add the following code in the preamble of your document:
\begin{longtable}{p{1.0\linewidth}}
\begin{Shaded}
\begin{Highlighting}[]

\CommentTok{% New definition of square root:}
\CommentTok{% it renames \textbackslash{}sqrt as \textbackslash{}oldsqrt}
\NormalTok{\textbackslash{}let\textbackslash{}oldsqrt\textbackslash{}sqrt}
\CommentTok
\NormalTok{\textbackslash{}setbox0=\textbackslash{}hbox\{$#1\textbackslash{}oldsqrt\{#2\textbackslash{},\}$\}\textbackslash{}dimen0=\textbackslash{}ht0}
\NormalTok{\textbackslash{}advance\textbackslash{}dimen0-0.2\textbackslash{}ht0}
\NormalTok{\textbackslash{}setbox2=\textbackslash{}hbox\{\textbackslash{}vrule height\textbackslash{}ht0 depth -\textbackslash{}dimen0\}}\CommentTok{%}
\NormalTok{\{\textbackslash{}box0\textbackslash{}lower0.4pt\textbackslash{}box2\}\}}
\end{Highlighting}
\end{Shaded}
\\


\begin{minipage}{0.62500\textwidth}
\begin{center}
\includegraphics[width=1.0\textwidth,height=6.5in,keepaspectratio]{../images/82.png}
\end{center}
\raggedright{}\myfigurewithcaption{82}{The new style is on left, the old one on right}
\end{minipage}\vspace{0.75cm}



\end{longtable}
This TeX code first renames the \LaTeXTT{\textbackslash{}sqrt} command as \LaTeXTT{\textbackslash{}oldsqrt}, then redefines \LaTeXTT{\textbackslash{}sqrt} in terms of the old one, adding something more. The new square root can be seen in the picture on the left, compared to the old one on the right. Unfortunately this code won\textquotesingle{}t work if you want to use multiple roots: if you try to write {$\sqrt[b]{a}$} as \LaTeXTT{\textbackslash{}sqrt{$\text{[}$}b{$\text{]}$}\{a\}} after you used the code above, you\textquotesingle{}ll just get a wrong output. In other words, you can redefine the square root this way only if you are not going to use multiple roots in the whole document.

An alternative piece of TeX code that does allow multiple roots is
\begin{longtable}{p{1.0\linewidth}}
\begin{Shaded}
\begin{Highlighting}[]

 
\NormalTok{\textbackslash{}usepackage\{letltxmacro\}}
\NormalTok{\textbackslash{}makeatletter}
\NormalTok{\textbackslash{}let\textbackslash{}oldr@@t\textbackslash{}r@@t}
\NormalTok{\textbackslash{}def\textbackslash{}r@@t#1#2\{}\CommentTok
\NormalTok{\{\textbackslash{}box0\textbackslash{}lower0.4pt\textbackslash{}box2\}\}}
\NormalTok{\textbackslash{}LetLtxMacro\{\textbackslash{}oldsqrt\}\{\textbackslash{}sqrt\}}
\NormalTok{\textbackslash{}renewcommand*\{\textbackslash{}sqrt\}[2][\textbackslash{} ]\{\textbackslash{}oldsqrt[#1]\{#2\} \}}
\NormalTok{\textbackslash{}makeatother}
 
 
\NormalTok{$\textbackslash{}sqrt[a]\{b\} \textbackslash{}quad \textbackslash{}oldsqrt[a]\{b\}$}
 
\end{Highlighting}
\end{Shaded}
\\


\begin{minipage}{1.0\linewidth}
\begin{center}
\includegraphics[width=1.0\linewidth,height=6.5in,keepaspectratio]{../images/83.png}
\end{center}
\raggedright{}\myfigurewithoutcaption{83}
\end{minipage}\vspace{0.75cm}



\end{longtable}
However this requires the \LaTeXTT{\textbackslash{}usepackage\{letltxmacro\}} package
\section{Sums and integrals}
\label{508}
The \LaTeXTT{\textbackslash{}sum} and \LaTeXTT{\textbackslash{}int} commands insert the sum and integral symbols respectively, with limits specified using the caret ({\ttfamily \setmainfont[Path=/usr/share/fonts/truetype/cmu/,UprightFont=cmunrm.ttf,BoldFont=cmunbx.ttf,ItalicFont=cmunti.ttf,BoldItalicFont=cmunbi.ttf]{cmuntt.ttf}\setmonofont[Path=/usr/share/fonts/truetype/cmu/,UprightFont=cmuntt.ttf,BoldFont=cmuntb.ttf,ItalicFont=cmunit.ttf,BoldItalicFont=cmuntx.ttf]{cmuntt.ttf}\ttfamily \^{}}\setmainfont[Path=/usr/share/fonts/truetype/cmu/,UprightFont=cmunrm.ttf,BoldFont=cmunbx.ttf,ItalicFont=cmunti.ttf,BoldItalicFont=cmunbi.ttf]{cmunrm.ttf}\setmonofont[Path=/usr/share/fonts/truetype/cmu/,UprightFont=cmuntt.ttf,BoldFont=cmuntb.ttf,ItalicFont=cmunit.ttf,BoldItalicFont=cmuntx.ttf]{cmunrm.ttf}) and underscore ({\ttfamily \setmainfont[Path=/usr/share/fonts/truetype/cmu/,UprightFont=cmunrm.ttf,BoldFont=cmunbx.ttf,ItalicFont=cmunti.ttf,BoldItalicFont=cmunbi.ttf]{cmuntt.ttf}\setmonofont[Path=/usr/share/fonts/truetype/cmu/,UprightFont=cmuntt.ttf,BoldFont=cmuntb.ttf,ItalicFont=cmunit.ttf,BoldItalicFont=cmuntx.ttf]{cmuntt.ttf}\ttfamily \_}\setmainfont[Path=/usr/share/fonts/truetype/cmu/,UprightFont=cmunrm.ttf,BoldFont=cmunbx.ttf,ItalicFont=cmunti.ttf,BoldItalicFont=cmunbi.ttf]{cmunrm.ttf}\setmonofont[Path=/usr/share/fonts/truetype/cmu/,UprightFont=cmuntt.ttf,BoldFont=cmuntb.ttf,ItalicFont=cmunit.ttf,BoldItalicFont=cmuntx.ttf]{cmunrm.ttf}).  The typical notation for sums is:
\begin{longtable}{p{1.0\linewidth}}
\begin{Shaded}
\begin{Highlighting}[]


 \NormalTok{\textbackslash{}sum_\{i=1\}^\{10\} t_i }
 
\end{Highlighting}
\end{Shaded}
\\
{$ \textstyle\sum_{i=1}^{10} t_i   \,$}

\end{longtable}
or
\begin{longtable}{p{1.0\linewidth}}
\begin{Shaded}
\begin{Highlighting}[]

 
\NormalTok{\textbackslash{}displaystyle\textbackslash{}sum_\{i=1\}^\{10\} t_i }
 
\end{Highlighting}
\end{Shaded}
\\
{$ \displaystyle\sum_{i=1}^{10} t_i \,$}

\end{longtable}


The limits for the integrals follow the same notation.  It\textquotesingle{}s also important to represent the integration variables with an upright d, which in math mode is obtained through the \textbackslash{}mathrm\{\} command, and with a small space separating it from the integrand, which is attained with the \textbackslash{}, command.
\begin{longtable}{p{1.0\linewidth}}
\begin{Shaded}
\begin{Highlighting}[]


 \NormalTok{\textbackslash{}int_0^\textbackslash{}infty \textbackslash{}mathrm\{e\}^\{-x\}\textbackslash{},\textbackslash{}mathrm\{d\}x}
 
\end{Highlighting}
\end{Shaded}
\\
{$ \int_0^\infty \mathrm{e}^{-x}\,\mathrm{d}x  \,$}

\end{longtable}

There are many other \symbol{34}big\symbol{34} commands which operate in a similar manner:{\scriptsize{}
\begin{longtable}{>{\RaggedRight}p{0.16814\linewidth}>{\RaggedRight}p{0.05780\linewidth}>{\RaggedRight}p{0.02708\linewidth}>{\RaggedRight}p{0.16814\linewidth}>{\RaggedRight}p{0.06856\linewidth}>{\RaggedRight}p{0.02708\linewidth}>{\RaggedRight}p{0.16814\linewidth}>{\RaggedRight}p{0.08650\linewidth}} 
\hspace*{0pt}\ignorespaces{}\hspace*{0pt} {\ttfamily \setmainfont[Path=/usr/share/fonts/truetype/cmu/,UprightFont=cmunrm.ttf,BoldFont=cmunbx.ttf,ItalicFont=cmunti.ttf,BoldItalicFont=cmunbi.ttf]{cmuntt.ttf}\setmonofont[Path=/usr/share/fonts/truetype/cmu/,UprightFont=cmuntt.ttf,BoldFont=cmuntb.ttf,ItalicFont=cmunit.ttf,BoldItalicFont=cmuntx.ttf]{cmuntt.ttf}\ttfamily \textbackslash{}sum}{$\text{ }$}\setmainfont[Path=/usr/share/fonts/truetype/cmu/,UprightFont=cmunrm.ttf,BoldFont=cmunbx.ttf,ItalicFont=cmunti.ttf,BoldItalicFont=cmunbi.ttf]{cmunrm.ttf}\setmonofont[Path=/usr/share/fonts/truetype/cmu/,UprightFont=cmuntt.ttf,BoldFont=cmuntb.ttf,ItalicFont=cmunit.ttf,BoldItalicFont=cmuntx.ttf]{cmunrm.ttf} &\hspace*{0pt}\ignorespaces{}\hspace*{0pt} {$\sum \,$} &\hspace*{0pt}\ignorespaces{}\hspace*{0pt}&\hspace*{0pt}\ignorespaces{}\hspace*{0pt} {\ttfamily \setmainfont[Path=/usr/share/fonts/truetype/cmu/,UprightFont=cmunrm.ttf,BoldFont=cmunbx.ttf,ItalicFont=cmunti.ttf,BoldItalicFont=cmunbi.ttf]{cmuntt.ttf}\setmonofont[Path=/usr/share/fonts/truetype/cmu/,UprightFont=cmuntt.ttf,BoldFont=cmuntb.ttf,ItalicFont=cmunit.ttf,BoldItalicFont=cmuntx.ttf]{cmuntt.ttf}\ttfamily \textbackslash{}prod}{$\text{ }$}\setmainfont[Path=/usr/share/fonts/truetype/cmu/,UprightFont=cmunrm.ttf,BoldFont=cmunbx.ttf,ItalicFont=cmunti.ttf,BoldItalicFont=cmunbi.ttf]{cmunrm.ttf}\setmonofont[Path=/usr/share/fonts/truetype/cmu/,UprightFont=cmuntt.ttf,BoldFont=cmuntb.ttf,ItalicFont=cmunit.ttf,BoldItalicFont=cmuntx.ttf]{cmunrm.ttf} &\hspace*{0pt}\ignorespaces{}\hspace*{0pt} {$\prod$} &\hspace*{0pt}\ignorespaces{}\hspace*{0pt}&\hspace*{0pt}\ignorespaces{}\hspace*{0pt} {\ttfamily \setmainfont[Path=/usr/share/fonts/truetype/cmu/,UprightFont=cmunrm.ttf,BoldFont=cmunbx.ttf,ItalicFont=cmunti.ttf,BoldItalicFont=cmunbi.ttf]{cmuntt.ttf}\setmonofont[Path=/usr/share/fonts/truetype/cmu/,UprightFont=cmuntt.ttf,BoldFont=cmuntb.ttf,ItalicFont=cmunit.ttf,BoldItalicFont=cmuntx.ttf]{cmuntt.ttf}\ttfamily \textbackslash{}coprod}{$\text{ }$}\setmainfont[Path=/usr/share/fonts/truetype/cmu/,UprightFont=cmunrm.ttf,BoldFont=cmunbx.ttf,ItalicFont=cmunti.ttf,BoldItalicFont=cmunbi.ttf]{cmunrm.ttf}\setmonofont[Path=/usr/share/fonts/truetype/cmu/,UprightFont=cmuntt.ttf,BoldFont=cmuntb.ttf,ItalicFont=cmunit.ttf,BoldItalicFont=cmuntx.ttf]{cmunrm.ttf} &\hspace*{0pt}\ignorespaces{}\hspace*{0pt} {$\coprod$} \\ \hspace*{0pt}\ignorespaces{}\hspace*{0pt} {\ttfamily \setmainfont[Path=/usr/share/fonts/truetype/cmu/,UprightFont=cmunrm.ttf,BoldFont=cmunbx.ttf,ItalicFont=cmunti.ttf,BoldItalicFont=cmunbi.ttf]{cmuntt.ttf}\setmonofont[Path=/usr/share/fonts/truetype/cmu/,UprightFont=cmuntt.ttf,BoldFont=cmuntb.ttf,ItalicFont=cmunit.ttf,BoldItalicFont=cmuntx.ttf]{cmuntt.ttf}\ttfamily \textbackslash{}bigoplus}{$\text{ }$}\setmainfont[Path=/usr/share/fonts/truetype/cmu/,UprightFont=cmunrm.ttf,BoldFont=cmunbx.ttf,ItalicFont=cmunti.ttf,BoldItalicFont=cmunbi.ttf]{cmunrm.ttf}\setmonofont[Path=/usr/share/fonts/truetype/cmu/,UprightFont=cmuntt.ttf,BoldFont=cmuntb.ttf,ItalicFont=cmunit.ttf,BoldItalicFont=cmuntx.ttf]{cmunrm.ttf} &\hspace*{0pt}\ignorespaces{}\hspace*{0pt} {$\bigoplus$} &\hspace*{0pt}\ignorespaces{}\hspace*{0pt}&\hspace*{0pt}\ignorespaces{}\hspace*{0pt} {\ttfamily \setmainfont[Path=/usr/share/fonts/truetype/cmu/,UprightFont=cmunrm.ttf,BoldFont=cmunbx.ttf,ItalicFont=cmunti.ttf,BoldItalicFont=cmunbi.ttf]{cmuntt.ttf}\setmonofont[Path=/usr/share/fonts/truetype/cmu/,UprightFont=cmuntt.ttf,BoldFont=cmuntb.ttf,ItalicFont=cmunit.ttf,BoldItalicFont=cmuntx.ttf]{cmuntt.ttf}\ttfamily \textbackslash{}bigotimes}{$\text{ }$}\setmainfont[Path=/usr/share/fonts/truetype/cmu/,UprightFont=cmunrm.ttf,BoldFont=cmunbx.ttf,ItalicFont=cmunti.ttf,BoldItalicFont=cmunbi.ttf]{cmunrm.ttf}\setmonofont[Path=/usr/share/fonts/truetype/cmu/,UprightFont=cmuntt.ttf,BoldFont=cmuntb.ttf,ItalicFont=cmunit.ttf,BoldItalicFont=cmuntx.ttf]{cmunrm.ttf} &\hspace*{0pt}\ignorespaces{}\hspace*{0pt} {$\bigotimes$} &\hspace*{0pt}\ignorespaces{}\hspace*{0pt}&\hspace*{0pt}\ignorespaces{}\hspace*{0pt} {\ttfamily \setmainfont[Path=/usr/share/fonts/truetype/cmu/,UprightFont=cmunrm.ttf,BoldFont=cmunbx.ttf,ItalicFont=cmunti.ttf,BoldItalicFont=cmunbi.ttf]{cmuntt.ttf}\setmonofont[Path=/usr/share/fonts/truetype/cmu/,UprightFont=cmuntt.ttf,BoldFont=cmuntb.ttf,ItalicFont=cmunit.ttf,BoldItalicFont=cmuntx.ttf]{cmuntt.ttf}\ttfamily \textbackslash{}bigodot}{$\text{ }$}\setmainfont[Path=/usr/share/fonts/truetype/cmu/,UprightFont=cmunrm.ttf,BoldFont=cmunbx.ttf,ItalicFont=cmunti.ttf,BoldItalicFont=cmunbi.ttf]{cmunrm.ttf}\setmonofont[Path=/usr/share/fonts/truetype/cmu/,UprightFont=cmuntt.ttf,BoldFont=cmuntb.ttf,ItalicFont=cmunit.ttf,BoldItalicFont=cmuntx.ttf]{cmunrm.ttf} &\hspace*{0pt}\ignorespaces{}\hspace*{0pt} {$\bigodot$} \\ \hspace*{0pt}\ignorespaces{}\hspace*{0pt} {\ttfamily \setmainfont[Path=/usr/share/fonts/truetype/cmu/,UprightFont=cmunrm.ttf,BoldFont=cmunbx.ttf,ItalicFont=cmunti.ttf,BoldItalicFont=cmunbi.ttf]{cmuntt.ttf}\setmonofont[Path=/usr/share/fonts/truetype/cmu/,UprightFont=cmuntt.ttf,BoldFont=cmuntb.ttf,ItalicFont=cmunit.ttf,BoldItalicFont=cmuntx.ttf]{cmuntt.ttf}\ttfamily \textbackslash{}bigcup}{$\text{ }$}\setmainfont[Path=/usr/share/fonts/truetype/cmu/,UprightFont=cmunrm.ttf,BoldFont=cmunbx.ttf,ItalicFont=cmunti.ttf,BoldItalicFont=cmunbi.ttf]{cmunrm.ttf}\setmonofont[Path=/usr/share/fonts/truetype/cmu/,UprightFont=cmuntt.ttf,BoldFont=cmuntb.ttf,ItalicFont=cmunit.ttf,BoldItalicFont=cmuntx.ttf]{cmunrm.ttf} &\hspace*{0pt}\ignorespaces{}\hspace*{0pt} {$\bigcup$} &\hspace*{0pt}\ignorespaces{}\hspace*{0pt}&\hspace*{0pt}\ignorespaces{}\hspace*{0pt} {\ttfamily \setmainfont[Path=/usr/share/fonts/truetype/cmu/,UprightFont=cmunrm.ttf,BoldFont=cmunbx.ttf,ItalicFont=cmunti.ttf,BoldItalicFont=cmunbi.ttf]{cmuntt.ttf}\setmonofont[Path=/usr/share/fonts/truetype/cmu/,UprightFont=cmuntt.ttf,BoldFont=cmuntb.ttf,ItalicFont=cmunit.ttf,BoldItalicFont=cmuntx.ttf]{cmuntt.ttf}\ttfamily \textbackslash{}bigcap}{$\text{ }$}\setmainfont[Path=/usr/share/fonts/truetype/cmu/,UprightFont=cmunrm.ttf,BoldFont=cmunbx.ttf,ItalicFont=cmunti.ttf,BoldItalicFont=cmunbi.ttf]{cmunrm.ttf}\setmonofont[Path=/usr/share/fonts/truetype/cmu/,UprightFont=cmuntt.ttf,BoldFont=cmuntb.ttf,ItalicFont=cmunit.ttf,BoldItalicFont=cmuntx.ttf]{cmunrm.ttf} &\hspace*{0pt}\ignorespaces{}\hspace*{0pt} {$\bigcap$} &\hspace*{0pt}\ignorespaces{}\hspace*{0pt}&\hspace*{0pt}\ignorespaces{}\hspace*{0pt} {\ttfamily \setmainfont[Path=/usr/share/fonts/truetype/cmu/,UprightFont=cmunrm.ttf,BoldFont=cmunbx.ttf,ItalicFont=cmunti.ttf,BoldItalicFont=cmunbi.ttf]{cmuntt.ttf}\setmonofont[Path=/usr/share/fonts/truetype/cmu/,UprightFont=cmuntt.ttf,BoldFont=cmuntb.ttf,ItalicFont=cmunit.ttf,BoldItalicFont=cmuntx.ttf]{cmuntt.ttf}\ttfamily \textbackslash{}biguplus}{$\text{ }$}\setmainfont[Path=/usr/share/fonts/truetype/cmu/,UprightFont=cmunrm.ttf,BoldFont=cmunbx.ttf,ItalicFont=cmunti.ttf,BoldItalicFont=cmunbi.ttf]{cmunrm.ttf}\setmonofont[Path=/usr/share/fonts/truetype/cmu/,UprightFont=cmuntt.ttf,BoldFont=cmuntb.ttf,ItalicFont=cmunit.ttf,BoldItalicFont=cmuntx.ttf]{cmunrm.ttf} &\hspace*{0pt}\ignorespaces{}\hspace*{0pt} {$\biguplus$} \\ \hspace*{0pt}\ignorespaces{}\hspace*{0pt} {\ttfamily \setmainfont[Path=/usr/share/fonts/truetype/cmu/,UprightFont=cmunrm.ttf,BoldFont=cmunbx.ttf,ItalicFont=cmunti.ttf,BoldItalicFont=cmunbi.ttf]{cmuntt.ttf}\setmonofont[Path=/usr/share/fonts/truetype/cmu/,UprightFont=cmuntt.ttf,BoldFont=cmuntb.ttf,ItalicFont=cmunit.ttf,BoldItalicFont=cmuntx.ttf]{cmuntt.ttf}\ttfamily \textbackslash{}bigsqcup}{$\text{ }$}\setmainfont[Path=/usr/share/fonts/truetype/cmu/,UprightFont=cmunrm.ttf,BoldFont=cmunbx.ttf,ItalicFont=cmunti.ttf,BoldItalicFont=cmunbi.ttf]{cmunrm.ttf}\setmonofont[Path=/usr/share/fonts/truetype/cmu/,UprightFont=cmuntt.ttf,BoldFont=cmuntb.ttf,ItalicFont=cmunit.ttf,BoldItalicFont=cmuntx.ttf]{cmunrm.ttf} &\hspace*{0pt}\ignorespaces{}\hspace*{0pt} {$\bigsqcup$} &\hspace*{0pt}\ignorespaces{}\hspace*{0pt}&\hspace*{0pt}\ignorespaces{}\hspace*{0pt} {\ttfamily \setmainfont[Path=/usr/share/fonts/truetype/cmu/,UprightFont=cmunrm.ttf,BoldFont=cmunbx.ttf,ItalicFont=cmunti.ttf,BoldItalicFont=cmunbi.ttf]{cmuntt.ttf}\setmonofont[Path=/usr/share/fonts/truetype/cmu/,UprightFont=cmuntt.ttf,BoldFont=cmuntb.ttf,ItalicFont=cmunit.ttf,BoldItalicFont=cmuntx.ttf]{cmuntt.ttf}\ttfamily \textbackslash{}bigvee}{$\text{ }$}\setmainfont[Path=/usr/share/fonts/truetype/cmu/,UprightFont=cmunrm.ttf,BoldFont=cmunbx.ttf,ItalicFont=cmunti.ttf,BoldItalicFont=cmunbi.ttf]{cmunrm.ttf}\setmonofont[Path=/usr/share/fonts/truetype/cmu/,UprightFont=cmuntt.ttf,BoldFont=cmuntb.ttf,ItalicFont=cmunit.ttf,BoldItalicFont=cmuntx.ttf]{cmunrm.ttf} &\hspace*{0pt}\ignorespaces{}\hspace*{0pt} {$\bigvee$} &\hspace*{0pt}\ignorespaces{}\hspace*{0pt}&\hspace*{0pt}\ignorespaces{}\hspace*{0pt} {\ttfamily \setmainfont[Path=/usr/share/fonts/truetype/cmu/,UprightFont=cmunrm.ttf,BoldFont=cmunbx.ttf,ItalicFont=cmunti.ttf,BoldItalicFont=cmunbi.ttf]{cmuntt.ttf}\setmonofont[Path=/usr/share/fonts/truetype/cmu/,UprightFont=cmuntt.ttf,BoldFont=cmuntb.ttf,ItalicFont=cmunit.ttf,BoldItalicFont=cmuntx.ttf]{cmuntt.ttf}\ttfamily \textbackslash{}bigwedge}{$\text{ }$}\setmainfont[Path=/usr/share/fonts/truetype/cmu/,UprightFont=cmunrm.ttf,BoldFont=cmunbx.ttf,ItalicFont=cmunti.ttf,BoldItalicFont=cmunbi.ttf]{cmunrm.ttf}\setmonofont[Path=/usr/share/fonts/truetype/cmu/,UprightFont=cmuntt.ttf,BoldFont=cmuntb.ttf,ItalicFont=cmunit.ttf,BoldItalicFont=cmuntx.ttf]{cmunrm.ttf} &\hspace*{0pt}\ignorespaces{}\hspace*{0pt} {$\bigwedge$} \\ \hspace*{0pt}\ignorespaces{}\hspace*{0pt} {\ttfamily \setmainfont[Path=/usr/share/fonts/truetype/cmu/,UprightFont=cmunrm.ttf,BoldFont=cmunbx.ttf,ItalicFont=cmunti.ttf,BoldItalicFont=cmunbi.ttf]{cmuntt.ttf}\setmonofont[Path=/usr/share/fonts/truetype/cmu/,UprightFont=cmuntt.ttf,BoldFont=cmuntb.ttf,ItalicFont=cmunit.ttf,BoldItalicFont=cmuntx.ttf]{cmuntt.ttf}\ttfamily \textbackslash{}int}{$\text{ }$}\setmainfont[Path=/usr/share/fonts/truetype/cmu/,UprightFont=cmunrm.ttf,BoldFont=cmunbx.ttf,ItalicFont=cmunti.ttf,BoldItalicFont=cmunbi.ttf]{cmunrm.ttf}\setmonofont[Path=/usr/share/fonts/truetype/cmu/,UprightFont=cmuntt.ttf,BoldFont=cmuntb.ttf,ItalicFont=cmunit.ttf,BoldItalicFont=cmuntx.ttf]{cmunrm.ttf} &\hspace*{0pt}\ignorespaces{}\hspace*{0pt} {$\int$} &\hspace*{0pt}\ignorespaces{}\hspace*{0pt}&\hspace*{0pt}\ignorespaces{}\hspace*{0pt} {\ttfamily \setmainfont[Path=/usr/share/fonts/truetype/cmu/,UprightFont=cmunrm.ttf,BoldFont=cmunbx.ttf,ItalicFont=cmunti.ttf,BoldItalicFont=cmunbi.ttf]{cmuntt.ttf}\setmonofont[Path=/usr/share/fonts/truetype/cmu/,UprightFont=cmuntt.ttf,BoldFont=cmuntb.ttf,ItalicFont=cmunit.ttf,BoldItalicFont=cmuntx.ttf]{cmuntt.ttf}\ttfamily \textbackslash{}oint}{$\text{ }$}\setmainfont[Path=/usr/share/fonts/truetype/cmu/,UprightFont=cmunrm.ttf,BoldFont=cmunbx.ttf,ItalicFont=cmunti.ttf,BoldItalicFont=cmunbi.ttf]{cmunrm.ttf}\setmonofont[Path=/usr/share/fonts/truetype/cmu/,UprightFont=cmuntt.ttf,BoldFont=cmuntb.ttf,ItalicFont=cmunit.ttf,BoldItalicFont=cmuntx.ttf]{cmunrm.ttf} &\hspace*{0pt}\ignorespaces{}\hspace*{0pt} {$\oint$} &\hspace*{0pt}\ignorespaces{}\hspace*{0pt}&\hspace*{0pt}\ignorespaces{}\hspace*{0pt} {\ttfamily \setmainfont[Path=/usr/share/fonts/truetype/cmu/,UprightFont=cmunrm.ttf,BoldFont=cmunbx.ttf,ItalicFont=cmunti.ttf,BoldItalicFont=cmunbi.ttf]{cmuntt.ttf}\setmonofont[Path=/usr/share/fonts/truetype/cmu/,UprightFont=cmuntt.ttf,BoldFont=cmuntb.ttf,ItalicFont=cmunit.ttf,BoldItalicFont=cmuntx.ttf]{cmuntt.ttf}\ttfamily \textbackslash{}iint}\myfootnote{\setmainfont[Path=/usr/share/fonts/truetype/cmu/,UprightFont=cmunrm.ttf,BoldFont=cmunbx.ttf,ItalicFont=cmunti.ttf,BoldItalicFont=cmunbi.ttf]{cmunrm.ttf}\setmonofont[Path=/usr/share/fonts/truetype/cmu/,UprightFont=cmuntt.ttf,BoldFont=cmuntb.ttf,ItalicFont=cmunit.ttf,BoldItalicFont=cmuntx.ttf]{cmunrm.ttf}requires the \LaTeXTT{amsmath} package} &\hspace*{0pt}\ignorespaces{}\hspace*{0pt} {$\iint$} \\ \hspace*{0pt}\ignorespaces{}\hspace*{0pt} {\ttfamily \setmainfont[Path=/usr/share/fonts/truetype/cmu/,UprightFont=cmunrm.ttf,BoldFont=cmunbx.ttf,ItalicFont=cmunti.ttf,BoldItalicFont=cmunbi.ttf]{cmuntt.ttf}\setmonofont[Path=/usr/share/fonts/truetype/cmu/,UprightFont=cmuntt.ttf,BoldFont=cmuntb.ttf,ItalicFont=cmunit.ttf,BoldItalicFont=cmuntx.ttf]{cmuntt.ttf}\ttfamily \textbackslash{}iiint}\myfootnote{\setmainfont[Path=/usr/share/fonts/truetype/cmu/,UprightFont=cmunrm.ttf,BoldFont=cmunbx.ttf,ItalicFont=cmunti.ttf,BoldItalicFont=cmunbi.ttf]{cmunrm.ttf}\setmonofont[Path=/usr/share/fonts/truetype/cmu/,UprightFont=cmuntt.ttf,BoldFont=cmuntb.ttf,ItalicFont=cmunit.ttf,BoldItalicFont=cmuntx.ttf]{cmunrm.ttf}requires the \LaTeXTT{amsmath} package} &\hspace*{0pt}\ignorespaces{}\hspace*{0pt} {$\iiint$} &\hspace*{0pt}\ignorespaces{}\hspace*{0pt}&\hspace*{0pt}\ignorespaces{}\hspace*{0pt} {\ttfamily \setmainfont[Path=/usr/share/fonts/truetype/cmu/,UprightFont=cmunrm.ttf,BoldFont=cmunbx.ttf,ItalicFont=cmunti.ttf,BoldItalicFont=cmunbi.ttf]{cmuntt.ttf}\setmonofont[Path=/usr/share/fonts/truetype/cmu/,UprightFont=cmuntt.ttf,BoldFont=cmuntb.ttf,ItalicFont=cmunit.ttf,BoldItalicFont=cmuntx.ttf]{cmuntt.ttf}\ttfamily \textbackslash{}iiiint}\myfootnote{\setmainfont[Path=/usr/share/fonts/truetype/cmu/,UprightFont=cmunrm.ttf,BoldFont=cmunbx.ttf,ItalicFont=cmunti.ttf,BoldItalicFont=cmunbi.ttf]{cmunrm.ttf}\setmonofont[Path=/usr/share/fonts/truetype/cmu/,UprightFont=cmuntt.ttf,BoldFont=cmuntb.ttf,ItalicFont=cmunit.ttf,BoldItalicFont=cmuntx.ttf]{cmunrm.ttf}requires the \LaTeXTT{amsmath} package} &\hspace*{0pt}\ignorespaces{}\hspace*{0pt} {$\iiiint$} &\hspace*{0pt}\ignorespaces{}\hspace*{0pt}&\hspace*{0pt}\ignorespaces{}\hspace*{0pt} {\ttfamily \setmainfont[Path=/usr/share/fonts/truetype/cmu/,UprightFont=cmunrm.ttf,BoldFont=cmunbx.ttf,ItalicFont=cmunti.ttf,BoldItalicFont=cmunbi.ttf]{cmuntt.ttf}\setmonofont[Path=/usr/share/fonts/truetype/cmu/,UprightFont=cmuntt.ttf,BoldFont=cmuntb.ttf,ItalicFont=cmunit.ttf,BoldItalicFont=cmuntx.ttf]{cmuntt.ttf}\ttfamily \textbackslash{}idotsint}\myfootnote{\setmainfont[Path=/usr/share/fonts/truetype/cmu/,UprightFont=cmunrm.ttf,BoldFont=cmunbx.ttf,ItalicFont=cmunti.ttf,BoldItalicFont=cmunbi.ttf]{cmunrm.ttf}\setmonofont[Path=/usr/share/fonts/truetype/cmu/,UprightFont=cmuntt.ttf,BoldFont=cmuntb.ttf,ItalicFont=cmunit.ttf,BoldItalicFont=cmuntx.ttf]{cmunrm.ttf}requires the \LaTeXTT{amsmath} package} &\hspace*{0pt}\ignorespaces{}\hspace*{0pt} {$\int  \cdots  \int$}  
\end{longtable}
}

For more integral symbols, including those not included by default in the Computer Modern font, try the \LaTeXTT{esint} package.

The \LaTeXTT{\textbackslash{}substack} command\myfootnote{requires the \LaTeXTT{amsmath} package} allows the use of \LaTeXTT{\textbackslash{}\textbackslash{}} to write the limits over multiple lines:
\begin{longtable}{p{1.0\linewidth}}
\begin{Shaded}
\begin{Highlighting}[]


 \NormalTok{\textbackslash{}sum_\{\textbackslash{}substack\{}
   \NormalTok{0<i<m \textbackslash{}\textbackslash{}}
   \NormalTok{0<j<n}
  \NormalTok{\}<!---->\} }
 \NormalTok{P(i,j)}
 
\end{Highlighting}
\end{Shaded}
\\
{$ \sum_{\overset{\scriptstyle 0<i<m} {\scriptstyle 0<j<n}} P(i,j)  \,$}

\end{longtable}

If you want the limits of an integral to be specified above and below the symbol (like the sum), use the \LaTeXTT{\textbackslash{}limits} command:
\begin{longtable}{p{1.0\linewidth}}
\begin{Shaded}
\begin{Highlighting}[]


 \NormalTok{\textbackslash{}int\textbackslash{}limits_a^b}
 
\end{Highlighting}
\end{Shaded}
\\
{$ \int\limits_a^b  \,$}

\end{longtable}
However if you want this to apply to ALL integrals, it is preferable to specify the \LaTeXTT{intlimits} option when loading the \LaTeXTT{amsmath} package:
\begin{Shaded}
\begin{Highlighting}[]

\NormalTok{\textbackslash{}usepackage[intlimits]\{amsmath\}}
\end{Highlighting}
\end{Shaded}


Subscripts and superscripts in other contexts as well as other parameters to \LaTeXTT{amsmath} package related to them are described in \mylref{547}{Advanced Mathematics} chapter.

For bigger integrals, you may use personal declarations, or the \LaTeXTT{bigints} package \myfootnote{ \myplainurl{http://hdl.handle.net/2268/6219}}.
\section{Brackets, braces and delimiters}
\label{509}
{\small {\itshape \setmainfont[Path=/usr/share/fonts/truetype/cmu/,UprightFont=cmunrm.ttf,BoldFont=cmunbx.ttf,ItalicFont=cmunti.ttf,BoldItalicFont=cmunbi.ttf]{cmunti.ttf}\setmonofont[Path=/usr/share/fonts/truetype/cmu/,UprightFont=cmuntt.ttf,BoldFont=cmuntb.ttf,ItalicFont=cmunit.ttf,BoldItalicFont=cmuntx.ttf]{cmunti.ttf}\itshape How to use braces in multi line equations is described in the \mylref{539}{Advanced Mathematics} chapter.}}\setmainfont[Path=/usr/share/fonts/truetype/cmu/,UprightFont=cmunrm.ttf,BoldFont=cmunbx.ttf,ItalicFont=cmunti.ttf,BoldItalicFont=cmunbi.ttf]{cmunrm.ttf}\setmonofont[Path=/usr/share/fonts/truetype/cmu/,UprightFont=cmuntt.ttf,BoldFont=cmuntb.ttf,ItalicFont=cmunit.ttf,BoldItalicFont=cmuntx.ttf]{cmunrm.ttf}

The use of delimiters such as brackets soon becomes important when dealing with anything but the most trivial equations. Without them, formulas can become ambiguous. Also, special types of mathematical structures, such as matrices, typically rely on delimiters to enclose them.

There are a variety of delimiters available for use in LaTeX:
\begin{longtable}{p{1.0\linewidth}}
\begin{Shaded}
\begin{Highlighting}[]

 
\NormalTok{( a ), [ b ], \textbackslash{}\{ c \textbackslash{}\},  d , \textbackslash{} e \textbackslash{},}
\NormalTok{\textbackslash{}langle f \textbackslash{}rangle, \textbackslash{}lfloor g \textbackslash{}rfloor,}
\NormalTok{\textbackslash{}lceil h \textbackslash{}rceil, \textbackslash{}ulcorner i \textbackslash{}urcorner}
 
\end{Highlighting}
\end{Shaded}
\\
{$ ( a ), [ b ], \{ c \}, | d |, \Vert e \Vert, \langle f \rangle, \lfloor g \rfloor, \lceil h \rceil, \ulcorner i \urcorner $}

\end{longtable}
where \textbackslash{}lbrack and \textbackslash{}rbrack may be used in place of {$\text{[}$} and {$\text{]}$}. 

\subsection{Automatic sizing}
\label{510}
Very often mathematical features will differ in size, in which case the delimiters surrounding the expression should vary accordingly. This can be done automatically using the \LaTeXTT{\textbackslash{}left}, \LaTeXTT{\textbackslash{}right}, and \LaTeXTT{\textbackslash{}middle} commands. Any of the previous delimiters may be used in combination with these:
\begin{longtable}{p{1.0\linewidth}}
\begin{Shaded}
\begin{Highlighting}[]


 \NormalTok{\textbackslash{}left(\textbackslash{}frac\{x^2\}\{y^3\}\textbackslash{}right)}
 
\end{Highlighting}
\end{Shaded}
\\
{$ \left(\frac{x^2}{y^3}\right)  \,$}

\end{longtable}
\begin{longtable}{p{1.0\linewidth}}
\begin{Shaded}
\begin{Highlighting}[]


 \NormalTok{P\textbackslash{}left(A=2\textbackslash{}middle\textbackslash{}frac\{A^2\}\{B\}>4\textbackslash{}right)}
 
\end{Highlighting}
\end{Shaded}
\\


\begin{minipage}{0.37500\textwidth}
\begin{center}
\includegraphics[width=1.0\textwidth,height=6.5in,keepaspectratio]{../images/84.png}
\end{center}
\raggedright{}\myfigurewithoutcaption{84}
\end{minipage}\vspace{0.75cm}



\end{longtable}
Curly braces are defined differently by using \LaTeXTT{\textbackslash{}left\textbackslash{}\{} and \LaTeXTT{\textbackslash{}right\textbackslash{}\}},

\begin{longtable}{p{1.0\linewidth}}
\begin{Shaded}
\begin{Highlighting}[]


 \NormalTok{\textbackslash{}left\textbackslash{}\{\textbackslash{}frac\{x^2\}\{y^3\}\textbackslash{}right\textbackslash{}\}}
 
\end{Highlighting}
\end{Shaded}
\\
{$ \left\{\frac{x^2}{y^3}\right\}  \,$}

\end{longtable}

If a delimiter on only one side of an expression is required, then an invisible delimiter on the other side may be denoted using a period ({\ttfamily \setmainfont[Path=/usr/share/fonts/truetype/cmu/,UprightFont=cmunrm.ttf,BoldFont=cmunbx.ttf,ItalicFont=cmunti.ttf,BoldItalicFont=cmunbi.ttf]{cmuntt.ttf}\setmonofont[Path=/usr/share/fonts/truetype/cmu/,UprightFont=cmuntt.ttf,BoldFont=cmuntb.ttf,ItalicFont=cmunit.ttf,BoldItalicFont=cmuntx.ttf]{cmuntt.ttf}\ttfamily .}\setmainfont[Path=/usr/share/fonts/truetype/cmu/,UprightFont=cmunrm.ttf,BoldFont=cmunbx.ttf,ItalicFont=cmunti.ttf,BoldItalicFont=cmunbi.ttf]{cmunrm.ttf}\setmonofont[Path=/usr/share/fonts/truetype/cmu/,UprightFont=cmuntt.ttf,BoldFont=cmuntb.ttf,ItalicFont=cmunit.ttf,BoldItalicFont=cmuntx.ttf]{cmunrm.ttf}).
\begin{longtable}{p{1.0\linewidth}}
\begin{Shaded}
\begin{Highlighting}[]


 \NormalTok{\textbackslash{}left.\textbackslash{}frac\{x^3\}\{3\}\textbackslash{}right_0^1}
 
\end{Highlighting}
\end{Shaded}
\\
{$  \left.\frac{x^3}{3}\right|_0^1  \,$}

\end{longtable}
\subsection{Manual sizing}
\label{511}
In certain cases, the sizing produced by the \LaTeXTT{\textbackslash{}left} and \LaTeXTT{\textbackslash{}right} commands may not be desirable, or you may simply want finer control over the delimiter sizes. In this case, the \LaTeXTT{\textbackslash{}big}, \LaTeXTT{\textbackslash{}Big}, \LaTeXTT{\textbackslash{}bigg} and \LaTeXTT{\textbackslash{}Bigg} modifier commands may be used:
\begin{longtable}{p{1.0\linewidth}}
\begin{Shaded}
\begin{Highlighting}[]


 \NormalTok{( \textbackslash{}big( \textbackslash{}Big( \textbackslash{}bigg( \textbackslash{}Bigg( }
 
\end{Highlighting}
\end{Shaded}
\\
{$ ( \big( \Big( \bigg( \Bigg(   \,$}

\end{longtable}
These commands are primarily useful when dealing with nested delimiters. For example, when typesetting
\begin{longtable}{p{1.0\linewidth}}
\begin{Shaded}
\begin{Highlighting}[]

 
\NormalTok{\textbackslash{}frac\{\textbackslash{}mathrm d\}\{\textbackslash{}mathrm d x\} \textbackslash{}left( k g(x) \textbackslash{}right)}
 
\end{Highlighting}
\end{Shaded}
\\
{$ \frac{\mathrm d}{\mathrm d x} \left( k g(x) \right) $}

\end{longtable}
we notice that the \LaTeXTT{\textbackslash{}left} and \LaTeXTT{\textbackslash{}right} commands produce the same size delimiters as those nested within it. This can be difficult to read. To fix this, we write
\begin{longtable}{p{1.0\linewidth}}
\begin{Shaded}
\begin{Highlighting}[]

 
\NormalTok{\textbackslash{}frac\{\textbackslash{}mathrm d\}\{\textbackslash{}mathrm d x\} \textbackslash{}big( k g(x) \textbackslash{}big)}
 
\end{Highlighting}
\end{Shaded}
\\
{$ \frac{\mathrm d}{\mathrm d x} \big( k g(x) \big) $}

\end{longtable}

Manual sizing can also be useful when an equation is too large, trails off the end of the page, and must be separated into two lines using an align command. Although the commands \LaTeXTT{\textbackslash{}left.} and \LaTeXTT{\textbackslash{}right.} can be used to balance the delimiters on each line, this may lead to wrong delimiter sizes.  Furthermore, manual sizing can be used to avoid overly large delimiters if an \LaTeXTT{\textbackslash{}underbrace} or a similar command appears between the delimiters.
\subsection{Typesetting intervals}
\label{512}

To denote open and half-{}open intervals, the notations {$\text{]}$}a,b{$\text{[}$}, (a,b), {$\text{]}$}a,b{$\text{]}$}, (a,b{$\text{]}$}, {$\text{[}$}a,b{$\text{[}$} and {$\text{[}$}a,b) are used. If the square bracket notation is used, then the interval must be put between curly braces ({\ttfamily \setmainfont[Path=/usr/share/fonts/truetype/cmu/,UprightFont=cmunrm.ttf,BoldFont=cmunbx.ttf,ItalicFont=cmunti.ttf,BoldItalicFont=cmunbi.ttf]{cmuntt.ttf}\setmonofont[Path=/usr/share/fonts/truetype/cmu/,UprightFont=cmuntt.ttf,BoldFont=cmuntb.ttf,ItalicFont=cmunit.ttf,BoldItalicFont=cmuntx.ttf]{cmuntt.ttf}\ttfamily \{}{$\text{ }$}\setmainfont[Path=/usr/share/fonts/truetype/cmu/,UprightFont=cmunrm.ttf,BoldFont=cmunbx.ttf,ItalicFont=cmunti.ttf,BoldItalicFont=cmunbi.ttf]{cmunrm.ttf}\setmonofont[Path=/usr/share/fonts/truetype/cmu/,UprightFont=cmuntt.ttf,BoldFont=cmuntb.ttf,ItalicFont=cmunit.ttf,BoldItalicFont=cmuntx.ttf]{cmunrm.ttf} and {\ttfamily \setmainfont[Path=/usr/share/fonts/truetype/cmu/,UprightFont=cmunrm.ttf,BoldFont=cmunbx.ttf,ItalicFont=cmunti.ttf,BoldItalicFont=cmunbi.ttf]{cmuntt.ttf}\setmonofont[Path=/usr/share/fonts/truetype/cmu/,UprightFont=cmuntt.ttf,BoldFont=cmuntb.ttf,ItalicFont=cmunit.ttf,BoldItalicFont=cmuntx.ttf]{cmuntt.ttf}\ttfamily \}}\setmainfont[Path=/usr/share/fonts/truetype/cmu/,UprightFont=cmunrm.ttf,BoldFont=cmunbx.ttf,ItalicFont=cmunti.ttf,BoldItalicFont=cmunbi.ttf]{cmunrm.ttf}\setmonofont[Path=/usr/share/fonts/truetype/cmu/,UprightFont=cmuntt.ttf,BoldFont=cmuntb.ttf,ItalicFont=cmunit.ttf,BoldItalicFont=cmuntx.ttf]{cmunrm.ttf}) in order to have correct spacing. Similarly, if a (half-{})open interval starts with a negative number, then the number including its minus-{}symbol must also be put between curly brackets, so that LaTeX understands that the minus-{}symbol is the unary operation. Compare:

\begin{longtable}{p{1.0\linewidth}}
\begin{Shaded}
\begin{Highlighting}[]

\NormalTok{x \textbackslash{}in [-1,1]}
\end{Highlighting}
\end{Shaded}
\\
{$x \in [-1,1]$}

\end{longtable}
\begin{longtable}{p{1.0\linewidth}}
\begin{Shaded}
\begin{Highlighting}[]

\NormalTok{x \textbackslash{}in \{[-1,1]\}}
\end{Highlighting}
\end{Shaded}
\\
{$x \in {[-1,1]}$}

\end{longtable}
\begin{longtable}{p{1.0\linewidth}}
\begin{Shaded}
\begin{Highlighting}[]

\NormalTok{x \textbackslash{}in \{[\{-1\},1]\}}
\end{Highlighting}
\end{Shaded}
\\
{$x \in {[{-1},1]}$}

\end{longtable}
\section{Matrices and arrays}
\label{513}
A basic matrix may be created using the \LaTeXTT{matrix} environment\myfootnote{requires the \LaTeXTT{amsmath} package}: in common with other table-{}like structures, entries are specified by row, with columns separated using an ampersand (\LaTeXTT{\&}) and a new rows separated with a double backslash (\LaTeXTT{\textbackslash{}\textbackslash{}})
\begin{longtable}{p{1.0\linewidth}}
\begin{Shaded}
\begin{Highlighting}[]


 \NormalTok{\textbackslash{}begin\{matrix\}}
  \NormalTok{a & b & c \textbackslash{}\textbackslash{}}
  \NormalTok{d & e & f \textbackslash{}\textbackslash{}}
  \NormalTok{g & h & i}
 \NormalTok{\textbackslash{}end\{matrix\}}
 
\end{Highlighting}
\end{Shaded}
\\
{$ \begin{matrix} a & b & c \\ d & e & f \\ g & h & i \end{matrix} $}

\end{longtable}

To specify alignment of columns in the table, use starred version\myfootnote{requires the \LaTeXTT{mathtools} package}:
\begin{longtable}{p{1.0\linewidth}}
\begin{Shaded}
\begin{Highlighting}[]


 \NormalTok{\textbackslash{}begin\{matrix\}}
  \NormalTok{-1 & 3 \textbackslash{}\textbackslash{}}
  \NormalTok{2 & -4}
 \NormalTok{\textbackslash{}end\{matrix\}}
 \NormalTok{=}
 \NormalTok{\textbackslash{}begin\{matrix*\}[r]}
  \NormalTok{-1 & 3 \textbackslash{}\textbackslash{}}
  \NormalTok{2 & -4}
 \NormalTok{\textbackslash{}end\{matrix*\}}
 
\end{Highlighting}
\end{Shaded}
\\

{$  \begin{matrix}   -1 & 3 \\   2 & -4  \end{matrix}  =  \begin{matrix}   -1 & \,3 \\   \,2 & -4  \end{matrix} $}

\end{longtable}

The alignment by default is \LaTeXTT{c} but it can be any column type valid in \LaTeXTT{array} environment.

However matrices are usually enclosed in delimiters of some kind, and while it is possible to use the \mylref{510}{{\ttfamily \setmainfont[Path=/usr/share/fonts/truetype/cmu/,UprightFont=cmunrm.ttf,BoldFont=cmunbx.ttf,ItalicFont=cmunti.ttf,BoldItalicFont=cmunbi.ttf]{cmuntt.ttf}\setmonofont[Path=/usr/share/fonts/truetype/cmu/,UprightFont=cmuntt.ttf,BoldFont=cmuntb.ttf,ItalicFont=cmunit.ttf,BoldItalicFont=cmuntx.ttf]{cmuntt.ttf}\ttfamily \textbackslash{}left}{$\text{ }$}\setmainfont[Path=/usr/share/fonts/truetype/cmu/,UprightFont=cmunrm.ttf,BoldFont=cmunbx.ttf,ItalicFont=cmunti.ttf,BoldItalicFont=cmunbi.ttf]{cmunrm.ttf}\setmonofont[Path=/usr/share/fonts/truetype/cmu/,UprightFont=cmuntt.ttf,BoldFont=cmuntb.ttf,ItalicFont=cmunit.ttf,BoldItalicFont=cmuntx.ttf]{cmunrm.ttf} and {\ttfamily \setmainfont[Path=/usr/share/fonts/truetype/cmu/,UprightFont=cmunrm.ttf,BoldFont=cmunbx.ttf,ItalicFont=cmunti.ttf,BoldItalicFont=cmunbi.ttf]{cmuntt.ttf}\setmonofont[Path=/usr/share/fonts/truetype/cmu/,UprightFont=cmuntt.ttf,BoldFont=cmuntb.ttf,ItalicFont=cmunit.ttf,BoldItalicFont=cmuntx.ttf]{cmuntt.ttf}\ttfamily \textbackslash{}right}{$\text{ }$}\setmainfont[Path=/usr/share/fonts/truetype/cmu/,UprightFont=cmunrm.ttf,BoldFont=cmunbx.ttf,ItalicFont=cmunti.ttf,BoldItalicFont=cmunbi.ttf]{cmunrm.ttf}\setmonofont[Path=/usr/share/fonts/truetype/cmu/,UprightFont=cmuntt.ttf,BoldFont=cmuntb.ttf,ItalicFont=cmunit.ttf,BoldItalicFont=cmuntx.ttf]{cmunrm.ttf} commands}, there are various other predefined environments which automatically include delimiters:
\begin{longtable}{|>{\RaggedRight}p{0.32858\linewidth}|>{\RaggedRight}p{0.25714\linewidth}|>{\RaggedRight}p{0.32858\linewidth}|} \hline 
{\bfseries \hspace*{0pt}\ignorespaces{}\hspace*{0pt} Environment name}&{\bfseries \hspace*{0pt}\ignorespaces{}\hspace*{0pt} Surrounding delimiter}&{\bfseries \hspace*{0pt}\ignorespaces{}\hspace*{0pt} Notes}\endhead  \hline \hspace*{0pt}\ignorespaces{}\hspace*{0pt} \LaTeXTT{pmatrix}\myfootnote{requires the \LaTeXTT{amsmath} package}&\hspace*{0pt}\ignorespaces{}\hspace*{0pt} {$( \, ) $}&\hspace*{0pt}\ignorespaces{}\hspace*{0pt} centers columns by default\\ \hline \hspace*{0pt}\ignorespaces{}\hspace*{0pt} \LaTeXTT{pmatrix*}\myfootnote{requires the \LaTeXTT{mathtools} package}&\hspace*{0pt}\ignorespaces{}\hspace*{0pt} {$( \, ) $}&\hspace*{0pt}\ignorespaces{}\hspace*{0pt} allows to specify alignment of columns in optional parameter\\ \hline \hspace*{0pt}\ignorespaces{}\hspace*{0pt} \LaTeXTT{bmatrix}\myfootnote{requires the \LaTeXTT{amsmath} package}&\hspace*{0pt}\ignorespaces{}\hspace*{0pt} {$[ \, ] $}&\hspace*{0pt}\ignorespaces{}\hspace*{0pt} centers columns by default\\ \hline \hspace*{0pt}\ignorespaces{}\hspace*{0pt} \LaTeXTT{bmatrix*}\myfootnote{requires the \LaTeXTT{mathtools} package}&\hspace*{0pt}\ignorespaces{}\hspace*{0pt} {$[ \, ] $}&\hspace*{0pt}\ignorespaces{}\hspace*{0pt} allows to specify alignment of columns in optional parameter\\ \hline \hspace*{0pt}\ignorespaces{}\hspace*{0pt} \LaTeXTT{Bmatrix}\myfootnote{requires the \LaTeXTT{amsmath} package}&\hspace*{0pt}\ignorespaces{}\hspace*{0pt} {$\{ \, \} $}&\hspace*{0pt}\ignorespaces{}\hspace*{0pt} centers columns by default\\ \hline \hspace*{0pt}\ignorespaces{}\hspace*{0pt} \LaTeXTT{Bmatrix*}\myfootnote{requires the \LaTeXTT{mathtools} package}&\hspace*{0pt}\ignorespaces{}\hspace*{0pt} {$\{ \, \} $}&\hspace*{0pt}\ignorespaces{}\hspace*{0pt} allows to specify alignment of columns in optional parameter\\ \hline \hspace*{0pt}\ignorespaces{}\hspace*{0pt} \LaTeXTT{vmatrix}\myfootnote{requires the \LaTeXTT{amsmath} package}&\hspace*{0pt}\ignorespaces{}\hspace*{0pt} {$| \, | $}&\hspace*{0pt}\ignorespaces{}\hspace*{0pt} centers columns by default\\ \hline \hspace*{0pt}\ignorespaces{}\hspace*{0pt} \LaTeXTT{vmatrix*}\myfootnote{requires the \LaTeXTT{mathtools} package}&\hspace*{0pt}\ignorespaces{}\hspace*{0pt} {$| \, | $}&\hspace*{0pt}\ignorespaces{}\hspace*{0pt} allows to specify alignment of columns in optional parameter\\ \hline \hspace*{0pt}\ignorespaces{}\hspace*{0pt} \LaTeXTT{Vmatrix}\myfootnote{requires the \LaTeXTT{amsmath} package}&\hspace*{0pt}\ignorespaces{}\hspace*{0pt} {$\Vert \, \Vert $}&\hspace*{0pt}\ignorespaces{}\hspace*{0pt} centers columns by default\\ \hline \hspace*{0pt}\ignorespaces{}\hspace*{0pt} \LaTeXTT{Vmatrix*}\myfootnote{requires the \LaTeXTT{mathtools} package}&\hspace*{0pt}\ignorespaces{}\hspace*{0pt} {$\Vert \, \Vert $}&\hspace*{0pt}\ignorespaces{}\hspace*{0pt} allows to specify alignment of colums in optional parameter\\ \hline 
\end{longtable}


When writing down arbitrary sized matrices, it is common to use horizontal, vertical and diagonal triplets of dots (known as \myhref{https://en.wikipedia.org/wiki/ellipsis}{ellipses}) to fill in certain columns and rows. These can be specified using the \LaTeXTT{\textbackslash{}cdots}, \LaTeXTT{\textbackslash{}vdots} and \LaTeXTT{\textbackslash{}ddots} respectively:
\begin{longtable}{p{1.0\linewidth}}
\begin{Shaded}
\begin{Highlighting}[]


 \NormalTok{A_\{m,n\} = }
 \NormalTok{\textbackslash{}begin\{pmatrix\}}
  \NormalTok{a_\{1,1\} & a_\{1,2\} & \textbackslash{}cdots & a_\{1,n\} \textbackslash{}\textbackslash{}}
  \NormalTok{a_\{2,1\} & a_\{2,2\} & \textbackslash{}cdots & a_\{2,n\} \textbackslash{}\textbackslash{}}
  \NormalTok{\textbackslash{}vdots  & \textbackslash{}vdots  & \textbackslash{}ddots & \textbackslash{}vdots  \textbackslash{}\textbackslash{}}
  \NormalTok{a_\{m,1\} & a_\{m,2\} & \textbackslash{}cdots & a_\{m,n\} }
 \NormalTok{\textbackslash{}end\{pmatrix\}}
 
\end{Highlighting}
\end{Shaded}
\\
{$ A_{m,n} =  \begin{pmatrix} a_{1,1} & a_{1,2} & \cdots & a_{1,n} \\ a_{2,1} & a_{2,2} & \cdots & a_{2,n} \\ \vdots  & \vdots  & \ddots & \vdots  \\ a_{m,1} & a_{m,2} & \cdots & a_{m,n}  \end{pmatrix} $}

\end{longtable}
In some cases you may want to have finer control of the alignment within each column, or want to insert lines between columns or rows. This can be achieved using the \LaTeXTT{array} environment, which is essentially a math-{}mode version of the \mylref{272}{{\ttfamily \setmainfont[Path=/usr/share/fonts/truetype/cmu/,UprightFont=cmunrm.ttf,BoldFont=cmunbx.ttf,ItalicFont=cmunti.ttf,BoldItalicFont=cmunbi.ttf]{cmuntt.ttf}\setmonofont[Path=/usr/share/fonts/truetype/cmu/,UprightFont=cmuntt.ttf,BoldFont=cmuntb.ttf,ItalicFont=cmunit.ttf,BoldItalicFont=cmuntx.ttf]{cmuntt.ttf}\ttfamily tabular}{$\text{ }$}\setmainfont[Path=/usr/share/fonts/truetype/cmu/,UprightFont=cmunrm.ttf,BoldFont=cmunbx.ttf,ItalicFont=cmunti.ttf,BoldItalicFont=cmunbi.ttf]{cmunrm.ttf}\setmonofont[Path=/usr/share/fonts/truetype/cmu/,UprightFont=cmuntt.ttf,BoldFont=cmuntb.ttf,ItalicFont=cmunit.ttf,BoldItalicFont=cmuntx.ttf]{cmunrm.ttf} environment}, which requires that the columns be pre-{}specified:
\begin{longtable}{p{1.0\linewidth}}
\begin{Shaded}
\begin{Highlighting}[]


 \NormalTok{\textbackslash{}begin\{array\}\{cc\}}
  \NormalTok{1 & 2 \textbackslash{}\textbackslash{} }
  \NormalTok{\textbackslash{}hline}
  \NormalTok{3 & 4}
 \NormalTok{\textbackslash{}end\{array\}}
 
\end{Highlighting}
\end{Shaded}
\\
{$ \begin{array}{c|c} 1 & 2 \\  \hline 3 & 4 \end{array} $}

\end{longtable}

You may see that the AMS matrix class of environments doesn\textquotesingle{}t leave enough space when used together with fractions resulting in output similar to this:

{$  M = \begin{bmatrix}        \frac{5}{6} & \frac{1}{6} & 0\\        \frac{5}{6} & 0           & \frac{1}{6}\\        0           & \frac{5}{6} & \frac{1}{6}      \end{bmatrix} $}

To counteract this problem, add additional leading space with the optional parameter to the \LaTeXTT{\textbackslash{}\textbackslash{}} command:

\begin{longtable}{p{1.0\linewidth}}
\begin{Shaded}
\begin{Highlighting}[]


 \NormalTok{M = \textbackslash{}begin\{bmatrix\}}
       \NormalTok{\textbackslash{}frac\{5\}\{6\} & \textbackslash{}frac\{1\}\{6\} & 0           \textbackslash{}\textbackslash{}[0.3em]}
       \NormalTok{\textbackslash{}frac\{5\}\{6\} & 0           & \textbackslash{}frac\{1\}\{6\} \textbackslash{}\textbackslash{}[0.3em]}
       \NormalTok{0           & \textbackslash{}frac\{5\}\{6\} & \textbackslash{}frac\{1\}\{6\}}
     \NormalTok{\textbackslash{}end\{bmatrix\}}
 
\end{Highlighting}
\end{Shaded}
\\
{$  M = \begin{bmatrix}        \frac{5}{6} & \frac{1}{6} & 0           \\[0.3em]        \frac{5}{6} & 0           & \frac{1}{6} \\[0.3em]        0           & \frac{5}{6} & \frac{1}{6}      \end{bmatrix}$}

\end{longtable}

If you need \symbol{34}border\symbol{34} or \symbol{34}indexes\symbol{34} on your matrix, plain TeX provides the macro \LaTeXTT{\textbackslash{}bordermatrix}
\begin{longtable}{p{1.0\linewidth}}
\begin{Shaded}
\begin{Highlighting}[]

 
\NormalTok{M = \textbackslash{}bordermatrix\{~ & x & y \textbackslash{}cr}
                  \NormalTok{A & 1 & 0 \textbackslash{}cr}
                  \NormalTok{B & 0 & 1 \textbackslash{}cr\}}
 
\end{Highlighting}
\end{Shaded}
\\


\begin{minipage}{0.37500\textwidth}
\begin{center}
\includegraphics[width=1.0\textwidth,height=6.5in,keepaspectratio]{../images/85.png}
\end{center}
\raggedright{}\myfigurewithoutcaption{85}
\end{minipage}\vspace{0.75cm}



\end{longtable}

\subsection{Matrices in running text}
\label{514}
To insert a small matrix, and not increase leading in the line containing it, use \LaTeXTT{smallmatrix} environment:

\begin{longtable}{p{1.0\linewidth}}
\begin{Shaded}
\begin{Highlighting}[]

\NormalTok{A matrix in text must be set smaller:}
\NormalTok{$\textbackslash{}bigl(\textbackslash{}begin\{smallmatrix\}}
\NormalTok{a&b \textbackslash{}\textbackslash{} c&d}
\NormalTok{\textbackslash{}end\{smallmatrix\} \textbackslash{}bigr)$}
\NormalTok{to not increase leading in a portion of text.}
\end{Highlighting}
\end{Shaded}
\\


\begin{minipage}{1.0\linewidth}
\begin{center}
\includegraphics[width=1.0\linewidth,height=6.5in,keepaspectratio]{../images/86.png}
\end{center}
\raggedright{}\myfigurewithoutcaption{86}
\end{minipage}\vspace{0.75cm}



\end{longtable}
\section{Adding text to equations}
\label{515}

The math environment differs from the text environment in the representation of text.  Here is an example of trying to represent text within the math environment:
\begin{longtable}{p{1.0\linewidth}}
\begin{Shaded}
\begin{Highlighting}[]


 \NormalTok{50 apples \textbackslash{}times 100 apples = lots of apples^2}
 
\end{Highlighting}
\end{Shaded}
\\
{$  50 apples \times 100 apples = lots of apples^2 \,$}

\end{longtable}

There are two noticeable problems: there are no spaces between words or numbers, and the letters are italicized and more spaced out than normal. Both issues are simply artifacts of the maths mode, in that it treats it as a mathematical expression: spaces are ignored (LaTeX spaces mathematics according to its own rules), and each character is a separate element (so are not positioned as closely as normal text).

There are a number of ways that text can be added properly. The typical way is to wrap the text with the \LaTeXTT{\textbackslash{}text\{...\}} command \myfootnote{requires the \LaTeXTT{amsmath} package} (a similar command is \LaTeXTT{\textbackslash{}mbox\{...\}}, though this causes problems with subscripts, and has a less descriptive name). Let\textquotesingle{}s see what happens when the above equation code is adapted:
\begin{longtable}{p{1.0\linewidth}}
\begin{Shaded}
\begin{Highlighting}[]


 \NormalTok{50 \textbackslash{}text\{apples\} \textbackslash{}times 100 \textbackslash{}text\{apples\} }
 \NormalTok{= \textbackslash{}text\{lots of apples\}^2}
 
\end{Highlighting}
\end{Shaded}
\\

{$  50 \text{apples} \times 100 \text{apples} = \text{lots of apples}^2 \,$}

\end{longtable}

The text looks better. However, there are no gaps between the numbers and the words. Unfortunately, you are required to explicitly add these. There are many ways to add spaces between maths elements, but for the sake of simplicity we may simply insert space characters into the \LaTeXTT{\textbackslash{}text} commands.
\begin{longtable}{p{1.0\linewidth}}
\begin{Shaded}
\begin{Highlighting}[]


 \NormalTok{50 \textbackslash{}text\{ apples\} \textbackslash{}times 100 \textbackslash{}text\{ apples\}}
 \NormalTok{= \textbackslash{}text\{lots of apples\}^2}
 
\end{Highlighting}
\end{Shaded}
\\

{$  50 \text{ apples} \times 100 \text{ apples} = \text{lots of apples}^2 \,$}

\end{longtable}
\subsection{Formatted text}
\label{516}
Using the \LaTeXTT{\textbackslash{}text} is fine and gets the basic result. Yet, there is an alternative that offers a little more flexibility. You may recall the introduction of \mylref{174}{font formatting commands}, such as \LaTeXTT{\textbackslash{}textrm}, \LaTeXTT{\textbackslash{}textit}, \LaTeXTT{\textbackslash{}textbf}, etc. These commands format the argument accordingly, e.g., \LaTeXTT{\textbackslash{}textbf\{bold text\}} gives {\bfseries \setmainfont[Path=/usr/share/fonts/truetype/cmu/,UprightFont=cmunrm.ttf,BoldFont=cmunbx.ttf,ItalicFont=cmunti.ttf,BoldItalicFont=cmunbi.ttf]{cmunbx.ttf}\setmonofont[Path=/usr/share/fonts/truetype/cmu/,UprightFont=cmuntt.ttf,BoldFont=cmuntb.ttf,ItalicFont=cmunit.ttf,BoldItalicFont=cmuntx.ttf]{cmunbx.ttf}\bfseries bold text}\setmainfont[Path=/usr/share/fonts/truetype/cmu/,UprightFont=cmunrm.ttf,BoldFont=cmunbx.ttf,ItalicFont=cmunti.ttf,BoldItalicFont=cmunbi.ttf]{cmunrm.ttf}\setmonofont[Path=/usr/share/fonts/truetype/cmu/,UprightFont=cmuntt.ttf,BoldFont=cmuntb.ttf,ItalicFont=cmunit.ttf,BoldItalicFont=cmuntx.ttf]{cmunrm.ttf}. These commands are equally valid within a maths environment to include text. The added benefit here is that you can have better control over the font formatting, rather than the standard text achieved with \LaTeXTT{\textbackslash{}text}.

\begin{longtable}{p{1.0\linewidth}}
\begin{Shaded}
\begin{Highlighting}[]


 \NormalTok{50 \textbackslash{}textrm\{ apples\} \textbackslash{}times 100}
 \NormalTok{\textbackslash{}textbf\{ apples\} = \textbackslash{}textit\{lots of apples\}^2}
 
\end{Highlighting}
\end{Shaded}
\\
{$  50 \textrm{ apples} \times 100 \textbf{ apples} = \textit{lots~of~apples}^2 \,$}

\end{longtable}
\section{Formatting mathematics symbols}
\label{517}
\begin{myquote}
\item{} {\itshape \setmainfont[Path=/usr/share/fonts/truetype/cmu/,UprightFont=cmunrm.ttf,BoldFont=cmunbx.ttf,ItalicFont=cmunti.ttf,BoldItalicFont=cmunbi.ttf]{cmunti.ttf}\setmonofont[Path=/usr/share/fonts/truetype/cmu/,UprightFont=cmuntt.ttf,BoldFont=cmuntb.ttf,ItalicFont=cmunit.ttf,BoldItalicFont=cmuntx.ttf]{cmunti.ttf}\itshape See also: \myhref{https://en.wikipedia.org/wiki/Mathematical\%20Alphanumeric\%20Symbols}{w:Mathematical Alphanumeric Symbols}, \myhref{https://en.wikipedia.org/wiki/Help\%3ADisplaying\%20a\%20formula\%23Alphabets\%20and\%20typefaces}{w:Help:Displaying a formula\#Alphabets and typefaces} and \myhref{https://en.wikipedia.org/wiki/Wikipedia\%3ALaTeX\%20symbols\%23Fonts}{w:Wikipedia:LaTeX symbols\#Fonts}}
\end{myquote}
\setmainfont[Path=/usr/share/fonts/truetype/cmu/,UprightFont=cmunrm.ttf,BoldFont=cmunbx.ttf,ItalicFont=cmunti.ttf,BoldItalicFont=cmunbi.ttf]{cmunrm.ttf}\setmonofont[Path=/usr/share/fonts/truetype/cmu/,UprightFont=cmuntt.ttf,BoldFont=cmuntb.ttf,ItalicFont=cmunit.ttf,BoldItalicFont=cmuntx.ttf]{cmunrm.ttf}
We can now format text; what about formatting mathematical expressions? There are a set of formatting commands very similar to the font formatting ones just used, except that they are specifically aimed at text in math mode (requires \LaTeXTT{amsfonts}){\scriptsize{}
{\scalefont{0.54654}\begin{longtable}{|>{\RaggedRight}p{0.21956\linewidth}|>{\RaggedRight}p{0.22205\linewidth}|>{\RaggedRight}p{0.22205\linewidth}|>{\RaggedRight}p{0.22205\linewidth}|} \hline 
{\bfseries \hspace*{0pt}\ignorespaces{}\hspace*{0pt}LaTeX command}&{\bfseries \hspace*{0pt}\ignorespaces{}\hspace*{0pt}Sample}&{\bfseries \hspace*{0pt}\ignorespaces{}\hspace*{0pt} Description}&{\bfseries \hspace*{0pt}\ignorespaces{}\hspace*{0pt} Common use}\endhead  \hline \hspace*{0pt}\ignorespaces{}\hspace*{0pt} \LaTeXTT{\textbackslash{}mathnormal\{…\}}\newline{}{\small (or simply omit any command)}&\hspace*{0pt}\ignorespaces{}\hspace*{0pt} {$ABCDEF~abcdef~123456\,$}&\hspace*{0pt}\ignorespaces{}\hspace*{0pt} The default math font&\hspace*{0pt}\ignorespaces{}\hspace*{0pt} Most mathematical notation\\ \hline \hspace*{0pt}\ignorespaces{}\hspace*{0pt} \LaTeXTT{\textbackslash{}mathrm\{…\}}&\hspace*{0pt}\ignorespaces{}\hspace*{0pt} {$\mathrm{ABCDEF~abcdef~123456}\,$}&\hspace*{0pt}\ignorespaces{}\hspace*{0pt} This is the default or normal font, unitalicised&\hspace*{0pt}\ignorespaces{}\hspace*{0pt} Units of measurement, one word functions\\ \hline \hspace*{0pt}\ignorespaces{}\hspace*{0pt} \LaTeXTT{\textbackslash{}mathit\{…\}}&\hspace*{0pt}\ignorespaces{}\hspace*{0pt} {$\mathit{ABCDEF~abcdef~123456}\,$}&\hspace*{0pt}\ignorespaces{}\hspace*{0pt} Italicised font&\hspace*{0pt}\ignorespaces{}\hspace*{0pt} Multi-{}letter function or variable names. Compared to {\ttfamily \setmainfont[Path=/usr/share/fonts/truetype/cmu/,UprightFont=cmunrm.ttf,BoldFont=cmunbx.ttf,ItalicFont=cmunti.ttf,BoldItalicFont=cmunbi.ttf]{cmuntt.ttf}\setmonofont[Path=/usr/share/fonts/truetype/cmu/,UprightFont=cmuntt.ttf,BoldFont=cmuntb.ttf,ItalicFont=cmunit.ttf,BoldItalicFont=cmuntx.ttf]{cmuntt.ttf}\ttfamily \textbackslash{}mathnormal}\setmainfont[Path=/usr/share/fonts/truetype/cmu/,UprightFont=cmunrm.ttf,BoldFont=cmunbx.ttf,ItalicFont=cmunti.ttf,BoldItalicFont=cmunbi.ttf]{cmunrm.ttf}\setmonofont[Path=/usr/share/fonts/truetype/cmu/,UprightFont=cmuntt.ttf,BoldFont=cmuntb.ttf,ItalicFont=cmunit.ttf,BoldItalicFont=cmuntx.ttf]{cmunrm.ttf}, words are spaced more naturally and numbers are italicized as well.\\ \hline \hspace*{0pt}\ignorespaces{}\hspace*{0pt} \LaTeXTT{\textbackslash{}mathbf\{…\}}&\hspace*{0pt}\ignorespaces{}\hspace*{0pt} {$\mathbf{ABCDEF~abcdef~123456}\,$}&\hspace*{0pt}\ignorespaces{}\hspace*{0pt} Bold font&\hspace*{0pt}\ignorespaces{}\hspace*{0pt} Vectors\\ \hline \hspace*{0pt}\ignorespaces{}\hspace*{0pt} \LaTeXTT{\textbackslash{}mathsf\{…\}}&\hspace*{0pt}\ignorespaces{}\hspace*{0pt} {$\mathsf{ABCDEF~abcdef~123456}\,$}&\hspace*{0pt}\ignorespaces{}\hspace*{0pt} \myhref{https://en.wikipedia.org/wiki/sans-serif}{Sans-{}serif}&\hspace*{0pt}\ignorespaces{}\hspace*{0pt}\\ \hline \hspace*{0pt}\ignorespaces{}\hspace*{0pt} \LaTeXTT{\textbackslash{}mathtt\{…\}}&\hspace*{0pt}\ignorespaces{}\hspace*{0pt} {$\mathtt{ABCDEF~abcdef~123456}\,$}&\hspace*{0pt}\ignorespaces{}\hspace*{0pt} \myhref{https://en.wikipedia.org/wiki/Monospace\%20font}{Monospace (fixed-{}width) font}&\hspace*{0pt}\ignorespaces{}\hspace*{0pt}\\ \hline \hspace*{0pt}\ignorespaces{}\hspace*{0pt} \LaTeXTT{\textbackslash{}mathfrak\{…\}}&\hspace*{0pt}\ignorespaces{}\hspace*{0pt} {$\mathfrak{ABCDEF~abcdef~123456}\,$}&\hspace*{0pt}\ignorespaces{}\hspace*{0pt} \myhref{https://en.wikipedia.org/wiki/Fraktur\%20\%28script\%29}{Fraktur}&\hspace*{0pt}\ignorespaces{}\hspace*{0pt} Almost canonical font for Lie algebras, with superscript used to denote \myhref{https://en.wikipedia.org/wiki/List\%20of\%20New\%20Testament\%20papyri}{New Testament papyri}, \myhref{https://en.wikipedia.org/wiki/Ideal\%20\%28ring\%20theory\%29}{ideals} in ring theory\\ \hline \hspace*{0pt}\ignorespaces{}\hspace*{0pt} \LaTeXTT{\textbackslash{}mathcal\{…\}}&\hspace*{0pt}\ignorespaces{}\hspace*{0pt} {$\mathcal{ABCDEF}\,$}&\hspace*{0pt}\ignorespaces{}\hspace*{0pt} Calligraphy (uppercase only)&\hspace*{0pt}\ignorespaces{}\hspace*{0pt} Often used for sheaves/schemes and categories, used to denote \myhref{https://en.wikipedia.org/wiki/Cryptography}{cryptological} concepts like an {\itshape \setmainfont[Path=/usr/share/fonts/truetype/cmu/,UprightFont=cmunrm.ttf,BoldFont=cmunbx.ttf,ItalicFont=cmunti.ttf,BoldItalicFont=cmunbi.ttf]{cmunti.ttf}\setmonofont[Path=/usr/share/fonts/truetype/cmu/,UprightFont=cmuntt.ttf,BoldFont=cmuntb.ttf,ItalicFont=cmunit.ttf,BoldItalicFont=cmuntx.ttf]{cmunti.ttf}\itshape alphabet of definition}{$\text{ }$}\setmainfont[Path=/usr/share/fonts/truetype/cmu/,UprightFont=cmunrm.ttf,BoldFont=cmunbx.ttf,ItalicFont=cmunti.ttf,BoldItalicFont=cmunbi.ttf]{cmunrm.ttf}\setmonofont[Path=/usr/share/fonts/truetype/cmu/,UprightFont=cmuntt.ttf,BoldFont=cmuntb.ttf,ItalicFont=cmunit.ttf,BoldItalicFont=cmuntx.ttf]{cmunrm.ttf} ({$\mathcal{A}$}), {\itshape \setmainfont[Path=/usr/share/fonts/truetype/cmu/,UprightFont=cmunrm.ttf,BoldFont=cmunbx.ttf,ItalicFont=cmunti.ttf,BoldItalicFont=cmunbi.ttf]{cmunti.ttf}\setmonofont[Path=/usr/share/fonts/truetype/cmu/,UprightFont=cmuntt.ttf,BoldFont=cmuntb.ttf,ItalicFont=cmunit.ttf,BoldItalicFont=cmuntx.ttf]{cmunti.ttf}\itshape message space}{$\text{ }$}\setmainfont[Path=/usr/share/fonts/truetype/cmu/,UprightFont=cmunrm.ttf,BoldFont=cmunbx.ttf,ItalicFont=cmunti.ttf,BoldItalicFont=cmunbi.ttf]{cmunrm.ttf}\setmonofont[Path=/usr/share/fonts/truetype/cmu/,UprightFont=cmuntt.ttf,BoldFont=cmuntb.ttf,ItalicFont=cmunit.ttf,BoldItalicFont=cmuntx.ttf]{cmunrm.ttf} ({$\mathcal{M}$}), {\itshape \setmainfont[Path=/usr/share/fonts/truetype/cmu/,UprightFont=cmunrm.ttf,BoldFont=cmunbx.ttf,ItalicFont=cmunti.ttf,BoldItalicFont=cmunbi.ttf]{cmunti.ttf}\setmonofont[Path=/usr/share/fonts/truetype/cmu/,UprightFont=cmuntt.ttf,BoldFont=cmuntb.ttf,ItalicFont=cmunit.ttf,BoldItalicFont=cmuntx.ttf]{cmunti.ttf}\itshape ciphertext space}{$\text{ }$}\setmainfont[Path=/usr/share/fonts/truetype/cmu/,UprightFont=cmunrm.ttf,BoldFont=cmunbx.ttf,ItalicFont=cmunti.ttf,BoldItalicFont=cmunbi.ttf]{cmunrm.ttf}\setmonofont[Path=/usr/share/fonts/truetype/cmu/,UprightFont=cmuntt.ttf,BoldFont=cmuntb.ttf,ItalicFont=cmunit.ttf,BoldItalicFont=cmuntx.ttf]{cmunrm.ttf} ({$\mathcal{C}$}) and {\itshape \myhref{https://en.wikipedia.org/wiki/key\%20space}{\setmainfont[Path=/usr/share/fonts/truetype/cmu/,UprightFont=cmunrm.ttf,BoldFont=cmunbx.ttf,ItalicFont=cmunti.ttf,BoldItalicFont=cmunbi.ttf]{cmunti.ttf}\setmonofont[Path=/usr/share/fonts/truetype/cmu/,UprightFont=cmuntt.ttf,BoldFont=cmuntb.ttf,ItalicFont=cmunit.ttf,BoldItalicFont=cmuntx.ttf]{cmunti.ttf}\itshape key space}} ({$\mathcal{K}$}); \myhref{https://en.wikipedia.org/wiki/Kleene\%27s\%20O}{Kleene\textquotesingle{}s {$\mathcal{O}$}}; \myhref{https://en.wikipedia.org/wiki/Description\%20logic\%23Naming\%20Convention}{naming convention in description logic}; \myhref{https://en.wikipedia.org/wiki/Laplace\%20transform}{Laplace transform} ({$\mathcal{L}$}) and \myhref{https://en.wikipedia.org/wiki/Fourier\%20transform}{Fourier transform} ({$\mathcal{F}$})\\ \hline \hspace*{0pt}\ignorespaces{}\hspace*{0pt} \LaTeXTT{\textbackslash{}mathbb\{…\}}\newline{}{\small (requires the \LaTeXTT{amsfonts} or \LaTeXTT{amssymb} package)}&\hspace*{0pt}\ignorespaces{}\hspace*{0pt} {$\mathbb{ABCDEF}\,$}&\hspace*{0pt}\ignorespaces{}\hspace*{0pt} \myhref{https://en.wikipedia.org/wiki/Blackboard\%20bold}{Blackboard bold} (uppercase only)&\hspace*{0pt}\ignorespaces{}\hspace*{0pt} Used to denote special sets (e.g. real numbers)\\ \hline \hspace*{0pt}\ignorespaces{}\hspace*{0pt} \LaTeXTT{\textbackslash{}mathscr\{…\}}\newline{}{\small (requires the \LaTeXTT{mathrsfs} package)}&\hspace*{0pt}\ignorespaces{}\hspace*{0pt} \begin{minipage}{1.0\linewidth}\begin{center}\includegraphics[width=1.0\linewidth,height=6.5in,keepaspectratio]{../images/87.png}\end{center}\myfigurewithoutcaption{87}\end{minipage}&\hspace*{0pt}\ignorespaces{}\hspace*{0pt} \myhref{https://en.wikipedia.org/wiki/Script\%20\%28typefaces\%29}{Script} (uppercase only)&\hspace*{0pt}\ignorespaces{}\hspace*{0pt} An alternative font for categories and sheaves.\\ \hline 
\end{longtable}
}}

These formatting commands can be wrapped around the entire equation, and not just on the textual elements: they only format letters, numbers, and uppercase Greek, and other math commands are unaffected. 

To bold lowercase Greek or other symbols use the \LaTeXTT{\textbackslash{}boldsymbol} command\myfootnote{requires the \LaTeXTT{amsmath} package}; this will only work if there exists a bold version of the symbol in the current font. As a last resort there is the \LaTeXTT{\textbackslash{}pmb} command\myfootnote{requires the \LaTeXTT{amsmath} package} (poor mans bold): this prints multiple versions of the character slightly offset against each other.
\begin{longtable}{p{1.0\linewidth}}
\begin{Shaded}
\begin{Highlighting}[]


 \NormalTok{\textbackslash{}boldsymbol\{\textbackslash{}beta\} = (\textbackslash{}beta_1,\textbackslash{}beta_2,\textbackslash{}dotsc,\textbackslash{}beta_n)}
 
\end{Highlighting}
\end{Shaded}
\\

{$  \boldsymbol{\beta} = (\beta_1,\beta_2,\dotsc,\beta_n) \,$}

\end{longtable}
To change the size of the fonts in math mode, see \mylref{552}{Changing font size}.
\subsection{Accents}
\label{518}
So what to do when you run out of symbols and fonts? Well the next step is to use accents:

\begin{longtable}{>{\RaggedRight}p{0.33308\linewidth}>{\RaggedRight}p{0.07382\linewidth}>{\RaggedRight}p{0.40766\linewidth}>{\RaggedRight}p{0.07115\linewidth}} 
\hspace*{0pt}\ignorespaces{}\hspace*{0pt} {\ttfamily \setmainfont[Path=/usr/share/fonts/truetype/cmu/,UprightFont=cmunrm.ttf,BoldFont=cmunbx.ttf,ItalicFont=cmunti.ttf,BoldItalicFont=cmunbi.ttf]{cmuntt.ttf}\setmonofont[Path=/usr/share/fonts/truetype/cmu/,UprightFont=cmuntt.ttf,BoldFont=cmuntb.ttf,ItalicFont=cmunit.ttf,BoldItalicFont=cmuntx.ttf]{cmuntt.ttf}\ttfamily a\textquotesingle{} or a\^{}\{\textbackslash{}prime\}}{$\text{ }$}\setmainfont[Path=/usr/share/fonts/truetype/cmu/,UprightFont=cmunrm.ttf,BoldFont=cmunbx.ttf,ItalicFont=cmunti.ttf,BoldItalicFont=cmunbi.ttf]{cmunrm.ttf}\setmonofont[Path=/usr/share/fonts/truetype/cmu/,UprightFont=cmuntt.ttf,BoldFont=cmuntb.ttf,ItalicFont=cmunit.ttf,BoldItalicFont=cmuntx.ttf]{cmunrm.ttf} &\hspace*{0pt}\ignorespaces{}\hspace*{0pt} {$a'\,$} &\hspace*{0pt}\ignorespaces{}\hspace*{0pt} {\ttfamily \setmainfont[Path=/usr/share/fonts/truetype/cmu/,UprightFont=cmunrm.ttf,BoldFont=cmunbx.ttf,ItalicFont=cmunti.ttf,BoldItalicFont=cmunbi.ttf]{cmuntt.ttf}\setmonofont[Path=/usr/share/fonts/truetype/cmu/,UprightFont=cmuntt.ttf,BoldFont=cmuntb.ttf,ItalicFont=cmunit.ttf,BoldItalicFont=cmuntx.ttf]{cmuntt.ttf}\ttfamily a\textquotesingle{}\textquotesingle{}}{$\text{ }$}\setmainfont[Path=/usr/share/fonts/truetype/cmu/,UprightFont=cmunrm.ttf,BoldFont=cmunbx.ttf,ItalicFont=cmunti.ttf,BoldItalicFont=cmunbi.ttf]{cmunrm.ttf}\setmonofont[Path=/usr/share/fonts/truetype/cmu/,UprightFont=cmuntt.ttf,BoldFont=cmuntb.ttf,ItalicFont=cmunit.ttf,BoldItalicFont=cmuntx.ttf]{cmunrm.ttf} &\hspace*{0pt}\ignorespaces{}\hspace*{0pt} {$a''\,$}\\ \hspace*{0pt}\ignorespaces{}\hspace*{0pt} {\ttfamily \setmainfont[Path=/usr/share/fonts/truetype/cmu/,UprightFont=cmunrm.ttf,BoldFont=cmunbx.ttf,ItalicFont=cmunti.ttf,BoldItalicFont=cmunbi.ttf]{cmuntt.ttf}\setmonofont[Path=/usr/share/fonts/truetype/cmu/,UprightFont=cmuntt.ttf,BoldFont=cmuntb.ttf,ItalicFont=cmunit.ttf,BoldItalicFont=cmuntx.ttf]{cmuntt.ttf}\ttfamily \textbackslash{}hat\{a\}}{$\text{ }$}\setmainfont[Path=/usr/share/fonts/truetype/cmu/,UprightFont=cmunrm.ttf,BoldFont=cmunbx.ttf,ItalicFont=cmunti.ttf,BoldItalicFont=cmunbi.ttf]{cmunrm.ttf}\setmonofont[Path=/usr/share/fonts/truetype/cmu/,UprightFont=cmuntt.ttf,BoldFont=cmuntb.ttf,ItalicFont=cmunit.ttf,BoldItalicFont=cmuntx.ttf]{cmunrm.ttf} &\hspace*{0pt}\ignorespaces{}\hspace*{0pt} {$\hat{a} \,$} &\hspace*{0pt}\ignorespaces{}\hspace*{0pt} {\ttfamily \setmainfont[Path=/usr/share/fonts/truetype/cmu/,UprightFont=cmunrm.ttf,BoldFont=cmunbx.ttf,ItalicFont=cmunti.ttf,BoldItalicFont=cmunbi.ttf]{cmuntt.ttf}\setmonofont[Path=/usr/share/fonts/truetype/cmu/,UprightFont=cmuntt.ttf,BoldFont=cmuntb.ttf,ItalicFont=cmunit.ttf,BoldItalicFont=cmuntx.ttf]{cmuntt.ttf}\ttfamily \textbackslash{}bar\{a\}}{$\text{ }$}\setmainfont[Path=/usr/share/fonts/truetype/cmu/,UprightFont=cmunrm.ttf,BoldFont=cmunbx.ttf,ItalicFont=cmunti.ttf,BoldItalicFont=cmunbi.ttf]{cmunrm.ttf}\setmonofont[Path=/usr/share/fonts/truetype/cmu/,UprightFont=cmuntt.ttf,BoldFont=cmuntb.ttf,ItalicFont=cmunit.ttf,BoldItalicFont=cmuntx.ttf]{cmunrm.ttf} &\hspace*{0pt}\ignorespaces{}\hspace*{0pt} {$\bar{a} \,$} \\ \hspace*{0pt}\ignorespaces{}\hspace*{0pt} {\ttfamily \setmainfont[Path=/usr/share/fonts/truetype/cmu/,UprightFont=cmunrm.ttf,BoldFont=cmunbx.ttf,ItalicFont=cmunti.ttf,BoldItalicFont=cmunbi.ttf]{cmuntt.ttf}\setmonofont[Path=/usr/share/fonts/truetype/cmu/,UprightFont=cmuntt.ttf,BoldFont=cmuntb.ttf,ItalicFont=cmunit.ttf,BoldItalicFont=cmuntx.ttf]{cmuntt.ttf}\ttfamily \textbackslash{}grave\{a\}}{$\text{ }$}\setmainfont[Path=/usr/share/fonts/truetype/cmu/,UprightFont=cmunrm.ttf,BoldFont=cmunbx.ttf,ItalicFont=cmunti.ttf,BoldItalicFont=cmunbi.ttf]{cmunrm.ttf}\setmonofont[Path=/usr/share/fonts/truetype/cmu/,UprightFont=cmuntt.ttf,BoldFont=cmuntb.ttf,ItalicFont=cmunit.ttf,BoldItalicFont=cmuntx.ttf]{cmunrm.ttf} &\hspace*{0pt}\ignorespaces{}\hspace*{0pt} {$\grave{a} \,$} &\hspace*{0pt}\ignorespaces{}\hspace*{0pt} {\ttfamily \setmainfont[Path=/usr/share/fonts/truetype/cmu/,UprightFont=cmunrm.ttf,BoldFont=cmunbx.ttf,ItalicFont=cmunti.ttf,BoldItalicFont=cmunbi.ttf]{cmuntt.ttf}\setmonofont[Path=/usr/share/fonts/truetype/cmu/,UprightFont=cmuntt.ttf,BoldFont=cmuntb.ttf,ItalicFont=cmunit.ttf,BoldItalicFont=cmuntx.ttf]{cmuntt.ttf}\ttfamily \textbackslash{}acute\{a\}}{$\text{ }$}\setmainfont[Path=/usr/share/fonts/truetype/cmu/,UprightFont=cmunrm.ttf,BoldFont=cmunbx.ttf,ItalicFont=cmunti.ttf,BoldItalicFont=cmunbi.ttf]{cmunrm.ttf}\setmonofont[Path=/usr/share/fonts/truetype/cmu/,UprightFont=cmuntt.ttf,BoldFont=cmuntb.ttf,ItalicFont=cmunit.ttf,BoldItalicFont=cmuntx.ttf]{cmunrm.ttf} &\hspace*{0pt}\ignorespaces{}\hspace*{0pt} {$\acute{a} \,$} \\ \hspace*{0pt}\ignorespaces{}\hspace*{0pt} {\ttfamily \setmainfont[Path=/usr/share/fonts/truetype/cmu/,UprightFont=cmunrm.ttf,BoldFont=cmunbx.ttf,ItalicFont=cmunti.ttf,BoldItalicFont=cmunbi.ttf]{cmuntt.ttf}\setmonofont[Path=/usr/share/fonts/truetype/cmu/,UprightFont=cmuntt.ttf,BoldFont=cmuntb.ttf,ItalicFont=cmunit.ttf,BoldItalicFont=cmuntx.ttf]{cmuntt.ttf}\ttfamily \textbackslash{}dot\{a\}}{$\text{ }$}\setmainfont[Path=/usr/share/fonts/truetype/cmu/,UprightFont=cmunrm.ttf,BoldFont=cmunbx.ttf,ItalicFont=cmunti.ttf,BoldItalicFont=cmunbi.ttf]{cmunrm.ttf}\setmonofont[Path=/usr/share/fonts/truetype/cmu/,UprightFont=cmuntt.ttf,BoldFont=cmuntb.ttf,ItalicFont=cmunit.ttf,BoldItalicFont=cmuntx.ttf]{cmunrm.ttf} &\hspace*{0pt}\ignorespaces{}\hspace*{0pt} {$\dot{a} \,$} &\hspace*{0pt}\ignorespaces{}\hspace*{0pt} {\ttfamily \setmainfont[Path=/usr/share/fonts/truetype/cmu/,UprightFont=cmunrm.ttf,BoldFont=cmunbx.ttf,ItalicFont=cmunti.ttf,BoldItalicFont=cmunbi.ttf]{cmuntt.ttf}\setmonofont[Path=/usr/share/fonts/truetype/cmu/,UprightFont=cmuntt.ttf,BoldFont=cmuntb.ttf,ItalicFont=cmunit.ttf,BoldItalicFont=cmuntx.ttf]{cmuntt.ttf}\ttfamily \textbackslash{}ddot\{a\}}{$\text{ }$}\setmainfont[Path=/usr/share/fonts/truetype/cmu/,UprightFont=cmunrm.ttf,BoldFont=cmunbx.ttf,ItalicFont=cmunti.ttf,BoldItalicFont=cmunbi.ttf]{cmunrm.ttf}\setmonofont[Path=/usr/share/fonts/truetype/cmu/,UprightFont=cmuntt.ttf,BoldFont=cmuntb.ttf,ItalicFont=cmunit.ttf,BoldItalicFont=cmuntx.ttf]{cmunrm.ttf} &\hspace*{0pt}\ignorespaces{}\hspace*{0pt} {$\ddot{a} \,$} \\ \hspace*{0pt}\ignorespaces{}\hspace*{0pt} {\ttfamily \setmainfont[Path=/usr/share/fonts/truetype/cmu/,UprightFont=cmunrm.ttf,BoldFont=cmunbx.ttf,ItalicFont=cmunti.ttf,BoldItalicFont=cmunbi.ttf]{cmuntt.ttf}\setmonofont[Path=/usr/share/fonts/truetype/cmu/,UprightFont=cmuntt.ttf,BoldFont=cmuntb.ttf,ItalicFont=cmunit.ttf,BoldItalicFont=cmuntx.ttf]{cmuntt.ttf}\ttfamily \textbackslash{}not\{a\}}{$\text{ }$}\setmainfont[Path=/usr/share/fonts/truetype/cmu/,UprightFont=cmunrm.ttf,BoldFont=cmunbx.ttf,ItalicFont=cmunti.ttf,BoldItalicFont=cmunbi.ttf]{cmunrm.ttf}\setmonofont[Path=/usr/share/fonts/truetype/cmu/,UprightFont=cmuntt.ttf,BoldFont=cmuntb.ttf,ItalicFont=cmunit.ttf,BoldItalicFont=cmuntx.ttf]{cmunrm.ttf} &\hspace*{0pt}\ignorespaces{}\hspace*{0pt} {$\not{a} \,$} &\hspace*{0pt}\ignorespaces{}\hspace*{0pt} {\ttfamily \setmainfont[Path=/usr/share/fonts/truetype/cmu/,UprightFont=cmunrm.ttf,BoldFont=cmunbx.ttf,ItalicFont=cmunti.ttf,BoldItalicFont=cmunbi.ttf]{cmuntt.ttf}\setmonofont[Path=/usr/share/fonts/truetype/cmu/,UprightFont=cmuntt.ttf,BoldFont=cmuntb.ttf,ItalicFont=cmunit.ttf,BoldItalicFont=cmuntx.ttf]{cmuntt.ttf}\ttfamily \textbackslash{}mathring\{a\}}{$\text{ }$}\setmainfont[Path=/usr/share/fonts/truetype/cmu/,UprightFont=cmunrm.ttf,BoldFont=cmunbx.ttf,ItalicFont=cmunti.ttf,BoldItalicFont=cmunbi.ttf]{cmunrm.ttf}\setmonofont[Path=/usr/share/fonts/truetype/cmu/,UprightFont=cmuntt.ttf,BoldFont=cmuntb.ttf,ItalicFont=cmunit.ttf,BoldItalicFont=cmuntx.ttf]{cmunrm.ttf} &\hspace*{0pt}\ignorespaces{}\hspace*{0pt}\\ \hspace*{0pt}\ignorespaces{}\hspace*{0pt} {\ttfamily \setmainfont[Path=/usr/share/fonts/truetype/cmu/,UprightFont=cmunrm.ttf,BoldFont=cmunbx.ttf,ItalicFont=cmunti.ttf,BoldItalicFont=cmunbi.ttf]{cmuntt.ttf}\setmonofont[Path=/usr/share/fonts/truetype/cmu/,UprightFont=cmuntt.ttf,BoldFont=cmuntb.ttf,ItalicFont=cmunit.ttf,BoldItalicFont=cmuntx.ttf]{cmuntt.ttf}\ttfamily \textbackslash{}overrightarrow\{AB\}}{$\text{ }$}\setmainfont[Path=/usr/share/fonts/truetype/cmu/,UprightFont=cmunrm.ttf,BoldFont=cmunbx.ttf,ItalicFont=cmunti.ttf,BoldItalicFont=cmunbi.ttf]{cmunrm.ttf}\setmonofont[Path=/usr/share/fonts/truetype/cmu/,UprightFont=cmuntt.ttf,BoldFont=cmuntb.ttf,ItalicFont=cmunit.ttf,BoldItalicFont=cmuntx.ttf]{cmunrm.ttf} &\hspace*{0pt}\ignorespaces{}\hspace*{0pt} {$\overrightarrow{AB} \,$} &\hspace*{0pt}\ignorespaces{}\hspace*{0pt} {\ttfamily \setmainfont[Path=/usr/share/fonts/truetype/cmu/,UprightFont=cmunrm.ttf,BoldFont=cmunbx.ttf,ItalicFont=cmunti.ttf,BoldItalicFont=cmunbi.ttf]{cmuntt.ttf}\setmonofont[Path=/usr/share/fonts/truetype/cmu/,UprightFont=cmuntt.ttf,BoldFont=cmuntb.ttf,ItalicFont=cmunit.ttf,BoldItalicFont=cmuntx.ttf]{cmuntt.ttf}\ttfamily \textbackslash{}overleftarrow\{AB\}}{$\text{ }$}\setmainfont[Path=/usr/share/fonts/truetype/cmu/,UprightFont=cmunrm.ttf,BoldFont=cmunbx.ttf,ItalicFont=cmunti.ttf,BoldItalicFont=cmunbi.ttf]{cmunrm.ttf}\setmonofont[Path=/usr/share/fonts/truetype/cmu/,UprightFont=cmuntt.ttf,BoldFont=cmuntb.ttf,ItalicFont=cmunit.ttf,BoldItalicFont=cmuntx.ttf]{cmunrm.ttf} &\hspace*{0pt}\ignorespaces{}\hspace*{0pt} {$\overleftarrow{AB} \,$} \\ \hspace*{0pt}\ignorespaces{}\hspace*{0pt} {\ttfamily \setmainfont[Path=/usr/share/fonts/truetype/cmu/,UprightFont=cmunrm.ttf,BoldFont=cmunbx.ttf,ItalicFont=cmunti.ttf,BoldItalicFont=cmunbi.ttf]{cmuntt.ttf}\setmonofont[Path=/usr/share/fonts/truetype/cmu/,UprightFont=cmuntt.ttf,BoldFont=cmuntb.ttf,ItalicFont=cmunit.ttf,BoldItalicFont=cmuntx.ttf]{cmuntt.ttf}\ttfamily a\textquotesingle{}\textquotesingle{}\textquotesingle{}}{$\text{ }$}\setmainfont[Path=/usr/share/fonts/truetype/cmu/,UprightFont=cmunrm.ttf,BoldFont=cmunbx.ttf,ItalicFont=cmunti.ttf,BoldItalicFont=cmunbi.ttf]{cmunrm.ttf}\setmonofont[Path=/usr/share/fonts/truetype/cmu/,UprightFont=cmuntt.ttf,BoldFont=cmuntb.ttf,ItalicFont=cmunit.ttf,BoldItalicFont=cmuntx.ttf]{cmunrm.ttf} &\hspace*{0pt}\ignorespaces{}\hspace*{0pt} {$a'''\,$}&\hspace*{0pt}\ignorespaces{}\hspace*{0pt} {\ttfamily \setmainfont[Path=/usr/share/fonts/truetype/cmu/,UprightFont=cmunrm.ttf,BoldFont=cmunbx.ttf,ItalicFont=cmunti.ttf,BoldItalicFont=cmunbi.ttf]{cmuntt.ttf}\setmonofont[Path=/usr/share/fonts/truetype/cmu/,UprightFont=cmuntt.ttf,BoldFont=cmuntb.ttf,ItalicFont=cmunit.ttf,BoldItalicFont=cmuntx.ttf]{cmuntt.ttf}\ttfamily a\textquotesingle{}\textquotesingle{}\textquotesingle{}\textquotesingle{}}{$\text{ }$}\setmainfont[Path=/usr/share/fonts/truetype/cmu/,UprightFont=cmunrm.ttf,BoldFont=cmunbx.ttf,ItalicFont=cmunti.ttf,BoldItalicFont=cmunbi.ttf]{cmunrm.ttf}\setmonofont[Path=/usr/share/fonts/truetype/cmu/,UprightFont=cmuntt.ttf,BoldFont=cmuntb.ttf,ItalicFont=cmunit.ttf,BoldItalicFont=cmuntx.ttf]{cmunrm.ttf} &\hspace*{0pt}\ignorespaces{}\hspace*{0pt} {$a''''\,$}\\ \hspace*{0pt}\ignorespaces{}\hspace*{0pt} {\ttfamily \setmainfont[Path=/usr/share/fonts/truetype/cmu/,UprightFont=cmunrm.ttf,BoldFont=cmunbx.ttf,ItalicFont=cmunti.ttf,BoldItalicFont=cmunbi.ttf]{cmuntt.ttf}\setmonofont[Path=/usr/share/fonts/truetype/cmu/,UprightFont=cmuntt.ttf,BoldFont=cmuntb.ttf,ItalicFont=cmunit.ttf,BoldItalicFont=cmuntx.ttf]{cmuntt.ttf}\ttfamily \textbackslash{}overline\{aaa\}}{$\text{ }$}\setmainfont[Path=/usr/share/fonts/truetype/cmu/,UprightFont=cmunrm.ttf,BoldFont=cmunbx.ttf,ItalicFont=cmunti.ttf,BoldItalicFont=cmunbi.ttf]{cmunrm.ttf}\setmonofont[Path=/usr/share/fonts/truetype/cmu/,UprightFont=cmuntt.ttf,BoldFont=cmuntb.ttf,ItalicFont=cmunit.ttf,BoldItalicFont=cmuntx.ttf]{cmunrm.ttf} &\hspace*{0pt}\ignorespaces{}\hspace*{0pt} {$\overline{aaa} \,$} &\hspace*{0pt}\ignorespaces{}\hspace*{0pt} {\ttfamily \setmainfont[Path=/usr/share/fonts/truetype/cmu/,UprightFont=cmunrm.ttf,BoldFont=cmunbx.ttf,ItalicFont=cmunti.ttf,BoldItalicFont=cmunbi.ttf]{cmuntt.ttf}\setmonofont[Path=/usr/share/fonts/truetype/cmu/,UprightFont=cmuntt.ttf,BoldFont=cmuntb.ttf,ItalicFont=cmunit.ttf,BoldItalicFont=cmuntx.ttf]{cmuntt.ttf}\ttfamily \textbackslash{}check\{a\}}{$\text{ }$}\setmainfont[Path=/usr/share/fonts/truetype/cmu/,UprightFont=cmunrm.ttf,BoldFont=cmunbx.ttf,ItalicFont=cmunti.ttf,BoldItalicFont=cmunbi.ttf]{cmunrm.ttf}\setmonofont[Path=/usr/share/fonts/truetype/cmu/,UprightFont=cmuntt.ttf,BoldFont=cmuntb.ttf,ItalicFont=cmunit.ttf,BoldItalicFont=cmuntx.ttf]{cmunrm.ttf} &\hspace*{0pt}\ignorespaces{}\hspace*{0pt} {$\check{a} \,$} \\ \hspace*{0pt}\ignorespaces{}\hspace*{0pt} {\ttfamily \setmainfont[Path=/usr/share/fonts/truetype/cmu/,UprightFont=cmunrm.ttf,BoldFont=cmunbx.ttf,ItalicFont=cmunti.ttf,BoldItalicFont=cmunbi.ttf]{cmuntt.ttf}\setmonofont[Path=/usr/share/fonts/truetype/cmu/,UprightFont=cmuntt.ttf,BoldFont=cmuntb.ttf,ItalicFont=cmunit.ttf,BoldItalicFont=cmuntx.ttf]{cmuntt.ttf}\ttfamily \textbackslash{}breve\{a\}}{$\text{ }$}\setmainfont[Path=/usr/share/fonts/truetype/cmu/,UprightFont=cmunrm.ttf,BoldFont=cmunbx.ttf,ItalicFont=cmunti.ttf,BoldItalicFont=cmunbi.ttf]{cmunrm.ttf}\setmonofont[Path=/usr/share/fonts/truetype/cmu/,UprightFont=cmuntt.ttf,BoldFont=cmuntb.ttf,ItalicFont=cmunit.ttf,BoldItalicFont=cmuntx.ttf]{cmunrm.ttf} &\hspace*{0pt}\ignorespaces{}\hspace*{0pt} {$\breve{a} \,$} &\hspace*{0pt}\ignorespaces{}\hspace*{0pt} {\ttfamily \setmainfont[Path=/usr/share/fonts/truetype/cmu/,UprightFont=cmunrm.ttf,BoldFont=cmunbx.ttf,ItalicFont=cmunti.ttf,BoldItalicFont=cmunbi.ttf]{cmuntt.ttf}\setmonofont[Path=/usr/share/fonts/truetype/cmu/,UprightFont=cmuntt.ttf,BoldFont=cmuntb.ttf,ItalicFont=cmunit.ttf,BoldItalicFont=cmuntx.ttf]{cmuntt.ttf}\ttfamily \textbackslash{}vec\{a\}}{$\text{ }$}\setmainfont[Path=/usr/share/fonts/truetype/cmu/,UprightFont=cmunrm.ttf,BoldFont=cmunbx.ttf,ItalicFont=cmunti.ttf,BoldItalicFont=cmunbi.ttf]{cmunrm.ttf}\setmonofont[Path=/usr/share/fonts/truetype/cmu/,UprightFont=cmuntt.ttf,BoldFont=cmuntb.ttf,ItalicFont=cmunit.ttf,BoldItalicFont=cmuntx.ttf]{cmunrm.ttf} &\hspace*{0pt}\ignorespaces{}\hspace*{0pt} {$\vec{a} \,$} \\ \hspace*{0pt}\ignorespaces{}\hspace*{0pt} {\ttfamily \setmainfont[Path=/usr/share/fonts/truetype/cmu/,UprightFont=cmunrm.ttf,BoldFont=cmunbx.ttf,ItalicFont=cmunti.ttf,BoldItalicFont=cmunbi.ttf]{cmuntt.ttf}\setmonofont[Path=/usr/share/fonts/truetype/cmu/,UprightFont=cmuntt.ttf,BoldFont=cmuntb.ttf,ItalicFont=cmunit.ttf,BoldItalicFont=cmuntx.ttf]{cmuntt.ttf}\ttfamily \textbackslash{}dddot\{a\}}\myfootnote{\setmainfont[Path=/usr/share/fonts/truetype/cmu/,UprightFont=cmunrm.ttf,BoldFont=cmunbx.ttf,ItalicFont=cmunti.ttf,BoldItalicFont=cmunbi.ttf]{cmunrm.ttf}\setmonofont[Path=/usr/share/fonts/truetype/cmu/,UprightFont=cmuntt.ttf,BoldFont=cmuntb.ttf,ItalicFont=cmunit.ttf,BoldItalicFont=cmuntx.ttf]{cmunrm.ttf}requires the \LaTeXTT{amsmath} package} &\hspace*{0pt}\ignorespaces{}\hspace*{0pt}&\hspace*{0pt}\ignorespaces{}\hspace*{0pt} {\ttfamily \setmainfont[Path=/usr/share/fonts/truetype/cmu/,UprightFont=cmunrm.ttf,BoldFont=cmunbx.ttf,ItalicFont=cmunti.ttf,BoldItalicFont=cmunbi.ttf]{cmuntt.ttf}\setmonofont[Path=/usr/share/fonts/truetype/cmu/,UprightFont=cmuntt.ttf,BoldFont=cmuntb.ttf,ItalicFont=cmunit.ttf,BoldItalicFont=cmuntx.ttf]{cmuntt.ttf}\ttfamily \textbackslash{}ddddot\{a\}}\myfootnote{\setmainfont[Path=/usr/share/fonts/truetype/cmu/,UprightFont=cmunrm.ttf,BoldFont=cmunbx.ttf,ItalicFont=cmunti.ttf,BoldItalicFont=cmunbi.ttf]{cmunrm.ttf}\setmonofont[Path=/usr/share/fonts/truetype/cmu/,UprightFont=cmuntt.ttf,BoldFont=cmuntb.ttf,ItalicFont=cmunit.ttf,BoldItalicFont=cmuntx.ttf]{cmunrm.ttf}requires the \LaTeXTT{amsmath} package} &\hspace*{0pt}\ignorespaces{}\hspace*{0pt} \\ \hspace*{0pt}\ignorespaces{}\hspace*{0pt} {\ttfamily \setmainfont[Path=/usr/share/fonts/truetype/cmu/,UprightFont=cmunrm.ttf,BoldFont=cmunbx.ttf,ItalicFont=cmunti.ttf,BoldItalicFont=cmunbi.ttf]{cmuntt.ttf}\setmonofont[Path=/usr/share/fonts/truetype/cmu/,UprightFont=cmuntt.ttf,BoldFont=cmuntb.ttf,ItalicFont=cmunit.ttf,BoldItalicFont=cmuntx.ttf]{cmuntt.ttf}\ttfamily \textbackslash{}widehat\{AAA\}}{$\text{ }$}\setmainfont[Path=/usr/share/fonts/truetype/cmu/,UprightFont=cmunrm.ttf,BoldFont=cmunbx.ttf,ItalicFont=cmunti.ttf,BoldItalicFont=cmunbi.ttf]{cmunrm.ttf}\setmonofont[Path=/usr/share/fonts/truetype/cmu/,UprightFont=cmuntt.ttf,BoldFont=cmuntb.ttf,ItalicFont=cmunit.ttf,BoldItalicFont=cmuntx.ttf]{cmunrm.ttf} &\hspace*{0pt}\ignorespaces{}\hspace*{0pt} {$\widehat{AAA} \,$} &\hspace*{0pt}\ignorespaces{}\hspace*{0pt} {\ttfamily \setmainfont[Path=/usr/share/fonts/truetype/cmu/,UprightFont=cmunrm.ttf,BoldFont=cmunbx.ttf,ItalicFont=cmunti.ttf,BoldItalicFont=cmunbi.ttf]{cmuntt.ttf}\setmonofont[Path=/usr/share/fonts/truetype/cmu/,UprightFont=cmuntt.ttf,BoldFont=cmuntb.ttf,ItalicFont=cmunit.ttf,BoldItalicFont=cmuntx.ttf]{cmuntt.ttf}\ttfamily \textbackslash{}widetilde\{AAA\}}{$\text{ }$}\setmainfont[Path=/usr/share/fonts/truetype/cmu/,UprightFont=cmunrm.ttf,BoldFont=cmunbx.ttf,ItalicFont=cmunti.ttf,BoldItalicFont=cmunbi.ttf]{cmunrm.ttf}\setmonofont[Path=/usr/share/fonts/truetype/cmu/,UprightFont=cmuntt.ttf,BoldFont=cmuntb.ttf,ItalicFont=cmunit.ttf,BoldItalicFont=cmuntx.ttf]{cmunrm.ttf} &\hspace*{0pt}\ignorespaces{}\hspace*{0pt} {$\widetilde{AAA}$}\\ \hspace*{0pt}\ignorespaces{}\hspace*{0pt} {\ttfamily \setmainfont[Path=/usr/share/fonts/truetype/cmu/,UprightFont=cmunrm.ttf,BoldFont=cmunbx.ttf,ItalicFont=cmunti.ttf,BoldItalicFont=cmunbi.ttf]{cmuntt.ttf}\setmonofont[Path=/usr/share/fonts/truetype/cmu/,UprightFont=cmuntt.ttf,BoldFont=cmuntb.ttf,ItalicFont=cmunit.ttf,BoldItalicFont=cmuntx.ttf]{cmuntt.ttf}\ttfamily \textbackslash{}widehat\{AAA\}}{$\text{ }$}\setmainfont[Path=/usr/share/fonts/truetype/cmu/,UprightFont=cmunrm.ttf,BoldFont=cmunbx.ttf,ItalicFont=cmunti.ttf,BoldItalicFont=cmunbi.ttf]{cmunrm.ttf}\setmonofont[Path=/usr/share/fonts/truetype/cmu/,UprightFont=cmuntt.ttf,BoldFont=cmuntb.ttf,ItalicFont=cmunit.ttf,BoldItalicFont=cmuntx.ttf]{cmunrm.ttf} &\hspace*{0pt}\ignorespaces{}\hspace*{0pt} {$\widehat{AAA} \,$} &\hspace*{0pt}\ignorespaces{}\hspace*{0pt} {\ttfamily \setmainfont[Path=/usr/share/fonts/truetype/cmu/,UprightFont=cmunrm.ttf,BoldFont=cmunbx.ttf,ItalicFont=cmunti.ttf,BoldItalicFont=cmunbi.ttf]{cmuntt.ttf}\setmonofont[Path=/usr/share/fonts/truetype/cmu/,UprightFont=cmuntt.ttf,BoldFont=cmuntb.ttf,ItalicFont=cmunit.ttf,BoldItalicFont=cmuntx.ttf]{cmuntt.ttf}\ttfamily \textbackslash{}stackrel\textbackslash{}frown\{AAA\}}{$\text{ }$}\setmainfont[Path=/usr/share/fonts/truetype/cmu/,UprightFont=cmunrm.ttf,BoldFont=cmunbx.ttf,ItalicFont=cmunti.ttf,BoldItalicFont=cmunbi.ttf]{cmunrm.ttf}\setmonofont[Path=/usr/share/fonts/truetype/cmu/,UprightFont=cmuntt.ttf,BoldFont=cmuntb.ttf,ItalicFont=cmunit.ttf,BoldItalicFont=cmuntx.ttf]{cmunrm.ttf} &\hspace*{0pt}\ignorespaces{}\hspace*{0pt} {$\stackrel\frown{AAA}$}\\ \hspace*{0pt}\ignorespaces{}\hspace*{0pt} {\ttfamily \setmainfont[Path=/usr/share/fonts/truetype/cmu/,UprightFont=cmunrm.ttf,BoldFont=cmunbx.ttf,ItalicFont=cmunti.ttf,BoldItalicFont=cmunbi.ttf]{cmuntt.ttf}\setmonofont[Path=/usr/share/fonts/truetype/cmu/,UprightFont=cmuntt.ttf,BoldFont=cmuntb.ttf,ItalicFont=cmunit.ttf,BoldItalicFont=cmuntx.ttf]{cmuntt.ttf}\ttfamily \textbackslash{}tilde\{a\}}{$\text{ }$}\setmainfont[Path=/usr/share/fonts/truetype/cmu/,UprightFont=cmunrm.ttf,BoldFont=cmunbx.ttf,ItalicFont=cmunti.ttf,BoldItalicFont=cmunbi.ttf]{cmunrm.ttf}\setmonofont[Path=/usr/share/fonts/truetype/cmu/,UprightFont=cmuntt.ttf,BoldFont=cmuntb.ttf,ItalicFont=cmunit.ttf,BoldItalicFont=cmuntx.ttf]{cmunrm.ttf} &\hspace*{0pt}\ignorespaces{}\hspace*{0pt} {$\tilde{a} \,$}&\hspace*{0pt}\ignorespaces{}\hspace*{0pt} {\ttfamily \setmainfont[Path=/usr/share/fonts/truetype/cmu/,UprightFont=cmunrm.ttf,BoldFont=cmunbx.ttf,ItalicFont=cmunti.ttf,BoldItalicFont=cmunbi.ttf]{cmuntt.ttf}\setmonofont[Path=/usr/share/fonts/truetype/cmu/,UprightFont=cmuntt.ttf,BoldFont=cmuntb.ttf,ItalicFont=cmunit.ttf,BoldItalicFont=cmuntx.ttf]{cmuntt.ttf}\ttfamily \textbackslash{}underline\{a\}}{$\text{ }$}\setmainfont[Path=/usr/share/fonts/truetype/cmu/,UprightFont=cmunrm.ttf,BoldFont=cmunbx.ttf,ItalicFont=cmunti.ttf,BoldItalicFont=cmunbi.ttf]{cmunrm.ttf}\setmonofont[Path=/usr/share/fonts/truetype/cmu/,UprightFont=cmuntt.ttf,BoldFont=cmuntb.ttf,ItalicFont=cmunit.ttf,BoldItalicFont=cmuntx.ttf]{cmunrm.ttf} &\hspace*{0pt}\ignorespaces{}\hspace*{0pt} {$\underline{a} \,$}  
\end{longtable}

\section{Color}
\label{519}
The package \LaTeXTT{xcolor}, described in \mylref{148}{Colors}, allows us to add color to our equations. For example,
\begin{longtable}{p{1.0\linewidth}}
\begin{Shaded}
\begin{Highlighting}[]

\NormalTok{k = \{\textbackslash{}color\{red\}x\} \textbackslash{}mathbin\{\textbackslash{}color\{blue\}-\} 2}
\end{Highlighting}
\end{Shaded}
\\
{$ k = {\color{red}x} \mathbin{\color{blue}-} 2 $}

\end{longtable}
The only problem is that this disrupts the default \LatexSymbol{} formatting around the \LaTeXTT{-{}} operator. To fix this, we enclose it in a \LaTeXTT{\textbackslash{}mathbin} environment, since \LaTeXTT{-{}} is a binary operator. This process is described \myhref{http://tex.stackexchange.com/questions/21598/how-to-color-math-symbols}{here}.
\section{Plus and minus signs}
\label{520}

LaTeX deals with the + and − signs in two possible ways. The most common is as a binary operator. When two maths elements appear on either side of the sign, it is assumed to be a binary operator, and as such, allocates some space either side of the sign. The alternative way is a sign designation. This is when you state whether a mathematical quantity is either positive or negative. This is common for the latter, as in maths, such elements are assumed to be positive unless a − is prefixed to it. In this instance, you want the sign to appear close to the appropriate element to show their association. If you put a + or a − with nothing before it but you want it to be handled like a binary operator you can add an {\itshape \setmainfont[Path=/usr/share/fonts/truetype/cmu/,UprightFont=cmunrm.ttf,BoldFont=cmunbx.ttf,ItalicFont=cmunti.ttf,BoldItalicFont=cmunbi.ttf]{cmunti.ttf}\setmonofont[Path=/usr/share/fonts/truetype/cmu/,UprightFont=cmuntt.ttf,BoldFont=cmuntb.ttf,ItalicFont=cmunit.ttf,BoldItalicFont=cmuntx.ttf]{cmunti.ttf}\itshape invisible}{$\text{ }$}\setmainfont[Path=/usr/share/fonts/truetype/cmu/,UprightFont=cmunrm.ttf,BoldFont=cmunbx.ttf,ItalicFont=cmunti.ttf,BoldItalicFont=cmunbi.ttf]{cmunrm.ttf}\setmonofont[Path=/usr/share/fonts/truetype/cmu/,UprightFont=cmuntt.ttf,BoldFont=cmuntb.ttf,ItalicFont=cmunit.ttf,BoldItalicFont=cmuntx.ttf]{cmunrm.ttf} character before the operator using \LaTeXTT{\{\}}. This can be useful if you are writing multiple-{}line formulas, and a new line could start with a = or a +, for example, then you can fix some strange alignments adding the invisible character where necessary.

A plus-{}minus sign is written as:
\begin{longtable}{p{1.0\linewidth}}
\begin{Shaded}
\begin{Highlighting}[]


 \NormalTok{\textbackslash{}pm}
 
\end{Highlighting}
\end{Shaded}
\\
{$ \pm $}

\end{longtable}

Similarly, there exists also a minus-{}plus sign:
\begin{longtable}{p{1.0\linewidth}}
\begin{Shaded}
\begin{Highlighting}[]


 \NormalTok{\textbackslash{}mp}
 
\end{Highlighting}
\end{Shaded}
\\
{$ \mp $}

\end{longtable}
\section{Controlling horizontal spacing}
\label{521}

LaTeX is obviously pretty good at typesetting maths—it was one of the chief aims of the core TeX system that LaTeX extends. However, it can\textquotesingle{}t always be relied upon to accurately interpret formulas in the way you did. It has to make certain assumptions when there are ambiguous expressions. The result tends to be slightly incorrect horizontal spacing. In these events, the output is still satisfactory, yet any perfectionists will no doubt wish to {\itshape \setmainfont[Path=/usr/share/fonts/truetype/cmu/,UprightFont=cmunrm.ttf,BoldFont=cmunbx.ttf,ItalicFont=cmunti.ttf,BoldItalicFont=cmunbi.ttf]{cmunti.ttf}\setmonofont[Path=/usr/share/fonts/truetype/cmu/,UprightFont=cmuntt.ttf,BoldFont=cmuntb.ttf,ItalicFont=cmunit.ttf,BoldItalicFont=cmuntx.ttf]{cmunti.ttf}\itshape fine-{}tune}{$\text{ }$}\setmainfont[Path=/usr/share/fonts/truetype/cmu/,UprightFont=cmunrm.ttf,BoldFont=cmunbx.ttf,ItalicFont=cmunti.ttf,BoldItalicFont=cmunbi.ttf]{cmunrm.ttf}\setmonofont[Path=/usr/share/fonts/truetype/cmu/,UprightFont=cmuntt.ttf,BoldFont=cmuntb.ttf,ItalicFont=cmunit.ttf,BoldItalicFont=cmuntx.ttf]{cmunrm.ttf} their formulas to ensure spacing is correct. These are generally very subtle adjustments.

There are other occasions where LaTeX has done its job correctly, but you just want to add some space, maybe to add a comment of some kind. For example, in the following equation, it is preferable to ensure there is a decent amount of space between the maths and the text.

\begin{longtable}{p{1.0\linewidth}}
\begin{Shaded}
\begin{Highlighting}[]

 
\NormalTok{\textbackslash{}[ f(n) =}
  \NormalTok{\textbackslash{}begin\{cases\}}
    \NormalTok{n/2       & \textbackslash{}quad \textbackslash{}text\{if \} n \textbackslash{}text\{ is even\}\textbackslash{}\textbackslash{}}
    \NormalTok{-(n+1)/2  & \textbackslash{}quad \textbackslash{}text\{if \} n \textbackslash{}text\{ is odd\}\textbackslash{}\textbackslash{}}
  \NormalTok{\textbackslash{}end\{cases\}}
\NormalTok{\textbackslash{}]}
 
 
\end{Highlighting}
\end{Shaded}
\\

{$ f(n) =   \begin{cases}     n/2       & \quad \text{if } n \text{ is even}\\     -(n+1)/2  & \quad \text{if } n \text{ is odd}\\   \end{cases} $}

\end{longtable}

This code produces errors with Miktex 2.9 and does not yield the results seen on the right.
Use \textbackslash{}mathrm instead of just \textbackslash{}text.


(Note that this particular example can be expressed in more elegant code by the \LaTeXTT{cases} construct provided by the \LaTeXTT{amsmath} package described in \mylref{538}{Advanced Mathematics} chapter.)

LaTeX has defined two commands that can be used anywhere in documents (not just maths) to insert some horizontal space. They are \LaTeXTT{\textbackslash{}quad} and \LaTeXTT{\textbackslash{}qquad}

A \LaTeXTT{\textbackslash{}quad} is a space equal to the current font size. So, if you are using an 11pt font, then the space provided by \LaTeXTT{\textbackslash{}quad} will also be 11pt (horizontally, of course.) The \LaTeXTT{\textbackslash{}qquad} gives twice that amount. As you can see from the code from the above example, \LaTeXTT{\textbackslash{}quad}s were used to add some separation between the maths and the text.

OK, so back to the fine tuning as mentioned at the beginning of the document. A good example would be displaying the simple equation for the indefinite integral of {\itshape \setmainfont[Path=/usr/share/fonts/truetype/cmu/,UprightFont=cmunrm.ttf,BoldFont=cmunbx.ttf,ItalicFont=cmunti.ttf,BoldItalicFont=cmunbi.ttf]{cmunti.ttf}\setmonofont[Path=/usr/share/fonts/truetype/cmu/,UprightFont=cmuntt.ttf,BoldFont=cmuntb.ttf,ItalicFont=cmunit.ttf,BoldItalicFont=cmuntx.ttf]{cmunti.ttf}\itshape y}{$\text{ }$}\setmainfont[Path=/usr/share/fonts/truetype/cmu/,UprightFont=cmunrm.ttf,BoldFont=cmunbx.ttf,ItalicFont=cmunti.ttf,BoldItalicFont=cmunbi.ttf]{cmunrm.ttf}\setmonofont[Path=/usr/share/fonts/truetype/cmu/,UprightFont=cmuntt.ttf,BoldFont=cmuntb.ttf,ItalicFont=cmunit.ttf,BoldItalicFont=cmuntx.ttf]{cmunrm.ttf} with respect to {\itshape \setmainfont[Path=/usr/share/fonts/truetype/cmu/,UprightFont=cmunrm.ttf,BoldFont=cmunbx.ttf,ItalicFont=cmunti.ttf,BoldItalicFont=cmunbi.ttf]{cmunti.ttf}\setmonofont[Path=/usr/share/fonts/truetype/cmu/,UprightFont=cmuntt.ttf,BoldFont=cmuntb.ttf,ItalicFont=cmunit.ttf,BoldItalicFont=cmuntx.ttf]{cmunti.ttf}\itshape x}\setmainfont[Path=/usr/share/fonts/truetype/cmu/,UprightFont=cmunrm.ttf,BoldFont=cmunbx.ttf,ItalicFont=cmunti.ttf,BoldItalicFont=cmunbi.ttf]{cmunrm.ttf}\setmonofont[Path=/usr/share/fonts/truetype/cmu/,UprightFont=cmuntt.ttf,BoldFont=cmuntb.ttf,ItalicFont=cmunit.ttf,BoldItalicFont=cmuntx.ttf]{cmunrm.ttf}:

{$\int y\, \mathrm{d}x$}

If you were to try this, you may write:

\begin{longtable}{p{1.0\linewidth}}
\begin{Shaded}
\begin{Highlighting}[]

 \NormalTok{\textbackslash{}int y \textbackslash{}mathrm\{d\}x }
\end{Highlighting}
\end{Shaded}
\\

{$\int y \mathrm{d}x$}

\end{longtable}

However, this doesn\textquotesingle{}t give the correct result. LaTeX doesn\textquotesingle{}t respect the white-{}space left in the code to signify that the {\itshape \setmainfont[Path=/usr/share/fonts/truetype/cmu/,UprightFont=cmunrm.ttf,BoldFont=cmunbx.ttf,ItalicFont=cmunti.ttf,BoldItalicFont=cmunbi.ttf]{cmunti.ttf}\setmonofont[Path=/usr/share/fonts/truetype/cmu/,UprightFont=cmuntt.ttf,BoldFont=cmuntb.ttf,ItalicFont=cmunit.ttf,BoldItalicFont=cmuntx.ttf]{cmunti.ttf}\itshape y}{$\text{ }$}\setmainfont[Path=/usr/share/fonts/truetype/cmu/,UprightFont=cmunrm.ttf,BoldFont=cmunbx.ttf,ItalicFont=cmunti.ttf,BoldItalicFont=cmunbi.ttf]{cmunrm.ttf}\setmonofont[Path=/usr/share/fonts/truetype/cmu/,UprightFont=cmuntt.ttf,BoldFont=cmuntb.ttf,ItalicFont=cmunit.ttf,BoldItalicFont=cmuntx.ttf]{cmunrm.ttf} and the d{\itshape \setmainfont[Path=/usr/share/fonts/truetype/cmu/,UprightFont=cmunrm.ttf,BoldFont=cmunbx.ttf,ItalicFont=cmunti.ttf,BoldItalicFont=cmunbi.ttf]{cmunti.ttf}\setmonofont[Path=/usr/share/fonts/truetype/cmu/,UprightFont=cmuntt.ttf,BoldFont=cmuntb.ttf,ItalicFont=cmunit.ttf,BoldItalicFont=cmuntx.ttf]{cmunti.ttf}\itshape x}{$\text{ }$}\setmainfont[Path=/usr/share/fonts/truetype/cmu/,UprightFont=cmunrm.ttf,BoldFont=cmunbx.ttf,ItalicFont=cmunti.ttf,BoldItalicFont=cmunbi.ttf]{cmunrm.ttf}\setmonofont[Path=/usr/share/fonts/truetype/cmu/,UprightFont=cmuntt.ttf,BoldFont=cmuntb.ttf,ItalicFont=cmunit.ttf,BoldItalicFont=cmuntx.ttf]{cmunrm.ttf} are independent entities. Instead, it lumps them altogether. A \LaTeXTT{\textbackslash{}quad} would clearly be overkill in this situation—what is needed are some small spaces to be utilized in this type of instance, and that\textquotesingle{}s what LaTeX provides:

\begin{longtable}{|>{\RaggedRight}p{0.26107\linewidth}|>{\RaggedRight}p{0.31674\linewidth}|>{\RaggedRight}p{0.33648\linewidth}|} \hline 
{\bfseries \hspace*{0pt}\ignorespaces{}\hspace*{0pt} Command}&{\bfseries \hspace*{0pt}\ignorespaces{}\hspace*{0pt} Description}&{\bfseries \hspace*{0pt}\ignorespaces{}\hspace*{0pt} Size}\endhead  \hline \hspace*{0pt}\ignorespaces{}\hspace*{0pt} \LaTeXTT{\textbackslash{},}&\hspace*{0pt}\ignorespaces{}\hspace*{0pt} small space&\hspace*{0pt}\ignorespaces{}\hspace*{0pt} 3/18 of a quad\\ \hline \hspace*{0pt}\ignorespaces{}\hspace*{0pt} \LaTeXTT{\textbackslash{}:}&\hspace*{0pt}\ignorespaces{}\hspace*{0pt} medium space&\hspace*{0pt}\ignorespaces{}\hspace*{0pt} 4/18 of a quad\\ \hline \hspace*{0pt}\ignorespaces{}\hspace*{0pt} \LaTeXTT{\textbackslash{};}&\hspace*{0pt}\ignorespaces{}\hspace*{0pt} large space&\hspace*{0pt}\ignorespaces{}\hspace*{0pt} 5/18 of a quad\\ \hline \hspace*{0pt}\ignorespaces{}\hspace*{0pt} \LaTeXTT{\textbackslash{}!}&\hspace*{0pt}\ignorespaces{}\hspace*{0pt} negative space&\hspace*{0pt}\ignorespaces{}\hspace*{0pt} -{}3/18 of a quad\\ \hline 
\end{longtable}


NB you can use more than one command in a sequence to achieve a greater space if necessary.

So, to rectify the current problem:

\begin{longtable}{p{1.0\linewidth}}
\begin{Shaded}
\begin{Highlighting}[]

 \NormalTok{\textbackslash{}int y\textbackslash{}, \textbackslash{}mathrm\{d\}x }
\end{Highlighting}
\end{Shaded}
\\

{$\int y\, \mathrm{d}x$}

\end{longtable}
\begin{longtable}{p{1.0\linewidth}}
\begin{Shaded}
\begin{Highlighting}[]

 \NormalTok{\textbackslash{}int y\textbackslash{}: \textbackslash{}mathrm\{d\}x }
\end{Highlighting}
\end{Shaded}
\\

{$\int y \mathrm{d}x$}

\end{longtable}
\begin{longtable}{p{1.0\linewidth}}
\begin{Shaded}
\begin{Highlighting}[]

 \NormalTok{\textbackslash{}int y\textbackslash{}; \textbackslash{}mathrm\{d\}x }
\end{Highlighting}
\end{Shaded}
\\

{$\int y \mathrm{d}x$}

\end{longtable}

The negative space may seem like an odd thing to use, however, it wouldn\textquotesingle{}t be there if it didn\textquotesingle{}t have {\itshape \setmainfont[Path=/usr/share/fonts/truetype/cmu/,UprightFont=cmunrm.ttf,BoldFont=cmunbx.ttf,ItalicFont=cmunti.ttf,BoldItalicFont=cmunbi.ttf]{cmunti.ttf}\setmonofont[Path=/usr/share/fonts/truetype/cmu/,UprightFont=cmuntt.ttf,BoldFont=cmuntb.ttf,ItalicFont=cmunit.ttf,BoldItalicFont=cmuntx.ttf]{cmunti.ttf}\itshape some}{$\text{ }$}\setmainfont[Path=/usr/share/fonts/truetype/cmu/,UprightFont=cmunrm.ttf,BoldFont=cmunbx.ttf,ItalicFont=cmunti.ttf,BoldItalicFont=cmunbi.ttf]{cmunrm.ttf}\setmonofont[Path=/usr/share/fonts/truetype/cmu/,UprightFont=cmuntt.ttf,BoldFont=cmuntb.ttf,ItalicFont=cmunit.ttf,BoldItalicFont=cmuntx.ttf]{cmunrm.ttf} use! Take the following example:

\begin{longtable}{p{1.0\linewidth}}
\begin{Shaded}
\begin{Highlighting}[]


  \NormalTok{\textbackslash{}left(}
    \NormalTok{\textbackslash{}begin\{array\}\{c\}}
      \NormalTok{n \textbackslash{}\textbackslash{}}
      \NormalTok{r}
    \NormalTok{\textbackslash{}end\{array\}}
  \NormalTok{\textbackslash{}right) = \textbackslash{}frac\{n!\}\{r!(n-r)!\}}
 
\end{Highlighting}
\end{Shaded}
\\

{$\left(    \begin{matrix}      n \\      r    \end{matrix}    \right) = \frac{n!}{r!(n-r)!}$}

\end{longtable}


The matrix-{}like expression for representing binomial coefficients is too padded. There is too much space between the brackets and the actual contents within. This can easily be corrected by adding a few negative spaces after the left bracket and before the right bracket.

\begin{longtable}{p{1.0\linewidth}}
\begin{Shaded}
\begin{Highlighting}[]


  \NormalTok{\textbackslash{}left(\textbackslash{}!}
    \NormalTok{\textbackslash{}begin\{array\}\{c\}}
      \NormalTok{n \textbackslash{}\textbackslash{}}
      \NormalTok{r}
    \NormalTok{\textbackslash{}end\{array\}}
  \NormalTok{\textbackslash{}!\textbackslash{}right) = \textbackslash{}frac\{n!\}\{r!(n-r)!\}}
 
\end{Highlighting}
\end{Shaded}
\\

{$\left(    \begin{matrix}      n \\      r    \end{matrix}    \right) = \frac{n!}{r!(n-r)!} $}

\end{longtable}

In any case, adding some spaces manually should be avoided whenever possible: it makes the source code more complex and it\textquotesingle{}s against the basic principles of a What You See is What You Mean approach. The best thing to do is to define some commands using all the spaces you want and then, when you use your command, you don\textquotesingle{}t have to add any other space. Later, if you change your mind about the length of the horizontal space, you can easily change it modifying only the command you defined before. Let us use an example: you want the {\itshape \setmainfont[Path=/usr/share/fonts/truetype/cmu/,UprightFont=cmunrm.ttf,BoldFont=cmunbx.ttf,ItalicFont=cmunti.ttf,BoldItalicFont=cmunbi.ttf]{cmunti.ttf}\setmonofont[Path=/usr/share/fonts/truetype/cmu/,UprightFont=cmuntt.ttf,BoldFont=cmuntb.ttf,ItalicFont=cmunit.ttf,BoldItalicFont=cmuntx.ttf]{cmunti.ttf}\itshape d}{$\text{ }$}\setmainfont[Path=/usr/share/fonts/truetype/cmu/,UprightFont=cmunrm.ttf,BoldFont=cmunbx.ttf,ItalicFont=cmunti.ttf,BoldItalicFont=cmunbi.ttf]{cmunrm.ttf}\setmonofont[Path=/usr/share/fonts/truetype/cmu/,UprightFont=cmuntt.ttf,BoldFont=cmuntb.ttf,ItalicFont=cmunit.ttf,BoldItalicFont=cmuntx.ttf]{cmunrm.ttf} of a {\itshape \setmainfont[Path=/usr/share/fonts/truetype/cmu/,UprightFont=cmunrm.ttf,BoldFont=cmunbx.ttf,ItalicFont=cmunti.ttf,BoldItalicFont=cmunbi.ttf]{cmunti.ttf}\setmonofont[Path=/usr/share/fonts/truetype/cmu/,UprightFont=cmuntt.ttf,BoldFont=cmuntb.ttf,ItalicFont=cmunit.ttf,BoldItalicFont=cmuntx.ttf]{cmunti.ttf}\itshape dx}{$\text{ }$}\setmainfont[Path=/usr/share/fonts/truetype/cmu/,UprightFont=cmunrm.ttf,BoldFont=cmunbx.ttf,ItalicFont=cmunti.ttf,BoldItalicFont=cmunbi.ttf]{cmunrm.ttf}\setmonofont[Path=/usr/share/fonts/truetype/cmu/,UprightFont=cmuntt.ttf,BoldFont=cmuntb.ttf,ItalicFont=cmunit.ttf,BoldItalicFont=cmuntx.ttf]{cmunrm.ttf} in an integral to be in roman font and a small space away from the rest. If you want to type an integral like \LaTeXTT{\textbackslash{}int x \textbackslash{}, \textbackslash{}mathrm\{d\} x}, you can define a command like this:
\begin{Shaded}
\begin{Highlighting}[]

\NormalTok{\textbackslash{}newcommand\{\textbackslash{}dd\}\{\textbackslash{}mathop\{\}\textbackslash{},\textbackslash{}mathrm\{d\}\}}
\end{Highlighting}
\end{Shaded}

in the preamble of your document. We have chosen \LaTeXTT{\textbackslash{}dd} just because it reminds the \symbol{34}d\symbol{34} it replaces and it is fast to type. Doing so, the code for your integral becomes \LaTeXTT{\textbackslash{}int x \textbackslash{}dd x}. Now, whenever you write an integral, you just have to use the \LaTeXTT{\textbackslash{}dd} instead of the \symbol{34}d\symbol{34}, and all your integrals will have the same style. If you change your mind, you just have to change the definition in the preamble, and all your integrals will be changed accordingly.
\section{Manually Specifying Formula Style}
\label{522}
To manually display a fragment of a formula using text style, surround the fragment with curly braces and prefix the fragment with \LaTeXTT{\textbackslash{}textstyle}. The braces are required because the \LaTeXTT{\textbackslash{}textstyle} macro changes the state of the renderer, rendering all subsequent mathematics in text style. The braces limit this change of state to just the fragment enclosed within. For example, to use text style for just the summation symbol in a sum, one would enter
\begin{Shaded}
\begin{Highlighting}[]

\NormalTok{\textbackslash{}begin\{equation\}}
   \NormalTok{C^i_j = \{\textbackslash{}textstyle \textbackslash{}sum_k\} A^i_k B^k_j}
\NormalTok{\textbackslash{}end\{equation\}}
\end{Highlighting}
\end{Shaded}

The same thing as a command would look like this:
\begin{Shaded}
\begin{Highlighting}[]

\NormalTok{\textbackslash{}newcommand\{\textbackslash{}tsum\}[1]\}}
\end{Highlighting}
\end{Shaded}

Note the extra braces. Just one set around the expression won\textquotesingle{}t be enough. That would cause all math after \LaTeXTT{\textbackslash{}tsum k} to be displayed using text style.

To display part of a formula using display style, do the same thing, but use \LaTeXTT{\textbackslash{}displaystyle} instead.
\section{Advanced Mathematics: AMS Math package}
\label{523}

The AMS (\myhref{https://en.wikipedia.org/wiki/American\%20Mathematical\%20Society}{American Mathematical Society}) mathematics package is a powerful package that creates a higher layer of abstraction over mathematical LaTeX language; if you use it it will make your life easier. Some commands \LaTeXTT{amsmath} introduces will make other plain LaTeX commands obsolete: in order to keep consistency in the final output you\textquotesingle{}d better use \LaTeXTT{amsmath} commands whenever possible. If you do so, you will get an elegant output without worrying about alignment and other details, keeping your source code readable. If you want to use it, you have to add this in the preamble:
\begin{Shaded}
\begin{Highlighting}[]

\NormalTok{\textbackslash{}usepackage\{amsmath\}}
\end{Highlighting}
\end{Shaded}

\subsection{Introducing dots in formulas}
\label{524}
\LaTeXTT{amsmath} defines also the \LaTeXTT{\textbackslash{}dots} command, that is a generalization of the existing \LaTeXTT{\textbackslash{}ldots}. You can use \LaTeXTT{\textbackslash{}dots} in both text and math mode and LaTeX will replace it with three dots \symbol{34}…\symbol{34} but it will decide according to the context whether to put it on the bottom (like \LaTeXTT{\textbackslash{}ldots}) or centered (like \LaTeXTT{\textbackslash{}cdots}).
\subsection{Dots}
\label{525}

LaTeX gives you several commands to insert dots (ellipses) in your formulae. This can be particularly useful if you have to type big matrices omitting elements. First of all, here are the main dots-{}related commands LaTeX provides:

\begin{longtable}{|>{\RaggedRight}p{0.20018\linewidth}|>{\RaggedRight}p{0.10187\linewidth}|>{\RaggedRight}p{0.61223\linewidth}|} \hline 
{\bfseries \hspace*{0pt}\ignorespaces{}\hspace*{0pt} Code }&{\bfseries \hspace*{0pt}\ignorespaces{}\hspace*{0pt} Output }&{\bfseries \hspace*{0pt}\ignorespaces{}\hspace*{0pt} Comment}\endhead  \hline \hspace*{0pt}\ignorespaces{}\hspace*{0pt} \LaTeXTT{\textbackslash{}dots} &\hspace*{0pt}\ignorespaces{}\hspace*{0pt} {$\dots$} &\hspace*{0pt}\ignorespaces{}\hspace*{0pt} generic dots (ellipsis), to be used in text (outside formulae as well). It automatically manages whitespaces before and after itself according to the context, it\textquotesingle{}s a higher level command.\\ \hline \hspace*{0pt}\ignorespaces{}\hspace*{0pt} \LaTeXTT{\textbackslash{}ldots}&\hspace*{0pt}\ignorespaces{}\hspace*{0pt} {$\ldots$} &\hspace*{0pt}\ignorespaces{}\hspace*{0pt} the output is similar to the previous one, but there is no automatic whitespace management; it works at a lower level.\\ \hline \hspace*{0pt}\ignorespaces{}\hspace*{0pt} \LaTeXTT{\textbackslash{}cdots} &\hspace*{0pt}\ignorespaces{}\hspace*{0pt} {$\cdots$} &\hspace*{0pt}\ignorespaces{}\hspace*{0pt} These dots are centered relative to the height of a letter. There is also the binary multiplication operator, {\ttfamily \setmainfont[Path=/usr/share/fonts/truetype/cmu/,UprightFont=cmunrm.ttf,BoldFont=cmunbx.ttf,ItalicFont=cmunti.ttf,BoldItalicFont=cmunbi.ttf]{cmuntt.ttf}\setmonofont[Path=/usr/share/fonts/truetype/cmu/,UprightFont=cmuntt.ttf,BoldFont=cmuntb.ttf,ItalicFont=cmunit.ttf,BoldItalicFont=cmuntx.ttf]{cmuntt.ttf}\ttfamily \textbackslash{}cdot}\setmainfont[Path=/usr/share/fonts/truetype/cmu/,UprightFont=cmunrm.ttf,BoldFont=cmunbx.ttf,ItalicFont=cmunti.ttf,BoldItalicFont=cmunbi.ttf]{cmunrm.ttf}\setmonofont[Path=/usr/share/fonts/truetype/cmu/,UprightFont=cmuntt.ttf,BoldFont=cmuntb.ttf,ItalicFont=cmunit.ttf,BoldItalicFont=cmuntx.ttf]{cmunrm.ttf}, mentioned below.\\ \hline \hspace*{0pt}\ignorespaces{}\hspace*{0pt} \LaTeXTT{\textbackslash{}vdots} &\hspace*{0pt}\ignorespaces{}\hspace*{0pt} {$\vdots$} &\hspace*{0pt}\ignorespaces{}\hspace*{0pt} vertical dots\\ \hline \hspace*{0pt}\ignorespaces{}\hspace*{0pt} \LaTeXTT{\textbackslash{}ddots} &\hspace*{0pt}\ignorespaces{}\hspace*{0pt} {$\ddots$} &\hspace*{0pt}\ignorespaces{}\hspace*{0pt} diagonal dots\\ \hline \hspace*{0pt}\ignorespaces{}\hspace*{0pt} \LaTeXTT{\textbackslash{}iddots} &\hspace*{0pt}\ignorespaces{}\hspace*{0pt}  &\hspace*{0pt}\ignorespaces{}\hspace*{0pt} inverse diagonal dots (requires the \LaTeXTT{mathdots} package)\\ \hline \hspace*{0pt}\ignorespaces{}\hspace*{0pt} \LaTeXTT{\textbackslash{}hdotsfor\{n\}} &\hspace*{0pt}\ignorespaces{}\hspace*{0pt} {$\ldots \ldots$}&\hspace*{0pt}\ignorespaces{}\hspace*{0pt} to be used in matrices, it creates a row of dots spanning {\itshape \setmainfont[Path=/usr/share/fonts/truetype/cmu/,UprightFont=cmunrm.ttf,BoldFont=cmunbx.ttf,ItalicFont=cmunti.ttf,BoldItalicFont=cmunbi.ttf]{cmunti.ttf}\setmonofont[Path=/usr/share/fonts/truetype/cmu/,UprightFont=cmuntt.ttf,BoldFont=cmuntb.ttf,ItalicFont=cmunit.ttf,BoldItalicFont=cmuntx.ttf]{cmunti.ttf}\itshape n}{$\text{ }$}\setmainfont[Path=/usr/share/fonts/truetype/cmu/,UprightFont=cmunrm.ttf,BoldFont=cmunbx.ttf,ItalicFont=cmunti.ttf,BoldItalicFont=cmunbi.ttf]{cmunrm.ttf}\setmonofont[Path=/usr/share/fonts/truetype/cmu/,UprightFont=cmuntt.ttf,BoldFont=cmuntb.ttf,ItalicFont=cmunit.ttf,BoldItalicFont=cmuntx.ttf]{cmunrm.ttf} columns. \\ \hline 
\end{longtable}


Instead of using \LaTeXTT{\textbackslash{}ldots} and \LaTeXTT{\textbackslash{}cdots}, you should use the semantically oriented commands. It makes it possible to adapt your document to different conventions on the fly, in case (for example) you have to submit it to a publisher who insists on following house tradition in this respect. The default treatment for the various kinds follows American Mathematical Society conventions.

\begin{longtable}{|>{\RaggedRight}p{0.24346\linewidth}|>{\RaggedRight}p{0.33542\linewidth}|>{\RaggedRight}p{0.33542\linewidth}|} \hline 
{\bfseries \hspace*{0pt}\ignorespaces{}\hspace*{0pt} Code }&{\bfseries \hspace*{0pt}\ignorespaces{}\hspace*{0pt} Output }&{\bfseries \hspace*{0pt}\ignorespaces{}\hspace*{0pt} Comment}\endhead  \hline \hspace*{0pt}\ignorespaces{}\hspace*{0pt} \LaTeXTT{A\_1,A\_2,\textbackslash{}dotsc,} &\hspace*{0pt}\ignorespaces{}\hspace*{0pt} \begin{minipage}{1.0\linewidth}\begin{center}\includegraphics[width=1.0\linewidth,height=6.5in,keepaspectratio]{../images/88.png}\end{center}\myfigurewithoutcaption{88}\end{minipage} &\hspace*{0pt}\ignorespaces{}\hspace*{0pt} for \symbol{34}dots with commas\symbol{34}\\ \hline \hspace*{0pt}\ignorespaces{}\hspace*{0pt} \LaTeXTT{A\_1+\textbackslash{}dotsb+A\_N} &\hspace*{0pt}\ignorespaces{}\hspace*{0pt} \begin{minipage}{1.0\linewidth}\begin{center}\includegraphics[width=1.0\linewidth,height=6.5in,keepaspectratio]{../images/89.png}\end{center}\myfigurewithoutcaption{89}\end{minipage} &\hspace*{0pt}\ignorespaces{}\hspace*{0pt} for \symbol{34}dots with binary operators/relations\symbol{34}\\ \hline \hspace*{0pt}\ignorespaces{}\hspace*{0pt} \LaTeXTT{A\_1 \textbackslash{}dotsm A\_N} &\hspace*{0pt}\ignorespaces{}\hspace*{0pt} \begin{minipage}{1.0\linewidth}\begin{center}\includegraphics[width=1.0\linewidth,height=6.5in,keepaspectratio]{../images/90.png}\end{center}\myfigurewithoutcaption{90}\end{minipage} &\hspace*{0pt}\ignorespaces{}\hspace*{0pt} for \symbol{34}multiplication dots\symbol{34}\\ \hline \hspace*{0pt}\ignorespaces{}\hspace*{0pt} \LaTeXTT{\textbackslash{}int\_a\^{}b \textbackslash{}dotsi} &\hspace*{0pt}\ignorespaces{}\hspace*{0pt} \begin{minipage}{1.0\linewidth}\begin{center}\includegraphics[width=1.0\linewidth,height=6.5in,keepaspectratio]{../images/91.png}\end{center}\myfigurewithoutcaption{91}\end{minipage} &\hspace*{0pt}\ignorespaces{}\hspace*{0pt} for \symbol{34}dots with integrals\symbol{34}\\ \hline \hspace*{0pt}\ignorespaces{}\hspace*{0pt} \LaTeXTT{A\_1\textbackslash{}dotso A\_N} &\hspace*{0pt}\ignorespaces{}\hspace*{0pt} \begin{minipage}{1.0\linewidth}\begin{center}\includegraphics[width=1.0\linewidth,height=6.5in,keepaspectratio]{../images/92.png}\end{center}\myfigurewithoutcaption{92}\end{minipage} &\hspace*{0pt}\ignorespaces{}\hspace*{0pt} for \symbol{34}other dots\symbol{34} (none of the above)\\ \hline 
\end{longtable}

\subsection{Write an equation with the align environment}
\label{526}
How to write an equation with the align environment with the \LaTeXTT{amsmath} package is described in \mylref{538}{Advanced Mathematics}.
\section{List of Mathematical Symbols}
\label{527}
All the pre-{}defined mathematical symbols from the \textbackslash{}TeX\textbackslash{} package are listed below. More symbols are available from extra packages.
{\scriptsize{}
{\scalefont{0.58691}\begin{longtable}{|>{\RaggedRight}p{0.05588\linewidth}|>{\RaggedRight}p{0.05634\linewidth}|>{\RaggedRight}p{0.01693\linewidth}|>{\RaggedRight}p{0.05588\linewidth}|>{\RaggedRight}p{0.10037\linewidth}|>{\RaggedRight}p{0.01693\linewidth}|>{\RaggedRight}p{0.05588\linewidth}|>{\RaggedRight}p{0.06920\linewidth}|>{\RaggedRight}p{0.01693\linewidth}|>{\RaggedRight}p{0.05588\linewidth}|>{\RaggedRight}p{0.05634\linewidth}|>{\RaggedRight}p{0.01693\linewidth}|>{\RaggedRight}p{0.05588\linewidth}|>{\RaggedRight}p{0.05634\linewidth}|} \hline 
\multicolumn{14}{|>{\RaggedRight}p{0.97143\linewidth}|}{{\bfseries \hspace*{0pt}\ignorespaces{}\hspace*{0pt} Relation Symbols}}\\ \hline {\bfseries \hspace*{0pt}\ignorespaces{}\hspace*{0pt} Symbol }&{\bfseries \hspace*{0pt}\ignorespaces{}\hspace*{0pt} Script}&\multirow{10}{\linewidth}{\hspace*{0pt}\ignorespaces{}\hspace*{0pt} {\mbox{$~$}}}&{\bfseries \hspace*{0pt}\ignorespaces{}\hspace*{0pt} Symbol }&{\bfseries \hspace*{0pt}\ignorespaces{}\hspace*{0pt} Script}&\multirow{10}{\linewidth}{\hspace*{0pt}\ignorespaces{}\hspace*{0pt} {\mbox{$~$}}}&{\bfseries \hspace*{0pt}\ignorespaces{}\hspace*{0pt} Symbol }&{\bfseries \hspace*{0pt}\ignorespaces{}\hspace*{0pt} Script}&\multirow{10}{\linewidth}{\hspace*{0pt}\ignorespaces{}\hspace*{0pt} {\mbox{$~$}}}&{\bfseries \hspace*{0pt}\ignorespaces{}\hspace*{0pt} Symbol }&{\bfseries \hspace*{0pt}\ignorespaces{}\hspace*{0pt} Script}&\multirow{10}{\linewidth}{\hspace*{0pt}\ignorespaces{}\hspace*{0pt} {\mbox{$~$}}}&{\bfseries \hspace*{0pt}\ignorespaces{}\hspace*{0pt} Symbol }&{\bfseries \hspace*{0pt}\ignorespaces{}\hspace*{0pt} Script}\\ \cline{1-1}\cline{2-2}\cline{4-4}\cline{5-5}\cline{7-7}\cline{8-8}\cline{10-10}\cline{11-11}\cline{13-13}\cline{14-14} \hspace*{0pt}\ignorespaces{}\hspace*{0pt} {$<\,$}&\hspace*{0pt}\ignorespaces{}\hspace*{0pt}{\ttfamily \setmainfont[Path=/usr/share/fonts/truetype/cmu/,UprightFont=cmunrm.ttf,BoldFont=cmunbx.ttf,ItalicFont=cmunti.ttf,BoldItalicFont=cmunbi.ttf]{cmuntt.ttf}\setmonofont[Path=/usr/share/fonts/truetype/cmu/,UprightFont=cmuntt.ttf,BoldFont=cmuntb.ttf,ItalicFont=cmunit.ttf,BoldItalicFont=cmuntx.ttf]{cmuntt.ttf}\ttfamily <{}}&\multicolumn{1}{|c|}{}&\hspace*{0pt}\ignorespaces{}\hspace*{0pt}{$\text{ }$}\setmainfont[Path=/usr/share/fonts/truetype/cmu/,UprightFont=cmunrm.ttf,BoldFont=cmunbx.ttf,ItalicFont=cmunti.ttf,BoldItalicFont=cmunbi.ttf]{cmunrm.ttf}\setmonofont[Path=/usr/share/fonts/truetype/cmu/,UprightFont=cmuntt.ttf,BoldFont=cmuntb.ttf,ItalicFont=cmunit.ttf,BoldItalicFont=cmuntx.ttf]{cmunrm.ttf} {$>\,$}&\hspace*{0pt}\ignorespaces{}\hspace*{0pt}{\ttfamily \setmainfont[Path=/usr/share/fonts/truetype/cmu/,UprightFont=cmunrm.ttf,BoldFont=cmunbx.ttf,ItalicFont=cmunti.ttf,BoldItalicFont=cmunbi.ttf]{cmuntt.ttf}\setmonofont[Path=/usr/share/fonts/truetype/cmu/,UprightFont=cmuntt.ttf,BoldFont=cmuntb.ttf,ItalicFont=cmunit.ttf,BoldItalicFont=cmuntx.ttf]{cmuntt.ttf}\ttfamily >{}}&\multicolumn{1}{|c|}{}&\hspace*{0pt}\ignorespaces{}\hspace*{0pt}{$\text{ }$}\setmainfont[Path=/usr/share/fonts/truetype/cmu/,UprightFont=cmunrm.ttf,BoldFont=cmunbx.ttf,ItalicFont=cmunti.ttf,BoldItalicFont=cmunbi.ttf]{cmunrm.ttf}\setmonofont[Path=/usr/share/fonts/truetype/cmu/,UprightFont=cmuntt.ttf,BoldFont=cmuntb.ttf,ItalicFont=cmunit.ttf,BoldItalicFont=cmuntx.ttf]{cmunrm.ttf} {$=\,$}&\hspace*{0pt}\ignorespaces{}\hspace*{0pt}{\ttfamily \setmainfont[Path=/usr/share/fonts/truetype/cmu/,UprightFont=cmunrm.ttf,BoldFont=cmunbx.ttf,ItalicFont=cmunti.ttf,BoldItalicFont=cmunbi.ttf]{cmuntt.ttf}\setmonofont[Path=/usr/share/fonts/truetype/cmu/,UprightFont=cmuntt.ttf,BoldFont=cmuntb.ttf,ItalicFont=cmunit.ttf,BoldItalicFont=cmuntx.ttf]{cmuntt.ttf}\ttfamily =}&\multicolumn{1}{|c|}{}&\hspace*{0pt}\ignorespaces{}\hspace*{0pt}{$\text{ }$}\setmainfont[Path=/usr/share/fonts/truetype/cmu/,UprightFont=cmunrm.ttf,BoldFont=cmunbx.ttf,ItalicFont=cmunti.ttf,BoldItalicFont=cmunbi.ttf]{cmunrm.ttf}\setmonofont[Path=/usr/share/fonts/truetype/cmu/,UprightFont=cmuntt.ttf,BoldFont=cmuntb.ttf,ItalicFont=cmunit.ttf,BoldItalicFont=cmuntx.ttf]{cmunrm.ttf} {$\parallel\,$}&\hspace*{0pt}\ignorespaces{}\hspace*{0pt}{\ttfamily \setmainfont[Path=/usr/share/fonts/truetype/cmu/,UprightFont=cmunrm.ttf,BoldFont=cmunbx.ttf,ItalicFont=cmunti.ttf,BoldItalicFont=cmunbi.ttf]{cmuntt.ttf}\setmonofont[Path=/usr/share/fonts/truetype/cmu/,UprightFont=cmuntt.ttf,BoldFont=cmuntb.ttf,ItalicFont=cmunit.ttf,BoldItalicFont=cmuntx.ttf]{cmuntt.ttf}\ttfamily \textbackslash{}parallel}&\multicolumn{1}{|c|}{}&\hspace*{0pt}\ignorespaces{}\hspace*{0pt}{$\text{ }$}\setmainfont[Path=/usr/share/fonts/truetype/cmu/,UprightFont=cmunrm.ttf,BoldFont=cmunbx.ttf,ItalicFont=cmunti.ttf,BoldItalicFont=cmunbi.ttf]{cmunrm.ttf}\setmonofont[Path=/usr/share/fonts/truetype/cmu/,UprightFont=cmuntt.ttf,BoldFont=cmuntb.ttf,ItalicFont=cmunit.ttf,BoldItalicFont=cmuntx.ttf]{cmunrm.ttf} {$\nparallel\,$}&\hspace*{0pt}\ignorespaces{}\hspace*{0pt}{\ttfamily \setmainfont[Path=/usr/share/fonts/truetype/cmu/,UprightFont=cmunrm.ttf,BoldFont=cmunbx.ttf,ItalicFont=cmunti.ttf,BoldItalicFont=cmunbi.ttf]{cmuntt.ttf}\setmonofont[Path=/usr/share/fonts/truetype/cmu/,UprightFont=cmuntt.ttf,BoldFont=cmuntb.ttf,ItalicFont=cmunit.ttf,BoldItalicFont=cmuntx.ttf]{cmuntt.ttf}\ttfamily \textbackslash{}nparallel}\\ \cline{1-1}\cline{2-2}\cline{4-4}\cline{5-5}\cline{7-7}\cline{8-8}\cline{10-10}\cline{11-11}\cline{13-13}\cline{14-14} \hspace*{0pt}\ignorespaces{}\hspace*{0pt}{$\text{ }$}\setmainfont[Path=/usr/share/fonts/truetype/cmu/,UprightFont=cmunrm.ttf,BoldFont=cmunbx.ttf,ItalicFont=cmunti.ttf,BoldItalicFont=cmunbi.ttf]{cmunrm.ttf}\setmonofont[Path=/usr/share/fonts/truetype/cmu/,UprightFont=cmuntt.ttf,BoldFont=cmuntb.ttf,ItalicFont=cmunit.ttf,BoldItalicFont=cmuntx.ttf]{cmunrm.ttf} {$\leq\,$}&\hspace*{0pt}\ignorespaces{}\hspace*{0pt}{\ttfamily \setmainfont[Path=/usr/share/fonts/truetype/cmu/,UprightFont=cmunrm.ttf,BoldFont=cmunbx.ttf,ItalicFont=cmunti.ttf,BoldItalicFont=cmunbi.ttf]{cmuntt.ttf}\setmonofont[Path=/usr/share/fonts/truetype/cmu/,UprightFont=cmuntt.ttf,BoldFont=cmuntb.ttf,ItalicFont=cmunit.ttf,BoldItalicFont=cmuntx.ttf]{cmuntt.ttf}\ttfamily \textbackslash{}leq}&\multicolumn{1}{|c|}{}&\hspace*{0pt}\ignorespaces{}\hspace*{0pt}{$\text{ }$}\setmainfont[Path=/usr/share/fonts/truetype/cmu/,UprightFont=cmunrm.ttf,BoldFont=cmunbx.ttf,ItalicFont=cmunti.ttf,BoldItalicFont=cmunbi.ttf]{cmunrm.ttf}\setmonofont[Path=/usr/share/fonts/truetype/cmu/,UprightFont=cmuntt.ttf,BoldFont=cmuntb.ttf,ItalicFont=cmunit.ttf,BoldItalicFont=cmuntx.ttf]{cmunrm.ttf} {$\geq\,$}&\hspace*{0pt}\ignorespaces{}\hspace*{0pt}{\ttfamily \setmainfont[Path=/usr/share/fonts/truetype/cmu/,UprightFont=cmunrm.ttf,BoldFont=cmunbx.ttf,ItalicFont=cmunti.ttf,BoldItalicFont=cmunbi.ttf]{cmuntt.ttf}\setmonofont[Path=/usr/share/fonts/truetype/cmu/,UprightFont=cmuntt.ttf,BoldFont=cmuntb.ttf,ItalicFont=cmunit.ttf,BoldItalicFont=cmuntx.ttf]{cmuntt.ttf}\ttfamily \textbackslash{}geq}&\multicolumn{1}{|c|}{}&\hspace*{0pt}\ignorespaces{}\hspace*{0pt}{$\text{ }$}\setmainfont[Path=/usr/share/fonts/truetype/cmu/,UprightFont=cmunrm.ttf,BoldFont=cmunbx.ttf,ItalicFont=cmunti.ttf,BoldItalicFont=cmunbi.ttf]{cmunrm.ttf}\setmonofont[Path=/usr/share/fonts/truetype/cmu/,UprightFont=cmuntt.ttf,BoldFont=cmuntb.ttf,ItalicFont=cmunit.ttf,BoldItalicFont=cmuntx.ttf]{cmunrm.ttf} {$\doteq\,$}&\hspace*{0pt}\ignorespaces{}\hspace*{0pt}{\ttfamily \setmainfont[Path=/usr/share/fonts/truetype/cmu/,UprightFont=cmunrm.ttf,BoldFont=cmunbx.ttf,ItalicFont=cmunti.ttf,BoldItalicFont=cmunbi.ttf]{cmuntt.ttf}\setmonofont[Path=/usr/share/fonts/truetype/cmu/,UprightFont=cmuntt.ttf,BoldFont=cmuntb.ttf,ItalicFont=cmunit.ttf,BoldItalicFont=cmuntx.ttf]{cmuntt.ttf}\ttfamily \textbackslash{}doteq}&\multicolumn{1}{|c|}{}&\hspace*{0pt}\ignorespaces{}\hspace*{0pt}{$\text{ }$}\setmainfont[Path=/usr/share/fonts/truetype/cmu/,UprightFont=cmunrm.ttf,BoldFont=cmunbx.ttf,ItalicFont=cmunti.ttf,BoldItalicFont=cmunbi.ttf]{cmunrm.ttf}\setmonofont[Path=/usr/share/fonts/truetype/cmu/,UprightFont=cmuntt.ttf,BoldFont=cmuntb.ttf,ItalicFont=cmunit.ttf,BoldItalicFont=cmuntx.ttf]{cmunrm.ttf} {$\asymp\,$}&\hspace*{0pt}\ignorespaces{}\hspace*{0pt}{\ttfamily \setmainfont[Path=/usr/share/fonts/truetype/cmu/,UprightFont=cmunrm.ttf,BoldFont=cmunbx.ttf,ItalicFont=cmunti.ttf,BoldItalicFont=cmunbi.ttf]{cmuntt.ttf}\setmonofont[Path=/usr/share/fonts/truetype/cmu/,UprightFont=cmuntt.ttf,BoldFont=cmuntb.ttf,ItalicFont=cmunit.ttf,BoldItalicFont=cmuntx.ttf]{cmuntt.ttf}\ttfamily \textbackslash{}asymp}&\multicolumn{1}{|c|}{}&\hspace*{0pt}\ignorespaces{}\hspace*{0pt}{$\text{ }$}\setmainfont[Path=/usr/share/fonts/truetype/cmu/,UprightFont=cmunrm.ttf,BoldFont=cmunbx.ttf,ItalicFont=cmunti.ttf,BoldItalicFont=cmunbi.ttf]{cmunrm.ttf}\setmonofont[Path=/usr/share/fonts/truetype/cmu/,UprightFont=cmuntt.ttf,BoldFont=cmuntb.ttf,ItalicFont=cmunit.ttf,BoldItalicFont=cmuntx.ttf]{cmunrm.ttf} {$\bowtie\,$}&\hspace*{0pt}\ignorespaces{}\hspace*{0pt}{\ttfamily \setmainfont[Path=/usr/share/fonts/truetype/cmu/,UprightFont=cmunrm.ttf,BoldFont=cmunbx.ttf,ItalicFont=cmunti.ttf,BoldItalicFont=cmunbi.ttf]{cmuntt.ttf}\setmonofont[Path=/usr/share/fonts/truetype/cmu/,UprightFont=cmuntt.ttf,BoldFont=cmuntb.ttf,ItalicFont=cmunit.ttf,BoldItalicFont=cmuntx.ttf]{cmuntt.ttf}\ttfamily \textbackslash{}bowtie}\\ \cline{1-1}\cline{2-2}\cline{4-4}\cline{5-5}\cline{7-7}\cline{8-8}\cline{10-10}\cline{11-11}\cline{13-13}\cline{14-14} \hspace*{0pt}\ignorespaces{}\hspace*{0pt}{$\text{ }$}\setmainfont[Path=/usr/share/fonts/truetype/cmu/,UprightFont=cmunrm.ttf,BoldFont=cmunbx.ttf,ItalicFont=cmunti.ttf,BoldItalicFont=cmunbi.ttf]{cmunrm.ttf}\setmonofont[Path=/usr/share/fonts/truetype/cmu/,UprightFont=cmuntt.ttf,BoldFont=cmuntb.ttf,ItalicFont=cmunit.ttf,BoldItalicFont=cmuntx.ttf]{cmunrm.ttf} {$\ll\,$}&\hspace*{0pt}\ignorespaces{}\hspace*{0pt}{\ttfamily \setmainfont[Path=/usr/share/fonts/truetype/cmu/,UprightFont=cmunrm.ttf,BoldFont=cmunbx.ttf,ItalicFont=cmunti.ttf,BoldItalicFont=cmunbi.ttf]{cmuntt.ttf}\setmonofont[Path=/usr/share/fonts/truetype/cmu/,UprightFont=cmuntt.ttf,BoldFont=cmuntb.ttf,ItalicFont=cmunit.ttf,BoldItalicFont=cmuntx.ttf]{cmuntt.ttf}\ttfamily \textbackslash{}ll}&\multicolumn{1}{|c|}{}&\hspace*{0pt}\ignorespaces{}\hspace*{0pt}{$\text{ }$}\setmainfont[Path=/usr/share/fonts/truetype/cmu/,UprightFont=cmunrm.ttf,BoldFont=cmunbx.ttf,ItalicFont=cmunti.ttf,BoldItalicFont=cmunbi.ttf]{cmunrm.ttf}\setmonofont[Path=/usr/share/fonts/truetype/cmu/,UprightFont=cmuntt.ttf,BoldFont=cmuntb.ttf,ItalicFont=cmunit.ttf,BoldItalicFont=cmuntx.ttf]{cmunrm.ttf} {$\gg\,$}&\hspace*{0pt}\ignorespaces{}\hspace*{0pt}{\ttfamily \setmainfont[Path=/usr/share/fonts/truetype/cmu/,UprightFont=cmunrm.ttf,BoldFont=cmunbx.ttf,ItalicFont=cmunti.ttf,BoldItalicFont=cmunbi.ttf]{cmuntt.ttf}\setmonofont[Path=/usr/share/fonts/truetype/cmu/,UprightFont=cmuntt.ttf,BoldFont=cmuntb.ttf,ItalicFont=cmunit.ttf,BoldItalicFont=cmuntx.ttf]{cmuntt.ttf}\ttfamily \textbackslash{}gg}&\multicolumn{1}{|c|}{}&\hspace*{0pt}\ignorespaces{}\hspace*{0pt}{$\text{ }$}\setmainfont[Path=/usr/share/fonts/truetype/cmu/,UprightFont=cmunrm.ttf,BoldFont=cmunbx.ttf,ItalicFont=cmunti.ttf,BoldItalicFont=cmunbi.ttf]{cmunrm.ttf}\setmonofont[Path=/usr/share/fonts/truetype/cmu/,UprightFont=cmuntt.ttf,BoldFont=cmuntb.ttf,ItalicFont=cmunit.ttf,BoldItalicFont=cmuntx.ttf]{cmunrm.ttf} {$\equiv\,$}&\hspace*{0pt}\ignorespaces{}\hspace*{0pt}{\ttfamily \setmainfont[Path=/usr/share/fonts/truetype/cmu/,UprightFont=cmunrm.ttf,BoldFont=cmunbx.ttf,ItalicFont=cmunti.ttf,BoldItalicFont=cmunbi.ttf]{cmuntt.ttf}\setmonofont[Path=/usr/share/fonts/truetype/cmu/,UprightFont=cmuntt.ttf,BoldFont=cmuntb.ttf,ItalicFont=cmunit.ttf,BoldItalicFont=cmuntx.ttf]{cmuntt.ttf}\ttfamily \textbackslash{}equiv}&\multicolumn{1}{|c|}{}&\hspace*{0pt}\ignorespaces{}\hspace*{0pt}{$\text{ }$}\setmainfont[Path=/usr/share/fonts/truetype/cmu/,UprightFont=cmunrm.ttf,BoldFont=cmunbx.ttf,ItalicFont=cmunti.ttf,BoldItalicFont=cmunbi.ttf]{cmunrm.ttf}\setmonofont[Path=/usr/share/fonts/truetype/cmu/,UprightFont=cmuntt.ttf,BoldFont=cmuntb.ttf,ItalicFont=cmunit.ttf,BoldItalicFont=cmuntx.ttf]{cmunrm.ttf} {$\vdash\,$}&\hspace*{0pt}\ignorespaces{}\hspace*{0pt}{\ttfamily \setmainfont[Path=/usr/share/fonts/truetype/cmu/,UprightFont=cmunrm.ttf,BoldFont=cmunbx.ttf,ItalicFont=cmunti.ttf,BoldItalicFont=cmunbi.ttf]{cmuntt.ttf}\setmonofont[Path=/usr/share/fonts/truetype/cmu/,UprightFont=cmuntt.ttf,BoldFont=cmuntb.ttf,ItalicFont=cmunit.ttf,BoldItalicFont=cmuntx.ttf]{cmuntt.ttf}\ttfamily \textbackslash{}vdash}&\multicolumn{1}{|c|}{}&\hspace*{0pt}\ignorespaces{}\hspace*{0pt}{$\text{ }$}\setmainfont[Path=/usr/share/fonts/truetype/cmu/,UprightFont=cmunrm.ttf,BoldFont=cmunbx.ttf,ItalicFont=cmunti.ttf,BoldItalicFont=cmunbi.ttf]{cmunrm.ttf}\setmonofont[Path=/usr/share/fonts/truetype/cmu/,UprightFont=cmuntt.ttf,BoldFont=cmuntb.ttf,ItalicFont=cmunit.ttf,BoldItalicFont=cmuntx.ttf]{cmunrm.ttf} {$\dashv\,$}&\hspace*{0pt}\ignorespaces{}\hspace*{0pt}{\ttfamily \setmainfont[Path=/usr/share/fonts/truetype/cmu/,UprightFont=cmunrm.ttf,BoldFont=cmunbx.ttf,ItalicFont=cmunti.ttf,BoldItalicFont=cmunbi.ttf]{cmuntt.ttf}\setmonofont[Path=/usr/share/fonts/truetype/cmu/,UprightFont=cmuntt.ttf,BoldFont=cmuntb.ttf,ItalicFont=cmunit.ttf,BoldItalicFont=cmuntx.ttf]{cmuntt.ttf}\ttfamily \textbackslash{}dashv}\\ \cline{1-1}\cline{2-2}\cline{4-4}\cline{5-5}\cline{7-7}\cline{8-8}\cline{10-10}\cline{11-11}\cline{13-13}\cline{14-14} \hspace*{0pt}\ignorespaces{}\hspace*{0pt}{$\text{ }$}\setmainfont[Path=/usr/share/fonts/truetype/cmu/,UprightFont=cmunrm.ttf,BoldFont=cmunbx.ttf,ItalicFont=cmunti.ttf,BoldItalicFont=cmunbi.ttf]{cmunrm.ttf}\setmonofont[Path=/usr/share/fonts/truetype/cmu/,UprightFont=cmuntt.ttf,BoldFont=cmuntb.ttf,ItalicFont=cmunit.ttf,BoldItalicFont=cmuntx.ttf]{cmunrm.ttf} {$\subset\,$}&\hspace*{0pt}\ignorespaces{}\hspace*{0pt}{\ttfamily \setmainfont[Path=/usr/share/fonts/truetype/cmu/,UprightFont=cmunrm.ttf,BoldFont=cmunbx.ttf,ItalicFont=cmunti.ttf,BoldItalicFont=cmunbi.ttf]{cmuntt.ttf}\setmonofont[Path=/usr/share/fonts/truetype/cmu/,UprightFont=cmuntt.ttf,BoldFont=cmuntb.ttf,ItalicFont=cmunit.ttf,BoldItalicFont=cmuntx.ttf]{cmuntt.ttf}\ttfamily \textbackslash{}subset}&\multicolumn{1}{|c|}{}&\hspace*{0pt}\ignorespaces{}\hspace*{0pt}{$\text{ }$}\setmainfont[Path=/usr/share/fonts/truetype/cmu/,UprightFont=cmunrm.ttf,BoldFont=cmunbx.ttf,ItalicFont=cmunti.ttf,BoldItalicFont=cmunbi.ttf]{cmunrm.ttf}\setmonofont[Path=/usr/share/fonts/truetype/cmu/,UprightFont=cmuntt.ttf,BoldFont=cmuntb.ttf,ItalicFont=cmunit.ttf,BoldItalicFont=cmuntx.ttf]{cmunrm.ttf} {$\supset\,$}&\hspace*{0pt}\ignorespaces{}\hspace*{0pt}{\ttfamily \setmainfont[Path=/usr/share/fonts/truetype/cmu/,UprightFont=cmunrm.ttf,BoldFont=cmunbx.ttf,ItalicFont=cmunti.ttf,BoldItalicFont=cmunbi.ttf]{cmuntt.ttf}\setmonofont[Path=/usr/share/fonts/truetype/cmu/,UprightFont=cmuntt.ttf,BoldFont=cmuntb.ttf,ItalicFont=cmunit.ttf,BoldItalicFont=cmuntx.ttf]{cmuntt.ttf}\ttfamily \textbackslash{}supset}&\multicolumn{1}{|c|}{}&\hspace*{0pt}\ignorespaces{}\hspace*{0pt}{$\text{ }$}\setmainfont[Path=/usr/share/fonts/truetype/cmu/,UprightFont=cmunrm.ttf,BoldFont=cmunbx.ttf,ItalicFont=cmunti.ttf,BoldItalicFont=cmunbi.ttf]{cmunrm.ttf}\setmonofont[Path=/usr/share/fonts/truetype/cmu/,UprightFont=cmuntt.ttf,BoldFont=cmuntb.ttf,ItalicFont=cmunit.ttf,BoldItalicFont=cmuntx.ttf]{cmunrm.ttf} {$\approx\,$}&\hspace*{0pt}\ignorespaces{}\hspace*{0pt}{\ttfamily \setmainfont[Path=/usr/share/fonts/truetype/cmu/,UprightFont=cmunrm.ttf,BoldFont=cmunbx.ttf,ItalicFont=cmunti.ttf,BoldItalicFont=cmunbi.ttf]{cmuntt.ttf}\setmonofont[Path=/usr/share/fonts/truetype/cmu/,UprightFont=cmuntt.ttf,BoldFont=cmuntb.ttf,ItalicFont=cmunit.ttf,BoldItalicFont=cmuntx.ttf]{cmuntt.ttf}\ttfamily \textbackslash{}approx}&\multicolumn{1}{|c|}{}&\hspace*{0pt}\ignorespaces{}\hspace*{0pt}{$\text{ }$}\setmainfont[Path=/usr/share/fonts/truetype/cmu/,UprightFont=cmunrm.ttf,BoldFont=cmunbx.ttf,ItalicFont=cmunti.ttf,BoldItalicFont=cmunbi.ttf]{cmunrm.ttf}\setmonofont[Path=/usr/share/fonts/truetype/cmu/,UprightFont=cmuntt.ttf,BoldFont=cmuntb.ttf,ItalicFont=cmunit.ttf,BoldItalicFont=cmuntx.ttf]{cmunrm.ttf} {$\in\,$}&\hspace*{0pt}\ignorespaces{}\hspace*{0pt}{\ttfamily \setmainfont[Path=/usr/share/fonts/truetype/cmu/,UprightFont=cmunrm.ttf,BoldFont=cmunbx.ttf,ItalicFont=cmunti.ttf,BoldItalicFont=cmunbi.ttf]{cmuntt.ttf}\setmonofont[Path=/usr/share/fonts/truetype/cmu/,UprightFont=cmuntt.ttf,BoldFont=cmuntb.ttf,ItalicFont=cmunit.ttf,BoldItalicFont=cmuntx.ttf]{cmuntt.ttf}\ttfamily \textbackslash{}in}&\multicolumn{1}{|c|}{}&\hspace*{0pt}\ignorespaces{}\hspace*{0pt}{$\text{ }$}\setmainfont[Path=/usr/share/fonts/truetype/cmu/,UprightFont=cmunrm.ttf,BoldFont=cmunbx.ttf,ItalicFont=cmunti.ttf,BoldItalicFont=cmunbi.ttf]{cmunrm.ttf}\setmonofont[Path=/usr/share/fonts/truetype/cmu/,UprightFont=cmuntt.ttf,BoldFont=cmuntb.ttf,ItalicFont=cmunit.ttf,BoldItalicFont=cmuntx.ttf]{cmunrm.ttf} {$\ni\,$}&\hspace*{0pt}\ignorespaces{}\hspace*{0pt}{\ttfamily \setmainfont[Path=/usr/share/fonts/truetype/cmu/,UprightFont=cmunrm.ttf,BoldFont=cmunbx.ttf,ItalicFont=cmunti.ttf,BoldItalicFont=cmunbi.ttf]{cmuntt.ttf}\setmonofont[Path=/usr/share/fonts/truetype/cmu/,UprightFont=cmuntt.ttf,BoldFont=cmuntb.ttf,ItalicFont=cmunit.ttf,BoldItalicFont=cmuntx.ttf]{cmuntt.ttf}\ttfamily \textbackslash{}ni}\\ \cline{1-1}\cline{2-2}\cline{4-4}\cline{5-5}\cline{7-7}\cline{8-8}\cline{10-10}\cline{11-11}\cline{13-13}\cline{14-14} \hspace*{0pt}\ignorespaces{}\hspace*{0pt}{$\text{ }$}\setmainfont[Path=/usr/share/fonts/truetype/cmu/,UprightFont=cmunrm.ttf,BoldFont=cmunbx.ttf,ItalicFont=cmunti.ttf,BoldItalicFont=cmunbi.ttf]{cmunrm.ttf}\setmonofont[Path=/usr/share/fonts/truetype/cmu/,UprightFont=cmuntt.ttf,BoldFont=cmuntb.ttf,ItalicFont=cmunit.ttf,BoldItalicFont=cmuntx.ttf]{cmunrm.ttf} {$\subseteq\,$}&\hspace*{0pt}\ignorespaces{}\hspace*{0pt}{\ttfamily \setmainfont[Path=/usr/share/fonts/truetype/cmu/,UprightFont=cmunrm.ttf,BoldFont=cmunbx.ttf,ItalicFont=cmunti.ttf,BoldItalicFont=cmunbi.ttf]{cmuntt.ttf}\setmonofont[Path=/usr/share/fonts/truetype/cmu/,UprightFont=cmuntt.ttf,BoldFont=cmuntb.ttf,ItalicFont=cmunit.ttf,BoldItalicFont=cmuntx.ttf]{cmuntt.ttf}\ttfamily \textbackslash{}subseteq}&\multicolumn{1}{|c|}{}&\hspace*{0pt}\ignorespaces{}\hspace*{0pt}{$\text{ }$}\setmainfont[Path=/usr/share/fonts/truetype/cmu/,UprightFont=cmunrm.ttf,BoldFont=cmunbx.ttf,ItalicFont=cmunti.ttf,BoldItalicFont=cmunbi.ttf]{cmunrm.ttf}\setmonofont[Path=/usr/share/fonts/truetype/cmu/,UprightFont=cmuntt.ttf,BoldFont=cmuntb.ttf,ItalicFont=cmunit.ttf,BoldItalicFont=cmuntx.ttf]{cmunrm.ttf} {$\supseteq\,$}&\hspace*{0pt}\ignorespaces{}\hspace*{0pt}{\ttfamily \setmainfont[Path=/usr/share/fonts/truetype/cmu/,UprightFont=cmunrm.ttf,BoldFont=cmunbx.ttf,ItalicFont=cmunti.ttf,BoldItalicFont=cmunbi.ttf]{cmuntt.ttf}\setmonofont[Path=/usr/share/fonts/truetype/cmu/,UprightFont=cmuntt.ttf,BoldFont=cmuntb.ttf,ItalicFont=cmunit.ttf,BoldItalicFont=cmuntx.ttf]{cmuntt.ttf}\ttfamily \textbackslash{}supseteq}&\multicolumn{1}{|c|}{}&\hspace*{0pt}\ignorespaces{}\hspace*{0pt}{$\text{ }$}\setmainfont[Path=/usr/share/fonts/truetype/cmu/,UprightFont=cmunrm.ttf,BoldFont=cmunbx.ttf,ItalicFont=cmunti.ttf,BoldItalicFont=cmunbi.ttf]{cmunrm.ttf}\setmonofont[Path=/usr/share/fonts/truetype/cmu/,UprightFont=cmuntt.ttf,BoldFont=cmuntb.ttf,ItalicFont=cmunit.ttf,BoldItalicFont=cmuntx.ttf]{cmunrm.ttf} {$\cong\,$}&\hspace*{0pt}\ignorespaces{}\hspace*{0pt}{\ttfamily \setmainfont[Path=/usr/share/fonts/truetype/cmu/,UprightFont=cmunrm.ttf,BoldFont=cmunbx.ttf,ItalicFont=cmunti.ttf,BoldItalicFont=cmunbi.ttf]{cmuntt.ttf}\setmonofont[Path=/usr/share/fonts/truetype/cmu/,UprightFont=cmuntt.ttf,BoldFont=cmuntb.ttf,ItalicFont=cmunit.ttf,BoldItalicFont=cmuntx.ttf]{cmuntt.ttf}\ttfamily \textbackslash{}cong}&\multicolumn{1}{|c|}{}&\hspace*{0pt}\ignorespaces{}\hspace*{0pt}{$\text{ }$}\setmainfont[Path=/usr/share/fonts/truetype/cmu/,UprightFont=cmunrm.ttf,BoldFont=cmunbx.ttf,ItalicFont=cmunti.ttf,BoldItalicFont=cmunbi.ttf]{cmunrm.ttf}\setmonofont[Path=/usr/share/fonts/truetype/cmu/,UprightFont=cmuntt.ttf,BoldFont=cmuntb.ttf,ItalicFont=cmunit.ttf,BoldItalicFont=cmuntx.ttf]{cmunrm.ttf} {$\smile\,$}&\hspace*{0pt}\ignorespaces{}\hspace*{0pt}{\ttfamily \setmainfont[Path=/usr/share/fonts/truetype/cmu/,UprightFont=cmunrm.ttf,BoldFont=cmunbx.ttf,ItalicFont=cmunti.ttf,BoldItalicFont=cmunbi.ttf]{cmuntt.ttf}\setmonofont[Path=/usr/share/fonts/truetype/cmu/,UprightFont=cmuntt.ttf,BoldFont=cmuntb.ttf,ItalicFont=cmunit.ttf,BoldItalicFont=cmuntx.ttf]{cmuntt.ttf}\ttfamily \textbackslash{}smile}&\multicolumn{1}{|c|}{}&\hspace*{0pt}\ignorespaces{}\hspace*{0pt}{$\text{ }$}\setmainfont[Path=/usr/share/fonts/truetype/cmu/,UprightFont=cmunrm.ttf,BoldFont=cmunbx.ttf,ItalicFont=cmunti.ttf,BoldItalicFont=cmunbi.ttf]{cmunrm.ttf}\setmonofont[Path=/usr/share/fonts/truetype/cmu/,UprightFont=cmuntt.ttf,BoldFont=cmuntb.ttf,ItalicFont=cmunit.ttf,BoldItalicFont=cmuntx.ttf]{cmunrm.ttf} {$\frown\,$}&\hspace*{0pt}\ignorespaces{}\hspace*{0pt}{\ttfamily \setmainfont[Path=/usr/share/fonts/truetype/cmu/,UprightFont=cmunrm.ttf,BoldFont=cmunbx.ttf,ItalicFont=cmunti.ttf,BoldItalicFont=cmunbi.ttf]{cmuntt.ttf}\setmonofont[Path=/usr/share/fonts/truetype/cmu/,UprightFont=cmuntt.ttf,BoldFont=cmuntb.ttf,ItalicFont=cmunit.ttf,BoldItalicFont=cmuntx.ttf]{cmuntt.ttf}\ttfamily \textbackslash{}frown}\\ \cline{1-1}\cline{2-2}\cline{4-4}\cline{5-5}\cline{7-7}\cline{8-8}\cline{10-10}\cline{11-11}\cline{13-13}\cline{14-14} \hspace*{0pt}\ignorespaces{}\hspace*{0pt}{$\text{ }$}\setmainfont[Path=/usr/share/fonts/truetype/cmu/,UprightFont=cmunrm.ttf,BoldFont=cmunbx.ttf,ItalicFont=cmunti.ttf,BoldItalicFont=cmunbi.ttf]{cmunrm.ttf}\setmonofont[Path=/usr/share/fonts/truetype/cmu/,UprightFont=cmuntt.ttf,BoldFont=cmuntb.ttf,ItalicFont=cmunit.ttf,BoldItalicFont=cmuntx.ttf]{cmunrm.ttf} {$\nsubseteq\,$}&\hspace*{0pt}\ignorespaces{}\hspace*{0pt}{\ttfamily \setmainfont[Path=/usr/share/fonts/truetype/cmu/,UprightFont=cmunrm.ttf,BoldFont=cmunbx.ttf,ItalicFont=cmunti.ttf,BoldItalicFont=cmunbi.ttf]{cmuntt.ttf}\setmonofont[Path=/usr/share/fonts/truetype/cmu/,UprightFont=cmuntt.ttf,BoldFont=cmuntb.ttf,ItalicFont=cmunit.ttf,BoldItalicFont=cmuntx.ttf]{cmuntt.ttf}\ttfamily \textbackslash{}nsubseteq}&\multicolumn{1}{|c|}{}&\hspace*{0pt}\ignorespaces{}\hspace*{0pt}{$\text{ }$}\setmainfont[Path=/usr/share/fonts/truetype/cmu/,UprightFont=cmunrm.ttf,BoldFont=cmunbx.ttf,ItalicFont=cmunti.ttf,BoldItalicFont=cmunbi.ttf]{cmunrm.ttf}\setmonofont[Path=/usr/share/fonts/truetype/cmu/,UprightFont=cmuntt.ttf,BoldFont=cmuntb.ttf,ItalicFont=cmunit.ttf,BoldItalicFont=cmuntx.ttf]{cmunrm.ttf} {$\nsupseteq\,$}&\hspace*{0pt}\ignorespaces{}\hspace*{0pt}{\ttfamily \setmainfont[Path=/usr/share/fonts/truetype/cmu/,UprightFont=cmunrm.ttf,BoldFont=cmunbx.ttf,ItalicFont=cmunti.ttf,BoldItalicFont=cmunbi.ttf]{cmuntt.ttf}\setmonofont[Path=/usr/share/fonts/truetype/cmu/,UprightFont=cmuntt.ttf,BoldFont=cmuntb.ttf,ItalicFont=cmunit.ttf,BoldItalicFont=cmuntx.ttf]{cmuntt.ttf}\ttfamily \textbackslash{}nsupseteq}&\multicolumn{1}{|c|}{}&\hspace*{0pt}\ignorespaces{}\hspace*{0pt}{$\text{ }$}\setmainfont[Path=/usr/share/fonts/truetype/cmu/,UprightFont=cmunrm.ttf,BoldFont=cmunbx.ttf,ItalicFont=cmunti.ttf,BoldItalicFont=cmunbi.ttf]{cmunrm.ttf}\setmonofont[Path=/usr/share/fonts/truetype/cmu/,UprightFont=cmuntt.ttf,BoldFont=cmuntb.ttf,ItalicFont=cmunit.ttf,BoldItalicFont=cmuntx.ttf]{cmunrm.ttf} {$\simeq\,$}&\hspace*{0pt}\ignorespaces{}\hspace*{0pt}{\ttfamily \setmainfont[Path=/usr/share/fonts/truetype/cmu/,UprightFont=cmunrm.ttf,BoldFont=cmunbx.ttf,ItalicFont=cmunti.ttf,BoldItalicFont=cmunbi.ttf]{cmuntt.ttf}\setmonofont[Path=/usr/share/fonts/truetype/cmu/,UprightFont=cmuntt.ttf,BoldFont=cmuntb.ttf,ItalicFont=cmunit.ttf,BoldItalicFont=cmuntx.ttf]{cmuntt.ttf}\ttfamily \textbackslash{}simeq}&\multicolumn{1}{|c|}{}&\hspace*{0pt}\ignorespaces{}\hspace*{0pt}{$\text{ }$}\setmainfont[Path=/usr/share/fonts/truetype/cmu/,UprightFont=cmunrm.ttf,BoldFont=cmunbx.ttf,ItalicFont=cmunti.ttf,BoldItalicFont=cmunbi.ttf]{cmunrm.ttf}\setmonofont[Path=/usr/share/fonts/truetype/cmu/,UprightFont=cmuntt.ttf,BoldFont=cmuntb.ttf,ItalicFont=cmunit.ttf,BoldItalicFont=cmuntx.ttf]{cmunrm.ttf} {$\models\,$}&\hspace*{0pt}\ignorespaces{}\hspace*{0pt}{\ttfamily \setmainfont[Path=/usr/share/fonts/truetype/cmu/,UprightFont=cmunrm.ttf,BoldFont=cmunbx.ttf,ItalicFont=cmunti.ttf,BoldItalicFont=cmunbi.ttf]{cmuntt.ttf}\setmonofont[Path=/usr/share/fonts/truetype/cmu/,UprightFont=cmuntt.ttf,BoldFont=cmuntb.ttf,ItalicFont=cmunit.ttf,BoldItalicFont=cmuntx.ttf]{cmuntt.ttf}\ttfamily \textbackslash{}models}&\multicolumn{1}{|c|}{}&\hspace*{0pt}\ignorespaces{}\hspace*{0pt}{$\text{ }$}\setmainfont[Path=/usr/share/fonts/truetype/cmu/,UprightFont=cmunrm.ttf,BoldFont=cmunbx.ttf,ItalicFont=cmunti.ttf,BoldItalicFont=cmunbi.ttf]{cmunrm.ttf}\setmonofont[Path=/usr/share/fonts/truetype/cmu/,UprightFont=cmuntt.ttf,BoldFont=cmuntb.ttf,ItalicFont=cmunit.ttf,BoldItalicFont=cmuntx.ttf]{cmunrm.ttf} {$\notin\,$}&\hspace*{0pt}\ignorespaces{}\hspace*{0pt}{\ttfamily \setmainfont[Path=/usr/share/fonts/truetype/cmu/,UprightFont=cmunrm.ttf,BoldFont=cmunbx.ttf,ItalicFont=cmunti.ttf,BoldItalicFont=cmunbi.ttf]{cmuntt.ttf}\setmonofont[Path=/usr/share/fonts/truetype/cmu/,UprightFont=cmuntt.ttf,BoldFont=cmuntb.ttf,ItalicFont=cmunit.ttf,BoldItalicFont=cmuntx.ttf]{cmuntt.ttf}\ttfamily \textbackslash{}notin}\\ \cline{1-1}\cline{2-2}\cline{4-4}\cline{5-5}\cline{7-7}\cline{8-8}\cline{10-10}\cline{11-11}\cline{13-13}\cline{14-14} \hspace*{0pt}\ignorespaces{}\hspace*{0pt}{$\text{ }$}\setmainfont[Path=/usr/share/fonts/truetype/cmu/,UprightFont=cmunrm.ttf,BoldFont=cmunbx.ttf,ItalicFont=cmunti.ttf,BoldItalicFont=cmunbi.ttf]{cmunrm.ttf}\setmonofont[Path=/usr/share/fonts/truetype/cmu/,UprightFont=cmuntt.ttf,BoldFont=cmuntb.ttf,ItalicFont=cmunit.ttf,BoldItalicFont=cmuntx.ttf]{cmunrm.ttf} {$\sqsubset\,$}&\hspace*{0pt}\ignorespaces{}\hspace*{0pt}{\ttfamily \setmainfont[Path=/usr/share/fonts/truetype/cmu/,UprightFont=cmunrm.ttf,BoldFont=cmunbx.ttf,ItalicFont=cmunti.ttf,BoldItalicFont=cmunbi.ttf]{cmuntt.ttf}\setmonofont[Path=/usr/share/fonts/truetype/cmu/,UprightFont=cmuntt.ttf,BoldFont=cmuntb.ttf,ItalicFont=cmunit.ttf,BoldItalicFont=cmuntx.ttf]{cmuntt.ttf}\ttfamily \textbackslash{}sqsubset}&\multicolumn{1}{|c|}{}&\hspace*{0pt}\ignorespaces{}\hspace*{0pt}{$\text{ }$}\setmainfont[Path=/usr/share/fonts/truetype/cmu/,UprightFont=cmunrm.ttf,BoldFont=cmunbx.ttf,ItalicFont=cmunti.ttf,BoldItalicFont=cmunbi.ttf]{cmunrm.ttf}\setmonofont[Path=/usr/share/fonts/truetype/cmu/,UprightFont=cmuntt.ttf,BoldFont=cmuntb.ttf,ItalicFont=cmunit.ttf,BoldItalicFont=cmuntx.ttf]{cmunrm.ttf} {$\sqsupset\,$}&\hspace*{0pt}\ignorespaces{}\hspace*{0pt}{\ttfamily \setmainfont[Path=/usr/share/fonts/truetype/cmu/,UprightFont=cmunrm.ttf,BoldFont=cmunbx.ttf,ItalicFont=cmunti.ttf,BoldItalicFont=cmunbi.ttf]{cmuntt.ttf}\setmonofont[Path=/usr/share/fonts/truetype/cmu/,UprightFont=cmuntt.ttf,BoldFont=cmuntb.ttf,ItalicFont=cmunit.ttf,BoldItalicFont=cmuntx.ttf]{cmuntt.ttf}\ttfamily \textbackslash{}sqsupset}&\multicolumn{1}{|c|}{}&\hspace*{0pt}\ignorespaces{}\hspace*{0pt}{$\text{ }$}\setmainfont[Path=/usr/share/fonts/truetype/cmu/,UprightFont=cmunrm.ttf,BoldFont=cmunbx.ttf,ItalicFont=cmunti.ttf,BoldItalicFont=cmunbi.ttf]{cmunrm.ttf}\setmonofont[Path=/usr/share/fonts/truetype/cmu/,UprightFont=cmuntt.ttf,BoldFont=cmuntb.ttf,ItalicFont=cmunit.ttf,BoldItalicFont=cmuntx.ttf]{cmunrm.ttf} {$\sim\,$}&\hspace*{0pt}\ignorespaces{}\hspace*{0pt}{\ttfamily \setmainfont[Path=/usr/share/fonts/truetype/cmu/,UprightFont=cmunrm.ttf,BoldFont=cmunbx.ttf,ItalicFont=cmunti.ttf,BoldItalicFont=cmunbi.ttf]{cmuntt.ttf}\setmonofont[Path=/usr/share/fonts/truetype/cmu/,UprightFont=cmuntt.ttf,BoldFont=cmuntb.ttf,ItalicFont=cmunit.ttf,BoldItalicFont=cmuntx.ttf]{cmuntt.ttf}\ttfamily \textbackslash{}sim}&\multicolumn{1}{|c|}{}&\hspace*{0pt}\ignorespaces{}\hspace*{0pt}{$\text{ }$}\setmainfont[Path=/usr/share/fonts/truetype/cmu/,UprightFont=cmunrm.ttf,BoldFont=cmunbx.ttf,ItalicFont=cmunti.ttf,BoldItalicFont=cmunbi.ttf]{cmunrm.ttf}\setmonofont[Path=/usr/share/fonts/truetype/cmu/,UprightFont=cmuntt.ttf,BoldFont=cmuntb.ttf,ItalicFont=cmunit.ttf,BoldItalicFont=cmuntx.ttf]{cmunrm.ttf} {$\perp\,$}&\hspace*{0pt}\ignorespaces{}\hspace*{0pt}{\ttfamily \setmainfont[Path=/usr/share/fonts/truetype/cmu/,UprightFont=cmunrm.ttf,BoldFont=cmunbx.ttf,ItalicFont=cmunti.ttf,BoldItalicFont=cmunbi.ttf]{cmuntt.ttf}\setmonofont[Path=/usr/share/fonts/truetype/cmu/,UprightFont=cmuntt.ttf,BoldFont=cmuntb.ttf,ItalicFont=cmunit.ttf,BoldItalicFont=cmuntx.ttf]{cmuntt.ttf}\ttfamily \textbackslash{}perp}&\multicolumn{1}{|c|}{}&\hspace*{0pt}\ignorespaces{}\hspace*{0pt}{$\text{ }$}\setmainfont[Path=/usr/share/fonts/truetype/cmu/,UprightFont=cmunrm.ttf,BoldFont=cmunbx.ttf,ItalicFont=cmunti.ttf,BoldItalicFont=cmunbi.ttf]{cmunrm.ttf}\setmonofont[Path=/usr/share/fonts/truetype/cmu/,UprightFont=cmuntt.ttf,BoldFont=cmuntb.ttf,ItalicFont=cmunit.ttf,BoldItalicFont=cmuntx.ttf]{cmunrm.ttf} {$\mid\,$}&\hspace*{0pt}\ignorespaces{}\hspace*{0pt}{\ttfamily \setmainfont[Path=/usr/share/fonts/truetype/cmu/,UprightFont=cmunrm.ttf,BoldFont=cmunbx.ttf,ItalicFont=cmunti.ttf,BoldItalicFont=cmunbi.ttf]{cmuntt.ttf}\setmonofont[Path=/usr/share/fonts/truetype/cmu/,UprightFont=cmuntt.ttf,BoldFont=cmuntb.ttf,ItalicFont=cmunit.ttf,BoldItalicFont=cmuntx.ttf]{cmuntt.ttf}\ttfamily \textbackslash{}mid}\\ \cline{1-1}\cline{2-2}\cline{4-4}\cline{5-5}\cline{7-7}\cline{8-8}\cline{10-10}\cline{11-11}\cline{13-13}\cline{14-14} \hspace*{0pt}\ignorespaces{}\hspace*{0pt}{$\text{ }$}\setmainfont[Path=/usr/share/fonts/truetype/cmu/,UprightFont=cmunrm.ttf,BoldFont=cmunbx.ttf,ItalicFont=cmunti.ttf,BoldItalicFont=cmunbi.ttf]{cmunrm.ttf}\setmonofont[Path=/usr/share/fonts/truetype/cmu/,UprightFont=cmuntt.ttf,BoldFont=cmuntb.ttf,ItalicFont=cmunit.ttf,BoldItalicFont=cmuntx.ttf]{cmunrm.ttf} {$\sqsubseteq\,$}&\hspace*{0pt}\ignorespaces{}\hspace*{0pt}{\ttfamily \setmainfont[Path=/usr/share/fonts/truetype/cmu/,UprightFont=cmunrm.ttf,BoldFont=cmunbx.ttf,ItalicFont=cmunti.ttf,BoldItalicFont=cmunbi.ttf]{cmuntt.ttf}\setmonofont[Path=/usr/share/fonts/truetype/cmu/,UprightFont=cmuntt.ttf,BoldFont=cmuntb.ttf,ItalicFont=cmunit.ttf,BoldItalicFont=cmuntx.ttf]{cmuntt.ttf}\ttfamily \textbackslash{}sqsubseteq}&\multicolumn{1}{|c|}{}&\hspace*{0pt}\ignorespaces{}\hspace*{0pt}{$\text{ }$}\setmainfont[Path=/usr/share/fonts/truetype/cmu/,UprightFont=cmunrm.ttf,BoldFont=cmunbx.ttf,ItalicFont=cmunti.ttf,BoldItalicFont=cmunbi.ttf]{cmunrm.ttf}\setmonofont[Path=/usr/share/fonts/truetype/cmu/,UprightFont=cmuntt.ttf,BoldFont=cmuntb.ttf,ItalicFont=cmunit.ttf,BoldItalicFont=cmuntx.ttf]{cmunrm.ttf} {$\sqsupseteq\,$}&\hspace*{0pt}\ignorespaces{}\hspace*{0pt}{\ttfamily \setmainfont[Path=/usr/share/fonts/truetype/cmu/,UprightFont=cmunrm.ttf,BoldFont=cmunbx.ttf,ItalicFont=cmunti.ttf,BoldItalicFont=cmunbi.ttf]{cmuntt.ttf}\setmonofont[Path=/usr/share/fonts/truetype/cmu/,UprightFont=cmuntt.ttf,BoldFont=cmuntb.ttf,ItalicFont=cmunit.ttf,BoldItalicFont=cmuntx.ttf]{cmuntt.ttf}\ttfamily \textbackslash{}sqsupseteq}&\multicolumn{1}{|c|}{}&\hspace*{0pt}\ignorespaces{}\hspace*{0pt}{$\text{ }$}\setmainfont[Path=/usr/share/fonts/truetype/cmu/,UprightFont=cmunrm.ttf,BoldFont=cmunbx.ttf,ItalicFont=cmunti.ttf,BoldItalicFont=cmunbi.ttf]{cmunrm.ttf}\setmonofont[Path=/usr/share/fonts/truetype/cmu/,UprightFont=cmuntt.ttf,BoldFont=cmuntb.ttf,ItalicFont=cmunit.ttf,BoldItalicFont=cmuntx.ttf]{cmunrm.ttf} {$\propto\,$}&\hspace*{0pt}\ignorespaces{}\hspace*{0pt}{\ttfamily \setmainfont[Path=/usr/share/fonts/truetype/cmu/,UprightFont=cmunrm.ttf,BoldFont=cmunbx.ttf,ItalicFont=cmunti.ttf,BoldItalicFont=cmunbi.ttf]{cmuntt.ttf}\setmonofont[Path=/usr/share/fonts/truetype/cmu/,UprightFont=cmuntt.ttf,BoldFont=cmuntb.ttf,ItalicFont=cmunit.ttf,BoldItalicFont=cmuntx.ttf]{cmuntt.ttf}\ttfamily \textbackslash{}propto}&\multicolumn{1}{|c|}{}&\hspace*{0pt}\ignorespaces{}\hspace*{0pt}{$\text{ }$}\setmainfont[Path=/usr/share/fonts/truetype/cmu/,UprightFont=cmunrm.ttf,BoldFont=cmunbx.ttf,ItalicFont=cmunti.ttf,BoldItalicFont=cmunbi.ttf]{cmunrm.ttf}\setmonofont[Path=/usr/share/fonts/truetype/cmu/,UprightFont=cmuntt.ttf,BoldFont=cmuntb.ttf,ItalicFont=cmunit.ttf,BoldItalicFont=cmuntx.ttf]{cmunrm.ttf} {$\prec\,$}&\hspace*{0pt}\ignorespaces{}\hspace*{0pt}{\ttfamily \setmainfont[Path=/usr/share/fonts/truetype/cmu/,UprightFont=cmunrm.ttf,BoldFont=cmunbx.ttf,ItalicFont=cmunti.ttf,BoldItalicFont=cmunbi.ttf]{cmuntt.ttf}\setmonofont[Path=/usr/share/fonts/truetype/cmu/,UprightFont=cmuntt.ttf,BoldFont=cmuntb.ttf,ItalicFont=cmunit.ttf,BoldItalicFont=cmuntx.ttf]{cmuntt.ttf}\ttfamily \textbackslash{}prec}&\multicolumn{1}{|c|}{}&\hspace*{0pt}\ignorespaces{}\hspace*{0pt}{$\text{ }$}\setmainfont[Path=/usr/share/fonts/truetype/cmu/,UprightFont=cmunrm.ttf,BoldFont=cmunbx.ttf,ItalicFont=cmunti.ttf,BoldItalicFont=cmunbi.ttf]{cmunrm.ttf}\setmonofont[Path=/usr/share/fonts/truetype/cmu/,UprightFont=cmuntt.ttf,BoldFont=cmuntb.ttf,ItalicFont=cmunit.ttf,BoldItalicFont=cmuntx.ttf]{cmunrm.ttf} {$\succ\,$}&\hspace*{0pt}\ignorespaces{}\hspace*{0pt}{\ttfamily \setmainfont[Path=/usr/share/fonts/truetype/cmu/,UprightFont=cmunrm.ttf,BoldFont=cmunbx.ttf,ItalicFont=cmunti.ttf,BoldItalicFont=cmunbi.ttf]{cmuntt.ttf}\setmonofont[Path=/usr/share/fonts/truetype/cmu/,UprightFont=cmuntt.ttf,BoldFont=cmuntb.ttf,ItalicFont=cmunit.ttf,BoldItalicFont=cmuntx.ttf]{cmuntt.ttf}\ttfamily \textbackslash{}succ}\\ \cline{1-1}\cline{2-2}\cline{4-4}\cline{5-5}\cline{7-7}\cline{8-8}\cline{10-10}\cline{11-11}\cline{13-13}\cline{14-14} \hspace*{0pt}\ignorespaces{}\hspace*{0pt}{$\text{ }$}\setmainfont[Path=/usr/share/fonts/truetype/cmu/,UprightFont=cmunrm.ttf,BoldFont=cmunbx.ttf,ItalicFont=cmunti.ttf,BoldItalicFont=cmunbi.ttf]{cmunrm.ttf}\setmonofont[Path=/usr/share/fonts/truetype/cmu/,UprightFont=cmuntt.ttf,BoldFont=cmuntb.ttf,ItalicFont=cmunit.ttf,BoldItalicFont=cmuntx.ttf]{cmunrm.ttf} {$\preceq\,$}&\hspace*{0pt}\ignorespaces{}\hspace*{0pt}{\ttfamily \setmainfont[Path=/usr/share/fonts/truetype/cmu/,UprightFont=cmunrm.ttf,BoldFont=cmunbx.ttf,ItalicFont=cmunti.ttf,BoldItalicFont=cmunbi.ttf]{cmuntt.ttf}\setmonofont[Path=/usr/share/fonts/truetype/cmu/,UprightFont=cmuntt.ttf,BoldFont=cmuntb.ttf,ItalicFont=cmunit.ttf,BoldItalicFont=cmuntx.ttf]{cmuntt.ttf}\ttfamily \textbackslash{}preceq}&\multicolumn{1}{|c|}{}&\hspace*{0pt}\ignorespaces{}\hspace*{0pt}{$\text{ }$}\setmainfont[Path=/usr/share/fonts/truetype/cmu/,UprightFont=cmunrm.ttf,BoldFont=cmunbx.ttf,ItalicFont=cmunti.ttf,BoldItalicFont=cmunbi.ttf]{cmunrm.ttf}\setmonofont[Path=/usr/share/fonts/truetype/cmu/,UprightFont=cmuntt.ttf,BoldFont=cmuntb.ttf,ItalicFont=cmunit.ttf,BoldItalicFont=cmuntx.ttf]{cmunrm.ttf} {$\succeq\,$}&\hspace*{0pt}\ignorespaces{}\hspace*{0pt}{\ttfamily \setmainfont[Path=/usr/share/fonts/truetype/cmu/,UprightFont=cmunrm.ttf,BoldFont=cmunbx.ttf,ItalicFont=cmunti.ttf,BoldItalicFont=cmunbi.ttf]{cmuntt.ttf}\setmonofont[Path=/usr/share/fonts/truetype/cmu/,UprightFont=cmuntt.ttf,BoldFont=cmuntb.ttf,ItalicFont=cmunit.ttf,BoldItalicFont=cmuntx.ttf]{cmuntt.ttf}\ttfamily \textbackslash{}succeq}&\multicolumn{1}{|c|}{}&\hspace*{0pt}\ignorespaces{}\hspace*{0pt}{$\text{ }$}\setmainfont[Path=/usr/share/fonts/truetype/cmu/,UprightFont=cmunrm.ttf,BoldFont=cmunbx.ttf,ItalicFont=cmunti.ttf,BoldItalicFont=cmunbi.ttf]{cmunrm.ttf}\setmonofont[Path=/usr/share/fonts/truetype/cmu/,UprightFont=cmuntt.ttf,BoldFont=cmuntb.ttf,ItalicFont=cmunit.ttf,BoldItalicFont=cmuntx.ttf]{cmunrm.ttf} {$\neq\,$}&\hspace*{0pt}\ignorespaces{}\hspace*{0pt}{\ttfamily \setmainfont[Path=/usr/share/fonts/truetype/cmu/,UprightFont=cmunrm.ttf,BoldFont=cmunbx.ttf,ItalicFont=cmunti.ttf,BoldItalicFont=cmunbi.ttf]{cmuntt.ttf}\setmonofont[Path=/usr/share/fonts/truetype/cmu/,UprightFont=cmuntt.ttf,BoldFont=cmuntb.ttf,ItalicFont=cmunit.ttf,BoldItalicFont=cmuntx.ttf]{cmuntt.ttf}\ttfamily \textbackslash{}neq}&\multicolumn{1}{|c|}{}&\hspace*{0pt}\ignorespaces{}\hspace*{0pt}{$\text{ }$}\setmainfont[Path=/usr/share/fonts/truetype/cmu/,UprightFont=cmunrm.ttf,BoldFont=cmunbx.ttf,ItalicFont=cmunti.ttf,BoldItalicFont=cmunbi.ttf]{cmunrm.ttf}\setmonofont[Path=/usr/share/fonts/truetype/cmu/,UprightFont=cmuntt.ttf,BoldFont=cmuntb.ttf,ItalicFont=cmunit.ttf,BoldItalicFont=cmuntx.ttf]{cmunrm.ttf} {$\sphericalangle\,$}&\hspace*{0pt}\ignorespaces{}\hspace*{0pt}{\ttfamily \setmainfont[Path=/usr/share/fonts/truetype/cmu/,UprightFont=cmunrm.ttf,BoldFont=cmunbx.ttf,ItalicFont=cmunti.ttf,BoldItalicFont=cmunbi.ttf]{cmuntt.ttf}\setmonofont[Path=/usr/share/fonts/truetype/cmu/,UprightFont=cmuntt.ttf,BoldFont=cmuntb.ttf,ItalicFont=cmunit.ttf,BoldItalicFont=cmuntx.ttf]{cmuntt.ttf}\ttfamily \textbackslash{}sphericalangle}&\multicolumn{1}{|c|}{}&\hspace*{0pt}\ignorespaces{}\hspace*{0pt}{$\text{ }$}\setmainfont[Path=/usr/share/fonts/truetype/cmu/,UprightFont=cmunrm.ttf,BoldFont=cmunbx.ttf,ItalicFont=cmunti.ttf,BoldItalicFont=cmunbi.ttf]{cmunrm.ttf}\setmonofont[Path=/usr/share/fonts/truetype/cmu/,UprightFont=cmuntt.ttf,BoldFont=cmuntb.ttf,ItalicFont=cmunit.ttf,BoldItalicFont=cmuntx.ttf]{cmunrm.ttf} {$\measuredangle\,$}&\hspace*{0pt}\ignorespaces{}\hspace*{0pt}{\ttfamily \setmainfont[Path=/usr/share/fonts/truetype/cmu/,UprightFont=cmunrm.ttf,BoldFont=cmunbx.ttf,ItalicFont=cmunti.ttf,BoldItalicFont=cmunbi.ttf]{cmuntt.ttf}\setmonofont[Path=/usr/share/fonts/truetype/cmu/,UprightFont=cmuntt.ttf,BoldFont=cmuntb.ttf,ItalicFont=cmunit.ttf,BoldItalicFont=cmuntx.ttf]{cmuntt.ttf}\ttfamily \textbackslash{}measuredangle}\\ \hline 
\end{longtable}
}}\setmainfont[Path=/usr/share/fonts/truetype/cmu/,UprightFont=cmunrm.ttf,BoldFont=cmunbx.ttf,ItalicFont=cmunti.ttf,BoldItalicFont=cmunbi.ttf]{cmunrm.ttf}\setmonofont[Path=/usr/share/fonts/truetype/cmu/,UprightFont=cmuntt.ttf,BoldFont=cmuntb.ttf,ItalicFont=cmunit.ttf,BoldItalicFont=cmuntx.ttf]{cmunrm.ttf}
{\scriptsize{}
{\scalefont{0.75315}\begin{longtable}{|>{\RaggedRight}p{0.08002\linewidth}|>{\RaggedRight}p{0.06957\linewidth}|>{\RaggedRight}p{0.02425\linewidth}|>{\RaggedRight}p{0.08002\linewidth}|>{\RaggedRight}p{0.08420\linewidth}|>{\RaggedRight}p{0.02425\linewidth}|>{\RaggedRight}p{0.08002\linewidth}|>{\RaggedRight}p{0.06957\linewidth}|>{\RaggedRight}p{0.02425\linewidth}|>{\RaggedRight}p{0.08002\linewidth}|>{\RaggedRight}p{0.06957\linewidth}|} \hline 
\multicolumn{11}{|>{\RaggedRight}p{0.97143\linewidth}|}{{\bfseries \hspace*{0pt}\ignorespaces{}\hspace*{0pt} Binary Operations}}\\ \hline {\bfseries \hspace*{0pt}\ignorespaces{}\hspace*{0pt} Symbol }&{\bfseries \hspace*{0pt}\ignorespaces{}\hspace*{0pt} Script}&\multirow{9}{\linewidth}{\hspace*{0pt}\ignorespaces{}\hspace*{0pt} {\mbox{$~$}}}&{\bfseries \hspace*{0pt}\ignorespaces{}\hspace*{0pt} Symbol }&{\bfseries \hspace*{0pt}\ignorespaces{}\hspace*{0pt} Script}&\multirow{9}{\linewidth}{\hspace*{0pt}\ignorespaces{}\hspace*{0pt} {\mbox{$~$}}}&{\bfseries \hspace*{0pt}\ignorespaces{}\hspace*{0pt} Symbol }&{\bfseries \hspace*{0pt}\ignorespaces{}\hspace*{0pt} Script}&\multirow{9}{\linewidth}{\hspace*{0pt}\ignorespaces{}\hspace*{0pt} {\mbox{$~$}}}&{\bfseries \hspace*{0pt}\ignorespaces{}\hspace*{0pt} Symbol }&{\bfseries \hspace*{0pt}\ignorespaces{}\hspace*{0pt} Script}\\ \cline{1-1}\cline{2-2}\cline{4-4}\cline{5-5}\cline{7-7}\cline{8-8}\cline{10-10}\cline{11-11} \hspace*{0pt}\ignorespaces{}\hspace*{0pt} {$\pm\,$} &\hspace*{0pt}\ignorespaces{}\hspace*{0pt} {\ttfamily \setmainfont[Path=/usr/share/fonts/truetype/cmu/,UprightFont=cmunrm.ttf,BoldFont=cmunbx.ttf,ItalicFont=cmunti.ttf,BoldItalicFont=cmunbi.ttf]{cmuntt.ttf}\setmonofont[Path=/usr/share/fonts/truetype/cmu/,UprightFont=cmuntt.ttf,BoldFont=cmuntb.ttf,ItalicFont=cmunit.ttf,BoldItalicFont=cmuntx.ttf]{cmuntt.ttf}\ttfamily \textbackslash{}pm}&\multicolumn{1}{|c|}{}&\hspace*{0pt}\ignorespaces{}\hspace*{0pt}{$\text{ }$}\setmainfont[Path=/usr/share/fonts/truetype/cmu/,UprightFont=cmunrm.ttf,BoldFont=cmunbx.ttf,ItalicFont=cmunti.ttf,BoldItalicFont=cmunbi.ttf]{cmunrm.ttf}\setmonofont[Path=/usr/share/fonts/truetype/cmu/,UprightFont=cmuntt.ttf,BoldFont=cmuntb.ttf,ItalicFont=cmunit.ttf,BoldItalicFont=cmuntx.ttf]{cmunrm.ttf} {$\cap\,$} &\hspace*{0pt}\ignorespaces{}\hspace*{0pt} {\ttfamily \setmainfont[Path=/usr/share/fonts/truetype/cmu/,UprightFont=cmunrm.ttf,BoldFont=cmunbx.ttf,ItalicFont=cmunti.ttf,BoldItalicFont=cmunbi.ttf]{cmuntt.ttf}\setmonofont[Path=/usr/share/fonts/truetype/cmu/,UprightFont=cmuntt.ttf,BoldFont=cmuntb.ttf,ItalicFont=cmunit.ttf,BoldItalicFont=cmuntx.ttf]{cmuntt.ttf}\ttfamily \textbackslash{}cap}&\multicolumn{1}{|c|}{}&\hspace*{0pt}\ignorespaces{}\hspace*{0pt}{$\text{ }$}\setmainfont[Path=/usr/share/fonts/truetype/cmu/,UprightFont=cmunrm.ttf,BoldFont=cmunbx.ttf,ItalicFont=cmunti.ttf,BoldItalicFont=cmunbi.ttf]{cmunrm.ttf}\setmonofont[Path=/usr/share/fonts/truetype/cmu/,UprightFont=cmuntt.ttf,BoldFont=cmuntb.ttf,ItalicFont=cmunit.ttf,BoldItalicFont=cmuntx.ttf]{cmunrm.ttf} {$\diamond\,$} &\hspace*{0pt}\ignorespaces{}\hspace*{0pt} {\ttfamily \setmainfont[Path=/usr/share/fonts/truetype/cmu/,UprightFont=cmunrm.ttf,BoldFont=cmunbx.ttf,ItalicFont=cmunti.ttf,BoldItalicFont=cmunbi.ttf]{cmuntt.ttf}\setmonofont[Path=/usr/share/fonts/truetype/cmu/,UprightFont=cmuntt.ttf,BoldFont=cmuntb.ttf,ItalicFont=cmunit.ttf,BoldItalicFont=cmuntx.ttf]{cmuntt.ttf}\ttfamily \textbackslash{}diamond}&\multicolumn{1}{|c|}{}&\hspace*{0pt}\ignorespaces{}\hspace*{0pt}{$\text{ }$}\setmainfont[Path=/usr/share/fonts/truetype/cmu/,UprightFont=cmunrm.ttf,BoldFont=cmunbx.ttf,ItalicFont=cmunti.ttf,BoldItalicFont=cmunbi.ttf]{cmunrm.ttf}\setmonofont[Path=/usr/share/fonts/truetype/cmu/,UprightFont=cmuntt.ttf,BoldFont=cmuntb.ttf,ItalicFont=cmunit.ttf,BoldItalicFont=cmuntx.ttf]{cmunrm.ttf} {$\oplus\,$} &\hspace*{0pt}\ignorespaces{}\hspace*{0pt} {\ttfamily \setmainfont[Path=/usr/share/fonts/truetype/cmu/,UprightFont=cmunrm.ttf,BoldFont=cmunbx.ttf,ItalicFont=cmunti.ttf,BoldItalicFont=cmunbi.ttf]{cmuntt.ttf}\setmonofont[Path=/usr/share/fonts/truetype/cmu/,UprightFont=cmuntt.ttf,BoldFont=cmuntb.ttf,ItalicFont=cmunit.ttf,BoldItalicFont=cmuntx.ttf]{cmuntt.ttf}\ttfamily \textbackslash{}oplus}{$\text{ }$}\setmainfont[Path=/usr/share/fonts/truetype/cmu/,UprightFont=cmunrm.ttf,BoldFont=cmunbx.ttf,ItalicFont=cmunti.ttf,BoldItalicFont=cmunbi.ttf]{cmunrm.ttf}\setmonofont[Path=/usr/share/fonts/truetype/cmu/,UprightFont=cmuntt.ttf,BoldFont=cmuntb.ttf,ItalicFont=cmunit.ttf,BoldItalicFont=cmuntx.ttf]{cmunrm.ttf} \\ \cline{1-1}\cline{2-2}\cline{4-4}\cline{5-5}\cline{7-7}\cline{8-8}\cline{10-10}\cline{11-11} \hspace*{0pt}\ignorespaces{}\hspace*{0pt} {$\mp\,$} &\hspace*{0pt}\ignorespaces{}\hspace*{0pt} {\ttfamily \setmainfont[Path=/usr/share/fonts/truetype/cmu/,UprightFont=cmunrm.ttf,BoldFont=cmunbx.ttf,ItalicFont=cmunti.ttf,BoldItalicFont=cmunbi.ttf]{cmuntt.ttf}\setmonofont[Path=/usr/share/fonts/truetype/cmu/,UprightFont=cmuntt.ttf,BoldFont=cmuntb.ttf,ItalicFont=cmunit.ttf,BoldItalicFont=cmuntx.ttf]{cmuntt.ttf}\ttfamily \textbackslash{}mp}&\multicolumn{1}{|c|}{}&\hspace*{0pt}\ignorespaces{}\hspace*{0pt}{$\text{ }$}\setmainfont[Path=/usr/share/fonts/truetype/cmu/,UprightFont=cmunrm.ttf,BoldFont=cmunbx.ttf,ItalicFont=cmunti.ttf,BoldItalicFont=cmunbi.ttf]{cmunrm.ttf}\setmonofont[Path=/usr/share/fonts/truetype/cmu/,UprightFont=cmuntt.ttf,BoldFont=cmuntb.ttf,ItalicFont=cmunit.ttf,BoldItalicFont=cmuntx.ttf]{cmunrm.ttf} {$\cup\,$} &\hspace*{0pt}\ignorespaces{}\hspace*{0pt} {\ttfamily \setmainfont[Path=/usr/share/fonts/truetype/cmu/,UprightFont=cmunrm.ttf,BoldFont=cmunbx.ttf,ItalicFont=cmunti.ttf,BoldItalicFont=cmunbi.ttf]{cmuntt.ttf}\setmonofont[Path=/usr/share/fonts/truetype/cmu/,UprightFont=cmuntt.ttf,BoldFont=cmuntb.ttf,ItalicFont=cmunit.ttf,BoldItalicFont=cmuntx.ttf]{cmuntt.ttf}\ttfamily \textbackslash{}cup}&\multicolumn{1}{|c|}{}&\hspace*{0pt}\ignorespaces{}\hspace*{0pt}{$\text{ }$}\setmainfont[Path=/usr/share/fonts/truetype/cmu/,UprightFont=cmunrm.ttf,BoldFont=cmunbx.ttf,ItalicFont=cmunti.ttf,BoldItalicFont=cmunbi.ttf]{cmunrm.ttf}\setmonofont[Path=/usr/share/fonts/truetype/cmu/,UprightFont=cmuntt.ttf,BoldFont=cmuntb.ttf,ItalicFont=cmunit.ttf,BoldItalicFont=cmuntx.ttf]{cmunrm.ttf} {$\bigtriangleup\,$} &\hspace*{0pt}\ignorespaces{}\hspace*{0pt} {\ttfamily \setmainfont[Path=/usr/share/fonts/truetype/cmu/,UprightFont=cmunrm.ttf,BoldFont=cmunbx.ttf,ItalicFont=cmunti.ttf,BoldItalicFont=cmunbi.ttf]{cmuntt.ttf}\setmonofont[Path=/usr/share/fonts/truetype/cmu/,UprightFont=cmuntt.ttf,BoldFont=cmuntb.ttf,ItalicFont=cmunit.ttf,BoldItalicFont=cmuntx.ttf]{cmuntt.ttf}\ttfamily \textbackslash{}bigtriangleup}&\multicolumn{1}{|c|}{}&\hspace*{0pt}\ignorespaces{}\hspace*{0pt}{$\text{ }$}\setmainfont[Path=/usr/share/fonts/truetype/cmu/,UprightFont=cmunrm.ttf,BoldFont=cmunbx.ttf,ItalicFont=cmunti.ttf,BoldItalicFont=cmunbi.ttf]{cmunrm.ttf}\setmonofont[Path=/usr/share/fonts/truetype/cmu/,UprightFont=cmuntt.ttf,BoldFont=cmuntb.ttf,ItalicFont=cmunit.ttf,BoldItalicFont=cmuntx.ttf]{cmunrm.ttf} {$\ominus\,$} &\hspace*{0pt}\ignorespaces{}\hspace*{0pt} {\ttfamily \setmainfont[Path=/usr/share/fonts/truetype/cmu/,UprightFont=cmunrm.ttf,BoldFont=cmunbx.ttf,ItalicFont=cmunti.ttf,BoldItalicFont=cmunbi.ttf]{cmuntt.ttf}\setmonofont[Path=/usr/share/fonts/truetype/cmu/,UprightFont=cmuntt.ttf,BoldFont=cmuntb.ttf,ItalicFont=cmunit.ttf,BoldItalicFont=cmuntx.ttf]{cmuntt.ttf}\ttfamily \textbackslash{}ominus}{$\text{ }$}\setmainfont[Path=/usr/share/fonts/truetype/cmu/,UprightFont=cmunrm.ttf,BoldFont=cmunbx.ttf,ItalicFont=cmunti.ttf,BoldItalicFont=cmunbi.ttf]{cmunrm.ttf}\setmonofont[Path=/usr/share/fonts/truetype/cmu/,UprightFont=cmuntt.ttf,BoldFont=cmuntb.ttf,ItalicFont=cmunit.ttf,BoldItalicFont=cmuntx.ttf]{cmunrm.ttf} \\ \cline{1-1}\cline{2-2}\cline{4-4}\cline{5-5}\cline{7-7}\cline{8-8}\cline{10-10}\cline{11-11} \hspace*{0pt}\ignorespaces{}\hspace*{0pt} {$\times\,$} &\hspace*{0pt}\ignorespaces{}\hspace*{0pt} {\ttfamily \setmainfont[Path=/usr/share/fonts/truetype/cmu/,UprightFont=cmunrm.ttf,BoldFont=cmunbx.ttf,ItalicFont=cmunti.ttf,BoldItalicFont=cmunbi.ttf]{cmuntt.ttf}\setmonofont[Path=/usr/share/fonts/truetype/cmu/,UprightFont=cmuntt.ttf,BoldFont=cmuntb.ttf,ItalicFont=cmunit.ttf,BoldItalicFont=cmuntx.ttf]{cmuntt.ttf}\ttfamily \textbackslash{}times}&\multicolumn{1}{|c|}{}&\hspace*{0pt}\ignorespaces{}\hspace*{0pt}{$\text{ }$}\setmainfont[Path=/usr/share/fonts/truetype/cmu/,UprightFont=cmunrm.ttf,BoldFont=cmunbx.ttf,ItalicFont=cmunti.ttf,BoldItalicFont=cmunbi.ttf]{cmunrm.ttf}\setmonofont[Path=/usr/share/fonts/truetype/cmu/,UprightFont=cmuntt.ttf,BoldFont=cmuntb.ttf,ItalicFont=cmunit.ttf,BoldItalicFont=cmuntx.ttf]{cmunrm.ttf} {$\uplus\,$} &\hspace*{0pt}\ignorespaces{}\hspace*{0pt} {\ttfamily \setmainfont[Path=/usr/share/fonts/truetype/cmu/,UprightFont=cmunrm.ttf,BoldFont=cmunbx.ttf,ItalicFont=cmunti.ttf,BoldItalicFont=cmunbi.ttf]{cmuntt.ttf}\setmonofont[Path=/usr/share/fonts/truetype/cmu/,UprightFont=cmuntt.ttf,BoldFont=cmuntb.ttf,ItalicFont=cmunit.ttf,BoldItalicFont=cmuntx.ttf]{cmuntt.ttf}\ttfamily \textbackslash{}uplus}&\multicolumn{1}{|c|}{}&\hspace*{0pt}\ignorespaces{}\hspace*{0pt}{$\text{ }$}\setmainfont[Path=/usr/share/fonts/truetype/cmu/,UprightFont=cmunrm.ttf,BoldFont=cmunbx.ttf,ItalicFont=cmunti.ttf,BoldItalicFont=cmunbi.ttf]{cmunrm.ttf}\setmonofont[Path=/usr/share/fonts/truetype/cmu/,UprightFont=cmuntt.ttf,BoldFont=cmuntb.ttf,ItalicFont=cmunit.ttf,BoldItalicFont=cmuntx.ttf]{cmunrm.ttf} {$\bigtriangledown\,$} &\hspace*{0pt}\ignorespaces{}\hspace*{0pt} {\ttfamily \setmainfont[Path=/usr/share/fonts/truetype/cmu/,UprightFont=cmunrm.ttf,BoldFont=cmunbx.ttf,ItalicFont=cmunti.ttf,BoldItalicFont=cmunbi.ttf]{cmuntt.ttf}\setmonofont[Path=/usr/share/fonts/truetype/cmu/,UprightFont=cmuntt.ttf,BoldFont=cmuntb.ttf,ItalicFont=cmunit.ttf,BoldItalicFont=cmuntx.ttf]{cmuntt.ttf}\ttfamily \textbackslash{}bigtriangledown}&\multicolumn{1}{|c|}{}&\hspace*{0pt}\ignorespaces{}\hspace*{0pt}{$\text{ }$}\setmainfont[Path=/usr/share/fonts/truetype/cmu/,UprightFont=cmunrm.ttf,BoldFont=cmunbx.ttf,ItalicFont=cmunti.ttf,BoldItalicFont=cmunbi.ttf]{cmunrm.ttf}\setmonofont[Path=/usr/share/fonts/truetype/cmu/,UprightFont=cmuntt.ttf,BoldFont=cmuntb.ttf,ItalicFont=cmunit.ttf,BoldItalicFont=cmuntx.ttf]{cmunrm.ttf} {$\otimes\,$} &\hspace*{0pt}\ignorespaces{}\hspace*{0pt} {\ttfamily \setmainfont[Path=/usr/share/fonts/truetype/cmu/,UprightFont=cmunrm.ttf,BoldFont=cmunbx.ttf,ItalicFont=cmunti.ttf,BoldItalicFont=cmunbi.ttf]{cmuntt.ttf}\setmonofont[Path=/usr/share/fonts/truetype/cmu/,UprightFont=cmuntt.ttf,BoldFont=cmuntb.ttf,ItalicFont=cmunit.ttf,BoldItalicFont=cmuntx.ttf]{cmuntt.ttf}\ttfamily \textbackslash{}otimes}{$\text{ }$}\setmainfont[Path=/usr/share/fonts/truetype/cmu/,UprightFont=cmunrm.ttf,BoldFont=cmunbx.ttf,ItalicFont=cmunti.ttf,BoldItalicFont=cmunbi.ttf]{cmunrm.ttf}\setmonofont[Path=/usr/share/fonts/truetype/cmu/,UprightFont=cmuntt.ttf,BoldFont=cmuntb.ttf,ItalicFont=cmunit.ttf,BoldItalicFont=cmuntx.ttf]{cmunrm.ttf} \\ \cline{1-1}\cline{2-2}\cline{4-4}\cline{5-5}\cline{7-7}\cline{8-8}\cline{10-10}\cline{11-11} \hspace*{0pt}\ignorespaces{}\hspace*{0pt} {$\div\,$} &\hspace*{0pt}\ignorespaces{}\hspace*{0pt} {\ttfamily \setmainfont[Path=/usr/share/fonts/truetype/cmu/,UprightFont=cmunrm.ttf,BoldFont=cmunbx.ttf,ItalicFont=cmunti.ttf,BoldItalicFont=cmunbi.ttf]{cmuntt.ttf}\setmonofont[Path=/usr/share/fonts/truetype/cmu/,UprightFont=cmuntt.ttf,BoldFont=cmuntb.ttf,ItalicFont=cmunit.ttf,BoldItalicFont=cmuntx.ttf]{cmuntt.ttf}\ttfamily \textbackslash{}div}&\multicolumn{1}{|c|}{}&\hspace*{0pt}\ignorespaces{}\hspace*{0pt}{$\text{ }$}\setmainfont[Path=/usr/share/fonts/truetype/cmu/,UprightFont=cmunrm.ttf,BoldFont=cmunbx.ttf,ItalicFont=cmunti.ttf,BoldItalicFont=cmunbi.ttf]{cmunrm.ttf}\setmonofont[Path=/usr/share/fonts/truetype/cmu/,UprightFont=cmuntt.ttf,BoldFont=cmuntb.ttf,ItalicFont=cmunit.ttf,BoldItalicFont=cmuntx.ttf]{cmunrm.ttf} {$\sqcap\,$} &\hspace*{0pt}\ignorespaces{}\hspace*{0pt} {\ttfamily \setmainfont[Path=/usr/share/fonts/truetype/cmu/,UprightFont=cmunrm.ttf,BoldFont=cmunbx.ttf,ItalicFont=cmunti.ttf,BoldItalicFont=cmunbi.ttf]{cmuntt.ttf}\setmonofont[Path=/usr/share/fonts/truetype/cmu/,UprightFont=cmuntt.ttf,BoldFont=cmuntb.ttf,ItalicFont=cmunit.ttf,BoldItalicFont=cmuntx.ttf]{cmuntt.ttf}\ttfamily \textbackslash{}sqcap}&\multicolumn{1}{|c|}{}&\hspace*{0pt}\ignorespaces{}\hspace*{0pt}{$\text{ }$}\setmainfont[Path=/usr/share/fonts/truetype/cmu/,UprightFont=cmunrm.ttf,BoldFont=cmunbx.ttf,ItalicFont=cmunti.ttf,BoldItalicFont=cmunbi.ttf]{cmunrm.ttf}\setmonofont[Path=/usr/share/fonts/truetype/cmu/,UprightFont=cmuntt.ttf,BoldFont=cmuntb.ttf,ItalicFont=cmunit.ttf,BoldItalicFont=cmuntx.ttf]{cmunrm.ttf} {$\triangleleft\,$} &\hspace*{0pt}\ignorespaces{}\hspace*{0pt} {\ttfamily \setmainfont[Path=/usr/share/fonts/truetype/cmu/,UprightFont=cmunrm.ttf,BoldFont=cmunbx.ttf,ItalicFont=cmunti.ttf,BoldItalicFont=cmunbi.ttf]{cmuntt.ttf}\setmonofont[Path=/usr/share/fonts/truetype/cmu/,UprightFont=cmuntt.ttf,BoldFont=cmuntb.ttf,ItalicFont=cmunit.ttf,BoldItalicFont=cmuntx.ttf]{cmuntt.ttf}\ttfamily \textbackslash{}triangleleft}&\multicolumn{1}{|c|}{}&\hspace*{0pt}\ignorespaces{}\hspace*{0pt}{$\text{ }$}\setmainfont[Path=/usr/share/fonts/truetype/cmu/,UprightFont=cmunrm.ttf,BoldFont=cmunbx.ttf,ItalicFont=cmunti.ttf,BoldItalicFont=cmunbi.ttf]{cmunrm.ttf}\setmonofont[Path=/usr/share/fonts/truetype/cmu/,UprightFont=cmuntt.ttf,BoldFont=cmuntb.ttf,ItalicFont=cmunit.ttf,BoldItalicFont=cmuntx.ttf]{cmunrm.ttf} {$\oslash\,$} &\hspace*{0pt}\ignorespaces{}\hspace*{0pt} {\ttfamily \setmainfont[Path=/usr/share/fonts/truetype/cmu/,UprightFont=cmunrm.ttf,BoldFont=cmunbx.ttf,ItalicFont=cmunti.ttf,BoldItalicFont=cmunbi.ttf]{cmuntt.ttf}\setmonofont[Path=/usr/share/fonts/truetype/cmu/,UprightFont=cmuntt.ttf,BoldFont=cmuntb.ttf,ItalicFont=cmunit.ttf,BoldItalicFont=cmuntx.ttf]{cmuntt.ttf}\ttfamily \textbackslash{}oslash}{$\text{ }$}\setmainfont[Path=/usr/share/fonts/truetype/cmu/,UprightFont=cmunrm.ttf,BoldFont=cmunbx.ttf,ItalicFont=cmunti.ttf,BoldItalicFont=cmunbi.ttf]{cmunrm.ttf}\setmonofont[Path=/usr/share/fonts/truetype/cmu/,UprightFont=cmuntt.ttf,BoldFont=cmuntb.ttf,ItalicFont=cmunit.ttf,BoldItalicFont=cmuntx.ttf]{cmunrm.ttf} \\ \cline{1-1}\cline{2-2}\cline{4-4}\cline{5-5}\cline{7-7}\cline{8-8}\cline{10-10}\cline{11-11} \hspace*{0pt}\ignorespaces{}\hspace*{0pt} {$\ast\,$} &\hspace*{0pt}\ignorespaces{}\hspace*{0pt} {\ttfamily \setmainfont[Path=/usr/share/fonts/truetype/cmu/,UprightFont=cmunrm.ttf,BoldFont=cmunbx.ttf,ItalicFont=cmunti.ttf,BoldItalicFont=cmunbi.ttf]{cmuntt.ttf}\setmonofont[Path=/usr/share/fonts/truetype/cmu/,UprightFont=cmuntt.ttf,BoldFont=cmuntb.ttf,ItalicFont=cmunit.ttf,BoldItalicFont=cmuntx.ttf]{cmuntt.ttf}\ttfamily \textbackslash{}ast}&\multicolumn{1}{|c|}{}&\hspace*{0pt}\ignorespaces{}\hspace*{0pt}{$\text{ }$}\setmainfont[Path=/usr/share/fonts/truetype/cmu/,UprightFont=cmunrm.ttf,BoldFont=cmunbx.ttf,ItalicFont=cmunti.ttf,BoldItalicFont=cmunbi.ttf]{cmunrm.ttf}\setmonofont[Path=/usr/share/fonts/truetype/cmu/,UprightFont=cmuntt.ttf,BoldFont=cmuntb.ttf,ItalicFont=cmunit.ttf,BoldItalicFont=cmuntx.ttf]{cmunrm.ttf} {$\sqcup\,$} &\hspace*{0pt}\ignorespaces{}\hspace*{0pt} {\ttfamily \setmainfont[Path=/usr/share/fonts/truetype/cmu/,UprightFont=cmunrm.ttf,BoldFont=cmunbx.ttf,ItalicFont=cmunti.ttf,BoldItalicFont=cmunbi.ttf]{cmuntt.ttf}\setmonofont[Path=/usr/share/fonts/truetype/cmu/,UprightFont=cmuntt.ttf,BoldFont=cmuntb.ttf,ItalicFont=cmunit.ttf,BoldItalicFont=cmuntx.ttf]{cmuntt.ttf}\ttfamily \textbackslash{}sqcup}&\multicolumn{1}{|c|}{}&\hspace*{0pt}\ignorespaces{}\hspace*{0pt}{$\text{ }$}\setmainfont[Path=/usr/share/fonts/truetype/cmu/,UprightFont=cmunrm.ttf,BoldFont=cmunbx.ttf,ItalicFont=cmunti.ttf,BoldItalicFont=cmunbi.ttf]{cmunrm.ttf}\setmonofont[Path=/usr/share/fonts/truetype/cmu/,UprightFont=cmuntt.ttf,BoldFont=cmuntb.ttf,ItalicFont=cmunit.ttf,BoldItalicFont=cmuntx.ttf]{cmunrm.ttf} {$\triangleright\,$} &\hspace*{0pt}\ignorespaces{}\hspace*{0pt} {\ttfamily \setmainfont[Path=/usr/share/fonts/truetype/cmu/,UprightFont=cmunrm.ttf,BoldFont=cmunbx.ttf,ItalicFont=cmunti.ttf,BoldItalicFont=cmunbi.ttf]{cmuntt.ttf}\setmonofont[Path=/usr/share/fonts/truetype/cmu/,UprightFont=cmuntt.ttf,BoldFont=cmuntb.ttf,ItalicFont=cmunit.ttf,BoldItalicFont=cmuntx.ttf]{cmuntt.ttf}\ttfamily \textbackslash{}triangleright}&\multicolumn{1}{|c|}{}&\hspace*{0pt}\ignorespaces{}\hspace*{0pt}{$\text{ }$}\setmainfont[Path=/usr/share/fonts/truetype/cmu/,UprightFont=cmunrm.ttf,BoldFont=cmunbx.ttf,ItalicFont=cmunti.ttf,BoldItalicFont=cmunbi.ttf]{cmunrm.ttf}\setmonofont[Path=/usr/share/fonts/truetype/cmu/,UprightFont=cmuntt.ttf,BoldFont=cmuntb.ttf,ItalicFont=cmunit.ttf,BoldItalicFont=cmuntx.ttf]{cmunrm.ttf} {$\odot\,$} &\hspace*{0pt}\ignorespaces{}\hspace*{0pt} {\ttfamily \setmainfont[Path=/usr/share/fonts/truetype/cmu/,UprightFont=cmunrm.ttf,BoldFont=cmunbx.ttf,ItalicFont=cmunti.ttf,BoldItalicFont=cmunbi.ttf]{cmuntt.ttf}\setmonofont[Path=/usr/share/fonts/truetype/cmu/,UprightFont=cmuntt.ttf,BoldFont=cmuntb.ttf,ItalicFont=cmunit.ttf,BoldItalicFont=cmuntx.ttf]{cmuntt.ttf}\ttfamily \textbackslash{}odot}{$\text{ }$}\setmainfont[Path=/usr/share/fonts/truetype/cmu/,UprightFont=cmunrm.ttf,BoldFont=cmunbx.ttf,ItalicFont=cmunti.ttf,BoldItalicFont=cmunbi.ttf]{cmunrm.ttf}\setmonofont[Path=/usr/share/fonts/truetype/cmu/,UprightFont=cmuntt.ttf,BoldFont=cmuntb.ttf,ItalicFont=cmunit.ttf,BoldItalicFont=cmuntx.ttf]{cmunrm.ttf} \\ \cline{1-1}\cline{2-2}\cline{4-4}\cline{5-5}\cline{7-7}\cline{8-8}\cline{10-10}\cline{11-11} \hspace*{0pt}\ignorespaces{}\hspace*{0pt} {$\star\,$} &\hspace*{0pt}\ignorespaces{}\hspace*{0pt} {\ttfamily \setmainfont[Path=/usr/share/fonts/truetype/cmu/,UprightFont=cmunrm.ttf,BoldFont=cmunbx.ttf,ItalicFont=cmunti.ttf,BoldItalicFont=cmunbi.ttf]{cmuntt.ttf}\setmonofont[Path=/usr/share/fonts/truetype/cmu/,UprightFont=cmuntt.ttf,BoldFont=cmuntb.ttf,ItalicFont=cmunit.ttf,BoldItalicFont=cmuntx.ttf]{cmuntt.ttf}\ttfamily \textbackslash{}star}&\multicolumn{1}{|c|}{}&\hspace*{0pt}\ignorespaces{}\hspace*{0pt}{$\text{ }$}\setmainfont[Path=/usr/share/fonts/truetype/cmu/,UprightFont=cmunrm.ttf,BoldFont=cmunbx.ttf,ItalicFont=cmunti.ttf,BoldItalicFont=cmunbi.ttf]{cmunrm.ttf}\setmonofont[Path=/usr/share/fonts/truetype/cmu/,UprightFont=cmuntt.ttf,BoldFont=cmuntb.ttf,ItalicFont=cmunit.ttf,BoldItalicFont=cmuntx.ttf]{cmunrm.ttf} {$\vee\,$} &\hspace*{0pt}\ignorespaces{}\hspace*{0pt} {\ttfamily \setmainfont[Path=/usr/share/fonts/truetype/cmu/,UprightFont=cmunrm.ttf,BoldFont=cmunbx.ttf,ItalicFont=cmunti.ttf,BoldItalicFont=cmunbi.ttf]{cmuntt.ttf}\setmonofont[Path=/usr/share/fonts/truetype/cmu/,UprightFont=cmuntt.ttf,BoldFont=cmuntb.ttf,ItalicFont=cmunit.ttf,BoldItalicFont=cmuntx.ttf]{cmuntt.ttf}\ttfamily \textbackslash{}vee}&\multicolumn{1}{|c|}{}&\hspace*{0pt}\ignorespaces{}\hspace*{0pt}{$\text{ }$}\setmainfont[Path=/usr/share/fonts/truetype/cmu/,UprightFont=cmunrm.ttf,BoldFont=cmunbx.ttf,ItalicFont=cmunti.ttf,BoldItalicFont=cmunbi.ttf]{cmunrm.ttf}\setmonofont[Path=/usr/share/fonts/truetype/cmu/,UprightFont=cmuntt.ttf,BoldFont=cmuntb.ttf,ItalicFont=cmunit.ttf,BoldItalicFont=cmuntx.ttf]{cmunrm.ttf} {$\bigcirc\,$} &\hspace*{0pt}\ignorespaces{}\hspace*{0pt} {\ttfamily \setmainfont[Path=/usr/share/fonts/truetype/cmu/,UprightFont=cmunrm.ttf,BoldFont=cmunbx.ttf,ItalicFont=cmunti.ttf,BoldItalicFont=cmunbi.ttf]{cmuntt.ttf}\setmonofont[Path=/usr/share/fonts/truetype/cmu/,UprightFont=cmuntt.ttf,BoldFont=cmuntb.ttf,ItalicFont=cmunit.ttf,BoldItalicFont=cmuntx.ttf]{cmuntt.ttf}\ttfamily \textbackslash{}bigcirc}&\multicolumn{1}{|c|}{}&\hspace*{0pt}\ignorespaces{}\hspace*{0pt}{$\text{ }$}\setmainfont[Path=/usr/share/fonts/truetype/cmu/,UprightFont=cmunrm.ttf,BoldFont=cmunbx.ttf,ItalicFont=cmunti.ttf,BoldItalicFont=cmunbi.ttf]{cmunrm.ttf}\setmonofont[Path=/usr/share/fonts/truetype/cmu/,UprightFont=cmuntt.ttf,BoldFont=cmuntb.ttf,ItalicFont=cmunit.ttf,BoldItalicFont=cmuntx.ttf]{cmunrm.ttf} {$\circ\,$} &\hspace*{0pt}\ignorespaces{}\hspace*{0pt} {\ttfamily \setmainfont[Path=/usr/share/fonts/truetype/cmu/,UprightFont=cmunrm.ttf,BoldFont=cmunbx.ttf,ItalicFont=cmunti.ttf,BoldItalicFont=cmunbi.ttf]{cmuntt.ttf}\setmonofont[Path=/usr/share/fonts/truetype/cmu/,UprightFont=cmuntt.ttf,BoldFont=cmuntb.ttf,ItalicFont=cmunit.ttf,BoldItalicFont=cmuntx.ttf]{cmuntt.ttf}\ttfamily \textbackslash{}circ}{$\text{ }$}\setmainfont[Path=/usr/share/fonts/truetype/cmu/,UprightFont=cmunrm.ttf,BoldFont=cmunbx.ttf,ItalicFont=cmunti.ttf,BoldItalicFont=cmunbi.ttf]{cmunrm.ttf}\setmonofont[Path=/usr/share/fonts/truetype/cmu/,UprightFont=cmuntt.ttf,BoldFont=cmuntb.ttf,ItalicFont=cmunit.ttf,BoldItalicFont=cmuntx.ttf]{cmunrm.ttf} \\ \cline{1-1}\cline{2-2}\cline{4-4}\cline{5-5}\cline{7-7}\cline{8-8}\cline{10-10}\cline{11-11} \hspace*{0pt}\ignorespaces{}\hspace*{0pt} {$\dagger\,$} &\hspace*{0pt}\ignorespaces{}\hspace*{0pt} {\ttfamily \setmainfont[Path=/usr/share/fonts/truetype/cmu/,UprightFont=cmunrm.ttf,BoldFont=cmunbx.ttf,ItalicFont=cmunti.ttf,BoldItalicFont=cmunbi.ttf]{cmuntt.ttf}\setmonofont[Path=/usr/share/fonts/truetype/cmu/,UprightFont=cmuntt.ttf,BoldFont=cmuntb.ttf,ItalicFont=cmunit.ttf,BoldItalicFont=cmuntx.ttf]{cmuntt.ttf}\ttfamily \textbackslash{}dagger}&\multicolumn{1}{|c|}{}&\hspace*{0pt}\ignorespaces{}\hspace*{0pt}{$\text{ }$}\setmainfont[Path=/usr/share/fonts/truetype/cmu/,UprightFont=cmunrm.ttf,BoldFont=cmunbx.ttf,ItalicFont=cmunti.ttf,BoldItalicFont=cmunbi.ttf]{cmunrm.ttf}\setmonofont[Path=/usr/share/fonts/truetype/cmu/,UprightFont=cmuntt.ttf,BoldFont=cmuntb.ttf,ItalicFont=cmunit.ttf,BoldItalicFont=cmuntx.ttf]{cmunrm.ttf} {$\wedge\,$} &\hspace*{0pt}\ignorespaces{}\hspace*{0pt} {\ttfamily \setmainfont[Path=/usr/share/fonts/truetype/cmu/,UprightFont=cmunrm.ttf,BoldFont=cmunbx.ttf,ItalicFont=cmunti.ttf,BoldItalicFont=cmunbi.ttf]{cmuntt.ttf}\setmonofont[Path=/usr/share/fonts/truetype/cmu/,UprightFont=cmuntt.ttf,BoldFont=cmuntb.ttf,ItalicFont=cmunit.ttf,BoldItalicFont=cmuntx.ttf]{cmuntt.ttf}\ttfamily \textbackslash{}wedge}&\multicolumn{1}{|c|}{}&\hspace*{0pt}\ignorespaces{}\hspace*{0pt}{$\text{ }$}\setmainfont[Path=/usr/share/fonts/truetype/cmu/,UprightFont=cmunrm.ttf,BoldFont=cmunbx.ttf,ItalicFont=cmunti.ttf,BoldItalicFont=cmunbi.ttf]{cmunrm.ttf}\setmonofont[Path=/usr/share/fonts/truetype/cmu/,UprightFont=cmuntt.ttf,BoldFont=cmuntb.ttf,ItalicFont=cmunit.ttf,BoldItalicFont=cmuntx.ttf]{cmunrm.ttf} {$\bullet\,$} &\hspace*{0pt}\ignorespaces{}\hspace*{0pt} {\ttfamily \setmainfont[Path=/usr/share/fonts/truetype/cmu/,UprightFont=cmunrm.ttf,BoldFont=cmunbx.ttf,ItalicFont=cmunti.ttf,BoldItalicFont=cmunbi.ttf]{cmuntt.ttf}\setmonofont[Path=/usr/share/fonts/truetype/cmu/,UprightFont=cmuntt.ttf,BoldFont=cmuntb.ttf,ItalicFont=cmunit.ttf,BoldItalicFont=cmuntx.ttf]{cmuntt.ttf}\ttfamily \textbackslash{}bullet}&\multicolumn{1}{|c|}{}&\hspace*{0pt}\ignorespaces{}\hspace*{0pt}{$\text{ }$}\setmainfont[Path=/usr/share/fonts/truetype/cmu/,UprightFont=cmunrm.ttf,BoldFont=cmunbx.ttf,ItalicFont=cmunti.ttf,BoldItalicFont=cmunbi.ttf]{cmunrm.ttf}\setmonofont[Path=/usr/share/fonts/truetype/cmu/,UprightFont=cmuntt.ttf,BoldFont=cmuntb.ttf,ItalicFont=cmunit.ttf,BoldItalicFont=cmuntx.ttf]{cmunrm.ttf} {$\setminus\,$} &\hspace*{0pt}\ignorespaces{}\hspace*{0pt} {\ttfamily \setmainfont[Path=/usr/share/fonts/truetype/cmu/,UprightFont=cmunrm.ttf,BoldFont=cmunbx.ttf,ItalicFont=cmunti.ttf,BoldItalicFont=cmunbi.ttf]{cmuntt.ttf}\setmonofont[Path=/usr/share/fonts/truetype/cmu/,UprightFont=cmuntt.ttf,BoldFont=cmuntb.ttf,ItalicFont=cmunit.ttf,BoldItalicFont=cmuntx.ttf]{cmuntt.ttf}\ttfamily \textbackslash{}setminus}{$\text{ }$}\setmainfont[Path=/usr/share/fonts/truetype/cmu/,UprightFont=cmunrm.ttf,BoldFont=cmunbx.ttf,ItalicFont=cmunti.ttf,BoldItalicFont=cmunbi.ttf]{cmunrm.ttf}\setmonofont[Path=/usr/share/fonts/truetype/cmu/,UprightFont=cmuntt.ttf,BoldFont=cmuntb.ttf,ItalicFont=cmunit.ttf,BoldItalicFont=cmuntx.ttf]{cmunrm.ttf} \\ \cline{1-1}\cline{2-2}\cline{4-4}\cline{5-5}\cline{7-7}\cline{8-8}\cline{10-10}\cline{11-11} \hspace*{0pt}\ignorespaces{}\hspace*{0pt} {$\ddagger\,$} &\hspace*{0pt}\ignorespaces{}\hspace*{0pt} {\ttfamily \setmainfont[Path=/usr/share/fonts/truetype/cmu/,UprightFont=cmunrm.ttf,BoldFont=cmunbx.ttf,ItalicFont=cmunti.ttf,BoldItalicFont=cmunbi.ttf]{cmuntt.ttf}\setmonofont[Path=/usr/share/fonts/truetype/cmu/,UprightFont=cmuntt.ttf,BoldFont=cmuntb.ttf,ItalicFont=cmunit.ttf,BoldItalicFont=cmuntx.ttf]{cmuntt.ttf}\ttfamily \textbackslash{}ddagger}&\multicolumn{1}{|c|}{}&\hspace*{0pt}\ignorespaces{}\hspace*{0pt}{$\text{ }$}\setmainfont[Path=/usr/share/fonts/truetype/cmu/,UprightFont=cmunrm.ttf,BoldFont=cmunbx.ttf,ItalicFont=cmunti.ttf,BoldItalicFont=cmunbi.ttf]{cmunrm.ttf}\setmonofont[Path=/usr/share/fonts/truetype/cmu/,UprightFont=cmuntt.ttf,BoldFont=cmuntb.ttf,ItalicFont=cmunit.ttf,BoldItalicFont=cmuntx.ttf]{cmunrm.ttf} {$\cdot\,$} &\hspace*{0pt}\ignorespaces{}\hspace*{0pt} {\ttfamily \setmainfont[Path=/usr/share/fonts/truetype/cmu/,UprightFont=cmunrm.ttf,BoldFont=cmunbx.ttf,ItalicFont=cmunti.ttf,BoldItalicFont=cmunbi.ttf]{cmuntt.ttf}\setmonofont[Path=/usr/share/fonts/truetype/cmu/,UprightFont=cmuntt.ttf,BoldFont=cmuntb.ttf,ItalicFont=cmunit.ttf,BoldItalicFont=cmuntx.ttf]{cmuntt.ttf}\ttfamily \textbackslash{}cdot}&\multicolumn{1}{|c|}{}&\hspace*{0pt}\ignorespaces{}\hspace*{0pt}{$\text{ }$}\setmainfont[Path=/usr/share/fonts/truetype/cmu/,UprightFont=cmunrm.ttf,BoldFont=cmunbx.ttf,ItalicFont=cmunti.ttf,BoldItalicFont=cmunbi.ttf]{cmunrm.ttf}\setmonofont[Path=/usr/share/fonts/truetype/cmu/,UprightFont=cmuntt.ttf,BoldFont=cmuntb.ttf,ItalicFont=cmunit.ttf,BoldItalicFont=cmuntx.ttf]{cmunrm.ttf} {$\wr\,$} &\hspace*{0pt}\ignorespaces{}\hspace*{0pt} {\ttfamily \setmainfont[Path=/usr/share/fonts/truetype/cmu/,UprightFont=cmunrm.ttf,BoldFont=cmunbx.ttf,ItalicFont=cmunti.ttf,BoldItalicFont=cmunbi.ttf]{cmuntt.ttf}\setmonofont[Path=/usr/share/fonts/truetype/cmu/,UprightFont=cmuntt.ttf,BoldFont=cmuntb.ttf,ItalicFont=cmunit.ttf,BoldItalicFont=cmuntx.ttf]{cmuntt.ttf}\ttfamily \textbackslash{}wr}&\multicolumn{1}{|c|}{}&\hspace*{0pt}\ignorespaces{}\hspace*{0pt}{$\text{ }$}\setmainfont[Path=/usr/share/fonts/truetype/cmu/,UprightFont=cmunrm.ttf,BoldFont=cmunbx.ttf,ItalicFont=cmunti.ttf,BoldItalicFont=cmunbi.ttf]{cmunrm.ttf}\setmonofont[Path=/usr/share/fonts/truetype/cmu/,UprightFont=cmuntt.ttf,BoldFont=cmuntb.ttf,ItalicFont=cmunit.ttf,BoldItalicFont=cmuntx.ttf]{cmunrm.ttf} {$\amalg\,$} &\hspace*{0pt}\ignorespaces{}\hspace*{0pt} {\ttfamily \setmainfont[Path=/usr/share/fonts/truetype/cmu/,UprightFont=cmunrm.ttf,BoldFont=cmunbx.ttf,ItalicFont=cmunti.ttf,BoldItalicFont=cmunbi.ttf]{cmuntt.ttf}\setmonofont[Path=/usr/share/fonts/truetype/cmu/,UprightFont=cmuntt.ttf,BoldFont=cmuntb.ttf,ItalicFont=cmunit.ttf,BoldItalicFont=cmuntx.ttf]{cmuntt.ttf}\ttfamily \textbackslash{}amalg}{$\text{ }$}\setmainfont[Path=/usr/share/fonts/truetype/cmu/,UprightFont=cmunrm.ttf,BoldFont=cmunbx.ttf,ItalicFont=cmunti.ttf,BoldItalicFont=cmunbi.ttf]{cmunrm.ttf}\setmonofont[Path=/usr/share/fonts/truetype/cmu/,UprightFont=cmuntt.ttf,BoldFont=cmuntb.ttf,ItalicFont=cmunit.ttf,BoldItalicFont=cmuntx.ttf]{cmunrm.ttf} \\ \hline 
\end{longtable}
}}
{\scriptsize{}
\begin{longtable}{|>{\RaggedRight}p{0.10341\linewidth}|>{\RaggedRight}p{0.13407\linewidth}|>{\RaggedRight}p{0.03133\linewidth}|>{\RaggedRight}p{0.10913\linewidth}|>{\RaggedRight}p{0.47920\linewidth}|} \hline 
\multicolumn{5}{|>{\RaggedRight}p{0.97143\linewidth}|}{{\bfseries \hspace*{0pt}\ignorespaces{}\hspace*{0pt} Set and/or Logic Notation}}\\ \hline {\bfseries \hspace*{0pt}\ignorespaces{}\hspace*{0pt} Symbol }&{\bfseries \hspace*{0pt}\ignorespaces{}\hspace*{0pt} Script}&\multirow{12}{\linewidth}{\hspace*{0pt}\ignorespaces{}\hspace*{0pt} {\mbox{$~$}}}&{\bfseries \hspace*{0pt}\ignorespaces{}\hspace*{0pt} Symbol }&{\bfseries \hspace*{0pt}\ignorespaces{}\hspace*{0pt} Script}\\ \cline{1-1}\cline{2-2}\cline{4-4}\cline{5-5} \hspace*{0pt}\ignorespaces{}\hspace*{0pt} {$\exists\,$} &\hspace*{0pt}\ignorespaces{}\hspace*{0pt} {\ttfamily \setmainfont[Path=/usr/share/fonts/truetype/cmu/,UprightFont=cmunrm.ttf,BoldFont=cmunbx.ttf,ItalicFont=cmunti.ttf,BoldItalicFont=cmunbi.ttf]{cmuntt.ttf}\setmonofont[Path=/usr/share/fonts/truetype/cmu/,UprightFont=cmuntt.ttf,BoldFont=cmuntb.ttf,ItalicFont=cmunit.ttf,BoldItalicFont=cmuntx.ttf]{cmuntt.ttf}\ttfamily \textbackslash{}exists}&\multicolumn{1}{|c|}{}&\hspace*{0pt}\ignorespaces{}\hspace*{0pt}{$\text{ }$}\setmainfont[Path=/usr/share/fonts/truetype/cmu/,UprightFont=cmunrm.ttf,BoldFont=cmunbx.ttf,ItalicFont=cmunti.ttf,BoldItalicFont=cmunbi.ttf]{cmunrm.ttf}\setmonofont[Path=/usr/share/fonts/truetype/cmu/,UprightFont=cmuntt.ttf,BoldFont=cmuntb.ttf,ItalicFont=cmunit.ttf,BoldItalicFont=cmuntx.ttf]{cmunrm.ttf} {$\rightarrow\,$} &\hspace*{0pt}\ignorespaces{}\hspace*{0pt} {\ttfamily \setmainfont[Path=/usr/share/fonts/truetype/cmu/,UprightFont=cmunrm.ttf,BoldFont=cmunbx.ttf,ItalicFont=cmunti.ttf,BoldItalicFont=cmunbi.ttf]{cmuntt.ttf}\setmonofont[Path=/usr/share/fonts/truetype/cmu/,UprightFont=cmuntt.ttf,BoldFont=cmuntb.ttf,ItalicFont=cmunit.ttf,BoldItalicFont=cmuntx.ttf]{cmuntt.ttf}\ttfamily \textbackslash{}rightarrow}{$\text{ }$}\setmainfont[Path=/usr/share/fonts/truetype/cmu/,UprightFont=cmunrm.ttf,BoldFont=cmunbx.ttf,ItalicFont=cmunti.ttf,BoldItalicFont=cmunbi.ttf]{cmunrm.ttf}\setmonofont[Path=/usr/share/fonts/truetype/cmu/,UprightFont=cmuntt.ttf,BoldFont=cmuntb.ttf,ItalicFont=cmunit.ttf,BoldItalicFont=cmuntx.ttf]{cmunrm.ttf} or {\ttfamily \setmainfont[Path=/usr/share/fonts/truetype/cmu/,UprightFont=cmunrm.ttf,BoldFont=cmunbx.ttf,ItalicFont=cmunti.ttf,BoldItalicFont=cmunbi.ttf]{cmuntt.ttf}\setmonofont[Path=/usr/share/fonts/truetype/cmu/,UprightFont=cmuntt.ttf,BoldFont=cmuntb.ttf,ItalicFont=cmunit.ttf,BoldItalicFont=cmuntx.ttf]{cmuntt.ttf}\ttfamily \textbackslash{}to}\\ \cline{1-1}\cline{2-2}\cline{4-4}\cline{5-5} \hspace*{0pt}\ignorespaces{}\hspace*{0pt}{$\text{ }$}\setmainfont[Path=/usr/share/fonts/truetype/cmu/,UprightFont=cmunrm.ttf,BoldFont=cmunbx.ttf,ItalicFont=cmunti.ttf,BoldItalicFont=cmunbi.ttf]{cmunrm.ttf}\setmonofont[Path=/usr/share/fonts/truetype/cmu/,UprightFont=cmuntt.ttf,BoldFont=cmuntb.ttf,ItalicFont=cmunit.ttf,BoldItalicFont=cmuntx.ttf]{cmunrm.ttf} {$\nexists\,$} &\hspace*{0pt}\ignorespaces{}\hspace*{0pt} {\ttfamily \setmainfont[Path=/usr/share/fonts/truetype/cmu/,UprightFont=cmunrm.ttf,BoldFont=cmunbx.ttf,ItalicFont=cmunti.ttf,BoldItalicFont=cmunbi.ttf]{cmuntt.ttf}\setmonofont[Path=/usr/share/fonts/truetype/cmu/,UprightFont=cmuntt.ttf,BoldFont=cmuntb.ttf,ItalicFont=cmunit.ttf,BoldItalicFont=cmuntx.ttf]{cmuntt.ttf}\ttfamily \textbackslash{}nexists}&\multicolumn{1}{|c|}{}&\hspace*{0pt}\ignorespaces{}\hspace*{0pt}{$\text{ }$}\setmainfont[Path=/usr/share/fonts/truetype/cmu/,UprightFont=cmunrm.ttf,BoldFont=cmunbx.ttf,ItalicFont=cmunti.ttf,BoldItalicFont=cmunbi.ttf]{cmunrm.ttf}\setmonofont[Path=/usr/share/fonts/truetype/cmu/,UprightFont=cmuntt.ttf,BoldFont=cmuntb.ttf,ItalicFont=cmunit.ttf,BoldItalicFont=cmuntx.ttf]{cmunrm.ttf} {$\leftarrow\,$} &\hspace*{0pt}\ignorespaces{}\hspace*{0pt} {\ttfamily \setmainfont[Path=/usr/share/fonts/truetype/cmu/,UprightFont=cmunrm.ttf,BoldFont=cmunbx.ttf,ItalicFont=cmunti.ttf,BoldItalicFont=cmunbi.ttf]{cmuntt.ttf}\setmonofont[Path=/usr/share/fonts/truetype/cmu/,UprightFont=cmuntt.ttf,BoldFont=cmuntb.ttf,ItalicFont=cmunit.ttf,BoldItalicFont=cmuntx.ttf]{cmuntt.ttf}\ttfamily \textbackslash{}leftarrow}{$\text{ }$}\setmainfont[Path=/usr/share/fonts/truetype/cmu/,UprightFont=cmunrm.ttf,BoldFont=cmunbx.ttf,ItalicFont=cmunti.ttf,BoldItalicFont=cmunbi.ttf]{cmunrm.ttf}\setmonofont[Path=/usr/share/fonts/truetype/cmu/,UprightFont=cmuntt.ttf,BoldFont=cmuntb.ttf,ItalicFont=cmunit.ttf,BoldItalicFont=cmuntx.ttf]{cmunrm.ttf} or {\ttfamily \setmainfont[Path=/usr/share/fonts/truetype/cmu/,UprightFont=cmunrm.ttf,BoldFont=cmunbx.ttf,ItalicFont=cmunti.ttf,BoldItalicFont=cmunbi.ttf]{cmuntt.ttf}\setmonofont[Path=/usr/share/fonts/truetype/cmu/,UprightFont=cmuntt.ttf,BoldFont=cmuntb.ttf,ItalicFont=cmunit.ttf,BoldItalicFont=cmuntx.ttf]{cmuntt.ttf}\ttfamily \textbackslash{}gets}\\ \cline{1-1}\cline{2-2}\cline{4-4}\cline{5-5} \hspace*{0pt}\ignorespaces{}\hspace*{0pt}{$\text{ }$}\setmainfont[Path=/usr/share/fonts/truetype/cmu/,UprightFont=cmunrm.ttf,BoldFont=cmunbx.ttf,ItalicFont=cmunti.ttf,BoldItalicFont=cmunbi.ttf]{cmunrm.ttf}\setmonofont[Path=/usr/share/fonts/truetype/cmu/,UprightFont=cmuntt.ttf,BoldFont=cmuntb.ttf,ItalicFont=cmunit.ttf,BoldItalicFont=cmuntx.ttf]{cmunrm.ttf} {$\forall\,$} &\hspace*{0pt}\ignorespaces{}\hspace*{0pt} {\ttfamily \setmainfont[Path=/usr/share/fonts/truetype/cmu/,UprightFont=cmunrm.ttf,BoldFont=cmunbx.ttf,ItalicFont=cmunti.ttf,BoldItalicFont=cmunbi.ttf]{cmuntt.ttf}\setmonofont[Path=/usr/share/fonts/truetype/cmu/,UprightFont=cmuntt.ttf,BoldFont=cmuntb.ttf,ItalicFont=cmunit.ttf,BoldItalicFont=cmuntx.ttf]{cmuntt.ttf}\ttfamily \textbackslash{}forall}&\multicolumn{1}{|c|}{}&\hspace*{0pt}\ignorespaces{}\hspace*{0pt}{$\text{ }$}\setmainfont[Path=/usr/share/fonts/truetype/cmu/,UprightFont=cmunrm.ttf,BoldFont=cmunbx.ttf,ItalicFont=cmunti.ttf,BoldItalicFont=cmunbi.ttf]{cmunrm.ttf}\setmonofont[Path=/usr/share/fonts/truetype/cmu/,UprightFont=cmuntt.ttf,BoldFont=cmuntb.ttf,ItalicFont=cmunit.ttf,BoldItalicFont=cmuntx.ttf]{cmunrm.ttf} {$\mapsto\,$} &\hspace*{0pt}\ignorespaces{}\hspace*{0pt} {\ttfamily \setmainfont[Path=/usr/share/fonts/truetype/cmu/,UprightFont=cmunrm.ttf,BoldFont=cmunbx.ttf,ItalicFont=cmunti.ttf,BoldItalicFont=cmunbi.ttf]{cmuntt.ttf}\setmonofont[Path=/usr/share/fonts/truetype/cmu/,UprightFont=cmuntt.ttf,BoldFont=cmuntb.ttf,ItalicFont=cmunit.ttf,BoldItalicFont=cmuntx.ttf]{cmuntt.ttf}\ttfamily \textbackslash{}mapsto}\\ \cline{1-1}\cline{2-2}\cline{4-4}\cline{5-5} \hspace*{0pt}\ignorespaces{}\hspace*{0pt}{$\text{ }$}\setmainfont[Path=/usr/share/fonts/truetype/cmu/,UprightFont=cmunrm.ttf,BoldFont=cmunbx.ttf,ItalicFont=cmunti.ttf,BoldItalicFont=cmunbi.ttf]{cmunrm.ttf}\setmonofont[Path=/usr/share/fonts/truetype/cmu/,UprightFont=cmuntt.ttf,BoldFont=cmuntb.ttf,ItalicFont=cmunit.ttf,BoldItalicFont=cmuntx.ttf]{cmunrm.ttf} {$\neg\,$} &\hspace*{0pt}\ignorespaces{}\hspace*{0pt} {\ttfamily \setmainfont[Path=/usr/share/fonts/truetype/cmu/,UprightFont=cmunrm.ttf,BoldFont=cmunbx.ttf,ItalicFont=cmunti.ttf,BoldItalicFont=cmunbi.ttf]{cmuntt.ttf}\setmonofont[Path=/usr/share/fonts/truetype/cmu/,UprightFont=cmuntt.ttf,BoldFont=cmuntb.ttf,ItalicFont=cmunit.ttf,BoldItalicFont=cmuntx.ttf]{cmuntt.ttf}\ttfamily \textbackslash{}neg}&\multicolumn{1}{|c|}{}&\hspace*{0pt}\ignorespaces{}\hspace*{0pt}{$\text{ }$}\setmainfont[Path=/usr/share/fonts/truetype/cmu/,UprightFont=cmunrm.ttf,BoldFont=cmunbx.ttf,ItalicFont=cmunti.ttf,BoldItalicFont=cmunbi.ttf]{cmunrm.ttf}\setmonofont[Path=/usr/share/fonts/truetype/cmu/,UprightFont=cmuntt.ttf,BoldFont=cmuntb.ttf,ItalicFont=cmunit.ttf,BoldItalicFont=cmuntx.ttf]{cmunrm.ttf} {$\implies\,$} &\hspace*{0pt}\ignorespaces{}\hspace*{0pt} {\ttfamily \setmainfont[Path=/usr/share/fonts/truetype/cmu/,UprightFont=cmunrm.ttf,BoldFont=cmunbx.ttf,ItalicFont=cmunti.ttf,BoldItalicFont=cmunbi.ttf]{cmuntt.ttf}\setmonofont[Path=/usr/share/fonts/truetype/cmu/,UprightFont=cmuntt.ttf,BoldFont=cmuntb.ttf,ItalicFont=cmunit.ttf,BoldItalicFont=cmuntx.ttf]{cmuntt.ttf}\ttfamily \textbackslash{}implies}\\ \cline{1-1}\cline{2-2}\cline{4-4}\cline{5-5} \hspace*{0pt}\ignorespaces{}\hspace*{0pt}{$\text{ }$}\setmainfont[Path=/usr/share/fonts/truetype/cmu/,UprightFont=cmunrm.ttf,BoldFont=cmunbx.ttf,ItalicFont=cmunti.ttf,BoldItalicFont=cmunbi.ttf]{cmunrm.ttf}\setmonofont[Path=/usr/share/fonts/truetype/cmu/,UprightFont=cmuntt.ttf,BoldFont=cmuntb.ttf,ItalicFont=cmunit.ttf,BoldItalicFont=cmuntx.ttf]{cmunrm.ttf} {$\subset\,$} &\hspace*{0pt}\ignorespaces{}\hspace*{0pt} {\ttfamily \setmainfont[Path=/usr/share/fonts/truetype/cmu/,UprightFont=cmunrm.ttf,BoldFont=cmunbx.ttf,ItalicFont=cmunti.ttf,BoldItalicFont=cmunbi.ttf]{cmuntt.ttf}\setmonofont[Path=/usr/share/fonts/truetype/cmu/,UprightFont=cmuntt.ttf,BoldFont=cmuntb.ttf,ItalicFont=cmunit.ttf,BoldItalicFont=cmuntx.ttf]{cmuntt.ttf}\ttfamily \textbackslash{}subset}&\multicolumn{1}{|c|}{}&\hspace*{0pt}\ignorespaces{}\hspace*{0pt}{$\text{ }$}\setmainfont[Path=/usr/share/fonts/truetype/cmu/,UprightFont=cmunrm.ttf,BoldFont=cmunbx.ttf,ItalicFont=cmunti.ttf,BoldItalicFont=cmunbi.ttf]{cmunrm.ttf}\setmonofont[Path=/usr/share/fonts/truetype/cmu/,UprightFont=cmuntt.ttf,BoldFont=cmuntb.ttf,ItalicFont=cmunit.ttf,BoldItalicFont=cmuntx.ttf]{cmunrm.ttf} {$\Rightarrow\,$} &\hspace*{0pt}\ignorespaces{}\hspace*{0pt} {\ttfamily \setmainfont[Path=/usr/share/fonts/truetype/cmu/,UprightFont=cmunrm.ttf,BoldFont=cmunbx.ttf,ItalicFont=cmunti.ttf,BoldItalicFont=cmunbi.ttf]{cmuntt.ttf}\setmonofont[Path=/usr/share/fonts/truetype/cmu/,UprightFont=cmuntt.ttf,BoldFont=cmuntb.ttf,ItalicFont=cmunit.ttf,BoldItalicFont=cmuntx.ttf]{cmuntt.ttf}\ttfamily \textbackslash{}Rightarrow}{$\text{ }$}\setmainfont[Path=/usr/share/fonts/truetype/cmu/,UprightFont=cmunrm.ttf,BoldFont=cmunbx.ttf,ItalicFont=cmunti.ttf,BoldItalicFont=cmunbi.ttf]{cmunrm.ttf}\setmonofont[Path=/usr/share/fonts/truetype/cmu/,UprightFont=cmuntt.ttf,BoldFont=cmuntb.ttf,ItalicFont=cmunit.ttf,BoldItalicFont=cmuntx.ttf]{cmunrm.ttf} or {\ttfamily \setmainfont[Path=/usr/share/fonts/truetype/cmu/,UprightFont=cmunrm.ttf,BoldFont=cmunbx.ttf,ItalicFont=cmunti.ttf,BoldItalicFont=cmunbi.ttf]{cmuntt.ttf}\setmonofont[Path=/usr/share/fonts/truetype/cmu/,UprightFont=cmuntt.ttf,BoldFont=cmuntb.ttf,ItalicFont=cmunit.ttf,BoldItalicFont=cmuntx.ttf]{cmuntt.ttf}\ttfamily \textbackslash{}implies}\\ \cline{1-1}\cline{2-2}\cline{4-4}\cline{5-5} \hspace*{0pt}\ignorespaces{}\hspace*{0pt}{$\text{ }$}\setmainfont[Path=/usr/share/fonts/truetype/cmu/,UprightFont=cmunrm.ttf,BoldFont=cmunbx.ttf,ItalicFont=cmunti.ttf,BoldItalicFont=cmunbi.ttf]{cmunrm.ttf}\setmonofont[Path=/usr/share/fonts/truetype/cmu/,UprightFont=cmuntt.ttf,BoldFont=cmuntb.ttf,ItalicFont=cmunit.ttf,BoldItalicFont=cmuntx.ttf]{cmunrm.ttf} {$\supset\,$} &\hspace*{0pt}\ignorespaces{}\hspace*{0pt} {\ttfamily \setmainfont[Path=/usr/share/fonts/truetype/cmu/,UprightFont=cmunrm.ttf,BoldFont=cmunbx.ttf,ItalicFont=cmunti.ttf,BoldItalicFont=cmunbi.ttf]{cmuntt.ttf}\setmonofont[Path=/usr/share/fonts/truetype/cmu/,UprightFont=cmuntt.ttf,BoldFont=cmuntb.ttf,ItalicFont=cmunit.ttf,BoldItalicFont=cmuntx.ttf]{cmuntt.ttf}\ttfamily \textbackslash{}supset}&\multicolumn{1}{|c|}{}&\hspace*{0pt}\ignorespaces{}\hspace*{0pt}{$\text{ }$}\setmainfont[Path=/usr/share/fonts/truetype/cmu/,UprightFont=cmunrm.ttf,BoldFont=cmunbx.ttf,ItalicFont=cmunti.ttf,BoldItalicFont=cmunbi.ttf]{cmunrm.ttf}\setmonofont[Path=/usr/share/fonts/truetype/cmu/,UprightFont=cmuntt.ttf,BoldFont=cmuntb.ttf,ItalicFont=cmunit.ttf,BoldItalicFont=cmuntx.ttf]{cmunrm.ttf} {$\leftrightarrow\,$} &\hspace*{0pt}\ignorespaces{}\hspace*{0pt} {\ttfamily \setmainfont[Path=/usr/share/fonts/truetype/cmu/,UprightFont=cmunrm.ttf,BoldFont=cmunbx.ttf,ItalicFont=cmunti.ttf,BoldItalicFont=cmunbi.ttf]{cmuntt.ttf}\setmonofont[Path=/usr/share/fonts/truetype/cmu/,UprightFont=cmuntt.ttf,BoldFont=cmuntb.ttf,ItalicFont=cmunit.ttf,BoldItalicFont=cmuntx.ttf]{cmuntt.ttf}\ttfamily \textbackslash{}leftrightarrow}\\ \cline{1-1}\cline{2-2}\cline{4-4}\cline{5-5} \hspace*{0pt}\ignorespaces{}\hspace*{0pt}{$\text{ }$}\setmainfont[Path=/usr/share/fonts/truetype/cmu/,UprightFont=cmunrm.ttf,BoldFont=cmunbx.ttf,ItalicFont=cmunti.ttf,BoldItalicFont=cmunbi.ttf]{cmunrm.ttf}\setmonofont[Path=/usr/share/fonts/truetype/cmu/,UprightFont=cmuntt.ttf,BoldFont=cmuntb.ttf,ItalicFont=cmunit.ttf,BoldItalicFont=cmuntx.ttf]{cmunrm.ttf} {$\in$}&\hspace*{0pt}\ignorespaces{}\hspace*{0pt}{\ttfamily \setmainfont[Path=/usr/share/fonts/truetype/cmu/,UprightFont=cmunrm.ttf,BoldFont=cmunbx.ttf,ItalicFont=cmunti.ttf,BoldItalicFont=cmunbi.ttf]{cmuntt.ttf}\setmonofont[Path=/usr/share/fonts/truetype/cmu/,UprightFont=cmuntt.ttf,BoldFont=cmuntb.ttf,ItalicFont=cmunit.ttf,BoldItalicFont=cmuntx.ttf]{cmuntt.ttf}\ttfamily \textbackslash{}in}&\multicolumn{1}{|c|}{}&\hspace*{0pt}\ignorespaces{}\hspace*{0pt}{$\text{ }$}\setmainfont[Path=/usr/share/fonts/truetype/cmu/,UprightFont=cmunrm.ttf,BoldFont=cmunbx.ttf,ItalicFont=cmunti.ttf,BoldItalicFont=cmunbi.ttf]{cmunrm.ttf}\setmonofont[Path=/usr/share/fonts/truetype/cmu/,UprightFont=cmuntt.ttf,BoldFont=cmuntb.ttf,ItalicFont=cmunit.ttf,BoldItalicFont=cmuntx.ttf]{cmunrm.ttf} {$\iff\,$} &\hspace*{0pt}\ignorespaces{}\hspace*{0pt} {\ttfamily \setmainfont[Path=/usr/share/fonts/truetype/cmu/,UprightFont=cmunrm.ttf,BoldFont=cmunbx.ttf,ItalicFont=cmunti.ttf,BoldItalicFont=cmunbi.ttf]{cmuntt.ttf}\setmonofont[Path=/usr/share/fonts/truetype/cmu/,UprightFont=cmuntt.ttf,BoldFont=cmuntb.ttf,ItalicFont=cmunit.ttf,BoldItalicFont=cmuntx.ttf]{cmuntt.ttf}\ttfamily \textbackslash{}iff}\\ \cline{1-1}\cline{2-2}\cline{4-4}\cline{5-5} \hspace*{0pt}\ignorespaces{}\hspace*{0pt}{$\text{ }$}\setmainfont[Path=/usr/share/fonts/truetype/cmu/,UprightFont=cmunrm.ttf,BoldFont=cmunbx.ttf,ItalicFont=cmunti.ttf,BoldItalicFont=cmunbi.ttf]{cmunrm.ttf}\setmonofont[Path=/usr/share/fonts/truetype/cmu/,UprightFont=cmuntt.ttf,BoldFont=cmuntb.ttf,ItalicFont=cmunit.ttf,BoldItalicFont=cmuntx.ttf]{cmunrm.ttf} {$\notin\,$} &\hspace*{0pt}\ignorespaces{}\hspace*{0pt} {\ttfamily \setmainfont[Path=/usr/share/fonts/truetype/cmu/,UprightFont=cmunrm.ttf,BoldFont=cmunbx.ttf,ItalicFont=cmunti.ttf,BoldItalicFont=cmunbi.ttf]{cmuntt.ttf}\setmonofont[Path=/usr/share/fonts/truetype/cmu/,UprightFont=cmuntt.ttf,BoldFont=cmuntb.ttf,ItalicFont=cmunit.ttf,BoldItalicFont=cmuntx.ttf]{cmuntt.ttf}\ttfamily \textbackslash{}notin}&\multicolumn{1}{|c|}{}&\hspace*{0pt}\ignorespaces{}\hspace*{0pt}{$\text{ }$}\setmainfont[Path=/usr/share/fonts/truetype/cmu/,UprightFont=cmunrm.ttf,BoldFont=cmunbx.ttf,ItalicFont=cmunti.ttf,BoldItalicFont=cmunbi.ttf]{cmunrm.ttf}\setmonofont[Path=/usr/share/fonts/truetype/cmu/,UprightFont=cmuntt.ttf,BoldFont=cmuntb.ttf,ItalicFont=cmunit.ttf,BoldItalicFont=cmuntx.ttf]{cmunrm.ttf} {$\Leftrightarrow\,$} &\hspace*{0pt}\ignorespaces{}\hspace*{0pt} {\ttfamily \setmainfont[Path=/usr/share/fonts/truetype/cmu/,UprightFont=cmunrm.ttf,BoldFont=cmunbx.ttf,ItalicFont=cmunti.ttf,BoldItalicFont=cmunbi.ttf]{cmuntt.ttf}\setmonofont[Path=/usr/share/fonts/truetype/cmu/,UprightFont=cmuntt.ttf,BoldFont=cmuntb.ttf,ItalicFont=cmunit.ttf,BoldItalicFont=cmuntx.ttf]{cmuntt.ttf}\ttfamily \textbackslash{}Leftrightarrow}{$\text{ }$}\setmainfont[Path=/usr/share/fonts/truetype/cmu/,UprightFont=cmunrm.ttf,BoldFont=cmunbx.ttf,ItalicFont=cmunti.ttf,BoldItalicFont=cmunbi.ttf]{cmunrm.ttf}\setmonofont[Path=/usr/share/fonts/truetype/cmu/,UprightFont=cmuntt.ttf,BoldFont=cmuntb.ttf,ItalicFont=cmunit.ttf,BoldItalicFont=cmuntx.ttf]{cmunrm.ttf} (preferred for equivalence (iff))\\ \cline{1-1}\cline{2-2}\cline{4-4}\cline{5-5} \hspace*{0pt}\ignorespaces{}\hspace*{0pt} {$\ni\,$} &\hspace*{0pt}\ignorespaces{}\hspace*{0pt} {\ttfamily \setmainfont[Path=/usr/share/fonts/truetype/cmu/,UprightFont=cmunrm.ttf,BoldFont=cmunbx.ttf,ItalicFont=cmunti.ttf,BoldItalicFont=cmunbi.ttf]{cmuntt.ttf}\setmonofont[Path=/usr/share/fonts/truetype/cmu/,UprightFont=cmuntt.ttf,BoldFont=cmuntb.ttf,ItalicFont=cmunit.ttf,BoldItalicFont=cmuntx.ttf]{cmuntt.ttf}\ttfamily \textbackslash{}ni}&\multicolumn{1}{|c|}{}&\hspace*{0pt}\ignorespaces{}\hspace*{0pt}{$\text{ }$}\setmainfont[Path=/usr/share/fonts/truetype/cmu/,UprightFont=cmunrm.ttf,BoldFont=cmunbx.ttf,ItalicFont=cmunti.ttf,BoldItalicFont=cmunbi.ttf]{cmunrm.ttf}\setmonofont[Path=/usr/share/fonts/truetype/cmu/,UprightFont=cmuntt.ttf,BoldFont=cmuntb.ttf,ItalicFont=cmunit.ttf,BoldItalicFont=cmuntx.ttf]{cmunrm.ttf} {$\top\,$} &\hspace*{0pt}\ignorespaces{}\hspace*{0pt} {\ttfamily \setmainfont[Path=/usr/share/fonts/truetype/cmu/,UprightFont=cmunrm.ttf,BoldFont=cmunbx.ttf,ItalicFont=cmunti.ttf,BoldItalicFont=cmunbi.ttf]{cmuntt.ttf}\setmonofont[Path=/usr/share/fonts/truetype/cmu/,UprightFont=cmuntt.ttf,BoldFont=cmuntb.ttf,ItalicFont=cmunit.ttf,BoldItalicFont=cmuntx.ttf]{cmuntt.ttf}\ttfamily \textbackslash{}top}\\ \cline{1-1}\cline{2-2}\cline{4-4}\cline{5-5} \hspace*{0pt}\ignorespaces{}\hspace*{0pt}{$\text{ }$}\setmainfont[Path=/usr/share/fonts/truetype/cmu/,UprightFont=cmunrm.ttf,BoldFont=cmunbx.ttf,ItalicFont=cmunti.ttf,BoldItalicFont=cmunbi.ttf]{cmunrm.ttf}\setmonofont[Path=/usr/share/fonts/truetype/cmu/,UprightFont=cmuntt.ttf,BoldFont=cmuntb.ttf,ItalicFont=cmunit.ttf,BoldItalicFont=cmuntx.ttf]{cmunrm.ttf} {$\land\,$} &\hspace*{0pt}\ignorespaces{}\hspace*{0pt} {\ttfamily \setmainfont[Path=/usr/share/fonts/truetype/cmu/,UprightFont=cmunrm.ttf,BoldFont=cmunbx.ttf,ItalicFont=cmunti.ttf,BoldItalicFont=cmunbi.ttf]{cmuntt.ttf}\setmonofont[Path=/usr/share/fonts/truetype/cmu/,UprightFont=cmuntt.ttf,BoldFont=cmuntb.ttf,ItalicFont=cmunit.ttf,BoldItalicFont=cmuntx.ttf]{cmuntt.ttf}\ttfamily \textbackslash{}land}&\multicolumn{1}{|c|}{}&\hspace*{0pt}\ignorespaces{}\hspace*{0pt}{$\text{ }$}\setmainfont[Path=/usr/share/fonts/truetype/cmu/,UprightFont=cmunrm.ttf,BoldFont=cmunbx.ttf,ItalicFont=cmunti.ttf,BoldItalicFont=cmunbi.ttf]{cmunrm.ttf}\setmonofont[Path=/usr/share/fonts/truetype/cmu/,UprightFont=cmuntt.ttf,BoldFont=cmuntb.ttf,ItalicFont=cmunit.ttf,BoldItalicFont=cmuntx.ttf]{cmunrm.ttf} {$\bot\,$} &\hspace*{0pt}\ignorespaces{}\hspace*{0pt} {\ttfamily \setmainfont[Path=/usr/share/fonts/truetype/cmu/,UprightFont=cmunrm.ttf,BoldFont=cmunbx.ttf,ItalicFont=cmunti.ttf,BoldItalicFont=cmunbi.ttf]{cmuntt.ttf}\setmonofont[Path=/usr/share/fonts/truetype/cmu/,UprightFont=cmuntt.ttf,BoldFont=cmuntb.ttf,ItalicFont=cmunit.ttf,BoldItalicFont=cmuntx.ttf]{cmuntt.ttf}\ttfamily \textbackslash{}bot}\\ \cline{1-1}\cline{2-2}\cline{4-4}\cline{5-5} \hspace*{0pt}\ignorespaces{}\hspace*{0pt}{$\text{ }$}\setmainfont[Path=/usr/share/fonts/truetype/cmu/,UprightFont=cmunrm.ttf,BoldFont=cmunbx.ttf,ItalicFont=cmunti.ttf,BoldItalicFont=cmunbi.ttf]{cmunrm.ttf}\setmonofont[Path=/usr/share/fonts/truetype/cmu/,UprightFont=cmuntt.ttf,BoldFont=cmuntb.ttf,ItalicFont=cmunit.ttf,BoldItalicFont=cmuntx.ttf]{cmunrm.ttf} {$\lor\,$} &\hspace*{0pt}\ignorespaces{}\hspace*{0pt} {\ttfamily \setmainfont[Path=/usr/share/fonts/truetype/cmu/,UprightFont=cmunrm.ttf,BoldFont=cmunbx.ttf,ItalicFont=cmunti.ttf,BoldItalicFont=cmunbi.ttf]{cmuntt.ttf}\setmonofont[Path=/usr/share/fonts/truetype/cmu/,UprightFont=cmuntt.ttf,BoldFont=cmuntb.ttf,ItalicFont=cmunit.ttf,BoldItalicFont=cmuntx.ttf]{cmuntt.ttf}\ttfamily \textbackslash{}lor}&\multicolumn{1}{|c|}{}&\hspace*{0pt}\ignorespaces{}\hspace*{0pt}{$\text{ }$}\setmainfont[Path=/usr/share/fonts/truetype/cmu/,UprightFont=cmunrm.ttf,BoldFont=cmunbx.ttf,ItalicFont=cmunti.ttf,BoldItalicFont=cmunbi.ttf]{cmunrm.ttf}\setmonofont[Path=/usr/share/fonts/truetype/cmu/,UprightFont=cmuntt.ttf,BoldFont=cmuntb.ttf,ItalicFont=cmunit.ttf,BoldItalicFont=cmuntx.ttf]{cmunrm.ttf} {$\emptyset\,$} and {$\varnothing\,$} &\hspace*{0pt}\ignorespaces{}\hspace*{0pt} {\ttfamily \setmainfont[Path=/usr/share/fonts/truetype/cmu/,UprightFont=cmunrm.ttf,BoldFont=cmunbx.ttf,ItalicFont=cmunti.ttf,BoldItalicFont=cmunbi.ttf]{cmuntt.ttf}\setmonofont[Path=/usr/share/fonts/truetype/cmu/,UprightFont=cmuntt.ttf,BoldFont=cmuntb.ttf,ItalicFont=cmunit.ttf,BoldItalicFont=cmuntx.ttf]{cmuntt.ttf}\ttfamily \textbackslash{}emptyset}{$\text{ }$}\setmainfont[Path=/usr/share/fonts/truetype/cmu/,UprightFont=cmunrm.ttf,BoldFont=cmunbx.ttf,ItalicFont=cmunti.ttf,BoldItalicFont=cmunbi.ttf]{cmunrm.ttf}\setmonofont[Path=/usr/share/fonts/truetype/cmu/,UprightFont=cmuntt.ttf,BoldFont=cmuntb.ttf,ItalicFont=cmunit.ttf,BoldItalicFont=cmuntx.ttf]{cmunrm.ttf} and {\ttfamily \setmainfont[Path=/usr/share/fonts/truetype/cmu/,UprightFont=cmunrm.ttf,BoldFont=cmunbx.ttf,ItalicFont=cmunti.ttf,BoldItalicFont=cmunbi.ttf]{cmuntt.ttf}\setmonofont[Path=/usr/share/fonts/truetype/cmu/,UprightFont=cmuntt.ttf,BoldFont=cmuntb.ttf,ItalicFont=cmunit.ttf,BoldItalicFont=cmuntx.ttf]{cmuntt.ttf}\ttfamily \textbackslash{}varnothing}\\ \hline 
\end{longtable}
}\setmainfont[Path=/usr/share/fonts/truetype/cmu/,UprightFont=cmunrm.ttf,BoldFont=cmunbx.ttf,ItalicFont=cmunti.ttf,BoldItalicFont=cmunbi.ttf]{cmunrm.ttf}\setmonofont[Path=/usr/share/fonts/truetype/cmu/,UprightFont=cmuntt.ttf,BoldFont=cmuntb.ttf,ItalicFont=cmunit.ttf,BoldItalicFont=cmuntx.ttf]{cmunrm.ttf}
{\scriptsize{}
{\scalefont{0.75315}\begin{longtable}{|>{\RaggedRight}p{0.08002\linewidth}|>{\RaggedRight}p{0.06631\linewidth}|>{\RaggedRight}p{0.02425\linewidth}|>{\RaggedRight}p{0.08002\linewidth}|>{\RaggedRight}p{0.06631\linewidth}|>{\RaggedRight}p{0.02425\linewidth}|>{\RaggedRight}p{0.08002\linewidth}|>{\RaggedRight}p{0.09397\linewidth}|>{\RaggedRight}p{0.02425\linewidth}|>{\RaggedRight}p{0.08002\linewidth}|>{\RaggedRight}p{0.06631\linewidth}|} \hline 
\multicolumn{11}{|>{\RaggedRight}p{0.97143\linewidth}|}{{\bfseries \hspace*{0pt}\ignorespaces{}\hspace*{0pt} Delimiters}}\\ \hline {\bfseries \hspace*{0pt}\ignorespaces{}\hspace*{0pt} Symbol }&{\bfseries \hspace*{0pt}\ignorespaces{}\hspace*{0pt} Script}&\multirow{5}{\linewidth}{\hspace*{0pt}\ignorespaces{}\hspace*{0pt} {\mbox{$~$}}}&{\bfseries \hspace*{0pt}\ignorespaces{}\hspace*{0pt} Symbol }&{\bfseries \hspace*{0pt}\ignorespaces{}\hspace*{0pt} Script}&\multirow{5}{\linewidth}{\hspace*{0pt}\ignorespaces{}\hspace*{0pt} {\mbox{$~$}}}&{\bfseries \hspace*{0pt}\ignorespaces{}\hspace*{0pt} Symbol }&{\bfseries \hspace*{0pt}\ignorespaces{}\hspace*{0pt} Script}&\multirow{5}{\linewidth}{\hspace*{0pt}\ignorespaces{}\hspace*{0pt} {\mbox{$~$}}}&{\bfseries \hspace*{0pt}\ignorespaces{}\hspace*{0pt} Symbol }&{\bfseries \hspace*{0pt}\ignorespaces{}\hspace*{0pt} Script}\\ \cline{1-1}\cline{2-2}\cline{4-4}\cline{5-5}\cline{7-7}\cline{8-8}\cline{10-10}\cline{11-11} \hspace*{0pt}\ignorespaces{}\hspace*{0pt} {$|\,$} &\hspace*{0pt}\ignorespaces{}\hspace*{0pt} {\ttfamily \setmainfont[Path=/usr/share/fonts/truetype/cmu/,UprightFont=cmunrm.ttf,BoldFont=cmunbx.ttf,ItalicFont=cmunti.ttf,BoldItalicFont=cmunbi.ttf]{cmuntt.ttf}\setmonofont[Path=/usr/share/fonts/truetype/cmu/,UprightFont=cmuntt.ttf,BoldFont=cmuntb.ttf,ItalicFont=cmunit.ttf,BoldItalicFont=cmuntx.ttf]{cmuntt.ttf}\ttfamily |}{$\text{ }$}\setmainfont[Path=/usr/share/fonts/truetype/cmu/,UprightFont=cmunrm.ttf,BoldFont=cmunbx.ttf,ItalicFont=cmunti.ttf,BoldItalicFont=cmunbi.ttf]{cmunrm.ttf}\setmonofont[Path=/usr/share/fonts/truetype/cmu/,UprightFont=cmuntt.ttf,BoldFont=cmuntb.ttf,ItalicFont=cmunit.ttf,BoldItalicFont=cmuntx.ttf]{cmunrm.ttf} or {\ttfamily \setmainfont[Path=/usr/share/fonts/truetype/cmu/,UprightFont=cmunrm.ttf,BoldFont=cmunbx.ttf,ItalicFont=cmunti.ttf,BoldItalicFont=cmunbi.ttf]{cmuntt.ttf}\setmonofont[Path=/usr/share/fonts/truetype/cmu/,UprightFont=cmuntt.ttf,BoldFont=cmuntb.ttf,ItalicFont=cmunit.ttf,BoldItalicFont=cmuntx.ttf]{cmuntt.ttf}\ttfamily \textbackslash{}mid}{$\text{ }$}\setmainfont[Path=/usr/share/fonts/truetype/cmu/,UprightFont=cmunrm.ttf,BoldFont=cmunbx.ttf,ItalicFont=cmunti.ttf,BoldItalicFont=cmunbi.ttf]{cmunrm.ttf}\setmonofont[Path=/usr/share/fonts/truetype/cmu/,UprightFont=cmuntt.ttf,BoldFont=cmuntb.ttf,ItalicFont=cmunit.ttf,BoldItalicFont=cmuntx.ttf]{cmunrm.ttf} (difference in spacing)&\multicolumn{1}{|c|}{}&\hspace*{0pt}\ignorespaces{}\hspace*{0pt} {$\Vert\,$} &\hspace*{0pt}\ignorespaces{}\hspace*{0pt} {\ttfamily \setmainfont[Path=/usr/share/fonts/truetype/cmu/,UprightFont=cmunrm.ttf,BoldFont=cmunbx.ttf,ItalicFont=cmunti.ttf,BoldItalicFont=cmunbi.ttf]{cmuntt.ttf}\setmonofont[Path=/usr/share/fonts/truetype/cmu/,UprightFont=cmuntt.ttf,BoldFont=cmuntb.ttf,ItalicFont=cmunit.ttf,BoldItalicFont=cmuntx.ttf]{cmuntt.ttf}\ttfamily \textbackslash{}|}&\multicolumn{1}{|c|}{}&\hspace*{0pt}\ignorespaces{}\hspace*{0pt}{$\text{ }$}\setmainfont[Path=/usr/share/fonts/truetype/cmu/,UprightFont=cmunrm.ttf,BoldFont=cmunbx.ttf,ItalicFont=cmunti.ttf,BoldItalicFont=cmunbi.ttf]{cmunrm.ttf}\setmonofont[Path=/usr/share/fonts/truetype/cmu/,UprightFont=cmuntt.ttf,BoldFont=cmuntb.ttf,ItalicFont=cmunit.ttf,BoldItalicFont=cmuntx.ttf]{cmunrm.ttf} {$/\,$} &\hspace*{0pt}\ignorespaces{}\hspace*{0pt} {\ttfamily \setmainfont[Path=/usr/share/fonts/truetype/cmu/,UprightFont=cmunrm.ttf,BoldFont=cmunbx.ttf,ItalicFont=cmunti.ttf,BoldItalicFont=cmunbi.ttf]{cmuntt.ttf}\setmonofont[Path=/usr/share/fonts/truetype/cmu/,UprightFont=cmuntt.ttf,BoldFont=cmuntb.ttf,ItalicFont=cmunit.ttf,BoldItalicFont=cmuntx.ttf]{cmuntt.ttf}\ttfamily /}&\multicolumn{1}{|c|}{}&\hspace*{0pt}\ignorespaces{}\hspace*{0pt}{$\text{ }$}\setmainfont[Path=/usr/share/fonts/truetype/cmu/,UprightFont=cmunrm.ttf,BoldFont=cmunbx.ttf,ItalicFont=cmunti.ttf,BoldItalicFont=cmunbi.ttf]{cmunrm.ttf}\setmonofont[Path=/usr/share/fonts/truetype/cmu/,UprightFont=cmuntt.ttf,BoldFont=cmuntb.ttf,ItalicFont=cmunit.ttf,BoldItalicFont=cmuntx.ttf]{cmunrm.ttf} {$\backslash\,$} &\hspace*{0pt}\ignorespaces{}\hspace*{0pt} {\ttfamily \setmainfont[Path=/usr/share/fonts/truetype/cmu/,UprightFont=cmunrm.ttf,BoldFont=cmunbx.ttf,ItalicFont=cmunti.ttf,BoldItalicFont=cmunbi.ttf]{cmuntt.ttf}\setmonofont[Path=/usr/share/fonts/truetype/cmu/,UprightFont=cmuntt.ttf,BoldFont=cmuntb.ttf,ItalicFont=cmunit.ttf,BoldItalicFont=cmuntx.ttf]{cmuntt.ttf}\ttfamily \textbackslash{}backslash}\\ \cline{1-1}\cline{2-2}\cline{4-4}\cline{5-5}\cline{7-7}\cline{8-8}\cline{10-10}\cline{11-11} \hspace*{0pt}\ignorespaces{}\hspace*{0pt}{$\text{ }$}\setmainfont[Path=/usr/share/fonts/truetype/cmu/,UprightFont=cmunrm.ttf,BoldFont=cmunbx.ttf,ItalicFont=cmunti.ttf,BoldItalicFont=cmunbi.ttf]{cmunrm.ttf}\setmonofont[Path=/usr/share/fonts/truetype/cmu/,UprightFont=cmuntt.ttf,BoldFont=cmuntb.ttf,ItalicFont=cmunit.ttf,BoldItalicFont=cmuntx.ttf]{cmunrm.ttf} {$\{\,$} &\hspace*{0pt}\ignorespaces{}\hspace*{0pt} {\ttfamily \setmainfont[Path=/usr/share/fonts/truetype/cmu/,UprightFont=cmunrm.ttf,BoldFont=cmunbx.ttf,ItalicFont=cmunti.ttf,BoldItalicFont=cmunbi.ttf]{cmuntt.ttf}\setmonofont[Path=/usr/share/fonts/truetype/cmu/,UprightFont=cmuntt.ttf,BoldFont=cmuntb.ttf,ItalicFont=cmunit.ttf,BoldItalicFont=cmuntx.ttf]{cmuntt.ttf}\ttfamily \textbackslash{}\{}&\multicolumn{1}{|c|}{}&\hspace*{0pt}\ignorespaces{}\hspace*{0pt}{$\text{ }$}\setmainfont[Path=/usr/share/fonts/truetype/cmu/,UprightFont=cmunrm.ttf,BoldFont=cmunbx.ttf,ItalicFont=cmunti.ttf,BoldItalicFont=cmunbi.ttf]{cmunrm.ttf}\setmonofont[Path=/usr/share/fonts/truetype/cmu/,UprightFont=cmuntt.ttf,BoldFont=cmuntb.ttf,ItalicFont=cmunit.ttf,BoldItalicFont=cmuntx.ttf]{cmunrm.ttf} {$\}\,$} &\hspace*{0pt}\ignorespaces{}\hspace*{0pt} {\ttfamily \setmainfont[Path=/usr/share/fonts/truetype/cmu/,UprightFont=cmunrm.ttf,BoldFont=cmunbx.ttf,ItalicFont=cmunti.ttf,BoldItalicFont=cmunbi.ttf]{cmuntt.ttf}\setmonofont[Path=/usr/share/fonts/truetype/cmu/,UprightFont=cmuntt.ttf,BoldFont=cmuntb.ttf,ItalicFont=cmunit.ttf,BoldItalicFont=cmuntx.ttf]{cmuntt.ttf}\ttfamily \textbackslash{}\}}&\multicolumn{1}{|c|}{}&\hspace*{0pt}\ignorespaces{}\hspace*{0pt}{$\text{ }$}\setmainfont[Path=/usr/share/fonts/truetype/cmu/,UprightFont=cmunrm.ttf,BoldFont=cmunbx.ttf,ItalicFont=cmunti.ttf,BoldItalicFont=cmunbi.ttf]{cmunrm.ttf}\setmonofont[Path=/usr/share/fonts/truetype/cmu/,UprightFont=cmuntt.ttf,BoldFont=cmuntb.ttf,ItalicFont=cmunit.ttf,BoldItalicFont=cmuntx.ttf]{cmunrm.ttf} {$\langle\,$} &\hspace*{0pt}\ignorespaces{}\hspace*{0pt} {\ttfamily \setmainfont[Path=/usr/share/fonts/truetype/cmu/,UprightFont=cmunrm.ttf,BoldFont=cmunbx.ttf,ItalicFont=cmunti.ttf,BoldItalicFont=cmunbi.ttf]{cmuntt.ttf}\setmonofont[Path=/usr/share/fonts/truetype/cmu/,UprightFont=cmuntt.ttf,BoldFont=cmuntb.ttf,ItalicFont=cmunit.ttf,BoldItalicFont=cmuntx.ttf]{cmuntt.ttf}\ttfamily \textbackslash{}langle}&\multicolumn{1}{|c|}{}&\hspace*{0pt}\ignorespaces{}\hspace*{0pt}{$\text{ }$}\setmainfont[Path=/usr/share/fonts/truetype/cmu/,UprightFont=cmunrm.ttf,BoldFont=cmunbx.ttf,ItalicFont=cmunti.ttf,BoldItalicFont=cmunbi.ttf]{cmunrm.ttf}\setmonofont[Path=/usr/share/fonts/truetype/cmu/,UprightFont=cmuntt.ttf,BoldFont=cmuntb.ttf,ItalicFont=cmunit.ttf,BoldItalicFont=cmuntx.ttf]{cmunrm.ttf} {$\rangle\,$} &\hspace*{0pt}\ignorespaces{}\hspace*{0pt} {\ttfamily \setmainfont[Path=/usr/share/fonts/truetype/cmu/,UprightFont=cmunrm.ttf,BoldFont=cmunbx.ttf,ItalicFont=cmunti.ttf,BoldItalicFont=cmunbi.ttf]{cmuntt.ttf}\setmonofont[Path=/usr/share/fonts/truetype/cmu/,UprightFont=cmuntt.ttf,BoldFont=cmuntb.ttf,ItalicFont=cmunit.ttf,BoldItalicFont=cmuntx.ttf]{cmuntt.ttf}\ttfamily \textbackslash{}rangle}\\ \cline{1-1}\cline{2-2}\cline{4-4}\cline{5-5}\cline{7-7}\cline{8-8}\cline{10-10}\cline{11-11} \hspace*{0pt}\ignorespaces{}\hspace*{0pt}{$\text{ }$}\setmainfont[Path=/usr/share/fonts/truetype/cmu/,UprightFont=cmunrm.ttf,BoldFont=cmunbx.ttf,ItalicFont=cmunti.ttf,BoldItalicFont=cmunbi.ttf]{cmunrm.ttf}\setmonofont[Path=/usr/share/fonts/truetype/cmu/,UprightFont=cmuntt.ttf,BoldFont=cmuntb.ttf,ItalicFont=cmunit.ttf,BoldItalicFont=cmuntx.ttf]{cmunrm.ttf} {$\uparrow\,$} &\hspace*{0pt}\ignorespaces{}\hspace*{0pt} {\ttfamily \setmainfont[Path=/usr/share/fonts/truetype/cmu/,UprightFont=cmunrm.ttf,BoldFont=cmunbx.ttf,ItalicFont=cmunti.ttf,BoldItalicFont=cmunbi.ttf]{cmuntt.ttf}\setmonofont[Path=/usr/share/fonts/truetype/cmu/,UprightFont=cmuntt.ttf,BoldFont=cmuntb.ttf,ItalicFont=cmunit.ttf,BoldItalicFont=cmuntx.ttf]{cmuntt.ttf}\ttfamily \textbackslash{}uparrow}&\multicolumn{1}{|c|}{}&\hspace*{0pt}\ignorespaces{}\hspace*{0pt}{$\text{ }$}\setmainfont[Path=/usr/share/fonts/truetype/cmu/,UprightFont=cmunrm.ttf,BoldFont=cmunbx.ttf,ItalicFont=cmunti.ttf,BoldItalicFont=cmunbi.ttf]{cmunrm.ttf}\setmonofont[Path=/usr/share/fonts/truetype/cmu/,UprightFont=cmuntt.ttf,BoldFont=cmuntb.ttf,ItalicFont=cmunit.ttf,BoldItalicFont=cmuntx.ttf]{cmunrm.ttf} {$\Uparrow\,$} &\hspace*{0pt}\ignorespaces{}\hspace*{0pt} {\ttfamily \setmainfont[Path=/usr/share/fonts/truetype/cmu/,UprightFont=cmunrm.ttf,BoldFont=cmunbx.ttf,ItalicFont=cmunti.ttf,BoldItalicFont=cmunbi.ttf]{cmuntt.ttf}\setmonofont[Path=/usr/share/fonts/truetype/cmu/,UprightFont=cmuntt.ttf,BoldFont=cmuntb.ttf,ItalicFont=cmunit.ttf,BoldItalicFont=cmuntx.ttf]{cmuntt.ttf}\ttfamily \textbackslash{}Uparrow}&\multicolumn{1}{|c|}{}&\hspace*{0pt}\ignorespaces{}\hspace*{0pt}{$\text{ }$}\setmainfont[Path=/usr/share/fonts/truetype/cmu/,UprightFont=cmunrm.ttf,BoldFont=cmunbx.ttf,ItalicFont=cmunti.ttf,BoldItalicFont=cmunbi.ttf]{cmunrm.ttf}\setmonofont[Path=/usr/share/fonts/truetype/cmu/,UprightFont=cmuntt.ttf,BoldFont=cmuntb.ttf,ItalicFont=cmunit.ttf,BoldItalicFont=cmuntx.ttf]{cmunrm.ttf} {$\lceil\,$} &\hspace*{0pt}\ignorespaces{}\hspace*{0pt} {\ttfamily \setmainfont[Path=/usr/share/fonts/truetype/cmu/,UprightFont=cmunrm.ttf,BoldFont=cmunbx.ttf,ItalicFont=cmunti.ttf,BoldItalicFont=cmunbi.ttf]{cmuntt.ttf}\setmonofont[Path=/usr/share/fonts/truetype/cmu/,UprightFont=cmuntt.ttf,BoldFont=cmuntb.ttf,ItalicFont=cmunit.ttf,BoldItalicFont=cmuntx.ttf]{cmuntt.ttf}\ttfamily \textbackslash{}lceil}&\multicolumn{1}{|c|}{}&\hspace*{0pt}\ignorespaces{}\hspace*{0pt}{$\text{ }$}\setmainfont[Path=/usr/share/fonts/truetype/cmu/,UprightFont=cmunrm.ttf,BoldFont=cmunbx.ttf,ItalicFont=cmunti.ttf,BoldItalicFont=cmunbi.ttf]{cmunrm.ttf}\setmonofont[Path=/usr/share/fonts/truetype/cmu/,UprightFont=cmuntt.ttf,BoldFont=cmuntb.ttf,ItalicFont=cmunit.ttf,BoldItalicFont=cmuntx.ttf]{cmunrm.ttf} {$\rceil\,$} &\hspace*{0pt}\ignorespaces{}\hspace*{0pt} {\ttfamily \setmainfont[Path=/usr/share/fonts/truetype/cmu/,UprightFont=cmunrm.ttf,BoldFont=cmunbx.ttf,ItalicFont=cmunti.ttf,BoldItalicFont=cmunbi.ttf]{cmuntt.ttf}\setmonofont[Path=/usr/share/fonts/truetype/cmu/,UprightFont=cmuntt.ttf,BoldFont=cmuntb.ttf,ItalicFont=cmunit.ttf,BoldItalicFont=cmuntx.ttf]{cmuntt.ttf}\ttfamily \textbackslash{}rceil}\\ \cline{1-1}\cline{2-2}\cline{4-4}\cline{5-5}\cline{7-7}\cline{8-8}\cline{10-10}\cline{11-11} \hspace*{0pt}\ignorespaces{}\hspace*{0pt}{$\text{ }$}\setmainfont[Path=/usr/share/fonts/truetype/cmu/,UprightFont=cmunrm.ttf,BoldFont=cmunbx.ttf,ItalicFont=cmunti.ttf,BoldItalicFont=cmunbi.ttf]{cmunrm.ttf}\setmonofont[Path=/usr/share/fonts/truetype/cmu/,UprightFont=cmuntt.ttf,BoldFont=cmuntb.ttf,ItalicFont=cmunit.ttf,BoldItalicFont=cmuntx.ttf]{cmunrm.ttf} {$\downarrow\,$} &\hspace*{0pt}\ignorespaces{}\hspace*{0pt} {\ttfamily \setmainfont[Path=/usr/share/fonts/truetype/cmu/,UprightFont=cmunrm.ttf,BoldFont=cmunbx.ttf,ItalicFont=cmunti.ttf,BoldItalicFont=cmunbi.ttf]{cmuntt.ttf}\setmonofont[Path=/usr/share/fonts/truetype/cmu/,UprightFont=cmuntt.ttf,BoldFont=cmuntb.ttf,ItalicFont=cmunit.ttf,BoldItalicFont=cmuntx.ttf]{cmuntt.ttf}\ttfamily \textbackslash{}downarrow}&\multicolumn{1}{|c|}{}&\hspace*{0pt}\ignorespaces{}\hspace*{0pt}{$\text{ }$}\setmainfont[Path=/usr/share/fonts/truetype/cmu/,UprightFont=cmunrm.ttf,BoldFont=cmunbx.ttf,ItalicFont=cmunti.ttf,BoldItalicFont=cmunbi.ttf]{cmunrm.ttf}\setmonofont[Path=/usr/share/fonts/truetype/cmu/,UprightFont=cmuntt.ttf,BoldFont=cmuntb.ttf,ItalicFont=cmunit.ttf,BoldItalicFont=cmuntx.ttf]{cmunrm.ttf} {$\Downarrow\,$} &\hspace*{0pt}\ignorespaces{}\hspace*{0pt} {\ttfamily \setmainfont[Path=/usr/share/fonts/truetype/cmu/,UprightFont=cmunrm.ttf,BoldFont=cmunbx.ttf,ItalicFont=cmunti.ttf,BoldItalicFont=cmunbi.ttf]{cmuntt.ttf}\setmonofont[Path=/usr/share/fonts/truetype/cmu/,UprightFont=cmuntt.ttf,BoldFont=cmuntb.ttf,ItalicFont=cmunit.ttf,BoldItalicFont=cmuntx.ttf]{cmuntt.ttf}\ttfamily \textbackslash{}Downarrow}&\multicolumn{1}{|c|}{}&\hspace*{0pt}\ignorespaces{}\hspace*{0pt}{$\text{ }$}\setmainfont[Path=/usr/share/fonts/truetype/cmu/,UprightFont=cmunrm.ttf,BoldFont=cmunbx.ttf,ItalicFont=cmunti.ttf,BoldItalicFont=cmunbi.ttf]{cmunrm.ttf}\setmonofont[Path=/usr/share/fonts/truetype/cmu/,UprightFont=cmuntt.ttf,BoldFont=cmuntb.ttf,ItalicFont=cmunit.ttf,BoldItalicFont=cmuntx.ttf]{cmunrm.ttf} {$\lfloor\,$} &\hspace*{0pt}\ignorespaces{}\hspace*{0pt} {\ttfamily \setmainfont[Path=/usr/share/fonts/truetype/cmu/,UprightFont=cmunrm.ttf,BoldFont=cmunbx.ttf,ItalicFont=cmunti.ttf,BoldItalicFont=cmunbi.ttf]{cmuntt.ttf}\setmonofont[Path=/usr/share/fonts/truetype/cmu/,UprightFont=cmuntt.ttf,BoldFont=cmuntb.ttf,ItalicFont=cmunit.ttf,BoldItalicFont=cmuntx.ttf]{cmuntt.ttf}\ttfamily \textbackslash{}lfloor}&\multicolumn{1}{|c|}{}&\hspace*{0pt}\ignorespaces{}\hspace*{0pt}{$\text{ }$}\setmainfont[Path=/usr/share/fonts/truetype/cmu/,UprightFont=cmunrm.ttf,BoldFont=cmunbx.ttf,ItalicFont=cmunti.ttf,BoldItalicFont=cmunbi.ttf]{cmunrm.ttf}\setmonofont[Path=/usr/share/fonts/truetype/cmu/,UprightFont=cmuntt.ttf,BoldFont=cmuntb.ttf,ItalicFont=cmunit.ttf,BoldItalicFont=cmuntx.ttf]{cmunrm.ttf} {$\rfloor\,$} &\hspace*{0pt}\ignorespaces{}\hspace*{0pt} {\ttfamily \setmainfont[Path=/usr/share/fonts/truetype/cmu/,UprightFont=cmunrm.ttf,BoldFont=cmunbx.ttf,ItalicFont=cmunti.ttf,BoldItalicFont=cmunbi.ttf]{cmuntt.ttf}\setmonofont[Path=/usr/share/fonts/truetype/cmu/,UprightFont=cmuntt.ttf,BoldFont=cmuntb.ttf,ItalicFont=cmunit.ttf,BoldItalicFont=cmuntx.ttf]{cmuntt.ttf}\ttfamily \textbackslash{}rfloor}\\ \hline 
\end{longtable}
}}\setmainfont[Path=/usr/share/fonts/truetype/cmu/,UprightFont=cmunrm.ttf,BoldFont=cmunbx.ttf,ItalicFont=cmunti.ttf,BoldItalicFont=cmunbi.ttf]{cmunrm.ttf}\setmonofont[Path=/usr/share/fonts/truetype/cmu/,UprightFont=cmuntt.ttf,BoldFont=cmuntb.ttf,ItalicFont=cmunit.ttf,BoldItalicFont=cmuntx.ttf]{cmunrm.ttf}

Note: To use the Greek Letters in LaTeX that have the same appearance as their Roman equivalent, just use the Roman form: e.g., A instead of Alpha, B instead of Beta, etc.{\scriptsize{}
\begin{longtable}{|>{\RaggedRight}p{0.14415\linewidth}|>{\RaggedRight}p{0.26625\linewidth}|>{\RaggedRight}p{0.03133\linewidth}|>{\RaggedRight}p{0.14917\linewidth}|>{\RaggedRight}p{0.26625\linewidth}|} \hline 
\multicolumn{5}{|>{\RaggedRight}p{0.97143\linewidth}|}{{\bfseries \hspace*{0pt}\ignorespaces{}\hspace*{0pt} Greek Letters}}\\ \hline {\bfseries \hspace*{0pt}\ignorespaces{}\hspace*{0pt} Symbol }&{\bfseries \hspace*{0pt}\ignorespaces{}\hspace*{0pt} Script}&\multirow{13}{\linewidth}{\hspace*{0pt}\ignorespaces{}\hspace*{0pt} {\mbox{$~$}}}&{\bfseries \hspace*{0pt}\ignorespaces{}\hspace*{0pt} Symbol }&{\bfseries \hspace*{0pt}\ignorespaces{}\hspace*{0pt} Script}\\ \cline{1-1}\cline{2-2}\cline{4-4}\cline{5-5} \hspace*{0pt}\ignorespaces{}\hspace*{0pt} {$\Alpha\,$} and {$\alpha\,$}&\hspace*{0pt}\ignorespaces{}\hspace*{0pt} {\ttfamily \setmainfont[Path=/usr/share/fonts/truetype/cmu/,UprightFont=cmunrm.ttf,BoldFont=cmunbx.ttf,ItalicFont=cmunti.ttf,BoldItalicFont=cmunbi.ttf]{cmuntt.ttf}\setmonofont[Path=/usr/share/fonts/truetype/cmu/,UprightFont=cmuntt.ttf,BoldFont=cmuntb.ttf,ItalicFont=cmunit.ttf,BoldItalicFont=cmuntx.ttf]{cmuntt.ttf}\ttfamily A}{$\text{ }$}\setmainfont[Path=/usr/share/fonts/truetype/cmu/,UprightFont=cmunrm.ttf,BoldFont=cmunbx.ttf,ItalicFont=cmunti.ttf,BoldItalicFont=cmunbi.ttf]{cmunrm.ttf}\setmonofont[Path=/usr/share/fonts/truetype/cmu/,UprightFont=cmuntt.ttf,BoldFont=cmuntb.ttf,ItalicFont=cmunit.ttf,BoldItalicFont=cmuntx.ttf]{cmunrm.ttf} and {\ttfamily \setmainfont[Path=/usr/share/fonts/truetype/cmu/,UprightFont=cmunrm.ttf,BoldFont=cmunbx.ttf,ItalicFont=cmunti.ttf,BoldItalicFont=cmunbi.ttf]{cmuntt.ttf}\setmonofont[Path=/usr/share/fonts/truetype/cmu/,UprightFont=cmuntt.ttf,BoldFont=cmuntb.ttf,ItalicFont=cmunit.ttf,BoldItalicFont=cmuntx.ttf]{cmuntt.ttf}\ttfamily \textbackslash{}alpha}&\multicolumn{1}{|c|}{}&\hspace*{0pt}\ignorespaces{}\hspace*{0pt}{$\text{ }$}\setmainfont[Path=/usr/share/fonts/truetype/cmu/,UprightFont=cmunrm.ttf,BoldFont=cmunbx.ttf,ItalicFont=cmunti.ttf,BoldItalicFont=cmunbi.ttf]{cmunrm.ttf}\setmonofont[Path=/usr/share/fonts/truetype/cmu/,UprightFont=cmuntt.ttf,BoldFont=cmuntb.ttf,ItalicFont=cmunit.ttf,BoldItalicFont=cmuntx.ttf]{cmunrm.ttf} {$\Nu\,$} and {$\nu\,$}&\hspace*{0pt}\ignorespaces{}\hspace*{0pt} {\ttfamily \setmainfont[Path=/usr/share/fonts/truetype/cmu/,UprightFont=cmunrm.ttf,BoldFont=cmunbx.ttf,ItalicFont=cmunti.ttf,BoldItalicFont=cmunbi.ttf]{cmuntt.ttf}\setmonofont[Path=/usr/share/fonts/truetype/cmu/,UprightFont=cmuntt.ttf,BoldFont=cmuntb.ttf,ItalicFont=cmunit.ttf,BoldItalicFont=cmuntx.ttf]{cmuntt.ttf}\ttfamily N}{$\text{ }$}\setmainfont[Path=/usr/share/fonts/truetype/cmu/,UprightFont=cmunrm.ttf,BoldFont=cmunbx.ttf,ItalicFont=cmunti.ttf,BoldItalicFont=cmunbi.ttf]{cmunrm.ttf}\setmonofont[Path=/usr/share/fonts/truetype/cmu/,UprightFont=cmuntt.ttf,BoldFont=cmuntb.ttf,ItalicFont=cmunit.ttf,BoldItalicFont=cmuntx.ttf]{cmunrm.ttf} and {\ttfamily \setmainfont[Path=/usr/share/fonts/truetype/cmu/,UprightFont=cmunrm.ttf,BoldFont=cmunbx.ttf,ItalicFont=cmunti.ttf,BoldItalicFont=cmunbi.ttf]{cmuntt.ttf}\setmonofont[Path=/usr/share/fonts/truetype/cmu/,UprightFont=cmuntt.ttf,BoldFont=cmuntb.ttf,ItalicFont=cmunit.ttf,BoldItalicFont=cmuntx.ttf]{cmuntt.ttf}\ttfamily \textbackslash{}nu}\\ \cline{1-1}\cline{2-2}\cline{4-4}\cline{5-5} \hspace*{0pt}\ignorespaces{}\hspace*{0pt}{$\text{ }$}\setmainfont[Path=/usr/share/fonts/truetype/cmu/,UprightFont=cmunrm.ttf,BoldFont=cmunbx.ttf,ItalicFont=cmunti.ttf,BoldItalicFont=cmunbi.ttf]{cmunrm.ttf}\setmonofont[Path=/usr/share/fonts/truetype/cmu/,UprightFont=cmuntt.ttf,BoldFont=cmuntb.ttf,ItalicFont=cmunit.ttf,BoldItalicFont=cmuntx.ttf]{cmunrm.ttf} {$\Beta\,$} and {$\beta\,$}&\hspace*{0pt}\ignorespaces{}\hspace*{0pt} {\ttfamily \setmainfont[Path=/usr/share/fonts/truetype/cmu/,UprightFont=cmunrm.ttf,BoldFont=cmunbx.ttf,ItalicFont=cmunti.ttf,BoldItalicFont=cmunbi.ttf]{cmuntt.ttf}\setmonofont[Path=/usr/share/fonts/truetype/cmu/,UprightFont=cmuntt.ttf,BoldFont=cmuntb.ttf,ItalicFont=cmunit.ttf,BoldItalicFont=cmuntx.ttf]{cmuntt.ttf}\ttfamily B}{$\text{ }$}\setmainfont[Path=/usr/share/fonts/truetype/cmu/,UprightFont=cmunrm.ttf,BoldFont=cmunbx.ttf,ItalicFont=cmunti.ttf,BoldItalicFont=cmunbi.ttf]{cmunrm.ttf}\setmonofont[Path=/usr/share/fonts/truetype/cmu/,UprightFont=cmuntt.ttf,BoldFont=cmuntb.ttf,ItalicFont=cmunit.ttf,BoldItalicFont=cmuntx.ttf]{cmunrm.ttf} and {\ttfamily \setmainfont[Path=/usr/share/fonts/truetype/cmu/,UprightFont=cmunrm.ttf,BoldFont=cmunbx.ttf,ItalicFont=cmunti.ttf,BoldItalicFont=cmunbi.ttf]{cmuntt.ttf}\setmonofont[Path=/usr/share/fonts/truetype/cmu/,UprightFont=cmuntt.ttf,BoldFont=cmuntb.ttf,ItalicFont=cmunit.ttf,BoldItalicFont=cmuntx.ttf]{cmuntt.ttf}\ttfamily \textbackslash{}beta}&\multicolumn{1}{|c|}{}&\hspace*{0pt}\ignorespaces{}\hspace*{0pt}{$\text{ }$}\setmainfont[Path=/usr/share/fonts/truetype/cmu/,UprightFont=cmunrm.ttf,BoldFont=cmunbx.ttf,ItalicFont=cmunti.ttf,BoldItalicFont=cmunbi.ttf]{cmunrm.ttf}\setmonofont[Path=/usr/share/fonts/truetype/cmu/,UprightFont=cmuntt.ttf,BoldFont=cmuntb.ttf,ItalicFont=cmunit.ttf,BoldItalicFont=cmuntx.ttf]{cmunrm.ttf} {$\Xi\,$} and {$\xi\,$}&\hspace*{0pt}\ignorespaces{}\hspace*{0pt} {\ttfamily \setmainfont[Path=/usr/share/fonts/truetype/cmu/,UprightFont=cmunrm.ttf,BoldFont=cmunbx.ttf,ItalicFont=cmunti.ttf,BoldItalicFont=cmunbi.ttf]{cmuntt.ttf}\setmonofont[Path=/usr/share/fonts/truetype/cmu/,UprightFont=cmuntt.ttf,BoldFont=cmuntb.ttf,ItalicFont=cmunit.ttf,BoldItalicFont=cmuntx.ttf]{cmuntt.ttf}\ttfamily \textbackslash{}Xi}{$\text{ }$}\setmainfont[Path=/usr/share/fonts/truetype/cmu/,UprightFont=cmunrm.ttf,BoldFont=cmunbx.ttf,ItalicFont=cmunti.ttf,BoldItalicFont=cmunbi.ttf]{cmunrm.ttf}\setmonofont[Path=/usr/share/fonts/truetype/cmu/,UprightFont=cmuntt.ttf,BoldFont=cmuntb.ttf,ItalicFont=cmunit.ttf,BoldItalicFont=cmuntx.ttf]{cmunrm.ttf} and {\ttfamily \setmainfont[Path=/usr/share/fonts/truetype/cmu/,UprightFont=cmunrm.ttf,BoldFont=cmunbx.ttf,ItalicFont=cmunti.ttf,BoldItalicFont=cmunbi.ttf]{cmuntt.ttf}\setmonofont[Path=/usr/share/fonts/truetype/cmu/,UprightFont=cmuntt.ttf,BoldFont=cmuntb.ttf,ItalicFont=cmunit.ttf,BoldItalicFont=cmuntx.ttf]{cmuntt.ttf}\ttfamily \textbackslash{}xi}\\ \cline{1-1}\cline{2-2}\cline{4-4}\cline{5-5} \hspace*{0pt}\ignorespaces{}\hspace*{0pt}{$\text{ }$}\setmainfont[Path=/usr/share/fonts/truetype/cmu/,UprightFont=cmunrm.ttf,BoldFont=cmunbx.ttf,ItalicFont=cmunti.ttf,BoldItalicFont=cmunbi.ttf]{cmunrm.ttf}\setmonofont[Path=/usr/share/fonts/truetype/cmu/,UprightFont=cmuntt.ttf,BoldFont=cmuntb.ttf,ItalicFont=cmunit.ttf,BoldItalicFont=cmuntx.ttf]{cmunrm.ttf} {$\Gamma\,$} and {$\gamma\,$}&\hspace*{0pt}\ignorespaces{}\hspace*{0pt} {\ttfamily \setmainfont[Path=/usr/share/fonts/truetype/cmu/,UprightFont=cmunrm.ttf,BoldFont=cmunbx.ttf,ItalicFont=cmunti.ttf,BoldItalicFont=cmunbi.ttf]{cmuntt.ttf}\setmonofont[Path=/usr/share/fonts/truetype/cmu/,UprightFont=cmuntt.ttf,BoldFont=cmuntb.ttf,ItalicFont=cmunit.ttf,BoldItalicFont=cmuntx.ttf]{cmuntt.ttf}\ttfamily \textbackslash{}Gamma}{$\text{ }$}\setmainfont[Path=/usr/share/fonts/truetype/cmu/,UprightFont=cmunrm.ttf,BoldFont=cmunbx.ttf,ItalicFont=cmunti.ttf,BoldItalicFont=cmunbi.ttf]{cmunrm.ttf}\setmonofont[Path=/usr/share/fonts/truetype/cmu/,UprightFont=cmuntt.ttf,BoldFont=cmuntb.ttf,ItalicFont=cmunit.ttf,BoldItalicFont=cmuntx.ttf]{cmunrm.ttf} and {\ttfamily \setmainfont[Path=/usr/share/fonts/truetype/cmu/,UprightFont=cmunrm.ttf,BoldFont=cmunbx.ttf,ItalicFont=cmunti.ttf,BoldItalicFont=cmunbi.ttf]{cmuntt.ttf}\setmonofont[Path=/usr/share/fonts/truetype/cmu/,UprightFont=cmuntt.ttf,BoldFont=cmuntb.ttf,ItalicFont=cmunit.ttf,BoldItalicFont=cmuntx.ttf]{cmuntt.ttf}\ttfamily \textbackslash{}gamma}&\multicolumn{1}{|c|}{}&\hspace*{0pt}\ignorespaces{}\hspace*{0pt}{$\text{ }$}\setmainfont[Path=/usr/share/fonts/truetype/cmu/,UprightFont=cmunrm.ttf,BoldFont=cmunbx.ttf,ItalicFont=cmunti.ttf,BoldItalicFont=cmunbi.ttf]{cmunrm.ttf}\setmonofont[Path=/usr/share/fonts/truetype/cmu/,UprightFont=cmuntt.ttf,BoldFont=cmuntb.ttf,ItalicFont=cmunit.ttf,BoldItalicFont=cmuntx.ttf]{cmunrm.ttf} {$\Omicron\,$} and {$\omicron\,$}&\hspace*{0pt}\ignorespaces{}\hspace*{0pt} {\ttfamily \setmainfont[Path=/usr/share/fonts/truetype/cmu/,UprightFont=cmunrm.ttf,BoldFont=cmunbx.ttf,ItalicFont=cmunti.ttf,BoldItalicFont=cmunbi.ttf]{cmuntt.ttf}\setmonofont[Path=/usr/share/fonts/truetype/cmu/,UprightFont=cmuntt.ttf,BoldFont=cmuntb.ttf,ItalicFont=cmunit.ttf,BoldItalicFont=cmuntx.ttf]{cmuntt.ttf}\ttfamily O}{$\text{ }$}\setmainfont[Path=/usr/share/fonts/truetype/cmu/,UprightFont=cmunrm.ttf,BoldFont=cmunbx.ttf,ItalicFont=cmunti.ttf,BoldItalicFont=cmunbi.ttf]{cmunrm.ttf}\setmonofont[Path=/usr/share/fonts/truetype/cmu/,UprightFont=cmuntt.ttf,BoldFont=cmuntb.ttf,ItalicFont=cmunit.ttf,BoldItalicFont=cmuntx.ttf]{cmunrm.ttf} and {\ttfamily \setmainfont[Path=/usr/share/fonts/truetype/cmu/,UprightFont=cmunrm.ttf,BoldFont=cmunbx.ttf,ItalicFont=cmunti.ttf,BoldItalicFont=cmunbi.ttf]{cmuntt.ttf}\setmonofont[Path=/usr/share/fonts/truetype/cmu/,UprightFont=cmuntt.ttf,BoldFont=cmuntb.ttf,ItalicFont=cmunit.ttf,BoldItalicFont=cmuntx.ttf]{cmuntt.ttf}\ttfamily o}\\ \cline{1-1}\cline{2-2}\cline{4-4}\cline{5-5} \hspace*{0pt}\ignorespaces{}\hspace*{0pt}{$\text{ }$}\setmainfont[Path=/usr/share/fonts/truetype/cmu/,UprightFont=cmunrm.ttf,BoldFont=cmunbx.ttf,ItalicFont=cmunti.ttf,BoldItalicFont=cmunbi.ttf]{cmunrm.ttf}\setmonofont[Path=/usr/share/fonts/truetype/cmu/,UprightFont=cmuntt.ttf,BoldFont=cmuntb.ttf,ItalicFont=cmunit.ttf,BoldItalicFont=cmuntx.ttf]{cmunrm.ttf} {$\Delta\,$} and {$\delta\,$}&\hspace*{0pt}\ignorespaces{}\hspace*{0pt} {\ttfamily \setmainfont[Path=/usr/share/fonts/truetype/cmu/,UprightFont=cmunrm.ttf,BoldFont=cmunbx.ttf,ItalicFont=cmunti.ttf,BoldItalicFont=cmunbi.ttf]{cmuntt.ttf}\setmonofont[Path=/usr/share/fonts/truetype/cmu/,UprightFont=cmuntt.ttf,BoldFont=cmuntb.ttf,ItalicFont=cmunit.ttf,BoldItalicFont=cmuntx.ttf]{cmuntt.ttf}\ttfamily \textbackslash{}Delta}{$\text{ }$}\setmainfont[Path=/usr/share/fonts/truetype/cmu/,UprightFont=cmunrm.ttf,BoldFont=cmunbx.ttf,ItalicFont=cmunti.ttf,BoldItalicFont=cmunbi.ttf]{cmunrm.ttf}\setmonofont[Path=/usr/share/fonts/truetype/cmu/,UprightFont=cmuntt.ttf,BoldFont=cmuntb.ttf,ItalicFont=cmunit.ttf,BoldItalicFont=cmuntx.ttf]{cmunrm.ttf} and {\ttfamily \setmainfont[Path=/usr/share/fonts/truetype/cmu/,UprightFont=cmunrm.ttf,BoldFont=cmunbx.ttf,ItalicFont=cmunti.ttf,BoldItalicFont=cmunbi.ttf]{cmuntt.ttf}\setmonofont[Path=/usr/share/fonts/truetype/cmu/,UprightFont=cmuntt.ttf,BoldFont=cmuntb.ttf,ItalicFont=cmunit.ttf,BoldItalicFont=cmuntx.ttf]{cmuntt.ttf}\ttfamily \textbackslash{}delta}&\multicolumn{1}{|c|}{}&\hspace*{0pt}\ignorespaces{}\hspace*{0pt}{$\text{ }$}\setmainfont[Path=/usr/share/fonts/truetype/cmu/,UprightFont=cmunrm.ttf,BoldFont=cmunbx.ttf,ItalicFont=cmunti.ttf,BoldItalicFont=cmunbi.ttf]{cmunrm.ttf}\setmonofont[Path=/usr/share/fonts/truetype/cmu/,UprightFont=cmuntt.ttf,BoldFont=cmuntb.ttf,ItalicFont=cmunit.ttf,BoldItalicFont=cmuntx.ttf]{cmunrm.ttf} {$\Pi\,$}, {$\pi\,$} and {$\varpi$}&\hspace*{0pt}\ignorespaces{}\hspace*{0pt} {\ttfamily \setmainfont[Path=/usr/share/fonts/truetype/cmu/,UprightFont=cmunrm.ttf,BoldFont=cmunbx.ttf,ItalicFont=cmunti.ttf,BoldItalicFont=cmunbi.ttf]{cmuntt.ttf}\setmonofont[Path=/usr/share/fonts/truetype/cmu/,UprightFont=cmuntt.ttf,BoldFont=cmuntb.ttf,ItalicFont=cmunit.ttf,BoldItalicFont=cmuntx.ttf]{cmuntt.ttf}\ttfamily \textbackslash{}Pi}\setmainfont[Path=/usr/share/fonts/truetype/cmu/,UprightFont=cmunrm.ttf,BoldFont=cmunbx.ttf,ItalicFont=cmunti.ttf,BoldItalicFont=cmunbi.ttf]{cmunrm.ttf}\setmonofont[Path=/usr/share/fonts/truetype/cmu/,UprightFont=cmuntt.ttf,BoldFont=cmuntb.ttf,ItalicFont=cmunit.ttf,BoldItalicFont=cmuntx.ttf]{cmunrm.ttf}, {\ttfamily \setmainfont[Path=/usr/share/fonts/truetype/cmu/,UprightFont=cmunrm.ttf,BoldFont=cmunbx.ttf,ItalicFont=cmunti.ttf,BoldItalicFont=cmunbi.ttf]{cmuntt.ttf}\setmonofont[Path=/usr/share/fonts/truetype/cmu/,UprightFont=cmuntt.ttf,BoldFont=cmuntb.ttf,ItalicFont=cmunit.ttf,BoldItalicFont=cmuntx.ttf]{cmuntt.ttf}\ttfamily \textbackslash{}pi}{$\text{ }$}\setmainfont[Path=/usr/share/fonts/truetype/cmu/,UprightFont=cmunrm.ttf,BoldFont=cmunbx.ttf,ItalicFont=cmunti.ttf,BoldItalicFont=cmunbi.ttf]{cmunrm.ttf}\setmonofont[Path=/usr/share/fonts/truetype/cmu/,UprightFont=cmuntt.ttf,BoldFont=cmuntb.ttf,ItalicFont=cmunit.ttf,BoldItalicFont=cmuntx.ttf]{cmunrm.ttf} and {\ttfamily \setmainfont[Path=/usr/share/fonts/truetype/cmu/,UprightFont=cmunrm.ttf,BoldFont=cmunbx.ttf,ItalicFont=cmunti.ttf,BoldItalicFont=cmunbi.ttf]{cmuntt.ttf}\setmonofont[Path=/usr/share/fonts/truetype/cmu/,UprightFont=cmuntt.ttf,BoldFont=cmuntb.ttf,ItalicFont=cmunit.ttf,BoldItalicFont=cmuntx.ttf]{cmuntt.ttf}\ttfamily \textbackslash{}varpi}\\ \cline{1-1}\cline{2-2}\cline{4-4}\cline{5-5} \hspace*{0pt}\ignorespaces{}\hspace*{0pt}{$\text{ }$}\setmainfont[Path=/usr/share/fonts/truetype/cmu/,UprightFont=cmunrm.ttf,BoldFont=cmunbx.ttf,ItalicFont=cmunti.ttf,BoldItalicFont=cmunbi.ttf]{cmunrm.ttf}\setmonofont[Path=/usr/share/fonts/truetype/cmu/,UprightFont=cmuntt.ttf,BoldFont=cmuntb.ttf,ItalicFont=cmunit.ttf,BoldItalicFont=cmuntx.ttf]{cmunrm.ttf} {$\Epsilon\,$}, {$\epsilon\,$} and {$\varepsilon\,$}&\hspace*{0pt}\ignorespaces{}\hspace*{0pt} {\ttfamily \setmainfont[Path=/usr/share/fonts/truetype/cmu/,UprightFont=cmunrm.ttf,BoldFont=cmunbx.ttf,ItalicFont=cmunti.ttf,BoldItalicFont=cmunbi.ttf]{cmuntt.ttf}\setmonofont[Path=/usr/share/fonts/truetype/cmu/,UprightFont=cmuntt.ttf,BoldFont=cmuntb.ttf,ItalicFont=cmunit.ttf,BoldItalicFont=cmuntx.ttf]{cmuntt.ttf}\ttfamily E}\setmainfont[Path=/usr/share/fonts/truetype/cmu/,UprightFont=cmunrm.ttf,BoldFont=cmunbx.ttf,ItalicFont=cmunti.ttf,BoldItalicFont=cmunbi.ttf]{cmunrm.ttf}\setmonofont[Path=/usr/share/fonts/truetype/cmu/,UprightFont=cmuntt.ttf,BoldFont=cmuntb.ttf,ItalicFont=cmunit.ttf,BoldItalicFont=cmuntx.ttf]{cmunrm.ttf}, {\ttfamily \setmainfont[Path=/usr/share/fonts/truetype/cmu/,UprightFont=cmunrm.ttf,BoldFont=cmunbx.ttf,ItalicFont=cmunti.ttf,BoldItalicFont=cmunbi.ttf]{cmuntt.ttf}\setmonofont[Path=/usr/share/fonts/truetype/cmu/,UprightFont=cmuntt.ttf,BoldFont=cmuntb.ttf,ItalicFont=cmunit.ttf,BoldItalicFont=cmuntx.ttf]{cmuntt.ttf}\ttfamily \textbackslash{}epsilon}{$\text{ }$}\setmainfont[Path=/usr/share/fonts/truetype/cmu/,UprightFont=cmunrm.ttf,BoldFont=cmunbx.ttf,ItalicFont=cmunti.ttf,BoldItalicFont=cmunbi.ttf]{cmunrm.ttf}\setmonofont[Path=/usr/share/fonts/truetype/cmu/,UprightFont=cmuntt.ttf,BoldFont=cmuntb.ttf,ItalicFont=cmunit.ttf,BoldItalicFont=cmuntx.ttf]{cmunrm.ttf} and {\ttfamily \setmainfont[Path=/usr/share/fonts/truetype/cmu/,UprightFont=cmunrm.ttf,BoldFont=cmunbx.ttf,ItalicFont=cmunti.ttf,BoldItalicFont=cmunbi.ttf]{cmuntt.ttf}\setmonofont[Path=/usr/share/fonts/truetype/cmu/,UprightFont=cmuntt.ttf,BoldFont=cmuntb.ttf,ItalicFont=cmunit.ttf,BoldItalicFont=cmuntx.ttf]{cmuntt.ttf}\ttfamily \textbackslash{}varepsilon}&\multicolumn{1}{|c|}{}&\hspace*{0pt}\ignorespaces{}\hspace*{0pt}{$\text{ }$}\setmainfont[Path=/usr/share/fonts/truetype/cmu/,UprightFont=cmunrm.ttf,BoldFont=cmunbx.ttf,ItalicFont=cmunti.ttf,BoldItalicFont=cmunbi.ttf]{cmunrm.ttf}\setmonofont[Path=/usr/share/fonts/truetype/cmu/,UprightFont=cmuntt.ttf,BoldFont=cmuntb.ttf,ItalicFont=cmunit.ttf,BoldItalicFont=cmuntx.ttf]{cmunrm.ttf} {$\Rho\,$}, {$\rho\,$} and {$\varrho\,$}&\hspace*{0pt}\ignorespaces{}\hspace*{0pt} {\ttfamily \setmainfont[Path=/usr/share/fonts/truetype/cmu/,UprightFont=cmunrm.ttf,BoldFont=cmunbx.ttf,ItalicFont=cmunti.ttf,BoldItalicFont=cmunbi.ttf]{cmuntt.ttf}\setmonofont[Path=/usr/share/fonts/truetype/cmu/,UprightFont=cmuntt.ttf,BoldFont=cmuntb.ttf,ItalicFont=cmunit.ttf,BoldItalicFont=cmuntx.ttf]{cmuntt.ttf}\ttfamily P}\setmainfont[Path=/usr/share/fonts/truetype/cmu/,UprightFont=cmunrm.ttf,BoldFont=cmunbx.ttf,ItalicFont=cmunti.ttf,BoldItalicFont=cmunbi.ttf]{cmunrm.ttf}\setmonofont[Path=/usr/share/fonts/truetype/cmu/,UprightFont=cmuntt.ttf,BoldFont=cmuntb.ttf,ItalicFont=cmunit.ttf,BoldItalicFont=cmuntx.ttf]{cmunrm.ttf}, {\ttfamily \setmainfont[Path=/usr/share/fonts/truetype/cmu/,UprightFont=cmunrm.ttf,BoldFont=cmunbx.ttf,ItalicFont=cmunti.ttf,BoldItalicFont=cmunbi.ttf]{cmuntt.ttf}\setmonofont[Path=/usr/share/fonts/truetype/cmu/,UprightFont=cmuntt.ttf,BoldFont=cmuntb.ttf,ItalicFont=cmunit.ttf,BoldItalicFont=cmuntx.ttf]{cmuntt.ttf}\ttfamily \textbackslash{}rho}{$\text{ }$}\setmainfont[Path=/usr/share/fonts/truetype/cmu/,UprightFont=cmunrm.ttf,BoldFont=cmunbx.ttf,ItalicFont=cmunti.ttf,BoldItalicFont=cmunbi.ttf]{cmunrm.ttf}\setmonofont[Path=/usr/share/fonts/truetype/cmu/,UprightFont=cmuntt.ttf,BoldFont=cmuntb.ttf,ItalicFont=cmunit.ttf,BoldItalicFont=cmuntx.ttf]{cmunrm.ttf} and {\ttfamily \setmainfont[Path=/usr/share/fonts/truetype/cmu/,UprightFont=cmunrm.ttf,BoldFont=cmunbx.ttf,ItalicFont=cmunti.ttf,BoldItalicFont=cmunbi.ttf]{cmuntt.ttf}\setmonofont[Path=/usr/share/fonts/truetype/cmu/,UprightFont=cmuntt.ttf,BoldFont=cmuntb.ttf,ItalicFont=cmunit.ttf,BoldItalicFont=cmuntx.ttf]{cmuntt.ttf}\ttfamily \textbackslash{}varrho}\\ \cline{1-1}\cline{2-2}\cline{4-4}\cline{5-5} \hspace*{0pt}\ignorespaces{}\hspace*{0pt}{$\text{ }$}\setmainfont[Path=/usr/share/fonts/truetype/cmu/,UprightFont=cmunrm.ttf,BoldFont=cmunbx.ttf,ItalicFont=cmunti.ttf,BoldItalicFont=cmunbi.ttf]{cmunrm.ttf}\setmonofont[Path=/usr/share/fonts/truetype/cmu/,UprightFont=cmuntt.ttf,BoldFont=cmuntb.ttf,ItalicFont=cmunit.ttf,BoldItalicFont=cmuntx.ttf]{cmunrm.ttf} {$\Zeta\,$} and {$\zeta\,$}&\hspace*{0pt}\ignorespaces{}\hspace*{0pt} {\ttfamily \setmainfont[Path=/usr/share/fonts/truetype/cmu/,UprightFont=cmunrm.ttf,BoldFont=cmunbx.ttf,ItalicFont=cmunti.ttf,BoldItalicFont=cmunbi.ttf]{cmuntt.ttf}\setmonofont[Path=/usr/share/fonts/truetype/cmu/,UprightFont=cmuntt.ttf,BoldFont=cmuntb.ttf,ItalicFont=cmunit.ttf,BoldItalicFont=cmuntx.ttf]{cmuntt.ttf}\ttfamily Z}{$\text{ }$}\setmainfont[Path=/usr/share/fonts/truetype/cmu/,UprightFont=cmunrm.ttf,BoldFont=cmunbx.ttf,ItalicFont=cmunti.ttf,BoldItalicFont=cmunbi.ttf]{cmunrm.ttf}\setmonofont[Path=/usr/share/fonts/truetype/cmu/,UprightFont=cmuntt.ttf,BoldFont=cmuntb.ttf,ItalicFont=cmunit.ttf,BoldItalicFont=cmuntx.ttf]{cmunrm.ttf} and {\ttfamily \setmainfont[Path=/usr/share/fonts/truetype/cmu/,UprightFont=cmunrm.ttf,BoldFont=cmunbx.ttf,ItalicFont=cmunti.ttf,BoldItalicFont=cmunbi.ttf]{cmuntt.ttf}\setmonofont[Path=/usr/share/fonts/truetype/cmu/,UprightFont=cmuntt.ttf,BoldFont=cmuntb.ttf,ItalicFont=cmunit.ttf,BoldItalicFont=cmuntx.ttf]{cmuntt.ttf}\ttfamily \textbackslash{}zeta}&\multicolumn{1}{|c|}{}&\hspace*{0pt}\ignorespaces{}\hspace*{0pt}{$\text{ }$}\setmainfont[Path=/usr/share/fonts/truetype/cmu/,UprightFont=cmunrm.ttf,BoldFont=cmunbx.ttf,ItalicFont=cmunti.ttf,BoldItalicFont=cmunbi.ttf]{cmunrm.ttf}\setmonofont[Path=/usr/share/fonts/truetype/cmu/,UprightFont=cmuntt.ttf,BoldFont=cmuntb.ttf,ItalicFont=cmunit.ttf,BoldItalicFont=cmuntx.ttf]{cmunrm.ttf} {$\Sigma\,$}, {$\sigma\,$} and {$\varsigma\,$}&\hspace*{0pt}\ignorespaces{}\hspace*{0pt} {\ttfamily \setmainfont[Path=/usr/share/fonts/truetype/cmu/,UprightFont=cmunrm.ttf,BoldFont=cmunbx.ttf,ItalicFont=cmunti.ttf,BoldItalicFont=cmunbi.ttf]{cmuntt.ttf}\setmonofont[Path=/usr/share/fonts/truetype/cmu/,UprightFont=cmuntt.ttf,BoldFont=cmuntb.ttf,ItalicFont=cmunit.ttf,BoldItalicFont=cmuntx.ttf]{cmuntt.ttf}\ttfamily \textbackslash{}Sigma}\setmainfont[Path=/usr/share/fonts/truetype/cmu/,UprightFont=cmunrm.ttf,BoldFont=cmunbx.ttf,ItalicFont=cmunti.ttf,BoldItalicFont=cmunbi.ttf]{cmunrm.ttf}\setmonofont[Path=/usr/share/fonts/truetype/cmu/,UprightFont=cmuntt.ttf,BoldFont=cmuntb.ttf,ItalicFont=cmunit.ttf,BoldItalicFont=cmuntx.ttf]{cmunrm.ttf}, {\ttfamily \setmainfont[Path=/usr/share/fonts/truetype/cmu/,UprightFont=cmunrm.ttf,BoldFont=cmunbx.ttf,ItalicFont=cmunti.ttf,BoldItalicFont=cmunbi.ttf]{cmuntt.ttf}\setmonofont[Path=/usr/share/fonts/truetype/cmu/,UprightFont=cmuntt.ttf,BoldFont=cmuntb.ttf,ItalicFont=cmunit.ttf,BoldItalicFont=cmuntx.ttf]{cmuntt.ttf}\ttfamily \textbackslash{}sigma}{$\text{ }$}\setmainfont[Path=/usr/share/fonts/truetype/cmu/,UprightFont=cmunrm.ttf,BoldFont=cmunbx.ttf,ItalicFont=cmunti.ttf,BoldItalicFont=cmunbi.ttf]{cmunrm.ttf}\setmonofont[Path=/usr/share/fonts/truetype/cmu/,UprightFont=cmuntt.ttf,BoldFont=cmuntb.ttf,ItalicFont=cmunit.ttf,BoldItalicFont=cmuntx.ttf]{cmunrm.ttf} and {\ttfamily \setmainfont[Path=/usr/share/fonts/truetype/cmu/,UprightFont=cmunrm.ttf,BoldFont=cmunbx.ttf,ItalicFont=cmunti.ttf,BoldItalicFont=cmunbi.ttf]{cmuntt.ttf}\setmonofont[Path=/usr/share/fonts/truetype/cmu/,UprightFont=cmuntt.ttf,BoldFont=cmuntb.ttf,ItalicFont=cmunit.ttf,BoldItalicFont=cmuntx.ttf]{cmuntt.ttf}\ttfamily \textbackslash{}varsigma}\\ \cline{1-1}\cline{2-2}\cline{4-4}\cline{5-5} \hspace*{0pt}\ignorespaces{}\hspace*{0pt}{$\text{ }$}\setmainfont[Path=/usr/share/fonts/truetype/cmu/,UprightFont=cmunrm.ttf,BoldFont=cmunbx.ttf,ItalicFont=cmunti.ttf,BoldItalicFont=cmunbi.ttf]{cmunrm.ttf}\setmonofont[Path=/usr/share/fonts/truetype/cmu/,UprightFont=cmuntt.ttf,BoldFont=cmuntb.ttf,ItalicFont=cmunit.ttf,BoldItalicFont=cmuntx.ttf]{cmunrm.ttf} {$\Eta\,$} and {$\eta\,$}&\hspace*{0pt}\ignorespaces{}\hspace*{0pt} {\ttfamily \setmainfont[Path=/usr/share/fonts/truetype/cmu/,UprightFont=cmunrm.ttf,BoldFont=cmunbx.ttf,ItalicFont=cmunti.ttf,BoldItalicFont=cmunbi.ttf]{cmuntt.ttf}\setmonofont[Path=/usr/share/fonts/truetype/cmu/,UprightFont=cmuntt.ttf,BoldFont=cmuntb.ttf,ItalicFont=cmunit.ttf,BoldItalicFont=cmuntx.ttf]{cmuntt.ttf}\ttfamily H}{$\text{ }$}\setmainfont[Path=/usr/share/fonts/truetype/cmu/,UprightFont=cmunrm.ttf,BoldFont=cmunbx.ttf,ItalicFont=cmunti.ttf,BoldItalicFont=cmunbi.ttf]{cmunrm.ttf}\setmonofont[Path=/usr/share/fonts/truetype/cmu/,UprightFont=cmuntt.ttf,BoldFont=cmuntb.ttf,ItalicFont=cmunit.ttf,BoldItalicFont=cmuntx.ttf]{cmunrm.ttf} and {\ttfamily \setmainfont[Path=/usr/share/fonts/truetype/cmu/,UprightFont=cmunrm.ttf,BoldFont=cmunbx.ttf,ItalicFont=cmunti.ttf,BoldItalicFont=cmunbi.ttf]{cmuntt.ttf}\setmonofont[Path=/usr/share/fonts/truetype/cmu/,UprightFont=cmuntt.ttf,BoldFont=cmuntb.ttf,ItalicFont=cmunit.ttf,BoldItalicFont=cmuntx.ttf]{cmuntt.ttf}\ttfamily \textbackslash{}eta}&\multicolumn{1}{|c|}{}&\hspace*{0pt}\ignorespaces{}\hspace*{0pt}{$\text{ }$}\setmainfont[Path=/usr/share/fonts/truetype/cmu/,UprightFont=cmunrm.ttf,BoldFont=cmunbx.ttf,ItalicFont=cmunti.ttf,BoldItalicFont=cmunbi.ttf]{cmunrm.ttf}\setmonofont[Path=/usr/share/fonts/truetype/cmu/,UprightFont=cmuntt.ttf,BoldFont=cmuntb.ttf,ItalicFont=cmunit.ttf,BoldItalicFont=cmuntx.ttf]{cmunrm.ttf} {$\Tau\,$} and {$\tau\,$}&\hspace*{0pt}\ignorespaces{}\hspace*{0pt} {\ttfamily \setmainfont[Path=/usr/share/fonts/truetype/cmu/,UprightFont=cmunrm.ttf,BoldFont=cmunbx.ttf,ItalicFont=cmunti.ttf,BoldItalicFont=cmunbi.ttf]{cmuntt.ttf}\setmonofont[Path=/usr/share/fonts/truetype/cmu/,UprightFont=cmuntt.ttf,BoldFont=cmuntb.ttf,ItalicFont=cmunit.ttf,BoldItalicFont=cmuntx.ttf]{cmuntt.ttf}\ttfamily T}{$\text{ }$}\setmainfont[Path=/usr/share/fonts/truetype/cmu/,UprightFont=cmunrm.ttf,BoldFont=cmunbx.ttf,ItalicFont=cmunti.ttf,BoldItalicFont=cmunbi.ttf]{cmunrm.ttf}\setmonofont[Path=/usr/share/fonts/truetype/cmu/,UprightFont=cmuntt.ttf,BoldFont=cmuntb.ttf,ItalicFont=cmunit.ttf,BoldItalicFont=cmuntx.ttf]{cmunrm.ttf} and {\ttfamily \setmainfont[Path=/usr/share/fonts/truetype/cmu/,UprightFont=cmunrm.ttf,BoldFont=cmunbx.ttf,ItalicFont=cmunti.ttf,BoldItalicFont=cmunbi.ttf]{cmuntt.ttf}\setmonofont[Path=/usr/share/fonts/truetype/cmu/,UprightFont=cmuntt.ttf,BoldFont=cmuntb.ttf,ItalicFont=cmunit.ttf,BoldItalicFont=cmuntx.ttf]{cmuntt.ttf}\ttfamily \textbackslash{}tau}\\ \cline{1-1}\cline{2-2}\cline{4-4}\cline{5-5} \hspace*{0pt}\ignorespaces{}\hspace*{0pt}{$\text{ }$}\setmainfont[Path=/usr/share/fonts/truetype/cmu/,UprightFont=cmunrm.ttf,BoldFont=cmunbx.ttf,ItalicFont=cmunti.ttf,BoldItalicFont=cmunbi.ttf]{cmunrm.ttf}\setmonofont[Path=/usr/share/fonts/truetype/cmu/,UprightFont=cmuntt.ttf,BoldFont=cmuntb.ttf,ItalicFont=cmunit.ttf,BoldItalicFont=cmuntx.ttf]{cmunrm.ttf} {$\Theta\,$}, {$\theta\,$} and {$\vartheta\,$}&\hspace*{0pt}\ignorespaces{}\hspace*{0pt} {\ttfamily \setmainfont[Path=/usr/share/fonts/truetype/cmu/,UprightFont=cmunrm.ttf,BoldFont=cmunbx.ttf,ItalicFont=cmunti.ttf,BoldItalicFont=cmunbi.ttf]{cmuntt.ttf}\setmonofont[Path=/usr/share/fonts/truetype/cmu/,UprightFont=cmuntt.ttf,BoldFont=cmuntb.ttf,ItalicFont=cmunit.ttf,BoldItalicFont=cmuntx.ttf]{cmuntt.ttf}\ttfamily \textbackslash{}Theta}\setmainfont[Path=/usr/share/fonts/truetype/cmu/,UprightFont=cmunrm.ttf,BoldFont=cmunbx.ttf,ItalicFont=cmunti.ttf,BoldItalicFont=cmunbi.ttf]{cmunrm.ttf}\setmonofont[Path=/usr/share/fonts/truetype/cmu/,UprightFont=cmuntt.ttf,BoldFont=cmuntb.ttf,ItalicFont=cmunit.ttf,BoldItalicFont=cmuntx.ttf]{cmunrm.ttf}, {\ttfamily \setmainfont[Path=/usr/share/fonts/truetype/cmu/,UprightFont=cmunrm.ttf,BoldFont=cmunbx.ttf,ItalicFont=cmunti.ttf,BoldItalicFont=cmunbi.ttf]{cmuntt.ttf}\setmonofont[Path=/usr/share/fonts/truetype/cmu/,UprightFont=cmuntt.ttf,BoldFont=cmuntb.ttf,ItalicFont=cmunit.ttf,BoldItalicFont=cmuntx.ttf]{cmuntt.ttf}\ttfamily \textbackslash{}theta}{$\text{ }$}\setmainfont[Path=/usr/share/fonts/truetype/cmu/,UprightFont=cmunrm.ttf,BoldFont=cmunbx.ttf,ItalicFont=cmunti.ttf,BoldItalicFont=cmunbi.ttf]{cmunrm.ttf}\setmonofont[Path=/usr/share/fonts/truetype/cmu/,UprightFont=cmuntt.ttf,BoldFont=cmuntb.ttf,ItalicFont=cmunit.ttf,BoldItalicFont=cmuntx.ttf]{cmunrm.ttf} and {\ttfamily \setmainfont[Path=/usr/share/fonts/truetype/cmu/,UprightFont=cmunrm.ttf,BoldFont=cmunbx.ttf,ItalicFont=cmunti.ttf,BoldItalicFont=cmunbi.ttf]{cmuntt.ttf}\setmonofont[Path=/usr/share/fonts/truetype/cmu/,UprightFont=cmuntt.ttf,BoldFont=cmuntb.ttf,ItalicFont=cmunit.ttf,BoldItalicFont=cmuntx.ttf]{cmuntt.ttf}\ttfamily \textbackslash{}vartheta}&\multicolumn{1}{|c|}{}&\hspace*{0pt}\ignorespaces{}\hspace*{0pt}{$\text{ }$}\setmainfont[Path=/usr/share/fonts/truetype/cmu/,UprightFont=cmunrm.ttf,BoldFont=cmunbx.ttf,ItalicFont=cmunti.ttf,BoldItalicFont=cmunbi.ttf]{cmunrm.ttf}\setmonofont[Path=/usr/share/fonts/truetype/cmu/,UprightFont=cmuntt.ttf,BoldFont=cmuntb.ttf,ItalicFont=cmunit.ttf,BoldItalicFont=cmuntx.ttf]{cmunrm.ttf} {$\Upsilon\,$} and {$\upsilon\,$}&\hspace*{0pt}\ignorespaces{}\hspace*{0pt} {\ttfamily \setmainfont[Path=/usr/share/fonts/truetype/cmu/,UprightFont=cmunrm.ttf,BoldFont=cmunbx.ttf,ItalicFont=cmunti.ttf,BoldItalicFont=cmunbi.ttf]{cmuntt.ttf}\setmonofont[Path=/usr/share/fonts/truetype/cmu/,UprightFont=cmuntt.ttf,BoldFont=cmuntb.ttf,ItalicFont=cmunit.ttf,BoldItalicFont=cmuntx.ttf]{cmuntt.ttf}\ttfamily \textbackslash{}Upsilon}{$\text{ }$}\setmainfont[Path=/usr/share/fonts/truetype/cmu/,UprightFont=cmunrm.ttf,BoldFont=cmunbx.ttf,ItalicFont=cmunti.ttf,BoldItalicFont=cmunbi.ttf]{cmunrm.ttf}\setmonofont[Path=/usr/share/fonts/truetype/cmu/,UprightFont=cmuntt.ttf,BoldFont=cmuntb.ttf,ItalicFont=cmunit.ttf,BoldItalicFont=cmuntx.ttf]{cmunrm.ttf} and {\ttfamily \setmainfont[Path=/usr/share/fonts/truetype/cmu/,UprightFont=cmunrm.ttf,BoldFont=cmunbx.ttf,ItalicFont=cmunti.ttf,BoldItalicFont=cmunbi.ttf]{cmuntt.ttf}\setmonofont[Path=/usr/share/fonts/truetype/cmu/,UprightFont=cmuntt.ttf,BoldFont=cmuntb.ttf,ItalicFont=cmunit.ttf,BoldItalicFont=cmuntx.ttf]{cmuntt.ttf}\ttfamily \textbackslash{}upsilon}\\ \cline{1-1}\cline{2-2}\cline{4-4}\cline{5-5} \hspace*{0pt}\ignorespaces{}\hspace*{0pt}{$\text{ }$}\setmainfont[Path=/usr/share/fonts/truetype/cmu/,UprightFont=cmunrm.ttf,BoldFont=cmunbx.ttf,ItalicFont=cmunti.ttf,BoldItalicFont=cmunbi.ttf]{cmunrm.ttf}\setmonofont[Path=/usr/share/fonts/truetype/cmu/,UprightFont=cmuntt.ttf,BoldFont=cmuntb.ttf,ItalicFont=cmunit.ttf,BoldItalicFont=cmuntx.ttf]{cmunrm.ttf} {$\Iota\,$} and {$\iota\,$}&\hspace*{0pt}\ignorespaces{}\hspace*{0pt} {\ttfamily \setmainfont[Path=/usr/share/fonts/truetype/cmu/,UprightFont=cmunrm.ttf,BoldFont=cmunbx.ttf,ItalicFont=cmunti.ttf,BoldItalicFont=cmunbi.ttf]{cmuntt.ttf}\setmonofont[Path=/usr/share/fonts/truetype/cmu/,UprightFont=cmuntt.ttf,BoldFont=cmuntb.ttf,ItalicFont=cmunit.ttf,BoldItalicFont=cmuntx.ttf]{cmuntt.ttf}\ttfamily I}{$\text{ }$}\setmainfont[Path=/usr/share/fonts/truetype/cmu/,UprightFont=cmunrm.ttf,BoldFont=cmunbx.ttf,ItalicFont=cmunti.ttf,BoldItalicFont=cmunbi.ttf]{cmunrm.ttf}\setmonofont[Path=/usr/share/fonts/truetype/cmu/,UprightFont=cmuntt.ttf,BoldFont=cmuntb.ttf,ItalicFont=cmunit.ttf,BoldItalicFont=cmuntx.ttf]{cmunrm.ttf} and {\ttfamily \setmainfont[Path=/usr/share/fonts/truetype/cmu/,UprightFont=cmunrm.ttf,BoldFont=cmunbx.ttf,ItalicFont=cmunti.ttf,BoldItalicFont=cmunbi.ttf]{cmuntt.ttf}\setmonofont[Path=/usr/share/fonts/truetype/cmu/,UprightFont=cmuntt.ttf,BoldFont=cmuntb.ttf,ItalicFont=cmunit.ttf,BoldItalicFont=cmuntx.ttf]{cmuntt.ttf}\ttfamily \textbackslash{}iota}&\multicolumn{1}{|c|}{}&\hspace*{0pt}\ignorespaces{}\hspace*{0pt}{$\text{ }$}\setmainfont[Path=/usr/share/fonts/truetype/cmu/,UprightFont=cmunrm.ttf,BoldFont=cmunbx.ttf,ItalicFont=cmunti.ttf,BoldItalicFont=cmunbi.ttf]{cmunrm.ttf}\setmonofont[Path=/usr/share/fonts/truetype/cmu/,UprightFont=cmuntt.ttf,BoldFont=cmuntb.ttf,ItalicFont=cmunit.ttf,BoldItalicFont=cmuntx.ttf]{cmunrm.ttf} {$\Phi\,$}, {$\phi\,$}, and {$\varphi\,$}&\hspace*{0pt}\ignorespaces{}\hspace*{0pt} {\ttfamily \setmainfont[Path=/usr/share/fonts/truetype/cmu/,UprightFont=cmunrm.ttf,BoldFont=cmunbx.ttf,ItalicFont=cmunti.ttf,BoldItalicFont=cmunbi.ttf]{cmuntt.ttf}\setmonofont[Path=/usr/share/fonts/truetype/cmu/,UprightFont=cmuntt.ttf,BoldFont=cmuntb.ttf,ItalicFont=cmunit.ttf,BoldItalicFont=cmuntx.ttf]{cmuntt.ttf}\ttfamily \textbackslash{}Phi}\setmainfont[Path=/usr/share/fonts/truetype/cmu/,UprightFont=cmunrm.ttf,BoldFont=cmunbx.ttf,ItalicFont=cmunti.ttf,BoldItalicFont=cmunbi.ttf]{cmunrm.ttf}\setmonofont[Path=/usr/share/fonts/truetype/cmu/,UprightFont=cmuntt.ttf,BoldFont=cmuntb.ttf,ItalicFont=cmunit.ttf,BoldItalicFont=cmuntx.ttf]{cmunrm.ttf}, {\ttfamily \setmainfont[Path=/usr/share/fonts/truetype/cmu/,UprightFont=cmunrm.ttf,BoldFont=cmunbx.ttf,ItalicFont=cmunti.ttf,BoldItalicFont=cmunbi.ttf]{cmuntt.ttf}\setmonofont[Path=/usr/share/fonts/truetype/cmu/,UprightFont=cmuntt.ttf,BoldFont=cmuntb.ttf,ItalicFont=cmunit.ttf,BoldItalicFont=cmuntx.ttf]{cmuntt.ttf}\ttfamily \textbackslash{}phi}{$\text{ }$}\setmainfont[Path=/usr/share/fonts/truetype/cmu/,UprightFont=cmunrm.ttf,BoldFont=cmunbx.ttf,ItalicFont=cmunti.ttf,BoldItalicFont=cmunbi.ttf]{cmunrm.ttf}\setmonofont[Path=/usr/share/fonts/truetype/cmu/,UprightFont=cmuntt.ttf,BoldFont=cmuntb.ttf,ItalicFont=cmunit.ttf,BoldItalicFont=cmuntx.ttf]{cmunrm.ttf} and {\ttfamily \setmainfont[Path=/usr/share/fonts/truetype/cmu/,UprightFont=cmunrm.ttf,BoldFont=cmunbx.ttf,ItalicFont=cmunti.ttf,BoldItalicFont=cmunbi.ttf]{cmuntt.ttf}\setmonofont[Path=/usr/share/fonts/truetype/cmu/,UprightFont=cmuntt.ttf,BoldFont=cmuntb.ttf,ItalicFont=cmunit.ttf,BoldItalicFont=cmuntx.ttf]{cmuntt.ttf}\ttfamily \textbackslash{}varphi}\\ \cline{1-1}\cline{2-2}\cline{4-4}\cline{5-5} \hspace*{0pt}\ignorespaces{}\hspace*{0pt}{$\text{ }$}\setmainfont[Path=/usr/share/fonts/truetype/cmu/,UprightFont=cmunrm.ttf,BoldFont=cmunbx.ttf,ItalicFont=cmunti.ttf,BoldItalicFont=cmunbi.ttf]{cmunrm.ttf}\setmonofont[Path=/usr/share/fonts/truetype/cmu/,UprightFont=cmuntt.ttf,BoldFont=cmuntb.ttf,ItalicFont=cmunit.ttf,BoldItalicFont=cmuntx.ttf]{cmunrm.ttf} {$\Kappa\,$}, {$\kappa\,$} and {$\varkappa\,$} &\hspace*{0pt}\ignorespaces{}\hspace*{0pt} {\ttfamily \setmainfont[Path=/usr/share/fonts/truetype/cmu/,UprightFont=cmunrm.ttf,BoldFont=cmunbx.ttf,ItalicFont=cmunti.ttf,BoldItalicFont=cmunbi.ttf]{cmuntt.ttf}\setmonofont[Path=/usr/share/fonts/truetype/cmu/,UprightFont=cmuntt.ttf,BoldFont=cmuntb.ttf,ItalicFont=cmunit.ttf,BoldItalicFont=cmuntx.ttf]{cmuntt.ttf}\ttfamily K}\setmainfont[Path=/usr/share/fonts/truetype/cmu/,UprightFont=cmunrm.ttf,BoldFont=cmunbx.ttf,ItalicFont=cmunti.ttf,BoldItalicFont=cmunbi.ttf]{cmunrm.ttf}\setmonofont[Path=/usr/share/fonts/truetype/cmu/,UprightFont=cmuntt.ttf,BoldFont=cmuntb.ttf,ItalicFont=cmunit.ttf,BoldItalicFont=cmuntx.ttf]{cmunrm.ttf}, {\ttfamily \setmainfont[Path=/usr/share/fonts/truetype/cmu/,UprightFont=cmunrm.ttf,BoldFont=cmunbx.ttf,ItalicFont=cmunti.ttf,BoldItalicFont=cmunbi.ttf]{cmuntt.ttf}\setmonofont[Path=/usr/share/fonts/truetype/cmu/,UprightFont=cmuntt.ttf,BoldFont=cmuntb.ttf,ItalicFont=cmunit.ttf,BoldItalicFont=cmuntx.ttf]{cmuntt.ttf}\ttfamily \textbackslash{}kappa}{$\text{ }$}\setmainfont[Path=/usr/share/fonts/truetype/cmu/,UprightFont=cmunrm.ttf,BoldFont=cmunbx.ttf,ItalicFont=cmunti.ttf,BoldItalicFont=cmunbi.ttf]{cmunrm.ttf}\setmonofont[Path=/usr/share/fonts/truetype/cmu/,UprightFont=cmuntt.ttf,BoldFont=cmuntb.ttf,ItalicFont=cmunit.ttf,BoldItalicFont=cmuntx.ttf]{cmunrm.ttf} and {\ttfamily \setmainfont[Path=/usr/share/fonts/truetype/cmu/,UprightFont=cmunrm.ttf,BoldFont=cmunbx.ttf,ItalicFont=cmunti.ttf,BoldItalicFont=cmunbi.ttf]{cmuntt.ttf}\setmonofont[Path=/usr/share/fonts/truetype/cmu/,UprightFont=cmuntt.ttf,BoldFont=cmuntb.ttf,ItalicFont=cmunit.ttf,BoldItalicFont=cmuntx.ttf]{cmuntt.ttf}\ttfamily \textbackslash{}varkappa}&\multicolumn{1}{|c|}{}&\hspace*{0pt}\ignorespaces{}\hspace*{0pt}{$\text{ }$}\setmainfont[Path=/usr/share/fonts/truetype/cmu/,UprightFont=cmunrm.ttf,BoldFont=cmunbx.ttf,ItalicFont=cmunti.ttf,BoldItalicFont=cmunbi.ttf]{cmunrm.ttf}\setmonofont[Path=/usr/share/fonts/truetype/cmu/,UprightFont=cmuntt.ttf,BoldFont=cmuntb.ttf,ItalicFont=cmunit.ttf,BoldItalicFont=cmuntx.ttf]{cmunrm.ttf} {$\Chi\,$} and {$\chi\,$}&\hspace*{0pt}\ignorespaces{}\hspace*{0pt} {\ttfamily \setmainfont[Path=/usr/share/fonts/truetype/cmu/,UprightFont=cmunrm.ttf,BoldFont=cmunbx.ttf,ItalicFont=cmunti.ttf,BoldItalicFont=cmunbi.ttf]{cmuntt.ttf}\setmonofont[Path=/usr/share/fonts/truetype/cmu/,UprightFont=cmuntt.ttf,BoldFont=cmuntb.ttf,ItalicFont=cmunit.ttf,BoldItalicFont=cmuntx.ttf]{cmuntt.ttf}\ttfamily X}{$\text{ }$}\setmainfont[Path=/usr/share/fonts/truetype/cmu/,UprightFont=cmunrm.ttf,BoldFont=cmunbx.ttf,ItalicFont=cmunti.ttf,BoldItalicFont=cmunbi.ttf]{cmunrm.ttf}\setmonofont[Path=/usr/share/fonts/truetype/cmu/,UprightFont=cmuntt.ttf,BoldFont=cmuntb.ttf,ItalicFont=cmunit.ttf,BoldItalicFont=cmuntx.ttf]{cmunrm.ttf} and {\ttfamily \setmainfont[Path=/usr/share/fonts/truetype/cmu/,UprightFont=cmunrm.ttf,BoldFont=cmunbx.ttf,ItalicFont=cmunti.ttf,BoldItalicFont=cmunbi.ttf]{cmuntt.ttf}\setmonofont[Path=/usr/share/fonts/truetype/cmu/,UprightFont=cmuntt.ttf,BoldFont=cmuntb.ttf,ItalicFont=cmunit.ttf,BoldItalicFont=cmuntx.ttf]{cmuntt.ttf}\ttfamily \textbackslash{}chi}\\ \cline{1-1}\cline{2-2}\cline{4-4}\cline{5-5} \hspace*{0pt}\ignorespaces{}\hspace*{0pt}{$\text{ }$}\setmainfont[Path=/usr/share/fonts/truetype/cmu/,UprightFont=cmunrm.ttf,BoldFont=cmunbx.ttf,ItalicFont=cmunti.ttf,BoldItalicFont=cmunbi.ttf]{cmunrm.ttf}\setmonofont[Path=/usr/share/fonts/truetype/cmu/,UprightFont=cmuntt.ttf,BoldFont=cmuntb.ttf,ItalicFont=cmunit.ttf,BoldItalicFont=cmuntx.ttf]{cmunrm.ttf} {$\Lambda\,$} and {$\lambda\,$}&\hspace*{0pt}\ignorespaces{}\hspace*{0pt} {\ttfamily \setmainfont[Path=/usr/share/fonts/truetype/cmu/,UprightFont=cmunrm.ttf,BoldFont=cmunbx.ttf,ItalicFont=cmunti.ttf,BoldItalicFont=cmunbi.ttf]{cmuntt.ttf}\setmonofont[Path=/usr/share/fonts/truetype/cmu/,UprightFont=cmuntt.ttf,BoldFont=cmuntb.ttf,ItalicFont=cmunit.ttf,BoldItalicFont=cmuntx.ttf]{cmuntt.ttf}\ttfamily \textbackslash{}Lambda}{$\text{ }$}\setmainfont[Path=/usr/share/fonts/truetype/cmu/,UprightFont=cmunrm.ttf,BoldFont=cmunbx.ttf,ItalicFont=cmunti.ttf,BoldItalicFont=cmunbi.ttf]{cmunrm.ttf}\setmonofont[Path=/usr/share/fonts/truetype/cmu/,UprightFont=cmuntt.ttf,BoldFont=cmuntb.ttf,ItalicFont=cmunit.ttf,BoldItalicFont=cmuntx.ttf]{cmunrm.ttf} and {\ttfamily \setmainfont[Path=/usr/share/fonts/truetype/cmu/,UprightFont=cmunrm.ttf,BoldFont=cmunbx.ttf,ItalicFont=cmunti.ttf,BoldItalicFont=cmunbi.ttf]{cmuntt.ttf}\setmonofont[Path=/usr/share/fonts/truetype/cmu/,UprightFont=cmuntt.ttf,BoldFont=cmuntb.ttf,ItalicFont=cmunit.ttf,BoldItalicFont=cmuntx.ttf]{cmuntt.ttf}\ttfamily \textbackslash{}lambda}&\multicolumn{1}{|c|}{}&\hspace*{0pt}\ignorespaces{}\hspace*{0pt}{$\text{ }$}\setmainfont[Path=/usr/share/fonts/truetype/cmu/,UprightFont=cmunrm.ttf,BoldFont=cmunbx.ttf,ItalicFont=cmunti.ttf,BoldItalicFont=cmunbi.ttf]{cmunrm.ttf}\setmonofont[Path=/usr/share/fonts/truetype/cmu/,UprightFont=cmuntt.ttf,BoldFont=cmuntb.ttf,ItalicFont=cmunit.ttf,BoldItalicFont=cmuntx.ttf]{cmunrm.ttf} {$\Psi\,$} and {$\psi\,$}&\hspace*{0pt}\ignorespaces{}\hspace*{0pt} {\ttfamily \setmainfont[Path=/usr/share/fonts/truetype/cmu/,UprightFont=cmunrm.ttf,BoldFont=cmunbx.ttf,ItalicFont=cmunti.ttf,BoldItalicFont=cmunbi.ttf]{cmuntt.ttf}\setmonofont[Path=/usr/share/fonts/truetype/cmu/,UprightFont=cmuntt.ttf,BoldFont=cmuntb.ttf,ItalicFont=cmunit.ttf,BoldItalicFont=cmuntx.ttf]{cmuntt.ttf}\ttfamily \textbackslash{}Psi}{$\text{ }$}\setmainfont[Path=/usr/share/fonts/truetype/cmu/,UprightFont=cmunrm.ttf,BoldFont=cmunbx.ttf,ItalicFont=cmunti.ttf,BoldItalicFont=cmunbi.ttf]{cmunrm.ttf}\setmonofont[Path=/usr/share/fonts/truetype/cmu/,UprightFont=cmuntt.ttf,BoldFont=cmuntb.ttf,ItalicFont=cmunit.ttf,BoldItalicFont=cmuntx.ttf]{cmunrm.ttf} and {\ttfamily \setmainfont[Path=/usr/share/fonts/truetype/cmu/,UprightFont=cmunrm.ttf,BoldFont=cmunbx.ttf,ItalicFont=cmunti.ttf,BoldItalicFont=cmunbi.ttf]{cmuntt.ttf}\setmonofont[Path=/usr/share/fonts/truetype/cmu/,UprightFont=cmuntt.ttf,BoldFont=cmuntb.ttf,ItalicFont=cmunit.ttf,BoldItalicFont=cmuntx.ttf]{cmuntt.ttf}\ttfamily \textbackslash{}psi}\\ \cline{1-1}\cline{2-2}\cline{4-4}\cline{5-5} \hspace*{0pt}\ignorespaces{}\hspace*{0pt}{$\text{ }$}\setmainfont[Path=/usr/share/fonts/truetype/cmu/,UprightFont=cmunrm.ttf,BoldFont=cmunbx.ttf,ItalicFont=cmunti.ttf,BoldItalicFont=cmunbi.ttf]{cmunrm.ttf}\setmonofont[Path=/usr/share/fonts/truetype/cmu/,UprightFont=cmuntt.ttf,BoldFont=cmuntb.ttf,ItalicFont=cmunit.ttf,BoldItalicFont=cmuntx.ttf]{cmunrm.ttf} {$\Mu\,$} and {$\mu\,$}&\hspace*{0pt}\ignorespaces{}\hspace*{0pt} {\ttfamily \setmainfont[Path=/usr/share/fonts/truetype/cmu/,UprightFont=cmunrm.ttf,BoldFont=cmunbx.ttf,ItalicFont=cmunti.ttf,BoldItalicFont=cmunbi.ttf]{cmuntt.ttf}\setmonofont[Path=/usr/share/fonts/truetype/cmu/,UprightFont=cmuntt.ttf,BoldFont=cmuntb.ttf,ItalicFont=cmunit.ttf,BoldItalicFont=cmuntx.ttf]{cmuntt.ttf}\ttfamily M}{$\text{ }$}\setmainfont[Path=/usr/share/fonts/truetype/cmu/,UprightFont=cmunrm.ttf,BoldFont=cmunbx.ttf,ItalicFont=cmunti.ttf,BoldItalicFont=cmunbi.ttf]{cmunrm.ttf}\setmonofont[Path=/usr/share/fonts/truetype/cmu/,UprightFont=cmuntt.ttf,BoldFont=cmuntb.ttf,ItalicFont=cmunit.ttf,BoldItalicFont=cmuntx.ttf]{cmunrm.ttf} and {\ttfamily \setmainfont[Path=/usr/share/fonts/truetype/cmu/,UprightFont=cmunrm.ttf,BoldFont=cmunbx.ttf,ItalicFont=cmunti.ttf,BoldItalicFont=cmunbi.ttf]{cmuntt.ttf}\setmonofont[Path=/usr/share/fonts/truetype/cmu/,UprightFont=cmuntt.ttf,BoldFont=cmuntb.ttf,ItalicFont=cmunit.ttf,BoldItalicFont=cmuntx.ttf]{cmuntt.ttf}\ttfamily \textbackslash{}mu}&\multicolumn{1}{|c|}{}&\hspace*{0pt}\ignorespaces{}\hspace*{0pt}{$\text{ }$}\setmainfont[Path=/usr/share/fonts/truetype/cmu/,UprightFont=cmunrm.ttf,BoldFont=cmunbx.ttf,ItalicFont=cmunti.ttf,BoldItalicFont=cmunbi.ttf]{cmunrm.ttf}\setmonofont[Path=/usr/share/fonts/truetype/cmu/,UprightFont=cmuntt.ttf,BoldFont=cmuntb.ttf,ItalicFont=cmunit.ttf,BoldItalicFont=cmuntx.ttf]{cmunrm.ttf} {$\Omega\,$} and {$\omega\,$}&\hspace*{0pt}\ignorespaces{}\hspace*{0pt} {\ttfamily \setmainfont[Path=/usr/share/fonts/truetype/cmu/,UprightFont=cmunrm.ttf,BoldFont=cmunbx.ttf,ItalicFont=cmunti.ttf,BoldItalicFont=cmunbi.ttf]{cmuntt.ttf}\setmonofont[Path=/usr/share/fonts/truetype/cmu/,UprightFont=cmuntt.ttf,BoldFont=cmuntb.ttf,ItalicFont=cmunit.ttf,BoldItalicFont=cmuntx.ttf]{cmuntt.ttf}\ttfamily \textbackslash{}Omega}{$\text{ }$}\setmainfont[Path=/usr/share/fonts/truetype/cmu/,UprightFont=cmunrm.ttf,BoldFont=cmunbx.ttf,ItalicFont=cmunti.ttf,BoldItalicFont=cmunbi.ttf]{cmunrm.ttf}\setmonofont[Path=/usr/share/fonts/truetype/cmu/,UprightFont=cmuntt.ttf,BoldFont=cmuntb.ttf,ItalicFont=cmunit.ttf,BoldItalicFont=cmuntx.ttf]{cmunrm.ttf} and {\ttfamily \setmainfont[Path=/usr/share/fonts/truetype/cmu/,UprightFont=cmunrm.ttf,BoldFont=cmunbx.ttf,ItalicFont=cmunti.ttf,BoldItalicFont=cmunbi.ttf]{cmuntt.ttf}\setmonofont[Path=/usr/share/fonts/truetype/cmu/,UprightFont=cmuntt.ttf,BoldFont=cmuntb.ttf,ItalicFont=cmunit.ttf,BoldItalicFont=cmuntx.ttf]{cmuntt.ttf}\ttfamily \textbackslash{}omega}\\ \hline 
\end{longtable}
}\setmainfont[Path=/usr/share/fonts/truetype/cmu/,UprightFont=cmunrm.ttf,BoldFont=cmunbx.ttf,ItalicFont=cmunti.ttf,BoldItalicFont=cmunbi.ttf]{cmunrm.ttf}\setmonofont[Path=/usr/share/fonts/truetype/cmu/,UprightFont=cmuntt.ttf,BoldFont=cmuntb.ttf,ItalicFont=cmunit.ttf,BoldItalicFont=cmuntx.ttf]{cmunrm.ttf}
{\scriptsize{}
{\scalefont{0.65311}\begin{longtable}{|>{\RaggedRight}p{0.06518\linewidth}|>{\RaggedRight}p{0.05199\linewidth}|>{\RaggedRight}p{0.01975\linewidth}|>{\RaggedRight}p{0.06518\linewidth}|>{\RaggedRight}p{0.05199\linewidth}|>{\RaggedRight}p{0.01975\linewidth}|>{\RaggedRight}p{0.06518\linewidth}|>{\RaggedRight}p{0.05630\linewidth}|>{\RaggedRight}p{0.01975\linewidth}|>{\RaggedRight}p{0.06518\linewidth}|>{\RaggedRight}p{0.05199\linewidth}|>{\RaggedRight}p{0.01975\linewidth}|>{\RaggedRight}p{0.06518\linewidth}|>{\RaggedRight}p{0.06858\linewidth}|} \hline 
\multicolumn{14}{|>{\RaggedRight}p{0.97143\linewidth}|}{{\bfseries \hspace*{0pt}\ignorespaces{}\hspace*{0pt} Other symbols}}\\ \hline {\bfseries \hspace*{0pt}\ignorespaces{}\hspace*{0pt} Symbol }&{\bfseries \hspace*{0pt}\ignorespaces{}\hspace*{0pt} Script}&\multirow{4}{\linewidth}{\hspace*{0pt}\ignorespaces{}\hspace*{0pt} {\mbox{$~$}}}&{\bfseries \hspace*{0pt}\ignorespaces{}\hspace*{0pt} Symbol }&{\bfseries \hspace*{0pt}\ignorespaces{}\hspace*{0pt} Script}&\multirow{4}{\linewidth}{\hspace*{0pt}\ignorespaces{}\hspace*{0pt} {\mbox{$~$}}}&{\bfseries \hspace*{0pt}\ignorespaces{}\hspace*{0pt} Symbol }&{\bfseries \hspace*{0pt}\ignorespaces{}\hspace*{0pt} Script}&\multirow{4}{\linewidth}{\hspace*{0pt}\ignorespaces{}\hspace*{0pt} {\mbox{$~$}}}&{\bfseries \hspace*{0pt}\ignorespaces{}\hspace*{0pt} Symbol }&{\bfseries \hspace*{0pt}\ignorespaces{}\hspace*{0pt} Script}&\multirow{4}{\linewidth}{\hspace*{0pt}\ignorespaces{}\hspace*{0pt} {\mbox{$~$}}}&{\bfseries \hspace*{0pt}\ignorespaces{}\hspace*{0pt} Symbol }&{\bfseries \hspace*{0pt}\ignorespaces{}\hspace*{0pt} Script}\\ \cline{1-1}\cline{2-2}\cline{4-4}\cline{5-5}\cline{7-7}\cline{8-8}\cline{10-10}\cline{11-11}\cline{13-13}\cline{14-14} \hspace*{0pt}\ignorespaces{}\hspace*{0pt} {$\partial\,$} &\hspace*{0pt}\ignorespaces{}\hspace*{0pt} {\ttfamily \setmainfont[Path=/usr/share/fonts/truetype/cmu/,UprightFont=cmunrm.ttf,BoldFont=cmunbx.ttf,ItalicFont=cmunti.ttf,BoldItalicFont=cmunbi.ttf]{cmuntt.ttf}\setmonofont[Path=/usr/share/fonts/truetype/cmu/,UprightFont=cmuntt.ttf,BoldFont=cmuntb.ttf,ItalicFont=cmunit.ttf,BoldItalicFont=cmuntx.ttf]{cmuntt.ttf}\ttfamily \textbackslash{}partial}&\multicolumn{1}{|c|}{}&\hspace*{0pt}\ignorespaces{}\hspace*{0pt}{$\text{ }$}\setmainfont[Path=/usr/share/fonts/truetype/cmu/,UprightFont=cmunrm.ttf,BoldFont=cmunbx.ttf,ItalicFont=cmunti.ttf,BoldItalicFont=cmunbi.ttf]{cmunrm.ttf}\setmonofont[Path=/usr/share/fonts/truetype/cmu/,UprightFont=cmuntt.ttf,BoldFont=cmuntb.ttf,ItalicFont=cmunit.ttf,BoldItalicFont=cmuntx.ttf]{cmunrm.ttf} {$\imath\,$} &\hspace*{0pt}\ignorespaces{}\hspace*{0pt} {\ttfamily \setmainfont[Path=/usr/share/fonts/truetype/cmu/,UprightFont=cmunrm.ttf,BoldFont=cmunbx.ttf,ItalicFont=cmunti.ttf,BoldItalicFont=cmunbi.ttf]{cmuntt.ttf}\setmonofont[Path=/usr/share/fonts/truetype/cmu/,UprightFont=cmuntt.ttf,BoldFont=cmuntb.ttf,ItalicFont=cmunit.ttf,BoldItalicFont=cmuntx.ttf]{cmuntt.ttf}\ttfamily \textbackslash{}imath}&\multicolumn{1}{|c|}{}&\hspace*{0pt}\ignorespaces{}\hspace*{0pt}{$\text{ }$}\setmainfont[Path=/usr/share/fonts/truetype/cmu/,UprightFont=cmunrm.ttf,BoldFont=cmunbx.ttf,ItalicFont=cmunti.ttf,BoldItalicFont=cmunbi.ttf]{cmunrm.ttf}\setmonofont[Path=/usr/share/fonts/truetype/cmu/,UprightFont=cmuntt.ttf,BoldFont=cmuntb.ttf,ItalicFont=cmunit.ttf,BoldItalicFont=cmuntx.ttf]{cmunrm.ttf} {$\Re\,$} &\hspace*{0pt}\ignorespaces{}\hspace*{0pt} {\ttfamily \setmainfont[Path=/usr/share/fonts/truetype/cmu/,UprightFont=cmunrm.ttf,BoldFont=cmunbx.ttf,ItalicFont=cmunti.ttf,BoldItalicFont=cmunbi.ttf]{cmuntt.ttf}\setmonofont[Path=/usr/share/fonts/truetype/cmu/,UprightFont=cmuntt.ttf,BoldFont=cmuntb.ttf,ItalicFont=cmunit.ttf,BoldItalicFont=cmuntx.ttf]{cmuntt.ttf}\ttfamily \textbackslash{}Re}&\multicolumn{1}{|c|}{}&\hspace*{0pt}\ignorespaces{}\hspace*{0pt}{$\text{ }$}\setmainfont[Path=/usr/share/fonts/truetype/cmu/,UprightFont=cmunrm.ttf,BoldFont=cmunbx.ttf,ItalicFont=cmunti.ttf,BoldItalicFont=cmunbi.ttf]{cmunrm.ttf}\setmonofont[Path=/usr/share/fonts/truetype/cmu/,UprightFont=cmuntt.ttf,BoldFont=cmuntb.ttf,ItalicFont=cmunit.ttf,BoldItalicFont=cmuntx.ttf]{cmunrm.ttf} {$\nabla\,$} &\hspace*{0pt}\ignorespaces{}\hspace*{0pt} {\ttfamily \setmainfont[Path=/usr/share/fonts/truetype/cmu/,UprightFont=cmunrm.ttf,BoldFont=cmunbx.ttf,ItalicFont=cmunti.ttf,BoldItalicFont=cmunbi.ttf]{cmuntt.ttf}\setmonofont[Path=/usr/share/fonts/truetype/cmu/,UprightFont=cmuntt.ttf,BoldFont=cmuntb.ttf,ItalicFont=cmunit.ttf,BoldItalicFont=cmuntx.ttf]{cmuntt.ttf}\ttfamily \textbackslash{}nabla}&\multicolumn{1}{|c|}{}&\hspace*{0pt}\ignorespaces{}\hspace*{0pt}{$\text{ }$}\setmainfont[Path=/usr/share/fonts/truetype/cmu/,UprightFont=cmunrm.ttf,BoldFont=cmunbx.ttf,ItalicFont=cmunti.ttf,BoldItalicFont=cmunbi.ttf]{cmunrm.ttf}\setmonofont[Path=/usr/share/fonts/truetype/cmu/,UprightFont=cmuntt.ttf,BoldFont=cmuntb.ttf,ItalicFont=cmunit.ttf,BoldItalicFont=cmuntx.ttf]{cmunrm.ttf} {$\aleph\,$} &\hspace*{0pt}\ignorespaces{}\hspace*{0pt} {\ttfamily \setmainfont[Path=/usr/share/fonts/truetype/cmu/,UprightFont=cmunrm.ttf,BoldFont=cmunbx.ttf,ItalicFont=cmunti.ttf,BoldItalicFont=cmunbi.ttf]{cmuntt.ttf}\setmonofont[Path=/usr/share/fonts/truetype/cmu/,UprightFont=cmuntt.ttf,BoldFont=cmuntb.ttf,ItalicFont=cmunit.ttf,BoldItalicFont=cmuntx.ttf]{cmuntt.ttf}\ttfamily \textbackslash{}aleph}\\ \cline{1-1}\cline{2-2}\cline{4-4}\cline{5-5}\cline{7-7}\cline{8-8}\cline{10-10}\cline{11-11}\cline{13-13}\cline{14-14} \hspace*{0pt}\ignorespaces{}\hspace*{0pt}{$\text{ }$}\setmainfont[Path=/usr/share/fonts/truetype/cmu/,UprightFont=cmunrm.ttf,BoldFont=cmunbx.ttf,ItalicFont=cmunti.ttf,BoldItalicFont=cmunbi.ttf]{cmunrm.ttf}\setmonofont[Path=/usr/share/fonts/truetype/cmu/,UprightFont=cmuntt.ttf,BoldFont=cmuntb.ttf,ItalicFont=cmunit.ttf,BoldItalicFont=cmuntx.ttf]{cmunrm.ttf} {$\eth\,$} &\hspace*{0pt}\ignorespaces{}\hspace*{0pt} {\ttfamily \setmainfont[Path=/usr/share/fonts/truetype/cmu/,UprightFont=cmunrm.ttf,BoldFont=cmunbx.ttf,ItalicFont=cmunti.ttf,BoldItalicFont=cmunbi.ttf]{cmuntt.ttf}\setmonofont[Path=/usr/share/fonts/truetype/cmu/,UprightFont=cmuntt.ttf,BoldFont=cmuntb.ttf,ItalicFont=cmunit.ttf,BoldItalicFont=cmuntx.ttf]{cmuntt.ttf}\ttfamily \textbackslash{}eth}&\multicolumn{1}{|c|}{}&\hspace*{0pt}\ignorespaces{}\hspace*{0pt}{$\text{ }$}\setmainfont[Path=/usr/share/fonts/truetype/cmu/,UprightFont=cmunrm.ttf,BoldFont=cmunbx.ttf,ItalicFont=cmunti.ttf,BoldItalicFont=cmunbi.ttf]{cmunrm.ttf}\setmonofont[Path=/usr/share/fonts/truetype/cmu/,UprightFont=cmuntt.ttf,BoldFont=cmuntb.ttf,ItalicFont=cmunit.ttf,BoldItalicFont=cmuntx.ttf]{cmunrm.ttf} {$\jmath\,$} &\hspace*{0pt}\ignorespaces{}\hspace*{0pt} {\ttfamily \setmainfont[Path=/usr/share/fonts/truetype/cmu/,UprightFont=cmunrm.ttf,BoldFont=cmunbx.ttf,ItalicFont=cmunti.ttf,BoldItalicFont=cmunbi.ttf]{cmuntt.ttf}\setmonofont[Path=/usr/share/fonts/truetype/cmu/,UprightFont=cmuntt.ttf,BoldFont=cmuntb.ttf,ItalicFont=cmunit.ttf,BoldItalicFont=cmuntx.ttf]{cmuntt.ttf}\ttfamily \textbackslash{}jmath}&\multicolumn{1}{|c|}{}&\hspace*{0pt}\ignorespaces{}\hspace*{0pt}{$\text{ }$}\setmainfont[Path=/usr/share/fonts/truetype/cmu/,UprightFont=cmunrm.ttf,BoldFont=cmunbx.ttf,ItalicFont=cmunti.ttf,BoldItalicFont=cmunbi.ttf]{cmunrm.ttf}\setmonofont[Path=/usr/share/fonts/truetype/cmu/,UprightFont=cmuntt.ttf,BoldFont=cmuntb.ttf,ItalicFont=cmunit.ttf,BoldItalicFont=cmuntx.ttf]{cmunrm.ttf} {$\Im\,$} &\hspace*{0pt}\ignorespaces{}\hspace*{0pt} {\ttfamily \setmainfont[Path=/usr/share/fonts/truetype/cmu/,UprightFont=cmunrm.ttf,BoldFont=cmunbx.ttf,ItalicFont=cmunti.ttf,BoldItalicFont=cmunbi.ttf]{cmuntt.ttf}\setmonofont[Path=/usr/share/fonts/truetype/cmu/,UprightFont=cmuntt.ttf,BoldFont=cmuntb.ttf,ItalicFont=cmunit.ttf,BoldItalicFont=cmuntx.ttf]{cmuntt.ttf}\ttfamily \textbackslash{}Im}&\multicolumn{1}{|c|}{}&\hspace*{0pt}\ignorespaces{}\hspace*{0pt}{$\text{ }$}\setmainfont[Path=/usr/share/fonts/truetype/cmu/,UprightFont=cmunrm.ttf,BoldFont=cmunbx.ttf,ItalicFont=cmunti.ttf,BoldItalicFont=cmunbi.ttf]{cmunrm.ttf}\setmonofont[Path=/usr/share/fonts/truetype/cmu/,UprightFont=cmuntt.ttf,BoldFont=cmuntb.ttf,ItalicFont=cmunit.ttf,BoldItalicFont=cmuntx.ttf]{cmunrm.ttf} {$\Box\,$} &\hspace*{0pt}\ignorespaces{}\hspace*{0pt} {\ttfamily \setmainfont[Path=/usr/share/fonts/truetype/cmu/,UprightFont=cmunrm.ttf,BoldFont=cmunbx.ttf,ItalicFont=cmunti.ttf,BoldItalicFont=cmunbi.ttf]{cmuntt.ttf}\setmonofont[Path=/usr/share/fonts/truetype/cmu/,UprightFont=cmuntt.ttf,BoldFont=cmuntb.ttf,ItalicFont=cmunit.ttf,BoldItalicFont=cmuntx.ttf]{cmuntt.ttf}\ttfamily \textbackslash{}Box}&\multicolumn{1}{|c|}{}&\hspace*{0pt}\ignorespaces{}\hspace*{0pt}{$\text{ }$}\setmainfont[Path=/usr/share/fonts/truetype/cmu/,UprightFont=cmunrm.ttf,BoldFont=cmunbx.ttf,ItalicFont=cmunti.ttf,BoldItalicFont=cmunbi.ttf]{cmunrm.ttf}\setmonofont[Path=/usr/share/fonts/truetype/cmu/,UprightFont=cmuntt.ttf,BoldFont=cmuntb.ttf,ItalicFont=cmunit.ttf,BoldItalicFont=cmuntx.ttf]{cmunrm.ttf} {$\beth\,$} &\hspace*{0pt}\ignorespaces{}\hspace*{0pt} {\ttfamily \setmainfont[Path=/usr/share/fonts/truetype/cmu/,UprightFont=cmunrm.ttf,BoldFont=cmunbx.ttf,ItalicFont=cmunti.ttf,BoldItalicFont=cmunbi.ttf]{cmuntt.ttf}\setmonofont[Path=/usr/share/fonts/truetype/cmu/,UprightFont=cmuntt.ttf,BoldFont=cmuntb.ttf,ItalicFont=cmunit.ttf,BoldItalicFont=cmuntx.ttf]{cmuntt.ttf}\ttfamily \textbackslash{}beth}\\ \cline{1-1}\cline{2-2}\cline{4-4}\cline{5-5}\cline{7-7}\cline{8-8}\cline{10-10}\cline{11-11}\cline{13-13}\cline{14-14} \hspace*{0pt}\ignorespaces{}\hspace*{0pt}{$\text{ }$}\setmainfont[Path=/usr/share/fonts/truetype/cmu/,UprightFont=cmunrm.ttf,BoldFont=cmunbx.ttf,ItalicFont=cmunti.ttf,BoldItalicFont=cmunbi.ttf]{cmunrm.ttf}\setmonofont[Path=/usr/share/fonts/truetype/cmu/,UprightFont=cmuntt.ttf,BoldFont=cmuntb.ttf,ItalicFont=cmunit.ttf,BoldItalicFont=cmuntx.ttf]{cmunrm.ttf} {$\hbar\,$} &\hspace*{0pt}\ignorespaces{}\hspace*{0pt} {\ttfamily \setmainfont[Path=/usr/share/fonts/truetype/cmu/,UprightFont=cmunrm.ttf,BoldFont=cmunbx.ttf,ItalicFont=cmunti.ttf,BoldItalicFont=cmunbi.ttf]{cmuntt.ttf}\setmonofont[Path=/usr/share/fonts/truetype/cmu/,UprightFont=cmuntt.ttf,BoldFont=cmuntb.ttf,ItalicFont=cmunit.ttf,BoldItalicFont=cmuntx.ttf]{cmuntt.ttf}\ttfamily \textbackslash{}hbar}&\multicolumn{1}{|c|}{}&\hspace*{0pt}\ignorespaces{}\hspace*{0pt}{$\text{ }$}\setmainfont[Path=/usr/share/fonts/truetype/cmu/,UprightFont=cmunrm.ttf,BoldFont=cmunbx.ttf,ItalicFont=cmunti.ttf,BoldItalicFont=cmunbi.ttf]{cmunrm.ttf}\setmonofont[Path=/usr/share/fonts/truetype/cmu/,UprightFont=cmuntt.ttf,BoldFont=cmuntb.ttf,ItalicFont=cmunit.ttf,BoldItalicFont=cmuntx.ttf]{cmunrm.ttf} {$\ell\,$} &\hspace*{0pt}\ignorespaces{}\hspace*{0pt} {\ttfamily \setmainfont[Path=/usr/share/fonts/truetype/cmu/,UprightFont=cmunrm.ttf,BoldFont=cmunbx.ttf,ItalicFont=cmunti.ttf,BoldItalicFont=cmunbi.ttf]{cmuntt.ttf}\setmonofont[Path=/usr/share/fonts/truetype/cmu/,UprightFont=cmuntt.ttf,BoldFont=cmuntb.ttf,ItalicFont=cmunit.ttf,BoldItalicFont=cmuntx.ttf]{cmuntt.ttf}\ttfamily \textbackslash{}ell}&\multicolumn{1}{|c|}{}&\hspace*{0pt}\ignorespaces{}\hspace*{0pt}{$\text{ }$}\setmainfont[Path=/usr/share/fonts/truetype/cmu/,UprightFont=cmunrm.ttf,BoldFont=cmunbx.ttf,ItalicFont=cmunti.ttf,BoldItalicFont=cmunbi.ttf]{cmunrm.ttf}\setmonofont[Path=/usr/share/fonts/truetype/cmu/,UprightFont=cmuntt.ttf,BoldFont=cmuntb.ttf,ItalicFont=cmunit.ttf,BoldItalicFont=cmuntx.ttf]{cmunrm.ttf} {$\wp\,$} &\hspace*{0pt}\ignorespaces{}\hspace*{0pt} {\ttfamily \setmainfont[Path=/usr/share/fonts/truetype/cmu/,UprightFont=cmunrm.ttf,BoldFont=cmunbx.ttf,ItalicFont=cmunti.ttf,BoldItalicFont=cmunbi.ttf]{cmuntt.ttf}\setmonofont[Path=/usr/share/fonts/truetype/cmu/,UprightFont=cmuntt.ttf,BoldFont=cmuntb.ttf,ItalicFont=cmunit.ttf,BoldItalicFont=cmuntx.ttf]{cmuntt.ttf}\ttfamily \textbackslash{}wp}&\multicolumn{1}{|c|}{}&\hspace*{0pt}\ignorespaces{}\hspace*{0pt}{$\text{ }$}\setmainfont[Path=/usr/share/fonts/truetype/cmu/,UprightFont=cmunrm.ttf,BoldFont=cmunbx.ttf,ItalicFont=cmunti.ttf,BoldItalicFont=cmunbi.ttf]{cmunrm.ttf}\setmonofont[Path=/usr/share/fonts/truetype/cmu/,UprightFont=cmuntt.ttf,BoldFont=cmuntb.ttf,ItalicFont=cmunit.ttf,BoldItalicFont=cmuntx.ttf]{cmunrm.ttf} {$\infty\,$} &\hspace*{0pt}\ignorespaces{}\hspace*{0pt} {\ttfamily \setmainfont[Path=/usr/share/fonts/truetype/cmu/,UprightFont=cmunrm.ttf,BoldFont=cmunbx.ttf,ItalicFont=cmunti.ttf,BoldItalicFont=cmunbi.ttf]{cmuntt.ttf}\setmonofont[Path=/usr/share/fonts/truetype/cmu/,UprightFont=cmuntt.ttf,BoldFont=cmuntb.ttf,ItalicFont=cmunit.ttf,BoldItalicFont=cmuntx.ttf]{cmuntt.ttf}\ttfamily \textbackslash{}infty}&\multicolumn{1}{|c|}{}&\hspace*{0pt}\ignorespaces{}\hspace*{0pt}{$\text{ }$}\setmainfont[Path=/usr/share/fonts/truetype/cmu/,UprightFont=cmunrm.ttf,BoldFont=cmunbx.ttf,ItalicFont=cmunti.ttf,BoldItalicFont=cmunbi.ttf]{cmunrm.ttf}\setmonofont[Path=/usr/share/fonts/truetype/cmu/,UprightFont=cmuntt.ttf,BoldFont=cmuntb.ttf,ItalicFont=cmunit.ttf,BoldItalicFont=cmuntx.ttf]{cmunrm.ttf} {$\gimel\,$} &\hspace*{0pt}\ignorespaces{}\hspace*{0pt} {\ttfamily \setmainfont[Path=/usr/share/fonts/truetype/cmu/,UprightFont=cmunrm.ttf,BoldFont=cmunbx.ttf,ItalicFont=cmunti.ttf,BoldItalicFont=cmunbi.ttf]{cmuntt.ttf}\setmonofont[Path=/usr/share/fonts/truetype/cmu/,UprightFont=cmuntt.ttf,BoldFont=cmuntb.ttf,ItalicFont=cmunit.ttf,BoldItalicFont=cmuntx.ttf]{cmuntt.ttf}\ttfamily \textbackslash{}gimel}\\ \hline 
\end{longtable}
}}\setmainfont[Path=/usr/share/fonts/truetype/cmu/,UprightFont=cmunrm.ttf,BoldFont=cmunbx.ttf,ItalicFont=cmunti.ttf,BoldItalicFont=cmunbi.ttf]{cmunrm.ttf}\setmonofont[Path=/usr/share/fonts/truetype/cmu/,UprightFont=cmuntt.ttf,BoldFont=cmuntb.ttf,ItalicFont=cmunit.ttf,BoldItalicFont=cmuntx.ttf]{cmunrm.ttf}
{\scriptsize{}
{\scalefont{0.80877}\begin{longtable}{|>{\RaggedRight}p{0.06405\linewidth}|>{\RaggedRight}p{0.07658\linewidth}|>{\RaggedRight}p{0.02686\linewidth}|>{\RaggedRight}p{0.06405\linewidth}|>{\RaggedRight}p{0.06405\linewidth}|>{\RaggedRight}p{0.02686\linewidth}|>{\RaggedRight}p{0.08866\linewidth}|>{\RaggedRight}p{0.08248\linewidth}|>{\RaggedRight}p{0.02686\linewidth}|>{\RaggedRight}p{0.08866\linewidth}|>{\RaggedRight}p{0.07658\linewidth}|} \hline 
\multicolumn{11}{|>{\RaggedRight}p{0.97143\linewidth}|}{{\bfseries \hspace*{0pt}\ignorespaces{}\hspace*{0pt}Trigonometric Functions}}\\ \hline {\bfseries \hspace*{0pt}\ignorespaces{}\hspace*{0pt} Symbol }&{\bfseries \hspace*{0pt}\ignorespaces{}\hspace*{0pt} Script}&\multirow{5}{\linewidth}{\hspace*{0pt}\ignorespaces{}\hspace*{0pt} {\mbox{$~$}}}&{\bfseries \hspace*{0pt}\ignorespaces{}\hspace*{0pt} Symbol }&{\bfseries \hspace*{0pt}\ignorespaces{}\hspace*{0pt} Script}&\multirow{5}{\linewidth}{\hspace*{0pt}\ignorespaces{}\hspace*{0pt} {\mbox{$~$}}}&{\bfseries \hspace*{0pt}\ignorespaces{}\hspace*{0pt} Symbol }&{\bfseries \hspace*{0pt}\ignorespaces{}\hspace*{0pt} Script}&\multirow{5}{\linewidth}{\hspace*{0pt}\ignorespaces{}\hspace*{0pt} {\mbox{$~$}}}&{\bfseries \hspace*{0pt}\ignorespaces{}\hspace*{0pt} Symbol }&{\bfseries \hspace*{0pt}\ignorespaces{}\hspace*{0pt} Script}\\ \cline{1-1}\cline{2-2}\cline{4-4}\cline{5-5}\cline{7-7}\cline{8-8}\cline{10-10}\cline{11-11} \hspace*{0pt}\ignorespaces{}\hspace*{0pt} {$\sin\,$} &\hspace*{0pt}\ignorespaces{}\hspace*{0pt} {\ttfamily \setmainfont[Path=/usr/share/fonts/truetype/cmu/,UprightFont=cmunrm.ttf,BoldFont=cmunbx.ttf,ItalicFont=cmunti.ttf,BoldItalicFont=cmunbi.ttf]{cmuntt.ttf}\setmonofont[Path=/usr/share/fonts/truetype/cmu/,UprightFont=cmuntt.ttf,BoldFont=cmuntb.ttf,ItalicFont=cmunit.ttf,BoldItalicFont=cmuntx.ttf]{cmuntt.ttf}\ttfamily \textbackslash{}sin}&\multicolumn{1}{|c|}{}&\hspace*{0pt}\ignorespaces{}\hspace*{0pt}{$\text{ }$}\setmainfont[Path=/usr/share/fonts/truetype/cmu/,UprightFont=cmunrm.ttf,BoldFont=cmunbx.ttf,ItalicFont=cmunti.ttf,BoldItalicFont=cmunbi.ttf]{cmunrm.ttf}\setmonofont[Path=/usr/share/fonts/truetype/cmu/,UprightFont=cmuntt.ttf,BoldFont=cmuntb.ttf,ItalicFont=cmunit.ttf,BoldItalicFont=cmuntx.ttf]{cmunrm.ttf} {$\arcsin\,$} &\hspace*{0pt}\ignorespaces{}\hspace*{0pt} {\ttfamily \setmainfont[Path=/usr/share/fonts/truetype/cmu/,UprightFont=cmunrm.ttf,BoldFont=cmunbx.ttf,ItalicFont=cmunti.ttf,BoldItalicFont=cmunbi.ttf]{cmuntt.ttf}\setmonofont[Path=/usr/share/fonts/truetype/cmu/,UprightFont=cmuntt.ttf,BoldFont=cmuntb.ttf,ItalicFont=cmunit.ttf,BoldItalicFont=cmuntx.ttf]{cmuntt.ttf}\ttfamily \textbackslash{}arcsin}&\multicolumn{1}{|c|}{}&\hspace*{0pt}\ignorespaces{}\hspace*{0pt}{$\text{ }$}\setmainfont[Path=/usr/share/fonts/truetype/cmu/,UprightFont=cmunrm.ttf,BoldFont=cmunbx.ttf,ItalicFont=cmunti.ttf,BoldItalicFont=cmunbi.ttf]{cmunrm.ttf}\setmonofont[Path=/usr/share/fonts/truetype/cmu/,UprightFont=cmuntt.ttf,BoldFont=cmuntb.ttf,ItalicFont=cmunit.ttf,BoldItalicFont=cmuntx.ttf]{cmunrm.ttf} {$\sinh\,$} &\hspace*{0pt}\ignorespaces{}\hspace*{0pt} {\ttfamily \setmainfont[Path=/usr/share/fonts/truetype/cmu/,UprightFont=cmunrm.ttf,BoldFont=cmunbx.ttf,ItalicFont=cmunti.ttf,BoldItalicFont=cmunbi.ttf]{cmuntt.ttf}\setmonofont[Path=/usr/share/fonts/truetype/cmu/,UprightFont=cmuntt.ttf,BoldFont=cmuntb.ttf,ItalicFont=cmunit.ttf,BoldItalicFont=cmuntx.ttf]{cmuntt.ttf}\ttfamily \textbackslash{}sinh}&\multicolumn{1}{|c|}{}&\hspace*{0pt}\ignorespaces{}\hspace*{0pt}{$\text{ }$}\setmainfont[Path=/usr/share/fonts/truetype/cmu/,UprightFont=cmunrm.ttf,BoldFont=cmunbx.ttf,ItalicFont=cmunti.ttf,BoldItalicFont=cmunbi.ttf]{cmunrm.ttf}\setmonofont[Path=/usr/share/fonts/truetype/cmu/,UprightFont=cmuntt.ttf,BoldFont=cmuntb.ttf,ItalicFont=cmunit.ttf,BoldItalicFont=cmuntx.ttf]{cmunrm.ttf} {$\sec\,$} &\hspace*{0pt}\ignorespaces{}\hspace*{0pt} {\ttfamily \setmainfont[Path=/usr/share/fonts/truetype/cmu/,UprightFont=cmunrm.ttf,BoldFont=cmunbx.ttf,ItalicFont=cmunti.ttf,BoldItalicFont=cmunbi.ttf]{cmuntt.ttf}\setmonofont[Path=/usr/share/fonts/truetype/cmu/,UprightFont=cmuntt.ttf,BoldFont=cmuntb.ttf,ItalicFont=cmunit.ttf,BoldItalicFont=cmuntx.ttf]{cmuntt.ttf}\ttfamily \textbackslash{}sec}\\ \cline{1-1}\cline{2-2}\cline{4-4}\cline{5-5}\cline{7-7}\cline{8-8}\cline{10-10}\cline{11-11} \hspace*{0pt}\ignorespaces{}\hspace*{0pt}{$\text{ }$}\setmainfont[Path=/usr/share/fonts/truetype/cmu/,UprightFont=cmunrm.ttf,BoldFont=cmunbx.ttf,ItalicFont=cmunti.ttf,BoldItalicFont=cmunbi.ttf]{cmunrm.ttf}\setmonofont[Path=/usr/share/fonts/truetype/cmu/,UprightFont=cmuntt.ttf,BoldFont=cmuntb.ttf,ItalicFont=cmunit.ttf,BoldItalicFont=cmuntx.ttf]{cmunrm.ttf} {$\cos\,$} &\hspace*{0pt}\ignorespaces{}\hspace*{0pt} {\ttfamily \setmainfont[Path=/usr/share/fonts/truetype/cmu/,UprightFont=cmunrm.ttf,BoldFont=cmunbx.ttf,ItalicFont=cmunti.ttf,BoldItalicFont=cmunbi.ttf]{cmuntt.ttf}\setmonofont[Path=/usr/share/fonts/truetype/cmu/,UprightFont=cmuntt.ttf,BoldFont=cmuntb.ttf,ItalicFont=cmunit.ttf,BoldItalicFont=cmuntx.ttf]{cmuntt.ttf}\ttfamily \textbackslash{}cos}&\multicolumn{1}{|c|}{}&\hspace*{0pt}\ignorespaces{}\hspace*{0pt}{$\text{ }$}\setmainfont[Path=/usr/share/fonts/truetype/cmu/,UprightFont=cmunrm.ttf,BoldFont=cmunbx.ttf,ItalicFont=cmunti.ttf,BoldItalicFont=cmunbi.ttf]{cmunrm.ttf}\setmonofont[Path=/usr/share/fonts/truetype/cmu/,UprightFont=cmuntt.ttf,BoldFont=cmuntb.ttf,ItalicFont=cmunit.ttf,BoldItalicFont=cmuntx.ttf]{cmunrm.ttf} {$\arccos\,$} &\hspace*{0pt}\ignorespaces{}\hspace*{0pt} {\ttfamily \setmainfont[Path=/usr/share/fonts/truetype/cmu/,UprightFont=cmunrm.ttf,BoldFont=cmunbx.ttf,ItalicFont=cmunti.ttf,BoldItalicFont=cmunbi.ttf]{cmuntt.ttf}\setmonofont[Path=/usr/share/fonts/truetype/cmu/,UprightFont=cmuntt.ttf,BoldFont=cmuntb.ttf,ItalicFont=cmunit.ttf,BoldItalicFont=cmuntx.ttf]{cmuntt.ttf}\ttfamily \textbackslash{}arccos}&\multicolumn{1}{|c|}{}&\hspace*{0pt}\ignorespaces{}\hspace*{0pt}{$\text{ }$}\setmainfont[Path=/usr/share/fonts/truetype/cmu/,UprightFont=cmunrm.ttf,BoldFont=cmunbx.ttf,ItalicFont=cmunti.ttf,BoldItalicFont=cmunbi.ttf]{cmunrm.ttf}\setmonofont[Path=/usr/share/fonts/truetype/cmu/,UprightFont=cmuntt.ttf,BoldFont=cmuntb.ttf,ItalicFont=cmunit.ttf,BoldItalicFont=cmuntx.ttf]{cmunrm.ttf} {$\cosh\,$} &\hspace*{0pt}\ignorespaces{}\hspace*{0pt} {\ttfamily \setmainfont[Path=/usr/share/fonts/truetype/cmu/,UprightFont=cmunrm.ttf,BoldFont=cmunbx.ttf,ItalicFont=cmunti.ttf,BoldItalicFont=cmunbi.ttf]{cmuntt.ttf}\setmonofont[Path=/usr/share/fonts/truetype/cmu/,UprightFont=cmuntt.ttf,BoldFont=cmuntb.ttf,ItalicFont=cmunit.ttf,BoldItalicFont=cmuntx.ttf]{cmuntt.ttf}\ttfamily \textbackslash{}cosh}&\multicolumn{1}{|c|}{}&\hspace*{0pt}\ignorespaces{}\hspace*{0pt}{$\text{ }$}\setmainfont[Path=/usr/share/fonts/truetype/cmu/,UprightFont=cmunrm.ttf,BoldFont=cmunbx.ttf,ItalicFont=cmunti.ttf,BoldItalicFont=cmunbi.ttf]{cmunrm.ttf}\setmonofont[Path=/usr/share/fonts/truetype/cmu/,UprightFont=cmuntt.ttf,BoldFont=cmuntb.ttf,ItalicFont=cmunit.ttf,BoldItalicFont=cmuntx.ttf]{cmunrm.ttf} {$\csc\,$} &\hspace*{0pt}\ignorespaces{}\hspace*{0pt} {\ttfamily \setmainfont[Path=/usr/share/fonts/truetype/cmu/,UprightFont=cmunrm.ttf,BoldFont=cmunbx.ttf,ItalicFont=cmunti.ttf,BoldItalicFont=cmunbi.ttf]{cmuntt.ttf}\setmonofont[Path=/usr/share/fonts/truetype/cmu/,UprightFont=cmuntt.ttf,BoldFont=cmuntb.ttf,ItalicFont=cmunit.ttf,BoldItalicFont=cmuntx.ttf]{cmuntt.ttf}\ttfamily \textbackslash{}csc}\\ \cline{1-1}\cline{2-2}\cline{4-4}\cline{5-5}\cline{7-7}\cline{8-8}\cline{10-10}\cline{11-11} \hspace*{0pt}\ignorespaces{}\hspace*{0pt}{$\text{ }$}\setmainfont[Path=/usr/share/fonts/truetype/cmu/,UprightFont=cmunrm.ttf,BoldFont=cmunbx.ttf,ItalicFont=cmunti.ttf,BoldItalicFont=cmunbi.ttf]{cmunrm.ttf}\setmonofont[Path=/usr/share/fonts/truetype/cmu/,UprightFont=cmuntt.ttf,BoldFont=cmuntb.ttf,ItalicFont=cmunit.ttf,BoldItalicFont=cmuntx.ttf]{cmunrm.ttf} {$\tan\,$} &\hspace*{0pt}\ignorespaces{}\hspace*{0pt} {\ttfamily \setmainfont[Path=/usr/share/fonts/truetype/cmu/,UprightFont=cmunrm.ttf,BoldFont=cmunbx.ttf,ItalicFont=cmunti.ttf,BoldItalicFont=cmunbi.ttf]{cmuntt.ttf}\setmonofont[Path=/usr/share/fonts/truetype/cmu/,UprightFont=cmuntt.ttf,BoldFont=cmuntb.ttf,ItalicFont=cmunit.ttf,BoldItalicFont=cmuntx.ttf]{cmuntt.ttf}\ttfamily \textbackslash{}tan}&\multicolumn{1}{|c|}{}&\hspace*{0pt}\ignorespaces{}\hspace*{0pt}{$\text{ }$}\setmainfont[Path=/usr/share/fonts/truetype/cmu/,UprightFont=cmunrm.ttf,BoldFont=cmunbx.ttf,ItalicFont=cmunti.ttf,BoldItalicFont=cmunbi.ttf]{cmunrm.ttf}\setmonofont[Path=/usr/share/fonts/truetype/cmu/,UprightFont=cmuntt.ttf,BoldFont=cmuntb.ttf,ItalicFont=cmunit.ttf,BoldItalicFont=cmuntx.ttf]{cmunrm.ttf} {$\arctan\,$} &\hspace*{0pt}\ignorespaces{}\hspace*{0pt} {\ttfamily \setmainfont[Path=/usr/share/fonts/truetype/cmu/,UprightFont=cmunrm.ttf,BoldFont=cmunbx.ttf,ItalicFont=cmunti.ttf,BoldItalicFont=cmunbi.ttf]{cmuntt.ttf}\setmonofont[Path=/usr/share/fonts/truetype/cmu/,UprightFont=cmuntt.ttf,BoldFont=cmuntb.ttf,ItalicFont=cmunit.ttf,BoldItalicFont=cmuntx.ttf]{cmuntt.ttf}\ttfamily \textbackslash{}arctan}&\multicolumn{1}{|c|}{}&\hspace*{0pt}\ignorespaces{}\hspace*{0pt}{$\text{ }$}\setmainfont[Path=/usr/share/fonts/truetype/cmu/,UprightFont=cmunrm.ttf,BoldFont=cmunbx.ttf,ItalicFont=cmunti.ttf,BoldItalicFont=cmunbi.ttf]{cmunrm.ttf}\setmonofont[Path=/usr/share/fonts/truetype/cmu/,UprightFont=cmuntt.ttf,BoldFont=cmuntb.ttf,ItalicFont=cmunit.ttf,BoldItalicFont=cmuntx.ttf]{cmunrm.ttf} {$\tanh\,$} &\hspace*{0pt}\ignorespaces{}\hspace*{0pt} {\ttfamily \setmainfont[Path=/usr/share/fonts/truetype/cmu/,UprightFont=cmunrm.ttf,BoldFont=cmunbx.ttf,ItalicFont=cmunti.ttf,BoldItalicFont=cmunbi.ttf]{cmuntt.ttf}\setmonofont[Path=/usr/share/fonts/truetype/cmu/,UprightFont=cmuntt.ttf,BoldFont=cmuntb.ttf,ItalicFont=cmunit.ttf,BoldItalicFont=cmuntx.ttf]{cmuntt.ttf}\ttfamily \textbackslash{}tanh}&\multicolumn{1}{|c|}{}&\hspace*{0pt}\ignorespaces{}\hspace*{0pt}{$\text{ }$}\setmainfont[Path=/usr/share/fonts/truetype/cmu/,UprightFont=cmunrm.ttf,BoldFont=cmunbx.ttf,ItalicFont=cmunti.ttf,BoldItalicFont=cmunbi.ttf]{cmunrm.ttf}\setmonofont[Path=/usr/share/fonts/truetype/cmu/,UprightFont=cmuntt.ttf,BoldFont=cmuntb.ttf,ItalicFont=cmunit.ttf,BoldItalicFont=cmuntx.ttf]{cmunrm.ttf} &\hspace*{0pt}\ignorespaces{}\hspace*{0pt}\\ \cline{1-1}\cline{2-2}\cline{4-4}\cline{5-5}\cline{7-7}\cline{8-8}\cline{10-10}\cline{11-11} \hspace*{0pt}\ignorespaces{}\hspace*{0pt} {$\cot\,$} &\hspace*{0pt}\ignorespaces{}\hspace*{0pt} {\ttfamily \setmainfont[Path=/usr/share/fonts/truetype/cmu/,UprightFont=cmunrm.ttf,BoldFont=cmunbx.ttf,ItalicFont=cmunti.ttf,BoldItalicFont=cmunbi.ttf]{cmuntt.ttf}\setmonofont[Path=/usr/share/fonts/truetype/cmu/,UprightFont=cmuntt.ttf,BoldFont=cmuntb.ttf,ItalicFont=cmunit.ttf,BoldItalicFont=cmuntx.ttf]{cmuntt.ttf}\ttfamily \textbackslash{}cot}&\multicolumn{1}{|c|}{}&\hspace*{0pt}\ignorespaces{}\hspace*{0pt}{$\text{ }$}\setmainfont[Path=/usr/share/fonts/truetype/cmu/,UprightFont=cmunrm.ttf,BoldFont=cmunbx.ttf,ItalicFont=cmunti.ttf,BoldItalicFont=cmunbi.ttf]{cmunrm.ttf}\setmonofont[Path=/usr/share/fonts/truetype/cmu/,UprightFont=cmuntt.ttf,BoldFont=cmuntb.ttf,ItalicFont=cmunit.ttf,BoldItalicFont=cmuntx.ttf]{cmunrm.ttf} {$\arccot\,$} &\hspace*{0pt}\ignorespaces{}\hspace*{0pt} {\ttfamily \setmainfont[Path=/usr/share/fonts/truetype/cmu/,UprightFont=cmunrm.ttf,BoldFont=cmunbx.ttf,ItalicFont=cmunti.ttf,BoldItalicFont=cmunbi.ttf]{cmuntt.ttf}\setmonofont[Path=/usr/share/fonts/truetype/cmu/,UprightFont=cmuntt.ttf,BoldFont=cmuntb.ttf,ItalicFont=cmunit.ttf,BoldItalicFont=cmuntx.ttf]{cmuntt.ttf}\ttfamily \textbackslash{}arccot}&\multicolumn{1}{|c|}{}&\hspace*{0pt}\ignorespaces{}\hspace*{0pt}{$\text{ }$}\setmainfont[Path=/usr/share/fonts/truetype/cmu/,UprightFont=cmunrm.ttf,BoldFont=cmunbx.ttf,ItalicFont=cmunti.ttf,BoldItalicFont=cmunbi.ttf]{cmunrm.ttf}\setmonofont[Path=/usr/share/fonts/truetype/cmu/,UprightFont=cmuntt.ttf,BoldFont=cmuntb.ttf,ItalicFont=cmunit.ttf,BoldItalicFont=cmuntx.ttf]{cmunrm.ttf} {$\coth\,$} &\hspace*{0pt}\ignorespaces{}\hspace*{0pt} {\ttfamily \setmainfont[Path=/usr/share/fonts/truetype/cmu/,UprightFont=cmunrm.ttf,BoldFont=cmunbx.ttf,ItalicFont=cmunti.ttf,BoldItalicFont=cmunbi.ttf]{cmuntt.ttf}\setmonofont[Path=/usr/share/fonts/truetype/cmu/,UprightFont=cmuntt.ttf,BoldFont=cmuntb.ttf,ItalicFont=cmunit.ttf,BoldItalicFont=cmuntx.ttf]{cmuntt.ttf}\ttfamily \textbackslash{}coth}&\multicolumn{1}{|c|}{}&\hspace*{0pt}\ignorespaces{}\hspace*{0pt}{$\text{ }$}\setmainfont[Path=/usr/share/fonts/truetype/cmu/,UprightFont=cmunrm.ttf,BoldFont=cmunbx.ttf,ItalicFont=cmunti.ttf,BoldItalicFont=cmunbi.ttf]{cmunrm.ttf}\setmonofont[Path=/usr/share/fonts/truetype/cmu/,UprightFont=cmuntt.ttf,BoldFont=cmuntb.ttf,ItalicFont=cmunit.ttf,BoldItalicFont=cmuntx.ttf]{cmunrm.ttf} &\hspace*{0pt}\ignorespaces{}\hspace*{0pt}\\ \hline 
\end{longtable}
}}

If LaTeX does not include a command for the mathematical operator you want to use, for example {\ttfamily \setmainfont[Path=/usr/share/fonts/truetype/cmu/,UprightFont=cmunrm.ttf,BoldFont=cmunbx.ttf,ItalicFont=cmunti.ttf,BoldItalicFont=cmunbi.ttf]{cmuntt.ttf}\setmonofont[Path=/usr/share/fonts/truetype/cmu/,UprightFont=cmuntt.ttf,BoldFont=cmuntb.ttf,ItalicFont=cmunit.ttf,BoldItalicFont=cmuntx.ttf]{cmuntt.ttf}\ttfamily \textbackslash{}cis}{$\text{ }$}\setmainfont[Path=/usr/share/fonts/truetype/cmu/,UprightFont=cmunrm.ttf,BoldFont=cmunbx.ttf,ItalicFont=cmunti.ttf,BoldItalicFont=cmunbi.ttf]{cmunrm.ttf}\setmonofont[Path=/usr/share/fonts/truetype/cmu/,UprightFont=cmuntt.ttf,BoldFont=cmuntb.ttf,ItalicFont=cmunit.ttf,BoldItalicFont=cmuntx.ttf]{cmunrm.ttf} ({\bfseries \setmainfont[Path=/usr/share/fonts/truetype/cmu/,UprightFont=cmunrm.ttf,BoldFont=cmunbx.ttf,ItalicFont=cmunti.ttf,BoldItalicFont=cmunbi.ttf]{cmunbx.ttf}\setmonofont[Path=/usr/share/fonts/truetype/cmu/,UprightFont=cmuntt.ttf,BoldFont=cmuntb.ttf,ItalicFont=cmunit.ttf,BoldItalicFont=cmuntx.ttf]{cmunbx.ttf}\bfseries c}\setmainfont[Path=/usr/share/fonts/truetype/cmu/,UprightFont=cmunrm.ttf,BoldFont=cmunbx.ttf,ItalicFont=cmunti.ttf,BoldItalicFont=cmunbi.ttf]{cmunrm.ttf}\setmonofont[Path=/usr/share/fonts/truetype/cmu/,UprightFont=cmuntt.ttf,BoldFont=cmuntb.ttf,ItalicFont=cmunit.ttf,BoldItalicFont=cmuntx.ttf]{cmunrm.ttf}osine plus {\bfseries \setmainfont[Path=/usr/share/fonts/truetype/cmu/,UprightFont=cmunrm.ttf,BoldFont=cmunbx.ttf,ItalicFont=cmunti.ttf,BoldItalicFont=cmunbi.ttf]{cmunbx.ttf}\setmonofont[Path=/usr/share/fonts/truetype/cmu/,UprightFont=cmuntt.ttf,BoldFont=cmuntb.ttf,ItalicFont=cmunit.ttf,BoldItalicFont=cmuntx.ttf]{cmunbx.ttf}\bfseries i}{$\text{ }$}\setmainfont[Path=/usr/share/fonts/truetype/cmu/,UprightFont=cmunrm.ttf,BoldFont=cmunbx.ttf,ItalicFont=cmunti.ttf,BoldItalicFont=cmunbi.ttf]{cmunrm.ttf}\setmonofont[Path=/usr/share/fonts/truetype/cmu/,UprightFont=cmuntt.ttf,BoldFont=cmuntb.ttf,ItalicFont=cmunit.ttf,BoldItalicFont=cmuntx.ttf]{cmunrm.ttf} times {\bfseries \setmainfont[Path=/usr/share/fonts/truetype/cmu/,UprightFont=cmunrm.ttf,BoldFont=cmunbx.ttf,ItalicFont=cmunti.ttf,BoldItalicFont=cmunbi.ttf]{cmunbx.ttf}\setmonofont[Path=/usr/share/fonts/truetype/cmu/,UprightFont=cmuntt.ttf,BoldFont=cmuntb.ttf,ItalicFont=cmunit.ttf,BoldItalicFont=cmuntx.ttf]{cmunbx.ttf}\bfseries s}\setmainfont[Path=/usr/share/fonts/truetype/cmu/,UprightFont=cmunrm.ttf,BoldFont=cmunbx.ttf,ItalicFont=cmunti.ttf,BoldItalicFont=cmunbi.ttf]{cmunrm.ttf}\setmonofont[Path=/usr/share/fonts/truetype/cmu/,UprightFont=cmuntt.ttf,BoldFont=cmuntb.ttf,ItalicFont=cmunit.ttf,BoldItalicFont=cmuntx.ttf]{cmunrm.ttf}ine), add to your preamble:\\

\TemplateSpaceIndent{$\text{ }${}\textbackslash{}DeclareMathOperator\textbackslash{}cis\{cis\}}


You can then use {\ttfamily \setmainfont[Path=/usr/share/fonts/truetype/cmu/,UprightFont=cmunrm.ttf,BoldFont=cmunbx.ttf,ItalicFont=cmunti.ttf,BoldItalicFont=cmunbi.ttf]{cmuntt.ttf}\setmonofont[Path=/usr/share/fonts/truetype/cmu/,UprightFont=cmuntt.ttf,BoldFont=cmuntb.ttf,ItalicFont=cmunit.ttf,BoldItalicFont=cmuntx.ttf]{cmuntt.ttf}\ttfamily \textbackslash{}cis}{$\text{ }$}\setmainfont[Path=/usr/share/fonts/truetype/cmu/,UprightFont=cmunrm.ttf,BoldFont=cmunbx.ttf,ItalicFont=cmunti.ttf,BoldItalicFont=cmunbi.ttf]{cmunrm.ttf}\setmonofont[Path=/usr/share/fonts/truetype/cmu/,UprightFont=cmuntt.ttf,BoldFont=cmuntb.ttf,ItalicFont=cmunit.ttf,BoldItalicFont=cmuntx.ttf]{cmunrm.ttf} in the document just like {\ttfamily \setmainfont[Path=/usr/share/fonts/truetype/cmu/,UprightFont=cmunrm.ttf,BoldFont=cmunbx.ttf,ItalicFont=cmunti.ttf,BoldItalicFont=cmunbi.ttf]{cmuntt.ttf}\setmonofont[Path=/usr/share/fonts/truetype/cmu/,UprightFont=cmuntt.ttf,BoldFont=cmuntb.ttf,ItalicFont=cmunit.ttf,BoldItalicFont=cmuntx.ttf]{cmuntt.ttf}\ttfamily \textbackslash{}cos}{$\text{ }$}\setmainfont[Path=/usr/share/fonts/truetype/cmu/,UprightFont=cmunrm.ttf,BoldFont=cmunbx.ttf,ItalicFont=cmunti.ttf,BoldItalicFont=cmunbi.ttf]{cmunrm.ttf}\setmonofont[Path=/usr/share/fonts/truetype/cmu/,UprightFont=cmuntt.ttf,BoldFont=cmuntb.ttf,ItalicFont=cmunit.ttf,BoldItalicFont=cmuntx.ttf]{cmunrm.ttf} or any other mathematical operator.



\LaTeXNullTemplate{}
\section{Summary}
\label{528}

As you begin to see, typesetting math can be tricky at times. However, because LaTeX provides so much control, you can get professional quality mathematics typesetting with relatively little effort (once you\textquotesingle{}ve had a bit of practice, of course!). It would be possible to keep going and going with math topics because it seems potentially limitless. However, with this tutorial, you should be able to get along sufficiently.

\LaTeXNullTemplate{}
\section{Notes}
\label{529}

\section{Further reading}
\label{530}
\begin{myitemize}
\item{}  \myhref{https://en.meta.org/wiki/Help\%3ADisplaying\%20a\%20formula}{meta:Help:Displaying a formula}: Wikimedia uses a subset of LaTeX commands.
\end{myitemize}

\section{External links}
\label{531}

\begin{myitemize}
\item{}  \myhref{http://www.artofproblemsolving.com/Wiki/index.php/LaTeX:Symbols}{LaTeX maths symbols}
\item{}  \myhref{http://detexify.kirelabs.org}{detexify}: applet for looking up LaTeX symbols by drawing them
\item{}  {$\text{[}$}ftp://ftp.ams.org/pub/tex/doc/amsmath/amsldoc.pdf {\ttfamily \setmainfont[Path=/usr/share/fonts/truetype/cmu/,UprightFont=cmunrm.ttf,BoldFont=cmunbx.ttf,ItalicFont=cmunti.ttf,BoldItalicFont=cmunbi.ttf]{cmuntt.ttf}\setmonofont[Path=/usr/share/fonts/truetype/cmu/,UprightFont=cmuntt.ttf,BoldFont=cmuntb.ttf,ItalicFont=cmunit.ttf,BoldItalicFont=cmuntx.ttf]{cmuntt.ttf}\ttfamily amsmath}{$\text{ }$}\setmainfont[Path=/usr/share/fonts/truetype/cmu/,UprightFont=cmunrm.ttf,BoldFont=cmunbx.ttf,ItalicFont=cmunti.ttf,BoldItalicFont=cmunbi.ttf]{cmunrm.ttf}\setmonofont[Path=/usr/share/fonts/truetype/cmu/,UprightFont=cmuntt.ttf,BoldFont=cmuntb.ttf,ItalicFont=cmunit.ttf,BoldItalicFont=cmuntx.ttf]{cmunrm.ttf} documentation{$\text{]}$}
\item{}  \myhref{http://www.thestudentroom.co.uk/wiki/LaTeX}{LaTeX -{} The Student Room}
\item{}  \myhref{http://www.ctan.org/tex-archive/info/symbols/comprehensive}{The Comprehensive LaTeX Symbol List}
\item{}  \myhref{http://mathlex.org/latex}{MathLex -{} LaTeX math translator and equation builder}
\end{myitemize}




\myhref{https://pl.wikibooks.org/wiki/LaTeX\%2FMatematyka}{pl:LaTeX/Matematyka}\chapter{Advanced Mathematics}

\myminitoc
\label{532}

\label{533}

This page outlines some more advanced uses of mathematics markup using LaTeX. In particular it makes heavy use of the AMS-{}LaTeX packages supplied by the \myhref{https://en.wikipedia.org/wiki/American\%20Mathematical\%20Society}{American Mathematical Society}.
\section{Equation numbering}
\label{534}
The {\ttfamily \setmainfont[Path=/usr/share/fonts/truetype/cmu/,UprightFont=cmunrm.ttf,BoldFont=cmunbx.ttf,ItalicFont=cmunti.ttf,BoldItalicFont=cmunbi.ttf]{cmuntt.ttf}\setmonofont[Path=/usr/share/fonts/truetype/cmu/,UprightFont=cmuntt.ttf,BoldFont=cmuntb.ttf,ItalicFont=cmunit.ttf,BoldItalicFont=cmuntx.ttf]{cmuntt.ttf}\ttfamily equation}{$\text{ }$}\setmainfont[Path=/usr/share/fonts/truetype/cmu/,UprightFont=cmunrm.ttf,BoldFont=cmunbx.ttf,ItalicFont=cmunti.ttf,BoldItalicFont=cmunbi.ttf]{cmunrm.ttf}\setmonofont[Path=/usr/share/fonts/truetype/cmu/,UprightFont=cmuntt.ttf,BoldFont=cmuntb.ttf,ItalicFont=cmunit.ttf,BoldItalicFont=cmuntx.ttf]{cmunrm.ttf} environment automatically numbers your equation:
\begin{longtable}{p{1.0\linewidth}}
\begin{Shaded}
\begin{Highlighting}[]
\NormalTok{\textbackslash{}begin\{equation\} }
 \NormalTok{f(x)=(x+a)(x+b)}
\NormalTok{\textbackslash{}end\{equation\}}
\end{Highlighting}
\end{Shaded}
\\
{$ {f(x)}=(x+a)(x+b) {\color{White} ww} (1) \, $}
\end{longtable}

You can also use the \LaTeXTT{\textbackslash{}label} and \LaTeXTT{\textbackslash{}ref} (or \LaTeXTT{\textbackslash{}eqref} from the \LaTeXTT{amsmath} package) commands to label and reference equations, respectively. For equation number 1, \LaTeXTT{\textbackslash{}ref} results in {$1\,$} and \LaTeXTT{\textbackslash{}eqref} results in {$(1)\,$}:
\begin{longtable}{p{1.0\linewidth}}
\begin{Shaded}
\begin{Highlighting}[]

\NormalTok{\textbackslash{}begin\{equation\} \textbackslash{}label\{eq:someequation\}}
\NormalTok{5^2 - 5 = 20}
\NormalTok{\textbackslash{}end\{equation\}}
 
\NormalTok{this references the equation \textbackslash{}ref\{eq:someequation\}.}
\end{Highlighting}
\end{Shaded}
\\
{$5^2 - 5 = 20\qquad(1)$}$\text{ }$\newline{}

{$\text{this references equation 1.}\,$}

\end{longtable}

\begin{longtable}{p{1.0\linewidth}}
\begin{Shaded}
\begin{Highlighting}[]

\NormalTok{\textbackslash{}begin\{equation\} \textbackslash{}label\{eq:erl\}}
\NormalTok{a = bq + r}
\NormalTok{\textbackslash{}end\{equation\}}
 
\NormalTok{where \textbackslash{}eqref\{eq:erl\} is true if $a$ and $b$ are integers with $b \textbackslash{}neq c$.}
\end{Highlighting}
\end{Shaded}
\\
{$a = bq + r\qquad(1)$}$\text{ }$\newline{}

{$\text{where (1) is true if }a\text{ and }b\,$}$\text{ }$\newline{}
{$\text{are integers with }b \neq c.$}

\end{longtable}

Further information is provided in the \mylref{417}{labels and cross-{}referencing} chapter.

To have the enumeration follow from your section or subsection heading, you must use the \LaTeXTT{amsmath} package or use AMS class documents. 
Then enter 
\begin{Shaded}
\begin{Highlighting}[]

\NormalTok{\textbackslash{}numberwithin\{equation\}\{section\}}
\end{Highlighting}
\end{Shaded}

to the preamble to get enumeration at the section level or
\begin{Shaded}
\begin{Highlighting}[]

\NormalTok{\textbackslash{}numberwithin\{equation\}\{subsection\}}
\end{Highlighting}
\end{Shaded}

to have the enumeration go to the subsection level.

\begin{longtable}{p{1.0\linewidth}}
\begin{Shaded}
\begin{Highlighting}[]
\NormalTok{\textbackslash{}documentclass[12pt]\{article\}}
\NormalTok{\textbackslash{}usepackage\{amsmath\}}
 \NormalTok{\textbackslash{}numberwithin\{equation\}\{subsection\}}
 \NormalTok{\textbackslash{}begin\{document\}}
 \NormalTok{\textbackslash{}section\{First Section\}}

 \NormalTok{\textbackslash{}subsection\{A subsection\}}
 \NormalTok{\textbackslash{}begin\{equation\}}
  \NormalTok{L' = \{L\}\{\textbackslash{}sqrt\{1-\textbackslash{}frac\{v^2\}\{c^2\}<!-- -->\}<!-- -->\}}
 \NormalTok{\textbackslash{}end\{equation\}}
\NormalTok{\textbackslash{}end\{document\}}
\end{Highlighting}
\end{Shaded}
\\

{$ L' = {L}{\sqrt{1-\frac{v^2}{c^2} } } {\color{White} ww} (1.1.1) \, $}
\end{longtable}

If the style you follow requires putting dots after ordinals (as it is required at least in Polish typography), the \LaTeXTT{\textbackslash{}numberwithin\{equation\}\{subsection\}} command in the preamble will result in the equation number in the above example being rendered as follows: (1.1..1).

To remove the duplicate dot, add the following command immediately after
\LaTeXTT{\textbackslash{}numberwithin\{equation\}\{section\}}:
\begin{Shaded}
\begin{Highlighting}[]
\NormalTok{\textbackslash{}renewcommand\{\textbackslash{}theequation\}\{\textbackslash{}thesection\textbackslash{}arabic\{equation\}\} }
\end{Highlighting}
\end{Shaded}


For a numbering scheme using \LaTeXTT{\textbackslash{}numberwithin\{equation\}\{subsection\}}, use:
\begin{Shaded}
\begin{Highlighting}[]
\NormalTok{\textbackslash{}renewcommand\{\textbackslash{}theequation\}\{\textbackslash{}thesubsection\textbackslash{}arabic\{equation\}\} }
\end{Highlighting}
\end{Shaded}

in the preamble of the document.

Note: Although it may look like the \LaTeXTT{\textbackslash{}renewcommand} works by itself, it won\textquotesingle{}t reset the equation number with each new section. It must be used together with manual equation number resetting after each new section beginning, or with the much cleaner \LaTeXTT{\textbackslash{}numberwithin}.
\subsection{Subordinate equation numbering}
\label{535}
To number subordinate equations in a numbered equation environment, place the part of document containing them in a \LaTeXTT{subequations} environment:
\begin{longtable}{p{1.0\linewidth}}
\begin{Shaded}
\begin{Highlighting}[]
\NormalTok{\textbackslash{}begin\{subequations\}}
\NormalTok{Maxwell's equations:}
\NormalTok{\textbackslash{}begin\{align\}}
        \NormalTok{B'&=-\textbackslash{}nabla \textbackslash{}times E,\textbackslash{}\textbackslash{}}
        \NormalTok{E'&=\textbackslash{}nabla \textbackslash{}times B - 4\textbackslash{}pi j,}
\NormalTok{\textbackslash{}end\{align\}}
\NormalTok{\textbackslash{}end\{subequations\}}
\end{Highlighting}
\end{Shaded}
\\

{$\text{Maxwell's equations:}\,$}

{$ \begin{aligned}         B'&=-\nabla \times E, &\quad&\mathrm{(1.1a)}\\         E'&=\nabla \times B - 4\pi j, &&\mathrm{(1.1b)} \end{aligned} $}

\end{longtable}

Referencing subordinate equations can be done using either of two methods: adding a label after the \LaTeXTT{\textbackslash{}begin\{subequations\}} command, which will reference the main equation (1.1 above), or adding a label at the end of each line, before the \LaTeXTT{\textbackslash{}\textbackslash{}} command, which will reference the sub-{}equation (1.1a or 1.1b above). It is possible to add both labels in case both types of references are needed.
\section{Vertically aligning displayed mathematics}
\label{536}
A problem often encountered with displayed environments (\LaTeXTT{displaymath} and \LaTeXTT{equation}) is the lack of any ability to span multiple lines. While it is possible to define lines individually, these will not be aligned.
\subsection{Above and below}
\label{537}
The \LaTeXTT{\textbackslash{}overset} and \LaTeXTT{\textbackslash{}underset} commands\myfootnote{requires the \LaTeXTT{amsmath} package}  typeset symbols above and below expressions. 
Without AmsTex the same result of  \LaTeXTT{\textbackslash{}overset}  can be obtained with  \LaTeXTT{\textbackslash{}stackrel}.
This can be particularly useful for creating new binary relations:
\begin{longtable}{p{1.0\linewidth}}
\begin{Shaded}
\begin{Highlighting}[]

\NormalTok{\textbackslash{}[}
 \NormalTok{A \textbackslash{}overset\{!\}\{=\} B; A \textbackslash{}stackrel\{!\}\{=\} B}
\NormalTok{\textbackslash{}]}
\end{Highlighting}
\end{Shaded}
\\

{$  A \overset{!}{=} B;~~  A \stackrel{!}{=} B \,$}

\end{longtable}
or to show usage of \myhref{https://en.wikipedia.org/wiki/L\%27H\%C3\%B4pital\%27s_rule}{L\textquotesingle{}Hôpital\textquotesingle{}s rule}:
\begin{longtable}{p{1.0\linewidth}}
\begin{Shaded}
\begin{Highlighting}[]

\NormalTok{\textbackslash{}[}
 \NormalTok{\textbackslash{}lim_\{x\textbackslash{}to 0\}\{\textbackslash{}frac\{e^x-1\}\{2x\}<!-- -->\}}
 \NormalTok{\textbackslash{}overset\{\textbackslash{}left[\textbackslash{}frac\{0\}\{0\}\textbackslash{}right]\}\{\textbackslash{}underset\{\textbackslash{}mathrm\{H\}<!-- -->\}\{=\}<!-- -->\}}
 \NormalTok{\textbackslash{}lim_\{x\textbackslash{}to 0\}\{\textbackslash{}frac\{e^x\}\{2\}<!-- -->\}=\{\textbackslash{}frac\{1\}\{2\}<!-- -->\}}
\NormalTok{\textbackslash{}]}
\end{Highlighting}
\end{Shaded}
\\

{$  \lim_{x\to 0}{\frac{e^x-1}{2x} }  \overset{\left[\frac{0}{0}\right]}{\underset{\mathrm{H} }{=} }  \lim_{x\to 0}{\frac{e^x}{2} }={\frac{1}{2} } $}

\end{longtable}

It is convenient to define a new operator that will set the equals sign with H and the provided fraction:
\begin{Shaded}
\begin{Highlighting}[]

\NormalTok{\textbackslash{}newcommand\{\textbackslash{}Heq\}[1]\{\textbackslash{}overset\{\textbackslash{}left[#1\textbackslash{}right]\}\{\textbackslash{}underset\{\textbackslash{}mathrm\{H\}\}\{=\}\}\}}
\end{Highlighting}
\end{Shaded}

which reduces the above example to:
\begin{Shaded}
\begin{Highlighting}[]

\NormalTok{\textbackslash{}[}
 \NormalTok{\textbackslash{}lim_\{x\textbackslash{}to 0\}\{\textbackslash{}frac\{e^x-1\}\{2x\}\}}
 \NormalTok{\textbackslash{}Heq\{\textbackslash{}frac\{0\}\{0\}\}}
 \NormalTok{\textbackslash{}lim_\{x\textbackslash{}to 0\}\{\textbackslash{}frac\{e^x\}\{2\}\}=\{\textbackslash{}frac\{1\}\{2\}\}}
\NormalTok{\textbackslash{}]}
\end{Highlighting}
\end{Shaded}


If the purpose is to make comments on particular parts of an equation, the \LaTeXTT{\textbackslash{}overbrace} and \LaTeXTT{\textbackslash{}underbrace} commands may be more useful. However, they have a different syntax (and can be aligned with the \LaTeXTT{\textbackslash{}vphantom} command):
\begin{longtable}{p{1.0\linewidth}}
\begin{Shaded}
\begin{Highlighting}[]

\NormalTok{\textbackslash{}[}
 \NormalTok{z = \textbackslash{}overbrace\{}
   \NormalTok{\textbackslash{}underbrace\{x\}_\textbackslash{}text\{real\} + i}
   \NormalTok{\textbackslash{}underbrace\{y\}_\textbackslash{}text\{imaginary\}}
  \NormalTok{\}^\textbackslash{}text\{complex number\}}
\NormalTok{\textbackslash{}]}
\end{Highlighting}
\end{Shaded}
\\
{$  z = \overbrace{    \underbrace{x}_\text{real} + i    \underbrace{y}_\text{imaginary}   }^\text{complex number} $}
\end{longtable}

Sometimes the comments are longer than the formula being commented on, which can cause spacing problems. These can be removed using the \LaTeXTT{\textbackslash{}mathclap} command\myfootnote{requires the \LaTeXTT{mathtools} package}:
\begin{longtable}{p{1.0\linewidth}}
\begin{Shaded}
\begin{Highlighting}[]

\NormalTok{\textbackslash{}[}
 \NormalTok{y = a + f(\textbackslash{}underbrace\{b x\}_\{}
                    \NormalTok{\textbackslash{}ge 0 \textbackslash{}text\{ by assumption\}<!-- -->\}) }
   \NormalTok{= a + f(\textbackslash{}underbrace\{b x\}_\{}
          \NormalTok{\textbackslash{}mathclap\{\textbackslash{}ge 0 \textbackslash{}text\{ by assumption\}<!-- -->\}<!-- -->\})}
\NormalTok{\textbackslash{}]}
\end{Highlighting}
\end{Shaded}
\\



\begin{minipage}{0.87500\textwidth}
\begin{center}
\includegraphics[width=1.0\textwidth,height=6.5in,keepaspectratio]{../images/93.png}
\end{center}
\raggedright{}\myfigurewithoutcaption{93}
\end{minipage}\vspace{0.75cm}



\end{longtable}

Alternatively, to use brackets instead of braces use \LaTeXTT{\textbackslash{}underbracket} and \LaTeXTT{\textbackslash{}overbracket} commands\myfootnote{requires the \LaTeXTT{mathtools} package}:
\begin{longtable}{p{1.0\linewidth}}
\begin{Shaded}
\begin{Highlighting}[]

\NormalTok{\textbackslash{}[}
 \NormalTok{z = \textbackslash{}overbracket[3pt]\{}
     \NormalTok{\textbackslash{}underbracket\{x\}_\{\textbackslash{}text\{real\}<!---->\} +}
     \NormalTok{\textbackslash{}underbracket[0.5pt][7pt]\{iy\}_\{\textbackslash{}text\{imaginary\}<!---->\}}
     \NormalTok{\}^\{\textbackslash{}text\{complex number\}<!---->\} }
\NormalTok{\textbackslash{}]}
\end{Highlighting}
\end{Shaded}
\\


\begin{minipage}{0.42500\textwidth}
\begin{center}
\includegraphics[width=1.0\textwidth,height=6.5in,keepaspectratio]{../images/94.png}
\end{center}
\raggedright{}\myfigurewithoutcaption{94}
\end{minipage}\vspace{0.75cm}


\end{longtable}

The optional arguments set the rule thickness and bracket height respectively:
\begin{Shaded}
\begin{Highlighting}[]

\NormalTok{\textbackslash{}underbracket[rule thickness][bracket height]\{argument\}_\{text below\}}
\end{Highlighting}
\end{Shaded}


The \LaTeXTT{\textbackslash{}xleftarrow} and \LaTeXTT{\textbackslash{}xrightarrow} commands\myfootnote{requires the \LaTeXTT{amsmath} package} produce arrows which extend to the length of the text. Yet again, the syntax is different: the optional argument (using {\ttfamily \setmainfont[Path=/usr/share/fonts/truetype/cmu/,UprightFont=cmunrm.ttf,BoldFont=cmunbx.ttf,ItalicFont=cmunti.ttf,BoldItalicFont=cmunbi.ttf]{cmuntt.ttf}\setmonofont[Path=/usr/share/fonts/truetype/cmu/,UprightFont=cmuntt.ttf,BoldFont=cmuntb.ttf,ItalicFont=cmunit.ttf,BoldItalicFont=cmuntx.ttf]{cmuntt.ttf}\ttfamily {$\text{[}$}}{$\text{ }$}\setmainfont[Path=/usr/share/fonts/truetype/cmu/,UprightFont=cmunrm.ttf,BoldFont=cmunbx.ttf,ItalicFont=cmunti.ttf,BoldItalicFont=cmunbi.ttf]{cmunrm.ttf}\setmonofont[Path=/usr/share/fonts/truetype/cmu/,UprightFont=cmuntt.ttf,BoldFont=cmuntb.ttf,ItalicFont=cmunit.ttf,BoldItalicFont=cmuntx.ttf]{cmunrm.ttf} and {\ttfamily \setmainfont[Path=/usr/share/fonts/truetype/cmu/,UprightFont=cmunrm.ttf,BoldFont=cmunbx.ttf,ItalicFont=cmunti.ttf,BoldItalicFont=cmunbi.ttf]{cmuntt.ttf}\setmonofont[Path=/usr/share/fonts/truetype/cmu/,UprightFont=cmuntt.ttf,BoldFont=cmuntb.ttf,ItalicFont=cmunit.ttf,BoldItalicFont=cmuntx.ttf]{cmuntt.ttf}\ttfamily {$\text{]}$}}\setmainfont[Path=/usr/share/fonts/truetype/cmu/,UprightFont=cmunrm.ttf,BoldFont=cmunbx.ttf,ItalicFont=cmunti.ttf,BoldItalicFont=cmunbi.ttf]{cmunrm.ttf}\setmonofont[Path=/usr/share/fonts/truetype/cmu/,UprightFont=cmuntt.ttf,BoldFont=cmuntb.ttf,ItalicFont=cmunit.ttf,BoldItalicFont=cmuntx.ttf]{cmunrm.ttf}) specifies the subscript, and the mandatory argument (using {\ttfamily \setmainfont[Path=/usr/share/fonts/truetype/cmu/,UprightFont=cmunrm.ttf,BoldFont=cmunbx.ttf,ItalicFont=cmunti.ttf,BoldItalicFont=cmunbi.ttf]{cmuntt.ttf}\setmonofont[Path=/usr/share/fonts/truetype/cmu/,UprightFont=cmuntt.ttf,BoldFont=cmuntb.ttf,ItalicFont=cmunit.ttf,BoldItalicFont=cmuntx.ttf]{cmuntt.ttf}\ttfamily \{}{$\text{ }$}\setmainfont[Path=/usr/share/fonts/truetype/cmu/,UprightFont=cmunrm.ttf,BoldFont=cmunbx.ttf,ItalicFont=cmunti.ttf,BoldItalicFont=cmunbi.ttf]{cmunrm.ttf}\setmonofont[Path=/usr/share/fonts/truetype/cmu/,UprightFont=cmuntt.ttf,BoldFont=cmuntb.ttf,ItalicFont=cmunit.ttf,BoldItalicFont=cmuntx.ttf]{cmunrm.ttf} and {\ttfamily \setmainfont[Path=/usr/share/fonts/truetype/cmu/,UprightFont=cmunrm.ttf,BoldFont=cmunbx.ttf,ItalicFont=cmunti.ttf,BoldItalicFont=cmunbi.ttf]{cmuntt.ttf}\setmonofont[Path=/usr/share/fonts/truetype/cmu/,UprightFont=cmuntt.ttf,BoldFont=cmuntb.ttf,ItalicFont=cmunit.ttf,BoldItalicFont=cmuntx.ttf]{cmuntt.ttf}\ttfamily \}}\setmainfont[Path=/usr/share/fonts/truetype/cmu/,UprightFont=cmunrm.ttf,BoldFont=cmunbx.ttf,ItalicFont=cmunti.ttf,BoldItalicFont=cmunbi.ttf]{cmunrm.ttf}\setmonofont[Path=/usr/share/fonts/truetype/cmu/,UprightFont=cmuntt.ttf,BoldFont=cmuntb.ttf,ItalicFont=cmunit.ttf,BoldItalicFont=cmuntx.ttf]{cmunrm.ttf}) specifies the superscript (which can be left empty by inserting a blank space).
\begin{longtable}{p{1.0\linewidth}}
\begin{Shaded}
\begin{Highlighting}[]

\NormalTok{\textbackslash{}[}
 \NormalTok{A \textbackslash{}xleftarrow\{\textbackslash{}text\{this way\}<!-- -->\} B }
  \NormalTok{\textbackslash{}xrightarrow[\textbackslash{}text\{or that way\}]\{ \} C}
\NormalTok{\textbackslash{}]}
\end{Highlighting}
\end{Shaded}
\\

{$ A \xleftarrow{\text{this way}} B \xrightarrow[\text{or that way}]{} C \,$}

\end{longtable}

For more extensible arrows, you must use the \LaTeXTT{mathtools} package:
\begin{longtable}{p{1.0\linewidth}}
\begin{Shaded}
\begin{Highlighting}[]

\NormalTok{\textbackslash{}[}
 \NormalTok{a \textbackslash{}xleftrightarrow[under]\{over\} b\textbackslash{}\textbackslash{}}
\CommentTok
 \NormalTok{B \textbackslash{}xRightarrow[under]\{over\} C\textbackslash{}\textbackslash{}}
\CommentTok
 \NormalTok{D \textbackslash{}xhookleftarrow[under]\{over\} E\textbackslash{}\textbackslash{}}
\CommentTok
 \NormalTok{F \textbackslash{}xmapsto[under]\{over\} G\textbackslash{}\textbackslash{}}
\NormalTok{\textbackslash{}]}
\end{Highlighting}
\end{Shaded}
\\


\begin{minipage}{0.22500\textwidth}
\begin{center}
\includegraphics[width=1.0\textwidth,height=6.5in,keepaspectratio]{../images/95.png}
\end{center}
\raggedright{}\myfigurewithoutcaption{95}
\end{minipage}\vspace{0.75cm}


\end{longtable}

and for harpoons:
\begin{longtable}{p{1.0\linewidth}}
\begin{Shaded}
\begin{Highlighting}[]

\NormalTok{\textbackslash{}[}
 \NormalTok{H \textbackslash{}xrightharpoondown[under]\{over\} I\textbackslash{}\textbackslash{}}
\CommentTok
 \NormalTok{J \textbackslash{}xleftharpoondown[under]\{over\} K\textbackslash{}\textbackslash{}}
\CommentTok
 \NormalTok{L \textbackslash{}xrightleftharpoons[under]\{over\} M\textbackslash{}\textbackslash{}}
\CommentTok{%}
 \NormalTok{M \textbackslash{}xleftrightharpoons[under]\{over\} N}
\NormalTok{\textbackslash{}]}
\end{Highlighting}
\end{Shaded}
\\


\begin{minipage}{0.22500\textwidth}
\begin{center}
\includegraphics[width=1.0\textwidth,height=6.5in,keepaspectratio]{../images/96.png}
\end{center}
\raggedright{}\myfigurewithoutcaption{96}
\end{minipage}\vspace{0.75cm}


\end{longtable}
\subsection{\LaTeXTT{align} and \LaTeXTT{align*}}
\label{538}
The \LaTeXTT{align} and \LaTeXTT{align*} environments, available through the \LaTeXTT{amsmath} package, are used for arranging equations of multiple lines. As with matrices and tables, \LaTeXTT{\textbackslash{}\textbackslash{}} specifies a line break, and \LaTeXTT{\&} is used to indicate the point at which the lines should be aligned.

The \LaTeXTT{align*} environment is used like the \LaTeXTT{displaymath} or \LaTeXTT{equation*} environment:
\begin{longtable}{p{1.0\linewidth}}
\begin{Shaded}
\begin{Highlighting}[]

\NormalTok{\textbackslash{}begin\{align*\}}
 \NormalTok{f(x) &= (x+a)(x+b) \textbackslash{}\textbackslash{}}
 \NormalTok{&= x^2 + (a+b)x + ab}
\NormalTok{\textbackslash{}end\{align*\}}
\end{Highlighting}
\end{Shaded}
\\

{$\begin{aligned} f(x) &= (x+a)(x+b) \\ &= x^2 + (a+b)x + ab \end{aligned}\,$}

\end{longtable}
Note that the \LaTeXTT{align} environment must not be nested inside an \LaTeXTT{equation} (or similar) environment.  Instead, \LaTeXTT{align} is a replacement for such environments; the contents inside an \LaTeXTT{align} are automatically placed in math mode.

\LaTeXTT{align*} suppresses numbering.  To force numbering on a specific line, use the \LaTeXTT{\textbackslash{}tag\{...\}} command before the line break.

\LaTeXTT{align} is similar, but automatically numbers each line like the \LaTeXTT{equation} environment.  Individual lines may be referred to by placing a \LaTeXTT{\textbackslash{}label\{...\}} before the line break. The \LaTeXTT{\textbackslash{}nonumber} or \LaTeXTT{\textbackslash{}notag} command can be used to suppress the number for a given line:
\begin{longtable}{p{1.0\linewidth}}
\begin{Shaded}
\begin{Highlighting}[]

\NormalTok{\textbackslash{}begin\{align\}}
 \NormalTok{f(x) &= x^4 + 7x^3 + 2x^2 \textbackslash{}nonumber \textbackslash{}\textbackslash{}}
 \NormalTok{&\textbackslash{}qquad \{\} + 10x + 12}
\NormalTok{\textbackslash{}end\{align\}}
\end{Highlighting}
\end{Shaded}
\\

{$\begin{aligned} f(x) &= x^4 + 7x^3 + 2x^2 \\ &\qquad {} + 10x + 12 \qquad \qquad (3) \end{aligned}$}

\end{longtable}
Notice that we\textquotesingle{}ve added some indenting on the second line.  Also, we need to insert the double braces ({\ttfamily \setmainfont[Path=/usr/share/fonts/truetype/cmu/,UprightFont=cmunrm.ttf,BoldFont=cmunbx.ttf,ItalicFont=cmunti.ttf,BoldItalicFont=cmunbi.ttf]{cmuntt.ttf}\setmonofont[Path=/usr/share/fonts/truetype/cmu/,UprightFont=cmuntt.ttf,BoldFont=cmuntb.ttf,ItalicFont=cmunit.ttf,BoldItalicFont=cmuntx.ttf]{cmuntt.ttf}\ttfamily \{\}}\setmainfont[Path=/usr/share/fonts/truetype/cmu/,UprightFont=cmunrm.ttf,BoldFont=cmunbx.ttf,ItalicFont=cmunti.ttf,BoldItalicFont=cmunbi.ttf]{cmunrm.ttf}\setmonofont[Path=/usr/share/fonts/truetype/cmu/,UprightFont=cmuntt.ttf,BoldFont=cmuntb.ttf,ItalicFont=cmunit.ttf,BoldItalicFont=cmuntx.ttf]{cmunrm.ttf}) before the + sign, otherwise latex won\textquotesingle{}t create the correct spacing after the + sign.  The reason for this is that without the braces, latex interprets the + sign as a unary operator, instead of the binary operator that it really is.

More complicated alignments are possible, with additional \LaTeXTT{\&}\textquotesingle{}s on a single line specifying multiple \symbol{34}equation columns\symbol{34}, each of which is aligned. The following example illustrates the alignment rule of \LaTeXTT{align*}:
\begin{longtable}{p{1.0\linewidth}}
\begin{Shaded}
\begin{Highlighting}[]

\NormalTok{\textbackslash{}begin\{align*\}}
 \NormalTok{f(x)  &= a x^2+b x +c   &   g(x)  &= d x^3 \textbackslash{}\textbackslash{}}
 \NormalTok{f'(x) &= 2 a x +b       &   g'(x) &= 3 d x^2}
\NormalTok{\textbackslash{}end\{align*\}}
\end{Highlighting}
\end{Shaded}
\\

{$\begin{aligned}  f(x)  &= a x^2+b x +c   &   g(x)  &= d x^3 \\  f'(x) &= 2 a x +b       &   g'(x) &= 3 d x^2 \end{aligned}\,$}

\end{longtable}
\subsection{Braces spanning multiple lines}
\label{539}
If you want a brace to continue across a new line, do the following:
\begin{longtable}{p{1.0\linewidth}}
\begin{Shaded}
\begin{Highlighting}[]

\NormalTok{\textbackslash{}begin\{align\}}
 \NormalTok{f(x) &= \textbackslash{}pi \textbackslash{}left\textbackslash{}\{ x^4 + 7x^3 + 2x^2 \textbackslash{}right.\textbackslash{}nonumber\textbackslash{}\textbackslash{}}
 \NormalTok{&\textbackslash{}qquad \textbackslash{}left. \{\} + 10x + 12 \textbackslash{}right\textbackslash{}\}}
\NormalTok{\textbackslash{}end\{align\}}
\end{Highlighting}
\end{Shaded}
\\

{$\begin{aligned} f(x) &= \pi \left\{ x^4 + 7x^3 + 2x^2 \right.\\ &\qquad \left. {} + 10x + 12 \right\}  \qquad \qquad (4) \end{aligned}$}

\end{longtable}

In this construction, the sizes of the left and right braces are not automatically equal, in spite of the use of \LaTeXTT{\textbackslash{}left\textbackslash{}\{} and \LaTeXTT{\textbackslash{}right\textbackslash{}\}}. This is because each line is typeset as a completely separate equation {\mbox{$\text{---}$}}notice the use of \LaTeXTT{\textbackslash{}right.} and \LaTeXTT{\textbackslash{}left.} so there are no unpaired \LaTeXTT{\textbackslash{}left} and \LaTeXTT{\textbackslash{}right} commands within a line (these aren\textquotesingle{}t needed if the formula is on one line). You can control the size of the braces manually with the \LaTeXTT{\textbackslash{}big}, \LaTeXTT{\textbackslash{}Big}, \LaTeXTT{\textbackslash{}bigg}, and \LaTeXTT{\textbackslash{}Bigg} commands.

Alternatively, the height of the taller equation can be replicated in the other using the \LaTeXTT{\textbackslash{}vphantom} command:
\begin{longtable}{p{1.0\linewidth}}
\begin{Shaded}
\begin{Highlighting}[]

\NormalTok{\textbackslash{}begin\{align\}}
 \NormalTok{A &=     \textbackslash{}left(\textbackslash{}int_t XXX       \textbackslash{}right.\textbackslash{}nonumber\textbackslash{}\textbackslash{}}
   \NormalTok{&\textbackslash{}qquad \textbackslash{}left.\textbackslash{}vphantom\{\textbackslash{}int_t\} YYY \textbackslash{}dots \textbackslash{}right)}
\NormalTok{\textbackslash{}end\{align\}}
\end{Highlighting}
\end{Shaded}
\\

{$ \begin{aligned}  A &=     \left(\int_t XXX\right.\\    &\qquad YYY \dots \biggr)\qquad\qquad \mathrm{(5)} \end{aligned} $}

\end{longtable}
\subsubsection{Using aligned braces for piecewise functions}
\label{540}
You can also use \LaTeXTT{\textbackslash{}left\textbackslash{}\{} and \LaTeXTT{\textbackslash{}right.} to typeset \myhref{https://en.wikipedia.org/wiki/piecewise\%20functions}{piecewise functions}:

\begin{longtable}{p{1.0\linewidth}}
\begin{Shaded}
\begin{Highlighting}[]

\NormalTok{\textbackslash{}[f(x) = \textbackslash{}left\textbackslash{}\{}
  \NormalTok{\textbackslash{}begin\{array\}\{lr\}}
    \NormalTok{x^2 & : x < 0\textbackslash{}\textbackslash{}}
    \NormalTok{x^3 & : x \textbackslash{}ge 0}
  \NormalTok{\textbackslash{}end\{array\}}
\NormalTok{\textbackslash{}right.}
\NormalTok{\textbackslash{}]}
\end{Highlighting}
\end{Shaded}
\\

{$ f(x) = \left\{   \begin{array}{lr}     x^2 & : x < 0\\     x^3 & : x \ge 0   \end{array} \right. $}

\end{longtable}
\subsection{The \LaTeXTT{cases} environment}
\label{541}
The \LaTeXTT{cases} environment\myfootnote{requires the \LaTeXTT{amsmath} package} allows the writing of piecewise functions:
\begin{longtable}{p{1.0\linewidth}}
\begin{Shaded}
\begin{Highlighting}[]

\NormalTok{\textbackslash{}[}
 \NormalTok{u(x) = }
  \NormalTok{\textbackslash{}begin\{cases\} }
   \NormalTok{\textbackslash{}exp\{x\} & \textbackslash{}text\{if \} x \textbackslash{}geq 0 \textbackslash{}\textbackslash{}}
   \NormalTok{1       & \textbackslash{}text\{if \} x < 0}
  \NormalTok{\textbackslash{}end\{cases\}}
\NormalTok{\textbackslash{}]}
\end{Highlighting}
\end{Shaded}
\\

{$ u(x) =  \begin{cases} \exp{x} & \text{if } x \geq 0 \\ 1       & \text{if } x < 0 \end{cases} $}

\end{longtable}
LaTeX will then take care of defining and or aligning the columns.

Within \LaTeXTT{cases}, text style math is used with results such as:

{$ a =  \begin{cases}   \int x\, \mathrm{d} x\\   b^2  \end{cases} $}

Display style may be used instead, by using the \LaTeXTT{dcases} environment\myfootnote{requires the \LaTeXTT{mathtools} package} from \LaTeXTT{mathtools}:
\begin{longtable}{p{1.0\linewidth}}
\begin{Shaded}
\begin{Highlighting}[]

\NormalTok{\textbackslash{}[}
 \NormalTok{a =}
   \NormalTok{\textbackslash{}begin\{dcases\}}
     \NormalTok{\textbackslash{}int x\textbackslash{}, \textbackslash{}mathrm\{d\} x\textbackslash{}\textbackslash{}}
     \NormalTok{b^2}
   \NormalTok{\textbackslash{}end\{dcases\}}
\NormalTok{\textbackslash{}]}
\end{Highlighting}
\end{Shaded}
\\

{$ a =  \begin{cases}   \displaystyle\int x\, \mathrm{d} x\\   \displaystyle b^2  \end{cases} $}
\end{longtable}

Often the second column consists mostly of normal text.  To set it in the normal Roman font of the document, the \LaTeXTT{dcases*} environment may be used:\myfootnote{requires the \LaTeXTT{mathtools} package}
\begin{longtable}{p{1.0\linewidth}}
\begin{Shaded}
\begin{Highlighting}[]

\NormalTok{\textbackslash{}[}
 \NormalTok{f(x) = \textbackslash{}begin\{dcases*\}}
        \NormalTok{x  & when }\AlertTok{$}\NormalTok{x}\AlertTok{$} \NormalTok{is even\textbackslash{}\textbackslash{}}
        \NormalTok{-x & when }\AlertTok{$}\NormalTok{x}\AlertTok{$} \NormalTok{is odd}
        \NormalTok{\textbackslash{}end\{dcases*\}}
\NormalTok{\textbackslash{}]}
\end{Highlighting}
\end{Shaded}
\\

{$  f(x) = \begin{cases}  x  & \text{when }x\text{ is even}\\  -x & \text{when }x\text{ is odd}  \end{cases} $}
\end{longtable}
\subsection{Other environments}
\label{542}
Although \LaTeXTT{align} and \LaTeXTT{align*} are the most useful, there are several other environments that may also be of interest:
{\scalefont{0.72679}\begin{longtable}{|>{\RaggedRight}p{0.29967\linewidth}|>{\RaggedRight}p{0.30731\linewidth}|>{\RaggedRight}p{0.30731\linewidth}|} \hline 
{\bfseries \hspace*{0pt}\ignorespaces{}\hspace*{0pt} Environment name}&{\bfseries \hspace*{0pt}\ignorespaces{}\hspace*{0pt} Description}&{\bfseries \hspace*{0pt}\ignorespaces{}\hspace*{0pt} Notes}\endhead  \hline \hspace*{0pt}\ignorespaces{}\hspace*{0pt} \LaTeXTT{eqnarray} and \LaTeXTT{eqnarray*}&\hspace*{0pt}\ignorespaces{}\hspace*{0pt} Similar to \LaTeXTT{align} and \LaTeXTT{align*}&\hspace*{0pt}\ignorespaces{}\hspace*{0pt} Not recommended because spacing is inconsistent\\ \hline \hspace*{0pt}\ignorespaces{}\hspace*{0pt} \LaTeXTT{multline} and \LaTeXTT{multline*}\myfootnote{requires the \LaTeXTT{amsmath} package}&\hspace*{0pt}\ignorespaces{}\hspace*{0pt} First line left aligned, last line right aligned&\hspace*{0pt}\ignorespaces{}\hspace*{0pt} Equation number aligned vertically with first line and not centered as with other environments\\ \hline \hspace*{0pt}\ignorespaces{}\hspace*{0pt} \LaTeXTT{gather} and \LaTeXTT{gather*}\myfootnote{requires the \LaTeXTT{amsmath} package}&\hspace*{0pt}\ignorespaces{}\hspace*{0pt} Consecutive equations without alignment&\hspace*{0pt}\ignorespaces{}\hspace*{0pt} \\ \hline \hspace*{0pt}\ignorespaces{}\hspace*{0pt} \LaTeXTT{flalign} and \LaTeXTT{flalign*}\myfootnote{requires the \LaTeXTT{amsmath} package}&\hspace*{0pt}\ignorespaces{}\hspace*{0pt} Similar to \LaTeXTT{align}, but left aligns first equation column, and right aligns last column&\hspace*{0pt}\ignorespaces{}\hspace*{0pt} \\ \hline \hspace*{0pt}\ignorespaces{}\hspace*{0pt} \LaTeXTT{alignat} and \LaTeXTT{alignat*}\myfootnote{requires the \LaTeXTT{amsmath} package}&\hspace*{0pt}\ignorespaces{}\hspace*{0pt} Takes an argument specifying number of columns. Allows control of the horizontal space between equations&\hspace*{0pt}\ignorespaces{}\hspace*{0pt} This environment takes one argument, the number of “equation columns”: count the maximum number of \LaTeXTT{\&}s in any row, add 1 and divide by 2. {$\text{[}$}ftp://ftp.ams.org/ams/doc/amsmath/amsldoc.pdf{$\text{]}$}\\ \hline 
\end{longtable}
}

There are also a few environments that don\textquotesingle{}t form a math environment by themselves and can be used as building blocks for more elaborate structures:
\begin{longtable}{|>{\RaggedRight}p{0.40767\linewidth}|>{\RaggedRight}p{0.53519\linewidth}|} \hline 
{\bfseries \hspace*{0pt}\ignorespaces{}\hspace*{0pt} Math environment name}&{\bfseries \hspace*{0pt}\ignorespaces{}\hspace*{0pt} Description}\endhead  \hline \hspace*{0pt}\ignorespaces{}\hspace*{0pt} \LaTeXTT{gathered}\myfootnote{requires the \LaTeXTT{amsmath} package}&\hspace*{0pt}\ignorespaces{}\hspace*{0pt} Allows gathering equations to be set under each other and assigned a single equation number\\ \hline \hspace*{0pt}\ignorespaces{}\hspace*{0pt} \LaTeXTT{split}\myfootnote{requires the \LaTeXTT{amsmath} package}&\hspace*{0pt}\ignorespaces{}\hspace*{0pt} Similar to \LaTeXTT{align*}, but used inside another displayed mathematics environment\\ \hline \hspace*{0pt}\ignorespaces{}\hspace*{0pt} \LaTeXTT{aligned}\myfootnote{requires the \LaTeXTT{amsmath} package}&\hspace*{0pt}\ignorespaces{}\hspace*{0pt} Similar to \LaTeXTT{align}, to be used inside another mathematics environment.\\ \hline \hspace*{0pt}\ignorespaces{}\hspace*{0pt} \LaTeXTT{alignedat}\myfootnote{requires the \LaTeXTT{amsmath} package}&\hspace*{0pt}\ignorespaces{}\hspace*{0pt} Similar to \LaTeXTT{alignat}, and likewise takes an additional argument specifying the number of columns of equations to set.\\ \hline 
\end{longtable}


For example:
\begin{longtable}{p{1.0\linewidth}}
\begin{Shaded}
\begin{Highlighting}[]

\NormalTok{\textbackslash{}begin\{equation\}}
 \NormalTok{\textbackslash{}left.\textbackslash{}begin\{aligned\}}
        \NormalTok{B'&=-\textbackslash{}partial \textbackslash{}times E,\textbackslash{}\textbackslash{}}
        \NormalTok{E'&=\textbackslash{}partial \textbackslash{}times B - 4\textbackslash{}pi j,}
       \NormalTok{\textbackslash{}end\{aligned\}}
 \NormalTok{\textbackslash{}right\textbackslash{}\}}
 \NormalTok{\textbackslash{}qquad \textbackslash{}text\{Maxwell's equations\}}
\NormalTok{\textbackslash{}end\{equation\}}
\end{Highlighting}
\end{Shaded}
\\

{$ \left.\begin{aligned}         B'&=-\partial \times E,\\         E'&=\partial \times B - 4\pi j, \end{aligned}\right\}\quad\text{Maxwell}'\text{s equations}\qquad\mathrm{(1.1)} $}

\end{longtable}
\begin{longtable}{p{1.0\linewidth}}
\begin{Shaded}
\begin{Highlighting}[]

\NormalTok{\textbackslash{}begin\{alignat\}\{2\}}
 \NormalTok{\textbackslash{}sigma_1 &= x + y  &\textbackslash{}quad \textbackslash{}sigma_2 &= \textbackslash{}frac\{x\}\{y\} \textbackslash{}\textbackslash{}	}
 \NormalTok{\textbackslash{}sigma_1' &= \textbackslash{}frac\{\textbackslash{}partial x + y\}\{\textbackslash{}partial x\} & \textbackslash{}sigma_2' }
    \NormalTok{&= \textbackslash{}frac\{\textbackslash{}partial \textbackslash{}frac\{x\}\{y\}<!---->\}\{\textbackslash{}partial x\}}
\NormalTok{\textbackslash{}end\{alignat\}}
\end{Highlighting}
\end{Shaded}
\\

{$\begin{aligned}  \sigma_1 &= x + y  &\sigma_2 &= \frac{x}{y}  &\qquad&\qquad&(1)   \\	  \sigma_1' &= \frac{\partial x + y}{\partial x} & \sigma_2' &= \frac{\partial \frac{x}{y}}{\partial x} &&&(2) \end{aligned}$}

\end{longtable}
\begin{longtable}{p{1.0\linewidth}}
\begin{Shaded}
\begin{Highlighting}[]

\NormalTok{\textbackslash{}begin\{gather*\}}
\NormalTok{a_0=\textbackslash{}frac\{1\}\{\textbackslash{}pi\}\textbackslash{}int\textbackslash{}limits_\{-\textbackslash{}pi\}^\{\textbackslash{}pi\}f(x)\textbackslash{},\textbackslash{}mathrm\{d\}x\textbackslash{}\textbackslash{}[6pt]}
\NormalTok{\textbackslash{}begin\{split\}}
\NormalTok{a_n=\textbackslash{}frac\{1\}\{\textbackslash{}pi\}\textbackslash{}int\textbackslash{}limits_\{-\textbackslash{}pi\}^\{\textbackslash{}pi\}f(x)\textbackslash{}cos nx\textbackslash{},\textbackslash{}mathrm\{d\}x=\textbackslash{}\textbackslash{}}
\NormalTok{=\textbackslash{}frac\{1\}\{\textbackslash{}pi\}\textbackslash{}int\textbackslash{}limits_\{-\textbackslash{}pi\}^\{\textbackslash{}pi\}x^2\textbackslash{}cos nx\textbackslash{},\textbackslash{}mathrm\{d\}x}
\NormalTok{\textbackslash{}end\{split\}\textbackslash{}\textbackslash{}[6pt]}
\NormalTok{\textbackslash{}begin\{split\}}
\NormalTok{b_n=\textbackslash{}frac\{1\}\{\textbackslash{}pi\}\textbackslash{}int\textbackslash{}limits_\{-\textbackslash{}pi\}^\{\textbackslash{}pi\}f(x)\textbackslash{}sin nx\textbackslash{},\textbackslash{}mathrm\{d\}x=\textbackslash{}\textbackslash{}}
\NormalTok{=\textbackslash{}frac\{1\}\{\textbackslash{}pi\}\textbackslash{}int\textbackslash{}limits_\{-\textbackslash{}pi\}^\{\textbackslash{}pi\}x^2\textbackslash{}sin nx\textbackslash{},\textbackslash{}mathrm\{d\}x}
\NormalTok{\textbackslash{}end\{split\}\textbackslash{}\textbackslash{}[6pt]}
\NormalTok{\textbackslash{}end\{gather*\}}
\end{Highlighting}
\end{Shaded}
\\


\begin{minipage}{0.37500\textwidth}
\begin{center}
\includegraphics[width=1.0\textwidth,height=6.5in,keepaspectratio]{../images/97.png}
\end{center}
\raggedright{}\myfigurewithoutcaption{97}
\end{minipage}\vspace{0.75cm}



\end{longtable}
\section{Indented Equations}
\label{543}

To indent an equation, you can set \LaTeXTT{fleqn} in the document class and then specify a certain value for the \LaTeXTT{\textbackslash{}mathindent} variable:
\begin{longtable}{p{1.0\linewidth}}
\begin{Shaded}
\begin{Highlighting}[]

\NormalTok{\textbackslash{}documentclass[a4paper,fleqn]\{report\}}
\NormalTok{\textbackslash{}usepackage\{amsmath\}}
\NormalTok{\textbackslash{}setlength\{\textbackslash{}mathindent\}\{1cm\}}
\NormalTok{\textbackslash{}begin\{document\}}
\NormalTok{\textbackslash{}noindent Euler's formula is given below:}
\NormalTok{\textbackslash{}begin\{equation*\}}
 \NormalTok{e^\{ix\} = \textbackslash{}cos\{x\} + i \textbackslash{}sin\{x\}.}
\NormalTok{\textbackslash{}end\{equation*\}}
\NormalTok{\textbackslash{}noindent This is a very important formula.}
\NormalTok{\textbackslash{}end\{document\}}
\end{Highlighting}
\end{Shaded}
\\



\begin{minipage}{1.0\linewidth}
\begin{center}
\includegraphics[width=1.0\linewidth,height=6.5in,keepaspectratio]{../images/98.png}
\end{center}
\raggedright{}\myfigurewithoutcaption{98}
\end{minipage}\vspace{0.75cm}



\end{longtable}
\section{Page breaks in math environments}
\label{544}
To suggest that LaTeX insert a page break inside an \LaTeXTT{amsmath} environment, you may use the \LaTeXTT{\textbackslash{}displaybreak} command before the line break. Just as with \LaTeXTT{\textbackslash{}pagebreak}, \LaTeXTT{\textbackslash{}displaybreak} can take an optional argument between 0 and 4 denoting the level of desirability of a page break. Whereas 0 means \symbol{34}it is permissible to break here\symbol{34}, 4 forces a break. No argument means the same as 4.

Alternatively, you may enable automatic page breaks in math environments with \LaTeXTT{\textbackslash{}allowdisplaybreaks}. It too can have an optional argument denoting the priority of page breaks in equations. Similarly, 1 means \symbol{34}allow page breaks but avoid them\symbol{34} and 4 means \symbol{34}break whenever you want\symbol{34}. You can prohibit a page break after a given line using \LaTeXTT{\textbackslash{}\textbackslash{}*}.

LaTeX will insert a page break into a long equation if it has additional text added using \LaTeXTT{\textbackslash{}intertext\{\}} without any additional commands.

Specific usage may look like this:
\begin{longtable}{p{1.0\linewidth}}
\begin{Shaded}
\begin{Highlighting}[]

\NormalTok{\textbackslash{}begin\{align*\}}
 \NormalTok{&\textbackslash{}vdots\textbackslash{}\textbackslash{} }
 \NormalTok{&=12+7 \textbackslash{}int_0^2}
  \NormalTok{\textbackslash{}left(}
    \NormalTok{-\textbackslash{}frac\{1\}\{4\}\textbackslash{}left(e^\{-4t_1\}+e^\{4t_1-8\}\textbackslash{}right)}
  \NormalTok{\textbackslash{}right)\textbackslash{},dt_1\textbackslash{}displaybreak[3]\textbackslash{}\textbackslash{}}
 \NormalTok{&= 12-\textbackslash{}frac\{7\}\{4\}\textbackslash{}int_0^2 \textbackslash{}left( e^\{-4t_1\}+e^\{4t_1-8\} \textbackslash{}right)\textbackslash{},dt_1\textbackslash{}\textbackslash{}}
 \NormalTok{&\textbackslash{}vdots }\CommentTok{% }
\NormalTok{\textbackslash{}end\{align*\}}
\end{Highlighting}
\end{Shaded}
\\


\begin{minipage}{1.0\linewidth}
\begin{center}
\includegraphics[width=1.0\linewidth,height=6.5in,keepaspectratio]{../images/99.png}
\end{center}
\raggedright{}\myfigurewithoutcaption{99}
\end{minipage}\vspace{0.75cm}



\end{longtable}

Page breaks before display maths (of all various forms) are controlled by \LaTeXTT{\textbackslash{}predisplaypenalty}.  Its default 10000 means never break immediately before a display.  Knuth ({\itshape \setmainfont[Path=/usr/share/fonts/truetype/cmu/,UprightFont=cmunrm.ttf,BoldFont=cmunbx.ttf,ItalicFont=cmunti.ttf,BoldItalicFont=cmunbi.ttf]{cmunti.ttf}\setmonofont[Path=/usr/share/fonts/truetype/cmu/,UprightFont=cmuntt.ttf,BoldFont=cmuntb.ttf,ItalicFont=cmunit.ttf,BoldItalicFont=cmuntx.ttf]{cmunti.ttf}\itshape TeXbook}{$\text{ }$}\setmainfont[Path=/usr/share/fonts/truetype/cmu/,UprightFont=cmunrm.ttf,BoldFont=cmunbx.ttf,ItalicFont=cmunti.ttf,BoldItalicFont=cmunbi.ttf]{cmunrm.ttf}\setmonofont[Path=/usr/share/fonts/truetype/cmu/,UprightFont=cmuntt.ttf,BoldFont=cmuntb.ttf,ItalicFont=cmunit.ttf,BoldItalicFont=cmuntx.ttf]{cmunrm.ttf} chapter 19) explains this as a printers\textquotesingle{} tradition not to have a displayed equation at the start of a page.  It can be relaxed with
\begin{Shaded}
\begin{Highlighting}[]

\NormalTok{\textbackslash{}predisplaypenalty=0}
\end{Highlighting}
\end{Shaded}

Sometimes an equation might look best kept together preceding text by a higher penalty, for example a single-{}line paragraph about a single-{}line equation, especially at the end of a section.
\section{Boxed Equations}
\label{545}
For a single equation or alignment building block, with the tag outside the box, use \LaTeXTT{\textbackslash{}boxed\{\}}:
\begin{longtable}{p{1.0\linewidth}}
\begin{Shaded}
\begin{Highlighting}[]

\NormalTok{\textbackslash{}begin\{equation\}}
 \NormalTok{\textbackslash{}boxed\{x^2+y^2 = z^2\}}
\NormalTok{\textbackslash{}end\{equation\}}
\end{Highlighting}
\end{Shaded}
\\



\begin{minipage}{0.62500\textwidth}
\begin{center}
\includegraphics[width=1.0\textwidth,height=6.5in,keepaspectratio]{../images/100.png}
\end{center}
\raggedright{}\myfigurewithoutcaption{100}
\end{minipage}\vspace{0.75cm}



\end{longtable}

If you want the entire line or several equations to be boxed, use a \LaTeXTT{minipage} inside an \LaTeXTT{\textbackslash{}fbox\{\}}:
\begin{longtable}{p{1.0\linewidth}}
\begin{Shaded}
\begin{Highlighting}[]

\NormalTok{\textbackslash{}fbox\{}
 \NormalTok{\textbackslash{}addtolength\{\textbackslash{}linewidth\}\{-2\textbackslash{}fboxsep\}}\CommentTok
 \NormalTok{\textbackslash{}begin\{minipage\}\{\textbackslash{}linewidth\}}
  \NormalTok{\textbackslash{}begin\{equation\}}
   \NormalTok{x^2+y^2=z^2}
  \NormalTok{\textbackslash{}end\{equation\}}
 \NormalTok{\textbackslash{}end\{minipage\}}
\NormalTok{\}}
\end{Highlighting}
\end{Shaded}
\\



\begin{minipage}{1.0\linewidth}
\begin{center}
\includegraphics[width=1.0\linewidth,height=6.5in,keepaspectratio]{../images/101.png}
\end{center}
\raggedright{}\myfigurewithoutcaption{101}
\end{minipage}\vspace{0.75cm}



\end{longtable}

There is also the mathtools \LaTeXTT{\textbackslash{}Aboxed\{\}} which is able to box across alignment marks:
\begin{longtable}{p{1.0\linewidth}}
\begin{Shaded}
\begin{Highlighting}[]

\NormalTok{\textbackslash{}begin\{align*\}}
\NormalTok{\textbackslash{}Aboxed\{ f(x) & = \textbackslash{}int h(x)\textbackslash{}, dx\} \textbackslash{}\textbackslash{}}
              \NormalTok{& = g(x)}
\NormalTok{\textbackslash{}end\{align*\}}
\end{Highlighting}
\end{Shaded}
\\


\begin{minipage}{0.45000\textwidth}
\begin{center}
\includegraphics[width=1.0\textwidth,height=6.5in,keepaspectratio]{../images/102.png}
\end{center}
\raggedright{}\myfigurewithoutcaption{102}
\end{minipage}\vspace{0.75cm}



\end{longtable}
\section{Custom operators}
\label{546}
Although many common \mylref{502}{operators} are available in LaTeX, sometimes you will need to write your own, e.g. to typeset the \myhref{https://en.wikipedia.org/wiki/Arg\%20max}{argmax} operator. The \LaTeXTT{\textbackslash{}operatorname} and \LaTeXTT{\textbackslash{}operatorname*} commands\myfootnote{requires the \LaTeXTT{amsmath} package} display custom operators; the {\ttfamily \setmainfont[Path=/usr/share/fonts/truetype/cmu/,UprightFont=cmunrm.ttf,BoldFont=cmunbx.ttf,ItalicFont=cmunti.ttf,BoldItalicFont=cmunbi.ttf]{cmuntt.ttf}\setmonofont[Path=/usr/share/fonts/truetype/cmu/,UprightFont=cmuntt.ttf,BoldFont=cmuntb.ttf,ItalicFont=cmunit.ttf,BoldItalicFont=cmuntx.ttf]{cmuntt.ttf}\ttfamily *}{$\text{ }$}\setmainfont[Path=/usr/share/fonts/truetype/cmu/,UprightFont=cmunrm.ttf,BoldFont=cmunbx.ttf,ItalicFont=cmunti.ttf,BoldItalicFont=cmunbi.ttf]{cmunrm.ttf}\setmonofont[Path=/usr/share/fonts/truetype/cmu/,UprightFont=cmuntt.ttf,BoldFont=cmuntb.ttf,ItalicFont=cmunit.ttf,BoldItalicFont=cmuntx.ttf]{cmunrm.ttf} version sets the underscored option underneath like the \LaTeXTT{\textbackslash{}lim} operator:
\begin{longtable}{p{1.0\linewidth}}
\begin{Shaded}
\begin{Highlighting}[]

\NormalTok{\textbackslash{}[}
 \NormalTok{\textbackslash{}operatorname\{arg\textbackslash{},max\}_a f(a) }
 \NormalTok{= \textbackslash{}operatorname*\{arg\textbackslash{},max\}_b f(b)}
\NormalTok{\textbackslash{}]}
\end{Highlighting}
\end{Shaded}
\\

{$\operatorname{arg\,max}_a f(a) = \operatorname*{arg\,max}_b f(b)$}

\end{longtable}

However, if the operator is frequently used, it is preferable to define a new operator that can be used throughout the entire document. The \LaTeXTT{\textbackslash{}DeclareMathOperator} and \LaTeXTT{\textbackslash{}DeclareMathOperator*} commands\myfootnote{requires the \LaTeXTT{amsmath} package} are specified in the header of the document:
\begin{Shaded}
\begin{Highlighting}[]

\NormalTok{\textbackslash{}DeclareMathOperator*\{\textbackslash{}argmax\}\{arg\textbackslash{},max\}}
\end{Highlighting}
\end{Shaded}

This defines a new command which may be referred to in the body:
\begin{longtable}{p{1.0\linewidth}}
\begin{Shaded}
\begin{Highlighting}[]

\NormalTok{\textbackslash{}[}
 \NormalTok{\textbackslash{}argmax_c f(c)}
\NormalTok{\textbackslash{}]}
\end{Highlighting}
\end{Shaded}
\\

{$\underset{c}{\operatorname{arg\,max}} f(c)$}

\end{longtable}
\section{Advanced formatting}
\label{547}\subsection{Limits}
\label{548}
There are defaults for placement of subscripts and superscripts.  For example, limits for the {\ttfamily \setmainfont[Path=/usr/share/fonts/truetype/cmu/,UprightFont=cmunrm.ttf,BoldFont=cmunbx.ttf,ItalicFont=cmunti.ttf,BoldItalicFont=cmunbi.ttf]{cmuntt.ttf}\setmonofont[Path=/usr/share/fonts/truetype/cmu/,UprightFont=cmuntt.ttf,BoldFont=cmuntb.ttf,ItalicFont=cmunit.ttf,BoldItalicFont=cmuntx.ttf]{cmuntt.ttf}\ttfamily lim}{$\text{ }$}\setmainfont[Path=/usr/share/fonts/truetype/cmu/,UprightFont=cmunrm.ttf,BoldFont=cmunbx.ttf,ItalicFont=cmunti.ttf,BoldItalicFont=cmunbi.ttf]{cmunrm.ttf}\setmonofont[Path=/usr/share/fonts/truetype/cmu/,UprightFont=cmuntt.ttf,BoldFont=cmuntb.ttf,ItalicFont=cmunit.ttf,BoldItalicFont=cmuntx.ttf]{cmunrm.ttf} operator are usually placed below the symbol:
\begin{longtable}{p{1.0\linewidth}}
\begin{Shaded}
\begin{Highlighting}[]

\NormalTok{\textbackslash{}begin\{equation\}}
  \NormalTok{\textbackslash{}lim_\{a\textbackslash{}to \textbackslash{}infty\} \textbackslash{}tfrac\{1\}\{a\}}
\NormalTok{\textbackslash{}end\{equation\}}
\end{Highlighting}
\end{Shaded}
\\

{$\lim_{a\to \infty} \tfrac{1}{a}$}

\end{longtable}

To override this behavior, use the \LaTeXTT{\textbackslash{}nolimits} operator:
\begin{longtable}{p{1.0\linewidth}}
\begin{Shaded}
\begin{Highlighting}[]

\NormalTok{\textbackslash{}begin\{equation\}}
  \NormalTok{\textbackslash{}lim\textbackslash{}nolimits_\{a\textbackslash{}to \textbackslash{}infty\} \textbackslash{}tfrac\{1\}\{a\}}
\NormalTok{\textbackslash{}end\{equation\}}
\end{Highlighting}
\end{Shaded}
\\

{$\lim\nolimits_{a\to \infty} \tfrac{1}{a}$}

\end{longtable}

A {\ttfamily \setmainfont[Path=/usr/share/fonts/truetype/cmu/,UprightFont=cmunrm.ttf,BoldFont=cmunbx.ttf,ItalicFont=cmunti.ttf,BoldItalicFont=cmunbi.ttf]{cmuntt.ttf}\setmonofont[Path=/usr/share/fonts/truetype/cmu/,UprightFont=cmuntt.ttf,BoldFont=cmuntb.ttf,ItalicFont=cmunit.ttf,BoldItalicFont=cmuntx.ttf]{cmuntt.ttf}\ttfamily lim}{$\text{ }$}\setmainfont[Path=/usr/share/fonts/truetype/cmu/,UprightFont=cmunrm.ttf,BoldFont=cmunbx.ttf,ItalicFont=cmunti.ttf,BoldItalicFont=cmunbi.ttf]{cmunrm.ttf}\setmonofont[Path=/usr/share/fonts/truetype/cmu/,UprightFont=cmuntt.ttf,BoldFont=cmuntb.ttf,ItalicFont=cmunit.ttf,BoldItalicFont=cmuntx.ttf]{cmunrm.ttf} in running text (inside \LaTeXTT{\${}...\${}}) will have its limits placed on the side, so that additional leading won\textquotesingle{}t be required. To override this behavior, use the \LaTeXTT{\textbackslash{}limits} command.

Similarly one can put subscripts under a symbol that usually has them on the side:
\begin{longtable}{p{1.0\linewidth}}
\begin{Shaded}
\begin{Highlighting}[]

\NormalTok{\textbackslash{}begin\{equation\}}
  \NormalTok{\textbackslash{}int_a^b x^2  \textbackslash{}mathrm\{d\} x}
\NormalTok{\textbackslash{}end\{equation\}}
\end{Highlighting}
\end{Shaded}
\\

{$\int_a^b x^2  \mathrm{d} x$}

\end{longtable}

Limits below and under:
\begin{longtable}{p{1.0\linewidth}}
\begin{Shaded}
\begin{Highlighting}[]

\NormalTok{\textbackslash{}begin\{equation\}}
  \NormalTok{\textbackslash{}int\textbackslash{}limits_a^b x^2  \textbackslash{}mathrm\{d\} x}
\NormalTok{\textbackslash{}end\{equation\}}
\end{Highlighting}
\end{Shaded}
\\

{$\int\limits_a^b x^2  \mathrm{d} x$}

\end{longtable}

To change the default placement of summation-{}type symbols to the side for every case, add the \LaTeXTT{nosumlimits} option to the \LaTeXTT{amsmath} package. To change the placement for integral symbols, add \LaTeXTT{intlimits} to the options. \LaTeXTT{nonamelimits} can be used to change the default for named operators like {\ttfamily \setmainfont[Path=/usr/share/fonts/truetype/cmu/,UprightFont=cmunrm.ttf,BoldFont=cmunbx.ttf,ItalicFont=cmunti.ttf,BoldItalicFont=cmunbi.ttf]{cmuntt.ttf}\setmonofont[Path=/usr/share/fonts/truetype/cmu/,UprightFont=cmuntt.ttf,BoldFont=cmuntb.ttf,ItalicFont=cmunit.ttf,BoldItalicFont=cmuntx.ttf]{cmuntt.ttf}\ttfamily det}\setmainfont[Path=/usr/share/fonts/truetype/cmu/,UprightFont=cmunrm.ttf,BoldFont=cmunbx.ttf,ItalicFont=cmunti.ttf,BoldItalicFont=cmunbi.ttf]{cmunrm.ttf}\setmonofont[Path=/usr/share/fonts/truetype/cmu/,UprightFont=cmuntt.ttf,BoldFont=cmuntb.ttf,ItalicFont=cmunit.ttf,BoldItalicFont=cmuntx.ttf]{cmunrm.ttf}, {\ttfamily \setmainfont[Path=/usr/share/fonts/truetype/cmu/,UprightFont=cmunrm.ttf,BoldFont=cmunbx.ttf,ItalicFont=cmunti.ttf,BoldItalicFont=cmunbi.ttf]{cmuntt.ttf}\setmonofont[Path=/usr/share/fonts/truetype/cmu/,UprightFont=cmuntt.ttf,BoldFont=cmuntb.ttf,ItalicFont=cmunit.ttf,BoldItalicFont=cmuntx.ttf]{cmuntt.ttf}\ttfamily min}\setmainfont[Path=/usr/share/fonts/truetype/cmu/,UprightFont=cmunrm.ttf,BoldFont=cmunbx.ttf,ItalicFont=cmunti.ttf,BoldItalicFont=cmunbi.ttf]{cmunrm.ttf}\setmonofont[Path=/usr/share/fonts/truetype/cmu/,UprightFont=cmuntt.ttf,BoldFont=cmuntb.ttf,ItalicFont=cmunit.ttf,BoldItalicFont=cmuntx.ttf]{cmunrm.ttf}, {\ttfamily \setmainfont[Path=/usr/share/fonts/truetype/cmu/,UprightFont=cmunrm.ttf,BoldFont=cmunbx.ttf,ItalicFont=cmunti.ttf,BoldItalicFont=cmunbi.ttf]{cmuntt.ttf}\setmonofont[Path=/usr/share/fonts/truetype/cmu/,UprightFont=cmuntt.ttf,BoldFont=cmuntb.ttf,ItalicFont=cmunit.ttf,BoldItalicFont=cmuntx.ttf]{cmuntt.ttf}\ttfamily lim}\setmainfont[Path=/usr/share/fonts/truetype/cmu/,UprightFont=cmunrm.ttf,BoldFont=cmunbx.ttf,ItalicFont=cmunti.ttf,BoldItalicFont=cmunbi.ttf]{cmunrm.ttf}\setmonofont[Path=/usr/share/fonts/truetype/cmu/,UprightFont=cmuntt.ttf,BoldFont=cmuntb.ttf,ItalicFont=cmunit.ttf,BoldItalicFont=cmuntx.ttf]{cmunrm.ttf}, etc.

To produce one-{}sided limits, use \LaTeXTT{\textbackslash{}underset} as follows:
\begin{longtable}{p{1.0\linewidth}}
\begin{Shaded}
\begin{Highlighting}[]

\NormalTok{\textbackslash{}begin\{equation\}}
  \NormalTok{\textbackslash{}lim_\{a \textbackslash{}underset\{>\}\{\textbackslash{}to\} 0\} \textbackslash{}frac\{1\}\{a\}}
\NormalTok{\textbackslash{}end\{equation\}}
\end{Highlighting}
\end{Shaded}
\\

{$\lim_{a \underset{>}{\to} 0} \frac{1}{a}$}

\end{longtable}
\subsection{Subscripts and superscripts}
\label{549}

You can place symbols in subscript or superscript (in summation style symbols) with \LaTeXTT{\textbackslash{}nolimits}:
\begin{longtable}{p{1.0\linewidth}}
\begin{Shaded}
\begin{Highlighting}[]

\NormalTok{\textbackslash{}begin\{equation\}}
  \NormalTok{\textbackslash{}sum\textbackslash{}nolimits' C_n}
\NormalTok{\textbackslash{}end\{equation\}}
\end{Highlighting}
\end{Shaded}
\\

{$\sum\nolimits' C_n$}

\end{longtable}

It\textquotesingle{}s impossible to mix them with typical usage of such symbols:
\begin{longtable}{p{1.0\linewidth}}
\begin{Shaded}
\begin{Highlighting}[]

\NormalTok{\textbackslash{}begin\{equation\}}
  \NormalTok{\textbackslash{}sum_\{n=1\}\textbackslash{}nolimits' C_n}
\NormalTok{\textbackslash{}end\{equation\}}
\end{Highlighting}
\end{Shaded}
\\

{$\sum_{n=1}\nolimits' C_n$}

\end{longtable}

To add both a prime and a limit to a symbol, one might use the \LaTeXTT{\textbackslash{}sideset} command:
\begin{longtable}{p{1.0\linewidth}}
\begin{Shaded}
\begin{Highlighting}[]

\NormalTok{\textbackslash{}begin\{equation\}}
  \NormalTok{\textbackslash{}sideset\{\}\{'\}\textbackslash{}sum_\{n=1\}C_n}
\NormalTok{\textbackslash{}end\{equation\}}
\end{Highlighting}
\end{Shaded}
\\

{$\sideset{}{'}\sum_{n=1}C_n$}

\end{longtable}

It is very flexible: for example, to put letters in each corner of the symbol use this command:
\begin{longtable}{p{1.0\linewidth}}
\begin{Shaded}
\begin{Highlighting}[]

\NormalTok{\textbackslash{}begin\{equation\}}
  \NormalTok{\textbackslash{}sideset\{_a^b\}\{_c^d\}\textbackslash{}sum}
\NormalTok{\textbackslash{}end\{equation\}}
\end{Highlighting}
\end{Shaded}
\\

{$\sideset{_a^b}{_c^d}\sum$}

\end{longtable}

If you wish to place them on the corners of an arbitrary symbol, you should use \LaTeXTT{\textbackslash{}fourIdx} from the \LaTeXTT{fouridx} package.

But a simple grouping can also solve the problem:
\begin{longtable}{p{1.0\linewidth}}
\begin{Shaded}
\begin{Highlighting}[]

\NormalTok{\textbackslash{}begin\{equation\}}
  \NormalTok{\{\textbackslash{}sum\textbackslash{}limits_\{n=1\} \}'C_n}
\NormalTok{\textbackslash{}end\{equation\}}
\end{Highlighting}
\end{Shaded}
\\

{${\sum\limits_{n=1} }'C_n$}

\end{longtable}
since a math operator can be used with limits or no limits. If you want to change its state, simply group it. You can make it another math operator if you want, and then you can have limits and then limits again.
\subsection{Multiline subscripts}
\label{550}
To produce multiline subscript, use the \LaTeXTT{\textbackslash{}substack} command:
\begin{longtable}{p{1.0\linewidth}}
\begin{Shaded}
\begin{Highlighting}[]

\NormalTok{\textbackslash{}begin\{equation\}}
  \NormalTok{\textbackslash{}prod_\{\textbackslash{}substack\{}
            \NormalTok{1\textbackslash{}le i \textbackslash{}le n\textbackslash{}\textbackslash{}}
            \NormalTok{1\textbackslash{}le j \textbackslash{}le m\}<!---->\}}
     \NormalTok{M_\{i,j\}}
\NormalTok{\textbackslash{}end\{equation\}}
\end{Highlighting}
\end{Shaded}
\\

{$\prod_{1 \le i \le n \atop 1 \le j \le m \ } M_{i,j}$}

\end{longtable}
\section{Text in aligned math display}
\label{551}
To add small interjections in math environments, use the \LaTeXTT{\textbackslash{}intertext} command:
\begin{longtable}{p{1.0\linewidth}}
\begin{Shaded}
\begin{Highlighting}[]

\NormalTok{\textbackslash{}begin\{minipage\}\{3in\}}
\NormalTok{\textbackslash{}begin\{align*\}}
\NormalTok{\textbackslash{}intertext\{If\}}
   \NormalTok{A &= \textbackslash{}sigma_1+\textbackslash{}sigma_2\textbackslash{}\textbackslash{}}
   \NormalTok{B &= \textbackslash{}rho_1+\textbackslash{}rho_2\textbackslash{}\textbackslash{}}
\NormalTok{\textbackslash{}intertext\{then\}}
\NormalTok{C(x) &= e^\{Ax^2+\textbackslash{}pi\}+B}
\NormalTok{\textbackslash{}end\{align*\} }
\NormalTok{\textbackslash{}end\{minipage\}}
\end{Highlighting}
\end{Shaded}
\\



\begin{minipage}{0.62500\textwidth}
\begin{center}
\includegraphics[width=1.0\textwidth,height=6.5in,keepaspectratio]{../images/103.png}
\end{center}
\raggedright{}\myfigurewithoutcaption{103}
\end{minipage}\vspace{0.75cm}



\end{longtable}
Note that any usage of this command does not change the alignment.

Also, in the above example, the command \LaTeXTT{\textbackslash{}shortintertext\{\}} from the \LaTeXTT{mathtools} package could have been used instead of intertext to reduce the amount of vertical white space added between the lines.
\section{Changing font size}
\label{552}

There may be a time when you would prefer to have some control over the font size. For example, using text-{}mode maths, by default a simple fraction will look like this: {$\textstyle \frac{a}{b}$}, whereas you may prefer to have it displayed larger, like when in display mode, but still keeping it in-{}line, like this: {$\displaystyle \frac{a}{b} $}.

A simple approach is to utilize the predefined sizes for maths elements:

\begin{longtable}{|>{\RaggedRight}p{0.41355\linewidth}|>{\RaggedRight}p{0.52930\linewidth}|} \hline 
{\bfseries \hspace*{0pt}\ignorespaces{}\hspace*{0pt} Size command}&{\bfseries \hspace*{0pt}\ignorespaces{}\hspace*{0pt} Description}\endhead  \hline \hspace*{0pt}\ignorespaces{}\hspace*{0pt} \LaTeXTT{\textbackslash{}displaystyle}&\hspace*{0pt}\ignorespaces{}\hspace*{0pt} Size for equations in display mode\\ \hline \hspace*{0pt}\ignorespaces{}\hspace*{0pt} \LaTeXTT{\textbackslash{}textstyle}&\hspace*{0pt}\ignorespaces{}\hspace*{0pt} Size for equations in text mode\\ \hline \hspace*{0pt}\ignorespaces{}\hspace*{0pt} \LaTeXTT{\textbackslash{}scriptstyle}&\hspace*{0pt}\ignorespaces{}\hspace*{0pt} Size for first sub/superscripts\\ \hline \hspace*{0pt}\ignorespaces{}\hspace*{0pt} \LaTeXTT{\textbackslash{}scriptscriptstyle}&\hspace*{0pt}\ignorespaces{}\hspace*{0pt} Size for subsequent sub/superscripts\\ \hline 
\end{longtable}


A classic example to see this in use is typesetting continued fractions (though it\textquotesingle{}s better to use the \LaTeXTT{\textbackslash{}cfrac} command\myfootnote{requires the \LaTeXTT{amsmath} package} described in the \mylref{505}{Mathematics} chapter instead of the method provided below). The following code provides an example.
\begin{longtable}{p{1.0\linewidth}}
\begin{Shaded}
\begin{Highlighting}[]

\NormalTok{\textbackslash{}begin\{equation\}}
  \NormalTok{x = a_0 + \textbackslash{}frac\{1\}\{a_1 + \textbackslash{}frac\{1\}\{a_2 + \textbackslash{}frac\{1\}\{a_3 + a_4\}<!---->\}<!---->\}}
\NormalTok{\textbackslash{}end\{equation\}}
\end{Highlighting}
\end{Shaded}
\\

{$x = a_0 + \frac{1}{a_1 + \frac{1}{a_2 + \frac{1}{a_3 + a_4}}}$}

\end{longtable}

As you can see, as the fractions continue, they get smaller (although they will not get any smaller than in this example, where they have reached the \LaTeXTT{\textbackslash{}scriptstyle} limit). If you want to keep the size consistent, you could declare each fraction to use the display style instead; e.g.
\begin{longtable}{p{1.0\linewidth}}
\begin{Shaded}
\begin{Highlighting}[]

\NormalTok{\textbackslash{}begin\{equation\}}
  \NormalTok{x = a_0 + \textbackslash{}frac\{1\}\{\textbackslash{}displaystyle a_1 }
          \NormalTok{+ \textbackslash{}frac\{1\}\{\textbackslash{}displaystyle a_2 }
          \NormalTok{+ \textbackslash{}frac\{1\}\{\textbackslash{}displaystyle a_3 + a_4\}<!---->\}<!---->\}}
\NormalTok{\textbackslash{}end\{equation\}}
\end{Highlighting}
\end{Shaded}
\\

{$   x = a_0 + \frac{1}{\displaystyle a_1             + \frac{1}{\displaystyle a_2             + \frac{1}{\displaystyle a_3 + a_4}}} $}

\end{longtable}

Another approach is to use the \LaTeXTT{\textbackslash{}DeclareMathSizes} command to select your preferred sizes. You can only define sizes for \LaTeXTT{\textbackslash{}displaystyle}, \LaTeXTT{\textbackslash{}textstyle}, etc. One potential downside is that this command sets the global maths sizes, as it can only be used in the document preamble.

But it\textquotesingle{}s fairly easy to use: \LaTeXTT{\textbackslash{}DeclareMathSizes\{ds\}\{ts\}\{ss\}\{sss\}}, where {\itshape \setmainfont[Path=/usr/share/fonts/truetype/cmu/,UprightFont=cmunrm.ttf,BoldFont=cmunbx.ttf,ItalicFont=cmunti.ttf,BoldItalicFont=cmunbi.ttf]{cmunti.ttf}\setmonofont[Path=/usr/share/fonts/truetype/cmu/,UprightFont=cmuntt.ttf,BoldFont=cmuntb.ttf,ItalicFont=cmunit.ttf,BoldItalicFont=cmuntx.ttf]{cmunti.ttf}\itshape ds}{$\text{ }$}\setmainfont[Path=/usr/share/fonts/truetype/cmu/,UprightFont=cmunrm.ttf,BoldFont=cmunbx.ttf,ItalicFont=cmunti.ttf,BoldItalicFont=cmunbi.ttf]{cmunrm.ttf}\setmonofont[Path=/usr/share/fonts/truetype/cmu/,UprightFont=cmuntt.ttf,BoldFont=cmuntb.ttf,ItalicFont=cmunit.ttf,BoldItalicFont=cmuntx.ttf]{cmunrm.ttf} is the {\itshape \setmainfont[Path=/usr/share/fonts/truetype/cmu/,UprightFont=cmunrm.ttf,BoldFont=cmunbx.ttf,ItalicFont=cmunti.ttf,BoldItalicFont=cmunbi.ttf]{cmunti.ttf}\setmonofont[Path=/usr/share/fonts/truetype/cmu/,UprightFont=cmuntt.ttf,BoldFont=cmuntb.ttf,ItalicFont=cmunit.ttf,BoldItalicFont=cmuntx.ttf]{cmunti.ttf}\itshape display size}\setmainfont[Path=/usr/share/fonts/truetype/cmu/,UprightFont=cmunrm.ttf,BoldFont=cmunbx.ttf,ItalicFont=cmunti.ttf,BoldItalicFont=cmunbi.ttf]{cmunrm.ttf}\setmonofont[Path=/usr/share/fonts/truetype/cmu/,UprightFont=cmuntt.ttf,BoldFont=cmuntb.ttf,ItalicFont=cmunit.ttf,BoldItalicFont=cmuntx.ttf]{cmunrm.ttf}, {\itshape \setmainfont[Path=/usr/share/fonts/truetype/cmu/,UprightFont=cmunrm.ttf,BoldFont=cmunbx.ttf,ItalicFont=cmunti.ttf,BoldItalicFont=cmunbi.ttf]{cmunti.ttf}\setmonofont[Path=/usr/share/fonts/truetype/cmu/,UprightFont=cmuntt.ttf,BoldFont=cmuntb.ttf,ItalicFont=cmunit.ttf,BoldItalicFont=cmuntx.ttf]{cmunti.ttf}\itshape ts}{$\text{ }$}\setmainfont[Path=/usr/share/fonts/truetype/cmu/,UprightFont=cmunrm.ttf,BoldFont=cmunbx.ttf,ItalicFont=cmunti.ttf,BoldItalicFont=cmunbi.ttf]{cmunrm.ttf}\setmonofont[Path=/usr/share/fonts/truetype/cmu/,UprightFont=cmuntt.ttf,BoldFont=cmuntb.ttf,ItalicFont=cmunit.ttf,BoldItalicFont=cmuntx.ttf]{cmunrm.ttf} is the {\itshape \setmainfont[Path=/usr/share/fonts/truetype/cmu/,UprightFont=cmunrm.ttf,BoldFont=cmunbx.ttf,ItalicFont=cmunti.ttf,BoldItalicFont=cmunbi.ttf]{cmunti.ttf}\setmonofont[Path=/usr/share/fonts/truetype/cmu/,UprightFont=cmuntt.ttf,BoldFont=cmuntb.ttf,ItalicFont=cmunit.ttf,BoldItalicFont=cmuntx.ttf]{cmunti.ttf}\itshape text size}\setmainfont[Path=/usr/share/fonts/truetype/cmu/,UprightFont=cmunrm.ttf,BoldFont=cmunbx.ttf,ItalicFont=cmunti.ttf,BoldItalicFont=cmunbi.ttf]{cmunrm.ttf}\setmonofont[Path=/usr/share/fonts/truetype/cmu/,UprightFont=cmuntt.ttf,BoldFont=cmuntb.ttf,ItalicFont=cmunit.ttf,BoldItalicFont=cmuntx.ttf]{cmunrm.ttf}, etc. The values you input are assumed to be point (pt) size.

Note that the changes only take place if the value in the first argument matches the current document text size. It is therefore common to see a set of declarations in the preamble, in the event of the main font being changed. E.g.,
\begin{Shaded}
\begin{Highlighting}[]

\NormalTok{\textbackslash{}DeclareMathSizes\{10\}\{18\}\{12\}\{8\}   }\CommentTok{% For size 10 text}
\NormalTok{\textbackslash{}DeclareMathSizes\{11\}\{19\}\{13\}\{9\}   }\CommentTok{% For size 11 text}
\NormalTok{\textbackslash{}DeclareMathSizes\{12\}\{20\}\{14\}\{10\}  }\CommentTok{% For size 12 text}
\end{Highlighting}
\end{Shaded}

\section{Forcing \textbackslash{}displaystyle for all math in a document}
\label{553}
Put
\begin{Shaded}
\begin{Highlighting}[]

\NormalTok{\textbackslash{}everymath\{\textbackslash{}displaystyle\}}
\end{Highlighting}
\end{Shaded}

before \begin{Shaded}
\begin{Highlighting}[]
 \NormalTok{\textbackslash{}begin\{document\} }
\end{Highlighting}
\end{Shaded}
 to force all math to \begin{Shaded}
\begin{Highlighting}[]
 \NormalTok{\textbackslash{}displaystyle }
\end{Highlighting}
\end{Shaded}
.
\section{Adjusting vertical white space around displayed math}
\label{554}
There are four parameters that control the vertical white space around displayed math:
\begin{Shaded}
\begin{Highlighting}[]

\NormalTok{\textbackslash{}abovedisplayskip=12pt}
\NormalTok{\textbackslash{}belowdisplayskip=12pt}
\NormalTok{\textbackslash{}abovedisplayshortskip=0pt}
\NormalTok{\textbackslash{}belowdisplayshortskip=7pt}
\end{Highlighting}
\end{Shaded}

Short skips are used if the preceding line ends, horizontally, before the formula. These parameters must be set after \begin{Shaded}
\begin{Highlighting}[]
 \NormalTok{\textbackslash{}begin\{document\} }
\end{Highlighting}
\end{Shaded}
.


\LaTeXNullTemplate{}
\section{Notes}
\label{555}


\chapter{Theorems}

\myminitoc
\label{556}

\label{557}


With \symbol{34}\myhref{https://en.wikipedia.org/wiki/Theorem}{theorem}\symbol{34} we can mean any kind of labelled enunciation that we want to look separated from the rest of the text and with sequential numbers next to it. This approach is commonly used for theorems in mathematics, but can be used for anything. LaTeX provides a command that will let you easily define any theorem-{}like enunciation.
\section{Basic theorems}
\label{558}
First of all, make sure you have the amsthm package enabled:


\begin{Shaded}
\begin{Highlighting}[]

\NormalTok{\textbackslash{}usepackage\{amsthm\}}\newline
\end{Highlighting}
\end{Shaded}


The easiest is the following:

\begin{Shaded}
\begin{Highlighting}[]

\NormalTok{\textbackslash{}newtheorem\{name\}\{Printed\ensuremath{\text{ }}output\}}\newline
\end{Highlighting}
\end{Shaded}

put it in the preamble. The first argument is the name you will use to reference it, the second argument is the output LaTeX will print whenever you use it. For example:

\begin{Shaded}
\begin{Highlighting}[]

\NormalTok{\textbackslash{}newtheorem\{mydef\}\{Definition\}}\newline
\end{Highlighting}
\end{Shaded}


will define the {\ttfamily \setmainfont[Path=/usr/share/fonts/truetype/cmu/,UprightFont=cmunrm.ttf,BoldFont=cmunbx.ttf,ItalicFont=cmunti.ttf,BoldItalicFont=cmunbi.ttf]{cmuntt.ttf}\setmonofont[Path=/usr/share/fonts/truetype/cmu/,UprightFont=cmuntt.ttf,BoldFont=cmuntb.ttf,ItalicFont=cmunit.ttf,BoldItalicFont=cmuntx.ttf]{cmuntt.ttf}\ttfamily mydef}{$\text{ }$}\setmainfont[Path=/usr/share/fonts/truetype/cmu/,UprightFont=cmunrm.ttf,BoldFont=cmunbx.ttf,ItalicFont=cmunti.ttf,BoldItalicFont=cmunbi.ttf]{cmunrm.ttf}\setmonofont[Path=/usr/share/fonts/truetype/cmu/,UprightFont=cmuntt.ttf,BoldFont=cmuntb.ttf,ItalicFont=cmunit.ttf,BoldItalicFont=cmuntx.ttf]{cmunrm.ttf} environment; if you use it like this:

\begin{Shaded}
\begin{Highlighting}[]

\NormalTok{\textbackslash{}begin\{mydef\}}\newline
\NormalTok{Here\ensuremath{\text{ }}is\ensuremath{\text{ }}a\ensuremath{\text{ }}new\ensuremath{\text{ }}definition}\newline
\NormalTok{\textbackslash{}end\{mydef\}}\newline
\end{Highlighting}
\end{Shaded}


It will look like this:
\begin{myquote}
\item{} {\bfseries \setmainfont[Path=/usr/share/fonts/truetype/cmu/,UprightFont=cmunrm.ttf,BoldFont=cmunbx.ttf,ItalicFont=cmunti.ttf,BoldItalicFont=cmunbi.ttf]{cmunbx.ttf}\setmonofont[Path=/usr/share/fonts/truetype/cmu/,UprightFont=cmuntt.ttf,BoldFont=cmuntb.ttf,ItalicFont=cmunit.ttf,BoldItalicFont=cmuntx.ttf]{cmunbx.ttf}\bfseries Definition 3}{$\text{ }$}\setmainfont[Path=/usr/share/fonts/truetype/cmu/,UprightFont=cmunrm.ttf,BoldFont=cmunbx.ttf,ItalicFont=cmunti.ttf,BoldItalicFont=cmunbi.ttf]{cmunrm.ttf}\setmonofont[Path=/usr/share/fonts/truetype/cmu/,UprightFont=cmuntt.ttf,BoldFont=cmuntb.ttf,ItalicFont=cmunit.ttf,BoldItalicFont=cmuntx.ttf]{cmunrm.ttf} {\itshape \setmainfont[Path=/usr/share/fonts/truetype/cmu/,UprightFont=cmunrm.ttf,BoldFont=cmunbx.ttf,ItalicFont=cmunti.ttf,BoldItalicFont=cmunbi.ttf]{cmunti.ttf}\setmonofont[Path=/usr/share/fonts/truetype/cmu/,UprightFont=cmuntt.ttf,BoldFont=cmuntb.ttf,ItalicFont=cmunit.ttf,BoldItalicFont=cmuntx.ttf]{cmunti.ttf}\itshape Here is a new definition}
\end{myquote}
\setmainfont[Path=/usr/share/fonts/truetype/cmu/,UprightFont=cmunrm.ttf,BoldFont=cmunbx.ttf,ItalicFont=cmunti.ttf,BoldItalicFont=cmunbi.ttf]{cmunrm.ttf}\setmonofont[Path=/usr/share/fonts/truetype/cmu/,UprightFont=cmuntt.ttf,BoldFont=cmuntb.ttf,ItalicFont=cmunit.ttf,BoldItalicFont=cmuntx.ttf]{cmunrm.ttf}
with line breaks separating it from the rest of the text.
\section{Theorem counters}
\label{559}
Often the counters are determined by section, for example \symbol{34}Theorem 2.3\symbol{34} refers to the 3rd theorem in the 2nd section of a document. In this case, specify the theorem as follows:

\begin{Shaded}
\begin{Highlighting}[]

\NormalTok{\textbackslash{}newtheorem\{name\}\{Printed\ensuremath{\text{ }}output\}[numberby]}\newline
\end{Highlighting}
\end{Shaded}


where {\itshape \setmainfont[Path=/usr/share/fonts/truetype/cmu/,UprightFont=cmunrm.ttf,BoldFont=cmunbx.ttf,ItalicFont=cmunti.ttf,BoldItalicFont=cmunbi.ttf]{cmunti.ttf}\setmonofont[Path=/usr/share/fonts/truetype/cmu/,UprightFont=cmuntt.ttf,BoldFont=cmuntb.ttf,ItalicFont=cmunit.ttf,BoldItalicFont=cmuntx.ttf]{cmunti.ttf}\itshape numberby}{$\text{ }$}\setmainfont[Path=/usr/share/fonts/truetype/cmu/,UprightFont=cmunrm.ttf,BoldFont=cmunbx.ttf,ItalicFont=cmunti.ttf,BoldItalicFont=cmunbi.ttf]{cmunrm.ttf}\setmonofont[Path=/usr/share/fonts/truetype/cmu/,UprightFont=cmuntt.ttf,BoldFont=cmuntb.ttf,ItalicFont=cmunit.ttf,BoldItalicFont=cmuntx.ttf]{cmunrm.ttf} is the name of the \mylref{97}{section level} (section/subsection/etc.) at which the numbering is to take place.

By default, each theorem uses its own counter. However it is common for similar types of theorems (e.g. Theorems, Lemmas and Corollaries) to share a counter. In this case, define subsequent theorems as:

\begin{Shaded}
\begin{Highlighting}[]

\NormalTok{\textbackslash{}newtheorem\{name\}[counter]\{Printed\ensuremath{\text{ }}output\}}\newline
\end{Highlighting}
\end{Shaded}


where {\itshape \setmainfont[Path=/usr/share/fonts/truetype/cmu/,UprightFont=cmunrm.ttf,BoldFont=cmunbx.ttf,ItalicFont=cmunti.ttf,BoldItalicFont=cmunbi.ttf]{cmunti.ttf}\setmonofont[Path=/usr/share/fonts/truetype/cmu/,UprightFont=cmuntt.ttf,BoldFont=cmuntb.ttf,ItalicFont=cmunit.ttf,BoldItalicFont=cmuntx.ttf]{cmunti.ttf}\itshape counter}{$\text{ }$}\setmainfont[Path=/usr/share/fonts/truetype/cmu/,UprightFont=cmunrm.ttf,BoldFont=cmunbx.ttf,ItalicFont=cmunti.ttf,BoldItalicFont=cmunbi.ttf]{cmunrm.ttf}\setmonofont[Path=/usr/share/fonts/truetype/cmu/,UprightFont=cmuntt.ttf,BoldFont=cmuntb.ttf,ItalicFont=cmunit.ttf,BoldItalicFont=cmuntx.ttf]{cmunrm.ttf} is the name of the counter to be used. Usually this will be the name of the master theorem.

The \textbackslash{}newtheorem command may have at most one optional argument.

You can also create a theorem environment that is not numbered by using the {\ttfamily \setmainfont[Path=/usr/share/fonts/truetype/cmu/,UprightFont=cmunrm.ttf,BoldFont=cmunbx.ttf,ItalicFont=cmunti.ttf,BoldItalicFont=cmunbi.ttf]{cmuntt.ttf}\setmonofont[Path=/usr/share/fonts/truetype/cmu/,UprightFont=cmuntt.ttf,BoldFont=cmuntb.ttf,ItalicFont=cmunit.ttf,BoldItalicFont=cmuntx.ttf]{cmuntt.ttf}\ttfamily newtheorem*}{$\text{ }$}\setmainfont[Path=/usr/share/fonts/truetype/cmu/,UprightFont=cmunrm.ttf,BoldFont=cmunbx.ttf,ItalicFont=cmunti.ttf,BoldItalicFont=cmunbi.ttf]{cmunrm.ttf}\setmonofont[Path=/usr/share/fonts/truetype/cmu/,UprightFont=cmuntt.ttf,BoldFont=cmuntb.ttf,ItalicFont=cmunit.ttf,BoldItalicFont=cmuntx.ttf]{cmunrm.ttf} command\myfootnote{Requires the {\ttfamily \setmainfont[Path=/usr/share/fonts/truetype/cmu/,UprightFont=cmunrm.ttf,BoldFont=cmunbx.ttf,ItalicFont=cmunti.ttf,BoldItalicFont=cmunbi.ttf]{cmuntt.ttf}\setmonofont[Path=/usr/share/fonts/truetype/cmu/,UprightFont=cmuntt.ttf,BoldFont=cmuntb.ttf,ItalicFont=cmunit.ttf,BoldItalicFont=cmuntx.ttf]{cmuntt.ttf}\ttfamily amsthm}{$\text{ }$}\setmainfont[Path=/usr/share/fonts/truetype/cmu/,UprightFont=cmunrm.ttf,BoldFont=cmunbx.ttf,ItalicFont=cmunti.ttf,BoldItalicFont=cmunbi.ttf]{cmunrm.ttf}\setmonofont[Path=/usr/share/fonts/truetype/cmu/,UprightFont=cmuntt.ttf,BoldFont=cmuntb.ttf,ItalicFont=cmunit.ttf,BoldItalicFont=cmuntx.ttf]{cmunrm.ttf} package}.  For instance,

\begin{Shaded}
\begin{Highlighting}[]

\NormalTok{\textbackslash{}newtheorem*\{mydef\}\{Definition\}}\newline
\end{Highlighting}
\end{Shaded}

defines the {\ttfamily \setmainfont[Path=/usr/share/fonts/truetype/cmu/,UprightFont=cmunrm.ttf,BoldFont=cmunbx.ttf,ItalicFont=cmunti.ttf,BoldItalicFont=cmunbi.ttf]{cmuntt.ttf}\setmonofont[Path=/usr/share/fonts/truetype/cmu/,UprightFont=cmuntt.ttf,BoldFont=cmuntb.ttf,ItalicFont=cmunit.ttf,BoldItalicFont=cmuntx.ttf]{cmuntt.ttf}\ttfamily mydef}{$\text{ }$}\setmainfont[Path=/usr/share/fonts/truetype/cmu/,UprightFont=cmunrm.ttf,BoldFont=cmunbx.ttf,ItalicFont=cmunti.ttf,BoldItalicFont=cmunbi.ttf]{cmunrm.ttf}\setmonofont[Path=/usr/share/fonts/truetype/cmu/,UprightFont=cmuntt.ttf,BoldFont=cmuntb.ttf,ItalicFont=cmunit.ttf,BoldItalicFont=cmuntx.ttf]{cmunrm.ttf} environment, which will generate definitions without numbering. This requires {\ttfamily \setmainfont[Path=/usr/share/fonts/truetype/cmu/,UprightFont=cmunrm.ttf,BoldFont=cmunbx.ttf,ItalicFont=cmunti.ttf,BoldItalicFont=cmunbi.ttf]{cmuntt.ttf}\setmonofont[Path=/usr/share/fonts/truetype/cmu/,UprightFont=cmuntt.ttf,BoldFont=cmuntb.ttf,ItalicFont=cmunit.ttf,BoldItalicFont=cmuntx.ttf]{cmuntt.ttf}\ttfamily amsthm}{$\text{ }$}\setmainfont[Path=/usr/share/fonts/truetype/cmu/,UprightFont=cmunrm.ttf,BoldFont=cmunbx.ttf,ItalicFont=cmunti.ttf,BoldItalicFont=cmunbi.ttf]{cmunrm.ttf}\setmonofont[Path=/usr/share/fonts/truetype/cmu/,UprightFont=cmuntt.ttf,BoldFont=cmuntb.ttf,ItalicFont=cmunit.ttf,BoldItalicFont=cmuntx.ttf]{cmunrm.ttf} package.
\section{Proofs}
\label{560}
The {\ttfamily \setmainfont[Path=/usr/share/fonts/truetype/cmu/,UprightFont=cmunrm.ttf,BoldFont=cmunbx.ttf,ItalicFont=cmunti.ttf,BoldItalicFont=cmunbi.ttf]{cmuntt.ttf}\setmonofont[Path=/usr/share/fonts/truetype/cmu/,UprightFont=cmuntt.ttf,BoldFont=cmuntb.ttf,ItalicFont=cmunit.ttf,BoldItalicFont=cmuntx.ttf]{cmuntt.ttf}\ttfamily proof}{$\text{ }$}\setmainfont[Path=/usr/share/fonts/truetype/cmu/,UprightFont=cmunrm.ttf,BoldFont=cmunbx.ttf,ItalicFont=cmunti.ttf,BoldItalicFont=cmunbi.ttf]{cmunrm.ttf}\setmonofont[Path=/usr/share/fonts/truetype/cmu/,UprightFont=cmuntt.ttf,BoldFont=cmuntb.ttf,ItalicFont=cmunit.ttf,BoldItalicFont=cmuntx.ttf]{cmunrm.ttf} environment\myfootnote{Requires the {\ttfamily \setmainfont[Path=/usr/share/fonts/truetype/cmu/,UprightFont=cmunrm.ttf,BoldFont=cmunbx.ttf,ItalicFont=cmunti.ttf,BoldItalicFont=cmunbi.ttf]{cmuntt.ttf}\setmonofont[Path=/usr/share/fonts/truetype/cmu/,UprightFont=cmuntt.ttf,BoldFont=cmuntb.ttf,ItalicFont=cmunit.ttf,BoldItalicFont=cmuntx.ttf]{cmuntt.ttf}\ttfamily amsthm}{$\text{ }$}\setmainfont[Path=/usr/share/fonts/truetype/cmu/,UprightFont=cmunrm.ttf,BoldFont=cmunbx.ttf,ItalicFont=cmunti.ttf,BoldItalicFont=cmunbi.ttf]{cmunrm.ttf}\setmonofont[Path=/usr/share/fonts/truetype/cmu/,UprightFont=cmuntt.ttf,BoldFont=cmuntb.ttf,ItalicFont=cmunit.ttf,BoldItalicFont=cmuntx.ttf]{cmunrm.ttf} package} can be used for adding the proof of a theorem. The basic usage is:

\begin{Shaded}
\begin{Highlighting}[]

\NormalTok{\textbackslash{}begin\{proof\}}\newline
\NormalTok{Here\ensuremath{\text{ }}is\ensuremath{\text{ }}my\ensuremath{\text{ }}proof}\newline
\NormalTok{\textbackslash{}end\{proof\}}\newline
\end{Highlighting}
\end{Shaded}


It just adds {\itshape \setmainfont[Path=/usr/share/fonts/truetype/cmu/,UprightFont=cmunrm.ttf,BoldFont=cmunbx.ttf,ItalicFont=cmunti.ttf,BoldItalicFont=cmunbi.ttf]{cmunti.ttf}\setmonofont[Path=/usr/share/fonts/truetype/cmu/,UprightFont=cmuntt.ttf,BoldFont=cmuntb.ttf,ItalicFont=cmunit.ttf,BoldItalicFont=cmuntx.ttf]{cmunti.ttf}\itshape Proof}{$\text{ }$}\setmainfont[Path=/usr/share/fonts/truetype/cmu/,UprightFont=cmunrm.ttf,BoldFont=cmunbx.ttf,ItalicFont=cmunti.ttf,BoldItalicFont=cmunbi.ttf]{cmunrm.ttf}\setmonofont[Path=/usr/share/fonts/truetype/cmu/,UprightFont=cmuntt.ttf,BoldFont=cmuntb.ttf,ItalicFont=cmunit.ttf,BoldItalicFont=cmuntx.ttf]{cmunrm.ttf} in italics at the beginning of the text given as argument and a white square (\myhref{https://en.wikipedia.org/wiki/Q.E.D.}{Q.E.D.} symbol, also known as a \myhref{https://en.wikipedia.org/wiki/Tombstone\%20\%28typography\%29}{tombstone}) at the end of it. If you are writing in another language than English, just use \mylref{209}{babel} with the right argument and the word {\itshape \setmainfont[Path=/usr/share/fonts/truetype/cmu/,UprightFont=cmunrm.ttf,BoldFont=cmunbx.ttf,ItalicFont=cmunti.ttf,BoldItalicFont=cmunbi.ttf]{cmunti.ttf}\setmonofont[Path=/usr/share/fonts/truetype/cmu/,UprightFont=cmuntt.ttf,BoldFont=cmuntb.ttf,ItalicFont=cmunit.ttf,BoldItalicFont=cmuntx.ttf]{cmunti.ttf}\itshape Proof}{$\text{ }$}\setmainfont[Path=/usr/share/fonts/truetype/cmu/,UprightFont=cmunrm.ttf,BoldFont=cmunbx.ttf,ItalicFont=cmunti.ttf,BoldItalicFont=cmunbi.ttf]{cmunrm.ttf}\setmonofont[Path=/usr/share/fonts/truetype/cmu/,UprightFont=cmuntt.ttf,BoldFont=cmuntb.ttf,ItalicFont=cmunit.ttf,BoldItalicFont=cmuntx.ttf]{cmunrm.ttf} printed in the output will be translated accordingly; anyway, in the source the name of the environment remains {\ttfamily \setmainfont[Path=/usr/share/fonts/truetype/cmu/,UprightFont=cmunrm.ttf,BoldFont=cmunbx.ttf,ItalicFont=cmunti.ttf,BoldItalicFont=cmunbi.ttf]{cmuntt.ttf}\setmonofont[Path=/usr/share/fonts/truetype/cmu/,UprightFont=cmuntt.ttf,BoldFont=cmuntb.ttf,ItalicFont=cmunit.ttf,BoldItalicFont=cmuntx.ttf]{cmuntt.ttf}\ttfamily proof}\setmainfont[Path=/usr/share/fonts/truetype/cmu/,UprightFont=cmunrm.ttf,BoldFont=cmunbx.ttf,ItalicFont=cmunti.ttf,BoldItalicFont=cmunbi.ttf]{cmunrm.ttf}\setmonofont[Path=/usr/share/fonts/truetype/cmu/,UprightFont=cmuntt.ttf,BoldFont=cmuntb.ttf,ItalicFont=cmunit.ttf,BoldItalicFont=cmuntx.ttf]{cmunrm.ttf}.

If you would like to manually name the proof, include the name in square brackets:

\begin{Shaded}
\begin{Highlighting}[]

\NormalTok{\textbackslash{}begin\{proof\}[Proof\ensuremath{\text{ }}of\ensuremath{\text{ }}important\ensuremath{\text{ }}theorem]}\newline
\NormalTok{Here\ensuremath{\text{ }}is\ensuremath{\text{ }}my\ensuremath{\text{ }}important\ensuremath{\text{ }}proof}\newline
\NormalTok{\textbackslash{}end\{proof\}}\newline
\end{Highlighting}
\end{Shaded}


If the last line of the proof is displayed math then the Q.E.D. symbol will appear on a subsequent empty line. To put the Q.E.D. symbol at the end of the last line, use the {\ttfamily \setmainfont[Path=/usr/share/fonts/truetype/cmu/,UprightFont=cmunrm.ttf,BoldFont=cmunbx.ttf,ItalicFont=cmunti.ttf,BoldItalicFont=cmunbi.ttf]{cmuntt.ttf}\setmonofont[Path=/usr/share/fonts/truetype/cmu/,UprightFont=cmuntt.ttf,BoldFont=cmuntb.ttf,ItalicFont=cmunit.ttf,BoldItalicFont=cmuntx.ttf]{cmuntt.ttf}\ttfamily \textbackslash{}qedhere}{$\text{ }$}\setmainfont[Path=/usr/share/fonts/truetype/cmu/,UprightFont=cmunrm.ttf,BoldFont=cmunbx.ttf,ItalicFont=cmunti.ttf,BoldItalicFont=cmunbi.ttf]{cmunrm.ttf}\setmonofont[Path=/usr/share/fonts/truetype/cmu/,UprightFont=cmuntt.ttf,BoldFont=cmuntb.ttf,ItalicFont=cmunit.ttf,BoldItalicFont=cmuntx.ttf]{cmunrm.ttf} command:

\begin{Shaded}
\begin{Highlighting}[]

\NormalTok{\textbackslash{}begin\{proof\}}\newline
\NormalTok{Here\ensuremath{\text{ }}is\ensuremath{\text{ }}my\ensuremath{\text{ }}proof:}\newline
\NormalTok{\textbackslash{}[}\newline
\NormalTok{a^2\ensuremath{\text{ }}+\ensuremath{\text{ }}b^2\ensuremath{\text{ }}=\ensuremath{\text{ }}c^2\ensuremath{\text{ }}\textbackslash{}qedhere}\newline
\NormalTok{\textbackslash{}]}\newline
\NormalTok{\textbackslash{}end\{proof\}}\newline
\end{Highlighting}
\end{Shaded}

The method above does not work with the deprecated environment {\ttfamily \setmainfont[Path=/usr/share/fonts/truetype/cmu/,UprightFont=cmunrm.ttf,BoldFont=cmunbx.ttf,ItalicFont=cmunti.ttf,BoldItalicFont=cmunbi.ttf]{cmuntt.ttf}\setmonofont[Path=/usr/share/fonts/truetype/cmu/,UprightFont=cmuntt.ttf,BoldFont=cmuntb.ttf,ItalicFont=cmunit.ttf,BoldItalicFont=cmuntx.ttf]{cmuntt.ttf}\ttfamily eqnarray*}\setmainfont[Path=/usr/share/fonts/truetype/cmu/,UprightFont=cmunrm.ttf,BoldFont=cmunbx.ttf,ItalicFont=cmunti.ttf,BoldItalicFont=cmunbi.ttf]{cmunrm.ttf}\setmonofont[Path=/usr/share/fonts/truetype/cmu/,UprightFont=cmuntt.ttf,BoldFont=cmuntb.ttf,ItalicFont=cmunit.ttf,BoldItalicFont=cmuntx.ttf]{cmunrm.ttf}. Use {\ttfamily \setmainfont[Path=/usr/share/fonts/truetype/cmu/,UprightFont=cmunrm.ttf,BoldFont=cmunbx.ttf,ItalicFont=cmunti.ttf,BoldItalicFont=cmunbi.ttf]{cmuntt.ttf}\setmonofont[Path=/usr/share/fonts/truetype/cmu/,UprightFont=cmuntt.ttf,BoldFont=cmuntb.ttf,ItalicFont=cmunit.ttf,BoldItalicFont=cmuntx.ttf]{cmuntt.ttf}\ttfamily align*}{$\text{ }$}\setmainfont[Path=/usr/share/fonts/truetype/cmu/,UprightFont=cmunrm.ttf,BoldFont=cmunbx.ttf,ItalicFont=cmunti.ttf,BoldItalicFont=cmunbi.ttf]{cmunrm.ttf}\setmonofont[Path=/usr/share/fonts/truetype/cmu/,UprightFont=cmuntt.ttf,BoldFont=cmuntb.ttf,ItalicFont=cmunit.ttf,BoldItalicFont=cmuntx.ttf]{cmunrm.ttf} instead.

To use a custom Q.E.D. symbol, redefine the {\ttfamily \setmainfont[Path=/usr/share/fonts/truetype/cmu/,UprightFont=cmunrm.ttf,BoldFont=cmunbx.ttf,ItalicFont=cmunti.ttf,BoldItalicFont=cmunbi.ttf]{cmuntt.ttf}\setmonofont[Path=/usr/share/fonts/truetype/cmu/,UprightFont=cmuntt.ttf,BoldFont=cmuntb.ttf,ItalicFont=cmunit.ttf,BoldItalicFont=cmuntx.ttf]{cmuntt.ttf}\ttfamily \textbackslash{}qedsymbol}{$\text{ }$}\setmainfont[Path=/usr/share/fonts/truetype/cmu/,UprightFont=cmunrm.ttf,BoldFont=cmunbx.ttf,ItalicFont=cmunti.ttf,BoldItalicFont=cmunbi.ttf]{cmunrm.ttf}\setmonofont[Path=/usr/share/fonts/truetype/cmu/,UprightFont=cmuntt.ttf,BoldFont=cmuntb.ttf,ItalicFont=cmunit.ttf,BoldItalicFont=cmuntx.ttf]{cmunrm.ttf} command. To hide the Q.E.D. symbol altogether, redefine it to be blank:

\begin{Shaded}
\begin{Highlighting}[]

\NormalTok{\textbackslash{}renewcommand\{\textbackslash{}qedsymbol\}\{\}}\newline
\end{Highlighting}
\end{Shaded}

\section{Theorem styles}
\label{561}
It adds the possibility to change the output of the environments defined by {\ttfamily \setmainfont[Path=/usr/share/fonts/truetype/cmu/,UprightFont=cmunrm.ttf,BoldFont=cmunbx.ttf,ItalicFont=cmunti.ttf,BoldItalicFont=cmunbi.ttf]{cmuntt.ttf}\setmonofont[Path=/usr/share/fonts/truetype/cmu/,UprightFont=cmuntt.ttf,BoldFont=cmuntb.ttf,ItalicFont=cmunit.ttf,BoldItalicFont=cmuntx.ttf]{cmuntt.ttf}\ttfamily \textbackslash{}newtheorem}{$\text{ }$}\setmainfont[Path=/usr/share/fonts/truetype/cmu/,UprightFont=cmunrm.ttf,BoldFont=cmunbx.ttf,ItalicFont=cmunti.ttf,BoldItalicFont=cmunbi.ttf]{cmunrm.ttf}\setmonofont[Path=/usr/share/fonts/truetype/cmu/,UprightFont=cmuntt.ttf,BoldFont=cmuntb.ttf,ItalicFont=cmunit.ttf,BoldItalicFont=cmuntx.ttf]{cmunrm.ttf} using the {\ttfamily \setmainfont[Path=/usr/share/fonts/truetype/cmu/,UprightFont=cmunrm.ttf,BoldFont=cmunbx.ttf,ItalicFont=cmunti.ttf,BoldItalicFont=cmunbi.ttf]{cmuntt.ttf}\setmonofont[Path=/usr/share/fonts/truetype/cmu/,UprightFont=cmuntt.ttf,BoldFont=cmuntb.ttf,ItalicFont=cmunit.ttf,BoldItalicFont=cmuntx.ttf]{cmuntt.ttf}\ttfamily \textbackslash{}theoremstyle}{$\text{ }$}\setmainfont[Path=/usr/share/fonts/truetype/cmu/,UprightFont=cmunrm.ttf,BoldFont=cmunbx.ttf,ItalicFont=cmunti.ttf,BoldItalicFont=cmunbi.ttf]{cmunrm.ttf}\setmonofont[Path=/usr/share/fonts/truetype/cmu/,UprightFont=cmuntt.ttf,BoldFont=cmuntb.ttf,ItalicFont=cmunit.ttf,BoldItalicFont=cmuntx.ttf]{cmunrm.ttf} command\myfootnote{Requires the {\ttfamily \setmainfont[Path=/usr/share/fonts/truetype/cmu/,UprightFont=cmunrm.ttf,BoldFont=cmunbx.ttf,ItalicFont=cmunti.ttf,BoldItalicFont=cmunbi.ttf]{cmuntt.ttf}\setmonofont[Path=/usr/share/fonts/truetype/cmu/,UprightFont=cmuntt.ttf,BoldFont=cmuntb.ttf,ItalicFont=cmunit.ttf,BoldItalicFont=cmuntx.ttf]{cmuntt.ttf}\ttfamily amsthm}{$\text{ }$}\setmainfont[Path=/usr/share/fonts/truetype/cmu/,UprightFont=cmunrm.ttf,BoldFont=cmunbx.ttf,ItalicFont=cmunti.ttf,BoldItalicFont=cmunbi.ttf]{cmunrm.ttf}\setmonofont[Path=/usr/share/fonts/truetype/cmu/,UprightFont=cmuntt.ttf,BoldFont=cmuntb.ttf,ItalicFont=cmunit.ttf,BoldItalicFont=cmuntx.ttf]{cmunrm.ttf} package} in the header:

\begin{Shaded}
\begin{Highlighting}[]

\NormalTok{\textbackslash{}theoremstyle\{stylename\}}\newline
\end{Highlighting}
\end{Shaded}

the argument is the style you want to use. All subsequently defined theorems will use this style. Here is a list of the possible pre-{}defined styles:

\begin{longtable}{|>{\RaggedRight}p{0.17133\linewidth}|>{\RaggedRight}p{0.39896\linewidth}|>{\RaggedRight}p{0.34400\linewidth}|} \hline 
{\bfseries \hspace*{0pt}\ignorespaces{}\hspace*{0pt}{\ttfamily \setmainfont[Path=/usr/share/fonts/truetype/cmu/,UprightFont=cmunrm.ttf,BoldFont=cmunbx.ttf,ItalicFont=cmunti.ttf,BoldItalicFont=cmunbi.ttf]{cmuntt.ttf}\setmonofont[Path=/usr/share/fonts/truetype/cmu/,UprightFont=cmuntt.ttf,BoldFont=cmuntb.ttf,ItalicFont=cmunit.ttf,BoldItalicFont=cmuntx.ttf]{cmuntt.ttf}\ttfamily stylename}}&{\bfseries \hspace*{0pt}\ignorespaces{}\hspace*{0pt}\setmainfont[Path=/usr/share/fonts/truetype/cmu/,UprightFont=cmunrm.ttf,BoldFont=cmunbx.ttf,ItalicFont=cmunti.ttf,BoldItalicFont=cmunbi.ttf]{cmunrm.ttf}\setmonofont[Path=/usr/share/fonts/truetype/cmu/,UprightFont=cmuntt.ttf,BoldFont=cmuntb.ttf,ItalicFont=cmunit.ttf,BoldItalicFont=cmuntx.ttf]{cmunrm.ttf}Description}&{\bfseries \hspace*{0pt}\ignorespaces{}\hspace*{0pt}Appearance}\endhead  \hline \hspace*{0pt}\ignorespaces{}\hspace*{0pt}{\ttfamily \setmainfont[Path=/usr/share/fonts/truetype/cmu/,UprightFont=cmunrm.ttf,BoldFont=cmunbx.ttf,ItalicFont=cmunti.ttf,BoldItalicFont=cmunbi.ttf]{cmuntt.ttf}\setmonofont[Path=/usr/share/fonts/truetype/cmu/,UprightFont=cmuntt.ttf,BoldFont=cmuntb.ttf,ItalicFont=cmunit.ttf,BoldItalicFont=cmuntx.ttf]{cmuntt.ttf}\ttfamily plain}&\hspace*{0pt}\ignorespaces{}\hspace*{0pt}{$\text{ }$}\setmainfont[Path=/usr/share/fonts/truetype/cmu/,UprightFont=cmunrm.ttf,BoldFont=cmunbx.ttf,ItalicFont=cmunti.ttf,BoldItalicFont=cmunbi.ttf]{cmunrm.ttf}\setmonofont[Path=/usr/share/fonts/truetype/cmu/,UprightFont=cmuntt.ttf,BoldFont=cmuntb.ttf,ItalicFont=cmunit.ttf,BoldItalicFont=cmuntx.ttf]{cmunrm.ttf} Used for theorems, lemmas, propositions, etc. (default)&\hspace*{0pt}\ignorespaces{}\hspace*{0pt}{\bfseries \setmainfont[Path=/usr/share/fonts/truetype/cmu/,UprightFont=cmunrm.ttf,BoldFont=cmunbx.ttf,ItalicFont=cmunti.ttf,BoldItalicFont=cmunbi.ttf]{cmunbx.ttf}\setmonofont[Path=/usr/share/fonts/truetype/cmu/,UprightFont=cmuntt.ttf,BoldFont=cmuntb.ttf,ItalicFont=cmunit.ttf,BoldItalicFont=cmuntx.ttf]{cmunbx.ttf}\bfseries Theorem 1.}{$\text{ }$}\setmainfont[Path=/usr/share/fonts/truetype/cmu/,UprightFont=cmunrm.ttf,BoldFont=cmunbx.ttf,ItalicFont=cmunti.ttf,BoldItalicFont=cmunbi.ttf]{cmunrm.ttf}\setmonofont[Path=/usr/share/fonts/truetype/cmu/,UprightFont=cmuntt.ttf,BoldFont=cmuntb.ttf,ItalicFont=cmunit.ttf,BoldItalicFont=cmuntx.ttf]{cmunrm.ttf} {\itshape \setmainfont[Path=/usr/share/fonts/truetype/cmu/,UprightFont=cmunrm.ttf,BoldFont=cmunbx.ttf,ItalicFont=cmunti.ttf,BoldItalicFont=cmunbi.ttf]{cmunti.ttf}\setmonofont[Path=/usr/share/fonts/truetype/cmu/,UprightFont=cmuntt.ttf,BoldFont=cmuntb.ttf,ItalicFont=cmunit.ttf,BoldItalicFont=cmuntx.ttf]{cmunti.ttf}\itshape Theorem text.}\\ \hline \hspace*{0pt}\ignorespaces{}\hspace*{0pt}{\ttfamily \setmainfont[Path=/usr/share/fonts/truetype/cmu/,UprightFont=cmunrm.ttf,BoldFont=cmunbx.ttf,ItalicFont=cmunti.ttf,BoldItalicFont=cmunbi.ttf]{cmuntt.ttf}\setmonofont[Path=/usr/share/fonts/truetype/cmu/,UprightFont=cmuntt.ttf,BoldFont=cmuntb.ttf,ItalicFont=cmunit.ttf,BoldItalicFont=cmuntx.ttf]{cmuntt.ttf}\ttfamily definition}&\hspace*{0pt}\ignorespaces{}\hspace*{0pt}\setmainfont[Path=/usr/share/fonts/truetype/cmu/,UprightFont=cmunrm.ttf,BoldFont=cmunbx.ttf,ItalicFont=cmunti.ttf,BoldItalicFont=cmunbi.ttf]{cmunrm.ttf}\setmonofont[Path=/usr/share/fonts/truetype/cmu/,UprightFont=cmuntt.ttf,BoldFont=cmuntb.ttf,ItalicFont=cmunit.ttf,BoldItalicFont=cmuntx.ttf]{cmunrm.ttf}Used for definitions and examples&\hspace*{0pt}\ignorespaces{}\hspace*{0pt}{\bfseries \setmainfont[Path=/usr/share/fonts/truetype/cmu/,UprightFont=cmunrm.ttf,BoldFont=cmunbx.ttf,ItalicFont=cmunti.ttf,BoldItalicFont=cmunbi.ttf]{cmunbx.ttf}\setmonofont[Path=/usr/share/fonts/truetype/cmu/,UprightFont=cmuntt.ttf,BoldFont=cmuntb.ttf,ItalicFont=cmunit.ttf,BoldItalicFont=cmuntx.ttf]{cmunbx.ttf}\bfseries Definition 2.}{$\text{ }$}\setmainfont[Path=/usr/share/fonts/truetype/cmu/,UprightFont=cmunrm.ttf,BoldFont=cmunbx.ttf,ItalicFont=cmunti.ttf,BoldItalicFont=cmunbi.ttf]{cmunrm.ttf}\setmonofont[Path=/usr/share/fonts/truetype/cmu/,UprightFont=cmuntt.ttf,BoldFont=cmuntb.ttf,ItalicFont=cmunit.ttf,BoldItalicFont=cmuntx.ttf]{cmunrm.ttf} Definition text.\\ \hline \hspace*{0pt}\ignorespaces{}\hspace*{0pt}{\ttfamily \setmainfont[Path=/usr/share/fonts/truetype/cmu/,UprightFont=cmunrm.ttf,BoldFont=cmunbx.ttf,ItalicFont=cmunti.ttf,BoldItalicFont=cmunbi.ttf]{cmuntt.ttf}\setmonofont[Path=/usr/share/fonts/truetype/cmu/,UprightFont=cmuntt.ttf,BoldFont=cmuntb.ttf,ItalicFont=cmunit.ttf,BoldItalicFont=cmuntx.ttf]{cmuntt.ttf}\ttfamily remark}&\hspace*{0pt}\ignorespaces{}\hspace*{0pt}\setmainfont[Path=/usr/share/fonts/truetype/cmu/,UprightFont=cmunrm.ttf,BoldFont=cmunbx.ttf,ItalicFont=cmunti.ttf,BoldItalicFont=cmunbi.ttf]{cmunrm.ttf}\setmonofont[Path=/usr/share/fonts/truetype/cmu/,UprightFont=cmuntt.ttf,BoldFont=cmuntb.ttf,ItalicFont=cmunit.ttf,BoldItalicFont=cmuntx.ttf]{cmunrm.ttf}Used for remarks and notes&\hspace*{0pt}\ignorespaces{}\hspace*{0pt}{\itshape \setmainfont[Path=/usr/share/fonts/truetype/cmu/,UprightFont=cmunrm.ttf,BoldFont=cmunbx.ttf,ItalicFont=cmunti.ttf,BoldItalicFont=cmunbi.ttf]{cmunti.ttf}\setmonofont[Path=/usr/share/fonts/truetype/cmu/,UprightFont=cmuntt.ttf,BoldFont=cmuntb.ttf,ItalicFont=cmunit.ttf,BoldItalicFont=cmuntx.ttf]{cmunti.ttf}\itshape Remark}{$\text{ }$}\setmainfont[Path=/usr/share/fonts/truetype/cmu/,UprightFont=cmunrm.ttf,BoldFont=cmunbx.ttf,ItalicFont=cmunti.ttf,BoldItalicFont=cmunbi.ttf]{cmunrm.ttf}\setmonofont[Path=/usr/share/fonts/truetype/cmu/,UprightFont=cmuntt.ttf,BoldFont=cmuntb.ttf,ItalicFont=cmunit.ttf,BoldItalicFont=cmuntx.ttf]{cmunrm.ttf} 3. Remark text.\\ \hline 
\end{longtable}

\subsection{Custom styles}
\label{562}
To define your own style, use the {\ttfamily \setmainfont[Path=/usr/share/fonts/truetype/cmu/,UprightFont=cmunrm.ttf,BoldFont=cmunbx.ttf,ItalicFont=cmunti.ttf,BoldItalicFont=cmunbi.ttf]{cmuntt.ttf}\setmonofont[Path=/usr/share/fonts/truetype/cmu/,UprightFont=cmuntt.ttf,BoldFont=cmuntb.ttf,ItalicFont=cmunit.ttf,BoldItalicFont=cmuntx.ttf]{cmuntt.ttf}\ttfamily \textbackslash{}newtheoremstyle}{$\text{ }$}\setmainfont[Path=/usr/share/fonts/truetype/cmu/,UprightFont=cmunrm.ttf,BoldFont=cmunbx.ttf,ItalicFont=cmunti.ttf,BoldItalicFont=cmunbi.ttf]{cmunrm.ttf}\setmonofont[Path=/usr/share/fonts/truetype/cmu/,UprightFont=cmuntt.ttf,BoldFont=cmuntb.ttf,ItalicFont=cmunit.ttf,BoldItalicFont=cmuntx.ttf]{cmunrm.ttf} command\myfootnote{Requires the {\ttfamily \setmainfont[Path=/usr/share/fonts/truetype/cmu/,UprightFont=cmunrm.ttf,BoldFont=cmunbx.ttf,ItalicFont=cmunti.ttf,BoldItalicFont=cmunbi.ttf]{cmuntt.ttf}\setmonofont[Path=/usr/share/fonts/truetype/cmu/,UprightFont=cmuntt.ttf,BoldFont=cmuntb.ttf,ItalicFont=cmunit.ttf,BoldItalicFont=cmuntx.ttf]{cmuntt.ttf}\ttfamily amsthm}{$\text{ }$}\setmainfont[Path=/usr/share/fonts/truetype/cmu/,UprightFont=cmunrm.ttf,BoldFont=cmunbx.ttf,ItalicFont=cmunti.ttf,BoldItalicFont=cmunbi.ttf]{cmunrm.ttf}\setmonofont[Path=/usr/share/fonts/truetype/cmu/,UprightFont=cmuntt.ttf,BoldFont=cmuntb.ttf,ItalicFont=cmunit.ttf,BoldItalicFont=cmuntx.ttf]{cmunrm.ttf} package}:

\begin{Shaded}
\begin{Highlighting}[]

\NormalTok{\textbackslash{}newtheoremstyle\{stylename\}}\CommentTok{\%\ensuremath{\text{ }}name\ensuremath{\text{ }}of\ensuremath{\text{ }}the\ensuremath{\text{ }}style\ensuremath{\text{ }}to\ensuremath{\text{ }}be\ensuremath{\text{ }}used}\newline
\ensuremath{\text{ }}\ensuremath{\text{ }}\NormalTok{\{spaceabove\}}\CommentTok{\%\ensuremath{\text{ }}measure\ensuremath{\text{ }}of\ensuremath{\text{ }}space\ensuremath{\text{ }}to\ensuremath{\text{ }}leave\ensuremath{\text{ }}above\ensuremath{\text{ }}the\ensuremath{\text{ }}theorem.\ensuremath{\text{ }}E.g.:\ensuremath{\text{ }}3pt}\newline
\ensuremath{\text{ }}\ensuremath{\text{ }}\NormalTok{\{spacebelow\}}\CommentTok{\%\ensuremath{\text{ }}measure\ensuremath{\text{ }}of\ensuremath{\text{ }}space\ensuremath{\text{ }}to\ensuremath{\text{ }}leave\ensuremath{\text{ }}below\ensuremath{\text{ }}the\ensuremath{\text{ }}theorem.\ensuremath{\text{ }}E.g.:\ensuremath{\text{ }}3pt}\newline
\ensuremath{\text{ }}\ensuremath{\text{ }}\NormalTok{\{bodyfont\}}\CommentTok{\%\ensuremath{\text{ }}name\ensuremath{\text{ }}of\ensuremath{\text{ }}font\ensuremath{\text{ }}to\ensuremath{\text{ }}use\ensuremath{\text{ }}in\ensuremath{\text{ }}the\ensuremath{\text{ }}body\ensuremath{\text{ }}of\ensuremath{\text{ }}the\ensuremath{\text{ }}theorem}\newline
\ensuremath{\text{ }}\ensuremath{\text{ }}\NormalTok{\{indent\}}\CommentTok{\%\ensuremath{\text{ }}measure\ensuremath{\text{ }}of\ensuremath{\text{ }}space\ensuremath{\text{ }}to\ensuremath{\text{ }}indent}\newline
\ensuremath{\text{ }}\ensuremath{\text{ }}\NormalTok{\{headfont\}}\CommentTok{\%\ensuremath{\text{ }}name\ensuremath{\text{ }}of\ensuremath{\text{ }}head\ensuremath{\text{ }}font}\newline
\ensuremath{\text{ }}\ensuremath{\text{ }}\NormalTok{\{headpunctuation\}}\CommentTok{\%\ensuremath{\text{ }}punctuation\ensuremath{\text{ }}between\ensuremath{\text{ }}head\ensuremath{\text{ }}and\ensuremath{\text{ }}body}\newline
\ensuremath{\text{ }}\ensuremath{\text{ }}\NormalTok{\{headspace\}}\CommentTok{\%\ensuremath{\text{ }}space\ensuremath{\text{ }}after\ensuremath{\text{ }}theorem\ensuremath{\text{ }}head;\ensuremath{\text{ }}"\ensuremath{\text{ }}"\ensuremath{\text{ }}=\ensuremath{\text{ }}normal\ensuremath{\text{ }}interword\ensuremath{\text{ }}space}\newline
\ensuremath{\text{ }}\ensuremath{\text{ }}\NormalTok{\{headspec\}}\CommentTok{\%\ensuremath{\text{ }}Manually\ensuremath{\text{ }}specify\ensuremath{\text{ }}head}\newline
\end{Highlighting}
\end{Shaded}

(Any arguments that are left blank will assume their default value).  Here is an example {\itshape \setmainfont[Path=/usr/share/fonts/truetype/cmu/,UprightFont=cmunrm.ttf,BoldFont=cmunbx.ttf,ItalicFont=cmunti.ttf,BoldItalicFont=cmunbi.ttf]{cmunti.ttf}\setmonofont[Path=/usr/share/fonts/truetype/cmu/,UprightFont=cmuntt.ttf,BoldFont=cmuntb.ttf,ItalicFont=cmunit.ttf,BoldItalicFont=cmuntx.ttf]{cmunti.ttf}\itshape headspec}\setmainfont[Path=/usr/share/fonts/truetype/cmu/,UprightFont=cmunrm.ttf,BoldFont=cmunbx.ttf,ItalicFont=cmunti.ttf,BoldItalicFont=cmunbi.ttf]{cmunrm.ttf}\setmonofont[Path=/usr/share/fonts/truetype/cmu/,UprightFont=cmuntt.ttf,BoldFont=cmuntb.ttf,ItalicFont=cmunit.ttf,BoldItalicFont=cmuntx.ttf]{cmunrm.ttf}:

\begin{Shaded}
\begin{Highlighting}[]

\NormalTok{\textbackslash{}thmname\{#1\}\textbackslash{}thmnumber\{\ensuremath{\text{ }}#2\}:\textbackslash{}thmnote\{\ensuremath{\text{ }}#3\}}\newline
\end{Highlighting}
\end{Shaded}

which would look something like:$\text{ }$\newline{}

{\bfseries \setmainfont[Path=/usr/share/fonts/truetype/cmu/,UprightFont=cmunrm.ttf,BoldFont=cmunbx.ttf,ItalicFont=cmunti.ttf,BoldItalicFont=cmunbi.ttf]{cmunbx.ttf}\setmonofont[Path=/usr/share/fonts/truetype/cmu/,UprightFont=cmuntt.ttf,BoldFont=cmuntb.ttf,ItalicFont=cmunit.ttf,BoldItalicFont=cmuntx.ttf]{cmunbx.ttf}\bfseries Definition 2}\setmainfont[Path=/usr/share/fonts/truetype/cmu/,UprightFont=cmunrm.ttf,BoldFont=cmunbx.ttf,ItalicFont=cmunti.ttf,BoldItalicFont=cmunbi.ttf]{cmunrm.ttf}\setmonofont[Path=/usr/share/fonts/truetype/cmu/,UprightFont=cmuntt.ttf,BoldFont=cmuntb.ttf,ItalicFont=cmunit.ttf,BoldItalicFont=cmuntx.ttf]{cmunrm.ttf}: Topology$\text{ }$\newline{}

for the following:

\begin{Shaded}
\begin{Highlighting}[]

\NormalTok{\textbackslash{}begin\{definition\}[Topology]...}\newline
\end{Highlighting}
\end{Shaded}

(The note argument, which in this case is Topology, is always optional, but will not appear by default unless you specify it as above in the head spec).$\text{ }$\newline{}

\section{Conflicts}
\label{563}

The theorem environment conflicts with other environments,  for example {\itshape \setmainfont[Path=/usr/share/fonts/truetype/cmu/,UprightFont=cmunrm.ttf,BoldFont=cmunbx.ttf,ItalicFont=cmunti.ttf,BoldItalicFont=cmunbi.ttf]{cmunti.ttf}\setmonofont[Path=/usr/share/fonts/truetype/cmu/,UprightFont=cmuntt.ttf,BoldFont=cmuntb.ttf,ItalicFont=cmunit.ttf,BoldItalicFont=cmuntx.ttf]{cmunti.ttf}\itshape wrapfigure}\setmainfont[Path=/usr/share/fonts/truetype/cmu/,UprightFont=cmunrm.ttf,BoldFont=cmunbx.ttf,ItalicFont=cmunti.ttf,BoldItalicFont=cmunbi.ttf]{cmunrm.ttf}\setmonofont[Path=/usr/share/fonts/truetype/cmu/,UprightFont=cmuntt.ttf,BoldFont=cmuntb.ttf,ItalicFont=cmunit.ttf,BoldItalicFont=cmuntx.ttf]{cmunrm.ttf}.
A work around is to redefine theorem, for example the following way:

\begin{Shaded}
\begin{Highlighting}[]

\CommentTok{\%\ensuremath{\text{ }}Fix\ensuremath{\text{ }}latex}\newline
\NormalTok{\textbackslash{}def\textbackslash{}smallskip\{\textbackslash{}vskip\textbackslash{}smallskipamount\}}\newline
\NormalTok{\textbackslash{}def\textbackslash{}medskip\{\textbackslash{}vskip\textbackslash{}medskipamount\}}\newline
\NormalTok{\textbackslash{}def\textbackslash{}bigskip\{\textbackslash{}vskip\textbackslash{}bigskipamount\}}\newline
\ensuremath{\text{ }}\newline
\CommentTok{\%\ensuremath{\text{ }}Hand\ensuremath{\text{ }}made\ensuremath{\text{ }}theorem}\newline
\NormalTok{\textbackslash{}newcounter\{thm\}[section]}\newline
\NormalTok{\textbackslash{}renewcommand\{\textbackslash{}thethm\}\{\textbackslash{}thesection.\textbackslash{}arabic\{thm\}\}}\newline
\NormalTok{\textbackslash{}def\textbackslash{}claim#1\{\textbackslash{}par\textbackslash{}medskip\textbackslash{}noindent\textbackslash{}refstepcounter\{thm\}\textbackslash{}hbox\{\textbackslash{}bf}\newline
\ensuremath{\text{ }}\NormalTok{\textbackslash{}arabic\{chapter\}.\textbackslash{}arabic\{section\}.\textbackslash{}arabic\{thm\}.\ensuremath{\text{ }}#1.\}}\newline
\NormalTok{\textbackslash{}it\textbackslash{}\ensuremath{\text{ }}}\CommentTok{\%\textbackslash{}ignorespaces}\newline
\NormalTok{\}}\newline
\NormalTok{\textbackslash{}def\textbackslash{}endclaim\{}\newline
\NormalTok{\textbackslash{}par\textbackslash{}medskip\}}\newline
\NormalTok{\textbackslash{}newenvironment\{thm\}\{\textbackslash{}claim\}\{\textbackslash{}endclaim\}}\newline
\end{Highlighting}
\end{Shaded}


In this case theorem looks like:


\begin{Shaded}
\begin{Highlighting}[]

\NormalTok{\textbackslash{}begin\{thm\}\{Claim\}\textbackslash{}label\{lyt-prob\}\ensuremath{\text{ }}}\newline
\NormalTok{Let\ensuremath{\text{ }}it\ensuremath{\text{ }}be.}\newline
\NormalTok{Then\ensuremath{\text{ }}you\ensuremath{\text{ }}know.}\newline
\NormalTok{\textbackslash{}end\{thm\}}\newline
\end{Highlighting}
\end{Shaded}

\section{Notes}
\label{564}

\section{External links}
\label{565}
\begin{myitemize}
\item{}  {$\text{[}$}ftp://ftp.ams.org/pub/tex/doc/amscls/amsthdoc.pdf {\ttfamily \setmainfont[Path=/usr/share/fonts/truetype/cmu/,UprightFont=cmunrm.ttf,BoldFont=cmunbx.ttf,ItalicFont=cmunti.ttf,BoldItalicFont=cmunbi.ttf]{cmuntt.ttf}\setmonofont[Path=/usr/share/fonts/truetype/cmu/,UprightFont=cmuntt.ttf,BoldFont=cmuntb.ttf,ItalicFont=cmunit.ttf,BoldItalicFont=cmuntx.ttf]{cmuntt.ttf}\ttfamily amsthm}{$\text{ }$}\setmainfont[Path=/usr/share/fonts/truetype/cmu/,UprightFont=cmunrm.ttf,BoldFont=cmunbx.ttf,ItalicFont=cmunti.ttf,BoldItalicFont=cmunbi.ttf]{cmunrm.ttf}\setmonofont[Path=/usr/share/fonts/truetype/cmu/,UprightFont=cmuntt.ttf,BoldFont=cmuntb.ttf,ItalicFont=cmunit.ttf,BoldItalicFont=cmuntx.ttf]{cmunrm.ttf} documentation{$\text{]}$}
\end{myitemize}




\myhref{https://sr.wikibooks.org/wiki/LaTeX\%2F\%D0\%A2\%D0\%B5\%D0\%BE\%D1\%80\%D0\%B5\%D0\%BC\%D0\%B5}{sr:LaTeX/Теореме}\chapter{Chemical Graphics}

\myminitoc
\label{566}

\label{567}


\LaTeXNullTemplate{}

\myhref{http://www.ctan.org/tex-archive/macros/latex/contrib/chemfig/}{chemfig} is a package used to draw 2D chemical structures. It is an alternative to \myhref{http://www.2k-software.de/ingo/ochem.html}{ochem}. Whereas ochem requires Perl to draw chemical structures, chemfig uses the \myhref{http://az.ctan.org/pkg/pgf}{tikz} package to produce its graphics. chemfig is used by adding the following to the preamble:


\begin{Shaded}
\begin{Highlighting}[]

\NormalTok{\textbackslash{}usepackage\{chemfig\}}\newline
\end{Highlighting}
\end{Shaded}

\section{Basic Usage}
\label{568}

The primary command used in this package is {\ttfamily \setmainfont[Path=/usr/share/fonts/truetype/cmu/,UprightFont=cmunrm.ttf,BoldFont=cmunbx.ttf,ItalicFont=cmunti.ttf,BoldItalicFont=cmunbi.ttf]{cmuntt.ttf}\setmonofont[Path=/usr/share/fonts/truetype/cmu/,UprightFont=cmuntt.ttf,BoldFont=cmuntb.ttf,ItalicFont=cmunit.ttf,BoldItalicFont=cmuntx.ttf]{cmuntt.ttf}\ttfamily \textbackslash{}chemfig\{\}}\setmainfont[Path=/usr/share/fonts/truetype/cmu/,UprightFont=cmunrm.ttf,BoldFont=cmunbx.ttf,ItalicFont=cmunti.ttf,BoldItalicFont=cmunbi.ttf]{cmunrm.ttf}\setmonofont[Path=/usr/share/fonts/truetype/cmu/,UprightFont=cmuntt.ttf,BoldFont=cmuntb.ttf,ItalicFont=cmunit.ttf,BoldItalicFont=cmuntx.ttf]{cmunrm.ttf}:


\begin{Shaded}
\begin{Highlighting}[]

\NormalTok{\textbackslash{}chemfig\{<atom1><bond\ensuremath{\text{ }}type>[<angle>,<coeff>,<tikz\ensuremath{\text{ }}code>]<atom2>\}}\newline
\end{Highlighting}
\end{Shaded}


<{}angle>{} is the bond angle between two atoms (or nodes). There are three types of angles: absolute, relative, and predefined. Absolute angles give a precise angle (generally, 0 to 360, though they can also be negative), and are represented with the syntax {\ttfamily \setmainfont[Path=/usr/share/fonts/truetype/cmu/,UprightFont=cmunrm.ttf,BoldFont=cmunbx.ttf,ItalicFont=cmunti.ttf,BoldItalicFont=cmunbi.ttf]{cmuntt.ttf}\setmonofont[Path=/usr/share/fonts/truetype/cmu/,UprightFont=cmuntt.ttf,BoldFont=cmuntb.ttf,ItalicFont=cmunit.ttf,BoldItalicFont=cmuntx.ttf]{cmuntt.ttf}\ttfamily {$\text{[}$}:<{}absolute angle>{}{$\text{]}$}}\setmainfont[Path=/usr/share/fonts/truetype/cmu/,UprightFont=cmunrm.ttf,BoldFont=cmunbx.ttf,ItalicFont=cmunti.ttf,BoldItalicFont=cmunbi.ttf]{cmunrm.ttf}\setmonofont[Path=/usr/share/fonts/truetype/cmu/,UprightFont=cmuntt.ttf,BoldFont=cmuntb.ttf,ItalicFont=cmunit.ttf,BoldItalicFont=cmuntx.ttf]{cmunrm.ttf}. Relative angles require the syntax {\ttfamily \setmainfont[Path=/usr/share/fonts/truetype/cmu/,UprightFont=cmunrm.ttf,BoldFont=cmunbx.ttf,ItalicFont=cmunti.ttf,BoldItalicFont=cmunbi.ttf]{cmuntt.ttf}\setmonofont[Path=/usr/share/fonts/truetype/cmu/,UprightFont=cmuntt.ttf,BoldFont=cmuntb.ttf,ItalicFont=cmunit.ttf,BoldItalicFont=cmuntx.ttf]{cmuntt.ttf}\ttfamily {$\text{[}$}::<{}relative angle>{}{$\text{]}$}}{$\text{ }$}\setmainfont[Path=/usr/share/fonts/truetype/cmu/,UprightFont=cmunrm.ttf,BoldFont=cmunbx.ttf,ItalicFont=cmunti.ttf,BoldItalicFont=cmunbi.ttf]{cmunrm.ttf}\setmonofont[Path=/usr/share/fonts/truetype/cmu/,UprightFont=cmuntt.ttf,BoldFont=cmuntb.ttf,ItalicFont=cmunit.ttf,BoldItalicFont=cmuntx.ttf]{cmunrm.ttf} and produce an angle relative to the angle of the preceding bond. Finally, predefined angles are whole numbers from 0 to 7 indicating intervals of 45 degrees. These are produced with the syntax {\ttfamily \setmainfont[Path=/usr/share/fonts/truetype/cmu/,UprightFont=cmunrm.ttf,BoldFont=cmunbx.ttf,ItalicFont=cmunti.ttf,BoldItalicFont=cmunbi.ttf]{cmuntt.ttf}\setmonofont[Path=/usr/share/fonts/truetype/cmu/,UprightFont=cmuntt.ttf,BoldFont=cmuntb.ttf,ItalicFont=cmunit.ttf,BoldItalicFont=cmuntx.ttf]{cmuntt.ttf}\ttfamily {$\text{[}$}<{} predefined angle>{}{$\text{]}$}}\setmainfont[Path=/usr/share/fonts/truetype/cmu/,UprightFont=cmunrm.ttf,BoldFont=cmunbx.ttf,ItalicFont=cmunti.ttf,BoldItalicFont=cmunbi.ttf]{cmunrm.ttf}\setmonofont[Path=/usr/share/fonts/truetype/cmu/,UprightFont=cmuntt.ttf,BoldFont=cmuntb.ttf,ItalicFont=cmunit.ttf,BoldItalicFont=cmuntx.ttf]{cmunrm.ttf}. The predefined angles and their corresponding absolute angles are represented in the diagram below.




\begin{longtable}{p{1.0\linewidth}}
\begin{Shaded}
\begin{Highlighting}[]
\NormalTok{\textbackslash{}chemfig\{(-[:0,1.5,,,draw=none]\textbackslash{}scriptstyle\textbackslash{}color\{red\}0)}
\NormalTok{(-[1]1)(-[:45,1.5,,,draw=none]\textbackslash{}scriptstyle\textbackslash{}color\{red\}45)}
\NormalTok{(-[2]2)(-[:90,1.5,,,draw=none]\textbackslash{}scriptstyle\textbackslash{}color\{red\}90)}
\NormalTok{(-[3]3)(-[:135,1.5,,,draw=none]\textbackslash{}scriptstyle\textbackslash{}color\{red\}135)}
\NormalTok{(-[4]4)(-[:180,1.5,,,draw=none]\textbackslash{}scriptstyle\textbackslash{}color\{red\}180)}
\NormalTok{(-[5]5)(-[:225,1.5,,,draw=none]\textbackslash{}scriptstyle\textbackslash{}color\{red\}225)}
\NormalTok{(-[6]6)(-[:270,1.5,,,draw=none]\textbackslash{}scriptstyle\textbackslash{}color\{red\}270)}
\NormalTok{(-[7]7)(-[:315,1.5,,,draw=none]\textbackslash{}scriptstyle\textbackslash{}color\{red\}315)}
\NormalTok{-0\}}
\end{Highlighting}
\end{Shaded}
\\


\begin{minipage}{0.62500\textwidth}
\begin{center}
\includegraphics[width=1.0\textwidth,height=6.5in,keepaspectratio]{../images/104.png}
\end{center}
\raggedright{}\myfigurewithoutcaption{104}
\end{minipage}\vspace{0.75cm}


\end{longtable}


<{}bond type>{} describes the bond attaching <{}atom1>{} and <{}atom2>{}. There are 9 different bond types:


\begin{longtable}{p{1.0\linewidth}}
\begin{Shaded}
\begin{Highlighting}[]
\NormalTok{\textbackslash{}chemfig\{A-B\}\textbackslash{}\textbackslash{}}
\NormalTok{\textbackslash{}chemfig\{A=B\}\textbackslash{}\textbackslash{}}
\NormalTok{\textbackslash{}chemfig\{A~B\}\textbackslash{}\textbackslash{}}
\NormalTok{\textbackslash{}chemfig\{A>B\}\textbackslash{}\textbackslash{}}
\NormalTok{\textbackslash{}chemfig\{A<B\}\textbackslash{}\textbackslash{}}
\NormalTok{\textbackslash{}chemfig\{A>:B\}\textbackslash{}\textbackslash{}}
\NormalTok{\textbackslash{}chemfig\{A<:B\}\textbackslash{}\textbackslash{}}
\NormalTok{\textbackslash{}chemfig\{A>B\}\textbackslash{}\textbackslash{}}
\NormalTok{\textbackslash{}chemfig\{A<B\}\textbackslash{}\textbackslash{} }
\end{Highlighting}
\end{Shaded}
\\


\begin{minipage}{0.25000\textwidth}
\begin{center}
\includegraphics[width=1.0\textwidth,height=6.5in,keepaspectratio]{../images/105.png}
\end{center}
\raggedright{}\myfigurewithoutcaption{105}
\end{minipage}\vspace{0.75cm}


\end{longtable}

\textbackslash{}chemfig\{C(-{}{$\text{[}$}:0{$\text{]}$}H)(-{}{$\text{[}$}:90{$\text{]}$}H)(-{}{$\text{[}$}:180{$\text{]}$}H)(-{}{$\text{[}$}:270{$\text{]}$}H)\}

<{}coeff>{} represents the factor by which the bond\textquotesingle{}s length will be multiplied. 

<{}tikz code>{} includes additional options regarding the color or style of the bond. 


A methane molecule, for instance, can be produced with the following code: 


\begin{longtable}{p{1.0\linewidth}}
\begin{Shaded}
\begin{Highlighting}[]
\NormalTok{\textbackslash{}chemfig\{C(-[:0]H)(-[:90]H)(-[:180]H)(-[:270]H)\}}
\end{Highlighting}
\end{Shaded}
\\


\begin{minipage}{0.40000\textwidth}
\begin{center}
\includegraphics[width=1.0\textwidth,height=6.5in,keepaspectratio]{../images/106.png}
\end{center}
\raggedright{}\myfigurewithoutcaption{106}
\end{minipage}\vspace{0.75cm}


\end{longtable}



Linear molecules (such as methane) are a weak example of this, but molecules are formed in chemfig by nesting.
\section{Skeletal Diagrams}
\label{569}

Skeleton diagrams can be produced as follows:


\begin{longtable}{p{1.0\linewidth}}
\begin{Shaded}
\begin{Highlighting}[]
\NormalTok{\textbackslash{}chemfig\{-[:30]-[:-30]-[:30]\}}
\end{Highlighting}
\end{Shaded}
\\


\begin{minipage}{0.40000\textwidth}
\begin{center}
\includegraphics[width=1.0\textwidth,height=6.5in,keepaspectratio]{../images/107.png}
\end{center}
\raggedright{}\myfigurewithoutcaption{107}
\end{minipage}\vspace{0.75cm}


\end{longtable}


\begin{longtable}{p{1.0\linewidth}}
\begin{Shaded}
\begin{Highlighting}[]
\NormalTok{\textbackslash{}chemfig\{-[:30]=[:-30]-[:30]\}}
\end{Highlighting}
\end{Shaded}
\\


\begin{minipage}{0.40000\textwidth}
\begin{center}
\includegraphics[width=1.0\textwidth,height=6.5in,keepaspectratio]{../images/108.png}
\end{center}
\raggedright{}\myfigurewithoutcaption{108}
\end{minipage}\vspace{0.75cm}


\end{longtable}
\section{Rings}
\label{570}

Rings follow the syntax {\ttfamily \setmainfont[Path=/usr/share/fonts/truetype/cmu/,UprightFont=cmunrm.ttf,BoldFont=cmunbx.ttf,ItalicFont=cmunti.ttf,BoldItalicFont=cmunbi.ttf]{cmuntt.ttf}\setmonofont[Path=/usr/share/fonts/truetype/cmu/,UprightFont=cmuntt.ttf,BoldFont=cmuntb.ttf,ItalicFont=cmunit.ttf,BoldItalicFont=cmuntx.ttf]{cmuntt.ttf}\ttfamily <{}atom>{}*<{}n>{}(code)}\setmainfont[Path=/usr/share/fonts/truetype/cmu/,UprightFont=cmunrm.ttf,BoldFont=cmunbx.ttf,ItalicFont=cmunti.ttf,BoldItalicFont=cmunbi.ttf]{cmunrm.ttf}\setmonofont[Path=/usr/share/fonts/truetype/cmu/,UprightFont=cmuntt.ttf,BoldFont=cmuntb.ttf,ItalicFont=cmunit.ttf,BoldItalicFont=cmuntx.ttf]{cmunrm.ttf}, where \symbol{34}n\symbol{34} indicates the number of sides in the ring and \symbol{34}code\symbol{34} represents the specific content of each ring (bonds and atoms).

\begin{longtable}{p{1.0\linewidth}}
\begin{Shaded}
\begin{Highlighting}[]
\NormalTok{\textbackslash{}chemfig\{A*6(-B-C-D-E-F-)\}}
\end{Highlighting}
\end{Shaded}
\\


\begin{minipage}{0.30000\textwidth}
\begin{center}
\includegraphics[width=1.0\textwidth,height=6.5in,keepaspectratio]{../images/109.png}
\end{center}
\raggedright{}\myfigurewithoutcaption{109}
\end{minipage}\vspace{0.75cm}


\end{longtable}

\begin{longtable}{p{1.0\linewidth}}
\begin{Shaded}
\begin{Highlighting}[]
\NormalTok{\textbackslash{}chemfig\{A*5(-B-C-D-E-)\}}
\end{Highlighting}
\end{Shaded}
\\


\begin{minipage}{0.30000\textwidth}
\begin{center}
\includegraphics[width=1.0\textwidth,height=6.5in,keepaspectratio]{../images/110.png}
\end{center}
\raggedright{}\myfigurewithoutcaption{110}
\end{minipage}\vspace{0.75cm}


\end{longtable}

\begin{longtable}{p{1.0\linewidth}}
\begin{Shaded}
\begin{Highlighting}[]
\NormalTok{\textbackslash{}chemfig\{*6(=-=-=-)\}}
\end{Highlighting}
\end{Shaded}
\\


\begin{minipage}{0.30000\textwidth}
\begin{center}
\includegraphics[width=1.0\textwidth,height=6.5in,keepaspectratio]{../images/111.png}
\end{center}
\raggedright{}\myfigurewithoutcaption{111}
\end{minipage}\vspace{0.75cm}


\end{longtable}

\begin{longtable}{p{1.0\linewidth}}
\begin{Shaded}
\begin{Highlighting}[]
\NormalTok{\textbackslash{}chemfig\{**5(------)\}}
\end{Highlighting}
\end{Shaded}
\\


\begin{minipage}{0.30000\textwidth}
\begin{center}
\includegraphics[width=1.0\textwidth,height=6.5in,keepaspectratio]{../images/112.png}
\end{center}
\raggedright{}\myfigurewithoutcaption{112}
\end{minipage}\vspace{0.75cm}


\end{longtable}
\section{Lewis Structures}
\label{571}

Lewis structures use the syntax \textbackslash{}lewis\{<{}n1>{}<{}n2>{}...<{}ni>{},<{}atom>{}\}, where <{}ni>{} is a number between 0 and 7 representing the position of the electrons. By default, the electrons are represented by a dash (-{}). Appending a period (.) or colon (:) after a number will display single and paired electrons respectively.

\begin{longtable}{p{1.0\linewidth}}
\begin{Shaded}
\begin{Highlighting}[]
\NormalTok{\textbackslash{}lewis\{0.2.4.6.,C\}}
\end{Highlighting}
\end{Shaded}
\\


\begin{minipage}{0.30000\textwidth}
\begin{center}
\includegraphics[width=1.0\textwidth,height=6.5in,keepaspectratio]{../images/113.png}
\end{center}
\raggedright{}\myfigurewithoutcaption{113}
\end{minipage}\vspace{0.75cm}


\end{longtable}


Lewis structures can also be included within \textbackslash{}chemfig\{\}.


\begin{longtable}{p{1.0\linewidth}}
\begin{Shaded}
\begin{Highlighting}[]
\NormalTok{\textbackslash{}chemfig\{H-[:52.24]\textbackslash{}lewis\{1:3:,O\}-[::-104.48]H\}}
\end{Highlighting}
\end{Shaded}
\\


\begin{minipage}{0.30000\textwidth}
\begin{center}
\includegraphics[width=1.0\textwidth,height=6.5in,keepaspectratio]{../images/114.png}
\end{center}
\raggedright{}\myfigurewithoutcaption{114}
\end{minipage}\vspace{0.75cm}


\end{longtable}
\section{Ions}
\label{572}

For example, consider an acetate ion: 

\begin{longtable}{p{1.0\linewidth}}
\begin{Shaded}
\begin{Highlighting}[]
\NormalTok{\textbackslash{}chemfig\{-(-[1]O^\{-\})=[7]O\}}
\end{Highlighting}
\end{Shaded}
\\


\begin{minipage}{0.30000\textwidth}
\begin{center}
\includegraphics[width=1.0\textwidth,height=6.5in,keepaspectratio]{../images/115.png}
\end{center}
\raggedright{}\myfigurewithoutcaption{115}
\end{minipage}\vspace{0.75cm}


\end{longtable}

Because the chemfig commands enters the math mode, ion charges can be added as superscripts (one caveat: a negative ion requires that the minus sign be enclosed in brackets, as in the example). 

The charge of an ion can be circled by using {\ttfamily \setmainfont[Path=/usr/share/fonts/truetype/cmu/,UprightFont=cmunrm.ttf,BoldFont=cmunbx.ttf,ItalicFont=cmunti.ttf,BoldItalicFont=cmunbi.ttf]{cmuntt.ttf}\setmonofont[Path=/usr/share/fonts/truetype/cmu/,UprightFont=cmuntt.ttf,BoldFont=cmuntb.ttf,ItalicFont=cmunit.ttf,BoldItalicFont=cmuntx.ttf]{cmuntt.ttf}\ttfamily \textbackslash{}oplus}{$\text{ }$}\setmainfont[Path=/usr/share/fonts/truetype/cmu/,UprightFont=cmunrm.ttf,BoldFont=cmunbx.ttf,ItalicFont=cmunti.ttf,BoldItalicFont=cmunbi.ttf]{cmunrm.ttf}\setmonofont[Path=/usr/share/fonts/truetype/cmu/,UprightFont=cmuntt.ttf,BoldFont=cmuntb.ttf,ItalicFont=cmunit.ttf,BoldItalicFont=cmuntx.ttf]{cmunrm.ttf} and {\ttfamily \setmainfont[Path=/usr/share/fonts/truetype/cmu/,UprightFont=cmunrm.ttf,BoldFont=cmunbx.ttf,ItalicFont=cmunti.ttf,BoldItalicFont=cmunbi.ttf]{cmuntt.ttf}\setmonofont[Path=/usr/share/fonts/truetype/cmu/,UprightFont=cmuntt.ttf,BoldFont=cmuntb.ttf,ItalicFont=cmunit.ttf,BoldItalicFont=cmuntx.ttf]{cmuntt.ttf}\ttfamily \textbackslash{}ominus}\setmainfont[Path=/usr/share/fonts/truetype/cmu/,UprightFont=cmunrm.ttf,BoldFont=cmunbx.ttf,ItalicFont=cmunti.ttf,BoldItalicFont=cmunbi.ttf]{cmunrm.ttf}\setmonofont[Path=/usr/share/fonts/truetype/cmu/,UprightFont=cmuntt.ttf,BoldFont=cmuntb.ttf,ItalicFont=cmunit.ttf,BoldItalicFont=cmuntx.ttf]{cmunrm.ttf}:  

\begin{longtable}{p{1.0\linewidth}}
\begin{Shaded}
\begin{Highlighting}[]
\NormalTok{\textbackslash{}chemfig\{-(-[1]O^\{\textbackslash{}ominus\})=[7]O\}}
\end{Highlighting}
\end{Shaded}
\\


\begin{minipage}{0.30000\textwidth}
\begin{center}
\includegraphics[width=1.0\textwidth,height=6.5in,keepaspectratio]{../images/116.png}
\end{center}
\raggedright{}\myfigurewithoutcaption{116}
\end{minipage}\vspace{0.75cm}


\end{longtable}

Alternatively, charges can be placed above ions using \textbackslash{}chemabove\{\}\{\}: 

\begin{longtable}{p{1.0\linewidth}}
\begin{Shaded}
\begin{Highlighting}[]
\NormalTok{\textbackslash{}chemfig\{-\textbackslash{}chemabove\{N\}\{\textbackslash{}scriptstyle\textbackslash{}oplus\}(=[1]O)-[7]O^\{\textbackslash{}ominus\}\} }
\end{Highlighting}
\end{Shaded}
\\


\begin{minipage}{0.30000\textwidth}
\begin{center}
\includegraphics[width=1.0\textwidth,height=6.5in,keepaspectratio]{../images/117.png}
\end{center}
\raggedright{}\myfigurewithoutcaption{117}
\end{minipage}\vspace{0.75cm}


\end{longtable}
\section{Resonance Structures and Formal Charges}
\label{573}

Resonance structures require a few math commands:


\begin{Shaded}
\begin{Highlighting}[]

\CommentTok{\%\ensuremath{\text{ }}see\ensuremath{\text{ }}"Advanced\ensuremath{\text{ }}Mathematics"\ensuremath{\text{ }}for\ensuremath{\text{ }}use\ensuremath{\text{ }}of\ensuremath{\text{ }}\textbackslash{}left\ensuremath{\text{ }}and\ensuremath{\text{ }}\textbackslash{}right}\newline
\CommentTok{\%\ensuremath{\text{ }}add\ensuremath{\text{ }}to\ensuremath{\text{ }}preamble:}\newline
\CommentTok{\%	\textbackslash{}usepackage\{mathtools\}	\%\ensuremath{\text{ }}\textbackslash{}Longleftrightarrow}\newline
\NormalTok{\$\textbackslash{}left\textbackslash{}\{\textbackslash{}chemfig\{O-N(=[:60]O)-[:300]O\}\textbackslash{}right\textbackslash{}\}}\newline
\NormalTok{\textbackslash{}Longleftrightarrow\ensuremath{\text{ }}}\newline
\NormalTok{\textbackslash{}left\textbackslash{}\{\textbackslash{}chemfig\{O=N(-[:60]O)-[:300]O\}\textbackslash{}right\textbackslash{}\}\ensuremath{\text{ }}}\newline
\NormalTok{\textbackslash{}Longleftrightarrow\ensuremath{\text{ }}}\newline
\NormalTok{\textbackslash{}left\textbackslash{}\{\textbackslash{}chemfig\{O-N(-[:60]O)=[:300]O\}\textbackslash{}right\textbackslash{}\}\$}\newline
\ensuremath{\text{ }}\newline
\end{Highlighting}
\end{Shaded}
\section{Chemical Reactions}
\label{574}


Commands {\ttfamily \setmainfont[Path=/usr/share/fonts/truetype/cmu/,UprightFont=cmunrm.ttf,BoldFont=cmunbx.ttf,ItalicFont=cmunti.ttf,BoldItalicFont=cmunbi.ttf]{cmuntt.ttf}\setmonofont[Path=/usr/share/fonts/truetype/cmu/,UprightFont=cmuntt.ttf,BoldFont=cmuntb.ttf,ItalicFont=cmunit.ttf,BoldItalicFont=cmuntx.ttf]{cmuntt.ttf}\ttfamily \textbackslash{}chemrel}{$\text{ }$}\setmainfont[Path=/usr/share/fonts/truetype/cmu/,UprightFont=cmunrm.ttf,BoldFont=cmunbx.ttf,ItalicFont=cmunti.ttf,BoldItalicFont=cmunbi.ttf]{cmunrm.ttf}\setmonofont[Path=/usr/share/fonts/truetype/cmu/,UprightFont=cmuntt.ttf,BoldFont=cmuntb.ttf,ItalicFont=cmunit.ttf,BoldItalicFont=cmuntx.ttf]{cmunrm.ttf} and {\ttfamily \setmainfont[Path=/usr/share/fonts/truetype/cmu/,UprightFont=cmunrm.ttf,BoldFont=cmunbx.ttf,ItalicFont=cmunti.ttf,BoldItalicFont=cmunbi.ttf]{cmuntt.ttf}\setmonofont[Path=/usr/share/fonts/truetype/cmu/,UprightFont=cmuntt.ttf,BoldFont=cmuntb.ttf,ItalicFont=cmunit.ttf,BoldItalicFont=cmuntx.ttf]{cmuntt.ttf}\ttfamily \textbackslash{}chemsign}{$\text{ }$}\setmainfont[Path=/usr/share/fonts/truetype/cmu/,UprightFont=cmunrm.ttf,BoldFont=cmunbx.ttf,ItalicFont=cmunti.ttf,BoldItalicFont=cmunbi.ttf]{cmunrm.ttf}\setmonofont[Path=/usr/share/fonts/truetype/cmu/,UprightFont=cmuntt.ttf,BoldFont=cmuntb.ttf,ItalicFont=cmunit.ttf,BoldItalicFont=cmuntx.ttf]{cmunrm.ttf} were removed from chemfig package in latest versions, so in order to draw chemical reactions, one must instead use respectively {\ttfamily \setmainfont[Path=/usr/share/fonts/truetype/cmu/,UprightFont=cmunrm.ttf,BoldFont=cmunbx.ttf,ItalicFont=cmunti.ttf,BoldItalicFont=cmunbi.ttf]{cmuntt.ttf}\setmonofont[Path=/usr/share/fonts/truetype/cmu/,UprightFont=cmuntt.ttf,BoldFont=cmuntb.ttf,ItalicFont=cmunit.ttf,BoldItalicFont=cmuntx.ttf]{cmuntt.ttf}\ttfamily \textbackslash{}arrow}{$\text{ }$}\setmainfont[Path=/usr/share/fonts/truetype/cmu/,UprightFont=cmunrm.ttf,BoldFont=cmunbx.ttf,ItalicFont=cmunti.ttf,BoldItalicFont=cmunbi.ttf]{cmunrm.ttf}\setmonofont[Path=/usr/share/fonts/truetype/cmu/,UprightFont=cmuntt.ttf,BoldFont=cmuntb.ttf,ItalicFont=cmunit.ttf,BoldItalicFont=cmuntx.ttf]{cmunrm.ttf} and {\ttfamily \setmainfont[Path=/usr/share/fonts/truetype/cmu/,UprightFont=cmunrm.ttf,BoldFont=cmunbx.ttf,ItalicFont=cmunti.ttf,BoldItalicFont=cmunbi.ttf]{cmuntt.ttf}\setmonofont[Path=/usr/share/fonts/truetype/cmu/,UprightFont=cmuntt.ttf,BoldFont=cmuntb.ttf,ItalicFont=cmunit.ttf,BoldItalicFont=cmuntx.ttf]{cmuntt.ttf}\ttfamily \textbackslash{}+}{$\text{ }$}\setmainfont[Path=/usr/share/fonts/truetype/cmu/,UprightFont=cmunrm.ttf,BoldFont=cmunbx.ttf,ItalicFont=cmunti.ttf,BoldItalicFont=cmunbi.ttf]{cmunrm.ttf}\setmonofont[Path=/usr/share/fonts/truetype/cmu/,UprightFont=cmuntt.ttf,BoldFont=cmuntb.ttf,ItalicFont=cmunit.ttf,BoldItalicFont=cmuntx.ttf]{cmunrm.ttf} commands in a block surrounded with {\ttfamily \setmainfont[Path=/usr/share/fonts/truetype/cmu/,UprightFont=cmunrm.ttf,BoldFont=cmunbx.ttf,ItalicFont=cmunti.ttf,BoldItalicFont=cmunbi.ttf]{cmuntt.ttf}\setmonofont[Path=/usr/share/fonts/truetype/cmu/,UprightFont=cmuntt.ttf,BoldFont=cmuntb.ttf,ItalicFont=cmunit.ttf,BoldItalicFont=cmuntx.ttf]{cmuntt.ttf}\ttfamily \textbackslash{}schemestart}{$\text{ }$}\setmainfont[Path=/usr/share/fonts/truetype/cmu/,UprightFont=cmunrm.ttf,BoldFont=cmunbx.ttf,ItalicFont=cmunti.ttf,BoldItalicFont=cmunbi.ttf]{cmunrm.ttf}\setmonofont[Path=/usr/share/fonts/truetype/cmu/,UprightFont=cmuntt.ttf,BoldFont=cmuntb.ttf,ItalicFont=cmunit.ttf,BoldItalicFont=cmuntx.ttf]{cmunrm.ttf} and {\ttfamily \setmainfont[Path=/usr/share/fonts/truetype/cmu/,UprightFont=cmunrm.ttf,BoldFont=cmunbx.ttf,ItalicFont=cmunti.ttf,BoldItalicFont=cmunbi.ttf]{cmuntt.ttf}\setmonofont[Path=/usr/share/fonts/truetype/cmu/,UprightFont=cmuntt.ttf,BoldFont=cmuntb.ttf,ItalicFont=cmunit.ttf,BoldItalicFont=cmuntx.ttf]{cmuntt.ttf}\ttfamily \textbackslash{}schemestop}\setmainfont[Path=/usr/share/fonts/truetype/cmu/,UprightFont=cmunrm.ttf,BoldFont=cmunbx.ttf,ItalicFont=cmunti.ttf,BoldItalicFont=cmunbi.ttf]{cmunrm.ttf}\setmonofont[Path=/usr/share/fonts/truetype/cmu/,UprightFont=cmuntt.ttf,BoldFont=cmuntb.ttf,ItalicFont=cmunit.ttf,BoldItalicFont=cmuntx.ttf]{cmunrm.ttf}.

There are a few types of arrows that can be drawn with the {\ttfamily \setmainfont[Path=/usr/share/fonts/truetype/cmu/,UprightFont=cmunrm.ttf,BoldFont=cmunbx.ttf,ItalicFont=cmunti.ttf,BoldItalicFont=cmunbi.ttf]{cmuntt.ttf}\setmonofont[Path=/usr/share/fonts/truetype/cmu/,UprightFont=cmuntt.ttf,BoldFont=cmuntb.ttf,ItalicFont=cmunit.ttf,BoldItalicFont=cmuntx.ttf]{cmuntt.ttf}\ttfamily \textbackslash{}arrow}{$\text{ }$}\setmainfont[Path=/usr/share/fonts/truetype/cmu/,UprightFont=cmunrm.ttf,BoldFont=cmunbx.ttf,ItalicFont=cmunti.ttf,BoldItalicFont=cmunbi.ttf]{cmunrm.ttf}\setmonofont[Path=/usr/share/fonts/truetype/cmu/,UprightFont=cmuntt.ttf,BoldFont=cmuntb.ttf,ItalicFont=cmunit.ttf,BoldItalicFont=cmuntx.ttf]{cmunrm.ttf} command:

\begin{longtable}{p{1.0\linewidth}}
\begin{Shaded}
\begin{Highlighting}[]

\NormalTok{\textbackslash{}schemestart A\textbackslash{}arrow\{->\}B\textbackslash{}schemestop\textbackslash{}par }\CommentTok{% by default}
\NormalTok{\textbackslash{}schemestart A\textbackslash{}arrow\{-/>\}B \textbackslash{}schemestop\textbackslash{}par}
\NormalTok{\textbackslash{}schemestart A\textbackslash{}arrow\{<-\}B \textbackslash{}schemestop\textbackslash{}par}
\NormalTok{\textbackslash{}schemestart A\textbackslash{}arrow\{<->\}B \textbackslash{}schemestop\textbackslash{}par}
\NormalTok{\textbackslash{}schemestart A\textbackslash{}arrow\{<=>\}B \textbackslash{}schemestop\textbackslash{}par}
\NormalTok{\textbackslash{}schemestart A\textbackslash{}arrow\{<->>\}B \textbackslash{}schemestop\textbackslash{}par}
\NormalTok{\textbackslash{}schemestart A\textbackslash{}arrow\{<<->\}B \textbackslash{}schemestop\textbackslash{}par}
\NormalTok{\textbackslash{}schemestart A\textbackslash{}arrow\{0\}B \textbackslash{}schemestop\textbackslash{}par}
\NormalTok{\textbackslash{}schemestart A\textbackslash{}arrow\{-U>\}B \textbackslash{}schemestop\textbackslash{}par}
\NormalTok{\textbackslash{}schemestart}
\NormalTok{A\textbackslash{}arrow[,,->] B\textbackslash{}arrow[,,-\{Triangle[slant=0.5,blue,width=10pt]\}]}
\NormalTok{C\textbackslash{}arrow[,,-\{CF[sharp]\}] D \textbackslash{}+ E}
\NormalTok{\textbackslash{}schemestop}
\end{Highlighting}
\end{Shaded}
\\


\begin{minipage}{1.0\linewidth}
\begin{center}
\includegraphics[width=1.0\linewidth,height=6.5in,keepaspectratio]{../images/118.png}
\end{center}
\raggedright{}\myfigurewithcaption{118}{Exemples de l\textquotesingle{}instruction \textbackslash{}arrow et \textbackslash{}+ du paquet chemfig en \textbackslash{}LaTeX}
\end{minipage}\vspace{0.75cm}


\end{longtable}

For more details on the {\ttfamily \setmainfont[Path=/usr/share/fonts/truetype/cmu/,UprightFont=cmunrm.ttf,BoldFont=cmunbx.ttf,ItalicFont=cmunti.ttf,BoldItalicFont=cmunbi.ttf]{cmuntt.ttf}\setmonofont[Path=/usr/share/fonts/truetype/cmu/,UprightFont=cmuntt.ttf,BoldFont=cmuntb.ttf,ItalicFont=cmunit.ttf,BoldItalicFont=cmuntx.ttf]{cmuntt.ttf}\ttfamily \textbackslash{}arrow}{$\text{ }$}\setmainfont[Path=/usr/share/fonts/truetype/cmu/,UprightFont=cmunrm.ttf,BoldFont=cmunbx.ttf,ItalicFont=cmunti.ttf,BoldItalicFont=cmunbi.ttf]{cmunrm.ttf}\setmonofont[Path=/usr/share/fonts/truetype/cmu/,UprightFont=cmuntt.ttf,BoldFont=cmuntb.ttf,ItalicFont=cmunit.ttf,BoldItalicFont=cmuntx.ttf]{cmunrm.ttf} command and chemical reactions in chemfig in general, consult the Part IV \symbol{34}Reaction schemes\symbol{34} of the \myhref{http://mirror.ibcp.fr/pub/CTAN/macros/latex/contrib/chemfig/chemfig_doc_en.pdf}{chemfig documentation file}.
\subsection{Older versions}
\label{575}
Chemical reactions can be created with the following commands: 


\begin{Shaded}
\begin{Highlighting}[]

\NormalTok{\textbackslash{}chemrel[<arg1>][<arg2>]\{<arrow\ensuremath{\text{ }}code>\}}\newline
\end{Highlighting}
\end{Shaded}



\begin{Shaded}
\begin{Highlighting}[]

\NormalTok{\textbackslash{}chemsign+	}\CommentTok{\%\ensuremath{\text{ }}produces\ensuremath{\text{ }}a\ensuremath{\text{ }}+}\newline
\end{Highlighting}
\end{Shaded}


In {\ttfamily \setmainfont[Path=/usr/share/fonts/truetype/cmu/,UprightFont=cmunrm.ttf,BoldFont=cmunbx.ttf,ItalicFont=cmunti.ttf,BoldItalicFont=cmunbi.ttf]{cmuntt.ttf}\setmonofont[Path=/usr/share/fonts/truetype/cmu/,UprightFont=cmuntt.ttf,BoldFont=cmuntb.ttf,ItalicFont=cmunit.ttf,BoldItalicFont=cmuntx.ttf]{cmuntt.ttf}\ttfamily \textbackslash{}chemrel\{\}}\setmainfont[Path=/usr/share/fonts/truetype/cmu/,UprightFont=cmunrm.ttf,BoldFont=cmunbx.ttf,ItalicFont=cmunti.ttf,BoldItalicFont=cmunbi.ttf]{cmunrm.ttf}\setmonofont[Path=/usr/share/fonts/truetype/cmu/,UprightFont=cmuntt.ttf,BoldFont=cmuntb.ttf,ItalicFont=cmunit.ttf,BoldItalicFont=cmuntx.ttf]{cmunrm.ttf}, <{}arg1>{} and <{}arg2>{} represent text placed above and below the arrow, respectively.

There are four types of arrows that can be produced with {\ttfamily \setmainfont[Path=/usr/share/fonts/truetype/cmu/,UprightFont=cmunrm.ttf,BoldFont=cmunbx.ttf,ItalicFont=cmunti.ttf,BoldItalicFont=cmunbi.ttf]{cmuntt.ttf}\setmonofont[Path=/usr/share/fonts/truetype/cmu/,UprightFont=cmuntt.ttf,BoldFont=cmuntb.ttf,ItalicFont=cmunit.ttf,BoldItalicFont=cmuntx.ttf]{cmuntt.ttf}\ttfamily \textbackslash{}chemrel\{\}}\setmainfont[Path=/usr/share/fonts/truetype/cmu/,UprightFont=cmunrm.ttf,BoldFont=cmunbx.ttf,ItalicFont=cmunti.ttf,BoldItalicFont=cmunbi.ttf]{cmunrm.ttf}\setmonofont[Path=/usr/share/fonts/truetype/cmu/,UprightFont=cmuntt.ttf,BoldFont=cmuntb.ttf,ItalicFont=cmunit.ttf,BoldItalicFont=cmuntx.ttf]{cmunrm.ttf}:


\begin{Shaded}
\begin{Highlighting}[]

\NormalTok{A\textbackslash{}chemrel\{->\}B\textbackslash{}par\ensuremath{\text{ }}}\newline
\NormalTok{A\textbackslash{}chemrel\{<-\}B\textbackslash{}par\ensuremath{\text{ }}}\newline
\NormalTok{A\textbackslash{}chemrel\{<->\}B\textbackslash{}par\ensuremath{\text{ }}}\newline
\NormalTok{A\textbackslash{}chemrel\{<>\}B}\newline
\end{Highlighting}
\end{Shaded}

\section{Naming Chemical Graphics}
\label{576}

Molecules can be named with the command


\begin{Shaded}
\begin{Highlighting}[]

\NormalTok{\textbackslash{}chemname[<dim>]\{\textbackslash{}chemfig\{<code\ensuremath{\text{ }}of\ensuremath{\text{ }}the\ensuremath{\text{ }}molecule>\}\}\{<name>\}}\newline
\end{Highlighting}
\end{Shaded}


<{}dim>{} is inserted between the bottom of the molecule and the top of the name defined by <{}name>{}. It is 1.5ex by default. 

<{}name>{} will be centered relative to the molecule it describes. 


\begin{Shaded}
\begin{Highlighting}[]

\NormalTok{\textbackslash{}chemname\{\textbackslash{}chemfig\{R-C(-[:-30]OH)=[:30]O\}\}\{Carboxylic\ensuremath{\text{ }}acid\}\ensuremath{\text{ }}}\newline
\NormalTok{\textbackslash{}chemsign\{+\}\ensuremath{\text{ }}}\newline
\NormalTok{\textbackslash{}chemname\{\textbackslash{}chemfig\{R’OH\}\}\{Alcohol\}\ensuremath{\text{ }}}\newline
\NormalTok{\textbackslash{}chemrel\{->\}\ensuremath{\text{ }}}\newline
\NormalTok{\textbackslash{}chemname\{\textbackslash{}chemfig\{R-C(-[:-30]OR’)=[:30]O\}\}\{Ester\}\ensuremath{\text{ }}}\newline
\NormalTok{\textbackslash{}chemsign\{+\}\ensuremath{\text{ }}}\newline
\NormalTok{\textbackslash{}chemname\{\textbackslash{}chemfig\{H_2O\}\}\{Water\}\ensuremath{\text{ }}}\newline
\end{Highlighting}
\end{Shaded}


In the reaction above, {\ttfamily \setmainfont[Path=/usr/share/fonts/truetype/cmu/,UprightFont=cmunrm.ttf,BoldFont=cmunbx.ttf,ItalicFont=cmunti.ttf,BoldItalicFont=cmunbi.ttf]{cmuntt.ttf}\setmonofont[Path=/usr/share/fonts/truetype/cmu/,UprightFont=cmuntt.ttf,BoldFont=cmuntb.ttf,ItalicFont=cmunit.ttf,BoldItalicFont=cmuntx.ttf]{cmuntt.ttf}\ttfamily \textbackslash{}chemname\{\}}{$\text{ }$}\setmainfont[Path=/usr/share/fonts/truetype/cmu/,UprightFont=cmunrm.ttf,BoldFont=cmunbx.ttf,ItalicFont=cmunti.ttf,BoldItalicFont=cmunbi.ttf]{cmunrm.ttf}\setmonofont[Path=/usr/share/fonts/truetype/cmu/,UprightFont=cmuntt.ttf,BoldFont=cmuntb.ttf,ItalicFont=cmunit.ttf,BoldItalicFont=cmuntx.ttf]{cmunrm.ttf} inserts 1.5ex plus the depth of the carboxylic acid molecule in between each molecule and their respective names. This is because the graphic for the first molecule in the reaction (carboxylic acid) extends deeper than the rest of the molecules. A different result is produced by putting the alcohol first:


\begin{Shaded}
\begin{Highlighting}[]

\NormalTok{\textbackslash{}chemname\{\textbackslash{}chemfig\{R’OH\}\}\{Alcohol\}\ensuremath{\text{ }}}\newline
\NormalTok{\textbackslash{}chemsign\{+\}\ensuremath{\text{ }}}\newline
\NormalTok{\textbackslash{}chemname\{\textbackslash{}chemfig\{R-C(-[:-30]OH)=[:30]O\}\}\{Carboxylic\ensuremath{\text{ }}acid\}\ensuremath{\text{ }}}\newline
\NormalTok{\textbackslash{}chemrel\{->\}\ensuremath{\text{ }}}\newline
\NormalTok{\textbackslash{}chemname\{\textbackslash{}chemfig\{R-C(-[:-30]OR’)=[:30]O\}\}\{Ester\}\ensuremath{\text{ }}}\newline
\NormalTok{\textbackslash{}chemsign\{+\}\ensuremath{\text{ }}}\newline
\NormalTok{\textbackslash{}chemname\{\textbackslash{}chemfig\{H_2O\}\}\{Water\}\ensuremath{\text{ }}}\newline
\end{Highlighting}
\end{Shaded}


This is fixed by adding {\ttfamily \setmainfont[Path=/usr/share/fonts/truetype/cmu/,UprightFont=cmunrm.ttf,BoldFont=cmunbx.ttf,ItalicFont=cmunti.ttf,BoldItalicFont=cmunbi.ttf]{cmuntt.ttf}\setmonofont[Path=/usr/share/fonts/truetype/cmu/,UprightFont=cmuntt.ttf,BoldFont=cmuntb.ttf,ItalicFont=cmunit.ttf,BoldItalicFont=cmuntx.ttf]{cmuntt.ttf}\ttfamily \textbackslash{}chemnameinit\{<{}deepest molecule>{}\}}{$\text{ }$}\setmainfont[Path=/usr/share/fonts/truetype/cmu/,UprightFont=cmunrm.ttf,BoldFont=cmunbx.ttf,ItalicFont=cmunti.ttf,BoldItalicFont=cmunbi.ttf]{cmunrm.ttf}\setmonofont[Path=/usr/share/fonts/truetype/cmu/,UprightFont=cmuntt.ttf,BoldFont=cmuntb.ttf,ItalicFont=cmunit.ttf,BoldItalicFont=cmuntx.ttf]{cmunrm.ttf} before the first instance of {\ttfamily \setmainfont[Path=/usr/share/fonts/truetype/cmu/,UprightFont=cmunrm.ttf,BoldFont=cmunbx.ttf,ItalicFont=cmunti.ttf,BoldItalicFont=cmunbi.ttf]{cmuntt.ttf}\setmonofont[Path=/usr/share/fonts/truetype/cmu/,UprightFont=cmuntt.ttf,BoldFont=cmuntb.ttf,ItalicFont=cmunit.ttf,BoldItalicFont=cmuntx.ttf]{cmuntt.ttf}\ttfamily \textbackslash{}chemname\{\}}{$\text{ }$}\setmainfont[Path=/usr/share/fonts/truetype/cmu/,UprightFont=cmunrm.ttf,BoldFont=cmunbx.ttf,ItalicFont=cmunti.ttf,BoldItalicFont=cmunbi.ttf]{cmunrm.ttf}\setmonofont[Path=/usr/share/fonts/truetype/cmu/,UprightFont=cmuntt.ttf,BoldFont=cmuntb.ttf,ItalicFont=cmunit.ttf,BoldItalicFont=cmuntx.ttf]{cmunrm.ttf} in a reaction and by adding {\ttfamily \setmainfont[Path=/usr/share/fonts/truetype/cmu/,UprightFont=cmunrm.ttf,BoldFont=cmunbx.ttf,ItalicFont=cmunti.ttf,BoldItalicFont=cmunbi.ttf]{cmuntt.ttf}\setmonofont[Path=/usr/share/fonts/truetype/cmu/,UprightFont=cmuntt.ttf,BoldFont=cmuntb.ttf,ItalicFont=cmunit.ttf,BoldItalicFont=cmuntx.ttf]{cmuntt.ttf}\ttfamily \textbackslash{}chemnameinit\{\}}{$\text{ }$}\setmainfont[Path=/usr/share/fonts/truetype/cmu/,UprightFont=cmunrm.ttf,BoldFont=cmunbx.ttf,ItalicFont=cmunti.ttf,BoldItalicFont=cmunbi.ttf]{cmunrm.ttf}\setmonofont[Path=/usr/share/fonts/truetype/cmu/,UprightFont=cmuntt.ttf,BoldFont=cmuntb.ttf,ItalicFont=cmunit.ttf,BoldItalicFont=cmuntx.ttf]{cmunrm.ttf} after the reaction: 


\begin{Shaded}
\begin{Highlighting}[]

\NormalTok{\textbackslash{}chemnameinit\{\textbackslash{}chemfig\{R-C(-[:-30]OH)=[:30]O\}\}\ensuremath{\text{ }}}\newline
\NormalTok{\textbackslash{}chemname\{\textbackslash{}chemfig\{R’OH\}\}\{Alcohol\}\ensuremath{\text{ }}}\newline
\NormalTok{\textbackslash{}chemsign\{+\}\ensuremath{\text{ }}}\newline
\NormalTok{\textbackslash{}chemname\{\textbackslash{}chemfig\{R-C(-[:-30]OH)=[:30]O\}\}\{Carboxylic\ensuremath{\text{ }}acid\}\ensuremath{\text{ }}}\newline
\NormalTok{\textbackslash{}chemrel\{->\}\ensuremath{\text{ }}}\newline
\NormalTok{\textbackslash{}chemname\{\textbackslash{}chemfig\{R-C(-[:-30]OR’)=[:30]O\}\}\{Ester\}\ensuremath{\text{ }}}\newline
\NormalTok{\textbackslash{}chemsign\{+\}\ensuremath{\text{ }}}\newline
\NormalTok{\textbackslash{}chemname\{\textbackslash{}chemfig\{H_2O\}\}\{Water\}\ensuremath{\text{ }}}\newline
\NormalTok{\textbackslash{}chemnameinit\{\}\ensuremath{\text{ }}}\newline
\end{Highlighting}
\end{Shaded}


Lastly, adding {\ttfamily \setmainfont[Path=/usr/share/fonts/truetype/cmu/,UprightFont=cmunrm.ttf,BoldFont=cmunbx.ttf,ItalicFont=cmunti.ttf,BoldItalicFont=cmunbi.ttf]{cmuntt.ttf}\setmonofont[Path=/usr/share/fonts/truetype/cmu/,UprightFont=cmuntt.ttf,BoldFont=cmuntb.ttf,ItalicFont=cmunit.ttf,BoldItalicFont=cmuntx.ttf]{cmuntt.ttf}\ttfamily \textbackslash{}\textbackslash{}}{$\text{ }$}\setmainfont[Path=/usr/share/fonts/truetype/cmu/,UprightFont=cmunrm.ttf,BoldFont=cmunbx.ttf,ItalicFont=cmunti.ttf,BoldItalicFont=cmunbi.ttf]{cmunrm.ttf}\setmonofont[Path=/usr/share/fonts/truetype/cmu/,UprightFont=cmuntt.ttf,BoldFont=cmuntb.ttf,ItalicFont=cmunit.ttf,BoldItalicFont=cmuntx.ttf]{cmunrm.ttf} in <{}name>{} will produce a line-{}break, allowing the name to span multiple lines.
\section{Advanced Graphics}
\label{577}

For advanced commands and examples, refer to the \myhref{http://mirrors.ctan.org/macros/generic/chemfig/chemfig-en.pdf}{chemfig manual}, where a more thorough and complete introduction to the package can be found.


\section{mhchem Package}
\label{578}

\myhref{http://www.ctan.org/tex-archive/macros/latex/contrib/mhchem/}{mhchem} is a package used to typeset chemical formulae and equations. As well as typeset basic 2D chemical structures. To use this package, add the following to your preamble:


\begin{Shaded}
\begin{Highlighting}[]

\NormalTok{\textbackslash{}usepackage[version=3]\{mhchem\}}\newline
\end{Highlighting}
\end{Shaded}


Chemical species are included using the {\ttfamily \setmainfont[Path=/usr/share/fonts/truetype/cmu/,UprightFont=cmunrm.ttf,BoldFont=cmunbx.ttf,ItalicFont=cmunti.ttf,BoldItalicFont=cmunbi.ttf]{cmuntt.ttf}\setmonofont[Path=/usr/share/fonts/truetype/cmu/,UprightFont=cmuntt.ttf,BoldFont=cmuntb.ttf,ItalicFont=cmunit.ttf,BoldItalicFont=cmuntx.ttf]{cmuntt.ttf}\ttfamily \textbackslash{}ce}{$\text{ }$}\setmainfont[Path=/usr/share/fonts/truetype/cmu/,UprightFont=cmunrm.ttf,BoldFont=cmunbx.ttf,ItalicFont=cmunti.ttf,BoldItalicFont=cmunbi.ttf]{cmunrm.ttf}\setmonofont[Path=/usr/share/fonts/truetype/cmu/,UprightFont=cmuntt.ttf,BoldFont=cmuntb.ttf,ItalicFont=cmunit.ttf,BoldItalicFont=cmuntx.ttf]{cmunrm.ttf} command. For example



\begin{Shaded}
\begin{Highlighting}[]

\NormalTok{\textbackslash{}ce\{3H2O\}\ensuremath{\text{ }}\textbackslash{}\textbackslash{}}\newline
\NormalTok{\textbackslash{}ce\{1/2H2O\}\ensuremath{\text{ }}\textbackslash{}\textbackslash{}}\newline
\NormalTok{\textbackslash{}ce\{AgCl2-\}\ensuremath{\text{ }}\textbackslash{}\textbackslash{}}\newline
\NormalTok{\textbackslash{}ce\{H2_\{(aq)\}\}\ensuremath{\text{ }}\textbackslash{}\textbackslash{}}\newline
\end{Highlighting}
\end{Shaded}


renders:
\\

\TemplateSpaceIndent{$\text{ }${}<{}ce>{}3H2O<{}/ce>{}$\text{ }$\newline{}
$\text{ }${}<{}ce>{}1/2H2O<{}/ce>{}$\text{ }$\newline{}
$\text{ }${}<{}ce>{}AgCl2-{}<{}/ce>{}$\text{ }$\newline{}
$\text{ }${}<{}ce>{}H2\_\{(aq)\}<{}/ce>{}}


For more examples, see \myhref{https://en.meta.org/wiki/Help\%3ADisplaying\%20a\%20formula\%23Chemistry\%202}{meta:Help:Displaying a formula\#Chemistry 2}.

A few things here are automatically typeset; The 2 in {\ttfamily \setmainfont[Path=/usr/share/fonts/truetype/cmu/,UprightFont=cmunrm.ttf,BoldFont=cmunbx.ttf,ItalicFont=cmunti.ttf,BoldItalicFont=cmunbi.ttf]{cmuntt.ttf}\setmonofont[Path=/usr/share/fonts/truetype/cmu/,UprightFont=cmuntt.ttf,BoldFont=cmuntb.ttf,ItalicFont=cmunit.ttf,BoldItalicFont=cmuntx.ttf]{cmuntt.ttf}\ttfamily \textbackslash{}ce\{H2O\}}{$\text{ }$}\setmainfont[Path=/usr/share/fonts/truetype/cmu/,UprightFont=cmunrm.ttf,BoldFont=cmunbx.ttf,ItalicFont=cmunti.ttf,BoldItalicFont=cmunbi.ttf]{cmunrm.ttf}\setmonofont[Path=/usr/share/fonts/truetype/cmu/,UprightFont=cmuntt.ttf,BoldFont=cmuntb.ttf,ItalicFont=cmunit.ttf,BoldItalicFont=cmuntx.ttf]{cmunrm.ttf} is automatically subscripted without requiring additional commands. The amount of the species precedes the formula. 1/2 and other fractional amounts are automatically typeset as in {\ttfamily \setmainfont[Path=/usr/share/fonts/truetype/cmu/,UprightFont=cmunrm.ttf,BoldFont=cmunbx.ttf,ItalicFont=cmunti.ttf,BoldItalicFont=cmunbi.ttf]{cmuntt.ttf}\setmonofont[Path=/usr/share/fonts/truetype/cmu/,UprightFont=cmuntt.ttf,BoldFont=cmuntb.ttf,ItalicFont=cmunit.ttf,BoldItalicFont=cmuntx.ttf]{cmuntt.ttf}\ttfamily \textbackslash{}ce\{1/2H2O\}}\setmainfont[Path=/usr/share/fonts/truetype/cmu/,UprightFont=cmunrm.ttf,BoldFont=cmunbx.ttf,ItalicFont=cmunti.ttf,BoldItalicFont=cmunbi.ttf]{cmunrm.ttf}\setmonofont[Path=/usr/share/fonts/truetype/cmu/,UprightFont=cmuntt.ttf,BoldFont=cmuntb.ttf,ItalicFont=cmunit.ttf,BoldItalicFont=cmuntx.ttf]{cmunrm.ttf}. The charge in {\ttfamily \setmainfont[Path=/usr/share/fonts/truetype/cmu/,UprightFont=cmunrm.ttf,BoldFont=cmunbx.ttf,ItalicFont=cmunti.ttf,BoldItalicFont=cmunbi.ttf]{cmuntt.ttf}\setmonofont[Path=/usr/share/fonts/truetype/cmu/,UprightFont=cmuntt.ttf,BoldFont=cmuntb.ttf,ItalicFont=cmunit.ttf,BoldItalicFont=cmuntx.ttf]{cmuntt.ttf}\ttfamily \textbackslash{}ce\{AgCl2-{}\}}{$\text{ }$}\setmainfont[Path=/usr/share/fonts/truetype/cmu/,UprightFont=cmunrm.ttf,BoldFont=cmunbx.ttf,ItalicFont=cmunti.ttf,BoldItalicFont=cmunbi.ttf]{cmunrm.ttf}\setmonofont[Path=/usr/share/fonts/truetype/cmu/,UprightFont=cmuntt.ttf,BoldFont=cmuntb.ttf,ItalicFont=cmunit.ttf,BoldItalicFont=cmuntx.ttf]{cmunrm.ttf}  is automatically superscripted. If the charge is neither 1 or -{}1, a {\ttfamily \setmainfont[Path=/usr/share/fonts/truetype/cmu/,UprightFont=cmunrm.ttf,BoldFont=cmunbx.ttf,ItalicFont=cmunti.ttf,BoldItalicFont=cmunbi.ttf]{cmuntt.ttf}\setmonofont[Path=/usr/share/fonts/truetype/cmu/,UprightFont=cmuntt.ttf,BoldFont=cmuntb.ttf,ItalicFont=cmunit.ttf,BoldItalicFont=cmuntx.ttf]{cmuntt.ttf}\ttfamily \^{}}{$\text{ }$}\setmainfont[Path=/usr/share/fonts/truetype/cmu/,UprightFont=cmunrm.ttf,BoldFont=cmunbx.ttf,ItalicFont=cmunti.ttf,BoldItalicFont=cmunbi.ttf]{cmunrm.ttf}\setmonofont[Path=/usr/share/fonts/truetype/cmu/,UprightFont=cmuntt.ttf,BoldFont=cmuntb.ttf,ItalicFont=cmunit.ttf,BoldItalicFont=cmuntx.ttf]{cmunrm.ttf} will superscript it, as in {\ttfamily \setmainfont[Path=/usr/share/fonts/truetype/cmu/,UprightFont=cmunrm.ttf,BoldFont=cmunbx.ttf,ItalicFont=cmunti.ttf,BoldItalicFont=cmunbi.ttf]{cmuntt.ttf}\setmonofont[Path=/usr/share/fonts/truetype/cmu/,UprightFont=cmuntt.ttf,BoldFont=cmuntb.ttf,ItalicFont=cmunit.ttf,BoldItalicFont=cmuntx.ttf]{cmuntt.ttf}\ttfamily \textbackslash{}ce\{AgCl2-{}\}}\setmainfont[Path=/usr/share/fonts/truetype/cmu/,UprightFont=cmunrm.ttf,BoldFont=cmunbx.ttf,ItalicFont=cmunti.ttf,BoldItalicFont=cmunbi.ttf]{cmunrm.ttf}\setmonofont[Path=/usr/share/fonts/truetype/cmu/,UprightFont=cmuntt.ttf,BoldFont=cmuntb.ttf,ItalicFont=cmunit.ttf,BoldItalicFont=cmuntx.ttf]{cmunrm.ttf}. The phase is not automatically subscripted and needs to be enclosed in parenthesis preceded with a {\ttfamily \setmainfont[Path=/usr/share/fonts/truetype/cmu/,UprightFont=cmunrm.ttf,BoldFont=cmunbx.ttf,ItalicFont=cmunti.ttf,BoldItalicFont=cmunbi.ttf]{cmuntt.ttf}\setmonofont[Path=/usr/share/fonts/truetype/cmu/,UprightFont=cmuntt.ttf,BoldFont=cmuntb.ttf,ItalicFont=cmunit.ttf,BoldItalicFont=cmuntx.ttf]{cmuntt.ttf}\ttfamily \_}{$\text{ }$}\setmainfont[Path=/usr/share/fonts/truetype/cmu/,UprightFont=cmunrm.ttf,BoldFont=cmunbx.ttf,ItalicFont=cmunti.ttf,BoldItalicFont=cmunbi.ttf]{cmunrm.ttf}\setmonofont[Path=/usr/share/fonts/truetype/cmu/,UprightFont=cmuntt.ttf,BoldFont=cmuntb.ttf,ItalicFont=cmunit.ttf,BoldItalicFont=cmuntx.ttf]{cmunrm.ttf} as in {\ttfamily \setmainfont[Path=/usr/share/fonts/truetype/cmu/,UprightFont=cmunrm.ttf,BoldFont=cmunbx.ttf,ItalicFont=cmunti.ttf,BoldItalicFont=cmunbi.ttf]{cmuntt.ttf}\setmonofont[Path=/usr/share/fonts/truetype/cmu/,UprightFont=cmuntt.ttf,BoldFont=cmuntb.ttf,ItalicFont=cmunit.ttf,BoldItalicFont=cmuntx.ttf]{cmuntt.ttf}\ttfamily \textbackslash{}ce\{H2\_\{(aq)\}}\setmainfont[Path=/usr/share/fonts/truetype/cmu/,UprightFont=cmunrm.ttf,BoldFont=cmunbx.ttf,ItalicFont=cmunti.ttf,BoldItalicFont=cmunbi.ttf]{cmunrm.ttf}\setmonofont[Path=/usr/share/fonts/truetype/cmu/,UprightFont=cmuntt.ttf,BoldFont=cmuntb.ttf,ItalicFont=cmunit.ttf,BoldItalicFont=cmuntx.ttf]{cmunrm.ttf}.

Since February 2016, the mhchem package is also available in TeX in MediaWiki sites like Wikipedia, using the tag {\ttfamily \setmainfont[Path=/usr/share/fonts/truetype/cmu/,UprightFont=cmunrm.ttf,BoldFont=cmunbx.ttf,ItalicFont=cmunti.ttf,BoldItalicFont=cmunbi.ttf]{cmuntt.ttf}\setmonofont[Path=/usr/share/fonts/truetype/cmu/,UprightFont=cmuntt.ttf,BoldFont=cmuntb.ttf,ItalicFont=cmunit.ttf,BoldItalicFont=cmuntx.ttf]{cmuntt.ttf}\ttfamily <{}ce>{}...<{}/ce>{}}\setmainfont[Path=/usr/share/fonts/truetype/cmu/,UprightFont=cmunrm.ttf,BoldFont=cmunbx.ttf,ItalicFont=cmunti.ttf,BoldItalicFont=cmunbi.ttf]{cmunrm.ttf}\setmonofont[Path=/usr/share/fonts/truetype/cmu/,UprightFont=cmuntt.ttf,BoldFont=cmuntb.ttf,ItalicFont=cmunit.ttf,BoldItalicFont=cmuntx.ttf]{cmunrm.ttf}.
\section{XyMTeX package}
\label{579}

The following code produces the image for \myhref{https://en.wikipedia.org/wiki/Corticosterone}{corticosterone} below. 


\begin{Shaded}
\begin{Highlighting}[]

\NormalTok{\textbackslash{}documentclass\{letter\}}\newline
\NormalTok{\textbackslash{}usepackage\{epic,carom\}}\newline
\NormalTok{\textbackslash{}pagestyle\{empty\}}\newline
\NormalTok{\textbackslash{}begin\{document\}}\newline
\NormalTok{\textbackslash{}begin\{picture\}(1000,500)}\newline
\ensuremath{\text{ }}\ensuremath{\text{ }}\ensuremath{\text{ }}\NormalTok{\textbackslash{}put(0,0)\{\textbackslash{}ste}\newline
\NormalTok{roid[d]\{3D==O;\{\{10\}\}==\textbackslash{}lmoiety\{H\$_\{3\}\$C\};\{\{13\}\}==\textbackslash{}lmoiety\{H\$_\{3\}\$C\};\{\{11\}\}==HO\}\}}\newline
\ensuremath{\text{ }}\ensuremath{\text{ }}\ensuremath{\text{ }}\NormalTok{\textbackslash{}put(684,606)\{\textbackslash{}sixunitv\{\}\{2D==O;1==OH\}\{cdef\}\}}\newline
\NormalTok{\textbackslash{}end\{picture\}}\newline
\NormalTok{\textbackslash{}end\{document\}}\newline
\end{Highlighting}
\end{Shaded}




\begin{minipage}{1.0\linewidth}
\begin{center}
\includegraphics[width=1.0\linewidth,height=6.5in,keepaspectratio]{../images/119.png}
\end{center}
\raggedright{}\myfigurewithcaption{119}{Corticosterone as rendered by XyMTeX}
\end{minipage}\vspace{0.75cm}



\myhref{https://sr.wikibooks.org/wiki/LaTeX\%2F\%D0\%A5\%D0\%B5\%D0\%BC\%D0\%B8\%D1\%98\%D1\%81\%D0\%BA\%D0\%B0\%20\%D0\%B3\%D1\%80\%D0\%B0\%D1\%84\%D0\%B8\%D0\%BA\%D0\%B0}{sr:LaTeX/Хемијска графика}\chapter{Algorithms}

\myminitoc
\label{580}

\label{581}


LaTeX has several packages for typesetting algorithms in form of \symbol{34}\myhref{https://en.wikipedia.org/wiki/pseudocode}{pseudocode}\symbol{34}.  They provide stylistic enhancements over a uniform style (i.e., all in typewriter font) so that constructs such as loops or conditionals are visually separated from other text. The pseudocode is usually put in an {\itshape \setmainfont[Path=/usr/share/fonts/truetype/cmu/,UprightFont=cmunrm.ttf,BoldFont=cmunbx.ttf,ItalicFont=cmunti.ttf,BoldItalicFont=cmunbi.ttf]{cmunti.ttf}\setmonofont[Path=/usr/share/fonts/truetype/cmu/,UprightFont=cmuntt.ttf,BoldFont=cmuntb.ttf,ItalicFont=cmunit.ttf,BoldItalicFont=cmuntx.ttf]{cmunti.ttf}\itshape algorithm}{$\text{ }$}\setmainfont[Path=/usr/share/fonts/truetype/cmu/,UprightFont=cmunrm.ttf,BoldFont=cmunbx.ttf,ItalicFont=cmunti.ttf,BoldItalicFont=cmunbi.ttf]{cmunrm.ttf}\setmonofont[Path=/usr/share/fonts/truetype/cmu/,UprightFont=cmuntt.ttf,BoldFont=cmuntb.ttf,ItalicFont=cmunit.ttf,BoldItalicFont=cmuntx.ttf]{cmunrm.ttf} environment.
For typesetting {\itshape \setmainfont[Path=/usr/share/fonts/truetype/cmu/,UprightFont=cmunrm.ttf,BoldFont=cmunbx.ttf,ItalicFont=cmunti.ttf,BoldItalicFont=cmunbi.ttf]{cmunti.ttf}\setmonofont[Path=/usr/share/fonts/truetype/cmu/,UprightFont=cmuntt.ttf,BoldFont=cmuntb.ttf,ItalicFont=cmunit.ttf,BoldItalicFont=cmuntx.ttf]{cmunti.ttf}\itshape real}{$\text{ }$}\setmainfont[Path=/usr/share/fonts/truetype/cmu/,UprightFont=cmunrm.ttf,BoldFont=cmunbx.ttf,ItalicFont=cmunti.ttf,BoldItalicFont=cmunbi.ttf]{cmunrm.ttf}\setmonofont[Path=/usr/share/fonts/truetype/cmu/,UprightFont=cmuntt.ttf,BoldFont=cmuntb.ttf,ItalicFont=cmunit.ttf,BoldItalicFont=cmuntx.ttf]{cmunrm.ttf} code, written in a {\itshape \setmainfont[Path=/usr/share/fonts/truetype/cmu/,UprightFont=cmunrm.ttf,BoldFont=cmunbx.ttf,ItalicFont=cmunti.ttf,BoldItalicFont=cmunbi.ttf]{cmunti.ttf}\setmonofont[Path=/usr/share/fonts/truetype/cmu/,UprightFont=cmuntt.ttf,BoldFont=cmuntb.ttf,ItalicFont=cmunit.ttf,BoldItalicFont=cmuntx.ttf]{cmunti.ttf}\itshape real}{$\text{ }$}\setmainfont[Path=/usr/share/fonts/truetype/cmu/,UprightFont=cmunrm.ttf,BoldFont=cmunbx.ttf,ItalicFont=cmunti.ttf,BoldItalicFont=cmunbi.ttf]{cmunrm.ttf}\setmonofont[Path=/usr/share/fonts/truetype/cmu/,UprightFont=cmuntt.ttf,BoldFont=cmuntb.ttf,ItalicFont=cmunit.ttf,BoldItalicFont=cmuntx.ttf]{cmunrm.ttf} programming language, consider the {\itshape \setmainfont[Path=/usr/share/fonts/truetype/cmu/,UprightFont=cmunrm.ttf,BoldFont=cmunbx.ttf,ItalicFont=cmunti.ttf,BoldItalicFont=cmunbi.ttf]{cmunti.ttf}\setmonofont[Path=/usr/share/fonts/truetype/cmu/,UprightFont=cmuntt.ttf,BoldFont=cmuntb.ttf,ItalicFont=cmunit.ttf,BoldItalicFont=cmuntx.ttf]{cmunti.ttf}\itshape listings}{$\text{ }$}\setmainfont[Path=/usr/share/fonts/truetype/cmu/,UprightFont=cmunrm.ttf,BoldFont=cmunbx.ttf,ItalicFont=cmunti.ttf,BoldItalicFont=cmunbi.ttf]{cmunrm.ttf}\setmonofont[Path=/usr/share/fonts/truetype/cmu/,UprightFont=cmuntt.ttf,BoldFont=cmuntb.ttf,ItalicFont=cmunit.ttf,BoldItalicFont=cmuntx.ttf]{cmunrm.ttf} package described in \mylref{593}{Source Code Listings}.
\section{Typesetting}
\label{582}
There are four notable packages {\itshape \setmainfont[Path=/usr/share/fonts/truetype/cmu/,UprightFont=cmunrm.ttf,BoldFont=cmunbx.ttf,ItalicFont=cmunti.ttf,BoldItalicFont=cmunbi.ttf]{cmunti.ttf}\setmonofont[Path=/usr/share/fonts/truetype/cmu/,UprightFont=cmuntt.ttf,BoldFont=cmuntb.ttf,ItalicFont=cmunit.ttf,BoldItalicFont=cmuntx.ttf]{cmunti.ttf}\itshape algorithmic}\setmainfont[Path=/usr/share/fonts/truetype/cmu/,UprightFont=cmunrm.ttf,BoldFont=cmunbx.ttf,ItalicFont=cmunti.ttf,BoldItalicFont=cmunbi.ttf]{cmunrm.ttf}\setmonofont[Path=/usr/share/fonts/truetype/cmu/,UprightFont=cmuntt.ttf,BoldFont=cmuntb.ttf,ItalicFont=cmunit.ttf,BoldItalicFont=cmuntx.ttf]{cmunrm.ttf}, {\itshape \setmainfont[Path=/usr/share/fonts/truetype/cmu/,UprightFont=cmunrm.ttf,BoldFont=cmunbx.ttf,ItalicFont=cmunti.ttf,BoldItalicFont=cmunbi.ttf]{cmunti.ttf}\setmonofont[Path=/usr/share/fonts/truetype/cmu/,UprightFont=cmuntt.ttf,BoldFont=cmuntb.ttf,ItalicFont=cmunit.ttf,BoldItalicFont=cmuntx.ttf]{cmunti.ttf}\itshape algorithm2e}\setmainfont[Path=/usr/share/fonts/truetype/cmu/,UprightFont=cmunrm.ttf,BoldFont=cmunbx.ttf,ItalicFont=cmunti.ttf,BoldItalicFont=cmunbi.ttf]{cmunrm.ttf}\setmonofont[Path=/usr/share/fonts/truetype/cmu/,UprightFont=cmuntt.ttf,BoldFont=cmuntb.ttf,ItalicFont=cmunit.ttf,BoldItalicFont=cmuntx.ttf]{cmunrm.ttf}, {\itshape \setmainfont[Path=/usr/share/fonts/truetype/cmu/,UprightFont=cmunrm.ttf,BoldFont=cmunbx.ttf,ItalicFont=cmunti.ttf,BoldItalicFont=cmunbi.ttf]{cmunti.ttf}\setmonofont[Path=/usr/share/fonts/truetype/cmu/,UprightFont=cmuntt.ttf,BoldFont=cmuntb.ttf,ItalicFont=cmunit.ttf,BoldItalicFont=cmuntx.ttf]{cmunti.ttf}\itshape algorithmicx}\setmainfont[Path=/usr/share/fonts/truetype/cmu/,UprightFont=cmunrm.ttf,BoldFont=cmunbx.ttf,ItalicFont=cmunti.ttf,BoldItalicFont=cmunbi.ttf]{cmunrm.ttf}\setmonofont[Path=/usr/share/fonts/truetype/cmu/,UprightFont=cmuntt.ttf,BoldFont=cmuntb.ttf,ItalicFont=cmunit.ttf,BoldItalicFont=cmuntx.ttf]{cmunrm.ttf}, and {\itshape \setmainfont[Path=/usr/share/fonts/truetype/cmu/,UprightFont=cmunrm.ttf,BoldFont=cmunbx.ttf,ItalicFont=cmunti.ttf,BoldItalicFont=cmunbi.ttf]{cmunti.ttf}\setmonofont[Path=/usr/share/fonts/truetype/cmu/,UprightFont=cmuntt.ttf,BoldFont=cmuntb.ttf,ItalicFont=cmunit.ttf,BoldItalicFont=cmuntx.ttf]{cmunti.ttf}\itshape program}\setmainfont[Path=/usr/share/fonts/truetype/cmu/,UprightFont=cmunrm.ttf,BoldFont=cmunbx.ttf,ItalicFont=cmunti.ttf,BoldItalicFont=cmunbi.ttf]{cmunrm.ttf}\setmonofont[Path=/usr/share/fonts/truetype/cmu/,UprightFont=cmuntt.ttf,BoldFont=cmuntb.ttf,ItalicFont=cmunit.ttf,BoldItalicFont=cmuntx.ttf]{cmunrm.ttf},\subsection{Typesetting using the {\ttfamily \setmainfont[Path=/usr/share/fonts/truetype/cmu/,UprightFont=cmunrm.ttf,BoldFont=cmunbx.ttf,ItalicFont=cmunti.ttf,BoldItalicFont=cmunbi.ttf]{cmuntt.ttf}\setmonofont[Path=/usr/share/fonts/truetype/cmu/,UprightFont=cmuntt.ttf,BoldFont=cmuntb.ttf,ItalicFont=cmunit.ttf,BoldItalicFont=cmuntx.ttf]{cmuntt.ttf}\ttfamily algorithmic}{$\text{ }$}\setmainfont[Path=/usr/share/fonts/truetype/cmu/,UprightFont=cmunrm.ttf,BoldFont=cmunbx.ttf,ItalicFont=cmunti.ttf,BoldItalicFont=cmunbi.ttf]{cmunrm.ttf}\setmonofont[Path=/usr/share/fonts/truetype/cmu/,UprightFont=cmuntt.ttf,BoldFont=cmuntb.ttf,ItalicFont=cmunit.ttf,BoldItalicFont=cmuntx.ttf]{cmunrm.ttf} package}
\label{583}
The {\ttfamily \setmainfont[Path=/usr/share/fonts/truetype/cmu/,UprightFont=cmunrm.ttf,BoldFont=cmunbx.ttf,ItalicFont=cmunti.ttf,BoldItalicFont=cmunbi.ttf]{cmuntt.ttf}\setmonofont[Path=/usr/share/fonts/truetype/cmu/,UprightFont=cmuntt.ttf,BoldFont=cmuntb.ttf,ItalicFont=cmunit.ttf,BoldItalicFont=cmuntx.ttf]{cmuntt.ttf}\ttfamily algorithmic}{$\text{ }$}\setmainfont[Path=/usr/share/fonts/truetype/cmu/,UprightFont=cmunrm.ttf,BoldFont=cmunbx.ttf,ItalicFont=cmunti.ttf,BoldItalicFont=cmunbi.ttf]{cmunrm.ttf}\setmonofont[Path=/usr/share/fonts/truetype/cmu/,UprightFont=cmuntt.ttf,BoldFont=cmuntb.ttf,ItalicFont=cmunit.ttf,BoldItalicFont=cmuntx.ttf]{cmunrm.ttf} package uses a different set of commands than the {\ttfamily \setmainfont[Path=/usr/share/fonts/truetype/cmu/,UprightFont=cmunrm.ttf,BoldFont=cmunbx.ttf,ItalicFont=cmunti.ttf,BoldItalicFont=cmunbi.ttf]{cmuntt.ttf}\setmonofont[Path=/usr/share/fonts/truetype/cmu/,UprightFont=cmuntt.ttf,BoldFont=cmuntb.ttf,ItalicFont=cmunit.ttf,BoldItalicFont=cmuntx.ttf]{cmuntt.ttf}\ttfamily algorithmicx}{$\text{ }$}\setmainfont[Path=/usr/share/fonts/truetype/cmu/,UprightFont=cmunrm.ttf,BoldFont=cmunbx.ttf,ItalicFont=cmunti.ttf,BoldItalicFont=cmunbi.ttf]{cmunrm.ttf}\setmonofont[Path=/usr/share/fonts/truetype/cmu/,UprightFont=cmuntt.ttf,BoldFont=cmuntb.ttf,ItalicFont=cmunit.ttf,BoldItalicFont=cmuntx.ttf]{cmunrm.ttf} package. This is not compatible with {\ttfamily \setmainfont[Path=/usr/share/fonts/truetype/cmu/,UprightFont=cmunrm.ttf,BoldFont=cmunbx.ttf,ItalicFont=cmunti.ttf,BoldItalicFont=cmunbi.ttf]{cmuntt.ttf}\setmonofont[Path=/usr/share/fonts/truetype/cmu/,UprightFont=cmuntt.ttf,BoldFont=cmuntb.ttf,ItalicFont=cmunit.ttf,BoldItalicFont=cmuntx.ttf]{cmuntt.ttf}\ttfamily revtex4-{}1}\setmainfont[Path=/usr/share/fonts/truetype/cmu/,UprightFont=cmunrm.ttf,BoldFont=cmunbx.ttf,ItalicFont=cmunti.ttf,BoldItalicFont=cmunbi.ttf]{cmunrm.ttf}\setmonofont[Path=/usr/share/fonts/truetype/cmu/,UprightFont=cmuntt.ttf,BoldFont=cmuntb.ttf,ItalicFont=cmunit.ttf,BoldItalicFont=cmuntx.ttf]{cmunrm.ttf}.
Basic commands are:

\begin{Shaded}
\begin{Highlighting}[]

\ensuremath{\text{ }}\NormalTok{\textbackslash{}STATE\ensuremath{\text{ }}<text>}\newline
\ensuremath{\text{ }}\NormalTok{\textbackslash{}IF\{<condition>\}\ensuremath{\text{ }}\textbackslash{}STATE\ensuremath{\text{ }}\{<text>\}\ensuremath{\text{ }}\textbackslash{}ELSE\ensuremath{\text{ }}\textbackslash{}STATE\{<text>\}\ensuremath{\text{ }}\textbackslash{}ENDIF}\newline
\ensuremath{\text{ }}\NormalTok{\textbackslash{}IF\{<condition>\}\ensuremath{\text{ }}\textbackslash{}STATE\ensuremath{\text{ }}\{<text>\}\ensuremath{\text{ }}\textbackslash{}ELSIF\{<condition>\}\ensuremath{\text{ }}\textbackslash{}STATE\{<text>\}\ensuremath{\text{ }}\textbackslash{}ENDIF}\newline
\ensuremath{\text{ }}\NormalTok{\textbackslash{}FOR\{<condition>\}\ensuremath{\text{ }}\textbackslash{}STATE\ensuremath{\text{ }}\{<text>\}\ensuremath{\text{ }}\textbackslash{}ENDFOR}\newline
\ensuremath{\text{ }}\NormalTok{\textbackslash{}FOR\{<condition>\ensuremath{\text{ }}\textbackslash{}TO\ensuremath{\text{ }}<condition>\ensuremath{\text{ }}\}\ensuremath{\text{ }}\textbackslash{}STATE\ensuremath{\text{ }}\{<text>\}\ensuremath{\text{ }}\textbackslash{}ENDFOR}\newline
\ensuremath{\text{ }}\NormalTok{\textbackslash{}FORALL\{<condition>\}\ensuremath{\text{ }}\textbackslash{}STATE\{<text>\}\ensuremath{\text{ }}\textbackslash{}ENDFOR}\newline
\ensuremath{\text{ }}\NormalTok{\textbackslash{}WHILE\{<condition>\}\ensuremath{\text{ }}\textbackslash{}STATE\{<text>\}\ensuremath{\text{ }}\textbackslash{}ENDWHILE}\newline
\ensuremath{\text{ }}\NormalTok{\textbackslash{}REPEAT\ensuremath{\text{ }}\textbackslash{}STATE\{<text>\}\ensuremath{\text{ }}\textbackslash{}UNTIL\{<condition>\}}\newline
\ensuremath{\text{ }}\NormalTok{\textbackslash{}LOOP\ensuremath{\text{ }}\textbackslash{}STATE\{<text>\}\ensuremath{\text{ }}\textbackslash{}ENDLOOP}\newline
\ensuremath{\text{ }}\NormalTok{\textbackslash{}REQUIRE\ensuremath{\text{ }}<text>}\newline
\ensuremath{\text{ }}\NormalTok{\textbackslash{}ENSURE\ensuremath{\text{ }}<text>}\newline
\ensuremath{\text{ }}\NormalTok{\textbackslash{}RETURN\ensuremath{\text{ }}<text>}\newline
\ensuremath{\text{ }}\NormalTok{\textbackslash{}PRINT\ensuremath{\text{ }}<text>}\newline
\ensuremath{\text{ }}\NormalTok{\textbackslash{}COMMENT\{<text>\}}\newline
\ensuremath{\text{ }}\NormalTok{\textbackslash{}AND,\ensuremath{\text{ }}\textbackslash{}OR,\ensuremath{\text{ }}\textbackslash{}XOR,\ensuremath{\text{ }}\textbackslash{}NOT,\ensuremath{\text{ }}\textbackslash{}TO,\ensuremath{\text{ }}\textbackslash{}TRUE,\ensuremath{\text{ }}\textbackslash{}FALSE}\newline
\end{Highlighting}
\end{Shaded}

Complete documentation is listed at \myplainurl{http://mirror.ctan.org/tex-archive/macros/latex/contrib/algorithms/algorithms.pdf}. Most commands are similar to the {\ttfamily \setmainfont[Path=/usr/share/fonts/truetype/cmu/,UprightFont=cmunrm.ttf,BoldFont=cmunbx.ttf,ItalicFont=cmunti.ttf,BoldItalicFont=cmunbi.ttf]{cmuntt.ttf}\setmonofont[Path=/usr/share/fonts/truetype/cmu/,UprightFont=cmuntt.ttf,BoldFont=cmuntb.ttf,ItalicFont=cmunit.ttf,BoldItalicFont=cmuntx.ttf]{cmuntt.ttf}\ttfamily algorithmicx}{$\text{ }$}\setmainfont[Path=/usr/share/fonts/truetype/cmu/,UprightFont=cmunrm.ttf,BoldFont=cmunbx.ttf,ItalicFont=cmunti.ttf,BoldItalicFont=cmunbi.ttf]{cmunrm.ttf}\setmonofont[Path=/usr/share/fonts/truetype/cmu/,UprightFont=cmuntt.ttf,BoldFont=cmuntb.ttf,ItalicFont=cmunit.ttf,BoldItalicFont=cmuntx.ttf]{cmunrm.ttf} equivalents, but with different capitalization.
The package {\ttfamily \setmainfont[Path=/usr/share/fonts/truetype/cmu/,UprightFont=cmunrm.ttf,BoldFont=cmunbx.ttf,ItalicFont=cmunti.ttf,BoldItalicFont=cmunbi.ttf]{cmuntt.ttf}\setmonofont[Path=/usr/share/fonts/truetype/cmu/,UprightFont=cmuntt.ttf,BoldFont=cmuntb.ttf,ItalicFont=cmunit.ttf,BoldItalicFont=cmuntx.ttf]{cmuntt.ttf}\ttfamily algorithms bundle}{$\text{ }$}\setmainfont[Path=/usr/share/fonts/truetype/cmu/,UprightFont=cmunrm.ttf,BoldFont=cmunbx.ttf,ItalicFont=cmunti.ttf,BoldItalicFont=cmunbi.ttf]{cmunrm.ttf}\setmonofont[Path=/usr/share/fonts/truetype/cmu/,UprightFont=cmuntt.ttf,BoldFont=cmuntb.ttf,ItalicFont=cmunit.ttf,BoldItalicFont=cmuntx.ttf]{cmunrm.ttf} at the \myhref{http://mirror.ctan.org/tex-archive/macros/latex/contrib/algorithms/}{ctan repository}, dated 2009-{}08-{}24, describes both the {\ttfamily \setmainfont[Path=/usr/share/fonts/truetype/cmu/,UprightFont=cmunrm.ttf,BoldFont=cmunbx.ttf,ItalicFont=cmunti.ttf,BoldItalicFont=cmunbi.ttf]{cmuntt.ttf}\setmonofont[Path=/usr/share/fonts/truetype/cmu/,UprightFont=cmuntt.ttf,BoldFont=cmuntb.ttf,ItalicFont=cmunit.ttf,BoldItalicFont=cmuntx.ttf]{cmuntt.ttf}\ttfamily algorithmic}{$\text{ }$}\setmainfont[Path=/usr/share/fonts/truetype/cmu/,UprightFont=cmunrm.ttf,BoldFont=cmunbx.ttf,ItalicFont=cmunti.ttf,BoldItalicFont=cmunbi.ttf]{cmunrm.ttf}\setmonofont[Path=/usr/share/fonts/truetype/cmu/,UprightFont=cmuntt.ttf,BoldFont=cmuntb.ttf,ItalicFont=cmunit.ttf,BoldItalicFont=cmuntx.ttf]{cmunrm.ttf} environment (for typesetting algorithms) and the {\ttfamily \setmainfont[Path=/usr/share/fonts/truetype/cmu/,UprightFont=cmunrm.ttf,BoldFont=cmunbx.ttf,ItalicFont=cmunti.ttf,BoldItalicFont=cmunbi.ttf]{cmuntt.ttf}\setmonofont[Path=/usr/share/fonts/truetype/cmu/,UprightFont=cmuntt.ttf,BoldFont=cmuntb.ttf,ItalicFont=cmunit.ttf,BoldItalicFont=cmuntx.ttf]{cmuntt.ttf}\ttfamily algorithm}{$\text{ }$}\setmainfont[Path=/usr/share/fonts/truetype/cmu/,UprightFont=cmunrm.ttf,BoldFont=cmunbx.ttf,ItalicFont=cmunti.ttf,BoldItalicFont=cmunbi.ttf]{cmunrm.ttf}\setmonofont[Path=/usr/share/fonts/truetype/cmu/,UprightFont=cmuntt.ttf,BoldFont=cmuntb.ttf,ItalicFont=cmunit.ttf,BoldItalicFont=cmuntx.ttf]{cmunrm.ttf} floating wrapper (see \mylref{586}{below}) which is designed to wrap around the algorithmic environment.

The {\ttfamily \setmainfont[Path=/usr/share/fonts/truetype/cmu/,UprightFont=cmunrm.ttf,BoldFont=cmunbx.ttf,ItalicFont=cmunti.ttf,BoldItalicFont=cmunbi.ttf]{cmuntt.ttf}\setmonofont[Path=/usr/share/fonts/truetype/cmu/,UprightFont=cmuntt.ttf,BoldFont=cmuntb.ttf,ItalicFont=cmunit.ttf,BoldItalicFont=cmuntx.ttf]{cmuntt.ttf}\ttfamily algorithmic}{$\text{ }$}\setmainfont[Path=/usr/share/fonts/truetype/cmu/,UprightFont=cmunrm.ttf,BoldFont=cmunbx.ttf,ItalicFont=cmunti.ttf,BoldItalicFont=cmunbi.ttf]{cmunrm.ttf}\setmonofont[Path=/usr/share/fonts/truetype/cmu/,UprightFont=cmuntt.ttf,BoldFont=cmuntb.ttf,ItalicFont=cmunit.ttf,BoldItalicFont=cmuntx.ttf]{cmunrm.ttf} package is suggested for \myhref{http://ieeexplore.ieee.org/xpl/periodicals.jsp}{IEEE journals} as it is a part of their default style sheet.\myfootnote{\myplainurl{http://www.ctan.org/tex-archive/macros/latex/contrib/IEEEtran}}

How to rename require/ensure to input/output:


\begin{Shaded}
\begin{Highlighting}[]

\NormalTok{\textbackslash{}floatname\{algorithm\}\{Procedure\}}\newline
\NormalTok{\textbackslash{}renewcommand\{\textbackslash{}algorithmicrequire\}\{\textbackslash{}textbf\{Input:\}\}}\newline
\NormalTok{\textbackslash{}renewcommand\{\textbackslash{}algorithmicensure\}\{\textbackslash{}textbf\{Output:\}\}}\newline
\end{Highlighting}
\end{Shaded}

\subsection{Typesetting using the {\ttfamily \setmainfont[Path=/usr/share/fonts/truetype/cmu/,UprightFont=cmunrm.ttf,BoldFont=cmunbx.ttf,ItalicFont=cmunti.ttf,BoldItalicFont=cmunbi.ttf]{cmuntt.ttf}\setmonofont[Path=/usr/share/fonts/truetype/cmu/,UprightFont=cmuntt.ttf,BoldFont=cmuntb.ttf,ItalicFont=cmunit.ttf,BoldItalicFont=cmuntx.ttf]{cmuntt.ttf}\ttfamily algorithm2e}{$\text{ }$}\setmainfont[Path=/usr/share/fonts/truetype/cmu/,UprightFont=cmunrm.ttf,BoldFont=cmunbx.ttf,ItalicFont=cmunti.ttf,BoldItalicFont=cmunbi.ttf]{cmunrm.ttf}\setmonofont[Path=/usr/share/fonts/truetype/cmu/,UprightFont=cmuntt.ttf,BoldFont=cmuntb.ttf,ItalicFont=cmunit.ttf,BoldItalicFont=cmuntx.ttf]{cmunrm.ttf} package}
\label{584}
The {\ttfamily \setmainfont[Path=/usr/share/fonts/truetype/cmu/,UprightFont=cmunrm.ttf,BoldFont=cmunbx.ttf,ItalicFont=cmunti.ttf,BoldItalicFont=cmunbi.ttf]{cmuntt.ttf}\setmonofont[Path=/usr/share/fonts/truetype/cmu/,UprightFont=cmuntt.ttf,BoldFont=cmuntb.ttf,ItalicFont=cmunit.ttf,BoldItalicFont=cmuntx.ttf]{cmuntt.ttf}\ttfamily algorithm2e}{$\text{ }$}\setmainfont[Path=/usr/share/fonts/truetype/cmu/,UprightFont=cmunrm.ttf,BoldFont=cmunbx.ttf,ItalicFont=cmunti.ttf,BoldItalicFont=cmunbi.ttf]{cmunrm.ttf}\setmonofont[Path=/usr/share/fonts/truetype/cmu/,UprightFont=cmuntt.ttf,BoldFont=cmuntb.ttf,ItalicFont=cmunit.ttf,BoldItalicFont=cmuntx.ttf]{cmunrm.ttf} package (first released 1995, latest updated January 2013 according to the \myhref{http://mirror.ctan.org/tex-archive/macros/latex/contrib/algorithm2e/doc/algorithm2e.pdf}{v5.0 manual}) allows typesetting algorithms with a lot of customization. Like {\ttfamily \setmainfont[Path=/usr/share/fonts/truetype/cmu/,UprightFont=cmunrm.ttf,BoldFont=cmunbx.ttf,ItalicFont=cmunti.ttf,BoldItalicFont=cmunbi.ttf]{cmuntt.ttf}\setmonofont[Path=/usr/share/fonts/truetype/cmu/,UprightFont=cmuntt.ttf,BoldFont=cmuntb.ttf,ItalicFont=cmunit.ttf,BoldItalicFont=cmuntx.ttf]{cmuntt.ttf}\ttfamily algorithmic}\setmainfont[Path=/usr/share/fonts/truetype/cmu/,UprightFont=cmunrm.ttf,BoldFont=cmunbx.ttf,ItalicFont=cmunti.ttf,BoldItalicFont=cmunbi.ttf]{cmunrm.ttf}\setmonofont[Path=/usr/share/fonts/truetype/cmu/,UprightFont=cmuntt.ttf,BoldFont=cmuntb.ttf,ItalicFont=cmunit.ttf,BoldItalicFont=cmuntx.ttf]{cmunrm.ttf}, this package is also not compatible with Revtex-{}4.1.\myfootnote{\myplainurl{http://tex.stackexchange.com/questions/70181/revtex4-1-and-algorithm2e-indentation-clash}}

Unlike {\ttfamily \setmainfont[Path=/usr/share/fonts/truetype/cmu/,UprightFont=cmunrm.ttf,BoldFont=cmunbx.ttf,ItalicFont=cmunti.ttf,BoldItalicFont=cmunbi.ttf]{cmuntt.ttf}\setmonofont[Path=/usr/share/fonts/truetype/cmu/,UprightFont=cmuntt.ttf,BoldFont=cmuntb.ttf,ItalicFont=cmunit.ttf,BoldItalicFont=cmuntx.ttf]{cmuntt.ttf}\ttfamily algorithmic}\setmainfont[Path=/usr/share/fonts/truetype/cmu/,UprightFont=cmunrm.ttf,BoldFont=cmunbx.ttf,ItalicFont=cmunti.ttf,BoldItalicFont=cmunbi.ttf]{cmunrm.ttf}\setmonofont[Path=/usr/share/fonts/truetype/cmu/,UprightFont=cmuntt.ttf,BoldFont=cmuntb.ttf,ItalicFont=cmunit.ttf,BoldItalicFont=cmuntx.ttf]{cmunrm.ttf}, {\ttfamily \setmainfont[Path=/usr/share/fonts/truetype/cmu/,UprightFont=cmunrm.ttf,BoldFont=cmunbx.ttf,ItalicFont=cmunti.ttf,BoldItalicFont=cmunbi.ttf]{cmuntt.ttf}\setmonofont[Path=/usr/share/fonts/truetype/cmu/,UprightFont=cmuntt.ttf,BoldFont=cmuntb.ttf,ItalicFont=cmunit.ttf,BoldItalicFont=cmuntx.ttf]{cmuntt.ttf}\ttfamily algorithm2e}{$\text{ }$}\setmainfont[Path=/usr/share/fonts/truetype/cmu/,UprightFont=cmunrm.ttf,BoldFont=cmunbx.ttf,ItalicFont=cmunti.ttf,BoldItalicFont=cmunbi.ttf]{cmunrm.ttf}\setmonofont[Path=/usr/share/fonts/truetype/cmu/,UprightFont=cmuntt.ttf,BoldFont=cmuntb.ttf,ItalicFont=cmunit.ttf,BoldItalicFont=cmuntx.ttf]{cmunrm.ttf} provides a relatively huge number of customization options to the algorithm suiting to the needs of various users.
The \myhref{http://mirror.ctan.org/tex-archive/macros/latex/contrib/algorithm2e/doc/algorithm2e.pdf}{CTAN-{}manual} provides a comprehensible list of examples and full set of controls.

Typically, the usage between {\ttfamily \setmainfont[Path=/usr/share/fonts/truetype/cmu/,UprightFont=cmunrm.ttf,BoldFont=cmunbx.ttf,ItalicFont=cmunti.ttf,BoldItalicFont=cmunbi.ttf]{cmuntt.ttf}\setmonofont[Path=/usr/share/fonts/truetype/cmu/,UprightFont=cmuntt.ttf,BoldFont=cmuntb.ttf,ItalicFont=cmunit.ttf,BoldItalicFont=cmuntx.ttf]{cmuntt.ttf}\ttfamily \textbackslash{}begin\{algorithm\}}{$\text{ }$}\setmainfont[Path=/usr/share/fonts/truetype/cmu/,UprightFont=cmunrm.ttf,BoldFont=cmunbx.ttf,ItalicFont=cmunti.ttf,BoldItalicFont=cmunbi.ttf]{cmunrm.ttf}\setmonofont[Path=/usr/share/fonts/truetype/cmu/,UprightFont=cmuntt.ttf,BoldFont=cmuntb.ttf,ItalicFont=cmunit.ttf,BoldItalicFont=cmuntx.ttf]{cmunrm.ttf} and {\ttfamily \setmainfont[Path=/usr/share/fonts/truetype/cmu/,UprightFont=cmunrm.ttf,BoldFont=cmunbx.ttf,ItalicFont=cmunti.ttf,BoldItalicFont=cmunbi.ttf]{cmuntt.ttf}\setmonofont[Path=/usr/share/fonts/truetype/cmu/,UprightFont=cmuntt.ttf,BoldFont=cmuntb.ttf,ItalicFont=cmunit.ttf,BoldItalicFont=cmuntx.ttf]{cmuntt.ttf}\ttfamily \textbackslash{}end\{algorithm\}}{$\text{ }$}\setmainfont[Path=/usr/share/fonts/truetype/cmu/,UprightFont=cmunrm.ttf,BoldFont=cmunbx.ttf,ItalicFont=cmunti.ttf,BoldItalicFont=cmunbi.ttf]{cmunrm.ttf}\setmonofont[Path=/usr/share/fonts/truetype/cmu/,UprightFont=cmuntt.ttf,BoldFont=cmuntb.ttf,ItalicFont=cmunit.ttf,BoldItalicFont=cmuntx.ttf]{cmunrm.ttf} would be $\text{ }$\newline{}

1. Declaring a set of keywords(to typeset as functions/operators), layout controls, caption, title, header text (which appears before the algorithm\textquotesingle{}s main steps e.g.: Input,Output) $\text{ }$\newline{}

2. Writing the main steps of the algorithm, with each step ending with a \textbackslash{}; $\text{ }$\newline{}

This may be taken in analogy with writing a latex-{}preamble before we start the actual document.

The package is loaded like

\begin{Shaded}
\begin{Highlighting}[]

\NormalTok{\textbackslash{}usepackage[]\{algorithm2e\}}\newline
\end{Highlighting}
\end{Shaded}

and a simple example, taken from the v4.01 manual, is

\begin{Shaded}
\begin{Highlighting}[]

\NormalTok{\textbackslash{}begin\{algorithm\}[H]}\newline
\ensuremath{\text{ }}\NormalTok{\textbackslash{}KwData\{this\ensuremath{\text{ }}text\}}\newline
\ensuremath{\text{ }}\NormalTok{\textbackslash{}KwResult\{how\ensuremath{\text{ }}to\ensuremath{\text{ }}write\ensuremath{\text{ }}algorithm\ensuremath{\text{ }}with\ensuremath{\text{ }}\textbackslash{}LaTeX2e\ensuremath{\text{ }}\}}\newline
\ensuremath{\text{ }}\NormalTok{initialization\textbackslash{};}\newline
\ensuremath{\text{ }}\NormalTok{\textbackslash{}While\{not\ensuremath{\text{ }}at\ensuremath{\text{ }}end\ensuremath{\text{ }}of\ensuremath{\text{ }}this\ensuremath{\text{ }}document\}\{}\newline
\ensuremath{\text{ }}\ensuremath{\text{ }}\NormalTok{read\ensuremath{\text{ }}current\textbackslash{};}\newline
\ensuremath{\text{ }}\ensuremath{\text{ }}\NormalTok{\textbackslash{}eIf\{understand\}\{}\newline
\ensuremath{\text{ }}\ensuremath{\text{ }}\ensuremath{\text{ }}\NormalTok{go\ensuremath{\text{ }}to\ensuremath{\text{ }}next\ensuremath{\text{ }}section\textbackslash{};}\newline
\ensuremath{\text{ }}\ensuremath{\text{ }}\ensuremath{\text{ }}\NormalTok{current\ensuremath{\text{ }}section\ensuremath{\text{ }}becomes\ensuremath{\text{ }}this\ensuremath{\text{ }}one\textbackslash{};}\newline
\ensuremath{\text{ }}\ensuremath{\text{ }}\ensuremath{\text{ }}\NormalTok{\}\{}\newline
\ensuremath{\text{ }}\ensuremath{\text{ }}\ensuremath{\text{ }}\NormalTok{go\ensuremath{\text{ }}back\ensuremath{\text{ }}to\ensuremath{\text{ }}the\ensuremath{\text{ }}beginning\ensuremath{\text{ }}of\ensuremath{\text{ }}current\ensuremath{\text{ }}section\textbackslash{};}\newline
\ensuremath{\text{ }}\ensuremath{\text{ }}\NormalTok{\}}\newline
\ensuremath{\text{ }}\NormalTok{\}}\newline
\ensuremath{\text{ }}\NormalTok{\textbackslash{}caption\{How\ensuremath{\text{ }}to\ensuremath{\text{ }}write\ensuremath{\text{ }}algorithms\}}\newline
\NormalTok{\textbackslash{}end\{algorithm\}}\newline
\end{Highlighting}
\end{Shaded}

which produces



\begin{minipage}{0.75000\textwidth}
\begin{center}
\includegraphics[width=1.0\textwidth,height=6.5in,keepaspectratio]{../images/120.png}
\end{center}
\raggedright{}\myfigurewithoutcaption{120}
\end{minipage}\vspace{0.75cm}



More details are in the manual hosted on the \myhref{http://mirror.ctan.org/tex-archive/macros/latex/contrib/algorithm2e/doc/algorithm2e.pdf}{ctan website}.
\subsection{Typesetting using the {\ttfamily \setmainfont[Path=/usr/share/fonts/truetype/cmu/,UprightFont=cmunrm.ttf,BoldFont=cmunbx.ttf,ItalicFont=cmunti.ttf,BoldItalicFont=cmunbi.ttf]{cmuntt.ttf}\setmonofont[Path=/usr/share/fonts/truetype/cmu/,UprightFont=cmuntt.ttf,BoldFont=cmuntb.ttf,ItalicFont=cmunit.ttf,BoldItalicFont=cmuntx.ttf]{cmuntt.ttf}\ttfamily algorithmicx}{$\text{ }$}\setmainfont[Path=/usr/share/fonts/truetype/cmu/,UprightFont=cmunrm.ttf,BoldFont=cmunbx.ttf,ItalicFont=cmunti.ttf,BoldItalicFont=cmunbi.ttf]{cmunrm.ttf}\setmonofont[Path=/usr/share/fonts/truetype/cmu/,UprightFont=cmuntt.ttf,BoldFont=cmuntb.ttf,ItalicFont=cmunit.ttf,BoldItalicFont=cmuntx.ttf]{cmunrm.ttf} package}
\label{585}

The {\ttfamily \myhref{https://www.ctan.org/tex-archive/macros/latex/contrib/algorithmicx/}{\setmainfont[Path=/usr/share/fonts/truetype/cmu/,UprightFont=cmunrm.ttf,BoldFont=cmunbx.ttf,ItalicFont=cmunti.ttf,BoldItalicFont=cmunbi.ttf]{cmuntt.ttf}\setmonofont[Path=/usr/share/fonts/truetype/cmu/,UprightFont=cmuntt.ttf,BoldFont=cmuntb.ttf,ItalicFont=cmunit.ttf,BoldItalicFont=cmuntx.ttf]{cmuntt.ttf}\ttfamily algorithmicx}} package provides a number of popular constructs for algorithm designs.  Put {\ttfamily \setmainfont[Path=/usr/share/fonts/truetype/cmu/,UprightFont=cmunrm.ttf,BoldFont=cmunbx.ttf,ItalicFont=cmunti.ttf,BoldItalicFont=cmunbi.ttf]{cmuntt.ttf}\setmonofont[Path=/usr/share/fonts/truetype/cmu/,UprightFont=cmuntt.ttf,BoldFont=cmuntb.ttf,ItalicFont=cmunit.ttf,BoldItalicFont=cmuntx.ttf]{cmuntt.ttf}\ttfamily \textbackslash{}usepackage\{algpseudocode\}}{$\text{ }$}\setmainfont[Path=/usr/share/fonts/truetype/cmu/,UprightFont=cmunrm.ttf,BoldFont=cmunbx.ttf,ItalicFont=cmunti.ttf,BoldItalicFont=cmunbi.ttf]{cmunrm.ttf}\setmonofont[Path=/usr/share/fonts/truetype/cmu/,UprightFont=cmuntt.ttf,BoldFont=cmuntb.ttf,ItalicFont=cmunit.ttf,BoldItalicFont=cmuntx.ttf]{cmunrm.ttf} in the preamble to use the algorithmic environment to write algorithm pseudocode ({\ttfamily \setmainfont[Path=/usr/share/fonts/truetype/cmu/,UprightFont=cmunrm.ttf,BoldFont=cmunbx.ttf,ItalicFont=cmunti.ttf,BoldItalicFont=cmunbi.ttf]{cmuntt.ttf}\setmonofont[Path=/usr/share/fonts/truetype/cmu/,UprightFont=cmuntt.ttf,BoldFont=cmuntb.ttf,ItalicFont=cmunit.ttf,BoldItalicFont=cmuntx.ttf]{cmuntt.ttf}\ttfamily \textbackslash{}begin\{algorithmic\}...\textbackslash{}end\{algorithmic\}}\setmainfont[Path=/usr/share/fonts/truetype/cmu/,UprightFont=cmunrm.ttf,BoldFont=cmunbx.ttf,ItalicFont=cmunti.ttf,BoldItalicFont=cmunbi.ttf]{cmunrm.ttf}\setmonofont[Path=/usr/share/fonts/truetype/cmu/,UprightFont=cmuntt.ttf,BoldFont=cmuntb.ttf,ItalicFont=cmunit.ttf,BoldItalicFont=cmuntx.ttf]{cmunrm.ttf}).  You might want to use the algorithm environment ({\ttfamily \setmainfont[Path=/usr/share/fonts/truetype/cmu/,UprightFont=cmunrm.ttf,BoldFont=cmunbx.ttf,ItalicFont=cmunti.ttf,BoldItalicFont=cmunbi.ttf]{cmuntt.ttf}\setmonofont[Path=/usr/share/fonts/truetype/cmu/,UprightFont=cmuntt.ttf,BoldFont=cmuntb.ttf,ItalicFont=cmunit.ttf,BoldItalicFont=cmuntx.ttf]{cmuntt.ttf}\ttfamily \textbackslash{}usepackage\{algorithm\}}\setmainfont[Path=/usr/share/fonts/truetype/cmu/,UprightFont=cmunrm.ttf,BoldFont=cmunbx.ttf,ItalicFont=cmunti.ttf,BoldItalicFont=cmunbi.ttf]{cmunrm.ttf}\setmonofont[Path=/usr/share/fonts/truetype/cmu/,UprightFont=cmuntt.ttf,BoldFont=cmuntb.ttf,ItalicFont=cmunit.ttf,BoldItalicFont=cmuntx.ttf]{cmunrm.ttf}) to wrap your algorithmic code in an algorithm environment ({\ttfamily \setmainfont[Path=/usr/share/fonts/truetype/cmu/,UprightFont=cmunrm.ttf,BoldFont=cmunbx.ttf,ItalicFont=cmunti.ttf,BoldItalicFont=cmunbi.ttf]{cmuntt.ttf}\setmonofont[Path=/usr/share/fonts/truetype/cmu/,UprightFont=cmuntt.ttf,BoldFont=cmuntb.ttf,ItalicFont=cmunit.ttf,BoldItalicFont=cmuntx.ttf]{cmuntt.ttf}\ttfamily \textbackslash{}begin\{algorithm\}...\textbackslash{}end\{algorithm\}}\setmainfont[Path=/usr/share/fonts/truetype/cmu/,UprightFont=cmunrm.ttf,BoldFont=cmunbx.ttf,ItalicFont=cmunti.ttf,BoldItalicFont=cmunbi.ttf]{cmunrm.ttf}\setmonofont[Path=/usr/share/fonts/truetype/cmu/,UprightFont=cmuntt.ttf,BoldFont=cmuntb.ttf,ItalicFont=cmunit.ttf,BoldItalicFont=cmuntx.ttf]{cmunrm.ttf}) to produce a floating environment with numbered algorithms.

The command {\ttfamily \setmainfont[Path=/usr/share/fonts/truetype/cmu/,UprightFont=cmunrm.ttf,BoldFont=cmunbx.ttf,ItalicFont=cmunti.ttf,BoldItalicFont=cmunbi.ttf]{cmuntt.ttf}\setmonofont[Path=/usr/share/fonts/truetype/cmu/,UprightFont=cmuntt.ttf,BoldFont=cmuntb.ttf,ItalicFont=cmunit.ttf,BoldItalicFont=cmuntx.ttf]{cmuntt.ttf}\ttfamily \textbackslash{}begin\{algorithmic\}}{$\text{ }$}\setmainfont[Path=/usr/share/fonts/truetype/cmu/,UprightFont=cmunrm.ttf,BoldFont=cmunbx.ttf,ItalicFont=cmunti.ttf,BoldItalicFont=cmunbi.ttf]{cmunrm.ttf}\setmonofont[Path=/usr/share/fonts/truetype/cmu/,UprightFont=cmuntt.ttf,BoldFont=cmuntb.ttf,ItalicFont=cmunit.ttf,BoldItalicFont=cmuntx.ttf]{cmunrm.ttf} can be given the optional argument of a positive integer, which if given will cause line numbering to occur at multiples of that integer. E.g. {\ttfamily \setmainfont[Path=/usr/share/fonts/truetype/cmu/,UprightFont=cmunrm.ttf,BoldFont=cmunbx.ttf,ItalicFont=cmunti.ttf,BoldItalicFont=cmunbi.ttf]{cmuntt.ttf}\setmonofont[Path=/usr/share/fonts/truetype/cmu/,UprightFont=cmuntt.ttf,BoldFont=cmuntb.ttf,ItalicFont=cmunit.ttf,BoldItalicFont=cmuntx.ttf]{cmuntt.ttf}\ttfamily \textbackslash{}begin\{algorithmic\}{$\text{[}$}5{$\text{]}$}}{$\text{ }$}\setmainfont[Path=/usr/share/fonts/truetype/cmu/,UprightFont=cmunrm.ttf,BoldFont=cmunbx.ttf,ItalicFont=cmunti.ttf,BoldItalicFont=cmunbi.ttf]{cmunrm.ttf}\setmonofont[Path=/usr/share/fonts/truetype/cmu/,UprightFont=cmuntt.ttf,BoldFont=cmuntb.ttf,ItalicFont=cmunit.ttf,BoldItalicFont=cmuntx.ttf]{cmunrm.ttf} will enter the algorithmic environment and number every fifth line.

Below is an example of typesetting a basic algorithm using the {\ttfamily \setmainfont[Path=/usr/share/fonts/truetype/cmu/,UprightFont=cmunrm.ttf,BoldFont=cmunbx.ttf,ItalicFont=cmunti.ttf,BoldItalicFont=cmunbi.ttf]{cmuntt.ttf}\setmonofont[Path=/usr/share/fonts/truetype/cmu/,UprightFont=cmuntt.ttf,BoldFont=cmuntb.ttf,ItalicFont=cmunit.ttf,BoldItalicFont=cmuntx.ttf]{cmuntt.ttf}\ttfamily algorithmicx}{$\text{ }$}\setmainfont[Path=/usr/share/fonts/truetype/cmu/,UprightFont=cmunrm.ttf,BoldFont=cmunbx.ttf,ItalicFont=cmunti.ttf,BoldItalicFont=cmunbi.ttf]{cmunrm.ttf}\setmonofont[Path=/usr/share/fonts/truetype/cmu/,UprightFont=cmuntt.ttf,BoldFont=cmuntb.ttf,ItalicFont=cmunit.ttf,BoldItalicFont=cmuntx.ttf]{cmunrm.ttf} package (remember to add the {\ttfamily \setmainfont[Path=/usr/share/fonts/truetype/cmu/,UprightFont=cmunrm.ttf,BoldFont=cmunbx.ttf,ItalicFont=cmunti.ttf,BoldItalicFont=cmunbi.ttf]{cmuntt.ttf}\setmonofont[Path=/usr/share/fonts/truetype/cmu/,UprightFont=cmuntt.ttf,BoldFont=cmuntb.ttf,ItalicFont=cmunit.ttf,BoldItalicFont=cmuntx.ttf]{cmuntt.ttf}\ttfamily \textbackslash{}usepackage\{algpseudocode\}}{$\text{ }$}\setmainfont[Path=/usr/share/fonts/truetype/cmu/,UprightFont=cmunrm.ttf,BoldFont=cmunbx.ttf,ItalicFont=cmunti.ttf,BoldItalicFont=cmunbi.ttf]{cmunrm.ttf}\setmonofont[Path=/usr/share/fonts/truetype/cmu/,UprightFont=cmuntt.ttf,BoldFont=cmuntb.ttf,ItalicFont=cmunit.ttf,BoldItalicFont=cmuntx.ttf]{cmunrm.ttf} statement to your document preamble):


\begin{Shaded}
\begin{Highlighting}[]

\NormalTok{\textbackslash{}begin\{algorithmic\}}\newline
\NormalTok{\textbackslash{}If\ensuremath{\text{ }}\{\$i\textbackslash{}geq\ensuremath{\text{ }}maxval\$\}}\newline
\ensuremath{\text{ }}\ensuremath{\text{ }}\ensuremath{\text{ }}\ensuremath{\text{ }}\NormalTok{\textbackslash{}State\ensuremath{\text{ }}\$i\textbackslash{}gets\ensuremath{\text{ }}0\$}\newline
\NormalTok{\textbackslash{}Else}\newline
\ensuremath{\text{ }}\ensuremath{\text{ }}\ensuremath{\text{ }}\ensuremath{\text{ }}\NormalTok{\textbackslash{}If\ensuremath{\text{ }}\{\$i+k\textbackslash{}leq\ensuremath{\text{ }}maxval\$\}}\newline
\ensuremath{\text{ }}\ensuremath{\text{ }}\ensuremath{\text{ }}\ensuremath{\text{ }}\ensuremath{\text{ }}\ensuremath{\text{ }}\ensuremath{\text{ }}\ensuremath{\text{ }}\NormalTok{\textbackslash{}State\ensuremath{\text{ }}\$i\textbackslash{}gets\ensuremath{\text{ }}i+k\$}\newline
\ensuremath{\text{ }}\ensuremath{\text{ }}\ensuremath{\text{ }}\ensuremath{\text{ }}\NormalTok{\textbackslash{}EndIf}\newline
\NormalTok{\textbackslash{}EndIf}\newline
\NormalTok{\textbackslash{}end\{algorithmic\}}\newline
\end{Highlighting}
\end{Shaded}


The LaTeX source can be written to a format familiar to programmers so that it is easy to read. This will not, however, affect the final layout in the document.



\begin{minipage}{0.75000\textwidth}
\begin{center}
\includegraphics[width=1.0\textwidth,height=6.5in,keepaspectratio]{../images/121.png}
\end{center}
\raggedright{}\myfigurewithoutcaption{121}
\end{minipage}\vspace{0.75cm}



Basic commands have the following syntax:

Statement (\textbackslash{}State causes a new line, can also be used in front of other commands)

\begin{Shaded}
\begin{Highlighting}[]

\NormalTok{\textbackslash{}State\ensuremath{\text{ }}\$x\textbackslash{}gets\ensuremath{\text{ }}<value>\$}\newline
\end{Highlighting}
\end{Shaded}


Three forms of if-{}statements:

\begin{Shaded}
\begin{Highlighting}[]

\NormalTok{\textbackslash{}If\{<condition>\}\ensuremath{\text{ }}<text>\ensuremath{\text{ }}\textbackslash{}EndIf}\newline
\end{Highlighting}
\end{Shaded}


\begin{Shaded}
\begin{Highlighting}[]

\NormalTok{\textbackslash{}If\{<condition>\}\ensuremath{\text{ }}<text>\ensuremath{\text{ }}\textbackslash{}Else\ensuremath{\text{ }}<text>\ensuremath{\text{ }}\textbackslash{}EndIf}\newline
\end{Highlighting}
\end{Shaded}


\begin{Shaded}
\begin{Highlighting}[]

\NormalTok{\textbackslash{}If\{<condition>\}\ensuremath{\text{ }}<text>\ensuremath{\text{ }}\textbackslash{}ElsIf\{<condition>\}\ensuremath{\text{ }}<text>\ensuremath{\text{ }}\textbackslash{}Else\ensuremath{\text{ }}<text>\ensuremath{\text{ }}\textbackslash{}EndIf}\newline
\end{Highlighting}
\end{Shaded}

The third form accepts as many {\ttfamily \setmainfont[Path=/usr/share/fonts/truetype/cmu/,UprightFont=cmunrm.ttf,BoldFont=cmunbx.ttf,ItalicFont=cmunti.ttf,BoldItalicFont=cmunbi.ttf]{cmuntt.ttf}\setmonofont[Path=/usr/share/fonts/truetype/cmu/,UprightFont=cmuntt.ttf,BoldFont=cmuntb.ttf,ItalicFont=cmunit.ttf,BoldItalicFont=cmuntx.ttf]{cmuntt.ttf}\ttfamily \textbackslash{}ElsIf\{\}}{$\text{ }$}\setmainfont[Path=/usr/share/fonts/truetype/cmu/,UprightFont=cmunrm.ttf,BoldFont=cmunbx.ttf,ItalicFont=cmunti.ttf,BoldItalicFont=cmunbi.ttf]{cmunrm.ttf}\setmonofont[Path=/usr/share/fonts/truetype/cmu/,UprightFont=cmuntt.ttf,BoldFont=cmuntb.ttf,ItalicFont=cmunit.ttf,BoldItalicFont=cmuntx.ttf]{cmunrm.ttf} clauses as required. Note that it is {\ttfamily \setmainfont[Path=/usr/share/fonts/truetype/cmu/,UprightFont=cmunrm.ttf,BoldFont=cmunbx.ttf,ItalicFont=cmunti.ttf,BoldItalicFont=cmunbi.ttf]{cmuntt.ttf}\setmonofont[Path=/usr/share/fonts/truetype/cmu/,UprightFont=cmuntt.ttf,BoldFont=cmuntb.ttf,ItalicFont=cmunit.ttf,BoldItalicFont=cmuntx.ttf]{cmuntt.ttf}\ttfamily \textbackslash{}ElsIf}{$\text{ }$}\setmainfont[Path=/usr/share/fonts/truetype/cmu/,UprightFont=cmunrm.ttf,BoldFont=cmunbx.ttf,ItalicFont=cmunti.ttf,BoldItalicFont=cmunbi.ttf]{cmunrm.ttf}\setmonofont[Path=/usr/share/fonts/truetype/cmu/,UprightFont=cmuntt.ttf,BoldFont=cmuntb.ttf,ItalicFont=cmunit.ttf,BoldItalicFont=cmuntx.ttf]{cmunrm.ttf} and not {\ttfamily \setmainfont[Path=/usr/share/fonts/truetype/cmu/,UprightFont=cmunrm.ttf,BoldFont=cmunbx.ttf,ItalicFont=cmunti.ttf,BoldItalicFont=cmunbi.ttf]{cmuntt.ttf}\setmonofont[Path=/usr/share/fonts/truetype/cmu/,UprightFont=cmuntt.ttf,BoldFont=cmuntb.ttf,ItalicFont=cmunit.ttf,BoldItalicFont=cmuntx.ttf]{cmuntt.ttf}\ttfamily \textbackslash{}ElseIf}\setmainfont[Path=/usr/share/fonts/truetype/cmu/,UprightFont=cmunrm.ttf,BoldFont=cmunbx.ttf,ItalicFont=cmunti.ttf,BoldItalicFont=cmunbi.ttf]{cmunrm.ttf}\setmonofont[Path=/usr/share/fonts/truetype/cmu/,UprightFont=cmuntt.ttf,BoldFont=cmuntb.ttf,ItalicFont=cmunit.ttf,BoldItalicFont=cmuntx.ttf]{cmunrm.ttf}.

Loops:

\begin{Shaded}
\begin{Highlighting}[]

\NormalTok{\textbackslash{}For\{<condition>\}\ensuremath{\text{ }}<text>\ensuremath{\text{ }}\textbackslash{}EndFor}\newline
\end{Highlighting}
\end{Shaded}


\begin{Shaded}
\begin{Highlighting}[]

\NormalTok{\textbackslash{}ForAll\{<condition>\}\ensuremath{\text{ }}<text>\ensuremath{\text{ }}\textbackslash{}EndFor}\newline
\end{Highlighting}
\end{Shaded}


\begin{Shaded}
\begin{Highlighting}[]

\NormalTok{\textbackslash{}While\{<condition>\}\ensuremath{\text{ }}<text>\ensuremath{\text{ }}\textbackslash{}EndWhile}\newline
\end{Highlighting}
\end{Shaded}


\begin{Shaded}
\begin{Highlighting}[]

\NormalTok{\textbackslash{}Repeat\ensuremath{\text{ }}<text>\ensuremath{\text{ }}\textbackslash{}Until\{<condition>\}}\newline
\end{Highlighting}
\end{Shaded}


\begin{Shaded}
\begin{Highlighting}[]

\NormalTok{\textbackslash{}Loop\ensuremath{\text{ }}<text>\ensuremath{\text{ }}\textbackslash{}EndLoop}\newline
\end{Highlighting}
\end{Shaded}


Pre-{} and postcondition:

\begin{Shaded}
\begin{Highlighting}[]

\NormalTok{\textbackslash{}Require\ensuremath{\text{ }}<text>}\newline
\end{Highlighting}
\end{Shaded}


\begin{Shaded}
\begin{Highlighting}[]

\NormalTok{\textbackslash{}Ensure\ensuremath{\text{ }}<text>}\newline
\end{Highlighting}
\end{Shaded}


Functions

\begin{Shaded}
\begin{Highlighting}[]

\NormalTok{\textbackslash{}Function\{<name>\}\{<params>\}\ensuremath{\text{ }}<body>\ensuremath{\text{ }}\textbackslash{}EndFunction}\newline
\end{Highlighting}
\end{Shaded}


\begin{Shaded}
\begin{Highlighting}[]

\NormalTok{\textbackslash{}Return\ensuremath{\text{ }}<text>}\newline
\end{Highlighting}
\end{Shaded}


\begin{Shaded}
\begin{Highlighting}[]

\NormalTok{\textbackslash{}Call\{<name>\}\{<params>\}}\newline
\end{Highlighting}
\end{Shaded}


This command will usually be used in conjunction with a {\ttfamily \setmainfont[Path=/usr/share/fonts/truetype/cmu/,UprightFont=cmunrm.ttf,BoldFont=cmunbx.ttf,ItalicFont=cmunti.ttf,BoldItalicFont=cmunbi.ttf]{cmuntt.ttf}\setmonofont[Path=/usr/share/fonts/truetype/cmu/,UprightFont=cmuntt.ttf,BoldFont=cmuntb.ttf,ItalicFont=cmunit.ttf,BoldItalicFont=cmuntx.ttf]{cmuntt.ttf}\ttfamily \textbackslash{}State}{$\text{ }$}\setmainfont[Path=/usr/share/fonts/truetype/cmu/,UprightFont=cmunrm.ttf,BoldFont=cmunbx.ttf,ItalicFont=cmunti.ttf,BoldItalicFont=cmunbi.ttf]{cmunrm.ttf}\setmonofont[Path=/usr/share/fonts/truetype/cmu/,UprightFont=cmuntt.ttf,BoldFont=cmuntb.ttf,ItalicFont=cmunit.ttf,BoldItalicFont=cmuntx.ttf]{cmunrm.ttf} command as follows:

\begin{Shaded}
\begin{Highlighting}[]

\NormalTok{\textbackslash{}Function\{Increment\}\{\$a\$\}}\newline
\ensuremath{\text{ }}\ensuremath{\text{ }}\ensuremath{\text{ }}\ensuremath{\text{ }}\NormalTok{\textbackslash{}State\ensuremath{\text{ }}\$a\ensuremath{\text{ }}\textbackslash{}gets\ensuremath{\text{ }}a+1\$}\newline
\ensuremath{\text{ }}\ensuremath{\text{ }}\ensuremath{\text{ }}\ensuremath{\text{ }}\NormalTok{\textbackslash{}State\ensuremath{\text{ }}\textbackslash{}Return\ensuremath{\text{ }}\$a\$}\newline
\NormalTok{\textbackslash{}EndFunction}\newline
\end{Highlighting}
\end{Shaded}


Comments:

\begin{Shaded}
\begin{Highlighting}[]

\NormalTok{\textbackslash{}Comment\{<text>\}}\newline
\end{Highlighting}
\end{Shaded}


Note to users who switched from the old {\ttfamily \setmainfont[Path=/usr/share/fonts/truetype/cmu/,UprightFont=cmunrm.ttf,BoldFont=cmunbx.ttf,ItalicFont=cmunti.ttf,BoldItalicFont=cmunbi.ttf]{cmuntt.ttf}\setmonofont[Path=/usr/share/fonts/truetype/cmu/,UprightFont=cmuntt.ttf,BoldFont=cmuntb.ttf,ItalicFont=cmunit.ttf,BoldItalicFont=cmuntx.ttf]{cmuntt.ttf}\ttfamily algorithmic}{$\text{ }$}\setmainfont[Path=/usr/share/fonts/truetype/cmu/,UprightFont=cmunrm.ttf,BoldFont=cmunbx.ttf,ItalicFont=cmunti.ttf,BoldItalicFont=cmunbi.ttf]{cmunrm.ttf}\setmonofont[Path=/usr/share/fonts/truetype/cmu/,UprightFont=cmuntt.ttf,BoldFont=cmuntb.ttf,ItalicFont=cmunit.ttf,BoldItalicFont=cmuntx.ttf]{cmunrm.ttf} package: comments may be placed everywhere in the source; there are no limitations as in the old {\ttfamily \setmainfont[Path=/usr/share/fonts/truetype/cmu/,UprightFont=cmunrm.ttf,BoldFont=cmunbx.ttf,ItalicFont=cmunti.ttf,BoldItalicFont=cmunbi.ttf]{cmuntt.ttf}\setmonofont[Path=/usr/share/fonts/truetype/cmu/,UprightFont=cmuntt.ttf,BoldFont=cmuntb.ttf,ItalicFont=cmunit.ttf,BoldItalicFont=cmuntx.ttf]{cmuntt.ttf}\ttfamily algorithmic}{$\text{ }$}\setmainfont[Path=/usr/share/fonts/truetype/cmu/,UprightFont=cmunrm.ttf,BoldFont=cmunbx.ttf,ItalicFont=cmunti.ttf,BoldItalicFont=cmunbi.ttf]{cmunrm.ttf}\setmonofont[Path=/usr/share/fonts/truetype/cmu/,UprightFont=cmuntt.ttf,BoldFont=cmuntb.ttf,ItalicFont=cmunit.ttf,BoldItalicFont=cmuntx.ttf]{cmunrm.ttf} package.

The {\ttfamily \setmainfont[Path=/usr/share/fonts/truetype/cmu/,UprightFont=cmunrm.ttf,BoldFont=cmunbx.ttf,ItalicFont=cmunti.ttf,BoldItalicFont=cmunbi.ttf]{cmuntt.ttf}\setmonofont[Path=/usr/share/fonts/truetype/cmu/,UprightFont=cmuntt.ttf,BoldFont=cmuntb.ttf,ItalicFont=cmunit.ttf,BoldItalicFont=cmuntx.ttf]{cmuntt.ttf}\ttfamily algorithmicx}{$\text{ }$}\setmainfont[Path=/usr/share/fonts/truetype/cmu/,UprightFont=cmunrm.ttf,BoldFont=cmunbx.ttf,ItalicFont=cmunti.ttf,BoldItalicFont=cmunbi.ttf]{cmunrm.ttf}\setmonofont[Path=/usr/share/fonts/truetype/cmu/,UprightFont=cmuntt.ttf,BoldFont=cmuntb.ttf,ItalicFont=cmunit.ttf,BoldItalicFont=cmuntx.ttf]{cmunrm.ttf} package allows you to define your own environments.

To define blocks beginning with a starting command and ending with an ending command, use

\begin{Shaded}
\begin{Highlighting}[]

\NormalTok{\textbackslash{}algblock[<block>]\{<start>\}\{<end>\}}\newline
\end{Highlighting}
\end{Shaded}

This defines two commands {\ttfamily \setmainfont[Path=/usr/share/fonts/truetype/cmu/,UprightFont=cmunrm.ttf,BoldFont=cmunbx.ttf,ItalicFont=cmunti.ttf,BoldItalicFont=cmunbi.ttf]{cmuntt.ttf}\setmonofont[Path=/usr/share/fonts/truetype/cmu/,UprightFont=cmuntt.ttf,BoldFont=cmuntb.ttf,ItalicFont=cmunit.ttf,BoldItalicFont=cmuntx.ttf]{cmuntt.ttf}\ttfamily \textbackslash{}<{}start>{}}{$\text{ }$}\setmainfont[Path=/usr/share/fonts/truetype/cmu/,UprightFont=cmunrm.ttf,BoldFont=cmunbx.ttf,ItalicFont=cmunti.ttf,BoldItalicFont=cmunbi.ttf]{cmunrm.ttf}\setmonofont[Path=/usr/share/fonts/truetype/cmu/,UprightFont=cmuntt.ttf,BoldFont=cmuntb.ttf,ItalicFont=cmunit.ttf,BoldItalicFont=cmuntx.ttf]{cmunrm.ttf} and {\ttfamily \setmainfont[Path=/usr/share/fonts/truetype/cmu/,UprightFont=cmunrm.ttf,BoldFont=cmunbx.ttf,ItalicFont=cmunti.ttf,BoldItalicFont=cmunbi.ttf]{cmuntt.ttf}\setmonofont[Path=/usr/share/fonts/truetype/cmu/,UprightFont=cmuntt.ttf,BoldFont=cmuntb.ttf,ItalicFont=cmunit.ttf,BoldItalicFont=cmuntx.ttf]{cmuntt.ttf}\ttfamily \textbackslash{}<{}end>{}}{$\text{ }$}\setmainfont[Path=/usr/share/fonts/truetype/cmu/,UprightFont=cmunrm.ttf,BoldFont=cmunbx.ttf,ItalicFont=cmunti.ttf,BoldItalicFont=cmunbi.ttf]{cmunrm.ttf}\setmonofont[Path=/usr/share/fonts/truetype/cmu/,UprightFont=cmuntt.ttf,BoldFont=cmuntb.ttf,ItalicFont=cmunit.ttf,BoldItalicFont=cmuntx.ttf]{cmunrm.ttf} which have no parameters. The text displayed by them is {\ttfamily \setmainfont[Path=/usr/share/fonts/truetype/cmu/,UprightFont=cmunrm.ttf,BoldFont=cmunbx.ttf,ItalicFont=cmunti.ttf,BoldItalicFont=cmunbi.ttf]{cmuntt.ttf}\setmonofont[Path=/usr/share/fonts/truetype/cmu/,UprightFont=cmuntt.ttf,BoldFont=cmuntb.ttf,ItalicFont=cmunit.ttf,BoldItalicFont=cmuntx.ttf]{cmuntt.ttf}\ttfamily \textbackslash{}textbf\{<{}start>{}\}}{$\text{ }$}\setmainfont[Path=/usr/share/fonts/truetype/cmu/,UprightFont=cmunrm.ttf,BoldFont=cmunbx.ttf,ItalicFont=cmunti.ttf,BoldItalicFont=cmunbi.ttf]{cmunrm.ttf}\setmonofont[Path=/usr/share/fonts/truetype/cmu/,UprightFont=cmuntt.ttf,BoldFont=cmuntb.ttf,ItalicFont=cmunit.ttf,BoldItalicFont=cmuntx.ttf]{cmunrm.ttf} and {\ttfamily \setmainfont[Path=/usr/share/fonts/truetype/cmu/,UprightFont=cmunrm.ttf,BoldFont=cmunbx.ttf,ItalicFont=cmunti.ttf,BoldItalicFont=cmunbi.ttf]{cmuntt.ttf}\setmonofont[Path=/usr/share/fonts/truetype/cmu/,UprightFont=cmuntt.ttf,BoldFont=cmuntb.ttf,ItalicFont=cmunit.ttf,BoldItalicFont=cmuntx.ttf]{cmuntt.ttf}\ttfamily \textbackslash{}textbf\{<{}end>{}\}}\setmainfont[Path=/usr/share/fonts/truetype/cmu/,UprightFont=cmunrm.ttf,BoldFont=cmunbx.ttf,ItalicFont=cmunti.ttf,BoldItalicFont=cmunbi.ttf]{cmunrm.ttf}\setmonofont[Path=/usr/share/fonts/truetype/cmu/,UprightFont=cmuntt.ttf,BoldFont=cmuntb.ttf,ItalicFont=cmunit.ttf,BoldItalicFont=cmuntx.ttf]{cmunrm.ttf}.

With {\ttfamily \setmainfont[Path=/usr/share/fonts/truetype/cmu/,UprightFont=cmunrm.ttf,BoldFont=cmunbx.ttf,ItalicFont=cmunti.ttf,BoldItalicFont=cmunbi.ttf]{cmuntt.ttf}\setmonofont[Path=/usr/share/fonts/truetype/cmu/,UprightFont=cmuntt.ttf,BoldFont=cmuntb.ttf,ItalicFont=cmunit.ttf,BoldItalicFont=cmuntx.ttf]{cmuntt.ttf}\ttfamily \textbackslash{}algblockdefx}{$\text{ }$}\setmainfont[Path=/usr/share/fonts/truetype/cmu/,UprightFont=cmunrm.ttf,BoldFont=cmunbx.ttf,ItalicFont=cmunti.ttf,BoldItalicFont=cmunbi.ttf]{cmunrm.ttf}\setmonofont[Path=/usr/share/fonts/truetype/cmu/,UprightFont=cmuntt.ttf,BoldFont=cmuntb.ttf,ItalicFont=cmunit.ttf,BoldItalicFont=cmuntx.ttf]{cmunrm.ttf} you can give the text to be output by the starting and ending command and the number of parameters for these commands. In the text the n-{}th parameter is referenced by {\ttfamily \setmainfont[Path=/usr/share/fonts/truetype/cmu/,UprightFont=cmunrm.ttf,BoldFont=cmunbx.ttf,ItalicFont=cmunti.ttf,BoldItalicFont=cmunbi.ttf]{cmuntt.ttf}\setmonofont[Path=/usr/share/fonts/truetype/cmu/,UprightFont=cmuntt.ttf,BoldFont=cmuntb.ttf,ItalicFont=cmunit.ttf,BoldItalicFont=cmuntx.ttf]{cmuntt.ttf}\ttfamily \#n}\setmainfont[Path=/usr/share/fonts/truetype/cmu/,UprightFont=cmunrm.ttf,BoldFont=cmunbx.ttf,ItalicFont=cmunti.ttf,BoldItalicFont=cmunbi.ttf]{cmunrm.ttf}\setmonofont[Path=/usr/share/fonts/truetype/cmu/,UprightFont=cmuntt.ttf,BoldFont=cmuntb.ttf,ItalicFont=cmunit.ttf,BoldItalicFont=cmuntx.ttf]{cmunrm.ttf}.

\begin{Shaded}
\begin{Highlighting}[]

\NormalTok{\textbackslash{}algblockdefx[<block>]\{<start>\}\{<end>\}}\newline
\ensuremath{\text{ }}\ensuremath{\text{ }}\ensuremath{\text{ }}\ensuremath{\text{ }}\NormalTok{[<startparamcount>][<default\ensuremath{\text{ }}value>]\{<start\ensuremath{\text{ }}text>\}}\newline
\ensuremath{\text{ }}\ensuremath{\text{ }}\ensuremath{\text{ }}\ensuremath{\text{ }}\NormalTok{[<endparamcount>][<default\ensuremath{\text{ }}value>]\{<end\ensuremath{\text{ }}text>\}}\newline
\end{Highlighting}
\end{Shaded}


Example:

\begin{Shaded}
\begin{Highlighting}[]

\NormalTok{\textbackslash{}algblock[Name]\{Start\}\{End\}}\newline
\NormalTok{\textbackslash{}algblockdefx[NAME]\{START\}\{END\}}\CommentTok{\%}\newline
\ensuremath{\text{ }}\ensuremath{\text{ }}\ensuremath{\text{ }}\ensuremath{\text{ }}\NormalTok{[2][Unknown]\{Start\ensuremath{\text{ }}#1(#2)\}}\CommentTok{\%}\newline
\ensuremath{\text{ }}\ensuremath{\text{ }}\ensuremath{\text{ }}\ensuremath{\text{ }}\NormalTok{\{Ending\}}\newline
\NormalTok{\textbackslash{}algblockdefx[NAME]\{\}\{OTHEREND\}}\CommentTok{\%}\newline
\ensuremath{\text{ }}\ensuremath{\text{ }}\ensuremath{\text{ }}\ensuremath{\text{ }}\NormalTok{[1]\{Until\ensuremath{\text{ }}(#1)\}}\newline
\NormalTok{\textbackslash{}begin\{algorithmic\}}\newline
\NormalTok{\textbackslash{}Start}\newline
\ensuremath{\text{ }}\ensuremath{\text{ }}\ensuremath{\text{ }}\ensuremath{\text{ }}\NormalTok{\textbackslash{}Start}\newline
\ensuremath{\text{ }}\ensuremath{\text{ }}\ensuremath{\text{ }}\ensuremath{\text{ }}\ensuremath{\text{ }}\ensuremath{\text{ }}\ensuremath{\text{ }}\ensuremath{\text{ }}\NormalTok{\textbackslash{}START[One]\{x\}}\newline
\ensuremath{\text{ }}\ensuremath{\text{ }}\ensuremath{\text{ }}\ensuremath{\text{ }}\ensuremath{\text{ }}\ensuremath{\text{ }}\ensuremath{\text{ }}\ensuremath{\text{ }}\NormalTok{\textbackslash{}END}\newline
\ensuremath{\text{ }}\ensuremath{\text{ }}\ensuremath{\text{ }}\ensuremath{\text{ }}\ensuremath{\text{ }}\ensuremath{\text{ }}\ensuremath{\text{ }}\ensuremath{\text{ }}\NormalTok{\textbackslash{}START\{0\}}\newline
\ensuremath{\text{ }}\ensuremath{\text{ }}\ensuremath{\text{ }}\ensuremath{\text{ }}\ensuremath{\text{ }}\ensuremath{\text{ }}\ensuremath{\text{ }}\ensuremath{\text{ }}\NormalTok{\textbackslash{}OTHEREND\{\textbackslash{}texttt\{True\}\}}\newline
\ensuremath{\text{ }}\ensuremath{\text{ }}\ensuremath{\text{ }}\ensuremath{\text{ }}\NormalTok{\textbackslash{}End}\newline
\ensuremath{\text{ }}\ensuremath{\text{ }}\ensuremath{\text{ }}\ensuremath{\text{ }}\NormalTok{\textbackslash{}Start}\newline
\ensuremath{\text{ }}\ensuremath{\text{ }}\ensuremath{\text{ }}\ensuremath{\text{ }}\NormalTok{\textbackslash{}End}\newline
\NormalTok{\textbackslash{}End}\newline
\NormalTok{\textbackslash{}end\{algorithmic\}}\newline
\end{Highlighting}
\end{Shaded}


More advanced customization and other constructions are described in the {\ttfamily \setmainfont[Path=/usr/share/fonts/truetype/cmu/,UprightFont=cmunrm.ttf,BoldFont=cmunbx.ttf,ItalicFont=cmunti.ttf,BoldItalicFont=cmunbi.ttf]{cmuntt.ttf}\setmonofont[Path=/usr/share/fonts/truetype/cmu/,UprightFont=cmuntt.ttf,BoldFont=cmuntb.ttf,ItalicFont=cmunit.ttf,BoldItalicFont=cmuntx.ttf]{cmuntt.ttf}\ttfamily algorithmicx}{$\text{ }$}\setmainfont[Path=/usr/share/fonts/truetype/cmu/,UprightFont=cmunrm.ttf,BoldFont=cmunbx.ttf,ItalicFont=cmunti.ttf,BoldItalicFont=cmunbi.ttf]{cmunrm.ttf}\setmonofont[Path=/usr/share/fonts/truetype/cmu/,UprightFont=cmuntt.ttf,BoldFont=cmuntb.ttf,ItalicFont=cmunit.ttf,BoldItalicFont=cmuntx.ttf]{cmunrm.ttf} manual: \myplainurl{http://mirror.ctan.org/macros/latex/contrib/algorithmicx/algorithmicx.pdf}
\subsection{Typesetting using the {\ttfamily \setmainfont[Path=/usr/share/fonts/truetype/cmu/,UprightFont=cmunrm.ttf,BoldFont=cmunbx.ttf,ItalicFont=cmunti.ttf,BoldItalicFont=cmunbi.ttf]{cmuntt.ttf}\setmonofont[Path=/usr/share/fonts/truetype/cmu/,UprightFont=cmuntt.ttf,BoldFont=cmuntb.ttf,ItalicFont=cmunit.ttf,BoldItalicFont=cmuntx.ttf]{cmuntt.ttf}\ttfamily program}{$\text{ }$}\setmainfont[Path=/usr/share/fonts/truetype/cmu/,UprightFont=cmunrm.ttf,BoldFont=cmunbx.ttf,ItalicFont=cmunti.ttf,BoldItalicFont=cmunbi.ttf]{cmunrm.ttf}\setmonofont[Path=/usr/share/fonts/truetype/cmu/,UprightFont=cmuntt.ttf,BoldFont=cmuntb.ttf,ItalicFont=cmunit.ttf,BoldItalicFont=cmuntx.ttf]{cmunrm.ttf} package}
\label{586}

The {\ttfamily \setmainfont[Path=/usr/share/fonts/truetype/cmu/,UprightFont=cmunrm.ttf,BoldFont=cmunbx.ttf,ItalicFont=cmunti.ttf,BoldItalicFont=cmunbi.ttf]{cmuntt.ttf}\setmonofont[Path=/usr/share/fonts/truetype/cmu/,UprightFont=cmuntt.ttf,BoldFont=cmuntb.ttf,ItalicFont=cmunit.ttf,BoldItalicFont=cmuntx.ttf]{cmuntt.ttf}\ttfamily program}{$\text{ }$}\setmainfont[Path=/usr/share/fonts/truetype/cmu/,UprightFont=cmunrm.ttf,BoldFont=cmunbx.ttf,ItalicFont=cmunti.ttf,BoldItalicFont=cmunbi.ttf]{cmunrm.ttf}\setmonofont[Path=/usr/share/fonts/truetype/cmu/,UprightFont=cmuntt.ttf,BoldFont=cmuntb.ttf,ItalicFont=cmunit.ttf,BoldItalicFont=cmuntx.ttf]{cmunrm.ttf} package provides macros for typesetting algorithms.
Each line is set in math mode, so all the indentation and spacing is done automatically. 
The notation  {\ttfamily \setmainfont[Path=/usr/share/fonts/truetype/cmu/,UprightFont=cmunrm.ttf,BoldFont=cmunbx.ttf,ItalicFont=cmunti.ttf,BoldItalicFont=cmunbi.ttf]{cmuntt.ttf}\setmonofont[Path=/usr/share/fonts/truetype/cmu/,UprightFont=cmuntt.ttf,BoldFont=cmuntb.ttf,ItalicFont=cmunit.ttf,BoldItalicFont=cmuntx.ttf]{cmuntt.ttf}\ttfamily |variable\_name|}{$\text{ }$}\setmainfont[Path=/usr/share/fonts/truetype/cmu/,UprightFont=cmunrm.ttf,BoldFont=cmunbx.ttf,ItalicFont=cmunti.ttf,BoldItalicFont=cmunbi.ttf]{cmunrm.ttf}\setmonofont[Path=/usr/share/fonts/truetype/cmu/,UprightFont=cmuntt.ttf,BoldFont=cmuntb.ttf,ItalicFont=cmunit.ttf,BoldItalicFont=cmuntx.ttf]{cmunrm.ttf} can be used within normal text,
maths expressions or programs to indicate a variable name.
Use {\ttfamily \setmainfont[Path=/usr/share/fonts/truetype/cmu/,UprightFont=cmunrm.ttf,BoldFont=cmunbx.ttf,ItalicFont=cmunti.ttf,BoldItalicFont=cmunbi.ttf]{cmuntt.ttf}\setmonofont[Path=/usr/share/fonts/truetype/cmu/,UprightFont=cmuntt.ttf,BoldFont=cmuntb.ttf,ItalicFont=cmunit.ttf,BoldItalicFont=cmuntx.ttf]{cmuntt.ttf}\ttfamily \textbackslash{}origbar}{$\text{ }$}\setmainfont[Path=/usr/share/fonts/truetype/cmu/,UprightFont=cmunrm.ttf,BoldFont=cmunbx.ttf,ItalicFont=cmunti.ttf,BoldItalicFont=cmunbi.ttf]{cmunrm.ttf}\setmonofont[Path=/usr/share/fonts/truetype/cmu/,UprightFont=cmuntt.ttf,BoldFont=cmuntb.ttf,ItalicFont=cmunit.ttf,BoldItalicFont=cmuntx.ttf]{cmunrm.ttf} to get a normal {\ttfamily \setmainfont[Path=/usr/share/fonts/truetype/cmu/,UprightFont=cmunrm.ttf,BoldFont=cmunbx.ttf,ItalicFont=cmunti.ttf,BoldItalicFont=cmunbi.ttf]{cmuntt.ttf}\setmonofont[Path=/usr/share/fonts/truetype/cmu/,UprightFont=cmuntt.ttf,BoldFont=cmuntb.ttf,ItalicFont=cmunit.ttf,BoldItalicFont=cmuntx.ttf]{cmuntt.ttf}\ttfamily |}{$\text{ }$}\setmainfont[Path=/usr/share/fonts/truetype/cmu/,UprightFont=cmunrm.ttf,BoldFont=cmunbx.ttf,ItalicFont=cmunti.ttf,BoldItalicFont=cmunbi.ttf]{cmunrm.ttf}\setmonofont[Path=/usr/share/fonts/truetype/cmu/,UprightFont=cmuntt.ttf,BoldFont=cmuntb.ttf,ItalicFont=cmunit.ttf,BoldItalicFont=cmuntx.ttf]{cmunrm.ttf} symbol in a program.
The commands {\ttfamily \setmainfont[Path=/usr/share/fonts/truetype/cmu/,UprightFont=cmunrm.ttf,BoldFont=cmunbx.ttf,ItalicFont=cmunti.ttf,BoldItalicFont=cmunbi.ttf]{cmuntt.ttf}\setmonofont[Path=/usr/share/fonts/truetype/cmu/,UprightFont=cmuntt.ttf,BoldFont=cmuntb.ttf,ItalicFont=cmunit.ttf,BoldItalicFont=cmuntx.ttf]{cmuntt.ttf}\ttfamily \textbackslash{}A}\setmainfont[Path=/usr/share/fonts/truetype/cmu/,UprightFont=cmunrm.ttf,BoldFont=cmunbx.ttf,ItalicFont=cmunti.ttf,BoldItalicFont=cmunbi.ttf]{cmunrm.ttf}\setmonofont[Path=/usr/share/fonts/truetype/cmu/,UprightFont=cmuntt.ttf,BoldFont=cmuntb.ttf,ItalicFont=cmunit.ttf,BoldItalicFont=cmuntx.ttf]{cmunrm.ttf}, {\ttfamily \setmainfont[Path=/usr/share/fonts/truetype/cmu/,UprightFont=cmunrm.ttf,BoldFont=cmunbx.ttf,ItalicFont=cmunti.ttf,BoldItalicFont=cmunbi.ttf]{cmuntt.ttf}\setmonofont[Path=/usr/share/fonts/truetype/cmu/,UprightFont=cmuntt.ttf,BoldFont=cmuntb.ttf,ItalicFont=cmunit.ttf,BoldItalicFont=cmuntx.ttf]{cmuntt.ttf}\ttfamily \textbackslash{}B}\setmainfont[Path=/usr/share/fonts/truetype/cmu/,UprightFont=cmunrm.ttf,BoldFont=cmunbx.ttf,ItalicFont=cmunti.ttf,BoldItalicFont=cmunbi.ttf]{cmunrm.ttf}\setmonofont[Path=/usr/share/fonts/truetype/cmu/,UprightFont=cmuntt.ttf,BoldFont=cmuntb.ttf,ItalicFont=cmunit.ttf,BoldItalicFont=cmuntx.ttf]{cmunrm.ttf}, {\ttfamily \setmainfont[Path=/usr/share/fonts/truetype/cmu/,UprightFont=cmunrm.ttf,BoldFont=cmunbx.ttf,ItalicFont=cmunti.ttf,BoldItalicFont=cmunbi.ttf]{cmuntt.ttf}\setmonofont[Path=/usr/share/fonts/truetype/cmu/,UprightFont=cmuntt.ttf,BoldFont=cmuntb.ttf,ItalicFont=cmunit.ttf,BoldItalicFont=cmuntx.ttf]{cmuntt.ttf}\ttfamily \textbackslash{}P}\setmainfont[Path=/usr/share/fonts/truetype/cmu/,UprightFont=cmunrm.ttf,BoldFont=cmunbx.ttf,ItalicFont=cmunti.ttf,BoldItalicFont=cmunbi.ttf]{cmunrm.ttf}\setmonofont[Path=/usr/share/fonts/truetype/cmu/,UprightFont=cmuntt.ttf,BoldFont=cmuntb.ttf,ItalicFont=cmunit.ttf,BoldItalicFont=cmuntx.ttf]{cmunrm.ttf}, {\ttfamily \setmainfont[Path=/usr/share/fonts/truetype/cmu/,UprightFont=cmunrm.ttf,BoldFont=cmunbx.ttf,ItalicFont=cmunti.ttf,BoldItalicFont=cmunbi.ttf]{cmuntt.ttf}\setmonofont[Path=/usr/share/fonts/truetype/cmu/,UprightFont=cmuntt.ttf,BoldFont=cmuntb.ttf,ItalicFont=cmunit.ttf,BoldItalicFont=cmuntx.ttf]{cmuntt.ttf}\ttfamily \textbackslash{}Q}\setmainfont[Path=/usr/share/fonts/truetype/cmu/,UprightFont=cmunrm.ttf,BoldFont=cmunbx.ttf,ItalicFont=cmunti.ttf,BoldItalicFont=cmunbi.ttf]{cmunrm.ttf}\setmonofont[Path=/usr/share/fonts/truetype/cmu/,UprightFont=cmuntt.ttf,BoldFont=cmuntb.ttf,ItalicFont=cmunit.ttf,BoldItalicFont=cmuntx.ttf]{cmunrm.ttf}, {\ttfamily \setmainfont[Path=/usr/share/fonts/truetype/cmu/,UprightFont=cmunrm.ttf,BoldFont=cmunbx.ttf,ItalicFont=cmunti.ttf,BoldItalicFont=cmunbi.ttf]{cmuntt.ttf}\setmonofont[Path=/usr/share/fonts/truetype/cmu/,UprightFont=cmuntt.ttf,BoldFont=cmuntb.ttf,ItalicFont=cmunit.ttf,BoldItalicFont=cmuntx.ttf]{cmuntt.ttf}\ttfamily \textbackslash{}R}\setmainfont[Path=/usr/share/fonts/truetype/cmu/,UprightFont=cmunrm.ttf,BoldFont=cmunbx.ttf,ItalicFont=cmunti.ttf,BoldItalicFont=cmunbi.ttf]{cmunrm.ttf}\setmonofont[Path=/usr/share/fonts/truetype/cmu/,UprightFont=cmuntt.ttf,BoldFont=cmuntb.ttf,ItalicFont=cmunit.ttf,BoldItalicFont=cmuntx.ttf]{cmunrm.ttf}, {\ttfamily \setmainfont[Path=/usr/share/fonts/truetype/cmu/,UprightFont=cmunrm.ttf,BoldFont=cmunbx.ttf,ItalicFont=cmunti.ttf,BoldItalicFont=cmunbi.ttf]{cmuntt.ttf}\setmonofont[Path=/usr/share/fonts/truetype/cmu/,UprightFont=cmuntt.ttf,BoldFont=cmuntb.ttf,ItalicFont=cmunit.ttf,BoldItalicFont=cmuntx.ttf]{cmuntt.ttf}\ttfamily \textbackslash{}S}\setmainfont[Path=/usr/share/fonts/truetype/cmu/,UprightFont=cmunrm.ttf,BoldFont=cmunbx.ttf,ItalicFont=cmunti.ttf,BoldItalicFont=cmunbi.ttf]{cmunrm.ttf}\setmonofont[Path=/usr/share/fonts/truetype/cmu/,UprightFont=cmuntt.ttf,BoldFont=cmuntb.ttf,ItalicFont=cmunit.ttf,BoldItalicFont=cmuntx.ttf]{cmunrm.ttf}, {\ttfamily \setmainfont[Path=/usr/share/fonts/truetype/cmu/,UprightFont=cmunrm.ttf,BoldFont=cmunbx.ttf,ItalicFont=cmunti.ttf,BoldItalicFont=cmunbi.ttf]{cmuntt.ttf}\setmonofont[Path=/usr/share/fonts/truetype/cmu/,UprightFont=cmuntt.ttf,BoldFont=cmuntb.ttf,ItalicFont=cmunit.ttf,BoldItalicFont=cmuntx.ttf]{cmuntt.ttf}\ttfamily \textbackslash{}T}{$\text{ }$}\setmainfont[Path=/usr/share/fonts/truetype/cmu/,UprightFont=cmunrm.ttf,BoldFont=cmunbx.ttf,ItalicFont=cmunti.ttf,BoldItalicFont=cmunbi.ttf]{cmunrm.ttf}\setmonofont[Path=/usr/share/fonts/truetype/cmu/,UprightFont=cmuntt.ttf,BoldFont=cmuntb.ttf,ItalicFont=cmunit.ttf,BoldItalicFont=cmuntx.ttf]{cmunrm.ttf} and {\ttfamily \setmainfont[Path=/usr/share/fonts/truetype/cmu/,UprightFont=cmunrm.ttf,BoldFont=cmunbx.ttf,ItalicFont=cmunti.ttf,BoldItalicFont=cmunbi.ttf]{cmuntt.ttf}\setmonofont[Path=/usr/share/fonts/truetype/cmu/,UprightFont=cmuntt.ttf,BoldFont=cmuntb.ttf,ItalicFont=cmunit.ttf,BoldItalicFont=cmuntx.ttf]{cmuntt.ttf}\ttfamily \textbackslash{}Z}\setmainfont[Path=/usr/share/fonts/truetype/cmu/,UprightFont=cmunrm.ttf,BoldFont=cmunbx.ttf,ItalicFont=cmunti.ttf,BoldItalicFont=cmunbi.ttf]{cmunrm.ttf}\setmonofont[Path=/usr/share/fonts/truetype/cmu/,UprightFont=cmuntt.ttf,BoldFont=cmuntb.ttf,ItalicFont=cmunit.ttf,BoldItalicFont=cmuntx.ttf]{cmunrm.ttf}typeset the corresponding bold
letter with the next object as a subscript (eg {\ttfamily \setmainfont[Path=/usr/share/fonts/truetype/cmu/,UprightFont=cmunrm.ttf,BoldFont=cmunbx.ttf,ItalicFont=cmunti.ttf,BoldItalicFont=cmunbi.ttf]{cmuntt.ttf}\setmonofont[Path=/usr/share/fonts/truetype/cmu/,UprightFont=cmuntt.ttf,BoldFont=cmuntb.ttf,ItalicFont=cmunit.ttf,BoldItalicFont=cmuntx.ttf]{cmuntt.ttf}\ttfamily \textbackslash{}S1}{$\text{ }$}\setmainfont[Path=/usr/share/fonts/truetype/cmu/,UprightFont=cmunrm.ttf,BoldFont=cmunbx.ttf,ItalicFont=cmunti.ttf,BoldItalicFont=cmunbi.ttf]{cmunrm.ttf}\setmonofont[Path=/usr/share/fonts/truetype/cmu/,UprightFont=cmuntt.ttf,BoldFont=cmuntb.ttf,ItalicFont=cmunit.ttf,BoldItalicFont=cmuntx.ttf]{cmunrm.ttf} typesets {\ttfamily \setmainfont[Path=/usr/share/fonts/truetype/cmu/,UprightFont=cmunrm.ttf,BoldFont=cmunbx.ttf,ItalicFont=cmunti.ttf,BoldItalicFont=cmunbi.ttf]{cmuntt.ttf}\setmonofont[Path=/usr/share/fonts/truetype/cmu/,UprightFont=cmuntt.ttf,BoldFont=cmuntb.ttf,ItalicFont=cmunit.ttf,BoldItalicFont=cmuntx.ttf]{cmuntt.ttf}\ttfamily \{\textbackslash{}bf
S\${}\_1\${}\}}{$\text{ }$}\setmainfont[Path=/usr/share/fonts/truetype/cmu/,UprightFont=cmunrm.ttf,BoldFont=cmunbx.ttf,ItalicFont=cmunti.ttf,BoldItalicFont=cmunbi.ttf]{cmunrm.ttf}\setmonofont[Path=/usr/share/fonts/truetype/cmu/,UprightFont=cmuntt.ttf,BoldFont=cmuntb.ttf,ItalicFont=cmunit.ttf,BoldItalicFont=cmuntx.ttf]{cmunrm.ttf} etc). Primes work normally, eg {\ttfamily \setmainfont[Path=/usr/share/fonts/truetype/cmu/,UprightFont=cmunrm.ttf,BoldFont=cmunbx.ttf,ItalicFont=cmunti.ttf,BoldItalicFont=cmunbi.ttf]{cmuntt.ttf}\setmonofont[Path=/usr/share/fonts/truetype/cmu/,UprightFont=cmuntt.ttf,BoldFont=cmuntb.ttf,ItalicFont=cmunit.ttf,BoldItalicFont=cmuntx.ttf]{cmuntt.ttf}\ttfamily \textbackslash{}S‘‘}\setmainfont[Path=/usr/share/fonts/truetype/cmu/,UprightFont=cmunrm.ttf,BoldFont=cmunbx.ttf,ItalicFont=cmunti.ttf,BoldItalicFont=cmunbi.ttf]{cmunrm.ttf}\setmonofont[Path=/usr/share/fonts/truetype/cmu/,UprightFont=cmuntt.ttf,BoldFont=cmuntb.ttf,ItalicFont=cmunit.ttf,BoldItalicFont=cmuntx.ttf]{cmunrm.ttf}.

Below is an example of typesetting a basic algorithm using the
{\ttfamily \setmainfont[Path=/usr/share/fonts/truetype/cmu/,UprightFont=cmunrm.ttf,BoldFont=cmunbx.ttf,ItalicFont=cmunti.ttf,BoldItalicFont=cmunbi.ttf]{cmuntt.ttf}\setmonofont[Path=/usr/share/fonts/truetype/cmu/,UprightFont=cmuntt.ttf,BoldFont=cmuntb.ttf,ItalicFont=cmunit.ttf,BoldItalicFont=cmuntx.ttf]{cmuntt.ttf}\ttfamily program}{$\text{ }$}\setmainfont[Path=/usr/share/fonts/truetype/cmu/,UprightFont=cmunrm.ttf,BoldFont=cmunbx.ttf,ItalicFont=cmunti.ttf,BoldItalicFont=cmunbi.ttf]{cmunrm.ttf}\setmonofont[Path=/usr/share/fonts/truetype/cmu/,UprightFont=cmuntt.ttf,BoldFont=cmuntb.ttf,ItalicFont=cmunit.ttf,BoldItalicFont=cmuntx.ttf]{cmunrm.ttf} package (remember to add the
{\ttfamily \setmainfont[Path=/usr/share/fonts/truetype/cmu/,UprightFont=cmunrm.ttf,BoldFont=cmunbx.ttf,ItalicFont=cmunti.ttf,BoldItalicFont=cmunbi.ttf]{cmuntt.ttf}\setmonofont[Path=/usr/share/fonts/truetype/cmu/,UprightFont=cmuntt.ttf,BoldFont=cmuntb.ttf,ItalicFont=cmunit.ttf,BoldItalicFont=cmuntx.ttf]{cmuntt.ttf}\ttfamily \textbackslash{}usepackage\{program\}}{$\text{ }$}\setmainfont[Path=/usr/share/fonts/truetype/cmu/,UprightFont=cmunrm.ttf,BoldFont=cmunbx.ttf,ItalicFont=cmunti.ttf,BoldItalicFont=cmunbi.ttf]{cmunrm.ttf}\setmonofont[Path=/usr/share/fonts/truetype/cmu/,UprightFont=cmuntt.ttf,BoldFont=cmuntb.ttf,ItalicFont=cmunit.ttf,BoldItalicFont=cmuntx.ttf]{cmunrm.ttf} statement to your document
preamble):


\begin{Shaded}
\begin{Highlighting}[]

\NormalTok{\textbackslash{}begin\{program\}}\newline
\NormalTok{\textbackslash{}mbox\{A\ensuremath{\text{ }}fast\ensuremath{\text{ }}exponentiation\ensuremath{\text{ }}procedure:\}}\newline
\NormalTok{\textbackslash{}BEGIN\ensuremath{\text{ }}\textbackslash{}\textbackslash{}\ensuremath{\text{ }}}\CommentTok{\%}\newline
\ensuremath{\text{ }}\ensuremath{\text{ }}\NormalTok{\textbackslash{}FOR\ensuremath{\text{ }}i:=1\ensuremath{\text{ }}\textbackslash{}TO\ensuremath{\text{ }}10\ensuremath{\text{ }}\textbackslash{}STEP\ensuremath{\text{ }}1\ensuremath{\text{ }}\textbackslash{}DO}\newline
\ensuremath{\text{ }}\ensuremath{\text{ }}\ensuremath{\text{ }}\ensuremath{\text{ }}\ensuremath{\text{ }}\NormalTok{|expt|(2,i);\ensuremath{\text{ }}\textbackslash{}\textbackslash{}\ensuremath{\text{ }}|newline|()\ensuremath{\text{ }}\textbackslash{}OD\ensuremath{\text{ }}}\CommentTok{\%}\newline
\NormalTok{\textbackslash{}rcomment\{This\ensuremath{\text{ }}text\ensuremath{\text{ }}will\ensuremath{\text{ }}be\ensuremath{\text{ }}set\ensuremath{\text{ }}flush\ensuremath{\text{ }}to\ensuremath{\text{ }}the\ensuremath{\text{ }}right\ensuremath{\text{ }}margin\}}\newline
\NormalTok{\textbackslash{}WHERE}\newline
\NormalTok{\textbackslash{}PROC\ensuremath{\text{ }}|expt|(x,n)\ensuremath{\text{ }}\textbackslash{}BODY}\newline
\ensuremath{\text{ }}\ensuremath{\text{ }}\ensuremath{\text{ }}\ensuremath{\text{ }}\ensuremath{\text{ }}\ensuremath{\text{ }}\ensuremath{\text{ }}\ensuremath{\text{ }}\ensuremath{\text{ }}\ensuremath{\text{ }}\NormalTok{z:=1;}\newline
\ensuremath{\text{ }}\ensuremath{\text{ }}\ensuremath{\text{ }}\ensuremath{\text{ }}\ensuremath{\text{ }}\ensuremath{\text{ }}\ensuremath{\text{ }}\ensuremath{\text{ }}\ensuremath{\text{ }}\ensuremath{\text{ }}\NormalTok{\textbackslash{}DO\ensuremath{\text{ }}\textbackslash{}IF\ensuremath{\text{ }}n=0\ensuremath{\text{ }}\textbackslash{}THEN\ensuremath{\text{ }}\textbackslash{}EXIT\ensuremath{\text{ }}\textbackslash{}FI;}\newline
\ensuremath{\text{ }}\ensuremath{\text{ }}\ensuremath{\text{ }}\ensuremath{\text{ }}\ensuremath{\text{ }}\ensuremath{\text{ }}\ensuremath{\text{ }}\ensuremath{\text{ }}\ensuremath{\text{ }}\ensuremath{\text{ }}\ensuremath{\text{ }}\ensuremath{\text{ }}\ensuremath{\text{ }}\NormalTok{\textbackslash{}DO\ensuremath{\text{ }}\textbackslash{}IF\ensuremath{\text{ }}|odd|(n)\ensuremath{\text{ }}\textbackslash{}THEN\ensuremath{\text{ }}\textbackslash{}EXIT\ensuremath{\text{ }}\textbackslash{}FI;}\newline
\NormalTok{\textbackslash{}COMMENT\{This\ensuremath{\text{ }}is\ensuremath{\text{ }}a\ensuremath{\text{ }}comment\ensuremath{\text{ }}statement\};}\newline
\ensuremath{\text{ }}\ensuremath{\text{ }}\ensuremath{\text{ }}\ensuremath{\text{ }}\ensuremath{\text{ }}\ensuremath{\text{ }}\ensuremath{\text{ }}\ensuremath{\text{ }}\ensuremath{\text{ }}\ensuremath{\text{ }}\ensuremath{\text{ }}\ensuremath{\text{ }}\ensuremath{\text{ }}\ensuremath{\text{ }}\ensuremath{\text{ }}\ensuremath{\text{ }}\NormalTok{n:=n/2;\ensuremath{\text{ }}x:=x*x\ensuremath{\text{ }}\textbackslash{}OD;}\newline
\ensuremath{\text{ }}\ensuremath{\text{ }}\ensuremath{\text{ }}\ensuremath{\text{ }}\ensuremath{\text{ }}\ensuremath{\text{ }}\ensuremath{\text{ }}\ensuremath{\text{ }}\ensuremath{\text{ }}\ensuremath{\text{ }}\ensuremath{\text{ }}\ensuremath{\text{ }}\ensuremath{\text{ }}\NormalTok{\textbackslash{}\{\ensuremath{\text{ }}n>0\ensuremath{\text{ }}\textbackslash{}\};}\newline
\ensuremath{\text{ }}\ensuremath{\text{ }}\ensuremath{\text{ }}\ensuremath{\text{ }}\ensuremath{\text{ }}\ensuremath{\text{ }}\ensuremath{\text{ }}\ensuremath{\text{ }}\ensuremath{\text{ }}\ensuremath{\text{ }}\ensuremath{\text{ }}\ensuremath{\text{ }}\ensuremath{\text{ }}\NormalTok{n:=n-1;\ensuremath{\text{ }}z:=z*x\ensuremath{\text{ }}\textbackslash{}OD;}\newline
\ensuremath{\text{ }}\ensuremath{\text{ }}\ensuremath{\text{ }}\ensuremath{\text{ }}\ensuremath{\text{ }}\ensuremath{\text{ }}\ensuremath{\text{ }}\ensuremath{\text{ }}\ensuremath{\text{ }}\ensuremath{\text{ }}\NormalTok{|print|(z)\ensuremath{\text{ }}\textbackslash{}ENDPROC}\newline
\NormalTok{\textbackslash{}END}\newline
\NormalTok{\textbackslash{}end\{program\}}\newline
\end{Highlighting}
\end{Shaded}




\begin{minipage}{1.0\linewidth}
\begin{center}
\includegraphics[width=1.0\linewidth,height=6.5in,keepaspectratio]{../images/122.png}
\end{center}
\raggedright{}\myfigurewithoutcaption{122}
\end{minipage}\vspace{0.75cm}



The commands {\ttfamily \setmainfont[Path=/usr/share/fonts/truetype/cmu/,UprightFont=cmunrm.ttf,BoldFont=cmunbx.ttf,ItalicFont=cmunti.ttf,BoldItalicFont=cmunbi.ttf]{cmuntt.ttf}\setmonofont[Path=/usr/share/fonts/truetype/cmu/,UprightFont=cmuntt.ttf,BoldFont=cmuntb.ttf,ItalicFont=cmunit.ttf,BoldItalicFont=cmuntx.ttf]{cmuntt.ttf}\ttfamily \textbackslash{}(}{$\text{ }$}\setmainfont[Path=/usr/share/fonts/truetype/cmu/,UprightFont=cmunrm.ttf,BoldFont=cmunbx.ttf,ItalicFont=cmunti.ttf,BoldItalicFont=cmunbi.ttf]{cmunrm.ttf}\setmonofont[Path=/usr/share/fonts/truetype/cmu/,UprightFont=cmuntt.ttf,BoldFont=cmuntb.ttf,ItalicFont=cmunit.ttf,BoldItalicFont=cmuntx.ttf]{cmunrm.ttf} and {\ttfamily \setmainfont[Path=/usr/share/fonts/truetype/cmu/,UprightFont=cmunrm.ttf,BoldFont=cmunbx.ttf,ItalicFont=cmunti.ttf,BoldItalicFont=cmunbi.ttf]{cmuntt.ttf}\setmonofont[Path=/usr/share/fonts/truetype/cmu/,UprightFont=cmuntt.ttf,BoldFont=cmuntb.ttf,ItalicFont=cmunit.ttf,BoldItalicFont=cmuntx.ttf]{cmuntt.ttf}\ttfamily \textbackslash{})}{$\text{ }$}\setmainfont[Path=/usr/share/fonts/truetype/cmu/,UprightFont=cmunrm.ttf,BoldFont=cmunbx.ttf,ItalicFont=cmunti.ttf,BoldItalicFont=cmunbi.ttf]{cmunrm.ttf}\setmonofont[Path=/usr/share/fonts/truetype/cmu/,UprightFont=cmuntt.ttf,BoldFont=cmuntb.ttf,ItalicFont=cmunit.ttf,BoldItalicFont=cmuntx.ttf]{cmunrm.ttf} are redefined 
to typeset an algorithm in a minipage, so an algorithm
can appear as a single box in a formula. For example,
to state that a particular action system is equivalent
to a WHILE loop you can write:


\begin{Shaded}
\begin{Highlighting}[]

\NormalTok{\textbackslash{}[}\newline
\NormalTok{\textbackslash{}(\ensuremath{\text{ }}\textbackslash{}ACTIONS\ensuremath{\text{ }}A:}\newline
\ensuremath{\text{ }}\ensuremath{\text{ }}\ensuremath{\text{ }}\ensuremath{\text{ }}\ensuremath{\text{ }}\ensuremath{\text{ }}\ensuremath{\text{ }}\ensuremath{\text{ }}\NormalTok{A\ensuremath{\text{ }}\textbackslash{}EQ\ensuremath{\text{ }}\textbackslash{}IF\ensuremath{\text{ }}\textbackslash{}B\{\}\ensuremath{\text{ }}\textbackslash{}THEN\ensuremath{\text{ }}\textbackslash{}S\{\};\ensuremath{\text{ }}\textbackslash{}CALL\ensuremath{\text{ }}A}\newline
\ensuremath{\text{ }}\ensuremath{\text{ }}\ensuremath{\text{ }}\ensuremath{\text{ }}\ensuremath{\text{ }}\ensuremath{\text{ }}\ensuremath{\text{ }}\ensuremath{\text{ }}\ensuremath{\text{ }}\ensuremath{\text{ }}\ensuremath{\text{ }}\ensuremath{\text{ }}\ensuremath{\text{ }}\ensuremath{\text{ }}\ensuremath{\text{ }}\ensuremath{\text{ }}\ensuremath{\text{ }}\ensuremath{\text{ }}\ensuremath{\text{ }}\ensuremath{\text{ }}\ensuremath{\text{ }}\ensuremath{\text{ }}\ensuremath{\text{ }}\NormalTok{\textbackslash{}ELSE\ensuremath{\text{ }}\textbackslash{}CALL\ensuremath{\text{ }}Z\ensuremath{\text{ }}\textbackslash{}FI\ensuremath{\text{ }}\textbackslash{}QE}\newline
\ensuremath{\text{ }}\ensuremath{\text{ }}\ensuremath{\text{ }}\NormalTok{\textbackslash{}ENDACTIONS\ensuremath{\text{ }}}\AlertTok{\textbackslash{})}\newline
\NormalTok{\textbackslash{}EQT}\newline
\NormalTok{\textbackslash{}(\ensuremath{\text{ }}\textbackslash{}WHILE\ensuremath{\text{ }}\textbackslash{}B\{\}\ensuremath{\text{ }}\textbackslash{}DO\ensuremath{\text{ }}\textbackslash{}S\{\}\ensuremath{\text{ }}\textbackslash{}OD\ensuremath{\text{ }}}\AlertTok{\textbackslash{})}\newline
\NormalTok{\textbackslash{}]}\newline
\end{Highlighting}
\end{Shaded}


Dijkstra conditionals and loops:

\begin{Shaded}
\begin{Highlighting}[]

\NormalTok{\textbackslash{}begin\{program\}}\newline
\NormalTok{\textbackslash{}IF\ensuremath{\text{ }}x\ensuremath{\text{ }}=\ensuremath{\text{ }}1\ensuremath{\text{ }}\textbackslash{}AR\ensuremath{\text{ }}y:=y+1}\newline
\NormalTok{\textbackslash{}BAR\ensuremath{\text{ }}x\ensuremath{\text{ }}=\ensuremath{\text{ }}2\ensuremath{\text{ }}\textbackslash{}AR\ensuremath{\text{ }}y:=y^2}\newline
\NormalTok{\textbackslash{}utdots}\newline
\NormalTok{\textbackslash{}BAR\ensuremath{\text{ }}x\ensuremath{\text{ }}=\ensuremath{\text{ }}n\ensuremath{\text{ }}\textbackslash{}AR\ensuremath{\text{ }}y:=\textbackslash{}displaystyle\textbackslash{}sum_\{i=1\}^n\ensuremath{\text{ }}y_i\ensuremath{\text{ }}\textbackslash{}FI}\newline
\ensuremath{\text{ }}\newline
\NormalTok{\textbackslash{}DO\ensuremath{\text{ }}2\ensuremath{\text{ }}\textbackslash{}origbar\ensuremath{\text{ }}x\ensuremath{\text{ }}\textbackslash{}AND\ensuremath{\text{ }}x>0\ensuremath{\text{ }}\textbackslash{}AR\ensuremath{\text{ }}x:=\ensuremath{\text{ }}x/2}\newline
\NormalTok{\textbackslash{}BAR\ensuremath{\text{ }}\textbackslash{}NOT\ensuremath{\text{ }}2\ensuremath{\text{ }}\textbackslash{}origbar\ensuremath{\text{ }}x\ensuremath{\text{ }}\ensuremath{\text{ }}\ensuremath{\text{ }}\ensuremath{\text{ }}\textbackslash{}AR\ensuremath{\text{ }}x:=\ensuremath{\text{ }}\textbackslash{}modbar\{x+3\}\ensuremath{\text{ }}\textbackslash{}OD}\newline
\NormalTok{\textbackslash{}end\{program\}}\newline
\end{Highlighting}
\end{Shaded}


Loops with multiple exits:

\begin{Shaded}
\begin{Highlighting}[]

\NormalTok{\textbackslash{}begin\{program\}\ensuremath{\text{ }}}\newline
\NormalTok{\textbackslash{}DO\ensuremath{\text{ }}\textbackslash{}DO\ensuremath{\text{ }}\textbackslash{}IF\ensuremath{\text{ }}\textbackslash{}B1\ensuremath{\text{ }}\textbackslash{}THEN\ensuremath{\text{ }}\textbackslash{}EXIT\ensuremath{\text{ }}\textbackslash{}FI;}\newline
\ensuremath{\text{ }}\ensuremath{\text{ }}\ensuremath{\text{ }}\ensuremath{\text{ }}\ensuremath{\text{ }}\ensuremath{\text{ }}\ensuremath{\text{ }}\ensuremath{\text{ }}\NormalTok{\textbackslash{}S1;}\newline
\ensuremath{\text{ }}\ensuremath{\text{ }}\ensuremath{\text{ }}\ensuremath{\text{ }}\ensuremath{\text{ }}\ensuremath{\text{ }}\ensuremath{\text{ }}\ensuremath{\text{ }}\NormalTok{\textbackslash{}IF\ensuremath{\text{ }}\textbackslash{}B2\ensuremath{\text{ }}\textbackslash{}THEN\ensuremath{\text{ }}\textbackslash{}EXIT(2)\ensuremath{\text{ }}\textbackslash{}FI\ensuremath{\text{ }}\textbackslash{}OD;}\newline
\ensuremath{\text{ }}\ensuremath{\text{ }}\ensuremath{\text{ }}\ensuremath{\text{ }}\NormalTok{\textbackslash{}IF\ensuremath{\text{ }}\textbackslash{}B1\ensuremath{\text{ }}\textbackslash{}THEN\ensuremath{\text{ }}\textbackslash{}EXIT\ensuremath{\text{ }}\textbackslash{}FI\ensuremath{\text{ }}\textbackslash{}OD}\newline
\NormalTok{\textbackslash{}end\{program\}\ensuremath{\text{ }}}\newline
\end{Highlighting}
\end{Shaded}


A Reverse Engineering Example.

Here\textquotesingle{}s the original program:

\begin{Shaded}
\begin{Highlighting}[]

\NormalTok{\textbackslash{}begin\{program\}\ensuremath{\text{ }}}\newline
\ensuremath{\text{ }}\NormalTok{\textbackslash{}VAR\ensuremath{\text{ }}\textbackslash{}seq\{m\ensuremath{\text{ }}:=\ensuremath{\text{ }}0,\ensuremath{\text{ }}p\ensuremath{\text{ }}:=\ensuremath{\text{ }}0,\ensuremath{\text{ }}|last|\ensuremath{\text{ }}:=\ensuremath{\text{ }}``\ensuremath{\text{ }}\textquotesingle{}\textquotesingle{}\};\ensuremath{\text{ }}}\newline
\ensuremath{\text{ }}\NormalTok{\textbackslash{}ACTIONS\ensuremath{\text{ }}|prog|:\ensuremath{\text{ }}}\newline
\NormalTok{|prog|\ensuremath{\text{ }}\textbackslash{}ACTIONEQ\ensuremath{\text{ }}}\CommentTok{\%}\newline
\ensuremath{\text{ }}\ensuremath{\text{ }}\ensuremath{\text{ }}\ensuremath{\text{ }}\NormalTok{\textbackslash{}seq\{|line|\ensuremath{\text{ }}:=\ensuremath{\text{ }}``\ensuremath{\text{ }}\textquotesingle{}\textquotesingle{},\ensuremath{\text{ }}m\ensuremath{\text{ }}:=\ensuremath{\text{ }}0,\ensuremath{\text{ }}i\ensuremath{\text{ }}:=\ensuremath{\text{ }}1\};}\newline
\ensuremath{\text{ }}\ensuremath{\text{ }}\ensuremath{\text{ }}\ensuremath{\text{ }}\NormalTok{\textbackslash{}CALL\ensuremath{\text{ }}|inhere|\ensuremath{\text{ }}\textbackslash{}ENDACTION}\newline
\NormalTok{l\ensuremath{\text{ }}\textbackslash{}ACTIONEQ\ensuremath{\text{ }}}\CommentTok{\%}\newline
\ensuremath{\text{ }}\ensuremath{\text{ }}\ensuremath{\text{ }}\ensuremath{\text{ }}\NormalTok{i\ensuremath{\text{ }}:=\ensuremath{\text{ }}i+1;\ensuremath{\text{ }}}\newline
\ensuremath{\text{ }}\ensuremath{\text{ }}\ensuremath{\text{ }}\ensuremath{\text{ }}\NormalTok{\textbackslash{}IF\ensuremath{\text{ }}(i=(n+1))\ensuremath{\text{ }}\textbackslash{}THEN\ensuremath{\text{ }}\textbackslash{}CALL\ensuremath{\text{ }}|alldone|\ensuremath{\text{ }}\textbackslash{}FI\ensuremath{\text{ }};\ensuremath{\text{ }}}\newline
\ensuremath{\text{ }}\ensuremath{\text{ }}\ensuremath{\text{ }}\ensuremath{\text{ }}\NormalTok{m\ensuremath{\text{ }}:=\ensuremath{\text{ }}1;\ensuremath{\text{ }}}\newline
\ensuremath{\text{ }}\ensuremath{\text{ }}\ensuremath{\text{ }}\ensuremath{\text{ }}\NormalTok{\textbackslash{}IF\ensuremath{\text{ }}|item|[i]\ensuremath{\text{ }}\textbackslash{}neq\ensuremath{\text{ }}|last|}\newline
\ensuremath{\text{ }}\ensuremath{\text{ }}\ensuremath{\text{ }}\ensuremath{\text{ }}\ensuremath{\text{ }}\ensuremath{\text{ }}\ensuremath{\text{ }}\ensuremath{\text{ }}\NormalTok{\textbackslash{}THEN\ensuremath{\text{ }}|write|(|line|);\ensuremath{\text{ }}|line|\ensuremath{\text{ }}:=\ensuremath{\text{ }}``\ensuremath{\text{ }}\textquotesingle{}\textquotesingle{};\ensuremath{\text{ }}m\ensuremath{\text{ }}:=\ensuremath{\text{ }}0;}\newline
\ensuremath{\text{ }}\ensuremath{\text{ }}\ensuremath{\text{ }}\ensuremath{\text{ }}\ensuremath{\text{ }}\ensuremath{\text{ }}\ensuremath{\text{ }}\ensuremath{\text{ }}\ensuremath{\text{ }}\ensuremath{\text{ }}\ensuremath{\text{ }}\ensuremath{\text{ }}\ensuremath{\text{ }}\ensuremath{\text{ }}\NormalTok{\textbackslash{}CALL\ensuremath{\text{ }}|inhere|\ensuremath{\text{ }}\textbackslash{}FI\ensuremath{\text{ }};\ensuremath{\text{ }}}\newline
\ensuremath{\text{ }}\ensuremath{\text{ }}\ensuremath{\text{ }}\ensuremath{\text{ }}\NormalTok{\textbackslash{}CALL\ensuremath{\text{ }}|more|\ensuremath{\text{ }}\textbackslash{}ENDACTION}\newline
\NormalTok{|inhere|\ensuremath{\text{ }}\textbackslash{}ACTIONEQ\ensuremath{\text{ }}}\CommentTok{\%}\newline
\ensuremath{\text{ }}\ensuremath{\text{ }}\ensuremath{\text{ }}\ensuremath{\text{ }}\NormalTok{p\ensuremath{\text{ }}:=\ensuremath{\text{ }}|number|[i];\ensuremath{\text{ }}|line|\ensuremath{\text{ }}:=\ensuremath{\text{ }}|item|[i];}\newline
\ensuremath{\text{ }}\ensuremath{\text{ }}\ensuremath{\text{ }}\ensuremath{\text{ }}\NormalTok{|line|\ensuremath{\text{ }}:=\ensuremath{\text{ }}|line|\ensuremath{\text{ }}\textbackslash{}concat\ensuremath{\text{ }}``\ensuremath{\text{ }}\textquotesingle{}\textquotesingle{}\ensuremath{\text{ }}\textbackslash{}concat\ensuremath{\text{ }}p;}\newline
\ensuremath{\text{ }}\ensuremath{\text{ }}\ensuremath{\text{ }}\ensuremath{\text{ }}\NormalTok{\textbackslash{}CALL\ensuremath{\text{ }}|more|\ensuremath{\text{ }}\textbackslash{}ENDACTION}\newline
\NormalTok{|more|\ensuremath{\text{ }}\textbackslash{}ACTIONEQ\ensuremath{\text{ }}}\CommentTok{\%}\newline
\ensuremath{\text{ }}\ensuremath{\text{ }}\ensuremath{\text{ }}\ensuremath{\text{ }}\NormalTok{\textbackslash{}IF\ensuremath{\text{ }}(m=1)\ensuremath{\text{ }}\textbackslash{}THEN\ensuremath{\text{ }}p\ensuremath{\text{ }}:=\ensuremath{\text{ }}|number|[i];}\newline
\ensuremath{\text{ }}\ensuremath{\text{ }}\ensuremath{\text{ }}\ensuremath{\text{ }}\NormalTok{|line|\ensuremath{\text{ }}:=\ensuremath{\text{ }}|line|\ensuremath{\text{ }}\textbackslash{}concat\ensuremath{\text{ }}``,\ensuremath{\text{ }}\textquotesingle{}\textquotesingle{}\ensuremath{\text{ }}\textbackslash{}concat\ensuremath{\text{ }}p\ensuremath{\text{ }}\textbackslash{}FI\ensuremath{\text{ }};\ensuremath{\text{ }}}\newline
\ensuremath{\text{ }}\ensuremath{\text{ }}\ensuremath{\text{ }}\ensuremath{\text{ }}\NormalTok{|last|\ensuremath{\text{ }}:=\ensuremath{\text{ }}|item|[i];\ensuremath{\text{ }}}\newline
\ensuremath{\text{ }}\ensuremath{\text{ }}\ensuremath{\text{ }}\ensuremath{\text{ }}\NormalTok{\textbackslash{}CALL\ensuremath{\text{ }}l\ensuremath{\text{ }}\ensuremath{\text{ }}\textbackslash{}ENDACTION\ensuremath{\text{ }}\ensuremath{\text{ }}}\newline
\NormalTok{|alldone|\ensuremath{\text{ }}\textbackslash{}ACTIONEQ\ensuremath{\text{ }}|write|(|line|);\ensuremath{\text{ }}\textbackslash{}CALL\ensuremath{\text{ }}Z\ensuremath{\text{ }}\textbackslash{}ENDACTION\ensuremath{\text{ }}\textbackslash{}ENDACTIONS\ensuremath{\text{ }}\textbackslash{}END\ensuremath{\text{ }}}\newline
\NormalTok{\textbackslash{}end\{program\}\ensuremath{\text{ }}}\newline
\end{Highlighting}
\end{Shaded}


And here\textquotesingle{}s the transformed and corrected version:

\begin{Shaded}
\begin{Highlighting}[]

\NormalTok{\textbackslash{}begin\{program\}\ensuremath{\text{ }}}\newline
\NormalTok{\textbackslash{}seq\{|line|\ensuremath{\text{ }}:=\ensuremath{\text{ }}``\ensuremath{\text{ }}\textquotesingle{}\textquotesingle{},\ensuremath{\text{ }}i\ensuremath{\text{ }}:=\ensuremath{\text{ }}1\};}\newline
\NormalTok{\textbackslash{}WHILE\ensuremath{\text{ }}i\ensuremath{\text{ }}\textbackslash{}neq\ensuremath{\text{ }}n+1\ensuremath{\text{ }}\textbackslash{}DO\ensuremath{\text{ }}}\newline
\ensuremath{\text{ }}\ensuremath{\text{ }}\NormalTok{|line|\ensuremath{\text{ }}:=\ensuremath{\text{ }}|item|[i]\ensuremath{\text{ }}\textbackslash{}concat\ensuremath{\text{ }}``\ensuremath{\text{ }}\textquotesingle{}\textquotesingle{}\ensuremath{\text{ }}\textbackslash{}concat\ensuremath{\text{ }}|number|[i];\ensuremath{\text{ }}}\newline
\ensuremath{\text{ }}\ensuremath{\text{ }}\NormalTok{i\ensuremath{\text{ }}:=\ensuremath{\text{ }}i+1;\ensuremath{\text{ }}}\newline
\ensuremath{\text{ }}\ensuremath{\text{ }}\NormalTok{\textbackslash{}WHILE\ensuremath{\text{ }}i\ensuremath{\text{ }}\textbackslash{}neq\ensuremath{\text{ }}n+1\ensuremath{\text{ }}\textbackslash{}AND\ensuremath{\text{ }}|item|[i]\ensuremath{\text{ }}=\ensuremath{\text{ }}|item|[i-1]\ensuremath{\text{ }}\textbackslash{}DO\ensuremath{\text{ }}}\newline
\ensuremath{\text{ }}\ensuremath{\text{ }}\ensuremath{\text{ }}\ensuremath{\text{ }}\NormalTok{|line|\ensuremath{\text{ }}:=\ensuremath{\text{ }}|line|\ensuremath{\text{ }}\textbackslash{}concat\ensuremath{\text{ }}``,\ensuremath{\text{ }}\textquotesingle{}\textquotesingle{}\ensuremath{\text{ }}\textbackslash{}concat\ensuremath{\text{ }}|number|[i]);}\newline
\ensuremath{\text{ }}\ensuremath{\text{ }}\ensuremath{\text{ }}\ensuremath{\text{ }}\NormalTok{i\ensuremath{\text{ }}:=\ensuremath{\text{ }}i+1\ensuremath{\text{ }}\textbackslash{}OD\ensuremath{\text{ }};\ensuremath{\text{ }}}\newline
\ensuremath{\text{ }}\ensuremath{\text{ }}\NormalTok{|write|(|line|)\ensuremath{\text{ }}\textbackslash{}OD\ensuremath{\text{ }}}\newline
\NormalTok{\textbackslash{}end\{program\}}\newline
\end{Highlighting}
\end{Shaded}


The package also provides a macro for typesetting a set
like this: {\ttfamily \setmainfont[Path=/usr/share/fonts/truetype/cmu/,UprightFont=cmunrm.ttf,BoldFont=cmunbx.ttf,ItalicFont=cmunti.ttf,BoldItalicFont=cmunbi.ttf]{cmuntt.ttf}\setmonofont[Path=/usr/share/fonts/truetype/cmu/,UprightFont=cmuntt.ttf,BoldFont=cmuntb.ttf,ItalicFont=cmunit.ttf,BoldItalicFont=cmuntx.ttf]{cmuntt.ttf}\ttfamily \textbackslash{}set\{x \textbackslash{}in N | x >{} 0\}}\setmainfont[Path=/usr/share/fonts/truetype/cmu/,UprightFont=cmunrm.ttf,BoldFont=cmunbx.ttf,ItalicFont=cmunti.ttf,BoldItalicFont=cmunbi.ttf]{cmunrm.ttf}\setmonofont[Path=/usr/share/fonts/truetype/cmu/,UprightFont=cmuntt.ttf,BoldFont=cmuntb.ttf,ItalicFont=cmunit.ttf,BoldItalicFont=cmuntx.ttf]{cmunrm.ttf}.

Lines can be numbered by setting {\ttfamily \setmainfont[Path=/usr/share/fonts/truetype/cmu/,UprightFont=cmunrm.ttf,BoldFont=cmunbx.ttf,ItalicFont=cmunti.ttf,BoldItalicFont=cmunbi.ttf]{cmuntt.ttf}\setmonofont[Path=/usr/share/fonts/truetype/cmu/,UprightFont=cmuntt.ttf,BoldFont=cmuntb.ttf,ItalicFont=cmunit.ttf,BoldItalicFont=cmuntx.ttf]{cmuntt.ttf}\ttfamily \textbackslash{}NumberProgramstrue}\setmainfont[Path=/usr/share/fonts/truetype/cmu/,UprightFont=cmunrm.ttf,BoldFont=cmunbx.ttf,ItalicFont=cmunti.ttf,BoldItalicFont=cmunbi.ttf]{cmunrm.ttf}\setmonofont[Path=/usr/share/fonts/truetype/cmu/,UprightFont=cmuntt.ttf,BoldFont=cmuntb.ttf,ItalicFont=cmunit.ttf,BoldItalicFont=cmuntx.ttf]{cmunrm.ttf}
and numbering turned off with {\ttfamily \setmainfont[Path=/usr/share/fonts/truetype/cmu/,UprightFont=cmunrm.ttf,BoldFont=cmunbx.ttf,ItalicFont=cmunti.ttf,BoldItalicFont=cmunbi.ttf]{cmuntt.ttf}\setmonofont[Path=/usr/share/fonts/truetype/cmu/,UprightFont=cmuntt.ttf,BoldFont=cmuntb.ttf,ItalicFont=cmunit.ttf,BoldItalicFont=cmuntx.ttf]{cmuntt.ttf}\ttfamily \textbackslash{}NumberProgramsfalse}\setmainfont[Path=/usr/share/fonts/truetype/cmu/,UprightFont=cmunrm.ttf,BoldFont=cmunbx.ttf,ItalicFont=cmunti.ttf,BoldItalicFont=cmunbi.ttf]{cmunrm.ttf}\setmonofont[Path=/usr/share/fonts/truetype/cmu/,UprightFont=cmuntt.ttf,BoldFont=cmuntb.ttf,ItalicFont=cmunit.ttf,BoldItalicFont=cmuntx.ttf]{cmunrm.ttf}

\myhref{http://www.ctan.org/pkg/program}{Package page}

\myhref{http://mirror.ctan.org/macros/latex/contrib/program/program-doc.pdf}{Package documentation}
\section{The {\ttfamily \setmainfont[Path=/usr/share/fonts/truetype/cmu/,UprightFont=cmunrm.ttf,BoldFont=cmunbx.ttf,ItalicFont=cmunti.ttf,BoldItalicFont=cmunbi.ttf]{cmuntt.ttf}\setmonofont[Path=/usr/share/fonts/truetype/cmu/,UprightFont=cmuntt.ttf,BoldFont=cmuntb.ttf,ItalicFont=cmunit.ttf,BoldItalicFont=cmuntx.ttf]{cmuntt.ttf}\ttfamily algorithm}{$\text{ }$}\setmainfont[Path=/usr/share/fonts/truetype/cmu/,UprightFont=cmunrm.ttf,BoldFont=cmunbx.ttf,ItalicFont=cmunti.ttf,BoldItalicFont=cmunbi.ttf]{cmunrm.ttf}\setmonofont[Path=/usr/share/fonts/truetype/cmu/,UprightFont=cmuntt.ttf,BoldFont=cmuntb.ttf,ItalicFont=cmunit.ttf,BoldItalicFont=cmuntx.ttf]{cmunrm.ttf} environment}
\label{587}

It is often useful for the algorithm produced by {\ttfamily \setmainfont[Path=/usr/share/fonts/truetype/cmu/,UprightFont=cmunrm.ttf,BoldFont=cmunbx.ttf,ItalicFont=cmunti.ttf,BoldItalicFont=cmunbi.ttf]{cmuntt.ttf}\setmonofont[Path=/usr/share/fonts/truetype/cmu/,UprightFont=cmuntt.ttf,BoldFont=cmuntb.ttf,ItalicFont=cmunit.ttf,BoldItalicFont=cmuntx.ttf]{cmuntt.ttf}\ttfamily algorithmic}{$\text{ }$}\setmainfont[Path=/usr/share/fonts/truetype/cmu/,UprightFont=cmunrm.ttf,BoldFont=cmunbx.ttf,ItalicFont=cmunti.ttf,BoldItalicFont=cmunbi.ttf]{cmunrm.ttf}\setmonofont[Path=/usr/share/fonts/truetype/cmu/,UprightFont=cmuntt.ttf,BoldFont=cmuntb.ttf,ItalicFont=cmunit.ttf,BoldItalicFont=cmuntx.ttf]{cmunrm.ttf} to be \symbol{34}floated\symbol{34} to the optimal point
in the document to avoid it being split across pages. The {\ttfamily \setmainfont[Path=/usr/share/fonts/truetype/cmu/,UprightFont=cmunrm.ttf,BoldFont=cmunbx.ttf,ItalicFont=cmunti.ttf,BoldItalicFont=cmunbi.ttf]{cmuntt.ttf}\setmonofont[Path=/usr/share/fonts/truetype/cmu/,UprightFont=cmuntt.ttf,BoldFont=cmuntb.ttf,ItalicFont=cmunit.ttf,BoldItalicFont=cmuntx.ttf]{cmuntt.ttf}\ttfamily algorithm}{$\text{ }$}\setmainfont[Path=/usr/share/fonts/truetype/cmu/,UprightFont=cmunrm.ttf,BoldFont=cmunbx.ttf,ItalicFont=cmunti.ttf,BoldItalicFont=cmunbi.ttf]{cmunrm.ttf}\setmonofont[Path=/usr/share/fonts/truetype/cmu/,UprightFont=cmuntt.ttf,BoldFont=cmuntb.ttf,ItalicFont=cmunit.ttf,BoldItalicFont=cmuntx.ttf]{cmunrm.ttf} environment provides this and a few other useful features. Include it by adding the $\text{ }$\newline{}

{\ttfamily \setmainfont[Path=/usr/share/fonts/truetype/cmu/,UprightFont=cmunrm.ttf,BoldFont=cmunbx.ttf,ItalicFont=cmunti.ttf,BoldItalicFont=cmunbi.ttf]{cmuntt.ttf}\setmonofont[Path=/usr/share/fonts/truetype/cmu/,UprightFont=cmuntt.ttf,BoldFont=cmuntb.ttf,ItalicFont=cmunit.ttf,BoldItalicFont=cmuntx.ttf]{cmuntt.ttf}\ttfamily \textbackslash{}usepackage\{algorithm\}}\setmainfont[Path=/usr/share/fonts/truetype/cmu/,UprightFont=cmunrm.ttf,BoldFont=cmunbx.ttf,ItalicFont=cmunti.ttf,BoldItalicFont=cmunbi.ttf]{cmunrm.ttf}\setmonofont[Path=/usr/share/fonts/truetype/cmu/,UprightFont=cmuntt.ttf,BoldFont=cmuntb.ttf,ItalicFont=cmunit.ttf,BoldItalicFont=cmuntx.ttf]{cmunrm.ttf}
to your document\textquotesingle{}s preamble. It is entered into by

\begin{Shaded}
\begin{Highlighting}[]

\NormalTok{\textbackslash{}begin\{algorithm\}}\newline
\NormalTok{\textbackslash{}caption\{<your\ensuremath{\text{ }}caption\ensuremath{\text{ }}for\ensuremath{\text{ }}this\ensuremath{\text{ }}algorithm>\}}\newline
\NormalTok{\textbackslash{}label\{<your\ensuremath{\text{ }}label\ensuremath{\text{ }}for\ensuremath{\text{ }}references\ensuremath{\text{ }}later\ensuremath{\text{ }}in\ensuremath{\text{ }}your\ensuremath{\text{ }}document>\}}\newline
\NormalTok{\textbackslash{}begin\{algorithmic\}}\newline
\NormalTok{<algorithmic\ensuremath{\text{ }}environment>}\newline
\NormalTok{\textbackslash{}end\{algorithmic\}}\newline
\NormalTok{\textbackslash{}end\{algorithm\}}\newline
\end{Highlighting}
\end{Shaded}

\subsection{Algorithm numbering}
\label{588}
The default numbering system for the {\ttfamily \setmainfont[Path=/usr/share/fonts/truetype/cmu/,UprightFont=cmunrm.ttf,BoldFont=cmunbx.ttf,ItalicFont=cmunti.ttf,BoldItalicFont=cmunbi.ttf]{cmuntt.ttf}\setmonofont[Path=/usr/share/fonts/truetype/cmu/,UprightFont=cmuntt.ttf,BoldFont=cmuntb.ttf,ItalicFont=cmunit.ttf,BoldItalicFont=cmuntx.ttf]{cmuntt.ttf}\ttfamily algorithm}{$\text{ }$}\setmainfont[Path=/usr/share/fonts/truetype/cmu/,UprightFont=cmunrm.ttf,BoldFont=cmunbx.ttf,ItalicFont=cmunti.ttf,BoldItalicFont=cmunbi.ttf]{cmunrm.ttf}\setmonofont[Path=/usr/share/fonts/truetype/cmu/,UprightFont=cmuntt.ttf,BoldFont=cmuntb.ttf,ItalicFont=cmunit.ttf,BoldItalicFont=cmuntx.ttf]{cmunrm.ttf} package is to number algorithms sequentially. This is often not desirable, particularly in large documents where numbering according to chapter is more appropriate. The numbering of algorithms can be influenced by providing the name of the document component within which numbering should be recommenced. The legal values for this option are: part, chapter, section, subsection, subsubsection or nothing (default). For example:

\begin{Shaded}
\begin{Highlighting}[]

\NormalTok{\textbackslash{}usepackage[chapter]\{algorithm\}}\newline
\end{Highlighting}
\end{Shaded}

\subsection{List of algorithms}
\label{589}

When you use figures or tables, you can add a list of them close to the table of contents; the {\ttfamily \setmainfont[Path=/usr/share/fonts/truetype/cmu/,UprightFont=cmunrm.ttf,BoldFont=cmunbx.ttf,ItalicFont=cmunti.ttf,BoldItalicFont=cmunbi.ttf]{cmuntt.ttf}\setmonofont[Path=/usr/share/fonts/truetype/cmu/,UprightFont=cmuntt.ttf,BoldFont=cmuntb.ttf,ItalicFont=cmunit.ttf,BoldItalicFont=cmuntx.ttf]{cmuntt.ttf}\ttfamily algorithm}{$\text{ }$}\setmainfont[Path=/usr/share/fonts/truetype/cmu/,UprightFont=cmunrm.ttf,BoldFont=cmunbx.ttf,ItalicFont=cmunti.ttf,BoldItalicFont=cmunbi.ttf]{cmunrm.ttf}\setmonofont[Path=/usr/share/fonts/truetype/cmu/,UprightFont=cmuntt.ttf,BoldFont=cmuntb.ttf,ItalicFont=cmunit.ttf,BoldItalicFont=cmuntx.ttf]{cmunrm.ttf} package provides a similar command. Just put

\begin{Shaded}
\begin{Highlighting}[]

\NormalTok{\textbackslash{}listofalgorithms}\newline
\end{Highlighting}
\end{Shaded}

anywhere in the document, and LaTeX will print a list of the \symbol{34}algorithm\symbol{34} environments in the document with the corresponding page and the caption.
\subsection{An example from the manual}
\label{590}
This is an example taken from the manual (\mylref{586}{official manual, p.14})

\begin{Shaded}
\begin{Highlighting}[]

\NormalTok{\textbackslash{}begin\{algorithm\}\ensuremath{\text{ }}\ensuremath{\text{ }}\ensuremath{\text{ }}\ensuremath{\text{ }}\ensuremath{\text{ }}\ensuremath{\text{ }}\ensuremath{\text{ }}\ensuremath{\text{ }}\ensuremath{\text{ }}\ensuremath{\text{ }}\ensuremath{\text{ }}\ensuremath{\text{ }}\ensuremath{\text{ }}\ensuremath{\text{ }}\ensuremath{\text{ }}\ensuremath{\text{ }}\ensuremath{\text{ }}\ensuremath{\text{ }}\ensuremath{\text{ }}\ensuremath{\text{ }}\ensuremath{\text{ }}\ensuremath{\text{ }}}\CommentTok{\%\ensuremath{\text{ }}enter\ensuremath{\text{ }}the\ensuremath{\text{ }}algorithm\ensuremath{\text{ }}environment}\newline
\NormalTok{\textbackslash{}caption\{Calculate\ensuremath{\text{ }}\$y\ensuremath{\text{ }}=\ensuremath{\text{ }}x^n\$\}\ensuremath{\text{ }}\ensuremath{\text{ }}\ensuremath{\text{ }}\ensuremath{\text{ }}\ensuremath{\text{ }}\ensuremath{\text{ }}\ensuremath{\text{ }}\ensuremath{\text{ }}\ensuremath{\text{ }}\ensuremath{\text{ }}}\CommentTok{\%\ensuremath{\text{ }}give\ensuremath{\text{ }}the\ensuremath{\text{ }}algorithm\ensuremath{\text{ }}a\ensuremath{\text{ }}caption}\newline
\NormalTok{\textbackslash{}label\{alg1\}\ensuremath{\text{ }}\ensuremath{\text{ }}\ensuremath{\text{ }}\ensuremath{\text{ }}\ensuremath{\text{ }}\ensuremath{\text{ }}\ensuremath{\text{ }}\ensuremath{\text{ }}\ensuremath{\text{ }}\ensuremath{\text{ }}\ensuremath{\text{ }}\ensuremath{\text{ }}\ensuremath{\text{ }}\ensuremath{\text{ }}\ensuremath{\text{ }}\ensuremath{\text{ }}\ensuremath{\text{ }}\ensuremath{\text{ }}\ensuremath{\text{ }}\ensuremath{\text{ }}\ensuremath{\text{ }}\ensuremath{\text{ }}\ensuremath{\text{ }}\ensuremath{\text{ }}\ensuremath{\text{ }}\ensuremath{\text{ }}\ensuremath{\text{ }}}\CommentTok{\%\ensuremath{\text{ }}and\ensuremath{\text{ }}a\ensuremath{\text{ }}label\ensuremath{\text{ }}for\ensuremath{\text{ }}\textbackslash{}ref\{\}\ensuremath{\text{ }}commands\ensuremath{\text{ }}later}\newline
\ensuremath{\text{ }}\NormalTok{in\ensuremath{\text{ }}the\ensuremath{\text{ }}document}\newline
\NormalTok{\textbackslash{}begin\{algorithmic\}\ensuremath{\text{ }}\ensuremath{\text{ }}\ensuremath{\text{ }}\ensuremath{\text{ }}\ensuremath{\text{ }}\ensuremath{\text{ }}\ensuremath{\text{ }}\ensuremath{\text{ }}\ensuremath{\text{ }}\ensuremath{\text{ }}\ensuremath{\text{ }}\ensuremath{\text{ }}\ensuremath{\text{ }}\ensuremath{\text{ }}\ensuremath{\text{ }}\ensuremath{\text{ }}\ensuremath{\text{ }}\ensuremath{\text{ }}\ensuremath{\text{ }}\ensuremath{\text{ }}}\CommentTok{\%\ensuremath{\text{ }}enter\ensuremath{\text{ }}the\ensuremath{\text{ }}algorithmic\ensuremath{\text{ }}environment}\newline
\ensuremath{\text{ }}\ensuremath{\text{ }}\ensuremath{\text{ }}\ensuremath{\text{ }}\NormalTok{\textbackslash{}REQUIRE\ensuremath{\text{ }}\$n\ensuremath{\text{ }}\textbackslash{}geq\ensuremath{\text{ }}0\ensuremath{\text{ }}\textbackslash{}vee\ensuremath{\text{ }}x\ensuremath{\text{ }}\textbackslash{}neq\ensuremath{\text{ }}0\$}\newline
\ensuremath{\text{ }}\ensuremath{\text{ }}\ensuremath{\text{ }}\ensuremath{\text{ }}\NormalTok{\textbackslash{}ENSURE\ensuremath{\text{ }}\$y\ensuremath{\text{ }}=\ensuremath{\text{ }}x^n\$}\newline
\ensuremath{\text{ }}\ensuremath{\text{ }}\ensuremath{\text{ }}\ensuremath{\text{ }}\NormalTok{\textbackslash{}STATE\ensuremath{\text{ }}\$y\ensuremath{\text{ }}\textbackslash{}Leftarrow\ensuremath{\text{ }}1\$}\newline
\ensuremath{\text{ }}\ensuremath{\text{ }}\ensuremath{\text{ }}\ensuremath{\text{ }}\NormalTok{\textbackslash{}IF\{\$n\ensuremath{\text{ }}<\ensuremath{\text{ }}0\$\}}\newline
\ensuremath{\text{ }}\ensuremath{\text{ }}\ensuremath{\text{ }}\ensuremath{\text{ }}\ensuremath{\text{ }}\ensuremath{\text{ }}\ensuremath{\text{ }}\ensuremath{\text{ }}\NormalTok{\textbackslash{}STATE\ensuremath{\text{ }}\$X\ensuremath{\text{ }}\textbackslash{}Leftarrow\ensuremath{\text{ }}1\ensuremath{\text{ }}/\ensuremath{\text{ }}x\$}\newline
\ensuremath{\text{ }}\ensuremath{\text{ }}\ensuremath{\text{ }}\ensuremath{\text{ }}\ensuremath{\text{ }}\ensuremath{\text{ }}\ensuremath{\text{ }}\ensuremath{\text{ }}\NormalTok{\textbackslash{}STATE\ensuremath{\text{ }}\$N\ensuremath{\text{ }}\textbackslash{}Leftarrow\ensuremath{\text{ }}-n\$}\newline
\ensuremath{\text{ }}\ensuremath{\text{ }}\ensuremath{\text{ }}\ensuremath{\text{ }}\NormalTok{\textbackslash{}ELSE}\newline
\ensuremath{\text{ }}\ensuremath{\text{ }}\ensuremath{\text{ }}\ensuremath{\text{ }}\ensuremath{\text{ }}\ensuremath{\text{ }}\ensuremath{\text{ }}\ensuremath{\text{ }}\NormalTok{\textbackslash{}STATE\ensuremath{\text{ }}\$X\ensuremath{\text{ }}\textbackslash{}Leftarrow\ensuremath{\text{ }}x\$}\newline
\ensuremath{\text{ }}\ensuremath{\text{ }}\ensuremath{\text{ }}\ensuremath{\text{ }}\ensuremath{\text{ }}\ensuremath{\text{ }}\ensuremath{\text{ }}\ensuremath{\text{ }}\NormalTok{\textbackslash{}STATE\ensuremath{\text{ }}\$N\ensuremath{\text{ }}\textbackslash{}Leftarrow\ensuremath{\text{ }}n\$}\newline
\ensuremath{\text{ }}\ensuremath{\text{ }}\ensuremath{\text{ }}\ensuremath{\text{ }}\NormalTok{\textbackslash{}ENDIF}\newline
\ensuremath{\text{ }}\ensuremath{\text{ }}\ensuremath{\text{ }}\ensuremath{\text{ }}\NormalTok{\textbackslash{}WHILE\{\$N\ensuremath{\text{ }}\textbackslash{}neq\ensuremath{\text{ }}0\$\}}\newline
\ensuremath{\text{ }}\ensuremath{\text{ }}\ensuremath{\text{ }}\ensuremath{\text{ }}\ensuremath{\text{ }}\ensuremath{\text{ }}\ensuremath{\text{ }}\ensuremath{\text{ }}\NormalTok{\textbackslash{}IF\{\$N\$\ensuremath{\text{ }}is\ensuremath{\text{ }}even\}}\newline
\ensuremath{\text{ }}\ensuremath{\text{ }}\ensuremath{\text{ }}\ensuremath{\text{ }}\ensuremath{\text{ }}\ensuremath{\text{ }}\ensuremath{\text{ }}\ensuremath{\text{ }}\ensuremath{\text{ }}\ensuremath{\text{ }}\ensuremath{\text{ }}\ensuremath{\text{ }}\NormalTok{\textbackslash{}STATE\ensuremath{\text{ }}\$X\ensuremath{\text{ }}\textbackslash{}Leftarrow\ensuremath{\text{ }}X\ensuremath{\text{ }}\textbackslash{}times\ensuremath{\text{ }}X\$}\newline
\ensuremath{\text{ }}\ensuremath{\text{ }}\ensuremath{\text{ }}\ensuremath{\text{ }}\ensuremath{\text{ }}\ensuremath{\text{ }}\ensuremath{\text{ }}\ensuremath{\text{ }}\ensuremath{\text{ }}\ensuremath{\text{ }}\ensuremath{\text{ }}\ensuremath{\text{ }}\NormalTok{\textbackslash{}STATE\ensuremath{\text{ }}\$N\ensuremath{\text{ }}\textbackslash{}Leftarrow\ensuremath{\text{ }}N\ensuremath{\text{ }}/\ensuremath{\text{ }}2\$}\newline
\ensuremath{\text{ }}\ensuremath{\text{ }}\ensuremath{\text{ }}\ensuremath{\text{ }}\ensuremath{\text{ }}\ensuremath{\text{ }}\ensuremath{\text{ }}\ensuremath{\text{ }}\NormalTok{\textbackslash{}ELSE[\$N\$\ensuremath{\text{ }}is\ensuremath{\text{ }}odd]}\newline
\ensuremath{\text{ }}\ensuremath{\text{ }}\ensuremath{\text{ }}\ensuremath{\text{ }}\ensuremath{\text{ }}\ensuremath{\text{ }}\ensuremath{\text{ }}\ensuremath{\text{ }}\ensuremath{\text{ }}\ensuremath{\text{ }}\ensuremath{\text{ }}\ensuremath{\text{ }}\NormalTok{\textbackslash{}STATE\ensuremath{\text{ }}\$y\ensuremath{\text{ }}\textbackslash{}Leftarrow\ensuremath{\text{ }}y\ensuremath{\text{ }}\textbackslash{}times\ensuremath{\text{ }}X\$}\newline
\ensuremath{\text{ }}\ensuremath{\text{ }}\ensuremath{\text{ }}\ensuremath{\text{ }}\ensuremath{\text{ }}\ensuremath{\text{ }}\ensuremath{\text{ }}\ensuremath{\text{ }}\ensuremath{\text{ }}\ensuremath{\text{ }}\ensuremath{\text{ }}\ensuremath{\text{ }}\NormalTok{\textbackslash{}STATE\ensuremath{\text{ }}\$N\ensuremath{\text{ }}\textbackslash{}Leftarrow\ensuremath{\text{ }}N\ensuremath{\text{ }}-\ensuremath{\text{ }}1\$}\newline
\ensuremath{\text{ }}\ensuremath{\text{ }}\ensuremath{\text{ }}\ensuremath{\text{ }}\ensuremath{\text{ }}\ensuremath{\text{ }}\ensuremath{\text{ }}\ensuremath{\text{ }}\NormalTok{\textbackslash{}ENDIF}\newline
\ensuremath{\text{ }}\ensuremath{\text{ }}\ensuremath{\text{ }}\ensuremath{\text{ }}\NormalTok{\textbackslash{}ENDWHILE}\newline
\NormalTok{\textbackslash{}end\{algorithmic\}}\newline
\NormalTok{\textbackslash{}end\{algorithm\}}\newline
\end{Highlighting}
\end{Shaded}




{\bfseries
\begin{mydescription} The official manual is located at
\end{mydescription}
}
\begin{myquote}\item{} \myplainurl{http://mirrors.ctan.org/macros/latex/contrib/algorithms/algorithms.pdf}
\end{myquote}



\section{References}
\label{591}

\begin{myitemize}
\item{} \newline
 \quad {\scshape  The official manual for the {\ttfamily \setmainfont[Path=/usr/share/fonts/truetype/cmu/,UprightFont=cmunrm.ttf,BoldFont=cmunbx.ttf,ItalicFont=cmunti.ttf,BoldItalicFont=cmunbi.ttf]{cmuntt.ttf}\setmonofont[Path=/usr/share/fonts/truetype/cmu/,UprightFont=cmuntt.ttf,BoldFont=cmuntb.ttf,ItalicFont=cmunit.ttf,BoldItalicFont=cmuntx.ttf]{cmuntt.ttf}\ttfamily algorithms}{$\text{ }$}\setmainfont[Path=/usr/share/fonts/truetype/cmu/,UprightFont=cmunrm.ttf,BoldFont=cmunbx.ttf,ItalicFont=cmunti.ttf,BoldItalicFont=cmunbi.ttf]{cmunrm.ttf}\setmonofont[Path=/usr/share/fonts/truetype/cmu/,UprightFont=cmuntt.ttf,BoldFont=cmuntb.ttf,ItalicFont=cmunit.ttf,BoldItalicFont=cmuntx.ttf]{cmunrm.ttf} package, Rogério Brito (2009), \myplainurl{http://mirrors.ctan.org/macros/latex/contrib/algorithms/algorithms.pdf}}
\end{myitemize}



\begin{myquote}
\item{}
\end{myquote}

\LaTeXNullTemplate{}



\myhref{https://sr.wikibooks.org/wiki/LaTeX\%2F\%D0\%90\%D0\%BB\%D0\%B3\%D0\%BE\%D1\%80\%D0\%B8\%D1\%82\%D0\%BC\%D0\%B8}{sr:LaTeX/Алгоритми}\chapter{Source Code Listings}

\myminitoc
\label{592}

\label{593}

\section{Using the {\itshape \setmainfont[Path=/usr/share/fonts/truetype/cmu/,UprightFont=cmunrm.ttf,BoldFont=cmunbx.ttf,ItalicFont=cmunti.ttf,BoldItalicFont=cmunbi.ttf]{cmunti.ttf}\setmonofont[Path=/usr/share/fonts/truetype/cmu/,UprightFont=cmuntt.ttf,BoldFont=cmuntb.ttf,ItalicFont=cmunit.ttf,BoldItalicFont=cmuntx.ttf]{cmunti.ttf}\itshape listings}{$\text{ }$}\setmainfont[Path=/usr/share/fonts/truetype/cmu/,UprightFont=cmunrm.ttf,BoldFont=cmunbx.ttf,ItalicFont=cmunti.ttf,BoldItalicFont=cmunbi.ttf]{cmunrm.ttf}\setmonofont[Path=/usr/share/fonts/truetype/cmu/,UprightFont=cmuntt.ttf,BoldFont=cmuntb.ttf,ItalicFont=cmunit.ttf,BoldItalicFont=cmuntx.ttf]{cmunrm.ttf} package}
\label{594}

Using the package \LaTeXTT{listings} you can add non-{}formatted text as you would do with \LaTeXTT{\textbackslash{}begin\{verbatim\}} but its main aim is to include the source code of any programming language within your document. If you wish to include pseudocode or algorithms, you may find \myhref{https://en.wikibooks.org/wiki/LaTeX\%2FAlgorithms\%20and\%20Pseudocode}{Algorithms and Pseudocode} useful also.

To use the package, you need:
\begin{Shaded}
\begin{Highlighting}[]

\NormalTok{\textbackslash{}usepackage\{listings\}}
\end{Highlighting}
\end{Shaded}


The \LaTeXTT{listings} package supports highlighting of all the most common languages and it is highly customizable. If you just want to write code within your document the package provides the \LaTeXTT{lstlisting} environment:

\begin{Shaded}
\begin{Highlighting}[]

\NormalTok{\textbackslash{}begin\{lstlisting\}}
\NormalTok{Put your code here.}
\NormalTok{\textbackslash{}end\{lstlisting\}}
\end{Highlighting}
\end{Shaded}


Another possibility, that is very useful if you created a program on several files and you are still editing it, is to import the code from the source itself. This way, if you modify the source, you just have to recompile the LaTeX code and your document will be updated. The command is:

\begin{Shaded}
\begin{Highlighting}[]

\NormalTok{\textbackslash{}lstinputlisting\{source_filename.py\}}
\end{Highlighting}
\end{Shaded}


in the example there is a Python source, but it doesn\textquotesingle{}t matter: you can include any file but you have to write the full file name. It will be considered plain text and it will be highlighted according to your settings, that means it doesn\textquotesingle{}t recognize the programming language by itself. You can specify the language while including the file with the following command:

\begin{Shaded}
\begin{Highlighting}[]

\NormalTok{\textbackslash{}lstinputlisting[language=Python]\{source_filename.py\}}
\end{Highlighting}
\end{Shaded}


You can also specify a scope for the file.

\begin{Shaded}
\begin{Highlighting}[]

\NormalTok{\textbackslash{}lstinputlisting[language=Python, firstline=37, lastline=45]\{source_filename.py\}}
\end{Highlighting}
\end{Shaded}


This comes in handy if you are sure that the file will not change (at least before the specified lines). You may also omit the \LaTeXTT{firstline}  or \LaTeXTT{lastline} parameter: it means {\itshape \setmainfont[Path=/usr/share/fonts/truetype/cmu/,UprightFont=cmunrm.ttf,BoldFont=cmunbx.ttf,ItalicFont=cmunti.ttf,BoldItalicFont=cmunbi.ttf]{cmunti.ttf}\setmonofont[Path=/usr/share/fonts/truetype/cmu/,UprightFont=cmuntt.ttf,BoldFont=cmuntb.ttf,ItalicFont=cmunit.ttf,BoldItalicFont=cmuntx.ttf]{cmunti.ttf}\itshape everything up to or starting from this point}\setmainfont[Path=/usr/share/fonts/truetype/cmu/,UprightFont=cmunrm.ttf,BoldFont=cmunbx.ttf,ItalicFont=cmunti.ttf,BoldItalicFont=cmunbi.ttf]{cmunrm.ttf}\setmonofont[Path=/usr/share/fonts/truetype/cmu/,UprightFont=cmuntt.ttf,BoldFont=cmuntb.ttf,ItalicFont=cmunit.ttf,BoldItalicFont=cmuntx.ttf]{cmunrm.ttf}.

This is a basic example for some Pascal code:


\begin{Shaded}
\begin{Highlighting}[]

\NormalTok{\textbackslash{}documentclass\{article\}}\newline
\NormalTok{\textbackslash{}usepackage\{listings\}\ensuremath{\text{ }}\ensuremath{\text{ }}\ensuremath{\text{ }}\ensuremath{\text{ }}\ensuremath{\text{ }}\ensuremath{\text{ }}\ensuremath{\text{ }}\ensuremath{\text{ }}\ensuremath{\text{ }}\ensuremath{\text{ }}\ensuremath{\text{ }}\ensuremath{\text{ }}\ensuremath{\text{ }}}\CommentTok{\%\ensuremath{\text{ }}Include\ensuremath{\text{ }}the\ensuremath{\text{ }}listings-package}\newline
\NormalTok{\textbackslash{}begin\{document\}}\newline
\NormalTok{\textbackslash{}lstset\{language=Pascal\}\ensuremath{\text{ }}\ensuremath{\text{ }}\ensuremath{\text{ }}\ensuremath{\text{ }}\ensuremath{\text{ }}\ensuremath{\text{ }}\ensuremath{\text{ }}\ensuremath{\text{ }}\ensuremath{\text{ }}\ensuremath{\text{ }}}\CommentTok{\%\ensuremath{\text{ }}Set\ensuremath{\text{ }}your\ensuremath{\text{ }}language\ensuremath{\text{ }}(you\ensuremath{\text{ }}can\ensuremath{\text{ }}change\ensuremath{\text{ }}the}\newline
\ensuremath{\text{ }}\NormalTok{language\ensuremath{\text{ }}for\ensuremath{\text{ }}each\ensuremath{\text{ }}code-block\ensuremath{\text{ }}optionally)}\newline
\ensuremath{\text{ }}\newline
\NormalTok{\textbackslash{}begin\{lstlisting\}[frame=single]\ensuremath{\text{ }}\ensuremath{\text{ }}\%\ensuremath{\text{ }}Start\ensuremath{\text{ }}your\ensuremath{\text{ }}code-block}\newline
\NormalTok{for\ensuremath{\text{ }}i:=maxint\ensuremath{\text{ }}to\ensuremath{\text{ }}0\ensuremath{\text{ }}do}\newline
\NormalTok{begin}\newline
\NormalTok{\{\ensuremath{\text{ }}do\ensuremath{\text{ }}nothing\ensuremath{\text{ }}\}}\newline
\NormalTok{end;}\newline
\NormalTok{Write(\textquotesingle{}Case\ensuremath{\text{ }}insensitive\ensuremath{\text{ }}\textquotesingle{});}\newline
\NormalTok{Write(\textquotesingle{}Pascal\ensuremath{\text{ }}keywords.\textquotesingle{});}\newline
\NormalTok{\textbackslash{}end\{lstlisting\}}\newline
\ensuremath{\text{ }}\newline
\NormalTok{\textbackslash{}end\{document\}}\newline
\end{Highlighting}
\end{Shaded}




\begin{minipage}{1.0\linewidth}
\begin{center}
\includegraphics[width=1.0\linewidth,height=6.5in,keepaspectratio]{../images/123.png}
\end{center}
\raggedright{}\myfigurewithoutcaption{123}
\end{minipage}\vspace{0.75cm}


\subsection{Supported languages}
\label{595}

It supports the following programming languages: 

ABAP\setmainfont[Path=/usr/share/fonts/truetype/cmu/,UprightFont=cmunrm.ttf,BoldFont=cmunbx.ttf,ItalicFont=cmunti.ttf,BoldItalicFont=cmunbi.ttf]{cmunrm.ttf}\setmonofont[Path=/usr/share/fonts/truetype/cmu/,UprightFont=cmuntt.ttf,BoldFont=cmuntb.ttf,ItalicFont=cmunit.ttf,BoldItalicFont=cmuntx.ttf]{cmunrm.ttf}\textsuperscript{2,4}, ACSL, Ada\setmainfont[Path=/usr/share/fonts/truetype/cmu/,UprightFont=cmunrm.ttf,BoldFont=cmunbx.ttf,ItalicFont=cmunti.ttf,BoldItalicFont=cmunbi.ttf]{cmunrm.ttf}\setmonofont[Path=/usr/share/fonts/truetype/cmu/,UprightFont=cmuntt.ttf,BoldFont=cmuntb.ttf,ItalicFont=cmunit.ttf,BoldItalicFont=cmuntx.ttf]{cmunrm.ttf}\textsuperscript{4}, Algol\setmainfont[Path=/usr/share/fonts/truetype/cmu/,UprightFont=cmunrm.ttf,BoldFont=cmunbx.ttf,ItalicFont=cmunti.ttf,BoldItalicFont=cmunbi.ttf]{cmunrm.ttf}\setmonofont[Path=/usr/share/fonts/truetype/cmu/,UprightFont=cmuntt.ttf,BoldFont=cmuntb.ttf,ItalicFont=cmunit.ttf,BoldItalicFont=cmuntx.ttf]{cmunrm.ttf}\textsuperscript{4}, Ant, Assembler\setmainfont[Path=/usr/share/fonts/truetype/cmu/,UprightFont=cmunrm.ttf,BoldFont=cmunbx.ttf,ItalicFont=cmunti.ttf,BoldItalicFont=cmunbi.ttf]{cmunrm.ttf}\setmonofont[Path=/usr/share/fonts/truetype/cmu/,UprightFont=cmuntt.ttf,BoldFont=cmuntb.ttf,ItalicFont=cmunit.ttf,BoldItalicFont=cmuntx.ttf]{cmunrm.ttf}\textsuperscript{2,4}, Awk\setmainfont[Path=/usr/share/fonts/truetype/cmu/,UprightFont=cmunrm.ttf,BoldFont=cmunbx.ttf,ItalicFont=cmunti.ttf,BoldItalicFont=cmunbi.ttf]{cmunrm.ttf}\setmonofont[Path=/usr/share/fonts/truetype/cmu/,UprightFont=cmuntt.ttf,BoldFont=cmuntb.ttf,ItalicFont=cmunit.ttf,BoldItalicFont=cmuntx.ttf]{cmunrm.ttf}\textsuperscript{4}, bash, Basic\setmainfont[Path=/usr/share/fonts/truetype/cmu/,UprightFont=cmunrm.ttf,BoldFont=cmunbx.ttf,ItalicFont=cmunti.ttf,BoldItalicFont=cmunbi.ttf]{cmunrm.ttf}\setmonofont[Path=/usr/share/fonts/truetype/cmu/,UprightFont=cmuntt.ttf,BoldFont=cmuntb.ttf,ItalicFont=cmunit.ttf,BoldItalicFont=cmuntx.ttf]{cmunrm.ttf}\textsuperscript{2,4}, C\#\setmainfont[Path=/usr/share/fonts/truetype/cmu/,UprightFont=cmunrm.ttf,BoldFont=cmunbx.ttf,ItalicFont=cmunti.ttf,BoldItalicFont=cmunbi.ttf]{cmunrm.ttf}\setmonofont[Path=/usr/share/fonts/truetype/cmu/,UprightFont=cmuntt.ttf,BoldFont=cmuntb.ttf,ItalicFont=cmunit.ttf,BoldItalicFont=cmuntx.ttf]{cmunrm.ttf}\textsuperscript{5}, C++\setmainfont[Path=/usr/share/fonts/truetype/cmu/,UprightFont=cmunrm.ttf,BoldFont=cmunbx.ttf,ItalicFont=cmunti.ttf,BoldItalicFont=cmunbi.ttf]{cmunrm.ttf}\setmonofont[Path=/usr/share/fonts/truetype/cmu/,UprightFont=cmuntt.ttf,BoldFont=cmuntb.ttf,ItalicFont=cmunit.ttf,BoldItalicFont=cmuntx.ttf]{cmunrm.ttf}\textsuperscript{4}, C\setmainfont[Path=/usr/share/fonts/truetype/cmu/,UprightFont=cmunrm.ttf,BoldFont=cmunbx.ttf,ItalicFont=cmunti.ttf,BoldItalicFont=cmunbi.ttf]{cmunrm.ttf}\setmonofont[Path=/usr/share/fonts/truetype/cmu/,UprightFont=cmuntt.ttf,BoldFont=cmuntb.ttf,ItalicFont=cmunit.ttf,BoldItalicFont=cmuntx.ttf]{cmunrm.ttf}\textsuperscript{4}, Caml\setmainfont[Path=/usr/share/fonts/truetype/cmu/,UprightFont=cmunrm.ttf,BoldFont=cmunbx.ttf,ItalicFont=cmunti.ttf,BoldItalicFont=cmunbi.ttf]{cmunrm.ttf}\setmonofont[Path=/usr/share/fonts/truetype/cmu/,UprightFont=cmuntt.ttf,BoldFont=cmuntb.ttf,ItalicFont=cmunit.ttf,BoldItalicFont=cmuntx.ttf]{cmunrm.ttf}\textsuperscript{4}, Clean, Cobol\setmainfont[Path=/usr/share/fonts/truetype/cmu/,UprightFont=cmunrm.ttf,BoldFont=cmunbx.ttf,ItalicFont=cmunti.ttf,BoldItalicFont=cmunbi.ttf]{cmunrm.ttf}\setmonofont[Path=/usr/share/fonts/truetype/cmu/,UprightFont=cmuntt.ttf,BoldFont=cmuntb.ttf,ItalicFont=cmunit.ttf,BoldItalicFont=cmuntx.ttf]{cmunrm.ttf}\textsuperscript{4}, Comal, csh, Delphi, Eiffel, Elan, erlang, Euphoria, Fortran\setmainfont[Path=/usr/share/fonts/truetype/cmu/,UprightFont=cmunrm.ttf,BoldFont=cmunbx.ttf,ItalicFont=cmunti.ttf,BoldItalicFont=cmunbi.ttf]{cmunrm.ttf}\setmonofont[Path=/usr/share/fonts/truetype/cmu/,UprightFont=cmuntt.ttf,BoldFont=cmuntb.ttf,ItalicFont=cmunit.ttf,BoldItalicFont=cmuntx.ttf]{cmunrm.ttf}\textsuperscript{4}, GCL, Gnuplot, Haskell, HTML, IDL\setmainfont[Path=/usr/share/fonts/truetype/cmu/,UprightFont=cmunrm.ttf,BoldFont=cmunbx.ttf,ItalicFont=cmunti.ttf,BoldItalicFont=cmunbi.ttf]{cmunrm.ttf}\setmonofont[Path=/usr/share/fonts/truetype/cmu/,UprightFont=cmuntt.ttf,BoldFont=cmuntb.ttf,ItalicFont=cmunit.ttf,BoldItalicFont=cmuntx.ttf]{cmunrm.ttf}\textsuperscript{4}, inform, Java\setmainfont[Path=/usr/share/fonts/truetype/cmu/,UprightFont=cmunrm.ttf,BoldFont=cmunbx.ttf,ItalicFont=cmunti.ttf,BoldItalicFont=cmunbi.ttf]{cmunrm.ttf}\setmonofont[Path=/usr/share/fonts/truetype/cmu/,UprightFont=cmuntt.ttf,BoldFont=cmuntb.ttf,ItalicFont=cmunit.ttf,BoldItalicFont=cmuntx.ttf]{cmunrm.ttf}\textsuperscript{4}, JVMIS, ksh, Lisp\setmainfont[Path=/usr/share/fonts/truetype/cmu/,UprightFont=cmunrm.ttf,BoldFont=cmunbx.ttf,ItalicFont=cmunti.ttf,BoldItalicFont=cmunbi.ttf]{cmunrm.ttf}\setmonofont[Path=/usr/share/fonts/truetype/cmu/,UprightFont=cmuntt.ttf,BoldFont=cmuntb.ttf,ItalicFont=cmunit.ttf,BoldItalicFont=cmuntx.ttf]{cmunrm.ttf}\textsuperscript{4}, Logo, Lua\setmainfont[Path=/usr/share/fonts/truetype/cmu/,UprightFont=cmunrm.ttf,BoldFont=cmunbx.ttf,ItalicFont=cmunti.ttf,BoldItalicFont=cmunbi.ttf]{cmunrm.ttf}\setmonofont[Path=/usr/share/fonts/truetype/cmu/,UprightFont=cmuntt.ttf,BoldFont=cmuntb.ttf,ItalicFont=cmunit.ttf,BoldItalicFont=cmuntx.ttf]{cmunrm.ttf}\textsuperscript{2}, make\setmainfont[Path=/usr/share/fonts/truetype/cmu/,UprightFont=cmunrm.ttf,BoldFont=cmunbx.ttf,ItalicFont=cmunti.ttf,BoldItalicFont=cmunbi.ttf]{cmunrm.ttf}\setmonofont[Path=/usr/share/fonts/truetype/cmu/,UprightFont=cmuntt.ttf,BoldFont=cmuntb.ttf,ItalicFont=cmunit.ttf,BoldItalicFont=cmuntx.ttf]{cmunrm.ttf}\textsuperscript{4}, Mathematica\setmainfont[Path=/usr/share/fonts/truetype/cmu/,UprightFont=cmunrm.ttf,BoldFont=cmunbx.ttf,ItalicFont=cmunti.ttf,BoldItalicFont=cmunbi.ttf]{cmunrm.ttf}\setmonofont[Path=/usr/share/fonts/truetype/cmu/,UprightFont=cmuntt.ttf,BoldFont=cmuntb.ttf,ItalicFont=cmunit.ttf,BoldItalicFont=cmuntx.ttf]{cmunrm.ttf}\textsuperscript{1,4}, Matlab, Mercury, MetaPost, Miranda, Mizar, ML, Modelica\setmainfont[Path=/usr/share/fonts/truetype/cmu/,UprightFont=cmunrm.ttf,BoldFont=cmunbx.ttf,ItalicFont=cmunti.ttf,BoldItalicFont=cmunbi.ttf]{cmunrm.ttf}\setmonofont[Path=/usr/share/fonts/truetype/cmu/,UprightFont=cmuntt.ttf,BoldFont=cmuntb.ttf,ItalicFont=cmunit.ttf,BoldItalicFont=cmuntx.ttf]{cmunrm.ttf}\textsuperscript{3}, Modula-{}2, MuPAD, NASTRAN, Oberon-{}2, Objective C\setmainfont[Path=/usr/share/fonts/truetype/cmu/,UprightFont=cmunrm.ttf,BoldFont=cmunbx.ttf,ItalicFont=cmunti.ttf,BoldItalicFont=cmunbi.ttf]{cmunrm.ttf}\setmonofont[Path=/usr/share/fonts/truetype/cmu/,UprightFont=cmuntt.ttf,BoldFont=cmuntb.ttf,ItalicFont=cmunit.ttf,BoldItalicFont=cmuntx.ttf]{cmunrm.ttf}\textsuperscript{5} , OCL\setmainfont[Path=/usr/share/fonts/truetype/cmu/,UprightFont=cmunrm.ttf,BoldFont=cmunbx.ttf,ItalicFont=cmunti.ttf,BoldItalicFont=cmunbi.ttf]{cmunrm.ttf}\setmonofont[Path=/usr/share/fonts/truetype/cmu/,UprightFont=cmuntt.ttf,BoldFont=cmuntb.ttf,ItalicFont=cmunit.ttf,BoldItalicFont=cmuntx.ttf]{cmunrm.ttf}\textsuperscript{4}, Octave, Oz, Pascal\setmainfont[Path=/usr/share/fonts/truetype/cmu/,UprightFont=cmunrm.ttf,BoldFont=cmunbx.ttf,ItalicFont=cmunti.ttf,BoldItalicFont=cmunbi.ttf]{cmunrm.ttf}\setmonofont[Path=/usr/share/fonts/truetype/cmu/,UprightFont=cmuntt.ttf,BoldFont=cmuntb.ttf,ItalicFont=cmunit.ttf,BoldItalicFont=cmuntx.ttf]{cmunrm.ttf}\textsuperscript{4}, Perl, PHP, PL/I, Plasm, POV, Prolog, Promela, Python, R, Reduce, Rexx, RSL, Ruby, S\setmainfont[Path=/usr/share/fonts/truetype/cmu/,UprightFont=cmunrm.ttf,BoldFont=cmunbx.ttf,ItalicFont=cmunti.ttf,BoldItalicFont=cmunbi.ttf]{cmunrm.ttf}\setmonofont[Path=/usr/share/fonts/truetype/cmu/,UprightFont=cmuntt.ttf,BoldFont=cmuntb.ttf,ItalicFont=cmunit.ttf,BoldItalicFont=cmuntx.ttf]{cmunrm.ttf}\textsuperscript{4}, SAS, Scilab, sh, SHELXL, Simula\setmainfont[Path=/usr/share/fonts/truetype/cmu/,UprightFont=cmunrm.ttf,BoldFont=cmunbx.ttf,ItalicFont=cmunti.ttf,BoldItalicFont=cmunbi.ttf]{cmunrm.ttf}\setmonofont[Path=/usr/share/fonts/truetype/cmu/,UprightFont=cmuntt.ttf,BoldFont=cmuntb.ttf,ItalicFont=cmunit.ttf,BoldItalicFont=cmuntx.ttf]{cmunrm.ttf}\textsuperscript{4}, SQL, tcl\setmainfont[Path=/usr/share/fonts/truetype/cmu/,UprightFont=cmunrm.ttf,BoldFont=cmunbx.ttf,ItalicFont=cmunti.ttf,BoldItalicFont=cmunbi.ttf]{cmunrm.ttf}\setmonofont[Path=/usr/share/fonts/truetype/cmu/,UprightFont=cmuntt.ttf,BoldFont=cmuntb.ttf,ItalicFont=cmunit.ttf,BoldItalicFont=cmuntx.ttf]{cmunrm.ttf}\textsuperscript{4}, TeX\setmainfont[Path=/usr/share/fonts/truetype/cmu/,UprightFont=cmunrm.ttf,BoldFont=cmunbx.ttf,ItalicFont=cmunti.ttf,BoldItalicFont=cmunbi.ttf]{cmunrm.ttf}\setmonofont[Path=/usr/share/fonts/truetype/cmu/,UprightFont=cmuntt.ttf,BoldFont=cmuntb.ttf,ItalicFont=cmunit.ttf,BoldItalicFont=cmuntx.ttf]{cmunrm.ttf}\textsuperscript{4}, VBScript, Verilog, VHDL\setmainfont[Path=/usr/share/fonts/truetype/cmu/,UprightFont=cmunrm.ttf,BoldFont=cmunbx.ttf,ItalicFont=cmunti.ttf,BoldItalicFont=cmunbi.ttf]{cmunrm.ttf}\setmonofont[Path=/usr/share/fonts/truetype/cmu/,UprightFont=cmuntt.ttf,BoldFont=cmuntb.ttf,ItalicFont=cmunit.ttf,BoldItalicFont=cmuntx.ttf]{cmunrm.ttf}\textsuperscript{4}, VRML\setmainfont[Path=/usr/share/fonts/truetype/cmu/,UprightFont=cmunrm.ttf,BoldFont=cmunbx.ttf,ItalicFont=cmunti.ttf,BoldItalicFont=cmunbi.ttf]{cmunrm.ttf}\setmonofont[Path=/usr/share/fonts/truetype/cmu/,UprightFont=cmuntt.ttf,BoldFont=cmuntb.ttf,ItalicFont=cmunit.ttf,BoldItalicFont=cmuntx.ttf]{cmunrm.ttf}\textsuperscript{4}, XML, XSLT.

For some of them, several dialects are supported. For more information, refer to the documentation that comes with the package, it should be within your distribution under the name {\ttfamily \setmainfont[Path=/usr/share/fonts/truetype/cmu/,UprightFont=cmunrm.ttf,BoldFont=cmunbx.ttf,ItalicFont=cmunti.ttf,BoldItalicFont=cmunbi.ttf]{cmuntt.ttf}\setmonofont[Path=/usr/share/fonts/truetype/cmu/,UprightFont=cmuntt.ttf,BoldFont=cmuntb.ttf,ItalicFont=cmunit.ttf,BoldItalicFont=cmuntx.ttf]{cmuntt.ttf}\ttfamily listings-{}*.dvi}\setmainfont[Path=/usr/share/fonts/truetype/cmu/,UprightFont=cmunrm.ttf,BoldFont=cmunbx.ttf,ItalicFont=cmunti.ttf,BoldItalicFont=cmunbi.ttf]{cmunrm.ttf}\setmonofont[Path=/usr/share/fonts/truetype/cmu/,UprightFont=cmuntt.ttf,BoldFont=cmuntb.ttf,ItalicFont=cmunit.ttf,BoldItalicFont=cmuntx.ttf]{cmunrm.ttf}.
{\bfseries
\begin{mydescription}Notes
\end{mydescription}
}

\begin{myenumerate}
\item{}  It supports Mathematica code only if you are typing in plain text format. You can\textquotesingle{}t include *.NB files \LaTeXTT{\textbackslash{}lstinputlisting\{...\}} as you could with any other programming language, but Mathematica can export in a pretty-{}formatted LaTeX source.
\item{}  Specification of the dialect is mandatory for these languages (e.g. \LaTeXTT{language=\{{$\text{[}$}x86masm{$\text{]}$}Assembler\}}).
\item{}  Modelica is supported via the dtsyntax package available \myhref{https://code.google.com/p/dtsyntax/}{here}.
\item{}  For these languages, multiple dialects are supported. C, for example, has ANSI, Handel, Objective and Sharp. See p. 12 of the \myhref{http://mirrors.ctan.org/macros/latex/contrib/listings/listings.pdf}{listings manual} for an overview.
\item{}  Defined as a dialect of another language
\end{myenumerate}

\subsection{Settings}
\label{596}

You can modify several parameters that will affect how the code is shown. You can put the following code anywhere in the document (it doesn\textquotesingle{}t matter whether before or after \LaTeXTT{\textbackslash{}begin\{document\}}), change it according to your needs. The meaning is explained next to any line.


\begin{Shaded}
\begin{Highlighting}[]

\NormalTok{\textbackslash{}usepackage\{listings\}}\newline
\NormalTok{\textbackslash{}usepackage\{color\}}\newline
\ensuremath{\text{ }}\newline
\NormalTok{\textbackslash{}definecolor\{mygreen\}\{rgb\}\{0,0.6,0\}}\newline
\NormalTok{\textbackslash{}definecolor\{mygray\}\{rgb\}\{0.5,0.5,0.5\}}\newline
\NormalTok{\textbackslash{}definecolor\{mymauve\}\{rgb\}\{0.58,0,0.82\}}\newline
\ensuremath{\text{ }}\newline
\NormalTok{\textbackslash{}lstset\{\ensuremath{\text{ }}}\CommentTok{\%}\newline
\ensuremath{\text{ }}\ensuremath{\text{ }}\NormalTok{backgroundcolor=\textbackslash{}color\{white\},\ensuremath{\text{ }}\ensuremath{\text{ }}\ensuremath{\text{ }}}\CommentTok{\%\ensuremath{\text{ }}choose\ensuremath{\text{ }}the\ensuremath{\text{ }}background\ensuremath{\text{ }}color;\ensuremath{\text{ }}you\ensuremath{\text{ }}must\ensuremath{\text{ }}add}\newline
\ensuremath{\text{ }}\NormalTok{\textbackslash{}usepackage\{color\}\ensuremath{\text{ }}or\ensuremath{\text{ }}\textbackslash{}usepackage\{xcolor\}}\newline
\ensuremath{\text{ }}\ensuremath{\text{ }}\NormalTok{basicstyle=\textbackslash{}footnotesize,\ensuremath{\text{ }}\ensuremath{\text{ }}\ensuremath{\text{ }}\ensuremath{\text{ }}\ensuremath{\text{ }}\ensuremath{\text{ }}\ensuremath{\text{ }}\ensuremath{\text{ }}}\CommentTok{\%\ensuremath{\text{ }}the\ensuremath{\text{ }}size\ensuremath{\text{ }}of\ensuremath{\text{ }}the\ensuremath{\text{ }}fonts\ensuremath{\text{ }}that\ensuremath{\text{ }}are\ensuremath{\text{ }}used\ensuremath{\text{ }}for\ensuremath{\text{ }}the}\newline
\ensuremath{\text{ }}\NormalTok{code}\newline
\ensuremath{\text{ }}\ensuremath{\text{ }}\NormalTok{breakatwhitespace=false,\ensuremath{\text{ }}\ensuremath{\text{ }}\ensuremath{\text{ }}\ensuremath{\text{ }}\ensuremath{\text{ }}\ensuremath{\text{ }}\ensuremath{\text{ }}\ensuremath{\text{ }}\ensuremath{\text{ }}}\CommentTok{\%\ensuremath{\text{ }}sets\ensuremath{\text{ }}if\ensuremath{\text{ }}automatic\ensuremath{\text{ }}breaks\ensuremath{\text{ }}should\ensuremath{\text{ }}only\ensuremath{\text{ }}happen}\newline
\ensuremath{\text{ }}\NormalTok{at\ensuremath{\text{ }}whitespace}\newline
\ensuremath{\text{ }}\ensuremath{\text{ }}\NormalTok{breaklines=true,\ensuremath{\text{ }}\ensuremath{\text{ }}\ensuremath{\text{ }}\ensuremath{\text{ }}\ensuremath{\text{ }}\ensuremath{\text{ }}\ensuremath{\text{ }}\ensuremath{\text{ }}\ensuremath{\text{ }}\ensuremath{\text{ }}\ensuremath{\text{ }}\ensuremath{\text{ }}\ensuremath{\text{ }}\ensuremath{\text{ }}\ensuremath{\text{ }}\ensuremath{\text{ }}\ensuremath{\text{ }}}\CommentTok{\%\ensuremath{\text{ }}sets\ensuremath{\text{ }}automatic\ensuremath{\text{ }}line\ensuremath{\text{ }}breaking}\newline
\ensuremath{\text{ }}\ensuremath{\text{ }}\NormalTok{captionpos=b,\ensuremath{\text{ }}\ensuremath{\text{ }}\ensuremath{\text{ }}\ensuremath{\text{ }}\ensuremath{\text{ }}\ensuremath{\text{ }}\ensuremath{\text{ }}\ensuremath{\text{ }}\ensuremath{\text{ }}\ensuremath{\text{ }}\ensuremath{\text{ }}\ensuremath{\text{ }}\ensuremath{\text{ }}\ensuremath{\text{ }}\ensuremath{\text{ }}\ensuremath{\text{ }}\ensuremath{\text{ }}\ensuremath{\text{ }}\ensuremath{\text{ }}\ensuremath{\text{ }}}\CommentTok{\%\ensuremath{\text{ }}sets\ensuremath{\text{ }}the\ensuremath{\text{ }}caption-position\ensuremath{\text{ }}to\ensuremath{\text{ }}bottom}\newline
\ensuremath{\text{ }}\ensuremath{\text{ }}\NormalTok{commentstyle=\textbackslash{}color\{mygreen\},\ensuremath{\text{ }}\ensuremath{\text{ }}\ensuremath{\text{ }}\ensuremath{\text{ }}}\CommentTok{\%\ensuremath{\text{ }}comment\ensuremath{\text{ }}style}\newline
\ensuremath{\text{ }}\ensuremath{\text{ }}\NormalTok{deletekeywords=\{...\},\ensuremath{\text{ }}\ensuremath{\text{ }}\ensuremath{\text{ }}\ensuremath{\text{ }}\ensuremath{\text{ }}\ensuremath{\text{ }}\ensuremath{\text{ }}\ensuremath{\text{ }}\ensuremath{\text{ }}\ensuremath{\text{ }}\ensuremath{\text{ }}\ensuremath{\text{ }}}\CommentTok{\%\ensuremath{\text{ }}if\ensuremath{\text{ }}you\ensuremath{\text{ }}want\ensuremath{\text{ }}to\ensuremath{\text{ }}delete\ensuremath{\text{ }}keywords\ensuremath{\text{ }}from\ensuremath{\text{ }}the}\newline
\ensuremath{\text{ }}\NormalTok{given\ensuremath{\text{ }}language}\newline
\ensuremath{\text{ }}\ensuremath{\text{ }}\NormalTok{escapeinside=\{\textbackslash{}\%*\}\{*)\},\ensuremath{\text{ }}\ensuremath{\text{ }}\ensuremath{\text{ }}\ensuremath{\text{ }}\ensuremath{\text{ }}\ensuremath{\text{ }}\ensuremath{\text{ }}\ensuremath{\text{ }}\ensuremath{\text{ }}\ensuremath{\text{ }}}\CommentTok{\%\ensuremath{\text{ }}if\ensuremath{\text{ }}you\ensuremath{\text{ }}want\ensuremath{\text{ }}to\ensuremath{\text{ }}add\ensuremath{\text{ }}LaTeX\ensuremath{\text{ }}within\ensuremath{\text{ }}your\ensuremath{\text{ }}code}\newline
\ensuremath{\text{ }}\ensuremath{\text{ }}\NormalTok{extendedchars=true,\ensuremath{\text{ }}\ensuremath{\text{ }}\ensuremath{\text{ }}\ensuremath{\text{ }}\ensuremath{\text{ }}\ensuremath{\text{ }}\ensuremath{\text{ }}\ensuremath{\text{ }}\ensuremath{\text{ }}\ensuremath{\text{ }}\ensuremath{\text{ }}\ensuremath{\text{ }}\ensuremath{\text{ }}\ensuremath{\text{ }}}\CommentTok{\%\ensuremath{\text{ }}lets\ensuremath{\text{ }}you\ensuremath{\text{ }}use\ensuremath{\text{ }}non-ASCII\ensuremath{\text{ }}characters;\ensuremath{\text{ }}for}\newline
\ensuremath{\text{ }}\NormalTok{8-bits\ensuremath{\text{ }}encodings\ensuremath{\text{ }}only,\ensuremath{\text{ }}does\ensuremath{\text{ }}not\ensuremath{\text{ }}work\ensuremath{\text{ }}with\ensuremath{\text{ }}UTF-8}\newline
\ensuremath{\text{ }}\ensuremath{\text{ }}\NormalTok{frame=single,	\ensuremath{\text{ }}\ensuremath{\text{ }}\ensuremath{\text{ }}\ensuremath{\text{ }}\ensuremath{\text{ }}\ensuremath{\text{ }}\ensuremath{\text{ }}\ensuremath{\text{ }}\ensuremath{\text{ }}\ensuremath{\text{ }}\ensuremath{\text{ }}\ensuremath{\text{ }}\ensuremath{\text{ }}\ensuremath{\text{ }}\ensuremath{\text{ }}\ensuremath{\text{ }}\ensuremath{\text{ }}\ensuremath{\text{ }}\ensuremath{\text{ }}}\CommentTok{\%\ensuremath{\text{ }}adds\ensuremath{\text{ }}a\ensuremath{\text{ }}frame\ensuremath{\text{ }}around\ensuremath{\text{ }}the\ensuremath{\text{ }}code}\newline
\ensuremath{\text{ }}\ensuremath{\text{ }}\NormalTok{keepspaces=true,\ensuremath{\text{ }}\ensuremath{\text{ }}\ensuremath{\text{ }}\ensuremath{\text{ }}\ensuremath{\text{ }}\ensuremath{\text{ }}\ensuremath{\text{ }}\ensuremath{\text{ }}\ensuremath{\text{ }}\ensuremath{\text{ }}\ensuremath{\text{ }}\ensuremath{\text{ }}\ensuremath{\text{ }}\ensuremath{\text{ }}\ensuremath{\text{ }}\ensuremath{\text{ }}\ensuremath{\text{ }}}\CommentTok{\%\ensuremath{\text{ }}keeps\ensuremath{\text{ }}spaces\ensuremath{\text{ }}in\ensuremath{\text{ }}text,\ensuremath{\text{ }}useful\ensuremath{\text{ }}for\ensuremath{\text{ }}keeping}\newline
\ensuremath{\text{ }}\NormalTok{indentation\ensuremath{\text{ }}of\ensuremath{\text{ }}code\ensuremath{\text{ }}(possibly\ensuremath{\text{ }}needs\ensuremath{\text{ }}columns=flexible)}\newline
\ensuremath{\text{ }}\ensuremath{\text{ }}\NormalTok{keywordstyle=\textbackslash{}color\{blue\},\ensuremath{\text{ }}\ensuremath{\text{ }}\ensuremath{\text{ }}\ensuremath{\text{ }}\ensuremath{\text{ }}\ensuremath{\text{ }}\ensuremath{\text{ }}}\CommentTok{\%\ensuremath{\text{ }}keyword\ensuremath{\text{ }}style}\newline
\ensuremath{\text{ }}\ensuremath{\text{ }}\NormalTok{language=Octave,\ensuremath{\text{ }}\ensuremath{\text{ }}\ensuremath{\text{ }}\ensuremath{\text{ }}\ensuremath{\text{ }}\ensuremath{\text{ }}\ensuremath{\text{ }}\ensuremath{\text{ }}\ensuremath{\text{ }}\ensuremath{\text{ }}\ensuremath{\text{ }}\ensuremath{\text{ }}\ensuremath{\text{ }}\ensuremath{\text{ }}\ensuremath{\text{ }}\ensuremath{\text{ }}\ensuremath{\text{ }}}\CommentTok{\%\ensuremath{\text{ }}the\ensuremath{\text{ }}language\ensuremath{\text{ }}of\ensuremath{\text{ }}the\ensuremath{\text{ }}code}\newline
\ensuremath{\text{ }}\ensuremath{\text{ }}\NormalTok{otherkeywords=\{*,...\},\ensuremath{\text{ }}\ensuremath{\text{ }}\ensuremath{\text{ }}\ensuremath{\text{ }}\ensuremath{\text{ }}\ensuremath{\text{ }}\ensuremath{\text{ }}\ensuremath{\text{ }}\ensuremath{\text{ }}\ensuremath{\text{ }}\ensuremath{\text{ }}}\CommentTok{\%\ensuremath{\text{ }}if\ensuremath{\text{ }}you\ensuremath{\text{ }}want\ensuremath{\text{ }}to\ensuremath{\text{ }}add\ensuremath{\text{ }}more\ensuremath{\text{ }}keywords\ensuremath{\text{ }}to\ensuremath{\text{ }}the\ensuremath{\text{ }}set}\newline
\ensuremath{\text{ }}\ensuremath{\text{ }}\NormalTok{numbers=left,\ensuremath{\text{ }}\ensuremath{\text{ }}\ensuremath{\text{ }}\ensuremath{\text{ }}\ensuremath{\text{ }}\ensuremath{\text{ }}\ensuremath{\text{ }}\ensuremath{\text{ }}\ensuremath{\text{ }}\ensuremath{\text{ }}\ensuremath{\text{ }}\ensuremath{\text{ }}\ensuremath{\text{ }}\ensuremath{\text{ }}\ensuremath{\text{ }}\ensuremath{\text{ }}\ensuremath{\text{ }}\ensuremath{\text{ }}\ensuremath{\text{ }}\ensuremath{\text{ }}}\CommentTok{\%\ensuremath{\text{ }}where\ensuremath{\text{ }}to\ensuremath{\text{ }}put\ensuremath{\text{ }}the\ensuremath{\text{ }}line-numbers;\ensuremath{\text{ }}possible}\newline
\ensuremath{\text{ }}\NormalTok{values\ensuremath{\text{ }}are\ensuremath{\text{ }}(none,\ensuremath{\text{ }}left,\ensuremath{\text{ }}right)}\newline
\ensuremath{\text{ }}\ensuremath{\text{ }}\NormalTok{numbersep=5pt,\ensuremath{\text{ }}\ensuremath{\text{ }}\ensuremath{\text{ }}\ensuremath{\text{ }}\ensuremath{\text{ }}\ensuremath{\text{ }}\ensuremath{\text{ }}\ensuremath{\text{ }}\ensuremath{\text{ }}\ensuremath{\text{ }}\ensuremath{\text{ }}\ensuremath{\text{ }}\ensuremath{\text{ }}\ensuremath{\text{ }}\ensuremath{\text{ }}\ensuremath{\text{ }}\ensuremath{\text{ }}\ensuremath{\text{ }}\ensuremath{\text{ }}}\CommentTok{\%\ensuremath{\text{ }}how\ensuremath{\text{ }}far\ensuremath{\text{ }}the\ensuremath{\text{ }}line-numbers\ensuremath{\text{ }}are\ensuremath{\text{ }}from\ensuremath{\text{ }}the\ensuremath{\text{ }}code}\newline
\ensuremath{\text{ }}\ensuremath{\text{ }}\NormalTok{numberstyle=\textbackslash{}tiny\textbackslash{}color\{mygray\},\ensuremath{\text{ }}}\CommentTok{\%\ensuremath{\text{ }}the\ensuremath{\text{ }}style\ensuremath{\text{ }}that\ensuremath{\text{ }}is\ensuremath{\text{ }}used\ensuremath{\text{ }}for\ensuremath{\text{ }}the\ensuremath{\text{ }}line-numbers}\newline
\ensuremath{\text{ }}\ensuremath{\text{ }}\NormalTok{rulecolor=\textbackslash{}color\{black\},\ensuremath{\text{ }}\ensuremath{\text{ }}\ensuremath{\text{ }}\ensuremath{\text{ }}\ensuremath{\text{ }}\ensuremath{\text{ }}\ensuremath{\text{ }}\ensuremath{\text{ }}\ensuremath{\text{ }}}\CommentTok{\%\ensuremath{\text{ }}if\ensuremath{\text{ }}not\ensuremath{\text{ }}set,\ensuremath{\text{ }}the\ensuremath{\text{ }}frame-color\ensuremath{\text{ }}may\ensuremath{\text{ }}be\ensuremath{\text{ }}changed}\newline
\ensuremath{\text{ }}\NormalTok{on\ensuremath{\text{ }}line-breaks\ensuremath{\text{ }}within\ensuremath{\text{ }}not-black\ensuremath{\text{ }}text\ensuremath{\text{ }}(e.g.\ensuremath{\text{ }}comments\ensuremath{\text{ }}(green\ensuremath{\text{ }}here))}\newline
\ensuremath{\text{ }}\ensuremath{\text{ }}\NormalTok{showspaces=false,\ensuremath{\text{ }}\ensuremath{\text{ }}\ensuremath{\text{ }}\ensuremath{\text{ }}\ensuremath{\text{ }}\ensuremath{\text{ }}\ensuremath{\text{ }}\ensuremath{\text{ }}\ensuremath{\text{ }}\ensuremath{\text{ }}\ensuremath{\text{ }}\ensuremath{\text{ }}\ensuremath{\text{ }}\ensuremath{\text{ }}\ensuremath{\text{ }}\ensuremath{\text{ }}}\CommentTok{\%\ensuremath{\text{ }}show\ensuremath{\text{ }}spaces\ensuremath{\text{ }}everywhere\ensuremath{\text{ }}adding\ensuremath{\text{ }}particular}\newline
\ensuremath{\text{ }}\NormalTok{underscores;\ensuremath{\text{ }}it\ensuremath{\text{ }}overrides\ensuremath{\text{ }}\textquotesingle{}showstringspaces\textquotesingle{}}\newline
\ensuremath{\text{ }}\ensuremath{\text{ }}\NormalTok{showstringspaces=false,\ensuremath{\text{ }}\ensuremath{\text{ }}\ensuremath{\text{ }}\ensuremath{\text{ }}\ensuremath{\text{ }}\ensuremath{\text{ }}\ensuremath{\text{ }}\ensuremath{\text{ }}\ensuremath{\text{ }}\ensuremath{\text{ }}}\CommentTok{\%\ensuremath{\text{ }}underline\ensuremath{\text{ }}spaces\ensuremath{\text{ }}within\ensuremath{\text{ }}strings\ensuremath{\text{ }}only}\newline
\ensuremath{\text{ }}\ensuremath{\text{ }}\NormalTok{showtabs=false,\ensuremath{\text{ }}\ensuremath{\text{ }}\ensuremath{\text{ }}\ensuremath{\text{ }}\ensuremath{\text{ }}\ensuremath{\text{ }}\ensuremath{\text{ }}\ensuremath{\text{ }}\ensuremath{\text{ }}\ensuremath{\text{ }}\ensuremath{\text{ }}\ensuremath{\text{ }}\ensuremath{\text{ }}\ensuremath{\text{ }}\ensuremath{\text{ }}\ensuremath{\text{ }}\ensuremath{\text{ }}\ensuremath{\text{ }}}\CommentTok{\%\ensuremath{\text{ }}show\ensuremath{\text{ }}tabs\ensuremath{\text{ }}within\ensuremath{\text{ }}strings\ensuremath{\text{ }}adding\ensuremath{\text{ }}particular}\newline
\ensuremath{\text{ }}\NormalTok{underscores}\newline
\ensuremath{\text{ }}\ensuremath{\text{ }}\NormalTok{stepnumber=2,\ensuremath{\text{ }}\ensuremath{\text{ }}\ensuremath{\text{ }}\ensuremath{\text{ }}\ensuremath{\text{ }}\ensuremath{\text{ }}\ensuremath{\text{ }}\ensuremath{\text{ }}\ensuremath{\text{ }}\ensuremath{\text{ }}\ensuremath{\text{ }}\ensuremath{\text{ }}\ensuremath{\text{ }}\ensuremath{\text{ }}\ensuremath{\text{ }}\ensuremath{\text{ }}\ensuremath{\text{ }}\ensuremath{\text{ }}\ensuremath{\text{ }}\ensuremath{\text{ }}}\CommentTok{\%\ensuremath{\text{ }}the\ensuremath{\text{ }}step\ensuremath{\text{ }}between\ensuremath{\text{ }}two\ensuremath{\text{ }}line-numbers.\ensuremath{\text{ }}If\ensuremath{\text{ }}it\textquotesingle{}s}\newline
\ensuremath{\text{ }}\NormalTok{1,\ensuremath{\text{ }}each\ensuremath{\text{ }}line\ensuremath{\text{ }}will\ensuremath{\text{ }}be\ensuremath{\text{ }}numbered}\newline
\ensuremath{\text{ }}\ensuremath{\text{ }}\NormalTok{stringstyle=\textbackslash{}color\{mymauve\},\ensuremath{\text{ }}\ensuremath{\text{ }}\ensuremath{\text{ }}\ensuremath{\text{ }}\ensuremath{\text{ }}}\CommentTok{\%\ensuremath{\text{ }}string\ensuremath{\text{ }}literal\ensuremath{\text{ }}style}\newline
\ensuremath{\text{ }}\ensuremath{\text{ }}\NormalTok{tabsize=2,	\ensuremath{\text{ }}\ensuremath{\text{ }}\ensuremath{\text{ }}\ensuremath{\text{ }}\ensuremath{\text{ }}\ensuremath{\text{ }}\ensuremath{\text{ }}\ensuremath{\text{ }}\ensuremath{\text{ }}\ensuremath{\text{ }}\ensuremath{\text{ }}\ensuremath{\text{ }}\ensuremath{\text{ }}\ensuremath{\text{ }}\ensuremath{\text{ }}\ensuremath{\text{ }}\ensuremath{\text{ }}\ensuremath{\text{ }}\ensuremath{\text{ }}}\CommentTok{\%\ensuremath{\text{ }}sets\ensuremath{\text{ }}default\ensuremath{\text{ }}tabsize\ensuremath{\text{ }}to\ensuremath{\text{ }}2\ensuremath{\text{ }}spaces}\newline
\ensuremath{\text{ }}\ensuremath{\text{ }}\NormalTok{title=\textbackslash{}lstname\ensuremath{\text{ }}\ensuremath{\text{ }}\ensuremath{\text{ }}\ensuremath{\text{ }}\ensuremath{\text{ }}\ensuremath{\text{ }}\ensuremath{\text{ }}\ensuremath{\text{ }}\ensuremath{\text{ }}\ensuremath{\text{ }}\ensuremath{\text{ }}\ensuremath{\text{ }}\ensuremath{\text{ }}\ensuremath{\text{ }}\ensuremath{\text{ }}\ensuremath{\text{ }}\ensuremath{\text{ }}\ensuremath{\text{ }}\ensuremath{\text{ }}}\CommentTok{\%\ensuremath{\text{ }}show\ensuremath{\text{ }}the\ensuremath{\text{ }}filename\ensuremath{\text{ }}of\ensuremath{\text{ }}files\ensuremath{\text{ }}included\ensuremath{\text{ }}with}\newline
\ensuremath{\text{ }}\NormalTok{\textbackslash{}lstinputlisting;\ensuremath{\text{ }}also\ensuremath{\text{ }}try\ensuremath{\text{ }}caption\ensuremath{\text{ }}instead\ensuremath{\text{ }}of\ensuremath{\text{ }}title}\newline
\NormalTok{\}}\newline
\end{Highlighting}
\end{Shaded}

{\bfseries
\begin{mydescription}escapeinside
\end{mydescription}
}

The \LaTeXTT{escapeinside} line needs an explanation. The option \LaTeXTT{escapeinside=\{A\}\{B\}} will define delimiters for escaping into LaTeX code, {\itshape \setmainfont[Path=/usr/share/fonts/truetype/cmu/,UprightFont=cmunrm.ttf,BoldFont=cmunbx.ttf,ItalicFont=cmunti.ttf,BoldItalicFont=cmunbi.ttf]{cmunti.ttf}\setmonofont[Path=/usr/share/fonts/truetype/cmu/,UprightFont=cmuntt.ttf,BoldFont=cmuntb.ttf,ItalicFont=cmunit.ttf,BoldItalicFont=cmuntx.ttf]{cmunti.ttf}\itshape i.e.}{$\text{ }$}\setmainfont[Path=/usr/share/fonts/truetype/cmu/,UprightFont=cmunrm.ttf,BoldFont=cmunbx.ttf,ItalicFont=cmunti.ttf,BoldItalicFont=cmunbi.ttf]{cmunrm.ttf}\setmonofont[Path=/usr/share/fonts/truetype/cmu/,UprightFont=cmuntt.ttf,BoldFont=cmuntb.ttf,ItalicFont=cmunit.ttf,BoldItalicFont=cmuntx.ttf]{cmunrm.ttf} all the code between the string \symbol{34}A\symbol{34} and \symbol{34}B\symbol{34} will be parsed as LaTeX over the current {\itshape \setmainfont[Path=/usr/share/fonts/truetype/cmu/,UprightFont=cmunrm.ttf,BoldFont=cmunbx.ttf,ItalicFont=cmunti.ttf,BoldItalicFont=cmunbi.ttf]{cmunti.ttf}\setmonofont[Path=/usr/share/fonts/truetype/cmu/,UprightFont=cmuntt.ttf,BoldFont=cmuntb.ttf,ItalicFont=cmunit.ttf,BoldItalicFont=cmuntx.ttf]{cmunti.ttf}\itshape listings}{$\text{ }$}\setmainfont[Path=/usr/share/fonts/truetype/cmu/,UprightFont=cmunrm.ttf,BoldFont=cmunbx.ttf,ItalicFont=cmunti.ttf,BoldItalicFont=cmunbi.ttf]{cmunrm.ttf}\setmonofont[Path=/usr/share/fonts/truetype/cmu/,UprightFont=cmuntt.ttf,BoldFont=cmuntb.ttf,ItalicFont=cmunit.ttf,BoldItalicFont=cmuntx.ttf]{cmunrm.ttf} style. In the example above, the comments for {\itshape \setmainfont[Path=/usr/share/fonts/truetype/cmu/,UprightFont=cmunrm.ttf,BoldFont=cmunbx.ttf,ItalicFont=cmunti.ttf,BoldItalicFont=cmunbi.ttf]{cmunti.ttf}\setmonofont[Path=/usr/share/fonts/truetype/cmu/,UprightFont=cmuntt.ttf,BoldFont=cmuntb.ttf,ItalicFont=cmunit.ttf,BoldItalicFont=cmuntx.ttf]{cmunti.ttf}\itshape Octave}{$\text{ }$}\setmainfont[Path=/usr/share/fonts/truetype/cmu/,UprightFont=cmunrm.ttf,BoldFont=cmunbx.ttf,ItalicFont=cmunti.ttf,BoldItalicFont=cmunbi.ttf]{cmunrm.ttf}\setmonofont[Path=/usr/share/fonts/truetype/cmu/,UprightFont=cmuntt.ttf,BoldFont=cmuntb.ttf,ItalicFont=cmunit.ttf,BoldItalicFont=cmuntx.ttf]{cmunrm.ttf} start with \LaTeXTT{\%}, and they are going to be printed in the document unless they start with \LaTeXTT{\%*}, in which case they are read as LaTeX (with all LaTeX commands fulfilled) until they\textquotesingle{}re closed with another \LaTeXTT{*)}.
If you add the above paragraph, the following can be used to alter the settings within the code:

\begin{Shaded}
\begin{Highlighting}[]

\NormalTok{\textbackslash{}lstset\{language=C,caption=\{Descriptive Caption Text\},label=DescriptiveLabel\}}
\end{Highlighting}
\end{Shaded}


There are many more options, check the official documentation.
\subsection{Style definition}
\label{597}
The package lets you define styles, {\itshape \setmainfont[Path=/usr/share/fonts/truetype/cmu/,UprightFont=cmunrm.ttf,BoldFont=cmunbx.ttf,ItalicFont=cmunti.ttf,BoldItalicFont=cmunbi.ttf]{cmunti.ttf}\setmonofont[Path=/usr/share/fonts/truetype/cmu/,UprightFont=cmuntt.ttf,BoldFont=cmuntb.ttf,ItalicFont=cmunit.ttf,BoldItalicFont=cmuntx.ttf]{cmunti.ttf}\itshape i.e.}{$\text{ }$}\setmainfont[Path=/usr/share/fonts/truetype/cmu/,UprightFont=cmunrm.ttf,BoldFont=cmunbx.ttf,ItalicFont=cmunti.ttf,BoldItalicFont=cmunbi.ttf]{cmunrm.ttf}\setmonofont[Path=/usr/share/fonts/truetype/cmu/,UprightFont=cmuntt.ttf,BoldFont=cmuntb.ttf,ItalicFont=cmunit.ttf,BoldItalicFont=cmuntx.ttf]{cmunrm.ttf} profiles specifying a set of settings. 

Example


\begin{Shaded}
\begin{Highlighting}[]

\ensuremath{\text{ }}\newline
\NormalTok{\textbackslash{}lstdefinestyle\{customc\}\{}\newline
\ensuremath{\text{ }}\ensuremath{\text{ }}\NormalTok{belowcaptionskip=1\textbackslash{}baselineskip,}\newline
\ensuremath{\text{ }}\ensuremath{\text{ }}\NormalTok{breaklines=true,}\newline
\ensuremath{\text{ }}\ensuremath{\text{ }}\NormalTok{frame=L,}\newline
\ensuremath{\text{ }}\ensuremath{\text{ }}\NormalTok{xleftmargin=\textbackslash{}parindent,}\newline
\ensuremath{\text{ }}\ensuremath{\text{ }}\NormalTok{language=C,}\newline
\ensuremath{\text{ }}\ensuremath{\text{ }}\NormalTok{showstringspaces=false,}\newline
\ensuremath{\text{ }}\ensuremath{\text{ }}\NormalTok{basicstyle=\textbackslash{}footnotesize\textbackslash{}ttfamily,}\newline
\ensuremath{\text{ }}\ensuremath{\text{ }}\NormalTok{keywordstyle=\textbackslash{}bfseries\textbackslash{}color\{green!40!black\},}\newline
\ensuremath{\text{ }}\ensuremath{\text{ }}\NormalTok{commentstyle=\textbackslash{}itshape\textbackslash{}color\{purple!40!black\},}\newline
\ensuremath{\text{ }}\ensuremath{\text{ }}\NormalTok{identifierstyle=\textbackslash{}color\{blue\},}\newline
\ensuremath{\text{ }}\ensuremath{\text{ }}\NormalTok{stringstyle=\textbackslash{}color\{orange\},}\newline
\NormalTok{\}}\newline
\ensuremath{\text{ }}\newline
\NormalTok{\textbackslash{}lstdefinestyle\{customasm\}\{}\newline
\ensuremath{\text{ }}\ensuremath{\text{ }}\NormalTok{belowcaptionskip=1\textbackslash{}baselineskip,}\newline
\ensuremath{\text{ }}\ensuremath{\text{ }}\NormalTok{frame=L,}\newline
\ensuremath{\text{ }}\ensuremath{\text{ }}\NormalTok{xleftmargin=\textbackslash{}parindent,}\newline
\ensuremath{\text{ }}\ensuremath{\text{ }}\NormalTok{language=[x86masm]Assembler,}\newline
\ensuremath{\text{ }}\ensuremath{\text{ }}\NormalTok{basicstyle=\textbackslash{}footnotesize\textbackslash{}ttfamily,}\newline
\ensuremath{\text{ }}\ensuremath{\text{ }}\NormalTok{commentstyle=\textbackslash{}itshape\textbackslash{}color\{purple!40!black\},}\newline
\NormalTok{\}}\newline
\ensuremath{\text{ }}\newline
\NormalTok{\textbackslash{}lstset\{escapechar=@,style=customc\}}\newline
\end{Highlighting}
\end{Shaded}

In our example, we only set two options globally: the default style and the escape character. Usage:


\begin{Shaded}
\begin{Highlighting}[]

\NormalTok{\textbackslash{}begin\{lstlisting\}}\newline
\OtherTok{#include\ensuremath{\text{ }}<stdio.h>}\newline
\OtherTok{#define\ensuremath{\text{ }}N\ensuremath{\text{ }}10}\newline
\CommentTok{/*\ensuremath{\text{ }}Block}\newline
\CommentTok{\ensuremath{\text{ }}*\ensuremath{\text{ }}comment\ensuremath{\text{ }}*/}\newline
\ensuremath{\text{ }}\newline
\DataTypeTok{int}\ensuremath{\text{ }}\NormalTok{main()}\newline
\NormalTok{\{}\newline
\ensuremath{\text{ }}\ensuremath{\text{ }}\ensuremath{\text{ }}\ensuremath{\text{ }}\DataTypeTok{int}\ensuremath{\text{ }}\NormalTok{i;}\newline
\ensuremath{\text{ }}\newline
\ensuremath{\text{ }}\ensuremath{\text{ }}\ensuremath{\text{ }}\ensuremath{\text{ }}\CommentTok{//\ensuremath{\text{ }}Line\ensuremath{\text{ }}comment.}\newline
\ensuremath{\text{ }}\ensuremath{\text{ }}\ensuremath{\text{ }}\ensuremath{\text{ }}\NormalTok{puts(}\StringTok{"Hello\ensuremath{\text{ }}world!"}\NormalTok{);}\newline
\ensuremath{\text{ }}\ensuremath{\text{ }}\ensuremath{\text{ }}\ensuremath{\text{ }}\newline
\ensuremath{\text{ }}\ensuremath{\text{ }}\ensuremath{\text{ }}\ensuremath{\text{ }}\KeywordTok{for}\ensuremath{\text{ }}\NormalTok{(i\ensuremath{\text{ }}=\ensuremath{\text{ }}}\DecValTok{0}\NormalTok{;\ensuremath{\text{ }}i\ensuremath{\text{ }}<\ensuremath{\text{ }}N;\ensuremath{\text{ }}i++)}\newline
\ensuremath{\text{ }}\ensuremath{\text{ }}\ensuremath{\text{ }}\ensuremath{\text{ }}\NormalTok{\{}\newline
\ensuremath{\text{ }}\ensuremath{\text{ }}\ensuremath{\text{ }}\ensuremath{\text{ }}\ensuremath{\text{ }}\ensuremath{\text{ }}\ensuremath{\text{ }}\ensuremath{\text{ }}\NormalTok{puts(}\StringTok{"LaTeX\ensuremath{\text{ }}is\ensuremath{\text{ }}also\ensuremath{\text{ }}great\ensuremath{\text{ }}for\ensuremath{\text{ }}programmers!"}\NormalTok{);}\newline
\ensuremath{\text{ }}\ensuremath{\text{ }}\ensuremath{\text{ }}\ensuremath{\text{ }}\NormalTok{\}}\newline
\ensuremath{\text{ }}\newline
\ensuremath{\text{ }}\ensuremath{\text{ }}\ensuremath{\text{ }}\ensuremath{\text{ }}\KeywordTok{return}\ensuremath{\text{ }}\DecValTok{0}\NormalTok{;}\newline
\NormalTok{\}}\newline
\NormalTok{\textbackslash{}end\{lstlisting\}}\newline
\ensuremath{\text{ }}\newline
\NormalTok{\textbackslash{}lstinputlisting[caption=Scheduler,\ensuremath{\text{ }}style=customc]\{hello.c\}}\newline
\end{Highlighting}
\end{Shaded}


The C part will print as



\begin{minipage}{1.0\linewidth}
\begin{center}
\includegraphics[width=1.0\linewidth,height=6.5in,keepaspectratio]{../images/124.\SVGExtension}
\end{center}
\raggedright{}\myfigurewithoutcaption{124}
\end{minipage}\vspace{0.75cm}


\subsection{Automating file inclusion}
\label{598}

If you have a bunch of source files you want to include,  you may find yourself doing the same thing over and over again. This is where macros show their real power.


\begin{Shaded}
\begin{Highlighting}[]

\NormalTok{\textbackslash{}newcommand\{\textbackslash{}includecode\}[2][c]\{\textbackslash{}lstinputlisting[caption=#2,\ensuremath{\text{ }}escapechar=,}\newline
\ensuremath{\text{ }}\NormalTok{style=custom#1]\{#2\}<!---->\}}\newline
\CommentTok{\%\ensuremath{\text{ }}...}\newline
\ensuremath{\text{ }}\newline
\NormalTok{\textbackslash{}includecode\{sched.c\}}\newline
\NormalTok{\textbackslash{}includecode[asm]\{sched.s\}}\newline
\CommentTok{\%\ensuremath{\text{ }}...}\newline
\ensuremath{\text{ }}\newline
\NormalTok{\textbackslash{}lstlistoflistings}\newline
\end{Highlighting}
\end{Shaded}


In this example, we create one command to ease source code inclusion. We set the default style to be {\itshape \setmainfont[Path=/usr/share/fonts/truetype/cmu/,UprightFont=cmunrm.ttf,BoldFont=cmunbx.ttf,ItalicFont=cmunti.ttf,BoldItalicFont=cmunbi.ttf]{cmunti.ttf}\setmonofont[Path=/usr/share/fonts/truetype/cmu/,UprightFont=cmuntt.ttf,BoldFont=cmuntb.ttf,ItalicFont=cmunit.ttf,BoldItalicFont=cmuntx.ttf]{cmunti.ttf}\itshape customc}\setmainfont[Path=/usr/share/fonts/truetype/cmu/,UprightFont=cmunrm.ttf,BoldFont=cmunbx.ttf,ItalicFont=cmunti.ttf,BoldItalicFont=cmunbi.ttf]{cmunrm.ttf}\setmonofont[Path=/usr/share/fonts/truetype/cmu/,UprightFont=cmuntt.ttf,BoldFont=cmuntb.ttf,ItalicFont=cmunit.ttf,BoldItalicFont=cmuntx.ttf]{cmunrm.ttf}. All listings will have their name as caption: we do not have to write the file name twice thanks to the macro.
Finally we list all listings with this command from the \LaTeXTT{listings} package.

See \mylref{837}{Macros} for more details.
\subsection{Encoding issue}
\label{599}
By default, \LaTeXTT{listings} does not support multi-{}byte encoding for source code.
The \LaTeXTT{extendedchar} option only works for 8-{}bits encodings such as latin1.

To handle UTF-{}8, you should tell listings how to interpret the special characters by defining them like so


\begin{Shaded}
\begin{Highlighting}[]

\NormalTok{\textbackslash{}lstset\{literate=}\newline
\ensuremath{\text{ }}\ensuremath{\text{ }}\NormalTok{\{á\}\{\{\textbackslash{}\textquotesingle{}a\}\}1\ensuremath{\text{ }}\{é\}\{\{\textbackslash{}\textquotesingle{}e\}\}1\ensuremath{\text{ }}\{í\}\{\{\textbackslash{}\textquotesingle{}i\}\}1\ensuremath{\text{ }}\{ó\}\{\{\textbackslash{}\textquotesingle{}o\}\}1\ensuremath{\text{ }}\{ú\}\{\{\textbackslash{}\textquotesingle{}u\}\}1}\newline
\ensuremath{\text{ }}\ensuremath{\text{ }}\NormalTok{\{Á\}\{\{\textbackslash{}\textquotesingle{}A\}\}1\ensuremath{\text{ }}\{É\}\{\{\textbackslash{}\textquotesingle{}E\}\}1\ensuremath{\text{ }}\{Í\}\{\{\textbackslash{}\textquotesingle{}I\}\}1\ensuremath{\text{ }}\{Ó\}\{\{\textbackslash{}\textquotesingle{}O\}\}1\ensuremath{\text{ }}\{Ú\}\{\{\textbackslash{}\textquotesingle{}U\}\}1}\newline
\ensuremath{\text{ }}\ensuremath{\text{ }}\NormalTok{\{à\}\{\{\textbackslash{}`a\}\}1\ensuremath{\text{ }}\{è\}\{\{\textbackslash{}`e\}\}1\ensuremath{\text{ }}\{ì\}\{\{\textbackslash{}`i\}\}1\ensuremath{\text{ }}\{ò\}\{\{\textbackslash{}`o\}\}1\ensuremath{\text{ }}\{ù\}\{\{\textbackslash{}`u\}\}1}\newline
\ensuremath{\text{ }}\ensuremath{\text{ }}\NormalTok{\{À\}\{\{\textbackslash{}`A\}\}1\ensuremath{\text{ }}\{È\}\{\{\textbackslash{}\textquotesingle{}E\}\}1\ensuremath{\text{ }}\{Ì\}\{\{\textbackslash{}`I\}\}1\ensuremath{\text{ }}\{Ò\}\{\{\textbackslash{}`O\}\}1\ensuremath{\text{ }}\{Ù\}\{\{\textbackslash{}`U\}\}1}\newline
\ensuremath{\text{ }}\ensuremath{\text{ }}\NormalTok{\{ä\}\{\{\textbackslash{}"a\}\}1\ensuremath{\text{ }}\{ë\}\{\{\textbackslash{}"e\}\}1\ensuremath{\text{ }}\{ï\}\{\{\textbackslash{}"i\}\}1\ensuremath{\text{ }}\{ö\}\{\{\textbackslash{}"o\}\}1\ensuremath{\text{ }}\{ü\}\{\{\textbackslash{}"u\}\}1}\newline
\ensuremath{\text{ }}\ensuremath{\text{ }}\NormalTok{\{Ä\}\{\{\textbackslash{}"A\}\}1\ensuremath{\text{ }}\{Ë\}\{\{\textbackslash{}"E\}\}1\ensuremath{\text{ }}\{Ï\}\{\{\textbackslash{}"I\}\}1\ensuremath{\text{ }}\{Ö\}\{\{\textbackslash{}"O\}\}1\ensuremath{\text{ }}\{Ü\}\{\{\textbackslash{}"U\}\}1}\newline
\ensuremath{\text{ }}\ensuremath{\text{ }}\NormalTok{\{â\}\{\{\textbackslash{}^a\}\}1\ensuremath{\text{ }}\{ê\}\{\{\textbackslash{}^e\}\}1\ensuremath{\text{ }}\{î\}\{\{\textbackslash{}^i\}\}1\ensuremath{\text{ }}\{ô\}\{\{\textbackslash{}^o\}\}1\ensuremath{\text{ }}\{û\}\{\{\textbackslash{}^u\}\}1}\newline
\ensuremath{\text{ }}\ensuremath{\text{ }}\NormalTok{\{Â\}\{\{\textbackslash{}^A\}\}1\ensuremath{\text{ }}\{Ê\}\{\{\textbackslash{}^E\}\}1\ensuremath{\text{ }}\{Î\}\{\{\textbackslash{}^I\}\}1\ensuremath{\text{ }}\{Ô\}\{\{\textbackslash{}^O\}\}1\ensuremath{\text{ }}\{Û\}\{\{\textbackslash{}^U\}\}1}\newline
\ensuremath{\text{ }}\ensuremath{\text{ }}\NormalTok{\{œ\}\{\{\textbackslash{}oe\}\}1\ensuremath{\text{ }}\{Œ\}\{\{\textbackslash{}OE\}\}1\ensuremath{\text{ }}\{æ\}\{\{\textbackslash{}ae\}\}1\ensuremath{\text{ }}\{Æ\}\{\{\textbackslash{}AE\}\}1\ensuremath{\text{ }}\{ß\}\{\{\textbackslash{}ss\}\}1}\newline
\ensuremath{\text{ }}\ensuremath{\text{ }}\NormalTok{\{ű\}\{\{\textbackslash{}H\{u\}\}\}1\ensuremath{\text{ }}\{Ű\}\{\{\textbackslash{}H\{U\}\}\}1\ensuremath{\text{ }}\{ő\}\{\{\textbackslash{}H\{o\}\}\}1\ensuremath{\text{ }}\{Ő\}\{\{\textbackslash{}H\{O\}\}\}1}\newline
\ensuremath{\text{ }}\ensuremath{\text{ }}\NormalTok{\{ç\}\{\{\textbackslash{}c\ensuremath{\text{ }}c\}\}1\ensuremath{\text{ }}\{Ç\}\{\{\textbackslash{}c\ensuremath{\text{ }}C\}\}1\ensuremath{\text{ }}\{ø\}\{\{\textbackslash{}o\}\}1\ensuremath{\text{ }}\{å\}\{\{\textbackslash{}r\ensuremath{\text{ }}a\}\}1\ensuremath{\text{ }}\{Å\}\{\{\textbackslash{}r\ensuremath{\text{ }}A\}\}1}\newline
\ensuremath{\text{ }}\ensuremath{\text{ }}\NormalTok{\{€\}\{\{\textbackslash{}EUR\}\}1\ensuremath{\text{ }}\{£\}\{\{\textbackslash{}pounds\}\}1}\newline
\NormalTok{\}}\newline
\end{Highlighting}
\end{Shaded}


The above table will cover most characters in latin languages.
For a more detailed explanation of the usage of the \LaTeXTT{literate} option check section 6.4 in the \myhref{http://mirrors.ctan.org/macros/latex/contrib/listings/listings.pdf}{Listings Documentation}.

Another possibility is to replace \LaTeXTT{\textbackslash{}usepackage\{listings\}} (in the preamble) with \LaTeXTT{\textbackslash{}usepackage\{listingsutf8\}}.
\subsection{Customizing captions}
\label{600}

You can have fancy captions (or titles) for your listings using the \LaTeXTT{caption} package. Here is an example for \LaTeXTT{listings}.


\begin{Shaded}
\begin{Highlighting}[]

\NormalTok{\textbackslash{}usepackage\{caption\}}\newline
\NormalTok{\textbackslash{}usepackage\{listings\}}\newline
\ensuremath{\text{ }}\newline
\NormalTok{\textbackslash{}DeclareCaptionFont\{white\}\{\ensuremath{\text{ }}\textbackslash{}color\{white\}\ensuremath{\text{ }}\}}\newline
\NormalTok{\textbackslash{}DeclareCaptionFormat\{listing\}\{}\newline
\ensuremath{\text{ }}\ensuremath{\text{ }}\NormalTok{\textbackslash{}colorbox[cmyk]\{0.43,\ensuremath{\text{ }}0.35,\ensuremath{\text{ }}0.35,0.01\ensuremath{\text{ }}\}\{}\newline
\ensuremath{\text{ }}\ensuremath{\text{ }}\ensuremath{\text{ }}\ensuremath{\text{ }}\NormalTok{\textbackslash{}parbox\{\textbackslash{}textwidth\}\{\textbackslash{}hspace\{15pt\}#1#2#3\}}\newline
\ensuremath{\text{ }}\ensuremath{\text{ }}\NormalTok{\}}\newline
\NormalTok{\}}\newline
\NormalTok{\textbackslash{}captionsetup[lstlisting]\{\ensuremath{\text{ }}format=listing,\ensuremath{\text{ }}labelfont=white,\ensuremath{\text{ }}textfont=white,}\newline
\ensuremath{\text{ }}\NormalTok{singlelinecheck=false,\ensuremath{\text{ }}margin=0pt,\ensuremath{\text{ }}font=\{bf,footnotesize\}\ensuremath{\text{ }}\}}\newline
\ensuremath{\text{ }}\newline
\CommentTok{\%\ensuremath{\text{ }}...}\newline
\ensuremath{\text{ }}\newline
\NormalTok{\textbackslash{}lstinputlisting[caption=My\ensuremath{\text{ }}caption]\{sourcefile.lang\}}\newline
\end{Highlighting}
\end{Shaded}

\section{The {\itshape \setmainfont[Path=/usr/share/fonts/truetype/cmu/,UprightFont=cmunrm.ttf,BoldFont=cmunbx.ttf,ItalicFont=cmunti.ttf,BoldItalicFont=cmunbi.ttf]{cmunti.ttf}\setmonofont[Path=/usr/share/fonts/truetype/cmu/,UprightFont=cmuntt.ttf,BoldFont=cmuntb.ttf,ItalicFont=cmunit.ttf,BoldItalicFont=cmuntx.ttf]{cmunti.ttf}\itshape minted}{$\text{ }$}\setmainfont[Path=/usr/share/fonts/truetype/cmu/,UprightFont=cmunrm.ttf,BoldFont=cmunbx.ttf,ItalicFont=cmunti.ttf,BoldItalicFont=cmunbi.ttf]{cmunrm.ttf}\setmonofont[Path=/usr/share/fonts/truetype/cmu/,UprightFont=cmuntt.ttf,BoldFont=cmuntb.ttf,ItalicFont=cmunit.ttf,BoldItalicFont=cmuntx.ttf]{cmunrm.ttf} package}
\label{601}
\LaTeXTT{minted} is an alternative to \LaTeXTT{listings} which has become popular. It uses the external Python library \myhref{http://pygments.org/}{Pygments} for code highlighting, which as of Nov 2014 boasts over 300 supported languages and text formats.

As the package relies on external Python code, the setup require a few more steps than a usual LaTeX package, so please have a look at their \myhref{https://github.com/gpoore/minted}{GitHub repo} and their \myhref{https://github.com/gpoore/minted/blob/master/source/minted.pdf}{manual}.
\section{References}
\label{602}

A lot more detailed information can be found in a \myhref{http://mirror.hmc.edu/ctan/macros/latex/contrib/listings/listings.pdf}{PDF by Carsten Heinz and Brooks Moses}.

Details and documentation about the Listings package can be found at \myhref{http://www.ctan.org/tex-archive/macros/latex/contrib/listings/}{its CTAN website}.




\myhref{https://sr.wikibooks.org/wiki/LaTeX\%2F\%D0\%9B\%D0\%B8\%D1\%81\%D1\%82\%D0\%B8\%D1\%80\%D0\%B0\%D1\%9A\%D0\%B5\%20\%D0\%B8\%D0\%B7\%D0\%B2\%D0\%BE\%D1\%80\%D0\%BD\%D0\%BE\%D0\%B3\%20\%D0\%BA\%D0\%BE\%D0\%B4\%D0\%B0}{sr:LaTeX/Листирање изворног кода}\chapter{Linguistics}

\myminitoc
\label{603}

\label{604}


There are a number of LaTeX packages available for writing linguistics papers. Various packages have been created for enumerated examples, syntactic trees, OT tableaux, feature matrices, IPA fonts, and many other applications. Some packages such as the 
\begin{Shaded}
\begin{Highlighting}[]

\NormalTok{tipa}\newline
\end{Highlighting}
\end{Shaded}
 package are effectively standard within the field, while others will vary by author preference.

Some recommended packages:\myfootnote{\myplainurl{http://cl.indiana.edu/~md7/08/latex/slides.pdf} LaTeX for Linguists presentation}
\begin{myitemize}
\item{}  Glosses: \LaTeXTT{gb4e} or \LaTeXTT{Covington};
\item{}  IPA symbols: \LaTeXTT{tipa};
\item{}  OT Tableaux: \LaTeXTT{OTtablx};
\item{}  Syntactic trees: \LaTeXTT{qtree} + \LaTeXTT{tree-{}dvips} (for drawing arrows);
\end{myitemize}

\begin{myquote}
\item{} 
\begin{myitemize}
\item{}  Alternatively, \LaTeXTT{xyling} is very powerful but not as user friendly as \LaTeXTT{qtree};
\item{}  The \LaTeXTT{\myhref{http://ctan.org/tex-archive/macros/generic/diagrams/xypic/xy}{xy}} package itself has a steep learning curve, but allows a lot of control; for simplest trees use the xymatrix feature and arrows; 
\item{}  \LaTeXTT{\myhref{http://ctan.org/pkg/tikz-qtree}{tikz-{}qtree}} has the same syntax as qtree, but uses PGF/TikZ, which allows more options for drawing arrows, etc.
\end{myitemize}

\end{myquote}

\begin{myitemize}
\item{}  Dependency trees and bubble parses:
\end{myitemize}

\begin{myquote}
\item{} 
\begin{myitemize}
\item{}  The \LaTeXTT{\myhref{http://sourceforge.net/projects/tikz-dependency/}{TikZ-{}dependency}} package provides a high-{}level, convenient interface to draw dependency graphs. It is based on PGF/TikZ but does not require prior knowledge of TikZ in order to be used productively.
\end{myitemize}

\end{myquote}

\begin{myitemize}
\item{}  Attribute-{}Value Matrices (AVMs): \LaTeXTT{\myhref{http://nlp.stanford.edu/~manning/tex/avm.sty}{avm}}
\item{}  John Frampton\textquotesingle{}s expex: \LaTeXTT{\myhref{http://www.math.neu.edu/ling/tex/}{expex}}
\end{myitemize}

\section{Enumerated examples}
\label{605}
There are several commonly used packages for creating the kinds of numbered examples that are used in linguistics publications.
\subsection{{\ttfamily {$\text{ }$}\setmainfont[Path=/usr/share/fonts/truetype/cmu/,UprightFont=cmunrm.ttf,BoldFont=cmunbx.ttf,ItalicFont=cmunti.ttf,BoldItalicFont=cmunbi.ttf]{cmuntt.ttf}\setmonofont[Path=/usr/share/fonts/truetype/cmu/,UprightFont=cmuntt.ttf,BoldFont=cmuntb.ttf,ItalicFont=cmunit.ttf,BoldItalicFont=cmuntx.ttf]{cmuntt.ttf}\ttfamily  gb4e}}
\label{606}\setmainfont[Path=/usr/share/fonts/truetype/cmu/,UprightFont=cmunrm.ttf,BoldFont=cmunbx.ttf,ItalicFont=cmunti.ttf,BoldItalicFont=cmunbi.ttf]{cmunrm.ttf}\setmonofont[Path=/usr/share/fonts/truetype/cmu/,UprightFont=cmuntt.ttf,BoldFont=cmuntb.ttf,ItalicFont=cmunit.ttf,BoldItalicFont=cmuntx.ttf]{cmunrm.ttf}
The 
\begin{Shaded}
\begin{Highlighting}[]

\NormalTok{gb4e}\newline
\end{Highlighting}
\end{Shaded}
 package\myfootnote{\myplainurl{http://ctan.mines-albi.fr/help/Catalogue/entries/gb4e.html/} The gb4e package on CTAN} is called with:


\begin{Shaded}
\begin{Highlighting}[]

\NormalTok{\textbackslash{}usepackage\{gb4e\}}\newline
\end{Highlighting}
\end{Shaded}
 

IMPORTANT: If you use gb4e package, let it be {\bfseries \setmainfont[Path=/usr/share/fonts/truetype/cmu/,UprightFont=cmunrm.ttf,BoldFont=cmunbx.ttf,ItalicFont=cmunti.ttf,BoldItalicFont=cmunbi.ttf]{cmunbx.ttf}\setmonofont[Path=/usr/share/fonts/truetype/cmu/,UprightFont=cmuntt.ttf,BoldFont=cmuntb.ttf,ItalicFont=cmunit.ttf,BoldItalicFont=cmuntx.ttf]{cmunbx.ttf}\bfseries the last \textbackslash{}usepackage call}{$\text{ }$}\setmainfont[Path=/usr/share/fonts/truetype/cmu/,UprightFont=cmunrm.ttf,BoldFont=cmunbx.ttf,ItalicFont=cmunti.ttf,BoldItalicFont=cmunbi.ttf]{cmunrm.ttf}\setmonofont[Path=/usr/share/fonts/truetype/cmu/,UprightFont=cmuntt.ttf,BoldFont=cmuntb.ttf,ItalicFont=cmunit.ttf,BoldItalicFont=cmuntx.ttf]{cmunrm.ttf} in the document\textquotesingle{}s preamble. Otherwise you may get exceeded parameter stack size error.

Examples for this package are placed within the 
\begin{Shaded}
\begin{Highlighting}[]

\NormalTok{exe}\newline
\end{Highlighting}
\end{Shaded}
 environment, and each example is introduced with the \LaTeXTT{\textbackslash{}ex} command. 


\begin{Shaded}
\begin{Highlighting}[]

\NormalTok{\textbackslash{}begin\{exe\}}\newline
	\NormalTok{\textbackslash{}ex\ensuremath{\text{ }}This\ensuremath{\text{ }}is\ensuremath{\text{ }}an\ensuremath{\text{ }}example.}\newline
\NormalTok{\textbackslash{}end\{exe\}}\newline
\end{Highlighting}
\end{Shaded}


produces:




\begin{minipage}{0.50000\textwidth}
\begin{center}
\includegraphics[width=1.0\textwidth,height=6.5in,keepaspectratio]{../images/125.png}
\end{center}
\raggedright{}\myfigurewithoutcaption{125}
\end{minipage}\vspace{0.75cm}




Multiple examples can be included within the environment, and each will have its own number.


\begin{Shaded}
\begin{Highlighting}[]

\NormalTok{\textbackslash{}begin\{exe\}}\newline
	\NormalTok{\textbackslash{}ex\ensuremath{\text{ }}This\ensuremath{\text{ }}is\ensuremath{\text{ }}the\ensuremath{\text{ }}first\ensuremath{\text{ }}example.}\newline
	\NormalTok{\textbackslash{}ex\ensuremath{\text{ }}This\ensuremath{\text{ }}is\ensuremath{\text{ }}the\ensuremath{\text{ }}second\ensuremath{\text{ }}example.}\newline
	\NormalTok{\textbackslash{}ex\ensuremath{\text{ }}This\ensuremath{\text{ }}is\ensuremath{\text{ }}the\ensuremath{\text{ }}third.}\newline
\NormalTok{\textbackslash{}end\{exe\}}\newline
\end{Highlighting}
\end{Shaded}


produces:




\begin{minipage}{0.62500\textwidth}
\begin{center}
\includegraphics[width=1.0\textwidth,height=6.5in,keepaspectratio]{../images/126.png}
\end{center}
\raggedright{}\myfigurewithoutcaption{126}
\end{minipage}\vspace{0.75cm}




To create nested lists of examples, the 
\begin{Shaded}
\begin{Highlighting}[]

\NormalTok{xlist}\newline
\end{Highlighting}
\end{Shaded}
 enviroment is used.


\begin{Shaded}
\begin{Highlighting}[]

\NormalTok{\textbackslash{}begin\{exe\}}\newline
\ensuremath{\text{ }}\ensuremath{\text{ }}\ensuremath{\text{ }}\ensuremath{\text{ }}\NormalTok{\textbackslash{}ex\ensuremath{\text{ }}\textbackslash{}begin\{xlist\}}\newline
\ensuremath{\text{ }}\ensuremath{\text{ }}\ensuremath{\text{ }}\ensuremath{\text{ }}\ensuremath{\text{ }}\ensuremath{\text{ }}\ensuremath{\text{ }}\ensuremath{\text{ }}\NormalTok{\textbackslash{}ex\ensuremath{\text{ }}This\ensuremath{\text{ }}is\ensuremath{\text{ }}a\ensuremath{\text{ }}sub-example.}\newline
\ensuremath{\text{ }}\ensuremath{\text{ }}\ensuremath{\text{ }}\ensuremath{\text{ }}\ensuremath{\text{ }}\ensuremath{\text{ }}\ensuremath{\text{ }}\ensuremath{\text{ }}\NormalTok{\textbackslash{}ex\ensuremath{\text{ }}This\ensuremath{\text{ }}is\ensuremath{\text{ }}a\ensuremath{\text{ }}second\ensuremath{\text{ }}sub-example.}\newline
\ensuremath{\text{ }}\ensuremath{\text{ }}\ensuremath{\text{ }}\ensuremath{\text{ }}\ensuremath{\text{ }}\ensuremath{\text{ }}\ensuremath{\text{ }}\ensuremath{\text{ }}\NormalTok{\textbackslash{}ex\ensuremath{\text{ }}\textbackslash{}begin\{xlist\}}\newline
\ensuremath{\text{ }}\ensuremath{\text{ }}\ensuremath{\text{ }}\ensuremath{\text{ }}\ensuremath{\text{ }}\ensuremath{\text{ }}\ensuremath{\text{ }}\ensuremath{\text{ }}\ensuremath{\text{ }}\ensuremath{\text{ }}\ensuremath{\text{ }}\ensuremath{\text{ }}\NormalTok{\textbackslash{}ex\ensuremath{\text{ }}This\ensuremath{\text{ }}is\ensuremath{\text{ }}a\ensuremath{\text{ }}sub-sub-example.}\newline
\ensuremath{\text{ }}\ensuremath{\text{ }}\ensuremath{\text{ }}\ensuremath{\text{ }}\ensuremath{\text{ }}\ensuremath{\text{ }}\ensuremath{\text{ }}\ensuremath{\text{ }}\ensuremath{\text{ }}\ensuremath{\text{ }}\ensuremath{\text{ }}\ensuremath{\text{ }}\NormalTok{\textbackslash{}ex\ensuremath{\text{ }}This\ensuremath{\text{ }}is\ensuremath{\text{ }}a\ensuremath{\text{ }}second\ensuremath{\text{ }}sub-sub-example.}\newline
\ensuremath{\text{ }}\ensuremath{\text{ }}\ensuremath{\text{ }}\ensuremath{\text{ }}\ensuremath{\text{ }}\ensuremath{\text{ }}\ensuremath{\text{ }}\ensuremath{\text{ }}\NormalTok{\textbackslash{}end\{xlist\}}\newline
\ensuremath{\text{ }}\ensuremath{\text{ }}\ensuremath{\text{ }}\ensuremath{\text{ }}\NormalTok{\textbackslash{}end\{xlist\}}\newline
\NormalTok{\textbackslash{}end\{exe\}}\newline
\end{Highlighting}
\end{Shaded}


produces:




\begin{minipage}{0.87500\textwidth}
\begin{center}
\includegraphics[width=1.0\textwidth,height=6.5in,keepaspectratio]{../images/127.png}
\end{center}
\raggedright{}\myfigurewithoutcaption{127}
\end{minipage}\vspace{0.75cm}




For notating acceptability judgments, the \LaTeXTT{\textbackslash{}ex} command can take an optional argument. When including a judgment marker, the corresponding sentence must be surrounded by braces.


\begin{Shaded}
\begin{Highlighting}[]

\NormalTok{\textbackslash{}begin\{exe\}}\newline
	\NormalTok{\textbackslash{}ex\ensuremath{\text{ }}This\ensuremath{\text{ }}sentence\ensuremath{\text{ }}is\ensuremath{\text{ }}grammatical\ensuremath{\text{ }}English.}\newline
	\NormalTok{\textbackslash{}ex[*]\ensuremath{\text{ }}\{This\ensuremath{\text{ }}sentence\ensuremath{\text{ }}English\ensuremath{\text{ }}in\ensuremath{\text{ }}ungrammatical\ensuremath{\text{ }}is.\}}\newline
\NormalTok{\textbackslash{}end\{exe\}}\newline
\end{Highlighting}
\end{Shaded}


produces:




\begin{minipage}{1.0\linewidth}
\begin{center}
\includegraphics[width=1.0\linewidth,height=6.5in,keepaspectratio]{../images/128.png}
\end{center}
\raggedright{}\myfigurewithoutcaption{128}
\end{minipage}\vspace{0.75cm}




Referencing examples in text works as it does in normal LaTeX documents. See the \mylref{417}{labeling and cross-{}referencing} section for more details.


\begin{Shaded}
\begin{Highlighting}[]

\NormalTok{\textbackslash{}begin\{exe\}}\newline
	\NormalTok{\textbackslash{}ex\textbackslash{}label\{ex1\}\ensuremath{\text{ }}Godzilla\ensuremath{\text{ }}destroyed\ensuremath{\text{ }}the\ensuremath{\text{ }}city.}\newline
	\NormalTok{\textbackslash{}ex\textbackslash{}label\{ex2\}\ensuremath{\text{ }}Godzilla\ensuremath{\text{ }}roared.}\newline
\NormalTok{\textbackslash{}end\{exe\}}\newline
\NormalTok{Sentence\ensuremath{\text{ }}(\textbackslash{}ref\{ex1\})\ensuremath{\text{ }}contains\ensuremath{\text{ }}two\ensuremath{\text{ }}arguments,\ensuremath{\text{ }}but\ensuremath{\text{ }}(\textbackslash{}ref\{ex2\})\ensuremath{\text{ }}contains\ensuremath{\text{ }}only\ensuremath{\text{ }}one.}\newline
\end{Highlighting}
\end{Shaded}



Further details can be found in the full documentation available \myhref{http://ctan.mackichan.com/macros/latex/contrib/gb4e/gb4e-doc.pdf}{here}.
\subsection{{\ttfamily {$\text{ }$}\setmainfont[Path=/usr/share/fonts/truetype/cmu/,UprightFont=cmunrm.ttf,BoldFont=cmunbx.ttf,ItalicFont=cmunti.ttf,BoldItalicFont=cmunbi.ttf]{cmuntt.ttf}\setmonofont[Path=/usr/share/fonts/truetype/cmu/,UprightFont=cmuntt.ttf,BoldFont=cmuntb.ttf,ItalicFont=cmunit.ttf,BoldItalicFont=cmuntx.ttf]{cmuntt.ttf}\ttfamily  lingmacros}}
\label{607}\setmainfont[Path=/usr/share/fonts/truetype/cmu/,UprightFont=cmunrm.ttf,BoldFont=cmunbx.ttf,ItalicFont=cmunti.ttf,BoldItalicFont=cmunbi.ttf]{cmunrm.ttf}\setmonofont[Path=/usr/share/fonts/truetype/cmu/,UprightFont=cmuntt.ttf,BoldFont=cmuntb.ttf,ItalicFont=cmunit.ttf,BoldItalicFont=cmuntx.ttf]{cmunrm.ttf}
The 
\begin{Shaded}
\begin{Highlighting}[]

\NormalTok{lingmacros}\newline
\end{Highlighting}
\end{Shaded}
 package\myfootnote{\myplainurl{http://ctan.org/tex-archive/macros/latex209/contrib/trees/tree-dvips} The lingmacros package on CTAN} created by Emma Pease is an alternate method for example numbering. This package uses two main commands, \LaTeXTT{\textbackslash{}enumsentence} and \LaTeXTT{\textbackslash{}eenumsentence}. The former is used for singleton examples, while the latter command is used for nested examples.


\begin{Shaded}
\begin{Highlighting}[]

\NormalTok{\textbackslash{}enumsentence\{This\ensuremath{\text{ }}is\ensuremath{\text{ }}an\ensuremath{\text{ }}example.\}}\newline
\end{Highlighting}
\end{Shaded}




\begin{minipage}{0.50000\textwidth}
\begin{center}
\includegraphics[width=1.0\textwidth,height=6.5in,keepaspectratio]{../images/129.png}
\end{center}
\raggedright{}\myfigurewithoutcaption{129}
\end{minipage}\vspace{0.75cm}





\begin{Shaded}
\begin{Highlighting}[]

\NormalTok{\textbackslash{}enumsentence\{This\ensuremath{\text{ }}is\ensuremath{\text{ }}the\ensuremath{\text{ }}first\ensuremath{\text{ }}example.\}}\newline
\NormalTok{\textbackslash{}enumsentence\{This\ensuremath{\text{ }}is\ensuremath{\text{ }}the\ensuremath{\text{ }}second\ensuremath{\text{ }}example.\}}\newline
\NormalTok{\textbackslash{}enumsentence\{This\ensuremath{\text{ }}is\ensuremath{\text{ }}the\ensuremath{\text{ }}third.\}}\newline
\end{Highlighting}
\end{Shaded}

			



\begin{minipage}{0.62500\textwidth}
\begin{center}
\includegraphics[width=1.0\textwidth,height=6.5in,keepaspectratio]{../images/130.png}
\end{center}
\raggedright{}\myfigurewithoutcaption{130}
\end{minipage}\vspace{0.75cm}




Multiply nested examples make use of the normal LaTeX \mylref{186}{list environments}.


\begin{Shaded}
\begin{Highlighting}[]

\NormalTok{\textbackslash{}eenumsentence\{\textbackslash{}item\ensuremath{\text{ }}This\ensuremath{\text{ }}is\ensuremath{\text{ }}a\ensuremath{\text{ }}sub-example.}\newline
		\NormalTok{\textbackslash{}item\ensuremath{\text{ }}This\ensuremath{\text{ }}is\ensuremath{\text{ }}a\ensuremath{\text{ }}second\ensuremath{\text{ }}sub-example.}\newline
		\NormalTok{\textbackslash{}item\ensuremath{\text{ }}\textbackslash{}begin\{enumerate\}}\newline
			\NormalTok{\textbackslash{}item\ensuremath{\text{ }}This\ensuremath{\text{ }}is\ensuremath{\text{ }}sub-sub-example.}\newline
			\NormalTok{\textbackslash{}item\ensuremath{\text{ }}This\ensuremath{\text{ }}is\ensuremath{\text{ }}a\ensuremath{\text{ }}second\ensuremath{\text{ }}sub-sub-example.}\newline
			\NormalTok{\textbackslash{}end\{enumerate\}}\newline
\ensuremath{\text{ }}\ensuremath{\text{ }}\ensuremath{\text{ }}\ensuremath{\text{ }}\ensuremath{\text{ }}\ensuremath{\text{ }}\ensuremath{\text{ }}\ensuremath{\text{ }}\ensuremath{\text{ }}\ensuremath{\text{ }}\ensuremath{\text{ }}\ensuremath{\text{ }}\ensuremath{\text{ }}\ensuremath{\text{ }}\ensuremath{\text{ }}\ensuremath{\text{ }}\ensuremath{\text{ }}\NormalTok{\}}\newline
\end{Highlighting}
\end{Shaded}


produces:




\begin{minipage}{0.87500\textwidth}
\begin{center}
\includegraphics[width=1.0\textwidth,height=6.5in,keepaspectratio]{../images/131.png}
\end{center}
\raggedright{}\myfigurewithoutcaption{131}
\end{minipage}\vspace{0.75cm}




Full documentation can be found \myhref{http://mirrors.ibiblio.org/pub/mirrors/CTAN/macros/latex209/contrib/trees/tree-dvips/lingmacros-manual.pdf}{here}.
\section{Syntactic trees}
\label{608}
Often, linguists will have to illustrate the syntactic structure of a sentence. One device for doing this is through syntactic trees. Unfortunately, trees look very different in different grammar formalisms, and different LaTeX packages are suited for different formalisms.
\subsection{Constituent trees}
\label{609}
While there are several packages for drawing syntactic trees available for LaTeX, this article focuses on the qtree and xyling packages.
\subsubsection{qtree}
\label{610}
Drawing trees with qtree is relatively straightforward. First, the \LaTeXTT{qtree} package has to be included in the document\textquotesingle{}s preamble:

\begin{Shaded}
\begin{Highlighting}[]

\NormalTok{\textbackslash{}usepackage\{qtree\}}\newline
\end{Highlighting}
\end{Shaded}

A new tree is started using the {\bfseries \setmainfont[Path=/usr/share/fonts/truetype/cmu/,UprightFont=cmunrm.ttf,BoldFont=cmunbx.ttf,ItalicFont=cmunti.ttf,BoldItalicFont=cmunbi.ttf]{cmunbx.ttf}\setmonofont[Path=/usr/share/fonts/truetype/cmu/,UprightFont=cmuntt.ttf,BoldFont=cmuntb.ttf,ItalicFont=cmunit.ttf,BoldItalicFont=cmuntx.ttf]{cmunbx.ttf}\bfseries \textbackslash{}Tree}{$\text{ }$}\setmainfont[Path=/usr/share/fonts/truetype/cmu/,UprightFont=cmunrm.ttf,BoldFont=cmunbx.ttf,ItalicFont=cmunti.ttf,BoldItalicFont=cmunbi.ttf]{cmunrm.ttf}\setmonofont[Path=/usr/share/fonts/truetype/cmu/,UprightFont=cmuntt.ttf,BoldFont=cmuntb.ttf,ItalicFont=cmunit.ttf,BoldItalicFont=cmuntx.ttf]{cmunrm.ttf} command, each (sub-{})tree is indicated by brackets {\bfseries \setmainfont[Path=/usr/share/fonts/truetype/cmu/,UprightFont=cmunrm.ttf,BoldFont=cmunbx.ttf,ItalicFont=cmunti.ttf,BoldItalicFont=cmunbi.ttf]{cmunbx.ttf}\setmonofont[Path=/usr/share/fonts/truetype/cmu/,UprightFont=cmuntt.ttf,BoldFont=cmuntb.ttf,ItalicFont=cmunit.ttf,BoldItalicFont=cmuntx.ttf]{cmunbx.ttf}\bfseries {$\text{[}$} {$\text{]}$}}\setmainfont[Path=/usr/share/fonts/truetype/cmu/,UprightFont=cmunrm.ttf,BoldFont=cmunbx.ttf,ItalicFont=cmunti.ttf,BoldItalicFont=cmunbi.ttf]{cmunrm.ttf}\setmonofont[Path=/usr/share/fonts/truetype/cmu/,UprightFont=cmuntt.ttf,BoldFont=cmuntb.ttf,ItalicFont=cmunit.ttf,BoldItalicFont=cmuntx.ttf]{cmunrm.ttf}. The root of a (sub-{})tree is always preceded by a {\bfseries \setmainfont[Path=/usr/share/fonts/truetype/cmu/,UprightFont=cmunrm.ttf,BoldFont=cmunbx.ttf,ItalicFont=cmunti.ttf,BoldItalicFont=cmunbi.ttf]{cmunbx.ttf}\setmonofont[Path=/usr/share/fonts/truetype/cmu/,UprightFont=cmuntt.ttf,BoldFont=cmuntb.ttf,ItalicFont=cmunit.ttf,BoldItalicFont=cmuntx.ttf]{cmunbx.ttf}\bfseries .}\setmainfont[Path=/usr/share/fonts/truetype/cmu/,UprightFont=cmunrm.ttf,BoldFont=cmunbx.ttf,ItalicFont=cmunti.ttf,BoldItalicFont=cmunbi.ttf]{cmunrm.ttf}\setmonofont[Path=/usr/share/fonts/truetype/cmu/,UprightFont=cmuntt.ttf,BoldFont=cmuntb.ttf,ItalicFont=cmunit.ttf,BoldItalicFont=cmuntx.ttf]{cmunrm.ttf}, leaf nodes are simply expressed by their labels.

For example, the following code

\begin{Shaded}
\begin{Highlighting}[]

\NormalTok{\textbackslash{}Tree\ensuremath{\text{ }}[.S\ensuremath{\text{ }}[.NP\ensuremath{\text{ }}LaTeX\ensuremath{\text{ }}]\ensuremath{\text{ }}[.VP\ensuremath{\text{ }}[.V\ensuremath{\text{ }}is\ensuremath{\text{ }}]\ensuremath{\text{ }}[.NP\ensuremath{\text{ }}fun\ensuremath{\text{ }}]\ensuremath{\text{ }}]\ensuremath{\text{ }}]}\newline
\end{Highlighting}
\end{Shaded}

produces this syntactic tree as output:



\begin{minipage}{1.0\linewidth}
\begin{center}
\includegraphics[width=1.0\linewidth,height=6.5in,keepaspectratio]{../images/132.png}
\end{center}
\raggedright{}\myfigurewithoutcaption{132}
\end{minipage}\vspace{0.75cm}



Note that the spaces before the closing brackets are {\bfseries \setmainfont[Path=/usr/share/fonts/truetype/cmu/,UprightFont=cmunrm.ttf,BoldFont=cmunbx.ttf,ItalicFont=cmunti.ttf,BoldItalicFont=cmunbi.ttf]{cmunbx.ttf}\setmonofont[Path=/usr/share/fonts/truetype/cmu/,UprightFont=cmuntt.ttf,BoldFont=cmuntb.ttf,ItalicFont=cmunit.ttf,BoldItalicFont=cmuntx.ttf]{cmunbx.ttf}\bfseries mandatory}\setmainfont[Path=/usr/share/fonts/truetype/cmu/,UprightFont=cmunrm.ttf,BoldFont=cmunbx.ttf,ItalicFont=cmunti.ttf,BoldItalicFont=cmunbi.ttf]{cmunrm.ttf}\setmonofont[Path=/usr/share/fonts/truetype/cmu/,UprightFont=cmuntt.ttf,BoldFont=cmuntb.ttf,ItalicFont=cmunit.ttf,BoldItalicFont=cmuntx.ttf]{cmunrm.ttf}.

By default, qtree centers syntactic trees on the page. This behaviour can be turned off by either specifying the behaviour when loading the package

\begin{Shaded}
\begin{Highlighting}[]

\NormalTok{\textbackslash{}usepackage[nocenter]\{qtree\}\ensuremath{\text{ }}}\CommentTok{\%\ensuremath{\text{ }}do\ensuremath{\text{ }}not\ensuremath{\text{ }}center\ensuremath{\text{ }}trees}\newline
\end{Highlighting}
\end{Shaded}

or via the command

\begin{Shaded}
\begin{Highlighting}[]

\NormalTok{\textbackslash{}qtreecenterfalse\ensuremath{\text{ }}}\CommentTok{\%\ensuremath{\text{ }}do\ensuremath{\text{ }}not\ensuremath{\text{ }}center\ensuremath{\text{ }}trees\ensuremath{\text{ }}from\ensuremath{\text{ }}here\ensuremath{\text{ }}on}\newline
\end{Highlighting}
\end{Shaded}

anywhere in the document. The effect of the latter can be undone by using the command

\begin{Shaded}
\begin{Highlighting}[]

\NormalTok{\textbackslash{}qtreecentertrue\ensuremath{\text{ }}}\CommentTok{\%\ensuremath{\text{ }}center\ensuremath{\text{ }}trees\ensuremath{\text{ }}from\ensuremath{\text{ }}here\ensuremath{\text{ }}on}\newline
\end{Highlighting}
\end{Shaded}


IMPORTANT: If you use gb4e package, let it be the last {\bfseries \setmainfont[Path=/usr/share/fonts/truetype/cmu/,UprightFont=cmunrm.ttf,BoldFont=cmunbx.ttf,ItalicFont=cmunti.ttf,BoldItalicFont=cmunbi.ttf]{cmunbx.ttf}\setmonofont[Path=/usr/share/fonts/truetype/cmu/,UprightFont=cmuntt.ttf,BoldFont=cmuntb.ttf,ItalicFont=cmunit.ttf,BoldItalicFont=cmuntx.ttf]{cmunbx.ttf}\bfseries \textbackslash{}usepackage}{$\text{ }$}\setmainfont[Path=/usr/share/fonts/truetype/cmu/,UprightFont=cmunrm.ttf,BoldFont=cmunbx.ttf,ItalicFont=cmunti.ttf,BoldItalicFont=cmunbi.ttf]{cmunrm.ttf}\setmonofont[Path=/usr/share/fonts/truetype/cmu/,UprightFont=cmuntt.ttf,BoldFont=cmuntb.ttf,ItalicFont=cmunit.ttf,BoldItalicFont=cmuntx.ttf]{cmunrm.ttf} call in the document\textquotesingle{}s preamble. Otherwise you may get exceeded parameter stack size error.
\subsubsection{tikz-{}qtree}
\label{611}
Using the same syntax as qtree, tikz-{}qtree is another easy-{}to-{}use alternative for drawing syntactic trees.

For simple trees, tikz-{}qtree is completely interchangable with qtree. However, some of qtree\textquotesingle{}s advanced features are implemented in a different way, or not at all. On the other hand, tikz-{}qtree provides other features such as controlling the direction of the tree\textquotesingle{}s growth (top to bottom, left to right etc.) or different styles for edges.

To use the \LaTeXTT{tikz-{}qtree} package for drawing trees, put the following into the document\textquotesingle{}s preamble:

\begin{Shaded}
\begin{Highlighting}[]

\NormalTok{\textbackslash{}usepackage\{tikz\}}\newline
\NormalTok{\textbackslash{}usepackage\{tikz-qtree\}}\newline
\end{Highlighting}
\end{Shaded}


The syntax of \LaTeXTT{tikz-{}qtree} and result when drawing a simple tree is the same as for \LaTeXTT{qtree}.


\begin{Shaded}
\begin{Highlighting}[]

\NormalTok{\textbackslash{}Tree\ensuremath{\text{ }}[.S\ensuremath{\text{ }}[.NP\ensuremath{\text{ }}LaTeX\ensuremath{\text{ }}]\ensuremath{\text{ }}[.VP\ensuremath{\text{ }}[.V\ensuremath{\text{ }}is\ensuremath{\text{ }}]\ensuremath{\text{ }}[.NP\ensuremath{\text{ }}fun\ensuremath{\text{ }}]\ensuremath{\text{ }}]\ensuremath{\text{ }}]}\newline
\end{Highlighting}
\end{Shaded}




\begin{minipage}{1.0\linewidth}
\begin{center}
\includegraphics[width=1.0\linewidth,height=6.5in,keepaspectratio]{../images/133.png}
\end{center}
\raggedright{}\myfigurewithoutcaption{133}
\end{minipage}\vspace{0.75cm}



Note that, other than for qtree, trees are not centered by default. To center them, put them into a centered environment:


\begin{Shaded}
\begin{Highlighting}[]

\NormalTok{\textbackslash{}begin\{center\}}\newline
\NormalTok{\textbackslash{}Tree\ensuremath{\text{ }}[.S\ensuremath{\text{ }}[.NP\ensuremath{\text{ }}LaTeX\ensuremath{\text{ }}]\ensuremath{\text{ }}[.VP\ensuremath{\text{ }}[.V\ensuremath{\text{ }}is\ensuremath{\text{ }}]\ensuremath{\text{ }}[.NP\ensuremath{\text{ }}fun\ensuremath{\text{ }}]\ensuremath{\text{ }}]\ensuremath{\text{ }}]}\newline
\NormalTok{\textbackslash{}end\{center\}}\newline
\end{Highlighting}
\end{Shaded}


For setting the style of trees, tikz-{}qtree provides the {\bfseries \setmainfont[Path=/usr/share/fonts/truetype/cmu/,UprightFont=cmunrm.ttf,BoldFont=cmunbx.ttf,ItalicFont=cmunti.ttf,BoldItalicFont=cmunbi.ttf]{cmunbx.ttf}\setmonofont[Path=/usr/share/fonts/truetype/cmu/,UprightFont=cmuntt.ttf,BoldFont=cmuntb.ttf,ItalicFont=cmunit.ttf,BoldItalicFont=cmuntx.ttf]{cmunbx.ttf}\bfseries \textbackslash{}tikzset}{$\text{ }$}\setmainfont[Path=/usr/share/fonts/truetype/cmu/,UprightFont=cmunrm.ttf,BoldFont=cmunbx.ttf,ItalicFont=cmunti.ttf,BoldItalicFont=cmunbi.ttf]{cmunrm.ttf}\setmonofont[Path=/usr/share/fonts/truetype/cmu/,UprightFont=cmuntt.ttf,BoldFont=cmuntb.ttf,ItalicFont=cmunit.ttf,BoldItalicFont=cmuntx.ttf]{cmunrm.ttf} command.
For example, to make a tree grow from left to right instead of from top to bottom, use the following code:


\begin{Shaded}
\begin{Highlighting}[]

\NormalTok{\textbackslash{}tikzset\{grow\textquotesingle{}=right\}\ensuremath{\text{ }}}\CommentTok{\%\ensuremath{\text{ }}make\ensuremath{\text{ }}trees\ensuremath{\text{ }}grow\ensuremath{\text{ }}from\ensuremath{\text{ }}left\ensuremath{\text{ }}to\ensuremath{\text{ }}right}\newline
\NormalTok{\textbackslash{}tikzset\{every\ensuremath{\text{ }}tree\ensuremath{\text{ }}node/.style=\{anchor=base\ensuremath{\text{ }}west\}\}\ensuremath{\text{ }}}\CommentTok{\%\ensuremath{\text{ }}allign\ensuremath{\text{ }}nodes\ensuremath{\text{ }}of\ensuremath{\text{ }}the\ensuremath{\text{ }}tree}\newline
\ensuremath{\text{ }}\NormalTok{to\ensuremath{\text{ }}the\ensuremath{\text{ }}left\ensuremath{\text{ }}(west)}\newline
\NormalTok{\textbackslash{}Tree\ensuremath{\text{ }}[.S\ensuremath{\text{ }}[.NP\ensuremath{\text{ }}LaTeX\ensuremath{\text{ }}]\ensuremath{\text{ }}[.VP\ensuremath{\text{ }}[.V\ensuremath{\text{ }}is\ensuremath{\text{ }}]\ensuremath{\text{ }}[.NP\ensuremath{\text{ }}fun\ensuremath{\text{ }}]\ensuremath{\text{ }}]\ensuremath{\text{ }}]}\newline
\end{Highlighting}
\end{Shaded}




\begin{minipage}{1.0\linewidth}
\begin{center}
\includegraphics[width=1.0\linewidth,height=6.5in,keepaspectratio]{../images/134.png}
\end{center}
\raggedright{}\myfigurewithoutcaption{134}
\end{minipage}\vspace{0.75cm}



The above code changes the default orientation for {\bfseries \setmainfont[Path=/usr/share/fonts/truetype/cmu/,UprightFont=cmunrm.ttf,BoldFont=cmunbx.ttf,ItalicFont=cmunti.ttf,BoldItalicFont=cmunbi.ttf]{cmunbx.ttf}\setmonofont[Path=/usr/share/fonts/truetype/cmu/,UprightFont=cmuntt.ttf,BoldFont=cmuntb.ttf,ItalicFont=cmunit.ttf,BoldItalicFont=cmuntx.ttf]{cmunbx.ttf}\bfseries all}{$\text{ }$}\setmainfont[Path=/usr/share/fonts/truetype/cmu/,UprightFont=cmunrm.ttf,BoldFont=cmunbx.ttf,ItalicFont=cmunti.ttf,BoldItalicFont=cmunbi.ttf]{cmunrm.ttf}\setmonofont[Path=/usr/share/fonts/truetype/cmu/,UprightFont=cmuntt.ttf,BoldFont=cmuntb.ttf,ItalicFont=cmunit.ttf,BoldItalicFont=cmuntx.ttf]{cmunrm.ttf} trees that are defined after {\bfseries \setmainfont[Path=/usr/share/fonts/truetype/cmu/,UprightFont=cmunrm.ttf,BoldFont=cmunbx.ttf,ItalicFont=cmunti.ttf,BoldItalicFont=cmunbi.ttf]{cmunbx.ttf}\setmonofont[Path=/usr/share/fonts/truetype/cmu/,UprightFont=cmuntt.ttf,BoldFont=cmuntb.ttf,ItalicFont=cmunit.ttf,BoldItalicFont=cmuntx.ttf]{cmunbx.ttf}\bfseries \textbackslash{}tikzset}{$\text{ }$}\setmainfont[Path=/usr/share/fonts/truetype/cmu/,UprightFont=cmunrm.ttf,BoldFont=cmunbx.ttf,ItalicFont=cmunti.ttf,BoldItalicFont=cmunbi.ttf]{cmunrm.ttf}\setmonofont[Path=/usr/share/fonts/truetype/cmu/,UprightFont=cmuntt.ttf,BoldFont=cmuntb.ttf,ItalicFont=cmunit.ttf,BoldItalicFont=cmuntx.ttf]{cmunrm.ttf} commands. To only change the direction of a single tree, it has to be put into a {\bfseries \setmainfont[Path=/usr/share/fonts/truetype/cmu/,UprightFont=cmunrm.ttf,BoldFont=cmunbx.ttf,ItalicFont=cmunti.ttf,BoldItalicFont=cmunbi.ttf]{cmunbx.ttf}\setmonofont[Path=/usr/share/fonts/truetype/cmu/,UprightFont=cmuntt.ttf,BoldFont=cmuntb.ttf,ItalicFont=cmunit.ttf,BoldItalicFont=cmuntx.ttf]{cmunbx.ttf}\bfseries \textbackslash{}tikzpicture}{$\text{ }$}\setmainfont[Path=/usr/share/fonts/truetype/cmu/,UprightFont=cmunrm.ttf,BoldFont=cmunbx.ttf,ItalicFont=cmunti.ttf,BoldItalicFont=cmunbi.ttf]{cmunrm.ttf}\setmonofont[Path=/usr/share/fonts/truetype/cmu/,UprightFont=cmuntt.ttf,BoldFont=cmuntb.ttf,ItalicFont=cmunit.ttf,BoldItalicFont=cmuntx.ttf]{cmunrm.ttf} environment:


\begin{Shaded}
\begin{Highlighting}[]

\NormalTok{\textbackslash{}begin\{tikzpicture\}\ensuremath{\text{ }}}\CommentTok{\%\ensuremath{\text{ }}all\ensuremath{\text{ }}changes\ensuremath{\text{ }}only\ensuremath{\text{ }}affect\ensuremath{\text{ }}trees\ensuremath{\text{ }}within\ensuremath{\text{ }}this\ensuremath{\text{ }}environment}\newline
\NormalTok{\textbackslash{}tikzset\{grow\textquotesingle{}=right\}\ensuremath{\text{ }}}\CommentTok{\%\ensuremath{\text{ }}make\ensuremath{\text{ }}trees\ensuremath{\text{ }}grow\ensuremath{\text{ }}from\ensuremath{\text{ }}left\ensuremath{\text{ }}to\ensuremath{\text{ }}right}\newline
\NormalTok{\textbackslash{}tikzset\{every\ensuremath{\text{ }}tree\ensuremath{\text{ }}node/.style=\{anchor=base\ensuremath{\text{ }}west\}\}\ensuremath{\text{ }}}\CommentTok{\%\ensuremath{\text{ }}allign\ensuremath{\text{ }}nodes\ensuremath{\text{ }}of\ensuremath{\text{ }}the\ensuremath{\text{ }}tree}\newline
\ensuremath{\text{ }}\NormalTok{to\ensuremath{\text{ }}the\ensuremath{\text{ }}left\ensuremath{\text{ }}(west)}\newline
\NormalTok{\textbackslash{}Tree\ensuremath{\text{ }}[.S\ensuremath{\text{ }}[.NP\ensuremath{\text{ }}LaTeX\ensuremath{\text{ }}]\ensuremath{\text{ }}[.VP\ensuremath{\text{ }}[.V\ensuremath{\text{ }}is\ensuremath{\text{ }}]\ensuremath{\text{ }}[.NP\ensuremath{\text{ }}fun\ensuremath{\text{ }}]\ensuremath{\text{ }}]\ensuremath{\text{ }}]}\newline
\NormalTok{\textbackslash{}end\{tikzpicture\}}\newline
\end{Highlighting}
\end{Shaded}


\subsection{Dependency Trees}
\label{612}

Dependency trees can take multiple visual forms. Commonly, they quite resemble phrase structure trees. Alternatively, they can be captured by brackets drawn above running text.

\subsubsection{Two-{}dimensional Dependency Trees}
\label{613}

These can be either achieved using the fairly universal drawing package TikZ, like so:


\begin{Shaded}
\begin{Highlighting}[]

\CommentTok{\%\ensuremath{\text{ }}In\ensuremath{\text{ }}the\ensuremath{\text{ }}preamble:}\newline
\NormalTok{\textbackslash{}usepackage\{tikz\}}\newline
\ensuremath{\text{ }}\newline
\CommentTok{\%\ensuremath{\text{ }}In\ensuremath{\text{ }}the\ensuremath{\text{ }}document:}\newline
\NormalTok{\textbackslash{}begin\{tikzpicture\}}\newline
	\NormalTok{\textbackslash{}node\ensuremath{\text{ }}(is-root)\ensuremath{\text{ }}\{is\}}\newline
		\NormalTok{[sibling\ensuremath{\text{ }}distance=3cm]}\newline
		\NormalTok{child\ensuremath{\text{ }}\{\ensuremath{\text{ }}node\ensuremath{\text{ }}\{this\}\ensuremath{\text{ }}\}}\newline
		\NormalTok{child\ensuremath{\text{ }}\{}\newline
			\NormalTok{node\ensuremath{\text{ }}\{tree\}}\newline
				\NormalTok{[sibling\ensuremath{\text{ }}distance=1.5cm]}\newline
				\NormalTok{child\ensuremath{\text{ }}\{\ensuremath{\text{ }}node\ensuremath{\text{ }}\{an\}\ensuremath{\text{ }}\}}\newline
				\NormalTok{child\ensuremath{\text{ }}\{\ensuremath{\text{ }}node\ensuremath{\text{ }}\{example\}\ensuremath{\text{ }}\}}\newline
				\NormalTok{child\ensuremath{\text{ }}\{\ensuremath{\text{ }}node\ensuremath{\text{ }}\{.\}\ensuremath{\text{ }}\}}\newline
				\NormalTok{child[missing]}\newline
		\NormalTok{\};}\newline
	\NormalTok{\textbackslash{}path\ensuremath{\text{ }}(is-root)\ensuremath{\text{ }}+(0,-2.5\textbackslash{}tikzleveldistance)}\newline
		\NormalTok{node\ensuremath{\text{ }}\{\textbackslash{}textit\{This\ensuremath{\text{ }}is\ensuremath{\text{ }}an\ensuremath{\text{ }}example\ensuremath{\text{ }}tree.\}\};}\newline
\NormalTok{\textbackslash{}end\{tikzpicture\}}\newline
\end{Highlighting}
\end{Shaded}

which gives you the following drawing:



\begin{minipage}{1.0\linewidth}
\begin{center}
\includegraphics[width=1.0\linewidth,height=6.5in,keepaspectratio]{../images/135.png}
\end{center}
\raggedright{}\myfigurewithcaption{135}{A dependency tree created using TikZ}
\end{minipage}\vspace{0.75cm}



TikZ has the advantage that it allows for generating PDF directly from the LaTeX source, without need for any detour of compiling to DVI using {\ttfamily \setmainfont[Path=/usr/share/fonts/truetype/cmu/,UprightFont=cmunrm.ttf,BoldFont=cmunbx.ttf,ItalicFont=cmunti.ttf,BoldItalicFont=cmunbi.ttf]{cmuntt.ttf}\setmonofont[Path=/usr/share/fonts/truetype/cmu/,UprightFont=cmuntt.ttf,BoldFont=cmuntb.ttf,ItalicFont=cmunit.ttf,BoldItalicFont=cmuntx.ttf]{cmuntt.ttf}\ttfamily latex}\setmainfont[Path=/usr/share/fonts/truetype/cmu/,UprightFont=cmunrm.ttf,BoldFont=cmunbx.ttf,ItalicFont=cmunti.ttf,BoldItalicFont=cmunbi.ttf]{cmunrm.ttf}\setmonofont[Path=/usr/share/fonts/truetype/cmu/,UprightFont=cmuntt.ttf,BoldFont=cmuntb.ttf,ItalicFont=cmunit.ttf,BoldItalicFont=cmuntx.ttf]{cmunrm.ttf}, and then converting to PDF probably via PS using tools such as {\ttfamily \setmainfont[Path=/usr/share/fonts/truetype/cmu/,UprightFont=cmunrm.ttf,BoldFont=cmunbx.ttf,ItalicFont=cmunti.ttf,BoldItalicFont=cmunbi.ttf]{cmuntt.ttf}\setmonofont[Path=/usr/share/fonts/truetype/cmu/,UprightFont=cmuntt.ttf,BoldFont=cmuntb.ttf,ItalicFont=cmunit.ttf,BoldItalicFont=cmuntx.ttf]{cmuntt.ttf}\ttfamily dvips}{$\text{ }$}\setmainfont[Path=/usr/share/fonts/truetype/cmu/,UprightFont=cmunrm.ttf,BoldFont=cmunbx.ttf,ItalicFont=cmunti.ttf,BoldItalicFont=cmunbi.ttf]{cmunrm.ttf}\setmonofont[Path=/usr/share/fonts/truetype/cmu/,UprightFont=cmuntt.ttf,BoldFont=cmuntb.ttf,ItalicFont=cmunit.ttf,BoldItalicFont=cmuntx.ttf]{cmunrm.ttf} and {\ttfamily \setmainfont[Path=/usr/share/fonts/truetype/cmu/,UprightFont=cmunrm.ttf,BoldFont=cmunbx.ttf,ItalicFont=cmunti.ttf,BoldItalicFont=cmunbi.ttf]{cmuntt.ttf}\setmonofont[Path=/usr/share/fonts/truetype/cmu/,UprightFont=cmuntt.ttf,BoldFont=cmuntb.ttf,ItalicFont=cmunit.ttf,BoldItalicFont=cmuntx.ttf]{cmuntt.ttf}\ttfamily ps2pdf}\setmainfont[Path=/usr/share/fonts/truetype/cmu/,UprightFont=cmunrm.ttf,BoldFont=cmunbx.ttf,ItalicFont=cmunti.ttf,BoldItalicFont=cmunbi.ttf]{cmunrm.ttf}\setmonofont[Path=/usr/share/fonts/truetype/cmu/,UprightFont=cmuntt.ttf,BoldFont=cmuntb.ttf,ItalicFont=cmunit.ttf,BoldItalicFont=cmuntx.ttf]{cmunrm.ttf}. Latter is the case of another package based on the package {\bfseries \setmainfont[Path=/usr/share/fonts/truetype/cmu/,UprightFont=cmunrm.ttf,BoldFont=cmunbx.ttf,ItalicFont=cmunti.ttf,BoldItalicFont=cmunbi.ttf]{cmunbx.ttf}\setmonofont[Path=/usr/share/fonts/truetype/cmu/,UprightFont=cmuntt.ttf,BoldFont=cmuntb.ttf,ItalicFont=cmunit.ttf,BoldItalicFont=cmuntx.ttf]{cmunbx.ttf}\bfseries xy}\setmainfont[Path=/usr/share/fonts/truetype/cmu/,UprightFont=cmunrm.ttf,BoldFont=cmunbx.ttf,ItalicFont=cmunti.ttf,BoldItalicFont=cmunbi.ttf]{cmunrm.ttf}\setmonofont[Path=/usr/share/fonts/truetype/cmu/,UprightFont=cmuntt.ttf,BoldFont=cmuntb.ttf,ItalicFont=cmunit.ttf,BoldItalicFont=cmuntx.ttf]{cmunrm.ttf}, namely {\bfseries \setmainfont[Path=/usr/share/fonts/truetype/cmu/,UprightFont=cmunrm.ttf,BoldFont=cmunbx.ttf,ItalicFont=cmunti.ttf,BoldItalicFont=cmunbi.ttf]{cmunbx.ttf}\setmonofont[Path=/usr/share/fonts/truetype/cmu/,UprightFont=cmuntt.ttf,BoldFont=cmuntb.ttf,ItalicFont=cmunit.ttf,BoldItalicFont=cmuntx.ttf]{cmunbx.ttf}\bfseries xyling}\setmainfont[Path=/usr/share/fonts/truetype/cmu/,UprightFont=cmunrm.ttf,BoldFont=cmunbx.ttf,ItalicFont=cmunti.ttf,BoldItalicFont=cmunbi.ttf]{cmunrm.ttf}\setmonofont[Path=/usr/share/fonts/truetype/cmu/,UprightFont=cmuntt.ttf,BoldFont=cmuntb.ttf,ItalicFont=cmunit.ttf,BoldItalicFont=cmuntx.ttf]{cmunrm.ttf}.

The code for a similar tree using {\bfseries \setmainfont[Path=/usr/share/fonts/truetype/cmu/,UprightFont=cmunrm.ttf,BoldFont=cmunbx.ttf,ItalicFont=cmunti.ttf,BoldItalicFont=cmunbi.ttf]{cmunbx.ttf}\setmonofont[Path=/usr/share/fonts/truetype/cmu/,UprightFont=cmuntt.ttf,BoldFont=cmuntb.ttf,ItalicFont=cmunit.ttf,BoldItalicFont=cmuntx.ttf]{cmunbx.ttf}\bfseries xyling}{$\text{ }$}\setmainfont[Path=/usr/share/fonts/truetype/cmu/,UprightFont=cmunrm.ttf,BoldFont=cmunbx.ttf,ItalicFont=cmunti.ttf,BoldItalicFont=cmunbi.ttf]{cmunrm.ttf}\setmonofont[Path=/usr/share/fonts/truetype/cmu/,UprightFont=cmuntt.ttf,BoldFont=cmuntb.ttf,ItalicFont=cmunit.ttf,BoldItalicFont=cmuntx.ttf]{cmunrm.ttf} might look like:


\begin{Shaded}
\begin{Highlighting}[]

\CommentTok{\%\ensuremath{\text{ }}In\ensuremath{\text{ }}the\ensuremath{\text{ }}preamble:}\newline
\NormalTok{\textbackslash{}usepackage\{xyling\}}\newline
\ensuremath{\text{ }}\newline
\CommentTok{\%\ensuremath{\text{ }}In\ensuremath{\text{ }}the\ensuremath{\text{ }}document:}\newline
\NormalTok{\textbackslash{}Tree\{	\&\ensuremath{\text{ }}\textbackslash{}K\{is\}\textbackslash{}B\{dl\}\textbackslash{}B\{drr\}\ensuremath{\text{ }}\textbackslash{}\textbackslash{}}\newline
\ensuremath{\text{ }}\ensuremath{\text{ }}\ensuremath{\text{ }}\ensuremath{\text{ }}\ensuremath{\text{ }}\ensuremath{\text{ }}\ensuremath{\text{ }}\ensuremath{\text{ }}\NormalTok{\textbackslash{}K\{this\}\ensuremath{\text{ }}\&\&\&\ensuremath{\text{ }}\textbackslash{}K\{tree\}\textbackslash{}B\{dll\}\textbackslash{}B\{dl\}\textbackslash{}B\{dr\}\ensuremath{\text{ }}\textbackslash{}\textbackslash{}}\newline
\ensuremath{\text{ }}\ensuremath{\text{ }}\ensuremath{\text{ }}\ensuremath{\text{ }}\ensuremath{\text{ }}\ensuremath{\text{ }}\ensuremath{\text{ }}\ensuremath{\text{ }}\NormalTok{\&\ensuremath{\text{ }}\textbackslash{}K\{an\}\ensuremath{\text{ }}\&\ensuremath{\text{ }}\textbackslash{}K\{example\}\ensuremath{\text{ }}\&\&\ensuremath{\text{ }}\textbackslash{}K\{.\}\ensuremath{\text{ }}\}}\newline
\ensuremath{\text{ }}\newline
\NormalTok{\textbackslash{}medskip}\newline
\NormalTok{\textbackslash{}textit\{This\ensuremath{\text{ }}is\ensuremath{\text{ }}an\ensuremath{\text{ }}example\ensuremath{\text{ }}tree.\}}\newline
\end{Highlighting}
\end{Shaded}

which gives you a drawing like this:



\begin{minipage}{1.0\linewidth}
\begin{center}
\includegraphics[width=1.0\linewidth,height=6.5in,keepaspectratio]{../images/136.png}
\end{center}
\raggedright{}\myfigurewithcaption{136}{A dependency tree created using {\ttfamily \setmainfont[Path=/usr/share/fonts/truetype/cmu/,UprightFont=cmunrm.ttf,BoldFont=cmunbx.ttf,ItalicFont=cmunti.ttf,BoldItalicFont=cmunbi.ttf]{cmuntt.ttf}\setmonofont[Path=/usr/share/fonts/truetype/cmu/,UprightFont=cmuntt.ttf,BoldFont=cmuntb.ttf,ItalicFont=cmunit.ttf,BoldItalicFont=cmuntx.ttf]{cmuntt.ttf}\ttfamily xyling}}
\end{minipage}\vspace{0.75cm}


\subsubsection{Dependency Trees as Brackets above Text}
\label{614}

One way to typeset dependency brackets above running text is using the package {\bfseries \setmainfont[Path=/usr/share/fonts/truetype/cmu/,UprightFont=cmunrm.ttf,BoldFont=cmunbx.ttf,ItalicFont=cmunti.ttf,BoldItalicFont=cmunbi.ttf]{cmunbx.ttf}\setmonofont[Path=/usr/share/fonts/truetype/cmu/,UprightFont=cmuntt.ttf,BoldFont=cmuntb.ttf,ItalicFont=cmunit.ttf,BoldItalicFont=cmuntx.ttf]{cmunbx.ttf}\bfseries xytree}\setmainfont[Path=/usr/share/fonts/truetype/cmu/,UprightFont=cmunrm.ttf,BoldFont=cmunbx.ttf,ItalicFont=cmunti.ttf,BoldItalicFont=cmunbi.ttf]{cmunrm.ttf}\setmonofont[Path=/usr/share/fonts/truetype/cmu/,UprightFont=cmuntt.ttf,BoldFont=cmuntb.ttf,ItalicFont=cmunit.ttf,BoldItalicFont=cmuntx.ttf]{cmunrm.ttf}. It gives you fairly good control of how the brackets are typeset but requires compiling the LaTeX code to DVI (and perhaps converting to PDF using the tools {\ttfamily \setmainfont[Path=/usr/share/fonts/truetype/cmu/,UprightFont=cmunrm.ttf,BoldFont=cmunbx.ttf,ItalicFont=cmunti.ttf,BoldItalicFont=cmunbi.ttf]{cmuntt.ttf}\setmonofont[Path=/usr/share/fonts/truetype/cmu/,UprightFont=cmuntt.ttf,BoldFont=cmuntb.ttf,ItalicFont=cmunit.ttf,BoldItalicFont=cmuntx.ttf]{cmuntt.ttf}\ttfamily dvips}{$\text{ }$}\setmainfont[Path=/usr/share/fonts/truetype/cmu/,UprightFont=cmunrm.ttf,BoldFont=cmunbx.ttf,ItalicFont=cmunti.ttf,BoldItalicFont=cmunbi.ttf]{cmunrm.ttf}\setmonofont[Path=/usr/share/fonts/truetype/cmu/,UprightFont=cmuntt.ttf,BoldFont=cmuntb.ttf,ItalicFont=cmunit.ttf,BoldItalicFont=cmuntx.ttf]{cmunrm.ttf} and {\ttfamily \setmainfont[Path=/usr/share/fonts/truetype/cmu/,UprightFont=cmunrm.ttf,BoldFont=cmunbx.ttf,ItalicFont=cmunti.ttf,BoldItalicFont=cmunbi.ttf]{cmuntt.ttf}\setmonofont[Path=/usr/share/fonts/truetype/cmu/,UprightFont=cmuntt.ttf,BoldFont=cmuntb.ttf,ItalicFont=cmunit.ttf,BoldItalicFont=cmuntx.ttf]{cmuntt.ttf}\ttfamily ps2pdf}{$\text{ }$}\setmainfont[Path=/usr/share/fonts/truetype/cmu/,UprightFont=cmunrm.ttf,BoldFont=cmunbx.ttf,ItalicFont=cmunti.ttf,BoldItalicFont=cmunbi.ttf]{cmunrm.ttf}\setmonofont[Path=/usr/share/fonts/truetype/cmu/,UprightFont=cmuntt.ttf,BoldFont=cmuntb.ttf,ItalicFont=cmunit.ttf,BoldItalicFont=cmuntx.ttf]{cmunrm.ttf} later).

An example code:


\begin{Shaded}
\begin{Highlighting}[]

\CommentTok{\%\ensuremath{\text{ }}In\ensuremath{\text{ }}the\ensuremath{\text{ }}preamble:}\newline
\NormalTok{\textbackslash{}usepackage\{xytree\}}\newline
\ensuremath{\text{ }}\newline
\CommentTok{\%\ensuremath{\text{ }}In\ensuremath{\text{ }}the\ensuremath{\text{ }}document:}\newline
\NormalTok{\textbackslash{}xytext\{}\newline
\ensuremath{\text{ }}\ensuremath{\text{ }}\NormalTok{\textbackslash{}xybarnode\{Peter\}\ensuremath{\text{ }}\&~~~\&}\newline
\ensuremath{\text{ }}\ensuremath{\text{ }}\NormalTok{\textbackslash{}xybarnode\{and\}}\newline
\ensuremath{\text{ }}\ensuremath{\text{ }}\ensuremath{\text{ }}\ensuremath{\text{ }}\NormalTok{\textbackslash{}xybarconnect(UL,U)\{-2\}"_\{\textbackslash{}small\ensuremath{\text{ }}conj\}"}\newline
\ensuremath{\text{ }}\ensuremath{\text{ }}\ensuremath{\text{ }}\ensuremath{\text{ }}\NormalTok{\textbackslash{}xybarconnect(UR,U)\{2\}"^\{\textbackslash{}small\ensuremath{\text{ }}conj\}"}\newline
\ensuremath{\text{ }}\ensuremath{\text{ }}\ensuremath{\text{ }}\ensuremath{\text{ }}\NormalTok{\&~~~\&}\newline
\ensuremath{\text{ }}\ensuremath{\text{ }}\NormalTok{\textbackslash{}xybarnode\{Mary\}\ensuremath{\text{ }}\&~~~\&}\newline
\ensuremath{\text{ }}\ensuremath{\text{ }}\NormalTok{\textbackslash{}xybarnode\{bought\}}\newline
\ensuremath{\text{ }}\ensuremath{\text{ }}\ensuremath{\text{ }}\ensuremath{\text{ }}\NormalTok{\textbackslash{}xybarconnect[8](UL,U)\{-4\}"_\{\textbackslash{}small\ensuremath{\text{ }}subj\}"}\newline
\ensuremath{\text{ }}\ensuremath{\text{ }}\ensuremath{\text{ }}\ensuremath{\text{ }}\NormalTok{\textbackslash{}xybarconnect[13]\{6\}"^\{\textbackslash{}small\ensuremath{\text{ }}punct\}"}\newline
\ensuremath{\text{ }}\ensuremath{\text{ }}\ensuremath{\text{ }}\ensuremath{\text{ }}\NormalTok{\textbackslash{}xybarconnect[8](UR,U)\{4\}"^\{\textbackslash{}small\ensuremath{\text{ }}obj\}"}\newline
\ensuremath{\text{ }}\ensuremath{\text{ }}\ensuremath{\text{ }}\ensuremath{\text{ }}\NormalTok{\&~~~\&}\newline
\ensuremath{\text{ }}\ensuremath{\text{ }}\NormalTok{\textbackslash{}xybarnode\{a\}\ensuremath{\text{ }}\&~~~\&}\newline
\ensuremath{\text{ }}\ensuremath{\text{ }}\NormalTok{\textbackslash{}xybarnode\{car\}}\newline
\ensuremath{\text{ }}\ensuremath{\text{ }}\ensuremath{\text{ }}\ensuremath{\text{ }}\NormalTok{\textbackslash{}xybarconnect(UL,U)\{-2\}"_\{\textbackslash{}small\ensuremath{\text{ }}det\}"}\newline
\ensuremath{\text{ }}\ensuremath{\text{ }}\ensuremath{\text{ }}\ensuremath{\text{ }}\NormalTok{\&~~~\&}\newline
\ensuremath{\text{ }}\ensuremath{\text{ }}\NormalTok{\textbackslash{}xybarnode\{.\}}\newline
\NormalTok{\}}\newline
\end{Highlighting}
\end{Shaded}


results in:



\begin{minipage}{1.0\linewidth}
\begin{center}
\includegraphics[width=1.0\linewidth,height=6.5in,keepaspectratio]{../images/137.png}
\end{center}
\raggedright{}\myfigurewithcaption{137}{A dependency tree above running text created using {\ttfamily \setmainfont[Path=/usr/share/fonts/truetype/cmu/,UprightFont=cmunrm.ttf,BoldFont=cmunbx.ttf,ItalicFont=cmunti.ttf,BoldItalicFont=cmunbi.ttf]{cmuntt.ttf}\setmonofont[Path=/usr/share/fonts/truetype/cmu/,UprightFont=cmuntt.ttf,BoldFont=cmuntb.ttf,ItalicFont=cmunit.ttf,BoldItalicFont=cmuntx.ttf]{cmuntt.ttf}\ttfamily xytree}}
\end{minipage}\vspace{0.75cm}


\subsubsection{Dependency Trees using TikZ-{}dependency}
\label{615}

The package provides high level commands to design and style dependency graphs. To draw a graph, you only need to create a {\ttfamily \setmainfont[Path=/usr/share/fonts/truetype/cmu/,UprightFont=cmunrm.ttf,BoldFont=cmunbx.ttf,ItalicFont=cmunti.ttf,BoldItalicFont=cmunbi.ttf]{cmuntt.ttf}\setmonofont[Path=/usr/share/fonts/truetype/cmu/,UprightFont=cmuntt.ttf,BoldFont=cmuntb.ttf,ItalicFont=cmunit.ttf,BoldItalicFont=cmuntx.ttf]{cmuntt.ttf}\ttfamily dependency}{$\text{ }$}\setmainfont[Path=/usr/share/fonts/truetype/cmu/,UprightFont=cmunrm.ttf,BoldFont=cmunbx.ttf,ItalicFont=cmunti.ttf,BoldItalicFont=cmunbi.ttf]{cmunrm.ttf}\setmonofont[Path=/usr/share/fonts/truetype/cmu/,UprightFont=cmuntt.ttf,BoldFont=cmuntb.ttf,ItalicFont=cmunit.ttf,BoldItalicFont=cmuntx.ttf]{cmunrm.ttf} environment, write the text of the sentence within the {\ttfamily \setmainfont[Path=/usr/share/fonts/truetype/cmu/,UprightFont=cmunrm.ttf,BoldFont=cmunbx.ttf,ItalicFont=cmunti.ttf,BoldItalicFont=cmunbi.ttf]{cmuntt.ttf}\setmonofont[Path=/usr/share/fonts/truetype/cmu/,UprightFont=cmuntt.ttf,BoldFont=cmuntb.ttf,ItalicFont=cmunit.ttf,BoldItalicFont=cmuntx.ttf]{cmuntt.ttf}\ttfamily deptext}{$\text{ }$}\setmainfont[Path=/usr/share/fonts/truetype/cmu/,UprightFont=cmunrm.ttf,BoldFont=cmunbx.ttf,ItalicFont=cmunti.ttf,BoldItalicFont=cmunbi.ttf]{cmunrm.ttf}\setmonofont[Path=/usr/share/fonts/truetype/cmu/,UprightFont=cmuntt.ttf,BoldFont=cmuntb.ttf,ItalicFont=cmunit.ttf,BoldItalicFont=cmuntx.ttf]{cmunrm.ttf} environment and use {\ttfamily \setmainfont[Path=/usr/share/fonts/truetype/cmu/,UprightFont=cmunrm.ttf,BoldFont=cmunbx.ttf,ItalicFont=cmunti.ttf,BoldItalicFont=cmunbi.ttf]{cmuntt.ttf}\setmonofont[Path=/usr/share/fonts/truetype/cmu/,UprightFont=cmuntt.ttf,BoldFont=cmuntb.ttf,ItalicFont=cmunit.ttf,BoldItalicFont=cmuntx.ttf]{cmuntt.ttf}\ttfamily depedge}{$\text{ }$}\setmainfont[Path=/usr/share/fonts/truetype/cmu/,UprightFont=cmunrm.ttf,BoldFont=cmunbx.ttf,ItalicFont=cmunti.ttf,BoldItalicFont=cmunbi.ttf]{cmunrm.ttf}\setmonofont[Path=/usr/share/fonts/truetype/cmu/,UprightFont=cmuntt.ttf,BoldFont=cmuntb.ttf,ItalicFont=cmunit.ttf,BoldItalicFont=cmuntx.ttf]{cmunrm.ttf} commands to draw the edges. Global and local optional parameters can be used to style and fine tune the looks of the graph, as shown in the following example:


\begin{Shaded}
\begin{Highlighting}[]

\CommentTok{\%\ensuremath{\text{ }}In\ensuremath{\text{ }}the\ensuremath{\text{ }}preamble:}\newline
\NormalTok{\textbackslash{}usepackage\{tikz-dependency\}}\newline
\ensuremath{\text{ }}\newline
\CommentTok{\%\ensuremath{\text{ }}In\ensuremath{\text{ }}the\ensuremath{\text{ }}document:}\newline
\NormalTok{\textbackslash{}begin\{dependency\}[theme\ensuremath{\text{ }}=\ensuremath{\text{ }}simple]}\newline
\ensuremath{\text{ }}\ensuremath{\text{ }}\ensuremath{\text{ }}\NormalTok{\textbackslash{}begin\{deptext\}[column\ensuremath{\text{ }}sep=1em]}\newline
\ensuremath{\text{ }}\ensuremath{\text{ }}\ensuremath{\text{ }}\ensuremath{\text{ }}\ensuremath{\text{ }}\ensuremath{\text{ }}\NormalTok{A\ensuremath{\text{ }}\textbackslash{}\&\ensuremath{\text{ }}hearing\ensuremath{\text{ }}\textbackslash{}\&\ensuremath{\text{ }}is\ensuremath{\text{ }}\textbackslash{}\&\ensuremath{\text{ }}scheduled\ensuremath{\text{ }}\textbackslash{}\&\ensuremath{\text{ }}on\ensuremath{\text{ }}\textbackslash{}\&\ensuremath{\text{ }}the\ensuremath{\text{ }}\textbackslash{}\&\ensuremath{\text{ }}issue\ensuremath{\text{ }}\textbackslash{}\&\ensuremath{\text{ }}today\ensuremath{\text{ }}\textbackslash{}\&\ensuremath{\text{ }}.\ensuremath{\text{ }}\textbackslash{}\textbackslash{}}\newline
\ensuremath{\text{ }}\ensuremath{\text{ }}\ensuremath{\text{ }}\NormalTok{\textbackslash{}end\{deptext\}}\newline
\ensuremath{\text{ }}\ensuremath{\text{ }}\ensuremath{\text{ }}\NormalTok{\textbackslash{}deproot\{3\}\{ROOT\}}\newline
\ensuremath{\text{ }}\ensuremath{\text{ }}\ensuremath{\text{ }}\NormalTok{\textbackslash{}depedge\{2\}\{1\}\{ATT\}}\newline
\ensuremath{\text{ }}\ensuremath{\text{ }}\ensuremath{\text{ }}\NormalTok{\textbackslash{}depedge[edge\ensuremath{\text{ }}start\ensuremath{\text{ }}x\ensuremath{\text{ }}offset=-6pt]\{2\}\{5\}\{ATT\}}\newline
\ensuremath{\text{ }}\ensuremath{\text{ }}\ensuremath{\text{ }}\NormalTok{\textbackslash{}depedge\{3\}\{2\}\{SBJ\}}\newline
\ensuremath{\text{ }}\ensuremath{\text{ }}\ensuremath{\text{ }}\NormalTok{\textbackslash{}depedge\{3\}\{9\}\{PU\}}\newline
\ensuremath{\text{ }}\ensuremath{\text{ }}\ensuremath{\text{ }}\NormalTok{\textbackslash{}depedge\{3\}\{4\}\{VC\}}\newline
\ensuremath{\text{ }}\ensuremath{\text{ }}\ensuremath{\text{ }}\NormalTok{\textbackslash{}depedge\{4\}\{8\}\{TMP\}}\newline
\ensuremath{\text{ }}\ensuremath{\text{ }}\ensuremath{\text{ }}\NormalTok{\textbackslash{}depedge\{5\}\{7\}\{PC\}}\newline
\ensuremath{\text{ }}\ensuremath{\text{ }}\ensuremath{\text{ }}\NormalTok{\textbackslash{}depedge[arc\ensuremath{\text{ }}angle=50]\{7\}\{6\}\{ATT\}}\newline
\NormalTok{\textbackslash{}end\{dependency\}}\newline
\end{Highlighting}
\end{Shaded}


This code snippet would produce the following result:



\begin{minipage}{1.0\linewidth}
\begin{center}
\includegraphics[width=1.0\linewidth,height=6.5in,keepaspectratio]{../images/138.png}
\end{center}
\raggedright{}\myfigurewithcaption{138}{A dependency tree drawn with TikZ-{}dependency.}
\end{minipage}\vspace{0.75cm}


\section{Glosses}
\label{616}
Below, it is explained how to make glossed examples with different packages.
\subsection{With {\ttfamily \setmainfont[Path=/usr/share/fonts/truetype/cmu/,UprightFont=cmunrm.ttf,BoldFont=cmunbx.ttf,ItalicFont=cmunti.ttf,BoldItalicFont=cmunbi.ttf]{cmuntt.ttf}\setmonofont[Path=/usr/share/fonts/truetype/cmu/,UprightFont=cmuntt.ttf,BoldFont=cmuntb.ttf,ItalicFont=cmunit.ttf,BoldItalicFont=cmuntx.ttf]{cmuntt.ttf}\ttfamily gb4e}}
\label{617}{$\text{ }$}\setmainfont[Path=/usr/share/fonts/truetype/cmu/,UprightFont=cmunrm.ttf,BoldFont=cmunbx.ttf,ItalicFont=cmunti.ttf,BoldItalicFont=cmunbi.ttf]{cmunrm.ttf}\setmonofont[Path=/usr/share/fonts/truetype/cmu/,UprightFont=cmuntt.ttf,BoldFont=cmuntb.ttf,ItalicFont=cmunit.ttf,BoldItalicFont=cmuntx.ttf]{cmunrm.ttf} 
To create a glossed example, use the normal 
\begin{Shaded}
\begin{Highlighting}[]

\NormalTok{exe}\newline
\end{Highlighting}
\end{Shaded}
 environment. But after the \LaTeXTT{\textbackslash{}ex} tag, introduce the example and its gloss using \LaTeXTT{\textbackslash{}gll} and the translation after it with \LaTeXTT{\textbackslash{}trans} tag. 


\begin{Shaded}
\begin{Highlighting}[]

\NormalTok{\textbackslash{}begin\{exe\}}\newline
\NormalTok{\textbackslash{}ex\ensuremath{\text{ }}}\newline
\NormalTok{\textbackslash{}gll\ensuremath{\text{ }}Кот\ensuremath{\text{ }}ест\ensuremath{\text{ }}сметану\textbackslash{}\textbackslash{}}\newline
\NormalTok{cat.NOM\ensuremath{\text{ }}eat.3.SG.PRS\ensuremath{\text{ }}sour-cream.ACC\textbackslash{}\textbackslash{}}\newline
\NormalTok{\textbackslash{}trans\ensuremath{\text{ }}`The\ensuremath{\text{ }}cat\ensuremath{\text{ }}eats\ensuremath{\text{ }}sour\ensuremath{\text{ }}cream\textquotesingle{}}\newline
\NormalTok{\textbackslash{}end\{exe\}}\newline
\end{Highlighting}
\end{Shaded}


The code will produce the following output:



\begin{minipage}{1.0\linewidth}
\begin{center}
\includegraphics[width=1.0\linewidth,height=6.5in,keepaspectratio]{../images/139.png}
\end{center}
\raggedright{}\myfigurewithoutcaption{139}
\end{minipage}\vspace{0.75cm}




Vertically aligned glosses are separated by spaces, so if it\textquotesingle{}s necessary to include a space in part the gloss, simply enclose the connected parts inside braces.


\begin{Shaded}
\begin{Highlighting}[]

\NormalTok{\textbackslash{}begin\{exe\}}\newline
\NormalTok{\textbackslash{}ex	}\newline
\NormalTok{\textbackslash{}gll\ensuremath{\text{ }}Pekka\ensuremath{\text{ }}pel\textbackslash{}"astyi\ensuremath{\text{ }}karhusta.\textbackslash{}\textbackslash{}}\newline
\ensuremath{\text{ }}\ensuremath{\text{ }}\ensuremath{\text{ }}\ensuremath{\text{ }}\ensuremath{\text{ }}\NormalTok{Pekka\ensuremath{\text{ }}\{became\ensuremath{\text{ }}afraid\}\ensuremath{\text{ }}bear.ELA\textbackslash{}\textbackslash{}}\newline
\NormalTok{\textbackslash{}trans\ensuremath{\text{ }}`Pekka\ensuremath{\text{ }}became\ensuremath{\text{ }}afraid\ensuremath{\text{ }}because\ensuremath{\text{ }}of\ensuremath{\text{ }}the/a\ensuremath{\text{ }}bear.\textquotesingle{}}\newline
\NormalTok{\textbackslash{}end\{exe\}}\newline
\end{Highlighting}
\end{Shaded}

\subsection{With {\ttfamily \setmainfont[Path=/usr/share/fonts/truetype/cmu/,UprightFont=cmunrm.ttf,BoldFont=cmunbx.ttf,ItalicFont=cmunti.ttf,BoldItalicFont=cmunbi.ttf]{cmuntt.ttf}\setmonofont[Path=/usr/share/fonts/truetype/cmu/,UprightFont=cmuntt.ttf,BoldFont=cmuntb.ttf,ItalicFont=cmunit.ttf,BoldItalicFont=cmuntx.ttf]{cmuntt.ttf}\ttfamily lingmacros}}
\label{618}{$\text{ }$}\setmainfont[Path=/usr/share/fonts/truetype/cmu/,UprightFont=cmunrm.ttf,BoldFont=cmunbx.ttf,ItalicFont=cmunti.ttf,BoldItalicFont=cmunbi.ttf]{cmunrm.ttf}\setmonofont[Path=/usr/share/fonts/truetype/cmu/,UprightFont=cmuntt.ttf,BoldFont=cmuntb.ttf,ItalicFont=cmunit.ttf,BoldItalicFont=cmuntx.ttf]{cmunrm.ttf} 

The 
\begin{Shaded}
\begin{Highlighting}[]

\NormalTok{lingmacros}\newline
\end{Highlighting}
\end{Shaded}
 package uses the \LaTeXTT{\textbackslash{}shortex} command to introduce glossed examples inside the \LaTeXTT{\textbackslash{}enumsentence} and \LaTeXTT{\textbackslash{}eenumsentence} commands. This command takes four arguments and builds off the normal \mylref{272}{tabular} environment. Its first argument specifies the number of columns in the gloss. The second and third arguments give the text and its gloss respectively, and items within each column are divided by the usual {\ttfamily \setmainfont[Path=/usr/share/fonts/truetype/cmu/,UprightFont=cmunrm.ttf,BoldFont=cmunbx.ttf,ItalicFont=cmunti.ttf,BoldItalicFont=cmunbi.ttf]{cmuntt.ttf}\setmonofont[Path=/usr/share/fonts/truetype/cmu/,UprightFont=cmuntt.ttf,BoldFont=cmuntb.ttf,ItalicFont=cmunit.ttf,BoldItalicFont=cmuntx.ttf]{cmuntt.ttf}\ttfamily \&}{$\text{ }$}\setmainfont[Path=/usr/share/fonts/truetype/cmu/,UprightFont=cmunrm.ttf,BoldFont=cmunbx.ttf,ItalicFont=cmunti.ttf,BoldItalicFont=cmunbi.ttf]{cmunrm.ttf}\setmonofont[Path=/usr/share/fonts/truetype/cmu/,UprightFont=cmuntt.ttf,BoldFont=cmuntb.ttf,ItalicFont=cmunit.ttf,BoldItalicFont=cmuntx.ttf]{cmunrm.ttf} tabular separator. The fourth argument is the translation.


\begin{Shaded}
\begin{Highlighting}[]

\NormalTok{\textbackslash{}enumsentence\{\textbackslash{}shortex\{3\}}\newline
		\NormalTok{\{Pekka\ensuremath{\text{ }}\&\ensuremath{\text{ }}pel\textbackslash{}"astyi\ensuremath{\text{ }}\&\ensuremath{\text{ }}karhu-sta.\}}\newline
		\NormalTok{\{Pekka\ensuremath{\text{ }}\&\ensuremath{\text{ }}became\ensuremath{\text{ }}afraid\ensuremath{\text{ }}\&\ensuremath{\text{ }}bear.ELA\}}\newline
		\NormalTok{\{`Pekka\ensuremath{\text{ }}became\ensuremath{\text{ }}afraid\ensuremath{\text{ }}because\ensuremath{\text{ }}of\ensuremath{\text{ }}the/a\ensuremath{\text{ }}bear.\textquotesingle{}\}}\newline
		\NormalTok{\}}\newline
\end{Highlighting}
\end{Shaded}

\section{IPA characters}
\label{619}
The \LaTeXTT{tipa} package is the standard \LatexSymbol{} package for International Phonetic Alphabet symbols.  

\begin{Shaded}
\begin{Highlighting}[]
\NormalTok{\textbackslash{}usepackage\{tipa\} }
\end{Highlighting}
\end{Shaded}


There are two methods for getting IPA symbols into a document. The first way is to use the \LaTeXTT{IPA} environment.

\begin{Shaded}
\begin{Highlighting}[]

\NormalTok{\textbackslash{}begin\{IPA\}}
\NormalTok{text in IPA format here}
\NormalTok{\textbackslash{}end\{IPA\}}
\end{Highlighting}
\end{Shaded}


This method is useful for long stretches of text that need to be in IPA. Alternatively, there is the \LaTeXTT{\textbackslash{}textipa} command that will format the text in its argument into IPA. This command is similar to other \myhref{https://en.wikibooks.org/wiki/LaTeX\%2FFormatting\%23Fonts}{font typesetting commands}.  

\begin{Shaded}
\begin{Highlighting}[]

\NormalTok{\textbackslash{}textipa\{text in IPA format here\}}
\end{Highlighting}
\end{Shaded}

\subsection{Basic symbols}
\label{620}
The IPA format works by translating ASCII characters into corresponding IPA symbols. Lower case letters are rendered as usual, 

\begin{longtable}{p{1.0\linewidth}}
\begin{Shaded}
\begin{Highlighting}[]

\NormalTok{\textbackslash{}textipa\{abcdefghijklmnopqrstuvwxyz\}}
\end{Highlighting}
\end{Shaded}
\\


\begin{minipage}{1.0\linewidth}
\begin{center}
\includegraphics[width=1.0\linewidth,height=6.5in,keepaspectratio]{../images/140.png}
\end{center}
\raggedright{}\myfigurewithoutcaption{140}
\end{minipage}\vspace{0.75cm}



\end{longtable}

however capital letters are rendered differently.

\begin{longtable}{p{1.0\linewidth}}
\begin{Shaded}
\begin{Highlighting}[]

\NormalTok{\textbackslash{}textipa\{ABCDEFGHIJKLMNOPQRSTUVWXYZ\}}
\end{Highlighting}
\end{Shaded}
\\


\begin{minipage}{1.0\linewidth}
\begin{center}
\includegraphics[width=1.0\linewidth,height=6.5in,keepaspectratio]{../images/141.png}
\end{center}
\raggedright{}\myfigurewithoutcaption{141}
\end{minipage}\vspace{0.75cm}



\end{longtable}

Punctuation marks that are normally used in \LatexSymbol{} are also rendered faithfully in the \LaTeXTT{IPA} environment.

\begin{longtable}{p{1.0\linewidth}}
\begin{Shaded}
\begin{Highlighting}[]

\NormalTok{\textbackslash{}textipa\{! * + = ? . , / [ ] ( ) ` '  \}}
\end{Highlighting}
\end{Shaded}
\\


\begin{minipage}{0.37500\textwidth}
\begin{center}
\includegraphics[width=1.0\textwidth,height=6.5in,keepaspectratio]{../images/142.png}
\end{center}
\raggedright{}\myfigurewithoutcaption{142}
\end{minipage}\vspace{0.75cm}



\end{longtable}

Numerals and {\ttfamily \setmainfont[Path=/usr/share/fonts/truetype/cmu/,UprightFont=cmunrm.ttf,BoldFont=cmunbx.ttf,ItalicFont=cmunti.ttf,BoldItalicFont=cmunbi.ttf]{cmuntt.ttf}\setmonofont[Path=/usr/share/fonts/truetype/cmu/,UprightFont=cmuntt.ttf,BoldFont=cmuntb.ttf,ItalicFont=cmunit.ttf,BoldItalicFont=cmuntx.ttf]{cmuntt.ttf}\ttfamily @}{$\text{ }$}\setmainfont[Path=/usr/share/fonts/truetype/cmu/,UprightFont=cmunrm.ttf,BoldFont=cmunbx.ttf,ItalicFont=cmunti.ttf,BoldItalicFont=cmunbi.ttf]{cmunrm.ttf}\setmonofont[Path=/usr/share/fonts/truetype/cmu/,UprightFont=cmuntt.ttf,BoldFont=cmuntb.ttf,ItalicFont=cmunit.ttf,BoldItalicFont=cmuntx.ttf]{cmunrm.ttf} also have variants in the \LaTeXTT{IPA} environment.

\begin{longtable}{p{1.0\linewidth}}
\begin{Shaded}
\begin{Highlighting}[]

\NormalTok{\textbackslash{}textipa\{1234567890 @\}}
\end{Highlighting}
\end{Shaded}
\\


\begin{minipage}{0.31250\textwidth}
\begin{center}
\includegraphics[width=1.0\textwidth,height=6.5in,keepaspectratio]{../images/143.png}
\end{center}
\raggedright{}\myfigurewithoutcaption{143}
\end{minipage}\vspace{0.75cm}



\end{longtable}

In addition, there are a number of special macros for representing symbols that don\textquotesingle{}t have other associations, some of which are listed here. For a complete list see the official \LaTeXTT{TIPA} Manual\myfootnote{\myfnhref{http://mirrors.ctan.org/fonts/tipa/tipa/doc/tipaman.pdf}{TIPA manual}}.

The \LaTeXTT{\textbackslash{};} macro preceding a capital letter produces a small caps version of the letter.

\begin{longtable}{p{1.0\linewidth}}
\begin{Shaded}
\begin{Highlighting}[]

\NormalTok{\textbackslash{}textipa\{\textbackslash{};A \textbackslash{};B \textbackslash{};E \textbackslash{};G \textbackslash{};H \textbackslash{};I \textbackslash{};L \textbackslash{};R \textbackslash{};Y\}}
\end{Highlighting}
\end{Shaded}
\\
 

\begin{minipage}{0.37500\textwidth}
\begin{center}
\includegraphics[width=1.0\textwidth,height=6.5in,keepaspectratio]{../images/144.png}
\end{center}
\raggedright{}\myfigurewithoutcaption{144}
\end{minipage}\vspace{0.75cm}



\end{longtable}

The \LaTeXTT{\textbackslash{}:} macro produces retroflex symbols.

\begin{longtable}{p{1.0\linewidth}}
\begin{Shaded}
\begin{Highlighting}[]

\NormalTok{\textbackslash{}textipa\{\textbackslash{}:d \textbackslash{}:l \textbackslash{}:n \textbackslash{}:r \textbackslash{}:s \textbackslash{}:t \textbackslash{}:z\}}
\end{Highlighting}
\end{Shaded}
\\
 

\begin{minipage}{0.25000\textwidth}
\begin{center}
\includegraphics[width=1.0\textwidth,height=6.5in,keepaspectratio]{../images/145.png}
\end{center}
\raggedright{}\myfigurewithoutcaption{145}
\end{minipage}\vspace{0.75cm}



\end{longtable}

The \LaTeXTT{\textbackslash{}!} macro produces implosive symbols and the bilabial click.

\begin{longtable}{p{1.0\linewidth}}
\begin{Shaded}
\begin{Highlighting}[]

\NormalTok{\textbackslash{}textipa\{\textbackslash{}!b \textbackslash{}!d \textbackslash{}!g \textbackslash{}!j \textbackslash{}!G \textbackslash{}!o\}}
\end{Highlighting}
\end{Shaded}
\\
 

\begin{minipage}{0.25000\textwidth}
\begin{center}
\includegraphics[width=1.0\textwidth,height=6.5in,keepaspectratio]{../images/146.png}
\end{center}
\raggedright{}\myfigurewithoutcaption{146}
\end{minipage}\vspace{0.75cm}



\end{longtable}
\subsection{XeLaTeX}
\label{621}
Another way of inputting IPA symbols in a document is to use \LaTeXTT{XeTeX} as the compiler and insert the symbols directly, using a character map or an IPA keyboard.\myfootnote{For more information on IPA keyboards, see SIL \myplainurl{http://scripts.sil.org/uniipakeyboard}.}

\begin{Shaded}
\begin{Highlighting}[]
\NormalTok{You can type [fəˈnɛtɪk] symbols in a straightforward manner.}
\end{Highlighting}
\end{Shaded}

\section{Phonological rules}
\label{622}
Typesetting phonological rules can be done with the help of the \LaTeXTT{phonrule} package.\myfootnote{\myplainurl{https://www.ctan.org/pkg/phonrule?lang=en} The package on CTAN.}

Following is an example of what you can achieve with the package:

\begin{longtable}{p{1.0\linewidth}}
\begin{Shaded}
\begin{Highlighting}[]

\NormalTok{\textbackslash{}phonb\{\textbackslash{}phonfeat\{+stop \textbackslash{}\textbackslash{} +consonant \textbackslash{}\textbackslash{} +alveolar\} \}\{[ɾ]\}\{\textbackslash{}phonfeat\{+vowel \textbackslash{}\textbackslash{}}
 \NormalTok{+stressed\} \}\{\textbackslash{}phonfeat\{+vowel \textbackslash{}\textbackslash{} +stressed\} \}}
\end{Highlighting}
\end{Shaded}
\\
 

\begin{minipage}{0.75000\textwidth}
\begin{center}
\includegraphics[width=1.0\textwidth,height=6.5in,keepaspectratio]{../images/147.png}
\end{center}
\raggedright{}\myfigurewithoutcaption{147}
\end{minipage}\vspace{0.75cm}



\end{longtable}
\section{References}
\label{623}

\section{External links}
\label{624}

\begin{myitemize}
\item{}  \myhref{http://www.essex.ac.uk/linguistics/external/clmt/latex4ling/}{LaTeX for Linguists} 
\item{}  \myhref{http://www.ling.upenn.edu/advice/latex/qtree/}{The qtree package for drawing syntactic trees.}
\item{}  \myhref{http://www.ctan.org/tex-archive/macros/latex/contrib/gb4e/}{The gb4e package page on CTAN. }
\end{myitemize}



\mypart{Special Pages}\chapter{Indexing}

\myminitoc
\label{625}

\label{626}

Especially useful in printed books, an index is an alphabetical list of words and expressions with the pages of the book upon which they are to be found. LaTeX supports the creation of indices with its package \LaTeXTT{makeidx}, and its support program {\ttfamily \setmainfont[Path=/usr/share/fonts/truetype/cmu/,UprightFont=cmunrm.ttf,BoldFont=cmunbx.ttf,ItalicFont=cmunti.ttf,BoldItalicFont=cmunbi.ttf]{cmuntt.ttf}\setmonofont[Path=/usr/share/fonts/truetype/cmu/,UprightFont=cmuntt.ttf,BoldFont=cmuntb.ttf,ItalicFont=cmunit.ttf,BoldItalicFont=cmuntx.ttf]{cmuntt.ttf}\ttfamily makeindex}\setmainfont[Path=/usr/share/fonts/truetype/cmu/,UprightFont=cmunrm.ttf,BoldFont=cmunbx.ttf,ItalicFont=cmunti.ttf,BoldItalicFont=cmunbi.ttf]{cmunrm.ttf}\setmonofont[Path=/usr/share/fonts/truetype/cmu/,UprightFont=cmuntt.ttf,BoldFont=cmuntb.ttf,ItalicFont=cmunit.ttf,BoldItalicFont=cmuntx.ttf]{cmunrm.ttf}, called on some systems {\ttfamily \setmainfont[Path=/usr/share/fonts/truetype/cmu/,UprightFont=cmunrm.ttf,BoldFont=cmunbx.ttf,ItalicFont=cmunti.ttf,BoldItalicFont=cmunbi.ttf]{cmuntt.ttf}\setmonofont[Path=/usr/share/fonts/truetype/cmu/,UprightFont=cmuntt.ttf,BoldFont=cmuntb.ttf,ItalicFont=cmunit.ttf,BoldItalicFont=cmuntx.ttf]{cmuntt.ttf}\ttfamily makeidx}\setmainfont[Path=/usr/share/fonts/truetype/cmu/,UprightFont=cmunrm.ttf,BoldFont=cmunbx.ttf,ItalicFont=cmunti.ttf,BoldItalicFont=cmunbi.ttf]{cmunrm.ttf}\setmonofont[Path=/usr/share/fonts/truetype/cmu/,UprightFont=cmuntt.ttf,BoldFont=cmuntb.ttf,ItalicFont=cmunit.ttf,BoldItalicFont=cmuntx.ttf]{cmunrm.ttf}.
\section{Using {\ttfamily \setmainfont[Path=/usr/share/fonts/truetype/cmu/,UprightFont=cmunrm.ttf,BoldFont=cmunbx.ttf,ItalicFont=cmunti.ttf,BoldItalicFont=cmunbi.ttf]{cmuntt.ttf}\setmonofont[Path=/usr/share/fonts/truetype/cmu/,UprightFont=cmuntt.ttf,BoldFont=cmuntb.ttf,ItalicFont=cmunit.ttf,BoldItalicFont=cmuntx.ttf]{cmuntt.ttf}\ttfamily makeidx}{$\text{ }$}\setmainfont[Path=/usr/share/fonts/truetype/cmu/,UprightFont=cmunrm.ttf,BoldFont=cmunbx.ttf,ItalicFont=cmunti.ttf,BoldItalicFont=cmunbi.ttf]{cmunrm.ttf}\setmonofont[Path=/usr/share/fonts/truetype/cmu/,UprightFont=cmuntt.ttf,BoldFont=cmuntb.ttf,ItalicFont=cmunit.ttf,BoldItalicFont=cmuntx.ttf]{cmunrm.ttf}}
\label{627}
To enable the indexing feature of LaTeX, the \LaTeXTT{makeidx} package must be loaded in the \mylref{91}{preamble} with:
\begin{Shaded}
\begin{Highlighting}[]

\NormalTok{\textbackslash{}usepackage\{makeidx\}}
\end{Highlighting}
\end{Shaded}


and the special indexing commands must be enabled by putting the
\begin{Shaded}
\begin{Highlighting}[]

\NormalTok{\textbackslash{}makeindex}
\end{Highlighting}
\end{Shaded}


command into the input file preamble. This should be done within the preamble, since it tells LaTeX to create the files needed for indexing. To tell LaTeX what to index, use
\begin{Shaded}
\begin{Highlighting}[]

\NormalTok{\textbackslash{}index\{key\}}
\end{Highlighting}
\end{Shaded}


where {\itshape \setmainfont[Path=/usr/share/fonts/truetype/cmu/,UprightFont=cmunrm.ttf,BoldFont=cmunbx.ttf,ItalicFont=cmunti.ttf,BoldItalicFont=cmunbi.ttf]{cmunti.ttf}\setmonofont[Path=/usr/share/fonts/truetype/cmu/,UprightFont=cmuntt.ttf,BoldFont=cmuntb.ttf,ItalicFont=cmunit.ttf,BoldItalicFont=cmuntx.ttf]{cmunti.ttf}\itshape key}{$\text{ }$}\setmainfont[Path=/usr/share/fonts/truetype/cmu/,UprightFont=cmunrm.ttf,BoldFont=cmunbx.ttf,ItalicFont=cmunti.ttf,BoldItalicFont=cmunbi.ttf]{cmunrm.ttf}\setmonofont[Path=/usr/share/fonts/truetype/cmu/,UprightFont=cmuntt.ttf,BoldFont=cmuntb.ttf,ItalicFont=cmunit.ttf,BoldItalicFont=cmuntx.ttf]{cmunrm.ttf} is the index entry and does not appear in the final layout. You enter the index commands at the points in the text that you want to be referenced in the index, likely near the reason for the {\itshape \setmainfont[Path=/usr/share/fonts/truetype/cmu/,UprightFont=cmunrm.ttf,BoldFont=cmunbx.ttf,ItalicFont=cmunti.ttf,BoldItalicFont=cmunbi.ttf]{cmunti.ttf}\setmonofont[Path=/usr/share/fonts/truetype/cmu/,UprightFont=cmuntt.ttf,BoldFont=cmuntb.ttf,ItalicFont=cmunit.ttf,BoldItalicFont=cmuntx.ttf]{cmunti.ttf}\itshape key}\setmainfont[Path=/usr/share/fonts/truetype/cmu/,UprightFont=cmunrm.ttf,BoldFont=cmunbx.ttf,ItalicFont=cmunti.ttf,BoldItalicFont=cmunbi.ttf]{cmunrm.ttf}\setmonofont[Path=/usr/share/fonts/truetype/cmu/,UprightFont=cmuntt.ttf,BoldFont=cmuntb.ttf,ItalicFont=cmunit.ttf,BoldItalicFont=cmuntx.ttf]{cmunrm.ttf}. For example, the text
\begin{Shaded}
\begin{Highlighting}[]

\NormalTok{To solve various problems in physics, it can be advantageous}
\NormalTok{to express any arbitrary piecewise-smooth function as a}
\NormalTok{Fourier Series composed of multiples of sine and cosine functions.}
\end{Highlighting}
\end{Shaded}

can be re-{}written as
\begin{Shaded}
\begin{Highlighting}[]

\NormalTok{To solve various problems in physics, it can be advantageous}
\NormalTok{to express any arbitrary piecewise-smooth function as a Fourier Series}
\NormalTok{\textbackslash{}index\{Fourier Series\}}
\NormalTok{composed of multiples of sine and cosine functions.}
\end{Highlighting}
\end{Shaded}

to create an entry called \textquotesingle{}Fourier Series\textquotesingle{} with a reference to the target page. Multiple uses of {\itshape \setmainfont[Path=/usr/share/fonts/truetype/cmu/,UprightFont=cmunrm.ttf,BoldFont=cmunbx.ttf,ItalicFont=cmunti.ttf,BoldItalicFont=cmunbi.ttf]{cmunti.ttf}\setmonofont[Path=/usr/share/fonts/truetype/cmu/,UprightFont=cmuntt.ttf,BoldFont=cmuntb.ttf,ItalicFont=cmunit.ttf,BoldItalicFont=cmuntx.ttf]{cmunti.ttf}\itshape \textbackslash{}index}{$\text{ }$}\setmainfont[Path=/usr/share/fonts/truetype/cmu/,UprightFont=cmunrm.ttf,BoldFont=cmunbx.ttf,ItalicFont=cmunti.ttf,BoldItalicFont=cmunbi.ttf]{cmunrm.ttf}\setmonofont[Path=/usr/share/fonts/truetype/cmu/,UprightFont=cmuntt.ttf,BoldFont=cmuntb.ttf,ItalicFont=cmunit.ttf,BoldItalicFont=cmuntx.ttf]{cmunrm.ttf} with the same {\itshape \setmainfont[Path=/usr/share/fonts/truetype/cmu/,UprightFont=cmunrm.ttf,BoldFont=cmunbx.ttf,ItalicFont=cmunti.ttf,BoldItalicFont=cmunbi.ttf]{cmunti.ttf}\setmonofont[Path=/usr/share/fonts/truetype/cmu/,UprightFont=cmuntt.ttf,BoldFont=cmuntb.ttf,ItalicFont=cmunit.ttf,BoldItalicFont=cmuntx.ttf]{cmunti.ttf}\itshape key}{$\text{ }$}\setmainfont[Path=/usr/share/fonts/truetype/cmu/,UprightFont=cmunrm.ttf,BoldFont=cmunbx.ttf,ItalicFont=cmunti.ttf,BoldItalicFont=cmunbi.ttf]{cmunrm.ttf}\setmonofont[Path=/usr/share/fonts/truetype/cmu/,UprightFont=cmuntt.ttf,BoldFont=cmuntb.ttf,ItalicFont=cmunit.ttf,BoldItalicFont=cmuntx.ttf]{cmunrm.ttf} on different pages will add those target pages to the same index entry.


To show the index within the document, merely use the command
\begin{Shaded}
\begin{Highlighting}[]

\NormalTok{\textbackslash{}printindex}
\end{Highlighting}
\end{Shaded}

It is common to place it at the end of the document. The default index format is two columns.


The \LaTeXTT{showidx} package that comes with LaTeX prints out all index entries in the right margin of the text. This is quite useful for proofreading a document and verifying the index.
\subsection{Compiling indices}
\label{628}
When the input file is processed with LaTeX, each \LaTeXTT{\textbackslash{}index} command writes an appropriate index entry, together with the current page number, to a special file. The file has the same name as the LaTeX input file, but a different extension ({\ttfamily \setmainfont[Path=/usr/share/fonts/truetype/cmu/,UprightFont=cmunrm.ttf,BoldFont=cmunbx.ttf,ItalicFont=cmunti.ttf,BoldItalicFont=cmunbi.ttf]{cmuntt.ttf}\setmonofont[Path=/usr/share/fonts/truetype/cmu/,UprightFont=cmuntt.ttf,BoldFont=cmuntb.ttf,ItalicFont=cmunit.ttf,BoldItalicFont=cmuntx.ttf]{cmuntt.ttf}\ttfamily .idx}\setmainfont[Path=/usr/share/fonts/truetype/cmu/,UprightFont=cmunrm.ttf,BoldFont=cmunbx.ttf,ItalicFont=cmunti.ttf,BoldItalicFont=cmunbi.ttf]{cmunrm.ttf}\setmonofont[Path=/usr/share/fonts/truetype/cmu/,UprightFont=cmuntt.ttf,BoldFont=cmuntb.ttf,ItalicFont=cmunit.ttf,BoldItalicFont=cmuntx.ttf]{cmunrm.ttf}). This {\ttfamily \setmainfont[Path=/usr/share/fonts/truetype/cmu/,UprightFont=cmunrm.ttf,BoldFont=cmunbx.ttf,ItalicFont=cmunti.ttf,BoldItalicFont=cmunbi.ttf]{cmuntt.ttf}\setmonofont[Path=/usr/share/fonts/truetype/cmu/,UprightFont=cmuntt.ttf,BoldFont=cmuntb.ttf,ItalicFont=cmunit.ttf,BoldItalicFont=cmuntx.ttf]{cmuntt.ttf}\ttfamily .idx}{$\text{ }$}\setmainfont[Path=/usr/share/fonts/truetype/cmu/,UprightFont=cmunrm.ttf,BoldFont=cmunbx.ttf,ItalicFont=cmunti.ttf,BoldItalicFont=cmunbi.ttf]{cmunrm.ttf}\setmonofont[Path=/usr/share/fonts/truetype/cmu/,UprightFont=cmuntt.ttf,BoldFont=cmuntb.ttf,ItalicFont=cmunit.ttf,BoldItalicFont=cmuntx.ttf]{cmunrm.ttf} file can then be processed with the {\ttfamily \setmainfont[Path=/usr/share/fonts/truetype/cmu/,UprightFont=cmunrm.ttf,BoldFont=cmunbx.ttf,ItalicFont=cmunti.ttf,BoldItalicFont=cmunbi.ttf]{cmuntt.ttf}\setmonofont[Path=/usr/share/fonts/truetype/cmu/,UprightFont=cmuntt.ttf,BoldFont=cmuntb.ttf,ItalicFont=cmunit.ttf,BoldItalicFont=cmuntx.ttf]{cmuntt.ttf}\ttfamily makeindex}{$\text{ }$}\setmainfont[Path=/usr/share/fonts/truetype/cmu/,UprightFont=cmunrm.ttf,BoldFont=cmunbx.ttf,ItalicFont=cmunti.ttf,BoldItalicFont=cmunbi.ttf]{cmunrm.ttf}\setmonofont[Path=/usr/share/fonts/truetype/cmu/,UprightFont=cmuntt.ttf,BoldFont=cmuntb.ttf,ItalicFont=cmunit.ttf,BoldItalicFont=cmuntx.ttf]{cmunrm.ttf} program. Type in the command line:
\\

\TemplateSpaceIndent{$\text{ }${}{\ttfamily \setmainfont[Path=/usr/share/fonts/truetype/cmu/,UprightFont=cmunrm.ttf,BoldFont=cmunbx.ttf,ItalicFont=cmunti.ttf,BoldItalicFont=cmunbi.ttf]{cmuntt.ttf}\setmonofont[Path=/usr/share/fonts/truetype/cmu/,UprightFont=cmuntt.ttf,BoldFont=cmuntb.ttf,ItalicFont=cmunit.ttf,BoldItalicFont=cmuntx.ttf]{cmuntt.ttf}\ttfamily makeindex}$\text{ }${}{\itshape \setmainfont[Path=/usr/share/fonts/truetype/cmu/,UprightFont=cmunrm.ttf,BoldFont=cmunbx.ttf,ItalicFont=cmunti.ttf,BoldItalicFont=cmunbi.ttf]{cmunti.ttf}\setmonofont[Path=/usr/share/fonts/truetype/cmu/,UprightFont=cmuntt.ttf,BoldFont=cmuntb.ttf,ItalicFont=cmunit.ttf,BoldItalicFont=cmuntx.ttf]{cmunti.ttf}\itshape filename}}
\setmainfont[Path=/usr/share/fonts/truetype/cmu/,UprightFont=cmunrm.ttf,BoldFont=cmunbx.ttf,ItalicFont=cmunti.ttf,BoldItalicFont=cmunbi.ttf]{cmunrm.ttf}\setmonofont[Path=/usr/share/fonts/truetype/cmu/,UprightFont=cmuntt.ttf,BoldFont=cmuntb.ttf,ItalicFont=cmunit.ttf,BoldItalicFont=cmuntx.ttf]{cmunrm.ttf}

Note that {\itshape \setmainfont[Path=/usr/share/fonts/truetype/cmu/,UprightFont=cmunrm.ttf,BoldFont=cmunbx.ttf,ItalicFont=cmunti.ttf,BoldItalicFont=cmunbi.ttf]{cmunti.ttf}\setmonofont[Path=/usr/share/fonts/truetype/cmu/,UprightFont=cmuntt.ttf,BoldFont=cmuntb.ttf,ItalicFont=cmunit.ttf,BoldItalicFont=cmuntx.ttf]{cmunti.ttf}\itshape filename}{$\text{ }$}\setmainfont[Path=/usr/share/fonts/truetype/cmu/,UprightFont=cmunrm.ttf,BoldFont=cmunbx.ttf,ItalicFont=cmunti.ttf,BoldItalicFont=cmunbi.ttf]{cmunrm.ttf}\setmonofont[Path=/usr/share/fonts/truetype/cmu/,UprightFont=cmuntt.ttf,BoldFont=cmuntb.ttf,ItalicFont=cmunit.ttf,BoldItalicFont=cmuntx.ttf]{cmunrm.ttf} is without extension: the program will look for {\itshape \setmainfont[Path=/usr/share/fonts/truetype/cmu/,UprightFont=cmunrm.ttf,BoldFont=cmunbx.ttf,ItalicFont=cmunti.ttf,BoldItalicFont=cmunbi.ttf]{cmunti.ttf}\setmonofont[Path=/usr/share/fonts/truetype/cmu/,UprightFont=cmuntt.ttf,BoldFont=cmuntb.ttf,ItalicFont=cmunit.ttf,BoldItalicFont=cmuntx.ttf]{cmunti.ttf}\itshape filename.idx}{$\text{ }$}\setmainfont[Path=/usr/share/fonts/truetype/cmu/,UprightFont=cmunrm.ttf,BoldFont=cmunbx.ttf,ItalicFont=cmunti.ttf,BoldItalicFont=cmunbi.ttf]{cmunrm.ttf}\setmonofont[Path=/usr/share/fonts/truetype/cmu/,UprightFont=cmuntt.ttf,BoldFont=cmuntb.ttf,ItalicFont=cmunit.ttf,BoldItalicFont=cmuntx.ttf]{cmunrm.ttf} and use that. You can optionally pass {\itshape \setmainfont[Path=/usr/share/fonts/truetype/cmu/,UprightFont=cmunrm.ttf,BoldFont=cmunbx.ttf,ItalicFont=cmunti.ttf,BoldItalicFont=cmunbi.ttf]{cmunti.ttf}\setmonofont[Path=/usr/share/fonts/truetype/cmu/,UprightFont=cmuntt.ttf,BoldFont=cmuntb.ttf,ItalicFont=cmunit.ttf,BoldItalicFont=cmuntx.ttf]{cmunti.ttf}\itshape filename.idx}{$\text{ }$}\setmainfont[Path=/usr/share/fonts/truetype/cmu/,UprightFont=cmunrm.ttf,BoldFont=cmunbx.ttf,ItalicFont=cmunti.ttf,BoldItalicFont=cmunbi.ttf]{cmunrm.ttf}\setmonofont[Path=/usr/share/fonts/truetype/cmu/,UprightFont=cmuntt.ttf,BoldFont=cmuntb.ttf,ItalicFont=cmunit.ttf,BoldItalicFont=cmuntx.ttf]{cmunrm.ttf} directly to the program as an argument. The {\ttfamily \setmainfont[Path=/usr/share/fonts/truetype/cmu/,UprightFont=cmunrm.ttf,BoldFont=cmunbx.ttf,ItalicFont=cmunti.ttf,BoldItalicFont=cmunbi.ttf]{cmuntt.ttf}\setmonofont[Path=/usr/share/fonts/truetype/cmu/,UprightFont=cmuntt.ttf,BoldFont=cmuntb.ttf,ItalicFont=cmunit.ttf,BoldItalicFont=cmuntx.ttf]{cmuntt.ttf}\ttfamily makeindex}{$\text{ }$}\setmainfont[Path=/usr/share/fonts/truetype/cmu/,UprightFont=cmunrm.ttf,BoldFont=cmunbx.ttf,ItalicFont=cmunti.ttf,BoldItalicFont=cmunbi.ttf]{cmunrm.ttf}\setmonofont[Path=/usr/share/fonts/truetype/cmu/,UprightFont=cmuntt.ttf,BoldFont=cmuntb.ttf,ItalicFont=cmunit.ttf,BoldItalicFont=cmuntx.ttf]{cmunrm.ttf} program generates a sorted index with the same base file name, but this time with the extension {\ttfamily \setmainfont[Path=/usr/share/fonts/truetype/cmu/,UprightFont=cmunrm.ttf,BoldFont=cmunbx.ttf,ItalicFont=cmunti.ttf,BoldItalicFont=cmunbi.ttf]{cmuntt.ttf}\setmonofont[Path=/usr/share/fonts/truetype/cmu/,UprightFont=cmuntt.ttf,BoldFont=cmuntb.ttf,ItalicFont=cmunit.ttf,BoldItalicFont=cmuntx.ttf]{cmuntt.ttf}\ttfamily .ind}\setmainfont[Path=/usr/share/fonts/truetype/cmu/,UprightFont=cmunrm.ttf,BoldFont=cmunbx.ttf,ItalicFont=cmunti.ttf,BoldItalicFont=cmunbi.ttf]{cmunrm.ttf}\setmonofont[Path=/usr/share/fonts/truetype/cmu/,UprightFont=cmuntt.ttf,BoldFont=cmuntb.ttf,ItalicFont=cmunit.ttf,BoldItalicFont=cmuntx.ttf]{cmunrm.ttf}. If now the LaTeX input file is processed again, this sorted index gets included into the document at the point where LaTeX finds \LaTeXTT{\textbackslash{}printindex}.

The index created by latex with the default options may not look as nice or as suitable as you would like it. To improve the looks of the index {\ttfamily \setmainfont[Path=/usr/share/fonts/truetype/cmu/,UprightFont=cmunrm.ttf,BoldFont=cmunbx.ttf,ItalicFont=cmunti.ttf,BoldItalicFont=cmunbi.ttf]{cmuntt.ttf}\setmonofont[Path=/usr/share/fonts/truetype/cmu/,UprightFont=cmuntt.ttf,BoldFont=cmuntb.ttf,ItalicFont=cmunit.ttf,BoldItalicFont=cmuntx.ttf]{cmuntt.ttf}\ttfamily makeindex}{$\text{ }$}\setmainfont[Path=/usr/share/fonts/truetype/cmu/,UprightFont=cmunrm.ttf,BoldFont=cmunbx.ttf,ItalicFont=cmunti.ttf,BoldItalicFont=cmunbi.ttf]{cmunrm.ttf}\setmonofont[Path=/usr/share/fonts/truetype/cmu/,UprightFont=cmuntt.ttf,BoldFont=cmuntb.ttf,ItalicFont=cmunit.ttf,BoldItalicFont=cmuntx.ttf]{cmunrm.ttf} comes with a set of style files, usually located somewhere in the tex directory structure, usually below the {\ttfamily \setmainfont[Path=/usr/share/fonts/truetype/cmu/,UprightFont=cmunrm.ttf,BoldFont=cmunbx.ttf,ItalicFont=cmunti.ttf,BoldItalicFont=cmunbi.ttf]{cmuntt.ttf}\setmonofont[Path=/usr/share/fonts/truetype/cmu/,UprightFont=cmuntt.ttf,BoldFont=cmuntb.ttf,ItalicFont=cmunit.ttf,BoldItalicFont=cmuntx.ttf]{cmuntt.ttf}\ttfamily makeindex}{$\text{ }$}\setmainfont[Path=/usr/share/fonts/truetype/cmu/,UprightFont=cmunrm.ttf,BoldFont=cmunbx.ttf,ItalicFont=cmunti.ttf,BoldItalicFont=cmunbi.ttf]{cmunrm.ttf}\setmonofont[Path=/usr/share/fonts/truetype/cmu/,UprightFont=cmuntt.ttf,BoldFont=cmuntb.ttf,ItalicFont=cmunit.ttf,BoldItalicFont=cmuntx.ttf]{cmunrm.ttf} subdirectory. To tell {\ttfamily \setmainfont[Path=/usr/share/fonts/truetype/cmu/,UprightFont=cmunrm.ttf,BoldFont=cmunbx.ttf,ItalicFont=cmunti.ttf,BoldItalicFont=cmunbi.ttf]{cmuntt.ttf}\setmonofont[Path=/usr/share/fonts/truetype/cmu/,UprightFont=cmuntt.ttf,BoldFont=cmuntb.ttf,ItalicFont=cmunit.ttf,BoldItalicFont=cmuntx.ttf]{cmuntt.ttf}\ttfamily makeindex}{$\text{ }$}\setmainfont[Path=/usr/share/fonts/truetype/cmu/,UprightFont=cmunrm.ttf,BoldFont=cmunbx.ttf,ItalicFont=cmunti.ttf,BoldItalicFont=cmunbi.ttf]{cmunrm.ttf}\setmonofont[Path=/usr/share/fonts/truetype/cmu/,UprightFont=cmuntt.ttf,BoldFont=cmuntb.ttf,ItalicFont=cmunit.ttf,BoldItalicFont=cmuntx.ttf]{cmunrm.ttf} to use a specific style file, run it with the command line option:
\\

\TemplateSpaceIndent{$\text{ }${}$\text{ }${}{\ttfamily \setmainfont[Path=/usr/share/fonts/truetype/cmu/,UprightFont=cmunrm.ttf,BoldFont=cmunbx.ttf,ItalicFont=cmunti.ttf,BoldItalicFont=cmunbi.ttf]{cmuntt.ttf}\setmonofont[Path=/usr/share/fonts/truetype/cmu/,UprightFont=cmuntt.ttf,BoldFont=cmuntb.ttf,ItalicFont=cmunit.ttf,BoldItalicFont=cmuntx.ttf]{cmuntt.ttf}\ttfamily makeindex}$\text{ }${}\setmainfont[Path=/usr/share/fonts/truetype/cmu/,UprightFont=cmunrm.ttf,BoldFont=cmunbx.ttf,ItalicFont=cmunti.ttf,BoldItalicFont=cmunbi.ttf]{cmunrm.ttf}\setmonofont[Path=/usr/share/fonts/truetype/cmu/,UprightFont=cmuntt.ttf,BoldFont=cmuntb.ttf,ItalicFont=cmunit.ttf,BoldItalicFont=cmuntx.ttf]{cmunrm.ttf}-{}s$\text{ }${}{$\text{[}$}style$\text{ }${}file{$\text{]}$}$\text{ }${}{\itshape \setmainfont[Path=/usr/share/fonts/truetype/cmu/,UprightFont=cmunrm.ttf,BoldFont=cmunbx.ttf,ItalicFont=cmunti.ttf,BoldItalicFont=cmunbi.ttf]{cmunti.ttf}\setmonofont[Path=/usr/share/fonts/truetype/cmu/,UprightFont=cmuntt.ttf,BoldFont=cmuntb.ttf,ItalicFont=cmunit.ttf,BoldItalicFont=cmuntx.ttf]{cmunti.ttf}\itshape filename}}
\setmainfont[Path=/usr/share/fonts/truetype/cmu/,UprightFont=cmunrm.ttf,BoldFont=cmunbx.ttf,ItalicFont=cmunti.ttf,BoldItalicFont=cmunbi.ttf]{cmunrm.ttf}\setmonofont[Path=/usr/share/fonts/truetype/cmu/,UprightFont=cmuntt.ttf,BoldFont=cmuntb.ttf,ItalicFont=cmunit.ttf,BoldItalicFont=cmuntx.ttf]{cmunrm.ttf}

If you use a GUI for compiling latex and index files, you may have to set this in the options. Here are some configuration tips for typical tools:\subsubsection{MakeIndex settings in WinEdt}
\label{629}
Say you want to add an index style file named {\ttfamily \setmainfont[Path=/usr/share/fonts/truetype/cmu/,UprightFont=cmunrm.ttf,BoldFont=cmunbx.ttf,ItalicFont=cmunti.ttf,BoldItalicFont=cmunbi.ttf]{cmuntt.ttf}\setmonofont[Path=/usr/share/fonts/truetype/cmu/,UprightFont=cmuntt.ttf,BoldFont=cmuntb.ttf,ItalicFont=cmunit.ttf,BoldItalicFont=cmuntx.ttf]{cmuntt.ttf}\ttfamily simpleidx.ist}
\begin{myitemize}
\item{} {$\text{ }$}\setmainfont[Path=/usr/share/fonts/truetype/cmu/,UprightFont=cmunrm.ttf,BoldFont=cmunbx.ttf,ItalicFont=cmunti.ttf,BoldItalicFont=cmunbi.ttf]{cmunrm.ttf}\setmonofont[Path=/usr/share/fonts/truetype/cmu/,UprightFont=cmuntt.ttf,BoldFont=cmuntb.ttf,ItalicFont=cmunit.ttf,BoldItalicFont=cmuntx.ttf]{cmunrm.ttf} Texify/PDFTexify: Options→Execution Modes→Accessories→PDFTeXify, add to the Switches: {\ttfamily \setmainfont[Path=/usr/share/fonts/truetype/cmu/,UprightFont=cmunrm.ttf,BoldFont=cmunbx.ttf,ItalicFont=cmunti.ttf,BoldItalicFont=cmunbi.ttf]{cmuntt.ttf}\setmonofont[Path=/usr/share/fonts/truetype/cmu/,UprightFont=cmuntt.ttf,BoldFont=cmuntb.ttf,ItalicFont=cmunit.ttf,BoldItalicFont=cmuntx.ttf]{cmuntt.ttf}\ttfamily -{}-{}mkidx-{}option=\symbol{34}-{}s simpleidx.ist\symbol{34}}
\item{} {$\text{ }$}\setmainfont[Path=/usr/share/fonts/truetype/cmu/,UprightFont=cmunrm.ttf,BoldFont=cmunbx.ttf,ItalicFont=cmunti.ttf,BoldItalicFont=cmunbi.ttf]{cmunrm.ttf}\setmonofont[Path=/usr/share/fonts/truetype/cmu/,UprightFont=cmuntt.ttf,BoldFont=cmuntb.ttf,ItalicFont=cmunit.ttf,BoldItalicFont=cmuntx.ttf]{cmunrm.ttf} MakeIndex alone: Options→Execution Modes→Accessories→MakeIndex, add to command line: {\ttfamily \setmainfont[Path=/usr/share/fonts/truetype/cmu/,UprightFont=cmunrm.ttf,BoldFont=cmunbx.ttf,ItalicFont=cmunti.ttf,BoldItalicFont=cmunbi.ttf]{cmuntt.ttf}\setmonofont[Path=/usr/share/fonts/truetype/cmu/,UprightFont=cmuntt.ttf,BoldFont=cmuntb.ttf,ItalicFont=cmunit.ttf,BoldItalicFont=cmuntx.ttf]{cmuntt.ttf}\ttfamily -{}s simpleidx.ist}
\end{myitemize}
\setmainfont[Path=/usr/share/fonts/truetype/cmu/,UprightFont=cmunrm.ttf,BoldFont=cmunbx.ttf,ItalicFont=cmunti.ttf,BoldItalicFont=cmunbi.ttf]{cmunrm.ttf}\setmonofont[Path=/usr/share/fonts/truetype/cmu/,UprightFont=cmuntt.ttf,BoldFont=cmuntb.ttf,ItalicFont=cmunit.ttf,BoldItalicFont=cmuntx.ttf]{cmunrm.ttf}
\subsection{Sophisticated indexing}
\label{630}
Below are examples of \LaTeXTT{\textbackslash{}index} entries:

\begin{longtable}{|>{\RaggedRight}p{0.38196\linewidth}|>{\RaggedRight}p{0.21807\linewidth}|>{\RaggedRight}p{0.31426\linewidth}|} \hline 
{\bfseries \hspace*{0pt}\ignorespaces{}\hspace*{0pt}Example}&{\bfseries \hspace*{0pt}\ignorespaces{}\hspace*{0pt}Index Entry}&{\bfseries \hspace*{0pt}\ignorespaces{}\hspace*{0pt}Comment}\endhead  \hline \hspace*{0pt}\ignorespaces{}\hspace*{0pt}\LaTeXTT{\textbackslash{}index\{hello\}}&\hspace*{0pt}\ignorespaces{}\hspace*{0pt}hello, 1&\hspace*{0pt}\ignorespaces{}\hspace*{0pt}Plain entry\\ \hline \hspace*{0pt}\ignorespaces{}\hspace*{0pt}\LaTeXTT{\textbackslash{}index\{hello!Peter\}}&\hspace*{0pt}\ignorespaces{}\hspace*{0pt}{\mbox{$~$}}{\mbox{$~$}}Peter, 3&\hspace*{0pt}\ignorespaces{}\hspace*{0pt}Subentry under \textquotesingle{}hello\textquotesingle{}\\ \hline \hspace*{0pt}\ignorespaces{}\hspace*{0pt}\LaTeXTT{\textbackslash{}index\{hello!Sam@\textbackslash{}textsl\{Sam\}\}}&\hspace*{0pt}\ignorespaces{}\hspace*{0pt}{\mbox{$~$}}{\mbox{$~$}}{\itshape \setmainfont[Path=/usr/share/fonts/truetype/cmu/,UprightFont=cmunrm.ttf,BoldFont=cmunbx.ttf,ItalicFont=cmunti.ttf,BoldItalicFont=cmunbi.ttf]{cmunti.ttf}\setmonofont[Path=/usr/share/fonts/truetype/cmu/,UprightFont=cmuntt.ttf,BoldFont=cmuntb.ttf,ItalicFont=cmunit.ttf,BoldItalicFont=cmuntx.ttf]{cmunti.ttf}\itshape Sam}\setmainfont[Path=/usr/share/fonts/truetype/cmu/,UprightFont=cmunrm.ttf,BoldFont=cmunbx.ttf,ItalicFont=cmunti.ttf,BoldItalicFont=cmunbi.ttf]{cmunrm.ttf}\setmonofont[Path=/usr/share/fonts/truetype/cmu/,UprightFont=cmuntt.ttf,BoldFont=cmuntb.ttf,ItalicFont=cmunit.ttf,BoldItalicFont=cmuntx.ttf]{cmunrm.ttf}, 2&\hspace*{0pt}\ignorespaces{}\hspace*{0pt}Subentry formatted and sorted\\ \hline \hspace*{0pt}\ignorespaces{}\hspace*{0pt}\LaTeXTT{\textbackslash{}index\{Sam@\textbackslash{}textsl\{Sam\}\}}&\hspace*{0pt}\ignorespaces{}\hspace*{0pt}{\itshape \setmainfont[Path=/usr/share/fonts/truetype/cmu/,UprightFont=cmunrm.ttf,BoldFont=cmunbx.ttf,ItalicFont=cmunti.ttf,BoldItalicFont=cmunbi.ttf]{cmunti.ttf}\setmonofont[Path=/usr/share/fonts/truetype/cmu/,UprightFont=cmuntt.ttf,BoldFont=cmuntb.ttf,ItalicFont=cmunit.ttf,BoldItalicFont=cmuntx.ttf]{cmunti.ttf}\itshape Sam}\setmainfont[Path=/usr/share/fonts/truetype/cmu/,UprightFont=cmunrm.ttf,BoldFont=cmunbx.ttf,ItalicFont=cmunti.ttf,BoldItalicFont=cmunbi.ttf]{cmunrm.ttf}\setmonofont[Path=/usr/share/fonts/truetype/cmu/,UprightFont=cmuntt.ttf,BoldFont=cmuntb.ttf,ItalicFont=cmunit.ttf,BoldItalicFont=cmuntx.ttf]{cmunrm.ttf}, 2&\hspace*{0pt}\ignorespaces{}\hspace*{0pt}Formatted entry\\ \hline \hspace*{0pt}\ignorespaces{}\hspace*{0pt}\LaTeXTT{\textbackslash{}index\{Lin@\textbackslash{}textbf\{Lin\}\}}&\hspace*{0pt}\ignorespaces{}\hspace*{0pt}{\bfseries \setmainfont[Path=/usr/share/fonts/truetype/cmu/,UprightFont=cmunrm.ttf,BoldFont=cmunbx.ttf,ItalicFont=cmunti.ttf,BoldItalicFont=cmunbi.ttf]{cmunbx.ttf}\setmonofont[Path=/usr/share/fonts/truetype/cmu/,UprightFont=cmuntt.ttf,BoldFont=cmuntb.ttf,ItalicFont=cmunit.ttf,BoldItalicFont=cmuntx.ttf]{cmunbx.ttf}\bfseries Lin}\setmainfont[Path=/usr/share/fonts/truetype/cmu/,UprightFont=cmunrm.ttf,BoldFont=cmunbx.ttf,ItalicFont=cmunti.ttf,BoldItalicFont=cmunbi.ttf]{cmunrm.ttf}\setmonofont[Path=/usr/share/fonts/truetype/cmu/,UprightFont=cmuntt.ttf,BoldFont=cmuntb.ttf,ItalicFont=cmunit.ttf,BoldItalicFont=cmuntx.ttf]{cmunrm.ttf}, 7&\hspace*{0pt}\ignorespaces{}\hspace*{0pt}Same as above\\ \hline \hspace*{0pt}\ignorespaces{}\hspace*{0pt}\LaTeXTT{\textbackslash{}index\{Jenny|textbf\}}&\hspace*{0pt}\ignorespaces{}\hspace*{0pt}Jenny, {\bfseries \setmainfont[Path=/usr/share/fonts/truetype/cmu/,UprightFont=cmunrm.ttf,BoldFont=cmunbx.ttf,ItalicFont=cmunti.ttf,BoldItalicFont=cmunbi.ttf]{cmunbx.ttf}\setmonofont[Path=/usr/share/fonts/truetype/cmu/,UprightFont=cmuntt.ttf,BoldFont=cmuntb.ttf,ItalicFont=cmunit.ttf,BoldItalicFont=cmuntx.ttf]{cmunbx.ttf}\bfseries 3}&\hspace*{0pt}\ignorespaces{}\hspace*{0pt}\setmainfont[Path=/usr/share/fonts/truetype/cmu/,UprightFont=cmunrm.ttf,BoldFont=cmunbx.ttf,ItalicFont=cmunti.ttf,BoldItalicFont=cmunbi.ttf]{cmunrm.ttf}\setmonofont[Path=/usr/share/fonts/truetype/cmu/,UprightFont=cmuntt.ttf,BoldFont=cmuntb.ttf,ItalicFont=cmunit.ttf,BoldItalicFont=cmuntx.ttf]{cmunrm.ttf}Formatted page number\\ \hline \hspace*{0pt}\ignorespaces{}\hspace*{0pt}\LaTeXTT{\textbackslash{}index\{Joe|textit\}}&\hspace*{0pt}\ignorespaces{}\hspace*{0pt}Joe, {\itshape \setmainfont[Path=/usr/share/fonts/truetype/cmu/,UprightFont=cmunrm.ttf,BoldFont=cmunbx.ttf,ItalicFont=cmunti.ttf,BoldItalicFont=cmunbi.ttf]{cmunti.ttf}\setmonofont[Path=/usr/share/fonts/truetype/cmu/,UprightFont=cmuntt.ttf,BoldFont=cmuntb.ttf,ItalicFont=cmunit.ttf,BoldItalicFont=cmuntx.ttf]{cmunti.ttf}\itshape 5}&\hspace*{0pt}\ignorespaces{}\hspace*{0pt}\setmainfont[Path=/usr/share/fonts/truetype/cmu/,UprightFont=cmunrm.ttf,BoldFont=cmunbx.ttf,ItalicFont=cmunti.ttf,BoldItalicFont=cmunbi.ttf]{cmunrm.ttf}\setmonofont[Path=/usr/share/fonts/truetype/cmu/,UprightFont=cmuntt.ttf,BoldFont=cmuntb.ttf,ItalicFont=cmunit.ttf,BoldItalicFont=cmuntx.ttf]{cmunrm.ttf}Same as above\\ \hline \hspace*{0pt}\ignorespaces{}\hspace*{0pt}\LaTeXTT{\textbackslash{}index\{ecole@\textbackslash{}\textquotesingle{}ecole\}}&\hspace*{0pt}\ignorespaces{}\hspace*{0pt}école, 4&\hspace*{0pt}\ignorespaces{}\hspace*{0pt}Handling of accents\\ \hline \hspace*{0pt}\ignorespaces{}\hspace*{0pt}\LaTeXTT{\textbackslash{}index\{Peter|see \{hello\}\}}&\hspace*{0pt}\ignorespaces{}\hspace*{0pt}Peter, {\itshape \setmainfont[Path=/usr/share/fonts/truetype/cmu/,UprightFont=cmunrm.ttf,BoldFont=cmunbx.ttf,ItalicFont=cmunti.ttf,BoldItalicFont=cmunbi.ttf]{cmunti.ttf}\setmonofont[Path=/usr/share/fonts/truetype/cmu/,UprightFont=cmuntt.ttf,BoldFont=cmuntb.ttf,ItalicFont=cmunit.ttf,BoldItalicFont=cmuntx.ttf]{cmunti.ttf}\itshape see}{$\text{ }$}\setmainfont[Path=/usr/share/fonts/truetype/cmu/,UprightFont=cmunrm.ttf,BoldFont=cmunbx.ttf,ItalicFont=cmunti.ttf,BoldItalicFont=cmunbi.ttf]{cmunrm.ttf}\setmonofont[Path=/usr/share/fonts/truetype/cmu/,UprightFont=cmuntt.ttf,BoldFont=cmuntb.ttf,ItalicFont=cmunit.ttf,BoldItalicFont=cmuntx.ttf]{cmunrm.ttf} hello&\hspace*{0pt}\ignorespaces{}\hspace*{0pt}Cross-{}references\\ \hline \hspace*{0pt}\ignorespaces{}\hspace*{0pt}\LaTeXTT{\textbackslash{}index\{Jen|seealso\{Jenny\}\}}&\hspace*{0pt}\ignorespaces{}\hspace*{0pt}Jen, {\itshape \setmainfont[Path=/usr/share/fonts/truetype/cmu/,UprightFont=cmunrm.ttf,BoldFont=cmunbx.ttf,ItalicFont=cmunti.ttf,BoldItalicFont=cmunbi.ttf]{cmunti.ttf}\setmonofont[Path=/usr/share/fonts/truetype/cmu/,UprightFont=cmuntt.ttf,BoldFont=cmuntb.ttf,ItalicFont=cmunit.ttf,BoldItalicFont=cmuntx.ttf]{cmunti.ttf}\itshape see also}{$\text{ }$}\setmainfont[Path=/usr/share/fonts/truetype/cmu/,UprightFont=cmunrm.ttf,BoldFont=cmunbx.ttf,ItalicFont=cmunti.ttf,BoldItalicFont=cmunbi.ttf]{cmunrm.ttf}\setmonofont[Path=/usr/share/fonts/truetype/cmu/,UprightFont=cmuntt.ttf,BoldFont=cmuntb.ttf,ItalicFont=cmunit.ttf,BoldItalicFont=cmuntx.ttf]{cmunrm.ttf} Jenny&\hspace*{0pt}\ignorespaces{}\hspace*{0pt}Same as above\\ \hline 
\end{longtable}

\subsubsection{Subentries}
\label{631}
If some entry has subsections, these can be marked off with {\ttfamily \setmainfont[Path=/usr/share/fonts/truetype/cmu/,UprightFont=cmunrm.ttf,BoldFont=cmunbx.ttf,ItalicFont=cmunti.ttf,BoldItalicFont=cmunbi.ttf]{cmuntt.ttf}\setmonofont[Path=/usr/share/fonts/truetype/cmu/,UprightFont=cmuntt.ttf,BoldFont=cmuntb.ttf,ItalicFont=cmunit.ttf,BoldItalicFont=cmuntx.ttf]{cmuntt.ttf}\ttfamily !}\setmainfont[Path=/usr/share/fonts/truetype/cmu/,UprightFont=cmunrm.ttf,BoldFont=cmunbx.ttf,ItalicFont=cmunti.ttf,BoldItalicFont=cmunbi.ttf]{cmunrm.ttf}\setmonofont[Path=/usr/share/fonts/truetype/cmu/,UprightFont=cmuntt.ttf,BoldFont=cmuntb.ttf,ItalicFont=cmunit.ttf,BoldItalicFont=cmuntx.ttf]{cmunrm.ttf}. For example,
\begin{Shaded}
\begin{Highlighting}[]

\NormalTok{\textbackslash{}index\{encodings!input!cp850\}}
\end{Highlighting}
\end{Shaded}

would create an index entry with \textquotesingle{}cp850\textquotesingle{} categorized under \textquotesingle{}input\textquotesingle{} (which itself is categorized into \textquotesingle{}encodings\textquotesingle{}). These are called subsubentries and subentries in makeidx terminology.
\subsubsection{Controlling sorting}
\label{632}
In order to determine how an index key is sorted, place a value to sort by before the key with the {\ttfamily \setmainfont[Path=/usr/share/fonts/truetype/cmu/,UprightFont=cmunrm.ttf,BoldFont=cmunbx.ttf,ItalicFont=cmunti.ttf,BoldItalicFont=cmunbi.ttf]{cmuntt.ttf}\setmonofont[Path=/usr/share/fonts/truetype/cmu/,UprightFont=cmuntt.ttf,BoldFont=cmuntb.ttf,ItalicFont=cmunit.ttf,BoldItalicFont=cmuntx.ttf]{cmuntt.ttf}\ttfamily @}{$\text{ }$}\setmainfont[Path=/usr/share/fonts/truetype/cmu/,UprightFont=cmunrm.ttf,BoldFont=cmunbx.ttf,ItalicFont=cmunti.ttf,BoldItalicFont=cmunbi.ttf]{cmunrm.ttf}\setmonofont[Path=/usr/share/fonts/truetype/cmu/,UprightFont=cmuntt.ttf,BoldFont=cmuntb.ttf,ItalicFont=cmunit.ttf,BoldItalicFont=cmuntx.ttf]{cmunrm.ttf} as a separator. This is useful if there is any formatting or math mode, so one example may be
\begin{Shaded}
\begin{Highlighting}[]

\NormalTok{\textbackslash{}index\{F@$\textbackslash{}vec\{F\}$\}}
\end{Highlighting}
\end{Shaded}

so that the entry in the index will show as \textquotesingle{}{$\vec{F}$}\textquotesingle{} but be sorted as \textquotesingle{}F\textquotesingle{}.

To combine with the above feature for subentries, you should style the appropriate component(s):
\begin{Shaded}
\begin{Highlighting}[]

\NormalTok{\textbackslash{}index\{bug reports!In re code@\textbackslash{}emph\{In re\} code\}}
 
\NormalTok{\textbackslash{}index\{LaTeX@\textbackslash{}LaTeX!Typesetting engine\}}
\end{Highlighting}
\end{Shaded}

\subsubsection{Changing page number style}
\label{633}
To change the formatting of a page number, append a {\ttfamily \setmainfont[Path=/usr/share/fonts/truetype/cmu/,UprightFont=cmunrm.ttf,BoldFont=cmunbx.ttf,ItalicFont=cmunti.ttf,BoldItalicFont=cmunbi.ttf]{cmuntt.ttf}\setmonofont[Path=/usr/share/fonts/truetype/cmu/,UprightFont=cmuntt.ttf,BoldFont=cmuntb.ttf,ItalicFont=cmunit.ttf,BoldItalicFont=cmuntx.ttf]{cmuntt.ttf}\ttfamily |}{$\text{ }$}\setmainfont[Path=/usr/share/fonts/truetype/cmu/,UprightFont=cmunrm.ttf,BoldFont=cmunbx.ttf,ItalicFont=cmunti.ttf,BoldItalicFont=cmunbi.ttf]{cmunrm.ttf}\setmonofont[Path=/usr/share/fonts/truetype/cmu/,UprightFont=cmuntt.ttf,BoldFont=cmuntb.ttf,ItalicFont=cmunit.ttf,BoldItalicFont=cmuntx.ttf]{cmunrm.ttf} and the name of some command which does the formatting. This command should only accept one argument.

For example, if on page 3 of a book you introduce bulldogs and include the command
\begin{Shaded}
\begin{Highlighting}[]

\NormalTok{\textbackslash{}index\{bulldog\}}
\end{Highlighting}
\end{Shaded}

and on page 10 of the same book you wish to show the main section on bulldogs with a bold page number, use
\begin{Shaded}
\begin{Highlighting}[]

\NormalTok{\textbackslash{}index\{bulldogtextbf\}}
\end{Highlighting}
\end{Shaded}

This will appear in the index as
bulldog, 3, {\bfseries \setmainfont[Path=/usr/share/fonts/truetype/cmu/,UprightFont=cmunrm.ttf,BoldFont=cmunbx.ttf,ItalicFont=cmunti.ttf,BoldItalicFont=cmunbi.ttf]{cmunbx.ttf}\setmonofont[Path=/usr/share/fonts/truetype/cmu/,UprightFont=cmuntt.ttf,BoldFont=cmuntb.ttf,ItalicFont=cmunit.ttf,BoldItalicFont=cmuntx.ttf]{cmunbx.ttf}\bfseries 10}\setmainfont[Path=/usr/share/fonts/truetype/cmu/,UprightFont=cmunrm.ttf,BoldFont=cmunbx.ttf,ItalicFont=cmunti.ttf,BoldItalicFont=cmunbi.ttf]{cmunrm.ttf}\setmonofont[Path=/usr/share/fonts/truetype/cmu/,UprightFont=cmuntt.ttf,BoldFont=cmuntb.ttf,ItalicFont=cmunit.ttf,BoldItalicFont=cmuntx.ttf]{cmunrm.ttf}

If you use {\ttfamily \setmainfont[Path=/usr/share/fonts/truetype/cmu/,UprightFont=cmunrm.ttf,BoldFont=cmunbx.ttf,ItalicFont=cmunti.ttf,BoldItalicFont=cmunbi.ttf]{cmuntt.ttf}\setmonofont[Path=/usr/share/fonts/truetype/cmu/,UprightFont=cmuntt.ttf,BoldFont=cmuntb.ttf,ItalicFont=cmunit.ttf,BoldItalicFont=cmuntx.ttf]{cmuntt.ttf}\ttfamily texindy}{$\text{ }$}\setmainfont[Path=/usr/share/fonts/truetype/cmu/,UprightFont=cmunrm.ttf,BoldFont=cmunbx.ttf,ItalicFont=cmunti.ttf,BoldItalicFont=cmunbi.ttf]{cmunrm.ttf}\setmonofont[Path=/usr/share/fonts/truetype/cmu/,UprightFont=cmuntt.ttf,BoldFont=cmuntb.ttf,ItalicFont=cmunit.ttf,BoldItalicFont=cmuntx.ttf]{cmunrm.ttf} in place of {\ttfamily \setmainfont[Path=/usr/share/fonts/truetype/cmu/,UprightFont=cmunrm.ttf,BoldFont=cmunbx.ttf,ItalicFont=cmunti.ttf,BoldItalicFont=cmunbi.ttf]{cmuntt.ttf}\setmonofont[Path=/usr/share/fonts/truetype/cmu/,UprightFont=cmuntt.ttf,BoldFont=cmuntb.ttf,ItalicFont=cmunit.ttf,BoldItalicFont=cmuntx.ttf]{cmuntt.ttf}\ttfamily makeindex}\setmainfont[Path=/usr/share/fonts/truetype/cmu/,UprightFont=cmunrm.ttf,BoldFont=cmunbx.ttf,ItalicFont=cmunti.ttf,BoldItalicFont=cmunbi.ttf]{cmunrm.ttf}\setmonofont[Path=/usr/share/fonts/truetype/cmu/,UprightFont=cmuntt.ttf,BoldFont=cmuntb.ttf,ItalicFont=cmunit.ttf,BoldItalicFont=cmuntx.ttf]{cmunrm.ttf}, the classified entries will be sorted too, such that all the bolded entries will be placed before all others by default.
\subsubsection{Multiple pages}
\label{634}
To perform multi-{}page indexing, add a {\ttfamily \setmainfont[Path=/usr/share/fonts/truetype/cmu/,UprightFont=cmunrm.ttf,BoldFont=cmunbx.ttf,ItalicFont=cmunti.ttf,BoldItalicFont=cmunbi.ttf]{cmuntt.ttf}\setmonofont[Path=/usr/share/fonts/truetype/cmu/,UprightFont=cmuntt.ttf,BoldFont=cmuntb.ttf,ItalicFont=cmunit.ttf,BoldItalicFont=cmuntx.ttf]{cmuntt.ttf}\ttfamily |(}{$\text{ }$}\setmainfont[Path=/usr/share/fonts/truetype/cmu/,UprightFont=cmunrm.ttf,BoldFont=cmunbx.ttf,ItalicFont=cmunti.ttf,BoldItalicFont=cmunbi.ttf]{cmunrm.ttf}\setmonofont[Path=/usr/share/fonts/truetype/cmu/,UprightFont=cmuntt.ttf,BoldFont=cmuntb.ttf,ItalicFont=cmunit.ttf,BoldItalicFont=cmuntx.ttf]{cmunrm.ttf} and {\ttfamily \setmainfont[Path=/usr/share/fonts/truetype/cmu/,UprightFont=cmunrm.ttf,BoldFont=cmunbx.ttf,ItalicFont=cmunti.ttf,BoldItalicFont=cmunbi.ttf]{cmuntt.ttf}\setmonofont[Path=/usr/share/fonts/truetype/cmu/,UprightFont=cmuntt.ttf,BoldFont=cmuntb.ttf,ItalicFont=cmunit.ttf,BoldItalicFont=cmuntx.ttf]{cmuntt.ttf}\ttfamily |)}{$\text{ }$}\setmainfont[Path=/usr/share/fonts/truetype/cmu/,UprightFont=cmunrm.ttf,BoldFont=cmunbx.ttf,ItalicFont=cmunti.ttf,BoldItalicFont=cmunbi.ttf]{cmunrm.ttf}\setmonofont[Path=/usr/share/fonts/truetype/cmu/,UprightFont=cmuntt.ttf,BoldFont=cmuntb.ttf,ItalicFont=cmunit.ttf,BoldItalicFont=cmuntx.ttf]{cmunrm.ttf} to the end of the \LaTeXTT{\textbackslash{}index} command, as in
\begin{Shaded}
\begin{Highlighting}[]

\NormalTok{\textbackslash{}index\{Quantum Mechanics!History(\}}
\NormalTok{In 1901, Max Planck released his theory of radiation dependent on quantized}
 \NormalTok{energy.}
\NormalTok{While this explained the ultraviolet catastrophe in the spectrum of }
\NormalTok{blackbody radiation, this had far larger consequences as the beginnings of}
 \NormalTok{quantum mechanics.}
\NormalTok{...}
\NormalTok{\textbackslash{}index\{Quantum Mechanics!History)\}}
\end{Highlighting}
\end{Shaded}

The entry in the index for the subentry \textquotesingle{}History\textquotesingle{} will be the range of pages between the two \LaTeXTT{\textbackslash{}index} commands.
\subsubsection{Using special characters}
\label{635}
In order to place values with {\ttfamily \setmainfont[Path=/usr/share/fonts/truetype/cmu/,UprightFont=cmunrm.ttf,BoldFont=cmunbx.ttf,ItalicFont=cmunti.ttf,BoldItalicFont=cmunbi.ttf]{cmuntt.ttf}\setmonofont[Path=/usr/share/fonts/truetype/cmu/,UprightFont=cmuntt.ttf,BoldFont=cmuntb.ttf,ItalicFont=cmunit.ttf,BoldItalicFont=cmuntx.ttf]{cmuntt.ttf}\ttfamily !}\setmainfont[Path=/usr/share/fonts/truetype/cmu/,UprightFont=cmunrm.ttf,BoldFont=cmunbx.ttf,ItalicFont=cmunti.ttf,BoldItalicFont=cmunbi.ttf]{cmunrm.ttf}\setmonofont[Path=/usr/share/fonts/truetype/cmu/,UprightFont=cmuntt.ttf,BoldFont=cmuntb.ttf,ItalicFont=cmunit.ttf,BoldItalicFont=cmuntx.ttf]{cmunrm.ttf}, {\ttfamily \setmainfont[Path=/usr/share/fonts/truetype/cmu/,UprightFont=cmunrm.ttf,BoldFont=cmunbx.ttf,ItalicFont=cmunti.ttf,BoldItalicFont=cmunbi.ttf]{cmuntt.ttf}\setmonofont[Path=/usr/share/fonts/truetype/cmu/,UprightFont=cmuntt.ttf,BoldFont=cmuntb.ttf,ItalicFont=cmunit.ttf,BoldItalicFont=cmuntx.ttf]{cmuntt.ttf}\ttfamily @}\setmainfont[Path=/usr/share/fonts/truetype/cmu/,UprightFont=cmunrm.ttf,BoldFont=cmunbx.ttf,ItalicFont=cmunti.ttf,BoldItalicFont=cmunbi.ttf]{cmunrm.ttf}\setmonofont[Path=/usr/share/fonts/truetype/cmu/,UprightFont=cmuntt.ttf,BoldFont=cmuntb.ttf,ItalicFont=cmunit.ttf,BoldItalicFont=cmuntx.ttf]{cmunrm.ttf}, or {\ttfamily \setmainfont[Path=/usr/share/fonts/truetype/cmu/,UprightFont=cmunrm.ttf,BoldFont=cmunbx.ttf,ItalicFont=cmunti.ttf,BoldItalicFont=cmunbi.ttf]{cmuntt.ttf}\setmonofont[Path=/usr/share/fonts/truetype/cmu/,UprightFont=cmuntt.ttf,BoldFont=cmuntb.ttf,ItalicFont=cmunit.ttf,BoldItalicFont=cmuntx.ttf]{cmuntt.ttf}\ttfamily |}\setmainfont[Path=/usr/share/fonts/truetype/cmu/,UprightFont=cmunrm.ttf,BoldFont=cmunbx.ttf,ItalicFont=cmunti.ttf,BoldItalicFont=cmunbi.ttf]{cmunrm.ttf}\setmonofont[Path=/usr/share/fonts/truetype/cmu/,UprightFont=cmuntt.ttf,BoldFont=cmuntb.ttf,ItalicFont=cmunit.ttf,BoldItalicFont=cmuntx.ttf]{cmunrm.ttf}, which are otherwise escape characters, in the index, one must quote these characters in the \LaTeXTT{\textbackslash{}index} command by putting a double quotation mark ({\ttfamily \setmainfont[Path=/usr/share/fonts/truetype/cmu/,UprightFont=cmunrm.ttf,BoldFont=cmunbx.ttf,ItalicFont=cmunti.ttf,BoldItalicFont=cmunbi.ttf]{cmuntt.ttf}\setmonofont[Path=/usr/share/fonts/truetype/cmu/,UprightFont=cmuntt.ttf,BoldFont=cmuntb.ttf,ItalicFont=cmunit.ttf,BoldItalicFont=cmuntx.ttf]{cmuntt.ttf}\ttfamily \symbol{34}}\setmainfont[Path=/usr/share/fonts/truetype/cmu/,UprightFont=cmunrm.ttf,BoldFont=cmunbx.ttf,ItalicFont=cmunti.ttf,BoldItalicFont=cmunbi.ttf]{cmunrm.ttf}\setmonofont[Path=/usr/share/fonts/truetype/cmu/,UprightFont=cmuntt.ttf,BoldFont=cmuntb.ttf,ItalicFont=cmunit.ttf,BoldItalicFont=cmuntx.ttf]{cmunrm.ttf}) in front of them, and one can only place a {\ttfamily \setmainfont[Path=/usr/share/fonts/truetype/cmu/,UprightFont=cmunrm.ttf,BoldFont=cmunbx.ttf,ItalicFont=cmunti.ttf,BoldItalicFont=cmunbi.ttf]{cmuntt.ttf}\setmonofont[Path=/usr/share/fonts/truetype/cmu/,UprightFont=cmuntt.ttf,BoldFont=cmuntb.ttf,ItalicFont=cmunit.ttf,BoldItalicFont=cmuntx.ttf]{cmuntt.ttf}\ttfamily \symbol{34}}{$\text{ }$}\setmainfont[Path=/usr/share/fonts/truetype/cmu/,UprightFont=cmunrm.ttf,BoldFont=cmunbx.ttf,ItalicFont=cmunti.ttf,BoldItalicFont=cmunbi.ttf]{cmunrm.ttf}\setmonofont[Path=/usr/share/fonts/truetype/cmu/,UprightFont=cmuntt.ttf,BoldFont=cmuntb.ttf,ItalicFont=cmunit.ttf,BoldItalicFont=cmuntx.ttf]{cmunrm.ttf} in the index by quoting it (i.e., a key for \symbol{34} would be \LaTeXTT{\textbackslash{}index\{\symbol{34}\symbol{34}\}}).

This rule does not hold for \textbackslash{}\symbol{34}, so to put the letter ä in the index, one may still use \LaTeXTT{\textbackslash{}index\{a@\textbackslash{}\symbol{34}\{a\}\}}.
\section{Abbreviation list}
\label{636}

You can make a list of abbreviations with the package \LaTeXTT{nomencl} \myplainurl{http://www.ctan.org/tex-archive/macros/latex/contrib/nomencl/}.
You may also be interested in using the \LaTeXTT{glossaries} package described in the \mylref{643}{Glossary} chapter. Another option is the package \LaTeXTT{acronym} \myplainurl{http://www.ctan.org/pkg/acronym/}.

To enable the Nomenclature feature of LaTeX, the \LaTeXTT{nomencl} package must be loaded in the preamble with:
\begin{Shaded}
\begin{Highlighting}[]

\NormalTok{\textbackslash{}usepackage[⟨options ⟩]\{nomencl\}}
\NormalTok{\textbackslash{}makenomenclature}
\end{Highlighting}
\end{Shaded}


Issue the \LaTeXTT{\textbackslash{}nomenclature{$\text{[}$}\setmainfont[Path=/usr/share/fonts/truetype/freefont/,UprightFont=FreeSerif.ttf,BoldFont=FreeSerifBold.ttf,ItalicFont=FreeSerifItalic.ttf,BoldItalicFont=FreeSerifBoldItalic.ttf]{FreeSerif.ttf}\setmonofont[Path=/usr/share/fonts/truetype/freefont/,UprightFont=FreeMono.ttf,BoldFont=FreeMonoBold.ttf,ItalicFont=FreeMonoOblique.ttf,BoldItalicFont=FreeMonoBoldOblique.ttf]{FreeSerif.ttf}⟨\setmainfont[Path=/usr/share/fonts/truetype/cmu/,UprightFont=cmunrm.ttf,BoldFont=cmunbx.ttf,ItalicFont=cmunti.ttf,BoldItalicFont=cmunbi.ttf]{cmunrm.ttf}\setmonofont[Path=/usr/share/fonts/truetype/cmu/,UprightFont=cmuntt.ttf,BoldFont=cmuntb.ttf,ItalicFont=cmunit.ttf,BoldItalicFont=cmuntx.ttf]{cmunrm.ttf}prefix\setmainfont[Path=/usr/share/fonts/truetype/freefont/,UprightFont=FreeSerif.ttf,BoldFont=FreeSerifBold.ttf,ItalicFont=FreeSerifItalic.ttf,BoldItalicFont=FreeSerifBoldItalic.ttf]{FreeSerif.ttf}\setmonofont[Path=/usr/share/fonts/truetype/freefont/,UprightFont=FreeMono.ttf,BoldFont=FreeMonoBold.ttf,ItalicFont=FreeMonoOblique.ttf,BoldItalicFont=FreeMonoBoldOblique.ttf]{FreeSerif.ttf}⟩\setmainfont[Path=/usr/share/fonts/truetype/cmu/,UprightFont=cmunrm.ttf,BoldFont=cmunbx.ttf,ItalicFont=cmunti.ttf,BoldItalicFont=cmunbi.ttf]{cmunrm.ttf}\setmonofont[Path=/usr/share/fonts/truetype/cmu/,UprightFont=cmuntt.ttf,BoldFont=cmuntb.ttf,ItalicFont=cmunit.ttf,BoldItalicFont=cmuntx.ttf]{cmunrm.ttf}{$\text{]}$}\{\setmainfont[Path=/usr/share/fonts/truetype/freefont/,UprightFont=FreeSerif.ttf,BoldFont=FreeSerifBold.ttf,ItalicFont=FreeSerifItalic.ttf,BoldItalicFont=FreeSerifBoldItalic.ttf]{FreeSerif.ttf}\setmonofont[Path=/usr/share/fonts/truetype/freefont/,UprightFont=FreeMono.ttf,BoldFont=FreeMonoBold.ttf,ItalicFont=FreeMonoOblique.ttf,BoldItalicFont=FreeMonoBoldOblique.ttf]{FreeSerif.ttf}⟨\setmainfont[Path=/usr/share/fonts/truetype/cmu/,UprightFont=cmunrm.ttf,BoldFont=cmunbx.ttf,ItalicFont=cmunti.ttf,BoldItalicFont=cmunbi.ttf]{cmunrm.ttf}\setmonofont[Path=/usr/share/fonts/truetype/cmu/,UprightFont=cmuntt.ttf,BoldFont=cmuntb.ttf,ItalicFont=cmunit.ttf,BoldItalicFont=cmuntx.ttf]{cmunrm.ttf}symbol\setmainfont[Path=/usr/share/fonts/truetype/freefont/,UprightFont=FreeSerif.ttf,BoldFont=FreeSerifBold.ttf,ItalicFont=FreeSerifItalic.ttf,BoldItalicFont=FreeSerifBoldItalic.ttf]{FreeSerif.ttf}\setmonofont[Path=/usr/share/fonts/truetype/freefont/,UprightFont=FreeMono.ttf,BoldFont=FreeMonoBold.ttf,ItalicFont=FreeMonoOblique.ttf,BoldItalicFont=FreeMonoBoldOblique.ttf]{FreeSerif.ttf}⟩\setmainfont[Path=/usr/share/fonts/truetype/cmu/,UprightFont=cmunrm.ttf,BoldFont=cmunbx.ttf,ItalicFont=cmunti.ttf,BoldItalicFont=cmunbi.ttf]{cmunrm.ttf}\setmonofont[Path=/usr/share/fonts/truetype/cmu/,UprightFont=cmuntt.ttf,BoldFont=cmuntb.ttf,ItalicFont=cmunit.ttf,BoldItalicFont=cmuntx.ttf]{cmunrm.ttf}\}\{\setmainfont[Path=/usr/share/fonts/truetype/freefont/,UprightFont=FreeSerif.ttf,BoldFont=FreeSerifBold.ttf,ItalicFont=FreeSerifItalic.ttf,BoldItalicFont=FreeSerifBoldItalic.ttf]{FreeSerif.ttf}\setmonofont[Path=/usr/share/fonts/truetype/freefont/,UprightFont=FreeMono.ttf,BoldFont=FreeMonoBold.ttf,ItalicFont=FreeMonoOblique.ttf,BoldItalicFont=FreeMonoBoldOblique.ttf]{FreeSerif.ttf}⟨\setmainfont[Path=/usr/share/fonts/truetype/cmu/,UprightFont=cmunrm.ttf,BoldFont=cmunbx.ttf,ItalicFont=cmunti.ttf,BoldItalicFont=cmunbi.ttf]{cmunrm.ttf}\setmonofont[Path=/usr/share/fonts/truetype/cmu/,UprightFont=cmuntt.ttf,BoldFont=cmuntb.ttf,ItalicFont=cmunit.ttf,BoldItalicFont=cmuntx.ttf]{cmunrm.ttf}description\setmainfont[Path=/usr/share/fonts/truetype/freefont/,UprightFont=FreeSerif.ttf,BoldFont=FreeSerifBold.ttf,ItalicFont=FreeSerifItalic.ttf,BoldItalicFont=FreeSerifBoldItalic.ttf]{FreeSerif.ttf}\setmonofont[Path=/usr/share/fonts/truetype/freefont/,UprightFont=FreeMono.ttf,BoldFont=FreeMonoBold.ttf,ItalicFont=FreeMonoOblique.ttf,BoldItalicFont=FreeMonoBoldOblique.ttf]{FreeSerif.ttf}⟩\setmainfont[Path=/usr/share/fonts/truetype/cmu/,UprightFont=cmunrm.ttf,BoldFont=cmunbx.ttf,ItalicFont=cmunti.ttf,BoldItalicFont=cmunbi.ttf]{cmunrm.ttf}\setmonofont[Path=/usr/share/fonts/truetype/cmu/,UprightFont=cmuntt.ttf,BoldFont=cmuntb.ttf,ItalicFont=cmunit.ttf,BoldItalicFont=cmuntx.ttf]{cmunrm.ttf}\}} command for each symbol you want to have included in the nomenclature list. The best place for this command is immediately after you introduce the symbol for the first time. Put \LaTeXTT{\textbackslash{}printnomenclature} at the place you want to have your nomenclature list. 

Run LaTeX 2 times then
\\

\TemplateSpaceIndent{$\text{ }${}{\ttfamily \setmainfont[Path=/usr/share/fonts/truetype/cmu/,UprightFont=cmunrm.ttf,BoldFont=cmunbx.ttf,ItalicFont=cmunti.ttf,BoldItalicFont=cmunbi.ttf]{cmuntt.ttf}\setmonofont[Path=/usr/share/fonts/truetype/cmu/,UprightFont=cmuntt.ttf,BoldFont=cmuntb.ttf,ItalicFont=cmunit.ttf,BoldItalicFont=cmuntx.ttf]{cmuntt.ttf}\ttfamily makeindex}$\text{ }${}{\itshape \setmainfont[Path=/usr/share/fonts/truetype/cmu/,UprightFont=cmunrm.ttf,BoldFont=cmunbx.ttf,ItalicFont=cmunti.ttf,BoldItalicFont=cmunbi.ttf]{cmunti.ttf}\setmonofont[Path=/usr/share/fonts/truetype/cmu/,UprightFont=cmuntt.ttf,BoldFont=cmuntb.ttf,ItalicFont=cmunit.ttf,BoldItalicFont=cmuntx.ttf]{cmunti.ttf}\itshape filename.nlo}$\text{ }${}$\text{ }${}\setmainfont[Path=/usr/share/fonts/truetype/cmu/,UprightFont=cmunrm.ttf,BoldFont=cmunbx.ttf,ItalicFont=cmunti.ttf,BoldItalicFont=cmunbi.ttf]{cmunrm.ttf}\setmonofont[Path=/usr/share/fonts/truetype/cmu/,UprightFont=cmuntt.ttf,BoldFont=cmuntb.ttf,ItalicFont=cmunit.ttf,BoldItalicFont=cmuntx.ttf]{cmunrm.ttf}-{}s$\text{ }${}nomencl.ist$\text{ }${}-{}o$\text{ }${}{\itshape \setmainfont[Path=/usr/share/fonts/truetype/cmu/,UprightFont=cmunrm.ttf,BoldFont=cmunbx.ttf,ItalicFont=cmunti.ttf,BoldItalicFont=cmunbi.ttf]{cmunti.ttf}\setmonofont[Path=/usr/share/fonts/truetype/cmu/,UprightFont=cmuntt.ttf,BoldFont=cmuntb.ttf,ItalicFont=cmunit.ttf,BoldItalicFont=cmuntx.ttf]{cmunti.ttf}\itshape filename.nls}}
\setmainfont[Path=/usr/share/fonts/truetype/cmu/,UprightFont=cmunrm.ttf,BoldFont=cmunbx.ttf,ItalicFont=cmunti.ttf,BoldItalicFont=cmunbi.ttf]{cmunrm.ttf}\setmonofont[Path=/usr/share/fonts/truetype/cmu/,UprightFont=cmuntt.ttf,BoldFont=cmuntb.ttf,ItalicFont=cmunit.ttf,BoldItalicFont=cmuntx.ttf]{cmunrm.ttf}

followed by running LaTeX once again.

To add the abbreviation list to the table of content, \LaTeXTT{intoc} option can be used when declare the \LaTeXTT{nomencl} package,  i.e. \begin{Shaded}
\begin{Highlighting}[]
\NormalTok{\textbackslash{}usepackage[intoc]\{nomencl\} }
\end{Highlighting}
\end{Shaded}
 instead of using the code in \mylref{638}{Adding Index to Table Of Contents} section.

The title of the list can be changed using the following command:
\begin{Shaded}
\begin{Highlighting}[]
\NormalTok{\textbackslash{}renewcommand\{\textbackslash{}nomname\}\{List of Abbreviations\}}
\end{Highlighting}
\end{Shaded}

\section{Multiple indices}
\label{637}

If you need multiple indices you can use the package \LaTeXTT{multind} \myplainurl{http://www.tex.ac.uk/cgi-bin/texfaq2html?label=multind}. 

This package provides the same commands as \LaTeXTT{makeidx}, but now you also have to pass a name as the first argument to every command.
\begin{Shaded}
\begin{Highlighting}[]

\NormalTok{\textbackslash{}usepackage\{multind\}}
\NormalTok{\textbackslash{}makeindex\{books\}}
\NormalTok{\textbackslash{}makeindex\{authors\}}
\NormalTok{...}
\NormalTok{\textbackslash{}index\{books\}\{A book to index\}}
\NormalTok{\textbackslash{}index\{authors\}\{Put this author in the index\}}
\NormalTok{...}
\NormalTok{\textbackslash{}printindex\{books\}\{The Books index\}}
\NormalTok{\textbackslash{}printindex\{authors\}\{The Authors index\}}
\end{Highlighting}
\end{Shaded}

\section{Adding index to table of contents}
\label{638}

By default, Index won\textquotesingle{}t show in Table Of Contents, so you have to add it manually.

To add index as a chapter, use these commands:
\begin{Shaded}
\begin{Highlighting}[]

\NormalTok{\textbackslash{}clearpage}
\NormalTok{\textbackslash{}addcontentsline\{toc\}\{chapter\}\{Index\}}
\NormalTok{\textbackslash{}printindex}
\end{Highlighting}
\end{Shaded}


If you use the book class, you may want to start it on an odd page by using \LaTeXTT{\textbackslash{}cleardoublepage}.
\section{International indices}
\label{639}
If you want to sort entries that have international characters (such as ő, ą, ó, ç, etc.) you may find that the sorting \symbol{34}is not quite right\symbol{34}. In most cases the characters are treated as special characters and end up in the same group as @, ¶ or µ. In most languages that use Latin alphabet it\textquotesingle{}s not correct.
\subsection{Generating index}
\label{640}
Unfortunately, current version of {\ttfamily \setmainfont[Path=/usr/share/fonts/truetype/cmu/,UprightFont=cmunrm.ttf,BoldFont=cmunbx.ttf,ItalicFont=cmunti.ttf,BoldItalicFont=cmunbi.ttf]{cmuntt.ttf}\setmonofont[Path=/usr/share/fonts/truetype/cmu/,UprightFont=cmuntt.ttf,BoldFont=cmuntb.ttf,ItalicFont=cmunit.ttf,BoldItalicFont=cmuntx.ttf]{cmuntt.ttf}\ttfamily xindy}{$\text{ }$}\setmainfont[Path=/usr/share/fonts/truetype/cmu/,UprightFont=cmunrm.ttf,BoldFont=cmunbx.ttf,ItalicFont=cmunti.ttf,BoldItalicFont=cmunbi.ttf]{cmunrm.ttf}\setmonofont[Path=/usr/share/fonts/truetype/cmu/,UprightFont=cmuntt.ttf,BoldFont=cmuntb.ttf,ItalicFont=cmunit.ttf,BoldItalicFont=cmuntx.ttf]{cmunrm.ttf} and \LaTeXTT{hyperref} are incompatible. When you use \LaTeXTT{textbf} or \LaTeXTT{textit} modifiers, {\ttfamily \setmainfont[Path=/usr/share/fonts/truetype/cmu/,UprightFont=cmunrm.ttf,BoldFont=cmunbx.ttf,ItalicFont=cmunti.ttf,BoldItalicFont=cmunbi.ttf]{cmuntt.ttf}\setmonofont[Path=/usr/share/fonts/truetype/cmu/,UprightFont=cmuntt.ttf,BoldFont=cmuntb.ttf,ItalicFont=cmunit.ttf,BoldItalicFont=cmuntx.ttf]{cmuntt.ttf}\ttfamily texindy}{$\text{ }$}\setmainfont[Path=/usr/share/fonts/truetype/cmu/,UprightFont=cmunrm.ttf,BoldFont=cmunbx.ttf,ItalicFont=cmunti.ttf,BoldItalicFont=cmunbi.ttf]{cmunrm.ttf}\setmonofont[Path=/usr/share/fonts/truetype/cmu/,UprightFont=cmuntt.ttf,BoldFont=cmuntb.ttf,ItalicFont=cmunit.ttf,BoldItalicFont=cmuntx.ttf]{cmunrm.ttf} will print error message:{\ttfamily \setmainfont[Path=/usr/share/fonts/truetype/cmu/,UprightFont=cmunrm.ttf,BoldFont=cmunbx.ttf,ItalicFont=cmunti.ttf,BoldItalicFont=cmunbi.ttf]{cmuntt.ttf}\setmonofont[Path=/usr/share/fonts/truetype/cmu/,UprightFont=cmuntt.ttf,BoldFont=cmuntb.ttf,ItalicFont=cmunit.ttf,BoldItalicFont=cmuntx.ttf]{cmuntt.ttf}\ttfamily unknown cross-{}reference-{}class `hyperindexformat\textquotesingle{}! (ignored)}{$\text{ }$}\setmainfont[Path=/usr/share/fonts/truetype/cmu/,UprightFont=cmunrm.ttf,BoldFont=cmunbx.ttf,ItalicFont=cmunti.ttf,BoldItalicFont=cmunbi.ttf]{cmunrm.ttf}\setmonofont[Path=/usr/share/fonts/truetype/cmu/,UprightFont=cmuntt.ttf,BoldFont=cmuntb.ttf,ItalicFont=cmunit.ttf,BoldItalicFont=cmuntx.ttf]{cmunrm.ttf} and won\textquotesingle{}t add those pages to index. Work-{}around for this bug is described on the \myhref{https://en.wikibooks.org/wiki/Talk\%3ALaTeX\%2FIndexing\%23Texindy\%2C\%20hyperref\%20and\%20textbf\%2C\%20textit\%20modifiers}{talk page}.

To generate international index file you have to use {\ttfamily \setmainfont[Path=/usr/share/fonts/truetype/cmu/,UprightFont=cmunrm.ttf,BoldFont=cmunbx.ttf,ItalicFont=cmunti.ttf,BoldItalicFont=cmunbi.ttf]{cmuntt.ttf}\setmonofont[Path=/usr/share/fonts/truetype/cmu/,UprightFont=cmuntt.ttf,BoldFont=cmuntb.ttf,ItalicFont=cmunit.ttf,BoldItalicFont=cmuntx.ttf]{cmuntt.ttf}\ttfamily texindy}{$\text{ }$}\setmainfont[Path=/usr/share/fonts/truetype/cmu/,UprightFont=cmunrm.ttf,BoldFont=cmunbx.ttf,ItalicFont=cmunti.ttf,BoldItalicFont=cmunbi.ttf]{cmunrm.ttf}\setmonofont[Path=/usr/share/fonts/truetype/cmu/,UprightFont=cmuntt.ttf,BoldFont=cmuntb.ttf,ItalicFont=cmunit.ttf,BoldItalicFont=cmuntx.ttf]{cmunrm.ttf} instead of {\ttfamily \setmainfont[Path=/usr/share/fonts/truetype/cmu/,UprightFont=cmunrm.ttf,BoldFont=cmunbx.ttf,ItalicFont=cmunti.ttf,BoldItalicFont=cmunbi.ttf]{cmuntt.ttf}\setmonofont[Path=/usr/share/fonts/truetype/cmu/,UprightFont=cmuntt.ttf,BoldFont=cmuntb.ttf,ItalicFont=cmunit.ttf,BoldItalicFont=cmuntx.ttf]{cmuntt.ttf}\ttfamily makeindex}\setmainfont[Path=/usr/share/fonts/truetype/cmu/,UprightFont=cmunrm.ttf,BoldFont=cmunbx.ttf,ItalicFont=cmunti.ttf,BoldItalicFont=cmunbi.ttf]{cmunrm.ttf}\setmonofont[Path=/usr/share/fonts/truetype/cmu/,UprightFont=cmuntt.ttf,BoldFont=cmuntb.ttf,ItalicFont=cmunit.ttf,BoldItalicFont=cmuntx.ttf]{cmunrm.ttf}.

\myhref{http://xindy.sourceforge.net/}{xindy} is a much more extensible and robust indexing system than the {\ttfamily \setmainfont[Path=/usr/share/fonts/truetype/cmu/,UprightFont=cmunrm.ttf,BoldFont=cmunbx.ttf,ItalicFont=cmunti.ttf,BoldItalicFont=cmunbi.ttf]{cmuntt.ttf}\setmonofont[Path=/usr/share/fonts/truetype/cmu/,UprightFont=cmuntt.ttf,BoldFont=cmuntb.ttf,ItalicFont=cmunit.ttf,BoldItalicFont=cmuntx.ttf]{cmuntt.ttf}\ttfamily makeindex}{$\text{ }$}\setmainfont[Path=/usr/share/fonts/truetype/cmu/,UprightFont=cmunrm.ttf,BoldFont=cmunbx.ttf,ItalicFont=cmunti.ttf,BoldItalicFont=cmunbi.ttf]{cmunrm.ttf}\setmonofont[Path=/usr/share/fonts/truetype/cmu/,UprightFont=cmuntt.ttf,BoldFont=cmuntb.ttf,ItalicFont=cmunit.ttf,BoldItalicFont=cmuntx.ttf]{cmunrm.ttf} system.

For example, one does not need to write:
\begin{Shaded}
\begin{Highlighting}[]

\NormalTok{\textbackslash{}index\{Lin@\textbackslash{}textbf\{Lin\}\}}
\end{Highlighting}
\end{Shaded}

to get the {\ttfamily \setmainfont[Path=/usr/share/fonts/truetype/cmu/,UprightFont=cmunrm.ttf,BoldFont=cmunbx.ttf,ItalicFont=cmunti.ttf,BoldItalicFont=cmunbi.ttf]{cmuntt.ttf}\setmonofont[Path=/usr/share/fonts/truetype/cmu/,UprightFont=cmuntt.ttf,BoldFont=cmuntb.ttf,ItalicFont=cmunit.ttf,BoldItalicFont=cmuntx.ttf]{cmuntt.ttf}\ttfamily Lin}{$\text{ }$}\setmainfont[Path=/usr/share/fonts/truetype/cmu/,UprightFont=cmunrm.ttf,BoldFont=cmunbx.ttf,ItalicFont=cmunti.ttf,BoldItalicFont=cmunbi.ttf]{cmunrm.ttf}\setmonofont[Path=/usr/share/fonts/truetype/cmu/,UprightFont=cmuntt.ttf,BoldFont=cmuntb.ttf,ItalicFont=cmunit.ttf,BoldItalicFont=cmuntx.ttf]{cmunrm.ttf} entry after {\ttfamily \setmainfont[Path=/usr/share/fonts/truetype/cmu/,UprightFont=cmunrm.ttf,BoldFont=cmunbx.ttf,ItalicFont=cmunti.ttf,BoldItalicFont=cmunbi.ttf]{cmuntt.ttf}\setmonofont[Path=/usr/share/fonts/truetype/cmu/,UprightFont=cmuntt.ttf,BoldFont=cmuntb.ttf,ItalicFont=cmunit.ttf,BoldItalicFont=cmuntx.ttf]{cmuntt.ttf}\ttfamily LAN}{$\text{ }$}\setmainfont[Path=/usr/share/fonts/truetype/cmu/,UprightFont=cmunrm.ttf,BoldFont=cmunbx.ttf,ItalicFont=cmunti.ttf,BoldItalicFont=cmunbi.ttf]{cmunrm.ttf}\setmonofont[Path=/usr/share/fonts/truetype/cmu/,UprightFont=cmuntt.ttf,BoldFont=cmuntb.ttf,ItalicFont=cmunit.ttf,BoldItalicFont=cmuntx.ttf]{cmunrm.ttf} and before {\ttfamily \setmainfont[Path=/usr/share/fonts/truetype/cmu/,UprightFont=cmunrm.ttf,BoldFont=cmunbx.ttf,ItalicFont=cmunti.ttf,BoldItalicFont=cmunbi.ttf]{cmuntt.ttf}\setmonofont[Path=/usr/share/fonts/truetype/cmu/,UprightFont=cmuntt.ttf,BoldFont=cmuntb.ttf,ItalicFont=cmunit.ttf,BoldItalicFont=cmuntx.ttf]{cmuntt.ttf}\ttfamily LZA}\setmainfont[Path=/usr/share/fonts/truetype/cmu/,UprightFont=cmunrm.ttf,BoldFont=cmunbx.ttf,ItalicFont=cmunti.ttf,BoldItalicFont=cmunbi.ttf]{cmunrm.ttf}\setmonofont[Path=/usr/share/fonts/truetype/cmu/,UprightFont=cmuntt.ttf,BoldFont=cmuntb.ttf,ItalicFont=cmunit.ttf,BoldItalicFont=cmuntx.ttf]{cmunrm.ttf}, instead, it\textquotesingle{}s enough to write
\begin{Shaded}
\begin{Highlighting}[]

\NormalTok{\textbackslash{}index\{\textbackslash{}textbf\{Lin\}\}}
\end{Highlighting}
\end{Shaded}


But what is much more important, it can properly sort index files in many languages, not only English. 

Unfortunately, generating indices ready to use by {\ttfamily \setmainfont[Path=/usr/share/fonts/truetype/cmu/,UprightFont=cmunrm.ttf,BoldFont=cmunbx.ttf,ItalicFont=cmunti.ttf,BoldItalicFont=cmunbi.ttf]{cmuntt.ttf}\setmonofont[Path=/usr/share/fonts/truetype/cmu/,UprightFont=cmuntt.ttf,BoldFont=cmuntb.ttf,ItalicFont=cmunit.ttf,BoldItalicFont=cmuntx.ttf]{cmuntt.ttf}\ttfamily LaTeX}{$\text{ }$}\setmainfont[Path=/usr/share/fonts/truetype/cmu/,UprightFont=cmunrm.ttf,BoldFont=cmunbx.ttf,ItalicFont=cmunti.ttf,BoldItalicFont=cmunbi.ttf]{cmunrm.ttf}\setmonofont[Path=/usr/share/fonts/truetype/cmu/,UprightFont=cmuntt.ttf,BoldFont=cmuntb.ttf,ItalicFont=cmunit.ttf,BoldItalicFont=cmuntx.ttf]{cmunrm.ttf} using {\ttfamily \setmainfont[Path=/usr/share/fonts/truetype/cmu/,UprightFont=cmunrm.ttf,BoldFont=cmunbx.ttf,ItalicFont=cmunti.ttf,BoldItalicFont=cmunbi.ttf]{cmuntt.ttf}\setmonofont[Path=/usr/share/fonts/truetype/cmu/,UprightFont=cmuntt.ttf,BoldFont=cmuntb.ttf,ItalicFont=cmunit.ttf,BoldItalicFont=cmuntx.ttf]{cmuntt.ttf}\ttfamily xindy}{$\text{ }$}\setmainfont[Path=/usr/share/fonts/truetype/cmu/,UprightFont=cmunrm.ttf,BoldFont=cmunbx.ttf,ItalicFont=cmunti.ttf,BoldItalicFont=cmunbi.ttf]{cmunrm.ttf}\setmonofont[Path=/usr/share/fonts/truetype/cmu/,UprightFont=cmuntt.ttf,BoldFont=cmuntb.ttf,ItalicFont=cmunit.ttf,BoldItalicFont=cmuntx.ttf]{cmunrm.ttf} is a bit more complicated than with {\ttfamily \setmainfont[Path=/usr/share/fonts/truetype/cmu/,UprightFont=cmunrm.ttf,BoldFont=cmunbx.ttf,ItalicFont=cmunti.ttf,BoldItalicFont=cmunbi.ttf]{cmuntt.ttf}\setmonofont[Path=/usr/share/fonts/truetype/cmu/,UprightFont=cmuntt.ttf,BoldFont=cmuntb.ttf,ItalicFont=cmunit.ttf,BoldItalicFont=cmuntx.ttf]{cmuntt.ttf}\ttfamily makeindex}\setmainfont[Path=/usr/share/fonts/truetype/cmu/,UprightFont=cmunrm.ttf,BoldFont=cmunbx.ttf,ItalicFont=cmunti.ttf,BoldItalicFont=cmunbi.ttf]{cmunrm.ttf}\setmonofont[Path=/usr/share/fonts/truetype/cmu/,UprightFont=cmuntt.ttf,BoldFont=cmuntb.ttf,ItalicFont=cmunit.ttf,BoldItalicFont=cmuntx.ttf]{cmunrm.ttf}.

First, we need to know in what encoding the {\ttfamily \setmainfont[Path=/usr/share/fonts/truetype/cmu/,UprightFont=cmunrm.ttf,BoldFont=cmunbx.ttf,ItalicFont=cmunti.ttf,BoldItalicFont=cmunbi.ttf]{cmuntt.ttf}\setmonofont[Path=/usr/share/fonts/truetype/cmu/,UprightFont=cmuntt.ttf,BoldFont=cmuntb.ttf,ItalicFont=cmunit.ttf,BoldItalicFont=cmuntx.ttf]{cmuntt.ttf}\ttfamily .tex}{$\text{ }$}\setmainfont[Path=/usr/share/fonts/truetype/cmu/,UprightFont=cmunrm.ttf,BoldFont=cmunbx.ttf,ItalicFont=cmunti.ttf,BoldItalicFont=cmunbi.ttf]{cmunrm.ttf}\setmonofont[Path=/usr/share/fonts/truetype/cmu/,UprightFont=cmuntt.ttf,BoldFont=cmuntb.ttf,ItalicFont=cmunit.ttf,BoldItalicFont=cmuntx.ttf]{cmunrm.ttf} project file is saved. In most cases it will be UTF-{}8 or ISO-{}8859-{}1, though if you live, for example in Poland it may be ISO-{}8859-{}2 or CP-{}1250. Check the parameter to the {\ttfamily \setmainfont[Path=/usr/share/fonts/truetype/cmu/,UprightFont=cmunrm.ttf,BoldFont=cmunbx.ttf,ItalicFont=cmunti.ttf,BoldItalicFont=cmunbi.ttf]{cmuntt.ttf}\setmonofont[Path=/usr/share/fonts/truetype/cmu/,UprightFont=cmuntt.ttf,BoldFont=cmuntb.ttf,ItalicFont=cmunit.ttf,BoldItalicFont=cmuntx.ttf]{cmuntt.ttf}\ttfamily inputenc}{$\text{ }$}\setmainfont[Path=/usr/share/fonts/truetype/cmu/,UprightFont=cmunrm.ttf,BoldFont=cmunbx.ttf,ItalicFont=cmunti.ttf,BoldItalicFont=cmunbi.ttf]{cmunrm.ttf}\setmonofont[Path=/usr/share/fonts/truetype/cmu/,UprightFont=cmuntt.ttf,BoldFont=cmuntb.ttf,ItalicFont=cmunit.ttf,BoldItalicFont=cmuntx.ttf]{cmunrm.ttf} package.

Second, we need to know which language is prominently used in our document. {\ttfamily \setmainfont[Path=/usr/share/fonts/truetype/cmu/,UprightFont=cmunrm.ttf,BoldFont=cmunbx.ttf,ItalicFont=cmunti.ttf,BoldItalicFont=cmunbi.ttf]{cmuntt.ttf}\setmonofont[Path=/usr/share/fonts/truetype/cmu/,UprightFont=cmuntt.ttf,BoldFont=cmuntb.ttf,ItalicFont=cmunit.ttf,BoldItalicFont=cmuntx.ttf]{cmuntt.ttf}\ttfamily xindy}{$\text{ }$}\setmainfont[Path=/usr/share/fonts/truetype/cmu/,UprightFont=cmunrm.ttf,BoldFont=cmunbx.ttf,ItalicFont=cmunti.ttf,BoldItalicFont=cmunbi.ttf]{cmunrm.ttf}\setmonofont[Path=/usr/share/fonts/truetype/cmu/,UprightFont=cmuntt.ttf,BoldFont=cmuntb.ttf,ItalicFont=cmunit.ttf,BoldItalicFont=cmuntx.ttf]{cmunrm.ttf} can natively sort indices in Albanian, Belarusian, Bulgarian, Croatian, Czech, Danish, Dutch, English, Esperanto, Estonian, Finnish, French, Georgian, German, Greek, Gypsy, Hausa, Hebrew, Hungarian, Icelandic, Italian, Klingon, Kurdish, Latin, Latvian, Lithuanian, Macedonian, Mongolian, Norwegian, Polish, Portuguese, Romanian, Russian, Serbian Slovak, Slovenian, Sorbian, Spanish, Swedish, Turkish, Ukrainian and Vietnamese,

I don\textquotesingle{}t know if other languages have similar problems, but with Polish, if your {\ttfamily \setmainfont[Path=/usr/share/fonts/truetype/cmu/,UprightFont=cmunrm.ttf,BoldFont=cmunbx.ttf,ItalicFont=cmunti.ttf,BoldItalicFont=cmunbi.ttf]{cmuntt.ttf}\setmonofont[Path=/usr/share/fonts/truetype/cmu/,UprightFont=cmuntt.ttf,BoldFont=cmuntb.ttf,ItalicFont=cmunit.ttf,BoldItalicFont=cmuntx.ttf]{cmuntt.ttf}\ttfamily .tex}{$\text{ }$}\setmainfont[Path=/usr/share/fonts/truetype/cmu/,UprightFont=cmunrm.ttf,BoldFont=cmunbx.ttf,ItalicFont=cmunti.ttf,BoldItalicFont=cmunbi.ttf]{cmunrm.ttf}\setmonofont[Path=/usr/share/fonts/truetype/cmu/,UprightFont=cmuntt.ttf,BoldFont=cmuntb.ttf,ItalicFont=cmunit.ttf,BoldItalicFont=cmuntx.ttf]{cmunrm.ttf} is saved using UTF-{}8, the {\ttfamily \setmainfont[Path=/usr/share/fonts/truetype/cmu/,UprightFont=cmunrm.ttf,BoldFont=cmunbx.ttf,ItalicFont=cmunti.ttf,BoldItalicFont=cmunbi.ttf]{cmuntt.ttf}\setmonofont[Path=/usr/share/fonts/truetype/cmu/,UprightFont=cmuntt.ttf,BoldFont=cmuntb.ttf,ItalicFont=cmunit.ttf,BoldItalicFont=cmuntx.ttf]{cmuntt.ttf}\ttfamily .ind}{$\text{ }$}\setmainfont[Path=/usr/share/fonts/truetype/cmu/,UprightFont=cmunrm.ttf,BoldFont=cmunbx.ttf,ItalicFont=cmunti.ttf,BoldItalicFont=cmunbi.ttf]{cmunrm.ttf}\setmonofont[Path=/usr/share/fonts/truetype/cmu/,UprightFont=cmuntt.ttf,BoldFont=cmuntb.ttf,ItalicFont=cmunit.ttf,BoldItalicFont=cmuntx.ttf]{cmunrm.ttf} produced by texindy will be encoded in ISO-{}8859-{}2 if you use only {\ttfamily \setmainfont[Path=/usr/share/fonts/truetype/cmu/,UprightFont=cmunrm.ttf,BoldFont=cmunbx.ttf,ItalicFont=cmunti.ttf,BoldItalicFont=cmunbi.ttf]{cmuntt.ttf}\setmonofont[Path=/usr/share/fonts/truetype/cmu/,UprightFont=cmuntt.ttf,BoldFont=cmuntb.ttf,ItalicFont=cmunit.ttf,BoldItalicFont=cmuntx.ttf]{cmuntt.ttf}\ttfamily -{}L polish}\setmainfont[Path=/usr/share/fonts/truetype/cmu/,UprightFont=cmunrm.ttf,BoldFont=cmunbx.ttf,ItalicFont=cmunti.ttf,BoldItalicFont=cmunbi.ttf]{cmunrm.ttf}\setmonofont[Path=/usr/share/fonts/truetype/cmu/,UprightFont=cmuntt.ttf,BoldFont=cmuntb.ttf,ItalicFont=cmunit.ttf,BoldItalicFont=cmuntx.ttf]{cmunrm.ttf}.
While it\textquotesingle{}s not a problem for entries containing polish letters, as LaTeX internally encodes all letters to plain ASCII, it is for accented letters at beginning of words, they create new index entry groups, if you have, for example an \symbol{34}średnia\symbol{34} entry, you\textquotesingle{}ll get a \symbol{34}Ś\symbol{34} encoded in ISO-{}8859-{}2 {\ttfamily \setmainfont[Path=/usr/share/fonts/truetype/cmu/,UprightFont=cmunrm.ttf,BoldFont=cmunbx.ttf,ItalicFont=cmunti.ttf,BoldItalicFont=cmunbi.ttf]{cmuntt.ttf}\setmonofont[Path=/usr/share/fonts/truetype/cmu/,UprightFont=cmuntt.ttf,BoldFont=cmuntb.ttf,ItalicFont=cmunit.ttf,BoldItalicFont=cmuntx.ttf]{cmuntt.ttf}\ttfamily .ind}{$\text{ }$}\setmainfont[Path=/usr/share/fonts/truetype/cmu/,UprightFont=cmunrm.ttf,BoldFont=cmunbx.ttf,ItalicFont=cmunti.ttf,BoldItalicFont=cmunbi.ttf]{cmunrm.ttf}\setmonofont[Path=/usr/share/fonts/truetype/cmu/,UprightFont=cmuntt.ttf,BoldFont=cmuntb.ttf,ItalicFont=cmunit.ttf,BoldItalicFont=cmuntx.ttf]{cmunrm.ttf} file.
LaTeX doesn\textquotesingle{}t like if part of the file is in UTF-{}8 and part is in ISO-{}8859-{}2.
The obvious solution (adding {\ttfamily \setmainfont[Path=/usr/share/fonts/truetype/cmu/,UprightFont=cmunrm.ttf,BoldFont=cmunbx.ttf,ItalicFont=cmunti.ttf,BoldItalicFont=cmunbi.ttf]{cmuntt.ttf}\setmonofont[Path=/usr/share/fonts/truetype/cmu/,UprightFont=cmuntt.ttf,BoldFont=cmuntb.ttf,ItalicFont=cmunit.ttf,BoldItalicFont=cmuntx.ttf]{cmuntt.ttf}\ttfamily -{}C utf8}\setmainfont[Path=/usr/share/fonts/truetype/cmu/,UprightFont=cmunrm.ttf,BoldFont=cmunbx.ttf,ItalicFont=cmunti.ttf,BoldItalicFont=cmunbi.ttf]{cmunrm.ttf}\setmonofont[Path=/usr/share/fonts/truetype/cmu/,UprightFont=cmuntt.ttf,BoldFont=cmuntb.ttf,ItalicFont=cmunit.ttf,BoldItalicFont=cmuntx.ttf]{cmunrm.ttf}) doesn\textquotesingle{}t work, {\ttfamily \setmainfont[Path=/usr/share/fonts/truetype/cmu/,UprightFont=cmunrm.ttf,BoldFont=cmunbx.ttf,ItalicFont=cmunti.ttf,BoldItalicFont=cmunbi.ttf]{cmuntt.ttf}\setmonofont[Path=/usr/share/fonts/truetype/cmu/,UprightFont=cmuntt.ttf,BoldFont=cmuntb.ttf,ItalicFont=cmunit.ttf,BoldItalicFont=cmuntx.ttf]{cmuntt.ttf}\ttfamily texindy}{$\text{ }$}\setmainfont[Path=/usr/share/fonts/truetype/cmu/,UprightFont=cmunrm.ttf,BoldFont=cmunbx.ttf,ItalicFont=cmunti.ttf,BoldItalicFont=cmunbi.ttf]{cmunrm.ttf}\setmonofont[Path=/usr/share/fonts/truetype/cmu/,UprightFont=cmuntt.ttf,BoldFont=cmuntb.ttf,ItalicFont=cmunit.ttf,BoldItalicFont=cmuntx.ttf]{cmunrm.ttf} stops with\\

\TemplateSpaceIndent{$\text{ }${}ERROR:$\text{ }${}Could$\text{ }${}not$\text{ }${}find$\text{ }${}file$\text{ }${}\symbol{34}tex/inputenc/utf8.xdy\symbol{34}}

error.
The fix this, you have to load the definiton style for the headings using {\ttfamily \setmainfont[Path=/usr/share/fonts/truetype/cmu/,UprightFont=cmunrm.ttf,BoldFont=cmunbx.ttf,ItalicFont=cmunti.ttf,BoldItalicFont=cmunbi.ttf]{cmuntt.ttf}\setmonofont[Path=/usr/share/fonts/truetype/cmu/,UprightFont=cmuntt.ttf,BoldFont=cmuntb.ttf,ItalicFont=cmunit.ttf,BoldItalicFont=cmuntx.ttf]{cmuntt.ttf}\ttfamily -{}M switch}\setmainfont[Path=/usr/share/fonts/truetype/cmu/,UprightFont=cmunrm.ttf,BoldFont=cmunbx.ttf,ItalicFont=cmunti.ttf,BoldItalicFont=cmunbi.ttf]{cmunrm.ttf}\setmonofont[Path=/usr/share/fonts/truetype/cmu/,UprightFont=cmuntt.ttf,BoldFont=cmuntb.ttf,ItalicFont=cmunit.ttf,BoldItalicFont=cmuntx.ttf]{cmunrm.ttf}:\\

\TemplateSpaceIndent{$\text{ }${}-{}M$\text{ }${}lang/polish/utf8}


In the end we have to run such command:\\

\TemplateSpaceIndent{$\text{ }${}texindy$\text{ }${}-{}L$\text{ }${}polish$\text{ }${}-{}M$\text{ }${}lang/polish/utf8$\text{ }${}{\itshape \setmainfont[Path=/usr/share/fonts/truetype/cmu/,UprightFont=cmunrm.ttf,BoldFont=cmunbx.ttf,ItalicFont=cmunti.ttf,BoldItalicFont=cmunbi.ttf]{cmunti.ttf}\setmonofont[Path=/usr/share/fonts/truetype/cmu/,UprightFont=cmuntt.ttf,BoldFont=cmuntb.ttf,ItalicFont=cmunit.ttf,BoldItalicFont=cmuntx.ttf]{cmunti.ttf}\itshape filename.idx}}
\setmainfont[Path=/usr/share/fonts/truetype/cmu/,UprightFont=cmunrm.ttf,BoldFont=cmunbx.ttf,ItalicFont=cmunti.ttf,BoldItalicFont=cmunbi.ttf]{cmunrm.ttf}\setmonofont[Path=/usr/share/fonts/truetype/cmu/,UprightFont=cmuntt.ttf,BoldFont=cmuntb.ttf,ItalicFont=cmunit.ttf,BoldItalicFont=cmuntx.ttf]{cmunrm.ttf}

Additional way to fix this problem is use \symbol{34}iconv\symbol{34} to create utf8.xdy from latin2.xdy \\

\TemplateSpaceIndent{$\text{ }${}{\ttfamily {$\text{ }$}\setmainfont[Path=/usr/share/fonts/truetype/cmu/,UprightFont=cmunrm.ttf,BoldFont=cmunbx.ttf,ItalicFont=cmunti.ttf,BoldItalicFont=cmunbi.ttf]{cmuntt.ttf}\setmonofont[Path=/usr/share/fonts/truetype/cmu/,UprightFont=cmuntt.ttf,BoldFont=cmuntb.ttf,ItalicFont=cmunit.ttf,BoldItalicFont=cmuntx.ttf]{cmuntt.ttf}\ttfamily  iconv -{}f latin2 -{}t utf8 latin2.xdy >{}utf8.xdy}}
\setmainfont[Path=/usr/share/fonts/truetype/cmu/,UprightFont=cmunrm.ttf,BoldFont=cmunbx.ttf,ItalicFont=cmunti.ttf,BoldItalicFont=cmunbi.ttf]{cmunrm.ttf}\setmonofont[Path=/usr/share/fonts/truetype/cmu/,UprightFont=cmuntt.ttf,BoldFont=cmuntb.ttf,ItalicFont=cmunit.ttf,BoldItalicFont=cmuntx.ttf]{cmunrm.ttf}
in folder \\

\TemplateSpaceIndent{$\text{ }${}{\ttfamily {$\text{ }$}\setmainfont[Path=/usr/share/fonts/truetype/cmu/,UprightFont=cmunrm.ttf,BoldFont=cmunbx.ttf,ItalicFont=cmunti.ttf,BoldItalicFont=cmunbi.ttf]{cmuntt.ttf}\setmonofont[Path=/usr/share/fonts/truetype/cmu/,UprightFont=cmuntt.ttf,BoldFont=cmuntb.ttf,ItalicFont=cmunit.ttf,BoldItalicFont=cmuntx.ttf]{cmuntt.ttf}\ttfamily   /usr/share/xindy/tex/inputenc}}
\setmainfont[Path=/usr/share/fonts/truetype/cmu/,UprightFont=cmunrm.ttf,BoldFont=cmunbx.ttf,ItalicFont=cmunti.ttf,BoldItalicFont=cmunbi.ttf]{cmunrm.ttf}\setmonofont[Path=/usr/share/fonts/truetype/cmu/,UprightFont=cmuntt.ttf,BoldFont=cmuntb.ttf,ItalicFont=cmunit.ttf,BoldItalicFont=cmuntx.ttf]{cmunrm.ttf}
(You must have root privileges)

\subsubsection{xindy in kile}
\label{641}
To use {\ttfamily \setmainfont[Path=/usr/share/fonts/truetype/cmu/,UprightFont=cmunrm.ttf,BoldFont=cmunbx.ttf,ItalicFont=cmunti.ttf,BoldItalicFont=cmunbi.ttf]{cmuntt.ttf}\setmonofont[Path=/usr/share/fonts/truetype/cmu/,UprightFont=cmuntt.ttf,BoldFont=cmuntb.ttf,ItalicFont=cmunit.ttf,BoldItalicFont=cmuntx.ttf]{cmuntt.ttf}\ttfamily texindy}{$\text{ }$}\setmainfont[Path=/usr/share/fonts/truetype/cmu/,UprightFont=cmunrm.ttf,BoldFont=cmunbx.ttf,ItalicFont=cmunti.ttf,BoldItalicFont=cmunbi.ttf]{cmunrm.ttf}\setmonofont[Path=/usr/share/fonts/truetype/cmu/,UprightFont=cmuntt.ttf,BoldFont=cmuntb.ttf,ItalicFont=cmunit.ttf,BoldItalicFont=cmuntx.ttf]{cmunrm.ttf} instead of {\ttfamily \setmainfont[Path=/usr/share/fonts/truetype/cmu/,UprightFont=cmunrm.ttf,BoldFont=cmunbx.ttf,ItalicFont=cmunti.ttf,BoldItalicFont=cmunbi.ttf]{cmuntt.ttf}\setmonofont[Path=/usr/share/fonts/truetype/cmu/,UprightFont=cmuntt.ttf,BoldFont=cmuntb.ttf,ItalicFont=cmunit.ttf,BoldItalicFont=cmuntx.ttf]{cmuntt.ttf}\ttfamily makeindex}{$\text{ }$}\setmainfont[Path=/usr/share/fonts/truetype/cmu/,UprightFont=cmunrm.ttf,BoldFont=cmunbx.ttf,ItalicFont=cmunti.ttf,BoldItalicFont=cmunbi.ttf]{cmunrm.ttf}\setmonofont[Path=/usr/share/fonts/truetype/cmu/,UprightFont=cmuntt.ttf,BoldFont=cmuntb.ttf,ItalicFont=cmunit.ttf,BoldItalicFont=cmuntx.ttf]{cmunrm.ttf} in kile, you have to either redefine the MakeIndex tool in Settings → Configure Kile... → Tools → Build, or define new tool and redefine other tools to use it (for example by adding it to QuickBuild).

The {\ttfamily \setmainfont[Path=/usr/share/fonts/truetype/cmu/,UprightFont=cmunrm.ttf,BoldFont=cmunbx.ttf,ItalicFont=cmunti.ttf,BoldItalicFont=cmunbi.ttf]{cmuntt.ttf}\setmonofont[Path=/usr/share/fonts/truetype/cmu/,UprightFont=cmuntt.ttf,BoldFont=cmuntb.ttf,ItalicFont=cmunit.ttf,BoldItalicFont=cmuntx.ttf]{cmuntt.ttf}\ttfamily xindy}{$\text{ }$}\setmainfont[Path=/usr/share/fonts/truetype/cmu/,UprightFont=cmunrm.ttf,BoldFont=cmunbx.ttf,ItalicFont=cmunti.ttf,BoldItalicFont=cmunbi.ttf]{cmunrm.ttf}\setmonofont[Path=/usr/share/fonts/truetype/cmu/,UprightFont=cmuntt.ttf,BoldFont=cmuntb.ttf,ItalicFont=cmunit.ttf,BoldItalicFont=cmuntx.ttf]{cmunrm.ttf} definition should look similar to this:\\

\TemplateSpaceIndent{$\text{ }${}General:$\text{ }$\newline{}
$\text{ }${}$\text{ }${}Command:$\text{ }${}texindy$\text{ }$\newline{}
$\text{ }${}$\text{ }${}Options:$\text{ }${}-{}L$\text{ }${}polish$\text{ }${}-{}M$\text{ }${}lang/polish/utf8$\text{ }${}-{}I$\text{ }${}latex$\text{ }${}\textquotesingle{}\%S.idx\textquotesingle{}$\text{ }$\newline{}
$\text{ }${}Advanced:$\text{ }$\newline{}
$\text{ }${}$\text{ }${}Type:$\text{ }${}Run$\text{ }${}Outside$\text{ }${}of$\text{ }${}Kile$\text{ }$\newline{}
$\text{ }${}$\text{ }${}Class:$\text{ }${}Compile$\text{ }$\newline{}
$\text{ }${}$\text{ }${}Source$\text{ }${}extension:$\text{ }${}idx$\text{ }$\newline{}
$\text{ }${}$\text{ }${}Target$\text{ }${}extension:$\text{ }${}ind$\text{ }$\newline{}
$\text{ }${}$\text{ }${}Target$\text{ }${}file:$\text{ }${}<{}empty>{}$\text{ }$\newline{}
$\text{ }${}$\text{ }${}Relative$\text{ }${}dir:$\text{ }${}<{}empty>{}$\text{ }$\newline{}
$\text{ }${}$\text{ }${}State:$\text{ }${}Editor$\text{ }$\newline{}
$\text{ }${}Menu:$\text{ }$\newline{}
$\text{ }${}$\text{ }${}Add$\text{ }${}tool$\text{ }${}to$\text{ }${}Build$\text{ }${}menu:$\text{ }${}Compile$\text{ }$\newline{}
$\text{ }${}$\text{ }${}Icon:$\text{ }${}the$\text{ }${}one$\text{ }${}you$\text{ }${}like}





\myhref{https://sr.wikibooks.org/wiki/LaTeX\%2F\%D0\%98\%D0\%BD\%D0\%B4\%D0\%B5\%D0\%BA\%D1\%81\%D0\%B8\%D1\%80\%D0\%B0\%D1\%9A\%D0\%B5}{sr:LaTeX/Индексирање}\chapter{Glossary}

\myminitoc
\label{642}

\label{643}

Many technical documents use terms or acronyms unknown to the general population. It is common practice to add a glossary to make such documents more accessible.

The \LaTeXTT{glossaries} package can be used to create glossaries. It supports multiple glossaries, acronyms, and symbols. This package replaces the \LaTeXTT{glossary} package and can be used instead of the \LaTeXTT{nomencl} package.\myfootnote{\myplainurl{http://www.ctan.org/pkg/nomencl}} Users requiring a simpler solution should consider hand-{}coding their entries by using the {\ttfamily \mylref{191}{\setmainfont[Path=/usr/share/fonts/truetype/cmu/,UprightFont=cmunrm.ttf,BoldFont=cmunbx.ttf,ItalicFont=cmunti.ttf,BoldItalicFont=cmunbi.ttf]{cmuntt.ttf}\setmonofont[Path=/usr/share/fonts/truetype/cmu/,UprightFont=cmuntt.ttf,BoldFont=cmuntb.ttf,ItalicFont=cmunit.ttf,BoldItalicFont=cmuntx.ttf]{cmuntt.ttf}\ttfamily description}} environment, or the {\ttfamily \mylref{277}{\setmainfont[Path=/usr/share/fonts/truetype/cmu/,UprightFont=cmunrm.ttf,BoldFont=cmunbx.ttf,ItalicFont=cmunti.ttf,BoldItalicFont=cmunbi.ttf]{cmuntt.ttf}\setmonofont[Path=/usr/share/fonts/truetype/cmu/,UprightFont=cmuntt.ttf,BoldFont=cmuntb.ttf,ItalicFont=cmunit.ttf,BoldItalicFont=cmuntx.ttf]{cmuntt.ttf}\ttfamily longtabu}} environment provided by the \LaTeXTT{tabu} package.


\section{Jump start}
\label{644}

Place \LaTeXTT{\textbackslash{}usepackage\{glossaries\}} and \LaTeXTT{\textbackslash{}makeglossaries} in your preamble (after \LaTeXTT{\textbackslash{}usepackage\{hyperref\}} if present).
Then define any number of \LaTeXTT{\textbackslash{}newglossaryentry} and \LaTeXTT{\textbackslash{}newacronym} glossary and acronym entries in your preamble (recommended) or before first use in your document proper.
Finally add a \LaTeXTT{\textbackslash{}printglossaries} call to locate the glossaries list within your document structure.
Then pepper your writing with \LaTeXTT{\textbackslash{}gls\{mylabel\}} macros (and similar) to simultaneously insert your predefined text and build the associated glossary.
File processing must now include a call to {\ttfamily \setmainfont[Path=/usr/share/fonts/truetype/cmu/,UprightFont=cmunrm.ttf,BoldFont=cmunbx.ttf,ItalicFont=cmunti.ttf,BoldItalicFont=cmunbi.ttf]{cmuntt.ttf}\setmonofont[Path=/usr/share/fonts/truetype/cmu/,UprightFont=cmuntt.ttf,BoldFont=cmuntb.ttf,ItalicFont=cmunit.ttf,BoldItalicFont=cmuntx.ttf]{cmuntt.ttf}\ttfamily makeglossaries}{$\text{ }$}\setmainfont[Path=/usr/share/fonts/truetype/cmu/,UprightFont=cmunrm.ttf,BoldFont=cmunbx.ttf,ItalicFont=cmunti.ttf,BoldItalicFont=cmunbi.ttf]{cmunrm.ttf}\setmonofont[Path=/usr/share/fonts/truetype/cmu/,UprightFont=cmuntt.ttf,BoldFont=cmuntb.ttf,ItalicFont=cmunit.ttf,BoldItalicFont=cmuntx.ttf]{cmunrm.ttf} followed by at least one further invocation of {\ttfamily \setmainfont[Path=/usr/share/fonts/truetype/cmu/,UprightFont=cmunrm.ttf,BoldFont=cmunbx.ttf,ItalicFont=cmunti.ttf,BoldItalicFont=cmunbi.ttf]{cmuntt.ttf}\setmonofont[Path=/usr/share/fonts/truetype/cmu/,UprightFont=cmuntt.ttf,BoldFont=cmuntb.ttf,ItalicFont=cmunit.ttf,BoldItalicFont=cmuntx.ttf]{cmuntt.ttf}\ttfamily latex}{$\text{ }$}\setmainfont[Path=/usr/share/fonts/truetype/cmu/,UprightFont=cmunrm.ttf,BoldFont=cmunbx.ttf,ItalicFont=cmunti.ttf,BoldItalicFont=cmunbi.ttf]{cmunrm.ttf}\setmonofont[Path=/usr/share/fonts/truetype/cmu/,UprightFont=cmuntt.ttf,BoldFont=cmuntb.ttf,ItalicFont=cmunit.ttf,BoldItalicFont=cmuntx.ttf]{cmunrm.ttf} or {\ttfamily \setmainfont[Path=/usr/share/fonts/truetype/cmu/,UprightFont=cmunrm.ttf,BoldFont=cmunbx.ttf,ItalicFont=cmunti.ttf,BoldItalicFont=cmunbi.ttf]{cmuntt.ttf}\setmonofont[Path=/usr/share/fonts/truetype/cmu/,UprightFont=cmuntt.ttf,BoldFont=cmuntb.ttf,ItalicFont=cmunit.ttf,BoldItalicFont=cmuntx.ttf]{cmuntt.ttf}\ttfamily pdflatex}\setmainfont[Path=/usr/share/fonts/truetype/cmu/,UprightFont=cmunrm.ttf,BoldFont=cmunbx.ttf,ItalicFont=cmunti.ttf,BoldItalicFont=cmunbi.ttf]{cmunrm.ttf}\setmonofont[Path=/usr/share/fonts/truetype/cmu/,UprightFont=cmuntt.ttf,BoldFont=cmuntb.ttf,ItalicFont=cmunit.ttf,BoldItalicFont=cmuntx.ttf]{cmunrm.ttf}.
\section{Using \LaTeXTT{glossaries}}
\label{645}
To use the \LaTeXTT{glossaries} package, you have to load it explicitly:
\begin{Shaded}
\begin{Highlighting}[]

\NormalTok{\textbackslash{}usepackage\{glossaries\}}
\end{Highlighting}
\end{Shaded}

if you wish to use {\ttfamily \myhref{https://en.wikipedia.org/wiki/xindy}{\setmainfont[Path=/usr/share/fonts/truetype/cmu/,UprightFont=cmunrm.ttf,BoldFont=cmunbx.ttf,ItalicFont=cmunti.ttf,BoldItalicFont=cmunbi.ttf]{cmuntt.ttf}\setmonofont[Path=/usr/share/fonts/truetype/cmu/,UprightFont=cmuntt.ttf,BoldFont=cmuntb.ttf,ItalicFont=cmunit.ttf,BoldItalicFont=cmuntx.ttf]{cmuntt.ttf}\ttfamily xindy}} (recommended) for the indexing phase, as opposed to {\ttfamily \myhref{https://en.wikipedia.org/wiki/MakeIndex}{\setmainfont[Path=/usr/share/fonts/truetype/cmu/,UprightFont=cmunrm.ttf,BoldFont=cmunbx.ttf,ItalicFont=cmunti.ttf,BoldItalicFont=cmunbi.ttf]{cmuntt.ttf}\setmonofont[Path=/usr/share/fonts/truetype/cmu/,UprightFont=cmuntt.ttf,BoldFont=cmuntb.ttf,ItalicFont=cmunit.ttf,BoldItalicFont=cmuntx.ttf]{cmuntt.ttf}\ttfamily makeindex}} (the default), you need to specify the \LaTeXTT{xindy} option:
\begin{Shaded}
\begin{Highlighting}[]

\NormalTok{\textbackslash{}usepackage[xindy]\{glossaries\}}
\end{Highlighting}
\end{Shaded}

For the glossary to show up in your Table of Contents, you need to specify the \LaTeXTT{toc} option:
\begin{Shaded}
\begin{Highlighting}[]

\NormalTok{\textbackslash{}usepackage[toc]\{glossaries\}}
\end{Highlighting}
\end{Shaded}

See also \mylref{656}{Custom Name} at the bottom of this page.

Finally, place the following command in your document preamble in order to generate the glossary:
\begin{Shaded}
\begin{Highlighting}[]

\NormalTok{\textbackslash{}makeglossaries}
\end{Highlighting}
\end{Shaded}


Any links in resulting glossary will not be \symbol{34}clickable\symbol{34} unless you load the \LaTeXTT{glossaries} package {\itshape \setmainfont[Path=/usr/share/fonts/truetype/cmu/,UprightFont=cmunrm.ttf,BoldFont=cmunbx.ttf,ItalicFont=cmunti.ttf,BoldItalicFont=cmunbi.ttf]{cmunti.ttf}\setmonofont[Path=/usr/share/fonts/truetype/cmu/,UprightFont=cmuntt.ttf,BoldFont=cmuntb.ttf,ItalicFont=cmunit.ttf,BoldItalicFont=cmuntx.ttf]{cmunti.ttf}\itshape after}{$\text{ }$}\setmainfont[Path=/usr/share/fonts/truetype/cmu/,UprightFont=cmunrm.ttf,BoldFont=cmunbx.ttf,ItalicFont=cmunti.ttf,BoldItalicFont=cmunbi.ttf]{cmunrm.ttf}\setmonofont[Path=/usr/share/fonts/truetype/cmu/,UprightFont=cmuntt.ttf,BoldFont=cmuntb.ttf,ItalicFont=cmunit.ttf,BoldItalicFont=cmuntx.ttf]{cmunrm.ttf} the \LaTeXTT{hyperref} package.

In addition, users who wish to make use of {\ttfamily \setmainfont[Path=/usr/share/fonts/truetype/cmu/,UprightFont=cmunrm.ttf,BoldFont=cmunbx.ttf,ItalicFont=cmunti.ttf,BoldItalicFont=cmunbi.ttf]{cmuntt.ttf}\setmonofont[Path=/usr/share/fonts/truetype/cmu/,UprightFont=cmuntt.ttf,BoldFont=cmuntb.ttf,ItalicFont=cmunit.ttf,BoldItalicFont=cmuntx.ttf]{cmuntt.ttf}\ttfamily makeglossaries}{$\text{ }$}\setmainfont[Path=/usr/share/fonts/truetype/cmu/,UprightFont=cmunrm.ttf,BoldFont=cmunbx.ttf,ItalicFont=cmunti.ttf,BoldItalicFont=cmunbi.ttf]{cmunrm.ttf}\setmonofont[Path=/usr/share/fonts/truetype/cmu/,UprightFont=cmuntt.ttf,BoldFont=cmuntb.ttf,ItalicFont=cmunit.ttf,BoldItalicFont=cmuntx.ttf]{cmunrm.ttf} will need to have \myhref{https://en.wikibooks.org/wiki/Perl}{Perl} installed {\mbox{$\text{---}$}} this is not normally present by default on Microsoft Windows platforms.
That said, {\ttfamily \setmainfont[Path=/usr/share/fonts/truetype/cmu/,UprightFont=cmunrm.ttf,BoldFont=cmunbx.ttf,ItalicFont=cmunti.ttf,BoldItalicFont=cmunbi.ttf]{cmuntt.ttf}\setmonofont[Path=/usr/share/fonts/truetype/cmu/,UprightFont=cmuntt.ttf,BoldFont=cmuntb.ttf,ItalicFont=cmunit.ttf,BoldItalicFont=cmuntx.ttf]{cmuntt.ttf}\ttfamily makeglossaries}{$\text{ }$}\setmainfont[Path=/usr/share/fonts/truetype/cmu/,UprightFont=cmunrm.ttf,BoldFont=cmunbx.ttf,ItalicFont=cmunti.ttf,BoldItalicFont=cmunbi.ttf]{cmunrm.ttf}\setmonofont[Path=/usr/share/fonts/truetype/cmu/,UprightFont=cmuntt.ttf,BoldFont=cmuntb.ttf,ItalicFont=cmunit.ttf,BoldItalicFont=cmuntx.ttf]{cmunrm.ttf} simply provides a convenient interface to {\ttfamily \setmainfont[Path=/usr/share/fonts/truetype/cmu/,UprightFont=cmunrm.ttf,BoldFont=cmunbx.ttf,ItalicFont=cmunti.ttf,BoldItalicFont=cmunbi.ttf]{cmuntt.ttf}\setmonofont[Path=/usr/share/fonts/truetype/cmu/,UprightFont=cmuntt.ttf,BoldFont=cmuntb.ttf,ItalicFont=cmunit.ttf,BoldItalicFont=cmuntx.ttf]{cmuntt.ttf}\ttfamily makeindex}{$\text{ }$}\setmainfont[Path=/usr/share/fonts/truetype/cmu/,UprightFont=cmunrm.ttf,BoldFont=cmunbx.ttf,ItalicFont=cmunti.ttf,BoldItalicFont=cmunbi.ttf]{cmunrm.ttf}\setmonofont[Path=/usr/share/fonts/truetype/cmu/,UprightFont=cmuntt.ttf,BoldFont=cmuntb.ttf,ItalicFont=cmunit.ttf,BoldItalicFont=cmuntx.ttf]{cmunrm.ttf} and {\ttfamily \setmainfont[Path=/usr/share/fonts/truetype/cmu/,UprightFont=cmunrm.ttf,BoldFont=cmunbx.ttf,ItalicFont=cmunti.ttf,BoldItalicFont=cmunbi.ttf]{cmuntt.ttf}\setmonofont[Path=/usr/share/fonts/truetype/cmu/,UprightFont=cmuntt.ttf,BoldFont=cmuntb.ttf,ItalicFont=cmunit.ttf,BoldItalicFont=cmuntx.ttf]{cmuntt.ttf}\ttfamily xindy}{$\text{ }$}\setmainfont[Path=/usr/share/fonts/truetype/cmu/,UprightFont=cmunrm.ttf,BoldFont=cmunbx.ttf,ItalicFont=cmunti.ttf,BoldItalicFont=cmunbi.ttf]{cmunrm.ttf}\setmonofont[Path=/usr/share/fonts/truetype/cmu/,UprightFont=cmuntt.ttf,BoldFont=cmuntb.ttf,ItalicFont=cmunit.ttf,BoldItalicFont=cmuntx.ttf]{cmunrm.ttf} and is not essential.
\section{Defining glossary entries}
\label{646}
To use an entry from a glossary you first need to define it.
There are few ways to define an entry depending on what you define and how it is going to be used.

Note that a defined entry {\itshape \setmainfont[Path=/usr/share/fonts/truetype/cmu/,UprightFont=cmunrm.ttf,BoldFont=cmunbx.ttf,ItalicFont=cmunti.ttf,BoldItalicFont=cmunbi.ttf]{cmunti.ttf}\setmonofont[Path=/usr/share/fonts/truetype/cmu/,UprightFont=cmuntt.ttf,BoldFont=cmuntb.ttf,ItalicFont=cmunit.ttf,BoldItalicFont=cmuntx.ttf]{cmunti.ttf}\itshape won\textquotesingle{}t}{$\text{ }$}\setmainfont[Path=/usr/share/fonts/truetype/cmu/,UprightFont=cmunrm.ttf,BoldFont=cmunbx.ttf,ItalicFont=cmunti.ttf,BoldItalicFont=cmunbi.ttf]{cmunrm.ttf}\setmonofont[Path=/usr/share/fonts/truetype/cmu/,UprightFont=cmuntt.ttf,BoldFont=cmuntb.ttf,ItalicFont=cmunit.ttf,BoldItalicFont=cmuntx.ttf]{cmunrm.ttf} be included in the printed glossary {\itshape \setmainfont[Path=/usr/share/fonts/truetype/cmu/,UprightFont=cmunrm.ttf,BoldFont=cmunbx.ttf,ItalicFont=cmunti.ttf,BoldItalicFont=cmunbi.ttf]{cmunti.ttf}\setmonofont[Path=/usr/share/fonts/truetype/cmu/,UprightFont=cmuntt.ttf,BoldFont=cmuntb.ttf,ItalicFont=cmunit.ttf,BoldItalicFont=cmuntx.ttf]{cmunti.ttf}\itshape unless}{$\text{ }$}\setmainfont[Path=/usr/share/fonts/truetype/cmu/,UprightFont=cmunrm.ttf,BoldFont=cmunbx.ttf,ItalicFont=cmunti.ttf,BoldItalicFont=cmunbi.ttf]{cmunrm.ttf}\setmonofont[Path=/usr/share/fonts/truetype/cmu/,UprightFont=cmuntt.ttf,BoldFont=cmuntb.ttf,ItalicFont=cmunit.ttf,BoldItalicFont=cmuntx.ttf]{cmunrm.ttf} it is used in the document.
This enables you to create a glossary of general terms and just \LaTeXTT{\textbackslash{}include} it in all your documents.
\section{Defining terms}
\label{647}
To define a term in glossary you use the \LaTeXTT{\textbackslash{}newglossaryentry} macro:
\begin{Shaded}
\begin{Highlighting}[]

\NormalTok{\textbackslash{}newglossaryentry\{<label>\}\{<settings>\}}
\end{Highlighting}
\end{Shaded}

 is a unique label used to identify an entry in glossary, <{}settings>{} are comma separated {\ttfamily \setmainfont[Path=/usr/share/fonts/truetype/cmu/,UprightFont=cmunrm.ttf,BoldFont=cmunbx.ttf,ItalicFont=cmunti.ttf,BoldItalicFont=cmunbi.ttf]{cmuntt.ttf}\setmonofont[Path=/usr/share/fonts/truetype/cmu/,UprightFont=cmuntt.ttf,BoldFont=cmuntb.ttf,ItalicFont=cmunit.ttf,BoldItalicFont=cmuntx.ttf]{cmuntt.ttf}\ttfamily key=value}{$\text{ }$}\setmainfont[Path=/usr/share/fonts/truetype/cmu/,UprightFont=cmunrm.ttf,BoldFont=cmunbx.ttf,ItalicFont=cmunti.ttf,BoldItalicFont=cmunbi.ttf]{cmunrm.ttf}\setmonofont[Path=/usr/share/fonts/truetype/cmu/,UprightFont=cmuntt.ttf,BoldFont=cmuntb.ttf,ItalicFont=cmunit.ttf,BoldItalicFont=cmuntx.ttf]{cmunrm.ttf} pairs used to define an entry.

For example, to define a computer entry:
\begin{Shaded}
\begin{Highlighting}[]

\NormalTok{\textbackslash{}newglossaryentry\{computer\}}
\NormalTok{\{}
  \NormalTok{name=computer,}
  \NormalTok{description=\{is a programmable machine that receives input,}
               \NormalTok{stores and manipulates data, and provides}
               \NormalTok{output in a useful format\}}
\NormalTok{\}}
\end{Highlighting}
\end{Shaded}

The above example defines an entry that has the same label and entry name.
This is not always the case as the next entry will show:
\begin{Shaded}
\begin{Highlighting}[]

\NormalTok{\textbackslash{}newglossaryentry\{naiive\}}
\NormalTok{\{}
  \NormalTok{name=na\textbackslash{}"\{\textbackslash{}i\}ve,}
  \NormalTok{description=\{is a French loanword (adjective, form of naïf)}
               \NormalTok{indicating having or showing a lack of experience,}
               \NormalTok{understanding or sophistication\}}
\NormalTok{\}}
\end{Highlighting}
\end{Shaded}


When you define terms, you need to remember that they will be sorted by {\ttfamily \setmainfont[Path=/usr/share/fonts/truetype/cmu/,UprightFont=cmunrm.ttf,BoldFont=cmunbx.ttf,ItalicFont=cmunti.ttf,BoldItalicFont=cmunbi.ttf]{cmuntt.ttf}\setmonofont[Path=/usr/share/fonts/truetype/cmu/,UprightFont=cmuntt.ttf,BoldFont=cmuntb.ttf,ItalicFont=cmunit.ttf,BoldItalicFont=cmuntx.ttf]{cmuntt.ttf}\ttfamily makeindex}{$\text{ }$}\setmainfont[Path=/usr/share/fonts/truetype/cmu/,UprightFont=cmunrm.ttf,BoldFont=cmunbx.ttf,ItalicFont=cmunti.ttf,BoldItalicFont=cmunbi.ttf]{cmunrm.ttf}\setmonofont[Path=/usr/share/fonts/truetype/cmu/,UprightFont=cmuntt.ttf,BoldFont=cmuntb.ttf,ItalicFont=cmunit.ttf,BoldItalicFont=cmuntx.ttf]{cmunrm.ttf} or {\ttfamily \setmainfont[Path=/usr/share/fonts/truetype/cmu/,UprightFont=cmunrm.ttf,BoldFont=cmunbx.ttf,ItalicFont=cmunti.ttf,BoldItalicFont=cmunbi.ttf]{cmuntt.ttf}\setmonofont[Path=/usr/share/fonts/truetype/cmu/,UprightFont=cmuntt.ttf,BoldFont=cmuntb.ttf,ItalicFont=cmunit.ttf,BoldItalicFont=cmuntx.ttf]{cmuntt.ttf}\ttfamily xindy}\setmainfont[Path=/usr/share/fonts/truetype/cmu/,UprightFont=cmunrm.ttf,BoldFont=cmunbx.ttf,ItalicFont=cmunti.ttf,BoldItalicFont=cmunbi.ttf]{cmunrm.ttf}\setmonofont[Path=/usr/share/fonts/truetype/cmu/,UprightFont=cmuntt.ttf,BoldFont=cmuntb.ttf,ItalicFont=cmunit.ttf,BoldItalicFont=cmuntx.ttf]{cmunrm.ttf}.
While {\ttfamily \setmainfont[Path=/usr/share/fonts/truetype/cmu/,UprightFont=cmunrm.ttf,BoldFont=cmunbx.ttf,ItalicFont=cmunti.ttf,BoldItalicFont=cmunbi.ttf]{cmuntt.ttf}\setmonofont[Path=/usr/share/fonts/truetype/cmu/,UprightFont=cmuntt.ttf,BoldFont=cmuntb.ttf,ItalicFont=cmunit.ttf,BoldItalicFont=cmuntx.ttf]{cmuntt.ttf}\ttfamily xindy}{$\text{ }$}\setmainfont[Path=/usr/share/fonts/truetype/cmu/,UprightFont=cmunrm.ttf,BoldFont=cmunbx.ttf,ItalicFont=cmunti.ttf,BoldItalicFont=cmunbi.ttf]{cmunrm.ttf}\setmonofont[Path=/usr/share/fonts/truetype/cmu/,UprightFont=cmuntt.ttf,BoldFont=cmuntb.ttf,ItalicFont=cmunit.ttf,BoldItalicFont=cmuntx.ttf]{cmunrm.ttf} is a bit more LaTeX aware, it does it by omitting latex macros (\LaTeXTT{\textbackslash{}\symbol{34}\{\textbackslash{}i\}}) thus incorrectly sorting the above example as {\ttfamily \setmainfont[Path=/usr/share/fonts/truetype/cmu/,UprightFont=cmunrm.ttf,BoldFont=cmunbx.ttf,ItalicFont=cmunti.ttf,BoldItalicFont=cmunbi.ttf]{cmuntt.ttf}\setmonofont[Path=/usr/share/fonts/truetype/cmu/,UprightFont=cmuntt.ttf,BoldFont=cmuntb.ttf,ItalicFont=cmunit.ttf,BoldItalicFont=cmuntx.ttf]{cmuntt.ttf}\ttfamily nave}\setmainfont[Path=/usr/share/fonts/truetype/cmu/,UprightFont=cmunrm.ttf,BoldFont=cmunbx.ttf,ItalicFont=cmunti.ttf,BoldItalicFont=cmunbi.ttf]{cmunrm.ttf}\setmonofont[Path=/usr/share/fonts/truetype/cmu/,UprightFont=cmuntt.ttf,BoldFont=cmuntb.ttf,ItalicFont=cmunit.ttf,BoldItalicFont=cmuntx.ttf]{cmunrm.ttf}.
{\ttfamily \setmainfont[Path=/usr/share/fonts/truetype/cmu/,UprightFont=cmunrm.ttf,BoldFont=cmunbx.ttf,ItalicFont=cmunti.ttf,BoldItalicFont=cmunbi.ttf]{cmuntt.ttf}\setmonofont[Path=/usr/share/fonts/truetype/cmu/,UprightFont=cmuntt.ttf,BoldFont=cmuntb.ttf,ItalicFont=cmunit.ttf,BoldItalicFont=cmuntx.ttf]{cmuntt.ttf}\ttfamily makeindex}{$\text{ }$}\setmainfont[Path=/usr/share/fonts/truetype/cmu/,UprightFont=cmunrm.ttf,BoldFont=cmunbx.ttf,ItalicFont=cmunti.ttf,BoldItalicFont=cmunbi.ttf]{cmunrm.ttf}\setmonofont[Path=/usr/share/fonts/truetype/cmu/,UprightFont=cmuntt.ttf,BoldFont=cmuntb.ttf,ItalicFont=cmunit.ttf,BoldItalicFont=cmuntx.ttf]{cmunrm.ttf} won\textquotesingle{}t fare much better, because it doesn\textquotesingle{}t understand TeX macros, it will interpret the word exactly as it was defined, putting it inside symbol class, before words beginning with {\ttfamily \setmainfont[Path=/usr/share/fonts/truetype/cmu/,UprightFont=cmunrm.ttf,BoldFont=cmunbx.ttf,ItalicFont=cmunti.ttf,BoldItalicFont=cmunbi.ttf]{cmuntt.ttf}\setmonofont[Path=/usr/share/fonts/truetype/cmu/,UprightFont=cmuntt.ttf,BoldFont=cmuntb.ttf,ItalicFont=cmunit.ttf,BoldItalicFont=cmuntx.ttf]{cmuntt.ttf}\ttfamily naa}\setmainfont[Path=/usr/share/fonts/truetype/cmu/,UprightFont=cmunrm.ttf,BoldFont=cmunbx.ttf,ItalicFont=cmunti.ttf,BoldItalicFont=cmunbi.ttf]{cmunrm.ttf}\setmonofont[Path=/usr/share/fonts/truetype/cmu/,UprightFont=cmuntt.ttf,BoldFont=cmuntb.ttf,ItalicFont=cmunit.ttf,BoldItalicFont=cmuntx.ttf]{cmunrm.ttf}.
Therefore it\textquotesingle{}s needed to extend our example and specify how to sort the word:
\begin{Shaded}
\begin{Highlighting}[]

\NormalTok{\textbackslash{}newglossaryentry\{naiive\}}
\NormalTok{\{}
  \NormalTok{name=na\textbackslash{}"\{\textbackslash{}i\}ve,}
  \NormalTok{description=\{is a French loanword (adjective, form of naïf)}
               \NormalTok{indicating having or showing a lack of experience,}
               \NormalTok{understanding or sophistication\},}
  \NormalTok{sort=naiive}
\NormalTok{\}}
\end{Highlighting}
\end{Shaded}


You can also specify plural forms, if they are not formed by adding “s” (we will learn how to use them in next section):
\begin{Shaded}
\begin{Highlighting}[]

\NormalTok{\textbackslash{}newglossaryentry\{Linux\}}
\NormalTok{\{}
  \NormalTok{name=Linux,}
  \NormalTok{description=\{is a generic term referring to the family of Unix-like}
               \NormalTok{computer operating systems that use the Linux kernel\},}
  \NormalTok{plural=Linuces}
\NormalTok{\}}
\end{Highlighting}
\end{Shaded}


Or, for acronyms:
\begin{Shaded}
\begin{Highlighting}[]

\NormalTok{\textbackslash{}newacronym[longplural=\{Frames per Second\}]\{fpsLabel\}\{FPS\}\{Frame per Second\}}
\end{Highlighting}
\end{Shaded}


This will avoid the wrong long plural: Frame per Seconds.

So far, the glossary entries have been defined as key-{}value lists. Sometimes, a description is more complex than just a paragraph. For example, you may want to have multiple paragraphs, itemized lists, figures, tables, etc. For such glossary entries use the command {\ttfamily \setmainfont[Path=/usr/share/fonts/truetype/cmu/,UprightFont=cmunrm.ttf,BoldFont=cmunbx.ttf,ItalicFont=cmunti.ttf,BoldItalicFont=cmunbi.ttf]{cmuntt.ttf}\setmonofont[Path=/usr/share/fonts/truetype/cmu/,UprightFont=cmuntt.ttf,BoldFont=cmuntb.ttf,ItalicFont=cmunit.ttf,BoldItalicFont=cmuntx.ttf]{cmuntt.ttf}\ttfamily longnewglossaryentry}{$\text{ }$}\setmainfont[Path=/usr/share/fonts/truetype/cmu/,UprightFont=cmunrm.ttf,BoldFont=cmunbx.ttf,ItalicFont=cmunti.ttf,BoldItalicFont=cmunbi.ttf]{cmunrm.ttf}\setmonofont[Path=/usr/share/fonts/truetype/cmu/,UprightFont=cmuntt.ttf,BoldFont=cmuntb.ttf,ItalicFont=cmunit.ttf,BoldItalicFont=cmuntx.ttf]{cmunrm.ttf} in which the description follows the key-{}value list. The computer entry then looks like this:
\begin{Shaded}
\begin{Highlighting}[]

\NormalTok{\textbackslash{}longnewglossaryentry\{computer\}}
\NormalTok{\{}
  \NormalTok{name=computer}
\NormalTok{\}}
  \NormalTok{\{is a programmable machine that receives input,}
               \NormalTok{stores and manipulates data, and provides}
               \NormalTok{output in a useful format\}}
\end{Highlighting}
\end{Shaded}

\subsection{Defining symbols}
\label{648}
Defined entries can also be symbols:
\begin{Shaded}
\begin{Highlighting}[]

\NormalTok{\textbackslash{}newglossaryentry\{pi\}}
\NormalTok{\{}
  \NormalTok{name=\{\textbackslash{}ensuremath\{\textbackslash{}pi\}\},}
  \NormalTok{description=\{ratio of circumference of circle to its}
               \NormalTok{diameter\},}
  \NormalTok{sort=pi}
\NormalTok{\}}
\end{Highlighting}
\end{Shaded}


You can also define both a name and a symbol:
\begin{Shaded}
\begin{Highlighting}[]

\NormalTok{\textbackslash{}newglossaryentry\{real number\}}
\NormalTok{\{}
  \NormalTok{name=\{real number\},}
  \NormalTok{description=\{include both rational numbers, such as $42$ and }
               \NormalTok{$\textbackslash{}frac\{-23\}\{129\}$, and irrational numbers, }
               \NormalTok{such as $\textbackslash{}pi$ and the square root of two; or,}
               \NormalTok{a real number can be given by an infinite decimal}
               \NormalTok{representation, such as $2.4871773339\textbackslash{}ldots$ where}
               \NormalTok{the digits continue in some way; or, the real}
               \NormalTok{numbers may be thought of as points on an infinitely}
               \NormalTok{long number line\},}
  \NormalTok{symbol=\{\textbackslash{}ensuremath\{\textbackslash{}mathbb\{R\}\}\}}
\NormalTok{\}}
\end{Highlighting}
\end{Shaded}

Note that not all glossary styles show defined symbols.
\subsection{Defining acronyms}
\label{649}

To define a new acronym you use the \LaTeXTT{\textbackslash{}newacronym} macro:
\begin{Shaded}
\begin{Highlighting}[]

\NormalTok{\textbackslash{}newacronym\{<label>\}\{<abbrv>\}\{<full>\}}
\end{Highlighting}
\end{Shaded}

where  is the unique label identifying the acronym, <{}abbrv>{} is the abbreviated form of the acronym and <{}full>{} is the expanded text. For example:
\begin{Shaded}
\begin{Highlighting}[]

\NormalTok{\textbackslash{}newacronym\{lvm\}\{LVM\}\{Logical Volume Manager\}}
\end{Highlighting}
\end{Shaded}


Defined acronyms can be put in separate list if you use {\ttfamily \setmainfont[Path=/usr/share/fonts/truetype/cmu/,UprightFont=cmunrm.ttf,BoldFont=cmunbx.ttf,ItalicFont=cmunti.ttf,BoldItalicFont=cmunbi.ttf]{cmuntt.ttf}\setmonofont[Path=/usr/share/fonts/truetype/cmu/,UprightFont=cmuntt.ttf,BoldFont=cmuntb.ttf,ItalicFont=cmunit.ttf,BoldItalicFont=cmuntx.ttf]{cmuntt.ttf}\ttfamily acronym}{$\text{ }$}\setmainfont[Path=/usr/share/fonts/truetype/cmu/,UprightFont=cmunrm.ttf,BoldFont=cmunbx.ttf,ItalicFont=cmunti.ttf,BoldItalicFont=cmunbi.ttf]{cmunrm.ttf}\setmonofont[Path=/usr/share/fonts/truetype/cmu/,UprightFont=cmuntt.ttf,BoldFont=cmuntb.ttf,ItalicFont=cmunit.ttf,BoldItalicFont=cmuntx.ttf]{cmunrm.ttf} package option:
\begin{Shaded}
\begin{Highlighting}[]

\NormalTok{\textbackslash{}usepackage[acronym]\{glossaries\}}
\end{Highlighting}
\end{Shaded}

\section{Using defined terms}
\label{650}
When you have defined a term, you can use it in a document.
There are many different commands used to refer to glossary terms.
\subsection{General references}
\label{651}
A general reference is used with \LaTeXTT{\textbackslash{}gls} command.
If, for example, you have glossary entries defined as those above, you might use it in this way:
\begin{longtable}{p{1.0\linewidth}}
\begin{Shaded}
\begin{Highlighting}[]

\NormalTok{\textbackslash{}Gls\{naiive\} people don't know about}
\NormalTok{alternative \textbackslash{}gls\{computer\} operating systems:}
\NormalTok{\textbackslash{}glspl\{Linux\}, BSDs and GNU/Hurd.}
\end{Highlighting}
\end{Shaded}
\\

Naïve people don\textquotesingle{}t know about alternative computer opera-{}$\text{ }$\newline{}

ting systems: Linuces, BSDs and GNU/Hurd.

\end{longtable}

Description of commands used in above example:
\begin{Shaded}
\begin{Highlighting}[]

\NormalTok{\textbackslash{}gls\{<label>\}}
\end{Highlighting}
\end{Shaded}

This command prints the term associated with  passed as its argument.
If the \LaTeXTT{hyperref} package was loaded before \LaTeXTT{glossaries} it will also be hyperlinked to the entry in glossary.

\begin{Shaded}
\begin{Highlighting}[]

\NormalTok{\textbackslash{}glspl\{<label>\}}
\end{Highlighting}
\end{Shaded}

This command prints the plural of the defined term, other than that it behaves in the same way as \LaTeXTT{gls}.

\begin{Shaded}
\begin{Highlighting}[]

\NormalTok{\textbackslash{}Gls\{<label>\}}
\end{Highlighting}
\end{Shaded}

This command prints the singular form of the term with the first character converted to upper case.

\begin{Shaded}
\begin{Highlighting}[]

\NormalTok{\textbackslash{}Glspl\{<label>\}}
\end{Highlighting}
\end{Shaded}

This command prints the plural form with first letter of the term converted to upper case.

\begin{Shaded}
\begin{Highlighting}[]

\NormalTok{\textbackslash{}glslink\{<label>\}\{<alternate text>\}}
\end{Highlighting}
\end{Shaded}

This command creates the link as usual, but typesets the {\itshape \setmainfont[Path=/usr/share/fonts/truetype/cmu/,UprightFont=cmunrm.ttf,BoldFont=cmunbx.ttf,ItalicFont=cmunti.ttf,BoldItalicFont=cmunbi.ttf]{cmunti.ttf}\setmonofont[Path=/usr/share/fonts/truetype/cmu/,UprightFont=cmuntt.ttf,BoldFont=cmuntb.ttf,ItalicFont=cmunit.ttf,BoldItalicFont=cmuntx.ttf]{cmunti.ttf}\itshape alternate text}{$\text{ }$}\setmainfont[Path=/usr/share/fonts/truetype/cmu/,UprightFont=cmunrm.ttf,BoldFont=cmunbx.ttf,ItalicFont=cmunti.ttf,BoldItalicFont=cmunbi.ttf]{cmunrm.ttf}\setmonofont[Path=/usr/share/fonts/truetype/cmu/,UprightFont=cmuntt.ttf,BoldFont=cmuntb.ttf,ItalicFont=cmunit.ttf,BoldItalicFont=cmuntx.ttf]{cmunrm.ttf} instead.  It can also take several options which changes its default behavior (see the documentation).

\begin{Shaded}
\begin{Highlighting}[]

\NormalTok{\textbackslash{}glssymbol\{<label>\}}
\end{Highlighting}
\end{Shaded}

This command prints what ever is defined in \textbackslash{}newglossaryentry\{\}\{symbol=\{Output of glssymbol\}, ...\}

\begin{Shaded}
\begin{Highlighting}[]

\NormalTok{\textbackslash{}glsdesc\{<label>\}}
\end{Highlighting}
\end{Shaded}


This command prints what ever is defined in \textbackslash{}newglossaryentry\{\}\{description=\{Output of glsdesc\}, ...\}
\subsection{Referring acronyms}
\label{652}
Acronyms behave a bit differently than normal glossary terms.
On first use the \LaTeXTT{\textbackslash{}gls} command will display \symbol{34}<{}full>{} (<{}abbrv>{})\symbol{34}.
On subsequent uses only the abbreviation will be displayed.

To reset the first use of an acronym, use the command:
\begin{Shaded}
\begin{Highlighting}[]

\NormalTok{\textbackslash{}glsreset\{<label>\}}
\end{Highlighting}
\end{Shaded}

or, if you want to reset the use status of all acronyms:
\begin{Shaded}
\begin{Highlighting}[]

\NormalTok{\textbackslash{}glsresetall}
\end{Highlighting}
\end{Shaded}


If you just want to print the long version of an acronym without the abbreviation \symbol{34}<{}full>{}\symbol{34}, use :
\begin{Shaded}
\begin{Highlighting}[]

\NormalTok{\textbackslash{}acrlong\{<label>\}}
\end{Highlighting}
\end{Shaded}

If you just want to print the long version of an acronym with the abbreviation \symbol{34}<{}full>{} (<{}abbrv>{})\symbol{34}, use :
\begin{Shaded}
\begin{Highlighting}[]

\NormalTok{\textbackslash{}acrfull\{<label>\}}
\end{Highlighting}
\end{Shaded}

If you just want to print the abbreviation \symbol{34}<{}abbrv>{}\symbol{34}, use :
\begin{Shaded}
\begin{Highlighting}[]

\NormalTok{\textbackslash{}acrshort\{<label>\}}
\end{Highlighting}
\end{Shaded}

\chapter{Displaying the Glossary}

\myminitoc
\label{653}
To display the sorted list of terms you need to add:
\begin{Shaded}
\begin{Highlighting}[]

\NormalTok{\textbackslash{}printglossaries}
\end{Highlighting}
\end{Shaded}

at the place you want the glossary and the list of acronyms to appear.

If all entries are to be printed the command
\begin{Shaded}
\begin{Highlighting}[]

\NormalTok{\textbackslash{}glsaddall}
\end{Highlighting}
\end{Shaded}

can be inserted before \LaTeXTT{\textbackslash{}printglossaries}.
You may also want to use \LaTeXTT{\textbackslash{}usepackage{$\text{[}$}nonumberlist{$\text{]}$}\{glossaries\}} to suppress the location list within the glossary.
\subsection{Separate Glossary and List of Acronyms}
\label{654}

\LaTeXTT{\textbackslash{}printglossaries} will display all the glossaries in the order in which they were defined.\myfootnote{\myplainurl{http://mirror.ox.ac.uk/sites/ctan.org/macros/latex/contrib/glossaries/glossaries-user.html\#dx1-35001}} If no custom glossaries are defined, the default glossary and the list of acronyms will be displayed.

The glossary and the list of acronyms can be displayed separately in different places\myfootnote{\myplainurl{http://mirror.ox.ac.uk/sites/ctan.org/macros/latex/contrib/glossaries/glossaries-user.html\#dx1-43001}}:

\begin{Shaded}
\begin{Highlighting}[]

\NormalTok{\textbackslash{}usepackage[acronym]\{glossaries\}}
 
\NormalTok{\textbackslash{}printglossary[type=\textbackslash{}acronymtype] }\CommentTok{% prints just the list of acronyms}
 
\NormalTok{Some text between the list of acronyms and the glossary.}
 
\NormalTok{\textbackslash{}printglossary }\CommentTok{% if no option is supplied the default glossary is printed.}
\end{Highlighting}
\end{Shaded}

\subsubsection{Dual entries with reference to a glossary entry from an acronym}
\label{655}
It may be useful to have both an acronym and a glossary entry for the same term.
To link these two, define the acronym with a reference to the glossary entry like this:
\begin{Shaded}
\begin{Highlighting}[]

\NormalTok{\textbackslash{}newglossaryentry\{gls-OWD\} \{}
  \NormalTok{name=\{One-Way Delay\},}
  \NormalTok{description=\{The time a packet uses through a network from one host to}
 \NormalTok{another\},}
\NormalTok{\}}
\NormalTok{\textbackslash{}newacronym[see=\{[Glossary:]\{gls-OWD\}\}]\{OWD\}\{OWD\}\{One-Way}
 \NormalTok{Delay\textbackslash{}glsadd\{gls-OWD\}\}}
\end{Highlighting}
\end{Shaded}
\\

\TemplateSpaceIndent{$\text{ }${}Refer$\text{ }${}to$\text{ }${}acronym$\text{ }${}with$\text{ }${}\textbackslash{}gls\{OWD\}$\text{ }${}and$\text{ }${}the$\text{ }${}glossary$\text{ }${}with$\text{ }${}\textbackslash{}gls\{gls-{}OWD\}}


To make this easier, we can use this command (modified from example in the official docs):
\\

\TemplateSpaceIndent{$\text{ }${}Syntax:$\text{ }${}\textbackslash{}newdualentry{$\text{[}$}glossary$\text{ }${}options{$\text{]}$}{$\text{[}$}acronym$\text{ }$\newline{}
$\text{ }${}options{$\text{]}$}\{label\}\{abbrv\}\{long\}\{description\}}


\begin{Shaded}
\begin{Highlighting}[]
\NormalTok{\textbackslash{}usepackage\{xparse\}}
\NormalTok{\textbackslash{}DeclareDocumentCommand\{\textbackslash{}newdualentry\}\{ O\{\} O\{\} m m m m \} \{}
  \NormalTok{\textbackslash{}newglossaryentry\{gls-#3\}\{name=\{#5\},text=\{#5\textbackslash{}glsadd\{#3\}\},}
    \NormalTok{description=\{#6\},#1}
  \NormalTok{\}}
  \NormalTok{\textbackslash{}makeglossaries}
  \NormalTok{\textbackslash{}newacronym[see=\{[Glossary:]\{gls-#3\}\},#2]\{#3\}\{#4\}\{#5\textbackslash{}glsadd\{gls-#3\}\}}
\NormalTok{\}}
\end{Highlighting}
\end{Shaded}

then, define new (dual) entries for glossary and acronym list like this:
\begin{Shaded}
\begin{Highlighting}[]

\NormalTok{\textbackslash{}newdualentry\{OWD\} }\CommentTok{% label}
  \NormalTok{\{OWD\}            }\CommentTok{% abbreviation}
  \NormalTok{\{One-Way Delay\}  }\CommentTok
 \NormalTok{description}
\end{Highlighting}
\end{Shaded}

\subsection{Custom Name}
\label{656}

The name of the glossary section can be replaced with a custom name or translated to a different language. Add the option \LaTeXTT{title} to \LaTeXTT{\textbackslash{}printglossary} to specify the glossary\textquotesingle{}s title. Add the option \LaTeXTT{toctitle} to specify a the title used in the table of content (if not used, \LaTeXTT{title} is used as default). \myfootnote{User Manual for glossaries.sty v4.02 as of 2014.01.13 \myplainurl{http://mirror.ox.ac.uk/sites/ctan.org/macros/latex/contrib/glossaries/glossaries-user.html\#sec:printglossary} }

\begin{Shaded}
\begin{Highlighting}[]

\NormalTok{\textbackslash{}printglossary[title=List of Terms,toctitle=Terms and abbreviations]}
\end{Highlighting}
\end{Shaded}

\subsection{Remove the point}
\label{657}

To omit the \myhref{https://en.wikipedia.org/wiki/Full_stop}{dot} at the end of each description, use this code:

\begin{Shaded}
\begin{Highlighting}[]

\NormalTok{\textbackslash{}usepackage[nopostdot]\{glossaries\}}
\end{Highlighting}
\end{Shaded}

\subsection{Changing Glossary Entry Presentation Using Glossary Styles}
\label{658}

A number of pre-{}built styles are available, and can be changed easily using

\begin{Shaded}
\begin{Highlighting}[]

\CommentTok{% Must be issued before \textbackslash{}printglossaries}
\NormalTok{\textbackslash{}glossarystyle\{<newstyle>\}}
\end{Highlighting}
\end{Shaded}


Commonly used styles include list
\\

\TemplateSpaceIndent{$\text{ }${}{\bfseries \setmainfont[Path=/usr/share/fonts/truetype/cmu/,UprightFont=cmunrm.ttf,BoldFont=cmunbx.ttf,ItalicFont=cmunti.ttf,BoldItalicFont=cmunbi.ttf]{cmunbx.ttf}\setmonofont[Path=/usr/share/fonts/truetype/cmu/,UprightFont=cmuntt.ttf,BoldFont=cmuntb.ttf,ItalicFont=cmunit.ttf,BoldItalicFont=cmuntx.ttf]{cmunbx.ttf}\bfseries My Term}$\text{ }${}\setmainfont[Path=/usr/share/fonts/truetype/cmu/,UprightFont=cmunrm.ttf,BoldFont=cmunbx.ttf,ItalicFont=cmunti.ttf,BoldItalicFont=cmunbi.ttf]{cmunrm.ttf}\setmonofont[Path=/usr/share/fonts/truetype/cmu/,UprightFont=cmuntt.ttf,BoldFont=cmuntb.ttf,ItalicFont=cmunit.ttf,BoldItalicFont=cmuntx.ttf]{cmunrm.ttf}Has$\text{ }${}some$\text{ }${}long$\text{ }${}description$\text{ }${}7,$\text{ }${}9}


altlist (inserts newline after term and indents description)
\\

\TemplateSpaceIndent{$\text{ }${}{\bfseries \setmainfont[Path=/usr/share/fonts/truetype/cmu/,UprightFont=cmunrm.ttf,BoldFont=cmunbx.ttf,ItalicFont=cmunti.ttf,BoldItalicFont=cmunbi.ttf]{cmunbx.ttf}\setmonofont[Path=/usr/share/fonts/truetype/cmu/,UprightFont=cmuntt.ttf,BoldFont=cmuntb.ttf,ItalicFont=cmunit.ttf,BoldItalicFont=cmuntx.ttf]{cmunbx.ttf}\bfseries My Term}$\text{ }${}$\text{ }$\newline{}
$\text{ }${}$\text{ }${}$\text{ }${}$\text{ }${}\setmainfont[Path=/usr/share/fonts/truetype/cmu/,UprightFont=cmunrm.ttf,BoldFont=cmunbx.ttf,ItalicFont=cmunti.ttf,BoldItalicFont=cmunbi.ttf]{cmunrm.ttf}\setmonofont[Path=/usr/share/fonts/truetype/cmu/,UprightFont=cmuntt.ttf,BoldFont=cmuntb.ttf,ItalicFont=cmunit.ttf,BoldItalicFont=cmuntx.ttf]{cmunrm.ttf}Has$\text{ }${}some$\text{ }${}long$\text{ }${}description$\text{ }${}7,$\text{ }${}9}


altlistgroup or listgroup (group adds grouping based on the first letters of the terms)
\\

\TemplateSpaceIndent{$\text{ }${}M$\text{ }$\newline{}
$\text{ }${}{\bfseries \setmainfont[Path=/usr/share/fonts/truetype/cmu/,UprightFont=cmunrm.ttf,BoldFont=cmunbx.ttf,ItalicFont=cmunti.ttf,BoldItalicFont=cmunbi.ttf]{cmunbx.ttf}\setmonofont[Path=/usr/share/fonts/truetype/cmu/,UprightFont=cmuntt.ttf,BoldFont=cmuntb.ttf,ItalicFont=cmunit.ttf,BoldItalicFont=cmuntx.ttf]{cmunbx.ttf}\bfseries My First Term}$\text{ }${}$\text{ }$\newline{}
$\text{ }${}$\text{ }${}$\text{ }${}$\text{ }${}\setmainfont[Path=/usr/share/fonts/truetype/cmu/,UprightFont=cmunrm.ttf,BoldFont=cmunbx.ttf,ItalicFont=cmunti.ttf,BoldItalicFont=cmunbi.ttf]{cmunrm.ttf}\setmonofont[Path=/usr/share/fonts/truetype/cmu/,UprightFont=cmuntt.ttf,BoldFont=cmuntb.ttf,ItalicFont=cmunit.ttf,BoldItalicFont=cmuntx.ttf]{cmunrm.ttf}Has$\text{ }${}some$\text{ }${}long$\text{ }${}description$\text{ }${}7,$\text{ }${}9$\text{ }$\newline{}
$\text{ }${}{\bfseries \setmainfont[Path=/usr/share/fonts/truetype/cmu/,UprightFont=cmunrm.ttf,BoldFont=cmunbx.ttf,ItalicFont=cmunti.ttf,BoldItalicFont=cmunbi.ttf]{cmunbx.ttf}\setmonofont[Path=/usr/share/fonts/truetype/cmu/,UprightFont=cmuntt.ttf,BoldFont=cmuntb.ttf,ItalicFont=cmunit.ttf,BoldItalicFont=cmuntx.ttf]{cmunbx.ttf}\bfseries My Second Term}$\text{ }${}$\text{ }$\newline{}
$\text{ }${}$\text{ }${}$\text{ }${}$\text{ }${}\setmainfont[Path=/usr/share/fonts/truetype/cmu/,UprightFont=cmunrm.ttf,BoldFont=cmunbx.ttf,ItalicFont=cmunti.ttf,BoldItalicFont=cmunbi.ttf]{cmunrm.ttf}\setmonofont[Path=/usr/share/fonts/truetype/cmu/,UprightFont=cmuntt.ttf,BoldFont=cmuntb.ttf,ItalicFont=cmunit.ttf,BoldItalicFont=cmuntx.ttf]{cmunrm.ttf}Has$\text{ }${}some$\text{ }${}long$\text{ }${}description$\text{ }${}7,$\text{ }${}9}


altlisthypergroup or listhypergroup (hyper adds an hyperlinked \textquotesingle{}index\textquotesingle{} at the top of each glossary to jump to a group)
\\

\TemplateSpaceIndent{$\text{ }${}A|B|C|D|F|G|I|M|O|R|S|C|D|G|M|P$\text{ }$\newline{}
$\text{ }${}A$\text{ }$\newline{}
$\text{ }${}{\bfseries \setmainfont[Path=/usr/share/fonts/truetype/cmu/,UprightFont=cmunrm.ttf,BoldFont=cmunbx.ttf,ItalicFont=cmunti.ttf,BoldItalicFont=cmunbi.ttf]{cmunbx.ttf}\setmonofont[Path=/usr/share/fonts/truetype/cmu/,UprightFont=cmuntt.ttf,BoldFont=cmuntb.ttf,ItalicFont=cmunit.ttf,BoldItalicFont=cmuntx.ttf]{cmunbx.ttf}\bfseries A First term}$\text{ }${}$\text{ }$\newline{}
$\text{ }${}$\text{ }${}$\text{ }${}$\text{ }${}\setmainfont[Path=/usr/share/fonts/truetype/cmu/,UprightFont=cmunrm.ttf,BoldFont=cmunbx.ttf,ItalicFont=cmunti.ttf,BoldItalicFont=cmunbi.ttf]{cmunrm.ttf}\setmonofont[Path=/usr/share/fonts/truetype/cmu/,UprightFont=cmuntt.ttf,BoldFont=cmuntb.ttf,ItalicFont=cmunit.ttf,BoldItalicFont=cmuntx.ttf]{cmunrm.ttf}Has$\text{ }${}some$\text{ }${}long$\text{ }${}description$\text{ }${}7,$\text{ }${}9$\text{ }$\newline{}
$\text{ }${}B$\text{ }$\newline{}
$\text{ }${}{\bfseries \setmainfont[Path=/usr/share/fonts/truetype/cmu/,UprightFont=cmunrm.ttf,BoldFont=cmunbx.ttf,ItalicFont=cmunti.ttf,BoldItalicFont=cmunbi.ttf]{cmunbx.ttf}\setmonofont[Path=/usr/share/fonts/truetype/cmu/,UprightFont=cmuntt.ttf,BoldFont=cmuntb.ttf,ItalicFont=cmunit.ttf,BoldItalicFont=cmuntx.ttf]{cmunbx.ttf}\bfseries Barely missed first}$\text{ }${}$\text{ }$\newline{}
$\text{ }${}$\text{ }${}$\text{ }${}$\text{ }${}\setmainfont[Path=/usr/share/fonts/truetype/cmu/,UprightFont=cmunrm.ttf,BoldFont=cmunbx.ttf,ItalicFont=cmunti.ttf,BoldItalicFont=cmunbi.ttf]{cmunrm.ttf}\setmonofont[Path=/usr/share/fonts/truetype/cmu/,UprightFont=cmuntt.ttf,BoldFont=cmuntb.ttf,ItalicFont=cmunit.ttf,BoldItalicFont=cmuntx.ttf]{cmunrm.ttf}Has$\text{ }${}some$\text{ }${}long$\text{ }${}description$\text{ }${}7,$\text{ }${}9}

\section{Building your document}
\label{659}

Building your document and its glossary requires three steps:

\begin{myenumerate}
\item{}  build your LaTeX document {\mbox{$\text{---}$}} this will also generate the files needed by {\ttfamily \setmainfont[Path=/usr/share/fonts/truetype/cmu/,UprightFont=cmunrm.ttf,BoldFont=cmunbx.ttf,ItalicFont=cmunti.ttf,BoldItalicFont=cmunbi.ttf]{cmuntt.ttf}\setmonofont[Path=/usr/share/fonts/truetype/cmu/,UprightFont=cmuntt.ttf,BoldFont=cmuntb.ttf,ItalicFont=cmunit.ttf,BoldItalicFont=cmuntx.ttf]{cmuntt.ttf}\ttfamily makeglossaries}
\item{} {$\text{ }$}\setmainfont[Path=/usr/share/fonts/truetype/cmu/,UprightFont=cmunrm.ttf,BoldFont=cmunbx.ttf,ItalicFont=cmunti.ttf,BoldItalicFont=cmunbi.ttf]{cmunrm.ttf}\setmonofont[Path=/usr/share/fonts/truetype/cmu/,UprightFont=cmuntt.ttf,BoldFont=cmuntb.ttf,ItalicFont=cmunit.ttf,BoldItalicFont=cmuntx.ttf]{cmunrm.ttf} invoke {\ttfamily \setmainfont[Path=/usr/share/fonts/truetype/cmu/,UprightFont=cmunrm.ttf,BoldFont=cmunbx.ttf,ItalicFont=cmunti.ttf,BoldItalicFont=cmunbi.ttf]{cmuntt.ttf}\setmonofont[Path=/usr/share/fonts/truetype/cmu/,UprightFont=cmuntt.ttf,BoldFont=cmuntb.ttf,ItalicFont=cmunit.ttf,BoldItalicFont=cmuntx.ttf]{cmuntt.ttf}\ttfamily makeglossaries}{$\text{ }$}\setmainfont[Path=/usr/share/fonts/truetype/cmu/,UprightFont=cmunrm.ttf,BoldFont=cmunbx.ttf,ItalicFont=cmunti.ttf,BoldItalicFont=cmunbi.ttf]{cmunrm.ttf}\setmonofont[Path=/usr/share/fonts/truetype/cmu/,UprightFont=cmuntt.ttf,BoldFont=cmuntb.ttf,ItalicFont=cmunit.ttf,BoldItalicFont=cmuntx.ttf]{cmunrm.ttf} {\mbox{$\text{---}$}} a script which selects the correct character encodings and language settings and which will also run {\ttfamily \setmainfont[Path=/usr/share/fonts/truetype/cmu/,UprightFont=cmunrm.ttf,BoldFont=cmunbx.ttf,ItalicFont=cmunti.ttf,BoldItalicFont=cmunbi.ttf]{cmuntt.ttf}\setmonofont[Path=/usr/share/fonts/truetype/cmu/,UprightFont=cmuntt.ttf,BoldFont=cmuntb.ttf,ItalicFont=cmunit.ttf,BoldItalicFont=cmuntx.ttf]{cmuntt.ttf}\ttfamily xindy}{$\text{ }$}\setmainfont[Path=/usr/share/fonts/truetype/cmu/,UprightFont=cmunrm.ttf,BoldFont=cmunbx.ttf,ItalicFont=cmunti.ttf,BoldItalicFont=cmunbi.ttf]{cmunrm.ttf}\setmonofont[Path=/usr/share/fonts/truetype/cmu/,UprightFont=cmuntt.ttf,BoldFont=cmuntb.ttf,ItalicFont=cmunit.ttf,BoldItalicFont=cmuntx.ttf]{cmunrm.ttf} or {\ttfamily \setmainfont[Path=/usr/share/fonts/truetype/cmu/,UprightFont=cmunrm.ttf,BoldFont=cmunbx.ttf,ItalicFont=cmunti.ttf,BoldItalicFont=cmunbi.ttf]{cmuntt.ttf}\setmonofont[Path=/usr/share/fonts/truetype/cmu/,UprightFont=cmuntt.ttf,BoldFont=cmuntb.ttf,ItalicFont=cmunit.ttf,BoldItalicFont=cmuntx.ttf]{cmuntt.ttf}\ttfamily makeindex}{$\text{ }$}\setmainfont[Path=/usr/share/fonts/truetype/cmu/,UprightFont=cmunrm.ttf,BoldFont=cmunbx.ttf,ItalicFont=cmunti.ttf,BoldItalicFont=cmunbi.ttf]{cmunrm.ttf}\setmonofont[Path=/usr/share/fonts/truetype/cmu/,UprightFont=cmuntt.ttf,BoldFont=cmuntb.ttf,ItalicFont=cmunit.ttf,BoldItalicFont=cmuntx.ttf]{cmunrm.ttf} if these are specified in your document file
\item{}  build your LaTeX document again {\mbox{$\text{---}$}} to produce a document with glossary entries
\end{myenumerate}


Thus:

\TemplatePreformat{$\text{ }$\newline{}
latex$\text{ }${}doc$\text{ }$\newline{}
makeglossaries$\text{ }${}doc$\text{ }$\newline{}
latex$\text{ }${}doc$\text{ }$\newline{}
}

where {\ttfamily \setmainfont[Path=/usr/share/fonts/truetype/cmu/,UprightFont=cmunrm.ttf,BoldFont=cmunbx.ttf,ItalicFont=cmunti.ttf,BoldItalicFont=cmunbi.ttf]{cmuntt.ttf}\setmonofont[Path=/usr/share/fonts/truetype/cmu/,UprightFont=cmuntt.ttf,BoldFont=cmuntb.ttf,ItalicFont=cmunit.ttf,BoldItalicFont=cmuntx.ttf]{cmuntt.ttf}\ttfamily latex}{$\text{ }$}\setmainfont[Path=/usr/share/fonts/truetype/cmu/,UprightFont=cmunrm.ttf,BoldFont=cmunbx.ttf,ItalicFont=cmunti.ttf,BoldItalicFont=cmunbi.ttf]{cmunrm.ttf}\setmonofont[Path=/usr/share/fonts/truetype/cmu/,UprightFont=cmuntt.ttf,BoldFont=cmuntb.ttf,ItalicFont=cmunit.ttf,BoldItalicFont=cmuntx.ttf]{cmunrm.ttf} is your usual build call (perhaps {\ttfamily \setmainfont[Path=/usr/share/fonts/truetype/cmu/,UprightFont=cmunrm.ttf,BoldFont=cmunbx.ttf,ItalicFont=cmunti.ttf,BoldItalicFont=cmunbi.ttf]{cmuntt.ttf}\setmonofont[Path=/usr/share/fonts/truetype/cmu/,UprightFont=cmuntt.ttf,BoldFont=cmuntb.ttf,ItalicFont=cmunit.ttf,BoldItalicFont=cmuntx.ttf]{cmuntt.ttf}\ttfamily pdflatex}\setmainfont[Path=/usr/share/fonts/truetype/cmu/,UprightFont=cmunrm.ttf,BoldFont=cmunbx.ttf,ItalicFont=cmunti.ttf,BoldItalicFont=cmunbi.ttf]{cmunrm.ttf}\setmonofont[Path=/usr/share/fonts/truetype/cmu/,UprightFont=cmuntt.ttf,BoldFont=cmuntb.ttf,ItalicFont=cmunit.ttf,BoldItalicFont=cmuntx.ttf]{cmunrm.ttf}) and {\ttfamily \setmainfont[Path=/usr/share/fonts/truetype/cmu/,UprightFont=cmunrm.ttf,BoldFont=cmunbx.ttf,ItalicFont=cmunti.ttf,BoldItalicFont=cmunbi.ttf]{cmuntt.ttf}\setmonofont[Path=/usr/share/fonts/truetype/cmu/,UprightFont=cmuntt.ttf,BoldFont=cmuntb.ttf,ItalicFont=cmunit.ttf,BoldItalicFont=cmuntx.ttf]{cmuntt.ttf}\ttfamily doc}{$\text{ }$}\setmainfont[Path=/usr/share/fonts/truetype/cmu/,UprightFont=cmunrm.ttf,BoldFont=cmunbx.ttf,ItalicFont=cmunti.ttf,BoldItalicFont=cmunbi.ttf]{cmunrm.ttf}\setmonofont[Path=/usr/share/fonts/truetype/cmu/,UprightFont=cmuntt.ttf,BoldFont=cmuntb.ttf,ItalicFont=cmunit.ttf,BoldItalicFont=cmuntx.ttf]{cmunrm.ttf} is the name of your LaTeX master file.

If your entries are interlinked (entries themselves link to other entries with \LaTeXTT{\textbackslash{}gls} calls), you will need to run steps 1 and 2 twice, that is, in the following order: 1,{\mbox{$~$}}2,{\mbox{$~$}}1,{\mbox{$~$}}2,{\mbox{$~$}}3.

If you encounter problems, view the {\ttfamily \setmainfont[Path=/usr/share/fonts/truetype/cmu/,UprightFont=cmunrm.ttf,BoldFont=cmunbx.ttf,ItalicFont=cmunti.ttf,BoldItalicFont=cmunbi.ttf]{cmuntt.ttf}\setmonofont[Path=/usr/share/fonts/truetype/cmu/,UprightFont=cmuntt.ttf,BoldFont=cmuntb.ttf,ItalicFont=cmunit.ttf,BoldItalicFont=cmuntx.ttf]{cmuntt.ttf}\ttfamily doc.log}{$\text{ }$}\setmainfont[Path=/usr/share/fonts/truetype/cmu/,UprightFont=cmunrm.ttf,BoldFont=cmunbx.ttf,ItalicFont=cmunti.ttf,BoldItalicFont=cmunbi.ttf]{cmunrm.ttf}\setmonofont[Path=/usr/share/fonts/truetype/cmu/,UprightFont=cmuntt.ttf,BoldFont=cmuntb.ttf,ItalicFont=cmunit.ttf,BoldItalicFont=cmuntx.ttf]{cmunrm.ttf} and {\ttfamily \setmainfont[Path=/usr/share/fonts/truetype/cmu/,UprightFont=cmunrm.ttf,BoldFont=cmunbx.ttf,ItalicFont=cmunti.ttf,BoldItalicFont=cmunbi.ttf]{cmuntt.ttf}\setmonofont[Path=/usr/share/fonts/truetype/cmu/,UprightFont=cmuntt.ttf,BoldFont=cmuntb.ttf,ItalicFont=cmunit.ttf,BoldItalicFont=cmuntx.ttf]{cmuntt.ttf}\ttfamily doc.glg}{$\text{ }$}\setmainfont[Path=/usr/share/fonts/truetype/cmu/,UprightFont=cmunrm.ttf,BoldFont=cmunbx.ttf,ItalicFont=cmunti.ttf,BoldItalicFont=cmunbi.ttf]{cmunrm.ttf}\setmonofont[Path=/usr/share/fonts/truetype/cmu/,UprightFont=cmuntt.ttf,BoldFont=cmuntb.ttf,ItalicFont=cmunit.ttf,BoldItalicFont=cmuntx.ttf]{cmunrm.ttf} files in a text editor for clues.
\chapter{Example for use in windows with Texmaker}

\myminitoc
\label{660}
\section{Compile glossary with xindy -{} In Windows with Texmaker}
\label{661}
In TeX Live and since June 2015 in MikTeX {\bfseries {\ttfamily \setmainfont[Path=/usr/share/fonts/truetype/cmu/,UprightFont=cmunrm.ttf,BoldFont=cmunbx.ttf,ItalicFont=cmunti.ttf,BoldItalicFont=cmunbi.ttf]{cmuntb.ttf}\setmonofont[Path=/usr/share/fonts/truetype/cmu/,UprightFont=cmuntt.ttf,BoldFont=cmuntb.ttf,ItalicFont=cmunit.ttf,BoldItalicFont=cmuntx.ttf]{cmuntb.ttf}\ttfamily \bfseries xindy}}{$\text{ }$}\setmainfont[Path=/usr/share/fonts/truetype/cmu/,UprightFont=cmunrm.ttf,BoldFont=cmunbx.ttf,ItalicFont=cmunti.ttf,BoldItalicFont=cmunbi.ttf]{cmunrm.ttf}\setmonofont[Path=/usr/share/fonts/truetype/cmu/,UprightFont=cmuntt.ttf,BoldFont=cmuntb.ttf,ItalicFont=cmunit.ttf,BoldItalicFont=cmuntx.ttf]{cmunrm.ttf} is already included. 

There is only one issue with path of the install directory of MikTeX containing spaces. It can be solved via the following edit:
\myplainurl{http://tex.stackexchange.com/questions/251221/miktex-and-xindy-problems/251801\#251801}

You need to restart Texmaker after installation of {\ttfamily \setmainfont[Path=/usr/share/fonts/truetype/cmu/,UprightFont=cmunrm.ttf,BoldFont=cmunbx.ttf,ItalicFont=cmunti.ttf,BoldItalicFont=cmunbi.ttf]{cmuntt.ttf}\setmonofont[Path=/usr/share/fonts/truetype/cmu/,UprightFont=cmuntt.ttf,BoldFont=cmuntb.ttf,ItalicFont=cmunit.ttf,BoldItalicFont=cmuntx.ttf]{cmuntt.ttf}\ttfamily xindy}\setmainfont[Path=/usr/share/fonts/truetype/cmu/,UprightFont=cmunrm.ttf,BoldFont=cmunbx.ttf,ItalicFont=cmunti.ttf,BoldItalicFont=cmunbi.ttf]{cmunrm.ttf}\setmonofont[Path=/usr/share/fonts/truetype/cmu/,UprightFont=cmuntt.ttf,BoldFont=cmuntb.ttf,ItalicFont=cmunit.ttf,BoldItalicFont=cmuntx.ttf]{cmunrm.ttf}, to update PATH references to {\ttfamily \setmainfont[Path=/usr/share/fonts/truetype/cmu/,UprightFont=cmunrm.ttf,BoldFont=cmunbx.ttf,ItalicFont=cmunti.ttf,BoldItalicFont=cmunbi.ttf]{cmuntt.ttf}\setmonofont[Path=/usr/share/fonts/truetype/cmu/,UprightFont=cmuntt.ttf,BoldFont=cmuntb.ttf,ItalicFont=cmunit.ttf,BoldItalicFont=cmuntx.ttf]{cmuntt.ttf}\ttfamily xindy}{$\text{ }$}\setmainfont[Path=/usr/share/fonts/truetype/cmu/,UprightFont=cmunrm.ttf,BoldFont=cmunbx.ttf,ItalicFont=cmunti.ttf,BoldItalicFont=cmunbi.ttf]{cmunrm.ttf}\setmonofont[Path=/usr/share/fonts/truetype/cmu/,UprightFont=cmuntt.ttf,BoldFont=cmuntb.ttf,ItalicFont=cmunit.ttf,BoldItalicFont=cmuntx.ttf]{cmunrm.ttf} and Perl binaries.

Then, in Texmaker, go to {\bfseries \setmainfont[Path=/usr/share/fonts/truetype/cmu/,UprightFont=cmunrm.ttf,BoldFont=cmunbx.ttf,ItalicFont=cmunti.ttf,BoldItalicFont=cmunbi.ttf]{cmunbx.ttf}\setmonofont[Path=/usr/share/fonts/truetype/cmu/,UprightFont=cmuntt.ttf,BoldFont=cmuntb.ttf,ItalicFont=cmunit.ttf,BoldItalicFont=cmuntx.ttf]{cmunbx.ttf}\bfseries User}{$\text{ }$}\setmainfont[Path=/usr/share/fonts/truetype/cmu/,UprightFont=cmunrm.ttf,BoldFont=cmunbx.ttf,ItalicFont=cmunti.ttf,BoldItalicFont=cmunbi.ttf]{cmunrm.ttf}\setmonofont[Path=/usr/share/fonts/truetype/cmu/,UprightFont=cmuntt.ttf,BoldFont=cmuntb.ttf,ItalicFont=cmunit.ttf,BoldItalicFont=cmuntx.ttf]{cmunrm.ttf} -{}>{} {\bfseries \setmainfont[Path=/usr/share/fonts/truetype/cmu/,UprightFont=cmunrm.ttf,BoldFont=cmunbx.ttf,ItalicFont=cmunti.ttf,BoldItalicFont=cmunbi.ttf]{cmunbx.ttf}\setmonofont[Path=/usr/share/fonts/truetype/cmu/,UprightFont=cmuntt.ttf,BoldFont=cmuntb.ttf,ItalicFont=cmunit.ttf,BoldItalicFont=cmuntx.ttf]{cmunbx.ttf}\bfseries User Commands}{$\text{ }$}\setmainfont[Path=/usr/share/fonts/truetype/cmu/,UprightFont=cmunrm.ttf,BoldFont=cmunbx.ttf,ItalicFont=cmunti.ttf,BoldItalicFont=cmunbi.ttf]{cmunrm.ttf}\setmonofont[Path=/usr/share/fonts/truetype/cmu/,UprightFont=cmuntt.ttf,BoldFont=cmuntb.ttf,ItalicFont=cmunit.ttf,BoldItalicFont=cmuntx.ttf]{cmunrm.ttf} -{}>{} {\bfseries \setmainfont[Path=/usr/share/fonts/truetype/cmu/,UprightFont=cmunrm.ttf,BoldFont=cmunbx.ttf,ItalicFont=cmunti.ttf,BoldItalicFont=cmunbi.ttf]{cmunbx.ttf}\setmonofont[Path=/usr/share/fonts/truetype/cmu/,UprightFont=cmuntt.ttf,BoldFont=cmuntb.ttf,ItalicFont=cmunit.ttf,BoldItalicFont=cmuntx.ttf]{cmunbx.ttf}\bfseries Edit User Commands}\setmainfont[Path=/usr/share/fonts/truetype/cmu/,UprightFont=cmunrm.ttf,BoldFont=cmunbx.ttf,ItalicFont=cmunti.ttf,BoldItalicFont=cmunbi.ttf]{cmunrm.ttf}\setmonofont[Path=/usr/share/fonts/truetype/cmu/,UprightFont=cmuntt.ttf,BoldFont=cmuntb.ttf,ItalicFont=cmunit.ttf,BoldItalicFont=cmuntx.ttf]{cmunrm.ttf}. $\text{ }$\newline{}

Choose command 1$\text{ }$\newline{}
 
\begin{myenumerate}
\item{}  Menuitem = {\bfseries \setmainfont[Path=/usr/share/fonts/truetype/cmu/,UprightFont=cmunrm.ttf,BoldFont=cmunbx.ttf,ItalicFont=cmunti.ttf,BoldItalicFont=cmunbi.ttf]{cmunbx.ttf}\setmonofont[Path=/usr/share/fonts/truetype/cmu/,UprightFont=cmuntt.ttf,BoldFont=cmuntb.ttf,ItalicFont=cmunit.ttf,BoldItalicFont=cmuntx.ttf]{cmunbx.ttf}\bfseries makeglossaries}
\item{} {$\text{ }$}\setmainfont[Path=/usr/share/fonts/truetype/cmu/,UprightFont=cmunrm.ttf,BoldFont=cmunbx.ttf,ItalicFont=cmunti.ttf,BoldItalicFont=cmunbi.ttf]{cmunrm.ttf}\setmonofont[Path=/usr/share/fonts/truetype/cmu/,UprightFont=cmuntt.ttf,BoldFont=cmuntb.ttf,ItalicFont=cmunit.ttf,BoldItalicFont=cmuntx.ttf]{cmunrm.ttf} Command = {\bfseries \setmainfont[Path=/usr/share/fonts/truetype/cmu/,UprightFont=cmunrm.ttf,BoldFont=cmunbx.ttf,ItalicFont=cmunti.ttf,BoldItalicFont=cmunbi.ttf]{cmunbx.ttf}\setmonofont[Path=/usr/share/fonts/truetype/cmu/,UprightFont=cmuntt.ttf,BoldFont=cmuntb.ttf,ItalicFont=cmunit.ttf,BoldItalicFont=cmuntx.ttf]{cmunbx.ttf}\bfseries makeglossaries \%}
\end{myenumerate}
\setmainfont[Path=/usr/share/fonts/truetype/cmu/,UprightFont=cmunrm.ttf,BoldFont=cmunbx.ttf,ItalicFont=cmunti.ttf,BoldItalicFont=cmunbi.ttf]{cmunrm.ttf}\setmonofont[Path=/usr/share/fonts/truetype/cmu/,UprightFont=cmuntt.ttf,BoldFont=cmuntb.ttf,ItalicFont=cmunit.ttf,BoldItalicFont=cmuntx.ttf]{cmunrm.ttf}
Now push {\bfseries \setmainfont[Path=/usr/share/fonts/truetype/cmu/,UprightFont=cmunrm.ttf,BoldFont=cmunbx.ttf,ItalicFont=cmunti.ttf,BoldItalicFont=cmunbi.ttf]{cmunbx.ttf}\setmonofont[Path=/usr/share/fonts/truetype/cmu/,UprightFont=cmuntt.ttf,BoldFont=cmuntb.ttf,ItalicFont=cmunit.ttf,BoldItalicFont=cmuntx.ttf]{cmunbx.ttf}\bfseries Alt+Shift+F1}\setmainfont[Path=/usr/share/fonts/truetype/cmu/,UprightFont=cmunrm.ttf,BoldFont=cmunbx.ttf,ItalicFont=cmunti.ttf,BoldItalicFont=cmunbi.ttf]{cmunrm.ttf}\setmonofont[Path=/usr/share/fonts/truetype/cmu/,UprightFont=cmuntt.ttf,BoldFont=cmuntb.ttf,ItalicFont=cmunit.ttf,BoldItalicFont=cmuntx.ttf]{cmunrm.ttf}and then -{}>{}{\bfseries \setmainfont[Path=/usr/share/fonts/truetype/cmu/,UprightFont=cmunrm.ttf,BoldFont=cmunbx.ttf,ItalicFont=cmunti.ttf,BoldItalicFont=cmunbi.ttf]{cmunbx.ttf}\setmonofont[Path=/usr/share/fonts/truetype/cmu/,UprightFont=cmuntt.ttf,BoldFont=cmuntb.ttf,ItalicFont=cmunit.ttf,BoldItalicFont=cmuntx.ttf]{cmunbx.ttf}\bfseries F1}\setmainfont[Path=/usr/share/fonts/truetype/cmu/,UprightFont=cmunrm.ttf,BoldFont=cmunbx.ttf,ItalicFont=cmunti.ttf,BoldItalicFont=cmunbi.ttf]{cmunrm.ttf}\setmonofont[Path=/usr/share/fonts/truetype/cmu/,UprightFont=cmuntt.ttf,BoldFont=cmuntb.ttf,ItalicFont=cmunit.ttf,BoldItalicFont=cmuntx.ttf]{cmunrm.ttf}

{\bfseries \setmainfont[Path=/usr/share/fonts/truetype/cmu/,UprightFont=cmunrm.ttf,BoldFont=cmunbx.ttf,ItalicFont=cmunti.ttf,BoldItalicFont=cmunbi.ttf]{cmunbx.ttf}\setmonofont[Path=/usr/share/fonts/truetype/cmu/,UprightFont=cmuntt.ttf,BoldFont=cmuntb.ttf,ItalicFont=cmunit.ttf,BoldItalicFont=cmuntx.ttf]{cmunbx.ttf}\bfseries Note}\setmainfont[Path=/usr/share/fonts/truetype/cmu/,UprightFont=cmunrm.ttf,BoldFont=cmunbx.ttf,ItalicFont=cmunti.ttf,BoldItalicFont=cmunbi.ttf]{cmunrm.ttf}\setmonofont[Path=/usr/share/fonts/truetype/cmu/,UprightFont=cmuntt.ttf,BoldFont=cmuntb.ttf,ItalicFont=cmunit.ttf,BoldItalicFont=cmuntx.ttf]{cmunrm.ttf}, for use with the \symbol{34}use build directory\symbol{34} option of Texmaker: makeglossaries needs to find the aux file. Thankfully, while Texmaker does not help there, the option {\bfseries \setmainfont[Path=/usr/share/fonts/truetype/cmu/,UprightFont=cmunrm.ttf,BoldFont=cmunbx.ttf,ItalicFont=cmunti.ttf,BoldItalicFont=cmunbi.ttf]{cmunbx.ttf}\setmonofont[Path=/usr/share/fonts/truetype/cmu/,UprightFont=cmuntt.ttf,BoldFont=cmuntb.ttf,ItalicFont=cmunit.ttf,BoldItalicFont=cmuntx.ttf]{cmunbx.ttf}\bfseries -{}d 
\begin{myitemize}
\end{myitemize}}
\begin{myitemize}{$\text{ }$}\setmainfont[Path=/usr/share/fonts/truetype/cmu/,UprightFont=cmunrm.ttf,BoldFont=cmunbx.ttf,ItalicFont=cmunti.ttf,BoldItalicFont=cmunbi.ttf]{cmunrm.ttf}\setmonofont[Path=/usr/share/fonts/truetype/cmu/,UprightFont=cmuntt.ttf,BoldFont=cmuntb.ttf,ItalicFont=cmunit.ttf,BoldItalicFont=cmuntx.ttf]{cmunrm.ttf} of makeglossaries provides for the subdirectory case. Hence the Command in this case should be:$\text{ }$\newline{}

Command = {\bfseries \setmainfont[Path=/usr/share/fonts/truetype/cmu/,UprightFont=cmunrm.ttf,BoldFont=cmunbx.ttf,ItalicFont=cmunti.ttf,BoldItalicFont=cmunbi.ttf]{cmunbx.ttf}\setmonofont[Path=/usr/share/fonts/truetype/cmu/,UprightFont=cmuntt.ttf,BoldFont=cmuntb.ttf,ItalicFont=cmunit.ttf,BoldItalicFont=cmuntx.ttf]{cmunbx.ttf}\bfseries makeglossaries -{}d build \%}{$\text{ }$}\setmainfont[Path=/usr/share/fonts/truetype/cmu/,UprightFont=cmunrm.ttf,BoldFont=cmunbx.ttf,ItalicFont=cmunti.ttf,BoldItalicFont=cmunbi.ttf]{cmunrm.ttf}\setmonofont[Path=/usr/share/fonts/truetype/cmu/,UprightFont=cmuntt.ttf,BoldFont=cmuntb.ttf,ItalicFont=cmunit.ttf,BoldItalicFont=cmuntx.ttf]{cmunrm.ttf} instead.
\section{Document preamble}
\label{662}
In preamble should be included (note, {\bfseries \setmainfont[Path=/usr/share/fonts/truetype/cmu/,UprightFont=cmunrm.ttf,BoldFont=cmunbx.ttf,ItalicFont=cmunti.ttf,BoldItalicFont=cmunbi.ttf]{cmunbx.ttf}\setmonofont[Path=/usr/share/fonts/truetype/cmu/,UprightFont=cmuntt.ttf,BoldFont=cmuntb.ttf,ItalicFont=cmunit.ttf,BoldItalicFont=cmuntx.ttf]{cmunbx.ttf}\bfseries hyperref}{$\text{ }$}\setmainfont[Path=/usr/share/fonts/truetype/cmu/,UprightFont=cmunrm.ttf,BoldFont=cmunbx.ttf,ItalicFont=cmunti.ttf,BoldItalicFont=cmunbi.ttf]{cmunrm.ttf}\setmonofont[Path=/usr/share/fonts/truetype/cmu/,UprightFont=cmuntt.ttf,BoldFont=cmuntb.ttf,ItalicFont=cmunit.ttf,BoldItalicFont=cmuntx.ttf]{cmunrm.ttf} should be loaded {\bfseries \setmainfont[Path=/usr/share/fonts/truetype/cmu/,UprightFont=cmunrm.ttf,BoldFont=cmunbx.ttf,ItalicFont=cmunti.ttf,BoldItalicFont=cmunbi.ttf]{cmunbx.ttf}\setmonofont[Path=/usr/share/fonts/truetype/cmu/,UprightFont=cmuntt.ttf,BoldFont=cmuntb.ttf,ItalicFont=cmunit.ttf,BoldItalicFont=cmuntx.ttf]{cmunbx.ttf}\bfseries before}{$\text{ }$}\setmainfont[Path=/usr/share/fonts/truetype/cmu/,UprightFont=cmunrm.ttf,BoldFont=cmunbx.ttf,ItalicFont=cmunti.ttf,BoldItalicFont=cmunbi.ttf]{cmunrm.ttf}\setmonofont[Path=/usr/share/fonts/truetype/cmu/,UprightFont=cmuntt.ttf,BoldFont=cmuntb.ttf,ItalicFont=cmunit.ttf,BoldItalicFont=cmuntx.ttf]{cmunrm.ttf} the {\bfseries \setmainfont[Path=/usr/share/fonts/truetype/cmu/,UprightFont=cmunrm.ttf,BoldFont=cmunbx.ttf,ItalicFont=cmunti.ttf,BoldItalicFont=cmunbi.ttf]{cmunbx.ttf}\setmonofont[Path=/usr/share/fonts/truetype/cmu/,UprightFont=cmuntt.ttf,BoldFont=cmuntb.ttf,ItalicFont=cmunit.ttf,BoldItalicFont=cmuntx.ttf]{cmunbx.ttf}\bfseries glossaries}\setmainfont[Path=/usr/share/fonts/truetype/cmu/,UprightFont=cmunrm.ttf,BoldFont=cmunbx.ttf,ItalicFont=cmunti.ttf,BoldItalicFont=cmunbi.ttf]{cmunrm.ttf}\setmonofont[Path=/usr/share/fonts/truetype/cmu/,UprightFont=cmuntt.ttf,BoldFont=cmuntb.ttf,ItalicFont=cmunit.ttf,BoldItalicFont=cmuntx.ttf]{cmunrm.ttf}):\\

\TemplateSpaceIndent{$\text{ }${}\textbackslash{}usepackage{$\text{[}$}nomain,acronym,xindy,toc{$\text{]}$}\{glossaries\}$\text{ }${}\%$\text{ }${}nomain,$\text{ }${}if$\text{ }${}you$\text{ }${}define$\text{ }$\newline{}
$\text{ }${}glossaries$\text{ }${}in$\text{ }${}a$\text{ }${}file,$\text{ }${}and$\text{ }${}you$\text{ }${}use$\text{ }${}\textbackslash{}include\{INP-{}00-{}glossary\}$\text{ }$\newline{}
$\text{ }${}\textbackslash{}makeglossaries$\text{ }$\newline{}
$\text{ }${}\textbackslash{}usepackage{$\text{[}$}xindy{$\text{]}$}\{imakeidx\}$\text{ }$\newline{}
$\text{ }${}\textbackslash{}makeindex}

\section{Glossary definitions}
\label{663}
Write all your glossaries/acronyms in a file: Ex: {\bfseries \setmainfont[Path=/usr/share/fonts/truetype/cmu/,UprightFont=cmunrm.ttf,BoldFont=cmunbx.ttf,ItalicFont=cmunti.ttf,BoldItalicFont=cmunbi.ttf]{cmunbx.ttf}\setmonofont[Path=/usr/share/fonts/truetype/cmu/,UprightFont=cmuntt.ttf,BoldFont=cmuntb.ttf,ItalicFont=cmunit.ttf,BoldItalicFont=cmuntx.ttf]{cmunbx.ttf}\bfseries INP-{}00-{}glossary.tex}\setmainfont[Path=/usr/share/fonts/truetype/cmu/,UprightFont=cmunrm.ttf,BoldFont=cmunbx.ttf,ItalicFont=cmunti.ttf,BoldItalicFont=cmunbi.ttf]{cmunrm.ttf}\setmonofont[Path=/usr/share/fonts/truetype/cmu/,UprightFont=cmuntt.ttf,BoldFont=cmuntb.ttf,ItalicFont=cmunit.ttf,BoldItalicFont=cmuntx.ttf]{cmunrm.ttf}
 \textbackslash{}newacronym\{ddye\}\{D\${}\_\{\textbackslash{}text\{dye\}\}\${}\}\{donor dye, ex. Alexa 488\}
 \textbackslash{}newacronym{$\text{[}$}description=\{\textbackslash{}glslink\{r0\}\{F\textbackslash{}\symbol{34}\{o\}rster distance\}\}{$\text{]}$}\{R0\}\{\${}R\_\{0\}\${}\}\{F\textbackslash{}\symbol{34}\{o\}rster distance\}
 \textbackslash{}newglossaryentry\{r0\}\{name=\textbackslash{}glslink\{R0\}\{\textbackslash{}ensuremath\{R\_\{0\}\}\},text=F\textbackslash{}\symbol{34}\{o\}rster distance,description=\{F\textbackslash{}\symbol{34}\{o\}rster distance, where 50\textbackslash{}\% ...\}, sort=R\}
 \textbackslash{}newglossaryentry\{kdeac\}\{name=\textbackslash{}glslink\{R0\}\{\textbackslash{}ensuremath\{k\_\{DEAC\}\}\},text=\${}k\_\{DEAC\}\${}, description=\{is the rate of deactivation from ... and emission)\}, sort=k\}
\section{Include glossary definitions and print glossary}
\label{664}
Include glossary definitions in the preamble (Before \symbol{34}\textbackslash{}begin\{document\}\symbol{34}) \\

\TemplateSpaceIndent{$\text{ }${}{\bfseries \setmainfont[Path=/usr/share/fonts/truetype/cmu/,UprightFont=cmunrm.ttf,BoldFont=cmunbx.ttf,ItalicFont=cmunti.ttf,BoldItalicFont=cmunbi.ttf]{cmunbx.ttf}\setmonofont[Path=/usr/share/fonts/truetype/cmu/,UprightFont=cmuntt.ttf,BoldFont=cmuntb.ttf,ItalicFont=cmunit.ttf,BoldItalicFont=cmuntx.ttf]{cmunbx.ttf}\bfseries \textbackslash{}loadglsentries{$\text{[}$}main{$\text{]}$}\{INP-{}00-{}glossary\}}$\text{ }$\newline{}
$\text{ }${}\setmainfont[Path=/usr/share/fonts/truetype/cmu/,UprightFont=cmunrm.ttf,BoldFont=cmunbx.ttf,ItalicFont=cmunti.ttf,BoldItalicFont=cmunbi.ttf]{cmunrm.ttf}\setmonofont[Path=/usr/share/fonts/truetype/cmu/,UprightFont=cmuntt.ttf,BoldFont=cmuntb.ttf,ItalicFont=cmunit.ttf,BoldItalicFont=cmuntx.ttf]{cmunrm.ttf}\%$\text{ }${}or$\text{ }${}using$\text{ }${}\textbackslash{}input:$\text{ }$\newline{}
$\text{ }${}\%{\bfseries \setmainfont[Path=/usr/share/fonts/truetype/cmu/,UprightFont=cmunrm.ttf,BoldFont=cmunbx.ttf,ItalicFont=cmunti.ttf,BoldItalicFont=cmunbi.ttf]{cmunbx.ttf}\setmonofont[Path=/usr/share/fonts/truetype/cmu/,UprightFont=cmuntt.ttf,BoldFont=cmuntb.ttf,ItalicFont=cmunit.ttf,BoldItalicFont=cmuntx.ttf]{cmunbx.ttf}\bfseries \textbackslash{}input\{INP-{}00-{}glossary\}}$\text{ }$\newline{}
$\text{ }${}$\text{ }$\newline{}
$\text{ }${}\setmainfont[Path=/usr/share/fonts/truetype/cmu/,UprightFont=cmunrm.ttf,BoldFont=cmunbx.ttf,ItalicFont=cmunti.ttf,BoldItalicFont=cmunbi.ttf]{cmunrm.ttf}\setmonofont[Path=/usr/share/fonts/truetype/cmu/,UprightFont=cmuntt.ttf,BoldFont=cmuntb.ttf,ItalicFont=cmunit.ttf,BoldItalicFont=cmuntx.ttf]{cmunrm.ttf}\textbackslash{}begin\{document\}}


Print glossaries, near end\\

\TemplateSpaceIndent{$\text{ }${}\textbackslash{}appendix$\text{ }$\newline{}
$\text{ }${}\textbackslash{}bibliographystyle\{plainnat\}$\text{ }$\newline{}
$\text{ }${}\textbackslash{}bibliography\{bibtex\}$\text{ }$\newline{}
$\text{ }${}\textbackslash{}printindex$\text{ }$\newline{}
$\text{ }${}{\bfseries \setmainfont[Path=/usr/share/fonts/truetype/cmu/,UprightFont=cmunrm.ttf,BoldFont=cmunbx.ttf,ItalicFont=cmunti.ttf,BoldItalicFont=cmunbi.ttf]{cmunbx.ttf}\setmonofont[Path=/usr/share/fonts/truetype/cmu/,UprightFont=cmuntt.ttf,BoldFont=cmuntb.ttf,ItalicFont=cmunit.ttf,BoldItalicFont=cmuntx.ttf]{cmunbx.ttf}\bfseries \textbackslash{}printglossaries}$\text{ }$\newline{}
$\text{ }${}\setmainfont[Path=/usr/share/fonts/truetype/cmu/,UprightFont=cmunrm.ttf,BoldFont=cmunbx.ttf,ItalicFont=cmunti.ttf,BoldItalicFont=cmunbi.ttf]{cmunrm.ttf}\setmonofont[Path=/usr/share/fonts/truetype/cmu/,UprightFont=cmuntt.ttf,BoldFont=cmuntb.ttf,ItalicFont=cmunit.ttf,BoldItalicFont=cmuntx.ttf]{cmunrm.ttf}\textbackslash{}end\{document\}}

\section{References}
\label{665}
\LaTeXNullTemplate{}
\begin{myitemize}
\item{}  \newline
 \quad {\scshape The \LaTeXTT{glossaries} documentation, \myplainurl{http://tug.ctan.org/tex-archive/macros/latex/contrib/glossaries/}}
\end{myitemize}


\begin{myitemize}
\item{}  \newline
 \quad {\scshape {\itshape \setmainfont[Path=/usr/share/fonts/truetype/cmu/,UprightFont=cmunrm.ttf,BoldFont=cmunbx.ttf,ItalicFont=cmunti.ttf,BoldItalicFont=cmunbi.ttf]{cmunti.ttf}\setmonofont[Path=/usr/share/fonts/truetype/cmu/,UprightFont=cmuntt.ttf,BoldFont=cmuntb.ttf,ItalicFont=cmunit.ttf,BoldItalicFont=cmuntx.ttf]{cmunti.ttf}\itshape Using LaTeX to Write a PhD Thesis}\setmainfont[Path=/usr/share/fonts/truetype/cmu/,UprightFont=cmunrm.ttf,BoldFont=cmunbx.ttf,ItalicFont=cmunti.ttf,BoldItalicFont=cmunbi.ttf]{cmunrm.ttf}\setmonofont[Path=/usr/share/fonts/truetype/cmu/,UprightFont=cmuntt.ttf,BoldFont=cmuntb.ttf,ItalicFont=cmunit.ttf,BoldItalicFont=cmuntx.ttf]{cmunrm.ttf}, Nicola L.C. Talbot, \myplainurl{http://theoval.cmp.uea.ac.uk/~nlct/latex/thesis/node25.html}}
\end{myitemize}


\begin{myitemize}
\item{}  \newline
 \quad {\scshape {\itshape \setmainfont[Path=/usr/share/fonts/truetype/cmu/,UprightFont=cmunrm.ttf,BoldFont=cmunbx.ttf,ItalicFont=cmunti.ttf,BoldItalicFont=cmunbi.ttf]{cmunti.ttf}\setmonofont[Path=/usr/share/fonts/truetype/cmu/,UprightFont=cmuntt.ttf,BoldFont=cmuntb.ttf,ItalicFont=cmunit.ttf,BoldItalicFont=cmuntx.ttf]{cmunti.ttf}\itshape glossaries FAQ}\setmainfont[Path=/usr/share/fonts/truetype/cmu/,UprightFont=cmunrm.ttf,BoldFont=cmunbx.ttf,ItalicFont=cmunti.ttf,BoldItalicFont=cmunbi.ttf]{cmunrm.ttf}\setmonofont[Path=/usr/share/fonts/truetype/cmu/,UprightFont=cmuntt.ttf,BoldFont=cmuntb.ttf,ItalicFont=cmunit.ttf,BoldItalicFont=cmuntx.ttf]{cmunrm.ttf}, Nicola L. C. Talbot, \myhref{http://www.dickimaw-books.com/faqs/glossariesfaq.html}{glossaries FAQ}}
\end{myitemize}


\begin{myitemize}
\item{}  \newline
 \quad {\scshape {\itshape \setmainfont[Path=/usr/share/fonts/truetype/cmu/,UprightFont=cmunrm.ttf,BoldFont=cmunbx.ttf,ItalicFont=cmunti.ttf,BoldItalicFont=cmunbi.ttf]{cmunti.ttf}\setmonofont[Path=/usr/share/fonts/truetype/cmu/,UprightFont=cmuntt.ttf,BoldFont=cmuntb.ttf,ItalicFont=cmunit.ttf,BoldItalicFont=cmuntx.ttf]{cmunti.ttf}\itshape Glossaries, Nomenclature, Lists of Symbols and Acronyms}\setmainfont[Path=/usr/share/fonts/truetype/cmu/,UprightFont=cmunrm.ttf,BoldFont=cmunbx.ttf,ItalicFont=cmunti.ttf,BoldItalicFont=cmunbi.ttf]{cmunrm.ttf}\setmonofont[Path=/usr/share/fonts/truetype/cmu/,UprightFont=cmuntt.ttf,BoldFont=cmuntb.ttf,ItalicFont=cmunit.ttf,BoldItalicFont=cmuntx.ttf]{cmunrm.ttf}, Nicola L. C. Talbot, \myhref{http://www.latex-community.org/know-how/263-glossaries-nomenclature-lists-of-symbols-and-acronyms}{link}}
\end{myitemize}



\begin{myquote}
\item{}
\end{myquote}

\LaTeXNullTemplate{}
\LaTeXNullTemplate{}



\myhref{https://sr.wikibooks.org/wiki/LaTeX\%2F\%D0\%A0\%D0\%B5\%D1\%87\%D0\%BD\%D0\%B8\%D0\%BA}{sr:LaTeX/Речник}\chapter{Bibliography Management}

\myminitoc
\label{666}

\label{667}




UNKNOWN TEMPLATE  
outdated

{}

 


UNKNOWN TEMPLATE  
rewrite

{}



For any academic/research writing, incorporating references into a document is an important task.  Fortunately, LaTeX has a variety of features that make dealing with references much simpler, including built-{}in support for citing references.  However, a much more powerful and flexible solution is achieved thanks to an auxiliary tool called \myhref{http://www.bibtex.org}{BibTeX} (which comes bundled as standard with LaTeX).  Recently, BibTeX has been succeeded by BibLaTeX, a tool configurable within LaTeX syntax.

BibTeX provides for the storage of all references in an external, flat-{}file database.  (BibLaTeX uses this same syntax.)  This database can be referenced in any LaTeX document, and citations made to any record that is contained within the file.  This is often more convenient than embedding them at the end of every document written; a centralized bibliography source can be linked to as many documents as desired (write once, read many!).  Of course, bibliographies can be split over as many files as one wishes, so there can be a file containing sources concerning topic A ({\ttfamily \setmainfont[Path=/usr/share/fonts/truetype/cmu/,UprightFont=cmunrm.ttf,BoldFont=cmunbx.ttf,ItalicFont=cmunti.ttf,BoldItalicFont=cmunbi.ttf]{cmuntt.ttf}\setmonofont[Path=/usr/share/fonts/truetype/cmu/,UprightFont=cmuntt.ttf,BoldFont=cmuntb.ttf,ItalicFont=cmunit.ttf,BoldItalicFont=cmuntx.ttf]{cmuntt.ttf}\ttfamily a.bib}\setmainfont[Path=/usr/share/fonts/truetype/cmu/,UprightFont=cmunrm.ttf,BoldFont=cmunbx.ttf,ItalicFont=cmunti.ttf,BoldItalicFont=cmunbi.ttf]{cmunrm.ttf}\setmonofont[Path=/usr/share/fonts/truetype/cmu/,UprightFont=cmuntt.ttf,BoldFont=cmuntb.ttf,ItalicFont=cmunit.ttf,BoldItalicFont=cmuntx.ttf]{cmunrm.ttf}) and another concerning topic B ({\ttfamily \setmainfont[Path=/usr/share/fonts/truetype/cmu/,UprightFont=cmunrm.ttf,BoldFont=cmunbx.ttf,ItalicFont=cmunti.ttf,BoldItalicFont=cmunbi.ttf]{cmuntt.ttf}\setmonofont[Path=/usr/share/fonts/truetype/cmu/,UprightFont=cmuntt.ttf,BoldFont=cmuntb.ttf,ItalicFont=cmunit.ttf,BoldItalicFont=cmuntx.ttf]{cmuntt.ttf}\ttfamily b.bib}\setmainfont[Path=/usr/share/fonts/truetype/cmu/,UprightFont=cmunrm.ttf,BoldFont=cmunbx.ttf,ItalicFont=cmunti.ttf,BoldItalicFont=cmunbi.ttf]{cmunrm.ttf}\setmonofont[Path=/usr/share/fonts/truetype/cmu/,UprightFont=cmuntt.ttf,BoldFont=cmuntb.ttf,ItalicFont=cmunit.ttf,BoldItalicFont=cmuntx.ttf]{cmunrm.ttf}).  When writing about topic AB, both of these files can be linked into the document (perhaps in addition to sources {\ttfamily \setmainfont[Path=/usr/share/fonts/truetype/cmu/,UprightFont=cmunrm.ttf,BoldFont=cmunbx.ttf,ItalicFont=cmunti.ttf,BoldItalicFont=cmunbi.ttf]{cmuntt.ttf}\setmonofont[Path=/usr/share/fonts/truetype/cmu/,UprightFont=cmuntt.ttf,BoldFont=cmuntb.ttf,ItalicFont=cmunit.ttf,BoldItalicFont=cmuntx.ttf]{cmuntt.ttf}\ttfamily ab.bib}{$\text{ }$}\setmainfont[Path=/usr/share/fonts/truetype/cmu/,UprightFont=cmunrm.ttf,BoldFont=cmunbx.ttf,ItalicFont=cmunti.ttf,BoldItalicFont=cmunbi.ttf]{cmunrm.ttf}\setmonofont[Path=/usr/share/fonts/truetype/cmu/,UprightFont=cmuntt.ttf,BoldFont=cmuntb.ttf,ItalicFont=cmunit.ttf,BoldItalicFont=cmuntx.ttf]{cmunrm.ttf} specific to topic AB).
\section{Embedded system}
\label{668}

If you are writing only one or two documents and aren\textquotesingle{}t planning on writing more on the same subject for a long time, you might not want to waste time creating a database of references you are never going to use. In this case you should consider using the basic and simple bibliography support that is embedded within LaTeX.

LaTeX provides an environment called {\ttfamily \setmainfont[Path=/usr/share/fonts/truetype/cmu/,UprightFont=cmunrm.ttf,BoldFont=cmunbx.ttf,ItalicFont=cmunti.ttf,BoldItalicFont=cmunbi.ttf]{cmuntt.ttf}\setmonofont[Path=/usr/share/fonts/truetype/cmu/,UprightFont=cmuntt.ttf,BoldFont=cmuntb.ttf,ItalicFont=cmunit.ttf,BoldItalicFont=cmuntx.ttf]{cmuntt.ttf}\ttfamily thebibliography}{$\text{ }$}\setmainfont[Path=/usr/share/fonts/truetype/cmu/,UprightFont=cmunrm.ttf,BoldFont=cmunbx.ttf,ItalicFont=cmunti.ttf,BoldItalicFont=cmunbi.ttf]{cmunrm.ttf}\setmonofont[Path=/usr/share/fonts/truetype/cmu/,UprightFont=cmuntt.ttf,BoldFont=cmuntb.ttf,ItalicFont=cmunit.ttf,BoldItalicFont=cmuntx.ttf]{cmunrm.ttf} that you have to use where you want the bibliography; that usually means at the very end of your document, just before the {\ttfamily \setmainfont[Path=/usr/share/fonts/truetype/cmu/,UprightFont=cmunrm.ttf,BoldFont=cmunbx.ttf,ItalicFont=cmunti.ttf,BoldItalicFont=cmunbi.ttf]{cmuntt.ttf}\setmonofont[Path=/usr/share/fonts/truetype/cmu/,UprightFont=cmuntt.ttf,BoldFont=cmuntb.ttf,ItalicFont=cmunit.ttf,BoldItalicFont=cmuntx.ttf]{cmuntt.ttf}\ttfamily \textbackslash{}end\{document\}}{$\text{ }$}\setmainfont[Path=/usr/share/fonts/truetype/cmu/,UprightFont=cmunrm.ttf,BoldFont=cmunbx.ttf,ItalicFont=cmunti.ttf,BoldItalicFont=cmunbi.ttf]{cmunrm.ttf}\setmonofont[Path=/usr/share/fonts/truetype/cmu/,UprightFont=cmuntt.ttf,BoldFont=cmuntb.ttf,ItalicFont=cmunit.ttf,BoldItalicFont=cmuntx.ttf]{cmunrm.ttf} command. Here is a practical example:


\begin{Shaded}
\begin{Highlighting}[]

\NormalTok{\textbackslash{}begin\{thebibliography\}\{9\}}\newline
\ensuremath{\text{ }}\newline
\NormalTok{\textbackslash{}bibitem\{lamport94\}}\newline
\ensuremath{\text{ }}\ensuremath{\text{ }}\NormalTok{Leslie\ensuremath{\text{ }}Lamport,}\newline
\ensuremath{\text{ }}\ensuremath{\text{ }}\NormalTok{\textbackslash{}emph\{\textbackslash{}LaTeX:\ensuremath{\text{ }}a\ensuremath{\text{ }}document\ensuremath{\text{ }}preparation\ensuremath{\text{ }}system\},}\newline
\ensuremath{\text{ }}\ensuremath{\text{ }}\NormalTok{Addison\ensuremath{\text{ }}Wesley,\ensuremath{\text{ }}Massachusetts,}\newline
\ensuremath{\text{ }}\ensuremath{\text{ }}\NormalTok{2nd\ensuremath{\text{ }}edition,}\newline
\ensuremath{\text{ }}\ensuremath{\text{ }}\NormalTok{1994.}\newline
\ensuremath{\text{ }}\newline
\NormalTok{\textbackslash{}end\{thebibliography\}}\newline
\end{Highlighting}
\end{Shaded}


OK, so what is going on here? The first thing to notice is the establishment of the environment. {\ttfamily \setmainfont[Path=/usr/share/fonts/truetype/cmu/,UprightFont=cmunrm.ttf,BoldFont=cmunbx.ttf,ItalicFont=cmunti.ttf,BoldItalicFont=cmunbi.ttf]{cmuntt.ttf}\setmonofont[Path=/usr/share/fonts/truetype/cmu/,UprightFont=cmuntt.ttf,BoldFont=cmuntb.ttf,ItalicFont=cmunit.ttf,BoldItalicFont=cmuntx.ttf]{cmuntt.ttf}\ttfamily thebibliography}{$\text{ }$}\setmainfont[Path=/usr/share/fonts/truetype/cmu/,UprightFont=cmunrm.ttf,BoldFont=cmunbx.ttf,ItalicFont=cmunti.ttf,BoldItalicFont=cmunbi.ttf]{cmunrm.ttf}\setmonofont[Path=/usr/share/fonts/truetype/cmu/,UprightFont=cmuntt.ttf,BoldFont=cmuntb.ttf,ItalicFont=cmunit.ttf,BoldItalicFont=cmuntx.ttf]{cmunrm.ttf} is a keyword that LaTeX recognizes as everything between the begin and end tags as being data for the bibliography. The mandatory argument, which I supplied after the begin statement, is telling LaTeX how wide the item label will be when printed. Note however, that the number itself is not the parameter, but the number of digits is. Therefore, I am effectively telling LaTeX that I will only need reference labels of one character in length, which ultimately means no more than nine references in total. If you want more than nine, then input any two-{}digit number, such as \textquotesingle{}56\textquotesingle{} which allows up to 99 references.

Next is the actual reference entry itself. This is prefixed with the {\ttfamily \setmainfont[Path=/usr/share/fonts/truetype/cmu/,UprightFont=cmunrm.ttf,BoldFont=cmunbx.ttf,ItalicFont=cmunti.ttf,BoldItalicFont=cmunbi.ttf]{cmuntt.ttf}\setmonofont[Path=/usr/share/fonts/truetype/cmu/,UprightFont=cmuntt.ttf,BoldFont=cmuntb.ttf,ItalicFont=cmunit.ttf,BoldItalicFont=cmuntx.ttf]{cmuntt.ttf}\ttfamily \textbackslash{}bibitem\{{\itshape \setmainfont[Path=/usr/share/fonts/truetype/cmu/,UprightFont=cmunrm.ttf,BoldFont=cmunbx.ttf,ItalicFont=cmunti.ttf,BoldItalicFont=cmunbi.ttf]{cmunit.ttf}\setmonofont[Path=/usr/share/fonts/truetype/cmu/,UprightFont=cmuntt.ttf,BoldFont=cmuntb.ttf,ItalicFont=cmunit.ttf,BoldItalicFont=cmuntx.ttf]{cmunit.ttf}\ttfamily \itshape cite\_key}\setmainfont[Path=/usr/share/fonts/truetype/cmu/,UprightFont=cmunrm.ttf,BoldFont=cmunbx.ttf,ItalicFont=cmunti.ttf,BoldItalicFont=cmunbi.ttf]{cmuntt.ttf}\setmonofont[Path=/usr/share/fonts/truetype/cmu/,UprightFont=cmuntt.ttf,BoldFont=cmuntb.ttf,ItalicFont=cmunit.ttf,BoldItalicFont=cmuntx.ttf]{cmuntt.ttf}\ttfamily \}}{$\text{ }$}\setmainfont[Path=/usr/share/fonts/truetype/cmu/,UprightFont=cmunrm.ttf,BoldFont=cmunbx.ttf,ItalicFont=cmunti.ttf,BoldItalicFont=cmunbi.ttf]{cmunrm.ttf}\setmonofont[Path=/usr/share/fonts/truetype/cmu/,UprightFont=cmuntt.ttf,BoldFont=cmuntb.ttf,ItalicFont=cmunit.ttf,BoldItalicFont=cmuntx.ttf]{cmunrm.ttf} command. The {\itshape \setmainfont[Path=/usr/share/fonts/truetype/cmu/,UprightFont=cmunrm.ttf,BoldFont=cmunbx.ttf,ItalicFont=cmunti.ttf,BoldItalicFont=cmunbi.ttf]{cmunti.ttf}\setmonofont[Path=/usr/share/fonts/truetype/cmu/,UprightFont=cmuntt.ttf,BoldFont=cmuntb.ttf,ItalicFont=cmunit.ttf,BoldItalicFont=cmuntx.ttf]{cmunti.ttf}\itshape cite\_key}{$\text{ }$}\setmainfont[Path=/usr/share/fonts/truetype/cmu/,UprightFont=cmunrm.ttf,BoldFont=cmunbx.ttf,ItalicFont=cmunti.ttf,BoldItalicFont=cmunbi.ttf]{cmunrm.ttf}\setmonofont[Path=/usr/share/fonts/truetype/cmu/,UprightFont=cmuntt.ttf,BoldFont=cmuntb.ttf,ItalicFont=cmunit.ttf,BoldItalicFont=cmuntx.ttf]{cmunrm.ttf} should be a unique identifier for that particular reference, and is often some sort of mnemonic consisting of any sequence of letters, numbers and punctuation symbols (although not a comma). I often use the surname of the first author, followed by the last two digits of the year (hence {\itshape \setmainfont[Path=/usr/share/fonts/truetype/cmu/,UprightFont=cmunrm.ttf,BoldFont=cmunbx.ttf,ItalicFont=cmunti.ttf,BoldItalicFont=cmunbi.ttf]{cmunti.ttf}\setmonofont[Path=/usr/share/fonts/truetype/cmu/,UprightFont=cmuntt.ttf,BoldFont=cmuntb.ttf,ItalicFont=cmunit.ttf,BoldItalicFont=cmuntx.ttf]{cmunti.ttf}\itshape lamport94}\setmainfont[Path=/usr/share/fonts/truetype/cmu/,UprightFont=cmunrm.ttf,BoldFont=cmunbx.ttf,ItalicFont=cmunti.ttf,BoldItalicFont=cmunbi.ttf]{cmunrm.ttf}\setmonofont[Path=/usr/share/fonts/truetype/cmu/,UprightFont=cmuntt.ttf,BoldFont=cmuntb.ttf,ItalicFont=cmunit.ttf,BoldItalicFont=cmuntx.ttf]{cmunrm.ttf}). If that author has produced more than one reference for a given year, then I add letters after, \textquotesingle{}a\textquotesingle{}, \textquotesingle{}b\textquotesingle{}, etc. But, you should do whatever works for you. Everything after the key is the reference itself. You need to type it as you want it to be presented. I have put the different parts of the reference, such as author, title, etc., on different lines for readability. These linebreaks are ignored by LaTeX. I wanted the title to be in italics, so I used the {\ttfamily \setmainfont[Path=/usr/share/fonts/truetype/cmu/,UprightFont=cmunrm.ttf,BoldFont=cmunbx.ttf,ItalicFont=cmunti.ttf,BoldItalicFont=cmunbi.ttf]{cmuntt.ttf}\setmonofont[Path=/usr/share/fonts/truetype/cmu/,UprightFont=cmuntt.ttf,BoldFont=cmuntb.ttf,ItalicFont=cmunit.ttf,BoldItalicFont=cmuntx.ttf]{cmuntt.ttf}\ttfamily \textbackslash{}emph\{\}}{$\text{ }$}\setmainfont[Path=/usr/share/fonts/truetype/cmu/,UprightFont=cmunrm.ttf,BoldFont=cmunbx.ttf,ItalicFont=cmunti.ttf,BoldItalicFont=cmunbi.ttf]{cmunrm.ttf}\setmonofont[Path=/usr/share/fonts/truetype/cmu/,UprightFont=cmuntt.ttf,BoldFont=cmuntb.ttf,ItalicFont=cmunit.ttf,BoldItalicFont=cmuntx.ttf]{cmunrm.ttf} command to achieve this.
\section{Citations}
\label{669}
To actually cite a given document is very easy. Go to the point where you want the citation to appear, and use the following: {\ttfamily \setmainfont[Path=/usr/share/fonts/truetype/cmu/,UprightFont=cmunrm.ttf,BoldFont=cmunbx.ttf,ItalicFont=cmunti.ttf,BoldItalicFont=cmunbi.ttf]{cmuntt.ttf}\setmonofont[Path=/usr/share/fonts/truetype/cmu/,UprightFont=cmuntt.ttf,BoldFont=cmuntb.ttf,ItalicFont=cmunit.ttf,BoldItalicFont=cmuntx.ttf]{cmuntt.ttf}\ttfamily \textbackslash{}cite\{{\itshape \setmainfont[Path=/usr/share/fonts/truetype/cmu/,UprightFont=cmunrm.ttf,BoldFont=cmunbx.ttf,ItalicFont=cmunti.ttf,BoldItalicFont=cmunbi.ttf]{cmunit.ttf}\setmonofont[Path=/usr/share/fonts/truetype/cmu/,UprightFont=cmuntt.ttf,BoldFont=cmuntb.ttf,ItalicFont=cmunit.ttf,BoldItalicFont=cmuntx.ttf]{cmunit.ttf}\ttfamily \itshape cite\_key}\setmainfont[Path=/usr/share/fonts/truetype/cmu/,UprightFont=cmunrm.ttf,BoldFont=cmunbx.ttf,ItalicFont=cmunti.ttf,BoldItalicFont=cmunbi.ttf]{cmuntt.ttf}\setmonofont[Path=/usr/share/fonts/truetype/cmu/,UprightFont=cmuntt.ttf,BoldFont=cmuntb.ttf,ItalicFont=cmunit.ttf,BoldItalicFont=cmuntx.ttf]{cmuntt.ttf}\ttfamily \}}\setmainfont[Path=/usr/share/fonts/truetype/cmu/,UprightFont=cmunrm.ttf,BoldFont=cmunbx.ttf,ItalicFont=cmunti.ttf,BoldItalicFont=cmunbi.ttf]{cmunrm.ttf}\setmonofont[Path=/usr/share/fonts/truetype/cmu/,UprightFont=cmuntt.ttf,BoldFont=cmuntb.ttf,ItalicFont=cmunit.ttf,BoldItalicFont=cmuntx.ttf]{cmunrm.ttf}, where the {\itshape \setmainfont[Path=/usr/share/fonts/truetype/cmu/,UprightFont=cmunrm.ttf,BoldFont=cmunbx.ttf,ItalicFont=cmunti.ttf,BoldItalicFont=cmunbi.ttf]{cmunti.ttf}\setmonofont[Path=/usr/share/fonts/truetype/cmu/,UprightFont=cmuntt.ttf,BoldFont=cmuntb.ttf,ItalicFont=cmunit.ttf,BoldItalicFont=cmuntx.ttf]{cmunti.ttf}\itshape cite\_key}{$\text{ }$}\setmainfont[Path=/usr/share/fonts/truetype/cmu/,UprightFont=cmunrm.ttf,BoldFont=cmunbx.ttf,ItalicFont=cmunti.ttf,BoldItalicFont=cmunbi.ttf]{cmunrm.ttf}\setmonofont[Path=/usr/share/fonts/truetype/cmu/,UprightFont=cmuntt.ttf,BoldFont=cmuntb.ttf,ItalicFont=cmunit.ttf,BoldItalicFont=cmuntx.ttf]{cmunrm.ttf} is that of the bibitem you wish to cite. When LaTeX processes the document, the citation will be cross-{}referenced with the bibitems and replaced with the appropriate number citation. The advantage here, once again, is that LaTeX looks after the numbering for you. If it were totally manual, then adding or removing a reference would be a real chore, as you would have to re-{}number all the citations by hand.


\begin{Shaded}
\begin{Highlighting}[]

\NormalTok{Instead\ensuremath{\text{ }}of\ensuremath{\text{ }}WYSIWYG\ensuremath{\text{ }}editors,\ensuremath{\text{ }}typesetting\ensuremath{\text{ }}systems\ensuremath{\text{ }}like\ensuremath{\text{ }}\textbackslash{}TeX\{\}\ensuremath{\text{ }}or\ensuremath{\text{ }}\textbackslash{}LaTeX\{\}}\newline
\ensuremath{\text{ }}\NormalTok{\textbackslash{}cite\{lamport94\}\ensuremath{\text{ }}can\ensuremath{\text{ }}be\ensuremath{\text{ }}used.}\newline
\end{Highlighting}
\end{Shaded}

\subsection{Referring more specifically}
\label{670}
Sometimes you want to refer to a certain page, figure or theorem in a text book. For that you can use the arguments to the {\ttfamily \setmainfont[Path=/usr/share/fonts/truetype/cmu/,UprightFont=cmunrm.ttf,BoldFont=cmunbx.ttf,ItalicFont=cmunti.ttf,BoldItalicFont=cmunbi.ttf]{cmuntt.ttf}\setmonofont[Path=/usr/share/fonts/truetype/cmu/,UprightFont=cmuntt.ttf,BoldFont=cmuntb.ttf,ItalicFont=cmunit.ttf,BoldItalicFont=cmuntx.ttf]{cmuntt.ttf}\ttfamily \textbackslash{}cite}{$\text{ }$}\setmainfont[Path=/usr/share/fonts/truetype/cmu/,UprightFont=cmunrm.ttf,BoldFont=cmunbx.ttf,ItalicFont=cmunti.ttf,BoldItalicFont=cmunbi.ttf]{cmunrm.ttf}\setmonofont[Path=/usr/share/fonts/truetype/cmu/,UprightFont=cmuntt.ttf,BoldFont=cmuntb.ttf,ItalicFont=cmunit.ttf,BoldItalicFont=cmuntx.ttf]{cmunrm.ttf} command:


\begin{Shaded}
\begin{Highlighting}[]

\NormalTok{\textbackslash{}cite[chapter,\ensuremath{\text{ }}p.~215]\{citation01\}}\newline
\end{Highlighting}
\end{Shaded}


The argument, \symbol{34}p. 215\symbol{34}, will show up inside the same brackets. Note the tilde in {$\text{[}$}p.\~{}215{$\text{]}$}, which replaces the end-{}of-{}sentence spacing with a non-{}breakable inter-{}word space. There are two reasons: end-{}of-{}sentence spacing is too wide, and \symbol{34}p.\symbol{34} should not be separated from the page number.
\subsection{Multiple citations}
\label{671}

When a sequence of multiple citations are needed, you should use a single {\ttfamily \setmainfont[Path=/usr/share/fonts/truetype/cmu/,UprightFont=cmunrm.ttf,BoldFont=cmunbx.ttf,ItalicFont=cmunti.ttf,BoldItalicFont=cmunbi.ttf]{cmuntt.ttf}\setmonofont[Path=/usr/share/fonts/truetype/cmu/,UprightFont=cmuntt.ttf,BoldFont=cmuntb.ttf,ItalicFont=cmunit.ttf,BoldItalicFont=cmuntx.ttf]{cmuntt.ttf}\ttfamily \textbackslash{}cite\{\}}{$\text{ }$}\setmainfont[Path=/usr/share/fonts/truetype/cmu/,UprightFont=cmunrm.ttf,BoldFont=cmunbx.ttf,ItalicFont=cmunti.ttf,BoldItalicFont=cmunbi.ttf]{cmunrm.ttf}\setmonofont[Path=/usr/share/fonts/truetype/cmu/,UprightFont=cmuntt.ttf,BoldFont=cmuntb.ttf,ItalicFont=cmunit.ttf,BoldItalicFont=cmuntx.ttf]{cmunrm.ttf} command. The citations are then separated by commas. Here\textquotesingle{}s an example: 


\begin{Shaded}
\begin{Highlighting}[]

\NormalTok{\textbackslash{}cite\{citation01,citation02,citation03\}}\newline
\end{Highlighting}
\end{Shaded}


The result will then be shown as citations inside the same brackets, depending on the citation style.
\subsection{Bibliography styles}
\label{672}

There are several different ways to format lists of bibliographic references and the citations to them in the text. These are called \myhref{https://en.wikipedia.org/wiki/Citation\%23Styles}{citation styles}, and consist of two parts: the format of the abbreviated citation (i.e. the marker that is inserted into the text to identify the entry in the list of references) and the format of the corresponding entry in the list of references, which includes full bibliographic details.

Abbreviated citations can be of two main types: numbered or textual. Numbered citations (also known as the \myhref{https://en.wikipedia.org/wiki/Vancouver\%20system}{Vancouver referencing system}) are numbered consecutively in order of appearance in the text, and consist in Arabic numerals in parentheses {\bfseries \setmainfont[Path=/usr/share/fonts/truetype/cmu/,UprightFont=cmunrm.ttf,BoldFont=cmunbx.ttf,ItalicFont=cmunti.ttf,BoldItalicFont=cmunbi.ttf]{cmunbx.ttf}\setmonofont[Path=/usr/share/fonts/truetype/cmu/,UprightFont=cmuntt.ttf,BoldFont=cmuntb.ttf,ItalicFont=cmunit.ttf,BoldItalicFont=cmuntx.ttf]{cmunbx.ttf}\bfseries (1)}\setmainfont[Path=/usr/share/fonts/truetype/cmu/,UprightFont=cmunrm.ttf,BoldFont=cmunbx.ttf,ItalicFont=cmunti.ttf,BoldItalicFont=cmunbi.ttf]{cmunrm.ttf}\setmonofont[Path=/usr/share/fonts/truetype/cmu/,UprightFont=cmuntt.ttf,BoldFont=cmuntb.ttf,ItalicFont=cmunit.ttf,BoldItalicFont=cmuntx.ttf]{cmunrm.ttf}, square brackets {\bfseries \setmainfont[Path=/usr/share/fonts/truetype/cmu/,UprightFont=cmunrm.ttf,BoldFont=cmunbx.ttf,ItalicFont=cmunti.ttf,BoldItalicFont=cmunbi.ttf]{cmunbx.ttf}\setmonofont[Path=/usr/share/fonts/truetype/cmu/,UprightFont=cmuntt.ttf,BoldFont=cmuntb.ttf,ItalicFont=cmunit.ttf,BoldItalicFont=cmuntx.ttf]{cmunbx.ttf}\bfseries {$\text{[}$}1{$\text{]}$}}\setmainfont[Path=/usr/share/fonts/truetype/cmu/,UprightFont=cmunrm.ttf,BoldFont=cmunbx.ttf,ItalicFont=cmunti.ttf,BoldItalicFont=cmunbi.ttf]{cmunrm.ttf}\setmonofont[Path=/usr/share/fonts/truetype/cmu/,UprightFont=cmuntt.ttf,BoldFont=cmuntb.ttf,ItalicFont=cmunit.ttf,BoldItalicFont=cmuntx.ttf]{cmunrm.ttf}, superscript{\bfseries \setmainfont[Path=/usr/share/fonts/truetype/cmu/,UprightFont=cmunrm.ttf,BoldFont=cmunbx.ttf,ItalicFont=cmunti.ttf,BoldItalicFont=cmunbi.ttf]{cmunrm.ttf}\setmonofont[Path=/usr/share/fonts/truetype/cmu/,UprightFont=cmuntt.ttf,BoldFont=cmuntb.ttf,ItalicFont=cmunit.ttf,BoldItalicFont=cmuntx.ttf]{cmunrm.ttf}\textsuperscript{\setmainfont[Path=/usr/share/fonts/truetype/cmu/,UprightFont=cmunrm.ttf,BoldFont=cmunbx.ttf,ItalicFont=cmunti.ttf,BoldItalicFont=cmunbi.ttf]{cmunbx.ttf}\setmonofont[Path=/usr/share/fonts/truetype/cmu/,UprightFont=cmuntt.ttf,BoldFont=cmuntb.ttf,ItalicFont=cmunit.ttf,BoldItalicFont=cmuntx.ttf]{cmunbx.ttf}\bfseries 1}}\setmainfont[Path=/usr/share/fonts/truetype/cmu/,UprightFont=cmunrm.ttf,BoldFont=cmunbx.ttf,ItalicFont=cmunti.ttf,BoldItalicFont=cmunbi.ttf]{cmunrm.ttf}\setmonofont[Path=/usr/share/fonts/truetype/cmu/,UprightFont=cmuntt.ttf,BoldFont=cmuntb.ttf,ItalicFont=cmunit.ttf,BoldItalicFont=cmuntx.ttf]{cmunrm.ttf}, or a combination thereof{\bfseries \setmainfont[Path=/usr/share/fonts/truetype/cmu/,UprightFont=cmunrm.ttf,BoldFont=cmunbx.ttf,ItalicFont=cmunti.ttf,BoldItalicFont=cmunbi.ttf]{cmunrm.ttf}\setmonofont[Path=/usr/share/fonts/truetype/cmu/,UprightFont=cmuntt.ttf,BoldFont=cmuntb.ttf,ItalicFont=cmunit.ttf,BoldItalicFont=cmuntx.ttf]{cmunrm.ttf}\textsuperscript{\setmainfont[Path=/usr/share/fonts/truetype/cmu/,UprightFont=cmunrm.ttf,BoldFont=cmunbx.ttf,ItalicFont=cmunti.ttf,BoldItalicFont=cmunbi.ttf]{cmunbx.ttf}\setmonofont[Path=/usr/share/fonts/truetype/cmu/,UprightFont=cmuntt.ttf,BoldFont=cmuntb.ttf,ItalicFont=cmunit.ttf,BoldItalicFont=cmuntx.ttf]{cmunbx.ttf}\bfseries {$\text{[}$}1{$\text{]}$}}}\setmainfont[Path=/usr/share/fonts/truetype/cmu/,UprightFont=cmunrm.ttf,BoldFont=cmunbx.ttf,ItalicFont=cmunti.ttf,BoldItalicFont=cmunbi.ttf]{cmunrm.ttf}\setmonofont[Path=/usr/share/fonts/truetype/cmu/,UprightFont=cmuntt.ttf,BoldFont=cmuntb.ttf,ItalicFont=cmunit.ttf,BoldItalicFont=cmuntx.ttf]{cmunrm.ttf}. Textual citations (also known as the \myhref{https://en.wikipedia.org/wiki/Parenthetical\%20referencing}{Harvard referencing system}) use the author surname and (usually) the year as the abbreviated form of the citation, which is normally fully {\bfseries \setmainfont[Path=/usr/share/fonts/truetype/cmu/,UprightFont=cmunrm.ttf,BoldFont=cmunbx.ttf,ItalicFont=cmunti.ttf,BoldItalicFont=cmunbi.ttf]{cmunbx.ttf}\setmonofont[Path=/usr/share/fonts/truetype/cmu/,UprightFont=cmuntt.ttf,BoldFont=cmuntb.ttf,ItalicFont=cmunit.ttf,BoldItalicFont=cmuntx.ttf]{cmunbx.ttf}\bfseries (Smith 

UNKNOWN TEMPLATE  
#time:Y

{ -{}10 years}

)}{$\text{ }$}\setmainfont[Path=/usr/share/fonts/truetype/cmu/,UprightFont=cmunrm.ttf,BoldFont=cmunbx.ttf,ItalicFont=cmunti.ttf,BoldItalicFont=cmunbi.ttf]{cmunrm.ttf}\setmonofont[Path=/usr/share/fonts/truetype/cmu/,UprightFont=cmuntt.ttf,BoldFont=cmuntb.ttf,ItalicFont=cmunit.ttf,BoldItalicFont=cmuntx.ttf]{cmunrm.ttf} or partially enclosed in parenthesis, as in {\bfseries \setmainfont[Path=/usr/share/fonts/truetype/cmu/,UprightFont=cmunrm.ttf,BoldFont=cmunbx.ttf,ItalicFont=cmunti.ttf,BoldItalicFont=cmunbi.ttf]{cmunbx.ttf}\setmonofont[Path=/usr/share/fonts/truetype/cmu/,UprightFont=cmuntt.ttf,BoldFont=cmuntb.ttf,ItalicFont=cmunit.ttf,BoldItalicFont=cmuntx.ttf]{cmunbx.ttf}\bfseries Smith (

UNKNOWN TEMPLATE  
#time:Y

{ -{}10 years}

)}\setmainfont[Path=/usr/share/fonts/truetype/cmu/,UprightFont=cmunrm.ttf,BoldFont=cmunbx.ttf,ItalicFont=cmunti.ttf,BoldItalicFont=cmunbi.ttf]{cmunrm.ttf}\setmonofont[Path=/usr/share/fonts/truetype/cmu/,UprightFont=cmuntt.ttf,BoldFont=cmuntb.ttf,ItalicFont=cmunit.ttf,BoldItalicFont=cmuntx.ttf]{cmunrm.ttf}. The latter form allows the citation to be integrated in the sentence it supports.



Below you can see three of the styles available with LaTeX:


\begin{minipage}{1.0\linewidth}
\begin{center}
\includegraphics[width=1.0\linewidth,height=6.5in,keepaspectratio]{../images/148.\SVGExtension}
\end{center}
\raggedright{}\myfigurewithcaption{148}{plain}
\end{minipage}\vspace{0.75cm}




\begin{minipage}{1.0\linewidth}
\begin{center}
\includegraphics[width=1.0\linewidth,height=6.5in,keepaspectratio]{../images/149.\SVGExtension}
\end{center}
\raggedright{}\myfigurewithcaption{149}{abbrv}
\end{minipage}\vspace{0.75cm}




\begin{minipage}{1.0\linewidth}
\begin{center}
\includegraphics[width=1.0\linewidth,height=6.5in,keepaspectratio]{../images/150.\SVGExtension}
\end{center}
\raggedright{}\myfigurewithcaption{150}{alpha}
\end{minipage}\vspace{0.75cm}


\LaTeXNullTemplate{}

Here are some more often used styles:
\begin{longtable}{|>{\RaggedRight}p{0.16787\linewidth}|>{\RaggedRight}p{0.29272\linewidth}|>{\RaggedRight}p{0.24663\linewidth}|>{\RaggedRight}p{0.17850\linewidth}|} \hline 
{\bfseries \hspace*{0pt}\ignorespaces{}\hspace*{0pt} Style Name }&{\bfseries \hspace*{0pt}\ignorespaces{}\hspace*{0pt} Author Name Format }&{\bfseries \hspace*{0pt}\ignorespaces{}\hspace*{0pt} Reference Format }&{\bfseries \hspace*{0pt}\ignorespaces{}\hspace*{0pt} Sorting}\endhead  \hline \hspace*{0pt}\ignorespaces{}\hspace*{0pt} plain &\hspace*{0pt}\ignorespaces{}\hspace*{0pt} Homer Jay Simpson &\hspace*{0pt}\ignorespaces{}\hspace*{0pt} \#ID\# &\hspace*{0pt}\ignorespaces{}\hspace*{0pt} by author\\ \hline \hspace*{0pt}\ignorespaces{}\hspace*{0pt} unsrt &\hspace*{0pt}\ignorespaces{}\hspace*{0pt} Homer Jay Simpson &\hspace*{0pt}\ignorespaces{}\hspace*{0pt} \#ID\# &\hspace*{0pt}\ignorespaces{}\hspace*{0pt} as referenced\\ \hline \hspace*{0pt}\ignorespaces{}\hspace*{0pt} abbrv &\hspace*{0pt}\ignorespaces{}\hspace*{0pt} H. J. Simpson &\hspace*{0pt}\ignorespaces{}\hspace*{0pt} \#ID\# &\hspace*{0pt}\ignorespaces{}\hspace*{0pt} by author\\ \hline \hspace*{0pt}\ignorespaces{}\hspace*{0pt} alpha &\hspace*{0pt}\ignorespaces{}\hspace*{0pt} Homer Jay Simpson &\hspace*{0pt}\ignorespaces{}\hspace*{0pt} Sim95 &\hspace*{0pt}\ignorespaces{}\hspace*{0pt} by author\\ \hline \hspace*{0pt}\ignorespaces{}\hspace*{0pt} abstract &\hspace*{0pt}\ignorespaces{}\hspace*{0pt} Homer Jay Simpson &\hspace*{0pt}\ignorespaces{}\hspace*{0pt} Simpson-{}1995a &\hspace*{0pt}\ignorespaces{}\hspace*{0pt} \\ \hline \hspace*{0pt}\ignorespaces{}\hspace*{0pt} acm &\hspace*{0pt}\ignorespaces{}\hspace*{0pt} Simpson, H. J. &\hspace*{0pt}\ignorespaces{}\hspace*{0pt} \#ID\# &\hspace*{0pt}\ignorespaces{}\hspace*{0pt}\\ \hline \hspace*{0pt}\ignorespaces{}\hspace*{0pt} authordate1 &\hspace*{0pt}\ignorespaces{}\hspace*{0pt} Simpson, Homer Jay &\hspace*{0pt}\ignorespaces{}\hspace*{0pt} Simpson, 1995 &\hspace*{0pt}\ignorespaces{}\hspace*{0pt}\\ \hline \hspace*{0pt}\ignorespaces{}\hspace*{0pt} apacite &\hspace*{0pt}\ignorespaces{}\hspace*{0pt} Simpson, H. J. (1995) &\hspace*{0pt}\ignorespaces{}\hspace*{0pt} Simpson1995 &\hspace*{0pt}\ignorespaces{}\hspace*{0pt}\\ \hline \hspace*{0pt}\ignorespaces{}\hspace*{0pt} named &\hspace*{0pt}\ignorespaces{}\hspace*{0pt} Homer Jay Simpson &\hspace*{0pt}\ignorespaces{}\hspace*{0pt} Simpson 1995 &\hspace*{0pt}\ignorespaces{}\hspace*{0pt}\\ \hline 
\end{longtable}


However, keep in mind that you will need to use the natbib package to use most of these.
\subsection{No cite}
\label{673}

If you only want a reference to appear in the bibliography, but not where it is referenced in the main text, then the {\ttfamily \setmainfont[Path=/usr/share/fonts/truetype/cmu/,UprightFont=cmunrm.ttf,BoldFont=cmunbx.ttf,ItalicFont=cmunti.ttf,BoldItalicFont=cmunbi.ttf]{cmuntt.ttf}\setmonofont[Path=/usr/share/fonts/truetype/cmu/,UprightFont=cmuntt.ttf,BoldFont=cmuntb.ttf,ItalicFont=cmunit.ttf,BoldItalicFont=cmuntx.ttf]{cmuntt.ttf}\ttfamily \textbackslash{}nocite\{\}}{$\text{ }$}\setmainfont[Path=/usr/share/fonts/truetype/cmu/,UprightFont=cmunrm.ttf,BoldFont=cmunbx.ttf,ItalicFont=cmunti.ttf,BoldItalicFont=cmunbi.ttf]{cmunrm.ttf}\setmonofont[Path=/usr/share/fonts/truetype/cmu/,UprightFont=cmuntt.ttf,BoldFont=cmuntb.ttf,ItalicFont=cmunit.ttf,BoldItalicFont=cmuntx.ttf]{cmunrm.ttf} command can be used, for example:


\begin{Shaded}
\begin{Highlighting}[]

\NormalTok{Lamport\ensuremath{\text{ }}showed\ensuremath{\text{ }}in\ensuremath{\text{ }}1995\ensuremath{\text{ }}something...\ensuremath{\text{ }}\ensuremath{\text{ }}\textbackslash{}nocite\{lamport95\}.}\newline
\end{Highlighting}
\end{Shaded}


A special version of the command, {\ttfamily \setmainfont[Path=/usr/share/fonts/truetype/cmu/,UprightFont=cmunrm.ttf,BoldFont=cmunbx.ttf,ItalicFont=cmunti.ttf,BoldItalicFont=cmunbi.ttf]{cmuntt.ttf}\setmonofont[Path=/usr/share/fonts/truetype/cmu/,UprightFont=cmuntt.ttf,BoldFont=cmuntb.ttf,ItalicFont=cmunit.ttf,BoldItalicFont=cmuntx.ttf]{cmuntt.ttf}\ttfamily \textbackslash{}nocite\{*\}}\setmainfont[Path=/usr/share/fonts/truetype/cmu/,UprightFont=cmunrm.ttf,BoldFont=cmunbx.ttf,ItalicFont=cmunti.ttf,BoldItalicFont=cmunbi.ttf]{cmunrm.ttf}\setmonofont[Path=/usr/share/fonts/truetype/cmu/,UprightFont=cmuntt.ttf,BoldFont=cmuntb.ttf,ItalicFont=cmunit.ttf,BoldItalicFont=cmuntx.ttf]{cmunrm.ttf}, includes all entries from the database, whether they are referenced in the document or not.
\subsection{Natbib}
\label{674}
\begin{longtable}{|>{\RaggedRight}p{0.33001\linewidth}|>{\RaggedRight}p{0.61285\linewidth}|} \hline 
\multicolumn{2}{|>{\RaggedRight}p{0.97143\linewidth}|}{{\bfseries \hspace*{0pt}\ignorespaces{}\hspace*{0pt} Natbib\textquotesingle{}s textual and parenthetical commands}}\\ \hline {\bfseries \hspace*{0pt}\ignorespaces{}\hspace*{0pt} Citation command}&{\bfseries \hspace*{0pt}\ignorespaces{}\hspace*{0pt} Output}\endhead  \hline \hspace*{0pt}\ignorespaces{}\hspace*{0pt} {\ttfamily \setmainfont[Path=/usr/share/fonts/truetype/cmu/,UprightFont=cmunrm.ttf,BoldFont=cmunbx.ttf,ItalicFont=cmunti.ttf,BoldItalicFont=cmunbi.ttf]{cmuntt.ttf}\setmonofont[Path=/usr/share/fonts/truetype/cmu/,UprightFont=cmuntt.ttf,BoldFont=cmuntb.ttf,ItalicFont=cmunit.ttf,BoldItalicFont=cmuntx.ttf]{cmuntt.ttf}\ttfamily \textbackslash{}citet\{goossens93\}}\newline{}{\ttfamily \textbackslash{}citep\{goossens93\}}&\hspace*{0pt}\ignorespaces{}\hspace*{0pt}{$\text{ }$}\setmainfont[Path=/usr/share/fonts/truetype/cmu/,UprightFont=cmunrm.ttf,BoldFont=cmunbx.ttf,ItalicFont=cmunti.ttf,BoldItalicFont=cmunbi.ttf]{cmunrm.ttf}\setmonofont[Path=/usr/share/fonts/truetype/cmu/,UprightFont=cmuntt.ttf,BoldFont=cmuntb.ttf,ItalicFont=cmunit.ttf,BoldItalicFont=cmuntx.ttf]{cmunrm.ttf} Goossens et al. (1993)\newline{}{\bfseries UNKNOWN TEMPLATE  red{\setmainfont[Path=/usr/share/fonts/truetype/cmu/,UprightFont=cmunrm.ttf,BoldFont=cmunbx.ttf,ItalicFont=cmunti.ttf,BoldItalicFont=cmunbi.ttf]{cmunbx.ttf}\setmonofont[Path=/usr/share/fonts/truetype/cmu/,UprightFont=cmuntt.ttf,BoldFont=cmuntb.ttf,ItalicFont=cmunit.ttf,BoldItalicFont=cmuntx.ttf]{cmunbx.ttf}\bfseries (}}\setmainfont[Path=/usr/share/fonts/truetype/cmu/,UprightFont=cmunrm.ttf,BoldFont=cmunbx.ttf,ItalicFont=cmunti.ttf,BoldItalicFont=cmunbi.ttf]{cmunrm.ttf}\setmonofont[Path=/usr/share/fonts/truetype/cmu/,UprightFont=cmuntt.ttf,BoldFont=cmuntb.ttf,ItalicFont=cmunit.ttf,BoldItalicFont=cmuntx.ttf]{cmunrm.ttf}Goossens et al.{\bfseries UNKNOWN TEMPLATE  red{\setmainfont[Path=/usr/share/fonts/truetype/cmu/,UprightFont=cmunrm.ttf,BoldFont=cmunbx.ttf,ItalicFont=cmunti.ttf,BoldItalicFont=cmunbi.ttf]{cmunbx.ttf}\setmonofont[Path=/usr/share/fonts/truetype/cmu/,UprightFont=cmuntt.ttf,BoldFont=cmuntb.ttf,ItalicFont=cmunit.ttf,BoldItalicFont=cmuntx.ttf]{cmunbx.ttf}\bfseries ,}}{$\text{ }$}\setmainfont[Path=/usr/share/fonts/truetype/cmu/,UprightFont=cmunrm.ttf,BoldFont=cmunbx.ttf,ItalicFont=cmunti.ttf,BoldItalicFont=cmunbi.ttf]{cmunrm.ttf}\setmonofont[Path=/usr/share/fonts/truetype/cmu/,UprightFont=cmuntt.ttf,BoldFont=cmuntb.ttf,ItalicFont=cmunit.ttf,BoldItalicFont=cmuntx.ttf]{cmunrm.ttf} 1993{\bfseries UNKNOWN TEMPLATE  red{\setmainfont[Path=/usr/share/fonts/truetype/cmu/,UprightFont=cmunrm.ttf,BoldFont=cmunbx.ttf,ItalicFont=cmunti.ttf,BoldItalicFont=cmunbi.ttf]{cmunbx.ttf}\setmonofont[Path=/usr/share/fonts/truetype/cmu/,UprightFont=cmuntt.ttf,BoldFont=cmuntb.ttf,ItalicFont=cmunit.ttf,BoldItalicFont=cmuntx.ttf]{cmunbx.ttf}\bfseries )}}\\ \hline \hspace*{0pt}\ignorespaces{}\hspace*{0pt}{$\text{ }$}\setmainfont[Path=/usr/share/fonts/truetype/cmu/,UprightFont=cmunrm.ttf,BoldFont=cmunbx.ttf,ItalicFont=cmunti.ttf,BoldItalicFont=cmunbi.ttf]{cmunrm.ttf}\setmonofont[Path=/usr/share/fonts/truetype/cmu/,UprightFont=cmuntt.ttf,BoldFont=cmuntb.ttf,ItalicFont=cmunit.ttf,BoldItalicFont=cmuntx.ttf]{cmunrm.ttf} {\ttfamily \setmainfont[Path=/usr/share/fonts/truetype/cmu/,UprightFont=cmunrm.ttf,BoldFont=cmunbx.ttf,ItalicFont=cmunti.ttf,BoldItalicFont=cmunbi.ttf]{cmuntt.ttf}\setmonofont[Path=/usr/share/fonts/truetype/cmu/,UprightFont=cmuntt.ttf,BoldFont=cmuntb.ttf,ItalicFont=cmunit.ttf,BoldItalicFont=cmuntx.ttf]{cmuntt.ttf}\ttfamily \textbackslash{}citet*\{goossens93\}}\newline{}{\ttfamily \textbackslash{}citep*\{goossens93\}}&\hspace*{0pt}\ignorespaces{}\hspace*{0pt}{$\text{ }$}\setmainfont[Path=/usr/share/fonts/truetype/cmu/,UprightFont=cmunrm.ttf,BoldFont=cmunbx.ttf,ItalicFont=cmunti.ttf,BoldItalicFont=cmunbi.ttf]{cmunrm.ttf}\setmonofont[Path=/usr/share/fonts/truetype/cmu/,UprightFont=cmuntt.ttf,BoldFont=cmuntb.ttf,ItalicFont=cmunit.ttf,BoldItalicFont=cmuntx.ttf]{cmunrm.ttf} Goossens, Mittlebach, and Samarin (1993)\newline{}{\bfseries UNKNOWN TEMPLATE  red{\setmainfont[Path=/usr/share/fonts/truetype/cmu/,UprightFont=cmunrm.ttf,BoldFont=cmunbx.ttf,ItalicFont=cmunti.ttf,BoldItalicFont=cmunbi.ttf]{cmunbx.ttf}\setmonofont[Path=/usr/share/fonts/truetype/cmu/,UprightFont=cmuntt.ttf,BoldFont=cmuntb.ttf,ItalicFont=cmunit.ttf,BoldItalicFont=cmuntx.ttf]{cmunbx.ttf}\bfseries (}}\setmainfont[Path=/usr/share/fonts/truetype/cmu/,UprightFont=cmunrm.ttf,BoldFont=cmunbx.ttf,ItalicFont=cmunti.ttf,BoldItalicFont=cmunbi.ttf]{cmunrm.ttf}\setmonofont[Path=/usr/share/fonts/truetype/cmu/,UprightFont=cmuntt.ttf,BoldFont=cmuntb.ttf,ItalicFont=cmunit.ttf,BoldItalicFont=cmuntx.ttf]{cmunrm.ttf}Goossens, Mittlebach, and Samarin, 1993{\bfseries UNKNOWN TEMPLATE  red{\setmainfont[Path=/usr/share/fonts/truetype/cmu/,UprightFont=cmunrm.ttf,BoldFont=cmunbx.ttf,ItalicFont=cmunti.ttf,BoldItalicFont=cmunbi.ttf]{cmunbx.ttf}\setmonofont[Path=/usr/share/fonts/truetype/cmu/,UprightFont=cmuntt.ttf,BoldFont=cmuntb.ttf,ItalicFont=cmunit.ttf,BoldItalicFont=cmuntx.ttf]{cmunbx.ttf}\bfseries )}}\\ \hline \hspace*{0pt}\ignorespaces{}\hspace*{0pt}{$\text{ }$}\setmainfont[Path=/usr/share/fonts/truetype/cmu/,UprightFont=cmunrm.ttf,BoldFont=cmunbx.ttf,ItalicFont=cmunti.ttf,BoldItalicFont=cmunbi.ttf]{cmunrm.ttf}\setmonofont[Path=/usr/share/fonts/truetype/cmu/,UprightFont=cmuntt.ttf,BoldFont=cmuntb.ttf,ItalicFont=cmunit.ttf,BoldItalicFont=cmuntx.ttf]{cmunrm.ttf} {\ttfamily \setmainfont[Path=/usr/share/fonts/truetype/cmu/,UprightFont=cmunrm.ttf,BoldFont=cmunbx.ttf,ItalicFont=cmunti.ttf,BoldItalicFont=cmunbi.ttf]{cmuntt.ttf}\setmonofont[Path=/usr/share/fonts/truetype/cmu/,UprightFont=cmuntt.ttf,BoldFont=cmuntb.ttf,ItalicFont=cmunit.ttf,BoldItalicFont=cmuntx.ttf]{cmuntt.ttf}\ttfamily \textbackslash{}citeauthor\{goossens93\}}\newline{}{\ttfamily \textbackslash{}citeauthor*\{goossens93\}}&\hspace*{0pt}\ignorespaces{}\hspace*{0pt}{$\text{ }$}\setmainfont[Path=/usr/share/fonts/truetype/cmu/,UprightFont=cmunrm.ttf,BoldFont=cmunbx.ttf,ItalicFont=cmunti.ttf,BoldItalicFont=cmunbi.ttf]{cmunrm.ttf}\setmonofont[Path=/usr/share/fonts/truetype/cmu/,UprightFont=cmuntt.ttf,BoldFont=cmuntb.ttf,ItalicFont=cmunit.ttf,BoldItalicFont=cmuntx.ttf]{cmunrm.ttf} Goossens et al.\newline{}GoossensUNKNOWN TEMPLATE  red{, Mittlebach, and Samarin}\\ \hline \hspace*{0pt}\ignorespaces{}\hspace*{0pt} {\ttfamily \setmainfont[Path=/usr/share/fonts/truetype/cmu/,UprightFont=cmunrm.ttf,BoldFont=cmunbx.ttf,ItalicFont=cmunti.ttf,BoldItalicFont=cmunbi.ttf]{cmuntt.ttf}\setmonofont[Path=/usr/share/fonts/truetype/cmu/,UprightFont=cmuntt.ttf,BoldFont=cmuntb.ttf,ItalicFont=cmunit.ttf,BoldItalicFont=cmuntx.ttf]{cmuntt.ttf}\ttfamily \textbackslash{}citeyear\{goossens93\}}\newline{}{\ttfamily \textbackslash{}citeyearpar\{goossens93\}}&\hspace*{0pt}\ignorespaces{}\hspace*{0pt}{$\text{ }$}\setmainfont[Path=/usr/share/fonts/truetype/cmu/,UprightFont=cmunrm.ttf,BoldFont=cmunbx.ttf,ItalicFont=cmunti.ttf,BoldItalicFont=cmunbi.ttf]{cmunrm.ttf}\setmonofont[Path=/usr/share/fonts/truetype/cmu/,UprightFont=cmuntt.ttf,BoldFont=cmuntb.ttf,ItalicFont=cmunit.ttf,BoldItalicFont=cmuntx.ttf]{cmunrm.ttf} 1993\newline{}{\bfseries UNKNOWN TEMPLATE  red{\setmainfont[Path=/usr/share/fonts/truetype/cmu/,UprightFont=cmunrm.ttf,BoldFont=cmunbx.ttf,ItalicFont=cmunti.ttf,BoldItalicFont=cmunbi.ttf]{cmunbx.ttf}\setmonofont[Path=/usr/share/fonts/truetype/cmu/,UprightFont=cmuntt.ttf,BoldFont=cmuntb.ttf,ItalicFont=cmunit.ttf,BoldItalicFont=cmuntx.ttf]{cmunbx.ttf}\bfseries (}}\setmainfont[Path=/usr/share/fonts/truetype/cmu/,UprightFont=cmunrm.ttf,BoldFont=cmunbx.ttf,ItalicFont=cmunti.ttf,BoldItalicFont=cmunbi.ttf]{cmunrm.ttf}\setmonofont[Path=/usr/share/fonts/truetype/cmu/,UprightFont=cmuntt.ttf,BoldFont=cmuntb.ttf,ItalicFont=cmunit.ttf,BoldItalicFont=cmuntx.ttf]{cmunrm.ttf}1993{\bfseries UNKNOWN TEMPLATE  red{\setmainfont[Path=/usr/share/fonts/truetype/cmu/,UprightFont=cmunrm.ttf,BoldFont=cmunbx.ttf,ItalicFont=cmunti.ttf,BoldItalicFont=cmunbi.ttf]{cmunbx.ttf}\setmonofont[Path=/usr/share/fonts/truetype/cmu/,UprightFont=cmuntt.ttf,BoldFont=cmuntb.ttf,ItalicFont=cmunit.ttf,BoldItalicFont=cmuntx.ttf]{cmunbx.ttf}\bfseries )}}\\ \hline \hspace*{0pt}\ignorespaces{}\hspace*{0pt}{$\text{ }$}\setmainfont[Path=/usr/share/fonts/truetype/cmu/,UprightFont=cmunrm.ttf,BoldFont=cmunbx.ttf,ItalicFont=cmunti.ttf,BoldItalicFont=cmunbi.ttf]{cmunrm.ttf}\setmonofont[Path=/usr/share/fonts/truetype/cmu/,UprightFont=cmuntt.ttf,BoldFont=cmuntb.ttf,ItalicFont=cmunit.ttf,BoldItalicFont=cmuntx.ttf]{cmunrm.ttf} {\ttfamily \setmainfont[Path=/usr/share/fonts/truetype/cmu/,UprightFont=cmunrm.ttf,BoldFont=cmunbx.ttf,ItalicFont=cmunti.ttf,BoldItalicFont=cmunbi.ttf]{cmuntt.ttf}\setmonofont[Path=/usr/share/fonts/truetype/cmu/,UprightFont=cmuntt.ttf,BoldFont=cmuntb.ttf,ItalicFont=cmunit.ttf,BoldItalicFont=cmuntx.ttf]{cmuntt.ttf}\ttfamily \textbackslash{}citealt\{goossens93\}}\newline{}{\ttfamily \textbackslash{}citealp\{goossens93\}}&\hspace*{0pt}\ignorespaces{}\hspace*{0pt}{$\text{ }$}\setmainfont[Path=/usr/share/fonts/truetype/cmu/,UprightFont=cmunrm.ttf,BoldFont=cmunbx.ttf,ItalicFont=cmunti.ttf,BoldItalicFont=cmunbi.ttf]{cmunrm.ttf}\setmonofont[Path=/usr/share/fonts/truetype/cmu/,UprightFont=cmuntt.ttf,BoldFont=cmuntb.ttf,ItalicFont=cmunit.ttf,BoldItalicFont=cmuntx.ttf]{cmunrm.ttf} Goossens et al. 1993\newline{}Goossens et al.{\bfseries UNKNOWN TEMPLATE  red{\setmainfont[Path=/usr/share/fonts/truetype/cmu/,UprightFont=cmunrm.ttf,BoldFont=cmunbx.ttf,ItalicFont=cmunti.ttf,BoldItalicFont=cmunbi.ttf]{cmunbx.ttf}\setmonofont[Path=/usr/share/fonts/truetype/cmu/,UprightFont=cmuntt.ttf,BoldFont=cmuntb.ttf,ItalicFont=cmunit.ttf,BoldItalicFont=cmuntx.ttf]{cmunbx.ttf}\bfseries ,}}{$\text{ }$}\setmainfont[Path=/usr/share/fonts/truetype/cmu/,UprightFont=cmunrm.ttf,BoldFont=cmunbx.ttf,ItalicFont=cmunti.ttf,BoldItalicFont=cmunbi.ttf]{cmunrm.ttf}\setmonofont[Path=/usr/share/fonts/truetype/cmu/,UprightFont=cmuntt.ttf,BoldFont=cmuntb.ttf,ItalicFont=cmunit.ttf,BoldItalicFont=cmuntx.ttf]{cmunrm.ttf} 1993\\ \hline \hspace*{0pt}\ignorespaces{}\hspace*{0pt} {\ttfamily \setmainfont[Path=/usr/share/fonts/truetype/cmu/,UprightFont=cmunrm.ttf,BoldFont=cmunbx.ttf,ItalicFont=cmunti.ttf,BoldItalicFont=cmunbi.ttf]{cmuntt.ttf}\setmonofont[Path=/usr/share/fonts/truetype/cmu/,UprightFont=cmuntt.ttf,BoldFont=cmuntb.ttf,ItalicFont=cmunit.ttf,BoldItalicFont=cmuntx.ttf]{cmuntt.ttf}\ttfamily \textbackslash{}citetext\{priv.\textbackslash{} comm.\}}&\hspace*{0pt}\ignorespaces{}\hspace*{0pt}{$\text{ }$}\setmainfont[Path=/usr/share/fonts/truetype/cmu/,UprightFont=cmunrm.ttf,BoldFont=cmunbx.ttf,ItalicFont=cmunti.ttf,BoldItalicFont=cmunbi.ttf]{cmunrm.ttf}\setmonofont[Path=/usr/share/fonts/truetype/cmu/,UprightFont=cmuntt.ttf,BoldFont=cmuntb.ttf,ItalicFont=cmunit.ttf,BoldItalicFont=cmuntx.ttf]{cmunrm.ttf} (priv. comm.)\\ \hline 
\end{longtable}

Using the standard LaTeX bibliography support, you will see that each reference is numbered and each citation corresponds to the numbers. The numeric style of citation is quite common in scientific writing. In other disciplines, the author-{}year style, e.g., (Roberts, 2003), such as {\itshape \setmainfont[Path=/usr/share/fonts/truetype/cmu/,UprightFont=cmunrm.ttf,BoldFont=cmunbx.ttf,ItalicFont=cmunti.ttf,BoldItalicFont=cmunbi.ttf]{cmunti.ttf}\setmonofont[Path=/usr/share/fonts/truetype/cmu/,UprightFont=cmuntt.ttf,BoldFont=cmuntb.ttf,ItalicFont=cmunit.ttf,BoldItalicFont=cmuntx.ttf]{cmunti.ttf}\itshape Harvard}{$\text{ }$}\setmainfont[Path=/usr/share/fonts/truetype/cmu/,UprightFont=cmunrm.ttf,BoldFont=cmunbx.ttf,ItalicFont=cmunti.ttf,BoldItalicFont=cmunbi.ttf]{cmunrm.ttf}\setmonofont[Path=/usr/share/fonts/truetype/cmu/,UprightFont=cmuntt.ttf,BoldFont=cmuntb.ttf,ItalicFont=cmunit.ttf,BoldItalicFont=cmuntx.ttf]{cmunrm.ttf} is preferred. A discussion about which is best will not occur here, but a possible way to get such an output is by the {\ttfamily \setmainfont[Path=/usr/share/fonts/truetype/cmu/,UprightFont=cmunrm.ttf,BoldFont=cmunbx.ttf,ItalicFont=cmunti.ttf,BoldItalicFont=cmunbi.ttf]{cmuntt.ttf}\setmonofont[Path=/usr/share/fonts/truetype/cmu/,UprightFont=cmuntt.ttf,BoldFont=cmuntb.ttf,ItalicFont=cmunit.ttf,BoldItalicFont=cmuntx.ttf]{cmuntt.ttf}\ttfamily natbib}{$\text{ }$}\setmainfont[Path=/usr/share/fonts/truetype/cmu/,UprightFont=cmunrm.ttf,BoldFont=cmunbx.ttf,ItalicFont=cmunti.ttf,BoldItalicFont=cmunbi.ttf]{cmunrm.ttf}\setmonofont[Path=/usr/share/fonts/truetype/cmu/,UprightFont=cmuntt.ttf,BoldFont=cmuntb.ttf,ItalicFont=cmunit.ttf,BoldItalicFont=cmuntx.ttf]{cmunrm.ttf} package. In fact, it can supersede LaTeX\textquotesingle{}s own citation commands, as Natbib allows the user to easily switch between Harvard or numeric.

The first job is to add the following to your preamble in order to get LaTeX to use the Natbib package:


\begin{Shaded}
\begin{Highlighting}[]

\NormalTok{\textbackslash{}usepackage[options]\{natbib\}}\newline
\end{Highlighting}
\end{Shaded}


Also, you need to change the bibliography style file to be used, so edit the appropriate line at the bottom of the file so that it reads: {\ttfamily \setmainfont[Path=/usr/share/fonts/truetype/cmu/,UprightFont=cmunrm.ttf,BoldFont=cmunbx.ttf,ItalicFont=cmunti.ttf,BoldItalicFont=cmunbi.ttf]{cmuntt.ttf}\setmonofont[Path=/usr/share/fonts/truetype/cmu/,UprightFont=cmuntt.ttf,BoldFont=cmuntb.ttf,ItalicFont=cmunit.ttf,BoldItalicFont=cmuntx.ttf]{cmuntt.ttf}\ttfamily \textbackslash{}bibliographystyle\{plainnat\}}\setmainfont[Path=/usr/share/fonts/truetype/cmu/,UprightFont=cmunrm.ttf,BoldFont=cmunbx.ttf,ItalicFont=cmunti.ttf,BoldItalicFont=cmunbi.ttf]{cmunrm.ttf}\setmonofont[Path=/usr/share/fonts/truetype/cmu/,UprightFont=cmuntt.ttf,BoldFont=cmuntb.ttf,ItalicFont=cmunit.ttf,BoldItalicFont=cmuntx.ttf]{cmunrm.ttf}. Once done, it is basically a matter of altering the existing {\ttfamily \setmainfont[Path=/usr/share/fonts/truetype/cmu/,UprightFont=cmunrm.ttf,BoldFont=cmunbx.ttf,ItalicFont=cmunti.ttf,BoldItalicFont=cmunbi.ttf]{cmuntt.ttf}\setmonofont[Path=/usr/share/fonts/truetype/cmu/,UprightFont=cmuntt.ttf,BoldFont=cmuntb.ttf,ItalicFont=cmunit.ttf,BoldItalicFont=cmuntx.ttf]{cmuntt.ttf}\ttfamily \textbackslash{}cite}{$\text{ }$}\setmainfont[Path=/usr/share/fonts/truetype/cmu/,UprightFont=cmunrm.ttf,BoldFont=cmunbx.ttf,ItalicFont=cmunti.ttf,BoldItalicFont=cmunbi.ttf]{cmunrm.ttf}\setmonofont[Path=/usr/share/fonts/truetype/cmu/,UprightFont=cmuntt.ttf,BoldFont=cmuntb.ttf,ItalicFont=cmunit.ttf,BoldItalicFont=cmuntx.ttf]{cmunrm.ttf} commands to display the type of citation you want.

\begin{longtable}{|>{\RaggedRight}p{0.13344\linewidth}|>{\RaggedRight}p{0.39042\linewidth}|>{\RaggedRight}p{0.39042\linewidth}|} \hline 
\multicolumn{3}{|>{\RaggedRight}p{0.97143\linewidth}|}{{\bfseries \hspace*{0pt}\ignorespaces{}\hspace*{0pt} Citation styles compatible with Natbib}}\\ \hline {\bfseries \hspace*{0pt}\ignorespaces{}\hspace*{0pt} Style }&{\bfseries \hspace*{0pt}\ignorespaces{}\hspace*{0pt} Source }&{\bfseries \hspace*{0pt}\ignorespaces{}\hspace*{0pt} Description}\endhead  \hline \hspace*{0pt}\ignorespaces{}\hspace*{0pt} plainnat &\hspace*{0pt}\ignorespaces{}\hspace*{0pt} Provided &\hspace*{0pt}\ignorespaces{}\hspace*{0pt} natbib-{}compatible version of plain\\ \hline \hspace*{0pt}\ignorespaces{}\hspace*{0pt} abbrvnat &\hspace*{0pt}\ignorespaces{}\hspace*{0pt} Provided &\hspace*{0pt}\ignorespaces{}\hspace*{0pt} natbib-{}compatible version of abbrv\\ \hline \hspace*{0pt}\ignorespaces{}\hspace*{0pt} unsrtnat &\hspace*{0pt}\ignorespaces{}\hspace*{0pt} Provided &\hspace*{0pt}\ignorespaces{}\hspace*{0pt} natbib-{}compatible version of unsrt\\ \hline \hspace*{0pt}\ignorespaces{}\hspace*{0pt} apsrev &\hspace*{0pt}\ignorespaces{}\hspace*{0pt} \myhref{http://authors.aps.org/revtex4/}{REVTeX 4 home page} &\hspace*{0pt}\ignorespaces{}\hspace*{0pt} natbib-{}compatible style for Physical Review journals\\ \hline \hspace*{0pt}\ignorespaces{}\hspace*{0pt} rmpaps &\hspace*{0pt}\ignorespaces{}\hspace*{0pt} \myhref{http://authors.aps.org/revtex4/}{REVTeX 4 home page} &\hspace*{0pt}\ignorespaces{}\hspace*{0pt} natbib-{}compatible style for Review of Modern Physics journals\\ \hline \hspace*{0pt}\ignorespaces{}\hspace*{0pt} IEEEtranN &\hspace*{0pt}\ignorespaces{}\hspace*{0pt} \myhref{http://www.ctan.org/tex-archive/help/Catalogue/entries/ieeetran.html}{TeX Catalogue entry} &\hspace*{0pt}\ignorespaces{}\hspace*{0pt} natbib-{}compatible style for IEEE publications\\ \hline \hspace*{0pt}\ignorespaces{}\hspace*{0pt} achemso &\hspace*{0pt}\ignorespaces{}\hspace*{0pt} \myhref{http://www.ctan.org/tex-archive/help/Catalogue/entries/achemso.html}{TeX Catalogue entry} &\hspace*{0pt}\ignorespaces{}\hspace*{0pt} natbib-{}compatible style for American Chemical Society journals\\ \hline \hspace*{0pt}\ignorespaces{}\hspace*{0pt} rsc &\hspace*{0pt}\ignorespaces{}\hspace*{0pt} \myhref{http://www.ctan.org/tex-archive/help/Catalogue/entries/rsc.html}{TeX Catalogue entry} &\hspace*{0pt}\ignorespaces{}\hspace*{0pt} natbib-{}compatible style for Royal Society of Chemistry journals\\ \hline 
\end{longtable}

\subsubsection{Customization}
\label{675}
\begin{longtable}{|>{\RaggedRight}p{0.37417\linewidth}|>{\RaggedRight}p{0.56869\linewidth}|} \hline 
\multicolumn{2}{|>{\RaggedRight}p{0.97143\linewidth}|}{{\bfseries \hspace*{0pt}\ignorespaces{}\hspace*{0pt} Natbib\textquotesingle{}s customization options}}\\ \hline {\bfseries \hspace*{0pt}\ignorespaces{}\hspace*{0pt} Option }&{\bfseries \hspace*{0pt}\ignorespaces{}\hspace*{0pt} Meaning}\endhead  \hline \hspace*{0pt}\ignorespaces{}\hspace*{0pt} {\ttfamily \setmainfont[Path=/usr/share/fonts/truetype/cmu/,UprightFont=cmunrm.ttf,BoldFont=cmunbx.ttf,ItalicFont=cmunti.ttf,BoldItalicFont=cmunbi.ttf]{cmuntt.ttf}\setmonofont[Path=/usr/share/fonts/truetype/cmu/,UprightFont=cmuntt.ttf,BoldFont=cmuntb.ttf,ItalicFont=cmunit.ttf,BoldItalicFont=cmuntx.ttf]{cmuntt.ttf}\ttfamily round}{$\text{ }$}\setmainfont[Path=/usr/share/fonts/truetype/cmu/,UprightFont=cmunrm.ttf,BoldFont=cmunbx.ttf,ItalicFont=cmunti.ttf,BoldItalicFont=cmunbi.ttf]{cmunrm.ttf}\setmonofont[Path=/usr/share/fonts/truetype/cmu/,UprightFont=cmuntt.ttf,BoldFont=cmuntb.ttf,ItalicFont=cmunit.ttf,BoldItalicFont=cmuntx.ttf]{cmunrm.ttf} : {\ttfamily \setmainfont[Path=/usr/share/fonts/truetype/cmu/,UprightFont=cmunrm.ttf,BoldFont=cmunbx.ttf,ItalicFont=cmunti.ttf,BoldItalicFont=cmunbi.ttf]{cmuntt.ttf}\setmonofont[Path=/usr/share/fonts/truetype/cmu/,UprightFont=cmuntt.ttf,BoldFont=cmuntb.ttf,ItalicFont=cmunit.ttf,BoldItalicFont=cmuntx.ttf]{cmuntt.ttf}\ttfamily square}{$\text{ }$}\setmainfont[Path=/usr/share/fonts/truetype/cmu/,UprightFont=cmunrm.ttf,BoldFont=cmunbx.ttf,ItalicFont=cmunti.ttf,BoldItalicFont=cmunbi.ttf]{cmunrm.ttf}\setmonofont[Path=/usr/share/fonts/truetype/cmu/,UprightFont=cmuntt.ttf,BoldFont=cmuntb.ttf,ItalicFont=cmunit.ttf,BoldItalicFont=cmuntx.ttf]{cmunrm.ttf} : {\ttfamily \setmainfont[Path=/usr/share/fonts/truetype/cmu/,UprightFont=cmunrm.ttf,BoldFont=cmunbx.ttf,ItalicFont=cmunti.ttf,BoldItalicFont=cmunbi.ttf]{cmuntt.ttf}\setmonofont[Path=/usr/share/fonts/truetype/cmu/,UprightFont=cmuntt.ttf,BoldFont=cmuntb.ttf,ItalicFont=cmunit.ttf,BoldItalicFont=cmuntx.ttf]{cmuntt.ttf}\ttfamily curly}{$\text{ }$}\setmainfont[Path=/usr/share/fonts/truetype/cmu/,UprightFont=cmunrm.ttf,BoldFont=cmunbx.ttf,ItalicFont=cmunti.ttf,BoldItalicFont=cmunbi.ttf]{cmunrm.ttf}\setmonofont[Path=/usr/share/fonts/truetype/cmu/,UprightFont=cmuntt.ttf,BoldFont=cmuntb.ttf,ItalicFont=cmunit.ttf,BoldItalicFont=cmuntx.ttf]{cmunrm.ttf} : {\ttfamily \setmainfont[Path=/usr/share/fonts/truetype/cmu/,UprightFont=cmunrm.ttf,BoldFont=cmunbx.ttf,ItalicFont=cmunti.ttf,BoldItalicFont=cmunbi.ttf]{cmuntt.ttf}\setmonofont[Path=/usr/share/fonts/truetype/cmu/,UprightFont=cmuntt.ttf,BoldFont=cmuntb.ttf,ItalicFont=cmunit.ttf,BoldItalicFont=cmuntx.ttf]{cmuntt.ttf}\ttfamily angle}{$\text{ }$}\setmainfont[Path=/usr/share/fonts/truetype/cmu/,UprightFont=cmunrm.ttf,BoldFont=cmunbx.ttf,ItalicFont=cmunti.ttf,BoldItalicFont=cmunbi.ttf]{cmunrm.ttf}\setmonofont[Path=/usr/share/fonts/truetype/cmu/,UprightFont=cmuntt.ttf,BoldFont=cmuntb.ttf,ItalicFont=cmunit.ttf,BoldItalicFont=cmuntx.ttf]{cmunrm.ttf} &\hspace*{0pt}\ignorespaces{}\hspace*{0pt} Parentheses () (default), square brackets {$\text{[}$}{$\text{]}$}, curly braces \{\} or angle brackets <{}>{} \\ \hline \hspace*{0pt}\ignorespaces{}\hspace*{0pt} {\ttfamily \setmainfont[Path=/usr/share/fonts/truetype/cmu/,UprightFont=cmunrm.ttf,BoldFont=cmunbx.ttf,ItalicFont=cmunti.ttf,BoldItalicFont=cmunbi.ttf]{cmuntt.ttf}\setmonofont[Path=/usr/share/fonts/truetype/cmu/,UprightFont=cmuntt.ttf,BoldFont=cmuntb.ttf,ItalicFont=cmunit.ttf,BoldItalicFont=cmuntx.ttf]{cmuntt.ttf}\ttfamily colon}{$\text{ }$}\setmainfont[Path=/usr/share/fonts/truetype/cmu/,UprightFont=cmunrm.ttf,BoldFont=cmunbx.ttf,ItalicFont=cmunti.ttf,BoldItalicFont=cmunbi.ttf]{cmunrm.ttf}\setmonofont[Path=/usr/share/fonts/truetype/cmu/,UprightFont=cmuntt.ttf,BoldFont=cmuntb.ttf,ItalicFont=cmunit.ttf,BoldItalicFont=cmuntx.ttf]{cmunrm.ttf} : {\ttfamily \setmainfont[Path=/usr/share/fonts/truetype/cmu/,UprightFont=cmunrm.ttf,BoldFont=cmunbx.ttf,ItalicFont=cmunti.ttf,BoldItalicFont=cmunbi.ttf]{cmuntt.ttf}\setmonofont[Path=/usr/share/fonts/truetype/cmu/,UprightFont=cmuntt.ttf,BoldFont=cmuntb.ttf,ItalicFont=cmunit.ttf,BoldItalicFont=cmuntx.ttf]{cmuntt.ttf}\ttfamily comma}{$\text{ }$}\setmainfont[Path=/usr/share/fonts/truetype/cmu/,UprightFont=cmunrm.ttf,BoldFont=cmunbx.ttf,ItalicFont=cmunti.ttf,BoldItalicFont=cmunbi.ttf]{cmunrm.ttf}\setmonofont[Path=/usr/share/fonts/truetype/cmu/,UprightFont=cmuntt.ttf,BoldFont=cmuntb.ttf,ItalicFont=cmunit.ttf,BoldItalicFont=cmuntx.ttf]{cmunrm.ttf} &\hspace*{0pt}\ignorespaces{}\hspace*{0pt} multiple citations are separated by semi-{}colons (default) or commas\\ \hline \hspace*{0pt}\ignorespaces{}\hspace*{0pt} {\ttfamily \setmainfont[Path=/usr/share/fonts/truetype/cmu/,UprightFont=cmunrm.ttf,BoldFont=cmunbx.ttf,ItalicFont=cmunti.ttf,BoldItalicFont=cmunbi.ttf]{cmuntt.ttf}\setmonofont[Path=/usr/share/fonts/truetype/cmu/,UprightFont=cmuntt.ttf,BoldFont=cmuntb.ttf,ItalicFont=cmunit.ttf,BoldItalicFont=cmuntx.ttf]{cmuntt.ttf}\ttfamily authoryear}{$\text{ }$}\setmainfont[Path=/usr/share/fonts/truetype/cmu/,UprightFont=cmunrm.ttf,BoldFont=cmunbx.ttf,ItalicFont=cmunti.ttf,BoldItalicFont=cmunbi.ttf]{cmunrm.ttf}\setmonofont[Path=/usr/share/fonts/truetype/cmu/,UprightFont=cmuntt.ttf,BoldFont=cmuntb.ttf,ItalicFont=cmunit.ttf,BoldItalicFont=cmuntx.ttf]{cmunrm.ttf} : {\ttfamily \setmainfont[Path=/usr/share/fonts/truetype/cmu/,UprightFont=cmunrm.ttf,BoldFont=cmunbx.ttf,ItalicFont=cmunti.ttf,BoldItalicFont=cmunbi.ttf]{cmuntt.ttf}\setmonofont[Path=/usr/share/fonts/truetype/cmu/,UprightFont=cmuntt.ttf,BoldFont=cmuntb.ttf,ItalicFont=cmunit.ttf,BoldItalicFont=cmuntx.ttf]{cmuntt.ttf}\ttfamily numbers}{$\text{ }$}\setmainfont[Path=/usr/share/fonts/truetype/cmu/,UprightFont=cmunrm.ttf,BoldFont=cmunbx.ttf,ItalicFont=cmunti.ttf,BoldItalicFont=cmunbi.ttf]{cmunrm.ttf}\setmonofont[Path=/usr/share/fonts/truetype/cmu/,UprightFont=cmuntt.ttf,BoldFont=cmuntb.ttf,ItalicFont=cmunit.ttf,BoldItalicFont=cmuntx.ttf]{cmunrm.ttf} : {\ttfamily \setmainfont[Path=/usr/share/fonts/truetype/cmu/,UprightFont=cmunrm.ttf,BoldFont=cmunbx.ttf,ItalicFont=cmunti.ttf,BoldItalicFont=cmunbi.ttf]{cmuntt.ttf}\setmonofont[Path=/usr/share/fonts/truetype/cmu/,UprightFont=cmuntt.ttf,BoldFont=cmuntb.ttf,ItalicFont=cmunit.ttf,BoldItalicFont=cmuntx.ttf]{cmuntt.ttf}\ttfamily super}{$\text{ }$}\setmainfont[Path=/usr/share/fonts/truetype/cmu/,UprightFont=cmunrm.ttf,BoldFont=cmunbx.ttf,ItalicFont=cmunti.ttf,BoldItalicFont=cmunbi.ttf]{cmunrm.ttf}\setmonofont[Path=/usr/share/fonts/truetype/cmu/,UprightFont=cmuntt.ttf,BoldFont=cmuntb.ttf,ItalicFont=cmunit.ttf,BoldItalicFont=cmuntx.ttf]{cmunrm.ttf} &\hspace*{0pt}\ignorespaces{}\hspace*{0pt} author year style citations (default), numeric citations or superscripted numeric citations\\ \hline \hspace*{0pt}\ignorespaces{}\hspace*{0pt} {\ttfamily \setmainfont[Path=/usr/share/fonts/truetype/cmu/,UprightFont=cmunrm.ttf,BoldFont=cmunbx.ttf,ItalicFont=cmunti.ttf,BoldItalicFont=cmunbi.ttf]{cmuntt.ttf}\setmonofont[Path=/usr/share/fonts/truetype/cmu/,UprightFont=cmuntt.ttf,BoldFont=cmuntb.ttf,ItalicFont=cmunit.ttf,BoldItalicFont=cmuntx.ttf]{cmuntt.ttf}\ttfamily sort}{$\text{ }$}\setmainfont[Path=/usr/share/fonts/truetype/cmu/,UprightFont=cmunrm.ttf,BoldFont=cmunbx.ttf,ItalicFont=cmunti.ttf,BoldItalicFont=cmunbi.ttf]{cmunrm.ttf}\setmonofont[Path=/usr/share/fonts/truetype/cmu/,UprightFont=cmuntt.ttf,BoldFont=cmuntb.ttf,ItalicFont=cmunit.ttf,BoldItalicFont=cmuntx.ttf]{cmunrm.ttf} : {\ttfamily \setmainfont[Path=/usr/share/fonts/truetype/cmu/,UprightFont=cmunrm.ttf,BoldFont=cmunbx.ttf,ItalicFont=cmunti.ttf,BoldItalicFont=cmunbi.ttf]{cmuntt.ttf}\setmonofont[Path=/usr/share/fonts/truetype/cmu/,UprightFont=cmuntt.ttf,BoldFont=cmuntb.ttf,ItalicFont=cmunit.ttf,BoldItalicFont=cmuntx.ttf]{cmuntt.ttf}\ttfamily sort\&compress}{$\text{ }$}\setmainfont[Path=/usr/share/fonts/truetype/cmu/,UprightFont=cmunrm.ttf,BoldFont=cmunbx.ttf,ItalicFont=cmunti.ttf,BoldItalicFont=cmunbi.ttf]{cmunrm.ttf}\setmonofont[Path=/usr/share/fonts/truetype/cmu/,UprightFont=cmuntt.ttf,BoldFont=cmuntb.ttf,ItalicFont=cmunit.ttf,BoldItalicFont=cmuntx.ttf]{cmunrm.ttf} &\hspace*{0pt}\ignorespaces{}\hspace*{0pt} multiple citations are sorted into the order in which they appear in the references section or also compressing multiple numeric citations where possible\\ \hline \hspace*{0pt}\ignorespaces{}\hspace*{0pt} {\ttfamily \setmainfont[Path=/usr/share/fonts/truetype/cmu/,UprightFont=cmunrm.ttf,BoldFont=cmunbx.ttf,ItalicFont=cmunti.ttf,BoldItalicFont=cmunbi.ttf]{cmuntt.ttf}\setmonofont[Path=/usr/share/fonts/truetype/cmu/,UprightFont=cmuntt.ttf,BoldFont=cmuntb.ttf,ItalicFont=cmunit.ttf,BoldItalicFont=cmuntx.ttf]{cmuntt.ttf}\ttfamily longnamesfirst}{$\text{ }$}\setmainfont[Path=/usr/share/fonts/truetype/cmu/,UprightFont=cmunrm.ttf,BoldFont=cmunbx.ttf,ItalicFont=cmunti.ttf,BoldItalicFont=cmunbi.ttf]{cmunrm.ttf}\setmonofont[Path=/usr/share/fonts/truetype/cmu/,UprightFont=cmuntt.ttf,BoldFont=cmuntb.ttf,ItalicFont=cmunit.ttf,BoldItalicFont=cmuntx.ttf]{cmunrm.ttf} &\hspace*{0pt}\ignorespaces{}\hspace*{0pt} the first citation of any reference will use the starred variant (full author list), subsequent citations will use the abbreviated {\itshape \setmainfont[Path=/usr/share/fonts/truetype/cmu/,UprightFont=cmunrm.ttf,BoldFont=cmunbx.ttf,ItalicFont=cmunti.ttf,BoldItalicFont=cmunbi.ttf]{cmunti.ttf}\setmonofont[Path=/usr/share/fonts/truetype/cmu/,UprightFont=cmuntt.ttf,BoldFont=cmuntb.ttf,ItalicFont=cmunit.ttf,BoldItalicFont=cmuntx.ttf]{cmunti.ttf}\itshape et al.}{$\text{ }$}\setmainfont[Path=/usr/share/fonts/truetype/cmu/,UprightFont=cmunrm.ttf,BoldFont=cmunbx.ttf,ItalicFont=cmunti.ttf,BoldItalicFont=cmunbi.ttf]{cmunrm.ttf}\setmonofont[Path=/usr/share/fonts/truetype/cmu/,UprightFont=cmuntt.ttf,BoldFont=cmuntb.ttf,ItalicFont=cmunit.ttf,BoldItalicFont=cmuntx.ttf]{cmunrm.ttf} style\\ \hline \hspace*{0pt}\ignorespaces{}\hspace*{0pt} {\ttfamily \setmainfont[Path=/usr/share/fonts/truetype/cmu/,UprightFont=cmunrm.ttf,BoldFont=cmunbx.ttf,ItalicFont=cmunti.ttf,BoldItalicFont=cmunbi.ttf]{cmuntt.ttf}\setmonofont[Path=/usr/share/fonts/truetype/cmu/,UprightFont=cmuntt.ttf,BoldFont=cmuntb.ttf,ItalicFont=cmunit.ttf,BoldItalicFont=cmuntx.ttf]{cmuntt.ttf}\ttfamily sectionbib}{$\text{ }$}\setmainfont[Path=/usr/share/fonts/truetype/cmu/,UprightFont=cmunrm.ttf,BoldFont=cmunbx.ttf,ItalicFont=cmunti.ttf,BoldItalicFont=cmunbi.ttf]{cmunrm.ttf}\setmonofont[Path=/usr/share/fonts/truetype/cmu/,UprightFont=cmuntt.ttf,BoldFont=cmuntb.ttf,ItalicFont=cmunit.ttf,BoldItalicFont=cmuntx.ttf]{cmunrm.ttf} &\hspace*{0pt}\ignorespaces{}\hspace*{0pt} for use with the chapterbib package. redefines \textbackslash{}thebibliography to issue \textbackslash{}section* instead of \textbackslash{}chapter*\\ \hline \hspace*{0pt}\ignorespaces{}\hspace*{0pt} {\ttfamily \setmainfont[Path=/usr/share/fonts/truetype/cmu/,UprightFont=cmunrm.ttf,BoldFont=cmunbx.ttf,ItalicFont=cmunti.ttf,BoldItalicFont=cmunbi.ttf]{cmuntt.ttf}\setmonofont[Path=/usr/share/fonts/truetype/cmu/,UprightFont=cmuntt.ttf,BoldFont=cmuntb.ttf,ItalicFont=cmunit.ttf,BoldItalicFont=cmuntx.ttf]{cmuntt.ttf}\ttfamily nonamebreak}{$\text{ }$}\setmainfont[Path=/usr/share/fonts/truetype/cmu/,UprightFont=cmunrm.ttf,BoldFont=cmunbx.ttf,ItalicFont=cmunti.ttf,BoldItalicFont=cmunbi.ttf]{cmunrm.ttf}\setmonofont[Path=/usr/share/fonts/truetype/cmu/,UprightFont=cmuntt.ttf,BoldFont=cmuntb.ttf,ItalicFont=cmunit.ttf,BoldItalicFont=cmuntx.ttf]{cmunrm.ttf} &\hspace*{0pt}\ignorespaces{}\hspace*{0pt} keeps all the authors’ names in a citation on one line to fix some hyperref problems -{} causes overfull hboxes\\ \hline 
\end{longtable}


The main commands simply add a {\itshape \setmainfont[Path=/usr/share/fonts/truetype/cmu/,UprightFont=cmunrm.ttf,BoldFont=cmunbx.ttf,ItalicFont=cmunti.ttf,BoldItalicFont=cmunbi.ttf]{cmunti.ttf}\setmonofont[Path=/usr/share/fonts/truetype/cmu/,UprightFont=cmuntt.ttf,BoldFont=cmuntb.ttf,ItalicFont=cmunit.ttf,BoldItalicFont=cmuntx.ttf]{cmunti.ttf}\itshape t}{$\text{ }$}\setmainfont[Path=/usr/share/fonts/truetype/cmu/,UprightFont=cmunrm.ttf,BoldFont=cmunbx.ttf,ItalicFont=cmunti.ttf,BoldItalicFont=cmunbi.ttf]{cmunrm.ttf}\setmonofont[Path=/usr/share/fonts/truetype/cmu/,UprightFont=cmuntt.ttf,BoldFont=cmuntb.ttf,ItalicFont=cmunit.ttf,BoldItalicFont=cmuntx.ttf]{cmunrm.ttf} for \textquotesingle{}textual\textquotesingle{} or {\itshape \setmainfont[Path=/usr/share/fonts/truetype/cmu/,UprightFont=cmunrm.ttf,BoldFont=cmunbx.ttf,ItalicFont=cmunti.ttf,BoldItalicFont=cmunbi.ttf]{cmunti.ttf}\setmonofont[Path=/usr/share/fonts/truetype/cmu/,UprightFont=cmuntt.ttf,BoldFont=cmuntb.ttf,ItalicFont=cmunit.ttf,BoldItalicFont=cmuntx.ttf]{cmunti.ttf}\itshape p}{$\text{ }$}\setmainfont[Path=/usr/share/fonts/truetype/cmu/,UprightFont=cmunrm.ttf,BoldFont=cmunbx.ttf,ItalicFont=cmunti.ttf,BoldItalicFont=cmunbi.ttf]{cmunrm.ttf}\setmonofont[Path=/usr/share/fonts/truetype/cmu/,UprightFont=cmuntt.ttf,BoldFont=cmuntb.ttf,ItalicFont=cmunit.ttf,BoldItalicFont=cmuntx.ttf]{cmunrm.ttf} for \textquotesingle{}parenthesized\textquotesingle{}, to the basic {\ttfamily \setmainfont[Path=/usr/share/fonts/truetype/cmu/,UprightFont=cmunrm.ttf,BoldFont=cmunbx.ttf,ItalicFont=cmunti.ttf,BoldItalicFont=cmunbi.ttf]{cmuntt.ttf}\setmonofont[Path=/usr/share/fonts/truetype/cmu/,UprightFont=cmuntt.ttf,BoldFont=cmuntb.ttf,ItalicFont=cmunit.ttf,BoldItalicFont=cmuntx.ttf]{cmuntt.ttf}\ttfamily \textbackslash{}cite}{$\text{ }$}\setmainfont[Path=/usr/share/fonts/truetype/cmu/,UprightFont=cmunrm.ttf,BoldFont=cmunbx.ttf,ItalicFont=cmunti.ttf,BoldItalicFont=cmunbi.ttf]{cmunrm.ttf}\setmonofont[Path=/usr/share/fonts/truetype/cmu/,UprightFont=cmuntt.ttf,BoldFont=cmuntb.ttf,ItalicFont=cmunit.ttf,BoldItalicFont=cmuntx.ttf]{cmunrm.ttf} command. You will also notice how Natbib by default will compress references with three or more authors to the more concise {\itshape \setmainfont[Path=/usr/share/fonts/truetype/cmu/,UprightFont=cmunrm.ttf,BoldFont=cmunbx.ttf,ItalicFont=cmunti.ttf,BoldItalicFont=cmunbi.ttf]{cmunti.ttf}\setmonofont[Path=/usr/share/fonts/truetype/cmu/,UprightFont=cmuntt.ttf,BoldFont=cmuntb.ttf,ItalicFont=cmunit.ttf,BoldItalicFont=cmuntx.ttf]{cmunti.ttf}\itshape 1st surname et al}{$\text{ }$}\setmainfont[Path=/usr/share/fonts/truetype/cmu/,UprightFont=cmunrm.ttf,BoldFont=cmunbx.ttf,ItalicFont=cmunti.ttf,BoldItalicFont=cmunbi.ttf]{cmunrm.ttf}\setmonofont[Path=/usr/share/fonts/truetype/cmu/,UprightFont=cmuntt.ttf,BoldFont=cmuntb.ttf,ItalicFont=cmunit.ttf,BoldItalicFont=cmuntx.ttf]{cmunrm.ttf} version. By adding an asterisk (*), you can override this default and list all authors associated with that citation. There are some other specialized commands that Natbib supports, listed in the table here. Keep in mind that for instance {\ttfamily \setmainfont[Path=/usr/share/fonts/truetype/cmu/,UprightFont=cmunrm.ttf,BoldFont=cmunbx.ttf,ItalicFont=cmunti.ttf,BoldItalicFont=cmunbi.ttf]{cmuntt.ttf}\setmonofont[Path=/usr/share/fonts/truetype/cmu/,UprightFont=cmuntt.ttf,BoldFont=cmuntb.ttf,ItalicFont=cmunit.ttf,BoldItalicFont=cmuntx.ttf]{cmuntt.ttf}\ttfamily abbrvnat}{$\text{ }$}\setmainfont[Path=/usr/share/fonts/truetype/cmu/,UprightFont=cmunrm.ttf,BoldFont=cmunbx.ttf,ItalicFont=cmunti.ttf,BoldItalicFont=cmunbi.ttf]{cmunrm.ttf}\setmonofont[Path=/usr/share/fonts/truetype/cmu/,UprightFont=cmuntt.ttf,BoldFont=cmuntb.ttf,ItalicFont=cmunit.ttf,BoldItalicFont=cmuntx.ttf]{cmunrm.ttf} does not support {\ttfamily \setmainfont[Path=/usr/share/fonts/truetype/cmu/,UprightFont=cmunrm.ttf,BoldFont=cmunbx.ttf,ItalicFont=cmunti.ttf,BoldItalicFont=cmunbi.ttf]{cmuntt.ttf}\setmonofont[Path=/usr/share/fonts/truetype/cmu/,UprightFont=cmuntt.ttf,BoldFont=cmuntb.ttf,ItalicFont=cmunit.ttf,BoldItalicFont=cmuntx.ttf]{cmuntt.ttf}\ttfamily \textbackslash{}citet*}{$\text{ }$}\setmainfont[Path=/usr/share/fonts/truetype/cmu/,UprightFont=cmunrm.ttf,BoldFont=cmunbx.ttf,ItalicFont=cmunti.ttf,BoldItalicFont=cmunbi.ttf]{cmunrm.ttf}\setmonofont[Path=/usr/share/fonts/truetype/cmu/,UprightFont=cmuntt.ttf,BoldFont=cmuntb.ttf,ItalicFont=cmunit.ttf,BoldItalicFont=cmuntx.ttf]{cmunrm.ttf} and will automatically choose between all authors and et al..

The final area that I wish to cover about Natbib is customizing its citation style. There is a command called {\ttfamily \setmainfont[Path=/usr/share/fonts/truetype/cmu/,UprightFont=cmunrm.ttf,BoldFont=cmunbx.ttf,ItalicFont=cmunti.ttf,BoldItalicFont=cmunbi.ttf]{cmuntt.ttf}\setmonofont[Path=/usr/share/fonts/truetype/cmu/,UprightFont=cmuntt.ttf,BoldFont=cmuntb.ttf,ItalicFont=cmunit.ttf,BoldItalicFont=cmuntx.ttf]{cmuntt.ttf}\ttfamily \textbackslash{}bibpunct}{$\text{ }$}\setmainfont[Path=/usr/share/fonts/truetype/cmu/,UprightFont=cmunrm.ttf,BoldFont=cmunbx.ttf,ItalicFont=cmunti.ttf,BoldItalicFont=cmunbi.ttf]{cmunrm.ttf}\setmonofont[Path=/usr/share/fonts/truetype/cmu/,UprightFont=cmuntt.ttf,BoldFont=cmuntb.ttf,ItalicFont=cmunit.ttf,BoldItalicFont=cmuntx.ttf]{cmunrm.ttf} that can be used to override the defaults and change certain settings. For example, I have put the following in the preamble:


\begin{Shaded}
\begin{Highlighting}[]

\NormalTok{\textbackslash{}bibpunct\{(\}\{)\}\{;\}\{a\}\{,\}\{,\}}\newline
\end{Highlighting}
\end{Shaded}


The command requires six mandatory parameters.

\begin{myenumerate}
\item{}  The symbol for the opening bracket.
\item{}  The symbol for the closing bracket.
\item{}  The symbol that appears between multiple citations.
\item{}  This argument takes a letter: 
\begin{myitemize}
\item{}  {\itshape \setmainfont[Path=/usr/share/fonts/truetype/cmu/,UprightFont=cmunrm.ttf,BoldFont=cmunbx.ttf,ItalicFont=cmunti.ttf,BoldItalicFont=cmunbi.ttf]{cmunti.ttf}\setmonofont[Path=/usr/share/fonts/truetype/cmu/,UprightFont=cmuntt.ttf,BoldFont=cmuntb.ttf,ItalicFont=cmunit.ttf,BoldItalicFont=cmuntx.ttf]{cmunti.ttf}\itshape n}{$\text{ }$}\setmainfont[Path=/usr/share/fonts/truetype/cmu/,UprightFont=cmunrm.ttf,BoldFont=cmunbx.ttf,ItalicFont=cmunti.ttf,BoldItalicFont=cmunbi.ttf]{cmunrm.ttf}\setmonofont[Path=/usr/share/fonts/truetype/cmu/,UprightFont=cmuntt.ttf,BoldFont=cmuntb.ttf,ItalicFont=cmunit.ttf,BoldItalicFont=cmuntx.ttf]{cmunrm.ttf} -{} numerical style.
\item{}  {\itshape \setmainfont[Path=/usr/share/fonts/truetype/cmu/,UprightFont=cmunrm.ttf,BoldFont=cmunbx.ttf,ItalicFont=cmunti.ttf,BoldItalicFont=cmunbi.ttf]{cmunti.ttf}\setmonofont[Path=/usr/share/fonts/truetype/cmu/,UprightFont=cmuntt.ttf,BoldFont=cmuntb.ttf,ItalicFont=cmunit.ttf,BoldItalicFont=cmuntx.ttf]{cmunti.ttf}\itshape s}{$\text{ }$}\setmainfont[Path=/usr/share/fonts/truetype/cmu/,UprightFont=cmunrm.ttf,BoldFont=cmunbx.ttf,ItalicFont=cmunti.ttf,BoldItalicFont=cmunbi.ttf]{cmunrm.ttf}\setmonofont[Path=/usr/share/fonts/truetype/cmu/,UprightFont=cmuntt.ttf,BoldFont=cmuntb.ttf,ItalicFont=cmunit.ttf,BoldItalicFont=cmuntx.ttf]{cmunrm.ttf} -{} numerical superscript style.
\item{}  {\itshape \setmainfont[Path=/usr/share/fonts/truetype/cmu/,UprightFont=cmunrm.ttf,BoldFont=cmunbx.ttf,ItalicFont=cmunti.ttf,BoldItalicFont=cmunbi.ttf]{cmunti.ttf}\setmonofont[Path=/usr/share/fonts/truetype/cmu/,UprightFont=cmuntt.ttf,BoldFont=cmuntb.ttf,ItalicFont=cmunit.ttf,BoldItalicFont=cmuntx.ttf]{cmunti.ttf}\itshape any other letter}{$\text{ }$}\setmainfont[Path=/usr/share/fonts/truetype/cmu/,UprightFont=cmunrm.ttf,BoldFont=cmunbx.ttf,ItalicFont=cmunti.ttf,BoldItalicFont=cmunbi.ttf]{cmunrm.ttf}\setmonofont[Path=/usr/share/fonts/truetype/cmu/,UprightFont=cmuntt.ttf,BoldFont=cmuntb.ttf,ItalicFont=cmunit.ttf,BoldItalicFont=cmuntx.ttf]{cmunrm.ttf} -{} author-{}year style.
\end{myitemize}

\item{}  The punctuation to appear between the author and the year (in parenthetical case only).
\item{}  The punctuation used between years, in multiple citations when there is a common author. e.g., (Chomsky 1956, 1957). If you want an extra space, then you need {\ttfamily \setmainfont[Path=/usr/share/fonts/truetype/cmu/,UprightFont=cmunrm.ttf,BoldFont=cmunbx.ttf,ItalicFont=cmunti.ttf,BoldItalicFont=cmunbi.ttf]{cmuntt.ttf}\setmonofont[Path=/usr/share/fonts/truetype/cmu/,UprightFont=cmuntt.ttf,BoldFont=cmuntb.ttf,ItalicFont=cmunit.ttf,BoldItalicFont=cmuntx.ttf]{cmuntt.ttf}\ttfamily \{,\~{}\}}\setmainfont[Path=/usr/share/fonts/truetype/cmu/,UprightFont=cmunrm.ttf,BoldFont=cmunbx.ttf,ItalicFont=cmunti.ttf,BoldItalicFont=cmunbi.ttf]{cmunrm.ttf}\setmonofont[Path=/usr/share/fonts/truetype/cmu/,UprightFont=cmuntt.ttf,BoldFont=cmuntb.ttf,ItalicFont=cmunit.ttf,BoldItalicFont=cmuntx.ttf]{cmunrm.ttf}.  
\end{myenumerate}


Some of the options controlled by {\ttfamily \setmainfont[Path=/usr/share/fonts/truetype/cmu/,UprightFont=cmunrm.ttf,BoldFont=cmunbx.ttf,ItalicFont=cmunti.ttf,BoldItalicFont=cmunbi.ttf]{cmuntt.ttf}\setmonofont[Path=/usr/share/fonts/truetype/cmu/,UprightFont=cmuntt.ttf,BoldFont=cmuntb.ttf,ItalicFont=cmunit.ttf,BoldItalicFont=cmuntx.ttf]{cmuntt.ttf}\ttfamily \textbackslash{}bibpunct}{$\text{ }$}\setmainfont[Path=/usr/share/fonts/truetype/cmu/,UprightFont=cmunrm.ttf,BoldFont=cmunbx.ttf,ItalicFont=cmunti.ttf,BoldItalicFont=cmunbi.ttf]{cmunrm.ttf}\setmonofont[Path=/usr/share/fonts/truetype/cmu/,UprightFont=cmuntt.ttf,BoldFont=cmuntb.ttf,ItalicFont=cmunit.ttf,BoldItalicFont=cmuntx.ttf]{cmunrm.ttf} are also accessible by passing options to the natbib package when it is loaded.  These options also allow some other aspect of the bibliography to be controlled, and can be seen in the table (right).

So as you can see, this package is quite flexible, especially as you can easily switch between different citation styles by changing a single parameter. Do have a look at the \myhref{http://www.ctex.org/documents/packages/bibref/natbib.pdf}{Natbib manual}, it\textquotesingle{}s a short document and you can learn even more about how to use it.

\LaTeXNullTemplate{}
\section{BibTeX}
\label{676}

I have previously introduced the idea of embedding references at the end of the document, and then using the {\ttfamily \setmainfont[Path=/usr/share/fonts/truetype/cmu/,UprightFont=cmunrm.ttf,BoldFont=cmunbx.ttf,ItalicFont=cmunti.ttf,BoldItalicFont=cmunbi.ttf]{cmuntt.ttf}\setmonofont[Path=/usr/share/fonts/truetype/cmu/,UprightFont=cmuntt.ttf,BoldFont=cmuntb.ttf,ItalicFont=cmunit.ttf,BoldItalicFont=cmuntx.ttf]{cmuntt.ttf}\ttfamily \textbackslash{}cite}{$\text{ }$}\setmainfont[Path=/usr/share/fonts/truetype/cmu/,UprightFont=cmunrm.ttf,BoldFont=cmunbx.ttf,ItalicFont=cmunti.ttf,BoldItalicFont=cmunbi.ttf]{cmunrm.ttf}\setmonofont[Path=/usr/share/fonts/truetype/cmu/,UprightFont=cmuntt.ttf,BoldFont=cmuntb.ttf,ItalicFont=cmunit.ttf,BoldItalicFont=cmuntx.ttf]{cmunrm.ttf} command to cite them within the text. In this tutorial, I want to do a little better than this method, as it\textquotesingle{}s not as flexible as it could be.  I will concentrate on using \myhref{https://en.wikipedia.org/wiki/BibTeX}{BibTeX}.

A BibTeX database is stored as a {\itshape \setmainfont[Path=/usr/share/fonts/truetype/cmu/,UprightFont=cmunrm.ttf,BoldFont=cmunbx.ttf,ItalicFont=cmunti.ttf,BoldItalicFont=cmunbi.ttf]{cmunti.ttf}\setmonofont[Path=/usr/share/fonts/truetype/cmu/,UprightFont=cmuntt.ttf,BoldFont=cmuntb.ttf,ItalicFont=cmunit.ttf,BoldItalicFont=cmuntx.ttf]{cmunti.ttf}\itshape .bib}{$\text{ }$}\setmainfont[Path=/usr/share/fonts/truetype/cmu/,UprightFont=cmunrm.ttf,BoldFont=cmunbx.ttf,ItalicFont=cmunti.ttf,BoldItalicFont=cmunbi.ttf]{cmunrm.ttf}\setmonofont[Path=/usr/share/fonts/truetype/cmu/,UprightFont=cmuntt.ttf,BoldFont=cmuntb.ttf,ItalicFont=cmunit.ttf,BoldItalicFont=cmuntx.ttf]{cmunrm.ttf} file. It is a plain text file, and so can be viewed and edited easily. The structure of the file is also quite simple. An example of a BibTeX entry:


\begin{Shaded}
\begin{Highlighting}[]

\KeywordTok{@article}\NormalTok{\{}\OtherTok{greenwade93}\NormalTok{,}\newline
\ensuremath{\text{ }}\ensuremath{\text{ }}\ensuremath{\text{ }}\ensuremath{\text{ }}\DataTypeTok{author}\ensuremath{\text{ }}\ensuremath{\text{ }}\NormalTok{=\ensuremath{\text{ }}"}\StringTok{George\ensuremath{\text{ }}D.\ensuremath{\text{ }}Greenwade}\NormalTok{",}\newline
\ensuremath{\text{ }}\ensuremath{\text{ }}\ensuremath{\text{ }}\ensuremath{\text{ }}\DataTypeTok{title}\ensuremath{\text{ }}\ensuremath{\text{ }}\ensuremath{\text{ }}\NormalTok{=\ensuremath{\text{ }}"}\StringTok{The\ensuremath{\text{ }}\{C\}omprehensive\ensuremath{\text{ }}\{T\}ex\ensuremath{\text{ }}\{A\}rchive\ensuremath{\text{ }}\{N\}etwork\ensuremath{\text{ }}(\{CTAN\})}\NormalTok{",}\newline
\ensuremath{\text{ }}\ensuremath{\text{ }}\ensuremath{\text{ }}\ensuremath{\text{ }}\DataTypeTok{year}\ensuremath{\text{ }}\ensuremath{\text{ }}\ensuremath{\text{ }}\ensuremath{\text{ }}\NormalTok{=\ensuremath{\text{ }}"}\StringTok{1993}\NormalTok{",}\newline
\ensuremath{\text{ }}\ensuremath{\text{ }}\ensuremath{\text{ }}\ensuremath{\text{ }}\DataTypeTok{journal}\ensuremath{\text{ }}\NormalTok{=\ensuremath{\text{ }}"}\StringTok{TUGBoat}\NormalTok{",}\newline
\ensuremath{\text{ }}\ensuremath{\text{ }}\ensuremath{\text{ }}\ensuremath{\text{ }}\DataTypeTok{volume}\ensuremath{\text{ }}\ensuremath{\text{ }}\NormalTok{=\ensuremath{\text{ }}"}\StringTok{14}\NormalTok{",}\newline
\ensuremath{\text{ }}\ensuremath{\text{ }}\ensuremath{\text{ }}\ensuremath{\text{ }}\DataTypeTok{number}\ensuremath{\text{ }}\ensuremath{\text{ }}\NormalTok{=\ensuremath{\text{ }}"}\StringTok{3}\NormalTok{",}\newline
\ensuremath{\text{ }}\ensuremath{\text{ }}\ensuremath{\text{ }}\ensuremath{\text{ }}\DataTypeTok{pages}\ensuremath{\text{ }}\ensuremath{\text{ }}\ensuremath{\text{ }}\NormalTok{=\ensuremath{\text{ }}"}\StringTok{342--351}\NormalTok{"}\newline
\NormalTok{\}}\newline
\end{Highlighting}
\end{Shaded}


Each entry begins with the declaration of the reference type, in the form of {\ttfamily \setmainfont[Path=/usr/share/fonts/truetype/cmu/,UprightFont=cmunrm.ttf,BoldFont=cmunbx.ttf,ItalicFont=cmunti.ttf,BoldItalicFont=cmunbi.ttf]{cmuntt.ttf}\setmonofont[Path=/usr/share/fonts/truetype/cmu/,UprightFont=cmuntt.ttf,BoldFont=cmuntb.ttf,ItalicFont=cmunit.ttf,BoldItalicFont=cmuntx.ttf]{cmuntt.ttf}\ttfamily @{\itshape \setmainfont[Path=/usr/share/fonts/truetype/cmu/,UprightFont=cmunrm.ttf,BoldFont=cmunbx.ttf,ItalicFont=cmunti.ttf,BoldItalicFont=cmunbi.ttf]{cmunit.ttf}\setmonofont[Path=/usr/share/fonts/truetype/cmu/,UprightFont=cmuntt.ttf,BoldFont=cmuntb.ttf,ItalicFont=cmunit.ttf,BoldItalicFont=cmuntx.ttf]{cmunit.ttf}\ttfamily \itshape type}}\setmainfont[Path=/usr/share/fonts/truetype/cmu/,UprightFont=cmunrm.ttf,BoldFont=cmunbx.ttf,ItalicFont=cmunti.ttf,BoldItalicFont=cmunbi.ttf]{cmunrm.ttf}\setmonofont[Path=/usr/share/fonts/truetype/cmu/,UprightFont=cmuntt.ttf,BoldFont=cmuntb.ttf,ItalicFont=cmunit.ttf,BoldItalicFont=cmuntx.ttf]{cmunrm.ttf}. BibTeX knows of practically all types you can think of, common ones are: {\itshape \setmainfont[Path=/usr/share/fonts/truetype/cmu/,UprightFont=cmunrm.ttf,BoldFont=cmunbx.ttf,ItalicFont=cmunti.ttf,BoldItalicFont=cmunbi.ttf]{cmunti.ttf}\setmonofont[Path=/usr/share/fonts/truetype/cmu/,UprightFont=cmuntt.ttf,BoldFont=cmuntb.ttf,ItalicFont=cmunit.ttf,BoldItalicFont=cmuntx.ttf]{cmunti.ttf}\itshape book}\setmainfont[Path=/usr/share/fonts/truetype/cmu/,UprightFont=cmunrm.ttf,BoldFont=cmunbx.ttf,ItalicFont=cmunti.ttf,BoldItalicFont=cmunbi.ttf]{cmunrm.ttf}\setmonofont[Path=/usr/share/fonts/truetype/cmu/,UprightFont=cmuntt.ttf,BoldFont=cmuntb.ttf,ItalicFont=cmunit.ttf,BoldItalicFont=cmuntx.ttf]{cmunrm.ttf}, {\itshape \setmainfont[Path=/usr/share/fonts/truetype/cmu/,UprightFont=cmunrm.ttf,BoldFont=cmunbx.ttf,ItalicFont=cmunti.ttf,BoldItalicFont=cmunbi.ttf]{cmunti.ttf}\setmonofont[Path=/usr/share/fonts/truetype/cmu/,UprightFont=cmuntt.ttf,BoldFont=cmuntb.ttf,ItalicFont=cmunit.ttf,BoldItalicFont=cmuntx.ttf]{cmunti.ttf}\itshape article}\setmainfont[Path=/usr/share/fonts/truetype/cmu/,UprightFont=cmunrm.ttf,BoldFont=cmunbx.ttf,ItalicFont=cmunti.ttf,BoldItalicFont=cmunbi.ttf]{cmunrm.ttf}\setmonofont[Path=/usr/share/fonts/truetype/cmu/,UprightFont=cmuntt.ttf,BoldFont=cmuntb.ttf,ItalicFont=cmunit.ttf,BoldItalicFont=cmuntx.ttf]{cmunrm.ttf}, and for papers presented at conferences, there is {\itshape \setmainfont[Path=/usr/share/fonts/truetype/cmu/,UprightFont=cmunrm.ttf,BoldFont=cmunbx.ttf,ItalicFont=cmunti.ttf,BoldItalicFont=cmunbi.ttf]{cmunti.ttf}\setmonofont[Path=/usr/share/fonts/truetype/cmu/,UprightFont=cmuntt.ttf,BoldFont=cmuntb.ttf,ItalicFont=cmunit.ttf,BoldItalicFont=cmuntx.ttf]{cmunti.ttf}\itshape inproceedings}\setmainfont[Path=/usr/share/fonts/truetype/cmu/,UprightFont=cmunrm.ttf,BoldFont=cmunbx.ttf,ItalicFont=cmunti.ttf,BoldItalicFont=cmunbi.ttf]{cmunrm.ttf}\setmonofont[Path=/usr/share/fonts/truetype/cmu/,UprightFont=cmuntt.ttf,BoldFont=cmuntb.ttf,ItalicFont=cmunit.ttf,BoldItalicFont=cmuntx.ttf]{cmunrm.ttf}. In this example, I have referred to an article within a journal.

After the type, you must have a left curly brace \textquotesingle{}{\ttfamily \setmainfont[Path=/usr/share/fonts/truetype/cmu/,UprightFont=cmunrm.ttf,BoldFont=cmunbx.ttf,ItalicFont=cmunti.ttf,BoldItalicFont=cmunbi.ttf]{cmuntt.ttf}\setmonofont[Path=/usr/share/fonts/truetype/cmu/,UprightFont=cmuntt.ttf,BoldFont=cmuntb.ttf,ItalicFont=cmunit.ttf,BoldItalicFont=cmuntx.ttf]{cmuntt.ttf}\ttfamily \{}\setmainfont[Path=/usr/share/fonts/truetype/cmu/,UprightFont=cmunrm.ttf,BoldFont=cmunbx.ttf,ItalicFont=cmunti.ttf,BoldItalicFont=cmunbi.ttf]{cmunrm.ttf}\setmonofont[Path=/usr/share/fonts/truetype/cmu/,UprightFont=cmuntt.ttf,BoldFont=cmuntb.ttf,ItalicFont=cmunit.ttf,BoldItalicFont=cmuntx.ttf]{cmunrm.ttf}\textquotesingle{} to signify the beginning of the reference attributes. The first one follows immediately after the brace, which is the {\itshape \setmainfont[Path=/usr/share/fonts/truetype/cmu/,UprightFont=cmunrm.ttf,BoldFont=cmunbx.ttf,ItalicFont=cmunti.ttf,BoldItalicFont=cmunbi.ttf]{cmunti.ttf}\setmonofont[Path=/usr/share/fonts/truetype/cmu/,UprightFont=cmuntt.ttf,BoldFont=cmuntb.ttf,ItalicFont=cmunit.ttf,BoldItalicFont=cmuntx.ttf]{cmunti.ttf}\itshape citation key}\setmainfont[Path=/usr/share/fonts/truetype/cmu/,UprightFont=cmunrm.ttf,BoldFont=cmunbx.ttf,ItalicFont=cmunti.ttf,BoldItalicFont=cmunbi.ttf]{cmunrm.ttf}\setmonofont[Path=/usr/share/fonts/truetype/cmu/,UprightFont=cmuntt.ttf,BoldFont=cmuntb.ttf,ItalicFont=cmunit.ttf,BoldItalicFont=cmuntx.ttf]{cmunrm.ttf}, or the {\itshape \setmainfont[Path=/usr/share/fonts/truetype/cmu/,UprightFont=cmunrm.ttf,BoldFont=cmunbx.ttf,ItalicFont=cmunti.ttf,BoldItalicFont=cmunbi.ttf]{cmunti.ttf}\setmonofont[Path=/usr/share/fonts/truetype/cmu/,UprightFont=cmuntt.ttf,BoldFont=cmuntb.ttf,ItalicFont=cmunit.ttf,BoldItalicFont=cmuntx.ttf]{cmunti.ttf}\itshape BibTeX key}\setmainfont[Path=/usr/share/fonts/truetype/cmu/,UprightFont=cmunrm.ttf,BoldFont=cmunbx.ttf,ItalicFont=cmunti.ttf,BoldItalicFont=cmunbi.ttf]{cmunrm.ttf}\setmonofont[Path=/usr/share/fonts/truetype/cmu/,UprightFont=cmuntt.ttf,BoldFont=cmuntb.ttf,ItalicFont=cmunit.ttf,BoldItalicFont=cmuntx.ttf]{cmunrm.ttf}. This key must be unique for all entries in your bibliography. It is this identifier that you will use within your document to cross-{}reference it to this entry. It is up to you as to how you wish to label each reference, but there is a loose standard in which you use the author\textquotesingle{}s surname, followed by the year of publication. This is the scheme that I use in this tutorial.

Next, it should be clear that what follows are the relevant fields and data for that particular reference. The field names on the left are \myhref{https://en.wikipedia.org/wiki/BibTeX\%23Bibliographic\%20information\%20file}{BibTeX keywords}. They are followed by an equals sign (=) where the value for that field is then placed. BibTeX expects you to explicitly label the beginning and end of each value. I personally use quotation marks (\symbol{34}), however, you also have the option of using curly braces (\textquotesingle{}\{\textquotesingle{}, \textquotesingle{}\}\textquotesingle{}). But as you will soon see, curly braces have other roles, within attributes, so I prefer not to use them for this job as they can get more confusing. A notable exception is when you want to use characters with umlauts (ü, ö, etc), since \myhref{https://en.wikibooks.org/wiki/..\%2FAccents}{their notation} is in the format {\ttfamily \setmainfont[Path=/usr/share/fonts/truetype/cmu/,UprightFont=cmunrm.ttf,BoldFont=cmunbx.ttf,ItalicFont=cmunti.ttf,BoldItalicFont=cmunbi.ttf]{cmuntt.ttf}\setmonofont[Path=/usr/share/fonts/truetype/cmu/,UprightFont=cmuntt.ttf,BoldFont=cmuntb.ttf,ItalicFont=cmunit.ttf,BoldItalicFont=cmuntx.ttf]{cmuntt.ttf}\ttfamily \textbackslash{}\symbol{34}\{o\}}\setmainfont[Path=/usr/share/fonts/truetype/cmu/,UprightFont=cmunrm.ttf,BoldFont=cmunbx.ttf,ItalicFont=cmunti.ttf,BoldItalicFont=cmunbi.ttf]{cmunrm.ttf}\setmonofont[Path=/usr/share/fonts/truetype/cmu/,UprightFont=cmuntt.ttf,BoldFont=cmuntb.ttf,ItalicFont=cmunit.ttf,BoldItalicFont=cmuntx.ttf]{cmunrm.ttf}, and the quotation mark will close the one opening the field, causing an error in the parsing of the reference.  Using {\ttfamily \setmainfont[Path=/usr/share/fonts/truetype/cmu/,UprightFont=cmunrm.ttf,BoldFont=cmunbx.ttf,ItalicFont=cmunti.ttf,BoldItalicFont=cmunbi.ttf]{cmuntt.ttf}\setmonofont[Path=/usr/share/fonts/truetype/cmu/,UprightFont=cmuntt.ttf,BoldFont=cmuntb.ttf,ItalicFont=cmunit.ttf,BoldItalicFont=cmuntx.ttf]{cmuntt.ttf}\ttfamily \textbackslash{}usepackage{$\text{[}$}utf8{$\text{]}$}\{inputenc\}}{$\text{ }$}\setmainfont[Path=/usr/share/fonts/truetype/cmu/,UprightFont=cmunrm.ttf,BoldFont=cmunbx.ttf,ItalicFont=cmunti.ttf,BoldItalicFont=cmunbi.ttf]{cmunrm.ttf}\setmonofont[Path=/usr/share/fonts/truetype/cmu/,UprightFont=cmuntt.ttf,BoldFont=cmuntb.ttf,ItalicFont=cmunit.ttf,BoldItalicFont=cmuntx.ttf]{cmunrm.ttf} in the preamble to the {\ttfamily \setmainfont[Path=/usr/share/fonts/truetype/cmu/,UprightFont=cmunrm.ttf,BoldFont=cmunbx.ttf,ItalicFont=cmunti.ttf,BoldItalicFont=cmunbi.ttf]{cmuntt.ttf}\setmonofont[Path=/usr/share/fonts/truetype/cmu/,UprightFont=cmuntt.ttf,BoldFont=cmuntb.ttf,ItalicFont=cmunit.ttf,BoldItalicFont=cmuntx.ttf]{cmuntt.ttf}\ttfamily .tex}{$\text{ }$}\setmainfont[Path=/usr/share/fonts/truetype/cmu/,UprightFont=cmunrm.ttf,BoldFont=cmunbx.ttf,ItalicFont=cmunti.ttf,BoldItalicFont=cmunbi.ttf]{cmunrm.ttf}\setmonofont[Path=/usr/share/fonts/truetype/cmu/,UprightFont=cmuntt.ttf,BoldFont=cmuntb.ttf,ItalicFont=cmunit.ttf,BoldItalicFont=cmuntx.ttf]{cmunrm.ttf} source file can get round this, as the accented characters can just be stored in the {\ttfamily \setmainfont[Path=/usr/share/fonts/truetype/cmu/,UprightFont=cmunrm.ttf,BoldFont=cmunbx.ttf,ItalicFont=cmunti.ttf,BoldItalicFont=cmunbi.ttf]{cmuntt.ttf}\setmonofont[Path=/usr/share/fonts/truetype/cmu/,UprightFont=cmuntt.ttf,BoldFont=cmuntb.ttf,ItalicFont=cmunit.ttf,BoldItalicFont=cmuntx.ttf]{cmuntt.ttf}\ttfamily .bib}{$\text{ }$}\setmainfont[Path=/usr/share/fonts/truetype/cmu/,UprightFont=cmunrm.ttf,BoldFont=cmunbx.ttf,ItalicFont=cmunti.ttf,BoldItalicFont=cmunbi.ttf]{cmunrm.ttf}\setmonofont[Path=/usr/share/fonts/truetype/cmu/,UprightFont=cmuntt.ttf,BoldFont=cmuntb.ttf,ItalicFont=cmunit.ttf,BoldItalicFont=cmuntx.ttf]{cmunrm.ttf} file without any need for special markup.  This allows a consistent format to be kept throughout the {\ttfamily \setmainfont[Path=/usr/share/fonts/truetype/cmu/,UprightFont=cmunrm.ttf,BoldFont=cmunbx.ttf,ItalicFont=cmunti.ttf,BoldItalicFont=cmunbi.ttf]{cmuntt.ttf}\setmonofont[Path=/usr/share/fonts/truetype/cmu/,UprightFont=cmuntt.ttf,BoldFont=cmuntb.ttf,ItalicFont=cmunit.ttf,BoldItalicFont=cmuntx.ttf]{cmuntt.ttf}\ttfamily .bib}{$\text{ }$}\setmainfont[Path=/usr/share/fonts/truetype/cmu/,UprightFont=cmunrm.ttf,BoldFont=cmunbx.ttf,ItalicFont=cmunti.ttf,BoldItalicFont=cmunbi.ttf]{cmunrm.ttf}\setmonofont[Path=/usr/share/fonts/truetype/cmu/,UprightFont=cmuntt.ttf,BoldFont=cmuntb.ttf,ItalicFont=cmunit.ttf,BoldItalicFont=cmuntx.ttf]{cmunrm.ttf} file, avoiding the need to use braces when there are umlauts to consider.

Remember that each attribute must be followed by a comma to delimit one from another. You do not need to add a comma to the last attribute, since the closing brace will tell BibTeX that there are no more attributes for this entry, although you won\textquotesingle{}t get an error if you do.

It can take a while to learn what the reference types are, and what fields each type has available (and which ones are required or optional, etc). So, look at this \myhref{http://newton.ex.ac.uk/tex/pack/bibtex/btxdoc/node6.html}{entry type reference} and also this \myhref{http://newton.ex.ac.uk/tex/pack/bibtex/btxdoc/node7.html}{field reference} for descriptions of all the fields. It may be worth bookmarking or printing these pages so that they are easily at hand when you need them. Much of the information contained therein is repeated in the following table for your convenience.
{\scriptsize{}
{\scalefont{0.52741}\begin{longtable}{|>{\RaggedRight}p{0.05411\linewidth}|>{\RaggedRight}p{0.04225\linewidth}|>{\RaggedRight}p{0.03585\linewidth}|>{\RaggedRight}p{0.04739\linewidth}|>{\RaggedRight}p{0.04501\linewidth}|>{\RaggedRight}p{0.06624\linewidth}|>{\RaggedRight}p{0.05411\linewidth}|>{\RaggedRight}p{0.04827\linewidth}|>{\RaggedRight}p{0.05411\linewidth}|>{\RaggedRight}p{0.03440\linewidth}|>{\RaggedRight}p{0.06887\linewidth}|>{\RaggedRight}p{0.06431\linewidth}|>{\RaggedRight}p{0.07079\linewidth}|} \hline 
\multicolumn{13}{|>{\RaggedRight}p{0.97143\linewidth}|}{{\bfseries \hspace*{0pt}\ignorespaces{}\hspace*{0pt} Standard BibTeX entry and field types}}\\ \hline {\bfseries \hspace*{0pt}\ignorespaces{}\hspace*{0pt}}&{\bfseries \hspace*{0pt}\ignorespaces{}\hspace*{0pt} article}&{\bfseries \hspace*{0pt}\ignorespaces{}\hspace*{0pt} book}&{\bfseries \hspace*{0pt}\ignorespaces{}\hspace*{0pt} booklet}&{\bfseries \hspace*{0pt}\ignorespaces{}\hspace*{0pt} inbook}&{\bfseries \hspace*{0pt}\ignorespaces{}\hspace*{0pt} incollection}&{\bfseries \hspace*{0pt}\ignorespaces{}\hspace*{0pt} inproceedings \setmainfont[Path=/usr/share/fonts/truetype/freefont/,UprightFont=FreeSerif.ttf,BoldFont=FreeSerifBold.ttf,ItalicFont=FreeSerifItalic.ttf,BoldItalicFont=FreeSerifBoldItalic.ttf]{FreeSerif.ttf}\setmonofont[Path=/usr/share/fonts/truetype/freefont/,UprightFont=FreeMono.ttf,BoldFont=FreeMonoBold.ttf,ItalicFont=FreeMonoOblique.ttf,BoldItalicFont=FreeMonoBoldOblique.ttf]{FreeSerif.ttf}≈\setmainfont[Path=/usr/share/fonts/truetype/cmu/,UprightFont=cmunrm.ttf,BoldFont=cmunbx.ttf,ItalicFont=cmunti.ttf,BoldItalicFont=cmunbi.ttf]{cmunrm.ttf}\setmonofont[Path=/usr/share/fonts/truetype/cmu/,UprightFont=cmuntt.ttf,BoldFont=cmuntb.ttf,ItalicFont=cmunit.ttf,BoldItalicFont=cmuntx.ttf]{cmunrm.ttf} conference}&{\bfseries \hspace*{0pt}\ignorespaces{}\hspace*{0pt} manual}&{\bfseries \hspace*{0pt}\ignorespaces{}\hspace*{0pt} mastersthesis, phdthesis}&{\bfseries \hspace*{0pt}\ignorespaces{}\hspace*{0pt} misc}&{\bfseries \hspace*{0pt}\ignorespaces{}\hspace*{0pt} proceedings}&{\bfseries \hspace*{0pt}\ignorespaces{}\hspace*{0pt} tech report}&{\bfseries \hspace*{0pt}\ignorespaces{}\hspace*{0pt} unpublished}\\ \hline {\bfseries \hspace*{0pt}\ignorespaces{}\hspace*{0pt} address}&\hspace*{0pt}\ignorespaces{}\hspace*{0pt} &\hspace*{0pt}\ignorespaces{}\hspace*{0pt} o &\hspace*{0pt}\ignorespaces{}\hspace*{0pt} o &\hspace*{0pt}\ignorespaces{}\hspace*{0pt} o &\hspace*{0pt}\ignorespaces{}\hspace*{0pt} o &\hspace*{0pt}\ignorespaces{}\hspace*{0pt} o &\hspace*{0pt}\ignorespaces{}\hspace*{0pt} o &\hspace*{0pt}\ignorespaces{}\hspace*{0pt} o &\hspace*{0pt}\ignorespaces{}\hspace*{0pt} &\hspace*{0pt}\ignorespaces{}\hspace*{0pt} o &\hspace*{0pt}\ignorespaces{}\hspace*{0pt} o &\hspace*{0pt}\ignorespaces{}\hspace*{0pt}\\ \hline {\bfseries \hspace*{0pt}\ignorespaces{}\hspace*{0pt} annote}&\hspace*{0pt}\ignorespaces{}\hspace*{0pt} &\hspace*{0pt}\ignorespaces{}\hspace*{0pt} &\hspace*{0pt}\ignorespaces{}\hspace*{0pt} &\hspace*{0pt}\ignorespaces{}\hspace*{0pt} &\hspace*{0pt}\ignorespaces{}\hspace*{0pt} &\hspace*{0pt}\ignorespaces{}\hspace*{0pt} &\hspace*{0pt}\ignorespaces{}\hspace*{0pt} &\hspace*{0pt}\ignorespaces{}\hspace*{0pt} &\hspace*{0pt}\ignorespaces{}\hspace*{0pt} &\hspace*{0pt}\ignorespaces{}\hspace*{0pt} &\hspace*{0pt}\ignorespaces{}\hspace*{0pt} &\hspace*{0pt}\ignorespaces{}\hspace*{0pt}\\ \hline {\bfseries \hspace*{0pt}\ignorespaces{}\hspace*{0pt} author}&\hspace*{0pt}\ignorespaces{}\hspace*{0pt} + &\hspace*{0pt}\ignorespaces{}\hspace*{0pt} * &\hspace*{0pt}\ignorespaces{}\hspace*{0pt} o &\hspace*{0pt}\ignorespaces{}\hspace*{0pt} *¹ &\hspace*{0pt}\ignorespaces{}\hspace*{0pt} + &\hspace*{0pt}\ignorespaces{}\hspace*{0pt} + &\hspace*{0pt}\ignorespaces{}\hspace*{0pt} o &\hspace*{0pt}\ignorespaces{}\hspace*{0pt} + &\hspace*{0pt}\ignorespaces{}\hspace*{0pt} o &\hspace*{0pt}\ignorespaces{}\hspace*{0pt} &\hspace*{0pt}\ignorespaces{}\hspace*{0pt} + &\hspace*{0pt}\ignorespaces{}\hspace*{0pt} +\\ \hline {\bfseries \hspace*{0pt}\ignorespaces{}\hspace*{0pt} booktitle}&\hspace*{0pt}\ignorespaces{}\hspace*{0pt} &\hspace*{0pt}\ignorespaces{}\hspace*{0pt} &\hspace*{0pt}\ignorespaces{}\hspace*{0pt} &\hspace*{0pt}\ignorespaces{}\hspace*{0pt} &\hspace*{0pt}\ignorespaces{}\hspace*{0pt} + &\hspace*{0pt}\ignorespaces{}\hspace*{0pt} + &\hspace*{0pt}\ignorespaces{}\hspace*{0pt} &\hspace*{0pt}\ignorespaces{}\hspace*{0pt} &\hspace*{0pt}\ignorespaces{}\hspace*{0pt} &\hspace*{0pt}\ignorespaces{}\hspace*{0pt} &\hspace*{0pt}\ignorespaces{}\hspace*{0pt} &\hspace*{0pt}\ignorespaces{}\hspace*{0pt}\\ \hline {\bfseries \hspace*{0pt}\ignorespaces{}\hspace*{0pt} chapter}&\hspace*{0pt}\ignorespaces{}\hspace*{0pt} &\hspace*{0pt}\ignorespaces{}\hspace*{0pt} &\hspace*{0pt}\ignorespaces{}\hspace*{0pt} &\hspace*{0pt}\ignorespaces{}\hspace*{0pt} *² &\hspace*{0pt}\ignorespaces{}\hspace*{0pt} o &\hspace*{0pt}\ignorespaces{}\hspace*{0pt} &\hspace*{0pt}\ignorespaces{}\hspace*{0pt} &\hspace*{0pt}\ignorespaces{}\hspace*{0pt} &\hspace*{0pt}\ignorespaces{}\hspace*{0pt} &\hspace*{0pt}\ignorespaces{}\hspace*{0pt} &\hspace*{0pt}\ignorespaces{}\hspace*{0pt} &\hspace*{0pt}\ignorespaces{}\hspace*{0pt}\\ \hline {\bfseries \hspace*{0pt}\ignorespaces{}\hspace*{0pt} crossref}&\hspace*{0pt}\ignorespaces{}\hspace*{0pt} &\hspace*{0pt}\ignorespaces{}\hspace*{0pt} &\hspace*{0pt}\ignorespaces{}\hspace*{0pt} &\hspace*{0pt}\ignorespaces{}\hspace*{0pt} &\hspace*{0pt}\ignorespaces{}\hspace*{0pt} &\hspace*{0pt}\ignorespaces{}\hspace*{0pt} &\hspace*{0pt}\ignorespaces{}\hspace*{0pt} &\hspace*{0pt}\ignorespaces{}\hspace*{0pt} &\hspace*{0pt}\ignorespaces{}\hspace*{0pt} &\hspace*{0pt}\ignorespaces{}\hspace*{0pt} &\hspace*{0pt}\ignorespaces{}\hspace*{0pt} &\hspace*{0pt}\ignorespaces{}\hspace*{0pt}\\ \hline {\bfseries \hspace*{0pt}\ignorespaces{}\hspace*{0pt} edition}&\hspace*{0pt}\ignorespaces{}\hspace*{0pt} &\hspace*{0pt}\ignorespaces{}\hspace*{0pt} o &\hspace*{0pt}\ignorespaces{}\hspace*{0pt} &\hspace*{0pt}\ignorespaces{}\hspace*{0pt} o &\hspace*{0pt}\ignorespaces{}\hspace*{0pt} o &\hspace*{0pt}\ignorespaces{}\hspace*{0pt} &\hspace*{0pt}\ignorespaces{}\hspace*{0pt} o &\hspace*{0pt}\ignorespaces{}\hspace*{0pt} &\hspace*{0pt}\ignorespaces{}\hspace*{0pt} &\hspace*{0pt}\ignorespaces{}\hspace*{0pt} &\hspace*{0pt}\ignorespaces{}\hspace*{0pt} &\hspace*{0pt}\ignorespaces{}\hspace*{0pt}\\ \hline {\bfseries \hspace*{0pt}\ignorespaces{}\hspace*{0pt} editor}&\hspace*{0pt}\ignorespaces{}\hspace*{0pt} &\hspace*{0pt}\ignorespaces{}\hspace*{0pt} * &\hspace*{0pt}\ignorespaces{}\hspace*{0pt} &\hspace*{0pt}\ignorespaces{}\hspace*{0pt} *¹ &\hspace*{0pt}\ignorespaces{}\hspace*{0pt} o &\hspace*{0pt}\ignorespaces{}\hspace*{0pt} o &\hspace*{0pt}\ignorespaces{}\hspace*{0pt} &\hspace*{0pt}\ignorespaces{}\hspace*{0pt} &\hspace*{0pt}\ignorespaces{}\hspace*{0pt} &\hspace*{0pt}\ignorespaces{}\hspace*{0pt} o &\hspace*{0pt}\ignorespaces{}\hspace*{0pt} &\hspace*{0pt}\ignorespaces{}\hspace*{0pt}\\ \hline {\bfseries \hspace*{0pt}\ignorespaces{}\hspace*{0pt} howpublished}&\hspace*{0pt}\ignorespaces{}\hspace*{0pt} &\hspace*{0pt}\ignorespaces{}\hspace*{0pt} &\hspace*{0pt}\ignorespaces{}\hspace*{0pt} o &\hspace*{0pt}\ignorespaces{}\hspace*{0pt} &\hspace*{0pt}\ignorespaces{}\hspace*{0pt} &\hspace*{0pt}\ignorespaces{}\hspace*{0pt} &\hspace*{0pt}\ignorespaces{}\hspace*{0pt} &\hspace*{0pt}\ignorespaces{}\hspace*{0pt} &\hspace*{0pt}\ignorespaces{}\hspace*{0pt} o &\hspace*{0pt}\ignorespaces{}\hspace*{0pt} &\hspace*{0pt}\ignorespaces{}\hspace*{0pt} &\hspace*{0pt}\ignorespaces{}\hspace*{0pt}\\ \hline {\bfseries \hspace*{0pt}\ignorespaces{}\hspace*{0pt} institution}&\hspace*{0pt}\ignorespaces{}\hspace*{0pt} &\hspace*{0pt}\ignorespaces{}\hspace*{0pt} &\hspace*{0pt}\ignorespaces{}\hspace*{0pt} &\hspace*{0pt}\ignorespaces{}\hspace*{0pt} &\hspace*{0pt}\ignorespaces{}\hspace*{0pt} &\hspace*{0pt}\ignorespaces{}\hspace*{0pt} &\hspace*{0pt}\ignorespaces{}\hspace*{0pt} &\hspace*{0pt}\ignorespaces{}\hspace*{0pt} &\hspace*{0pt}\ignorespaces{}\hspace*{0pt} &\hspace*{0pt}\ignorespaces{}\hspace*{0pt} &\hspace*{0pt}\ignorespaces{}\hspace*{0pt} + &\hspace*{0pt}\ignorespaces{}\hspace*{0pt}\\ \hline {\bfseries \hspace*{0pt}\ignorespaces{}\hspace*{0pt} journal}&\hspace*{0pt}\ignorespaces{}\hspace*{0pt} + &\hspace*{0pt}\ignorespaces{}\hspace*{0pt} &\hspace*{0pt}\ignorespaces{}\hspace*{0pt} &\hspace*{0pt}\ignorespaces{}\hspace*{0pt} &\hspace*{0pt}\ignorespaces{}\hspace*{0pt} &\hspace*{0pt}\ignorespaces{}\hspace*{0pt} &\hspace*{0pt}\ignorespaces{}\hspace*{0pt} &\hspace*{0pt}\ignorespaces{}\hspace*{0pt} &\hspace*{0pt}\ignorespaces{}\hspace*{0pt} &\hspace*{0pt}\ignorespaces{}\hspace*{0pt} &\hspace*{0pt}\ignorespaces{}\hspace*{0pt} &\hspace*{0pt}\ignorespaces{}\hspace*{0pt}\\ \hline {\bfseries \hspace*{0pt}\ignorespaces{}\hspace*{0pt} key}&\hspace*{0pt}\ignorespaces{}\hspace*{0pt} &\hspace*{0pt}\ignorespaces{}\hspace*{0pt} &\hspace*{0pt}\ignorespaces{}\hspace*{0pt} &\hspace*{0pt}\ignorespaces{}\hspace*{0pt} &\hspace*{0pt}\ignorespaces{}\hspace*{0pt} &\hspace*{0pt}\ignorespaces{}\hspace*{0pt} &\hspace*{0pt}\ignorespaces{}\hspace*{0pt} &\hspace*{0pt}\ignorespaces{}\hspace*{0pt} &\hspace*{0pt}\ignorespaces{}\hspace*{0pt} &\hspace*{0pt}\ignorespaces{}\hspace*{0pt} &\hspace*{0pt}\ignorespaces{}\hspace*{0pt} &\hspace*{0pt}\ignorespaces{}\hspace*{0pt}\\ \hline {\bfseries \hspace*{0pt}\ignorespaces{}\hspace*{0pt} month}&\hspace*{0pt}\ignorespaces{}\hspace*{0pt} o &\hspace*{0pt}\ignorespaces{}\hspace*{0pt} o &\hspace*{0pt}\ignorespaces{}\hspace*{0pt} o &\hspace*{0pt}\ignorespaces{}\hspace*{0pt} o &\hspace*{0pt}\ignorespaces{}\hspace*{0pt} o &\hspace*{0pt}\ignorespaces{}\hspace*{0pt} o &\hspace*{0pt}\ignorespaces{}\hspace*{0pt} o &\hspace*{0pt}\ignorespaces{}\hspace*{0pt} o &\hspace*{0pt}\ignorespaces{}\hspace*{0pt} o &\hspace*{0pt}\ignorespaces{}\hspace*{0pt} o &\hspace*{0pt}\ignorespaces{}\hspace*{0pt} o &\hspace*{0pt}\ignorespaces{}\hspace*{0pt}o\\ \hline {\bfseries \hspace*{0pt}\ignorespaces{}\hspace*{0pt} note}&\hspace*{0pt}\ignorespaces{}\hspace*{0pt} o &\hspace*{0pt}\ignorespaces{}\hspace*{0pt} o &\hspace*{0pt}\ignorespaces{}\hspace*{0pt} o &\hspace*{0pt}\ignorespaces{}\hspace*{0pt} o &\hspace*{0pt}\ignorespaces{}\hspace*{0pt} o &\hspace*{0pt}\ignorespaces{}\hspace*{0pt} o &\hspace*{0pt}\ignorespaces{}\hspace*{0pt} o &\hspace*{0pt}\ignorespaces{}\hspace*{0pt} o &\hspace*{0pt}\ignorespaces{}\hspace*{0pt} o &\hspace*{0pt}\ignorespaces{}\hspace*{0pt} o &\hspace*{0pt}\ignorespaces{}\hspace*{0pt} o &\hspace*{0pt}\ignorespaces{}\hspace*{0pt} +\\ \hline {\bfseries \hspace*{0pt}\ignorespaces{}\hspace*{0pt} number}&\hspace*{0pt}\ignorespaces{}\hspace*{0pt} o &\hspace*{0pt}\ignorespaces{}\hspace*{0pt} o &\hspace*{0pt}\ignorespaces{}\hspace*{0pt} &\hspace*{0pt}\ignorespaces{}\hspace*{0pt} o &\hspace*{0pt}\ignorespaces{}\hspace*{0pt} o &\hspace*{0pt}\ignorespaces{}\hspace*{0pt} o &\hspace*{0pt}\ignorespaces{}\hspace*{0pt} &\hspace*{0pt}\ignorespaces{}\hspace*{0pt} &\hspace*{0pt}\ignorespaces{}\hspace*{0pt} &\hspace*{0pt}\ignorespaces{}\hspace*{0pt} o &\hspace*{0pt}\ignorespaces{}\hspace*{0pt} o &\hspace*{0pt}\ignorespaces{}\hspace*{0pt}\\ \hline {\bfseries \hspace*{0pt}\ignorespaces{}\hspace*{0pt} organization}&\hspace*{0pt}\ignorespaces{}\hspace*{0pt} &\hspace*{0pt}\ignorespaces{}\hspace*{0pt} &\hspace*{0pt}\ignorespaces{}\hspace*{0pt} &\hspace*{0pt}\ignorespaces{}\hspace*{0pt} &\hspace*{0pt}\ignorespaces{}\hspace*{0pt} &\hspace*{0pt}\ignorespaces{}\hspace*{0pt} o &\hspace*{0pt}\ignorespaces{}\hspace*{0pt} o &\hspace*{0pt}\ignorespaces{}\hspace*{0pt} &\hspace*{0pt}\ignorespaces{}\hspace*{0pt} &\hspace*{0pt}\ignorespaces{}\hspace*{0pt} o &\hspace*{0pt}\ignorespaces{}\hspace*{0pt} &\hspace*{0pt}\ignorespaces{}\hspace*{0pt}\\ \hline {\bfseries \hspace*{0pt}\ignorespaces{}\hspace*{0pt} pages}&\hspace*{0pt}\ignorespaces{}\hspace*{0pt} o &\hspace*{0pt}\ignorespaces{}\hspace*{0pt} &\hspace*{0pt}\ignorespaces{}\hspace*{0pt} &\hspace*{0pt}\ignorespaces{}\hspace*{0pt} *² &\hspace*{0pt}\ignorespaces{}\hspace*{0pt} o &\hspace*{0pt}\ignorespaces{}\hspace*{0pt} o &\hspace*{0pt}\ignorespaces{}\hspace*{0pt} &\hspace*{0pt}\ignorespaces{}\hspace*{0pt} &\hspace*{0pt}\ignorespaces{}\hspace*{0pt} &\hspace*{0pt}\ignorespaces{}\hspace*{0pt} &\hspace*{0pt}\ignorespaces{}\hspace*{0pt} &\hspace*{0pt}\ignorespaces{}\hspace*{0pt}\\ \hline {\bfseries \hspace*{0pt}\ignorespaces{}\hspace*{0pt} publisher}&\hspace*{0pt}\ignorespaces{}\hspace*{0pt} &\hspace*{0pt}\ignorespaces{}\hspace*{0pt} + &\hspace*{0pt}\ignorespaces{}\hspace*{0pt} &\hspace*{0pt}\ignorespaces{}\hspace*{0pt} + &\hspace*{0pt}\ignorespaces{}\hspace*{0pt} + &\hspace*{0pt}\ignorespaces{}\hspace*{0pt} o &\hspace*{0pt}\ignorespaces{}\hspace*{0pt} &\hspace*{0pt}\ignorespaces{}\hspace*{0pt} &\hspace*{0pt}\ignorespaces{}\hspace*{0pt} &\hspace*{0pt}\ignorespaces{}\hspace*{0pt} o &\hspace*{0pt}\ignorespaces{}\hspace*{0pt} &\hspace*{0pt}\ignorespaces{}\hspace*{0pt}\\ \hline {\bfseries \hspace*{0pt}\ignorespaces{}\hspace*{0pt} school}&\hspace*{0pt}\ignorespaces{}\hspace*{0pt} &\hspace*{0pt}\ignorespaces{}\hspace*{0pt} &\hspace*{0pt}\ignorespaces{}\hspace*{0pt} &\hspace*{0pt}\ignorespaces{}\hspace*{0pt} &\hspace*{0pt}\ignorespaces{}\hspace*{0pt} &\hspace*{0pt}\ignorespaces{}\hspace*{0pt} &\hspace*{0pt}\ignorespaces{}\hspace*{0pt} &\hspace*{0pt}\ignorespaces{}\hspace*{0pt} + &\hspace*{0pt}\ignorespaces{}\hspace*{0pt} &\hspace*{0pt}\ignorespaces{}\hspace*{0pt} &\hspace*{0pt}\ignorespaces{}\hspace*{0pt} &\hspace*{0pt}\ignorespaces{}\hspace*{0pt}\\ \hline {\bfseries \hspace*{0pt}\ignorespaces{}\hspace*{0pt} series}&\hspace*{0pt}\ignorespaces{}\hspace*{0pt} &\hspace*{0pt}\ignorespaces{}\hspace*{0pt} o &\hspace*{0pt}\ignorespaces{}\hspace*{0pt} &\hspace*{0pt}\ignorespaces{}\hspace*{0pt} o &\hspace*{0pt}\ignorespaces{}\hspace*{0pt} o &\hspace*{0pt}\ignorespaces{}\hspace*{0pt} o &\hspace*{0pt}\ignorespaces{}\hspace*{0pt} &\hspace*{0pt}\ignorespaces{}\hspace*{0pt} &\hspace*{0pt}\ignorespaces{}\hspace*{0pt} &\hspace*{0pt}\ignorespaces{}\hspace*{0pt} o &\hspace*{0pt}\ignorespaces{}\hspace*{0pt} &\hspace*{0pt}\ignorespaces{}\hspace*{0pt}\\ \hline {\bfseries \hspace*{0pt}\ignorespaces{}\hspace*{0pt} title}&\hspace*{0pt}\ignorespaces{}\hspace*{0pt} + &\hspace*{0pt}\ignorespaces{}\hspace*{0pt} + &\hspace*{0pt}\ignorespaces{}\hspace*{0pt} + &\hspace*{0pt}\ignorespaces{}\hspace*{0pt} + &\hspace*{0pt}\ignorespaces{}\hspace*{0pt} + &\hspace*{0pt}\ignorespaces{}\hspace*{0pt} + &\hspace*{0pt}\ignorespaces{}\hspace*{0pt} + &\hspace*{0pt}\ignorespaces{}\hspace*{0pt} + &\hspace*{0pt}\ignorespaces{}\hspace*{0pt} o &\hspace*{0pt}\ignorespaces{}\hspace*{0pt} + &\hspace*{0pt}\ignorespaces{}\hspace*{0pt} + &\hspace*{0pt}\ignorespaces{}\hspace*{0pt} +\\ \hline {\bfseries \hspace*{0pt}\ignorespaces{}\hspace*{0pt} type}&\hspace*{0pt}\ignorespaces{}\hspace*{0pt} &\hspace*{0pt}\ignorespaces{}\hspace*{0pt} &\hspace*{0pt}\ignorespaces{}\hspace*{0pt} &\hspace*{0pt}\ignorespaces{}\hspace*{0pt} o &\hspace*{0pt}\ignorespaces{}\hspace*{0pt} o &\hspace*{0pt}\ignorespaces{}\hspace*{0pt} &\hspace*{0pt}\ignorespaces{}\hspace*{0pt} &\hspace*{0pt}\ignorespaces{}\hspace*{0pt} o &\hspace*{0pt}\ignorespaces{}\hspace*{0pt} &\hspace*{0pt}\ignorespaces{}\hspace*{0pt} &\hspace*{0pt}\ignorespaces{}\hspace*{0pt} o &\hspace*{0pt}\ignorespaces{}\hspace*{0pt}\\ \hline {\bfseries \hspace*{0pt}\ignorespaces{}\hspace*{0pt} volume}&\hspace*{0pt}\ignorespaces{}\hspace*{0pt} o &\hspace*{0pt}\ignorespaces{}\hspace*{0pt} o &\hspace*{0pt}\ignorespaces{}\hspace*{0pt} &\hspace*{0pt}\ignorespaces{}\hspace*{0pt} o &\hspace*{0pt}\ignorespaces{}\hspace*{0pt} o &\hspace*{0pt}\ignorespaces{}\hspace*{0pt} o &\hspace*{0pt}\ignorespaces{}\hspace*{0pt} &\hspace*{0pt}\ignorespaces{}\hspace*{0pt} &\hspace*{0pt}\ignorespaces{}\hspace*{0pt} &\hspace*{0pt}\ignorespaces{}\hspace*{0pt} o &\hspace*{0pt}\ignorespaces{}\hspace*{0pt} &\hspace*{0pt}\ignorespaces{}\hspace*{0pt}\\ \hline {\bfseries \hspace*{0pt}\ignorespaces{}\hspace*{0pt} year}&\hspace*{0pt}\ignorespaces{}\hspace*{0pt} + &\hspace*{0pt}\ignorespaces{}\hspace*{0pt} + &\hspace*{0pt}\ignorespaces{}\hspace*{0pt} o &\hspace*{0pt}\ignorespaces{}\hspace*{0pt} + &\hspace*{0pt}\ignorespaces{}\hspace*{0pt} + &\hspace*{0pt}\ignorespaces{}\hspace*{0pt} + &\hspace*{0pt}\ignorespaces{}\hspace*{0pt} o &\hspace*{0pt}\ignorespaces{}\hspace*{0pt} + &\hspace*{0pt}\ignorespaces{}\hspace*{0pt} o &\hspace*{0pt}\ignorespaces{}\hspace*{0pt} + &\hspace*{0pt}\ignorespaces{}\hspace*{0pt} + &\hspace*{0pt}\ignorespaces{}\hspace*{0pt}o\\ \hline 
\end{longtable}
}}
+ Required fields, O Optional fields
\subsection{Authors}
\label{677}

BibTeX can be quite clever with names of authors. It can accept names in {\itshape \setmainfont[Path=/usr/share/fonts/truetype/cmu/,UprightFont=cmunrm.ttf,BoldFont=cmunbx.ttf,ItalicFont=cmunti.ttf,BoldItalicFont=cmunbi.ttf]{cmunti.ttf}\setmonofont[Path=/usr/share/fonts/truetype/cmu/,UprightFont=cmuntt.ttf,BoldFont=cmuntb.ttf,ItalicFont=cmunit.ttf,BoldItalicFont=cmuntx.ttf]{cmunti.ttf}\itshape forename surname}{$\text{ }$}\setmainfont[Path=/usr/share/fonts/truetype/cmu/,UprightFont=cmunrm.ttf,BoldFont=cmunbx.ttf,ItalicFont=cmunti.ttf,BoldItalicFont=cmunbi.ttf]{cmunrm.ttf}\setmonofont[Path=/usr/share/fonts/truetype/cmu/,UprightFont=cmuntt.ttf,BoldFont=cmuntb.ttf,ItalicFont=cmunit.ttf,BoldItalicFont=cmuntx.ttf]{cmunrm.ttf} or {\itshape \setmainfont[Path=/usr/share/fonts/truetype/cmu/,UprightFont=cmunrm.ttf,BoldFont=cmunbx.ttf,ItalicFont=cmunti.ttf,BoldItalicFont=cmunbi.ttf]{cmunti.ttf}\setmonofont[Path=/usr/share/fonts/truetype/cmu/,UprightFont=cmuntt.ttf,BoldFont=cmuntb.ttf,ItalicFont=cmunit.ttf,BoldItalicFont=cmuntx.ttf]{cmunti.ttf}\itshape surname, forename}\setmainfont[Path=/usr/share/fonts/truetype/cmu/,UprightFont=cmunrm.ttf,BoldFont=cmunbx.ttf,ItalicFont=cmunti.ttf,BoldItalicFont=cmunbi.ttf]{cmunrm.ttf}\setmonofont[Path=/usr/share/fonts/truetype/cmu/,UprightFont=cmuntt.ttf,BoldFont=cmuntb.ttf,ItalicFont=cmunit.ttf,BoldItalicFont=cmuntx.ttf]{cmunrm.ttf}. I personally use the former, but remember that the order you input them (or any data within an entry for that matter) is customizable and so you can get BibTeX to manipulate the input and then output it however you like. If you use the {\itshape \setmainfont[Path=/usr/share/fonts/truetype/cmu/,UprightFont=cmunrm.ttf,BoldFont=cmunbx.ttf,ItalicFont=cmunti.ttf,BoldItalicFont=cmunbi.ttf]{cmunti.ttf}\setmonofont[Path=/usr/share/fonts/truetype/cmu/,UprightFont=cmuntt.ttf,BoldFont=cmuntb.ttf,ItalicFont=cmunit.ttf,BoldItalicFont=cmuntx.ttf]{cmunti.ttf}\itshape forename surname}{$\text{ }$}\setmainfont[Path=/usr/share/fonts/truetype/cmu/,UprightFont=cmunrm.ttf,BoldFont=cmunbx.ttf,ItalicFont=cmunti.ttf,BoldItalicFont=cmunbi.ttf]{cmunrm.ttf}\setmonofont[Path=/usr/share/fonts/truetype/cmu/,UprightFont=cmuntt.ttf,BoldFont=cmuntb.ttf,ItalicFont=cmunit.ttf,BoldItalicFont=cmuntx.ttf]{cmunrm.ttf} method, then you must be careful with a few special names, where there are compound surnames, for example \symbol{34}John von Neumann\symbol{34}. In this form, BibTeX assumes that the last word is the surname, and everything before is the forename, plus any middle names. You must therefore manually tell BibTeX to keep the \textquotesingle{}von\textquotesingle{} and \textquotesingle{}Neumann\textquotesingle{} together. This is achieved easily using curly braces. So the final result would be \symbol{34}John \{von Neumann\}\symbol{34}.  This is easily avoided with the {\itshape \setmainfont[Path=/usr/share/fonts/truetype/cmu/,UprightFont=cmunrm.ttf,BoldFont=cmunbx.ttf,ItalicFont=cmunti.ttf,BoldItalicFont=cmunbi.ttf]{cmunti.ttf}\setmonofont[Path=/usr/share/fonts/truetype/cmu/,UprightFont=cmuntt.ttf,BoldFont=cmuntb.ttf,ItalicFont=cmunit.ttf,BoldItalicFont=cmuntx.ttf]{cmunti.ttf}\itshape surname, forename}\setmainfont[Path=/usr/share/fonts/truetype/cmu/,UprightFont=cmunrm.ttf,BoldFont=cmunbx.ttf,ItalicFont=cmunti.ttf,BoldItalicFont=cmunbi.ttf]{cmunrm.ttf}\setmonofont[Path=/usr/share/fonts/truetype/cmu/,UprightFont=cmuntt.ttf,BoldFont=cmuntb.ttf,ItalicFont=cmunit.ttf,BoldItalicFont=cmuntx.ttf]{cmunrm.ttf}, since you have a comma to separate the surname from the forename.

Secondly, there is the issue of how to tell BibTeX when a reference has more than one author. This is very simply done by putting the keyword {\itshape \setmainfont[Path=/usr/share/fonts/truetype/cmu/,UprightFont=cmunrm.ttf,BoldFont=cmunbx.ttf,ItalicFont=cmunti.ttf,BoldItalicFont=cmunbi.ttf]{cmunti.ttf}\setmonofont[Path=/usr/share/fonts/truetype/cmu/,UprightFont=cmuntt.ttf,BoldFont=cmuntb.ttf,ItalicFont=cmunit.ttf,BoldItalicFont=cmuntx.ttf]{cmunti.ttf}\itshape and}{$\text{ }$}\setmainfont[Path=/usr/share/fonts/truetype/cmu/,UprightFont=cmunrm.ttf,BoldFont=cmunbx.ttf,ItalicFont=cmunti.ttf,BoldItalicFont=cmunbi.ttf]{cmunrm.ttf}\setmonofont[Path=/usr/share/fonts/truetype/cmu/,UprightFont=cmuntt.ttf,BoldFont=cmuntb.ttf,ItalicFont=cmunit.ttf,BoldItalicFont=cmuntx.ttf]{cmunrm.ttf} in between every author. As we can see from another example:


\begin{Shaded}
\begin{Highlighting}[]

\KeywordTok{@book}\NormalTok{\{}\OtherTok{goossens93}\NormalTok{,}\newline
\ensuremath{\text{ }}\ensuremath{\text{ }}\ensuremath{\text{ }}\ensuremath{\text{ }}\DataTypeTok{author}\ensuremath{\text{ }}\ensuremath{\text{ }}\ensuremath{\text{ }}\ensuremath{\text{ }}\NormalTok{=\ensuremath{\text{ }}"}\StringTok{Michel\ensuremath{\text{ }}Goossens\ensuremath{\text{ }}and\ensuremath{\text{ }}Frank\ensuremath{\text{ }}Mittelbach\ensuremath{\text{ }}and\ensuremath{\text{ }}Alexander\ensuremath{\text{ }}Samarin}\NormalTok{",}\newline
\ensuremath{\text{ }}\ensuremath{\text{ }}\ensuremath{\text{ }}\ensuremath{\text{ }}\DataTypeTok{title}\ensuremath{\text{ }}\ensuremath{\text{ }}\ensuremath{\text{ }}\ensuremath{\text{ }}\ensuremath{\text{ }}\NormalTok{=\ensuremath{\text{ }}"}\StringTok{The\ensuremath{\text{ }}LaTeX\ensuremath{\text{ }}Companion}\NormalTok{",}\newline
\ensuremath{\text{ }}\ensuremath{\text{ }}\ensuremath{\text{ }}\ensuremath{\text{ }}\DataTypeTok{year}\ensuremath{\text{ }}\ensuremath{\text{ }}\ensuremath{\text{ }}\ensuremath{\text{ }}\ensuremath{\text{ }}\ensuremath{\text{ }}\NormalTok{=\ensuremath{\text{ }}"}\StringTok{1993}\NormalTok{",}\newline
\ensuremath{\text{ }}\ensuremath{\text{ }}\ensuremath{\text{ }}\ensuremath{\text{ }}\DataTypeTok{publisher}\ensuremath{\text{ }}\NormalTok{=\ensuremath{\text{ }}"}\StringTok{Addison-Wesley}\NormalTok{",}\newline
\ensuremath{\text{ }}\ensuremath{\text{ }}\ensuremath{\text{ }}\ensuremath{\text{ }}\DataTypeTok{address}\ensuremath{\text{ }}\ensuremath{\text{ }}\ensuremath{\text{ }}\NormalTok{=\ensuremath{\text{ }}"}\StringTok{Reading,\ensuremath{\text{ }}Massachusetts}\NormalTok{"}\newline
\NormalTok{\}}\newline
\end{Highlighting}
\end{Shaded}


This book has three authors, and each is separated as described. Of course, when BibTeX processes and outputs this, there will only be an \textquotesingle{}and\textquotesingle{} between the penultimate and last authors, but within the .bib file, it needs the {\itshape \setmainfont[Path=/usr/share/fonts/truetype/cmu/,UprightFont=cmunrm.ttf,BoldFont=cmunbx.ttf,ItalicFont=cmunti.ttf,BoldItalicFont=cmunbi.ttf]{cmunti.ttf}\setmonofont[Path=/usr/share/fonts/truetype/cmu/,UprightFont=cmuntt.ttf,BoldFont=cmuntb.ttf,ItalicFont=cmunit.ttf,BoldItalicFont=cmuntx.ttf]{cmunti.ttf}\itshape and}\setmainfont[Path=/usr/share/fonts/truetype/cmu/,UprightFont=cmunrm.ttf,BoldFont=cmunbx.ttf,ItalicFont=cmunti.ttf,BoldItalicFont=cmunbi.ttf]{cmunrm.ttf}\setmonofont[Path=/usr/share/fonts/truetype/cmu/,UprightFont=cmuntt.ttf,BoldFont=cmuntb.ttf,ItalicFont=cmunit.ttf,BoldItalicFont=cmuntx.ttf]{cmunrm.ttf}s so that it can keep track of the individual authors.
\subsection{Standard templates}
\label{678}
Be careful if you copy the following templates, the \% sign is not valid to comment out lines in bibtex files. If you want to comment out a line, you have to put it outside the entry.
{\bfseries
\begin{mydescription}@article 
\end{mydescription}
}
\begin{myquote}\item{} An article from a magazine or a journal.
\end{myquote}

\begin{myquote}
\item{} 
\begin{myitemize}
\item{} Required fields: author, title, journal, year.
\item{} Optional fields: volume, number, pages, month, note.
\end{myitemize}

\end{myquote}


\begin{Shaded}
\begin{Highlighting}[]

\KeywordTok{@article}\NormalTok{\{}\OtherTok{Xarticle}\NormalTok{,}\newline
\ensuremath{\text{ }}\ensuremath{\text{ }}\ensuremath{\text{ }}\ensuremath{\text{ }}\DataTypeTok{author}\ensuremath{\text{ }}\ensuremath{\text{ }}\ensuremath{\text{ }}\ensuremath{\text{ }}\NormalTok{=\ensuremath{\text{ }}"",}\newline
\ensuremath{\text{ }}\ensuremath{\text{ }}\ensuremath{\text{ }}\ensuremath{\text{ }}\DataTypeTok{title}\ensuremath{\text{ }}\ensuremath{\text{ }}\ensuremath{\text{ }}\ensuremath{\text{ }}\ensuremath{\text{ }}\NormalTok{=\ensuremath{\text{ }}"",}\newline
\ensuremath{\text{ }}\ensuremath{\text{ }}\ensuremath{\text{ }}\ensuremath{\text{ }}\DataTypeTok{journal}\ensuremath{\text{ }}\ensuremath{\text{ }}\ensuremath{\text{ }}\NormalTok{=\ensuremath{\text{ }}"",}\newline
\ensuremath{\text{ }}\ensuremath{\text{ }}\ensuremath{\text{ }}\ensuremath{\text{ }}\AlertTok{\%}\StringTok{volume}\ensuremath{\text{ }}\ensuremath{\text{ }}\ensuremath{\text{ }}\NormalTok{=\ensuremath{\text{ }}"",}\newline
\ensuremath{\text{ }}\ensuremath{\text{ }}\ensuremath{\text{ }}\ensuremath{\text{ }}\AlertTok{\%}\StringTok{number}\ensuremath{\text{ }}\ensuremath{\text{ }}\ensuremath{\text{ }}\NormalTok{=\ensuremath{\text{ }}"",}\newline
\ensuremath{\text{ }}\ensuremath{\text{ }}\ensuremath{\text{ }}\ensuremath{\text{ }}\AlertTok{\%}\StringTok{pages}\ensuremath{\text{ }}\ensuremath{\text{ }}\ensuremath{\text{ }}\ensuremath{\text{ }}\NormalTok{=\ensuremath{\text{ }}"",}\newline
\ensuremath{\text{ }}\ensuremath{\text{ }}\ensuremath{\text{ }}\ensuremath{\text{ }}\DataTypeTok{year}\ensuremath{\text{ }}\ensuremath{\text{ }}\ensuremath{\text{ }}\ensuremath{\text{ }}\ensuremath{\text{ }}\ensuremath{\text{ }}\NormalTok{=\ensuremath{\text{ }}"}\StringTok{XXXX}\NormalTok{",}\newline
\ensuremath{\text{ }}\ensuremath{\text{ }}\ensuremath{\text{ }}\ensuremath{\text{ }}\AlertTok{\%}\StringTok{month}\ensuremath{\text{ }}\ensuremath{\text{ }}\ensuremath{\text{ }}\ensuremath{\text{ }}\NormalTok{=\ensuremath{\text{ }}"",}\newline
\ensuremath{\text{ }}\ensuremath{\text{ }}\ensuremath{\text{ }}\ensuremath{\text{ }}\AlertTok{\%}\StringTok{note}\ensuremath{\text{ }}\ensuremath{\text{ }}\ensuremath{\text{ }}\ensuremath{\text{ }}\ensuremath{\text{ }}\NormalTok{=\ensuremath{\text{ }}"",}\newline
\NormalTok{\}}\newline
\end{Highlighting}
\end{Shaded}

{\bfseries
\begin{mydescription}@book 
\end{mydescription}
}
\begin{myquote}\item{} A published book
\end{myquote}

\begin{myquote}
\item{} 
\begin{myitemize}
\item{} Required fields: author/editor, title, publisher, year.
\item{} Optional fields: volume/number, series, address, edition, month, note.
\end{myitemize}

\end{myquote}


\begin{Shaded}
\begin{Highlighting}[]

\KeywordTok{@book}\NormalTok{\{}\OtherTok{Xbook}\NormalTok{,}\newline
\ensuremath{\text{ }}\ensuremath{\text{ }}\ensuremath{\text{ }}\ensuremath{\text{ }}\DataTypeTok{author}\ensuremath{\text{ }}\ensuremath{\text{ }}\ensuremath{\text{ }}\ensuremath{\text{ }}\NormalTok{=\ensuremath{\text{ }}"",}\newline
\ensuremath{\text{ }}\ensuremath{\text{ }}\ensuremath{\text{ }}\ensuremath{\text{ }}\DataTypeTok{title}\ensuremath{\text{ }}\ensuremath{\text{ }}\ensuremath{\text{ }}\ensuremath{\text{ }}\ensuremath{\text{ }}\NormalTok{=\ensuremath{\text{ }}"",}\newline
\ensuremath{\text{ }}\ensuremath{\text{ }}\ensuremath{\text{ }}\ensuremath{\text{ }}\DataTypeTok{publisher}\ensuremath{\text{ }}\NormalTok{=\ensuremath{\text{ }}"",}\newline
\ensuremath{\text{ }}\ensuremath{\text{ }}\ensuremath{\text{ }}\ensuremath{\text{ }}\AlertTok{\%}\StringTok{volume}\ensuremath{\text{ }}\ensuremath{\text{ }}\ensuremath{\text{ }}\NormalTok{=\ensuremath{\text{ }}"",}\newline
\ensuremath{\text{ }}\ensuremath{\text{ }}\ensuremath{\text{ }}\ensuremath{\text{ }}\AlertTok{\%}\StringTok{number}\ensuremath{\text{ }}\ensuremath{\text{ }}\ensuremath{\text{ }}\NormalTok{=\ensuremath{\text{ }}"",}\newline
\ensuremath{\text{ }}\ensuremath{\text{ }}\ensuremath{\text{ }}\ensuremath{\text{ }}\AlertTok{\%}\StringTok{series}\ensuremath{\text{ }}\ensuremath{\text{ }}\ensuremath{\text{ }}\NormalTok{=\ensuremath{\text{ }}"",}\newline
\ensuremath{\text{ }}\ensuremath{\text{ }}\ensuremath{\text{ }}\ensuremath{\text{ }}\AlertTok{\%}\StringTok{address}\ensuremath{\text{ }}\ensuremath{\text{ }}\NormalTok{=\ensuremath{\text{ }}"",}\newline
\ensuremath{\text{ }}\ensuremath{\text{ }}\ensuremath{\text{ }}\ensuremath{\text{ }}\AlertTok{\%}\StringTok{edition}\ensuremath{\text{ }}\ensuremath{\text{ }}\NormalTok{=\ensuremath{\text{ }}"",}\newline
\ensuremath{\text{ }}\ensuremath{\text{ }}\ensuremath{\text{ }}\ensuremath{\text{ }}\DataTypeTok{year}\ensuremath{\text{ }}\ensuremath{\text{ }}\ensuremath{\text{ }}\ensuremath{\text{ }}\ensuremath{\text{ }}\ensuremath{\text{ }}\NormalTok{=\ensuremath{\text{ }}"}\StringTok{XXXX}\NormalTok{",}\newline
\ensuremath{\text{ }}\ensuremath{\text{ }}\ensuremath{\text{ }}\ensuremath{\text{ }}\AlertTok{\%}\StringTok{month}\ensuremath{\text{ }}\ensuremath{\text{ }}\ensuremath{\text{ }}\ensuremath{\text{ }}\NormalTok{=\ensuremath{\text{ }}"",}\newline
\ensuremath{\text{ }}\ensuremath{\text{ }}\ensuremath{\text{ }}\ensuremath{\text{ }}\AlertTok{\%}\StringTok{note}\ensuremath{\text{ }}\ensuremath{\text{ }}\ensuremath{\text{ }}\ensuremath{\text{ }}\ensuremath{\text{ }}\NormalTok{=\ensuremath{\text{ }}"",}\newline
\NormalTok{\}}\newline
\end{Highlighting}
\end{Shaded}

{\bfseries
\begin{mydescription}@booklet 
\end{mydescription}
}
\begin{myquote}\item{} A bound work without a named publisher or sponsor.
\end{myquote}

\begin{myquote}
\item{} 
\begin{myitemize}
\item{} Required fields: title.
\item{} Optional fields: author, howpublished, address, month, year, note.
\end{myitemize}

\end{myquote}


\begin{Shaded}
\begin{Highlighting}[]

\KeywordTok{@booklet}\NormalTok{\{}\OtherTok{Xbooklet}\NormalTok{,}\newline
\ensuremath{\text{ }}\ensuremath{\text{ }}\ensuremath{\text{ }}\ensuremath{\text{ }}\AlertTok{\%}\StringTok{author}\ensuremath{\text{ }}\ensuremath{\text{ }}\ensuremath{\text{ }}\NormalTok{=\ensuremath{\text{ }}"",}\newline
\ensuremath{\text{ }}\ensuremath{\text{ }}\ensuremath{\text{ }}\ensuremath{\text{ }}\DataTypeTok{title}\ensuremath{\text{ }}\ensuremath{\text{ }}\ensuremath{\text{ }}\ensuremath{\text{ }}\ensuremath{\text{ }}\NormalTok{=\ensuremath{\text{ }}"",}\newline
\ensuremath{\text{ }}\ensuremath{\text{ }}\ensuremath{\text{ }}\ensuremath{\text{ }}\AlertTok{\%}\StringTok{howpublished}\ensuremath{\text{ }}\ensuremath{\text{ }}\ensuremath{\text{ }}\NormalTok{=\ensuremath{\text{ }}"",}\newline
\ensuremath{\text{ }}\ensuremath{\text{ }}\ensuremath{\text{ }}\ensuremath{\text{ }}\AlertTok{\%}\StringTok{address}\ensuremath{\text{ }}\ensuremath{\text{ }}\NormalTok{=\ensuremath{\text{ }}"",}\newline
\ensuremath{\text{ }}\ensuremath{\text{ }}\ensuremath{\text{ }}\ensuremath{\text{ }}\AlertTok{\%}\StringTok{year}\ensuremath{\text{ }}\ensuremath{\text{ }}\ensuremath{\text{ }}\ensuremath{\text{ }}\ensuremath{\text{ }}\ensuremath{\text{ }}\NormalTok{=\ensuremath{\text{ }}"}\StringTok{XXXX}\NormalTok{",}\newline
\ensuremath{\text{ }}\ensuremath{\text{ }}\ensuremath{\text{ }}\ensuremath{\text{ }}\AlertTok{\%}\StringTok{month}\ensuremath{\text{ }}\ensuremath{\text{ }}\ensuremath{\text{ }}\ensuremath{\text{ }}\NormalTok{=\ensuremath{\text{ }}"",}\newline
\ensuremath{\text{ }}\ensuremath{\text{ }}\ensuremath{\text{ }}\ensuremath{\text{ }}\AlertTok{\%}\StringTok{note}\ensuremath{\text{ }}\ensuremath{\text{ }}\ensuremath{\text{ }}\ensuremath{\text{ }}\ensuremath{\text{ }}\NormalTok{=\ensuremath{\text{ }}"",}\newline
\NormalTok{\}}\newline
\end{Highlighting}
\end{Shaded}

{\bfseries
\begin{mydescription}@conference 
\end{mydescription}
}
\begin{myquote}\item{} Equal to inproceedings
\end{myquote}

\begin{myquote}
\item{} 
\begin{myitemize}
\item{} Required fields: author, title, booktitle, year.
\item{} Optional fields: editor, volume/number, series, pages, address, month, organization, publisher, note.
\end{myitemize}

\end{myquote}


\begin{Shaded}
\begin{Highlighting}[]

\KeywordTok{@conference}\NormalTok{\{}\OtherTok{Xconference}\NormalTok{,}\newline
\ensuremath{\text{ }}\ensuremath{\text{ }}\ensuremath{\text{ }}\ensuremath{\text{ }}\DataTypeTok{author}\ensuremath{\text{ }}\ensuremath{\text{ }}\ensuremath{\text{ }}\ensuremath{\text{ }}\NormalTok{=\ensuremath{\text{ }}"",}\newline
\ensuremath{\text{ }}\ensuremath{\text{ }}\ensuremath{\text{ }}\ensuremath{\text{ }}\DataTypeTok{title}\ensuremath{\text{ }}\ensuremath{\text{ }}\ensuremath{\text{ }}\ensuremath{\text{ }}\ensuremath{\text{ }}\NormalTok{=\ensuremath{\text{ }}"",}\newline
\ensuremath{\text{ }}\ensuremath{\text{ }}\ensuremath{\text{ }}\ensuremath{\text{ }}\DataTypeTok{booktitle}\ensuremath{\text{ }}\NormalTok{=\ensuremath{\text{ }}"",}\newline
\ensuremath{\text{ }}\ensuremath{\text{ }}\ensuremath{\text{ }}\ensuremath{\text{ }}\AlertTok{\%}\StringTok{editor}\ensuremath{\text{ }}\ensuremath{\text{ }}\ensuremath{\text{ }}\NormalTok{=\ensuremath{\text{ }}"",}\newline
\ensuremath{\text{ }}\ensuremath{\text{ }}\ensuremath{\text{ }}\ensuremath{\text{ }}\AlertTok{\%}\StringTok{volume}\ensuremath{\text{ }}\ensuremath{\text{ }}\ensuremath{\text{ }}\NormalTok{=\ensuremath{\text{ }}"",}\newline
\ensuremath{\text{ }}\ensuremath{\text{ }}\ensuremath{\text{ }}\ensuremath{\text{ }}\AlertTok{\%}\StringTok{number}\ensuremath{\text{ }}\ensuremath{\text{ }}\ensuremath{\text{ }}\NormalTok{=\ensuremath{\text{ }}"",}\newline
\ensuremath{\text{ }}\ensuremath{\text{ }}\ensuremath{\text{ }}\ensuremath{\text{ }}\AlertTok{\%}\StringTok{series}\ensuremath{\text{ }}\ensuremath{\text{ }}\ensuremath{\text{ }}\NormalTok{=\ensuremath{\text{ }}"",}\newline
\ensuremath{\text{ }}\ensuremath{\text{ }}\ensuremath{\text{ }}\ensuremath{\text{ }}\AlertTok{\%}\StringTok{pages}\ensuremath{\text{ }}\ensuremath{\text{ }}\ensuremath{\text{ }}\ensuremath{\text{ }}\NormalTok{=\ensuremath{\text{ }}"",}\newline
\ensuremath{\text{ }}\ensuremath{\text{ }}\ensuremath{\text{ }}\ensuremath{\text{ }}\AlertTok{\%}\StringTok{address}\ensuremath{\text{ }}\ensuremath{\text{ }}\NormalTok{=\ensuremath{\text{ }}"",}\newline
\ensuremath{\text{ }}\ensuremath{\text{ }}\ensuremath{\text{ }}\ensuremath{\text{ }}\DataTypeTok{year}\ensuremath{\text{ }}\ensuremath{\text{ }}\ensuremath{\text{ }}\ensuremath{\text{ }}\ensuremath{\text{ }}\ensuremath{\text{ }}\NormalTok{=\ensuremath{\text{ }}"}\StringTok{XXXX}\NormalTok{",}\newline
\ensuremath{\text{ }}\ensuremath{\text{ }}\ensuremath{\text{ }}\ensuremath{\text{ }}\AlertTok{\%}\StringTok{month}\ensuremath{\text{ }}\ensuremath{\text{ }}\ensuremath{\text{ }}\ensuremath{\text{ }}\NormalTok{=\ensuremath{\text{ }}"",}\newline
\ensuremath{\text{ }}\ensuremath{\text{ }}\ensuremath{\text{ }}\ensuremath{\text{ }}\AlertTok{\%}\StringTok{publisher}\NormalTok{=\ensuremath{\text{ }}"",}\newline
\ensuremath{\text{ }}\ensuremath{\text{ }}\ensuremath{\text{ }}\ensuremath{\text{ }}\AlertTok{\%}\StringTok{note}\ensuremath{\text{ }}\ensuremath{\text{ }}\ensuremath{\text{ }}\ensuremath{\text{ }}\ensuremath{\text{ }}\NormalTok{=\ensuremath{\text{ }}"",}\newline
\NormalTok{\}}\newline
\end{Highlighting}
\end{Shaded}

{\bfseries
\begin{mydescription}@inbook 
\end{mydescription}
}
\begin{myquote}\item{} A section of a book {\itshape \setmainfont[Path=/usr/share/fonts/truetype/cmu/,UprightFont=cmunrm.ttf,BoldFont=cmunbx.ttf,ItalicFont=cmunti.ttf,BoldItalicFont=cmunbi.ttf]{cmunti.ttf}\setmonofont[Path=/usr/share/fonts/truetype/cmu/,UprightFont=cmuntt.ttf,BoldFont=cmuntb.ttf,ItalicFont=cmunit.ttf,BoldItalicFont=cmuntx.ttf]{cmunti.ttf}\itshape without}{$\text{ }$}\setmainfont[Path=/usr/share/fonts/truetype/cmu/,UprightFont=cmunrm.ttf,BoldFont=cmunbx.ttf,ItalicFont=cmunti.ttf,BoldItalicFont=cmunbi.ttf]{cmunrm.ttf}\setmonofont[Path=/usr/share/fonts/truetype/cmu/,UprightFont=cmuntt.ttf,BoldFont=cmuntb.ttf,ItalicFont=cmunit.ttf,BoldItalicFont=cmuntx.ttf]{cmunrm.ttf} its own title.
\end{myquote}

\begin{myquote}
\item{} 
\begin{myitemize}
\item{} Required fields: author/editor, title, chapter and/or pages, publisher, year.
\item{} Optional fields: volume/number, series, type, address, edition, month, note.
\end{myitemize}

\end{myquote}


\begin{Shaded}
\begin{Highlighting}[]

\KeywordTok{@inbook}\NormalTok{\{}\OtherTok{Xinbook}\NormalTok{,}\newline
	\DataTypeTok{author}	\NormalTok{=\ensuremath{\text{ }}"",}\newline
	\DataTypeTok{editor}	\NormalTok{=\ensuremath{\text{ }}"",}\newline
	\DataTypeTok{title}	\NormalTok{=\ensuremath{\text{ }}"",}\newline
	\DataTypeTok{chapter}	\NormalTok{=\ensuremath{\text{ }}"",}\newline
	\DataTypeTok{pages}	\NormalTok{=\ensuremath{\text{ }}"",}\newline
	\DataTypeTok{publisher}\NormalTok{=\ensuremath{\text{ }}"",}\newline
	\AlertTok{\%}\StringTok{volume}	\NormalTok{=\ensuremath{\text{ }}"",}\newline
	\AlertTok{\%}\StringTok{number}	\NormalTok{=\ensuremath{\text{ }}"",}\newline
	\AlertTok{\%}\StringTok{series}	\NormalTok{=\ensuremath{\text{ }}"",}\newline
	\AlertTok{\%}\StringTok{type}	\NormalTok{=\ensuremath{\text{ }}"",}\newline
	\AlertTok{\%}\StringTok{address}\NormalTok{=\ensuremath{\text{ }}"",}\newline
	\AlertTok{\%}\StringTok{edition}\NormalTok{=\ensuremath{\text{ }}"",}\newline
	\DataTypeTok{year}	\NormalTok{=\ensuremath{\text{ }}"",}\newline
	\AlertTok{\%}\StringTok{month}	\NormalTok{=\ensuremath{\text{ }}"",}\newline
	\AlertTok{\%}\StringTok{note}	\NormalTok{=\ensuremath{\text{ }}"",}\newline
\NormalTok{\}}\newline
\end{Highlighting}
\end{Shaded}

{\bfseries
\begin{mydescription}@incollection 
\end{mydescription}
}
\begin{myquote}\item{} A section of a book having its own title.
\end{myquote}

\begin{myquote}
\item{} 
\begin{myitemize}
\item{} Required fields: author, title, booktitle, publisher, year.
\item{} Optional fields: editor, volume/number, series, type, chapter, pages, address, edition, month, note.
\end{myitemize}

\end{myquote}


\begin{Shaded}
\begin{Highlighting}[]

\KeywordTok{@incollection}\NormalTok{\{}\OtherTok{Xincollection}\NormalTok{,}\newline
	\DataTypeTok{author}	\NormalTok{=\ensuremath{\text{ }}"",}\newline
	\DataTypeTok{title}	\NormalTok{=\ensuremath{\text{ }}"",}\newline
	\DataTypeTok{booktitle}\NormalTok{=\ensuremath{\text{ }}"",}\newline
	\DataTypeTok{publisher}\NormalTok{=\ensuremath{\text{ }}"",}\newline
	\AlertTok{\%}\StringTok{editor}	\NormalTok{=\ensuremath{\text{ }}"",}\newline
	\AlertTok{\%}\StringTok{volume}	\NormalTok{=\ensuremath{\text{ }}"",}\newline
	\AlertTok{\%}\StringTok{number}	\NormalTok{=\ensuremath{\text{ }}"",}\newline
	\AlertTok{\%}\StringTok{series}	\NormalTok{=\ensuremath{\text{ }}"",}\newline
	\AlertTok{\%}\StringTok{type}	\NormalTok{=\ensuremath{\text{ }}"",}\newline
	\AlertTok{\%}\StringTok{chapter}\NormalTok{=\ensuremath{\text{ }}"",}\newline
	\AlertTok{\%}\StringTok{pages}	\NormalTok{=\ensuremath{\text{ }}"",}\newline
	\AlertTok{\%}\StringTok{address}\NormalTok{=\ensuremath{\text{ }}"",}\newline
	\AlertTok{\%}\StringTok{edition}\NormalTok{=\ensuremath{\text{ }}"",}\newline
	\DataTypeTok{year}	\NormalTok{=\ensuremath{\text{ }}"",}\newline
	\AlertTok{\%}\StringTok{month}	\NormalTok{=\ensuremath{\text{ }}"",}\newline
	\AlertTok{\%}\StringTok{note}	\NormalTok{=\ensuremath{\text{ }}"",}\newline
\NormalTok{\}}\newline
\end{Highlighting}
\end{Shaded}

{\bfseries
\begin{mydescription}@inproceedings 
\end{mydescription}
}
\begin{myquote}\item{} An article in a conference proceedings.
\end{myquote}

\begin{myquote}
\item{} 
\begin{myitemize}
\item{} Required fields: author, title, booktitle, year.
\item{} Optional fields: editor, volume/number, series, pages, address, month, organization, publisher, note.
\end{myitemize}

\end{myquote}


\begin{Shaded}
\begin{Highlighting}[]

\KeywordTok{@inproceedings}\NormalTok{\{}\OtherTok{Xinproceedings}\NormalTok{,}\newline
	\DataTypeTok{author}		\NormalTok{=\ensuremath{\text{ }}"",}\newline
	\DataTypeTok{title}		\NormalTok{=\ensuremath{\text{ }}"",}\newline
	\DataTypeTok{booktitle}	\NormalTok{=\ensuremath{\text{ }}"",}\newline
	\AlertTok{\%}\StringTok{editor}		\NormalTok{=\ensuremath{\text{ }}"",}\newline
	\AlertTok{\%}\StringTok{volume}		\NormalTok{=\ensuremath{\text{ }}"",}\newline
	\AlertTok{\%}\StringTok{number}		\NormalTok{=\ensuremath{\text{ }}"",}\newline
	\AlertTok{\%}\StringTok{series}		\NormalTok{=\ensuremath{\text{ }}"",}\newline
	\AlertTok{\%}\StringTok{pages}		\NormalTok{=\ensuremath{\text{ }}"",}\newline
	\AlertTok{\%}\StringTok{address}	\NormalTok{=\ensuremath{\text{ }}"",}\newline
	\AlertTok{\%}\StringTok{organization}	\NormalTok{=\ensuremath{\text{ }}"",}\newline
	\AlertTok{\%}\StringTok{publisher}	\NormalTok{=\ensuremath{\text{ }}"",}\newline
	\DataTypeTok{year}		\NormalTok{=\ensuremath{\text{ }}"",}\newline
	\AlertTok{\%}\StringTok{month}		\NormalTok{=\ensuremath{\text{ }}"",}\newline
	\AlertTok{\%}\StringTok{note}		\NormalTok{=\ensuremath{\text{ }}"",}\newline
\NormalTok{\}}\newline
\end{Highlighting}
\end{Shaded}

{\bfseries
\begin{mydescription}@manual 
\end{mydescription}
}
\begin{myquote}\item{} Technical manual
\end{myquote}

\begin{myquote}
\item{} 
\begin{myitemize}
\item{} Required fields: title.
\item{} Optional fields: author, organization, address, edition, month, year, note.
\end{myitemize}

\end{myquote}


\begin{Shaded}
\begin{Highlighting}[]

\KeywordTok{@manual}\NormalTok{\{}\OtherTok{Xmanual}\NormalTok{,}\newline
	\DataTypeTok{title}		\NormalTok{=\ensuremath{\text{ }}"",}\newline
	\AlertTok{\%}\StringTok{author}		\NormalTok{=\ensuremath{\text{ }}"",}\newline
	\AlertTok{\%}\StringTok{organization}	\NormalTok{=\ensuremath{\text{ }}"",}\newline
	\AlertTok{\%}\StringTok{address}	\NormalTok{=\ensuremath{\text{ }}"",}\newline
	\AlertTok{\%}\StringTok{edition}	\NormalTok{=\ensuremath{\text{ }}"",}\newline
	\DataTypeTok{year}		\NormalTok{=\ensuremath{\text{ }}"",}\newline
	\AlertTok{\%}\StringTok{month}		\NormalTok{=\ensuremath{\text{ }}"",}\newline
	\AlertTok{\%}\StringTok{note}		\NormalTok{=\ensuremath{\text{ }}"",}\newline
\NormalTok{\}}\newline
\end{Highlighting}
\end{Shaded}

{\bfseries
\begin{mydescription}@mastersthesis 
\end{mydescription}
}
\begin{myquote}\item{} Master\textquotesingle{}s thesis
\end{myquote}

\begin{myquote}
\item{} 
\begin{myitemize}
\item{} Required fields: author, title, school, year.
\item{} Optional fields: type (eg. \symbol{34}diploma thesis\symbol{34}), address, month, note.
\end{myitemize}

\end{myquote}


\begin{Shaded}
\begin{Highlighting}[]

\KeywordTok{@mastersthesis}\NormalTok{\{}\OtherTok{Xthesis}\NormalTok{,}\newline
\ensuremath{\text{ }}\ensuremath{\text{ }}\ensuremath{\text{ }}\ensuremath{\text{ }}\DataTypeTok{author}\ensuremath{\text{ }}\ensuremath{\text{ }}\ensuremath{\text{ }}\ensuremath{\text{ }}\NormalTok{=\ensuremath{\text{ }}"",}\newline
\ensuremath{\text{ }}\ensuremath{\text{ }}\ensuremath{\text{ }}\ensuremath{\text{ }}\DataTypeTok{title}\ensuremath{\text{ }}\ensuremath{\text{ }}\ensuremath{\text{ }}\ensuremath{\text{ }}\ensuremath{\text{ }}\NormalTok{=\ensuremath{\text{ }}"",}\newline
\ensuremath{\text{ }}\ensuremath{\text{ }}\ensuremath{\text{ }}\ensuremath{\text{ }}\DataTypeTok{school}\ensuremath{\text{ }}\ensuremath{\text{ }}\ensuremath{\text{ }}\ensuremath{\text{ }}\NormalTok{=\ensuremath{\text{ }}"",}\newline
\ensuremath{\text{ }}\ensuremath{\text{ }}\ensuremath{\text{ }}\ensuremath{\text{ }}\AlertTok{\%}\StringTok{type}\ensuremath{\text{ }}\ensuremath{\text{ }}\ensuremath{\text{ }}\ensuremath{\text{ }}\ensuremath{\text{ }}\NormalTok{=\ensuremath{\text{ }}"}\StringTok{diploma\ensuremath{\text{ }}thesis}\NormalTok{",}\newline
\ensuremath{\text{ }}\ensuremath{\text{ }}\ensuremath{\text{ }}\ensuremath{\text{ }}\AlertTok{\%}\StringTok{address}\ensuremath{\text{ }}\ensuremath{\text{ }}\NormalTok{=\ensuremath{\text{ }}"",}\newline
\ensuremath{\text{ }}\ensuremath{\text{ }}\ensuremath{\text{ }}\ensuremath{\text{ }}\DataTypeTok{year}\ensuremath{\text{ }}\ensuremath{\text{ }}\ensuremath{\text{ }}\ensuremath{\text{ }}\ensuremath{\text{ }}\ensuremath{\text{ }}\NormalTok{=\ensuremath{\text{ }}"}\StringTok{XXXX}\NormalTok{",}\newline
\ensuremath{\text{ }}\ensuremath{\text{ }}\ensuremath{\text{ }}\ensuremath{\text{ }}\AlertTok{\%}\StringTok{month}\ensuremath{\text{ }}\ensuremath{\text{ }}\ensuremath{\text{ }}\ensuremath{\text{ }}\NormalTok{=\ensuremath{\text{ }}"",}\newline
\ensuremath{\text{ }}\ensuremath{\text{ }}\ensuremath{\text{ }}\ensuremath{\text{ }}\AlertTok{\%}\StringTok{note}\ensuremath{\text{ }}\ensuremath{\text{ }}\ensuremath{\text{ }}\ensuremath{\text{ }}\ensuremath{\text{ }}\NormalTok{=\ensuremath{\text{ }}"",}\newline
\NormalTok{\}}\newline
\end{Highlighting}
\end{Shaded}

{\bfseries
\begin{mydescription}@misc 
\end{mydescription}
}
\begin{myquote}\item{} Template useful for other kinds of publication
\end{myquote}

\begin{myquote}
\item{} 
\begin{myitemize}
\item{} Required fields: none
\item{} Optional fields: author, title, howpublished, month, year, note.
\end{myitemize}

\end{myquote}


\begin{Shaded}
\begin{Highlighting}[]

\KeywordTok{@misc}\NormalTok{\{}\OtherTok{Xmisc}\NormalTok{,}\newline
\ensuremath{\text{ }}\ensuremath{\text{ }}\ensuremath{\text{ }}\ensuremath{\text{ }}\AlertTok{\%}\StringTok{author}\ensuremath{\text{ }}\ensuremath{\text{ }}\ensuremath{\text{ }}\ensuremath{\text{ }}\NormalTok{=\ensuremath{\text{ }}"",}\newline
\ensuremath{\text{ }}\ensuremath{\text{ }}\ensuremath{\text{ }}\ensuremath{\text{ }}\AlertTok{\%}\StringTok{title}\ensuremath{\text{ }}\ensuremath{\text{ }}\ensuremath{\text{ }}\ensuremath{\text{ }}\ensuremath{\text{ }}\NormalTok{=\ensuremath{\text{ }}"",}\newline
\ensuremath{\text{ }}\ensuremath{\text{ }}\ensuremath{\text{ }}\ensuremath{\text{ }}\AlertTok{\%}\StringTok{howpublished}\ensuremath{\text{ }}\NormalTok{=\ensuremath{\text{ }}"",}\newline
\ensuremath{\text{ }}\ensuremath{\text{ }}\ensuremath{\text{ }}\ensuremath{\text{ }}\AlertTok{\%}\StringTok{year}\ensuremath{\text{ }}\ensuremath{\text{ }}\ensuremath{\text{ }}\ensuremath{\text{ }}\ensuremath{\text{ }}\NormalTok{=\ensuremath{\text{ }}"}\StringTok{XXXX}\NormalTok{",}\newline
\ensuremath{\text{ }}\ensuremath{\text{ }}\ensuremath{\text{ }}\ensuremath{\text{ }}\AlertTok{\%}\StringTok{month}\ensuremath{\text{ }}\ensuremath{\text{ }}\ensuremath{\text{ }}\ensuremath{\text{ }}\NormalTok{=\ensuremath{\text{ }}"",}\newline
\ensuremath{\text{ }}\ensuremath{\text{ }}\ensuremath{\text{ }}\ensuremath{\text{ }}\AlertTok{\%}\StringTok{note}\ensuremath{\text{ }}\ensuremath{\text{ }}\ensuremath{\text{ }}\ensuremath{\text{ }}\ensuremath{\text{ }}\NormalTok{=\ensuremath{\text{ }}"",}\newline
\NormalTok{\}}\newline
\end{Highlighting}
\end{Shaded}

{\bfseries
\begin{mydescription}@phdthesis 
\end{mydescription}
}
\begin{myquote}\item{} Ph.D. thesis
\end{myquote}

\begin{myquote}
\item{} 
\begin{myitemize}
\item{} Required fields: author, title, year, school.
\item{} Optional fields: address, month, keywords, note.
\end{myitemize}

\end{myquote}


\begin{Shaded}
\begin{Highlighting}[]

\KeywordTok{@phdthesis}\NormalTok{\{}\OtherTok{Xphdthesis}\NormalTok{,}\newline
	\DataTypeTok{author}		\NormalTok{=\ensuremath{\text{ }}"",}\newline
	\DataTypeTok{title}		\NormalTok{=\ensuremath{\text{ }}"",}\newline
	\DataTypeTok{school}		\NormalTok{=\ensuremath{\text{ }}"",}\newline
	\AlertTok{\%}\StringTok{address}	\NormalTok{=\ensuremath{\text{ }}"",}\newline
	\DataTypeTok{year}		\NormalTok{=\ensuremath{\text{ }}"",}\newline
	\AlertTok{\%}\StringTok{month}		\NormalTok{=\ensuremath{\text{ }}"",}\newline
	\AlertTok{\%}\StringTok{keywords}	\NormalTok{=\ensuremath{\text{ }}"",}\newline
	\AlertTok{\%}\StringTok{note}		\NormalTok{=\ensuremath{\text{ }}"",}\newline
\NormalTok{\}}\newline
\end{Highlighting}
\end{Shaded}

{\bfseries
\begin{mydescription}@proceedings 
\end{mydescription}
}
\begin{myquote}\item{} The proceedings of a conference.
\end{myquote}

\begin{myquote}
\item{} 
\begin{myitemize}
\item{} Required fields: title, year.
\item{} Optional fields: editor, volume/number, series, address, month, organization, publisher, note.
\end{myitemize}

\end{myquote}


\begin{Shaded}
\begin{Highlighting}[]

\KeywordTok{@proceedings}\NormalTok{\{}\OtherTok{Xproceedings}\NormalTok{,}\newline
	\DataTypeTok{title}		\NormalTok{=\ensuremath{\text{ }}"",}\newline
	\AlertTok{\%}\StringTok{editor}		\NormalTok{=\ensuremath{\text{ }}"",}\newline
	\AlertTok{\%}\StringTok{volume}		\NormalTok{=\ensuremath{\text{ }}"",}\newline
	\AlertTok{\%}\StringTok{number}		\NormalTok{=\ensuremath{\text{ }}"",}\newline
	\AlertTok{\%}\StringTok{series}		\NormalTok{=\ensuremath{\text{ }}"",}\newline
	\AlertTok{\%}\StringTok{address}	\NormalTok{=\ensuremath{\text{ }}"",}\newline
	\AlertTok{\%}\StringTok{organization}	\NormalTok{=\ensuremath{\text{ }}"",}\newline
	\AlertTok{\%}\StringTok{publisher}	\NormalTok{=\ensuremath{\text{ }}"",}\newline
	\DataTypeTok{year}		\NormalTok{=\ensuremath{\text{ }}"",}\newline
	\AlertTok{\%}\StringTok{month}		\NormalTok{=\ensuremath{\text{ }}"",}\newline
	\AlertTok{\%}\StringTok{note}		\NormalTok{=\ensuremath{\text{ }}"",}\newline
\NormalTok{\}}\newline
\end{Highlighting}
\end{Shaded}

{\bfseries
\begin{mydescription}@techreport 
\end{mydescription}
}
\begin{myquote}\item{} Technical report from educational, commercial or standardization institution.
\end{myquote}

\begin{myquote}
\item{} 
\begin{myitemize}
\item{} Required fields: author, title, institution, year.
\item{} Optional fields: type, number, address, month, note.
\end{myitemize}

\end{myquote}


\begin{Shaded}
\begin{Highlighting}[]

\KeywordTok{@techreport}\NormalTok{\{}\OtherTok{Xtreport}\NormalTok{,}\newline
\ensuremath{\text{ }}\ensuremath{\text{ }}\ensuremath{\text{ }}\ensuremath{\text{ }}\DataTypeTok{author}\ensuremath{\text{ }}\ensuremath{\text{ }}\ensuremath{\text{ }}\ensuremath{\text{ }}\NormalTok{=\ensuremath{\text{ }}"",}\newline
\ensuremath{\text{ }}\ensuremath{\text{ }}\ensuremath{\text{ }}\ensuremath{\text{ }}\DataTypeTok{title}\ensuremath{\text{ }}\ensuremath{\text{ }}\ensuremath{\text{ }}\ensuremath{\text{ }}\ensuremath{\text{ }}\NormalTok{=\ensuremath{\text{ }}"",}\newline
\ensuremath{\text{ }}\ensuremath{\text{ }}\ensuremath{\text{ }}\ensuremath{\text{ }}\DataTypeTok{institution}\ensuremath{\text{ }}\NormalTok{=\ensuremath{\text{ }}"",}\newline
\ensuremath{\text{ }}\ensuremath{\text{ }}\ensuremath{\text{ }}\ensuremath{\text{ }}\AlertTok{\%}\StringTok{type}\ensuremath{\text{ }}\ensuremath{\text{ }}\ensuremath{\text{ }}\ensuremath{\text{ }}\ensuremath{\text{ }}\NormalTok{=\ensuremath{\text{ }}"",\ensuremath{\text{ }}}\newline
\ensuremath{\text{ }}\ensuremath{\text{ }}\ensuremath{\text{ }}\ensuremath{\text{ }}\AlertTok{\%}\StringTok{number}\ensuremath{\text{ }}\ensuremath{\text{ }}\ensuremath{\text{ }}\NormalTok{=\ensuremath{\text{ }}"",}\newline
\ensuremath{\text{ }}\ensuremath{\text{ }}\ensuremath{\text{ }}\ensuremath{\text{ }}\AlertTok{\%}\StringTok{address}\ensuremath{\text{ }}\ensuremath{\text{ }}\NormalTok{=\ensuremath{\text{ }}"",}\newline
\ensuremath{\text{ }}\ensuremath{\text{ }}\ensuremath{\text{ }}\ensuremath{\text{ }}\DataTypeTok{year}\ensuremath{\text{ }}\ensuremath{\text{ }}\ensuremath{\text{ }}\ensuremath{\text{ }}\ensuremath{\text{ }}\ensuremath{\text{ }}\NormalTok{=\ensuremath{\text{ }}"}\StringTok{XXXX}\NormalTok{",}\newline
\ensuremath{\text{ }}\ensuremath{\text{ }}\ensuremath{\text{ }}\ensuremath{\text{ }}\AlertTok{\%}\StringTok{month}\ensuremath{\text{ }}\ensuremath{\text{ }}\ensuremath{\text{ }}\ensuremath{\text{ }}\NormalTok{=\ensuremath{\text{ }}"",}\newline
\ensuremath{\text{ }}\ensuremath{\text{ }}\ensuremath{\text{ }}\ensuremath{\text{ }}\AlertTok{\%}\StringTok{note}\ensuremath{\text{ }}\ensuremath{\text{ }}\ensuremath{\text{ }}\ensuremath{\text{ }}\ensuremath{\text{ }}\NormalTok{=\ensuremath{\text{ }}"",}\newline
\NormalTok{\}}\newline
\end{Highlighting}
\end{Shaded}

{\bfseries
\begin{mydescription}@unpublished 
\end{mydescription}
}
\begin{myquote}\item{} An unpublished article, book, thesis, etc.
\end{myquote}

\begin{myquote}
\item{} 
\begin{myitemize}
\item{} Required fields: author, title, note.
\item{} Optional fields: month, year.
\end{myitemize}

\end{myquote}


\begin{Shaded}
\begin{Highlighting}[]

\KeywordTok{@unpublished}\NormalTok{\{}\OtherTok{Xunpublished}\NormalTok{,}\newline
	\DataTypeTok{author}	\NormalTok{=\ensuremath{\text{ }}"",}\newline
	\DataTypeTok{title}	\NormalTok{=\ensuremath{\text{ }}"",}\newline
	\AlertTok{\%}\StringTok{year}	\NormalTok{=\ensuremath{\text{ }}"",}\newline
	\AlertTok{\%}\StringTok{month}	\NormalTok{=\ensuremath{\text{ }}"",}\newline
	\DataTypeTok{note}	\NormalTok{=\ensuremath{\text{ }}"",}\newline
\NormalTok{\}}\newline
\end{Highlighting}
\end{Shaded}

\subsection{Not standard templates}
\label{679}{\bfseries
\begin{mydescription}@patent
\end{mydescription}
}

\begin{myquote}
\item{} BiBTeX entries can be exported from Google Patents.
\item{} (see \myhref{http://www.see-out.com/sandramau/bibpat.html}{Cite Patents with Bibtex} for an alternative)
\end{myquote}

{\bfseries
\begin{mydescription}@collection
\end{mydescription}
}

{\bfseries
\begin{mydescription}@electronic
\end{mydescription}
}

{\bfseries
\begin{mydescription}@Unpublished
\end{mydescription}
}

\begin{myquote}
\item{} For citing arXiv.org papers in a \myhref{https://journals.aps.org/revtex}{REVTEX}-{}style article
\item{} (see \myhref{https://d22izw7byeupn1.cloudfront.net/files/revtex/auguide4-1.pdf}{REVTEX Author\textquotesingle{}s guide})
\end{myquote}

\subsection{Preserving case of letters}
\label{680}

In the event that BibTeX has been set by the chosen style to not preserve all capitalization within titles, problems can occur, especially if you are referring to proper nouns, or acronyms. To tell BibTeX to keep them, use the good old curly braces around the letter in question, (or letters, if it\textquotesingle{}s an acronym) and all will be well!  It is even possible that lower-{}case letters may need to be preserved -{} for example if a chemical formula is used in a style that sets a title in all caps or small caps, or if \symbol{34}pH\symbol{34} is to be used in a style that capitalises all first letters.

\begin{myquote}
\item{} 
\begin{Shaded}
\begin{Highlighting}[]

\CommentTok{title\ensuremath{\text{ }}=\ensuremath{\text{ }}"The\ensuremath{\text{ }}\{LaTeX\}\ensuremath{\text{ }}Companion",}\newline
\end{Highlighting}
\end{Shaded}

\end{myquote}


However, {\bfseries \setmainfont[Path=/usr/share/fonts/truetype/cmu/,UprightFont=cmunrm.ttf,BoldFont=cmunbx.ttf,ItalicFont=cmunti.ttf,BoldItalicFont=cmunbi.ttf]{cmunbx.ttf}\setmonofont[Path=/usr/share/fonts/truetype/cmu/,UprightFont=cmuntt.ttf,BoldFont=cmuntb.ttf,ItalicFont=cmunit.ttf,BoldItalicFont=cmuntx.ttf]{cmunbx.ttf}\bfseries avoid}{$\text{ }$}\setmainfont[Path=/usr/share/fonts/truetype/cmu/,UprightFont=cmunrm.ttf,BoldFont=cmunbx.ttf,ItalicFont=cmunti.ttf,BoldItalicFont=cmunbi.ttf]{cmunrm.ttf}\setmonofont[Path=/usr/share/fonts/truetype/cmu/,UprightFont=cmuntt.ttf,BoldFont=cmuntb.ttf,ItalicFont=cmunit.ttf,BoldItalicFont=cmuntx.ttf]{cmunrm.ttf} putting the whole title in curly braces, as it will look odd if a different capitalization format is used:

\begin{myquote}
\item{} 
\begin{Shaded}
\begin{Highlighting}[]

\CommentTok{title\ensuremath{\text{ }}=\ensuremath{\text{ }}"\{The\ensuremath{\text{ }}LaTeX\ensuremath{\text{ }}Companion\}",}\newline
\end{Highlighting}
\end{Shaded}

\end{myquote}


For convenience though, many people simply put double curly braces, which may help when writing scientific articles for different magazines, conferences with different BibTex styles that do sometimes keep and sometimes not keep the capital letters:

\begin{myquote}
\item{} 
\begin{Shaded}
\begin{Highlighting}[]

\CommentTok{title\ensuremath{\text{ }}=\ensuremath{\text{ }}\{\{The\ensuremath{\text{ }}LaTeX\ensuremath{\text{ }}Companion\}\},}\newline
\end{Highlighting}
\end{Shaded}

\end{myquote}


As an alternative, try other BibTex styles or modify the existing.  The approach of putting only relevant text in curly brackets is the most feasible if using a template under the control of a publisher, such as for journal submissions. Using curly braces around single letters is also to be avoided if possible, as it may mess up the kerning, especially with biblatex,\myfootnote{\myfnhref{http://mirrors.ctan.org/macros/latex/contrib/biblatex/doc/biblatex.pdf}{The biblatex manual}} so the first step should generally be to enclose single words in braces.
\subsection{A few additional examples}
\label{681}

Below you will find a few additional examples of bibliography entries. The first one covers the case of multiple authors in the Surname, Firstname format, and the second one deals with the incollection case.


\begin{Shaded}
\begin{Highlighting}[]

\KeywordTok{@article}\NormalTok{\{}\OtherTok{AbedonHymanThomas2003}\NormalTok{,}\newline
\ensuremath{\text{ }}\ensuremath{\text{ }}\DataTypeTok{author}\ensuremath{\text{ }}\NormalTok{=\ensuremath{\text{ }}"}\StringTok{Abedon,\ensuremath{\text{ }}S.\ensuremath{\text{ }}T.\ensuremath{\text{ }}and\ensuremath{\text{ }}Hyman,\ensuremath{\text{ }}P.\ensuremath{\text{ }}and\ensuremath{\text{ }}Thomas,\ensuremath{\text{ }}C.}\NormalTok{",}\newline
\ensuremath{\text{ }}\ensuremath{\text{ }}\DataTypeTok{year}\ensuremath{\text{ }}\NormalTok{=\ensuremath{\text{ }}"}\StringTok{2003}\NormalTok{",}\newline
\ensuremath{\text{ }}\ensuremath{\text{ }}\DataTypeTok{title}\ensuremath{\text{ }}\NormalTok{=\ensuremath{\text{ }}"}\StringTok{Experimental\ensuremath{\text{ }}examination\ensuremath{\text{ }}of\ensuremath{\text{ }}bacteriophage\ensuremath{\text{ }}latent-period\ensuremath{\text{ }}evolution\ensuremath{\text{ }}as}\newline
\StringTok{\ensuremath{\text{ }}a\ensuremath{\text{ }}response\ensuremath{\text{ }}to\ensuremath{\text{ }}bacterial\ensuremath{\text{ }}availability}\NormalTok{",}\newline
\ensuremath{\text{ }}\ensuremath{\text{ }}\DataTypeTok{journal}\ensuremath{\text{ }}\NormalTok{=\ensuremath{\text{ }}"}\StringTok{Applied\ensuremath{\text{ }}and\ensuremath{\text{ }}Environmental\ensuremath{\text{ }}Microbiology}\NormalTok{",}\newline
\ensuremath{\text{ }}\ensuremath{\text{ }}\DataTypeTok{volume}\ensuremath{\text{ }}\NormalTok{=\ensuremath{\text{ }}"}\StringTok{69}\NormalTok{",}\newline
\ensuremath{\text{ }}\ensuremath{\text{ }}\DataTypeTok{pages}\ensuremath{\text{ }}\NormalTok{=\ensuremath{\text{ }}"}\StringTok{7499--7506}\NormalTok{"}\newline
\NormalTok{\}}\CommentTok{,}\newline
\CommentTok{\ensuremath{\text{ }}}\newline
\KeywordTok{@incollection}\NormalTok{\{}\OtherTok{Abedon1994}\NormalTok{,}\newline
\ensuremath{\text{ }}\ensuremath{\text{ }}\DataTypeTok{author}\ensuremath{\text{ }}\NormalTok{=\ensuremath{\text{ }}"}\StringTok{Abedon,\ensuremath{\text{ }}S.\ensuremath{\text{ }}T.}\NormalTok{",}\newline
\ensuremath{\text{ }}\ensuremath{\text{ }}\DataTypeTok{title}\ensuremath{\text{ }}\NormalTok{=\ensuremath{\text{ }}"}\StringTok{Lysis\ensuremath{\text{ }}and\ensuremath{\text{ }}the\ensuremath{\text{ }}interaction\ensuremath{\text{ }}between\ensuremath{\text{ }}free\ensuremath{\text{ }}phages\ensuremath{\text{ }}and\ensuremath{\text{ }}infected\ensuremath{\text{ }}cells}\NormalTok{",}\newline
\ensuremath{\text{ }}\ensuremath{\text{ }}\DataTypeTok{pages}\ensuremath{\text{ }}\NormalTok{=\ensuremath{\text{ }}"}\StringTok{397--405}\NormalTok{",}\newline
\ensuremath{\text{ }}\ensuremath{\text{ }}\DataTypeTok{booktitle}\ensuremath{\text{ }}\NormalTok{=\ensuremath{\text{ }}"}\StringTok{Molecular\ensuremath{\text{ }}biology\ensuremath{\text{ }}of\ensuremath{\text{ }}bacteriophage\ensuremath{\text{ }}T4}\NormalTok{",}\newline
\ensuremath{\text{ }}\ensuremath{\text{ }}\DataTypeTok{editor}\ensuremath{\text{ }}\NormalTok{=\ensuremath{\text{ }}"}\StringTok{Karam,\ensuremath{\text{ }}Jim\ensuremath{\text{ }}D.\ensuremath{\text{ }}Karam\ensuremath{\text{ }}and\ensuremath{\text{ }}Drake,\ensuremath{\text{ }}John\ensuremath{\text{ }}W.\ensuremath{\text{ }}and\ensuremath{\text{ }}Kreuzer,\ensuremath{\text{ }}Kenneth\ensuremath{\text{ }}N.\ensuremath{\text{ }}and}\newline
\StringTok{\ensuremath{\text{ }}Mosig,\ensuremath{\text{ }}Gisela}\newline
\StringTok{\ensuremath{\text{ }}\ensuremath{\text{ }}\ensuremath{\text{ }}\ensuremath{\text{ }}\ensuremath{\text{ }}\ensuremath{\text{ }}\ensuremath{\text{ }}\ensuremath{\text{ }}\ensuremath{\text{ }}\ensuremath{\text{ }}\ensuremath{\text{ }}\ensuremath{\text{ }}and\ensuremath{\text{ }}Hall,\ensuremath{\text{ }}Dwight\ensuremath{\text{ }}and\ensuremath{\text{ }}Eiserling,\ensuremath{\text{ }}Frederick\ensuremath{\text{ }}A.\ensuremath{\text{ }}and\ensuremath{\text{ }}Black,\ensuremath{\text{ }}Lindsay\ensuremath{\text{ }}W.}\newline
\StringTok{\ensuremath{\text{ }}and\ensuremath{\text{ }}Kutter,\ensuremath{\text{ }}Elizabeth}\newline
\StringTok{\ensuremath{\text{ }}\ensuremath{\text{ }}\ensuremath{\text{ }}\ensuremath{\text{ }}\ensuremath{\text{ }}\ensuremath{\text{ }}\ensuremath{\text{ }}\ensuremath{\text{ }}\ensuremath{\text{ }}\ensuremath{\text{ }}\ensuremath{\text{ }}\ensuremath{\text{ }}and\ensuremath{\text{ }}Carlson,\ensuremath{\text{ }}Karin\ensuremath{\text{ }}and\ensuremath{\text{ }}Miller,\ensuremath{\text{ }}Eric\ensuremath{\text{ }}S.\ensuremath{\text{ }}and\ensuremath{\text{ }}Spicer,\ensuremath{\text{ }}Eleanor}\NormalTok{",}\newline
\ensuremath{\text{ }}\ensuremath{\text{ }}\DataTypeTok{publisher}\ensuremath{\text{ }}\NormalTok{=\ensuremath{\text{ }}"}\StringTok{ASM\ensuremath{\text{ }}Press,\ensuremath{\text{ }}Washington\ensuremath{\text{ }}DC}\NormalTok{",}\newline
\ensuremath{\text{ }}\ensuremath{\text{ }}\DataTypeTok{year}\ensuremath{\text{ }}\NormalTok{=\ensuremath{\text{ }}"}\StringTok{1994}\NormalTok{"}\newline
\NormalTok{\}}\CommentTok{,}\newline
\end{Highlighting}
\end{Shaded}


If you have to cite a website you can use @misc, for example:


\begin{Shaded}
\begin{Highlighting}[]

\KeywordTok{@misc}\NormalTok{\{}\OtherTok{website:fermentas}\NormalTok{-}\OtherTok{lambda}\NormalTok{,}\newline
\ensuremath{\text{ }}\ensuremath{\text{ }}\ensuremath{\text{ }}\ensuremath{\text{ }}\ensuremath{\text{ }}\ensuremath{\text{ }}\DataTypeTok{author}\ensuremath{\text{ }}\NormalTok{=\ensuremath{\text{ }}"}\StringTok{Fermentas\ensuremath{\text{ }}Inc.}\NormalTok{",}\newline
\ensuremath{\text{ }}\ensuremath{\text{ }}\ensuremath{\text{ }}\ensuremath{\text{ }}\ensuremath{\text{ }}\ensuremath{\text{ }}\DataTypeTok{title}\ensuremath{\text{ }}\NormalTok{=\ensuremath{\text{ }}"}\StringTok{Phage\ensuremath{\text{ }}Lambda:\ensuremath{\text{ }}description\ensuremath{\text{ }}}\CharTok{\textbackslash{}\&}\StringTok{\ensuremath{\text{ }}restriction\ensuremath{\text{ }}map}\NormalTok{",}\newline
\ensuremath{\text{ }}\ensuremath{\text{ }}\ensuremath{\text{ }}\ensuremath{\text{ }}\ensuremath{\text{ }}\ensuremath{\text{ }}\DataTypeTok{month}\ensuremath{\text{ }}\NormalTok{=\ensuremath{\text{ }}"}\StringTok{November}\NormalTok{",}\newline
\ensuremath{\text{ }}\ensuremath{\text{ }}\ensuremath{\text{ }}\ensuremath{\text{ }}\ensuremath{\text{ }}\ensuremath{\text{ }}\DataTypeTok{year}\ensuremath{\text{ }}\NormalTok{=\ensuremath{\text{ }}"}\StringTok{2008}\NormalTok{",}\newline
\ensuremath{\text{ }}\ensuremath{\text{ }}\ensuremath{\text{ }}\ensuremath{\text{ }}\ensuremath{\text{ }}\ensuremath{\text{ }}\DataTypeTok{url}\ensuremath{\text{ }}\NormalTok{=\ensuremath{\text{ }}"}\StringTok{http://www.fermentas.com/techinfo/nucleicacids/maplambda.htm}\NormalTok{"}\newline
\NormalTok{\}}\CommentTok{,}\newline
\end{Highlighting}
\end{Shaded}


The note field comes in handy if you need to add unstructured information, for example that the corresponding issue of the journal has yet to appear:


\begin{Shaded}
\begin{Highlighting}[]

\KeywordTok{@article}\NormalTok{\{}\OtherTok{blackholes}\NormalTok{,}\newline
\ensuremath{\text{ }}\ensuremath{\text{ }}\ensuremath{\text{ }}\ensuremath{\text{ }}\ensuremath{\text{ }}\ensuremath{\text{ }}\DataTypeTok{author}\NormalTok{="}\StringTok{Rabbert\ensuremath{\text{ }}Klein}\NormalTok{",}\newline
\ensuremath{\text{ }}\ensuremath{\text{ }}\ensuremath{\text{ }}\ensuremath{\text{ }}\ensuremath{\text{ }}\ensuremath{\text{ }}\DataTypeTok{title}\NormalTok{="}\StringTok{Black\ensuremath{\text{ }}Holes\ensuremath{\text{ }}and\ensuremath{\text{ }}Their\ensuremath{\text{ }}Relation\ensuremath{\text{ }}to\ensuremath{\text{ }}Hiding\ensuremath{\text{ }}Eggs}\NormalTok{",}\newline
\ensuremath{\text{ }}\ensuremath{\text{ }}\ensuremath{\text{ }}\ensuremath{\text{ }}\ensuremath{\text{ }}\ensuremath{\text{ }}\DataTypeTok{journal}\NormalTok{="}\StringTok{Theoretical\ensuremath{\text{ }}Easter\ensuremath{\text{ }}Physics}\NormalTok{",}\newline
\ensuremath{\text{ }}\ensuremath{\text{ }}\ensuremath{\text{ }}\ensuremath{\text{ }}\ensuremath{\text{ }}\ensuremath{\text{ }}\DataTypeTok{publisher}\NormalTok{="}\StringTok{Eggs\ensuremath{\text{ }}Ltd.}\NormalTok{",}\newline
\ensuremath{\text{ }}\ensuremath{\text{ }}\ensuremath{\text{ }}\ensuremath{\text{ }}\ensuremath{\text{ }}\ensuremath{\text{ }}\DataTypeTok{year}\NormalTok{="}\StringTok{2010}\NormalTok{",}\newline
\ensuremath{\text{ }}\ensuremath{\text{ }}\ensuremath{\text{ }}\ensuremath{\text{ }}\ensuremath{\text{ }}\ensuremath{\text{ }}\DataTypeTok{note}\NormalTok{="}\StringTok{(to\ensuremath{\text{ }}appear)}\NormalTok{"}\newline
\NormalTok{\}}\newline
\end{Highlighting}
\end{Shaded}

\subsection{Getting current LaTeX document to use your .bib file}
\label{682}

At the end of your LaTeX file (that is, after the content, but before {\ttfamily \setmainfont[Path=/usr/share/fonts/truetype/cmu/,UprightFont=cmunrm.ttf,BoldFont=cmunbx.ttf,ItalicFont=cmunti.ttf,BoldItalicFont=cmunbi.ttf]{cmuntt.ttf}\setmonofont[Path=/usr/share/fonts/truetype/cmu/,UprightFont=cmuntt.ttf,BoldFont=cmuntb.ttf,ItalicFont=cmunit.ttf,BoldItalicFont=cmuntx.ttf]{cmuntt.ttf}\ttfamily \textbackslash{}end\{document\}}\setmainfont[Path=/usr/share/fonts/truetype/cmu/,UprightFont=cmunrm.ttf,BoldFont=cmunbx.ttf,ItalicFont=cmunti.ttf,BoldItalicFont=cmunbi.ttf]{cmunrm.ttf}\setmonofont[Path=/usr/share/fonts/truetype/cmu/,UprightFont=cmuntt.ttf,BoldFont=cmuntb.ttf,ItalicFont=cmunit.ttf,BoldItalicFont=cmuntx.ttf]{cmunrm.ttf}), you need to place the following commands:


\begin{Shaded}
\begin{Highlighting}[]

\NormalTok{\textbackslash{}bibliographystyle\{plain\}}\newline
\NormalTok{\textbackslash{}bibliography\{sample1,sample2,...,samplen\}\ensuremath{\text{ }}}\newline
\CommentTok{\%\ensuremath{\text{ }}Note\ensuremath{\text{ }}the\ensuremath{\text{ }}lack\ensuremath{\text{ }}of\ensuremath{\text{ }}whitespace\ensuremath{\text{ }}between\ensuremath{\text{ }}the\ensuremath{\text{ }}commas\ensuremath{\text{ }}and\ensuremath{\text{ }}the\ensuremath{\text{ }}next\ensuremath{\text{ }}bib\ensuremath{\text{ }}file.}\newline
\end{Highlighting}
\end{Shaded}


Bibliography styles are files recognized by BibTeX that tell it how to format the information stored in the {\ttfamily \setmainfont[Path=/usr/share/fonts/truetype/cmu/,UprightFont=cmunrm.ttf,BoldFont=cmunbx.ttf,ItalicFont=cmunti.ttf,BoldItalicFont=cmunbi.ttf]{cmuntt.ttf}\setmonofont[Path=/usr/share/fonts/truetype/cmu/,UprightFont=cmuntt.ttf,BoldFont=cmuntb.ttf,ItalicFont=cmunit.ttf,BoldItalicFont=cmuntx.ttf]{cmuntt.ttf}\ttfamily .bib}{$\text{ }$}\setmainfont[Path=/usr/share/fonts/truetype/cmu/,UprightFont=cmunrm.ttf,BoldFont=cmunbx.ttf,ItalicFont=cmunti.ttf,BoldItalicFont=cmunbi.ttf]{cmunrm.ttf}\setmonofont[Path=/usr/share/fonts/truetype/cmu/,UprightFont=cmuntt.ttf,BoldFont=cmuntb.ttf,ItalicFont=cmunit.ttf,BoldItalicFont=cmuntx.ttf]{cmunrm.ttf} file when processed for output. And so the first command listed above is declaring which style file to use. The style file in this instance is {\ttfamily \setmainfont[Path=/usr/share/fonts/truetype/cmu/,UprightFont=cmunrm.ttf,BoldFont=cmunbx.ttf,ItalicFont=cmunti.ttf,BoldItalicFont=cmunbi.ttf]{cmuntt.ttf}\setmonofont[Path=/usr/share/fonts/truetype/cmu/,UprightFont=cmuntt.ttf,BoldFont=cmuntb.ttf,ItalicFont=cmunit.ttf,BoldItalicFont=cmuntx.ttf]{cmuntt.ttf}\ttfamily plain.bst}{$\text{ }$}\setmainfont[Path=/usr/share/fonts/truetype/cmu/,UprightFont=cmunrm.ttf,BoldFont=cmunbx.ttf,ItalicFont=cmunti.ttf,BoldItalicFont=cmunbi.ttf]{cmunrm.ttf}\setmonofont[Path=/usr/share/fonts/truetype/cmu/,UprightFont=cmuntt.ttf,BoldFont=cmuntb.ttf,ItalicFont=cmunit.ttf,BoldItalicFont=cmuntx.ttf]{cmunrm.ttf} (which comes as standard with BibTeX). You do not need to add the .bst extension when using this command, as it is assumed. Despite its name, the plain style does a pretty good job (look at the output of this tutorial to see what I mean).

The second command is the one that actually specifies the {\ttfamily \setmainfont[Path=/usr/share/fonts/truetype/cmu/,UprightFont=cmunrm.ttf,BoldFont=cmunbx.ttf,ItalicFont=cmunti.ttf,BoldItalicFont=cmunbi.ttf]{cmuntt.ttf}\setmonofont[Path=/usr/share/fonts/truetype/cmu/,UprightFont=cmuntt.ttf,BoldFont=cmuntb.ttf,ItalicFont=cmunit.ttf,BoldItalicFont=cmuntx.ttf]{cmuntt.ttf}\ttfamily .bib}{$\text{ }$}\setmainfont[Path=/usr/share/fonts/truetype/cmu/,UprightFont=cmunrm.ttf,BoldFont=cmunbx.ttf,ItalicFont=cmunti.ttf,BoldItalicFont=cmunbi.ttf]{cmunrm.ttf}\setmonofont[Path=/usr/share/fonts/truetype/cmu/,UprightFont=cmuntt.ttf,BoldFont=cmuntb.ttf,ItalicFont=cmunit.ttf,BoldItalicFont=cmuntx.ttf]{cmunrm.ttf} file you wish to use. The ones I created for this tutorial were called {\ttfamily \setmainfont[Path=/usr/share/fonts/truetype/cmu/,UprightFont=cmunrm.ttf,BoldFont=cmunbx.ttf,ItalicFont=cmunti.ttf,BoldItalicFont=cmunbi.ttf]{cmuntt.ttf}\setmonofont[Path=/usr/share/fonts/truetype/cmu/,UprightFont=cmuntt.ttf,BoldFont=cmuntb.ttf,ItalicFont=cmunit.ttf,BoldItalicFont=cmuntx.ttf]{cmuntt.ttf}\ttfamily sample1.bib}\setmainfont[Path=/usr/share/fonts/truetype/cmu/,UprightFont=cmunrm.ttf,BoldFont=cmunbx.ttf,ItalicFont=cmunti.ttf,BoldItalicFont=cmunbi.ttf]{cmunrm.ttf}\setmonofont[Path=/usr/share/fonts/truetype/cmu/,UprightFont=cmuntt.ttf,BoldFont=cmuntb.ttf,ItalicFont=cmunit.ttf,BoldItalicFont=cmuntx.ttf]{cmunrm.ttf}, {\ttfamily \setmainfont[Path=/usr/share/fonts/truetype/cmu/,UprightFont=cmunrm.ttf,BoldFont=cmunbx.ttf,ItalicFont=cmunti.ttf,BoldItalicFont=cmunbi.ttf]{cmuntt.ttf}\setmonofont[Path=/usr/share/fonts/truetype/cmu/,UprightFont=cmuntt.ttf,BoldFont=cmuntb.ttf,ItalicFont=cmunit.ttf,BoldItalicFont=cmuntx.ttf]{cmuntt.ttf}\ttfamily sample2.bib}\setmainfont[Path=/usr/share/fonts/truetype/cmu/,UprightFont=cmunrm.ttf,BoldFont=cmunbx.ttf,ItalicFont=cmunti.ttf,BoldItalicFont=cmunbi.ttf]{cmunrm.ttf}\setmonofont[Path=/usr/share/fonts/truetype/cmu/,UprightFont=cmuntt.ttf,BoldFont=cmuntb.ttf,ItalicFont=cmunit.ttf,BoldItalicFont=cmuntx.ttf]{cmunrm.ttf}, . . ., {\ttfamily \setmainfont[Path=/usr/share/fonts/truetype/cmu/,UprightFont=cmunrm.ttf,BoldFont=cmunbx.ttf,ItalicFont=cmunti.ttf,BoldItalicFont=cmunbi.ttf]{cmuntt.ttf}\setmonofont[Path=/usr/share/fonts/truetype/cmu/,UprightFont=cmuntt.ttf,BoldFont=cmuntb.ttf,ItalicFont=cmunit.ttf,BoldItalicFont=cmuntx.ttf]{cmuntt.ttf}\ttfamily samplen.bib}\setmainfont[Path=/usr/share/fonts/truetype/cmu/,UprightFont=cmunrm.ttf,BoldFont=cmunbx.ttf,ItalicFont=cmunti.ttf,BoldItalicFont=cmunbi.ttf]{cmunrm.ttf}\setmonofont[Path=/usr/share/fonts/truetype/cmu/,UprightFont=cmuntt.ttf,BoldFont=cmuntb.ttf,ItalicFont=cmunit.ttf,BoldItalicFont=cmuntx.ttf]{cmunrm.ttf}, but once again, you don\textquotesingle{}t include the file extension. At the moment, the {\ttfamily \setmainfont[Path=/usr/share/fonts/truetype/cmu/,UprightFont=cmunrm.ttf,BoldFont=cmunbx.ttf,ItalicFont=cmunti.ttf,BoldItalicFont=cmunbi.ttf]{cmuntt.ttf}\setmonofont[Path=/usr/share/fonts/truetype/cmu/,UprightFont=cmuntt.ttf,BoldFont=cmuntb.ttf,ItalicFont=cmunit.ttf,BoldItalicFont=cmuntx.ttf]{cmuntt.ttf}\ttfamily .bib}{$\text{ }$}\setmainfont[Path=/usr/share/fonts/truetype/cmu/,UprightFont=cmunrm.ttf,BoldFont=cmunbx.ttf,ItalicFont=cmunti.ttf,BoldItalicFont=cmunbi.ttf]{cmunrm.ttf}\setmonofont[Path=/usr/share/fonts/truetype/cmu/,UprightFont=cmuntt.ttf,BoldFont=cmuntb.ttf,ItalicFont=cmunit.ttf,BoldItalicFont=cmuntx.ttf]{cmunrm.ttf} file is in the same directory as the LaTeX document too. However, if your .bib file was elsewhere (which makes sense if you intend to maintain a centralized database of references for all your research), you need to specify the path as well, e.g {\ttfamily \setmainfont[Path=/usr/share/fonts/truetype/cmu/,UprightFont=cmunrm.ttf,BoldFont=cmunbx.ttf,ItalicFont=cmunti.ttf,BoldItalicFont=cmunbi.ttf]{cmuntt.ttf}\setmonofont[Path=/usr/share/fonts/truetype/cmu/,UprightFont=cmuntt.ttf,BoldFont=cmuntb.ttf,ItalicFont=cmunit.ttf,BoldItalicFont=cmuntx.ttf]{cmuntt.ttf}\ttfamily \textbackslash{}bibliography\{/some/where/sample\}}{$\text{ }$}\setmainfont[Path=/usr/share/fonts/truetype/cmu/,UprightFont=cmunrm.ttf,BoldFont=cmunbx.ttf,ItalicFont=cmunti.ttf,BoldItalicFont=cmunbi.ttf]{cmunrm.ttf}\setmonofont[Path=/usr/share/fonts/truetype/cmu/,UprightFont=cmuntt.ttf,BoldFont=cmuntb.ttf,ItalicFont=cmunit.ttf,BoldItalicFont=cmuntx.ttf]{cmunrm.ttf} or {\ttfamily \setmainfont[Path=/usr/share/fonts/truetype/cmu/,UprightFont=cmunrm.ttf,BoldFont=cmunbx.ttf,ItalicFont=cmunti.ttf,BoldItalicFont=cmunbi.ttf]{cmuntt.ttf}\setmonofont[Path=/usr/share/fonts/truetype/cmu/,UprightFont=cmuntt.ttf,BoldFont=cmuntb.ttf,ItalicFont=cmunit.ttf,BoldItalicFont=cmuntx.ttf]{cmuntt.ttf}\ttfamily \textbackslash{}bibliography\{../sample1\}}{$\text{ }$}\setmainfont[Path=/usr/share/fonts/truetype/cmu/,UprightFont=cmunrm.ttf,BoldFont=cmunbx.ttf,ItalicFont=cmunti.ttf,BoldItalicFont=cmunbi.ttf]{cmunrm.ttf}\setmonofont[Path=/usr/share/fonts/truetype/cmu/,UprightFont=cmuntt.ttf,BoldFont=cmuntb.ttf,ItalicFont=cmunit.ttf,BoldItalicFont=cmuntx.ttf]{cmunrm.ttf} (if the {\ttfamily \setmainfont[Path=/usr/share/fonts/truetype/cmu/,UprightFont=cmunrm.ttf,BoldFont=cmunbx.ttf,ItalicFont=cmunti.ttf,BoldItalicFont=cmunbi.ttf]{cmuntt.ttf}\setmonofont[Path=/usr/share/fonts/truetype/cmu/,UprightFont=cmuntt.ttf,BoldFont=cmuntb.ttf,ItalicFont=cmunit.ttf,BoldItalicFont=cmuntx.ttf]{cmuntt.ttf}\ttfamily .bib}{$\text{ }$}\setmainfont[Path=/usr/share/fonts/truetype/cmu/,UprightFont=cmunrm.ttf,BoldFont=cmunbx.ttf,ItalicFont=cmunti.ttf,BoldItalicFont=cmunbi.ttf]{cmunrm.ttf}\setmonofont[Path=/usr/share/fonts/truetype/cmu/,UprightFont=cmuntt.ttf,BoldFont=cmuntb.ttf,ItalicFont=cmunit.ttf,BoldItalicFont=cmuntx.ttf]{cmunrm.ttf} file is in the parent directory of the {\ttfamily \setmainfont[Path=/usr/share/fonts/truetype/cmu/,UprightFont=cmunrm.ttf,BoldFont=cmunbx.ttf,ItalicFont=cmunti.ttf,BoldItalicFont=cmunbi.ttf]{cmuntt.ttf}\setmonofont[Path=/usr/share/fonts/truetype/cmu/,UprightFont=cmuntt.ttf,BoldFont=cmuntb.ttf,ItalicFont=cmunit.ttf,BoldItalicFont=cmuntx.ttf]{cmuntt.ttf}\ttfamily .tex}{$\text{ }$}\setmainfont[Path=/usr/share/fonts/truetype/cmu/,UprightFont=cmunrm.ttf,BoldFont=cmunbx.ttf,ItalicFont=cmunti.ttf,BoldItalicFont=cmunbi.ttf]{cmunrm.ttf}\setmonofont[Path=/usr/share/fonts/truetype/cmu/,UprightFont=cmuntt.ttf,BoldFont=cmuntb.ttf,ItalicFont=cmunit.ttf,BoldItalicFont=cmuntx.ttf]{cmunrm.ttf} document that calls it).

Now that LaTeX and BibTeX know where to look for the appropriate files, actually citing the references is fairly trivial. The {\ttfamily \setmainfont[Path=/usr/share/fonts/truetype/cmu/,UprightFont=cmunrm.ttf,BoldFont=cmunbx.ttf,ItalicFont=cmunti.ttf,BoldItalicFont=cmunbi.ttf]{cmuntt.ttf}\setmonofont[Path=/usr/share/fonts/truetype/cmu/,UprightFont=cmuntt.ttf,BoldFont=cmuntb.ttf,ItalicFont=cmunit.ttf,BoldItalicFont=cmuntx.ttf]{cmuntt.ttf}\ttfamily \textbackslash{}cite\{{\itshape \setmainfont[Path=/usr/share/fonts/truetype/cmu/,UprightFont=cmunrm.ttf,BoldFont=cmunbx.ttf,ItalicFont=cmunti.ttf,BoldItalicFont=cmunbi.ttf]{cmunit.ttf}\setmonofont[Path=/usr/share/fonts/truetype/cmu/,UprightFont=cmuntt.ttf,BoldFont=cmuntb.ttf,ItalicFont=cmunit.ttf,BoldItalicFont=cmuntx.ttf]{cmunit.ttf}\ttfamily \itshape ref\_key}\setmainfont[Path=/usr/share/fonts/truetype/cmu/,UprightFont=cmunrm.ttf,BoldFont=cmunbx.ttf,ItalicFont=cmunti.ttf,BoldItalicFont=cmunbi.ttf]{cmuntt.ttf}\setmonofont[Path=/usr/share/fonts/truetype/cmu/,UprightFont=cmuntt.ttf,BoldFont=cmuntb.ttf,ItalicFont=cmunit.ttf,BoldItalicFont=cmuntx.ttf]{cmuntt.ttf}\ttfamily \}}{$\text{ }$}\setmainfont[Path=/usr/share/fonts/truetype/cmu/,UprightFont=cmunrm.ttf,BoldFont=cmunbx.ttf,ItalicFont=cmunti.ttf,BoldItalicFont=cmunbi.ttf]{cmunrm.ttf}\setmonofont[Path=/usr/share/fonts/truetype/cmu/,UprightFont=cmuntt.ttf,BoldFont=cmuntb.ttf,ItalicFont=cmunit.ttf,BoldItalicFont=cmuntx.ttf]{cmunrm.ttf} is the command you need, making sure that the {\itshape \setmainfont[Path=/usr/share/fonts/truetype/cmu/,UprightFont=cmunrm.ttf,BoldFont=cmunbx.ttf,ItalicFont=cmunti.ttf,BoldItalicFont=cmunbi.ttf]{cmunti.ttf}\setmonofont[Path=/usr/share/fonts/truetype/cmu/,UprightFont=cmuntt.ttf,BoldFont=cmuntb.ttf,ItalicFont=cmunit.ttf,BoldItalicFont=cmuntx.ttf]{cmunti.ttf}\itshape ref\_key}{$\text{ }$}\setmainfont[Path=/usr/share/fonts/truetype/cmu/,UprightFont=cmunrm.ttf,BoldFont=cmunbx.ttf,ItalicFont=cmunti.ttf,BoldItalicFont=cmunbi.ttf]{cmunrm.ttf}\setmonofont[Path=/usr/share/fonts/truetype/cmu/,UprightFont=cmuntt.ttf,BoldFont=cmuntb.ttf,ItalicFont=cmunit.ttf,BoldItalicFont=cmuntx.ttf]{cmunrm.ttf} corresponds exactly to one of the entries in the .bib file. If you wish to cite more than one reference at the same time, do the following: {\ttfamily \setmainfont[Path=/usr/share/fonts/truetype/cmu/,UprightFont=cmunrm.ttf,BoldFont=cmunbx.ttf,ItalicFont=cmunti.ttf,BoldItalicFont=cmunbi.ttf]{cmuntt.ttf}\setmonofont[Path=/usr/share/fonts/truetype/cmu/,UprightFont=cmuntt.ttf,BoldFont=cmuntb.ttf,ItalicFont=cmunit.ttf,BoldItalicFont=cmuntx.ttf]{cmuntt.ttf}\ttfamily \textbackslash{}cite\{{\itshape \setmainfont[Path=/usr/share/fonts/truetype/cmu/,UprightFont=cmunrm.ttf,BoldFont=cmunbx.ttf,ItalicFont=cmunti.ttf,BoldItalicFont=cmunbi.ttf]{cmunit.ttf}\setmonofont[Path=/usr/share/fonts/truetype/cmu/,UprightFont=cmuntt.ttf,BoldFont=cmuntb.ttf,ItalicFont=cmunit.ttf,BoldItalicFont=cmuntx.ttf]{cmunit.ttf}\ttfamily \itshape ref\_key1}\setmainfont[Path=/usr/share/fonts/truetype/cmu/,UprightFont=cmunrm.ttf,BoldFont=cmunbx.ttf,ItalicFont=cmunti.ttf,BoldItalicFont=cmunbi.ttf]{cmuntt.ttf}\setmonofont[Path=/usr/share/fonts/truetype/cmu/,UprightFont=cmuntt.ttf,BoldFont=cmuntb.ttf,ItalicFont=cmunit.ttf,BoldItalicFont=cmuntx.ttf]{cmuntt.ttf}\ttfamily , {\itshape \setmainfont[Path=/usr/share/fonts/truetype/cmu/,UprightFont=cmunrm.ttf,BoldFont=cmunbx.ttf,ItalicFont=cmunti.ttf,BoldItalicFont=cmunbi.ttf]{cmunit.ttf}\setmonofont[Path=/usr/share/fonts/truetype/cmu/,UprightFont=cmuntt.ttf,BoldFont=cmuntb.ttf,ItalicFont=cmunit.ttf,BoldItalicFont=cmuntx.ttf]{cmunit.ttf}\ttfamily \itshape ref\_key2}\setmainfont[Path=/usr/share/fonts/truetype/cmu/,UprightFont=cmunrm.ttf,BoldFont=cmunbx.ttf,ItalicFont=cmunti.ttf,BoldItalicFont=cmunbi.ttf]{cmuntt.ttf}\setmonofont[Path=/usr/share/fonts/truetype/cmu/,UprightFont=cmuntt.ttf,BoldFont=cmuntb.ttf,ItalicFont=cmunit.ttf,BoldItalicFont=cmuntx.ttf]{cmuntt.ttf}\ttfamily , ..., {\itshape \setmainfont[Path=/usr/share/fonts/truetype/cmu/,UprightFont=cmunrm.ttf,BoldFont=cmunbx.ttf,ItalicFont=cmunti.ttf,BoldItalicFont=cmunbi.ttf]{cmunit.ttf}\setmonofont[Path=/usr/share/fonts/truetype/cmu/,UprightFont=cmuntt.ttf,BoldFont=cmuntb.ttf,ItalicFont=cmunit.ttf,BoldItalicFont=cmuntx.ttf]{cmunit.ttf}\ttfamily \itshape ref\_keyN}\setmainfont[Path=/usr/share/fonts/truetype/cmu/,UprightFont=cmunrm.ttf,BoldFont=cmunbx.ttf,ItalicFont=cmunti.ttf,BoldItalicFont=cmunbi.ttf]{cmuntt.ttf}\setmonofont[Path=/usr/share/fonts/truetype/cmu/,UprightFont=cmuntt.ttf,BoldFont=cmuntb.ttf,ItalicFont=cmunit.ttf,BoldItalicFont=cmuntx.ttf]{cmuntt.ttf}\ttfamily \}}\setmainfont[Path=/usr/share/fonts/truetype/cmu/,UprightFont=cmunrm.ttf,BoldFont=cmunbx.ttf,ItalicFont=cmunti.ttf,BoldItalicFont=cmunbi.ttf]{cmunrm.ttf}\setmonofont[Path=/usr/share/fonts/truetype/cmu/,UprightFont=cmuntt.ttf,BoldFont=cmuntb.ttf,ItalicFont=cmunit.ttf,BoldItalicFont=cmuntx.ttf]{cmunrm.ttf}.
\subsection{Why won\textquotesingle{}t LaTeX generate any output?}
\label{683}

The addition of BibTeX adds extra complexity for the processing of the source to the desired output. This is largely hidden from the user, but because of all the complexity of the referencing of citations from your source LaTeX file to the database entries in another file, you actually need multiple passes to accomplish the task. This means you have to run LaTeX a number of times.  Each pass will perform a particular task until it has managed to resolve all the citation references. Here\textquotesingle{}s what you need to type (into command line): 
\begin{myenumerate}
\item{}  {\ttfamily \setmainfont[Path=/usr/share/fonts/truetype/cmu/,UprightFont=cmunrm.ttf,BoldFont=cmunbx.ttf,ItalicFont=cmunti.ttf,BoldItalicFont=cmunbi.ttf]{cmuntt.ttf}\setmonofont[Path=/usr/share/fonts/truetype/cmu/,UprightFont=cmuntt.ttf,BoldFont=cmuntb.ttf,ItalicFont=cmunit.ttf,BoldItalicFont=cmuntx.ttf]{cmuntt.ttf}\ttfamily latex latex\_source\_code.tex}
\item{} {$\text{ }$}\setmainfont[Path=/usr/share/fonts/truetype/cmu/,UprightFont=cmunrm.ttf,BoldFont=cmunbx.ttf,ItalicFont=cmunti.ttf,BoldItalicFont=cmunbi.ttf]{cmunrm.ttf}\setmonofont[Path=/usr/share/fonts/truetype/cmu/,UprightFont=cmuntt.ttf,BoldFont=cmuntb.ttf,ItalicFont=cmunit.ttf,BoldItalicFont=cmuntx.ttf]{cmunrm.ttf} {\ttfamily \setmainfont[Path=/usr/share/fonts/truetype/cmu/,UprightFont=cmunrm.ttf,BoldFont=cmunbx.ttf,ItalicFont=cmunti.ttf,BoldItalicFont=cmunbi.ttf]{cmuntt.ttf}\setmonofont[Path=/usr/share/fonts/truetype/cmu/,UprightFont=cmuntt.ttf,BoldFont=cmuntb.ttf,ItalicFont=cmunit.ttf,BoldItalicFont=cmuntx.ttf]{cmuntt.ttf}\ttfamily bibtex latex\_source\_code.aux}
\item{} {$\text{ }$}\setmainfont[Path=/usr/share/fonts/truetype/cmu/,UprightFont=cmunrm.ttf,BoldFont=cmunbx.ttf,ItalicFont=cmunti.ttf,BoldItalicFont=cmunbi.ttf]{cmunrm.ttf}\setmonofont[Path=/usr/share/fonts/truetype/cmu/,UprightFont=cmuntt.ttf,BoldFont=cmuntb.ttf,ItalicFont=cmunit.ttf,BoldItalicFont=cmuntx.ttf]{cmunrm.ttf} {\ttfamily \setmainfont[Path=/usr/share/fonts/truetype/cmu/,UprightFont=cmunrm.ttf,BoldFont=cmunbx.ttf,ItalicFont=cmunti.ttf,BoldItalicFont=cmunbi.ttf]{cmuntt.ttf}\setmonofont[Path=/usr/share/fonts/truetype/cmu/,UprightFont=cmuntt.ttf,BoldFont=cmuntb.ttf,ItalicFont=cmunit.ttf,BoldItalicFont=cmuntx.ttf]{cmuntt.ttf}\ttfamily latex latex\_source\_code.tex}
\item{} {$\text{ }$}\setmainfont[Path=/usr/share/fonts/truetype/cmu/,UprightFont=cmunrm.ttf,BoldFont=cmunbx.ttf,ItalicFont=cmunti.ttf,BoldItalicFont=cmunbi.ttf]{cmunrm.ttf}\setmonofont[Path=/usr/share/fonts/truetype/cmu/,UprightFont=cmuntt.ttf,BoldFont=cmuntb.ttf,ItalicFont=cmunit.ttf,BoldItalicFont=cmuntx.ttf]{cmunrm.ttf} {\ttfamily \setmainfont[Path=/usr/share/fonts/truetype/cmu/,UprightFont=cmunrm.ttf,BoldFont=cmunbx.ttf,ItalicFont=cmunti.ttf,BoldItalicFont=cmunbi.ttf]{cmuntt.ttf}\setmonofont[Path=/usr/share/fonts/truetype/cmu/,UprightFont=cmuntt.ttf,BoldFont=cmuntb.ttf,ItalicFont=cmunit.ttf,BoldItalicFont=cmuntx.ttf]{cmuntt.ttf}\ttfamily latex latex\_source\_code.tex}
\end{myenumerate}
\setmainfont[Path=/usr/share/fonts/truetype/cmu/,UprightFont=cmunrm.ttf,BoldFont=cmunbx.ttf,ItalicFont=cmunti.ttf,BoldItalicFont=cmunbi.ttf]{cmunrm.ttf}\setmonofont[Path=/usr/share/fonts/truetype/cmu/,UprightFont=cmuntt.ttf,BoldFont=cmuntb.ttf,ItalicFont=cmunit.ttf,BoldItalicFont=cmuntx.ttf]{cmunrm.ttf}
(Extensions are optional, if you put them note that the bibtex command takes the AUX file as input.)

After the first LaTeX run, you will see errors such as:

\TemplatePreformat{$\text{ }$\newline{}
LaTeX$\text{ }${}Warning:$\text{ }${}Citation$\text{ }${}`lamport94\textquotesingle{}$\text{ }${}on$\text{ }${}page$\text{ }${}1$\text{ }${}undefined$\text{ }${}on$\text{ }${}input$\text{ }${}line$\text{ }${}21.$\text{ }$\newline{}
...$\text{ }$\newline{}
LaTeX$\text{ }${}Warning:$\text{ }${}There$\text{ }${}were$\text{ }${}undefined$\text{ }${}references.$\text{ }$\newline{}
}

The next step is to run bibtex on that same LaTeX source (or more precisely the corresponding AUX file, however not on the actual .bib file) to then define all the references within that document. You should see output like the following:

\TemplatePreformat{$\text{ }$\newline{}
This$\text{ }${}is$\text{ }${}BibTeX,$\text{ }${}Version$\text{ }${}0.99c$\text{ }${}(Web2C$\text{ }${}7.3.1)$\text{ }$\newline{}
The$\text{ }${}top-{}level$\text{ }${}auxiliary$\text{ }${}file:$\text{ }${}latex\_source\_code.aux$\text{ }$\newline{}
The$\text{ }${}style$\text{ }${}file:$\text{ }${}plain.bst$\text{ }$\newline{}
Database$\text{ }${}file$\text{ }${}\#1:$\text{ }${}sample.bib$\text{ }$\newline{}
}

The third step, which is invoking LaTeX for the second time will see more errors like \symbol{34}{\ttfamily \setmainfont[Path=/usr/share/fonts/truetype/cmu/,UprightFont=cmunrm.ttf,BoldFont=cmunbx.ttf,ItalicFont=cmunti.ttf,BoldItalicFont=cmunbi.ttf]{cmuntt.ttf}\setmonofont[Path=/usr/share/fonts/truetype/cmu/,UprightFont=cmuntt.ttf,BoldFont=cmuntb.ttf,ItalicFont=cmunit.ttf,BoldItalicFont=cmuntx.ttf]{cmuntt.ttf}\ttfamily LaTeX Warning: Label(s) may have changed. Rerun to get cross-{}references right.}\setmainfont[Path=/usr/share/fonts/truetype/cmu/,UprightFont=cmunrm.ttf,BoldFont=cmunbx.ttf,ItalicFont=cmunti.ttf,BoldItalicFont=cmunbi.ttf]{cmunrm.ttf}\setmonofont[Path=/usr/share/fonts/truetype/cmu/,UprightFont=cmuntt.ttf,BoldFont=cmuntb.ttf,ItalicFont=cmunit.ttf,BoldItalicFont=cmuntx.ttf]{cmunrm.ttf}\symbol{34}. Don\textquotesingle{}t be alarmed, it\textquotesingle{}s almost complete. As you can guess, all you have to do is follow its instructions, and run LaTeX for the third time, and the document will be output as expected, without further problems.

If you want a pdf output instead of a dvi output you can use {\ttfamily \setmainfont[Path=/usr/share/fonts/truetype/cmu/,UprightFont=cmunrm.ttf,BoldFont=cmunbx.ttf,ItalicFont=cmunti.ttf,BoldItalicFont=cmunbi.ttf]{cmuntt.ttf}\setmonofont[Path=/usr/share/fonts/truetype/cmu/,UprightFont=cmuntt.ttf,BoldFont=cmuntb.ttf,ItalicFont=cmunit.ttf,BoldItalicFont=cmuntx.ttf]{cmuntt.ttf}\ttfamily pdflatex}{$\text{ }$}\setmainfont[Path=/usr/share/fonts/truetype/cmu/,UprightFont=cmunrm.ttf,BoldFont=cmunbx.ttf,ItalicFont=cmunti.ttf,BoldItalicFont=cmunbi.ttf]{cmunrm.ttf}\setmonofont[Path=/usr/share/fonts/truetype/cmu/,UprightFont=cmuntt.ttf,BoldFont=cmuntb.ttf,ItalicFont=cmunit.ttf,BoldItalicFont=cmuntx.ttf]{cmunrm.ttf} instead of {\ttfamily \setmainfont[Path=/usr/share/fonts/truetype/cmu/,UprightFont=cmunrm.ttf,BoldFont=cmunbx.ttf,ItalicFont=cmunti.ttf,BoldItalicFont=cmunbi.ttf]{cmuntt.ttf}\setmonofont[Path=/usr/share/fonts/truetype/cmu/,UprightFont=cmuntt.ttf,BoldFont=cmuntb.ttf,ItalicFont=cmunit.ttf,BoldItalicFont=cmuntx.ttf]{cmuntt.ttf}\ttfamily latex}{$\text{ }$}\setmainfont[Path=/usr/share/fonts/truetype/cmu/,UprightFont=cmunrm.ttf,BoldFont=cmunbx.ttf,ItalicFont=cmunti.ttf,BoldItalicFont=cmunbi.ttf]{cmunrm.ttf}\setmonofont[Path=/usr/share/fonts/truetype/cmu/,UprightFont=cmuntt.ttf,BoldFont=cmuntb.ttf,ItalicFont=cmunit.ttf,BoldItalicFont=cmuntx.ttf]{cmunrm.ttf} as follows:

\begin{myenumerate}
\item{}  {\ttfamily \setmainfont[Path=/usr/share/fonts/truetype/cmu/,UprightFont=cmunrm.ttf,BoldFont=cmunbx.ttf,ItalicFont=cmunti.ttf,BoldItalicFont=cmunbi.ttf]{cmuntt.ttf}\setmonofont[Path=/usr/share/fonts/truetype/cmu/,UprightFont=cmuntt.ttf,BoldFont=cmuntb.ttf,ItalicFont=cmunit.ttf,BoldItalicFont=cmuntx.ttf]{cmuntt.ttf}\ttfamily pdflatex latex\_source\_code.tex}
\item{} {$\text{ }$}\setmainfont[Path=/usr/share/fonts/truetype/cmu/,UprightFont=cmunrm.ttf,BoldFont=cmunbx.ttf,ItalicFont=cmunti.ttf,BoldItalicFont=cmunbi.ttf]{cmunrm.ttf}\setmonofont[Path=/usr/share/fonts/truetype/cmu/,UprightFont=cmuntt.ttf,BoldFont=cmuntb.ttf,ItalicFont=cmunit.ttf,BoldItalicFont=cmuntx.ttf]{cmunrm.ttf} {\ttfamily \setmainfont[Path=/usr/share/fonts/truetype/cmu/,UprightFont=cmunrm.ttf,BoldFont=cmunbx.ttf,ItalicFont=cmunti.ttf,BoldItalicFont=cmunbi.ttf]{cmuntt.ttf}\setmonofont[Path=/usr/share/fonts/truetype/cmu/,UprightFont=cmuntt.ttf,BoldFont=cmuntb.ttf,ItalicFont=cmunit.ttf,BoldItalicFont=cmuntx.ttf]{cmuntt.ttf}\ttfamily bibtex latex\_source\_code.aux}
\item{} {$\text{ }$}\setmainfont[Path=/usr/share/fonts/truetype/cmu/,UprightFont=cmunrm.ttf,BoldFont=cmunbx.ttf,ItalicFont=cmunti.ttf,BoldItalicFont=cmunbi.ttf]{cmunrm.ttf}\setmonofont[Path=/usr/share/fonts/truetype/cmu/,UprightFont=cmuntt.ttf,BoldFont=cmuntb.ttf,ItalicFont=cmunit.ttf,BoldItalicFont=cmuntx.ttf]{cmunrm.ttf} {\ttfamily \setmainfont[Path=/usr/share/fonts/truetype/cmu/,UprightFont=cmunrm.ttf,BoldFont=cmunbx.ttf,ItalicFont=cmunti.ttf,BoldItalicFont=cmunbi.ttf]{cmuntt.ttf}\setmonofont[Path=/usr/share/fonts/truetype/cmu/,UprightFont=cmuntt.ttf,BoldFont=cmuntb.ttf,ItalicFont=cmunit.ttf,BoldItalicFont=cmuntx.ttf]{cmuntt.ttf}\ttfamily pdflatex latex\_source\_code.tex}
\item{} {$\text{ }$}\setmainfont[Path=/usr/share/fonts/truetype/cmu/,UprightFont=cmunrm.ttf,BoldFont=cmunbx.ttf,ItalicFont=cmunti.ttf,BoldItalicFont=cmunbi.ttf]{cmunrm.ttf}\setmonofont[Path=/usr/share/fonts/truetype/cmu/,UprightFont=cmuntt.ttf,BoldFont=cmuntb.ttf,ItalicFont=cmunit.ttf,BoldItalicFont=cmuntx.ttf]{cmunrm.ttf} {\ttfamily \setmainfont[Path=/usr/share/fonts/truetype/cmu/,UprightFont=cmunrm.ttf,BoldFont=cmunbx.ttf,ItalicFont=cmunti.ttf,BoldItalicFont=cmunbi.ttf]{cmuntt.ttf}\setmonofont[Path=/usr/share/fonts/truetype/cmu/,UprightFont=cmuntt.ttf,BoldFont=cmuntb.ttf,ItalicFont=cmunit.ttf,BoldItalicFont=cmuntx.ttf]{cmuntt.ttf}\ttfamily pdflatex latex\_source\_code.tex}
\end{myenumerate}
\setmainfont[Path=/usr/share/fonts/truetype/cmu/,UprightFont=cmunrm.ttf,BoldFont=cmunbx.ttf,ItalicFont=cmunti.ttf,BoldItalicFont=cmunbi.ttf]{cmunrm.ttf}\setmonofont[Path=/usr/share/fonts/truetype/cmu/,UprightFont=cmuntt.ttf,BoldFont=cmuntb.ttf,ItalicFont=cmunit.ttf,BoldItalicFont=cmuntx.ttf]{cmunrm.ttf}
(Extensions are optional, if you put them note that the bibtex command takes the AUX file as input.)

Note that if you are editing your source in vim and attempt to use command mode and the current file shortcut (\%) to process the document like this:

\begin{myenumerate}
\item{} {\ttfamily \setmainfont[Path=/usr/share/fonts/truetype/cmu/,UprightFont=cmunrm.ttf,BoldFont=cmunbx.ttf,ItalicFont=cmunti.ttf,BoldItalicFont=cmunbi.ttf]{cmuntt.ttf}\setmonofont[Path=/usr/share/fonts/truetype/cmu/,UprightFont=cmuntt.ttf,BoldFont=cmuntb.ttf,ItalicFont=cmunit.ttf,BoldItalicFont=cmuntx.ttf]{cmuntt.ttf}\ttfamily :! pdflatex \%}
\item{} {\ttfamily :! bibtex \%}
\end{myenumerate}
\setmainfont[Path=/usr/share/fonts/truetype/cmu/,UprightFont=cmunrm.ttf,BoldFont=cmunbx.ttf,ItalicFont=cmunti.ttf,BoldItalicFont=cmunbi.ttf]{cmunrm.ttf}\setmonofont[Path=/usr/share/fonts/truetype/cmu/,UprightFont=cmuntt.ttf,BoldFont=cmuntb.ttf,ItalicFont=cmunit.ttf,BoldItalicFont=cmuntx.ttf]{cmunrm.ttf}

You will get an error similar to this:  

\begin{myenumerate}
\item{}  {\ttfamily \setmainfont[Path=/usr/share/fonts/truetype/cmu/,UprightFont=cmunrm.ttf,BoldFont=cmunbx.ttf,ItalicFont=cmunti.ttf,BoldItalicFont=cmunbi.ttf]{cmuntt.ttf}\setmonofont[Path=/usr/share/fonts/truetype/cmu/,UprightFont=cmuntt.ttf,BoldFont=cmuntb.ttf,ItalicFont=cmunit.ttf,BoldItalicFont=cmuntx.ttf]{cmuntt.ttf}\ttfamily I couldn\textquotesingle{}t open file name \textquotesingle{}current\_file.tex.aux\textquotesingle{}}
\end{myenumerate}
\setmainfont[Path=/usr/share/fonts/truetype/cmu/,UprightFont=cmunrm.ttf,BoldFont=cmunbx.ttf,ItalicFont=cmunti.ttf,BoldItalicFont=cmunbi.ttf]{cmunrm.ttf}\setmonofont[Path=/usr/share/fonts/truetype/cmu/,UprightFont=cmuntt.ttf,BoldFont=cmuntb.ttf,ItalicFont=cmunit.ttf,BoldItalicFont=cmuntx.ttf]{cmunrm.ttf}

It appears that the file extension is included by default when the current file command (\%) is executed.  To process your document from within vim, you must explicitly name the file without the file extension for bibtex to work, as is shown below:

\begin{myenumerate}
\item{}  {\ttfamily \setmainfont[Path=/usr/share/fonts/truetype/cmu/,UprightFont=cmunrm.ttf,BoldFont=cmunbx.ttf,ItalicFont=cmunti.ttf,BoldItalicFont=cmunbi.ttf]{cmuntt.ttf}\setmonofont[Path=/usr/share/fonts/truetype/cmu/,UprightFont=cmuntt.ttf,BoldFont=cmuntb.ttf,ItalicFont=cmunit.ttf,BoldItalicFont=cmuntx.ttf]{cmuntt.ttf}\ttfamily :! pdflatex \%}
\item{} {$\text{ }$}\setmainfont[Path=/usr/share/fonts/truetype/cmu/,UprightFont=cmunrm.ttf,BoldFont=cmunbx.ttf,ItalicFont=cmunti.ttf,BoldItalicFont=cmunbi.ttf]{cmunrm.ttf}\setmonofont[Path=/usr/share/fonts/truetype/cmu/,UprightFont=cmuntt.ttf,BoldFont=cmuntb.ttf,ItalicFont=cmunit.ttf,BoldItalicFont=cmuntx.ttf]{cmunrm.ttf} {\ttfamily \setmainfont[Path=/usr/share/fonts/truetype/cmu/,UprightFont=cmunrm.ttf,BoldFont=cmunbx.ttf,ItalicFont=cmunti.ttf,BoldItalicFont=cmunbi.ttf]{cmuntt.ttf}\setmonofont[Path=/usr/share/fonts/truetype/cmu/,UprightFont=cmuntt.ttf,BoldFont=cmuntb.ttf,ItalicFont=cmunit.ttf,BoldItalicFont=cmuntx.ttf]{cmuntt.ttf}\ttfamily :! bibtex \%:r}{$\text{ }$}\setmainfont[Path=/usr/share/fonts/truetype/cmu/,UprightFont=cmunrm.ttf,BoldFont=cmunbx.ttf,ItalicFont=cmunti.ttf,BoldItalicFont=cmunbi.ttf]{cmunrm.ttf}\setmonofont[Path=/usr/share/fonts/truetype/cmu/,UprightFont=cmuntt.ttf,BoldFont=cmuntb.ttf,ItalicFont=cmunit.ttf,BoldItalicFont=cmuntx.ttf]{cmunrm.ttf} (without file extension, it looks for the AUX file as mentioned above)
\item{}  {\ttfamily \setmainfont[Path=/usr/share/fonts/truetype/cmu/,UprightFont=cmunrm.ttf,BoldFont=cmunbx.ttf,ItalicFont=cmunti.ttf,BoldItalicFont=cmunbi.ttf]{cmuntt.ttf}\setmonofont[Path=/usr/share/fonts/truetype/cmu/,UprightFont=cmuntt.ttf,BoldFont=cmuntb.ttf,ItalicFont=cmunit.ttf,BoldItalicFont=cmuntx.ttf]{cmuntt.ttf}\ttfamily :! pdflatex \%}
\item{} {$\text{ }$}\setmainfont[Path=/usr/share/fonts/truetype/cmu/,UprightFont=cmunrm.ttf,BoldFont=cmunbx.ttf,ItalicFont=cmunti.ttf,BoldItalicFont=cmunbi.ttf]{cmunrm.ttf}\setmonofont[Path=/usr/share/fonts/truetype/cmu/,UprightFont=cmuntt.ttf,BoldFont=cmuntb.ttf,ItalicFont=cmunit.ttf,BoldItalicFont=cmuntx.ttf]{cmunrm.ttf} {\ttfamily \setmainfont[Path=/usr/share/fonts/truetype/cmu/,UprightFont=cmunrm.ttf,BoldFont=cmunbx.ttf,ItalicFont=cmunti.ttf,BoldItalicFont=cmunbi.ttf]{cmuntt.ttf}\setmonofont[Path=/usr/share/fonts/truetype/cmu/,UprightFont=cmuntt.ttf,BoldFont=cmuntb.ttf,ItalicFont=cmunit.ttf,BoldItalicFont=cmuntx.ttf]{cmuntt.ttf}\ttfamily :! pdflatex \%}
\end{myenumerate}
\setmainfont[Path=/usr/share/fonts/truetype/cmu/,UprightFont=cmunrm.ttf,BoldFont=cmunbx.ttf,ItalicFont=cmunti.ttf,BoldItalicFont=cmunbi.ttf]{cmunrm.ttf}\setmonofont[Path=/usr/share/fonts/truetype/cmu/,UprightFont=cmuntt.ttf,BoldFont=cmuntb.ttf,ItalicFont=cmunit.ttf,BoldItalicFont=cmuntx.ttf]{cmunrm.ttf}

However, it is much easier to install the Vim-{}LaTeX plugin from \myhref{http://vim-latex.sourceforge.net/}{here}. This allows you to simply type \textbackslash{}ll when not in insert mode, and all the appropriate commands are automatically executed to compile the document. Vim-{}LaTeX even detects how many times it has to run pdflatex, and whether or not it has to run bibtex. This is just one of the many nice features of Vim-{}LaTeX, you can read the excellent \myhref{http://vim-latex.sourceforge.net/documentation/latex-suite-quickstart/}{Beginner\textquotesingle{}s Tutorial} for more about the many clever shortcuts Vim-{}LaTeX provides.

Another option exists if you are running Unix/Linux or any other platform where you have \myhref{http://en.wikipedia.org/wiki/Make_\%28software\%29}{make}. Then you can simply create a Makefile and use vim\textquotesingle{}s make command or use make in shell. The Makefile would then look like this:
\\

\TemplateSpaceIndent{$\text{ }${}latex\_source\_code.pdf:$\text{ }${}latex\_source\_code.tex$\text{ }${}latex\_source\_code.bib$\text{ }$\newline{}
$\text{ }${}${\text{ }}${}${\text{ }}${}${\text{ }}${}${\text{ }}${}pdflatex$\text{ }${}latex\_source\_code.tex$\text{ }$\newline{}
$\text{ }${}${\text{ }}${}${\text{ }}${}${\text{ }}${}${\text{ }}${}bibtex$\text{ }${}latex\_source\_code.aux$\text{ }$\newline{}
$\text{ }${}${\text{ }}${}${\text{ }}${}${\text{ }}${}${\text{ }}${}pdflatex$\text{ }${}latex\_source\_code.tex$\text{ }$\newline{}
$\text{ }${}${\text{ }}${}${\text{ }}${}${\text{ }}${}${\text{ }}${}pdflatex$\text{ }${}latex\_source\_code.tex}

\subsection{Including URLs in bibliography}
\label{684}
As you can see, there is no field for URLs. One possibility is to include Internet addresses in {\ttfamily \setmainfont[Path=/usr/share/fonts/truetype/cmu/,UprightFont=cmunrm.ttf,BoldFont=cmunbx.ttf,ItalicFont=cmunti.ttf,BoldItalicFont=cmunbi.ttf]{cmuntt.ttf}\setmonofont[Path=/usr/share/fonts/truetype/cmu/,UprightFont=cmuntt.ttf,BoldFont=cmuntb.ttf,ItalicFont=cmunit.ttf,BoldItalicFont=cmuntx.ttf]{cmuntt.ttf}\ttfamily howpublished}{$\text{ }$}\setmainfont[Path=/usr/share/fonts/truetype/cmu/,UprightFont=cmunrm.ttf,BoldFont=cmunbx.ttf,ItalicFont=cmunti.ttf,BoldItalicFont=cmunbi.ttf]{cmunrm.ttf}\setmonofont[Path=/usr/share/fonts/truetype/cmu/,UprightFont=cmuntt.ttf,BoldFont=cmuntb.ttf,ItalicFont=cmunit.ttf,BoldItalicFont=cmuntx.ttf]{cmunrm.ttf} field of {\ttfamily \setmainfont[Path=/usr/share/fonts/truetype/cmu/,UprightFont=cmunrm.ttf,BoldFont=cmunbx.ttf,ItalicFont=cmunti.ttf,BoldItalicFont=cmunbi.ttf]{cmuntt.ttf}\setmonofont[Path=/usr/share/fonts/truetype/cmu/,UprightFont=cmuntt.ttf,BoldFont=cmuntb.ttf,ItalicFont=cmunit.ttf,BoldItalicFont=cmuntx.ttf]{cmuntt.ttf}\ttfamily @misc}{$\text{ }$}\setmainfont[Path=/usr/share/fonts/truetype/cmu/,UprightFont=cmunrm.ttf,BoldFont=cmunbx.ttf,ItalicFont=cmunti.ttf,BoldItalicFont=cmunbi.ttf]{cmunrm.ttf}\setmonofont[Path=/usr/share/fonts/truetype/cmu/,UprightFont=cmuntt.ttf,BoldFont=cmuntb.ttf,ItalicFont=cmunit.ttf,BoldItalicFont=cmuntx.ttf]{cmunrm.ttf} or {\ttfamily \setmainfont[Path=/usr/share/fonts/truetype/cmu/,UprightFont=cmunrm.ttf,BoldFont=cmunbx.ttf,ItalicFont=cmunti.ttf,BoldItalicFont=cmunbi.ttf]{cmuntt.ttf}\setmonofont[Path=/usr/share/fonts/truetype/cmu/,UprightFont=cmuntt.ttf,BoldFont=cmuntb.ttf,ItalicFont=cmunit.ttf,BoldItalicFont=cmuntx.ttf]{cmuntt.ttf}\ttfamily note}{$\text{ }$}\setmainfont[Path=/usr/share/fonts/truetype/cmu/,UprightFont=cmunrm.ttf,BoldFont=cmunbx.ttf,ItalicFont=cmunti.ttf,BoldItalicFont=cmunbi.ttf]{cmunrm.ttf}\setmonofont[Path=/usr/share/fonts/truetype/cmu/,UprightFont=cmuntt.ttf,BoldFont=cmuntb.ttf,ItalicFont=cmunit.ttf,BoldItalicFont=cmuntx.ttf]{cmunrm.ttf} field of {\ttfamily \setmainfont[Path=/usr/share/fonts/truetype/cmu/,UprightFont=cmunrm.ttf,BoldFont=cmunbx.ttf,ItalicFont=cmunti.ttf,BoldItalicFont=cmunbi.ttf]{cmuntt.ttf}\setmonofont[Path=/usr/share/fonts/truetype/cmu/,UprightFont=cmuntt.ttf,BoldFont=cmuntb.ttf,ItalicFont=cmunit.ttf,BoldItalicFont=cmuntx.ttf]{cmuntt.ttf}\ttfamily @techreport}\setmainfont[Path=/usr/share/fonts/truetype/cmu/,UprightFont=cmunrm.ttf,BoldFont=cmunbx.ttf,ItalicFont=cmunti.ttf,BoldItalicFont=cmunbi.ttf]{cmunrm.ttf}\setmonofont[Path=/usr/share/fonts/truetype/cmu/,UprightFont=cmuntt.ttf,BoldFont=cmuntb.ttf,ItalicFont=cmunit.ttf,BoldItalicFont=cmuntx.ttf]{cmunrm.ttf}, {\ttfamily \setmainfont[Path=/usr/share/fonts/truetype/cmu/,UprightFont=cmunrm.ttf,BoldFont=cmunbx.ttf,ItalicFont=cmunti.ttf,BoldItalicFont=cmunbi.ttf]{cmuntt.ttf}\setmonofont[Path=/usr/share/fonts/truetype/cmu/,UprightFont=cmuntt.ttf,BoldFont=cmuntb.ttf,ItalicFont=cmunit.ttf,BoldItalicFont=cmuntx.ttf]{cmuntt.ttf}\ttfamily @article}\setmainfont[Path=/usr/share/fonts/truetype/cmu/,UprightFont=cmunrm.ttf,BoldFont=cmunbx.ttf,ItalicFont=cmunti.ttf,BoldItalicFont=cmunbi.ttf]{cmunrm.ttf}\setmonofont[Path=/usr/share/fonts/truetype/cmu/,UprightFont=cmuntt.ttf,BoldFont=cmuntb.ttf,ItalicFont=cmunit.ttf,BoldItalicFont=cmuntx.ttf]{cmunrm.ttf}, {\ttfamily \setmainfont[Path=/usr/share/fonts/truetype/cmu/,UprightFont=cmunrm.ttf,BoldFont=cmunbx.ttf,ItalicFont=cmunti.ttf,BoldItalicFont=cmunbi.ttf]{cmuntt.ttf}\setmonofont[Path=/usr/share/fonts/truetype/cmu/,UprightFont=cmuntt.ttf,BoldFont=cmuntb.ttf,ItalicFont=cmunit.ttf,BoldItalicFont=cmuntx.ttf]{cmuntt.ttf}\ttfamily @book}\setmainfont[Path=/usr/share/fonts/truetype/cmu/,UprightFont=cmunrm.ttf,BoldFont=cmunbx.ttf,ItalicFont=cmunti.ttf,BoldItalicFont=cmunbi.ttf]{cmunrm.ttf}\setmonofont[Path=/usr/share/fonts/truetype/cmu/,UprightFont=cmuntt.ttf,BoldFont=cmuntb.ttf,ItalicFont=cmunit.ttf,BoldItalicFont=cmuntx.ttf]{cmunrm.ttf}:


\begin{Shaded}
\begin{Highlighting}[]

\CommentTok{howpublished\ensuremath{\text{ }}=\ensuremath{\text{ }}"\textbackslash{}url\{http://www.example.com\}"}\newline
\end{Highlighting}
\end{Shaded}


Note the usage of {\ttfamily \setmainfont[Path=/usr/share/fonts/truetype/cmu/,UprightFont=cmunrm.ttf,BoldFont=cmunbx.ttf,ItalicFont=cmunti.ttf,BoldItalicFont=cmunbi.ttf]{cmuntt.ttf}\setmonofont[Path=/usr/share/fonts/truetype/cmu/,UprightFont=cmuntt.ttf,BoldFont=cmuntb.ttf,ItalicFont=cmunit.ttf,BoldItalicFont=cmuntx.ttf]{cmuntt.ttf}\ttfamily \textbackslash{}url}{$\text{ }$}\setmainfont[Path=/usr/share/fonts/truetype/cmu/,UprightFont=cmunrm.ttf,BoldFont=cmunbx.ttf,ItalicFont=cmunti.ttf,BoldItalicFont=cmunbi.ttf]{cmunrm.ttf}\setmonofont[Path=/usr/share/fonts/truetype/cmu/,UprightFont=cmuntt.ttf,BoldFont=cmuntb.ttf,ItalicFont=cmunit.ttf,BoldItalicFont=cmuntx.ttf]{cmunrm.ttf} command to ensure \myhref{https://en.wikibooks.org/wiki/LaTeX\%2FFormatting\%23Typesetting\%20URLs}{proper appearance of URLs}.

Another way is to use special field {\ttfamily \setmainfont[Path=/usr/share/fonts/truetype/cmu/,UprightFont=cmunrm.ttf,BoldFont=cmunbx.ttf,ItalicFont=cmunti.ttf,BoldItalicFont=cmunbi.ttf]{cmuntt.ttf}\setmonofont[Path=/usr/share/fonts/truetype/cmu/,UprightFont=cmuntt.ttf,BoldFont=cmuntb.ttf,ItalicFont=cmunit.ttf,BoldItalicFont=cmuntx.ttf]{cmuntt.ttf}\ttfamily url}{$\text{ }$}\setmainfont[Path=/usr/share/fonts/truetype/cmu/,UprightFont=cmunrm.ttf,BoldFont=cmunbx.ttf,ItalicFont=cmunti.ttf,BoldItalicFont=cmunbi.ttf]{cmunrm.ttf}\setmonofont[Path=/usr/share/fonts/truetype/cmu/,UprightFont=cmuntt.ttf,BoldFont=cmuntb.ttf,ItalicFont=cmunit.ttf,BoldItalicFont=cmuntx.ttf]{cmunrm.ttf} and make bibliography style recognise it.


\begin{Shaded}
\begin{Highlighting}[]

\CommentTok{\ensuremath{\text{ }}url\ensuremath{\text{ }}=\ensuremath{\text{ }}"http://www.example.com"}\newline
\end{Highlighting}
\end{Shaded}


You need to use {\ttfamily \setmainfont[Path=/usr/share/fonts/truetype/cmu/,UprightFont=cmunrm.ttf,BoldFont=cmunbx.ttf,ItalicFont=cmunti.ttf,BoldItalicFont=cmunbi.ttf]{cmuntt.ttf}\setmonofont[Path=/usr/share/fonts/truetype/cmu/,UprightFont=cmuntt.ttf,BoldFont=cmuntb.ttf,ItalicFont=cmunit.ttf,BoldItalicFont=cmuntx.ttf]{cmuntt.ttf}\ttfamily \textbackslash{}usepackage\{url\}}{$\text{ }$}\setmainfont[Path=/usr/share/fonts/truetype/cmu/,UprightFont=cmunrm.ttf,BoldFont=cmunbx.ttf,ItalicFont=cmunti.ttf,BoldItalicFont=cmunbi.ttf]{cmunrm.ttf}\setmonofont[Path=/usr/share/fonts/truetype/cmu/,UprightFont=cmuntt.ttf,BoldFont=cmuntb.ttf,ItalicFont=cmunit.ttf,BoldItalicFont=cmuntx.ttf]{cmunrm.ttf} in the first case or {\ttfamily \setmainfont[Path=/usr/share/fonts/truetype/cmu/,UprightFont=cmunrm.ttf,BoldFont=cmunbx.ttf,ItalicFont=cmunti.ttf,BoldItalicFont=cmunbi.ttf]{cmuntt.ttf}\setmonofont[Path=/usr/share/fonts/truetype/cmu/,UprightFont=cmuntt.ttf,BoldFont=cmuntb.ttf,ItalicFont=cmunit.ttf,BoldItalicFont=cmuntx.ttf]{cmuntt.ttf}\ttfamily \textbackslash{}usepackage\{hyperref\}}{$\text{ }$}\setmainfont[Path=/usr/share/fonts/truetype/cmu/,UprightFont=cmunrm.ttf,BoldFont=cmunbx.ttf,ItalicFont=cmunti.ttf,BoldItalicFont=cmunbi.ttf]{cmunrm.ttf}\setmonofont[Path=/usr/share/fonts/truetype/cmu/,UprightFont=cmuntt.ttf,BoldFont=cmuntb.ttf,ItalicFont=cmunit.ttf,BoldItalicFont=cmuntx.ttf]{cmunrm.ttf} in the second case.

Styles provided by Natbib (see below) handle this field, other styles can be modified using \myhref{http://purl.org/nxg/dist/urlbst}{urlbst} program. Modifications of three standard styles (plain, abbrv and alpha) are provided with urlbst.

If you need more help about URLs in bibliography, visit \myhref{http://www.tex.ac.uk/cgi-bin/texfaq2html?label=citeURL}{FAQ of UK List of TeX}.
\subsection{Customizing bibliography appearance}
\label{685}

One of the main advantages of BibTeX, especially for people who write many research papers, is the ability to customize your bibliography to suit the requirements of a given publication. You will notice how different publications tend to have their own style of formatting references, to which authors must adhere if they want their manuscripts published. In fact, established journals and conference organizers often will have created their own bibliography style (.bst file) for those users of BibTeX, to do all the hard work for you.

It can achieve this because of the nature of the .bib database, where all the information about your references is stored in a structured format, but nothing about style. This is a common theme in LaTeX in general, where it tries as much as possible to keep content and presentation separate.

A bibliography style file ({\ttfamily \setmainfont[Path=/usr/share/fonts/truetype/cmu/,UprightFont=cmunrm.ttf,BoldFont=cmunbx.ttf,ItalicFont=cmunti.ttf,BoldItalicFont=cmunbi.ttf]{cmuntt.ttf}\setmonofont[Path=/usr/share/fonts/truetype/cmu/,UprightFont=cmuntt.ttf,BoldFont=cmuntb.ttf,ItalicFont=cmunit.ttf,BoldItalicFont=cmuntx.ttf]{cmuntt.ttf}\ttfamily .bst}\setmainfont[Path=/usr/share/fonts/truetype/cmu/,UprightFont=cmunrm.ttf,BoldFont=cmunbx.ttf,ItalicFont=cmunti.ttf,BoldItalicFont=cmunbi.ttf]{cmunrm.ttf}\setmonofont[Path=/usr/share/fonts/truetype/cmu/,UprightFont=cmuntt.ttf,BoldFont=cmuntb.ttf,ItalicFont=cmunit.ttf,BoldItalicFont=cmuntx.ttf]{cmunrm.ttf}) will tell LaTeX how to format each attribute, what order to put them in, what punctuation to use in between particular attributes etc. Unfortunately, creating such a style by hand is not a trivial task. Which is why {\ttfamily \setmainfont[Path=/usr/share/fonts/truetype/cmu/,UprightFont=cmunrm.ttf,BoldFont=cmunbx.ttf,ItalicFont=cmunti.ttf,BoldItalicFont=cmunbi.ttf]{cmuntt.ttf}\setmonofont[Path=/usr/share/fonts/truetype/cmu/,UprightFont=cmuntt.ttf,BoldFont=cmuntb.ttf,ItalicFont=cmunit.ttf,BoldItalicFont=cmuntx.ttf]{cmuntt.ttf}\ttfamily Makebst}{$\text{ }$}\setmainfont[Path=/usr/share/fonts/truetype/cmu/,UprightFont=cmunrm.ttf,BoldFont=cmunbx.ttf,ItalicFont=cmunti.ttf,BoldItalicFont=cmunbi.ttf]{cmunrm.ttf}\setmonofont[Path=/usr/share/fonts/truetype/cmu/,UprightFont=cmuntt.ttf,BoldFont=cmuntb.ttf,ItalicFont=cmunit.ttf,BoldItalicFont=cmuntx.ttf]{cmunrm.ttf} (also known as {\itshape \setmainfont[Path=/usr/share/fonts/truetype/cmu/,UprightFont=cmunrm.ttf,BoldFont=cmunbx.ttf,ItalicFont=cmunti.ttf,BoldItalicFont=cmunbi.ttf]{cmunti.ttf}\setmonofont[Path=/usr/share/fonts/truetype/cmu/,UprightFont=cmuntt.ttf,BoldFont=cmuntb.ttf,ItalicFont=cmunit.ttf,BoldItalicFont=cmuntx.ttf]{cmunti.ttf}\itshape custom-{}bib}\setmainfont[Path=/usr/share/fonts/truetype/cmu/,UprightFont=cmunrm.ttf,BoldFont=cmunbx.ttf,ItalicFont=cmunti.ttf,BoldItalicFont=cmunbi.ttf]{cmunrm.ttf}\setmonofont[Path=/usr/share/fonts/truetype/cmu/,UprightFont=cmuntt.ttf,BoldFont=cmuntb.ttf,ItalicFont=cmunit.ttf,BoldItalicFont=cmuntx.ttf]{cmunrm.ttf}) is the tool we need.

{\ttfamily \setmainfont[Path=/usr/share/fonts/truetype/cmu/,UprightFont=cmunrm.ttf,BoldFont=cmunbx.ttf,ItalicFont=cmunti.ttf,BoldItalicFont=cmunbi.ttf]{cmuntt.ttf}\setmonofont[Path=/usr/share/fonts/truetype/cmu/,UprightFont=cmuntt.ttf,BoldFont=cmuntb.ttf,ItalicFont=cmunit.ttf,BoldItalicFont=cmuntx.ttf]{cmuntt.ttf}\ttfamily Makebst}{$\text{ }$}\setmainfont[Path=/usr/share/fonts/truetype/cmu/,UprightFont=cmunrm.ttf,BoldFont=cmunbx.ttf,ItalicFont=cmunti.ttf,BoldItalicFont=cmunbi.ttf]{cmunrm.ttf}\setmonofont[Path=/usr/share/fonts/truetype/cmu/,UprightFont=cmuntt.ttf,BoldFont=cmuntb.ttf,ItalicFont=cmunit.ttf,BoldItalicFont=cmuntx.ttf]{cmunrm.ttf} can be used to automatically generate a .bst file based on your needs. It is very simple, and actually asks you a series of questions about your preferences. Once complete, it will then output the appropriate style file for you to use.

It should be installed with the LaTeX distribution (otherwise, you can 
\myhref{http://www.mps.mpg.de/software/latex/localtex/localltx.html\#makebst}{download it}) and it\textquotesingle{}s very simple to initiate. At the command line, type:
\\

\TemplateSpaceIndent{$\text{ }${}latex$\text{ }${}makebst}


LaTeX will find the relevant file and the questioning process will begin. You will have to answer quite a few (although, note that the default answers are pretty sensible), which means it would be impractical to go through an example in this tutorial. However, it is fairly straight-{}forward. And if you require further guidance, then there is a comprehensive \myhref{http://www.mps.mpg.de/software/latex/localtex/doc/merlin.pdf}{manual} available. I\textquotesingle{}d recommend experimenting with it and seeing what the results are when applied to a LaTeX document.

If you are using a custom built .bst file, it is important that LaTeX can find it! So, make sure it\textquotesingle{}s in the same directory as the LaTeX source file, {\itshape \setmainfont[Path=/usr/share/fonts/truetype/cmu/,UprightFont=cmunrm.ttf,BoldFont=cmunbx.ttf,ItalicFont=cmunti.ttf,BoldItalicFont=cmunbi.ttf]{cmunti.ttf}\setmonofont[Path=/usr/share/fonts/truetype/cmu/,UprightFont=cmuntt.ttf,BoldFont=cmuntb.ttf,ItalicFont=cmunit.ttf,BoldItalicFont=cmuntx.ttf]{cmunti.ttf}\itshape unless}{$\text{ }$}\setmainfont[Path=/usr/share/fonts/truetype/cmu/,UprightFont=cmunrm.ttf,BoldFont=cmunbx.ttf,ItalicFont=cmunti.ttf,BoldItalicFont=cmunbi.ttf]{cmunrm.ttf}\setmonofont[Path=/usr/share/fonts/truetype/cmu/,UprightFont=cmuntt.ttf,BoldFont=cmuntb.ttf,ItalicFont=cmunit.ttf,BoldItalicFont=cmuntx.ttf]{cmunrm.ttf} you are using one of the standard style files (such as {\itshape \setmainfont[Path=/usr/share/fonts/truetype/cmu/,UprightFont=cmunrm.ttf,BoldFont=cmunbx.ttf,ItalicFont=cmunti.ttf,BoldItalicFont=cmunbi.ttf]{cmunti.ttf}\setmonofont[Path=/usr/share/fonts/truetype/cmu/,UprightFont=cmuntt.ttf,BoldFont=cmuntb.ttf,ItalicFont=cmunit.ttf,BoldItalicFont=cmuntx.ttf]{cmunti.ttf}\itshape plain}{$\text{ }$}\setmainfont[Path=/usr/share/fonts/truetype/cmu/,UprightFont=cmunrm.ttf,BoldFont=cmunbx.ttf,ItalicFont=cmunti.ttf,BoldItalicFont=cmunbi.ttf]{cmunrm.ttf}\setmonofont[Path=/usr/share/fonts/truetype/cmu/,UprightFont=cmuntt.ttf,BoldFont=cmuntb.ttf,ItalicFont=cmunit.ttf,BoldItalicFont=cmuntx.ttf]{cmunrm.ttf} or {\itshape \setmainfont[Path=/usr/share/fonts/truetype/cmu/,UprightFont=cmunrm.ttf,BoldFont=cmunbx.ttf,ItalicFont=cmunti.ttf,BoldItalicFont=cmunbi.ttf]{cmunti.ttf}\setmonofont[Path=/usr/share/fonts/truetype/cmu/,UprightFont=cmuntt.ttf,BoldFont=cmuntb.ttf,ItalicFont=cmunit.ttf,BoldItalicFont=cmuntx.ttf]{cmunti.ttf}\itshape plainnat}\setmainfont[Path=/usr/share/fonts/truetype/cmu/,UprightFont=cmunrm.ttf,BoldFont=cmunbx.ttf,ItalicFont=cmunti.ttf,BoldItalicFont=cmunbi.ttf]{cmunrm.ttf}\setmonofont[Path=/usr/share/fonts/truetype/cmu/,UprightFont=cmuntt.ttf,BoldFont=cmuntb.ttf,ItalicFont=cmunit.ttf,BoldItalicFont=cmuntx.ttf]{cmunrm.ttf}, that come bundled with LaTeX -{} these will be automatically found in the directories that they are installed. Also, make sure the name of the {\ttfamily \setmainfont[Path=/usr/share/fonts/truetype/cmu/,UprightFont=cmunrm.ttf,BoldFont=cmunbx.ttf,ItalicFont=cmunti.ttf,BoldItalicFont=cmunbi.ttf]{cmuntt.ttf}\setmonofont[Path=/usr/share/fonts/truetype/cmu/,UprightFont=cmuntt.ttf,BoldFont=cmuntb.ttf,ItalicFont=cmunit.ttf,BoldItalicFont=cmuntx.ttf]{cmuntt.ttf}\ttfamily .bst}{$\text{ }$}\setmainfont[Path=/usr/share/fonts/truetype/cmu/,UprightFont=cmunrm.ttf,BoldFont=cmunbx.ttf,ItalicFont=cmunti.ttf,BoldItalicFont=cmunbi.ttf]{cmunrm.ttf}\setmonofont[Path=/usr/share/fonts/truetype/cmu/,UprightFont=cmuntt.ttf,BoldFont=cmuntb.ttf,ItalicFont=cmunit.ttf,BoldItalicFont=cmuntx.ttf]{cmunrm.ttf} file you want to use is reflected in the {\ttfamily \setmainfont[Path=/usr/share/fonts/truetype/cmu/,UprightFont=cmunrm.ttf,BoldFont=cmunbx.ttf,ItalicFont=cmunti.ttf,BoldItalicFont=cmunbi.ttf]{cmuntt.ttf}\setmonofont[Path=/usr/share/fonts/truetype/cmu/,UprightFont=cmuntt.ttf,BoldFont=cmuntb.ttf,ItalicFont=cmunit.ttf,BoldItalicFont=cmuntx.ttf]{cmuntt.ttf}\ttfamily \textbackslash{}bibliographystyle\{style\}}{$\text{ }$}\setmainfont[Path=/usr/share/fonts/truetype/cmu/,UprightFont=cmunrm.ttf,BoldFont=cmunbx.ttf,ItalicFont=cmunti.ttf,BoldItalicFont=cmunbi.ttf]{cmunrm.ttf}\setmonofont[Path=/usr/share/fonts/truetype/cmu/,UprightFont=cmuntt.ttf,BoldFont=cmuntb.ttf,ItalicFont=cmunit.ttf,BoldItalicFont=cmuntx.ttf]{cmunrm.ttf} command (but don\textquotesingle{}t include the {\ttfamily \setmainfont[Path=/usr/share/fonts/truetype/cmu/,UprightFont=cmunrm.ttf,BoldFont=cmunbx.ttf,ItalicFont=cmunti.ttf,BoldItalicFont=cmunbi.ttf]{cmuntt.ttf}\setmonofont[Path=/usr/share/fonts/truetype/cmu/,UprightFont=cmuntt.ttf,BoldFont=cmuntb.ttf,ItalicFont=cmunit.ttf,BoldItalicFont=cmuntx.ttf]{cmuntt.ttf}\ttfamily .bst}{$\text{ }$}\setmainfont[Path=/usr/share/fonts/truetype/cmu/,UprightFont=cmunrm.ttf,BoldFont=cmunbx.ttf,ItalicFont=cmunti.ttf,BoldItalicFont=cmunbi.ttf]{cmunrm.ttf}\setmonofont[Path=/usr/share/fonts/truetype/cmu/,UprightFont=cmuntt.ttf,BoldFont=cmuntb.ttf,ItalicFont=cmunit.ttf,BoldItalicFont=cmuntx.ttf]{cmunrm.ttf} extension!).
\subsection{Localizing bibliography appearance}
\label{686}

When writing documents in languages other than English, you may find it desirable to adapt the appearance of your bibliography to the document language. This concerns words such as {\itshape \setmainfont[Path=/usr/share/fonts/truetype/cmu/,UprightFont=cmunrm.ttf,BoldFont=cmunbx.ttf,ItalicFont=cmunti.ttf,BoldItalicFont=cmunbi.ttf]{cmunti.ttf}\setmonofont[Path=/usr/share/fonts/truetype/cmu/,UprightFont=cmuntt.ttf,BoldFont=cmuntb.ttf,ItalicFont=cmunit.ttf,BoldItalicFont=cmuntx.ttf]{cmunti.ttf}\itshape editors}\setmainfont[Path=/usr/share/fonts/truetype/cmu/,UprightFont=cmunrm.ttf,BoldFont=cmunbx.ttf,ItalicFont=cmunti.ttf,BoldItalicFont=cmunbi.ttf]{cmunrm.ttf}\setmonofont[Path=/usr/share/fonts/truetype/cmu/,UprightFont=cmuntt.ttf,BoldFont=cmuntb.ttf,ItalicFont=cmunit.ttf,BoldItalicFont=cmuntx.ttf]{cmunrm.ttf}, {\itshape \setmainfont[Path=/usr/share/fonts/truetype/cmu/,UprightFont=cmunrm.ttf,BoldFont=cmunbx.ttf,ItalicFont=cmunti.ttf,BoldItalicFont=cmunbi.ttf]{cmunti.ttf}\setmonofont[Path=/usr/share/fonts/truetype/cmu/,UprightFont=cmuntt.ttf,BoldFont=cmuntb.ttf,ItalicFont=cmunit.ttf,BoldItalicFont=cmuntx.ttf]{cmunti.ttf}\itshape and}\setmainfont[Path=/usr/share/fonts/truetype/cmu/,UprightFont=cmunrm.ttf,BoldFont=cmunbx.ttf,ItalicFont=cmunti.ttf,BoldItalicFont=cmunbi.ttf]{cmunrm.ttf}\setmonofont[Path=/usr/share/fonts/truetype/cmu/,UprightFont=cmuntt.ttf,BoldFont=cmuntb.ttf,ItalicFont=cmunit.ttf,BoldItalicFont=cmuntx.ttf]{cmunrm.ttf}, or {\itshape \setmainfont[Path=/usr/share/fonts/truetype/cmu/,UprightFont=cmunrm.ttf,BoldFont=cmunbx.ttf,ItalicFont=cmunti.ttf,BoldItalicFont=cmunbi.ttf]{cmunti.ttf}\setmonofont[Path=/usr/share/fonts/truetype/cmu/,UprightFont=cmuntt.ttf,BoldFont=cmuntb.ttf,ItalicFont=cmunit.ttf,BoldItalicFont=cmuntx.ttf]{cmunti.ttf}\itshape in}{$\text{ }$}\setmainfont[Path=/usr/share/fonts/truetype/cmu/,UprightFont=cmunrm.ttf,BoldFont=cmunbx.ttf,ItalicFont=cmunti.ttf,BoldItalicFont=cmunbi.ttf]{cmunrm.ttf}\setmonofont[Path=/usr/share/fonts/truetype/cmu/,UprightFont=cmuntt.ttf,BoldFont=cmuntb.ttf,ItalicFont=cmunit.ttf,BoldItalicFont=cmuntx.ttf]{cmunrm.ttf} as well as a proper typographic layout. The \myhref{http://tug.ctan.org/tex-archive/biblio/bibtex/contrib/babelbib/}{{\ttfamily \setmainfont[Path=/usr/share/fonts/truetype/cmu/,UprightFont=cmunrm.ttf,BoldFont=cmunbx.ttf,ItalicFont=cmunti.ttf,BoldItalicFont=cmunbi.ttf]{cmuntt.ttf}\setmonofont[Path=/usr/share/fonts/truetype/cmu/,UprightFont=cmuntt.ttf,BoldFont=cmuntb.ttf,ItalicFont=cmunit.ttf,BoldItalicFont=cmuntx.ttf]{cmuntt.ttf}\ttfamily babelbib}{$\text{ }$}\setmainfont[Path=/usr/share/fonts/truetype/cmu/,UprightFont=cmunrm.ttf,BoldFont=cmunbx.ttf,ItalicFont=cmunti.ttf,BoldItalicFont=cmunbi.ttf]{cmunrm.ttf}\setmonofont[Path=/usr/share/fonts/truetype/cmu/,UprightFont=cmuntt.ttf,BoldFont=cmuntb.ttf,ItalicFont=cmunit.ttf,BoldItalicFont=cmuntx.ttf]{cmunrm.ttf} package} can be used here. For example, to layout the bibliography in German, add the following to the header:


\begin{Shaded}
\begin{Highlighting}[]

\NormalTok{\textbackslash{}usepackage[fixlanguage]\{babelbib\}}\newline
\NormalTok{\textbackslash{}selectbiblanguage\{german\}}\newline
\end{Highlighting}
\end{Shaded}


Alternatively, you can layout each bibliography entry according to the language of the cited document:


\begin{Shaded}
\begin{Highlighting}[]

\NormalTok{\textbackslash{}usepackage\{babelbib\}}\newline
\end{Highlighting}
\end{Shaded}


The language of an entry is specified as an additional field in the BibTeX entry:


\begin{Shaded}
\begin{Highlighting}[]

\KeywordTok{@article}\NormalTok{\{}\OtherTok{mueller08}\NormalTok{,}\newline
\ensuremath{\text{ }}\ensuremath{\text{ }}\AlertTok{\%}\ensuremath{\text{ }}\AlertTok{...}\newline
\ensuremath{\text{ }}\ensuremath{\text{ }}\DataTypeTok{language}\ensuremath{\text{ }}\NormalTok{=\ensuremath{\text{ }}\{german\}}\newline
\NormalTok{\}}\newline
\end{Highlighting}
\end{Shaded}


For {\ttfamily \setmainfont[Path=/usr/share/fonts/truetype/cmu/,UprightFont=cmunrm.ttf,BoldFont=cmunbx.ttf,ItalicFont=cmunti.ttf,BoldItalicFont=cmunbi.ttf]{cmuntt.ttf}\setmonofont[Path=/usr/share/fonts/truetype/cmu/,UprightFont=cmuntt.ttf,BoldFont=cmuntb.ttf,ItalicFont=cmunit.ttf,BoldItalicFont=cmuntx.ttf]{cmuntt.ttf}\ttfamily babelbib}{$\text{ }$}\setmainfont[Path=/usr/share/fonts/truetype/cmu/,UprightFont=cmunrm.ttf,BoldFont=cmunbx.ttf,ItalicFont=cmunti.ttf,BoldItalicFont=cmunbi.ttf]{cmunrm.ttf}\setmonofont[Path=/usr/share/fonts/truetype/cmu/,UprightFont=cmuntt.ttf,BoldFont=cmuntb.ttf,ItalicFont=cmunit.ttf,BoldItalicFont=cmuntx.ttf]{cmunrm.ttf} to take effect, a bibliography style supported by it -{} one of {\ttfamily \setmainfont[Path=/usr/share/fonts/truetype/cmu/,UprightFont=cmunrm.ttf,BoldFont=cmunbx.ttf,ItalicFont=cmunti.ttf,BoldItalicFont=cmunbi.ttf]{cmuntt.ttf}\setmonofont[Path=/usr/share/fonts/truetype/cmu/,UprightFont=cmuntt.ttf,BoldFont=cmuntb.ttf,ItalicFont=cmunit.ttf,BoldItalicFont=cmuntx.ttf]{cmuntt.ttf}\ttfamily babplain}\setmainfont[Path=/usr/share/fonts/truetype/cmu/,UprightFont=cmunrm.ttf,BoldFont=cmunbx.ttf,ItalicFont=cmunti.ttf,BoldItalicFont=cmunbi.ttf]{cmunrm.ttf}\setmonofont[Path=/usr/share/fonts/truetype/cmu/,UprightFont=cmuntt.ttf,BoldFont=cmuntb.ttf,ItalicFont=cmunit.ttf,BoldItalicFont=cmuntx.ttf]{cmunrm.ttf}, {\ttfamily \setmainfont[Path=/usr/share/fonts/truetype/cmu/,UprightFont=cmunrm.ttf,BoldFont=cmunbx.ttf,ItalicFont=cmunti.ttf,BoldItalicFont=cmunbi.ttf]{cmuntt.ttf}\setmonofont[Path=/usr/share/fonts/truetype/cmu/,UprightFont=cmuntt.ttf,BoldFont=cmuntb.ttf,ItalicFont=cmunit.ttf,BoldItalicFont=cmuntx.ttf]{cmuntt.ttf}\ttfamily babplai3}\setmainfont[Path=/usr/share/fonts/truetype/cmu/,UprightFont=cmunrm.ttf,BoldFont=cmunbx.ttf,ItalicFont=cmunti.ttf,BoldItalicFont=cmunbi.ttf]{cmunrm.ttf}\setmonofont[Path=/usr/share/fonts/truetype/cmu/,UprightFont=cmuntt.ttf,BoldFont=cmuntb.ttf,ItalicFont=cmunit.ttf,BoldItalicFont=cmuntx.ttf]{cmunrm.ttf}, {\ttfamily \setmainfont[Path=/usr/share/fonts/truetype/cmu/,UprightFont=cmunrm.ttf,BoldFont=cmunbx.ttf,ItalicFont=cmunti.ttf,BoldItalicFont=cmunbi.ttf]{cmuntt.ttf}\setmonofont[Path=/usr/share/fonts/truetype/cmu/,UprightFont=cmuntt.ttf,BoldFont=cmuntb.ttf,ItalicFont=cmunit.ttf,BoldItalicFont=cmuntx.ttf]{cmuntt.ttf}\ttfamily babalpha}\setmainfont[Path=/usr/share/fonts/truetype/cmu/,UprightFont=cmunrm.ttf,BoldFont=cmunbx.ttf,ItalicFont=cmunti.ttf,BoldItalicFont=cmunbi.ttf]{cmunrm.ttf}\setmonofont[Path=/usr/share/fonts/truetype/cmu/,UprightFont=cmuntt.ttf,BoldFont=cmuntb.ttf,ItalicFont=cmunit.ttf,BoldItalicFont=cmuntx.ttf]{cmunrm.ttf}, {\ttfamily \setmainfont[Path=/usr/share/fonts/truetype/cmu/,UprightFont=cmunrm.ttf,BoldFont=cmunbx.ttf,ItalicFont=cmunti.ttf,BoldItalicFont=cmunbi.ttf]{cmuntt.ttf}\setmonofont[Path=/usr/share/fonts/truetype/cmu/,UprightFont=cmuntt.ttf,BoldFont=cmuntb.ttf,ItalicFont=cmunit.ttf,BoldItalicFont=cmuntx.ttf]{cmuntt.ttf}\ttfamily babunsrt}\setmainfont[Path=/usr/share/fonts/truetype/cmu/,UprightFont=cmunrm.ttf,BoldFont=cmunbx.ttf,ItalicFont=cmunti.ttf,BoldItalicFont=cmunbi.ttf]{cmunrm.ttf}\setmonofont[Path=/usr/share/fonts/truetype/cmu/,UprightFont=cmuntt.ttf,BoldFont=cmuntb.ttf,ItalicFont=cmunit.ttf,BoldItalicFont=cmuntx.ttf]{cmunrm.ttf}, {\ttfamily \setmainfont[Path=/usr/share/fonts/truetype/cmu/,UprightFont=cmunrm.ttf,BoldFont=cmunbx.ttf,ItalicFont=cmunti.ttf,BoldItalicFont=cmunbi.ttf]{cmuntt.ttf}\setmonofont[Path=/usr/share/fonts/truetype/cmu/,UprightFont=cmuntt.ttf,BoldFont=cmuntb.ttf,ItalicFont=cmunit.ttf,BoldItalicFont=cmuntx.ttf]{cmuntt.ttf}\ttfamily bababbrv}\setmainfont[Path=/usr/share/fonts/truetype/cmu/,UprightFont=cmunrm.ttf,BoldFont=cmunbx.ttf,ItalicFont=cmunti.ttf,BoldItalicFont=cmunbi.ttf]{cmunrm.ttf}\setmonofont[Path=/usr/share/fonts/truetype/cmu/,UprightFont=cmuntt.ttf,BoldFont=cmuntb.ttf,ItalicFont=cmunit.ttf,BoldItalicFont=cmuntx.ttf]{cmunrm.ttf}, and {\ttfamily \setmainfont[Path=/usr/share/fonts/truetype/cmu/,UprightFont=cmunrm.ttf,BoldFont=cmunbx.ttf,ItalicFont=cmunti.ttf,BoldItalicFont=cmunbi.ttf]{cmuntt.ttf}\setmonofont[Path=/usr/share/fonts/truetype/cmu/,UprightFont=cmuntt.ttf,BoldFont=cmuntb.ttf,ItalicFont=cmunit.ttf,BoldItalicFont=cmuntx.ttf]{cmuntt.ttf}\ttfamily bababbr3}{$\text{ }$}\setmainfont[Path=/usr/share/fonts/truetype/cmu/,UprightFont=cmunrm.ttf,BoldFont=cmunbx.ttf,ItalicFont=cmunti.ttf,BoldItalicFont=cmunbi.ttf]{cmunrm.ttf}\setmonofont[Path=/usr/share/fonts/truetype/cmu/,UprightFont=cmuntt.ttf,BoldFont=cmuntb.ttf,ItalicFont=cmunit.ttf,BoldItalicFont=cmuntx.ttf]{cmunrm.ttf} -{} must be used:


\begin{Shaded}
\begin{Highlighting}[]

\NormalTok{\textbackslash{}bibliographystyle\{babplain\}}\newline
\NormalTok{\textbackslash{}bibliography\{sample\}}\newline
\end{Highlighting}
\end{Shaded}

\subsection{Showing unused items}
\label{687}

Usually LaTeX only displays the entries which are referred to with 
\begin{Shaded}
\begin{Highlighting}[]

\NormalTok{\textbackslash{}cite}\newline
\end{Highlighting}
\end{Shaded}
. It\textquotesingle{}s possible to make uncited entries visible:


\begin{Shaded}
\begin{Highlighting}[]

\NormalTok{\textbackslash{}nocite\{Name89\}\ensuremath{\text{ }}}\CommentTok{\%\ensuremath{\text{ }}Show\ensuremath{\text{ }}Bibliography\ensuremath{\text{ }}entry\ensuremath{\text{ }}of\ensuremath{\text{ }}Name89}\newline
\NormalTok{\textbackslash{}nocite\{*\}\ensuremath{\text{ }}}\CommentTok{\%\ensuremath{\text{ }}Show\ensuremath{\text{ }}all\ensuremath{\text{ }}Bib-entries}\newline
\end{Highlighting}
\end{Shaded}

\subsection{Getting bibliographic data}
\label{688}

Many online databases provide bibliographic data in BibTeX-{}Format, making it easy to build your own database.  
For example, \myhref{http://scholar.google.com}{Google Scholar} offers the option to return properly formatted output, which can also be turned on in the settings page.

One should be alert to the fact that bibliographic databases are frequently the product of several generations of automatic processing, and so the resulting BibTex code is prone to a variety of minor errors, especially in older entries.
\subsection{Helpful tools}
\label{689}
See also: \myhref{https://en.wikipedia.org/wiki/Comparison\%20of\%20reference\%20management\%20software}{w:en:Comparison of reference management software}


\begin{minipage}{1.0\linewidth}
\begin{center}
\includegraphics[width=1.0\linewidth,height=6.5in,keepaspectratio]{../images/151.jpg}
\end{center}
\raggedright{}\myfigurewithcaption{151}{Literatur-{}Generator}
\end{minipage}\vspace{0.75cm}




\begin{minipage}{1.0\linewidth}
\begin{center}
\includegraphics[width=1.0\linewidth,height=6.5in,keepaspectratio]{../images/152.png}
\end{center}
\raggedright{}\myfigurewithcaption{152}{JabRef}
\end{minipage}\vspace{0.75cm}




\begin{minipage}{1.0\linewidth}
\begin{center}
\includegraphics[width=1.0\linewidth,height=6.5in,keepaspectratio]{../images/153.png}
\end{center}
\raggedright{}\myfigurewithcaption{153}{BibDesk}
\end{minipage}\vspace{0.75cm}



\begin{myitemize}
\item{}  \myhref{http://bibdesk.sourceforge.net/}{BibDesk} BibDesk is a bibliographic reference manager for Mac OS X. It features a very usable user interface and provides a number of features like smart folders based on keywords and live tex display.
\item{}  \myhref{http://www.bibsonomy.org/}{BibSonomy} — A free social bookmark and publication management system based on BibTeX.
\item{}  \myhref{http://www.bibtexsearch.com/}{BibTeXSearch} BibTeXSearch is a free searchable BibTeX database spanning millions of academic records.
\item{}  \myhref{http://truben.no/latex/bibtex}{Bibtex Editor} -{} An online BibTeX entry generator and bibliography management system. Possible to import and export Bibtex files.
\item{} \myhref{http://www.mediawiki.org/wiki/Extension:Bibwiki}{Bibwiki} Bibwiki is a Specialpage for MediaWiki to manage BibTeX bibliographies. It offers a straightforward way to import and export bibliographic records.
\item{} \myhref{http://www.molspaces.com/cb2bib/}{cb2Bib} The cb2Bib is a tool for rapidly extracting unformatted, or unstandardized bibliographic references from email alerts, journal Web pages, and PDF files.
\item{}  \myhref{http://www.citavi.ch}{Citavi} Commercial software (with size-{}limited free demo version) which even searches libraries for citations and keeps all your knowledge in a database. Export of the database to all kinds of formats is possible. Works together with MS Word and Open Office Writer. Moreover plug ins for browsers and Acrobat Reader exist to automatically include references to your project. 
\item{}  \myhref{http://www.citeulike.org/}{CiteULike} CiteULike is a free online service to organise academic papers. It can export citations in BibTeX format, and can \symbol{34}scrape\symbol{34} BibTeX data from many popular websites.
\item{}  \myhref{http://stat.genopole.cnrs.fr/~cambroise/doku.php?id=softwares:dokuwikibibtexplugin}{DokuWiki} Bibtex is a DokuWiki plugin that allows for the inclusion of bibtex formatted citations in DokuWiki pages and displays them in APA format. Note: This Plugins is vulnerable to an XSS attack -{}>{} \myplainurl{http://www.dokuwiki.org/plugin:bibtex}
\item{}  \myhref{http://ebib.sourceforge.net/}{Ebib} {\mbox{$\text{---}$}} a BibTeX database manager for \myhref{https://en.wikipedia.org/wiki/Emacs}{Emacs}, well resolved and never more than a few keystrokes away.
\item{}  \myhref{http://www.jabref.org/}{JabRef} is a Java program (under the GPL license) which lets you search many bibliographic databases such as Medline, Citeseer, IEEEXplore and arXiv and feed and manage your BibTeX local databases with your selected articles. Based on BiBTeX, JabRef can export in many other output formats such as html, MS Word or EndNote.
\item{} \myhref{http://users.tpg.com.au/thachly/kbib/}{KBib} Another BibTeX editor for KDE. It has similar capabilities, and slightly different UI. Features include BibTeX reference generation from PDF files, plain text, DOI, arXiv \& PubMed IDs. Web queries to Google Scholar, PubMer, arXiv and a number of other services are also supported. 
\item{} \myhref{http://home.gna.org/kbibtex/}{KBibTeX} KBibTeX is a BibTeX editor for KDE to edit bibliographies used with LaTeX. Features include comfortable input masks, starting web queries (e. g. Google or PubMed) and exporting to PDF, PostScript, RTF and XML/HTML. As KBibTeX is using KDE\textquotesingle{}s KParts technology, KBibTeX can be embedded into Kile or Konqueror.
\item{}  \myhref{http://literatur-generator.de/}{Literatur-{}Generator} is a German-{}language online tool for creating a bibliography (Bibtex, Endnote, Din 1505, ...).
\item{}  \myhref{http://mendeley.com}{Mendeley} Mendeley is cost-{}free academic software for managing PDFs which can manage a bibliography in Open Office and read BibTeX.
\item{}  \myhref{http://www.qiqqa.com/}{Qiqqa} Qiqqa is a free research manager that has built-{}in support for automatically associating BibTeX records with your PDFs and a \textquotesingle{}BibTeX Sniffer\textquotesingle{} for helping you semi-{}automatically find BibTeX records.
\item{}  \myhref{http://icculus.org/referencer/index.html}{Referencer} Referencer is a Gnome application to organise documents or references, and ultimately generate a BibTeX bibliography file.
\item{}  \myhref{http://www.verzetteln.de/synapsen/}{Synapsen} —  Hypertextual Card Index / Reference Manager with special support for BiBTeX / biblatex, written in Java.
\item{}  \myhref{http://www.zotero.org/}{Zotero} Zotero is a free and open reference manager working as a Firefox plugin or standalone application, capable of importing and exporting bib files.
\end{myitemize}

\subsection{Summary}
\label{690}

Although it can take a little time to get to grips with BibTeX, in the long term, it\textquotesingle{}s an efficient way to handle your references. It\textquotesingle{}s not uncommon to find .bib files on websites that people compile as a list of their own publications, or a survey of relevant works within a given topic, etc. Or in those huge, online bibliography databases, you often find BibTeX versions of publications, so it\textquotesingle{}s a quick cut-{}and-{}paste into your own .bib file, and then no more hassle!

Having all your references in one place can be a big advantage. And having them in a structured form, that allows customizable output is another one. There are a variety of free utilities that can load your .bib files, and allow you to view them in a more efficient manner, as well as sort them and check for errors.
\section{Bibliography in the table of contents}
\label{691}

If you are writing a {\itshape \setmainfont[Path=/usr/share/fonts/truetype/cmu/,UprightFont=cmunrm.ttf,BoldFont=cmunbx.ttf,ItalicFont=cmunti.ttf,BoldItalicFont=cmunbi.ttf]{cmunti.ttf}\setmonofont[Path=/usr/share/fonts/truetype/cmu/,UprightFont=cmuntt.ttf,BoldFont=cmuntb.ttf,ItalicFont=cmunit.ttf,BoldItalicFont=cmuntx.ttf]{cmunti.ttf}\itshape book}{$\text{ }$}\setmainfont[Path=/usr/share/fonts/truetype/cmu/,UprightFont=cmunrm.ttf,BoldFont=cmunbx.ttf,ItalicFont=cmunti.ttf,BoldItalicFont=cmunbi.ttf]{cmunrm.ttf}\setmonofont[Path=/usr/share/fonts/truetype/cmu/,UprightFont=cmuntt.ttf,BoldFont=cmuntb.ttf,ItalicFont=cmunit.ttf,BoldItalicFont=cmuntx.ttf]{cmunrm.ttf} or {\itshape \setmainfont[Path=/usr/share/fonts/truetype/cmu/,UprightFont=cmunrm.ttf,BoldFont=cmunbx.ttf,ItalicFont=cmunti.ttf,BoldItalicFont=cmunbi.ttf]{cmunti.ttf}\setmonofont[Path=/usr/share/fonts/truetype/cmu/,UprightFont=cmuntt.ttf,BoldFont=cmuntb.ttf,ItalicFont=cmunit.ttf,BoldItalicFont=cmuntx.ttf]{cmunti.ttf}\itshape report}\setmainfont[Path=/usr/share/fonts/truetype/cmu/,UprightFont=cmunrm.ttf,BoldFont=cmunbx.ttf,ItalicFont=cmunti.ttf,BoldItalicFont=cmunbi.ttf]{cmunrm.ttf}\setmonofont[Path=/usr/share/fonts/truetype/cmu/,UprightFont=cmuntt.ttf,BoldFont=cmuntb.ttf,ItalicFont=cmunit.ttf,BoldItalicFont=cmuntx.ttf]{cmunrm.ttf}, you\textquotesingle{}ll likely insert your bibliography using something like:


\begin{Shaded}
\begin{Highlighting}[]

\NormalTok{\textbackslash{}begin\{thebibliography\}\{99\}}\newline
\NormalTok{\textbackslash{}bibitem\{bib:one_book\}\ensuremath{\text{ }}some\ensuremath{\text{ }}information}\newline
\NormalTok{\textbackslash{}bibitem\{bib:one_article\}\ensuremath{\text{ }}other\ensuremath{\text{ }}information}\newline
\NormalTok{\textbackslash{}end\{thebibliography\}}\newline
\end{Highlighting}
\end{Shaded}


Or, if you are using BibTeX, your references will be saved in a .bib file, and your TeX document will include the bibliography by these commands:


\begin{Shaded}
\begin{Highlighting}[]

\NormalTok{\textbackslash{}bibliographystyle\{plain\}}\newline
\NormalTok{\textbackslash{}bibliography\{mybibtexfile\}}\newline
\end{Highlighting}
\end{Shaded}


Both of these examples will create a chapter-{}like (or section-{}like) output showing all your references. But even though the resulting “References” looks like a chapter or section, it will not be handled quite the same: it will not appear in the Table of Contents.
\subsection{Using tocbibind}
\label{692}
The most comfortable way of adding your bibliography to the table of contents is to use the dedicated package {\bfseries \setmainfont[Path=/usr/share/fonts/truetype/cmu/,UprightFont=cmunrm.ttf,BoldFont=cmunbx.ttf,ItalicFont=cmunti.ttf,BoldItalicFont=cmunbi.ttf]{cmunbx.ttf}\setmonofont[Path=/usr/share/fonts/truetype/cmu/,UprightFont=cmuntt.ttf,BoldFont=cmuntb.ttf,ItalicFont=cmunit.ttf,BoldItalicFont=cmuntx.ttf]{cmunbx.ttf}\bfseries tocbibind}{$\text{ }$}\setmainfont[Path=/usr/share/fonts/truetype/cmu/,UprightFont=cmunrm.ttf,BoldFont=cmunbx.ttf,ItalicFont=cmunti.ttf,BoldItalicFont=cmunbi.ttf]{cmunrm.ttf}\setmonofont[Path=/usr/share/fonts/truetype/cmu/,UprightFont=cmuntt.ttf,BoldFont=cmuntb.ttf,ItalicFont=cmunit.ttf,BoldItalicFont=cmuntx.ttf]{cmunrm.ttf} that works with many standard document classes. Simply include this code in the preamble of your document:


\begin{Shaded}
\begin{Highlighting}[]

\NormalTok{\textbackslash{}usepackage[nottoc]\{tocbibind\}}\newline
\end{Highlighting}
\end{Shaded}


This will include the Bibliography in the Table of Contents without numbering. If you want to have proper numbering, include the following code in the preamble:


\begin{Shaded}
\begin{Highlighting}[]

\NormalTok{\textbackslash{}usepackage[nottoc,numbib]\{tocbibind\}}\newline
\end{Highlighting}
\end{Shaded}


The {\bfseries \setmainfont[Path=/usr/share/fonts/truetype/cmu/,UprightFont=cmunrm.ttf,BoldFont=cmunbx.ttf,ItalicFont=cmunti.ttf,BoldItalicFont=cmunbi.ttf]{cmunbx.ttf}\setmonofont[Path=/usr/share/fonts/truetype/cmu/,UprightFont=cmuntt.ttf,BoldFont=cmuntb.ttf,ItalicFont=cmunit.ttf,BoldItalicFont=cmuntx.ttf]{cmunbx.ttf}\bfseries tocbibind}{$\text{ }$}\setmainfont[Path=/usr/share/fonts/truetype/cmu/,UprightFont=cmunrm.ttf,BoldFont=cmunbx.ttf,ItalicFont=cmunti.ttf,BoldItalicFont=cmunbi.ttf]{cmunrm.ttf}\setmonofont[Path=/usr/share/fonts/truetype/cmu/,UprightFont=cmuntt.ttf,BoldFont=cmuntb.ttf,ItalicFont=cmunit.ttf,BoldItalicFont=cmuntx.ttf]{cmunrm.ttf} package can also handle including the List of Figures, List of Tables and the Table of Contents itself in the Table of Contents. It has many options for numbering, document structure etc. to fit almost any scenario. See \myhref{http://www.ctan.org/tex-archive/macros/latex/contrib/tocbibind}{the {\bfseries \setmainfont[Path=/usr/share/fonts/truetype/cmu/,UprightFont=cmunrm.ttf,BoldFont=cmunbx.ttf,ItalicFont=cmunti.ttf,BoldItalicFont=cmunbi.ttf]{cmunbx.ttf}\setmonofont[Path=/usr/share/fonts/truetype/cmu/,UprightFont=cmuntt.ttf,BoldFont=cmuntb.ttf,ItalicFont=cmunit.ttf,BoldItalicFont=cmuntx.ttf]{cmunbx.ttf}\bfseries tocbibind}{$\text{ }$}\setmainfont[Path=/usr/share/fonts/truetype/cmu/,UprightFont=cmunrm.ttf,BoldFont=cmunbx.ttf,ItalicFont=cmunti.ttf,BoldItalicFont=cmunbi.ttf]{cmunrm.ttf}\setmonofont[Path=/usr/share/fonts/truetype/cmu/,UprightFont=cmuntt.ttf,BoldFont=cmuntb.ttf,ItalicFont=cmunit.ttf,BoldItalicFont=cmuntx.ttf]{cmunrm.ttf} CTAN page} for detailed documentation.
\subsection{Other methods}
\label{693}\subsubsection{As unnumbered item}
\label{694}
If you want your bibliography to be in the table of contents, just add the following two lines just before the {\itshape \setmainfont[Path=/usr/share/fonts/truetype/cmu/,UprightFont=cmunrm.ttf,BoldFont=cmunbx.ttf,ItalicFont=cmunti.ttf,BoldItalicFont=cmunbi.ttf]{cmunti.ttf}\setmonofont[Path=/usr/share/fonts/truetype/cmu/,UprightFont=cmuntt.ttf,BoldFont=cmuntb.ttf,ItalicFont=cmunit.ttf,BoldItalicFont=cmuntx.ttf]{cmunti.ttf}\itshape thebibliography}{$\text{ }$}\setmainfont[Path=/usr/share/fonts/truetype/cmu/,UprightFont=cmunrm.ttf,BoldFont=cmunbx.ttf,ItalicFont=cmunti.ttf,BoldItalicFont=cmunbi.ttf]{cmunrm.ttf}\setmonofont[Path=/usr/share/fonts/truetype/cmu/,UprightFont=cmuntt.ttf,BoldFont=cmuntb.ttf,ItalicFont=cmunit.ttf,BoldItalicFont=cmuntx.ttf]{cmunrm.ttf} environment:


\begin{Shaded}
\begin{Highlighting}[]

\NormalTok{\textbackslash{}clearpage}\CommentTok{\%\ensuremath{\text{ }}or\ensuremath{\text{ }}cleardoublepage}\newline
\NormalTok{\textbackslash{}addcontentsline\{toc\}\{chapter\}\{Bibliography\}}\newline
\end{Highlighting}
\end{Shaded}


(OR 
\begin{Shaded}
\begin{Highlighting}[]

\NormalTok{\textbackslash{}addcontentsline\{toc\}\{section\}\{Bibliography\}}\newline
\end{Highlighting}
\end{Shaded}
 if you\textquotesingle{}re writing an {\itshape \setmainfont[Path=/usr/share/fonts/truetype/cmu/,UprightFont=cmunrm.ttf,BoldFont=cmunbx.ttf,ItalicFont=cmunti.ttf,BoldItalicFont=cmunbi.ttf]{cmunti.ttf}\setmonofont[Path=/usr/share/fonts/truetype/cmu/,UprightFont=cmuntt.ttf,BoldFont=cmuntb.ttf,ItalicFont=cmunit.ttf,BoldItalicFont=cmuntx.ttf]{cmunti.ttf}\itshape article}\setmainfont[Path=/usr/share/fonts/truetype/cmu/,UprightFont=cmunrm.ttf,BoldFont=cmunbx.ttf,ItalicFont=cmunti.ttf,BoldItalicFont=cmunbi.ttf]{cmunrm.ttf}\setmonofont[Path=/usr/share/fonts/truetype/cmu/,UprightFont=cmuntt.ttf,BoldFont=cmuntb.ttf,ItalicFont=cmunit.ttf,BoldItalicFont=cmuntx.ttf]{cmunrm.ttf})

The first line just terminates the current paragraph and page. If you are writing a {\itshape \setmainfont[Path=/usr/share/fonts/truetype/cmu/,UprightFont=cmunrm.ttf,BoldFont=cmunbx.ttf,ItalicFont=cmunti.ttf,BoldItalicFont=cmunbi.ttf]{cmunti.ttf}\setmonofont[Path=/usr/share/fonts/truetype/cmu/,UprightFont=cmuntt.ttf,BoldFont=cmuntb.ttf,ItalicFont=cmunit.ttf,BoldItalicFont=cmuntx.ttf]{cmunti.ttf}\itshape book}\setmainfont[Path=/usr/share/fonts/truetype/cmu/,UprightFont=cmunrm.ttf,BoldFont=cmunbx.ttf,ItalicFont=cmunti.ttf,BoldItalicFont=cmunbi.ttf]{cmunrm.ttf}\setmonofont[Path=/usr/share/fonts/truetype/cmu/,UprightFont=cmuntt.ttf,BoldFont=cmuntb.ttf,ItalicFont=cmunit.ttf,BoldItalicFont=cmuntx.ttf]{cmunrm.ttf}, use {\ttfamily \setmainfont[Path=/usr/share/fonts/truetype/cmu/,UprightFont=cmunrm.ttf,BoldFont=cmunbx.ttf,ItalicFont=cmunti.ttf,BoldItalicFont=cmunbi.ttf]{cmuntt.ttf}\setmonofont[Path=/usr/share/fonts/truetype/cmu/,UprightFont=cmuntt.ttf,BoldFont=cmuntb.ttf,ItalicFont=cmunit.ttf,BoldItalicFont=cmuntx.ttf]{cmuntt.ttf}\ttfamily \textbackslash{}cleardoublepage}{$\text{ }$}\setmainfont[Path=/usr/share/fonts/truetype/cmu/,UprightFont=cmunrm.ttf,BoldFont=cmunbx.ttf,ItalicFont=cmunti.ttf,BoldItalicFont=cmunbi.ttf]{cmunrm.ttf}\setmonofont[Path=/usr/share/fonts/truetype/cmu/,UprightFont=cmuntt.ttf,BoldFont=cmuntb.ttf,ItalicFont=cmunit.ttf,BoldItalicFont=cmuntx.ttf]{cmunrm.ttf} to match the style used. The second line will add a line in the Table of Contents (first option, {\itshape \setmainfont[Path=/usr/share/fonts/truetype/cmu/,UprightFont=cmunrm.ttf,BoldFont=cmunbx.ttf,ItalicFont=cmunti.ttf,BoldItalicFont=cmunbi.ttf]{cmunti.ttf}\setmonofont[Path=/usr/share/fonts/truetype/cmu/,UprightFont=cmuntt.ttf,BoldFont=cmuntb.ttf,ItalicFont=cmunit.ttf,BoldItalicFont=cmuntx.ttf]{cmunti.ttf}\itshape toc}\setmainfont[Path=/usr/share/fonts/truetype/cmu/,UprightFont=cmunrm.ttf,BoldFont=cmunbx.ttf,ItalicFont=cmunti.ttf,BoldItalicFont=cmunbi.ttf]{cmunrm.ttf}\setmonofont[Path=/usr/share/fonts/truetype/cmu/,UprightFont=cmuntt.ttf,BoldFont=cmuntb.ttf,ItalicFont=cmunit.ttf,BoldItalicFont=cmuntx.ttf]{cmunrm.ttf}), it will be like the ones created by chapters (second option, {\itshape \setmainfont[Path=/usr/share/fonts/truetype/cmu/,UprightFont=cmunrm.ttf,BoldFont=cmunbx.ttf,ItalicFont=cmunti.ttf,BoldItalicFont=cmunbi.ttf]{cmunti.ttf}\setmonofont[Path=/usr/share/fonts/truetype/cmu/,UprightFont=cmuntt.ttf,BoldFont=cmuntb.ttf,ItalicFont=cmunit.ttf,BoldItalicFont=cmuntx.ttf]{cmunti.ttf}\itshape chapter}\setmainfont[Path=/usr/share/fonts/truetype/cmu/,UprightFont=cmunrm.ttf,BoldFont=cmunbx.ttf,ItalicFont=cmunti.ttf,BoldItalicFont=cmunbi.ttf]{cmunrm.ttf}\setmonofont[Path=/usr/share/fonts/truetype/cmu/,UprightFont=cmuntt.ttf,BoldFont=cmuntb.ttf,ItalicFont=cmunit.ttf,BoldItalicFont=cmuntx.ttf]{cmunrm.ttf}), and the third argument will be printed on the corresponding line in the Table of Contents; here {\itshape \setmainfont[Path=/usr/share/fonts/truetype/cmu/,UprightFont=cmunrm.ttf,BoldFont=cmunbx.ttf,ItalicFont=cmunti.ttf,BoldItalicFont=cmunbi.ttf]{cmunti.ttf}\setmonofont[Path=/usr/share/fonts/truetype/cmu/,UprightFont=cmuntt.ttf,BoldFont=cmuntb.ttf,ItalicFont=cmunit.ttf,BoldItalicFont=cmuntx.ttf]{cmunti.ttf}\itshape Bibliography}{$\text{ }$}\setmainfont[Path=/usr/share/fonts/truetype/cmu/,UprightFont=cmunrm.ttf,BoldFont=cmunbx.ttf,ItalicFont=cmunti.ttf,BoldItalicFont=cmunbi.ttf]{cmunrm.ttf}\setmonofont[Path=/usr/share/fonts/truetype/cmu/,UprightFont=cmuntt.ttf,BoldFont=cmuntb.ttf,ItalicFont=cmunit.ttf,BoldItalicFont=cmuntx.ttf]{cmunrm.ttf} was chosen because it\textquotesingle{}s the same text the {\itshape \setmainfont[Path=/usr/share/fonts/truetype/cmu/,UprightFont=cmunrm.ttf,BoldFont=cmunbx.ttf,ItalicFont=cmunti.ttf,BoldItalicFont=cmunbi.ttf]{cmunti.ttf}\setmonofont[Path=/usr/share/fonts/truetype/cmu/,UprightFont=cmuntt.ttf,BoldFont=cmuntb.ttf,ItalicFont=cmunit.ttf,BoldItalicFont=cmuntx.ttf]{cmunti.ttf}\itshape thebibliography}{$\text{ }$}\setmainfont[Path=/usr/share/fonts/truetype/cmu/,UprightFont=cmunrm.ttf,BoldFont=cmunbx.ttf,ItalicFont=cmunti.ttf,BoldItalicFont=cmunbi.ttf]{cmunrm.ttf}\setmonofont[Path=/usr/share/fonts/truetype/cmu/,UprightFont=cmuntt.ttf,BoldFont=cmuntb.ttf,ItalicFont=cmunit.ttf,BoldItalicFont=cmuntx.ttf]{cmunrm.ttf} environment will automatically write when you use it, but you are free to write whatever you like. If you are using separate bib file, add these lines between {\ttfamily \setmainfont[Path=/usr/share/fonts/truetype/cmu/,UprightFont=cmunrm.ttf,BoldFont=cmunbx.ttf,ItalicFont=cmunti.ttf,BoldItalicFont=cmunbi.ttf]{cmuntt.ttf}\setmonofont[Path=/usr/share/fonts/truetype/cmu/,UprightFont=cmuntt.ttf,BoldFont=cmuntb.ttf,ItalicFont=cmunit.ttf,BoldItalicFont=cmuntx.ttf]{cmuntt.ttf}\ttfamily \textbackslash{}bibliographystyle}{$\text{ }$}\setmainfont[Path=/usr/share/fonts/truetype/cmu/,UprightFont=cmunrm.ttf,BoldFont=cmunbx.ttf,ItalicFont=cmunti.ttf,BoldItalicFont=cmunbi.ttf]{cmunrm.ttf}\setmonofont[Path=/usr/share/fonts/truetype/cmu/,UprightFont=cmuntt.ttf,BoldFont=cmuntb.ttf,ItalicFont=cmunit.ttf,BoldItalicFont=cmuntx.ttf]{cmunrm.ttf} and {\ttfamily \setmainfont[Path=/usr/share/fonts/truetype/cmu/,UprightFont=cmunrm.ttf,BoldFont=cmunbx.ttf,ItalicFont=cmunti.ttf,BoldItalicFont=cmunbi.ttf]{cmuntt.ttf}\setmonofont[Path=/usr/share/fonts/truetype/cmu/,UprightFont=cmuntt.ttf,BoldFont=cmuntb.ttf,ItalicFont=cmunit.ttf,BoldItalicFont=cmuntx.ttf]{cmuntt.ttf}\ttfamily \textbackslash{}bibliography}\setmainfont[Path=/usr/share/fonts/truetype/cmu/,UprightFont=cmunrm.ttf,BoldFont=cmunbx.ttf,ItalicFont=cmunti.ttf,BoldItalicFont=cmunbi.ttf]{cmunrm.ttf}\setmonofont[Path=/usr/share/fonts/truetype/cmu/,UprightFont=cmuntt.ttf,BoldFont=cmuntb.ttf,ItalicFont=cmunit.ttf,BoldItalicFont=cmuntx.ttf]{cmunrm.ttf}.

If you use \mylref{392}{hyperref} package, you should also use {\ttfamily \setmainfont[Path=/usr/share/fonts/truetype/cmu/,UprightFont=cmunrm.ttf,BoldFont=cmunbx.ttf,ItalicFont=cmunti.ttf,BoldItalicFont=cmunbi.ttf]{cmuntt.ttf}\setmonofont[Path=/usr/share/fonts/truetype/cmu/,UprightFont=cmuntt.ttf,BoldFont=cmuntb.ttf,ItalicFont=cmunit.ttf,BoldItalicFont=cmuntx.ttf]{cmuntt.ttf}\ttfamily \textbackslash{}phantomsection}{$\text{ }$}\setmainfont[Path=/usr/share/fonts/truetype/cmu/,UprightFont=cmunrm.ttf,BoldFont=cmunbx.ttf,ItalicFont=cmunti.ttf,BoldItalicFont=cmunbi.ttf]{cmunrm.ttf}\setmonofont[Path=/usr/share/fonts/truetype/cmu/,UprightFont=cmuntt.ttf,BoldFont=cmuntb.ttf,ItalicFont=cmunit.ttf,BoldItalicFont=cmuntx.ttf]{cmunrm.ttf} command to enable hyperlinking from the table of contents to bibliography.


\begin{Shaded}
\begin{Highlighting}[]

\NormalTok{\textbackslash{}clearpage}\CommentTok{\%\ensuremath{\text{ }}or\ensuremath{\text{ }}cleardoublepage}\newline
\NormalTok{\textbackslash{}phantomsection}\newline
\NormalTok{\textbackslash{}addcontentsline\{toc\}\{chapter\}\{Bibliography\}}\newline
\end{Highlighting}
\end{Shaded}


This trick is particularly useful when you have to insert the bibliography in the Table of Contents, but it can work for anything. When LaTeX finds the code above, it will record the info as described and the current page number, inserting a new line in the Contents page.
\subsubsection{As numbered item}
\label{695}
If you instead want bibliography to be numbered section or chapter, you\textquotesingle{}ll likely use this way:


\begin{Shaded}
\begin{Highlighting}[]

\NormalTok{\textbackslash{}cleardoublepage\ensuremath{\text{ }}}\CommentTok{\%\ensuremath{\text{ }}This\ensuremath{\text{ }}is\ensuremath{\text{ }}needed\ensuremath{\text{ }}if\ensuremath{\text{ }}the\ensuremath{\text{ }}book\ensuremath{\text{ }}class\ensuremath{\text{ }}is\ensuremath{\text{ }}used,\ensuremath{\text{ }}to\ensuremath{\text{ }}place\ensuremath{\text{ }}the\ensuremath{\text{ }}anchor}\newline
\ensuremath{\text{ }}\NormalTok{in\ensuremath{\text{ }}the\ensuremath{\text{ }}correct\ensuremath{\text{ }}page,}\newline
\ensuremath{\text{ }}\ensuremath{\text{ }}\ensuremath{\text{ }}\ensuremath{\text{ }}\ensuremath{\text{ }}\ensuremath{\text{ }}\ensuremath{\text{ }}\ensuremath{\text{ }}\ensuremath{\text{ }}\ensuremath{\text{ }}\ensuremath{\text{ }}\ensuremath{\text{ }}\ensuremath{\text{ }}\ensuremath{\text{ }}\ensuremath{\text{ }}\ensuremath{\text{ }}\ensuremath{\text{ }}\CommentTok{\%\ensuremath{\text{ }}because\ensuremath{\text{ }}the\ensuremath{\text{ }}bibliography\ensuremath{\text{ }}will\ensuremath{\text{ }}start\ensuremath{\text{ }}on\ensuremath{\text{ }}its\ensuremath{\text{ }}own\ensuremath{\text{ }}page.}\newline
\ensuremath{\text{ }}\ensuremath{\text{ }}\ensuremath{\text{ }}\ensuremath{\text{ }}\ensuremath{\text{ }}\ensuremath{\text{ }}\ensuremath{\text{ }}\ensuremath{\text{ }}\ensuremath{\text{ }}\ensuremath{\text{ }}\ensuremath{\text{ }}\ensuremath{\text{ }}\ensuremath{\text{ }}\ensuremath{\text{ }}\ensuremath{\text{ }}\ensuremath{\text{ }}\ensuremath{\text{ }}\CommentTok{\%\ensuremath{\text{ }}Use\ensuremath{\text{ }}\textbackslash{}clearpage\ensuremath{\text{ }}instead\ensuremath{\text{ }}if\ensuremath{\text{ }}the\ensuremath{\text{ }}document\ensuremath{\text{ }}class\ensuremath{\text{ }}uses\ensuremath{\text{ }}the}\newline
\ensuremath{\text{ }}\NormalTok{"oneside"\ensuremath{\text{ }}argument}\newline
\NormalTok{\textbackslash{}renewcommand*\{\textbackslash{}refname\}\{\}\ensuremath{\text{ }}}\CommentTok{\%\ensuremath{\text{ }}This\ensuremath{\text{ }}will\ensuremath{\text{ }}define\ensuremath{\text{ }}heading\ensuremath{\text{ }}of\ensuremath{\text{ }}bibliography\ensuremath{\text{ }}to\ensuremath{\text{ }}be}\newline
\ensuremath{\text{ }}\NormalTok{empty,\ensuremath{\text{ }}so\ensuremath{\text{ }}you\ensuremath{\text{ }}can...}\newline
\NormalTok{\textbackslash{}section\{Bibliography\}\ensuremath{\text{ }}\ensuremath{\text{ }}\ensuremath{\text{ }}\ensuremath{\text{ }}\ensuremath{\text{ }}}\CommentTok{\%\ensuremath{\text{ }}...place\ensuremath{\text{ }}a\ensuremath{\text{ }}normal\ensuremath{\text{ }}section\ensuremath{\text{ }}heading\ensuremath{\text{ }}before\ensuremath{\text{ }}the}\newline
\ensuremath{\text{ }}\NormalTok{bibliography\ensuremath{\text{ }}entries.}\newline
\ensuremath{\text{ }}\newline
\NormalTok{\textbackslash{}begin\{thebibliography\}\{99\}}\newline
\NormalTok{...}\newline
\NormalTok{\textbackslash{}end\{thebibliography\}}\newline
\end{Highlighting}
\end{Shaded}


Another even easier solution is to use 
\begin{Shaded}
\begin{Highlighting}[]

\NormalTok{\textbackslash{}section}\newline
\end{Highlighting}
\end{Shaded}
 inside of the 
\begin{Shaded}
\begin{Highlighting}[]

\NormalTok{\textbackslash{}renewcommand}\newline
\end{Highlighting}
\end{Shaded}
 block:


\begin{Shaded}
\begin{Highlighting}[]

\NormalTok{\textbackslash{}renewcommand\{\textbackslash{}refname\}\{\textbackslash{}section\{Sources\}\}\ensuremath{\text{ }}}\CommentTok{\%\ensuremath{\text{ }}Using\ensuremath{\text{ }}"Sources"\ensuremath{\text{ }}as\ensuremath{\text{ }}the\ensuremath{\text{ }}title\ensuremath{\text{ }}of\ensuremath{\text{ }}the}\newline
\ensuremath{\text{ }}\NormalTok{section}\newline
\NormalTok{\textbackslash{}begin\{thebibliography\}\{99\}}\newline
\NormalTok{...}\newline
\NormalTok{\textbackslash{}end\{thebibliography\}}\newline
\end{Highlighting}
\end{Shaded}


You may wish to use 
\begin{Shaded}
\begin{Highlighting}[]

\NormalTok{\textbackslash{}renewcommand*\{\textbackslash{}refname\}\{\textbackslash{}vspace*\{-1em\}\}}\newline
\end{Highlighting}
\end{Shaded}
 followed by 
\begin{Shaded}
\begin{Highlighting}[]

\NormalTok{\textbackslash{}vspace*\{-1em\}}\newline
\end{Highlighting}
\end{Shaded}
 to counteract the extra space the blank 
\begin{Shaded}
\begin{Highlighting}[]

\NormalTok{\textbackslash{}refname}\newline
\end{Highlighting}
\end{Shaded}
 inserts.

If you are using BibTeX, the 
\begin{Shaded}
\begin{Highlighting}[]

\NormalTok{\textbackslash{}bibliography}\newline
\end{Highlighting}
\end{Shaded}
 command, and the book or report class, you will need to redefine 
\begin{Shaded}
\begin{Highlighting}[]

\NormalTok{\textbackslash{}bibname}\newline
\end{Highlighting}
\end{Shaded}
 instead of 
\begin{Shaded}
\begin{Highlighting}[]

\NormalTok{\textbackslash{}refname}\newline
\end{Highlighting}
\end{Shaded}
 like so.


\begin{Shaded}
\begin{Highlighting}[]

\NormalTok{\textbackslash{}renewcommand\{\textbackslash{}bibname\}\{\textbackslash{}section\{Sources\}\}\ensuremath{\text{ }}}\CommentTok{\%\ensuremath{\text{ }}Redefine\ensuremath{\text{ }}bibname}\newline
\NormalTok{\textbackslash{}bibliographystyle\{IEEEtran\}\ensuremath{\text{ }}\ensuremath{\text{ }}\ensuremath{\text{ }}\ensuremath{\text{ }}\ensuremath{\text{ }}\ensuremath{\text{ }}\ensuremath{\text{ }}\ensuremath{\text{ }}\ensuremath{\text{ }}\ensuremath{\text{ }}\ensuremath{\text{ }}\ensuremath{\text{ }}\ensuremath{\text{ }}\ensuremath{\text{ }}\ensuremath{\text{ }}}\CommentTok{\%\ensuremath{\text{ }}Set\ensuremath{\text{ }}any\ensuremath{\text{ }}options\ensuremath{\text{ }}you\ensuremath{\text{ }}want}\newline
\NormalTok{\textbackslash{}bibliography\{your_bib_file_names\}\ensuremath{\text{ }}\ensuremath{\text{ }}\ensuremath{\text{ }}\ensuremath{\text{ }}\ensuremath{\text{ }}\ensuremath{\text{ }}\ensuremath{\text{ }}\ensuremath{\text{ }}\ensuremath{\text{ }}}\CommentTok{\%\ensuremath{\text{ }}Build\ensuremath{\text{ }}the\ensuremath{\text{ }}bibliography}\newline
\end{Highlighting}
\end{Shaded}

\section{biblatex}
\label{696}
As we said before, biblatex is widely considered the `successor\textquotesingle{} of BibTeX.  Intended as a full replacement for BibTeX, it is more configurable in its output and provides a multitude of new styles (for output) and fields (for the database) that can be used in a document.  For now, refer to its \myhref{http://www.ctan.org/pkg/biblatex}{comprehensive documentation on CTAN}.
\subsection{Entry and field types in .bib files}
\label{697}
The following table shows most field types. Some field types are lists, either {\ttfamily \setmainfont[Path=/usr/share/fonts/truetype/cmu/,UprightFont=cmunrm.ttf,BoldFont=cmunbx.ttf,ItalicFont=cmunti.ttf,BoldItalicFont=cmunbi.ttf]{cmuntt.ttf}\setmonofont[Path=/usr/share/fonts/truetype/cmu/,UprightFont=cmuntt.ttf,BoldFont=cmuntb.ttf,ItalicFont=cmunit.ttf,BoldItalicFont=cmuntx.ttf]{cmuntt.ttf}\ttfamily lists of person names}\setmainfont[Path=/usr/share/fonts/truetype/cmu/,UprightFont=cmunrm.ttf,BoldFont=cmunbx.ttf,ItalicFont=cmunti.ttf,BoldItalicFont=cmunbi.ttf]{cmunrm.ttf}\setmonofont[Path=/usr/share/fonts/truetype/cmu/,UprightFont=cmuntt.ttf,BoldFont=cmuntb.ttf,ItalicFont=cmunit.ttf,BoldItalicFont=cmuntx.ttf]{cmunrm.ttf}, others are {\ttfamily \setmainfont[Path=/usr/share/fonts/truetype/cmu/,UprightFont=cmunrm.ttf,BoldFont=cmunbx.ttf,ItalicFont=cmunti.ttf,BoldItalicFont=cmunbi.ttf]{cmuntt.ttf}\setmonofont[Path=/usr/share/fonts/truetype/cmu/,UprightFont=cmuntt.ttf,BoldFont=cmuntb.ttf,ItalicFont=cmunit.ttf,BoldItalicFont=cmuntx.ttf]{cmuntt.ttf}\ttfamily literal lists}\setmainfont[Path=/usr/share/fonts/truetype/cmu/,UprightFont=cmunrm.ttf,BoldFont=cmunbx.ttf,ItalicFont=cmunti.ttf,BoldItalicFont=cmunbi.ttf]{cmunrm.ttf}\setmonofont[Path=/usr/share/fonts/truetype/cmu/,UprightFont=cmuntt.ttf,BoldFont=cmuntb.ttf,ItalicFont=cmunit.ttf,BoldItalicFont=cmuntx.ttf]{cmunrm.ttf}. A {\ttfamily \setmainfont[Path=/usr/share/fonts/truetype/cmu/,UprightFont=cmunrm.ttf,BoldFont=cmunbx.ttf,ItalicFont=cmunti.ttf,BoldItalicFont=cmunbi.ttf]{cmuntt.ttf}\setmonofont[Path=/usr/share/fonts/truetype/cmu/,UprightFont=cmuntt.ttf,BoldFont=cmuntb.ttf,ItalicFont=cmunit.ttf,BoldItalicFont=cmuntx.ttf]{cmuntt.ttf}\ttfamily date}{$\text{ }$}\setmainfont[Path=/usr/share/fonts/truetype/cmu/,UprightFont=cmunrm.ttf,BoldFont=cmunbx.ttf,ItalicFont=cmunti.ttf,BoldItalicFont=cmunbi.ttf]{cmunrm.ttf}\setmonofont[Path=/usr/share/fonts/truetype/cmu/,UprightFont=cmuntt.ttf,BoldFont=cmuntb.ttf,ItalicFont=cmunit.ttf,BoldItalicFont=cmuntx.ttf]{cmunrm.ttf} can either be given in parts or full, some {\ttfamily \setmainfont[Path=/usr/share/fonts/truetype/cmu/,UprightFont=cmunrm.ttf,BoldFont=cmunbx.ttf,ItalicFont=cmunti.ttf,BoldItalicFont=cmunbi.ttf]{cmuntt.ttf}\setmonofont[Path=/usr/share/fonts/truetype/cmu/,UprightFont=cmuntt.ttf,BoldFont=cmuntb.ttf,ItalicFont=cmunit.ttf,BoldItalicFont=cmuntx.ttf]{cmuntt.ttf}\ttfamily keys}{$\text{ }$}\setmainfont[Path=/usr/share/fonts/truetype/cmu/,UprightFont=cmunrm.ttf,BoldFont=cmunbx.ttf,ItalicFont=cmunti.ttf,BoldItalicFont=cmunbi.ttf]{cmunrm.ttf}\setmonofont[Path=/usr/share/fonts/truetype/cmu/,UprightFont=cmuntt.ttf,BoldFont=cmuntb.ttf,ItalicFont=cmunit.ttf,BoldItalicFont=cmuntx.ttf]{cmunrm.ttf} are necessary, page references are provided as {\ttfamily \setmainfont[Path=/usr/share/fonts/truetype/cmu/,UprightFont=cmunrm.ttf,BoldFont=cmunbx.ttf,ItalicFont=cmunti.ttf,BoldItalicFont=cmunbi.ttf]{cmuntt.ttf}\setmonofont[Path=/usr/share/fonts/truetype/cmu/,UprightFont=cmuntt.ttf,BoldFont=cmuntb.ttf,ItalicFont=cmunit.ttf,BoldItalicFont=cmuntx.ttf]{cmuntt.ttf}\ttfamily ranges}{$\text{ }$}\setmainfont[Path=/usr/share/fonts/truetype/cmu/,UprightFont=cmunrm.ttf,BoldFont=cmunbx.ttf,ItalicFont=cmunti.ttf,BoldItalicFont=cmunbi.ttf]{cmunrm.ttf}\setmonofont[Path=/usr/share/fonts/truetype/cmu/,UprightFont=cmuntt.ttf,BoldFont=cmuntb.ttf,ItalicFont=cmunit.ttf,BoldItalicFont=cmuntx.ttf]{cmunrm.ttf} and certain special fields contain {\ttfamily \setmainfont[Path=/usr/share/fonts/truetype/cmu/,UprightFont=cmunrm.ttf,BoldFont=cmunbx.ttf,ItalicFont=cmunti.ttf,BoldItalicFont=cmunbi.ttf]{cmuntt.ttf}\setmonofont[Path=/usr/share/fonts/truetype/cmu/,UprightFont=cmuntt.ttf,BoldFont=cmuntb.ttf,ItalicFont=cmunit.ttf,BoldItalicFont=cmuntx.ttf]{cmuntt.ttf}\ttfamily verbatim}{$\text{ }$}\setmainfont[Path=/usr/share/fonts/truetype/cmu/,UprightFont=cmunrm.ttf,BoldFont=cmunbx.ttf,ItalicFont=cmunti.ttf,BoldItalicFont=cmunbi.ttf]{cmunrm.ttf}\setmonofont[Path=/usr/share/fonts/truetype/cmu/,UprightFont=cmuntt.ttf,BoldFont=cmuntb.ttf,ItalicFont=cmunit.ttf,BoldItalicFont=cmuntx.ttf]{cmunrm.ttf} code. There are many kinds of {\ttfamily \setmainfont[Path=/usr/share/fonts/truetype/cmu/,UprightFont=cmunrm.ttf,BoldFont=cmunbx.ttf,ItalicFont=cmunti.ttf,BoldItalicFont=cmunbi.ttf]{cmuntt.ttf}\setmonofont[Path=/usr/share/fonts/truetype/cmu/,UprightFont=cmuntt.ttf,BoldFont=cmuntb.ttf,ItalicFont=cmunit.ttf,BoldItalicFont=cmuntx.ttf]{cmuntt.ttf}\ttfamily titles}\setmainfont[Path=/usr/share/fonts/truetype/cmu/,UprightFont=cmunrm.ttf,BoldFont=cmunbx.ttf,ItalicFont=cmunti.ttf,BoldItalicFont=cmunbi.ttf]{cmunrm.ttf}\setmonofont[Path=/usr/share/fonts/truetype/cmu/,UprightFont=cmuntt.ttf,BoldFont=cmuntb.ttf,ItalicFont=cmunit.ttf,BoldItalicFont=cmuntx.ttf]{cmunrm.ttf}.

\begin{longtable}{|>{\RaggedRight}p{0.18455\linewidth}|>{\RaggedRight}p{0.20980\linewidth}|>{\RaggedRight}p{0.24568\linewidth}|>{\RaggedRight}p{0.24568\linewidth}|} \hline 
\multicolumn{4}{|>{\RaggedRight}p{0.97143\linewidth}|}{{\bfseries \hspace*{0pt}\ignorespaces{}\hspace*{0pt} Hierarchic entry types}}\\ \hline {\bfseries \hspace*{0pt}\ignorespaces{}\hspace*{0pt} Base type }&{\bfseries \hspace*{0pt}\ignorespaces{}\hspace*{0pt} Multi-{}volume }&{\bfseries \hspace*{0pt}\ignorespaces{}\hspace*{0pt} Standalone part thereof }&{\bfseries \hspace*{0pt}\ignorespaces{}\hspace*{0pt} Supplemental material therein}\endhead  \hline \hspace*{0pt}\ignorespaces{}\hspace*{0pt} {\ttfamily \setmainfont[Path=/usr/share/fonts/truetype/cmu/,UprightFont=cmunrm.ttf,BoldFont=cmunbx.ttf,ItalicFont=cmunti.ttf,BoldItalicFont=cmunbi.ttf]{cmuntt.ttf}\setmonofont[Path=/usr/share/fonts/truetype/cmu/,UprightFont=cmuntt.ttf,BoldFont=cmuntb.ttf,ItalicFont=cmunit.ttf,BoldItalicFont=cmuntx.ttf]{cmuntt.ttf}\ttfamily @book}{$\text{ }$}\setmainfont[Path=/usr/share/fonts/truetype/cmu/,UprightFont=cmunrm.ttf,BoldFont=cmunbx.ttf,ItalicFont=cmunti.ttf,BoldItalicFont=cmunbi.ttf]{cmunrm.ttf}\setmonofont[Path=/usr/share/fonts/truetype/cmu/,UprightFont=cmuntt.ttf,BoldFont=cmuntb.ttf,ItalicFont=cmunit.ttf,BoldItalicFont=cmuntx.ttf]{cmunrm.ttf} &\hspace*{0pt}\ignorespaces{}\hspace*{0pt} {\ttfamily \setmainfont[Path=/usr/share/fonts/truetype/cmu/,UprightFont=cmunrm.ttf,BoldFont=cmunbx.ttf,ItalicFont=cmunti.ttf,BoldItalicFont=cmunbi.ttf]{cmuntt.ttf}\setmonofont[Path=/usr/share/fonts/truetype/cmu/,UprightFont=cmuntt.ttf,BoldFont=cmuntb.ttf,ItalicFont=cmunit.ttf,BoldItalicFont=cmuntx.ttf]{cmuntt.ttf}\ttfamily @mvbook}{$\text{ }$}\setmainfont[Path=/usr/share/fonts/truetype/cmu/,UprightFont=cmunrm.ttf,BoldFont=cmunbx.ttf,ItalicFont=cmunti.ttf,BoldItalicFont=cmunbi.ttf]{cmunrm.ttf}\setmonofont[Path=/usr/share/fonts/truetype/cmu/,UprightFont=cmuntt.ttf,BoldFont=cmuntb.ttf,ItalicFont=cmunit.ttf,BoldItalicFont=cmuntx.ttf]{cmunrm.ttf} &\hspace*{0pt}\ignorespaces{}\hspace*{0pt} {\ttfamily \setmainfont[Path=/usr/share/fonts/truetype/cmu/,UprightFont=cmunrm.ttf,BoldFont=cmunbx.ttf,ItalicFont=cmunti.ttf,BoldItalicFont=cmunbi.ttf]{cmuntt.ttf}\setmonofont[Path=/usr/share/fonts/truetype/cmu/,UprightFont=cmuntt.ttf,BoldFont=cmuntb.ttf,ItalicFont=cmunit.ttf,BoldItalicFont=cmuntx.ttf]{cmuntt.ttf}\ttfamily @inbook}\setmainfont[Path=/usr/share/fonts/truetype/cmu/,UprightFont=cmunrm.ttf,BoldFont=cmunbx.ttf,ItalicFont=cmunti.ttf,BoldItalicFont=cmunbi.ttf]{cmunrm.ttf}\setmonofont[Path=/usr/share/fonts/truetype/cmu/,UprightFont=cmuntt.ttf,BoldFont=cmuntb.ttf,ItalicFont=cmunit.ttf,BoldItalicFont=cmuntx.ttf]{cmunrm.ttf}, {\ttfamily \setmainfont[Path=/usr/share/fonts/truetype/cmu/,UprightFont=cmunrm.ttf,BoldFont=cmunbx.ttf,ItalicFont=cmunti.ttf,BoldItalicFont=cmunbi.ttf]{cmuntt.ttf}\setmonofont[Path=/usr/share/fonts/truetype/cmu/,UprightFont=cmuntt.ttf,BoldFont=cmuntb.ttf,ItalicFont=cmunit.ttf,BoldItalicFont=cmuntx.ttf]{cmuntt.ttf}\ttfamily @bookinbook}{$\text{ }$}\setmainfont[Path=/usr/share/fonts/truetype/cmu/,UprightFont=cmunrm.ttf,BoldFont=cmunbx.ttf,ItalicFont=cmunti.ttf,BoldItalicFont=cmunbi.ttf]{cmunrm.ttf}\setmonofont[Path=/usr/share/fonts/truetype/cmu/,UprightFont=cmuntt.ttf,BoldFont=cmuntb.ttf,ItalicFont=cmunit.ttf,BoldItalicFont=cmuntx.ttf]{cmunrm.ttf} &\hspace*{0pt}\ignorespaces{}\hspace*{0pt} {\ttfamily \setmainfont[Path=/usr/share/fonts/truetype/cmu/,UprightFont=cmunrm.ttf,BoldFont=cmunbx.ttf,ItalicFont=cmunti.ttf,BoldItalicFont=cmunbi.ttf]{cmuntt.ttf}\setmonofont[Path=/usr/share/fonts/truetype/cmu/,UprightFont=cmuntt.ttf,BoldFont=cmuntb.ttf,ItalicFont=cmunit.ttf,BoldItalicFont=cmuntx.ttf]{cmuntt.ttf}\ttfamily @suppbook}{$\text{ }$}\setmainfont[Path=/usr/share/fonts/truetype/cmu/,UprightFont=cmunrm.ttf,BoldFont=cmunbx.ttf,ItalicFont=cmunti.ttf,BoldItalicFont=cmunbi.ttf]{cmunrm.ttf}\setmonofont[Path=/usr/share/fonts/truetype/cmu/,UprightFont=cmuntt.ttf,BoldFont=cmuntb.ttf,ItalicFont=cmunit.ttf,BoldItalicFont=cmuntx.ttf]{cmunrm.ttf} \\ \hline \hspace*{0pt}\ignorespaces{}\hspace*{0pt} {\ttfamily \setmainfont[Path=/usr/share/fonts/truetype/cmu/,UprightFont=cmunrm.ttf,BoldFont=cmunbx.ttf,ItalicFont=cmunti.ttf,BoldItalicFont=cmunbi.ttf]{cmuntt.ttf}\setmonofont[Path=/usr/share/fonts/truetype/cmu/,UprightFont=cmuntt.ttf,BoldFont=cmuntb.ttf,ItalicFont=cmunit.ttf,BoldItalicFont=cmuntx.ttf]{cmuntt.ttf}\ttfamily @periodical}{$\text{ }$}\setmainfont[Path=/usr/share/fonts/truetype/cmu/,UprightFont=cmunrm.ttf,BoldFont=cmunbx.ttf,ItalicFont=cmunti.ttf,BoldItalicFont=cmunbi.ttf]{cmunrm.ttf}\setmonofont[Path=/usr/share/fonts/truetype/cmu/,UprightFont=cmuntt.ttf,BoldFont=cmuntb.ttf,ItalicFont=cmunit.ttf,BoldItalicFont=cmuntx.ttf]{cmunrm.ttf} &\hspace*{0pt}\ignorespaces{}\hspace*{0pt} — &\hspace*{0pt}\ignorespaces{}\hspace*{0pt} {\ttfamily \setmainfont[Path=/usr/share/fonts/truetype/cmu/,UprightFont=cmunrm.ttf,BoldFont=cmunbx.ttf,ItalicFont=cmunti.ttf,BoldItalicFont=cmunbi.ttf]{cmuntt.ttf}\setmonofont[Path=/usr/share/fonts/truetype/cmu/,UprightFont=cmuntt.ttf,BoldFont=cmuntb.ttf,ItalicFont=cmunit.ttf,BoldItalicFont=cmuntx.ttf]{cmuntt.ttf}\ttfamily @article}{$\text{ }$}\setmainfont[Path=/usr/share/fonts/truetype/cmu/,UprightFont=cmunrm.ttf,BoldFont=cmunbx.ttf,ItalicFont=cmunti.ttf,BoldItalicFont=cmunbi.ttf]{cmunrm.ttf}\setmonofont[Path=/usr/share/fonts/truetype/cmu/,UprightFont=cmuntt.ttf,BoldFont=cmuntb.ttf,ItalicFont=cmunit.ttf,BoldItalicFont=cmuntx.ttf]{cmunrm.ttf} &\hspace*{0pt}\ignorespaces{}\hspace*{0pt} {\ttfamily \setmainfont[Path=/usr/share/fonts/truetype/cmu/,UprightFont=cmunrm.ttf,BoldFont=cmunbx.ttf,ItalicFont=cmunti.ttf,BoldItalicFont=cmunbi.ttf]{cmuntt.ttf}\setmonofont[Path=/usr/share/fonts/truetype/cmu/,UprightFont=cmuntt.ttf,BoldFont=cmuntb.ttf,ItalicFont=cmunit.ttf,BoldItalicFont=cmuntx.ttf]{cmuntt.ttf}\ttfamily @suppperiodical}{$\text{ }$}\setmainfont[Path=/usr/share/fonts/truetype/cmu/,UprightFont=cmunrm.ttf,BoldFont=cmunbx.ttf,ItalicFont=cmunti.ttf,BoldItalicFont=cmunbi.ttf]{cmunrm.ttf}\setmonofont[Path=/usr/share/fonts/truetype/cmu/,UprightFont=cmuntt.ttf,BoldFont=cmuntb.ttf,ItalicFont=cmunit.ttf,BoldItalicFont=cmuntx.ttf]{cmunrm.ttf} \\ \hline \hspace*{0pt}\ignorespaces{}\hspace*{0pt} {\ttfamily \setmainfont[Path=/usr/share/fonts/truetype/cmu/,UprightFont=cmunrm.ttf,BoldFont=cmunbx.ttf,ItalicFont=cmunti.ttf,BoldItalicFont=cmunbi.ttf]{cmuntt.ttf}\setmonofont[Path=/usr/share/fonts/truetype/cmu/,UprightFont=cmuntt.ttf,BoldFont=cmuntb.ttf,ItalicFont=cmunit.ttf,BoldItalicFont=cmuntx.ttf]{cmuntt.ttf}\ttfamily @collection}{$\text{ }$}\setmainfont[Path=/usr/share/fonts/truetype/cmu/,UprightFont=cmunrm.ttf,BoldFont=cmunbx.ttf,ItalicFont=cmunti.ttf,BoldItalicFont=cmunbi.ttf]{cmunrm.ttf}\setmonofont[Path=/usr/share/fonts/truetype/cmu/,UprightFont=cmuntt.ttf,BoldFont=cmuntb.ttf,ItalicFont=cmunit.ttf,BoldItalicFont=cmuntx.ttf]{cmunrm.ttf} &\hspace*{0pt}\ignorespaces{}\hspace*{0pt} {\ttfamily \setmainfont[Path=/usr/share/fonts/truetype/cmu/,UprightFont=cmunrm.ttf,BoldFont=cmunbx.ttf,ItalicFont=cmunti.ttf,BoldItalicFont=cmunbi.ttf]{cmuntt.ttf}\setmonofont[Path=/usr/share/fonts/truetype/cmu/,UprightFont=cmuntt.ttf,BoldFont=cmuntb.ttf,ItalicFont=cmunit.ttf,BoldItalicFont=cmuntx.ttf]{cmuntt.ttf}\ttfamily @mvcollection}{$\text{ }$}\setmainfont[Path=/usr/share/fonts/truetype/cmu/,UprightFont=cmunrm.ttf,BoldFont=cmunbx.ttf,ItalicFont=cmunti.ttf,BoldItalicFont=cmunbi.ttf]{cmunrm.ttf}\setmonofont[Path=/usr/share/fonts/truetype/cmu/,UprightFont=cmuntt.ttf,BoldFont=cmuntb.ttf,ItalicFont=cmunit.ttf,BoldItalicFont=cmuntx.ttf]{cmunrm.ttf} &\hspace*{0pt}\ignorespaces{}\hspace*{0pt} {\ttfamily \setmainfont[Path=/usr/share/fonts/truetype/cmu/,UprightFont=cmunrm.ttf,BoldFont=cmunbx.ttf,ItalicFont=cmunti.ttf,BoldItalicFont=cmunbi.ttf]{cmuntt.ttf}\setmonofont[Path=/usr/share/fonts/truetype/cmu/,UprightFont=cmuntt.ttf,BoldFont=cmuntb.ttf,ItalicFont=cmunit.ttf,BoldItalicFont=cmuntx.ttf]{cmuntt.ttf}\ttfamily @incollection}{$\text{ }$}\setmainfont[Path=/usr/share/fonts/truetype/cmu/,UprightFont=cmunrm.ttf,BoldFont=cmunbx.ttf,ItalicFont=cmunti.ttf,BoldItalicFont=cmunbi.ttf]{cmunrm.ttf}\setmonofont[Path=/usr/share/fonts/truetype/cmu/,UprightFont=cmuntt.ttf,BoldFont=cmuntb.ttf,ItalicFont=cmunit.ttf,BoldItalicFont=cmuntx.ttf]{cmunrm.ttf} &\hspace*{0pt}\ignorespaces{}\hspace*{0pt} {\ttfamily \setmainfont[Path=/usr/share/fonts/truetype/cmu/,UprightFont=cmunrm.ttf,BoldFont=cmunbx.ttf,ItalicFont=cmunti.ttf,BoldItalicFont=cmunbi.ttf]{cmuntt.ttf}\setmonofont[Path=/usr/share/fonts/truetype/cmu/,UprightFont=cmuntt.ttf,BoldFont=cmuntb.ttf,ItalicFont=cmunit.ttf,BoldItalicFont=cmuntx.ttf]{cmuntt.ttf}\ttfamily @suppcollection}{$\text{ }$}\setmainfont[Path=/usr/share/fonts/truetype/cmu/,UprightFont=cmunrm.ttf,BoldFont=cmunbx.ttf,ItalicFont=cmunti.ttf,BoldItalicFont=cmunbi.ttf]{cmunrm.ttf}\setmonofont[Path=/usr/share/fonts/truetype/cmu/,UprightFont=cmuntt.ttf,BoldFont=cmuntb.ttf,ItalicFont=cmunit.ttf,BoldItalicFont=cmuntx.ttf]{cmunrm.ttf} \\ \hline \hspace*{0pt}\ignorespaces{}\hspace*{0pt} {\ttfamily \setmainfont[Path=/usr/share/fonts/truetype/cmu/,UprightFont=cmunrm.ttf,BoldFont=cmunbx.ttf,ItalicFont=cmunti.ttf,BoldItalicFont=cmunbi.ttf]{cmuntt.ttf}\setmonofont[Path=/usr/share/fonts/truetype/cmu/,UprightFont=cmuntt.ttf,BoldFont=cmuntb.ttf,ItalicFont=cmunit.ttf,BoldItalicFont=cmuntx.ttf]{cmuntt.ttf}\ttfamily @reference}{$\text{ }$}\setmainfont[Path=/usr/share/fonts/truetype/cmu/,UprightFont=cmunrm.ttf,BoldFont=cmunbx.ttf,ItalicFont=cmunti.ttf,BoldItalicFont=cmunbi.ttf]{cmunrm.ttf}\setmonofont[Path=/usr/share/fonts/truetype/cmu/,UprightFont=cmuntt.ttf,BoldFont=cmuntb.ttf,ItalicFont=cmunit.ttf,BoldItalicFont=cmuntx.ttf]{cmunrm.ttf} &\hspace*{0pt}\ignorespaces{}\hspace*{0pt} {\ttfamily \setmainfont[Path=/usr/share/fonts/truetype/cmu/,UprightFont=cmunrm.ttf,BoldFont=cmunbx.ttf,ItalicFont=cmunti.ttf,BoldItalicFont=cmunbi.ttf]{cmuntt.ttf}\setmonofont[Path=/usr/share/fonts/truetype/cmu/,UprightFont=cmuntt.ttf,BoldFont=cmuntb.ttf,ItalicFont=cmunit.ttf,BoldItalicFont=cmuntx.ttf]{cmuntt.ttf}\ttfamily @mvreference}{$\text{ }$}\setmainfont[Path=/usr/share/fonts/truetype/cmu/,UprightFont=cmunrm.ttf,BoldFont=cmunbx.ttf,ItalicFont=cmunti.ttf,BoldItalicFont=cmunbi.ttf]{cmunrm.ttf}\setmonofont[Path=/usr/share/fonts/truetype/cmu/,UprightFont=cmuntt.ttf,BoldFont=cmuntb.ttf,ItalicFont=cmunit.ttf,BoldItalicFont=cmuntx.ttf]{cmunrm.ttf} &\hspace*{0pt}\ignorespaces{}\hspace*{0pt} {\ttfamily \setmainfont[Path=/usr/share/fonts/truetype/cmu/,UprightFont=cmunrm.ttf,BoldFont=cmunbx.ttf,ItalicFont=cmunti.ttf,BoldItalicFont=cmunbi.ttf]{cmuntt.ttf}\setmonofont[Path=/usr/share/fonts/truetype/cmu/,UprightFont=cmuntt.ttf,BoldFont=cmuntb.ttf,ItalicFont=cmunit.ttf,BoldItalicFont=cmuntx.ttf]{cmuntt.ttf}\ttfamily @inreference}{$\text{ }$}\setmainfont[Path=/usr/share/fonts/truetype/cmu/,UprightFont=cmunrm.ttf,BoldFont=cmunbx.ttf,ItalicFont=cmunti.ttf,BoldItalicFont=cmunbi.ttf]{cmunrm.ttf}\setmonofont[Path=/usr/share/fonts/truetype/cmu/,UprightFont=cmuntt.ttf,BoldFont=cmuntb.ttf,ItalicFont=cmunit.ttf,BoldItalicFont=cmuntx.ttf]{cmunrm.ttf} &\hspace*{0pt}\ignorespaces{}\hspace*{0pt} —\\ \hline \hspace*{0pt}\ignorespaces{}\hspace*{0pt} {\ttfamily \setmainfont[Path=/usr/share/fonts/truetype/cmu/,UprightFont=cmunrm.ttf,BoldFont=cmunbx.ttf,ItalicFont=cmunti.ttf,BoldItalicFont=cmunbi.ttf]{cmuntt.ttf}\setmonofont[Path=/usr/share/fonts/truetype/cmu/,UprightFont=cmuntt.ttf,BoldFont=cmuntb.ttf,ItalicFont=cmunit.ttf,BoldItalicFont=cmuntx.ttf]{cmuntt.ttf}\ttfamily @proceedings}{$\text{ }$}\setmainfont[Path=/usr/share/fonts/truetype/cmu/,UprightFont=cmunrm.ttf,BoldFont=cmunbx.ttf,ItalicFont=cmunti.ttf,BoldItalicFont=cmunbi.ttf]{cmunrm.ttf}\setmonofont[Path=/usr/share/fonts/truetype/cmu/,UprightFont=cmuntt.ttf,BoldFont=cmuntb.ttf,ItalicFont=cmunit.ttf,BoldItalicFont=cmuntx.ttf]{cmunrm.ttf} &\hspace*{0pt}\ignorespaces{}\hspace*{0pt} {\ttfamily \setmainfont[Path=/usr/share/fonts/truetype/cmu/,UprightFont=cmunrm.ttf,BoldFont=cmunbx.ttf,ItalicFont=cmunti.ttf,BoldItalicFont=cmunbi.ttf]{cmuntt.ttf}\setmonofont[Path=/usr/share/fonts/truetype/cmu/,UprightFont=cmuntt.ttf,BoldFont=cmuntb.ttf,ItalicFont=cmunit.ttf,BoldItalicFont=cmuntx.ttf]{cmuntt.ttf}\ttfamily @mvproceedings}{$\text{ }$}\setmainfont[Path=/usr/share/fonts/truetype/cmu/,UprightFont=cmunrm.ttf,BoldFont=cmunbx.ttf,ItalicFont=cmunti.ttf,BoldItalicFont=cmunbi.ttf]{cmunrm.ttf}\setmonofont[Path=/usr/share/fonts/truetype/cmu/,UprightFont=cmuntt.ttf,BoldFont=cmuntb.ttf,ItalicFont=cmunit.ttf,BoldItalicFont=cmuntx.ttf]{cmunrm.ttf} &\hspace*{0pt}\ignorespaces{}\hspace*{0pt} {\ttfamily \setmainfont[Path=/usr/share/fonts/truetype/cmu/,UprightFont=cmunrm.ttf,BoldFont=cmunbx.ttf,ItalicFont=cmunti.ttf,BoldItalicFont=cmunbi.ttf]{cmuntt.ttf}\setmonofont[Path=/usr/share/fonts/truetype/cmu/,UprightFont=cmuntt.ttf,BoldFont=cmuntb.ttf,ItalicFont=cmunit.ttf,BoldItalicFont=cmuntx.ttf]{cmuntt.ttf}\ttfamily @inproceedings}\setmainfont[Path=/usr/share/fonts/truetype/cmu/,UprightFont=cmunrm.ttf,BoldFont=cmunbx.ttf,ItalicFont=cmunti.ttf,BoldItalicFont=cmunbi.ttf]{cmunrm.ttf}\setmonofont[Path=/usr/share/fonts/truetype/cmu/,UprightFont=cmuntt.ttf,BoldFont=cmuntb.ttf,ItalicFont=cmunit.ttf,BoldItalicFont=cmuntx.ttf]{cmunrm.ttf}, {\ttfamily \setmainfont[Path=/usr/share/fonts/truetype/cmu/,UprightFont=cmunrm.ttf,BoldFont=cmunbx.ttf,ItalicFont=cmunti.ttf,BoldItalicFont=cmunbi.ttf]{cmuntt.ttf}\setmonofont[Path=/usr/share/fonts/truetype/cmu/,UprightFont=cmuntt.ttf,BoldFont=cmuntb.ttf,ItalicFont=cmunit.ttf,BoldItalicFont=cmuntx.ttf]{cmuntt.ttf}\ttfamily @conference}{$\text{ }$}\setmainfont[Path=/usr/share/fonts/truetype/cmu/,UprightFont=cmunrm.ttf,BoldFont=cmunbx.ttf,ItalicFont=cmunti.ttf,BoldItalicFont=cmunbi.ttf]{cmunrm.ttf}\setmonofont[Path=/usr/share/fonts/truetype/cmu/,UprightFont=cmuntt.ttf,BoldFont=cmuntb.ttf,ItalicFont=cmunit.ttf,BoldItalicFont=cmuntx.ttf]{cmunrm.ttf} &\hspace*{0pt}\ignorespaces{}\hspace*{0pt} —\\ \hline 
\end{longtable}

{\scriptsize{}
{\scalefont{0.50896}\begin{longtable}{|>{\RaggedRight}p{0.03512\linewidth}|>{\RaggedRight}p{0.04014\linewidth}|>{\RaggedRight}p{0.03405\linewidth}|>{\RaggedRight}p{0.02903\linewidth}|>{\RaggedRight}p{0.02436\linewidth}|>{\RaggedRight}p{0.02640\linewidth}|>{\RaggedRight}p{0.04275\linewidth}|>{\RaggedRight}p{0.02903\linewidth}|>{\RaggedRight}p{0.02436\linewidth}|>{\RaggedRight}p{0.03512\linewidth}|>{\RaggedRight}p{0.03268\linewidth}|>{\RaggedRight}p{0.03942\linewidth}|>{\RaggedRight}p{0.04031\linewidth}|>{\RaggedRight}p{0.04354\linewidth}|>{\RaggedRight}p{0.03512\linewidth}|>{\RaggedRight}p{0.02903\linewidth}|>{\RaggedRight}p{0.02436\linewidth}|>{\RaggedRight}p{0.03995\linewidth}|>{\RaggedRight}p{0.03730\linewidth}|>{\RaggedRight}p{0.04363\linewidth}|} \hline 
\multicolumn{20}{|>{\RaggedRight}p{0.97143\linewidth}|}{{\bfseries \hspace*{0pt}\ignorespaces{}\hspace*{0pt} Entry types in {\ttfamily \setmainfont[Path=/usr/share/fonts/truetype/cmu/,UprightFont=cmunrm.ttf,BoldFont=cmunbx.ttf,ItalicFont=cmunti.ttf,BoldItalicFont=cmunbi.ttf]{cmuntt.ttf}\setmonofont[Path=/usr/share/fonts/truetype/cmu/,UprightFont=cmuntt.ttf,BoldFont=cmuntb.ttf,ItalicFont=cmunit.ttf,BoldItalicFont=cmuntx.ttf]{cmuntt.ttf}\ttfamily .bib}{$\text{ }$}\setmainfont[Path=/usr/share/fonts/truetype/cmu/,UprightFont=cmunrm.ttf,BoldFont=cmunbx.ttf,ItalicFont=cmunti.ttf,BoldItalicFont=cmunbi.ttf]{cmunrm.ttf}\setmonofont[Path=/usr/share/fonts/truetype/cmu/,UprightFont=cmuntt.ttf,BoldFont=cmuntb.ttf,ItalicFont=cmunit.ttf,BoldItalicFont=cmuntx.ttf]{cmunrm.ttf} files known by biblatex and field types supported,\newline{}either required {\ttfamily \setmainfont[Path=/usr/share/fonts/truetype/cmu/,UprightFont=cmunrm.ttf,BoldFont=cmunbx.ttf,ItalicFont=cmunti.ttf,BoldItalicFont=cmunbi.ttf]{cmuntt.ttf}\setmonofont[Path=/usr/share/fonts/truetype/cmu/,UprightFont=cmuntt.ttf,BoldFont=cmuntb.ttf,ItalicFont=cmunit.ttf,BoldItalicFont=cmuntx.ttf]{cmuntt.ttf}\ttfamily +}\setmainfont[Path=/usr/share/fonts/truetype/cmu/,UprightFont=cmunrm.ttf,BoldFont=cmunbx.ttf,ItalicFont=cmunti.ttf,BoldItalicFont=cmunbi.ttf]{cmunrm.ttf}\setmonofont[Path=/usr/share/fonts/truetype/cmu/,UprightFont=cmuntt.ttf,BoldFont=cmuntb.ttf,ItalicFont=cmunit.ttf,BoldItalicFont=cmuntx.ttf]{cmunrm.ttf}, alternatively required {\ttfamily \setmainfont[Path=/usr/share/fonts/truetype/cmu/,UprightFont=cmunrm.ttf,BoldFont=cmunbx.ttf,ItalicFont=cmunti.ttf,BoldItalicFont=cmunbi.ttf]{cmuntt.ttf}\setmonofont[Path=/usr/share/fonts/truetype/cmu/,UprightFont=cmuntt.ttf,BoldFont=cmuntb.ttf,ItalicFont=cmunit.ttf,BoldItalicFont=cmuntx.ttf]{cmuntt.ttf}\ttfamily ±}\setmainfont[Path=/usr/share/fonts/truetype/cmu/,UprightFont=cmunrm.ttf,BoldFont=cmunbx.ttf,ItalicFont=cmunti.ttf,BoldItalicFont=cmunbi.ttf]{cmunrm.ttf}\setmonofont[Path=/usr/share/fonts/truetype/cmu/,UprightFont=cmuntt.ttf,BoldFont=cmuntb.ttf,ItalicFont=cmunit.ttf,BoldItalicFont=cmuntx.ttf]{cmunrm.ttf}, optional {\ttfamily \setmainfont[Path=/usr/share/fonts/truetype/cmu/,UprightFont=cmunrm.ttf,BoldFont=cmunbx.ttf,ItalicFont=cmunti.ttf,BoldItalicFont=cmunbi.ttf]{cmuntt.ttf}\setmonofont[Path=/usr/share/fonts/truetype/cmu/,UprightFont=cmuntt.ttf,BoldFont=cmuntb.ttf,ItalicFont=cmunit.ttf,BoldItalicFont=cmuntx.ttf]{cmuntt.ttf}\ttfamily \^{}}\setmainfont[Path=/usr/share/fonts/truetype/cmu/,UprightFont=cmunrm.ttf,BoldFont=cmunbx.ttf,ItalicFont=cmunti.ttf,BoldItalicFont=cmunbi.ttf]{cmunrm.ttf}\setmonofont[Path=/usr/share/fonts/truetype/cmu/,UprightFont=cmuntt.ttf,BoldFont=cmuntb.ttf,ItalicFont=cmunit.ttf,BoldItalicFont=cmuntx.ttf]{cmunrm.ttf}, not supported (empty) or forbidden {\ttfamily \setmainfont[Path=/usr/share/fonts/truetype/cmu/,UprightFont=cmunrm.ttf,BoldFont=cmunbx.ttf,ItalicFont=cmunti.ttf,BoldItalicFont=cmunbi.ttf]{cmuntt.ttf}\setmonofont[Path=/usr/share/fonts/truetype/cmu/,UprightFont=cmuntt.ttf,BoldFont=cmuntb.ttf,ItalicFont=cmunit.ttf,BoldItalicFont=cmuntx.ttf]{cmuntt.ttf}\ttfamily –}\setmainfont[Path=/usr/share/fonts/truetype/cmu/,UprightFont=cmunrm.ttf,BoldFont=cmunbx.ttf,ItalicFont=cmunti.ttf,BoldItalicFont=cmunbi.ttf]{cmunrm.ttf}\setmonofont[Path=/usr/share/fonts/truetype/cmu/,UprightFont=cmuntt.ttf,BoldFont=cmuntb.ttf,ItalicFont=cmunit.ttf,BoldItalicFont=cmuntx.ttf]{cmunrm.ttf};\newline{}some types have been shortened: dot ‘.’ truncates entry and tilde ‘\~{}’ repeats last full entry}}\\ \hline {\bfseries \hspace*{0pt}\ignorespaces{}\hspace*{0pt}}&{\bfseries \hspace*{0pt}\ignorespaces{}\hspace*{0pt} article}&{\bfseries \hspace*{0pt}\ignorespaces{}\hspace*{0pt} book }&{\bfseries \hspace*{0pt}\ignorespaces{}\hspace*{0pt} mv\~{} }&{\bfseries \hspace*{0pt}\ignorespaces{}\hspace*{0pt} in\~{} }&{\bfseries \hspace*{0pt}\ignorespaces{}\hspace*{0pt} \~{}let}&{\bfseries \hspace*{0pt}\ignorespaces{}\hspace*{0pt} collect. }&{\bfseries \hspace*{0pt}\ignorespaces{}\hspace*{0pt} mv\~{} }&{\bfseries \hspace*{0pt}\ignorespaces{}\hspace*{0pt} in\~{}}&{\bfseries \hspace*{0pt}\ignorespaces{}\hspace*{0pt} manual }&{\bfseries \hspace*{0pt}\ignorespaces{}\hspace*{0pt} misc }&{\bfseries \hspace*{0pt}\ignorespaces{}\hspace*{0pt} online }&{\bfseries \hspace*{0pt}\ignorespaces{}\hspace*{0pt} patent}&{\bfseries \hspace*{0pt}\ignorespaces{}\hspace*{0pt} period.}&{\bfseries \hspace*{0pt}\ignorespaces{}\hspace*{0pt} proceed. }&{\bfseries \hspace*{0pt}\ignorespaces{}\hspace*{0pt} mv\~{} }&{\bfseries \hspace*{0pt}\ignorespaces{}\hspace*{0pt} in\~{}}&{\bfseries \hspace*{0pt}\ignorespaces{}\hspace*{0pt} report}&{\bfseries \hspace*{0pt}\ignorespaces{}\hspace*{0pt} thesis}&{\bfseries \hspace*{0pt}\ignorespaces{}\hspace*{0pt} unpub.}\\ \hline {\bfseries \hspace*{0pt}\ignorespaces{}\hspace*{0pt} {\ttfamily \setmainfont[Path=/usr/share/fonts/truetype/cmu/,UprightFont=cmunrm.ttf,BoldFont=cmunbx.ttf,ItalicFont=cmunti.ttf,BoldItalicFont=cmunbi.ttf]{cmuntt.ttf}\setmonofont[Path=/usr/share/fonts/truetype/cmu/,UprightFont=cmuntt.ttf,BoldFont=cmuntb.ttf,ItalicFont=cmunit.ttf,BoldItalicFont=cmuntx.ttf]{cmuntt.ttf}\ttfamily author}\setmainfont[Path=/usr/share/fonts/truetype/cmu/,UprightFont=cmunrm.ttf,BoldFont=cmunbx.ttf,ItalicFont=cmunti.ttf,BoldItalicFont=cmunbi.ttf]{cmunrm.ttf}\setmonofont[Path=/usr/share/fonts/truetype/cmu/,UprightFont=cmuntt.ttf,BoldFont=cmuntb.ttf,ItalicFont=cmunit.ttf,BoldItalicFont=cmuntx.ttf]{cmunrm.ttf}, {\ttfamily \setmainfont[Path=/usr/share/fonts/truetype/cmu/,UprightFont=cmunrm.ttf,BoldFont=cmunbx.ttf,ItalicFont=cmunti.ttf,BoldItalicFont=cmunbi.ttf]{cmuntt.ttf}\setmonofont[Path=/usr/share/fonts/truetype/cmu/,UprightFont=cmuntt.ttf,BoldFont=cmuntb.ttf,ItalicFont=cmunit.ttf,BoldItalicFont=cmuntx.ttf]{cmuntt.ttf}\ttfamily authortype}}&\hspace*{0pt}\ignorespaces{}\hspace*{0pt}{$\text{ }$}\setmainfont[Path=/usr/share/fonts/truetype/cmu/,UprightFont=cmunrm.ttf,BoldFont=cmunbx.ttf,ItalicFont=cmunti.ttf,BoldItalicFont=cmunbi.ttf]{cmunrm.ttf}\setmonofont[Path=/usr/share/fonts/truetype/cmu/,UprightFont=cmuntt.ttf,BoldFont=cmuntb.ttf,ItalicFont=cmunit.ttf,BoldItalicFont=cmuntx.ttf]{cmunrm.ttf} + &\hspace*{0pt}\ignorespaces{}\hspace*{0pt} + &\hspace*{0pt}\ignorespaces{}\hspace*{0pt} + &\hspace*{0pt}\ignorespaces{}\hspace*{0pt} + &\hspace*{0pt}\ignorespaces{}\hspace*{0pt} ± &\hspace*{0pt}\ignorespaces{}\hspace*{0pt} – &\hspace*{0pt}\ignorespaces{}\hspace*{0pt} – &\hspace*{0pt}\ignorespaces{}\hspace*{0pt} + &\hspace*{0pt}\ignorespaces{}\hspace*{0pt} ± &\hspace*{0pt}\ignorespaces{}\hspace*{0pt} ± &\hspace*{0pt}\ignorespaces{}\hspace*{0pt} ± &\hspace*{0pt}\ignorespaces{}\hspace*{0pt} + &\hspace*{0pt}\ignorespaces{}\hspace*{0pt} – &\hspace*{0pt}\ignorespaces{}\hspace*{0pt} – &\hspace*{0pt}\ignorespaces{}\hspace*{0pt} – &\hspace*{0pt}\ignorespaces{}\hspace*{0pt} + &\hspace*{0pt}\ignorespaces{}\hspace*{0pt} + &\hspace*{0pt}\ignorespaces{}\hspace*{0pt} + &\hspace*{0pt}\ignorespaces{}\hspace*{0pt} +\\ \hline {\bfseries \hspace*{0pt}\ignorespaces{}\hspace*{0pt} {\ttfamily \setmainfont[Path=/usr/share/fonts/truetype/cmu/,UprightFont=cmunrm.ttf,BoldFont=cmunbx.ttf,ItalicFont=cmunti.ttf,BoldItalicFont=cmunbi.ttf]{cmuntt.ttf}\setmonofont[Path=/usr/share/fonts/truetype/cmu/,UprightFont=cmuntt.ttf,BoldFont=cmuntb.ttf,ItalicFont=cmunit.ttf,BoldItalicFont=cmuntx.ttf]{cmuntt.ttf}\ttfamily editor}\setmainfont[Path=/usr/share/fonts/truetype/cmu/,UprightFont=cmunrm.ttf,BoldFont=cmunbx.ttf,ItalicFont=cmunti.ttf,BoldItalicFont=cmunbi.ttf]{cmunrm.ttf}\setmonofont[Path=/usr/share/fonts/truetype/cmu/,UprightFont=cmuntt.ttf,BoldFont=cmuntb.ttf,ItalicFont=cmunit.ttf,BoldItalicFont=cmuntx.ttf]{cmunrm.ttf},  {\ttfamily \setmainfont[Path=/usr/share/fonts/truetype/cmu/,UprightFont=cmunrm.ttf,BoldFont=cmunbx.ttf,ItalicFont=cmunti.ttf,BoldItalicFont=cmunbi.ttf]{cmuntt.ttf}\setmonofont[Path=/usr/share/fonts/truetype/cmu/,UprightFont=cmuntt.ttf,BoldFont=cmuntb.ttf,ItalicFont=cmunit.ttf,BoldItalicFont=cmuntx.ttf]{cmuntt.ttf}\ttfamily editortype}}&\hspace*{0pt}\ignorespaces{}\hspace*{0pt}{$\text{ }$}\setmainfont[Path=/usr/share/fonts/truetype/cmu/,UprightFont=cmunrm.ttf,BoldFont=cmunbx.ttf,ItalicFont=cmunti.ttf,BoldItalicFont=cmunbi.ttf]{cmunrm.ttf}\setmonofont[Path=/usr/share/fonts/truetype/cmu/,UprightFont=cmuntt.ttf,BoldFont=cmuntb.ttf,ItalicFont=cmunit.ttf,BoldItalicFont=cmuntx.ttf]{cmunrm.ttf} \^{} &\hspace*{0pt}\ignorespaces{}\hspace*{0pt} \^{} &\hspace*{0pt}\ignorespaces{}\hspace*{0pt} \^{} &\hspace*{0pt}\ignorespaces{}\hspace*{0pt} \^{} &\hspace*{0pt}\ignorespaces{}\hspace*{0pt} ± &\hspace*{0pt}\ignorespaces{}\hspace*{0pt} + &\hspace*{0pt}\ignorespaces{}\hspace*{0pt} + &\hspace*{0pt}\ignorespaces{}\hspace*{0pt} + &\hspace*{0pt}\ignorespaces{}\hspace*{0pt} ± &\hspace*{0pt}\ignorespaces{}\hspace*{0pt} ± &\hspace*{0pt}\ignorespaces{}\hspace*{0pt} ± &\hspace*{0pt}\ignorespaces{}\hspace*{0pt}   &\hspace*{0pt}\ignorespaces{}\hspace*{0pt} + &\hspace*{0pt}\ignorespaces{}\hspace*{0pt} + &\hspace*{0pt}\ignorespaces{}\hspace*{0pt} + &\hspace*{0pt}\ignorespaces{}\hspace*{0pt} + &\hspace*{0pt}\ignorespaces{}\hspace*{0pt}   &\hspace*{0pt}\ignorespaces{}\hspace*{0pt}   &\hspace*{0pt}\ignorespaces{}\hspace*{0pt}\\ \hline {\bfseries \hspace*{0pt}\ignorespaces{}\hspace*{0pt} {\ttfamily \setmainfont[Path=/usr/share/fonts/truetype/cmu/,UprightFont=cmunrm.ttf,BoldFont=cmunbx.ttf,ItalicFont=cmunti.ttf,BoldItalicFont=cmunbi.ttf]{cmuntt.ttf}\setmonofont[Path=/usr/share/fonts/truetype/cmu/,UprightFont=cmuntt.ttf,BoldFont=cmuntb.ttf,ItalicFont=cmunit.ttf,BoldItalicFont=cmuntx.ttf]{cmuntt.ttf}\ttfamily editor{\itshape \setmainfont[Path=/usr/share/fonts/truetype/cmu/,UprightFont=cmunrm.ttf,BoldFont=cmunbx.ttf,ItalicFont=cmunti.ttf,BoldItalicFont=cmunbi.ttf]{cmunit.ttf}\setmonofont[Path=/usr/share/fonts/truetype/cmu/,UprightFont=cmuntt.ttf,BoldFont=cmuntb.ttf,ItalicFont=cmunit.ttf,BoldItalicFont=cmuntx.ttf]{cmunit.ttf}\ttfamily \itshape X}}\setmainfont[Path=/usr/share/fonts/truetype/cmu/,UprightFont=cmunrm.ttf,BoldFont=cmunbx.ttf,ItalicFont=cmunti.ttf,BoldItalicFont=cmunbi.ttf]{cmunrm.ttf}\setmonofont[Path=/usr/share/fonts/truetype/cmu/,UprightFont=cmuntt.ttf,BoldFont=cmuntb.ttf,ItalicFont=cmunit.ttf,BoldItalicFont=cmuntx.ttf]{cmunrm.ttf},  {\ttfamily \setmainfont[Path=/usr/share/fonts/truetype/cmu/,UprightFont=cmunrm.ttf,BoldFont=cmunbx.ttf,ItalicFont=cmunti.ttf,BoldItalicFont=cmunbi.ttf]{cmuntt.ttf}\setmonofont[Path=/usr/share/fonts/truetype/cmu/,UprightFont=cmuntt.ttf,BoldFont=cmuntb.ttf,ItalicFont=cmunit.ttf,BoldItalicFont=cmuntx.ttf]{cmuntt.ttf}\ttfamily editor{\itshape \setmainfont[Path=/usr/share/fonts/truetype/cmu/,UprightFont=cmunrm.ttf,BoldFont=cmunbx.ttf,ItalicFont=cmunti.ttf,BoldItalicFont=cmunbi.ttf]{cmunit.ttf}\setmonofont[Path=/usr/share/fonts/truetype/cmu/,UprightFont=cmuntt.ttf,BoldFont=cmuntb.ttf,ItalicFont=cmunit.ttf,BoldItalicFont=cmuntx.ttf]{cmunit.ttf}\ttfamily \itshape X}\setmainfont[Path=/usr/share/fonts/truetype/cmu/,UprightFont=cmunrm.ttf,BoldFont=cmunbx.ttf,ItalicFont=cmunti.ttf,BoldItalicFont=cmunbi.ttf]{cmuntt.ttf}\setmonofont[Path=/usr/share/fonts/truetype/cmu/,UprightFont=cmuntt.ttf,BoldFont=cmuntb.ttf,ItalicFont=cmunit.ttf,BoldItalicFont=cmuntx.ttf]{cmuntt.ttf}\ttfamily type}}&\hspace*{0pt}\ignorespaces{}\hspace*{0pt}{$\text{ }$}\setmainfont[Path=/usr/share/fonts/truetype/cmu/,UprightFont=cmunrm.ttf,BoldFont=cmunbx.ttf,ItalicFont=cmunti.ttf,BoldItalicFont=cmunbi.ttf]{cmunrm.ttf}\setmonofont[Path=/usr/share/fonts/truetype/cmu/,UprightFont=cmuntt.ttf,BoldFont=cmuntb.ttf,ItalicFont=cmunit.ttf,BoldItalicFont=cmuntx.ttf]{cmunrm.ttf} \^{} &\hspace*{0pt}\ignorespaces{}\hspace*{0pt} \^{} &\hspace*{0pt}\ignorespaces{}\hspace*{0pt} \^{} &\hspace*{0pt}\ignorespaces{}\hspace*{0pt} \^{} &\hspace*{0pt}\ignorespaces{}\hspace*{0pt}   &\hspace*{0pt}\ignorespaces{}\hspace*{0pt} \^{} &\hspace*{0pt}\ignorespaces{}\hspace*{0pt} \^{} &\hspace*{0pt}\ignorespaces{}\hspace*{0pt} \^{} &\hspace*{0pt}\ignorespaces{}\hspace*{0pt}   &\hspace*{0pt}\ignorespaces{}\hspace*{0pt}   &\hspace*{0pt}\ignorespaces{}\hspace*{0pt}   &\hspace*{0pt}\ignorespaces{}\hspace*{0pt}   &\hspace*{0pt}\ignorespaces{}\hspace*{0pt} \^{} &\hspace*{0pt}\ignorespaces{}\hspace*{0pt}   &\hspace*{0pt}\ignorespaces{}\hspace*{0pt}   &\hspace*{0pt}\ignorespaces{}\hspace*{0pt}   &\hspace*{0pt}\ignorespaces{}\hspace*{0pt}   &\hspace*{0pt}\ignorespaces{}\hspace*{0pt}   &\hspace*{0pt}\ignorespaces{}\hspace*{0pt}\\ \hline {\bfseries \hspace*{0pt}\ignorespaces{}\hspace*{0pt} {\ttfamily \setmainfont[Path=/usr/share/fonts/truetype/cmu/,UprightFont=cmunrm.ttf,BoldFont=cmunbx.ttf,ItalicFont=cmunti.ttf,BoldItalicFont=cmunbi.ttf]{cmuntt.ttf}\setmonofont[Path=/usr/share/fonts/truetype/cmu/,UprightFont=cmuntt.ttf,BoldFont=cmuntb.ttf,ItalicFont=cmunit.ttf,BoldItalicFont=cmuntx.ttf]{cmuntt.ttf}\ttfamily holder}}&\hspace*{0pt}\ignorespaces{}\hspace*{0pt}{$\text{ }$}\setmainfont[Path=/usr/share/fonts/truetype/cmu/,UprightFont=cmunrm.ttf,BoldFont=cmunbx.ttf,ItalicFont=cmunti.ttf,BoldItalicFont=cmunbi.ttf]{cmunrm.ttf}\setmonofont[Path=/usr/share/fonts/truetype/cmu/,UprightFont=cmuntt.ttf,BoldFont=cmuntb.ttf,ItalicFont=cmunit.ttf,BoldItalicFont=cmuntx.ttf]{cmunrm.ttf}   &\hspace*{0pt}\ignorespaces{}\hspace*{0pt}   &\hspace*{0pt}\ignorespaces{}\hspace*{0pt}   &\hspace*{0pt}\ignorespaces{}\hspace*{0pt}   &\hspace*{0pt}\ignorespaces{}\hspace*{0pt}   &\hspace*{0pt}\ignorespaces{}\hspace*{0pt}   &\hspace*{0pt}\ignorespaces{}\hspace*{0pt}   &\hspace*{0pt}\ignorespaces{}\hspace*{0pt}   &\hspace*{0pt}\ignorespaces{}\hspace*{0pt}   &\hspace*{0pt}\ignorespaces{}\hspace*{0pt}   &\hspace*{0pt}\ignorespaces{}\hspace*{0pt}   &\hspace*{0pt}\ignorespaces{}\hspace*{0pt} \^{} &\hspace*{0pt}\ignorespaces{}\hspace*{0pt}   &\hspace*{0pt}\ignorespaces{}\hspace*{0pt}   &\hspace*{0pt}\ignorespaces{}\hspace*{0pt}   &\hspace*{0pt}\ignorespaces{}\hspace*{0pt}   &\hspace*{0pt}\ignorespaces{}\hspace*{0pt}   &\hspace*{0pt}\ignorespaces{}\hspace*{0pt}   &\hspace*{0pt}\ignorespaces{}\hspace*{0pt}\\ \hline {\bfseries \hspace*{0pt}\ignorespaces{}\hspace*{0pt} {\ttfamily \setmainfont[Path=/usr/share/fonts/truetype/cmu/,UprightFont=cmunrm.ttf,BoldFont=cmunbx.ttf,ItalicFont=cmunti.ttf,BoldItalicFont=cmunbi.ttf]{cmuntt.ttf}\setmonofont[Path=/usr/share/fonts/truetype/cmu/,UprightFont=cmuntt.ttf,BoldFont=cmuntb.ttf,ItalicFont=cmunit.ttf,BoldItalicFont=cmuntx.ttf]{cmuntt.ttf}\ttfamily bookauthor}}&\hspace*{0pt}\ignorespaces{}\hspace*{0pt}{$\text{ }$}\setmainfont[Path=/usr/share/fonts/truetype/cmu/,UprightFont=cmunrm.ttf,BoldFont=cmunbx.ttf,ItalicFont=cmunti.ttf,BoldItalicFont=cmunbi.ttf]{cmunrm.ttf}\setmonofont[Path=/usr/share/fonts/truetype/cmu/,UprightFont=cmuntt.ttf,BoldFont=cmuntb.ttf,ItalicFont=cmunit.ttf,BoldItalicFont=cmuntx.ttf]{cmunrm.ttf}   &\hspace*{0pt}\ignorespaces{}\hspace*{0pt}   &\hspace*{0pt}\ignorespaces{}\hspace*{0pt}   &\hspace*{0pt}\ignorespaces{}\hspace*{0pt} \^{} &\hspace*{0pt}\ignorespaces{}\hspace*{0pt}   &\hspace*{0pt}\ignorespaces{}\hspace*{0pt}   &\hspace*{0pt}\ignorespaces{}\hspace*{0pt}   &\hspace*{0pt}\ignorespaces{}\hspace*{0pt}   &\hspace*{0pt}\ignorespaces{}\hspace*{0pt}   &\hspace*{0pt}\ignorespaces{}\hspace*{0pt}   &\hspace*{0pt}\ignorespaces{}\hspace*{0pt}   &\hspace*{0pt}\ignorespaces{}\hspace*{0pt}   &\hspace*{0pt}\ignorespaces{}\hspace*{0pt}   &\hspace*{0pt}\ignorespaces{}\hspace*{0pt}   &\hspace*{0pt}\ignorespaces{}\hspace*{0pt}   &\hspace*{0pt}\ignorespaces{}\hspace*{0pt}   &\hspace*{0pt}\ignorespaces{}\hspace*{0pt}   &\hspace*{0pt}\ignorespaces{}\hspace*{0pt}   &\hspace*{0pt}\ignorespaces{}\hspace*{0pt}\\ \hline {\bfseries \hspace*{0pt}\ignorespaces{}\hspace*{0pt} {\ttfamily \setmainfont[Path=/usr/share/fonts/truetype/cmu/,UprightFont=cmunrm.ttf,BoldFont=cmunbx.ttf,ItalicFont=cmunti.ttf,BoldItalicFont=cmunbi.ttf]{cmuntt.ttf}\setmonofont[Path=/usr/share/fonts/truetype/cmu/,UprightFont=cmuntt.ttf,BoldFont=cmuntb.ttf,ItalicFont=cmunit.ttf,BoldItalicFont=cmuntx.ttf]{cmuntt.ttf}\ttfamily annotator}\setmainfont[Path=/usr/share/fonts/truetype/cmu/,UprightFont=cmunrm.ttf,BoldFont=cmunbx.ttf,ItalicFont=cmunti.ttf,BoldItalicFont=cmunbi.ttf]{cmunrm.ttf}\setmonofont[Path=/usr/share/fonts/truetype/cmu/,UprightFont=cmuntt.ttf,BoldFont=cmuntb.ttf,ItalicFont=cmunit.ttf,BoldItalicFont=cmuntx.ttf]{cmunrm.ttf}, {\ttfamily \setmainfont[Path=/usr/share/fonts/truetype/cmu/,UprightFont=cmunrm.ttf,BoldFont=cmunbx.ttf,ItalicFont=cmunti.ttf,BoldItalicFont=cmunbi.ttf]{cmuntt.ttf}\setmonofont[Path=/usr/share/fonts/truetype/cmu/,UprightFont=cmuntt.ttf,BoldFont=cmuntb.ttf,ItalicFont=cmunit.ttf,BoldItalicFont=cmuntx.ttf]{cmuntt.ttf}\ttfamily commentator}}&\hspace*{0pt}\ignorespaces{}\hspace*{0pt}{$\text{ }$}\setmainfont[Path=/usr/share/fonts/truetype/cmu/,UprightFont=cmunrm.ttf,BoldFont=cmunbx.ttf,ItalicFont=cmunti.ttf,BoldItalicFont=cmunbi.ttf]{cmunrm.ttf}\setmonofont[Path=/usr/share/fonts/truetype/cmu/,UprightFont=cmuntt.ttf,BoldFont=cmuntb.ttf,ItalicFont=cmunit.ttf,BoldItalicFont=cmuntx.ttf]{cmunrm.ttf} \^{} &\hspace*{0pt}\ignorespaces{}\hspace*{0pt} \^{} &\hspace*{0pt}\ignorespaces{}\hspace*{0pt} \^{} &\hspace*{0pt}\ignorespaces{}\hspace*{0pt} \^{} &\hspace*{0pt}\ignorespaces{}\hspace*{0pt}   &\hspace*{0pt}\ignorespaces{}\hspace*{0pt} \^{} &\hspace*{0pt}\ignorespaces{}\hspace*{0pt} \^{} &\hspace*{0pt}\ignorespaces{}\hspace*{0pt} \^{} &\hspace*{0pt}\ignorespaces{}\hspace*{0pt}   &\hspace*{0pt}\ignorespaces{}\hspace*{0pt}   &\hspace*{0pt}\ignorespaces{}\hspace*{0pt}   &\hspace*{0pt}\ignorespaces{}\hspace*{0pt}   &\hspace*{0pt}\ignorespaces{}\hspace*{0pt}   &\hspace*{0pt}\ignorespaces{}\hspace*{0pt}   &\hspace*{0pt}\ignorespaces{}\hspace*{0pt}   &\hspace*{0pt}\ignorespaces{}\hspace*{0pt}   &\hspace*{0pt}\ignorespaces{}\hspace*{0pt}   &\hspace*{0pt}\ignorespaces{}\hspace*{0pt}   &\hspace*{0pt}\ignorespaces{}\hspace*{0pt}\\ \hline {\bfseries \hspace*{0pt}\ignorespaces{}\hspace*{0pt} {\ttfamily \setmainfont[Path=/usr/share/fonts/truetype/cmu/,UprightFont=cmunrm.ttf,BoldFont=cmunbx.ttf,ItalicFont=cmunti.ttf,BoldItalicFont=cmunbi.ttf]{cmuntt.ttf}\setmonofont[Path=/usr/share/fonts/truetype/cmu/,UprightFont=cmuntt.ttf,BoldFont=cmuntb.ttf,ItalicFont=cmunit.ttf,BoldItalicFont=cmuntx.ttf]{cmuntt.ttf}\ttfamily translator}\setmainfont[Path=/usr/share/fonts/truetype/cmu/,UprightFont=cmunrm.ttf,BoldFont=cmunbx.ttf,ItalicFont=cmunti.ttf,BoldItalicFont=cmunbi.ttf]{cmunrm.ttf}\setmonofont[Path=/usr/share/fonts/truetype/cmu/,UprightFont=cmuntt.ttf,BoldFont=cmuntb.ttf,ItalicFont=cmunit.ttf,BoldItalicFont=cmuntx.ttf]{cmunrm.ttf}, {\ttfamily \setmainfont[Path=/usr/share/fonts/truetype/cmu/,UprightFont=cmunrm.ttf,BoldFont=cmunbx.ttf,ItalicFont=cmunti.ttf,BoldItalicFont=cmunbi.ttf]{cmuntt.ttf}\setmonofont[Path=/usr/share/fonts/truetype/cmu/,UprightFont=cmuntt.ttf,BoldFont=cmuntb.ttf,ItalicFont=cmunit.ttf,BoldItalicFont=cmuntx.ttf]{cmuntt.ttf}\ttfamily origlanguage}}&\hspace*{0pt}\ignorespaces{}\hspace*{0pt}{$\text{ }$}\setmainfont[Path=/usr/share/fonts/truetype/cmu/,UprightFont=cmunrm.ttf,BoldFont=cmunbx.ttf,ItalicFont=cmunti.ttf,BoldItalicFont=cmunbi.ttf]{cmunrm.ttf}\setmonofont[Path=/usr/share/fonts/truetype/cmu/,UprightFont=cmuntt.ttf,BoldFont=cmuntb.ttf,ItalicFont=cmunit.ttf,BoldItalicFont=cmuntx.ttf]{cmunrm.ttf} \^{} &\hspace*{0pt}\ignorespaces{}\hspace*{0pt} \^{} &\hspace*{0pt}\ignorespaces{}\hspace*{0pt} \^{} &\hspace*{0pt}\ignorespaces{}\hspace*{0pt} \^{} &\hspace*{0pt}\ignorespaces{}\hspace*{0pt}   &\hspace*{0pt}\ignorespaces{}\hspace*{0pt} \^{} &\hspace*{0pt}\ignorespaces{}\hspace*{0pt} \^{} &\hspace*{0pt}\ignorespaces{}\hspace*{0pt} \^{} &\hspace*{0pt}\ignorespaces{}\hspace*{0pt}   &\hspace*{0pt}\ignorespaces{}\hspace*{0pt}   &\hspace*{0pt}\ignorespaces{}\hspace*{0pt}   &\hspace*{0pt}\ignorespaces{}\hspace*{0pt}   &\hspace*{0pt}\ignorespaces{}\hspace*{0pt}   &\hspace*{0pt}\ignorespaces{}\hspace*{0pt}   &\hspace*{0pt}\ignorespaces{}\hspace*{0pt}   &\hspace*{0pt}\ignorespaces{}\hspace*{0pt}   &\hspace*{0pt}\ignorespaces{}\hspace*{0pt}   &\hspace*{0pt}\ignorespaces{}\hspace*{0pt}   &\hspace*{0pt}\ignorespaces{}\hspace*{0pt}\\ \hline {\bfseries \hspace*{0pt}\ignorespaces{}\hspace*{0pt} {\ttfamily \setmainfont[Path=/usr/share/fonts/truetype/cmu/,UprightFont=cmunrm.ttf,BoldFont=cmunbx.ttf,ItalicFont=cmunti.ttf,BoldItalicFont=cmunbi.ttf]{cmuntt.ttf}\setmonofont[Path=/usr/share/fonts/truetype/cmu/,UprightFont=cmuntt.ttf,BoldFont=cmuntb.ttf,ItalicFont=cmunit.ttf,BoldItalicFont=cmuntx.ttf]{cmuntt.ttf}\ttfamily afterword}\setmainfont[Path=/usr/share/fonts/truetype/cmu/,UprightFont=cmunrm.ttf,BoldFont=cmunbx.ttf,ItalicFont=cmunti.ttf,BoldItalicFont=cmunbi.ttf]{cmunrm.ttf}\setmonofont[Path=/usr/share/fonts/truetype/cmu/,UprightFont=cmuntt.ttf,BoldFont=cmuntb.ttf,ItalicFont=cmunit.ttf,BoldItalicFont=cmuntx.ttf]{cmunrm.ttf}, {\ttfamily \setmainfont[Path=/usr/share/fonts/truetype/cmu/,UprightFont=cmunrm.ttf,BoldFont=cmunbx.ttf,ItalicFont=cmunti.ttf,BoldItalicFont=cmunbi.ttf]{cmuntt.ttf}\setmonofont[Path=/usr/share/fonts/truetype/cmu/,UprightFont=cmuntt.ttf,BoldFont=cmuntb.ttf,ItalicFont=cmunit.ttf,BoldItalicFont=cmuntx.ttf]{cmuntt.ttf}\ttfamily foreword}\setmainfont[Path=/usr/share/fonts/truetype/cmu/,UprightFont=cmunrm.ttf,BoldFont=cmunbx.ttf,ItalicFont=cmunti.ttf,BoldItalicFont=cmunbi.ttf]{cmunrm.ttf}\setmonofont[Path=/usr/share/fonts/truetype/cmu/,UprightFont=cmuntt.ttf,BoldFont=cmuntb.ttf,ItalicFont=cmunit.ttf,BoldItalicFont=cmuntx.ttf]{cmunrm.ttf}, {\ttfamily \setmainfont[Path=/usr/share/fonts/truetype/cmu/,UprightFont=cmunrm.ttf,BoldFont=cmunbx.ttf,ItalicFont=cmunti.ttf,BoldItalicFont=cmunbi.ttf]{cmuntt.ttf}\setmonofont[Path=/usr/share/fonts/truetype/cmu/,UprightFont=cmuntt.ttf,BoldFont=cmuntb.ttf,ItalicFont=cmunit.ttf,BoldItalicFont=cmuntx.ttf]{cmuntt.ttf}\ttfamily introduction}}&\hspace*{0pt}\ignorespaces{}\hspace*{0pt}{$\text{ }$}\setmainfont[Path=/usr/share/fonts/truetype/cmu/,UprightFont=cmunrm.ttf,BoldFont=cmunbx.ttf,ItalicFont=cmunti.ttf,BoldItalicFont=cmunbi.ttf]{cmunrm.ttf}\setmonofont[Path=/usr/share/fonts/truetype/cmu/,UprightFont=cmuntt.ttf,BoldFont=cmuntb.ttf,ItalicFont=cmunit.ttf,BoldItalicFont=cmuntx.ttf]{cmunrm.ttf}   &\hspace*{0pt}\ignorespaces{}\hspace*{0pt} \^{} &\hspace*{0pt}\ignorespaces{}\hspace*{0pt} \^{} &\hspace*{0pt}\ignorespaces{}\hspace*{0pt} \^{} &\hspace*{0pt}\ignorespaces{}\hspace*{0pt}   &\hspace*{0pt}\ignorespaces{}\hspace*{0pt} \^{} &\hspace*{0pt}\ignorespaces{}\hspace*{0pt} \^{} &\hspace*{0pt}\ignorespaces{}\hspace*{0pt} \^{} &\hspace*{0pt}\ignorespaces{}\hspace*{0pt}   &\hspace*{0pt}\ignorespaces{}\hspace*{0pt}   &\hspace*{0pt}\ignorespaces{}\hspace*{0pt}   &\hspace*{0pt}\ignorespaces{}\hspace*{0pt}   &\hspace*{0pt}\ignorespaces{}\hspace*{0pt}   &\hspace*{0pt}\ignorespaces{}\hspace*{0pt}   &\hspace*{0pt}\ignorespaces{}\hspace*{0pt}   &\hspace*{0pt}\ignorespaces{}\hspace*{0pt}   &\hspace*{0pt}\ignorespaces{}\hspace*{0pt}   &\hspace*{0pt}\ignorespaces{}\hspace*{0pt}   &\hspace*{0pt}\ignorespaces{}\hspace*{0pt}\\ \hline {\bfseries \hspace*{0pt}\ignorespaces{}\hspace*{0pt} {\ttfamily \setmainfont[Path=/usr/share/fonts/truetype/cmu/,UprightFont=cmunrm.ttf,BoldFont=cmunbx.ttf,ItalicFont=cmunti.ttf,BoldItalicFont=cmunbi.ttf]{cmuntt.ttf}\setmonofont[Path=/usr/share/fonts/truetype/cmu/,UprightFont=cmuntt.ttf,BoldFont=cmuntb.ttf,ItalicFont=cmunit.ttf,BoldItalicFont=cmuntx.ttf]{cmuntt.ttf}\ttfamily title}}&\hspace*{0pt}\ignorespaces{}\hspace*{0pt}{$\text{ }$}\setmainfont[Path=/usr/share/fonts/truetype/cmu/,UprightFont=cmunrm.ttf,BoldFont=cmunbx.ttf,ItalicFont=cmunti.ttf,BoldItalicFont=cmunbi.ttf]{cmunrm.ttf}\setmonofont[Path=/usr/share/fonts/truetype/cmu/,UprightFont=cmuntt.ttf,BoldFont=cmuntb.ttf,ItalicFont=cmunit.ttf,BoldItalicFont=cmuntx.ttf]{cmunrm.ttf} + &\hspace*{0pt}\ignorespaces{}\hspace*{0pt} + &\hspace*{0pt}\ignorespaces{}\hspace*{0pt} + &\hspace*{0pt}\ignorespaces{}\hspace*{0pt} + &\hspace*{0pt}\ignorespaces{}\hspace*{0pt} + &\hspace*{0pt}\ignorespaces{}\hspace*{0pt} + &\hspace*{0pt}\ignorespaces{}\hspace*{0pt} + &\hspace*{0pt}\ignorespaces{}\hspace*{0pt} + &\hspace*{0pt}\ignorespaces{}\hspace*{0pt} + &\hspace*{0pt}\ignorespaces{}\hspace*{0pt} + &\hspace*{0pt}\ignorespaces{}\hspace*{0pt} + &\hspace*{0pt}\ignorespaces{}\hspace*{0pt} + &\hspace*{0pt}\ignorespaces{}\hspace*{0pt} + &\hspace*{0pt}\ignorespaces{}\hspace*{0pt} + &\hspace*{0pt}\ignorespaces{}\hspace*{0pt} + &\hspace*{0pt}\ignorespaces{}\hspace*{0pt} + &\hspace*{0pt}\ignorespaces{}\hspace*{0pt} + &\hspace*{0pt}\ignorespaces{}\hspace*{0pt} + &\hspace*{0pt}\ignorespaces{}\hspace*{0pt} +\\ \hline {\bfseries \hspace*{0pt}\ignorespaces{}\hspace*{0pt} {\ttfamily \setmainfont[Path=/usr/share/fonts/truetype/cmu/,UprightFont=cmunrm.ttf,BoldFont=cmunbx.ttf,ItalicFont=cmunti.ttf,BoldItalicFont=cmunbi.ttf]{cmuntt.ttf}\setmonofont[Path=/usr/share/fonts/truetype/cmu/,UprightFont=cmuntt.ttf,BoldFont=cmuntb.ttf,ItalicFont=cmunit.ttf,BoldItalicFont=cmuntx.ttf]{cmuntt.ttf}\ttfamily titleaddon}\setmainfont[Path=/usr/share/fonts/truetype/cmu/,UprightFont=cmunrm.ttf,BoldFont=cmunbx.ttf,ItalicFont=cmunti.ttf,BoldItalicFont=cmunbi.ttf]{cmunrm.ttf}\setmonofont[Path=/usr/share/fonts/truetype/cmu/,UprightFont=cmuntt.ttf,BoldFont=cmuntb.ttf,ItalicFont=cmunit.ttf,BoldItalicFont=cmuntx.ttf]{cmunrm.ttf}, {\ttfamily \setmainfont[Path=/usr/share/fonts/truetype/cmu/,UprightFont=cmunrm.ttf,BoldFont=cmunbx.ttf,ItalicFont=cmunti.ttf,BoldItalicFont=cmunbi.ttf]{cmuntt.ttf}\setmonofont[Path=/usr/share/fonts/truetype/cmu/,UprightFont=cmuntt.ttf,BoldFont=cmuntb.ttf,ItalicFont=cmunit.ttf,BoldItalicFont=cmuntx.ttf]{cmuntt.ttf}\ttfamily subtitle}}&\hspace*{0pt}\ignorespaces{}\hspace*{0pt}{$\text{ }$}\setmainfont[Path=/usr/share/fonts/truetype/cmu/,UprightFont=cmunrm.ttf,BoldFont=cmunbx.ttf,ItalicFont=cmunti.ttf,BoldItalicFont=cmunbi.ttf]{cmunrm.ttf}\setmonofont[Path=/usr/share/fonts/truetype/cmu/,UprightFont=cmuntt.ttf,BoldFont=cmuntb.ttf,ItalicFont=cmunit.ttf,BoldItalicFont=cmuntx.ttf]{cmunrm.ttf} \^{} &\hspace*{0pt}\ignorespaces{}\hspace*{0pt} \^{} &\hspace*{0pt}\ignorespaces{}\hspace*{0pt} \^{} &\hspace*{0pt}\ignorespaces{}\hspace*{0pt} \^{} &\hspace*{0pt}\ignorespaces{}\hspace*{0pt} \^{} &\hspace*{0pt}\ignorespaces{}\hspace*{0pt} \^{} &\hspace*{0pt}\ignorespaces{}\hspace*{0pt} \^{} &\hspace*{0pt}\ignorespaces{}\hspace*{0pt} \^{} &\hspace*{0pt}\ignorespaces{}\hspace*{0pt} \^{} &\hspace*{0pt}\ignorespaces{}\hspace*{0pt} \^{} &\hspace*{0pt}\ignorespaces{}\hspace*{0pt} \^{} &\hspace*{0pt}\ignorespaces{}\hspace*{0pt} \^{} &\hspace*{0pt}\ignorespaces{}\hspace*{0pt} ˇ &\hspace*{0pt}\ignorespaces{}\hspace*{0pt} \^{} &\hspace*{0pt}\ignorespaces{}\hspace*{0pt} \^{} &\hspace*{0pt}\ignorespaces{}\hspace*{0pt} \^{} &\hspace*{0pt}\ignorespaces{}\hspace*{0pt} \^{} &\hspace*{0pt}\ignorespaces{}\hspace*{0pt} \^{} &\hspace*{0pt}\ignorespaces{}\hspace*{0pt} \^{}\\ \hline {\bfseries \hspace*{0pt}\ignorespaces{}\hspace*{0pt} {\ttfamily \setmainfont[Path=/usr/share/fonts/truetype/cmu/,UprightFont=cmunrm.ttf,BoldFont=cmunbx.ttf,ItalicFont=cmunti.ttf,BoldItalicFont=cmunbi.ttf]{cmuntt.ttf}\setmonofont[Path=/usr/share/fonts/truetype/cmu/,UprightFont=cmuntt.ttf,BoldFont=cmuntb.ttf,ItalicFont=cmunit.ttf,BoldItalicFont=cmuntx.ttf]{cmuntt.ttf}\ttfamily maintitle}\setmainfont[Path=/usr/share/fonts/truetype/cmu/,UprightFont=cmunrm.ttf,BoldFont=cmunbx.ttf,ItalicFont=cmunti.ttf,BoldItalicFont=cmunbi.ttf]{cmunrm.ttf}\setmonofont[Path=/usr/share/fonts/truetype/cmu/,UprightFont=cmuntt.ttf,BoldFont=cmuntb.ttf,ItalicFont=cmunit.ttf,BoldItalicFont=cmuntx.ttf]{cmunrm.ttf}, {\ttfamily \setmainfont[Path=/usr/share/fonts/truetype/cmu/,UprightFont=cmunrm.ttf,BoldFont=cmunbx.ttf,ItalicFont=cmunti.ttf,BoldItalicFont=cmunbi.ttf]{cmuntt.ttf}\setmonofont[Path=/usr/share/fonts/truetype/cmu/,UprightFont=cmuntt.ttf,BoldFont=cmuntb.ttf,ItalicFont=cmunit.ttf,BoldItalicFont=cmuntx.ttf]{cmuntt.ttf}\ttfamily mainsubtitle}\setmainfont[Path=/usr/share/fonts/truetype/cmu/,UprightFont=cmunrm.ttf,BoldFont=cmunbx.ttf,ItalicFont=cmunti.ttf,BoldItalicFont=cmunbi.ttf]{cmunrm.ttf}\setmonofont[Path=/usr/share/fonts/truetype/cmu/,UprightFont=cmuntt.ttf,BoldFont=cmuntb.ttf,ItalicFont=cmunit.ttf,BoldItalicFont=cmuntx.ttf]{cmunrm.ttf}, {\ttfamily \setmainfont[Path=/usr/share/fonts/truetype/cmu/,UprightFont=cmunrm.ttf,BoldFont=cmunbx.ttf,ItalicFont=cmunti.ttf,BoldItalicFont=cmunbi.ttf]{cmuntt.ttf}\setmonofont[Path=/usr/share/fonts/truetype/cmu/,UprightFont=cmuntt.ttf,BoldFont=cmuntb.ttf,ItalicFont=cmunit.ttf,BoldItalicFont=cmuntx.ttf]{cmuntt.ttf}\ttfamily maintitleaddon}}&\hspace*{0pt}\ignorespaces{}\hspace*{0pt}{$\text{ }$}\setmainfont[Path=/usr/share/fonts/truetype/cmu/,UprightFont=cmunrm.ttf,BoldFont=cmunbx.ttf,ItalicFont=cmunti.ttf,BoldItalicFont=cmunbi.ttf]{cmunrm.ttf}\setmonofont[Path=/usr/share/fonts/truetype/cmu/,UprightFont=cmuntt.ttf,BoldFont=cmuntb.ttf,ItalicFont=cmunit.ttf,BoldItalicFont=cmuntx.ttf]{cmunrm.ttf}   &\hspace*{0pt}\ignorespaces{}\hspace*{0pt} \^{} &\hspace*{0pt}\ignorespaces{}\hspace*{0pt} – &\hspace*{0pt}\ignorespaces{}\hspace*{0pt} \^{} &\hspace*{0pt}\ignorespaces{}\hspace*{0pt}   &\hspace*{0pt}\ignorespaces{}\hspace*{0pt} \^{} &\hspace*{0pt}\ignorespaces{}\hspace*{0pt} – &\hspace*{0pt}\ignorespaces{}\hspace*{0pt} \^{} &\hspace*{0pt}\ignorespaces{}\hspace*{0pt}   &\hspace*{0pt}\ignorespaces{}\hspace*{0pt}   &\hspace*{0pt}\ignorespaces{}\hspace*{0pt}   &\hspace*{0pt}\ignorespaces{}\hspace*{0pt}   &\hspace*{0pt}\ignorespaces{}\hspace*{0pt}   &\hspace*{0pt}\ignorespaces{}\hspace*{0pt} \^{} &\hspace*{0pt}\ignorespaces{}\hspace*{0pt} – &\hspace*{0pt}\ignorespaces{}\hspace*{0pt} \^{} &\hspace*{0pt}\ignorespaces{}\hspace*{0pt}   &\hspace*{0pt}\ignorespaces{}\hspace*{0pt}   &\hspace*{0pt}\ignorespaces{}\hspace*{0pt}\\ \hline {\bfseries \hspace*{0pt}\ignorespaces{}\hspace*{0pt} {\ttfamily \setmainfont[Path=/usr/share/fonts/truetype/cmu/,UprightFont=cmunrm.ttf,BoldFont=cmunbx.ttf,ItalicFont=cmunti.ttf,BoldItalicFont=cmunbi.ttf]{cmuntt.ttf}\setmonofont[Path=/usr/share/fonts/truetype/cmu/,UprightFont=cmuntt.ttf,BoldFont=cmuntb.ttf,ItalicFont=cmunit.ttf,BoldItalicFont=cmuntx.ttf]{cmuntt.ttf}\ttfamily booktitle}}&\hspace*{0pt}\ignorespaces{}\hspace*{0pt}{$\text{ }$}\setmainfont[Path=/usr/share/fonts/truetype/cmu/,UprightFont=cmunrm.ttf,BoldFont=cmunbx.ttf,ItalicFont=cmunti.ttf,BoldItalicFont=cmunbi.ttf]{cmunrm.ttf}\setmonofont[Path=/usr/share/fonts/truetype/cmu/,UprightFont=cmuntt.ttf,BoldFont=cmuntb.ttf,ItalicFont=cmunit.ttf,BoldItalicFont=cmuntx.ttf]{cmunrm.ttf}   &\hspace*{0pt}\ignorespaces{}\hspace*{0pt}   &\hspace*{0pt}\ignorespaces{}\hspace*{0pt}   &\hspace*{0pt}\ignorespaces{}\hspace*{0pt} + &\hspace*{0pt}\ignorespaces{}\hspace*{0pt}   &\hspace*{0pt}\ignorespaces{}\hspace*{0pt}   &\hspace*{0pt}\ignorespaces{}\hspace*{0pt}   &\hspace*{0pt}\ignorespaces{}\hspace*{0pt} + &\hspace*{0pt}\ignorespaces{}\hspace*{0pt}   &\hspace*{0pt}\ignorespaces{}\hspace*{0pt}   &\hspace*{0pt}\ignorespaces{}\hspace*{0pt}   &\hspace*{0pt}\ignorespaces{}\hspace*{0pt}   &\hspace*{0pt}\ignorespaces{}\hspace*{0pt}   &\hspace*{0pt}\ignorespaces{}\hspace*{0pt}   &\hspace*{0pt}\ignorespaces{}\hspace*{0pt}   &\hspace*{0pt}\ignorespaces{}\hspace*{0pt} + &\hspace*{0pt}\ignorespaces{}\hspace*{0pt}   &\hspace*{0pt}\ignorespaces{}\hspace*{0pt}   &\hspace*{0pt}\ignorespaces{}\hspace*{0pt}\\ \hline {\bfseries \hspace*{0pt}\ignorespaces{}\hspace*{0pt} {\ttfamily \setmainfont[Path=/usr/share/fonts/truetype/cmu/,UprightFont=cmunrm.ttf,BoldFont=cmunbx.ttf,ItalicFont=cmunti.ttf,BoldItalicFont=cmunbi.ttf]{cmuntt.ttf}\setmonofont[Path=/usr/share/fonts/truetype/cmu/,UprightFont=cmuntt.ttf,BoldFont=cmuntb.ttf,ItalicFont=cmunit.ttf,BoldItalicFont=cmuntx.ttf]{cmuntt.ttf}\ttfamily booksubtitle}\setmainfont[Path=/usr/share/fonts/truetype/cmu/,UprightFont=cmunrm.ttf,BoldFont=cmunbx.ttf,ItalicFont=cmunti.ttf,BoldItalicFont=cmunbi.ttf]{cmunrm.ttf}\setmonofont[Path=/usr/share/fonts/truetype/cmu/,UprightFont=cmuntt.ttf,BoldFont=cmuntb.ttf,ItalicFont=cmunit.ttf,BoldItalicFont=cmuntx.ttf]{cmunrm.ttf}, {\ttfamily \setmainfont[Path=/usr/share/fonts/truetype/cmu/,UprightFont=cmunrm.ttf,BoldFont=cmunbx.ttf,ItalicFont=cmunti.ttf,BoldItalicFont=cmunbi.ttf]{cmuntt.ttf}\setmonofont[Path=/usr/share/fonts/truetype/cmu/,UprightFont=cmuntt.ttf,BoldFont=cmuntb.ttf,ItalicFont=cmunit.ttf,BoldItalicFont=cmuntx.ttf]{cmuntt.ttf}\ttfamily booktitleaddon}}&\hspace*{0pt}\ignorespaces{}\hspace*{0pt}{$\text{ }$}\setmainfont[Path=/usr/share/fonts/truetype/cmu/,UprightFont=cmunrm.ttf,BoldFont=cmunbx.ttf,ItalicFont=cmunti.ttf,BoldItalicFont=cmunbi.ttf]{cmunrm.ttf}\setmonofont[Path=/usr/share/fonts/truetype/cmu/,UprightFont=cmuntt.ttf,BoldFont=cmuntb.ttf,ItalicFont=cmunit.ttf,BoldItalicFont=cmuntx.ttf]{cmunrm.ttf}   &\hspace*{0pt}\ignorespaces{}\hspace*{0pt}   &\hspace*{0pt}\ignorespaces{}\hspace*{0pt}   &\hspace*{0pt}\ignorespaces{}\hspace*{0pt} \^{} &\hspace*{0pt}\ignorespaces{}\hspace*{0pt}   &\hspace*{0pt}\ignorespaces{}\hspace*{0pt}   &\hspace*{0pt}\ignorespaces{}\hspace*{0pt}   &\hspace*{0pt}\ignorespaces{}\hspace*{0pt} \^{} &\hspace*{0pt}\ignorespaces{}\hspace*{0pt}   &\hspace*{0pt}\ignorespaces{}\hspace*{0pt}   &\hspace*{0pt}\ignorespaces{}\hspace*{0pt}   &\hspace*{0pt}\ignorespaces{}\hspace*{0pt}   &\hspace*{0pt}\ignorespaces{}\hspace*{0pt}   &\hspace*{0pt}\ignorespaces{}\hspace*{0pt}   &\hspace*{0pt}\ignorespaces{}\hspace*{0pt}   &\hspace*{0pt}\ignorespaces{}\hspace*{0pt} \^{} &\hspace*{0pt}\ignorespaces{}\hspace*{0pt}   &\hspace*{0pt}\ignorespaces{}\hspace*{0pt}   &\hspace*{0pt}\ignorespaces{}\hspace*{0pt}\\ \hline {\bfseries \hspace*{0pt}\ignorespaces{}\hspace*{0pt} {\ttfamily \setmainfont[Path=/usr/share/fonts/truetype/cmu/,UprightFont=cmunrm.ttf,BoldFont=cmunbx.ttf,ItalicFont=cmunti.ttf,BoldItalicFont=cmunbi.ttf]{cmuntt.ttf}\setmonofont[Path=/usr/share/fonts/truetype/cmu/,UprightFont=cmuntt.ttf,BoldFont=cmuntb.ttf,ItalicFont=cmunit.ttf,BoldItalicFont=cmuntx.ttf]{cmuntt.ttf}\ttfamily journalsubtitle}}&\hspace*{0pt}\ignorespaces{}\hspace*{0pt}{$\text{ }$}\setmainfont[Path=/usr/share/fonts/truetype/cmu/,UprightFont=cmunrm.ttf,BoldFont=cmunbx.ttf,ItalicFont=cmunti.ttf,BoldItalicFont=cmunbi.ttf]{cmunrm.ttf}\setmonofont[Path=/usr/share/fonts/truetype/cmu/,UprightFont=cmuntt.ttf,BoldFont=cmuntb.ttf,ItalicFont=cmunit.ttf,BoldItalicFont=cmuntx.ttf]{cmunrm.ttf} \^{} &\hspace*{0pt}\ignorespaces{}\hspace*{0pt}   &\hspace*{0pt}\ignorespaces{}\hspace*{0pt}   &\hspace*{0pt}\ignorespaces{}\hspace*{0pt}   &\hspace*{0pt}\ignorespaces{}\hspace*{0pt}   &\hspace*{0pt}\ignorespaces{}\hspace*{0pt}   &\hspace*{0pt}\ignorespaces{}\hspace*{0pt}   &\hspace*{0pt}\ignorespaces{}\hspace*{0pt}   &\hspace*{0pt}\ignorespaces{}\hspace*{0pt}   &\hspace*{0pt}\ignorespaces{}\hspace*{0pt}   &\hspace*{0pt}\ignorespaces{}\hspace*{0pt}   &\hspace*{0pt}\ignorespaces{}\hspace*{0pt}   &\hspace*{0pt}\ignorespaces{}\hspace*{0pt}   &\hspace*{0pt}\ignorespaces{}\hspace*{0pt}   &\hspace*{0pt}\ignorespaces{}\hspace*{0pt}   &\hspace*{0pt}\ignorespaces{}\hspace*{0pt}   &\hspace*{0pt}\ignorespaces{}\hspace*{0pt}   &\hspace*{0pt}\ignorespaces{}\hspace*{0pt}   &\hspace*{0pt}\ignorespaces{}\hspace*{0pt}\\ \hline {\bfseries \hspace*{0pt}\ignorespaces{}\hspace*{0pt} {\ttfamily \setmainfont[Path=/usr/share/fonts/truetype/cmu/,UprightFont=cmunrm.ttf,BoldFont=cmunbx.ttf,ItalicFont=cmunti.ttf,BoldItalicFont=cmunbi.ttf]{cmuntt.ttf}\setmonofont[Path=/usr/share/fonts/truetype/cmu/,UprightFont=cmuntt.ttf,BoldFont=cmuntb.ttf,ItalicFont=cmunit.ttf,BoldItalicFont=cmuntx.ttf]{cmuntt.ttf}\ttfamily journaltitle}}&\hspace*{0pt}\ignorespaces{}\hspace*{0pt}{$\text{ }$}\setmainfont[Path=/usr/share/fonts/truetype/cmu/,UprightFont=cmunrm.ttf,BoldFont=cmunbx.ttf,ItalicFont=cmunti.ttf,BoldItalicFont=cmunbi.ttf]{cmunrm.ttf}\setmonofont[Path=/usr/share/fonts/truetype/cmu/,UprightFont=cmuntt.ttf,BoldFont=cmuntb.ttf,ItalicFont=cmunit.ttf,BoldItalicFont=cmuntx.ttf]{cmunrm.ttf} + &\hspace*{0pt}\ignorespaces{}\hspace*{0pt}   &\hspace*{0pt}\ignorespaces{}\hspace*{0pt}   &\hspace*{0pt}\ignorespaces{}\hspace*{0pt}   &\hspace*{0pt}\ignorespaces{}\hspace*{0pt}   &\hspace*{0pt}\ignorespaces{}\hspace*{0pt}   &\hspace*{0pt}\ignorespaces{}\hspace*{0pt}   &\hspace*{0pt}\ignorespaces{}\hspace*{0pt}   &\hspace*{0pt}\ignorespaces{}\hspace*{0pt}   &\hspace*{0pt}\ignorespaces{}\hspace*{0pt}   &\hspace*{0pt}\ignorespaces{}\hspace*{0pt}   &\hspace*{0pt}\ignorespaces{}\hspace*{0pt}   &\hspace*{0pt}\ignorespaces{}\hspace*{0pt}   &\hspace*{0pt}\ignorespaces{}\hspace*{0pt}   &\hspace*{0pt}\ignorespaces{}\hspace*{0pt}   &\hspace*{0pt}\ignorespaces{}\hspace*{0pt}   &\hspace*{0pt}\ignorespaces{}\hspace*{0pt}   &\hspace*{0pt}\ignorespaces{}\hspace*{0pt}   &\hspace*{0pt}\ignorespaces{}\hspace*{0pt}\\ \hline {\bfseries \hspace*{0pt}\ignorespaces{}\hspace*{0pt} {\ttfamily \setmainfont[Path=/usr/share/fonts/truetype/cmu/,UprightFont=cmunrm.ttf,BoldFont=cmunbx.ttf,ItalicFont=cmunti.ttf,BoldItalicFont=cmunbi.ttf]{cmuntt.ttf}\setmonofont[Path=/usr/share/fonts/truetype/cmu/,UprightFont=cmuntt.ttf,BoldFont=cmuntb.ttf,ItalicFont=cmunit.ttf,BoldItalicFont=cmuntx.ttf]{cmuntt.ttf}\ttfamily eventdate}\setmainfont[Path=/usr/share/fonts/truetype/cmu/,UprightFont=cmunrm.ttf,BoldFont=cmunbx.ttf,ItalicFont=cmunti.ttf,BoldItalicFont=cmunbi.ttf]{cmunrm.ttf}\setmonofont[Path=/usr/share/fonts/truetype/cmu/,UprightFont=cmuntt.ttf,BoldFont=cmuntb.ttf,ItalicFont=cmunit.ttf,BoldItalicFont=cmuntx.ttf]{cmunrm.ttf}, {\ttfamily \setmainfont[Path=/usr/share/fonts/truetype/cmu/,UprightFont=cmunrm.ttf,BoldFont=cmunbx.ttf,ItalicFont=cmunti.ttf,BoldItalicFont=cmunbi.ttf]{cmuntt.ttf}\setmonofont[Path=/usr/share/fonts/truetype/cmu/,UprightFont=cmuntt.ttf,BoldFont=cmuntb.ttf,ItalicFont=cmunit.ttf,BoldItalicFont=cmuntx.ttf]{cmuntt.ttf}\ttfamily eventtitle}\setmainfont[Path=/usr/share/fonts/truetype/cmu/,UprightFont=cmunrm.ttf,BoldFont=cmunbx.ttf,ItalicFont=cmunti.ttf,BoldItalicFont=cmunbi.ttf]{cmunrm.ttf}\setmonofont[Path=/usr/share/fonts/truetype/cmu/,UprightFont=cmuntt.ttf,BoldFont=cmuntb.ttf,ItalicFont=cmunit.ttf,BoldItalicFont=cmuntx.ttf]{cmunrm.ttf}, {\ttfamily \setmainfont[Path=/usr/share/fonts/truetype/cmu/,UprightFont=cmunrm.ttf,BoldFont=cmunbx.ttf,ItalicFont=cmunti.ttf,BoldItalicFont=cmunbi.ttf]{cmuntt.ttf}\setmonofont[Path=/usr/share/fonts/truetype/cmu/,UprightFont=cmuntt.ttf,BoldFont=cmuntb.ttf,ItalicFont=cmunit.ttf,BoldItalicFont=cmuntx.ttf]{cmuntt.ttf}\ttfamily eventtitleaddon}\setmainfont[Path=/usr/share/fonts/truetype/cmu/,UprightFont=cmunrm.ttf,BoldFont=cmunbx.ttf,ItalicFont=cmunti.ttf,BoldItalicFont=cmunbi.ttf]{cmunrm.ttf}\setmonofont[Path=/usr/share/fonts/truetype/cmu/,UprightFont=cmuntt.ttf,BoldFont=cmuntb.ttf,ItalicFont=cmunit.ttf,BoldItalicFont=cmuntx.ttf]{cmunrm.ttf}, {\ttfamily \setmainfont[Path=/usr/share/fonts/truetype/cmu/,UprightFont=cmunrm.ttf,BoldFont=cmunbx.ttf,ItalicFont=cmunti.ttf,BoldItalicFont=cmunbi.ttf]{cmuntt.ttf}\setmonofont[Path=/usr/share/fonts/truetype/cmu/,UprightFont=cmuntt.ttf,BoldFont=cmuntb.ttf,ItalicFont=cmunit.ttf,BoldItalicFont=cmuntx.ttf]{cmuntt.ttf}\ttfamily venue}}&\hspace*{0pt}\ignorespaces{}\hspace*{0pt}{$\text{ }$}\setmainfont[Path=/usr/share/fonts/truetype/cmu/,UprightFont=cmunrm.ttf,BoldFont=cmunbx.ttf,ItalicFont=cmunti.ttf,BoldItalicFont=cmunbi.ttf]{cmunrm.ttf}\setmonofont[Path=/usr/share/fonts/truetype/cmu/,UprightFont=cmuntt.ttf,BoldFont=cmuntb.ttf,ItalicFont=cmunit.ttf,BoldItalicFont=cmuntx.ttf]{cmunrm.ttf}   &\hspace*{0pt}\ignorespaces{}\hspace*{0pt}   &\hspace*{0pt}\ignorespaces{}\hspace*{0pt}   &\hspace*{0pt}\ignorespaces{}\hspace*{0pt}   &\hspace*{0pt}\ignorespaces{}\hspace*{0pt}   &\hspace*{0pt}\ignorespaces{}\hspace*{0pt}   &\hspace*{0pt}\ignorespaces{}\hspace*{0pt}   &\hspace*{0pt}\ignorespaces{}\hspace*{0pt}   &\hspace*{0pt}\ignorespaces{}\hspace*{0pt}   &\hspace*{0pt}\ignorespaces{}\hspace*{0pt}   &\hspace*{0pt}\ignorespaces{}\hspace*{0pt}   &\hspace*{0pt}\ignorespaces{}\hspace*{0pt}   &\hspace*{0pt}\ignorespaces{}\hspace*{0pt}   &\hspace*{0pt}\ignorespaces{}\hspace*{0pt} \^{} &\hspace*{0pt}\ignorespaces{}\hspace*{0pt} \^{} &\hspace*{0pt}\ignorespaces{}\hspace*{0pt} \^{} &\hspace*{0pt}\ignorespaces{}\hspace*{0pt}   &\hspace*{0pt}\ignorespaces{}\hspace*{0pt}   &\hspace*{0pt}\ignorespaces{}\hspace*{0pt}\\ \hline {\bfseries \hspace*{0pt}\ignorespaces{}\hspace*{0pt} {\ttfamily \setmainfont[Path=/usr/share/fonts/truetype/cmu/,UprightFont=cmunrm.ttf,BoldFont=cmunbx.ttf,ItalicFont=cmunti.ttf,BoldItalicFont=cmunbi.ttf]{cmuntt.ttf}\setmonofont[Path=/usr/share/fonts/truetype/cmu/,UprightFont=cmuntt.ttf,BoldFont=cmuntb.ttf,ItalicFont=cmunit.ttf,BoldItalicFont=cmuntx.ttf]{cmuntt.ttf}\ttfamily date}\setmainfont[Path=/usr/share/fonts/truetype/cmu/,UprightFont=cmunrm.ttf,BoldFont=cmunbx.ttf,ItalicFont=cmunti.ttf,BoldItalicFont=cmunbi.ttf]{cmunrm.ttf}\setmonofont[Path=/usr/share/fonts/truetype/cmu/,UprightFont=cmuntt.ttf,BoldFont=cmuntb.ttf,ItalicFont=cmunit.ttf,BoldItalicFont=cmuntx.ttf]{cmunrm.ttf}, {\ttfamily \setmainfont[Path=/usr/share/fonts/truetype/cmu/,UprightFont=cmunrm.ttf,BoldFont=cmunbx.ttf,ItalicFont=cmunti.ttf,BoldItalicFont=cmunbi.ttf]{cmuntt.ttf}\setmonofont[Path=/usr/share/fonts/truetype/cmu/,UprightFont=cmuntt.ttf,BoldFont=cmuntb.ttf,ItalicFont=cmunit.ttf,BoldItalicFont=cmuntx.ttf]{cmuntt.ttf}\ttfamily year}}&\hspace*{0pt}\ignorespaces{}\hspace*{0pt}{$\text{ }$}\setmainfont[Path=/usr/share/fonts/truetype/cmu/,UprightFont=cmunrm.ttf,BoldFont=cmunbx.ttf,ItalicFont=cmunti.ttf,BoldItalicFont=cmunbi.ttf]{cmunrm.ttf}\setmonofont[Path=/usr/share/fonts/truetype/cmu/,UprightFont=cmuntt.ttf,BoldFont=cmuntb.ttf,ItalicFont=cmunit.ttf,BoldItalicFont=cmuntx.ttf]{cmunrm.ttf} ± &\hspace*{0pt}\ignorespaces{}\hspace*{0pt} ± &\hspace*{0pt}\ignorespaces{}\hspace*{0pt} ± &\hspace*{0pt}\ignorespaces{}\hspace*{0pt} ± &\hspace*{0pt}\ignorespaces{}\hspace*{0pt} ± &\hspace*{0pt}\ignorespaces{}\hspace*{0pt} ± &\hspace*{0pt}\ignorespaces{}\hspace*{0pt} ± &\hspace*{0pt}\ignorespaces{}\hspace*{0pt} ± &\hspace*{0pt}\ignorespaces{}\hspace*{0pt} ± &\hspace*{0pt}\ignorespaces{}\hspace*{0pt} ± &\hspace*{0pt}\ignorespaces{}\hspace*{0pt} ± &\hspace*{0pt}\ignorespaces{}\hspace*{0pt} ± &\hspace*{0pt}\ignorespaces{}\hspace*{0pt} ± &\hspace*{0pt}\ignorespaces{}\hspace*{0pt} ± &\hspace*{0pt}\ignorespaces{}\hspace*{0pt} ± &\hspace*{0pt}\ignorespaces{}\hspace*{0pt} ± &\hspace*{0pt}\ignorespaces{}\hspace*{0pt} ± &\hspace*{0pt}\ignorespaces{}\hspace*{0pt} ± &\hspace*{0pt}\ignorespaces{}\hspace*{0pt} ± \\ \hline {\bfseries \hspace*{0pt}\ignorespaces{}\hspace*{0pt} {\ttfamily \setmainfont[Path=/usr/share/fonts/truetype/cmu/,UprightFont=cmunrm.ttf,BoldFont=cmunbx.ttf,ItalicFont=cmunti.ttf,BoldItalicFont=cmunbi.ttf]{cmuntt.ttf}\setmonofont[Path=/usr/share/fonts/truetype/cmu/,UprightFont=cmuntt.ttf,BoldFont=cmuntb.ttf,ItalicFont=cmunit.ttf,BoldItalicFont=cmuntx.ttf]{cmuntt.ttf}\ttfamily month}}&\hspace*{0pt}\ignorespaces{}\hspace*{0pt}{$\text{ }$}\setmainfont[Path=/usr/share/fonts/truetype/cmu/,UprightFont=cmunrm.ttf,BoldFont=cmunbx.ttf,ItalicFont=cmunti.ttf,BoldItalicFont=cmunbi.ttf]{cmunrm.ttf}\setmonofont[Path=/usr/share/fonts/truetype/cmu/,UprightFont=cmuntt.ttf,BoldFont=cmuntb.ttf,ItalicFont=cmunit.ttf,BoldItalicFont=cmuntx.ttf]{cmunrm.ttf} \^{} &\hspace*{0pt}\ignorespaces{}\hspace*{0pt}   &\hspace*{0pt}\ignorespaces{}\hspace*{0pt}   &\hspace*{0pt}\ignorespaces{}\hspace*{0pt}   &\hspace*{0pt}\ignorespaces{}\hspace*{0pt}   &\hspace*{0pt}\ignorespaces{}\hspace*{0pt}   &\hspace*{0pt}\ignorespaces{}\hspace*{0pt}   &\hspace*{0pt}\ignorespaces{}\hspace*{0pt}   &\hspace*{0pt}\ignorespaces{}\hspace*{0pt}   &\hspace*{0pt}\ignorespaces{}\hspace*{0pt} \^{} &\hspace*{0pt}\ignorespaces{}\hspace*{0pt} \^{} &\hspace*{0pt}\ignorespaces{}\hspace*{0pt} \^{} &\hspace*{0pt}\ignorespaces{}\hspace*{0pt} \^{} &\hspace*{0pt}\ignorespaces{}\hspace*{0pt} \^{} &\hspace*{0pt}\ignorespaces{}\hspace*{0pt} \^{} &\hspace*{0pt}\ignorespaces{}\hspace*{0pt} \^{} &\hspace*{0pt}\ignorespaces{}\hspace*{0pt} \^{} &\hspace*{0pt}\ignorespaces{}\hspace*{0pt} \^{} &\hspace*{0pt}\ignorespaces{}\hspace*{0pt} \^{} \\ \hline {\bfseries \hspace*{0pt}\ignorespaces{}\hspace*{0pt} {\ttfamily \setmainfont[Path=/usr/share/fonts/truetype/cmu/,UprightFont=cmunrm.ttf,BoldFont=cmunbx.ttf,ItalicFont=cmunti.ttf,BoldItalicFont=cmunbi.ttf]{cmuntt.ttf}\setmonofont[Path=/usr/share/fonts/truetype/cmu/,UprightFont=cmuntt.ttf,BoldFont=cmuntb.ttf,ItalicFont=cmunit.ttf,BoldItalicFont=cmuntx.ttf]{cmuntt.ttf}\ttfamily edition}}&\hspace*{0pt}\ignorespaces{}\hspace*{0pt}{$\text{ }$}\setmainfont[Path=/usr/share/fonts/truetype/cmu/,UprightFont=cmunrm.ttf,BoldFont=cmunbx.ttf,ItalicFont=cmunti.ttf,BoldItalicFont=cmunbi.ttf]{cmunrm.ttf}\setmonofont[Path=/usr/share/fonts/truetype/cmu/,UprightFont=cmuntt.ttf,BoldFont=cmuntb.ttf,ItalicFont=cmunit.ttf,BoldItalicFont=cmuntx.ttf]{cmunrm.ttf}   &\hspace*{0pt}\ignorespaces{}\hspace*{0pt} \^{} &\hspace*{0pt}\ignorespaces{}\hspace*{0pt} \^{} &\hspace*{0pt}\ignorespaces{}\hspace*{0pt} \^{} &\hspace*{0pt}\ignorespaces{}\hspace*{0pt}   &\hspace*{0pt}\ignorespaces{}\hspace*{0pt} \^{} &\hspace*{0pt}\ignorespaces{}\hspace*{0pt} \^{} &\hspace*{0pt}\ignorespaces{}\hspace*{0pt} \^{} &\hspace*{0pt}\ignorespaces{}\hspace*{0pt}   &\hspace*{0pt}\ignorespaces{}\hspace*{0pt}   &\hspace*{0pt}\ignorespaces{}\hspace*{0pt}   &\hspace*{0pt}\ignorespaces{}\hspace*{0pt}   &\hspace*{0pt}\ignorespaces{}\hspace*{0pt}   &\hspace*{0pt}\ignorespaces{}\hspace*{0pt}   &\hspace*{0pt}\ignorespaces{}\hspace*{0pt}   &\hspace*{0pt}\ignorespaces{}\hspace*{0pt}   &\hspace*{0pt}\ignorespaces{}\hspace*{0pt}   &\hspace*{0pt}\ignorespaces{}\hspace*{0pt}   &\hspace*{0pt}\ignorespaces{}\hspace*{0pt}\\ \hline {\bfseries \hspace*{0pt}\ignorespaces{}\hspace*{0pt} {\ttfamily \setmainfont[Path=/usr/share/fonts/truetype/cmu/,UprightFont=cmunrm.ttf,BoldFont=cmunbx.ttf,ItalicFont=cmunti.ttf,BoldItalicFont=cmunbi.ttf]{cmuntt.ttf}\setmonofont[Path=/usr/share/fonts/truetype/cmu/,UprightFont=cmuntt.ttf,BoldFont=cmuntb.ttf,ItalicFont=cmunit.ttf,BoldItalicFont=cmuntx.ttf]{cmuntt.ttf}\ttfamily issue}\setmainfont[Path=/usr/share/fonts/truetype/cmu/,UprightFont=cmunrm.ttf,BoldFont=cmunbx.ttf,ItalicFont=cmunti.ttf,BoldItalicFont=cmunbi.ttf]{cmunrm.ttf}\setmonofont[Path=/usr/share/fonts/truetype/cmu/,UprightFont=cmuntt.ttf,BoldFont=cmuntb.ttf,ItalicFont=cmunit.ttf,BoldItalicFont=cmuntx.ttf]{cmunrm.ttf}, {\ttfamily \setmainfont[Path=/usr/share/fonts/truetype/cmu/,UprightFont=cmunrm.ttf,BoldFont=cmunbx.ttf,ItalicFont=cmunti.ttf,BoldItalicFont=cmunbi.ttf]{cmuntt.ttf}\setmonofont[Path=/usr/share/fonts/truetype/cmu/,UprightFont=cmuntt.ttf,BoldFont=cmuntb.ttf,ItalicFont=cmunit.ttf,BoldItalicFont=cmuntx.ttf]{cmuntt.ttf}\ttfamily issuetitle}\setmainfont[Path=/usr/share/fonts/truetype/cmu/,UprightFont=cmunrm.ttf,BoldFont=cmunbx.ttf,ItalicFont=cmunti.ttf,BoldItalicFont=cmunbi.ttf]{cmunrm.ttf}\setmonofont[Path=/usr/share/fonts/truetype/cmu/,UprightFont=cmuntt.ttf,BoldFont=cmuntb.ttf,ItalicFont=cmunit.ttf,BoldItalicFont=cmuntx.ttf]{cmunrm.ttf}, {\ttfamily \setmainfont[Path=/usr/share/fonts/truetype/cmu/,UprightFont=cmunrm.ttf,BoldFont=cmunbx.ttf,ItalicFont=cmunti.ttf,BoldItalicFont=cmunbi.ttf]{cmuntt.ttf}\setmonofont[Path=/usr/share/fonts/truetype/cmu/,UprightFont=cmuntt.ttf,BoldFont=cmuntb.ttf,ItalicFont=cmunit.ttf,BoldItalicFont=cmuntx.ttf]{cmuntt.ttf}\ttfamily issuesubtitle}}&\hspace*{0pt}\ignorespaces{}\hspace*{0pt}{$\text{ }$}\setmainfont[Path=/usr/share/fonts/truetype/cmu/,UprightFont=cmunrm.ttf,BoldFont=cmunbx.ttf,ItalicFont=cmunti.ttf,BoldItalicFont=cmunbi.ttf]{cmunrm.ttf}\setmonofont[Path=/usr/share/fonts/truetype/cmu/,UprightFont=cmuntt.ttf,BoldFont=cmuntb.ttf,ItalicFont=cmunit.ttf,BoldItalicFont=cmuntx.ttf]{cmunrm.ttf} \^{} &\hspace*{0pt}\ignorespaces{}\hspace*{0pt}   &\hspace*{0pt}\ignorespaces{}\hspace*{0pt}   &\hspace*{0pt}\ignorespaces{}\hspace*{0pt}   &\hspace*{0pt}\ignorespaces{}\hspace*{0pt}   &\hspace*{0pt}\ignorespaces{}\hspace*{0pt}   &\hspace*{0pt}\ignorespaces{}\hspace*{0pt}   &\hspace*{0pt}\ignorespaces{}\hspace*{0pt}   &\hspace*{0pt}\ignorespaces{}\hspace*{0pt}   &\hspace*{0pt}\ignorespaces{}\hspace*{0pt}   &\hspace*{0pt}\ignorespaces{}\hspace*{0pt}   &\hspace*{0pt}\ignorespaces{}\hspace*{0pt}   &\hspace*{0pt}\ignorespaces{}\hspace*{0pt} \^{} &\hspace*{0pt}\ignorespaces{}\hspace*{0pt}   &\hspace*{0pt}\ignorespaces{}\hspace*{0pt}   &\hspace*{0pt}\ignorespaces{}\hspace*{0pt}   &\hspace*{0pt}\ignorespaces{}\hspace*{0pt}   &\hspace*{0pt}\ignorespaces{}\hspace*{0pt}   &\hspace*{0pt}\ignorespaces{}\hspace*{0pt}\\ \hline {\bfseries \hspace*{0pt}\ignorespaces{}\hspace*{0pt} {\ttfamily \setmainfont[Path=/usr/share/fonts/truetype/cmu/,UprightFont=cmunrm.ttf,BoldFont=cmunbx.ttf,ItalicFont=cmunti.ttf,BoldItalicFont=cmunbi.ttf]{cmuntt.ttf}\setmonofont[Path=/usr/share/fonts/truetype/cmu/,UprightFont=cmuntt.ttf,BoldFont=cmuntb.ttf,ItalicFont=cmunit.ttf,BoldItalicFont=cmuntx.ttf]{cmuntt.ttf}\ttfamily number}}&\hspace*{0pt}\ignorespaces{}\hspace*{0pt}{$\text{ }$}\setmainfont[Path=/usr/share/fonts/truetype/cmu/,UprightFont=cmunrm.ttf,BoldFont=cmunbx.ttf,ItalicFont=cmunti.ttf,BoldItalicFont=cmunbi.ttf]{cmunrm.ttf}\setmonofont[Path=/usr/share/fonts/truetype/cmu/,UprightFont=cmuntt.ttf,BoldFont=cmuntb.ttf,ItalicFont=cmunit.ttf,BoldItalicFont=cmuntx.ttf]{cmunrm.ttf} \^{} &\hspace*{0pt}\ignorespaces{}\hspace*{0pt} \^{} &\hspace*{0pt}\ignorespaces{}\hspace*{0pt} \^{} &\hspace*{0pt}\ignorespaces{}\hspace*{0pt} \^{} &\hspace*{0pt}\ignorespaces{}\hspace*{0pt}   &\hspace*{0pt}\ignorespaces{}\hspace*{0pt} \^{} &\hspace*{0pt}\ignorespaces{}\hspace*{0pt} \^{} &\hspace*{0pt}\ignorespaces{}\hspace*{0pt} \^{} &\hspace*{0pt}\ignorespaces{}\hspace*{0pt}   &\hspace*{0pt}\ignorespaces{}\hspace*{0pt}   &\hspace*{0pt}\ignorespaces{}\hspace*{0pt}   &\hspace*{0pt}\ignorespaces{}\hspace*{0pt} + &\hspace*{0pt}\ignorespaces{}\hspace*{0pt} \^{} &\hspace*{0pt}\ignorespaces{}\hspace*{0pt} \^{} &\hspace*{0pt}\ignorespaces{}\hspace*{0pt} \^{} &\hspace*{0pt}\ignorespaces{}\hspace*{0pt} \^{} &\hspace*{0pt}\ignorespaces{}\hspace*{0pt} \^{} &\hspace*{0pt}\ignorespaces{}\hspace*{0pt}   &\hspace*{0pt}\ignorespaces{}\hspace*{0pt}\\ \hline {\bfseries \hspace*{0pt}\ignorespaces{}\hspace*{0pt} {\ttfamily \setmainfont[Path=/usr/share/fonts/truetype/cmu/,UprightFont=cmunrm.ttf,BoldFont=cmunbx.ttf,ItalicFont=cmunti.ttf,BoldItalicFont=cmunbi.ttf]{cmuntt.ttf}\setmonofont[Path=/usr/share/fonts/truetype/cmu/,UprightFont=cmuntt.ttf,BoldFont=cmuntb.ttf,ItalicFont=cmunit.ttf,BoldItalicFont=cmuntx.ttf]{cmuntt.ttf}\ttfamily series}}&\hspace*{0pt}\ignorespaces{}\hspace*{0pt}{$\text{ }$}\setmainfont[Path=/usr/share/fonts/truetype/cmu/,UprightFont=cmunrm.ttf,BoldFont=cmunbx.ttf,ItalicFont=cmunti.ttf,BoldItalicFont=cmunbi.ttf]{cmunrm.ttf}\setmonofont[Path=/usr/share/fonts/truetype/cmu/,UprightFont=cmuntt.ttf,BoldFont=cmuntb.ttf,ItalicFont=cmunit.ttf,BoldItalicFont=cmuntx.ttf]{cmunrm.ttf} \^{} &\hspace*{0pt}\ignorespaces{}\hspace*{0pt} \^{} &\hspace*{0pt}\ignorespaces{}\hspace*{0pt} \^{} &\hspace*{0pt}\ignorespaces{}\hspace*{0pt} \^{} &\hspace*{0pt}\ignorespaces{}\hspace*{0pt}   &\hspace*{0pt}\ignorespaces{}\hspace*{0pt} \^{} &\hspace*{0pt}\ignorespaces{}\hspace*{0pt} \^{} &\hspace*{0pt}\ignorespaces{}\hspace*{0pt} \^{} &\hspace*{0pt}\ignorespaces{}\hspace*{0pt} \^{} &\hspace*{0pt}\ignorespaces{}\hspace*{0pt}   &\hspace*{0pt}\ignorespaces{}\hspace*{0pt}   &\hspace*{0pt}\ignorespaces{}\hspace*{0pt}   &\hspace*{0pt}\ignorespaces{}\hspace*{0pt} \^{} &\hspace*{0pt}\ignorespaces{}\hspace*{0pt} \^{} &\hspace*{0pt}\ignorespaces{}\hspace*{0pt} \^{} &\hspace*{0pt}\ignorespaces{}\hspace*{0pt} \^{} &\hspace*{0pt}\ignorespaces{}\hspace*{0pt}   &\hspace*{0pt}\ignorespaces{}\hspace*{0pt}   &\hspace*{0pt}\ignorespaces{}\hspace*{0pt}\\ \hline {\bfseries \hspace*{0pt}\ignorespaces{}\hspace*{0pt} {\ttfamily \setmainfont[Path=/usr/share/fonts/truetype/cmu/,UprightFont=cmunrm.ttf,BoldFont=cmunbx.ttf,ItalicFont=cmunti.ttf,BoldItalicFont=cmunbi.ttf]{cmuntt.ttf}\setmonofont[Path=/usr/share/fonts/truetype/cmu/,UprightFont=cmuntt.ttf,BoldFont=cmuntb.ttf,ItalicFont=cmunit.ttf,BoldItalicFont=cmuntx.ttf]{cmuntt.ttf}\ttfamily chapter}}&\hspace*{0pt}\ignorespaces{}\hspace*{0pt}{$\text{ }$}\setmainfont[Path=/usr/share/fonts/truetype/cmu/,UprightFont=cmunrm.ttf,BoldFont=cmunbx.ttf,ItalicFont=cmunti.ttf,BoldItalicFont=cmunbi.ttf]{cmunrm.ttf}\setmonofont[Path=/usr/share/fonts/truetype/cmu/,UprightFont=cmuntt.ttf,BoldFont=cmuntb.ttf,ItalicFont=cmunit.ttf,BoldItalicFont=cmuntx.ttf]{cmunrm.ttf}   &\hspace*{0pt}\ignorespaces{}\hspace*{0pt} \^{} &\hspace*{0pt}\ignorespaces{}\hspace*{0pt} – &\hspace*{0pt}\ignorespaces{}\hspace*{0pt} \^{} &\hspace*{0pt}\ignorespaces{}\hspace*{0pt} \^{} &\hspace*{0pt}\ignorespaces{}\hspace*{0pt} \^{} &\hspace*{0pt}\ignorespaces{}\hspace*{0pt} – &\hspace*{0pt}\ignorespaces{}\hspace*{0pt} \^{} &\hspace*{0pt}\ignorespaces{}\hspace*{0pt} \^{} &\hspace*{0pt}\ignorespaces{}\hspace*{0pt}   &\hspace*{0pt}\ignorespaces{}\hspace*{0pt}   &\hspace*{0pt}\ignorespaces{}\hspace*{0pt}   &\hspace*{0pt}\ignorespaces{}\hspace*{0pt}   &\hspace*{0pt}\ignorespaces{}\hspace*{0pt} \^{} &\hspace*{0pt}\ignorespaces{}\hspace*{0pt} – &\hspace*{0pt}\ignorespaces{}\hspace*{0pt} \^{} &\hspace*{0pt}\ignorespaces{}\hspace*{0pt} \^{} &\hspace*{0pt}\ignorespaces{}\hspace*{0pt} \^{} &\hspace*{0pt}\ignorespaces{}\hspace*{0pt}\\ \hline {\bfseries \hspace*{0pt}\ignorespaces{}\hspace*{0pt} {\ttfamily \setmainfont[Path=/usr/share/fonts/truetype/cmu/,UprightFont=cmunrm.ttf,BoldFont=cmunbx.ttf,ItalicFont=cmunti.ttf,BoldItalicFont=cmunbi.ttf]{cmuntt.ttf}\setmonofont[Path=/usr/share/fonts/truetype/cmu/,UprightFont=cmuntt.ttf,BoldFont=cmuntb.ttf,ItalicFont=cmunit.ttf,BoldItalicFont=cmuntx.ttf]{cmuntt.ttf}\ttfamily part}}&\hspace*{0pt}\ignorespaces{}\hspace*{0pt}{$\text{ }$}\setmainfont[Path=/usr/share/fonts/truetype/cmu/,UprightFont=cmunrm.ttf,BoldFont=cmunbx.ttf,ItalicFont=cmunti.ttf,BoldItalicFont=cmunbi.ttf]{cmunrm.ttf}\setmonofont[Path=/usr/share/fonts/truetype/cmu/,UprightFont=cmuntt.ttf,BoldFont=cmuntb.ttf,ItalicFont=cmunit.ttf,BoldItalicFont=cmuntx.ttf]{cmunrm.ttf}   &\hspace*{0pt}\ignorespaces{}\hspace*{0pt} \^{} &\hspace*{0pt}\ignorespaces{}\hspace*{0pt} – &\hspace*{0pt}\ignorespaces{}\hspace*{0pt} \^{} &\hspace*{0pt}\ignorespaces{}\hspace*{0pt}   &\hspace*{0pt}\ignorespaces{}\hspace*{0pt} \^{} &\hspace*{0pt}\ignorespaces{}\hspace*{0pt} – &\hspace*{0pt}\ignorespaces{}\hspace*{0pt} \^{} &\hspace*{0pt}\ignorespaces{}\hspace*{0pt}   &\hspace*{0pt}\ignorespaces{}\hspace*{0pt}   &\hspace*{0pt}\ignorespaces{}\hspace*{0pt}   &\hspace*{0pt}\ignorespaces{}\hspace*{0pt}   &\hspace*{0pt}\ignorespaces{}\hspace*{0pt}   &\hspace*{0pt}\ignorespaces{}\hspace*{0pt} \^{} &\hspace*{0pt}\ignorespaces{}\hspace*{0pt} – &\hspace*{0pt}\ignorespaces{}\hspace*{0pt} \^{} &\hspace*{0pt}\ignorespaces{}\hspace*{0pt}   &\hspace*{0pt}\ignorespaces{}\hspace*{0pt}   &\hspace*{0pt}\ignorespaces{}\hspace*{0pt}\\ \hline {\bfseries \hspace*{0pt}\ignorespaces{}\hspace*{0pt} {\ttfamily \setmainfont[Path=/usr/share/fonts/truetype/cmu/,UprightFont=cmunrm.ttf,BoldFont=cmunbx.ttf,ItalicFont=cmunti.ttf,BoldItalicFont=cmunbi.ttf]{cmuntt.ttf}\setmonofont[Path=/usr/share/fonts/truetype/cmu/,UprightFont=cmuntt.ttf,BoldFont=cmuntb.ttf,ItalicFont=cmunit.ttf,BoldItalicFont=cmuntx.ttf]{cmuntt.ttf}\ttfamily volume}}&\hspace*{0pt}\ignorespaces{}\hspace*{0pt}{$\text{ }$}\setmainfont[Path=/usr/share/fonts/truetype/cmu/,UprightFont=cmunrm.ttf,BoldFont=cmunbx.ttf,ItalicFont=cmunti.ttf,BoldItalicFont=cmunbi.ttf]{cmunrm.ttf}\setmonofont[Path=/usr/share/fonts/truetype/cmu/,UprightFont=cmuntt.ttf,BoldFont=cmuntb.ttf,ItalicFont=cmunit.ttf,BoldItalicFont=cmuntx.ttf]{cmunrm.ttf} \^{} &\hspace*{0pt}\ignorespaces{}\hspace*{0pt} \^{} &\hspace*{0pt}\ignorespaces{}\hspace*{0pt} – &\hspace*{0pt}\ignorespaces{}\hspace*{0pt} \^{} &\hspace*{0pt}\ignorespaces{}\hspace*{0pt}   &\hspace*{0pt}\ignorespaces{}\hspace*{0pt} \^{} &\hspace*{0pt}\ignorespaces{}\hspace*{0pt} – &\hspace*{0pt}\ignorespaces{}\hspace*{0pt} \^{} &\hspace*{0pt}\ignorespaces{}\hspace*{0pt}   &\hspace*{0pt}\ignorespaces{}\hspace*{0pt}   &\hspace*{0pt}\ignorespaces{}\hspace*{0pt}   &\hspace*{0pt}\ignorespaces{}\hspace*{0pt}   &\hspace*{0pt}\ignorespaces{}\hspace*{0pt} \^{} &\hspace*{0pt}\ignorespaces{}\hspace*{0pt} \^{} &\hspace*{0pt}\ignorespaces{}\hspace*{0pt} – &\hspace*{0pt}\ignorespaces{}\hspace*{0pt} \^{} &\hspace*{0pt}\ignorespaces{}\hspace*{0pt}   &\hspace*{0pt}\ignorespaces{}\hspace*{0pt}   &\hspace*{0pt}\ignorespaces{}\hspace*{0pt}\\ \hline {\bfseries \hspace*{0pt}\ignorespaces{}\hspace*{0pt} {\ttfamily \setmainfont[Path=/usr/share/fonts/truetype/cmu/,UprightFont=cmunrm.ttf,BoldFont=cmunbx.ttf,ItalicFont=cmunti.ttf,BoldItalicFont=cmunbi.ttf]{cmuntt.ttf}\setmonofont[Path=/usr/share/fonts/truetype/cmu/,UprightFont=cmuntt.ttf,BoldFont=cmuntb.ttf,ItalicFont=cmunit.ttf,BoldItalicFont=cmuntx.ttf]{cmuntt.ttf}\ttfamily volumes}}&\hspace*{0pt}\ignorespaces{}\hspace*{0pt}{$\text{ }$}\setmainfont[Path=/usr/share/fonts/truetype/cmu/,UprightFont=cmunrm.ttf,BoldFont=cmunbx.ttf,ItalicFont=cmunti.ttf,BoldItalicFont=cmunbi.ttf]{cmunrm.ttf}\setmonofont[Path=/usr/share/fonts/truetype/cmu/,UprightFont=cmuntt.ttf,BoldFont=cmuntb.ttf,ItalicFont=cmunit.ttf,BoldItalicFont=cmuntx.ttf]{cmunrm.ttf}   &\hspace*{0pt}\ignorespaces{}\hspace*{0pt} \^{} &\hspace*{0pt}\ignorespaces{}\hspace*{0pt} \^{} &\hspace*{0pt}\ignorespaces{}\hspace*{0pt} \^{} &\hspace*{0pt}\ignorespaces{}\hspace*{0pt}   &\hspace*{0pt}\ignorespaces{}\hspace*{0pt} \^{} &\hspace*{0pt}\ignorespaces{}\hspace*{0pt} \^{} &\hspace*{0pt}\ignorespaces{}\hspace*{0pt} \^{} &\hspace*{0pt}\ignorespaces{}\hspace*{0pt}   &\hspace*{0pt}\ignorespaces{}\hspace*{0pt}   &\hspace*{0pt}\ignorespaces{}\hspace*{0pt}   &\hspace*{0pt}\ignorespaces{}\hspace*{0pt}   &\hspace*{0pt}\ignorespaces{}\hspace*{0pt}   &\hspace*{0pt}\ignorespaces{}\hspace*{0pt} \^{} &\hspace*{0pt}\ignorespaces{}\hspace*{0pt} \^{} &\hspace*{0pt}\ignorespaces{}\hspace*{0pt} \^{} &\hspace*{0pt}\ignorespaces{}\hspace*{0pt}   &\hspace*{0pt}\ignorespaces{}\hspace*{0pt}   &\hspace*{0pt}\ignorespaces{}\hspace*{0pt}\\ \hline {\bfseries \hspace*{0pt}\ignorespaces{}\hspace*{0pt} {\ttfamily \setmainfont[Path=/usr/share/fonts/truetype/cmu/,UprightFont=cmunrm.ttf,BoldFont=cmunbx.ttf,ItalicFont=cmunti.ttf,BoldItalicFont=cmunbi.ttf]{cmuntt.ttf}\setmonofont[Path=/usr/share/fonts/truetype/cmu/,UprightFont=cmuntt.ttf,BoldFont=cmuntb.ttf,ItalicFont=cmunit.ttf,BoldItalicFont=cmuntx.ttf]{cmuntt.ttf}\ttfamily version}}&\hspace*{0pt}\ignorespaces{}\hspace*{0pt}{$\text{ }$}\setmainfont[Path=/usr/share/fonts/truetype/cmu/,UprightFont=cmunrm.ttf,BoldFont=cmunbx.ttf,ItalicFont=cmunti.ttf,BoldItalicFont=cmunbi.ttf]{cmunrm.ttf}\setmonofont[Path=/usr/share/fonts/truetype/cmu/,UprightFont=cmuntt.ttf,BoldFont=cmuntb.ttf,ItalicFont=cmunit.ttf,BoldItalicFont=cmuntx.ttf]{cmunrm.ttf} \^{} &\hspace*{0pt}\ignorespaces{}\hspace*{0pt}   &\hspace*{0pt}\ignorespaces{}\hspace*{0pt}   &\hspace*{0pt}\ignorespaces{}\hspace*{0pt}   &\hspace*{0pt}\ignorespaces{}\hspace*{0pt}   &\hspace*{0pt}\ignorespaces{}\hspace*{0pt}   &\hspace*{0pt}\ignorespaces{}\hspace*{0pt}   &\hspace*{0pt}\ignorespaces{}\hspace*{0pt}   &\hspace*{0pt}\ignorespaces{}\hspace*{0pt} \^{} &\hspace*{0pt}\ignorespaces{}\hspace*{0pt} \^{} &\hspace*{0pt}\ignorespaces{}\hspace*{0pt} \^{} &\hspace*{0pt}\ignorespaces{}\hspace*{0pt} \^{} &\hspace*{0pt}\ignorespaces{}\hspace*{0pt}   &\hspace*{0pt}\ignorespaces{}\hspace*{0pt}   &\hspace*{0pt}\ignorespaces{}\hspace*{0pt}   &\hspace*{0pt}\ignorespaces{}\hspace*{0pt}   &\hspace*{0pt}\ignorespaces{}\hspace*{0pt} \^{} &\hspace*{0pt}\ignorespaces{}\hspace*{0pt}   &\hspace*{0pt}\ignorespaces{}\hspace*{0pt}\\ \hline {\bfseries \hspace*{0pt}\ignorespaces{}\hspace*{0pt} {\ttfamily \setmainfont[Path=/usr/share/fonts/truetype/cmu/,UprightFont=cmunrm.ttf,BoldFont=cmunbx.ttf,ItalicFont=cmunti.ttf,BoldItalicFont=cmunbi.ttf]{cmuntt.ttf}\setmonofont[Path=/usr/share/fonts/truetype/cmu/,UprightFont=cmuntt.ttf,BoldFont=cmuntb.ttf,ItalicFont=cmunit.ttf,BoldItalicFont=cmuntx.ttf]{cmuntt.ttf}\ttfamily doi}\setmainfont[Path=/usr/share/fonts/truetype/cmu/,UprightFont=cmunrm.ttf,BoldFont=cmunbx.ttf,ItalicFont=cmunti.ttf,BoldItalicFont=cmunbi.ttf]{cmunrm.ttf}\setmonofont[Path=/usr/share/fonts/truetype/cmu/,UprightFont=cmuntt.ttf,BoldFont=cmuntb.ttf,ItalicFont=cmunit.ttf,BoldItalicFont=cmuntx.ttf]{cmunrm.ttf}, {\ttfamily \setmainfont[Path=/usr/share/fonts/truetype/cmu/,UprightFont=cmunrm.ttf,BoldFont=cmunbx.ttf,ItalicFont=cmunti.ttf,BoldItalicFont=cmunbi.ttf]{cmuntt.ttf}\setmonofont[Path=/usr/share/fonts/truetype/cmu/,UprightFont=cmuntt.ttf,BoldFont=cmuntb.ttf,ItalicFont=cmunit.ttf,BoldItalicFont=cmuntx.ttf]{cmuntt.ttf}\ttfamily eprint}\setmainfont[Path=/usr/share/fonts/truetype/cmu/,UprightFont=cmunrm.ttf,BoldFont=cmunbx.ttf,ItalicFont=cmunti.ttf,BoldItalicFont=cmunbi.ttf]{cmunrm.ttf}\setmonofont[Path=/usr/share/fonts/truetype/cmu/,UprightFont=cmuntt.ttf,BoldFont=cmuntb.ttf,ItalicFont=cmunit.ttf,BoldItalicFont=cmuntx.ttf]{cmunrm.ttf}, {\ttfamily \setmainfont[Path=/usr/share/fonts/truetype/cmu/,UprightFont=cmunrm.ttf,BoldFont=cmunbx.ttf,ItalicFont=cmunti.ttf,BoldItalicFont=cmunbi.ttf]{cmuntt.ttf}\setmonofont[Path=/usr/share/fonts/truetype/cmu/,UprightFont=cmuntt.ttf,BoldFont=cmuntb.ttf,ItalicFont=cmunit.ttf,BoldItalicFont=cmuntx.ttf]{cmuntt.ttf}\ttfamily eprintclass}\setmainfont[Path=/usr/share/fonts/truetype/cmu/,UprightFont=cmunrm.ttf,BoldFont=cmunbx.ttf,ItalicFont=cmunti.ttf,BoldItalicFont=cmunbi.ttf]{cmunrm.ttf}\setmonofont[Path=/usr/share/fonts/truetype/cmu/,UprightFont=cmuntt.ttf,BoldFont=cmuntb.ttf,ItalicFont=cmunit.ttf,BoldItalicFont=cmuntx.ttf]{cmunrm.ttf}, {\ttfamily \setmainfont[Path=/usr/share/fonts/truetype/cmu/,UprightFont=cmunrm.ttf,BoldFont=cmunbx.ttf,ItalicFont=cmunti.ttf,BoldItalicFont=cmunbi.ttf]{cmuntt.ttf}\setmonofont[Path=/usr/share/fonts/truetype/cmu/,UprightFont=cmuntt.ttf,BoldFont=cmuntb.ttf,ItalicFont=cmunit.ttf,BoldItalicFont=cmuntx.ttf]{cmuntt.ttf}\ttfamily eprinttype}}&\hspace*{0pt}\ignorespaces{}\hspace*{0pt}{$\text{ }$}\setmainfont[Path=/usr/share/fonts/truetype/cmu/,UprightFont=cmunrm.ttf,BoldFont=cmunbx.ttf,ItalicFont=cmunti.ttf,BoldItalicFont=cmunbi.ttf]{cmunrm.ttf}\setmonofont[Path=/usr/share/fonts/truetype/cmu/,UprightFont=cmuntt.ttf,BoldFont=cmuntb.ttf,ItalicFont=cmunit.ttf,BoldItalicFont=cmuntx.ttf]{cmunrm.ttf} \^{} &\hspace*{0pt}\ignorespaces{}\hspace*{0pt} \^{} &\hspace*{0pt}\ignorespaces{}\hspace*{0pt} \^{} &\hspace*{0pt}\ignorespaces{}\hspace*{0pt} \^{} &\hspace*{0pt}\ignorespaces{}\hspace*{0pt} \^{} &\hspace*{0pt}\ignorespaces{}\hspace*{0pt} \^{} &\hspace*{0pt}\ignorespaces{}\hspace*{0pt} \^{} &\hspace*{0pt}\ignorespaces{}\hspace*{0pt} \^{} &\hspace*{0pt}\ignorespaces{}\hspace*{0pt} \^{} &\hspace*{0pt}\ignorespaces{}\hspace*{0pt} \^{} &\hspace*{0pt}\ignorespaces{}\hspace*{0pt}   &\hspace*{0pt}\ignorespaces{}\hspace*{0pt} \^{} &\hspace*{0pt}\ignorespaces{}\hspace*{0pt} \^{} &\hspace*{0pt}\ignorespaces{}\hspace*{0pt} \^{} &\hspace*{0pt}\ignorespaces{}\hspace*{0pt} \^{} &\hspace*{0pt}\ignorespaces{}\hspace*{0pt} \^{} &\hspace*{0pt}\ignorespaces{}\hspace*{0pt} \^{} &\hspace*{0pt}\ignorespaces{}\hspace*{0pt} \^{} &\hspace*{0pt}\ignorespaces{}\hspace*{0pt} \^{}\\ \hline {\bfseries \hspace*{0pt}\ignorespaces{}\hspace*{0pt} {\ttfamily \setmainfont[Path=/usr/share/fonts/truetype/cmu/,UprightFont=cmunrm.ttf,BoldFont=cmunbx.ttf,ItalicFont=cmunti.ttf,BoldItalicFont=cmunbi.ttf]{cmuntt.ttf}\setmonofont[Path=/usr/share/fonts/truetype/cmu/,UprightFont=cmuntt.ttf,BoldFont=cmuntb.ttf,ItalicFont=cmunit.ttf,BoldItalicFont=cmuntx.ttf]{cmuntt.ttf}\ttfamily eid}}&\hspace*{0pt}\ignorespaces{}\hspace*{0pt}{$\text{ }$}\setmainfont[Path=/usr/share/fonts/truetype/cmu/,UprightFont=cmunrm.ttf,BoldFont=cmunbx.ttf,ItalicFont=cmunti.ttf,BoldItalicFont=cmunbi.ttf]{cmunrm.ttf}\setmonofont[Path=/usr/share/fonts/truetype/cmu/,UprightFont=cmuntt.ttf,BoldFont=cmuntb.ttf,ItalicFont=cmunit.ttf,BoldItalicFont=cmuntx.ttf]{cmunrm.ttf} \^{} &\hspace*{0pt}\ignorespaces{}\hspace*{0pt}   &\hspace*{0pt}\ignorespaces{}\hspace*{0pt}   &\hspace*{0pt}\ignorespaces{}\hspace*{0pt}   &\hspace*{0pt}\ignorespaces{}\hspace*{0pt}   &\hspace*{0pt}\ignorespaces{}\hspace*{0pt}   &\hspace*{0pt}\ignorespaces{}\hspace*{0pt}   &\hspace*{0pt}\ignorespaces{}\hspace*{0pt}   &\hspace*{0pt}\ignorespaces{}\hspace*{0pt}   &\hspace*{0pt}\ignorespaces{}\hspace*{0pt}   &\hspace*{0pt}\ignorespaces{}\hspace*{0pt}   &\hspace*{0pt}\ignorespaces{}\hspace*{0pt}   &\hspace*{0pt}\ignorespaces{}\hspace*{0pt}   &\hspace*{0pt}\ignorespaces{}\hspace*{0pt}   &\hspace*{0pt}\ignorespaces{}\hspace*{0pt}   &\hspace*{0pt}\ignorespaces{}\hspace*{0pt}   &\hspace*{0pt}\ignorespaces{}\hspace*{0pt}   &\hspace*{0pt}\ignorespaces{}\hspace*{0pt}   &\hspace*{0pt}\ignorespaces{}\hspace*{0pt}\\ \hline {\bfseries \hspace*{0pt}\ignorespaces{}\hspace*{0pt} {\ttfamily \setmainfont[Path=/usr/share/fonts/truetype/cmu/,UprightFont=cmunrm.ttf,BoldFont=cmunbx.ttf,ItalicFont=cmunti.ttf,BoldItalicFont=cmunbi.ttf]{cmuntt.ttf}\setmonofont[Path=/usr/share/fonts/truetype/cmu/,UprightFont=cmuntt.ttf,BoldFont=cmuntb.ttf,ItalicFont=cmunit.ttf,BoldItalicFont=cmuntx.ttf]{cmuntt.ttf}\ttfamily isbn}}&\hspace*{0pt}\ignorespaces{}\hspace*{0pt}{$\text{ }$}\setmainfont[Path=/usr/share/fonts/truetype/cmu/,UprightFont=cmunrm.ttf,BoldFont=cmunbx.ttf,ItalicFont=cmunti.ttf,BoldItalicFont=cmunbi.ttf]{cmunrm.ttf}\setmonofont[Path=/usr/share/fonts/truetype/cmu/,UprightFont=cmuntt.ttf,BoldFont=cmuntb.ttf,ItalicFont=cmunit.ttf,BoldItalicFont=cmuntx.ttf]{cmunrm.ttf}   &\hspace*{0pt}\ignorespaces{}\hspace*{0pt} \^{} &\hspace*{0pt}\ignorespaces{}\hspace*{0pt} \^{} &\hspace*{0pt}\ignorespaces{}\hspace*{0pt} \^{} &\hspace*{0pt}\ignorespaces{}\hspace*{0pt}   &\hspace*{0pt}\ignorespaces{}\hspace*{0pt} \^{} &\hspace*{0pt}\ignorespaces{}\hspace*{0pt} \^{} &\hspace*{0pt}\ignorespaces{}\hspace*{0pt} \^{} &\hspace*{0pt}\ignorespaces{}\hspace*{0pt} \^{} &\hspace*{0pt}\ignorespaces{}\hspace*{0pt}   &\hspace*{0pt}\ignorespaces{}\hspace*{0pt}   &\hspace*{0pt}\ignorespaces{}\hspace*{0pt}   &\hspace*{0pt}\ignorespaces{}\hspace*{0pt}   &\hspace*{0pt}\ignorespaces{}\hspace*{0pt} \^{} &\hspace*{0pt}\ignorespaces{}\hspace*{0pt} \^{} &\hspace*{0pt}\ignorespaces{}\hspace*{0pt} \^{} &\hspace*{0pt}\ignorespaces{}\hspace*{0pt}   &\hspace*{0pt}\ignorespaces{}\hspace*{0pt} \^{} &\hspace*{0pt}\ignorespaces{}\hspace*{0pt} \^{} \\ \hline {\bfseries \hspace*{0pt}\ignorespaces{}\hspace*{0pt} {\ttfamily \setmainfont[Path=/usr/share/fonts/truetype/cmu/,UprightFont=cmunrm.ttf,BoldFont=cmunbx.ttf,ItalicFont=cmunti.ttf,BoldItalicFont=cmunbi.ttf]{cmuntt.ttf}\setmonofont[Path=/usr/share/fonts/truetype/cmu/,UprightFont=cmuntt.ttf,BoldFont=cmuntb.ttf,ItalicFont=cmunit.ttf,BoldItalicFont=cmuntx.ttf]{cmuntt.ttf}\ttfamily isrn}}&\hspace*{0pt}\ignorespaces{}\hspace*{0pt}{$\text{ }$}\setmainfont[Path=/usr/share/fonts/truetype/cmu/,UprightFont=cmunrm.ttf,BoldFont=cmunbx.ttf,ItalicFont=cmunti.ttf,BoldItalicFont=cmunbi.ttf]{cmunrm.ttf}\setmonofont[Path=/usr/share/fonts/truetype/cmu/,UprightFont=cmuntt.ttf,BoldFont=cmuntb.ttf,ItalicFont=cmunit.ttf,BoldItalicFont=cmuntx.ttf]{cmunrm.ttf}   &\hspace*{0pt}\ignorespaces{}\hspace*{0pt}   &\hspace*{0pt}\ignorespaces{}\hspace*{0pt}   &\hspace*{0pt}\ignorespaces{}\hspace*{0pt}   &\hspace*{0pt}\ignorespaces{}\hspace*{0pt}   &\hspace*{0pt}\ignorespaces{}\hspace*{0pt}   &\hspace*{0pt}\ignorespaces{}\hspace*{0pt}   &\hspace*{0pt}\ignorespaces{}\hspace*{0pt}   &\hspace*{0pt}\ignorespaces{}\hspace*{0pt}   &\hspace*{0pt}\ignorespaces{}\hspace*{0pt}   &\hspace*{0pt}\ignorespaces{}\hspace*{0pt}   &\hspace*{0pt}\ignorespaces{}\hspace*{0pt}   &\hspace*{0pt}\ignorespaces{}\hspace*{0pt}   &\hspace*{0pt}\ignorespaces{}\hspace*{0pt}   &\hspace*{0pt}\ignorespaces{}\hspace*{0pt}   &\hspace*{0pt}\ignorespaces{}\hspace*{0pt}   &\hspace*{0pt}\ignorespaces{}\hspace*{0pt} \^{} &\hspace*{0pt}\ignorespaces{}\hspace*{0pt}   &\hspace*{0pt}\ignorespaces{}\hspace*{0pt}\\ \hline {\bfseries \hspace*{0pt}\ignorespaces{}\hspace*{0pt} {\ttfamily \setmainfont[Path=/usr/share/fonts/truetype/cmu/,UprightFont=cmunrm.ttf,BoldFont=cmunbx.ttf,ItalicFont=cmunti.ttf,BoldItalicFont=cmunbi.ttf]{cmuntt.ttf}\setmonofont[Path=/usr/share/fonts/truetype/cmu/,UprightFont=cmuntt.ttf,BoldFont=cmuntb.ttf,ItalicFont=cmunit.ttf,BoldItalicFont=cmuntx.ttf]{cmuntt.ttf}\ttfamily issn}}&\hspace*{0pt}\ignorespaces{}\hspace*{0pt}{$\text{ }$}\setmainfont[Path=/usr/share/fonts/truetype/cmu/,UprightFont=cmunrm.ttf,BoldFont=cmunbx.ttf,ItalicFont=cmunti.ttf,BoldItalicFont=cmunbi.ttf]{cmunrm.ttf}\setmonofont[Path=/usr/share/fonts/truetype/cmu/,UprightFont=cmuntt.ttf,BoldFont=cmuntb.ttf,ItalicFont=cmunit.ttf,BoldItalicFont=cmuntx.ttf]{cmunrm.ttf} \^{} &\hspace*{0pt}\ignorespaces{}\hspace*{0pt}   &\hspace*{0pt}\ignorespaces{}\hspace*{0pt}   &\hspace*{0pt}\ignorespaces{}\hspace*{0pt}   &\hspace*{0pt}\ignorespaces{}\hspace*{0pt}   &\hspace*{0pt}\ignorespaces{}\hspace*{0pt}   &\hspace*{0pt}\ignorespaces{}\hspace*{0pt}   &\hspace*{0pt}\ignorespaces{}\hspace*{0pt}   &\hspace*{0pt}\ignorespaces{}\hspace*{0pt}   &\hspace*{0pt}\ignorespaces{}\hspace*{0pt}   &\hspace*{0pt}\ignorespaces{}\hspace*{0pt}   &\hspace*{0pt}\ignorespaces{}\hspace*{0pt}   &\hspace*{0pt}\ignorespaces{}\hspace*{0pt} \^{} &\hspace*{0pt}\ignorespaces{}\hspace*{0pt}   &\hspace*{0pt}\ignorespaces{}\hspace*{0pt}   &\hspace*{0pt}\ignorespaces{}\hspace*{0pt}   &\hspace*{0pt}\ignorespaces{}\hspace*{0pt}   &\hspace*{0pt}\ignorespaces{}\hspace*{0pt}   &\hspace*{0pt}\ignorespaces{}\hspace*{0pt}\\ \hline {\bfseries \hspace*{0pt}\ignorespaces{}\hspace*{0pt} {\ttfamily \setmainfont[Path=/usr/share/fonts/truetype/cmu/,UprightFont=cmunrm.ttf,BoldFont=cmunbx.ttf,ItalicFont=cmunti.ttf,BoldItalicFont=cmunbi.ttf]{cmuntt.ttf}\setmonofont[Path=/usr/share/fonts/truetype/cmu/,UprightFont=cmuntt.ttf,BoldFont=cmuntb.ttf,ItalicFont=cmunit.ttf,BoldItalicFont=cmuntx.ttf]{cmuntt.ttf}\ttfamily isan}\setmainfont[Path=/usr/share/fonts/truetype/cmu/,UprightFont=cmunrm.ttf,BoldFont=cmunbx.ttf,ItalicFont=cmunti.ttf,BoldItalicFont=cmunbi.ttf]{cmunrm.ttf}\setmonofont[Path=/usr/share/fonts/truetype/cmu/,UprightFont=cmuntt.ttf,BoldFont=cmuntb.ttf,ItalicFont=cmunit.ttf,BoldItalicFont=cmuntx.ttf]{cmunrm.ttf}, {\ttfamily \setmainfont[Path=/usr/share/fonts/truetype/cmu/,UprightFont=cmunrm.ttf,BoldFont=cmunbx.ttf,ItalicFont=cmunti.ttf,BoldItalicFont=cmunbi.ttf]{cmuntt.ttf}\setmonofont[Path=/usr/share/fonts/truetype/cmu/,UprightFont=cmuntt.ttf,BoldFont=cmuntb.ttf,ItalicFont=cmunit.ttf,BoldItalicFont=cmuntx.ttf]{cmuntt.ttf}\ttfamily ismn}\setmainfont[Path=/usr/share/fonts/truetype/cmu/,UprightFont=cmunrm.ttf,BoldFont=cmunbx.ttf,ItalicFont=cmunti.ttf,BoldItalicFont=cmunbi.ttf]{cmunrm.ttf}\setmonofont[Path=/usr/share/fonts/truetype/cmu/,UprightFont=cmuntt.ttf,BoldFont=cmuntb.ttf,ItalicFont=cmunit.ttf,BoldItalicFont=cmuntx.ttf]{cmunrm.ttf}, {\ttfamily \setmainfont[Path=/usr/share/fonts/truetype/cmu/,UprightFont=cmunrm.ttf,BoldFont=cmunbx.ttf,ItalicFont=cmunti.ttf,BoldItalicFont=cmunbi.ttf]{cmuntt.ttf}\setmonofont[Path=/usr/share/fonts/truetype/cmu/,UprightFont=cmuntt.ttf,BoldFont=cmuntb.ttf,ItalicFont=cmunit.ttf,BoldItalicFont=cmuntx.ttf]{cmuntt.ttf}\ttfamily iswc}}&\hspace*{0pt}\ignorespaces{}\hspace*{0pt}{$\text{ }$}\setmainfont[Path=/usr/share/fonts/truetype/cmu/,UprightFont=cmunrm.ttf,BoldFont=cmunbx.ttf,ItalicFont=cmunti.ttf,BoldItalicFont=cmunbi.ttf]{cmunrm.ttf}\setmonofont[Path=/usr/share/fonts/truetype/cmu/,UprightFont=cmuntt.ttf,BoldFont=cmuntb.ttf,ItalicFont=cmunit.ttf,BoldItalicFont=cmuntx.ttf]{cmunrm.ttf}   &\hspace*{0pt}\ignorespaces{}\hspace*{0pt}   &\hspace*{0pt}\ignorespaces{}\hspace*{0pt}   &\hspace*{0pt}\ignorespaces{}\hspace*{0pt}   &\hspace*{0pt}\ignorespaces{}\hspace*{0pt}   &\hspace*{0pt}\ignorespaces{}\hspace*{0pt}   &\hspace*{0pt}\ignorespaces{}\hspace*{0pt}   &\hspace*{0pt}\ignorespaces{}\hspace*{0pt}   &\hspace*{0pt}\ignorespaces{}\hspace*{0pt}   &\hspace*{0pt}\ignorespaces{}\hspace*{0pt}   &\hspace*{0pt}\ignorespaces{}\hspace*{0pt}   &\hspace*{0pt}\ignorespaces{}\hspace*{0pt}   &\hspace*{0pt}\ignorespaces{}\hspace*{0pt}   &\hspace*{0pt}\ignorespaces{}\hspace*{0pt}   &\hspace*{0pt}\ignorespaces{}\hspace*{0pt}   &\hspace*{0pt}\ignorespaces{}\hspace*{0pt}   &\hspace*{0pt}\ignorespaces{}\hspace*{0pt}   &\hspace*{0pt}\ignorespaces{}\hspace*{0pt}   &\hspace*{0pt}\ignorespaces{}\hspace*{0pt}\\ \hline {\bfseries \hspace*{0pt}\ignorespaces{}\hspace*{0pt} {\ttfamily \setmainfont[Path=/usr/share/fonts/truetype/cmu/,UprightFont=cmunrm.ttf,BoldFont=cmunbx.ttf,ItalicFont=cmunti.ttf,BoldItalicFont=cmunbi.ttf]{cmuntt.ttf}\setmonofont[Path=/usr/share/fonts/truetype/cmu/,UprightFont=cmuntt.ttf,BoldFont=cmuntb.ttf,ItalicFont=cmunit.ttf,BoldItalicFont=cmuntx.ttf]{cmuntt.ttf}\ttfamily url}}&\hspace*{0pt}\ignorespaces{}\hspace*{0pt}{$\text{ }$}\setmainfont[Path=/usr/share/fonts/truetype/cmu/,UprightFont=cmunrm.ttf,BoldFont=cmunbx.ttf,ItalicFont=cmunti.ttf,BoldItalicFont=cmunbi.ttf]{cmunrm.ttf}\setmonofont[Path=/usr/share/fonts/truetype/cmu/,UprightFont=cmuntt.ttf,BoldFont=cmuntb.ttf,ItalicFont=cmunit.ttf,BoldItalicFont=cmuntx.ttf]{cmunrm.ttf} \^{} &\hspace*{0pt}\ignorespaces{}\hspace*{0pt} \^{} &\hspace*{0pt}\ignorespaces{}\hspace*{0pt} \^{} &\hspace*{0pt}\ignorespaces{}\hspace*{0pt} \^{} &\hspace*{0pt}\ignorespaces{}\hspace*{0pt} \^{} &\hspace*{0pt}\ignorespaces{}\hspace*{0pt} \^{} &\hspace*{0pt}\ignorespaces{}\hspace*{0pt} \^{} &\hspace*{0pt}\ignorespaces{}\hspace*{0pt} \^{} &\hspace*{0pt}\ignorespaces{}\hspace*{0pt} \^{} &\hspace*{0pt}\ignorespaces{}\hspace*{0pt} \^{} &\hspace*{0pt}\ignorespaces{}\hspace*{0pt} + &\hspace*{0pt}\ignorespaces{}\hspace*{0pt} \^{} &\hspace*{0pt}\ignorespaces{}\hspace*{0pt} \^{} &\hspace*{0pt}\ignorespaces{}\hspace*{0pt} \^{} &\hspace*{0pt}\ignorespaces{}\hspace*{0pt} \^{} &\hspace*{0pt}\ignorespaces{}\hspace*{0pt} \^{} &\hspace*{0pt}\ignorespaces{}\hspace*{0pt} \^{} &\hspace*{0pt}\ignorespaces{}\hspace*{0pt} \^{} &\hspace*{0pt}\ignorespaces{}\hspace*{0pt} \^{} \\ \hline {\bfseries \hspace*{0pt}\ignorespaces{}\hspace*{0pt} {\ttfamily \setmainfont[Path=/usr/share/fonts/truetype/cmu/,UprightFont=cmunrm.ttf,BoldFont=cmunbx.ttf,ItalicFont=cmunti.ttf,BoldItalicFont=cmunbi.ttf]{cmuntt.ttf}\setmonofont[Path=/usr/share/fonts/truetype/cmu/,UprightFont=cmuntt.ttf,BoldFont=cmuntb.ttf,ItalicFont=cmunit.ttf,BoldItalicFont=cmuntx.ttf]{cmuntt.ttf}\ttfamily urldate}}&\hspace*{0pt}\ignorespaces{}\hspace*{0pt}{$\text{ }$}\setmainfont[Path=/usr/share/fonts/truetype/cmu/,UprightFont=cmunrm.ttf,BoldFont=cmunbx.ttf,ItalicFont=cmunti.ttf,BoldItalicFont=cmunbi.ttf]{cmunrm.ttf}\setmonofont[Path=/usr/share/fonts/truetype/cmu/,UprightFont=cmuntt.ttf,BoldFont=cmuntb.ttf,ItalicFont=cmunit.ttf,BoldItalicFont=cmuntx.ttf]{cmunrm.ttf} \^{} &\hspace*{0pt}\ignorespaces{}\hspace*{0pt} \^{} &\hspace*{0pt}\ignorespaces{}\hspace*{0pt} \^{} &\hspace*{0pt}\ignorespaces{}\hspace*{0pt} \^{} &\hspace*{0pt}\ignorespaces{}\hspace*{0pt} \^{} &\hspace*{0pt}\ignorespaces{}\hspace*{0pt} \^{} &\hspace*{0pt}\ignorespaces{}\hspace*{0pt} \^{} &\hspace*{0pt}\ignorespaces{}\hspace*{0pt} \^{} &\hspace*{0pt}\ignorespaces{}\hspace*{0pt} \^{} &\hspace*{0pt}\ignorespaces{}\hspace*{0pt} \^{} &\hspace*{0pt}\ignorespaces{}\hspace*{0pt} \^{} &\hspace*{0pt}\ignorespaces{}\hspace*{0pt} \^{} &\hspace*{0pt}\ignorespaces{}\hspace*{0pt} \^{} &\hspace*{0pt}\ignorespaces{}\hspace*{0pt} \^{} &\hspace*{0pt}\ignorespaces{}\hspace*{0pt} \^{} &\hspace*{0pt}\ignorespaces{}\hspace*{0pt} \^{} &\hspace*{0pt}\ignorespaces{}\hspace*{0pt} \^{} &\hspace*{0pt}\ignorespaces{}\hspace*{0pt} \^{} &\hspace*{0pt}\ignorespaces{}\hspace*{0pt} \^{} \\ \hline {\bfseries \hspace*{0pt}\ignorespaces{}\hspace*{0pt} {\ttfamily \setmainfont[Path=/usr/share/fonts/truetype/cmu/,UprightFont=cmunrm.ttf,BoldFont=cmunbx.ttf,ItalicFont=cmunti.ttf,BoldItalicFont=cmunbi.ttf]{cmuntt.ttf}\setmonofont[Path=/usr/share/fonts/truetype/cmu/,UprightFont=cmuntt.ttf,BoldFont=cmuntb.ttf,ItalicFont=cmunit.ttf,BoldItalicFont=cmuntx.ttf]{cmuntt.ttf}\ttfamily location}}&\hspace*{0pt}\ignorespaces{}\hspace*{0pt}{$\text{ }$}\setmainfont[Path=/usr/share/fonts/truetype/cmu/,UprightFont=cmunrm.ttf,BoldFont=cmunbx.ttf,ItalicFont=cmunti.ttf,BoldItalicFont=cmunbi.ttf]{cmunrm.ttf}\setmonofont[Path=/usr/share/fonts/truetype/cmu/,UprightFont=cmuntt.ttf,BoldFont=cmuntb.ttf,ItalicFont=cmunit.ttf,BoldItalicFont=cmuntx.ttf]{cmunrm.ttf}   &\hspace*{0pt}\ignorespaces{}\hspace*{0pt} \^{} &\hspace*{0pt}\ignorespaces{}\hspace*{0pt} \^{} &\hspace*{0pt}\ignorespaces{}\hspace*{0pt} \^{} &\hspace*{0pt}\ignorespaces{}\hspace*{0pt} \^{} &\hspace*{0pt}\ignorespaces{}\hspace*{0pt} \^{} &\hspace*{0pt}\ignorespaces{}\hspace*{0pt} \^{} &\hspace*{0pt}\ignorespaces{}\hspace*{0pt} \^{} &\hspace*{0pt}\ignorespaces{}\hspace*{0pt} \^{} &\hspace*{0pt}\ignorespaces{}\hspace*{0pt} \^{} &\hspace*{0pt}\ignorespaces{}\hspace*{0pt}   &\hspace*{0pt}\ignorespaces{}\hspace*{0pt} \^{} &\hspace*{0pt}\ignorespaces{}\hspace*{0pt}   &\hspace*{0pt}\ignorespaces{}\hspace*{0pt} \^{} &\hspace*{0pt}\ignorespaces{}\hspace*{0pt} \^{} &\hspace*{0pt}\ignorespaces{}\hspace*{0pt} \^{} &\hspace*{0pt}\ignorespaces{}\hspace*{0pt} \^{} &\hspace*{0pt}\ignorespaces{}\hspace*{0pt} \^{} &\hspace*{0pt}\ignorespaces{}\hspace*{0pt} \^{} \\ \hline {\bfseries \hspace*{0pt}\ignorespaces{}\hspace*{0pt} {\ttfamily \setmainfont[Path=/usr/share/fonts/truetype/cmu/,UprightFont=cmunrm.ttf,BoldFont=cmunbx.ttf,ItalicFont=cmunti.ttf,BoldItalicFont=cmunbi.ttf]{cmuntt.ttf}\setmonofont[Path=/usr/share/fonts/truetype/cmu/,UprightFont=cmuntt.ttf,BoldFont=cmuntb.ttf,ItalicFont=cmunit.ttf,BoldItalicFont=cmuntx.ttf]{cmuntt.ttf}\ttfamily publisher}}&\hspace*{0pt}\ignorespaces{}\hspace*{0pt}{$\text{ }$}\setmainfont[Path=/usr/share/fonts/truetype/cmu/,UprightFont=cmunrm.ttf,BoldFont=cmunbx.ttf,ItalicFont=cmunti.ttf,BoldItalicFont=cmunbi.ttf]{cmunrm.ttf}\setmonofont[Path=/usr/share/fonts/truetype/cmu/,UprightFont=cmuntt.ttf,BoldFont=cmuntb.ttf,ItalicFont=cmunit.ttf,BoldItalicFont=cmuntx.ttf]{cmunrm.ttf}   &\hspace*{0pt}\ignorespaces{}\hspace*{0pt} \^{} &\hspace*{0pt}\ignorespaces{}\hspace*{0pt} \^{} &\hspace*{0pt}\ignorespaces{}\hspace*{0pt} \^{} &\hspace*{0pt}\ignorespaces{}\hspace*{0pt}   &\hspace*{0pt}\ignorespaces{}\hspace*{0pt} \^{} &\hspace*{0pt}\ignorespaces{}\hspace*{0pt} \^{} &\hspace*{0pt}\ignorespaces{}\hspace*{0pt} \^{} &\hspace*{0pt}\ignorespaces{}\hspace*{0pt} \^{} &\hspace*{0pt}\ignorespaces{}\hspace*{0pt}   &\hspace*{0pt}\ignorespaces{}\hspace*{0pt}   &\hspace*{0pt}\ignorespaces{}\hspace*{0pt}   &\hspace*{0pt}\ignorespaces{}\hspace*{0pt}   &\hspace*{0pt}\ignorespaces{}\hspace*{0pt} \^{} &\hspace*{0pt}\ignorespaces{}\hspace*{0pt} \^{} &\hspace*{0pt}\ignorespaces{}\hspace*{0pt} \^{} &\hspace*{0pt}\ignorespaces{}\hspace*{0pt}   &\hspace*{0pt}\ignorespaces{}\hspace*{0pt}   &\hspace*{0pt}\ignorespaces{}\hspace*{0pt}\\ \hline {\bfseries \hspace*{0pt}\ignorespaces{}\hspace*{0pt} {\ttfamily \setmainfont[Path=/usr/share/fonts/truetype/cmu/,UprightFont=cmunrm.ttf,BoldFont=cmunbx.ttf,ItalicFont=cmunti.ttf,BoldItalicFont=cmunbi.ttf]{cmuntt.ttf}\setmonofont[Path=/usr/share/fonts/truetype/cmu/,UprightFont=cmuntt.ttf,BoldFont=cmuntb.ttf,ItalicFont=cmunit.ttf,BoldItalicFont=cmuntx.ttf]{cmuntt.ttf}\ttfamily organization}}&\hspace*{0pt}\ignorespaces{}\hspace*{0pt}{$\text{ }$}\setmainfont[Path=/usr/share/fonts/truetype/cmu/,UprightFont=cmunrm.ttf,BoldFont=cmunbx.ttf,ItalicFont=cmunti.ttf,BoldItalicFont=cmunbi.ttf]{cmunrm.ttf}\setmonofont[Path=/usr/share/fonts/truetype/cmu/,UprightFont=cmuntt.ttf,BoldFont=cmuntb.ttf,ItalicFont=cmunit.ttf,BoldItalicFont=cmuntx.ttf]{cmunrm.ttf}   &\hspace*{0pt}\ignorespaces{}\hspace*{0pt}   &\hspace*{0pt}\ignorespaces{}\hspace*{0pt}   &\hspace*{0pt}\ignorespaces{}\hspace*{0pt}   &\hspace*{0pt}\ignorespaces{}\hspace*{0pt}   &\hspace*{0pt}\ignorespaces{}\hspace*{0pt}   &\hspace*{0pt}\ignorespaces{}\hspace*{0pt}   &\hspace*{0pt}\ignorespaces{}\hspace*{0pt}   &\hspace*{0pt}\ignorespaces{}\hspace*{0pt} \^{} &\hspace*{0pt}\ignorespaces{}\hspace*{0pt} \^{} &\hspace*{0pt}\ignorespaces{}\hspace*{0pt} \^{} &\hspace*{0pt}\ignorespaces{}\hspace*{0pt}   &\hspace*{0pt}\ignorespaces{}\hspace*{0pt}   &\hspace*{0pt}\ignorespaces{}\hspace*{0pt} \^{} &\hspace*{0pt}\ignorespaces{}\hspace*{0pt} \^{} &\hspace*{0pt}\ignorespaces{}\hspace*{0pt} \^{} &\hspace*{0pt}\ignorespaces{}\hspace*{0pt}   &\hspace*{0pt}\ignorespaces{}\hspace*{0pt}   &\hspace*{0pt}\ignorespaces{}\hspace*{0pt}\\ \hline {\bfseries \hspace*{0pt}\ignorespaces{}\hspace*{0pt} {\ttfamily \setmainfont[Path=/usr/share/fonts/truetype/cmu/,UprightFont=cmunrm.ttf,BoldFont=cmunbx.ttf,ItalicFont=cmunti.ttf,BoldItalicFont=cmunbi.ttf]{cmuntt.ttf}\setmonofont[Path=/usr/share/fonts/truetype/cmu/,UprightFont=cmuntt.ttf,BoldFont=cmuntb.ttf,ItalicFont=cmunit.ttf,BoldItalicFont=cmuntx.ttf]{cmuntt.ttf}\ttfamily institution}}&\hspace*{0pt}\ignorespaces{}\hspace*{0pt}{$\text{ }$}\setmainfont[Path=/usr/share/fonts/truetype/cmu/,UprightFont=cmunrm.ttf,BoldFont=cmunbx.ttf,ItalicFont=cmunti.ttf,BoldItalicFont=cmunbi.ttf]{cmunrm.ttf}\setmonofont[Path=/usr/share/fonts/truetype/cmu/,UprightFont=cmuntt.ttf,BoldFont=cmuntb.ttf,ItalicFont=cmunit.ttf,BoldItalicFont=cmuntx.ttf]{cmunrm.ttf}   &\hspace*{0pt}\ignorespaces{}\hspace*{0pt}   &\hspace*{0pt}\ignorespaces{}\hspace*{0pt}   &\hspace*{0pt}\ignorespaces{}\hspace*{0pt}   &\hspace*{0pt}\ignorespaces{}\hspace*{0pt}   &\hspace*{0pt}\ignorespaces{}\hspace*{0pt}   &\hspace*{0pt}\ignorespaces{}\hspace*{0pt}   &\hspace*{0pt}\ignorespaces{}\hspace*{0pt}   &\hspace*{0pt}\ignorespaces{}\hspace*{0pt}   &\hspace*{0pt}\ignorespaces{}\hspace*{0pt}   &\hspace*{0pt}\ignorespaces{}\hspace*{0pt}   &\hspace*{0pt}\ignorespaces{}\hspace*{0pt}   &\hspace*{0pt}\ignorespaces{}\hspace*{0pt}   &\hspace*{0pt}\ignorespaces{}\hspace*{0pt}   &\hspace*{0pt}\ignorespaces{}\hspace*{0pt}   &\hspace*{0pt}\ignorespaces{}\hspace*{0pt}   &\hspace*{0pt}\ignorespaces{}\hspace*{0pt} + &\hspace*{0pt}\ignorespaces{}\hspace*{0pt} + &\hspace*{0pt}\ignorespaces{}\hspace*{0pt}\\ \hline {\bfseries \hspace*{0pt}\ignorespaces{}\hspace*{0pt} {\ttfamily \setmainfont[Path=/usr/share/fonts/truetype/cmu/,UprightFont=cmunrm.ttf,BoldFont=cmunbx.ttf,ItalicFont=cmunti.ttf,BoldItalicFont=cmunbi.ttf]{cmuntt.ttf}\setmonofont[Path=/usr/share/fonts/truetype/cmu/,UprightFont=cmuntt.ttf,BoldFont=cmuntb.ttf,ItalicFont=cmunit.ttf,BoldItalicFont=cmuntx.ttf]{cmuntt.ttf}\ttfamily type}}&\hspace*{0pt}\ignorespaces{}\hspace*{0pt}{$\text{ }$}\setmainfont[Path=/usr/share/fonts/truetype/cmu/,UprightFont=cmunrm.ttf,BoldFont=cmunbx.ttf,ItalicFont=cmunti.ttf,BoldItalicFont=cmunbi.ttf]{cmunrm.ttf}\setmonofont[Path=/usr/share/fonts/truetype/cmu/,UprightFont=cmuntt.ttf,BoldFont=cmuntb.ttf,ItalicFont=cmunit.ttf,BoldItalicFont=cmuntx.ttf]{cmunrm.ttf}   &\hspace*{0pt}\ignorespaces{}\hspace*{0pt}   &\hspace*{0pt}\ignorespaces{}\hspace*{0pt}   &\hspace*{0pt}\ignorespaces{}\hspace*{0pt}   &\hspace*{0pt}\ignorespaces{}\hspace*{0pt} \^{} &\hspace*{0pt}\ignorespaces{}\hspace*{0pt}   &\hspace*{0pt}\ignorespaces{}\hspace*{0pt}   &\hspace*{0pt}\ignorespaces{}\hspace*{0pt}   &\hspace*{0pt}\ignorespaces{}\hspace*{0pt} \^{} &\hspace*{0pt}\ignorespaces{}\hspace*{0pt} \^{} &\hspace*{0pt}\ignorespaces{}\hspace*{0pt}   &\hspace*{0pt}\ignorespaces{}\hspace*{0pt} \^{} &\hspace*{0pt}\ignorespaces{}\hspace*{0pt}   &\hspace*{0pt}\ignorespaces{}\hspace*{0pt}   &\hspace*{0pt}\ignorespaces{}\hspace*{0pt}   &\hspace*{0pt}\ignorespaces{}\hspace*{0pt}   &\hspace*{0pt}\ignorespaces{}\hspace*{0pt} + &\hspace*{0pt}\ignorespaces{}\hspace*{0pt} + &\hspace*{0pt}\ignorespaces{}\hspace*{0pt}\\ \hline {\bfseries \hspace*{0pt}\ignorespaces{}\hspace*{0pt} {\ttfamily \setmainfont[Path=/usr/share/fonts/truetype/cmu/,UprightFont=cmunrm.ttf,BoldFont=cmunbx.ttf,ItalicFont=cmunti.ttf,BoldItalicFont=cmunbi.ttf]{cmuntt.ttf}\setmonofont[Path=/usr/share/fonts/truetype/cmu/,UprightFont=cmuntt.ttf,BoldFont=cmuntb.ttf,ItalicFont=cmunit.ttf,BoldItalicFont=cmuntx.ttf]{cmuntt.ttf}\ttfamily howpublished}}&\hspace*{0pt}\ignorespaces{}\hspace*{0pt}{$\text{ }$}\setmainfont[Path=/usr/share/fonts/truetype/cmu/,UprightFont=cmunrm.ttf,BoldFont=cmunbx.ttf,ItalicFont=cmunti.ttf,BoldItalicFont=cmunbi.ttf]{cmunrm.ttf}\setmonofont[Path=/usr/share/fonts/truetype/cmu/,UprightFont=cmuntt.ttf,BoldFont=cmuntb.ttf,ItalicFont=cmunit.ttf,BoldItalicFont=cmuntx.ttf]{cmunrm.ttf}   &\hspace*{0pt}\ignorespaces{}\hspace*{0pt}   &\hspace*{0pt}\ignorespaces{}\hspace*{0pt}   &\hspace*{0pt}\ignorespaces{}\hspace*{0pt}   &\hspace*{0pt}\ignorespaces{}\hspace*{0pt} \^{} &\hspace*{0pt}\ignorespaces{}\hspace*{0pt}   &\hspace*{0pt}\ignorespaces{}\hspace*{0pt}   &\hspace*{0pt}\ignorespaces{}\hspace*{0pt}   &\hspace*{0pt}\ignorespaces{}\hspace*{0pt}   &\hspace*{0pt}\ignorespaces{}\hspace*{0pt} \^{} &\hspace*{0pt}\ignorespaces{}\hspace*{0pt}   &\hspace*{0pt}\ignorespaces{}\hspace*{0pt}   &\hspace*{0pt}\ignorespaces{}\hspace*{0pt}   &\hspace*{0pt}\ignorespaces{}\hspace*{0pt}   &\hspace*{0pt}\ignorespaces{}\hspace*{0pt}   &\hspace*{0pt}\ignorespaces{}\hspace*{0pt}   &\hspace*{0pt}\ignorespaces{}\hspace*{0pt}   &\hspace*{0pt}\ignorespaces{}\hspace*{0pt}   &\hspace*{0pt}\ignorespaces{}\hspace*{0pt} \^{} \\ \hline {\bfseries \hspace*{0pt}\ignorespaces{}\hspace*{0pt} {\ttfamily \setmainfont[Path=/usr/share/fonts/truetype/cmu/,UprightFont=cmunrm.ttf,BoldFont=cmunbx.ttf,ItalicFont=cmunti.ttf,BoldItalicFont=cmunbi.ttf]{cmuntt.ttf}\setmonofont[Path=/usr/share/fonts/truetype/cmu/,UprightFont=cmuntt.ttf,BoldFont=cmuntb.ttf,ItalicFont=cmunit.ttf,BoldItalicFont=cmuntx.ttf]{cmuntt.ttf}\ttfamily pages}}&\hspace*{0pt}\ignorespaces{}\hspace*{0pt}{$\text{ }$}\setmainfont[Path=/usr/share/fonts/truetype/cmu/,UprightFont=cmunrm.ttf,BoldFont=cmunbx.ttf,ItalicFont=cmunti.ttf,BoldItalicFont=cmunbi.ttf]{cmunrm.ttf}\setmonofont[Path=/usr/share/fonts/truetype/cmu/,UprightFont=cmuntt.ttf,BoldFont=cmuntb.ttf,ItalicFont=cmunit.ttf,BoldItalicFont=cmuntx.ttf]{cmunrm.ttf} \^{} &\hspace*{0pt}\ignorespaces{}\hspace*{0pt} \^{} &\hspace*{0pt}\ignorespaces{}\hspace*{0pt} – &\hspace*{0pt}\ignorespaces{}\hspace*{0pt} \^{} &\hspace*{0pt}\ignorespaces{}\hspace*{0pt} \^{} &\hspace*{0pt}\ignorespaces{}\hspace*{0pt} \^{} &\hspace*{0pt}\ignorespaces{}\hspace*{0pt} – &\hspace*{0pt}\ignorespaces{}\hspace*{0pt} \^{} &\hspace*{0pt}\ignorespaces{}\hspace*{0pt} \^{} &\hspace*{0pt}\ignorespaces{}\hspace*{0pt}   &\hspace*{0pt}\ignorespaces{}\hspace*{0pt}   &\hspace*{0pt}\ignorespaces{}\hspace*{0pt}   &\hspace*{0pt}\ignorespaces{}\hspace*{0pt}   &\hspace*{0pt}\ignorespaces{}\hspace*{0pt} \^{} &\hspace*{0pt}\ignorespaces{}\hspace*{0pt} – &\hspace*{0pt}\ignorespaces{}\hspace*{0pt} \^{} &\hspace*{0pt}\ignorespaces{}\hspace*{0pt} \^{} &\hspace*{0pt}\ignorespaces{}\hspace*{0pt} \^{} &\hspace*{0pt}\ignorespaces{}\hspace*{0pt}\\ \hline {\bfseries \hspace*{0pt}\ignorespaces{}\hspace*{0pt} {\ttfamily \setmainfont[Path=/usr/share/fonts/truetype/cmu/,UprightFont=cmunrm.ttf,BoldFont=cmunbx.ttf,ItalicFont=cmunti.ttf,BoldItalicFont=cmunbi.ttf]{cmuntt.ttf}\setmonofont[Path=/usr/share/fonts/truetype/cmu/,UprightFont=cmuntt.ttf,BoldFont=cmuntb.ttf,ItalicFont=cmunit.ttf,BoldItalicFont=cmuntx.ttf]{cmuntt.ttf}\ttfamily pagetotal}}&\hspace*{0pt}\ignorespaces{}\hspace*{0pt}{$\text{ }$}\setmainfont[Path=/usr/share/fonts/truetype/cmu/,UprightFont=cmunrm.ttf,BoldFont=cmunbx.ttf,ItalicFont=cmunti.ttf,BoldItalicFont=cmunbi.ttf]{cmunrm.ttf}\setmonofont[Path=/usr/share/fonts/truetype/cmu/,UprightFont=cmuntt.ttf,BoldFont=cmuntb.ttf,ItalicFont=cmunit.ttf,BoldItalicFont=cmuntx.ttf]{cmunrm.ttf}   &\hspace*{0pt}\ignorespaces{}\hspace*{0pt} \^{} &\hspace*{0pt}\ignorespaces{}\hspace*{0pt} \^{} &\hspace*{0pt}\ignorespaces{}\hspace*{0pt} – &\hspace*{0pt}\ignorespaces{}\hspace*{0pt} \^{} &\hspace*{0pt}\ignorespaces{}\hspace*{0pt} \^{} &\hspace*{0pt}\ignorespaces{}\hspace*{0pt} \^{} &\hspace*{0pt}\ignorespaces{}\hspace*{0pt} – &\hspace*{0pt}\ignorespaces{}\hspace*{0pt} \^{} &\hspace*{0pt}\ignorespaces{}\hspace*{0pt}   &\hspace*{0pt}\ignorespaces{}\hspace*{0pt}   &\hspace*{0pt}\ignorespaces{}\hspace*{0pt}   &\hspace*{0pt}\ignorespaces{}\hspace*{0pt}   &\hspace*{0pt}\ignorespaces{}\hspace*{0pt} \^{} &\hspace*{0pt}\ignorespaces{}\hspace*{0pt} \^{} &\hspace*{0pt}\ignorespaces{}\hspace*{0pt} – &\hspace*{0pt}\ignorespaces{}\hspace*{0pt} \^{} &\hspace*{0pt}\ignorespaces{}\hspace*{0pt} \^{} &\hspace*{0pt}\ignorespaces{}\hspace*{0pt}\\ \hline 
\end{longtable}
}}

Some entry types are hard to distinguish and are treated the same by standard styles:
\begin{myitemize}
\item{}  {\ttfamily \setmainfont[Path=/usr/share/fonts/truetype/cmu/,UprightFont=cmunrm.ttf,BoldFont=cmunbx.ttf,ItalicFont=cmunti.ttf,BoldItalicFont=cmunbi.ttf]{cmuntt.ttf}\setmonofont[Path=/usr/share/fonts/truetype/cmu/,UprightFont=cmuntt.ttf,BoldFont=cmuntb.ttf,ItalicFont=cmunit.ttf,BoldItalicFont=cmuntx.ttf]{cmuntt.ttf}\ttfamily @article}{$\text{ }$}\setmainfont[Path=/usr/share/fonts/truetype/cmu/,UprightFont=cmunrm.ttf,BoldFont=cmunbx.ttf,ItalicFont=cmunti.ttf,BoldItalicFont=cmunbi.ttf]{cmunrm.ttf}\setmonofont[Path=/usr/share/fonts/truetype/cmu/,UprightFont=cmuntt.ttf,BoldFont=cmuntb.ttf,ItalicFont=cmunit.ttf,BoldItalicFont=cmuntx.ttf]{cmunrm.ttf} is the same as hypothetic *{\ttfamily \setmainfont[Path=/usr/share/fonts/truetype/cmu/,UprightFont=cmunrm.ttf,BoldFont=cmunbx.ttf,ItalicFont=cmunti.ttf,BoldItalicFont=cmunbi.ttf]{cmuntt.ttf}\setmonofont[Path=/usr/share/fonts/truetype/cmu/,UprightFont=cmuntt.ttf,BoldFont=cmuntb.ttf,ItalicFont=cmunit.ttf,BoldItalicFont=cmuntx.ttf]{cmuntt.ttf}\ttfamily @inperiodical}{$\text{ }$}\setmainfont[Path=/usr/share/fonts/truetype/cmu/,UprightFont=cmunrm.ttf,BoldFont=cmunbx.ttf,ItalicFont=cmunti.ttf,BoldItalicFont=cmunbi.ttf]{cmunrm.ttf}\setmonofont[Path=/usr/share/fonts/truetype/cmu/,UprightFont=cmuntt.ttf,BoldFont=cmuntb.ttf,ItalicFont=cmunit.ttf,BoldItalicFont=cmuntx.ttf]{cmunrm.ttf} and therefore encompasses existing {\ttfamily \setmainfont[Path=/usr/share/fonts/truetype/cmu/,UprightFont=cmunrm.ttf,BoldFont=cmunbx.ttf,ItalicFont=cmunti.ttf,BoldItalicFont=cmunbi.ttf]{cmuntt.ttf}\setmonofont[Path=/usr/share/fonts/truetype/cmu/,UprightFont=cmuntt.ttf,BoldFont=cmuntb.ttf,ItalicFont=cmunit.ttf,BoldItalicFont=cmuntx.ttf]{cmuntt.ttf}\ttfamily @suppperiodical}
\item{} {$\text{ }$}\setmainfont[Path=/usr/share/fonts/truetype/cmu/,UprightFont=cmunrm.ttf,BoldFont=cmunbx.ttf,ItalicFont=cmunti.ttf,BoldItalicFont=cmunbi.ttf]{cmunrm.ttf}\setmonofont[Path=/usr/share/fonts/truetype/cmu/,UprightFont=cmuntt.ttf,BoldFont=cmuntb.ttf,ItalicFont=cmunit.ttf,BoldItalicFont=cmuntx.ttf]{cmunrm.ttf} {\ttfamily \setmainfont[Path=/usr/share/fonts/truetype/cmu/,UprightFont=cmunrm.ttf,BoldFont=cmunbx.ttf,ItalicFont=cmunti.ttf,BoldItalicFont=cmunbi.ttf]{cmuntt.ttf}\setmonofont[Path=/usr/share/fonts/truetype/cmu/,UprightFont=cmuntt.ttf,BoldFont=cmuntb.ttf,ItalicFont=cmunit.ttf,BoldItalicFont=cmuntx.ttf]{cmuntt.ttf}\ttfamily @inbook}{$\text{ }$}\setmainfont[Path=/usr/share/fonts/truetype/cmu/,UprightFont=cmunrm.ttf,BoldFont=cmunbx.ttf,ItalicFont=cmunti.ttf,BoldItalicFont=cmunbi.ttf]{cmunrm.ttf}\setmonofont[Path=/usr/share/fonts/truetype/cmu/,UprightFont=cmuntt.ttf,BoldFont=cmuntb.ttf,ItalicFont=cmunit.ttf,BoldItalicFont=cmuntx.ttf]{cmunrm.ttf} = {\ttfamily \setmainfont[Path=/usr/share/fonts/truetype/cmu/,UprightFont=cmunrm.ttf,BoldFont=cmunbx.ttf,ItalicFont=cmunti.ttf,BoldItalicFont=cmunbi.ttf]{cmuntt.ttf}\setmonofont[Path=/usr/share/fonts/truetype/cmu/,UprightFont=cmuntt.ttf,BoldFont=cmuntb.ttf,ItalicFont=cmunit.ttf,BoldItalicFont=cmuntx.ttf]{cmuntt.ttf}\ttfamily @bookinbook}{$\text{ }$}\setmainfont[Path=/usr/share/fonts/truetype/cmu/,UprightFont=cmunrm.ttf,BoldFont=cmunbx.ttf,ItalicFont=cmunti.ttf,BoldItalicFont=cmunbi.ttf]{cmunrm.ttf}\setmonofont[Path=/usr/share/fonts/truetype/cmu/,UprightFont=cmuntt.ttf,BoldFont=cmuntb.ttf,ItalicFont=cmunit.ttf,BoldItalicFont=cmuntx.ttf]{cmunrm.ttf} = {\ttfamily \setmainfont[Path=/usr/share/fonts/truetype/cmu/,UprightFont=cmunrm.ttf,BoldFont=cmunbx.ttf,ItalicFont=cmunti.ttf,BoldItalicFont=cmunbi.ttf]{cmuntt.ttf}\setmonofont[Path=/usr/share/fonts/truetype/cmu/,UprightFont=cmuntt.ttf,BoldFont=cmuntb.ttf,ItalicFont=cmunit.ttf,BoldItalicFont=cmuntx.ttf]{cmuntt.ttf}\ttfamily @suppbook}
\item{} {$\text{ }$}\setmainfont[Path=/usr/share/fonts/truetype/cmu/,UprightFont=cmunrm.ttf,BoldFont=cmunbx.ttf,ItalicFont=cmunti.ttf,BoldItalicFont=cmunbi.ttf]{cmunrm.ttf}\setmonofont[Path=/usr/share/fonts/truetype/cmu/,UprightFont=cmuntt.ttf,BoldFont=cmuntb.ttf,ItalicFont=cmunit.ttf,BoldItalicFont=cmuntx.ttf]{cmunrm.ttf} {\ttfamily \setmainfont[Path=/usr/share/fonts/truetype/cmu/,UprightFont=cmunrm.ttf,BoldFont=cmunbx.ttf,ItalicFont=cmunti.ttf,BoldItalicFont=cmunbi.ttf]{cmuntt.ttf}\setmonofont[Path=/usr/share/fonts/truetype/cmu/,UprightFont=cmuntt.ttf,BoldFont=cmuntb.ttf,ItalicFont=cmunit.ttf,BoldItalicFont=cmuntx.ttf]{cmuntt.ttf}\ttfamily @collection}{$\text{ }$}\setmainfont[Path=/usr/share/fonts/truetype/cmu/,UprightFont=cmunrm.ttf,BoldFont=cmunbx.ttf,ItalicFont=cmunti.ttf,BoldItalicFont=cmunbi.ttf]{cmunrm.ttf}\setmonofont[Path=/usr/share/fonts/truetype/cmu/,UprightFont=cmuntt.ttf,BoldFont=cmuntb.ttf,ItalicFont=cmunit.ttf,BoldItalicFont=cmuntx.ttf]{cmunrm.ttf} = {\ttfamily \setmainfont[Path=/usr/share/fonts/truetype/cmu/,UprightFont=cmunrm.ttf,BoldFont=cmunbx.ttf,ItalicFont=cmunti.ttf,BoldItalicFont=cmunbi.ttf]{cmuntt.ttf}\setmonofont[Path=/usr/share/fonts/truetype/cmu/,UprightFont=cmuntt.ttf,BoldFont=cmuntb.ttf,ItalicFont=cmunit.ttf,BoldItalicFont=cmuntx.ttf]{cmuntt.ttf}\ttfamily @reference}
\item{} {$\text{ }$}\setmainfont[Path=/usr/share/fonts/truetype/cmu/,UprightFont=cmunrm.ttf,BoldFont=cmunbx.ttf,ItalicFont=cmunti.ttf,BoldItalicFont=cmunbi.ttf]{cmunrm.ttf}\setmonofont[Path=/usr/share/fonts/truetype/cmu/,UprightFont=cmuntt.ttf,BoldFont=cmuntb.ttf,ItalicFont=cmunit.ttf,BoldItalicFont=cmuntx.ttf]{cmunrm.ttf} {\ttfamily \setmainfont[Path=/usr/share/fonts/truetype/cmu/,UprightFont=cmunrm.ttf,BoldFont=cmunbx.ttf,ItalicFont=cmunti.ttf,BoldItalicFont=cmunbi.ttf]{cmuntt.ttf}\setmonofont[Path=/usr/share/fonts/truetype/cmu/,UprightFont=cmuntt.ttf,BoldFont=cmuntb.ttf,ItalicFont=cmunit.ttf,BoldItalicFont=cmuntx.ttf]{cmuntt.ttf}\ttfamily @mvcollection}{$\text{ }$}\setmainfont[Path=/usr/share/fonts/truetype/cmu/,UprightFont=cmunrm.ttf,BoldFont=cmunbx.ttf,ItalicFont=cmunti.ttf,BoldItalicFont=cmunbi.ttf]{cmunrm.ttf}\setmonofont[Path=/usr/share/fonts/truetype/cmu/,UprightFont=cmuntt.ttf,BoldFont=cmuntb.ttf,ItalicFont=cmunit.ttf,BoldItalicFont=cmuntx.ttf]{cmunrm.ttf} = {\ttfamily \setmainfont[Path=/usr/share/fonts/truetype/cmu/,UprightFont=cmunrm.ttf,BoldFont=cmunbx.ttf,ItalicFont=cmunti.ttf,BoldItalicFont=cmunbi.ttf]{cmuntt.ttf}\setmonofont[Path=/usr/share/fonts/truetype/cmu/,UprightFont=cmuntt.ttf,BoldFont=cmuntb.ttf,ItalicFont=cmunit.ttf,BoldItalicFont=cmuntx.ttf]{cmuntt.ttf}\ttfamily @mvreference}
\item{} {$\text{ }$}\setmainfont[Path=/usr/share/fonts/truetype/cmu/,UprightFont=cmunrm.ttf,BoldFont=cmunbx.ttf,ItalicFont=cmunti.ttf,BoldItalicFont=cmunbi.ttf]{cmunrm.ttf}\setmonofont[Path=/usr/share/fonts/truetype/cmu/,UprightFont=cmuntt.ttf,BoldFont=cmuntb.ttf,ItalicFont=cmunit.ttf,BoldItalicFont=cmuntx.ttf]{cmunrm.ttf} {\ttfamily \setmainfont[Path=/usr/share/fonts/truetype/cmu/,UprightFont=cmunrm.ttf,BoldFont=cmunbx.ttf,ItalicFont=cmunti.ttf,BoldItalicFont=cmunbi.ttf]{cmuntt.ttf}\setmonofont[Path=/usr/share/fonts/truetype/cmu/,UprightFont=cmuntt.ttf,BoldFont=cmuntb.ttf,ItalicFont=cmunit.ttf,BoldItalicFont=cmuntx.ttf]{cmuntt.ttf}\ttfamily @incollection}{$\text{ }$}\setmainfont[Path=/usr/share/fonts/truetype/cmu/,UprightFont=cmunrm.ttf,BoldFont=cmunbx.ttf,ItalicFont=cmunti.ttf,BoldItalicFont=cmunbi.ttf]{cmunrm.ttf}\setmonofont[Path=/usr/share/fonts/truetype/cmu/,UprightFont=cmuntt.ttf,BoldFont=cmuntb.ttf,ItalicFont=cmunit.ttf,BoldItalicFont=cmuntx.ttf]{cmunrm.ttf} = {\ttfamily \setmainfont[Path=/usr/share/fonts/truetype/cmu/,UprightFont=cmunrm.ttf,BoldFont=cmunbx.ttf,ItalicFont=cmunti.ttf,BoldItalicFont=cmunbi.ttf]{cmuntt.ttf}\setmonofont[Path=/usr/share/fonts/truetype/cmu/,UprightFont=cmuntt.ttf,BoldFont=cmuntb.ttf,ItalicFont=cmunit.ttf,BoldItalicFont=cmuntx.ttf]{cmuntt.ttf}\ttfamily @suppcollection}{$\text{ }$}\setmainfont[Path=/usr/share/fonts/truetype/cmu/,UprightFont=cmunrm.ttf,BoldFont=cmunbx.ttf,ItalicFont=cmunti.ttf,BoldItalicFont=cmunbi.ttf]{cmunrm.ttf}\setmonofont[Path=/usr/share/fonts/truetype/cmu/,UprightFont=cmuntt.ttf,BoldFont=cmuntb.ttf,ItalicFont=cmunit.ttf,BoldItalicFont=cmuntx.ttf]{cmunrm.ttf} = {\ttfamily \setmainfont[Path=/usr/share/fonts/truetype/cmu/,UprightFont=cmunrm.ttf,BoldFont=cmunbx.ttf,ItalicFont=cmunti.ttf,BoldItalicFont=cmunbi.ttf]{cmuntt.ttf}\setmonofont[Path=/usr/share/fonts/truetype/cmu/,UprightFont=cmuntt.ttf,BoldFont=cmuntb.ttf,ItalicFont=cmunit.ttf,BoldItalicFont=cmuntx.ttf]{cmuntt.ttf}\ttfamily @inreference}
\item{} {$\text{ }$}\setmainfont[Path=/usr/share/fonts/truetype/cmu/,UprightFont=cmunrm.ttf,BoldFont=cmunbx.ttf,ItalicFont=cmunti.ttf,BoldItalicFont=cmunbi.ttf]{cmunrm.ttf}\setmonofont[Path=/usr/share/fonts/truetype/cmu/,UprightFont=cmuntt.ttf,BoldFont=cmuntb.ttf,ItalicFont=cmunit.ttf,BoldItalicFont=cmuntx.ttf]{cmunrm.ttf} {\ttfamily \setmainfont[Path=/usr/share/fonts/truetype/cmu/,UprightFont=cmunrm.ttf,BoldFont=cmunbx.ttf,ItalicFont=cmunti.ttf,BoldItalicFont=cmunbi.ttf]{cmuntt.ttf}\setmonofont[Path=/usr/share/fonts/truetype/cmu/,UprightFont=cmuntt.ttf,BoldFont=cmuntb.ttf,ItalicFont=cmunit.ttf,BoldItalicFont=cmuntx.ttf]{cmuntt.ttf}\ttfamily @online}{$\text{ }$}\setmainfont[Path=/usr/share/fonts/truetype/cmu/,UprightFont=cmunrm.ttf,BoldFont=cmunbx.ttf,ItalicFont=cmunti.ttf,BoldItalicFont=cmunbi.ttf]{cmunrm.ttf}\setmonofont[Path=/usr/share/fonts/truetype/cmu/,UprightFont=cmuntt.ttf,BoldFont=cmuntb.ttf,ItalicFont=cmunit.ttf,BoldItalicFont=cmuntx.ttf]{cmunrm.ttf} = {\ttfamily \setmainfont[Path=/usr/share/fonts/truetype/cmu/,UprightFont=cmunrm.ttf,BoldFont=cmunbx.ttf,ItalicFont=cmunti.ttf,BoldItalicFont=cmunbi.ttf]{cmuntt.ttf}\setmonofont[Path=/usr/share/fonts/truetype/cmu/,UprightFont=cmuntt.ttf,BoldFont=cmuntb.ttf,ItalicFont=cmunit.ttf,BoldItalicFont=cmuntx.ttf]{cmuntt.ttf}\ttfamily @electronic}{$\text{ }$}\setmainfont[Path=/usr/share/fonts/truetype/cmu/,UprightFont=cmunrm.ttf,BoldFont=cmunbx.ttf,ItalicFont=cmunti.ttf,BoldItalicFont=cmunbi.ttf]{cmunrm.ttf}\setmonofont[Path=/usr/share/fonts/truetype/cmu/,UprightFont=cmuntt.ttf,BoldFont=cmuntb.ttf,ItalicFont=cmunit.ttf,BoldItalicFont=cmuntx.ttf]{cmunrm.ttf} = {\ttfamily \setmainfont[Path=/usr/share/fonts/truetype/cmu/,UprightFont=cmunrm.ttf,BoldFont=cmunbx.ttf,ItalicFont=cmunti.ttf,BoldItalicFont=cmunbi.ttf]{cmuntt.ttf}\setmonofont[Path=/usr/share/fonts/truetype/cmu/,UprightFont=cmuntt.ttf,BoldFont=cmuntb.ttf,ItalicFont=cmunit.ttf,BoldItalicFont=cmuntx.ttf]{cmuntt.ttf}\ttfamily @www}
\item{} {$\text{ }$}\setmainfont[Path=/usr/share/fonts/truetype/cmu/,UprightFont=cmunrm.ttf,BoldFont=cmunbx.ttf,ItalicFont=cmunti.ttf,BoldItalicFont=cmunbi.ttf]{cmunrm.ttf}\setmonofont[Path=/usr/share/fonts/truetype/cmu/,UprightFont=cmuntt.ttf,BoldFont=cmuntb.ttf,ItalicFont=cmunit.ttf,BoldItalicFont=cmuntx.ttf]{cmunrm.ttf} {\ttfamily \setmainfont[Path=/usr/share/fonts/truetype/cmu/,UprightFont=cmunrm.ttf,BoldFont=cmunbx.ttf,ItalicFont=cmunti.ttf,BoldItalicFont=cmunbi.ttf]{cmuntt.ttf}\setmonofont[Path=/usr/share/fonts/truetype/cmu/,UprightFont=cmuntt.ttf,BoldFont=cmuntb.ttf,ItalicFont=cmunit.ttf,BoldItalicFont=cmuntx.ttf]{cmuntt.ttf}\ttfamily @report}{$\text{ }$}\setmainfont[Path=/usr/share/fonts/truetype/cmu/,UprightFont=cmunrm.ttf,BoldFont=cmunbx.ttf,ItalicFont=cmunti.ttf,BoldItalicFont=cmunbi.ttf]{cmunrm.ttf}\setmonofont[Path=/usr/share/fonts/truetype/cmu/,UprightFont=cmuntt.ttf,BoldFont=cmuntb.ttf,ItalicFont=cmunit.ttf,BoldItalicFont=cmuntx.ttf]{cmunrm.ttf} = {\ttfamily \setmainfont[Path=/usr/share/fonts/truetype/cmu/,UprightFont=cmunrm.ttf,BoldFont=cmunbx.ttf,ItalicFont=cmunti.ttf,BoldItalicFont=cmunbi.ttf]{cmuntt.ttf}\setmonofont[Path=/usr/share/fonts/truetype/cmu/,UprightFont=cmuntt.ttf,BoldFont=cmuntb.ttf,ItalicFont=cmunit.ttf,BoldItalicFont=cmuntx.ttf]{cmuntt.ttf}\ttfamily @techreport}
\item{} {$\text{ }$}\setmainfont[Path=/usr/share/fonts/truetype/cmu/,UprightFont=cmunrm.ttf,BoldFont=cmunbx.ttf,ItalicFont=cmunti.ttf,BoldItalicFont=cmunbi.ttf]{cmunrm.ttf}\setmonofont[Path=/usr/share/fonts/truetype/cmu/,UprightFont=cmuntt.ttf,BoldFont=cmuntb.ttf,ItalicFont=cmunit.ttf,BoldItalicFont=cmuntx.ttf]{cmunrm.ttf} {\ttfamily \setmainfont[Path=/usr/share/fonts/truetype/cmu/,UprightFont=cmunrm.ttf,BoldFont=cmunbx.ttf,ItalicFont=cmunti.ttf,BoldItalicFont=cmunbi.ttf]{cmuntt.ttf}\setmonofont[Path=/usr/share/fonts/truetype/cmu/,UprightFont=cmuntt.ttf,BoldFont=cmuntb.ttf,ItalicFont=cmunit.ttf,BoldItalicFont=cmuntx.ttf]{cmuntt.ttf}\ttfamily @thesis}{$\text{ }$}\setmainfont[Path=/usr/share/fonts/truetype/cmu/,UprightFont=cmunrm.ttf,BoldFont=cmunbx.ttf,ItalicFont=cmunti.ttf,BoldItalicFont=cmunbi.ttf]{cmunrm.ttf}\setmonofont[Path=/usr/share/fonts/truetype/cmu/,UprightFont=cmuntt.ttf,BoldFont=cmuntb.ttf,ItalicFont=cmunit.ttf,BoldItalicFont=cmuntx.ttf]{cmunrm.ttf} = {\ttfamily \setmainfont[Path=/usr/share/fonts/truetype/cmu/,UprightFont=cmunrm.ttf,BoldFont=cmunbx.ttf,ItalicFont=cmunti.ttf,BoldItalicFont=cmunbi.ttf]{cmuntt.ttf}\setmonofont[Path=/usr/share/fonts/truetype/cmu/,UprightFont=cmuntt.ttf,BoldFont=cmuntb.ttf,ItalicFont=cmunit.ttf,BoldItalicFont=cmuntx.ttf]{cmuntt.ttf}\ttfamily @mastersthesis}{$\text{ }$}\setmainfont[Path=/usr/share/fonts/truetype/cmu/,UprightFont=cmunrm.ttf,BoldFont=cmunbx.ttf,ItalicFont=cmunti.ttf,BoldItalicFont=cmunbi.ttf]{cmunrm.ttf}\setmonofont[Path=/usr/share/fonts/truetype/cmu/,UprightFont=cmuntt.ttf,BoldFont=cmuntb.ttf,ItalicFont=cmunit.ttf,BoldItalicFont=cmuntx.ttf]{cmunrm.ttf} = {\ttfamily \setmainfont[Path=/usr/share/fonts/truetype/cmu/,UprightFont=cmunrm.ttf,BoldFont=cmunbx.ttf,ItalicFont=cmunti.ttf,BoldItalicFont=cmunbi.ttf]{cmuntt.ttf}\setmonofont[Path=/usr/share/fonts/truetype/cmu/,UprightFont=cmuntt.ttf,BoldFont=cmuntb.ttf,ItalicFont=cmunit.ttf,BoldItalicFont=cmuntx.ttf]{cmuntt.ttf}\ttfamily @phdthesis}
\end{myitemize}
\setmainfont[Path=/usr/share/fonts/truetype/cmu/,UprightFont=cmunrm.ttf,BoldFont=cmunbx.ttf,ItalicFont=cmunti.ttf,BoldItalicFont=cmunbi.ttf]{cmunrm.ttf}\setmonofont[Path=/usr/share/fonts/truetype/cmu/,UprightFont=cmuntt.ttf,BoldFont=cmuntb.ttf,ItalicFont=cmunit.ttf,BoldItalicFont=cmuntx.ttf]{cmunrm.ttf}

Some field types are defined, but the documentation does not say which entry types they can be used with. This is either because they depend on another field being set to be useful or they can always be used in a user-{}defined manner, but will never be used in standard styles:
\begin{myitemize}
\item{}  {\ttfamily \setmainfont[Path=/usr/share/fonts/truetype/cmu/,UprightFont=cmunrm.ttf,BoldFont=cmunbx.ttf,ItalicFont=cmunti.ttf,BoldItalicFont=cmunbi.ttf]{cmuntt.ttf}\setmonofont[Path=/usr/share/fonts/truetype/cmu/,UprightFont=cmuntt.ttf,BoldFont=cmuntb.ttf,ItalicFont=cmunit.ttf,BoldItalicFont=cmuntx.ttf]{cmuntt.ttf}\ttfamily abstract}\setmainfont[Path=/usr/share/fonts/truetype/cmu/,UprightFont=cmunrm.ttf,BoldFont=cmunbx.ttf,ItalicFont=cmunti.ttf,BoldItalicFont=cmunbi.ttf]{cmunrm.ttf}\setmonofont[Path=/usr/share/fonts/truetype/cmu/,UprightFont=cmuntt.ttf,BoldFont=cmuntb.ttf,ItalicFont=cmunit.ttf,BoldItalicFont=cmuntx.ttf]{cmunrm.ttf},  {\ttfamily \setmainfont[Path=/usr/share/fonts/truetype/cmu/,UprightFont=cmunrm.ttf,BoldFont=cmunbx.ttf,ItalicFont=cmunti.ttf,BoldItalicFont=cmunbi.ttf]{cmuntt.ttf}\setmonofont[Path=/usr/share/fonts/truetype/cmu/,UprightFont=cmuntt.ttf,BoldFont=cmuntb.ttf,ItalicFont=cmunit.ttf,BoldItalicFont=cmuntx.ttf]{cmuntt.ttf}\ttfamily annotation}
\item{} {$\text{ }$}\setmainfont[Path=/usr/share/fonts/truetype/cmu/,UprightFont=cmunrm.ttf,BoldFont=cmunbx.ttf,ItalicFont=cmunti.ttf,BoldItalicFont=cmunbi.ttf]{cmunrm.ttf}\setmonofont[Path=/usr/share/fonts/truetype/cmu/,UprightFont=cmuntt.ttf,BoldFont=cmuntb.ttf,ItalicFont=cmunit.ttf,BoldItalicFont=cmuntx.ttf]{cmunrm.ttf} {\ttfamily \setmainfont[Path=/usr/share/fonts/truetype/cmu/,UprightFont=cmunrm.ttf,BoldFont=cmunbx.ttf,ItalicFont=cmunti.ttf,BoldItalicFont=cmunbi.ttf]{cmuntt.ttf}\setmonofont[Path=/usr/share/fonts/truetype/cmu/,UprightFont=cmuntt.ttf,BoldFont=cmuntb.ttf,ItalicFont=cmunit.ttf,BoldItalicFont=cmuntx.ttf]{cmuntt.ttf}\ttfamily entrysubtype}
\item{} {$\text{ }$}\setmainfont[Path=/usr/share/fonts/truetype/cmu/,UprightFont=cmunrm.ttf,BoldFont=cmunbx.ttf,ItalicFont=cmunti.ttf,BoldItalicFont=cmunbi.ttf]{cmunrm.ttf}\setmonofont[Path=/usr/share/fonts/truetype/cmu/,UprightFont=cmuntt.ttf,BoldFont=cmuntb.ttf,ItalicFont=cmunit.ttf,BoldItalicFont=cmuntx.ttf]{cmunrm.ttf} {\ttfamily \setmainfont[Path=/usr/share/fonts/truetype/cmu/,UprightFont=cmunrm.ttf,BoldFont=cmunbx.ttf,ItalicFont=cmunti.ttf,BoldItalicFont=cmunbi.ttf]{cmuntt.ttf}\setmonofont[Path=/usr/share/fonts/truetype/cmu/,UprightFont=cmuntt.ttf,BoldFont=cmuntb.ttf,ItalicFont=cmunit.ttf,BoldItalicFont=cmuntx.ttf]{cmuntt.ttf}\ttfamily file}
\item{} {$\text{ }$}\setmainfont[Path=/usr/share/fonts/truetype/cmu/,UprightFont=cmunrm.ttf,BoldFont=cmunbx.ttf,ItalicFont=cmunti.ttf,BoldItalicFont=cmunbi.ttf]{cmunrm.ttf}\setmonofont[Path=/usr/share/fonts/truetype/cmu/,UprightFont=cmuntt.ttf,BoldFont=cmuntb.ttf,ItalicFont=cmunit.ttf,BoldItalicFont=cmuntx.ttf]{cmunrm.ttf} {\ttfamily \setmainfont[Path=/usr/share/fonts/truetype/cmu/,UprightFont=cmunrm.ttf,BoldFont=cmunbx.ttf,ItalicFont=cmunti.ttf,BoldItalicFont=cmunbi.ttf]{cmuntt.ttf}\setmonofont[Path=/usr/share/fonts/truetype/cmu/,UprightFont=cmuntt.ttf,BoldFont=cmuntb.ttf,ItalicFont=cmunit.ttf,BoldItalicFont=cmuntx.ttf]{cmuntt.ttf}\ttfamily label}
\item{} {$\text{ }$}\setmainfont[Path=/usr/share/fonts/truetype/cmu/,UprightFont=cmunrm.ttf,BoldFont=cmunbx.ttf,ItalicFont=cmunti.ttf,BoldItalicFont=cmunbi.ttf]{cmunrm.ttf}\setmonofont[Path=/usr/share/fonts/truetype/cmu/,UprightFont=cmuntt.ttf,BoldFont=cmuntb.ttf,ItalicFont=cmunit.ttf,BoldItalicFont=cmuntx.ttf]{cmunrm.ttf} {\ttfamily \setmainfont[Path=/usr/share/fonts/truetype/cmu/,UprightFont=cmunrm.ttf,BoldFont=cmunbx.ttf,ItalicFont=cmunti.ttf,BoldItalicFont=cmunbi.ttf]{cmuntt.ttf}\setmonofont[Path=/usr/share/fonts/truetype/cmu/,UprightFont=cmuntt.ttf,BoldFont=cmuntb.ttf,ItalicFont=cmunit.ttf,BoldItalicFont=cmuntx.ttf]{cmuntt.ttf}\ttfamily library}
\item{} {$\text{ }$}\setmainfont[Path=/usr/share/fonts/truetype/cmu/,UprightFont=cmunrm.ttf,BoldFont=cmunbx.ttf,ItalicFont=cmunti.ttf,BoldItalicFont=cmunbi.ttf]{cmunrm.ttf}\setmonofont[Path=/usr/share/fonts/truetype/cmu/,UprightFont=cmuntt.ttf,BoldFont=cmuntb.ttf,ItalicFont=cmunit.ttf,BoldItalicFont=cmuntx.ttf]{cmunrm.ttf} {\ttfamily \setmainfont[Path=/usr/share/fonts/truetype/cmu/,UprightFont=cmunrm.ttf,BoldFont=cmunbx.ttf,ItalicFont=cmunti.ttf,BoldItalicFont=cmunbi.ttf]{cmuntt.ttf}\setmonofont[Path=/usr/share/fonts/truetype/cmu/,UprightFont=cmuntt.ttf,BoldFont=cmuntb.ttf,ItalicFont=cmunit.ttf,BoldItalicFont=cmuntx.ttf]{cmuntt.ttf}\ttfamily nameaddon}
\item{} {$\text{ }$}\setmainfont[Path=/usr/share/fonts/truetype/cmu/,UprightFont=cmunrm.ttf,BoldFont=cmunbx.ttf,ItalicFont=cmunti.ttf,BoldItalicFont=cmunbi.ttf]{cmunrm.ttf}\setmonofont[Path=/usr/share/fonts/truetype/cmu/,UprightFont=cmuntt.ttf,BoldFont=cmuntb.ttf,ItalicFont=cmunit.ttf,BoldItalicFont=cmuntx.ttf]{cmunrm.ttf} {\ttfamily \setmainfont[Path=/usr/share/fonts/truetype/cmu/,UprightFont=cmunrm.ttf,BoldFont=cmunbx.ttf,ItalicFont=cmunti.ttf,BoldItalicFont=cmunbi.ttf]{cmuntt.ttf}\setmonofont[Path=/usr/share/fonts/truetype/cmu/,UprightFont=cmuntt.ttf,BoldFont=cmuntb.ttf,ItalicFont=cmunit.ttf,BoldItalicFont=cmuntx.ttf]{cmuntt.ttf}\ttfamily origdate}\setmainfont[Path=/usr/share/fonts/truetype/cmu/,UprightFont=cmunrm.ttf,BoldFont=cmunbx.ttf,ItalicFont=cmunti.ttf,BoldItalicFont=cmunbi.ttf]{cmunrm.ttf}\setmonofont[Path=/usr/share/fonts/truetype/cmu/,UprightFont=cmuntt.ttf,BoldFont=cmuntb.ttf,ItalicFont=cmunit.ttf,BoldItalicFont=cmuntx.ttf]{cmunrm.ttf}, {\ttfamily \setmainfont[Path=/usr/share/fonts/truetype/cmu/,UprightFont=cmunrm.ttf,BoldFont=cmunbx.ttf,ItalicFont=cmunti.ttf,BoldItalicFont=cmunbi.ttf]{cmuntt.ttf}\setmonofont[Path=/usr/share/fonts/truetype/cmu/,UprightFont=cmuntt.ttf,BoldFont=cmuntb.ttf,ItalicFont=cmunit.ttf,BoldItalicFont=cmuntx.ttf]{cmuntt.ttf}\ttfamily origlocation}\setmainfont[Path=/usr/share/fonts/truetype/cmu/,UprightFont=cmunrm.ttf,BoldFont=cmunbx.ttf,ItalicFont=cmunti.ttf,BoldItalicFont=cmunbi.ttf]{cmunrm.ttf}\setmonofont[Path=/usr/share/fonts/truetype/cmu/,UprightFont=cmuntt.ttf,BoldFont=cmuntb.ttf,ItalicFont=cmunit.ttf,BoldItalicFont=cmuntx.ttf]{cmunrm.ttf}, {\ttfamily \setmainfont[Path=/usr/share/fonts/truetype/cmu/,UprightFont=cmunrm.ttf,BoldFont=cmunbx.ttf,ItalicFont=cmunti.ttf,BoldItalicFont=cmunbi.ttf]{cmuntt.ttf}\setmonofont[Path=/usr/share/fonts/truetype/cmu/,UprightFont=cmuntt.ttf,BoldFont=cmuntb.ttf,ItalicFont=cmunit.ttf,BoldItalicFont=cmuntx.ttf]{cmuntt.ttf}\ttfamily origpublisher}
\item{} {$\text{ }$}\setmainfont[Path=/usr/share/fonts/truetype/cmu/,UprightFont=cmunrm.ttf,BoldFont=cmunbx.ttf,ItalicFont=cmunti.ttf,BoldItalicFont=cmunbi.ttf]{cmunrm.ttf}\setmonofont[Path=/usr/share/fonts/truetype/cmu/,UprightFont=cmuntt.ttf,BoldFont=cmuntb.ttf,ItalicFont=cmunit.ttf,BoldItalicFont=cmuntx.ttf]{cmunrm.ttf} {\ttfamily \setmainfont[Path=/usr/share/fonts/truetype/cmu/,UprightFont=cmunrm.ttf,BoldFont=cmunbx.ttf,ItalicFont=cmunti.ttf,BoldItalicFont=cmunbi.ttf]{cmuntt.ttf}\setmonofont[Path=/usr/share/fonts/truetype/cmu/,UprightFont=cmuntt.ttf,BoldFont=cmuntb.ttf,ItalicFont=cmunit.ttf,BoldItalicFont=cmuntx.ttf]{cmuntt.ttf}\ttfamily origtitle}\setmainfont[Path=/usr/share/fonts/truetype/cmu/,UprightFont=cmunrm.ttf,BoldFont=cmunbx.ttf,ItalicFont=cmunti.ttf,BoldItalicFont=cmunbi.ttf]{cmunrm.ttf}\setmonofont[Path=/usr/share/fonts/truetype/cmu/,UprightFont=cmuntt.ttf,BoldFont=cmuntb.ttf,ItalicFont=cmunit.ttf,BoldItalicFont=cmuntx.ttf]{cmunrm.ttf}, {\ttfamily \setmainfont[Path=/usr/share/fonts/truetype/cmu/,UprightFont=cmunrm.ttf,BoldFont=cmunbx.ttf,ItalicFont=cmunti.ttf,BoldItalicFont=cmunbi.ttf]{cmuntt.ttf}\setmonofont[Path=/usr/share/fonts/truetype/cmu/,UprightFont=cmuntt.ttf,BoldFont=cmuntb.ttf,ItalicFont=cmunit.ttf,BoldItalicFont=cmuntx.ttf]{cmuntt.ttf}\ttfamily reprinttitle}\setmainfont[Path=/usr/share/fonts/truetype/cmu/,UprightFont=cmunrm.ttf,BoldFont=cmunbx.ttf,ItalicFont=cmunti.ttf,BoldItalicFont=cmunbi.ttf]{cmunrm.ttf}\setmonofont[Path=/usr/share/fonts/truetype/cmu/,UprightFont=cmuntt.ttf,BoldFont=cmuntb.ttf,ItalicFont=cmunit.ttf,BoldItalicFont=cmuntx.ttf]{cmunrm.ttf}, {\ttfamily \setmainfont[Path=/usr/share/fonts/truetype/cmu/,UprightFont=cmunrm.ttf,BoldFont=cmunbx.ttf,ItalicFont=cmunti.ttf,BoldItalicFont=cmunbi.ttf]{cmuntt.ttf}\setmonofont[Path=/usr/share/fonts/truetype/cmu/,UprightFont=cmuntt.ttf,BoldFont=cmuntb.ttf,ItalicFont=cmunit.ttf,BoldItalicFont=cmuntx.ttf]{cmuntt.ttf}\ttfamily indextitle}
\item{} {$\text{ }$}\setmainfont[Path=/usr/share/fonts/truetype/cmu/,UprightFont=cmunrm.ttf,BoldFont=cmunbx.ttf,ItalicFont=cmunti.ttf,BoldItalicFont=cmunbi.ttf]{cmunrm.ttf}\setmonofont[Path=/usr/share/fonts/truetype/cmu/,UprightFont=cmuntt.ttf,BoldFont=cmuntb.ttf,ItalicFont=cmunit.ttf,BoldItalicFont=cmuntx.ttf]{cmunrm.ttf} {\ttfamily \setmainfont[Path=/usr/share/fonts/truetype/cmu/,UprightFont=cmunrm.ttf,BoldFont=cmunbx.ttf,ItalicFont=cmunti.ttf,BoldItalicFont=cmunbi.ttf]{cmuntt.ttf}\setmonofont[Path=/usr/share/fonts/truetype/cmu/,UprightFont=cmuntt.ttf,BoldFont=cmuntb.ttf,ItalicFont=cmunit.ttf,BoldItalicFont=cmuntx.ttf]{cmuntt.ttf}\ttfamily pagination}\setmainfont[Path=/usr/share/fonts/truetype/cmu/,UprightFont=cmunrm.ttf,BoldFont=cmunbx.ttf,ItalicFont=cmunti.ttf,BoldItalicFont=cmunbi.ttf]{cmunrm.ttf}\setmonofont[Path=/usr/share/fonts/truetype/cmu/,UprightFont=cmuntt.ttf,BoldFont=cmuntb.ttf,ItalicFont=cmunit.ttf,BoldItalicFont=cmuntx.ttf]{cmunrm.ttf}, {\ttfamily \setmainfont[Path=/usr/share/fonts/truetype/cmu/,UprightFont=cmunrm.ttf,BoldFont=cmunbx.ttf,ItalicFont=cmunti.ttf,BoldItalicFont=cmunbi.ttf]{cmuntt.ttf}\setmonofont[Path=/usr/share/fonts/truetype/cmu/,UprightFont=cmuntt.ttf,BoldFont=cmuntb.ttf,ItalicFont=cmunit.ttf,BoldItalicFont=cmuntx.ttf]{cmuntt.ttf}\ttfamily bookpagination}
\item{} {$\text{ }$}\setmainfont[Path=/usr/share/fonts/truetype/cmu/,UprightFont=cmunrm.ttf,BoldFont=cmunbx.ttf,ItalicFont=cmunti.ttf,BoldItalicFont=cmunbi.ttf]{cmunrm.ttf}\setmonofont[Path=/usr/share/fonts/truetype/cmu/,UprightFont=cmuntt.ttf,BoldFont=cmuntb.ttf,ItalicFont=cmunit.ttf,BoldItalicFont=cmuntx.ttf]{cmunrm.ttf} {\ttfamily \setmainfont[Path=/usr/share/fonts/truetype/cmu/,UprightFont=cmunrm.ttf,BoldFont=cmunbx.ttf,ItalicFont=cmunti.ttf,BoldItalicFont=cmunbi.ttf]{cmuntt.ttf}\setmonofont[Path=/usr/share/fonts/truetype/cmu/,UprightFont=cmuntt.ttf,BoldFont=cmuntb.ttf,ItalicFont=cmunit.ttf,BoldItalicFont=cmuntx.ttf]{cmuntt.ttf}\ttfamily shortauthor}\setmainfont[Path=/usr/share/fonts/truetype/cmu/,UprightFont=cmunrm.ttf,BoldFont=cmunbx.ttf,ItalicFont=cmunti.ttf,BoldItalicFont=cmunbi.ttf]{cmunrm.ttf}\setmonofont[Path=/usr/share/fonts/truetype/cmu/,UprightFont=cmuntt.ttf,BoldFont=cmuntb.ttf,ItalicFont=cmunit.ttf,BoldItalicFont=cmuntx.ttf]{cmunrm.ttf}, {\ttfamily \setmainfont[Path=/usr/share/fonts/truetype/cmu/,UprightFont=cmunrm.ttf,BoldFont=cmunbx.ttf,ItalicFont=cmunti.ttf,BoldItalicFont=cmunbi.ttf]{cmuntt.ttf}\setmonofont[Path=/usr/share/fonts/truetype/cmu/,UprightFont=cmuntt.ttf,BoldFont=cmuntb.ttf,ItalicFont=cmunit.ttf,BoldItalicFont=cmuntx.ttf]{cmuntt.ttf}\ttfamily shorteditor}\setmainfont[Path=/usr/share/fonts/truetype/cmu/,UprightFont=cmunrm.ttf,BoldFont=cmunbx.ttf,ItalicFont=cmunti.ttf,BoldItalicFont=cmunbi.ttf]{cmunrm.ttf}\setmonofont[Path=/usr/share/fonts/truetype/cmu/,UprightFont=cmuntt.ttf,BoldFont=cmuntb.ttf,ItalicFont=cmunit.ttf,BoldItalicFont=cmuntx.ttf]{cmunrm.ttf}, {\ttfamily \setmainfont[Path=/usr/share/fonts/truetype/cmu/,UprightFont=cmunrm.ttf,BoldFont=cmunbx.ttf,ItalicFont=cmunti.ttf,BoldItalicFont=cmunbi.ttf]{cmuntt.ttf}\setmonofont[Path=/usr/share/fonts/truetype/cmu/,UprightFont=cmuntt.ttf,BoldFont=cmuntb.ttf,ItalicFont=cmunit.ttf,BoldItalicFont=cmuntx.ttf]{cmuntt.ttf}\ttfamily shorthand}\setmainfont[Path=/usr/share/fonts/truetype/cmu/,UprightFont=cmunrm.ttf,BoldFont=cmunbx.ttf,ItalicFont=cmunti.ttf,BoldItalicFont=cmunbi.ttf]{cmunrm.ttf}\setmonofont[Path=/usr/share/fonts/truetype/cmu/,UprightFont=cmuntt.ttf,BoldFont=cmuntb.ttf,ItalicFont=cmunit.ttf,BoldItalicFont=cmuntx.ttf]{cmunrm.ttf}, {\ttfamily \setmainfont[Path=/usr/share/fonts/truetype/cmu/,UprightFont=cmunrm.ttf,BoldFont=cmunbx.ttf,ItalicFont=cmunti.ttf,BoldItalicFont=cmunbi.ttf]{cmuntt.ttf}\setmonofont[Path=/usr/share/fonts/truetype/cmu/,UprightFont=cmuntt.ttf,BoldFont=cmuntb.ttf,ItalicFont=cmunit.ttf,BoldItalicFont=cmuntx.ttf]{cmuntt.ttf}\ttfamily shorthandintro}\setmainfont[Path=/usr/share/fonts/truetype/cmu/,UprightFont=cmunrm.ttf,BoldFont=cmunbx.ttf,ItalicFont=cmunti.ttf,BoldItalicFont=cmunbi.ttf]{cmunrm.ttf}\setmonofont[Path=/usr/share/fonts/truetype/cmu/,UprightFont=cmuntt.ttf,BoldFont=cmuntb.ttf,ItalicFont=cmunit.ttf,BoldItalicFont=cmuntx.ttf]{cmunrm.ttf}, {\ttfamily \setmainfont[Path=/usr/share/fonts/truetype/cmu/,UprightFont=cmunrm.ttf,BoldFont=cmunbx.ttf,ItalicFont=cmunti.ttf,BoldItalicFont=cmunbi.ttf]{cmuntt.ttf}\setmonofont[Path=/usr/share/fonts/truetype/cmu/,UprightFont=cmuntt.ttf,BoldFont=cmuntb.ttf,ItalicFont=cmunit.ttf,BoldItalicFont=cmuntx.ttf]{cmuntt.ttf}\ttfamily shortjournal}\setmainfont[Path=/usr/share/fonts/truetype/cmu/,UprightFont=cmunrm.ttf,BoldFont=cmunbx.ttf,ItalicFont=cmunti.ttf,BoldItalicFont=cmunbi.ttf]{cmunrm.ttf}\setmonofont[Path=/usr/share/fonts/truetype/cmu/,UprightFont=cmuntt.ttf,BoldFont=cmuntb.ttf,ItalicFont=cmunit.ttf,BoldItalicFont=cmuntx.ttf]{cmunrm.ttf}, {\ttfamily \setmainfont[Path=/usr/share/fonts/truetype/cmu/,UprightFont=cmunrm.ttf,BoldFont=cmunbx.ttf,ItalicFont=cmunti.ttf,BoldItalicFont=cmunbi.ttf]{cmuntt.ttf}\setmonofont[Path=/usr/share/fonts/truetype/cmu/,UprightFont=cmuntt.ttf,BoldFont=cmuntb.ttf,ItalicFont=cmunit.ttf,BoldItalicFont=cmuntx.ttf]{cmuntt.ttf}\ttfamily shortseries}{$\text{ }$}\setmainfont[Path=/usr/share/fonts/truetype/cmu/,UprightFont=cmunrm.ttf,BoldFont=cmunbx.ttf,ItalicFont=cmunti.ttf,BoldItalicFont=cmunbi.ttf]{cmunrm.ttf}\setmonofont[Path=/usr/share/fonts/truetype/cmu/,UprightFont=cmuntt.ttf,BoldFont=cmuntb.ttf,ItalicFont=cmunit.ttf,BoldItalicFont=cmuntx.ttf]{cmunrm.ttf} {\ttfamily \setmainfont[Path=/usr/share/fonts/truetype/cmu/,UprightFont=cmunrm.ttf,BoldFont=cmunbx.ttf,ItalicFont=cmunti.ttf,BoldItalicFont=cmunbi.ttf]{cmuntt.ttf}\setmonofont[Path=/usr/share/fonts/truetype/cmu/,UprightFont=cmuntt.ttf,BoldFont=cmuntb.ttf,ItalicFont=cmunit.ttf,BoldItalicFont=cmuntx.ttf]{cmuntt.ttf}\ttfamily shorttitle}
\end{myitemize}
\setmainfont[Path=/usr/share/fonts/truetype/cmu/,UprightFont=cmunrm.ttf,BoldFont=cmunbx.ttf,ItalicFont=cmunti.ttf,BoldItalicFont=cmunbi.ttf]{cmunrm.ttf}\setmonofont[Path=/usr/share/fonts/truetype/cmu/,UprightFont=cmuntt.ttf,BoldFont=cmuntb.ttf,ItalicFont=cmunit.ttf,BoldItalicFont=cmuntx.ttf]{cmunrm.ttf}

The only field that is always mandatory, is {\ttfamily \setmainfont[Path=/usr/share/fonts/truetype/cmu/,UprightFont=cmunrm.ttf,BoldFont=cmunbx.ttf,ItalicFont=cmunti.ttf,BoldItalicFont=cmunbi.ttf]{cmuntt.ttf}\setmonofont[Path=/usr/share/fonts/truetype/cmu/,UprightFont=cmuntt.ttf,BoldFont=cmuntb.ttf,ItalicFont=cmunit.ttf,BoldItalicFont=cmuntx.ttf]{cmuntt.ttf}\ttfamily title}\setmainfont[Path=/usr/share/fonts/truetype/cmu/,UprightFont=cmunrm.ttf,BoldFont=cmunbx.ttf,ItalicFont=cmunti.ttf,BoldItalicFont=cmunbi.ttf]{cmunrm.ttf}\setmonofont[Path=/usr/share/fonts/truetype/cmu/,UprightFont=cmuntt.ttf,BoldFont=cmuntb.ttf,ItalicFont=cmunit.ttf,BoldItalicFont=cmuntx.ttf]{cmunrm.ttf}. All entry types also require either {\ttfamily \setmainfont[Path=/usr/share/fonts/truetype/cmu/,UprightFont=cmunrm.ttf,BoldFont=cmunbx.ttf,ItalicFont=cmunti.ttf,BoldItalicFont=cmunbi.ttf]{cmuntt.ttf}\setmonofont[Path=/usr/share/fonts/truetype/cmu/,UprightFont=cmuntt.ttf,BoldFont=cmuntb.ttf,ItalicFont=cmunit.ttf,BoldItalicFont=cmuntx.ttf]{cmuntt.ttf}\ttfamily date}{$\text{ }$}\setmainfont[Path=/usr/share/fonts/truetype/cmu/,UprightFont=cmunrm.ttf,BoldFont=cmunbx.ttf,ItalicFont=cmunti.ttf,BoldItalicFont=cmunbi.ttf]{cmunrm.ttf}\setmonofont[Path=/usr/share/fonts/truetype/cmu/,UprightFont=cmuntt.ttf,BoldFont=cmuntb.ttf,ItalicFont=cmunit.ttf,BoldItalicFont=cmuntx.ttf]{cmunrm.ttf} or {\ttfamily \setmainfont[Path=/usr/share/fonts/truetype/cmu/,UprightFont=cmunrm.ttf,BoldFont=cmunbx.ttf,ItalicFont=cmunti.ttf,BoldItalicFont=cmunbi.ttf]{cmuntt.ttf}\setmonofont[Path=/usr/share/fonts/truetype/cmu/,UprightFont=cmuntt.ttf,BoldFont=cmuntb.ttf,ItalicFont=cmunit.ttf,BoldItalicFont=cmuntx.ttf]{cmuntt.ttf}\ttfamily year}{$\text{ }$}\setmainfont[Path=/usr/share/fonts/truetype/cmu/,UprightFont=cmunrm.ttf,BoldFont=cmunbx.ttf,ItalicFont=cmunti.ttf,BoldItalicFont=cmunbi.ttf]{cmunrm.ttf}\setmonofont[Path=/usr/share/fonts/truetype/cmu/,UprightFont=cmuntt.ttf,BoldFont=cmuntb.ttf,ItalicFont=cmunit.ttf,BoldItalicFont=cmuntx.ttf]{cmunrm.ttf} and they specify which of {\ttfamily \setmainfont[Path=/usr/share/fonts/truetype/cmu/,UprightFont=cmunrm.ttf,BoldFont=cmunbx.ttf,ItalicFont=cmunti.ttf,BoldItalicFont=cmunbi.ttf]{cmuntt.ttf}\setmonofont[Path=/usr/share/fonts/truetype/cmu/,UprightFont=cmuntt.ttf,BoldFont=cmuntb.ttf,ItalicFont=cmunit.ttf,BoldItalicFont=cmuntx.ttf]{cmuntt.ttf}\ttfamily author}{$\text{ }$}\setmainfont[Path=/usr/share/fonts/truetype/cmu/,UprightFont=cmunrm.ttf,BoldFont=cmunbx.ttf,ItalicFont=cmunti.ttf,BoldItalicFont=cmunbi.ttf]{cmunrm.ttf}\setmonofont[Path=/usr/share/fonts/truetype/cmu/,UprightFont=cmuntt.ttf,BoldFont=cmuntb.ttf,ItalicFont=cmunit.ttf,BoldItalicFont=cmuntx.ttf]{cmunrm.ttf} and {\ttfamily \setmainfont[Path=/usr/share/fonts/truetype/cmu/,UprightFont=cmunrm.ttf,BoldFont=cmunbx.ttf,ItalicFont=cmunti.ttf,BoldItalicFont=cmunbi.ttf]{cmuntt.ttf}\setmonofont[Path=/usr/share/fonts/truetype/cmu/,UprightFont=cmuntt.ttf,BoldFont=cmuntb.ttf,ItalicFont=cmunit.ttf,BoldItalicFont=cmuntx.ttf]{cmuntt.ttf}\ttfamily editor}{$\text{ }$}\setmainfont[Path=/usr/share/fonts/truetype/cmu/,UprightFont=cmunrm.ttf,BoldFont=cmunbx.ttf,ItalicFont=cmunti.ttf,BoldItalicFont=cmunbi.ttf]{cmunrm.ttf}\setmonofont[Path=/usr/share/fonts/truetype/cmu/,UprightFont=cmuntt.ttf,BoldFont=cmuntb.ttf,ItalicFont=cmunit.ttf,BoldItalicFont=cmuntx.ttf]{cmunrm.ttf} they expect or whether they can use both. Some field types can optionally be used with any entry type:
\begin{myitemize}
\item{}  {\ttfamily \setmainfont[Path=/usr/share/fonts/truetype/cmu/,UprightFont=cmunrm.ttf,BoldFont=cmunbx.ttf,ItalicFont=cmunti.ttf,BoldItalicFont=cmunbi.ttf]{cmuntt.ttf}\setmonofont[Path=/usr/share/fonts/truetype/cmu/,UprightFont=cmuntt.ttf,BoldFont=cmuntb.ttf,ItalicFont=cmunit.ttf,BoldItalicFont=cmuntx.ttf]{cmuntt.ttf}\ttfamily addendum}\setmainfont[Path=/usr/share/fonts/truetype/cmu/,UprightFont=cmunrm.ttf,BoldFont=cmunbx.ttf,ItalicFont=cmunti.ttf,BoldItalicFont=cmunbi.ttf]{cmunrm.ttf}\setmonofont[Path=/usr/share/fonts/truetype/cmu/,UprightFont=cmuntt.ttf,BoldFont=cmuntb.ttf,ItalicFont=cmunit.ttf,BoldItalicFont=cmuntx.ttf]{cmunrm.ttf}, {\ttfamily \setmainfont[Path=/usr/share/fonts/truetype/cmu/,UprightFont=cmunrm.ttf,BoldFont=cmunbx.ttf,ItalicFont=cmunti.ttf,BoldItalicFont=cmunbi.ttf]{cmuntt.ttf}\setmonofont[Path=/usr/share/fonts/truetype/cmu/,UprightFont=cmuntt.ttf,BoldFont=cmuntb.ttf,ItalicFont=cmunit.ttf,BoldItalicFont=cmuntx.ttf]{cmuntt.ttf}\ttfamily note}
\item{} {$\text{ }$}\setmainfont[Path=/usr/share/fonts/truetype/cmu/,UprightFont=cmunrm.ttf,BoldFont=cmunbx.ttf,ItalicFont=cmunti.ttf,BoldItalicFont=cmunbi.ttf]{cmunrm.ttf}\setmonofont[Path=/usr/share/fonts/truetype/cmu/,UprightFont=cmuntt.ttf,BoldFont=cmuntb.ttf,ItalicFont=cmunit.ttf,BoldItalicFont=cmuntx.ttf]{cmunrm.ttf} {\ttfamily \setmainfont[Path=/usr/share/fonts/truetype/cmu/,UprightFont=cmunrm.ttf,BoldFont=cmunbx.ttf,ItalicFont=cmunti.ttf,BoldItalicFont=cmunbi.ttf]{cmuntt.ttf}\setmonofont[Path=/usr/share/fonts/truetype/cmu/,UprightFont=cmuntt.ttf,BoldFont=cmuntb.ttf,ItalicFont=cmunit.ttf,BoldItalicFont=cmuntx.ttf]{cmuntt.ttf}\ttfamily language}
\item{} {$\text{ }$}\setmainfont[Path=/usr/share/fonts/truetype/cmu/,UprightFont=cmunrm.ttf,BoldFont=cmunbx.ttf,ItalicFont=cmunti.ttf,BoldItalicFont=cmunbi.ttf]{cmunrm.ttf}\setmonofont[Path=/usr/share/fonts/truetype/cmu/,UprightFont=cmuntt.ttf,BoldFont=cmuntb.ttf,ItalicFont=cmunit.ttf,BoldItalicFont=cmuntx.ttf]{cmunrm.ttf} {\ttfamily \setmainfont[Path=/usr/share/fonts/truetype/cmu/,UprightFont=cmunrm.ttf,BoldFont=cmunbx.ttf,ItalicFont=cmunti.ttf,BoldItalicFont=cmunbi.ttf]{cmuntt.ttf}\setmonofont[Path=/usr/share/fonts/truetype/cmu/,UprightFont=cmuntt.ttf,BoldFont=cmuntb.ttf,ItalicFont=cmunit.ttf,BoldItalicFont=cmuntx.ttf]{cmuntt.ttf}\ttfamily pubstate}
\item{} {$\text{ }$}\setmainfont[Path=/usr/share/fonts/truetype/cmu/,UprightFont=cmunrm.ttf,BoldFont=cmunbx.ttf,ItalicFont=cmunti.ttf,BoldItalicFont=cmunbi.ttf]{cmunrm.ttf}\setmonofont[Path=/usr/share/fonts/truetype/cmu/,UprightFont=cmuntt.ttf,BoldFont=cmuntb.ttf,ItalicFont=cmunit.ttf,BoldItalicFont=cmuntx.ttf]{cmunrm.ttf} {\ttfamily \setmainfont[Path=/usr/share/fonts/truetype/cmu/,UprightFont=cmunrm.ttf,BoldFont=cmunbx.ttf,ItalicFont=cmunti.ttf,BoldItalicFont=cmunbi.ttf]{cmuntt.ttf}\setmonofont[Path=/usr/share/fonts/truetype/cmu/,UprightFont=cmuntt.ttf,BoldFont=cmuntb.ttf,ItalicFont=cmunit.ttf,BoldItalicFont=cmuntx.ttf]{cmuntt.ttf}\ttfamily urldate}
\end{myitemize}
\setmainfont[Path=/usr/share/fonts/truetype/cmu/,UprightFont=cmunrm.ttf,BoldFont=cmunbx.ttf,ItalicFont=cmunti.ttf,BoldItalicFont=cmunbi.ttf]{cmunrm.ttf}\setmonofont[Path=/usr/share/fonts/truetype/cmu/,UprightFont=cmuntt.ttf,BoldFont=cmuntb.ttf,ItalicFont=cmunit.ttf,BoldItalicFont=cmuntx.ttf]{cmunrm.ttf}
All physical (print) entry types share further optional field types:
\begin{myitemize}
\item{}  {\ttfamily \setmainfont[Path=/usr/share/fonts/truetype/cmu/,UprightFont=cmunrm.ttf,BoldFont=cmunbx.ttf,ItalicFont=cmunti.ttf,BoldItalicFont=cmunbi.ttf]{cmuntt.ttf}\setmonofont[Path=/usr/share/fonts/truetype/cmu/,UprightFont=cmuntt.ttf,BoldFont=cmuntb.ttf,ItalicFont=cmunit.ttf,BoldItalicFont=cmuntx.ttf]{cmuntt.ttf}\ttfamily url}\setmainfont[Path=/usr/share/fonts/truetype/cmu/,UprightFont=cmunrm.ttf,BoldFont=cmunbx.ttf,ItalicFont=cmunti.ttf,BoldItalicFont=cmunbi.ttf]{cmunrm.ttf}\setmonofont[Path=/usr/share/fonts/truetype/cmu/,UprightFont=cmuntt.ttf,BoldFont=cmuntb.ttf,ItalicFont=cmunit.ttf,BoldItalicFont=cmuntx.ttf]{cmunrm.ttf}, {\ttfamily \setmainfont[Path=/usr/share/fonts/truetype/cmu/,UprightFont=cmunrm.ttf,BoldFont=cmunbx.ttf,ItalicFont=cmunti.ttf,BoldItalicFont=cmunbi.ttf]{cmuntt.ttf}\setmonofont[Path=/usr/share/fonts/truetype/cmu/,UprightFont=cmuntt.ttf,BoldFont=cmuntb.ttf,ItalicFont=cmunit.ttf,BoldItalicFont=cmuntx.ttf]{cmuntt.ttf}\ttfamily doi}
\item{} {$\text{ }$}\setmainfont[Path=/usr/share/fonts/truetype/cmu/,UprightFont=cmunrm.ttf,BoldFont=cmunbx.ttf,ItalicFont=cmunti.ttf,BoldItalicFont=cmunbi.ttf]{cmunrm.ttf}\setmonofont[Path=/usr/share/fonts/truetype/cmu/,UprightFont=cmuntt.ttf,BoldFont=cmuntb.ttf,ItalicFont=cmunit.ttf,BoldItalicFont=cmuntx.ttf]{cmunrm.ttf} {\ttfamily \setmainfont[Path=/usr/share/fonts/truetype/cmu/,UprightFont=cmunrm.ttf,BoldFont=cmunbx.ttf,ItalicFont=cmunti.ttf,BoldItalicFont=cmunbi.ttf]{cmuntt.ttf}\setmonofont[Path=/usr/share/fonts/truetype/cmu/,UprightFont=cmuntt.ttf,BoldFont=cmuntb.ttf,ItalicFont=cmunit.ttf,BoldItalicFont=cmuntx.ttf]{cmuntt.ttf}\ttfamily eprint}\setmainfont[Path=/usr/share/fonts/truetype/cmu/,UprightFont=cmunrm.ttf,BoldFont=cmunbx.ttf,ItalicFont=cmunti.ttf,BoldItalicFont=cmunbi.ttf]{cmunrm.ttf}\setmonofont[Path=/usr/share/fonts/truetype/cmu/,UprightFont=cmuntt.ttf,BoldFont=cmuntb.ttf,ItalicFont=cmunit.ttf,BoldItalicFont=cmuntx.ttf]{cmunrm.ttf}, {\ttfamily \setmainfont[Path=/usr/share/fonts/truetype/cmu/,UprightFont=cmunrm.ttf,BoldFont=cmunbx.ttf,ItalicFont=cmunti.ttf,BoldItalicFont=cmunbi.ttf]{cmuntt.ttf}\setmonofont[Path=/usr/share/fonts/truetype/cmu/,UprightFont=cmuntt.ttf,BoldFont=cmuntb.ttf,ItalicFont=cmunit.ttf,BoldItalicFont=cmuntx.ttf]{cmuntt.ttf}\ttfamily eprintclass}\setmainfont[Path=/usr/share/fonts/truetype/cmu/,UprightFont=cmunrm.ttf,BoldFont=cmunbx.ttf,ItalicFont=cmunti.ttf,BoldItalicFont=cmunbi.ttf]{cmunrm.ttf}\setmonofont[Path=/usr/share/fonts/truetype/cmu/,UprightFont=cmuntt.ttf,BoldFont=cmuntb.ttf,ItalicFont=cmunit.ttf,BoldItalicFont=cmuntx.ttf]{cmunrm.ttf}, {\ttfamily \setmainfont[Path=/usr/share/fonts/truetype/cmu/,UprightFont=cmunrm.ttf,BoldFont=cmunbx.ttf,ItalicFont=cmunti.ttf,BoldItalicFont=cmunbi.ttf]{cmuntt.ttf}\setmonofont[Path=/usr/share/fonts/truetype/cmu/,UprightFont=cmuntt.ttf,BoldFont=cmuntb.ttf,ItalicFont=cmunit.ttf,BoldItalicFont=cmuntx.ttf]{cmuntt.ttf}\ttfamily eprinttype}
\end{myitemize}
\setmainfont[Path=/usr/share/fonts/truetype/cmu/,UprightFont=cmunrm.ttf,BoldFont=cmunbx.ttf,ItalicFont=cmunti.ttf,BoldItalicFont=cmunbi.ttf]{cmunrm.ttf}\setmonofont[Path=/usr/share/fonts/truetype/cmu/,UprightFont=cmuntt.ttf,BoldFont=cmuntb.ttf,ItalicFont=cmunit.ttf,BoldItalicFont=cmuntx.ttf]{cmunrm.ttf}

Multimedia entry types 
\begin{myitemize}
\item{}  {\ttfamily \setmainfont[Path=/usr/share/fonts/truetype/cmu/,UprightFont=cmunrm.ttf,BoldFont=cmunbx.ttf,ItalicFont=cmunti.ttf,BoldItalicFont=cmunbi.ttf]{cmuntt.ttf}\setmonofont[Path=/usr/share/fonts/truetype/cmu/,UprightFont=cmuntt.ttf,BoldFont=cmuntb.ttf,ItalicFont=cmunit.ttf,BoldItalicFont=cmuntx.ttf]{cmuntt.ttf}\ttfamily @artwork}
\item{} {$\text{ }$}\setmainfont[Path=/usr/share/fonts/truetype/cmu/,UprightFont=cmunrm.ttf,BoldFont=cmunbx.ttf,ItalicFont=cmunti.ttf,BoldItalicFont=cmunbi.ttf]{cmunrm.ttf}\setmonofont[Path=/usr/share/fonts/truetype/cmu/,UprightFont=cmuntt.ttf,BoldFont=cmuntb.ttf,ItalicFont=cmunit.ttf,BoldItalicFont=cmuntx.ttf]{cmunrm.ttf} {\ttfamily \setmainfont[Path=/usr/share/fonts/truetype/cmu/,UprightFont=cmunrm.ttf,BoldFont=cmunbx.ttf,ItalicFont=cmunti.ttf,BoldItalicFont=cmunbi.ttf]{cmuntt.ttf}\setmonofont[Path=/usr/share/fonts/truetype/cmu/,UprightFont=cmuntt.ttf,BoldFont=cmuntb.ttf,ItalicFont=cmunit.ttf,BoldItalicFont=cmuntx.ttf]{cmuntt.ttf}\ttfamily @audio}
\item{} {$\text{ }$}\setmainfont[Path=/usr/share/fonts/truetype/cmu/,UprightFont=cmunrm.ttf,BoldFont=cmunbx.ttf,ItalicFont=cmunti.ttf,BoldItalicFont=cmunbi.ttf]{cmunrm.ttf}\setmonofont[Path=/usr/share/fonts/truetype/cmu/,UprightFont=cmuntt.ttf,BoldFont=cmuntb.ttf,ItalicFont=cmunit.ttf,BoldItalicFont=cmuntx.ttf]{cmunrm.ttf} {\ttfamily \setmainfont[Path=/usr/share/fonts/truetype/cmu/,UprightFont=cmunrm.ttf,BoldFont=cmunbx.ttf,ItalicFont=cmunti.ttf,BoldItalicFont=cmunbi.ttf]{cmuntt.ttf}\setmonofont[Path=/usr/share/fonts/truetype/cmu/,UprightFont=cmuntt.ttf,BoldFont=cmuntb.ttf,ItalicFont=cmunit.ttf,BoldItalicFont=cmuntx.ttf]{cmuntt.ttf}\ttfamily @image}
\item{} {$\text{ }$}\setmainfont[Path=/usr/share/fonts/truetype/cmu/,UprightFont=cmunrm.ttf,BoldFont=cmunbx.ttf,ItalicFont=cmunti.ttf,BoldItalicFont=cmunbi.ttf]{cmunrm.ttf}\setmonofont[Path=/usr/share/fonts/truetype/cmu/,UprightFont=cmuntt.ttf,BoldFont=cmuntb.ttf,ItalicFont=cmunit.ttf,BoldItalicFont=cmuntx.ttf]{cmunrm.ttf} {\ttfamily \setmainfont[Path=/usr/share/fonts/truetype/cmu/,UprightFont=cmunrm.ttf,BoldFont=cmunbx.ttf,ItalicFont=cmunti.ttf,BoldItalicFont=cmunbi.ttf]{cmuntt.ttf}\setmonofont[Path=/usr/share/fonts/truetype/cmu/,UprightFont=cmuntt.ttf,BoldFont=cmuntb.ttf,ItalicFont=cmunit.ttf,BoldItalicFont=cmuntx.ttf]{cmuntt.ttf}\ttfamily @movie}
\item{} {$\text{ }$}\setmainfont[Path=/usr/share/fonts/truetype/cmu/,UprightFont=cmunrm.ttf,BoldFont=cmunbx.ttf,ItalicFont=cmunti.ttf,BoldItalicFont=cmunbi.ttf]{cmunrm.ttf}\setmonofont[Path=/usr/share/fonts/truetype/cmu/,UprightFont=cmuntt.ttf,BoldFont=cmuntb.ttf,ItalicFont=cmunit.ttf,BoldItalicFont=cmuntx.ttf]{cmunrm.ttf} {\ttfamily \setmainfont[Path=/usr/share/fonts/truetype/cmu/,UprightFont=cmunrm.ttf,BoldFont=cmunbx.ttf,ItalicFont=cmunti.ttf,BoldItalicFont=cmunbi.ttf]{cmuntt.ttf}\setmonofont[Path=/usr/share/fonts/truetype/cmu/,UprightFont=cmuntt.ttf,BoldFont=cmuntb.ttf,ItalicFont=cmunit.ttf,BoldItalicFont=cmuntx.ttf]{cmuntt.ttf}\ttfamily @music}
\item{} {$\text{ }$}\setmainfont[Path=/usr/share/fonts/truetype/cmu/,UprightFont=cmunrm.ttf,BoldFont=cmunbx.ttf,ItalicFont=cmunti.ttf,BoldItalicFont=cmunbi.ttf]{cmunrm.ttf}\setmonofont[Path=/usr/share/fonts/truetype/cmu/,UprightFont=cmuntt.ttf,BoldFont=cmuntb.ttf,ItalicFont=cmunit.ttf,BoldItalicFont=cmuntx.ttf]{cmunrm.ttf} {\ttfamily \setmainfont[Path=/usr/share/fonts/truetype/cmu/,UprightFont=cmunrm.ttf,BoldFont=cmunbx.ttf,ItalicFont=cmunti.ttf,BoldItalicFont=cmunbi.ttf]{cmuntt.ttf}\setmonofont[Path=/usr/share/fonts/truetype/cmu/,UprightFont=cmuntt.ttf,BoldFont=cmuntb.ttf,ItalicFont=cmunit.ttf,BoldItalicFont=cmuntx.ttf]{cmuntt.ttf}\ttfamily @performance}
\item{} {$\text{ }$}\setmainfont[Path=/usr/share/fonts/truetype/cmu/,UprightFont=cmunrm.ttf,BoldFont=cmunbx.ttf,ItalicFont=cmunti.ttf,BoldItalicFont=cmunbi.ttf]{cmunrm.ttf}\setmonofont[Path=/usr/share/fonts/truetype/cmu/,UprightFont=cmuntt.ttf,BoldFont=cmuntb.ttf,ItalicFont=cmunit.ttf,BoldItalicFont=cmuntx.ttf]{cmunrm.ttf} {\ttfamily \setmainfont[Path=/usr/share/fonts/truetype/cmu/,UprightFont=cmunrm.ttf,BoldFont=cmunbx.ttf,ItalicFont=cmunti.ttf,BoldItalicFont=cmunbi.ttf]{cmuntt.ttf}\setmonofont[Path=/usr/share/fonts/truetype/cmu/,UprightFont=cmuntt.ttf,BoldFont=cmuntb.ttf,ItalicFont=cmunit.ttf,BoldItalicFont=cmuntx.ttf]{cmuntt.ttf}\ttfamily @video}
\item{} {$\text{ }$}\setmainfont[Path=/usr/share/fonts/truetype/cmu/,UprightFont=cmunrm.ttf,BoldFont=cmunbx.ttf,ItalicFont=cmunti.ttf,BoldItalicFont=cmunbi.ttf]{cmunrm.ttf}\setmonofont[Path=/usr/share/fonts/truetype/cmu/,UprightFont=cmuntt.ttf,BoldFont=cmuntb.ttf,ItalicFont=cmunit.ttf,BoldItalicFont=cmuntx.ttf]{cmunrm.ttf} {\ttfamily \setmainfont[Path=/usr/share/fonts/truetype/cmu/,UprightFont=cmunrm.ttf,BoldFont=cmunbx.ttf,ItalicFont=cmunti.ttf,BoldItalicFont=cmunbi.ttf]{cmuntt.ttf}\setmonofont[Path=/usr/share/fonts/truetype/cmu/,UprightFont=cmuntt.ttf,BoldFont=cmuntb.ttf,ItalicFont=cmunit.ttf,BoldItalicFont=cmuntx.ttf]{cmuntt.ttf}\ttfamily @software}
\end{myitemize}
\setmainfont[Path=/usr/share/fonts/truetype/cmu/,UprightFont=cmunrm.ttf,BoldFont=cmunbx.ttf,ItalicFont=cmunti.ttf,BoldItalicFont=cmunbi.ttf]{cmunrm.ttf}\setmonofont[Path=/usr/share/fonts/truetype/cmu/,UprightFont=cmuntt.ttf,BoldFont=cmuntb.ttf,ItalicFont=cmunit.ttf,BoldItalicFont=cmuntx.ttf]{cmunrm.ttf}
and legal entry types 
\begin{myitemize}
\item{}  {\ttfamily \setmainfont[Path=/usr/share/fonts/truetype/cmu/,UprightFont=cmunrm.ttf,BoldFont=cmunbx.ttf,ItalicFont=cmunti.ttf,BoldItalicFont=cmunbi.ttf]{cmuntt.ttf}\setmonofont[Path=/usr/share/fonts/truetype/cmu/,UprightFont=cmuntt.ttf,BoldFont=cmuntb.ttf,ItalicFont=cmunit.ttf,BoldItalicFont=cmuntx.ttf]{cmuntt.ttf}\ttfamily @commentary}
\item{} {$\text{ }$}\setmainfont[Path=/usr/share/fonts/truetype/cmu/,UprightFont=cmunrm.ttf,BoldFont=cmunbx.ttf,ItalicFont=cmunti.ttf,BoldItalicFont=cmunbi.ttf]{cmunrm.ttf}\setmonofont[Path=/usr/share/fonts/truetype/cmu/,UprightFont=cmuntt.ttf,BoldFont=cmuntb.ttf,ItalicFont=cmunit.ttf,BoldItalicFont=cmuntx.ttf]{cmunrm.ttf} {\ttfamily \setmainfont[Path=/usr/share/fonts/truetype/cmu/,UprightFont=cmunrm.ttf,BoldFont=cmunbx.ttf,ItalicFont=cmunti.ttf,BoldItalicFont=cmunbi.ttf]{cmuntt.ttf}\setmonofont[Path=/usr/share/fonts/truetype/cmu/,UprightFont=cmuntt.ttf,BoldFont=cmuntb.ttf,ItalicFont=cmunit.ttf,BoldItalicFont=cmuntx.ttf]{cmuntt.ttf}\ttfamily @jurisdiction}
\item{} {$\text{ }$}\setmainfont[Path=/usr/share/fonts/truetype/cmu/,UprightFont=cmunrm.ttf,BoldFont=cmunbx.ttf,ItalicFont=cmunti.ttf,BoldItalicFont=cmunbi.ttf]{cmunrm.ttf}\setmonofont[Path=/usr/share/fonts/truetype/cmu/,UprightFont=cmuntt.ttf,BoldFont=cmuntb.ttf,ItalicFont=cmunit.ttf,BoldItalicFont=cmuntx.ttf]{cmunrm.ttf} {\ttfamily \setmainfont[Path=/usr/share/fonts/truetype/cmu/,UprightFont=cmunrm.ttf,BoldFont=cmunbx.ttf,ItalicFont=cmunti.ttf,BoldItalicFont=cmunbi.ttf]{cmuntt.ttf}\setmonofont[Path=/usr/share/fonts/truetype/cmu/,UprightFont=cmuntt.ttf,BoldFont=cmuntb.ttf,ItalicFont=cmunit.ttf,BoldItalicFont=cmuntx.ttf]{cmuntt.ttf}\ttfamily @legislation}
\item{} {$\text{ }$}\setmainfont[Path=/usr/share/fonts/truetype/cmu/,UprightFont=cmunrm.ttf,BoldFont=cmunbx.ttf,ItalicFont=cmunti.ttf,BoldItalicFont=cmunbi.ttf]{cmunrm.ttf}\setmonofont[Path=/usr/share/fonts/truetype/cmu/,UprightFont=cmuntt.ttf,BoldFont=cmuntb.ttf,ItalicFont=cmunit.ttf,BoldItalicFont=cmuntx.ttf]{cmunrm.ttf} {\ttfamily \setmainfont[Path=/usr/share/fonts/truetype/cmu/,UprightFont=cmunrm.ttf,BoldFont=cmunbx.ttf,ItalicFont=cmunti.ttf,BoldItalicFont=cmunbi.ttf]{cmuntt.ttf}\setmonofont[Path=/usr/share/fonts/truetype/cmu/,UprightFont=cmuntt.ttf,BoldFont=cmuntb.ttf,ItalicFont=cmunit.ttf,BoldItalicFont=cmuntx.ttf]{cmuntt.ttf}\ttfamily @legal}
\item{} {$\text{ }$}\setmainfont[Path=/usr/share/fonts/truetype/cmu/,UprightFont=cmunrm.ttf,BoldFont=cmunbx.ttf,ItalicFont=cmunti.ttf,BoldItalicFont=cmunbi.ttf]{cmunrm.ttf}\setmonofont[Path=/usr/share/fonts/truetype/cmu/,UprightFont=cmuntt.ttf,BoldFont=cmuntb.ttf,ItalicFont=cmunit.ttf,BoldItalicFont=cmuntx.ttf]{cmunrm.ttf} {\ttfamily \setmainfont[Path=/usr/share/fonts/truetype/cmu/,UprightFont=cmunrm.ttf,BoldFont=cmunbx.ttf,ItalicFont=cmunti.ttf,BoldItalicFont=cmunbi.ttf]{cmuntt.ttf}\setmonofont[Path=/usr/share/fonts/truetype/cmu/,UprightFont=cmuntt.ttf,BoldFont=cmuntb.ttf,ItalicFont=cmunit.ttf,BoldItalicFont=cmuntx.ttf]{cmuntt.ttf}\ttfamily @letter}
\item{} {$\text{ }$}\setmainfont[Path=/usr/share/fonts/truetype/cmu/,UprightFont=cmunrm.ttf,BoldFont=cmunbx.ttf,ItalicFont=cmunti.ttf,BoldItalicFont=cmunbi.ttf]{cmunrm.ttf}\setmonofont[Path=/usr/share/fonts/truetype/cmu/,UprightFont=cmuntt.ttf,BoldFont=cmuntb.ttf,ItalicFont=cmunit.ttf,BoldItalicFont=cmuntx.ttf]{cmunrm.ttf} {\ttfamily \setmainfont[Path=/usr/share/fonts/truetype/cmu/,UprightFont=cmunrm.ttf,BoldFont=cmunbx.ttf,ItalicFont=cmunti.ttf,BoldItalicFont=cmunbi.ttf]{cmuntt.ttf}\setmonofont[Path=/usr/share/fonts/truetype/cmu/,UprightFont=cmuntt.ttf,BoldFont=cmuntb.ttf,ItalicFont=cmunit.ttf,BoldItalicFont=cmuntx.ttf]{cmuntt.ttf}\ttfamily @review}
\item{} {$\text{ }$}\setmainfont[Path=/usr/share/fonts/truetype/cmu/,UprightFont=cmunrm.ttf,BoldFont=cmunbx.ttf,ItalicFont=cmunti.ttf,BoldItalicFont=cmunbi.ttf]{cmunrm.ttf}\setmonofont[Path=/usr/share/fonts/truetype/cmu/,UprightFont=cmuntt.ttf,BoldFont=cmuntb.ttf,ItalicFont=cmunit.ttf,BoldItalicFont=cmuntx.ttf]{cmunrm.ttf} {\ttfamily \setmainfont[Path=/usr/share/fonts/truetype/cmu/,UprightFont=cmunrm.ttf,BoldFont=cmunbx.ttf,ItalicFont=cmunti.ttf,BoldItalicFont=cmunbi.ttf]{cmuntt.ttf}\setmonofont[Path=/usr/share/fonts/truetype/cmu/,UprightFont=cmuntt.ttf,BoldFont=cmuntb.ttf,ItalicFont=cmunit.ttf,BoldItalicFont=cmuntx.ttf]{cmuntt.ttf}\ttfamily @standard}
\end{myitemize}
\setmainfont[Path=/usr/share/fonts/truetype/cmu/,UprightFont=cmunrm.ttf,BoldFont=cmunbx.ttf,ItalicFont=cmunti.ttf,BoldItalicFont=cmunbi.ttf]{cmunrm.ttf}\setmonofont[Path=/usr/share/fonts/truetype/cmu/,UprightFont=cmuntt.ttf,BoldFont=cmuntb.ttf,ItalicFont=cmunit.ttf,BoldItalicFont=cmuntx.ttf]{cmunrm.ttf}
are defined, but not yet supported (well).

The entry types {\ttfamily \setmainfont[Path=/usr/share/fonts/truetype/cmu/,UprightFont=cmunrm.ttf,BoldFont=cmunbx.ttf,ItalicFont=cmunti.ttf,BoldItalicFont=cmunbi.ttf]{cmuntt.ttf}\setmonofont[Path=/usr/share/fonts/truetype/cmu/,UprightFont=cmuntt.ttf,BoldFont=cmuntb.ttf,ItalicFont=cmunit.ttf,BoldItalicFont=cmuntx.ttf]{cmuntt.ttf}\ttfamily @bibnote}\setmainfont[Path=/usr/share/fonts/truetype/cmu/,UprightFont=cmunrm.ttf,BoldFont=cmunbx.ttf,ItalicFont=cmunti.ttf,BoldItalicFont=cmunbi.ttf]{cmunrm.ttf}\setmonofont[Path=/usr/share/fonts/truetype/cmu/,UprightFont=cmuntt.ttf,BoldFont=cmuntb.ttf,ItalicFont=cmunit.ttf,BoldItalicFont=cmuntx.ttf]{cmunrm.ttf}, {\ttfamily \setmainfont[Path=/usr/share/fonts/truetype/cmu/,UprightFont=cmunrm.ttf,BoldFont=cmunbx.ttf,ItalicFont=cmunti.ttf,BoldItalicFont=cmunbi.ttf]{cmuntt.ttf}\setmonofont[Path=/usr/share/fonts/truetype/cmu/,UprightFont=cmuntt.ttf,BoldFont=cmuntb.ttf,ItalicFont=cmunit.ttf,BoldItalicFont=cmuntx.ttf]{cmuntt.ttf}\ttfamily @set}{$\text{ }$}\setmainfont[Path=/usr/share/fonts/truetype/cmu/,UprightFont=cmunrm.ttf,BoldFont=cmunbx.ttf,ItalicFont=cmunti.ttf,BoldItalicFont=cmunbi.ttf]{cmunrm.ttf}\setmonofont[Path=/usr/share/fonts/truetype/cmu/,UprightFont=cmuntt.ttf,BoldFont=cmuntb.ttf,ItalicFont=cmunit.ttf,BoldItalicFont=cmuntx.ttf]{cmunrm.ttf} and {\ttfamily \setmainfont[Path=/usr/share/fonts/truetype/cmu/,UprightFont=cmunrm.ttf,BoldFont=cmunbx.ttf,ItalicFont=cmunti.ttf,BoldItalicFont=cmunbi.ttf]{cmuntt.ttf}\setmonofont[Path=/usr/share/fonts/truetype/cmu/,UprightFont=cmuntt.ttf,BoldFont=cmuntb.ttf,ItalicFont=cmunit.ttf,BoldItalicFont=cmuntx.ttf]{cmuntt.ttf}\ttfamily @xdata}{$\text{ }$}\setmainfont[Path=/usr/share/fonts/truetype/cmu/,UprightFont=cmunrm.ttf,BoldFont=cmunbx.ttf,ItalicFont=cmunti.ttf,BoldItalicFont=cmunbi.ttf]{cmunrm.ttf}\setmonofont[Path=/usr/share/fonts/truetype/cmu/,UprightFont=cmuntt.ttf,BoldFont=cmuntb.ttf,ItalicFont=cmunit.ttf,BoldItalicFont=cmuntx.ttf]{cmunrm.ttf} are special.
\subsection{Printing bibliography}
\label{698}

Presuming we have defined our references in a file called references.bib, we add this to biblatex by adding the following to the preamble:

\begin{Shaded}
\begin{Highlighting}[]

\NormalTok{\textbackslash{}addbibresource\{references.bib\}}\newline
\end{Highlighting}
\end{Shaded}


Print the bibiography with this macro (usually at the end of the document body):

\begin{Shaded}
\begin{Highlighting}[]

\NormalTok{\textbackslash{}printbibliography}\newline
\end{Highlighting}
\end{Shaded}

\subsubsection{Printing separate bibliographies}
\label{699}

We want to separate the bibliography into papers, books and others
\begin{Shaded}
\begin{Highlighting}[]

\NormalTok{\textbackslash{}printbibliography[title=\{Book references\},type=book]}
\NormalTok{\textbackslash{}printbibliography[title=\{Article references\},type=article]}
\NormalTok{\textbackslash{}printbibliography[title=\{Other references\}, nottype=article, nottype=book]}
\end{Highlighting}
\end{Shaded}


If the bib entries are located in multiple files we can add them like this:
\begin{Shaded}
\begin{Highlighting}[]

\NormalTok{\textbackslash{}addbibresource\{references.bib\}}
\NormalTok{\textbackslash{}addbibresource\{other.bib\}}
\end{Highlighting}
\end{Shaded}


We can also filter on other fields, such as entrysubtype. If we define our online resources like this:
\begin{Shaded}
\begin{Highlighting}[]

\NormalTok{@misc\{some-resource,}
   \NormalTok{...}
   \NormalTok{entrysubtype = \{inet\},}
\NormalTok{\}}
\end{Highlighting}
\end{Shaded}

we filter with 
\begin{Shaded}
\begin{Highlighting}[]

\NormalTok{\textbackslash{}printbibliography[title=\{Online\ensuremath{\text{ }}resources\},\ensuremath{\text{ }}subtype=inet]}\newline
\end{Highlighting}
\end{Shaded}

\subsubsection{Example with prefix keys, subheadings and table of contents}
\label{700}

As the numbering of the bibliographies are independent, it can be useful to also separate the bibliographies using prefixnumbers such as a, b and c.
In addition we add a main heading for the bibliographies and add that to the table of contents.

To make \mylref{393}{Hyperref} links point to the correct bibliography section, we also add 
\begin{Shaded}
\begin{Highlighting}[]

\NormalTok{\textbackslash{}phantomsection}\newline
\end{Highlighting}
\end{Shaded}
 before printing each bibliography

\begin{Shaded}
\begin{Highlighting}[]

\NormalTok{\textbackslash{}printbibheading[}
\NormalTok{heading=bibintoc, }\CommentTok{% bibintoc adds the Bibliography to the table of contents}
\NormalTok{title=\{Resources\} }\CommentTok{% If we want to override the default title "Bibliography" }
\NormalTok{]}
\NormalTok{\textbackslash{}phantomsection }\CommentTok{% Requires hyperref package}
\NormalTok{\textbackslash{}printbibliography[title=\{Printed sources\}, heading=subbibliography,}
 \NormalTok{prefixnumbers=\{a\}, type=book, type=article]}
\NormalTok{\textbackslash{}phantomsection}
\NormalTok{\textbackslash{}printbibliography[title=\{Online resources\}, heading=subbibliography,}
 \NormalTok{prefixnumbers=\{c\}, subtype=inet]}
\NormalTok{\textbackslash{}phantomsection}
\NormalTok{\textbackslash{}printbibliography[title=\{Other\}, heading=subbibliography, prefixnumbers=\{c\},}
 \NormalTok{nottype=article, nottype=book, notsubtype=inet]}
\end{Highlighting}
\end{Shaded}

To add each of the bibliographies to the table of contents as sub-{}sections to the main Bibliography, replace {\ttfamily \setmainfont[Path=/usr/share/fonts/truetype/cmu/,UprightFont=cmunrm.ttf,BoldFont=cmunbx.ttf,ItalicFont=cmunti.ttf,BoldItalicFont=cmunbi.ttf]{cmuntt.ttf}\setmonofont[Path=/usr/share/fonts/truetype/cmu/,UprightFont=cmuntt.ttf,BoldFont=cmuntb.ttf,ItalicFont=cmunit.ttf,BoldItalicFont=cmuntx.ttf]{cmuntt.ttf}\ttfamily heading=subbibliography}{$\text{ }$}\setmainfont[Path=/usr/share/fonts/truetype/cmu/,UprightFont=cmunrm.ttf,BoldFont=cmunbx.ttf,ItalicFont=cmunti.ttf,BoldItalicFont=cmunbi.ttf]{cmunrm.ttf}\setmonofont[Path=/usr/share/fonts/truetype/cmu/,UprightFont=cmuntt.ttf,BoldFont=cmuntb.ttf,ItalicFont=cmunit.ttf,BoldItalicFont=cmuntx.ttf]{cmunrm.ttf} with {\ttfamily \setmainfont[Path=/usr/share/fonts/truetype/cmu/,UprightFont=cmunrm.ttf,BoldFont=cmunbx.ttf,ItalicFont=cmunti.ttf,BoldItalicFont=cmunbi.ttf]{cmuntt.ttf}\setmonofont[Path=/usr/share/fonts/truetype/cmu/,UprightFont=cmuntt.ttf,BoldFont=cmuntb.ttf,ItalicFont=cmunit.ttf,BoldItalicFont=cmuntx.ttf]{cmuntt.ttf}\ttfamily heading=subbibintoc}\setmainfont[Path=/usr/share/fonts/truetype/cmu/,UprightFont=cmunrm.ttf,BoldFont=cmunbx.ttf,ItalicFont=cmunti.ttf,BoldItalicFont=cmunbi.ttf]{cmunrm.ttf}\setmonofont[Path=/usr/share/fonts/truetype/cmu/,UprightFont=cmuntt.ttf,BoldFont=cmuntb.ttf,ItalicFont=cmunit.ttf,BoldItalicFont=cmuntx.ttf]{cmunrm.ttf}.
\section{Multiple bibliographies}
\label{701}
\subsection{Using {\ttfamily \setmainfont[Path=/usr/share/fonts/truetype/cmu/,UprightFont=cmunrm.ttf,BoldFont=cmunbx.ttf,ItalicFont=cmunti.ttf,BoldItalicFont=cmunbi.ttf]{cmuntt.ttf}\setmonofont[Path=/usr/share/fonts/truetype/cmu/,UprightFont=cmuntt.ttf,BoldFont=cmuntb.ttf,ItalicFont=cmunit.ttf,BoldItalicFont=cmuntx.ttf]{cmuntt.ttf}\ttfamily multibib}{$\text{ }$}\setmainfont[Path=/usr/share/fonts/truetype/cmu/,UprightFont=cmunrm.ttf,BoldFont=cmunbx.ttf,ItalicFont=cmunti.ttf,BoldItalicFont=cmunbi.ttf]{cmunrm.ttf}\setmonofont[Path=/usr/share/fonts/truetype/cmu/,UprightFont=cmuntt.ttf,BoldFont=cmuntb.ttf,ItalicFont=cmunit.ttf,BoldItalicFont=cmuntx.ttf]{cmunrm.ttf}}
\label{702}

This package is for multiple Bibliographies for different sections in your work. For example, you can generate a bibliography for each chapter.
You can find information about the package on CTAN\myfootnote{\myplainurl{http://ctan.org/pkg/multibib}}
\subsection{Using {\ttfamily \setmainfont[Path=/usr/share/fonts/truetype/cmu/,UprightFont=cmunrm.ttf,BoldFont=cmunbx.ttf,ItalicFont=cmunti.ttf,BoldItalicFont=cmunbi.ttf]{cmuntt.ttf}\setmonofont[Path=/usr/share/fonts/truetype/cmu/,UprightFont=cmuntt.ttf,BoldFont=cmuntb.ttf,ItalicFont=cmunit.ttf,BoldItalicFont=cmuntx.ttf]{cmuntt.ttf}\ttfamily bibtopic}{$\text{ }$}\setmainfont[Path=/usr/share/fonts/truetype/cmu/,UprightFont=cmunrm.ttf,BoldFont=cmunbx.ttf,ItalicFont=cmunti.ttf,BoldItalicFont=cmunbi.ttf]{cmunrm.ttf}\setmonofont[Path=/usr/share/fonts/truetype/cmu/,UprightFont=cmuntt.ttf,BoldFont=cmuntb.ttf,ItalicFont=cmunit.ttf,BoldItalicFont=cmuntx.ttf]{cmunrm.ttf}}
\label{703}

The bibtopic-{}Package\myfootnote{\myplainurl{http://ctan.org/pkg/bibtopic}} is created to differ the citations on more files, so that you can divide the bibliography into more parts.


\begin{Shaded}
\begin{Highlighting}[]

\NormalTok{\textbackslash{}documentclass[11pt]\{article\}}\newline
\NormalTok{\textbackslash{}usepackage\{bibtopic\}}\newline
\NormalTok{\textbackslash{}begin\{document\}}\newline
\ensuremath{\text{ }}\newline
\NormalTok{\textbackslash{}bibliographystyle\{alpha\}}\newline
\NormalTok{\textbackslash{}section\{Testing\}}\newline
\NormalTok{Let’s\ensuremath{\text{ }}cite\ensuremath{\text{ }}all\ensuremath{\text{ }}the\ensuremath{\text{ }}books:\ensuremath{\text{ }}\textbackslash{}cite\{ColBenh:93\}\ensuremath{\text{ }}and}\newline
\NormalTok{\textbackslash{}cite\{Munt:93\};\ensuremath{\text{ }}and\ensuremath{\text{ }}an\ensuremath{\text{ }}article:\ensuremath{\text{ }}\textbackslash{}cite\{RouxSmart:95\}.}\newline
\ensuremath{\text{ }}\newline
\NormalTok{File\ensuremath{\text{ }}books.bib\ensuremath{\text{ }}is\ensuremath{\text{ }}used\ensuremath{\text{ }}for\ensuremath{\text{ }}this\ensuremath{\text{ }}listing:}\newline
\NormalTok{\textbackslash{}begin\{btSect\}\{books\}}\newline
\ensuremath{\text{ }}\NormalTok{\textbackslash{}section\{References\ensuremath{\text{ }}from\ensuremath{\text{ }}books\}}\newline
\ensuremath{\text{ }}\NormalTok{\textbackslash{}btPrintCited}\newline
\NormalTok{\textbackslash{}end\{btSect\}}\newline
\NormalTok{Here,\ensuremath{\text{ }}the\ensuremath{\text{ }}articles.bib\ensuremath{\text{ }}is\ensuremath{\text{ }}used,\ensuremath{\text{ }}and\ensuremath{\text{ }}the\ensuremath{\text{ }}listing\ensuremath{\text{ }}is\ensuremath{\text{ }}in\ensuremath{\text{ }}plain-format\ensuremath{\text{ }}instead\ensuremath{\text{ }}of}\newline
\ensuremath{\text{ }}\NormalTok{the\ensuremath{\text{ }}standard\ensuremath{\text{ }}alpha.}\newline
\NormalTok{\textbackslash{}begin\{btSect\}[plain]\{articles\}}\newline
\ensuremath{\text{ }}\NormalTok{\textbackslash{}section\{References\ensuremath{\text{ }}from\ensuremath{\text{ }}articles\}}\newline
\ensuremath{\text{ }}\NormalTok{\textbackslash{}btPrintCited}\newline
\ensuremath{\text{ }}\NormalTok{\textbackslash{}section\{Articles\ensuremath{\text{ }}not\ensuremath{\text{ }}cited\}}\newline
\ensuremath{\text{ }}\NormalTok{\textbackslash{}btPrintNotCited}\newline
\NormalTok{\textbackslash{}end\{btSect\}}\newline
\NormalTok{Just\ensuremath{\text{ }}print\ensuremath{\text{ }}all\ensuremath{\text{ }}entries\ensuremath{\text{ }}here\ensuremath{\text{ }}with\ensuremath{\text{ }}\textbackslash{}btPrintAll}\newline
\NormalTok{\textbackslash{}begin\{btSect\}[plain]\{internet\}}\newline
\ensuremath{\text{ }}\NormalTok{\textbackslash{}section\{References\ensuremath{\text{ }}from\ensuremath{\text{ }}the\ensuremath{\text{ }}internet\}}\newline
\ensuremath{\text{ }}\NormalTok{\textbackslash{}btPrintAll}\newline
\NormalTok{\textbackslash{}end\{btSect\}}\newline
\NormalTok{\textbackslash{}end\{document\}}\newline
\end{Highlighting}
\end{Shaded}

\section{Notes and references}
\label{704}
\LaTeXNullTemplate{}
\ARoberts{}



\myhref{https://fr.wikibooks.org/wiki/LaTeX\%2FGestion\%20de\%20la\%20bibliographie}{fr:LaTeX/Gestion de la bibliographie}
\myhref{https://ru.wikibooks.org/wiki/LaTeX\%2F\%D0\%A3\%D0\%BF\%D1\%80\%D0\%B0\%D0\%B2\%D0\%BB\%D0\%B5\%D0\%BD\%D0\%B8\%D0\%B5\%20\%D0\%B1\%D0\%B8\%D0\%B1\%D0\%BB\%D0\%B8\%D0\%BE\%D0\%B3\%D1\%80\%D0\%B0\%D1\%84\%D0\%B8\%D0\%B5\%D0\%B9}{ru:LaTeX/Управление библиографией}\chapter{More Bibliographies}

\myminitoc
\label{705}

\label{706}

\LaTeXNullTemplate{}

This is a gentle introduction to using some of the bibliography functionality available to LaTeX users beyond the \mylref{667}{BibTeX} basics.  This introduction won\textquotesingle{}t be discussing how to create new styles or packages but rather how to use some existing ones.
It is worth noting that {\itshape \setmainfont[Path=/usr/share/fonts/truetype/cmu/,UprightFont=cmunrm.ttf,BoldFont=cmunbx.ttf,ItalicFont=cmunti.ttf,BoldItalicFont=cmunbi.ttf]{cmunti.ttf}\setmonofont[Path=/usr/share/fonts/truetype/cmu/,UprightFont=cmuntt.ttf,BoldFont=cmuntb.ttf,ItalicFont=cmunit.ttf,BoldItalicFont=cmuntx.ttf]{cmunti.ttf}\itshape Harvard}\setmainfont[Path=/usr/share/fonts/truetype/cmu/,UprightFont=cmunrm.ttf,BoldFont=cmunbx.ttf,ItalicFont=cmunti.ttf,BoldItalicFont=cmunbi.ttf]{cmunrm.ttf}\setmonofont[Path=/usr/share/fonts/truetype/cmu/,UprightFont=cmuntt.ttf,BoldFont=cmuntb.ttf,ItalicFont=cmunit.ttf,BoldItalicFont=cmuntx.ttf]{cmunrm.ttf}, for example, is a {\itshape \setmainfont[Path=/usr/share/fonts/truetype/cmu/,UprightFont=cmunrm.ttf,BoldFont=cmunbx.ttf,ItalicFont=cmunti.ttf,BoldItalicFont=cmunbi.ttf]{cmunti.ttf}\setmonofont[Path=/usr/share/fonts/truetype/cmu/,UprightFont=cmuntt.ttf,BoldFont=cmuntb.ttf,ItalicFont=cmunit.ttf,BoldItalicFont=cmuntx.ttf]{cmunti.ttf}\itshape citation}{$\text{ }$}\setmainfont[Path=/usr/share/fonts/truetype/cmu/,UprightFont=cmunrm.ttf,BoldFont=cmunbx.ttf,ItalicFont=cmunti.ttf,BoldItalicFont=cmunbi.ttf]{cmunrm.ttf}\setmonofont[Path=/usr/share/fonts/truetype/cmu/,UprightFont=cmuntt.ttf,BoldFont=cmuntb.ttf,ItalicFont=cmunit.ttf,BoldItalicFont=cmuntx.ttf]{cmunrm.ttf} style.  It is associated with an alphabetical reference list secondarily ordered on date, but the only strictly defined element of {\itshape \setmainfont[Path=/usr/share/fonts/truetype/cmu/,UprightFont=cmunrm.ttf,BoldFont=cmunbx.ttf,ItalicFont=cmunti.ttf,BoldItalicFont=cmunbi.ttf]{cmunti.ttf}\setmonofont[Path=/usr/share/fonts/truetype/cmu/,UprightFont=cmuntt.ttf,BoldFont=cmuntb.ttf,ItalicFont=cmunit.ttf,BoldItalicFont=cmuntx.ttf]{cmunti.ttf}\itshape Harvard}{$\text{ }$}\setmainfont[Path=/usr/share/fonts/truetype/cmu/,UprightFont=cmunrm.ttf,BoldFont=cmunbx.ttf,ItalicFont=cmunti.ttf,BoldItalicFont=cmunbi.ttf]{cmunrm.ttf}\setmonofont[Path=/usr/share/fonts/truetype/cmu/,UprightFont=cmuntt.ttf,BoldFont=cmuntb.ttf,ItalicFont=cmunit.ttf,BoldItalicFont=cmuntx.ttf]{cmunrm.ttf} style is the citation in {\itshape \setmainfont[Path=/usr/share/fonts/truetype/cmu/,UprightFont=cmunrm.ttf,BoldFont=cmunbx.ttf,ItalicFont=cmunti.ttf,BoldItalicFont=cmunbi.ttf]{cmunti.ttf}\setmonofont[Path=/usr/share/fonts/truetype/cmu/,UprightFont=cmuntt.ttf,BoldFont=cmuntb.ttf,ItalicFont=cmunit.ttf,BoldItalicFont=cmuntx.ttf]{cmunti.ttf}\itshape author-{}date}{$\text{ }$}\setmainfont[Path=/usr/share/fonts/truetype/cmu/,UprightFont=cmunrm.ttf,BoldFont=cmunbx.ttf,ItalicFont=cmunti.ttf,BoldItalicFont=cmunbi.ttf]{cmunrm.ttf}\setmonofont[Path=/usr/share/fonts/truetype/cmu/,UprightFont=cmuntt.ttf,BoldFont=cmuntb.ttf,ItalicFont=cmunit.ttf,BoldItalicFont=cmuntx.ttf]{cmunrm.ttf} format.
\section{The example data}
\label{707}
The database used for my examples contains just the following

\begin{Shaded}
\begin{Highlighting}[]

\NormalTok{@article\{Erdos65,}\newline
	\NormalTok{title\ensuremath{\text{ }}=\ensuremath{\text{ }}\{Some\ensuremath{\text{ }}very\ensuremath{\text{ }}hard\ensuremath{\text{ }}sums\},}\newline
	\NormalTok{journal=\{Difficult\ensuremath{\text{ }}Maths\ensuremath{\text{ }}Today\},}\newline
	\NormalTok{author=\{Paul\ensuremath{\text{ }}Erd\textbackslash{}H\{o\}s\ensuremath{\text{ }}and\ensuremath{\text{ }}Arend\ensuremath{\text{ }}Heyting\ensuremath{\text{ }}and\ensuremath{\text{ }}Luitzen\ensuremath{\text{ }}Egbertus\ensuremath{\text{ }}Brouwer\},}\newline
	\NormalTok{year=\{1930\},}\newline
	\NormalTok{pages=\{30\}}\newline
\NormalTok{\}}\newline
\end{Highlighting}
\end{Shaded}

\section{The limits of BibTeX styles}
\label{708}
Using cite.sty and BibTeX makes it very easy to produce {\bfseries \setmainfont[Path=/usr/share/fonts/truetype/cmu/,UprightFont=cmunrm.ttf,BoldFont=cmunbx.ttf,ItalicFont=cmunti.ttf,BoldItalicFont=cmunbi.ttf]{cmunbx.ttf}\setmonofont[Path=/usr/share/fonts/truetype/cmu/,UprightFont=cmuntt.ttf,BoldFont=cmuntb.ttf,ItalicFont=cmunit.ttf,BoldItalicFont=cmuntx.ttf]{cmunbx.ttf}\bfseries some}{$\text{ }$}\setmainfont[Path=/usr/share/fonts/truetype/cmu/,UprightFont=cmunrm.ttf,BoldFont=cmunbx.ttf,ItalicFont=cmunti.ttf,BoldItalicFont=cmunbi.ttf]{cmunrm.ttf}\setmonofont[Path=/usr/share/fonts/truetype/cmu/,UprightFont=cmuntt.ttf,BoldFont=cmuntb.ttf,ItalicFont=cmunit.ttf,BoldItalicFont=cmuntx.ttf]{cmunrm.ttf} bibliography styles.  But {\itshape \setmainfont[Path=/usr/share/fonts/truetype/cmu/,UprightFont=cmunrm.ttf,BoldFont=cmunbx.ttf,ItalicFont=cmunti.ttf,BoldItalicFont=cmunbi.ttf]{cmunti.ttf}\setmonofont[Path=/usr/share/fonts/truetype/cmu/,UprightFont=cmuntt.ttf,BoldFont=cmuntb.ttf,ItalicFont=cmunit.ttf,BoldItalicFont=cmuntx.ttf]{cmunti.ttf}\itshape author-{}date}{$\text{ }$}\setmainfont[Path=/usr/share/fonts/truetype/cmu/,UprightFont=cmunrm.ttf,BoldFont=cmunbx.ttf,ItalicFont=cmunti.ttf,BoldItalicFont=cmunbi.ttf]{cmunrm.ttf}\setmonofont[Path=/usr/share/fonts/truetype/cmu/,UprightFont=cmuntt.ttf,BoldFont=cmuntb.ttf,ItalicFont=cmunit.ttf,BoldItalicFont=cmuntx.ttf]{cmunrm.ttf} styles -{} for example the often mentioned, never defined \symbol{34}Harvard\symbol{34} -{} are not so easy.  It\textquotesingle{}s true that you can download some {\itshape \setmainfont[Path=/usr/share/fonts/truetype/cmu/,UprightFont=cmunrm.ttf,BoldFont=cmunbx.ttf,ItalicFont=cmunti.ttf,BoldItalicFont=cmunbi.ttf]{cmunti.ttf}\setmonofont[Path=/usr/share/fonts/truetype/cmu/,UprightFont=cmuntt.ttf,BoldFont=cmuntb.ttf,ItalicFont=cmunit.ttf,BoldItalicFont=cmuntx.ttf]{cmunti.ttf}\itshape .bst}{$\text{ }$}\setmainfont[Path=/usr/share/fonts/truetype/cmu/,UprightFont=cmunrm.ttf,BoldFont=cmunbx.ttf,ItalicFont=cmunti.ttf,BoldItalicFont=cmunbi.ttf]{cmunrm.ttf}\setmonofont[Path=/usr/share/fonts/truetype/cmu/,UprightFont=cmuntt.ttf,BoldFont=cmuntb.ttf,ItalicFont=cmunit.ttf,BoldItalicFont=cmuntx.ttf]{cmunrm.ttf} files from CTAN that will handle some variants but using them is not always straightforward.
This guide deals with {\itshape \setmainfont[Path=/usr/share/fonts/truetype/cmu/,UprightFont=cmunrm.ttf,BoldFont=cmunbx.ttf,ItalicFont=cmunti.ttf,BoldItalicFont=cmunbi.ttf]{cmunti.ttf}\setmonofont[Path=/usr/share/fonts/truetype/cmu/,UprightFont=cmuntt.ttf,BoldFont=cmuntb.ttf,ItalicFont=cmunit.ttf,BoldItalicFont=cmuntx.ttf]{cmunti.ttf}\itshape Natbib}{$\text{ }$}\setmainfont[Path=/usr/share/fonts/truetype/cmu/,UprightFont=cmunrm.ttf,BoldFont=cmunbx.ttf,ItalicFont=cmunti.ttf,BoldItalicFont=cmunbi.ttf]{cmunrm.ttf}\setmonofont[Path=/usr/share/fonts/truetype/cmu/,UprightFont=cmuntt.ttf,BoldFont=cmuntb.ttf,ItalicFont=cmunit.ttf,BoldItalicFont=cmuntx.ttf]{cmunrm.ttf} a supplementary package that can access {\itshape \setmainfont[Path=/usr/share/fonts/truetype/cmu/,UprightFont=cmunrm.ttf,BoldFont=cmunbx.ttf,ItalicFont=cmunti.ttf,BoldItalicFont=cmunbi.ttf]{cmunti.ttf}\setmonofont[Path=/usr/share/fonts/truetype/cmu/,UprightFont=cmuntt.ttf,BoldFont=cmuntb.ttf,ItalicFont=cmunit.ttf,BoldItalicFont=cmuntx.ttf]{cmunti.ttf}\itshape .bib}{$\text{ }$}\setmainfont[Path=/usr/share/fonts/truetype/cmu/,UprightFont=cmunrm.ttf,BoldFont=cmunbx.ttf,ItalicFont=cmunti.ttf,BoldItalicFont=cmunbi.ttf]{cmunrm.ttf}\setmonofont[Path=/usr/share/fonts/truetype/cmu/,UprightFont=cmuntt.ttf,BoldFont=cmuntb.ttf,ItalicFont=cmunit.ttf,BoldItalicFont=cmuntx.ttf]{cmunrm.ttf} files and has sophisticated functionality for producing custom or default author-{}year format citations and bibliographies as well as the numerical styles handled by BibTeX.
\section{Natbib}
\label{709}
Natbib is a package created by Patrick Daly as a replacement for the {\itshape \setmainfont[Path=/usr/share/fonts/truetype/cmu/,UprightFont=cmunrm.ttf,BoldFont=cmunbx.ttf,ItalicFont=cmunti.ttf,BoldItalicFont=cmunbi.ttf]{cmunti.ttf}\setmonofont[Path=/usr/share/fonts/truetype/cmu/,UprightFont=cmuntt.ttf,BoldFont=cmuntb.ttf,ItalicFont=cmunit.ttf,BoldItalicFont=cmuntx.ttf]{cmunti.ttf}\itshape cite.sty}{$\text{ }$}\setmainfont[Path=/usr/share/fonts/truetype/cmu/,UprightFont=cmunrm.ttf,BoldFont=cmunbx.ttf,ItalicFont=cmunti.ttf,BoldItalicFont=cmunbi.ttf]{cmunrm.ttf}\setmonofont[Path=/usr/share/fonts/truetype/cmu/,UprightFont=cmuntt.ttf,BoldFont=cmuntb.ttf,ItalicFont=cmunit.ttf,BoldItalicFont=cmuntx.ttf]{cmunrm.ttf} package when {\itshape \setmainfont[Path=/usr/share/fonts/truetype/cmu/,UprightFont=cmunrm.ttf,BoldFont=cmunbx.ttf,ItalicFont=cmunti.ttf,BoldItalicFont=cmunbi.ttf]{cmunti.ttf}\setmonofont[Path=/usr/share/fonts/truetype/cmu/,UprightFont=cmuntt.ttf,BoldFont=cmuntb.ttf,ItalicFont=cmunit.ttf,BoldItalicFont=cmuntx.ttf]{cmunti.ttf}\itshape author-{}date}{$\text{ }$}\setmainfont[Path=/usr/share/fonts/truetype/cmu/,UprightFont=cmunrm.ttf,BoldFont=cmunbx.ttf,ItalicFont=cmunti.ttf,BoldItalicFont=cmunbi.ttf]{cmunrm.ttf}\setmonofont[Path=/usr/share/fonts/truetype/cmu/,UprightFont=cmuntt.ttf,BoldFont=cmuntb.ttf,ItalicFont=cmunit.ttf,BoldItalicFont=cmuntx.ttf]{cmunrm.ttf} citation styles are required.  Natbib provides three associated bibliography styles:
\begin{myitemize}
\item{}  plainnat 
\item{}  abbrvnat
\item{}  unsrtnat
\end{myitemize}

which correspond to the three styles available by default in BibTeX where you have a plain numbered style, an abbreviated numbered style and an unsorted numbered style.
Alongside these new styles is an extended set of citation commands to provide flexible citation formats.  These are 
\begin{Shaded}
\begin{Highlighting}[]

\NormalTok{\textbackslash{}citet[]\{\}}\newline
\end{Highlighting}
\end{Shaded}
 and 
\begin{Shaded}
\begin{Highlighting}[]

\NormalTok{\textbackslash{}citep[]\{\}}\newline
\end{Highlighting}
\end{Shaded}
 each of which has a number of variants.\subsection{The Preamble}
\label{710}
All Natbib styles require that you load the package in your document preamble.  So, a skeleton LaTeX file with Natbib might look like this:

\begin{Shaded}
\begin{Highlighting}[]

\NormalTok{\textbackslash{}documentclass[]\{article\}}\newline
\NormalTok{\textbackslash{}usepackage[round]\{natbib\}}\newline
\ensuremath{\text{ }}\newline
\NormalTok{\textbackslash{}begin\{document\}}\newline
\ensuremath{\text{ }}\newline
\NormalTok{Document\ensuremath{\text{ }}body\ensuremath{\text{ }}text\ensuremath{\text{ }}with\ensuremath{\text{ }}citations.}\newline
\ensuremath{\text{ }}\newline
\NormalTok{\textbackslash{}bibliographystyle\{plainnat\}}\newline
\NormalTok{\textbackslash{}bibliography\{myrefs\}}\newline
\ensuremath{\text{ }}\newline
\NormalTok{\textbackslash{}end\{document\}}\newline
\end{Highlighting}
\end{Shaded}
\subsubsection{Options}
\label{711}
Options available with Natbib can be specified in the brackets on the \textbackslash{}usepackage command.  Among them are:
\begin{longtable}{>{\RaggedRight}p{0.16991\linewidth}>{\RaggedRight}p{0.77295\linewidth}} 
{\bfseries \hspace*{0pt}\ignorespaces{}\hspace*{0pt}Option }&{\bfseries \hspace*{0pt}\ignorespaces{}\hspace*{0pt}Effect}\endhead  \hspace*{0pt}\ignorespaces{}\hspace*{0pt} round &\hspace*{0pt}\ignorespaces{}\hspace*{0pt} ()\\ \hspace*{0pt}\ignorespaces{}\hspace*{0pt} square &\hspace*{0pt}\ignorespaces{}\hspace*{0pt} {$\text{[}$}{$\text{]}$}\\ \hspace*{0pt}\ignorespaces{}\hspace*{0pt} curly &\hspace*{0pt}\ignorespaces{}\hspace*{0pt} \{\}\\ \hspace*{0pt}\ignorespaces{}\hspace*{0pt} angle&\hspace*{0pt}\ignorespaces{}\hspace*{0pt} <{}>{}\\ \hspace*{0pt}\ignorespaces{}\hspace*{0pt} semicolon&\hspace*{0pt}\ignorespaces{}\hspace*{0pt} separate citations with {\bfseries {\itshape \setmainfont[Path=/usr/share/fonts/truetype/cmu/,UprightFont=cmunrm.ttf,BoldFont=cmunbx.ttf,ItalicFont=cmunti.ttf,BoldItalicFont=cmunbi.ttf]{cmunbi.ttf}\setmonofont[Path=/usr/share/fonts/truetype/cmu/,UprightFont=cmuntt.ttf,BoldFont=cmuntb.ttf,ItalicFont=cmunit.ttf,BoldItalicFont=cmuntx.ttf]{cmunbi.ttf}\bfseries \itshape ;}}{\itshape }\\ \hspace*{0pt}\ignorespaces{}\hspace*{0pt}{$\text{ }$}\setmainfont[Path=/usr/share/fonts/truetype/cmu/,UprightFont=cmunrm.ttf,BoldFont=cmunbx.ttf,ItalicFont=cmunti.ttf,BoldItalicFont=cmunbi.ttf]{cmunrm.ttf}\setmonofont[Path=/usr/share/fonts/truetype/cmu/,UprightFont=cmuntt.ttf,BoldFont=cmuntb.ttf,ItalicFont=cmunit.ttf,BoldItalicFont=cmuntx.ttf]{cmunrm.ttf} colon &\hspace*{0pt}\ignorespaces{}\hspace*{0pt} as semicolon\\ \hspace*{0pt}\ignorespaces{}\hspace*{0pt} comma &\hspace*{0pt}\ignorespaces{}\hspace*{0pt} separate with commas\\ \hspace*{0pt}\ignorespaces{}\hspace*{0pt} authoryear &\hspace*{0pt}\ignorespaces{}\hspace*{0pt} author-{}year citations\\ \hspace*{0pt}\ignorespaces{}\hspace*{0pt} numbers &\hspace*{0pt}\ignorespaces{}\hspace*{0pt} numeric citations\\ \hspace*{0pt}\ignorespaces{}\hspace*{0pt} super&\hspace*{0pt}\ignorespaces{}\hspace*{0pt} superscript citations\\ \hspace*{0pt}\ignorespaces{}\hspace*{0pt} sort &\hspace*{0pt}\ignorespaces{}\hspace*{0pt} multiple citations are ordered as in bibliography\\ \hspace*{0pt}\ignorespaces{}\hspace*{0pt}sort\&compress&\hspace*{0pt}\ignorespaces{}\hspace*{0pt}as {\bfseries {\itshape \setmainfont[Path=/usr/share/fonts/truetype/cmu/,UprightFont=cmunrm.ttf,BoldFont=cmunbx.ttf,ItalicFont=cmunti.ttf,BoldItalicFont=cmunbi.ttf]{cmunbi.ttf}\setmonofont[Path=/usr/share/fonts/truetype/cmu/,UprightFont=cmuntt.ttf,BoldFont=cmuntb.ttf,ItalicFont=cmunit.ttf,BoldItalicFont=cmuntx.ttf]{cmunbi.ttf}\bfseries \itshape sort}}{\itshape }{$\text{ }$}\setmainfont[Path=/usr/share/fonts/truetype/cmu/,UprightFont=cmunrm.ttf,BoldFont=cmunbx.ttf,ItalicFont=cmunti.ttf,BoldItalicFont=cmunbi.ttf]{cmunrm.ttf}\setmonofont[Path=/usr/share/fonts/truetype/cmu/,UprightFont=cmuntt.ttf,BoldFont=cmuntb.ttf,ItalicFont=cmunit.ttf,BoldItalicFont=cmuntx.ttf]{cmunrm.ttf} but number ranges are compressed and hyphenated\\ \hspace*{0pt}\ignorespaces{}\hspace*{0pt}compress&\hspace*{0pt}\ignorespaces{}\hspace*{0pt}number ranges are compressed and hyphenated but only where the \textquotesingle{}natural\textquotesingle{} sort produces a continuous range\\ \hspace*{0pt}\ignorespaces{}\hspace*{0pt}longnamesfirst&\hspace*{0pt}\ignorespaces{}\hspace*{0pt}first citation is full author list and subsequent citations are abbreviated\\ \hspace*{0pt}\ignorespaces{}\hspace*{0pt}sectionbib&\hspace*{0pt}\ignorespaces{}\hspace*{0pt}allows multiple bibliographies in the same document\\ \hspace*{0pt}\ignorespaces{}\hspace*{0pt}nonamebreak&\hspace*{0pt}\ignorespaces{}\hspace*{0pt}forces all author names onto one line\\ \hspace*{0pt}\ignorespaces{}\hspace*{0pt}merge&\hspace*{0pt}\ignorespaces{}\hspace*{0pt}merges a citation with a previous citation\\ \hspace*{0pt}\ignorespaces{}\hspace*{0pt}elide&\hspace*{0pt}\ignorespaces{}\hspace*{0pt}elides any repeated elements in merged references\\ \hspace*{0pt}\ignorespaces{}\hspace*{0pt}mcite&\hspace*{0pt}\ignorespaces{}\hspace*{0pt}ignore merge 
\end{longtable}

Clearly some of these options require explanation but that will be achieved via examples below.  For now, we just note that they can be passed through {\itshape \setmainfont[Path=/usr/share/fonts/truetype/cmu/,UprightFont=cmunrm.ttf,BoldFont=cmunbx.ttf,ItalicFont=cmunti.ttf,BoldItalicFont=cmunbi.ttf]{cmunti.ttf}\setmonofont[Path=/usr/share/fonts/truetype/cmu/,UprightFont=cmuntt.ttf,BoldFont=cmuntb.ttf,ItalicFont=cmunit.ttf,BoldItalicFont=cmuntx.ttf]{cmunti.ttf}\itshape \textbackslash{}usepackage{$\text{[}$}{$\text{]}$}\{\}}{$\text{ }$}\setmainfont[Path=/usr/share/fonts/truetype/cmu/,UprightFont=cmunrm.ttf,BoldFont=cmunbx.ttf,ItalicFont=cmunti.ttf,BoldItalicFont=cmunbi.ttf]{cmunrm.ttf}\setmonofont[Path=/usr/share/fonts/truetype/cmu/,UprightFont=cmuntt.ttf,BoldFont=cmuntb.ttf,ItalicFont=cmunit.ttf,BoldItalicFont=cmuntx.ttf]{cmunrm.ttf} in the preamble of your LaTeX file.
\section{Citation}
\label{712}\subsection{Basic Citation Commands}
\label{713}

To cite with Natbib, use the commands {\itshape \setmainfont[Path=/usr/share/fonts/truetype/cmu/,UprightFont=cmunrm.ttf,BoldFont=cmunbx.ttf,ItalicFont=cmunti.ttf,BoldItalicFont=cmunbi.ttf]{cmunti.ttf}\setmonofont[Path=/usr/share/fonts/truetype/cmu/,UprightFont=cmuntt.ttf,BoldFont=cmuntb.ttf,ItalicFont=cmunit.ttf,BoldItalicFont=cmuntx.ttf]{cmunti.ttf}\itshape \textbackslash{}citet}{$\text{ }$}\setmainfont[Path=/usr/share/fonts/truetype/cmu/,UprightFont=cmunrm.ttf,BoldFont=cmunbx.ttf,ItalicFont=cmunti.ttf,BoldItalicFont=cmunbi.ttf]{cmunrm.ttf}\setmonofont[Path=/usr/share/fonts/truetype/cmu/,UprightFont=cmuntt.ttf,BoldFont=cmuntb.ttf,ItalicFont=cmunit.ttf,BoldItalicFont=cmuntx.ttf]{cmunrm.ttf} or {\itshape \setmainfont[Path=/usr/share/fonts/truetype/cmu/,UprightFont=cmunrm.ttf,BoldFont=cmunbx.ttf,ItalicFont=cmunti.ttf,BoldItalicFont=cmunbi.ttf]{cmunti.ttf}\setmonofont[Path=/usr/share/fonts/truetype/cmu/,UprightFont=cmuntt.ttf,BoldFont=cmuntb.ttf,ItalicFont=cmunit.ttf,BoldItalicFont=cmuntx.ttf]{cmunti.ttf}\itshape \textbackslash{}citep}{$\text{ }$}\setmainfont[Path=/usr/share/fonts/truetype/cmu/,UprightFont=cmunrm.ttf,BoldFont=cmunbx.ttf,ItalicFont=cmunti.ttf,BoldItalicFont=cmunbi.ttf]{cmunrm.ttf}\setmonofont[Path=/usr/share/fonts/truetype/cmu/,UprightFont=cmuntt.ttf,BoldFont=cmuntb.ttf,ItalicFont=cmunit.ttf,BoldItalicFont=cmuntx.ttf]{cmunrm.ttf} in your document.  The \symbol{34}plain\symbol{34} versions of these commands produced abbreviated lists in the case of multiple authors but both have {\itshape \setmainfont[Path=/usr/share/fonts/truetype/cmu/,UprightFont=cmunrm.ttf,BoldFont=cmunbx.ttf,ItalicFont=cmunti.ttf,BoldItalicFont=cmunbi.ttf]{cmunti.ttf}\setmonofont[Path=/usr/share/fonts/truetype/cmu/,UprightFont=cmuntt.ttf,BoldFont=cmuntb.ttf,ItalicFont=cmunit.ttf,BoldItalicFont=cmuntx.ttf]{cmunti.ttf}\itshape *}{$\text{ }$}\setmainfont[Path=/usr/share/fonts/truetype/cmu/,UprightFont=cmunrm.ttf,BoldFont=cmunbx.ttf,ItalicFont=cmunti.ttf,BoldItalicFont=cmunbi.ttf]{cmunrm.ttf}\setmonofont[Path=/usr/share/fonts/truetype/cmu/,UprightFont=cmuntt.ttf,BoldFont=cmuntb.ttf,ItalicFont=cmunit.ttf,BoldItalicFont=cmuntx.ttf]{cmunrm.ttf} variants which result in full author listings.  We assume the use of the {\itshape \setmainfont[Path=/usr/share/fonts/truetype/cmu/,UprightFont=cmunrm.ttf,BoldFont=cmunbx.ttf,ItalicFont=cmunti.ttf,BoldItalicFont=cmunbi.ttf]{cmunti.ttf}\setmonofont[Path=/usr/share/fonts/truetype/cmu/,UprightFont=cmuntt.ttf,BoldFont=cmuntb.ttf,ItalicFont=cmunit.ttf,BoldItalicFont=cmuntx.ttf]{cmunti.ttf}\itshape round}{$\text{ }$}\setmainfont[Path=/usr/share/fonts/truetype/cmu/,UprightFont=cmunrm.ttf,BoldFont=cmunbx.ttf,ItalicFont=cmunti.ttf,BoldItalicFont=cmunbi.ttf]{cmunrm.ttf}\setmonofont[Path=/usr/share/fonts/truetype/cmu/,UprightFont=cmuntt.ttf,BoldFont=cmuntb.ttf,ItalicFont=cmunit.ttf,BoldItalicFont=cmuntx.ttf]{cmunrm.ttf} option in these examples.\subsubsection{\textbackslash{}citet and \textbackslash{}citet*}
\label{714}
The {\itshape \setmainfont[Path=/usr/share/fonts/truetype/cmu/,UprightFont=cmunrm.ttf,BoldFont=cmunbx.ttf,ItalicFont=cmunti.ttf,BoldItalicFont=cmunbi.ttf]{cmunti.ttf}\setmonofont[Path=/usr/share/fonts/truetype/cmu/,UprightFont=cmuntt.ttf,BoldFont=cmuntb.ttf,ItalicFont=cmunit.ttf,BoldItalicFont=cmuntx.ttf]{cmunti.ttf}\itshape \textbackslash{}citet}{$\text{ }$}\setmainfont[Path=/usr/share/fonts/truetype/cmu/,UprightFont=cmunrm.ttf,BoldFont=cmunbx.ttf,ItalicFont=cmunti.ttf,BoldItalicFont=cmunbi.ttf]{cmunrm.ttf}\setmonofont[Path=/usr/share/fonts/truetype/cmu/,UprightFont=cmuntt.ttf,BoldFont=cmuntb.ttf,ItalicFont=cmunit.ttf,BoldItalicFont=cmuntx.ttf]{cmunrm.ttf} command is used for {\itshape \setmainfont[Path=/usr/share/fonts/truetype/cmu/,UprightFont=cmunrm.ttf,BoldFont=cmunbx.ttf,ItalicFont=cmunti.ttf,BoldItalicFont=cmunbi.ttf]{cmunti.ttf}\setmonofont[Path=/usr/share/fonts/truetype/cmu/,UprightFont=cmuntt.ttf,BoldFont=cmuntb.ttf,ItalicFont=cmunit.ttf,BoldItalicFont=cmuntx.ttf]{cmunti.ttf}\itshape textual}{$\text{ }$}\setmainfont[Path=/usr/share/fonts/truetype/cmu/,UprightFont=cmunrm.ttf,BoldFont=cmunbx.ttf,ItalicFont=cmunti.ttf,BoldItalicFont=cmunbi.ttf]{cmunrm.ttf}\setmonofont[Path=/usr/share/fonts/truetype/cmu/,UprightFont=cmuntt.ttf,BoldFont=cmuntb.ttf,ItalicFont=cmunit.ttf,BoldItalicFont=cmuntx.ttf]{cmunrm.ttf} citations, that is to say that author names appear in the text outside of the parenthetical reference to the date of publication.  This command can take options for chapter, page numbers etc. Here are examples

\begin{longtable}{>{\RaggedRight}p{0.37937\linewidth}>{\RaggedRight}p{0.13716\linewidth}>{\RaggedRight}p{0.39776\linewidth}} 
\hspace*{0pt}\ignorespaces{}\hspace*{0pt}\textbackslash{}citet\{Erdos65\}&\hspace*{0pt}\ignorespaces{}\hspace*{0pt}produces&\hspace*{0pt}\ignorespaces{}\hspace*{0pt}Erdős et al. (1965)\\ \hspace*{0pt}\ignorespaces{}\hspace*{0pt}\textbackslash{}citet{$\text{[}$}chapter 2{$\text{]}$}\{Erdos65\}&\hspace*{0pt}\ignorespaces{}\hspace*{0pt}produces&\hspace*{0pt}\ignorespaces{}\hspace*{0pt}Erdős et al. (1965, chapter 2)\\ \hspace*{0pt}\ignorespaces{}\hspace*{0pt}\textbackslash{}citet{$\text{[}$}pp. 10-{}12{$\text{]}$}\{Erdos65\}&\hspace*{0pt}\ignorespaces{}\hspace*{0pt}produces&\hspace*{0pt}\ignorespaces{}\hspace*{0pt}Erdős et al. (1965, pp. 10-{}12)\\ \hspace*{0pt}\ignorespaces{}\hspace*{0pt}\textbackslash{}citet{$\text{[}$}see{$\text{]}$}{$\text{[}$}chap. 2{$\text{]}$}\{Erdos65\}&\hspace*{0pt}\ignorespaces{}\hspace*{0pt}produces&\hspace*{0pt}\ignorespaces{}\hspace*{0pt}Erdős et al. (see 1965, chap. 2) 
\end{longtable}

Here are the {\itshape \setmainfont[Path=/usr/share/fonts/truetype/cmu/,UprightFont=cmunrm.ttf,BoldFont=cmunbx.ttf,ItalicFont=cmunti.ttf,BoldItalicFont=cmunbi.ttf]{cmunti.ttf}\setmonofont[Path=/usr/share/fonts/truetype/cmu/,UprightFont=cmuntt.ttf,BoldFont=cmuntb.ttf,ItalicFont=cmunit.ttf,BoldItalicFont=cmuntx.ttf]{cmunti.ttf}\itshape \textbackslash{}citet*}{$\text{ }$}\setmainfont[Path=/usr/share/fonts/truetype/cmu/,UprightFont=cmunrm.ttf,BoldFont=cmunbx.ttf,ItalicFont=cmunti.ttf,BoldItalicFont=cmunbi.ttf]{cmunrm.ttf}\setmonofont[Path=/usr/share/fonts/truetype/cmu/,UprightFont=cmuntt.ttf,BoldFont=cmuntb.ttf,ItalicFont=cmunit.ttf,BoldItalicFont=cmuntx.ttf]{cmunrm.ttf} versions

\begin{longtable}{>{\RaggedRight}p{0.32165\linewidth}>{\RaggedRight}p{0.11195\linewidth}>{\RaggedRight}p{0.48069\linewidth}} 
\hspace*{0pt}\ignorespaces{}\hspace*{0pt}\textbackslash{}citet*\{Erdos65\}&\hspace*{0pt}\ignorespaces{}\hspace*{0pt}produces&\hspace*{0pt}\ignorespaces{}\hspace*{0pt}Erdős, Heyting and Brouwer (1965)\\ \hspace*{0pt}\ignorespaces{}\hspace*{0pt}\textbackslash{}citet*{$\text{[}$}chapter 2{$\text{]}$}\{Erdos65\}&\hspace*{0pt}\ignorespaces{}\hspace*{0pt}produces&\hspace*{0pt}\ignorespaces{}\hspace*{0pt}Erdős , Heyting and Brouwer (1965, chapter 2)\\ \hspace*{0pt}\ignorespaces{}\hspace*{0pt}\textbackslash{}citet*{$\text{[}$}pp. 10-{}12{$\text{]}$}\{Erdos65\}&\hspace*{0pt}\ignorespaces{}\hspace*{0pt}produces&\hspace*{0pt}\ignorespaces{}\hspace*{0pt}Erdős , Heyting and Brouwer (1965, pp. 10-{}12)\\ \hspace*{0pt}\ignorespaces{}\hspace*{0pt}\textbackslash{}citet*{$\text{[}$}see{$\text{]}$}{$\text{[}$}chap. 2{$\text{]}$}\{Erdos65\}&\hspace*{0pt}\ignorespaces{}\hspace*{0pt}produces&\hspace*{0pt}\ignorespaces{}\hspace*{0pt}Erdős , Heyting and Brouwer (see 1965, chap. 2) 
\end{longtable}

\subsubsection{\textbackslash{}citep and \textbackslash{}citep*}
\label{715}
The {\itshape \setmainfont[Path=/usr/share/fonts/truetype/cmu/,UprightFont=cmunrm.ttf,BoldFont=cmunbx.ttf,ItalicFont=cmunti.ttf,BoldItalicFont=cmunbi.ttf]{cmunti.ttf}\setmonofont[Path=/usr/share/fonts/truetype/cmu/,UprightFont=cmuntt.ttf,BoldFont=cmuntb.ttf,ItalicFont=cmunit.ttf,BoldItalicFont=cmuntx.ttf]{cmunti.ttf}\itshape \textbackslash{}citep}{$\text{ }$}\setmainfont[Path=/usr/share/fonts/truetype/cmu/,UprightFont=cmunrm.ttf,BoldFont=cmunbx.ttf,ItalicFont=cmunti.ttf,BoldItalicFont=cmunbi.ttf]{cmunrm.ttf}\setmonofont[Path=/usr/share/fonts/truetype/cmu/,UprightFont=cmuntt.ttf,BoldFont=cmuntb.ttf,ItalicFont=cmunit.ttf,BoldItalicFont=cmuntx.ttf]{cmunrm.ttf} command is used where the author name is to appear inside the parentheses alongside the date.

\begin{longtable}{>{\RaggedRight}p{0.38105\linewidth}>{\RaggedRight}p{0.13549\linewidth}>{\RaggedRight}p{0.39775\linewidth}} 
\hspace*{0pt}\ignorespaces{}\hspace*{0pt}\textbackslash{}citep\{Erdos65\}&\hspace*{0pt}\ignorespaces{}\hspace*{0pt}produces&\hspace*{0pt}\ignorespaces{}\hspace*{0pt}(Erdős et al. 1965)\\ \hspace*{0pt}\ignorespaces{}\hspace*{0pt}\textbackslash{}citep{$\text{[}$}chapter 2{$\text{]}$}\{Erdos65\}&\hspace*{0pt}\ignorespaces{}\hspace*{0pt}produces&\hspace*{0pt}\ignorespaces{}\hspace*{0pt}(Erdős et al. 1965, chapter 2)\\ \hspace*{0pt}\ignorespaces{}\hspace*{0pt}\textbackslash{}citep{$\text{[}$}pp. 10-{}12{$\text{]}$}\{Erdos65\}&\hspace*{0pt}\ignorespaces{}\hspace*{0pt}produces&\hspace*{0pt}\ignorespaces{}\hspace*{0pt}(Erdős et al. 1965, pp. 10-{}12)\\ \hspace*{0pt}\ignorespaces{}\hspace*{0pt}\textbackslash{}citep{$\text{[}$}see{$\text{]}$}{$\text{[}$}chap. 2{$\text{]}$}\{Erdos65\}&\hspace*{0pt}\ignorespaces{}\hspace*{0pt}produces&\hspace*{0pt}\ignorespaces{}\hspace*{0pt}(see Erdős et al., 1965, chap. 2)\\ \hspace*{0pt}\ignorespaces{}\hspace*{0pt}\textbackslash{}citep{$\text{[}$}e.g.{$\text{]}$}{$\text{[}$}{$\text{]}$}\{Erdos65\}&\hspace*{0pt}\ignorespaces{}\hspace*{0pt}produces&\hspace*{0pt}\ignorespaces{}\hspace*{0pt}(e.g. Erdős et al., 1965) 
\end{longtable}

Here are the {\itshape \setmainfont[Path=/usr/share/fonts/truetype/cmu/,UprightFont=cmunrm.ttf,BoldFont=cmunbx.ttf,ItalicFont=cmunti.ttf,BoldItalicFont=cmunbi.ttf]{cmunti.ttf}\setmonofont[Path=/usr/share/fonts/truetype/cmu/,UprightFont=cmuntt.ttf,BoldFont=cmuntb.ttf,ItalicFont=cmunit.ttf,BoldItalicFont=cmuntx.ttf]{cmunti.ttf}\itshape \textbackslash{}citep*}{$\text{ }$}\setmainfont[Path=/usr/share/fonts/truetype/cmu/,UprightFont=cmunrm.ttf,BoldFont=cmunbx.ttf,ItalicFont=cmunti.ttf,BoldItalicFont=cmunbi.ttf]{cmunrm.ttf}\setmonofont[Path=/usr/share/fonts/truetype/cmu/,UprightFont=cmuntt.ttf,BoldFont=cmuntb.ttf,ItalicFont=cmunit.ttf,BoldItalicFont=cmuntx.ttf]{cmunrm.ttf} versions

\begin{longtable}{>{\RaggedRight}p{0.32686\linewidth}>{\RaggedRight}p{0.11195\linewidth}>{\RaggedRight}p{0.47547\linewidth}} 
\hspace*{0pt}\ignorespaces{}\hspace*{0pt}\textbackslash{}citep*\{Erdos65\}&\hspace*{0pt}\ignorespaces{}\hspace*{0pt}produces&\hspace*{0pt}\ignorespaces{}\hspace*{0pt}(Erdős, Heyting and Brouwer 1965)\\ \hspace*{0pt}\ignorespaces{}\hspace*{0pt}\textbackslash{}citep*{$\text{[}$}chapter 2{$\text{]}$}\{Erdos65\}&\hspace*{0pt}\ignorespaces{}\hspace*{0pt}produces&\hspace*{0pt}\ignorespaces{}\hspace*{0pt}(Erdős, Heyting and Brouwer 1965, chapter 2)\\ \hspace*{0pt}\ignorespaces{}\hspace*{0pt}\textbackslash{}citep*{$\text{[}$}pp. 10-{}12{$\text{]}$}\{Erdos65\}&\hspace*{0pt}\ignorespaces{}\hspace*{0pt}produces&\hspace*{0pt}\ignorespaces{}\hspace*{0pt}(Erdős , Heyting and Brouwer 1965, pp. 10-{}12)\\ \hspace*{0pt}\ignorespaces{}\hspace*{0pt}\textbackslash{}citep*{$\text{[}$}see{$\text{]}$}{$\text{[}$}chap. 2{$\text{]}$}\{Erdos65\}&\hspace*{0pt}\ignorespaces{}\hspace*{0pt}produces&\hspace*{0pt}\ignorespaces{}\hspace*{0pt}(see Erdős , Heyting and Brouwer, 1965, chap. 2)\\ \hspace*{0pt}\ignorespaces{}\hspace*{0pt}\textbackslash{}citep*{$\text{[}$}e.g.{$\text{]}$}{$\text{[}$}{$\text{]}$}\{Erdos65\}&\hspace*{0pt}\ignorespaces{}\hspace*{0pt}produces&\hspace*{0pt}\ignorespaces{}\hspace*{0pt}(e.g. Erdős , Heyting and Brouwer, 1965) 
\end{longtable}

\subsection{The Reference List}
\label{716}
Having dealt with basic varieties of citation, we turn to the creation of the bibliography or reference list.
Inserting a correct and correctly formatted bibliography when using Natbib is no different than when using plain BibTeX.  There are two essential commands -{} 
\begin{Shaded}
\begin{Highlighting}[]

\NormalTok{\textbackslash{}bibliography\{mybibliographydatabase\}}\newline
\end{Highlighting}
\end{Shaded}
 which LaTeX interprets as an instruction to read a bibliographic database file (eg myrefs.bib) and insert the relevant data here, and 
\begin{Shaded}
\begin{Highlighting}[]

\NormalTok{\textbackslash{}bibliographystyle\{plainnat\}}\newline
\end{Highlighting}
\end{Shaded}
 which specifies how the data are to be presented.
Above the three basic Natbib styles were mentioned as analogues of the partially homonymous styles in BibTeX.  Let us imagine documents bearing citations as in \mylref{712}{the section about citation above}.  Here is, approximately, how these citations would appear in plainnat.


\begin{minipage}{1.0\linewidth}
\begin{center}
\includegraphics[width=1.0\linewidth,height=6.5in,keepaspectratio]{../images/154.png}
\end{center}
\raggedright{}\myfigurewithoutcaption{154}
\end{minipage}\vspace{0.75cm}


\subsection{What more is there?}
\label{717}
This covers the basic functionality provided by the package Natbib.  It may not, of course, provide what you are looking for.  If you don\textquotesingle{}t find what you want here then you should probably next investigate {\itshape \setmainfont[Path=/usr/share/fonts/truetype/cmu/,UprightFont=cmunrm.ttf,BoldFont=cmunbx.ttf,ItalicFont=cmunti.ttf,BoldItalicFont=cmunbi.ttf]{cmunti.ttf}\setmonofont[Path=/usr/share/fonts/truetype/cmu/,UprightFont=cmuntt.ttf,BoldFont=cmuntb.ttf,ItalicFont=cmunit.ttf,BoldItalicFont=cmuntx.ttf]{cmunti.ttf}\itshape harvard.sty}{$\text{ }$}\setmainfont[Path=/usr/share/fonts/truetype/cmu/,UprightFont=cmunrm.ttf,BoldFont=cmunbx.ttf,ItalicFont=cmunti.ttf,BoldItalicFont=cmunbi.ttf]{cmunrm.ttf}\setmonofont[Path=/usr/share/fonts/truetype/cmu/,UprightFont=cmuntt.ttf,BoldFont=cmuntb.ttf,ItalicFont=cmunit.ttf,BoldItalicFont=cmuntx.ttf]{cmunrm.ttf} which provides a slighly different set of author-{}date citation functions.  Providing a gentle guide to {\itshape \setmainfont[Path=/usr/share/fonts/truetype/cmu/,UprightFont=cmunrm.ttf,BoldFont=cmunbx.ttf,ItalicFont=cmunti.ttf,BoldItalicFont=cmunbi.ttf]{cmunti.ttf}\setmonofont[Path=/usr/share/fonts/truetype/cmu/,UprightFont=cmuntt.ttf,BoldFont=cmuntb.ttf,ItalicFont=cmunit.ttf,BoldItalicFont=cmuntx.ttf]{cmunti.ttf}\itshape harvard.sty}{$\text{ }$}\setmainfont[Path=/usr/share/fonts/truetype/cmu/,UprightFont=cmunrm.ttf,BoldFont=cmunbx.ttf,ItalicFont=cmunti.ttf,BoldItalicFont=cmunbi.ttf]{cmunrm.ttf}\setmonofont[Path=/usr/share/fonts/truetype/cmu/,UprightFont=cmuntt.ttf,BoldFont=cmuntb.ttf,ItalicFont=cmunit.ttf,BoldItalicFont=cmuntx.ttf]{cmunrm.ttf} is my next rainy day project.




\LaTeXNullTemplate{}
\mypart{Special Documents}\chapter{Letters}

\myminitoc
\label{718}

\label{719}


Sometimes the mundane things are the most painful. However, it doesn\textquotesingle{}t have to be that way because of evolved, user-{}friendly templates. Thankfully, LaTeX allows for very quick letter writing, with little hassle.
\section{The letter class}
\label{720}
To write letters use the standard document class {\itshape \setmainfont[Path=/usr/share/fonts/truetype/cmu/,UprightFont=cmunrm.ttf,BoldFont=cmunbx.ttf,ItalicFont=cmunti.ttf,BoldItalicFont=cmunbi.ttf]{cmunti.ttf}\setmonofont[Path=/usr/share/fonts/truetype/cmu/,UprightFont=cmuntt.ttf,BoldFont=cmuntb.ttf,ItalicFont=cmunit.ttf,BoldItalicFont=cmuntx.ttf]{cmunti.ttf}\itshape letter}\setmainfont[Path=/usr/share/fonts/truetype/cmu/,UprightFont=cmunrm.ttf,BoldFont=cmunbx.ttf,ItalicFont=cmunti.ttf,BoldItalicFont=cmunbi.ttf]{cmunrm.ttf}\setmonofont[Path=/usr/share/fonts/truetype/cmu/,UprightFont=cmuntt.ttf,BoldFont=cmuntb.ttf,ItalicFont=cmunit.ttf,BoldItalicFont=cmuntx.ttf]{cmunrm.ttf}.

You can write multiple letters in one LaTeX file -{} start each one with \LaTeXTT{\textbackslash{}begin\{letter\}\{\textquotesingle{}\textquotesingle{}recipient\textquotesingle{}\textquotesingle{}\}} and end with \LaTeXTT{\textbackslash{}end\{letter\}}. You can leave {\itshape \setmainfont[Path=/usr/share/fonts/truetype/cmu/,UprightFont=cmunrm.ttf,BoldFont=cmunbx.ttf,ItalicFont=cmunti.ttf,BoldItalicFont=cmunbi.ttf]{cmunti.ttf}\setmonofont[Path=/usr/share/fonts/truetype/cmu/,UprightFont=cmuntt.ttf,BoldFont=cmuntb.ttf,ItalicFont=cmunit.ttf,BoldItalicFont=cmuntx.ttf]{cmunti.ttf}\itshape recipient}{$\text{ }$}\setmainfont[Path=/usr/share/fonts/truetype/cmu/,UprightFont=cmunrm.ttf,BoldFont=cmunbx.ttf,ItalicFont=cmunti.ttf,BoldItalicFont=cmunbi.ttf]{cmunrm.ttf}\setmonofont[Path=/usr/share/fonts/truetype/cmu/,UprightFont=cmuntt.ttf,BoldFont=cmuntb.ttf,ItalicFont=cmunit.ttf,BoldItalicFont=cmuntx.ttf]{cmunrm.ttf} blank. Each letter consists of four parts.
\begin{myenumerate}
\item{}  Opening (like \LaTeXTT{\textbackslash{}opening\{Dear Sir or Madam,\}} or \LaTeXTT{\textbackslash{}opening\{Dear Kate,\}}).
\item{}  Main body (written as usual in LaTeX). If you want the same body in all the letters, you may want to consider putting the entire body in a new command like \LaTeXTT{\textbackslash{}newcommand\{\textbackslash{}BODY\}\{actual body\}} and then using \LaTeXTT{\textbackslash{}BODY} in all the letters.
\item{}  Closing (like \LaTeXTT{\textbackslash{}closing\{Yours sincerely,\}}).
\begin{myquote}
\item{}  LaTeX will leave some space after closing for your hand-{}written signature; then it will put your name and surname, if you have declared them.
\end{myquote}

\item{}  Additional elements: post scripta, carbon copy and list of enclosures.
\end{myenumerate}


If you want your name, address and telephone number to appear in the letter, you have to declare them first signature, address and telephone.

The output letter will look like this:



\begin{minipage}{0.37500\textwidth}
\begin{center}
\includegraphics[width=1.0\textwidth,height=6.5in,keepaspectratio]{../images/155.png}
\end{center}
\raggedright{}\myfigurewithcaption{155}{A sample letter.}
\end{minipage}\vspace{0.75cm}



Here is the example\textquotesingle{}s code:


\begin{Shaded}
\begin{Highlighting}[]

\NormalTok{\textbackslash{}documentclass\{letter\}}\newline
\NormalTok{\textbackslash{}usepackage\{hyperref\}}\newline
\NormalTok{\textbackslash{}signature\{Joe\ensuremath{\text{ }}Bloggs\}}\newline
\NormalTok{\textbackslash{}address\{21\ensuremath{\text{ }}Bridge\ensuremath{\text{ }}Street\ensuremath{\text{ }}\textbackslash{}\textbackslash{}\ensuremath{\text{ }}Smallville\ensuremath{\text{ }}\textbackslash{}\textbackslash{}\ensuremath{\text{ }}Dunwich\ensuremath{\text{ }}DU3\ensuremath{\text{ }}4WE\}}\newline
\NormalTok{\textbackslash{}begin\{document\}}\newline
\ensuremath{\text{ }}\newline
\NormalTok{\textbackslash{}begin\{letter\}\{Director\ensuremath{\text{ }}\textbackslash{}\textbackslash{}\ensuremath{\text{ }}Doe\ensuremath{\text{ }}\textbackslash{}\&\ensuremath{\text{ }}Co\ensuremath{\text{ }}\textbackslash{}\textbackslash{}\ensuremath{\text{ }}35\ensuremath{\text{ }}Anthony\ensuremath{\text{ }}Road}\newline
\NormalTok{\textbackslash{}\textbackslash{}\ensuremath{\text{ }}Newport\ensuremath{\text{ }}\textbackslash{}\textbackslash{}\ensuremath{\text{ }}Ipswich\ensuremath{\text{ }}IP3\ensuremath{\text{ }}5RT\}}\newline
\NormalTok{\textbackslash{}opening\{Dear\ensuremath{\text{ }}Sir\ensuremath{\text{ }}or\ensuremath{\text{ }}Madam:\}}\newline
\ensuremath{\text{ }}\newline
\NormalTok{I\ensuremath{\text{ }}am\ensuremath{\text{ }}writing\ensuremath{\text{ }}to\ensuremath{\text{ }}you\ensuremath{\text{ }}on\ensuremath{\text{ }}behalf\ensuremath{\text{ }}of\ensuremath{\text{ }}the\ensuremath{\text{ }}Wikipedia\ensuremath{\text{ }}project}\newline
\ensuremath{\text{ }}\NormalTok{(http://www.wikipedia.org/),}\newline
\NormalTok{an\ensuremath{\text{ }}endeavour\ensuremath{\text{ }}to\ensuremath{\text{ }}build\ensuremath{\text{ }}a\ensuremath{\text{ }}fully-fledged\ensuremath{\text{ }}multilingual\ensuremath{\text{ }}encyclopaedia\ensuremath{\text{ }}in\ensuremath{\text{ }}an\ensuremath{\text{ }}entirely}\newline
\NormalTok{open\ensuremath{\text{ }}manner,\ensuremath{\text{ }}to\ensuremath{\text{ }}ask\ensuremath{\text{ }}for\ensuremath{\text{ }}permission\ensuremath{\text{ }}to\ensuremath{\text{ }}use\ensuremath{\text{ }}your\ensuremath{\text{ }}copyrighted\ensuremath{\text{ }}material.}\newline
\ensuremath{\text{ }}\newline
\CommentTok{\%\ensuremath{\text{ }}The\ensuremath{\text{ }}\textbackslash{}ldots\ensuremath{\text{ }}command\ensuremath{\text{ }}produces\ensuremath{\text{ }}dots\ensuremath{\text{ }}in\ensuremath{\text{ }}a\ensuremath{\text{ }}way\ensuremath{\text{ }}that\ensuremath{\text{ }}will\ensuremath{\text{ }}not\ensuremath{\text{ }}upset}\newline
\CommentTok{\%\ensuremath{\text{ }}the\ensuremath{\text{ }}typesetting\ensuremath{\text{ }}of\ensuremath{\text{ }}the\ensuremath{\text{ }}document.}\newline
\NormalTok{\textbackslash{}ldots\ensuremath{\text{ }}}\newline
\ensuremath{\text{ }}\newline
\NormalTok{That\ensuremath{\text{ }}said,\ensuremath{\text{ }}allow\ensuremath{\text{ }}me\ensuremath{\text{ }}to\ensuremath{\text{ }}reiterate\ensuremath{\text{ }}that\ensuremath{\text{ }}your\ensuremath{\text{ }}material\ensuremath{\text{ }}will\ensuremath{\text{ }}be\ensuremath{\text{ }}used\ensuremath{\text{ }}to\ensuremath{\text{ }}the\ensuremath{\text{ }}noble}\newline
\ensuremath{\text{ }}\NormalTok{end\ensuremath{\text{ }}of}\newline
\NormalTok{providing\ensuremath{\text{ }}a\ensuremath{\text{ }}free\ensuremath{\text{ }}collection\ensuremath{\text{ }}of\ensuremath{\text{ }}knowledge\ensuremath{\text{ }}for\ensuremath{\text{ }}everyone;\ensuremath{\text{ }}naturally\ensuremath{\text{ }}enough,\ensuremath{\text{ }}only\ensuremath{\text{ }}if}\newline
\ensuremath{\text{ }}\NormalTok{you}\newline
\NormalTok{agree.\ensuremath{\text{ }}If\ensuremath{\text{ }}that\ensuremath{\text{ }}is\ensuremath{\text{ }}the\ensuremath{\text{ }}case,\ensuremath{\text{ }}could\ensuremath{\text{ }}you\ensuremath{\text{ }}kindly\ensuremath{\text{ }}fill\ensuremath{\text{ }}in\ensuremath{\text{ }}the\ensuremath{\text{ }}attached\ensuremath{\text{ }}form\ensuremath{\text{ }}and\ensuremath{\text{ }}post}\newline
\ensuremath{\text{ }}\NormalTok{it}\newline
\NormalTok{back\ensuremath{\text{ }}to\ensuremath{\text{ }}me?\ensuremath{\text{ }}We\ensuremath{\text{ }}shall\ensuremath{\text{ }}greatly\ensuremath{\text{ }}appreciate\ensuremath{\text{ }}it.}\newline
\ensuremath{\text{ }}\newline
\NormalTok{Thank\ensuremath{\text{ }}you\ensuremath{\text{ }}for\ensuremath{\text{ }}your\ensuremath{\text{ }}time\ensuremath{\text{ }}and\ensuremath{\text{ }}consideration.}\newline
\ensuremath{\text{ }}\newline
\NormalTok{I\ensuremath{\text{ }}look\ensuremath{\text{ }}forward\ensuremath{\text{ }}to\ensuremath{\text{ }}your\ensuremath{\text{ }}reply.}\newline
\ensuremath{\text{ }}\newline
\NormalTok{\textbackslash{}closing\{Yours\ensuremath{\text{ }}Faithfully,\}}\newline
\ensuremath{\text{ }}\newline
\NormalTok{\textbackslash{}ps}\newline
\ensuremath{\text{ }}\newline
\NormalTok{P.S.\ensuremath{\text{ }}You\ensuremath{\text{ }}can\ensuremath{\text{ }}find\ensuremath{\text{ }}the\ensuremath{\text{ }}full\ensuremath{\text{ }}text\ensuremath{\text{ }}of\ensuremath{\text{ }}GFDL\ensuremath{\text{ }}license\ensuremath{\text{ }}at}\newline
\NormalTok{\textbackslash{}url\{http://www.gnu.org/copyleft/fdl.html\}.}\newline
\ensuremath{\text{ }}\newline
\NormalTok{\textbackslash{}encl\{Copyright\ensuremath{\text{ }}permission\ensuremath{\text{ }}form\}}\newline
\ensuremath{\text{ }}\newline
\NormalTok{\textbackslash{}end\{letter\}}\newline
\NormalTok{\textbackslash{}end\{document\}}\newline
\end{Highlighting}
\end{Shaded}


To move the closing and signature parts to the left, insert the following before \LaTeXTT{\textbackslash{}begin\{document\}}:

\LaTeXTT{\textbackslash{}longindentation=0pt}

The amount of space to the left can be adjusted by increasing the 0pt.
\section{Envelopes}
\label{721}\subsection{Using the \LaTeXTT{envlab} package}
\label{722}
The \LaTeXTT{envlab} package provides customization to the \LaTeXTT{\textbackslash{}makelabels} command, allowing the user to print on any of an assortment of labels or envelope sizes.  For example, beginning your LaTeX file the following way produces a document which includes the letter and a business-{}size (\#10) envelope on the following page.

\begin{Shaded}
\begin{Highlighting}[]

\NormalTok{\textbackslash{}documentclass\{letter\}}
\NormalTok{\textbackslash{}usepackage[businessenvelope]\{envlab\}}
\NormalTok{\textbackslash{}makelabels}
\end{Highlighting}
\end{Shaded}


Refer to the \myhref{http://mirrors.ctan.org/macros/latex/contrib/envlab/elguide.pdf}{{\ttfamily \setmainfont[Path=/usr/share/fonts/truetype/cmu/,UprightFont=cmunrm.ttf,BoldFont=cmunbx.ttf,ItalicFont=cmunti.ttf,BoldItalicFont=cmunbi.ttf]{cmuntt.ttf}\setmonofont[Path=/usr/share/fonts/truetype/cmu/,UprightFont=cmuntt.ttf,BoldFont=cmuntb.ttf,ItalicFont=cmunit.ttf,BoldItalicFont=cmuntx.ttf]{cmuntt.ttf}\ttfamily envlab}{$\text{ }$}\setmainfont[Path=/usr/share/fonts/truetype/cmu/,UprightFont=cmunrm.ttf,BoldFont=cmunbx.ttf,ItalicFont=cmunti.ttf,BoldItalicFont=cmunbi.ttf]{cmunrm.ttf}\setmonofont[Path=/usr/share/fonts/truetype/cmu/,UprightFont=cmuntt.ttf,BoldFont=cmuntb.ttf,ItalicFont=cmunit.ttf,BoldItalicFont=cmuntx.ttf]{cmunrm.ttf} user guide} for more information about this capable package. Note that the \LaTeXTT{envlab} package has issues displaying characters outside the base ASCII character set, see \myhref{http://bugs.debian.org/cgi-bin/bugreport.cgi?bug=547978}{this bug report} for more information.
\subsection{Using the \LaTeXTT{geometry} package}
\label{723}
Here is a relatively simple envelope which uses the \LaTeXTT{geometry} package which is used because it vastly simplifies the task of rearranging things on the page (and the page itself).

\begin{Shaded}
\begin{Highlighting}[]

\CommentTok{% envelope.tex}
\NormalTok{\textbackslash{}documentclass\{letter\}}
\NormalTok{\textbackslash{}usepackage[}
\NormalTok{left=1in,top=0.15in,papersize=\{4.125in,9.5in\},landscape,twoside=false]\{geometry\}}
\NormalTok{\textbackslash{}setlength\textbackslash{}parskip\{0pt\}}
\NormalTok{\textbackslash{}pagestyle\{empty\}}
 
\NormalTok{\textbackslash{}begin\{document\}}
 
\NormalTok{FROM-NAME}
 
\NormalTok{FROM-STREET ADDRESS}
 
\NormalTok{FROM-CITY, STATE, \textbackslash{} ZIP}
 
\NormalTok{\textbackslash{}vspace\{1.0in\}\textbackslash{}large}
\NormalTok{\textbackslash{}setlength\textbackslash{}parindent\{3.6in\}}
 
\NormalTok{TO-NAME}
 
\NormalTok{TO-STREET ADDRESS}
 
\NormalTok{TO-CITY, STATE, \textbackslash{} ZIP}
 
\NormalTok{\textbackslash{}end\{document\}}
\end{Highlighting}
\end{Shaded}




\begin{minipage}{0.25000\textwidth}
\begin{center}
\includegraphics[width=1.0\textwidth,height=6.5in,keepaspectratio]{../images/156.jpg}
\end{center}
\raggedright{}\myfigurewithcaption{156}{A sample envelope to be printed in landscape mode.}
\end{minipage}\vspace{0.75cm}


\subsection{Printing}
\label{724}
The above will certainly take care of the spacing but the actual printing is between you and your printer.  One user reports that printing envelopes created with \LaTeXTT{envlab} is relatively painless.  If you use the \LaTeXTT{geometry} package, you may find the following commands useful for printing the envelope.
\\

\TemplateSpaceIndent{$\text{ }${}\${}$\text{ }${}pdflatex$\text{ }${}envelope.tex$\text{ }$\newline{}
$\text{ }${}\${}$\text{ }${}pdf2ps$\text{ }${}envelope.pdf$\text{ }$\newline{}
$\text{ }${}\${}$\text{ }${}lpr$\text{ }${}-{}o$\text{ }${}landscape$\text{ }${}envelope.ps}


Alternatively, you can use the latex dvi output driver.

In the first line, dvips command converts the .dvi file produced by latex into a .ps (PostScript) file. In the second line, the PostScript file is sent to the printer. \\

\TemplateSpaceIndent{$\text{ }${}\${}$\text{ }${}latex$\text{ }${}envelope.tex$\text{ }${}\&\&$\text{ }${}dvips$\text{ }${}-{}t$\text{ }${}unknown$\text{ }${}-{}T$\text{ }${}9.5in,4.125in$\text{ }${}envelope.dvi$\text{ }$\newline{}
$\text{ }${}\${}$\text{ }${}lpr$\text{ }${}-{}o$\text{ }${}landscape$\text{ }${}envelope.ps}


It is reported that {\ttfamily \setmainfont[Path=/usr/share/fonts/truetype/cmu/,UprightFont=cmunrm.ttf,BoldFont=cmunbx.ttf,ItalicFont=cmunti.ttf,BoldItalicFont=cmunbi.ttf]{cmuntt.ttf}\setmonofont[Path=/usr/share/fonts/truetype/cmu/,UprightFont=cmuntt.ttf,BoldFont=cmuntb.ttf,ItalicFont=cmunit.ttf,BoldItalicFont=cmuntx.ttf]{cmuntt.ttf}\ttfamily pdflatex}{$\text{ }$}\setmainfont[Path=/usr/share/fonts/truetype/cmu/,UprightFont=cmunrm.ttf,BoldFont=cmunbx.ttf,ItalicFont=cmunti.ttf,BoldItalicFont=cmunbi.ttf]{cmunrm.ttf}\setmonofont[Path=/usr/share/fonts/truetype/cmu/,UprightFont=cmuntt.ttf,BoldFont=cmuntb.ttf,ItalicFont=cmunit.ttf,BoldItalicFont=cmuntx.ttf]{cmunrm.ttf} creates the right page size but not {\ttfamily \setmainfont[Path=/usr/share/fonts/truetype/cmu/,UprightFont=cmunrm.ttf,BoldFont=cmunbx.ttf,ItalicFont=cmunti.ttf,BoldItalicFont=cmunbi.ttf]{cmuntt.ttf}\setmonofont[Path=/usr/share/fonts/truetype/cmu/,UprightFont=cmuntt.ttf,BoldFont=cmuntb.ttf,ItalicFont=cmunit.ttf,BoldItalicFont=cmuntx.ttf]{cmuntt.ttf}\ttfamily dvips}{$\text{ }$}\setmainfont[Path=/usr/share/fonts/truetype/cmu/,UprightFont=cmunrm.ttf,BoldFont=cmunbx.ttf,ItalicFont=cmunti.ttf,BoldItalicFont=cmunbi.ttf]{cmunrm.ttf}\setmonofont[Path=/usr/share/fonts/truetype/cmu/,UprightFont=cmuntt.ttf,BoldFont=cmuntb.ttf,ItalicFont=cmunit.ttf,BoldItalicFont=cmuntx.ttf]{cmunrm.ttf} despite what it says in the \LaTeXTT{geometry} manual. It will never work though unless your printer settings are adjusted to the correct page style. These settings depend on the printer filter you are using and in CUPS might be available on the {\ttfamily \setmainfont[Path=/usr/share/fonts/truetype/cmu/,UprightFont=cmunrm.ttf,BoldFont=cmunbx.ttf,ItalicFont=cmunti.ttf,BoldItalicFont=cmunbi.ttf]{cmuntt.ttf}\setmonofont[Path=/usr/share/fonts/truetype/cmu/,UprightFont=cmuntt.ttf,BoldFont=cmuntb.ttf,ItalicFont=cmunit.ttf,BoldItalicFont=cmuntx.ttf]{cmuntt.ttf}\ttfamily lpr}{$\text{ }$}\setmainfont[Path=/usr/share/fonts/truetype/cmu/,UprightFont=cmunrm.ttf,BoldFont=cmunbx.ttf,ItalicFont=cmunti.ttf,BoldItalicFont=cmunbi.ttf]{cmunrm.ttf}\setmonofont[Path=/usr/share/fonts/truetype/cmu/,UprightFont=cmuntt.ttf,BoldFont=cmuntb.ttf,ItalicFont=cmunit.ttf,BoldItalicFont=cmuntx.ttf]{cmunrm.ttf} command line.
\section{Windowed envelopes}
\label{725}
An alternative to separately printing addresses on envelopes is to use the letter class from the KOMA package.  It supports additional features like folding marks and the correct address placement for windowed envelopes.  Using the \LaTeXTT{scrlttr2} document class from the KOMA package the example letter code is:


\begin{Shaded}
\begin{Highlighting}[]

\CommentTok{\%\ensuremath{\text{ }}koma_env.tex}\newline
\NormalTok{\textbackslash{}documentclass[a4paper]\{scrlttr2\}}\newline
\NormalTok{\textbackslash{}usepackage\{lmodern\}}\newline
\NormalTok{\textbackslash{}usepackage[utf8]\{inputenc\}}\newline
\NormalTok{\textbackslash{}usepackage[T1]\{fontenc\}}\newline
\NormalTok{\textbackslash{}usepackage[english]\{babel\}}\newline
\NormalTok{\textbackslash{}usepackage\{url\}}\newline
\ensuremath{\text{ }}\newline
\ensuremath{\text{ }}\newline
\NormalTok{\textbackslash{}setkomavar\{fromname\}\{Joe\ensuremath{\text{ }}Bloggs\}}\newline
\NormalTok{\textbackslash{}setkomavar\{fromaddress\}\{21\ensuremath{\text{ }}Bridge\ensuremath{\text{ }}Street\ensuremath{\text{ }}\textbackslash{}\textbackslash{}\ensuremath{\text{ }}Smallville\ensuremath{\text{ }}\textbackslash{}\textbackslash{}\ensuremath{\text{ }}Dunwich\ensuremath{\text{ }}DU3\ensuremath{\text{ }}4WE\}}\newline
\NormalTok{\textbackslash{}setkomavar\{fromphone\}\{0123\ensuremath{\text{ }}45679\}}\newline
\ensuremath{\text{ }}\newline
\NormalTok{\textbackslash{}begin\{document\}}\newline
\ensuremath{\text{ }}\newline
\NormalTok{\textbackslash{}begin\{letter\}\{Director\ensuremath{\text{ }}\textbackslash{}\textbackslash{}\ensuremath{\text{ }}Doe\ensuremath{\text{ }}\textbackslash{}\&\ensuremath{\text{ }}Co\ensuremath{\text{ }}\textbackslash{}\textbackslash{}\ensuremath{\text{ }}35\ensuremath{\text{ }}Anthony\ensuremath{\text{ }}Road}\newline
\NormalTok{\textbackslash{}\textbackslash{}\ensuremath{\text{ }}Newport\ensuremath{\text{ }}\textbackslash{}\textbackslash{}\ensuremath{\text{ }}Ipswich\ensuremath{\text{ }}IP3\ensuremath{\text{ }}5RT\}}\newline
\ensuremath{\text{ }}\newline
\NormalTok{\textbackslash{}KOMAoptions\{fromphone=true,fromfax=false\}}\newline
\NormalTok{\textbackslash{}setkomavar\{subject\}\{Wikipedia\}}\newline
\NormalTok{\textbackslash{}setkomavar\{customer\}\{2342\}}\newline
\NormalTok{\textbackslash{}opening\{Dear\ensuremath{\text{ }}Sir\ensuremath{\text{ }}or\ensuremath{\text{ }}Madam,\}}\newline
\ensuremath{\text{ }}\newline
\NormalTok{I\ensuremath{\text{ }}am\ensuremath{\text{ }}writing\ensuremath{\text{ }}to\ensuremath{\text{ }}you\ensuremath{\text{ }}on\ensuremath{\text{ }}behalf\ensuremath{\text{ }}of\ensuremath{\text{ }}the\ensuremath{\text{ }}Wikipedia\ensuremath{\text{ }}project}\newline
\NormalTok{(\textbackslash{}url\{http://www.wikipedia.org/\}),\ensuremath{\text{ }}an\ensuremath{\text{ }}endeavour\ensuremath{\text{ }}to\ensuremath{\text{ }}build\ensuremath{\text{ }}a}\newline
\NormalTok{fully-fledged\ensuremath{\text{ }}multilingual\ensuremath{\text{ }}encyclopaedia\ensuremath{\text{ }}in\ensuremath{\text{ }}an\ensuremath{\text{ }}entirely\ensuremath{\text{ }}open}\newline
\NormalTok{manner,\ensuremath{\text{ }}to\ensuremath{\text{ }}ask\ensuremath{\text{ }}for\ensuremath{\text{ }}permission\ensuremath{\text{ }}to\ensuremath{\text{ }}use\ensuremath{\text{ }}your\ensuremath{\text{ }}copyrighted\ensuremath{\text{ }}material.}\newline
\ensuremath{\text{ }}\newline
\NormalTok{\textbackslash{}ldots\ensuremath{\text{ }}}\newline
\ensuremath{\text{ }}\newline
\NormalTok{That\ensuremath{\text{ }}said,\ensuremath{\text{ }}allow\ensuremath{\text{ }}me\ensuremath{\text{ }}to\ensuremath{\text{ }}reiterate\ensuremath{\text{ }}that\ensuremath{\text{ }}your\ensuremath{\text{ }}material\ensuremath{\text{ }}will\ensuremath{\text{ }}be\ensuremath{\text{ }}used}\newline
\NormalTok{to\ensuremath{\text{ }}the\ensuremath{\text{ }}noble\ensuremath{\text{ }}end\ensuremath{\text{ }}of\ensuremath{\text{ }}providing\ensuremath{\text{ }}a\ensuremath{\text{ }}free\ensuremath{\text{ }}collection\ensuremath{\text{ }}of\ensuremath{\text{ }}knowledge\ensuremath{\text{ }}for}\newline
\NormalTok{everyone;\ensuremath{\text{ }}naturally\ensuremath{\text{ }}enough,\ensuremath{\text{ }}only\ensuremath{\text{ }}if\ensuremath{\text{ }}you\ensuremath{\text{ }}agree.\ensuremath{\text{ }}If\ensuremath{\text{ }}that\ensuremath{\text{ }}is\ensuremath{\text{ }}the}\newline
\NormalTok{case,\ensuremath{\text{ }}could\ensuremath{\text{ }}you\ensuremath{\text{ }}kindly\ensuremath{\text{ }}fill\ensuremath{\text{ }}in\ensuremath{\text{ }}the\ensuremath{\text{ }}attached\ensuremath{\text{ }}form\ensuremath{\text{ }}and\ensuremath{\text{ }}post\ensuremath{\text{ }}it\ensuremath{\text{ }}back}\newline
\NormalTok{to\ensuremath{\text{ }}me?\ensuremath{\text{ }}We\ensuremath{\text{ }}shall\ensuremath{\text{ }}greatly\ensuremath{\text{ }}appreciate\ensuremath{\text{ }}it.}\newline
\ensuremath{\text{ }}\newline
\NormalTok{Thank\ensuremath{\text{ }}you\ensuremath{\text{ }}for\ensuremath{\text{ }}your\ensuremath{\text{ }}time\ensuremath{\text{ }}and\ensuremath{\text{ }}consideration.}\newline
\ensuremath{\text{ }}\newline
\NormalTok{I\ensuremath{\text{ }}look\ensuremath{\text{ }}forward\ensuremath{\text{ }}to\ensuremath{\text{ }}your\ensuremath{\text{ }}reply.}\newline
\ensuremath{\text{ }}\newline
\NormalTok{\textbackslash{}closing\{Yours\ensuremath{\text{ }}Faithfully,\}}\newline
\NormalTok{\textbackslash{}ps\{P.S.\ensuremath{\text{ }}You\ensuremath{\text{ }}can\ensuremath{\text{ }}find\ensuremath{\text{ }}the\ensuremath{\text{ }}full\ensuremath{\text{ }}text\ensuremath{\text{ }}of\ensuremath{\text{ }}GFDL\ensuremath{\text{ }}license\ensuremath{\text{ }}at}\newline
\NormalTok{\textbackslash{}url\{http://www.gnu.org/copyleft/fdl.html\}.\}}\newline
\NormalTok{\textbackslash{}encl\{Copyright\ensuremath{\text{ }}permission\ensuremath{\text{ }}form\}}\newline
\ensuremath{\text{ }}\newline
\NormalTok{\textbackslash{}end\{letter\}}\newline
\ensuremath{\text{ }}\newline
\NormalTok{\textbackslash{}end\{document\}}\newline
\end{Highlighting}
\end{Shaded}


The output is generated via\\

\TemplateSpaceIndent{$\text{ }${}\${}$\text{ }${}pdflatex$\text{ }${}koma\_env}




\begin{minipage}{0.37500\textwidth}
\begin{center}
\includegraphics[width=1.0\textwidth,height=6.5in,keepaspectratio]{../images/157.png}
\end{center}
\raggedright{}\myfigurewithcaption{157}{A sample letter with folding marks ready for standardized windowed envelopes.}
\end{minipage}\vspace{0.75cm}



Folding the print of the resulting file koma\_env.pdf according the folding marks it can be placed into standardized windowed envelopes DIN C6/5, DL, C4, C5 or C6.

In addition to the default, the KOMA-{}package includes predefined format definitions for different standardized Swiss and Japanese letter formats.
\section{Reference: {\ttfamily \setmainfont[Path=/usr/share/fonts/truetype/cmu/,UprightFont=cmunrm.ttf,BoldFont=cmunbx.ttf,ItalicFont=cmunti.ttf,BoldItalicFont=cmunbi.ttf]{cmuntt.ttf}\setmonofont[Path=/usr/share/fonts/truetype/cmu/,UprightFont=cmuntt.ttf,BoldFont=cmuntb.ttf,ItalicFont=cmunit.ttf,BoldItalicFont=cmuntx.ttf]{cmuntt.ttf}\ttfamily letter.cls}{$\text{ }$}\setmainfont[Path=/usr/share/fonts/truetype/cmu/,UprightFont=cmunrm.ttf,BoldFont=cmunbx.ttf,ItalicFont=cmunti.ttf,BoldItalicFont=cmunbi.ttf]{cmunrm.ttf}\setmonofont[Path=/usr/share/fonts/truetype/cmu/,UprightFont=cmuntt.ttf,BoldFont=cmuntb.ttf,ItalicFont=cmunit.ttf,BoldItalicFont=cmuntx.ttf]{cmunrm.ttf} commands}
\label{726}

\begin{longtable}{|>{\RaggedRight}p{0.27291\linewidth}|>{\RaggedRight}p{0.66995\linewidth}|} \hline 
{\bfseries \hspace*{0pt}\ignorespaces{}\hspace*{0pt} command }&{\bfseries \hspace*{0pt}\ignorespaces{}\hspace*{0pt} description}\endhead  \hline \hspace*{0pt}\ignorespaces{}\hspace*{0pt} {\ttfamily \setmainfont[Path=/usr/share/fonts/truetype/cmu/,UprightFont=cmunrm.ttf,BoldFont=cmunbx.ttf,ItalicFont=cmunti.ttf,BoldItalicFont=cmunbi.ttf]{cmuntt.ttf}\setmonofont[Path=/usr/share/fonts/truetype/cmu/,UprightFont=cmuntt.ttf,BoldFont=cmuntb.ttf,ItalicFont=cmunit.ttf,BoldItalicFont=cmuntx.ttf]{cmuntt.ttf}\ttfamily \textbackslash{}name\{\}}{$\text{ }$}\setmainfont[Path=/usr/share/fonts/truetype/cmu/,UprightFont=cmunrm.ttf,BoldFont=cmunbx.ttf,ItalicFont=cmunti.ttf,BoldItalicFont=cmunbi.ttf]{cmunrm.ttf}\setmonofont[Path=/usr/share/fonts/truetype/cmu/,UprightFont=cmuntt.ttf,BoldFont=cmuntb.ttf,ItalicFont=cmunit.ttf,BoldItalicFont=cmuntx.ttf]{cmunrm.ttf} &\hspace*{0pt}\ignorespaces{}\hspace*{0pt}\\ \hline \hspace*{0pt}\ignorespaces{}\hspace*{0pt} {\ttfamily \setmainfont[Path=/usr/share/fonts/truetype/cmu/,UprightFont=cmunrm.ttf,BoldFont=cmunbx.ttf,ItalicFont=cmunti.ttf,BoldItalicFont=cmunbi.ttf]{cmuntt.ttf}\setmonofont[Path=/usr/share/fonts/truetype/cmu/,UprightFont=cmuntt.ttf,BoldFont=cmuntb.ttf,ItalicFont=cmunit.ttf,BoldItalicFont=cmuntx.ttf]{cmuntt.ttf}\ttfamily \textbackslash{}signature\{\}}{$\text{ }$}\setmainfont[Path=/usr/share/fonts/truetype/cmu/,UprightFont=cmunrm.ttf,BoldFont=cmunbx.ttf,ItalicFont=cmunti.ttf,BoldItalicFont=cmunbi.ttf]{cmunrm.ttf}\setmonofont[Path=/usr/share/fonts/truetype/cmu/,UprightFont=cmuntt.ttf,BoldFont=cmuntb.ttf,ItalicFont=cmunit.ttf,BoldItalicFont=cmuntx.ttf]{cmunrm.ttf} &\hspace*{0pt}\ignorespaces{}\hspace*{0pt}\\ \hline \hspace*{0pt}\ignorespaces{}\hspace*{0pt} {\ttfamily \setmainfont[Path=/usr/share/fonts/truetype/cmu/,UprightFont=cmunrm.ttf,BoldFont=cmunbx.ttf,ItalicFont=cmunti.ttf,BoldItalicFont=cmunbi.ttf]{cmuntt.ttf}\setmonofont[Path=/usr/share/fonts/truetype/cmu/,UprightFont=cmuntt.ttf,BoldFont=cmuntb.ttf,ItalicFont=cmunit.ttf,BoldItalicFont=cmuntx.ttf]{cmuntt.ttf}\ttfamily \textbackslash{}address\{\}}{$\text{ }$}\setmainfont[Path=/usr/share/fonts/truetype/cmu/,UprightFont=cmunrm.ttf,BoldFont=cmunbx.ttf,ItalicFont=cmunti.ttf,BoldItalicFont=cmunbi.ttf]{cmunrm.ttf}\setmonofont[Path=/usr/share/fonts/truetype/cmu/,UprightFont=cmuntt.ttf,BoldFont=cmuntb.ttf,ItalicFont=cmunit.ttf,BoldItalicFont=cmuntx.ttf]{cmunrm.ttf} &\hspace*{0pt}\ignorespaces{}\hspace*{0pt}\\ \hline \hspace*{0pt}\ignorespaces{}\hspace*{0pt} {\ttfamily \setmainfont[Path=/usr/share/fonts/truetype/cmu/,UprightFont=cmunrm.ttf,BoldFont=cmunbx.ttf,ItalicFont=cmunti.ttf,BoldItalicFont=cmunbi.ttf]{cmuntt.ttf}\setmonofont[Path=/usr/share/fonts/truetype/cmu/,UprightFont=cmuntt.ttf,BoldFont=cmuntb.ttf,ItalicFont=cmunit.ttf,BoldItalicFont=cmuntx.ttf]{cmuntt.ttf}\ttfamily \textbackslash{}location\{\}}{$\text{ }$}\setmainfont[Path=/usr/share/fonts/truetype/cmu/,UprightFont=cmunrm.ttf,BoldFont=cmunbx.ttf,ItalicFont=cmunti.ttf,BoldItalicFont=cmunbi.ttf]{cmunrm.ttf}\setmonofont[Path=/usr/share/fonts/truetype/cmu/,UprightFont=cmuntt.ttf,BoldFont=cmuntb.ttf,ItalicFont=cmunit.ttf,BoldItalicFont=cmuntx.ttf]{cmunrm.ttf} &\hspace*{0pt}\ignorespaces{}\hspace*{0pt}\\ \hline \hspace*{0pt}\ignorespaces{}\hspace*{0pt} {\ttfamily \setmainfont[Path=/usr/share/fonts/truetype/cmu/,UprightFont=cmunrm.ttf,BoldFont=cmunbx.ttf,ItalicFont=cmunti.ttf,BoldItalicFont=cmunbi.ttf]{cmuntt.ttf}\setmonofont[Path=/usr/share/fonts/truetype/cmu/,UprightFont=cmuntt.ttf,BoldFont=cmuntb.ttf,ItalicFont=cmunit.ttf,BoldItalicFont=cmuntx.ttf]{cmuntt.ttf}\ttfamily \textbackslash{}telephone\{\}}{$\text{ }$}\setmainfont[Path=/usr/share/fonts/truetype/cmu/,UprightFont=cmunrm.ttf,BoldFont=cmunbx.ttf,ItalicFont=cmunti.ttf,BoldItalicFont=cmunbi.ttf]{cmunrm.ttf}\setmonofont[Path=/usr/share/fonts/truetype/cmu/,UprightFont=cmuntt.ttf,BoldFont=cmuntb.ttf,ItalicFont=cmunit.ttf,BoldItalicFont=cmuntx.ttf]{cmunrm.ttf} &\hspace*{0pt}\ignorespaces{}\hspace*{0pt}\\ \hline \hspace*{0pt}\ignorespaces{}\hspace*{0pt} {\ttfamily \setmainfont[Path=/usr/share/fonts/truetype/cmu/,UprightFont=cmunrm.ttf,BoldFont=cmunbx.ttf,ItalicFont=cmunti.ttf,BoldItalicFont=cmunbi.ttf]{cmuntt.ttf}\setmonofont[Path=/usr/share/fonts/truetype/cmu/,UprightFont=cmuntt.ttf,BoldFont=cmuntb.ttf,ItalicFont=cmunit.ttf,BoldItalicFont=cmuntx.ttf]{cmuntt.ttf}\ttfamily \textbackslash{}makelabels}{$\text{ }$}\setmainfont[Path=/usr/share/fonts/truetype/cmu/,UprightFont=cmunrm.ttf,BoldFont=cmunbx.ttf,ItalicFont=cmunti.ttf,BoldItalicFont=cmunbi.ttf]{cmunrm.ttf}\setmonofont[Path=/usr/share/fonts/truetype/cmu/,UprightFont=cmuntt.ttf,BoldFont=cmuntb.ttf,ItalicFont=cmunit.ttf,BoldItalicFont=cmuntx.ttf]{cmunrm.ttf} &\hspace*{0pt}\ignorespaces{}\hspace*{0pt}\\ \hline \hspace*{0pt}\ignorespaces{}\hspace*{0pt} {\ttfamily \setmainfont[Path=/usr/share/fonts/truetype/cmu/,UprightFont=cmunrm.ttf,BoldFont=cmunbx.ttf,ItalicFont=cmunti.ttf,BoldItalicFont=cmunbi.ttf]{cmuntt.ttf}\setmonofont[Path=/usr/share/fonts/truetype/cmu/,UprightFont=cmuntt.ttf,BoldFont=cmuntb.ttf,ItalicFont=cmunit.ttf,BoldItalicFont=cmuntx.ttf]{cmuntt.ttf}\ttfamily \textbackslash{}stopbreaks}{$\text{ }$}\setmainfont[Path=/usr/share/fonts/truetype/cmu/,UprightFont=cmunrm.ttf,BoldFont=cmunbx.ttf,ItalicFont=cmunti.ttf,BoldItalicFont=cmunbi.ttf]{cmunrm.ttf}\setmonofont[Path=/usr/share/fonts/truetype/cmu/,UprightFont=cmuntt.ttf,BoldFont=cmuntb.ttf,ItalicFont=cmunit.ttf,BoldItalicFont=cmuntx.ttf]{cmunrm.ttf} &\hspace*{0pt}\ignorespaces{}\hspace*{0pt}\\ \hline \hspace*{0pt}\ignorespaces{}\hspace*{0pt} {\ttfamily \setmainfont[Path=/usr/share/fonts/truetype/cmu/,UprightFont=cmunrm.ttf,BoldFont=cmunbx.ttf,ItalicFont=cmunti.ttf,BoldItalicFont=cmunbi.ttf]{cmuntt.ttf}\setmonofont[Path=/usr/share/fonts/truetype/cmu/,UprightFont=cmuntt.ttf,BoldFont=cmuntb.ttf,ItalicFont=cmunit.ttf,BoldItalicFont=cmuntx.ttf]{cmuntt.ttf}\ttfamily \textbackslash{}startbreaks}{$\text{ }$}\setmainfont[Path=/usr/share/fonts/truetype/cmu/,UprightFont=cmunrm.ttf,BoldFont=cmunbx.ttf,ItalicFont=cmunti.ttf,BoldItalicFont=cmunbi.ttf]{cmunrm.ttf}\setmonofont[Path=/usr/share/fonts/truetype/cmu/,UprightFont=cmuntt.ttf,BoldFont=cmuntb.ttf,ItalicFont=cmunit.ttf,BoldItalicFont=cmuntx.ttf]{cmunrm.ttf} &\hspace*{0pt}\ignorespaces{}\hspace*{0pt}\\ \hline \hspace*{0pt}\ignorespaces{}\hspace*{0pt} {\ttfamily \setmainfont[Path=/usr/share/fonts/truetype/cmu/,UprightFont=cmunrm.ttf,BoldFont=cmunbx.ttf,ItalicFont=cmunti.ttf,BoldItalicFont=cmunbi.ttf]{cmuntt.ttf}\setmonofont[Path=/usr/share/fonts/truetype/cmu/,UprightFont=cmuntt.ttf,BoldFont=cmuntb.ttf,ItalicFont=cmunit.ttf,BoldItalicFont=cmuntx.ttf]{cmuntt.ttf}\ttfamily \textbackslash{}opening\{\}}{$\text{ }$}\setmainfont[Path=/usr/share/fonts/truetype/cmu/,UprightFont=cmunrm.ttf,BoldFont=cmunbx.ttf,ItalicFont=cmunti.ttf,BoldItalicFont=cmunbi.ttf]{cmunrm.ttf}\setmonofont[Path=/usr/share/fonts/truetype/cmu/,UprightFont=cmuntt.ttf,BoldFont=cmuntb.ttf,ItalicFont=cmunit.ttf,BoldItalicFont=cmuntx.ttf]{cmunrm.ttf} &\hspace*{0pt}\ignorespaces{}\hspace*{0pt}\\ \hline \hspace*{0pt}\ignorespaces{}\hspace*{0pt} {\ttfamily \setmainfont[Path=/usr/share/fonts/truetype/cmu/,UprightFont=cmunrm.ttf,BoldFont=cmunbx.ttf,ItalicFont=cmunti.ttf,BoldItalicFont=cmunbi.ttf]{cmuntt.ttf}\setmonofont[Path=/usr/share/fonts/truetype/cmu/,UprightFont=cmuntt.ttf,BoldFont=cmuntb.ttf,ItalicFont=cmunit.ttf,BoldItalicFont=cmuntx.ttf]{cmuntt.ttf}\ttfamily \textbackslash{}closing\{\}}{$\text{ }$}\setmainfont[Path=/usr/share/fonts/truetype/cmu/,UprightFont=cmunrm.ttf,BoldFont=cmunbx.ttf,ItalicFont=cmunti.ttf,BoldItalicFont=cmunbi.ttf]{cmunrm.ttf}\setmonofont[Path=/usr/share/fonts/truetype/cmu/,UprightFont=cmuntt.ttf,BoldFont=cmuntb.ttf,ItalicFont=cmunit.ttf,BoldItalicFont=cmuntx.ttf]{cmunrm.ttf} &\hspace*{0pt}\ignorespaces{}\hspace*{0pt}\\ \hline \hspace*{0pt}\ignorespaces{}\hspace*{0pt} {\ttfamily \setmainfont[Path=/usr/share/fonts/truetype/cmu/,UprightFont=cmunrm.ttf,BoldFont=cmunbx.ttf,ItalicFont=cmunti.ttf,BoldItalicFont=cmunbi.ttf]{cmuntt.ttf}\setmonofont[Path=/usr/share/fonts/truetype/cmu/,UprightFont=cmuntt.ttf,BoldFont=cmuntb.ttf,ItalicFont=cmunit.ttf,BoldItalicFont=cmuntx.ttf]{cmuntt.ttf}\ttfamily \textbackslash{}cc\{\}}{$\text{ }$}\setmainfont[Path=/usr/share/fonts/truetype/cmu/,UprightFont=cmunrm.ttf,BoldFont=cmunbx.ttf,ItalicFont=cmunti.ttf,BoldItalicFont=cmunbi.ttf]{cmunrm.ttf}\setmonofont[Path=/usr/share/fonts/truetype/cmu/,UprightFont=cmuntt.ttf,BoldFont=cmuntb.ttf,ItalicFont=cmunit.ttf,BoldItalicFont=cmuntx.ttf]{cmunrm.ttf} &\hspace*{0pt}\ignorespaces{}\hspace*{0pt} Start a parbox introduced with \textbackslash{}ccname:\\ \hline \hspace*{0pt}\ignorespaces{}\hspace*{0pt} {\ttfamily \setmainfont[Path=/usr/share/fonts/truetype/cmu/,UprightFont=cmunrm.ttf,BoldFont=cmunbx.ttf,ItalicFont=cmunti.ttf,BoldItalicFont=cmunbi.ttf]{cmuntt.ttf}\setmonofont[Path=/usr/share/fonts/truetype/cmu/,UprightFont=cmuntt.ttf,BoldFont=cmuntb.ttf,ItalicFont=cmunit.ttf,BoldItalicFont=cmuntx.ttf]{cmuntt.ttf}\ttfamily \textbackslash{}encl\{\}}{$\text{ }$}\setmainfont[Path=/usr/share/fonts/truetype/cmu/,UprightFont=cmunrm.ttf,BoldFont=cmunbx.ttf,ItalicFont=cmunti.ttf,BoldItalicFont=cmunbi.ttf]{cmunrm.ttf}\setmonofont[Path=/usr/share/fonts/truetype/cmu/,UprightFont=cmuntt.ttf,BoldFont=cmuntb.ttf,ItalicFont=cmunit.ttf,BoldItalicFont=cmuntx.ttf]{cmunrm.ttf} &\hspace*{0pt}\ignorespaces{}\hspace*{0pt} Start a parbox introduced with \textbackslash{}enclname:\\ \hline \hspace*{0pt}\ignorespaces{}\hspace*{0pt} {\ttfamily \setmainfont[Path=/usr/share/fonts/truetype/cmu/,UprightFont=cmunrm.ttf,BoldFont=cmunbx.ttf,ItalicFont=cmunti.ttf,BoldItalicFont=cmunbi.ttf]{cmuntt.ttf}\setmonofont[Path=/usr/share/fonts/truetype/cmu/,UprightFont=cmuntt.ttf,BoldFont=cmuntb.ttf,ItalicFont=cmunit.ttf,BoldItalicFont=cmuntx.ttf]{cmuntt.ttf}\ttfamily \textbackslash{}ps}{$\text{ }$}\setmainfont[Path=/usr/share/fonts/truetype/cmu/,UprightFont=cmunrm.ttf,BoldFont=cmunbx.ttf,ItalicFont=cmunti.ttf,BoldItalicFont=cmunbi.ttf]{cmunrm.ttf}\setmonofont[Path=/usr/share/fonts/truetype/cmu/,UprightFont=cmuntt.ttf,BoldFont=cmuntb.ttf,ItalicFont=cmunit.ttf,BoldItalicFont=cmuntx.ttf]{cmunrm.ttf} &\hspace*{0pt}\ignorespaces{}\hspace*{0pt} Begins a new paragraph, normally at the close of the letter\\ \hline \hspace*{0pt}\ignorespaces{}\hspace*{0pt} {\ttfamily \setmainfont[Path=/usr/share/fonts/truetype/cmu/,UprightFont=cmunrm.ttf,BoldFont=cmunbx.ttf,ItalicFont=cmunti.ttf,BoldItalicFont=cmunbi.ttf]{cmuntt.ttf}\setmonofont[Path=/usr/share/fonts/truetype/cmu/,UprightFont=cmuntt.ttf,BoldFont=cmuntb.ttf,ItalicFont=cmunit.ttf,BoldItalicFont=cmuntx.ttf]{cmuntt.ttf}\ttfamily \textbackslash{}stopletter}{$\text{ }$}\setmainfont[Path=/usr/share/fonts/truetype/cmu/,UprightFont=cmunrm.ttf,BoldFont=cmunbx.ttf,ItalicFont=cmunti.ttf,BoldItalicFont=cmunbi.ttf]{cmunrm.ttf}\setmonofont[Path=/usr/share/fonts/truetype/cmu/,UprightFont=cmuntt.ttf,BoldFont=cmuntb.ttf,ItalicFont=cmunit.ttf,BoldItalicFont=cmuntx.ttf]{cmunrm.ttf} &\hspace*{0pt}\ignorespaces{}\hspace*{0pt} (empty)\\ \hline \hspace*{0pt}\ignorespaces{}\hspace*{0pt} {\ttfamily \setmainfont[Path=/usr/share/fonts/truetype/cmu/,UprightFont=cmunrm.ttf,BoldFont=cmunbx.ttf,ItalicFont=cmunti.ttf,BoldItalicFont=cmunbi.ttf]{cmuntt.ttf}\setmonofont[Path=/usr/share/fonts/truetype/cmu/,UprightFont=cmuntt.ttf,BoldFont=cmuntb.ttf,ItalicFont=cmunit.ttf,BoldItalicFont=cmuntx.ttf]{cmuntt.ttf}\ttfamily \textbackslash{}returnaddress}{$\text{ }$}\setmainfont[Path=/usr/share/fonts/truetype/cmu/,UprightFont=cmunrm.ttf,BoldFont=cmunbx.ttf,ItalicFont=cmunti.ttf,BoldItalicFont=cmunbi.ttf]{cmunrm.ttf}\setmonofont[Path=/usr/share/fonts/truetype/cmu/,UprightFont=cmuntt.ttf,BoldFont=cmuntb.ttf,ItalicFont=cmunit.ttf,BoldItalicFont=cmuntx.ttf]{cmunrm.ttf} &\hspace*{0pt}\ignorespaces{}\hspace*{0pt} (empty)\\ \hline \hspace*{0pt}\ignorespaces{}\hspace*{0pt} {\ttfamily \setmainfont[Path=/usr/share/fonts/truetype/cmu/,UprightFont=cmunrm.ttf,BoldFont=cmunbx.ttf,ItalicFont=cmunti.ttf,BoldItalicFont=cmunbi.ttf]{cmuntt.ttf}\setmonofont[Path=/usr/share/fonts/truetype/cmu/,UprightFont=cmuntt.ttf,BoldFont=cmuntb.ttf,ItalicFont=cmunit.ttf,BoldItalicFont=cmuntx.ttf]{cmuntt.ttf}\ttfamily \textbackslash{}startlabels}{$\text{ }$}\setmainfont[Path=/usr/share/fonts/truetype/cmu/,UprightFont=cmunrm.ttf,BoldFont=cmunbx.ttf,ItalicFont=cmunti.ttf,BoldItalicFont=cmunbi.ttf]{cmunrm.ttf}\setmonofont[Path=/usr/share/fonts/truetype/cmu/,UprightFont=cmuntt.ttf,BoldFont=cmuntb.ttf,ItalicFont=cmunit.ttf,BoldItalicFont=cmuntx.ttf]{cmunrm.ttf} &\hspace*{0pt}\ignorespaces{}\hspace*{0pt}\\ \hline \hspace*{0pt}\ignorespaces{}\hspace*{0pt} {\ttfamily \setmainfont[Path=/usr/share/fonts/truetype/cmu/,UprightFont=cmunrm.ttf,BoldFont=cmunbx.ttf,ItalicFont=cmunti.ttf,BoldItalicFont=cmunbi.ttf]{cmuntt.ttf}\setmonofont[Path=/usr/share/fonts/truetype/cmu/,UprightFont=cmuntt.ttf,BoldFont=cmuntb.ttf,ItalicFont=cmunit.ttf,BoldItalicFont=cmuntx.ttf]{cmuntt.ttf}\ttfamily \textbackslash{}mlabel\{\}\{\}}{$\text{ }$}\setmainfont[Path=/usr/share/fonts/truetype/cmu/,UprightFont=cmunrm.ttf,BoldFont=cmunbx.ttf,ItalicFont=cmunti.ttf,BoldItalicFont=cmunbi.ttf]{cmunrm.ttf}\setmonofont[Path=/usr/share/fonts/truetype/cmu/,UprightFont=cmuntt.ttf,BoldFont=cmuntb.ttf,ItalicFont=cmunit.ttf,BoldItalicFont=cmuntx.ttf]{cmunrm.ttf} &\hspace*{0pt}\ignorespaces{}\hspace*{0pt}\\ \hline \hspace*{0pt}\ignorespaces{}\hspace*{0pt} {\ttfamily \setmainfont[Path=/usr/share/fonts/truetype/cmu/,UprightFont=cmunrm.ttf,BoldFont=cmunbx.ttf,ItalicFont=cmunti.ttf,BoldItalicFont=cmunbi.ttf]{cmuntt.ttf}\setmonofont[Path=/usr/share/fonts/truetype/cmu/,UprightFont=cmuntt.ttf,BoldFont=cmuntb.ttf,ItalicFont=cmunit.ttf,BoldItalicFont=cmuntx.ttf]{cmuntt.ttf}\ttfamily \textbackslash{}descriptionlabel\{\}}{$\text{ }$}\setmainfont[Path=/usr/share/fonts/truetype/cmu/,UprightFont=cmunrm.ttf,BoldFont=cmunbx.ttf,ItalicFont=cmunti.ttf,BoldItalicFont=cmunbi.ttf]{cmunrm.ttf}\setmonofont[Path=/usr/share/fonts/truetype/cmu/,UprightFont=cmuntt.ttf,BoldFont=cmuntb.ttf,ItalicFont=cmunit.ttf,BoldItalicFont=cmuntx.ttf]{cmunrm.ttf} &\hspace*{0pt}\ignorespaces{}\hspace*{0pt}\\ \hline \hspace*{0pt}\ignorespaces{}\hspace*{0pt} {\ttfamily \setmainfont[Path=/usr/share/fonts/truetype/cmu/,UprightFont=cmunrm.ttf,BoldFont=cmunbx.ttf,ItalicFont=cmunti.ttf,BoldItalicFont=cmunbi.ttf]{cmuntt.ttf}\setmonofont[Path=/usr/share/fonts/truetype/cmu/,UprightFont=cmuntt.ttf,BoldFont=cmuntb.ttf,ItalicFont=cmunit.ttf,BoldItalicFont=cmuntx.ttf]{cmuntt.ttf}\ttfamily \textbackslash{}ccname}{$\text{ }$}\setmainfont[Path=/usr/share/fonts/truetype/cmu/,UprightFont=cmunrm.ttf,BoldFont=cmunbx.ttf,ItalicFont=cmunti.ttf,BoldItalicFont=cmunbi.ttf]{cmunrm.ttf}\setmonofont[Path=/usr/share/fonts/truetype/cmu/,UprightFont=cmuntt.ttf,BoldFont=cmuntb.ttf,ItalicFont=cmunit.ttf,BoldItalicFont=cmuntx.ttf]{cmunrm.ttf} &\hspace*{0pt}\ignorespaces{}\hspace*{0pt} \symbol{34}cc\symbol{34}\\ \hline \hspace*{0pt}\ignorespaces{}\hspace*{0pt} {\ttfamily \setmainfont[Path=/usr/share/fonts/truetype/cmu/,UprightFont=cmunrm.ttf,BoldFont=cmunbx.ttf,ItalicFont=cmunti.ttf,BoldItalicFont=cmunbi.ttf]{cmuntt.ttf}\setmonofont[Path=/usr/share/fonts/truetype/cmu/,UprightFont=cmuntt.ttf,BoldFont=cmuntb.ttf,ItalicFont=cmunit.ttf,BoldItalicFont=cmuntx.ttf]{cmuntt.ttf}\ttfamily \textbackslash{}enclname}{$\text{ }$}\setmainfont[Path=/usr/share/fonts/truetype/cmu/,UprightFont=cmunrm.ttf,BoldFont=cmunbx.ttf,ItalicFont=cmunti.ttf,BoldItalicFont=cmunbi.ttf]{cmunrm.ttf}\setmonofont[Path=/usr/share/fonts/truetype/cmu/,UprightFont=cmuntt.ttf,BoldFont=cmuntb.ttf,ItalicFont=cmunit.ttf,BoldItalicFont=cmuntx.ttf]{cmunrm.ttf} &\hspace*{0pt}\ignorespaces{}\hspace*{0pt} \symbol{34}encl\symbol{34}\\ \hline \hspace*{0pt}\ignorespaces{}\hspace*{0pt} {\ttfamily \setmainfont[Path=/usr/share/fonts/truetype/cmu/,UprightFont=cmunrm.ttf,BoldFont=cmunbx.ttf,ItalicFont=cmunti.ttf,BoldItalicFont=cmunbi.ttf]{cmuntt.ttf}\setmonofont[Path=/usr/share/fonts/truetype/cmu/,UprightFont=cmuntt.ttf,BoldFont=cmuntb.ttf,ItalicFont=cmunit.ttf,BoldItalicFont=cmuntx.ttf]{cmuntt.ttf}\ttfamily \textbackslash{}pagename}{$\text{ }$}\setmainfont[Path=/usr/share/fonts/truetype/cmu/,UprightFont=cmunrm.ttf,BoldFont=cmunbx.ttf,ItalicFont=cmunti.ttf,BoldItalicFont=cmunbi.ttf]{cmunrm.ttf}\setmonofont[Path=/usr/share/fonts/truetype/cmu/,UprightFont=cmuntt.ttf,BoldFont=cmuntb.ttf,ItalicFont=cmunit.ttf,BoldItalicFont=cmuntx.ttf]{cmunrm.ttf} &\hspace*{0pt}\ignorespaces{}\hspace*{0pt} \symbol{34}Page\symbol{34}\\ \hline \hspace*{0pt}\ignorespaces{}\hspace*{0pt} {\ttfamily \setmainfont[Path=/usr/share/fonts/truetype/cmu/,UprightFont=cmunrm.ttf,BoldFont=cmunbx.ttf,ItalicFont=cmunti.ttf,BoldItalicFont=cmunbi.ttf]{cmuntt.ttf}\setmonofont[Path=/usr/share/fonts/truetype/cmu/,UprightFont=cmuntt.ttf,BoldFont=cmuntb.ttf,ItalicFont=cmunit.ttf,BoldItalicFont=cmuntx.ttf]{cmuntt.ttf}\ttfamily \textbackslash{}headtoname}{$\text{ }$}\setmainfont[Path=/usr/share/fonts/truetype/cmu/,UprightFont=cmunrm.ttf,BoldFont=cmunbx.ttf,ItalicFont=cmunti.ttf,BoldItalicFont=cmunbi.ttf]{cmunrm.ttf}\setmonofont[Path=/usr/share/fonts/truetype/cmu/,UprightFont=cmuntt.ttf,BoldFont=cmuntb.ttf,ItalicFont=cmunit.ttf,BoldItalicFont=cmuntx.ttf]{cmunrm.ttf} &\hspace*{0pt}\ignorespaces{}\hspace*{0pt} \symbol{34}To\symbol{34}\\ \hline \hspace*{0pt}\ignorespaces{}\hspace*{0pt} {\ttfamily \setmainfont[Path=/usr/share/fonts/truetype/cmu/,UprightFont=cmunrm.ttf,BoldFont=cmunbx.ttf,ItalicFont=cmunti.ttf,BoldItalicFont=cmunbi.ttf]{cmuntt.ttf}\setmonofont[Path=/usr/share/fonts/truetype/cmu/,UprightFont=cmuntt.ttf,BoldFont=cmuntb.ttf,ItalicFont=cmunit.ttf,BoldItalicFont=cmuntx.ttf]{cmuntt.ttf}\ttfamily \textbackslash{}date\{\}}{$\text{ }$}\setmainfont[Path=/usr/share/fonts/truetype/cmu/,UprightFont=cmunrm.ttf,BoldFont=cmunbx.ttf,ItalicFont=cmunti.ttf,BoldItalicFont=cmunbi.ttf]{cmunrm.ttf}\setmonofont[Path=/usr/share/fonts/truetype/cmu/,UprightFont=cmuntt.ttf,BoldFont=cmuntb.ttf,ItalicFont=cmunit.ttf,BoldItalicFont=cmuntx.ttf]{cmunrm.ttf} &\hspace*{0pt}\ignorespaces{}\hspace*{0pt} Alter the date. See {\itshape \setmainfont[Path=/usr/share/fonts/truetype/cmu/,UprightFont=cmunrm.ttf,BoldFont=cmunbx.ttf,ItalicFont=cmunti.ttf,BoldItalicFont=cmunbi.ttf]{cmunti.ttf}\setmonofont[Path=/usr/share/fonts/truetype/cmu/,UprightFont=cmuntt.ttf,BoldFont=cmuntb.ttf,ItalicFont=cmunit.ttf,BoldItalicFont=cmuntx.ttf]{cmunti.ttf}\itshape datetime}{$\text{ }$}\setmainfont[Path=/usr/share/fonts/truetype/cmu/,UprightFont=cmunrm.ttf,BoldFont=cmunbx.ttf,ItalicFont=cmunti.ttf,BoldItalicFont=cmunbi.ttf]{cmunrm.ttf}\setmonofont[Path=/usr/share/fonts/truetype/cmu/,UprightFont=cmuntt.ttf,BoldFont=cmuntb.ttf,ItalicFont=cmunit.ttf,BoldItalicFont=cmuntx.ttf]{cmunrm.ttf} package for alternative formattings.\\ \hline \hspace*{0pt}\ignorespaces{}\hspace*{0pt} {\ttfamily \setmainfont[Path=/usr/share/fonts/truetype/cmu/,UprightFont=cmunrm.ttf,BoldFont=cmunbx.ttf,ItalicFont=cmunti.ttf,BoldItalicFont=cmunbi.ttf]{cmuntt.ttf}\setmonofont[Path=/usr/share/fonts/truetype/cmu/,UprightFont=cmuntt.ttf,BoldFont=cmuntb.ttf,ItalicFont=cmunit.ttf,BoldItalicFont=cmuntx.ttf]{cmuntt.ttf}\ttfamily \textbackslash{}today}{$\text{ }$}\setmainfont[Path=/usr/share/fonts/truetype/cmu/,UprightFont=cmunrm.ttf,BoldFont=cmunbx.ttf,ItalicFont=cmunti.ttf,BoldItalicFont=cmunbi.ttf]{cmunrm.ttf}\setmonofont[Path=/usr/share/fonts/truetype/cmu/,UprightFont=cmuntt.ttf,BoldFont=cmuntb.ttf,ItalicFont=cmunit.ttf,BoldItalicFont=cmuntx.ttf]{cmunrm.ttf} &\hspace*{0pt}\ignorespaces{}\hspace*{0pt} Long form date\\ \hline 
\end{longtable}


\begin{longtable}{|>{\RaggedRight}p{0.46206\linewidth}|>{\RaggedRight}p{0.48080\linewidth}|} \hline 
{\bfseries \hspace*{0pt}\ignorespaces{}\hspace*{0pt} environment }&{\bfseries \hspace*{0pt}\ignorespaces{}\hspace*{0pt} Description}\endhead  \hline \hspace*{0pt}\ignorespaces{}\hspace*{0pt} {\ttfamily \setmainfont[Path=/usr/share/fonts/truetype/cmu/,UprightFont=cmunrm.ttf,BoldFont=cmunbx.ttf,ItalicFont=cmunti.ttf,BoldItalicFont=cmunbi.ttf]{cmuntt.ttf}\setmonofont[Path=/usr/share/fonts/truetype/cmu/,UprightFont=cmuntt.ttf,BoldFont=cmuntb.ttf,ItalicFont=cmunit.ttf,BoldItalicFont=cmuntx.ttf]{cmuntt.ttf}\ttfamily letter\{\}}{$\text{ }$}\setmainfont[Path=/usr/share/fonts/truetype/cmu/,UprightFont=cmunrm.ttf,BoldFont=cmunbx.ttf,ItalicFont=cmunti.ttf,BoldItalicFont=cmunbi.ttf]{cmunrm.ttf}\setmonofont[Path=/usr/share/fonts/truetype/cmu/,UprightFont=cmuntt.ttf,BoldFont=cmuntb.ttf,ItalicFont=cmunit.ttf,BoldItalicFont=cmuntx.ttf]{cmunrm.ttf} &\hspace*{0pt}\ignorespaces{}\hspace*{0pt} See main article\\ \hline \hspace*{0pt}\ignorespaces{}\hspace*{0pt} {\ttfamily \setmainfont[Path=/usr/share/fonts/truetype/cmu/,UprightFont=cmunrm.ttf,BoldFont=cmunbx.ttf,ItalicFont=cmunti.ttf,BoldItalicFont=cmunbi.ttf]{cmuntt.ttf}\setmonofont[Path=/usr/share/fonts/truetype/cmu/,UprightFont=cmuntt.ttf,BoldFont=cmuntb.ttf,ItalicFont=cmunit.ttf,BoldItalicFont=cmuntx.ttf]{cmuntt.ttf}\ttfamily description}{$\text{ }$}\setmainfont[Path=/usr/share/fonts/truetype/cmu/,UprightFont=cmunrm.ttf,BoldFont=cmunbx.ttf,ItalicFont=cmunti.ttf,BoldItalicFont=cmunbi.ttf]{cmunrm.ttf}\setmonofont[Path=/usr/share/fonts/truetype/cmu/,UprightFont=cmuntt.ttf,BoldFont=cmuntb.ttf,ItalicFont=cmunit.ttf,BoldItalicFont=cmuntx.ttf]{cmunrm.ttf} &\hspace*{0pt}\ignorespaces{}\hspace*{0pt}\\ \hline \hspace*{0pt}\ignorespaces{}\hspace*{0pt} {\ttfamily \setmainfont[Path=/usr/share/fonts/truetype/cmu/,UprightFont=cmunrm.ttf,BoldFont=cmunbx.ttf,ItalicFont=cmunti.ttf,BoldItalicFont=cmunbi.ttf]{cmuntt.ttf}\setmonofont[Path=/usr/share/fonts/truetype/cmu/,UprightFont=cmuntt.ttf,BoldFont=cmuntb.ttf,ItalicFont=cmunit.ttf,BoldItalicFont=cmuntx.ttf]{cmuntt.ttf}\ttfamily verse}{$\text{ }$}\setmainfont[Path=/usr/share/fonts/truetype/cmu/,UprightFont=cmunrm.ttf,BoldFont=cmunbx.ttf,ItalicFont=cmunti.ttf,BoldItalicFont=cmunbi.ttf]{cmunrm.ttf}\setmonofont[Path=/usr/share/fonts/truetype/cmu/,UprightFont=cmuntt.ttf,BoldFont=cmuntb.ttf,ItalicFont=cmunit.ttf,BoldItalicFont=cmuntx.ttf]{cmunrm.ttf} &\hspace*{0pt}\ignorespaces{}\hspace*{0pt}\\ \hline \hspace*{0pt}\ignorespaces{}\hspace*{0pt} {\ttfamily \setmainfont[Path=/usr/share/fonts/truetype/cmu/,UprightFont=cmunrm.ttf,BoldFont=cmunbx.ttf,ItalicFont=cmunti.ttf,BoldItalicFont=cmunbi.ttf]{cmuntt.ttf}\setmonofont[Path=/usr/share/fonts/truetype/cmu/,UprightFont=cmuntt.ttf,BoldFont=cmuntb.ttf,ItalicFont=cmunit.ttf,BoldItalicFont=cmuntx.ttf]{cmuntt.ttf}\ttfamily quotation}{$\text{ }$}\setmainfont[Path=/usr/share/fonts/truetype/cmu/,UprightFont=cmunrm.ttf,BoldFont=cmunbx.ttf,ItalicFont=cmunti.ttf,BoldItalicFont=cmunbi.ttf]{cmunrm.ttf}\setmonofont[Path=/usr/share/fonts/truetype/cmu/,UprightFont=cmuntt.ttf,BoldFont=cmuntb.ttf,ItalicFont=cmunit.ttf,BoldItalicFont=cmuntx.ttf]{cmunrm.ttf} &\hspace*{0pt}\ignorespaces{}\hspace*{0pt}\\ \hline \hspace*{0pt}\ignorespaces{}\hspace*{0pt} {\ttfamily \setmainfont[Path=/usr/share/fonts/truetype/cmu/,UprightFont=cmunrm.ttf,BoldFont=cmunbx.ttf,ItalicFont=cmunti.ttf,BoldItalicFont=cmunbi.ttf]{cmuntt.ttf}\setmonofont[Path=/usr/share/fonts/truetype/cmu/,UprightFont=cmuntt.ttf,BoldFont=cmuntb.ttf,ItalicFont=cmunit.ttf,BoldItalicFont=cmuntx.ttf]{cmuntt.ttf}\ttfamily quote}{$\text{ }$}\setmainfont[Path=/usr/share/fonts/truetype/cmu/,UprightFont=cmunrm.ttf,BoldFont=cmunbx.ttf,ItalicFont=cmunti.ttf,BoldItalicFont=cmunbi.ttf]{cmunrm.ttf}\setmonofont[Path=/usr/share/fonts/truetype/cmu/,UprightFont=cmuntt.ttf,BoldFont=cmuntb.ttf,ItalicFont=cmunit.ttf,BoldItalicFont=cmuntx.ttf]{cmunrm.ttf} &\hspace*{0pt}\ignorespaces{}\hspace*{0pt}\\ \hline 
\end{longtable}

\section{Sources}
\label{727}
\begin{myitemize}
\item{}  \myhref{http://mirrors.ctan.org/macros/latex/contrib/koma-script/doc/scrguien.pdf}{KOMA-{}Script -{} The Guide}
\end{myitemize}



\begin{myquote}
\item{}
\end{myquote}

\LaTeXNullTemplate{}



\myhref{https://pl.wikibooks.org/wiki/LaTeX\%2FPisanie\%20list\%C3\%B3w}{pl:LaTeX/Pisanie listów}
\myhref{https://fr.wikibooks.org/wiki/LaTeX\%2FLettre}{fr:LaTeX/Lettre}

\myhref{https://sr.wikibooks.org/wiki/LaTeX\%2F\%D0\%9F\%D0\%B8\%D1\%81\%D0\%BC\%D0\%B0}{sr:LaTeX/Писма}\chapter{Presentations}

\myminitoc
\label{728}

\label{729}
\LaTeXNullTemplate{}



LaTeX can be used for creating presentations. There are several packages for the task, including the {\ttfamily \setmainfont[Path=/usr/share/fonts/truetype/cmu/,UprightFont=cmunrm.ttf,BoldFont=cmunbx.ttf,ItalicFont=cmunti.ttf,BoldItalicFont=cmunbi.ttf]{cmuntt.ttf}\setmonofont[Path=/usr/share/fonts/truetype/cmu/,UprightFont=cmuntt.ttf,BoldFont=cmuntb.ttf,ItalicFont=cmunit.ttf,BoldItalicFont=cmuntx.ttf]{cmuntt.ttf}\ttfamily beamer}{$\text{ }$}\setmainfont[Path=/usr/share/fonts/truetype/cmu/,UprightFont=cmunrm.ttf,BoldFont=cmunbx.ttf,ItalicFont=cmunti.ttf,BoldItalicFont=cmunbi.ttf]{cmunrm.ttf}\setmonofont[Path=/usr/share/fonts/truetype/cmu/,UprightFont=cmuntt.ttf,BoldFont=cmuntb.ttf,ItalicFont=cmunit.ttf,BoldItalicFont=cmuntx.ttf]{cmunrm.ttf} package.
\section{The Beamer package}
\label{730}

The beamer package is provided with most LaTeX distributions, but is also available from \myhref{http://www.ctan.org/tex-archive/macros/latex/contrib/beamer/}{CTAN}. If you use MikTeX, all you have to do is to include the beamer package and let LaTeX download all wanted packages automatically. The \myhref{http://www.ctan.org/tex-archive/macros/latex/contrib/beamer/doc/beameruserguide.pdf}{documentation} explains the features in great detail. You can also have a look at the PracTex article {\bfseries \setmainfont[Path=/usr/share/fonts/truetype/cmu/,UprightFont=cmunrm.ttf,BoldFont=cmunbx.ttf,ItalicFont=cmunti.ttf,BoldItalicFont=cmunbi.ttf]{cmunbx.ttf}\setmonofont[Path=/usr/share/fonts/truetype/cmu/,UprightFont=cmuntt.ttf,BoldFont=cmuntb.ttf,ItalicFont=cmunit.ttf,BoldItalicFont=cmuntx.ttf]{cmunbx.ttf}\bfseries Beamer by Example}\setmainfont[Path=/usr/share/fonts/truetype/cmu/,UprightFont=cmunrm.ttf,BoldFont=cmunbx.ttf,ItalicFont=cmunti.ttf,BoldItalicFont=cmunbi.ttf]{cmunrm.ttf}\setmonofont[Path=/usr/share/fonts/truetype/cmu/,UprightFont=cmuntt.ttf,BoldFont=cmuntb.ttf,ItalicFont=cmunit.ttf,BoldItalicFont=cmuntx.ttf]{cmunrm.ttf}.\myfootnote{Andrew Mertz and William Slough,  \myfnhref{https://tug.org/pracjourn/2005-4/mertz/}{{\itshape \setmainfont[Path=/usr/share/fonts/truetype/cmu/,UprightFont=cmunrm.ttf,BoldFont=cmunbx.ttf,ItalicFont=cmunti.ttf,BoldItalicFont=cmunbi.ttf]{cmunti.ttf}\setmonofont[Path=/usr/share/fonts/truetype/cmu/,UprightFont=cmuntt.ttf,BoldFont=cmuntb.ttf,ItalicFont=cmunit.ttf,BoldItalicFont=cmuntx.ttf]{cmunti.ttf}\itshape Beamer by Example}{$\text{ }$}\setmainfont[Path=/usr/share/fonts/truetype/cmu/,UprightFont=cmunrm.ttf,BoldFont=cmunbx.ttf,ItalicFont=cmunti.ttf,BoldItalicFont=cmunbi.ttf]{cmunrm.ttf}\setmonofont[Path=/usr/share/fonts/truetype/cmu/,UprightFont=cmuntt.ttf,BoldFont=cmuntb.ttf,ItalicFont=cmunit.ttf,BoldItalicFont=cmuntx.ttf]{cmunrm.ttf} }}

The {\ttfamily \setmainfont[Path=/usr/share/fonts/truetype/cmu/,UprightFont=cmunrm.ttf,BoldFont=cmunbx.ttf,ItalicFont=cmunti.ttf,BoldItalicFont=cmunbi.ttf]{cmuntt.ttf}\setmonofont[Path=/usr/share/fonts/truetype/cmu/,UprightFont=cmuntt.ttf,BoldFont=cmuntb.ttf,ItalicFont=cmunit.ttf,BoldItalicFont=cmuntx.ttf]{cmuntt.ttf}\ttfamily beamer}{$\text{ }$}\setmainfont[Path=/usr/share/fonts/truetype/cmu/,UprightFont=cmunrm.ttf,BoldFont=cmunbx.ttf,ItalicFont=cmunti.ttf,BoldItalicFont=cmunbi.ttf]{cmunrm.ttf}\setmonofont[Path=/usr/share/fonts/truetype/cmu/,UprightFont=cmuntt.ttf,BoldFont=cmuntb.ttf,ItalicFont=cmunit.ttf,BoldItalicFont=cmuntx.ttf]{cmunrm.ttf} package also loads many useful packages including {\ttfamily \setmainfont[Path=/usr/share/fonts/truetype/cmu/,UprightFont=cmunrm.ttf,BoldFont=cmunbx.ttf,ItalicFont=cmunti.ttf,BoldItalicFont=cmunbi.ttf]{cmuntt.ttf}\setmonofont[Path=/usr/share/fonts/truetype/cmu/,UprightFont=cmuntt.ttf,BoldFont=cmuntb.ttf,ItalicFont=cmunit.ttf,BoldItalicFont=cmuntx.ttf]{cmuntt.ttf}\ttfamily hyperref}\setmainfont[Path=/usr/share/fonts/truetype/cmu/,UprightFont=cmunrm.ttf,BoldFont=cmunbx.ttf,ItalicFont=cmunti.ttf,BoldItalicFont=cmunbi.ttf]{cmunrm.ttf}\setmonofont[Path=/usr/share/fonts/truetype/cmu/,UprightFont=cmuntt.ttf,BoldFont=cmuntb.ttf,ItalicFont=cmunit.ttf,BoldItalicFont=cmuntx.ttf]{cmunrm.ttf}.
\subsection{Introductory example}
\label{731}

The beamer package is loaded by calling the {\ttfamily \setmainfont[Path=/usr/share/fonts/truetype/cmu/,UprightFont=cmunrm.ttf,BoldFont=cmunbx.ttf,ItalicFont=cmunti.ttf,BoldItalicFont=cmunbi.ttf]{cmuntt.ttf}\setmonofont[Path=/usr/share/fonts/truetype/cmu/,UprightFont=cmuntt.ttf,BoldFont=cmuntb.ttf,ItalicFont=cmunit.ttf,BoldItalicFont=cmuntx.ttf]{cmuntt.ttf}\ttfamily beamer}{$\text{ }$}\setmainfont[Path=/usr/share/fonts/truetype/cmu/,UprightFont=cmunrm.ttf,BoldFont=cmunbx.ttf,ItalicFont=cmunti.ttf,BoldItalicFont=cmunbi.ttf]{cmunrm.ttf}\setmonofont[Path=/usr/share/fonts/truetype/cmu/,UprightFont=cmuntt.ttf,BoldFont=cmuntb.ttf,ItalicFont=cmunit.ttf,BoldItalicFont=cmuntx.ttf]{cmunrm.ttf} class:


\begin{Shaded}
\begin{Highlighting}[]

\NormalTok{\textbackslash{}documentclass\{beamer\}}\newline
\end{Highlighting}
\end{Shaded}


The usual header information may then be specified. Note that if you are compiling with XeTeX then you should use


\begin{Shaded}
\begin{Highlighting}[]

\NormalTok{\textbackslash{}documentclass[xetex,mathserif,serif]\{beamer\}}\newline
\end{Highlighting}
\end{Shaded}


Inside the {\ttfamily \setmainfont[Path=/usr/share/fonts/truetype/cmu/,UprightFont=cmunrm.ttf,BoldFont=cmunbx.ttf,ItalicFont=cmunti.ttf,BoldItalicFont=cmunbi.ttf]{cmuntt.ttf}\setmonofont[Path=/usr/share/fonts/truetype/cmu/,UprightFont=cmuntt.ttf,BoldFont=cmuntb.ttf,ItalicFont=cmunit.ttf,BoldItalicFont=cmuntx.ttf]{cmuntt.ttf}\ttfamily document}{$\text{ }$}\setmainfont[Path=/usr/share/fonts/truetype/cmu/,UprightFont=cmunrm.ttf,BoldFont=cmunbx.ttf,ItalicFont=cmunti.ttf,BoldItalicFont=cmunbi.ttf]{cmunrm.ttf}\setmonofont[Path=/usr/share/fonts/truetype/cmu/,UprightFont=cmuntt.ttf,BoldFont=cmuntb.ttf,ItalicFont=cmunit.ttf,BoldItalicFont=cmuntx.ttf]{cmunrm.ttf} environment, multiple {\ttfamily \setmainfont[Path=/usr/share/fonts/truetype/cmu/,UprightFont=cmunrm.ttf,BoldFont=cmunbx.ttf,ItalicFont=cmunti.ttf,BoldItalicFont=cmunbi.ttf]{cmuntt.ttf}\setmonofont[Path=/usr/share/fonts/truetype/cmu/,UprightFont=cmuntt.ttf,BoldFont=cmuntb.ttf,ItalicFont=cmunit.ttf,BoldItalicFont=cmuntx.ttf]{cmuntt.ttf}\ttfamily frame}{$\text{ }$}\setmainfont[Path=/usr/share/fonts/truetype/cmu/,UprightFont=cmunrm.ttf,BoldFont=cmunbx.ttf,ItalicFont=cmunti.ttf,BoldItalicFont=cmunbi.ttf]{cmunrm.ttf}\setmonofont[Path=/usr/share/fonts/truetype/cmu/,UprightFont=cmuntt.ttf,BoldFont=cmuntb.ttf,ItalicFont=cmunit.ttf,BoldItalicFont=cmuntx.ttf]{cmunrm.ttf} environments specify the content to be put on each slide. The {\ttfamily \setmainfont[Path=/usr/share/fonts/truetype/cmu/,UprightFont=cmunrm.ttf,BoldFont=cmunbx.ttf,ItalicFont=cmunti.ttf,BoldItalicFont=cmunbi.ttf]{cmuntt.ttf}\setmonofont[Path=/usr/share/fonts/truetype/cmu/,UprightFont=cmuntt.ttf,BoldFont=cmuntb.ttf,ItalicFont=cmunit.ttf,BoldItalicFont=cmuntx.ttf]{cmuntt.ttf}\ttfamily frametitle}{$\text{ }$}\setmainfont[Path=/usr/share/fonts/truetype/cmu/,UprightFont=cmunrm.ttf,BoldFont=cmunbx.ttf,ItalicFont=cmunti.ttf,BoldItalicFont=cmunbi.ttf]{cmunrm.ttf}\setmonofont[Path=/usr/share/fonts/truetype/cmu/,UprightFont=cmuntt.ttf,BoldFont=cmuntb.ttf,ItalicFont=cmunit.ttf,BoldItalicFont=cmuntx.ttf]{cmunrm.ttf} command specifies the title for each slide (see image):


\begin{Shaded}
\begin{Highlighting}[]

\NormalTok{\textbackslash{}begin\{document\}}\newline
\ensuremath{\text{ }}\ensuremath{\text{ }}\NormalTok{\textbackslash{}begin\{frame\}}\newline
\ensuremath{\text{ }}\ensuremath{\text{ }}\ensuremath{\text{ }}\ensuremath{\text{ }}\NormalTok{\textbackslash{}frametitle\{This\ensuremath{\text{ }}is\ensuremath{\text{ }}the\ensuremath{\text{ }}first\ensuremath{\text{ }}slide\}}\newline
\ensuremath{\text{ }}\ensuremath{\text{ }}\ensuremath{\text{ }}\ensuremath{\text{ }}\CommentTok{\%Content\ensuremath{\text{ }}goes\ensuremath{\text{ }}here}\newline
\ensuremath{\text{ }}\ensuremath{\text{ }}\NormalTok{\textbackslash{}end\{frame\}}\newline
\ensuremath{\text{ }}\ensuremath{\text{ }}\NormalTok{\textbackslash{}begin\{frame\}}\newline
\ensuremath{\text{ }}\ensuremath{\text{ }}\ensuremath{\text{ }}\ensuremath{\text{ }}\NormalTok{\textbackslash{}frametitle\{This\ensuremath{\text{ }}is\ensuremath{\text{ }}the\ensuremath{\text{ }}second\ensuremath{\text{ }}slide\}}\newline
\ensuremath{\text{ }}\ensuremath{\text{ }}\ensuremath{\text{ }}\ensuremath{\text{ }}\NormalTok{\textbackslash{}framesubtitle\{A\ensuremath{\text{ }}bit\ensuremath{\text{ }}more\ensuremath{\text{ }}information\ensuremath{\text{ }}about\ensuremath{\text{ }}this\}}\newline
\ensuremath{\text{ }}\ensuremath{\text{ }}\ensuremath{\text{ }}\ensuremath{\text{ }}\CommentTok{\%More\ensuremath{\text{ }}content\ensuremath{\text{ }}goes\ensuremath{\text{ }}here}\newline
\ensuremath{\text{ }}\ensuremath{\text{ }}\NormalTok{\textbackslash{}end\{frame\}}\newline
\CommentTok{\%\ensuremath{\text{ }}etc}\newline
\NormalTok{\textbackslash{}end\{document\}}\newline
\end{Highlighting}
\end{Shaded}




\begin{minipage}{1.0\linewidth}
\begin{center}
\includegraphics[width=1.0\linewidth,height=6.5in,keepaspectratio]{../images/158.png}
\end{center}
\raggedright{}\myfigurewithoutcaption{158}
\end{minipage}\vspace{0.75cm}



The usual environments ({\ttfamily \setmainfont[Path=/usr/share/fonts/truetype/cmu/,UprightFont=cmunrm.ttf,BoldFont=cmunbx.ttf,ItalicFont=cmunti.ttf,BoldItalicFont=cmunbi.ttf]{cmuntt.ttf}\setmonofont[Path=/usr/share/fonts/truetype/cmu/,UprightFont=cmuntt.ttf,BoldFont=cmuntb.ttf,ItalicFont=cmunit.ttf,BoldItalicFont=cmuntx.ttf]{cmuntt.ttf}\ttfamily itemize}\setmainfont[Path=/usr/share/fonts/truetype/cmu/,UprightFont=cmunrm.ttf,BoldFont=cmunbx.ttf,ItalicFont=cmunti.ttf,BoldItalicFont=cmunbi.ttf]{cmunrm.ttf}\setmonofont[Path=/usr/share/fonts/truetype/cmu/,UprightFont=cmuntt.ttf,BoldFont=cmuntb.ttf,ItalicFont=cmunit.ttf,BoldItalicFont=cmuntx.ttf]{cmunrm.ttf}, {\ttfamily \setmainfont[Path=/usr/share/fonts/truetype/cmu/,UprightFont=cmunrm.ttf,BoldFont=cmunbx.ttf,ItalicFont=cmunti.ttf,BoldItalicFont=cmunbi.ttf]{cmuntt.ttf}\setmonofont[Path=/usr/share/fonts/truetype/cmu/,UprightFont=cmuntt.ttf,BoldFont=cmuntb.ttf,ItalicFont=cmunit.ttf,BoldItalicFont=cmuntx.ttf]{cmuntt.ttf}\ttfamily enumerate}\setmainfont[Path=/usr/share/fonts/truetype/cmu/,UprightFont=cmunrm.ttf,BoldFont=cmunbx.ttf,ItalicFont=cmunti.ttf,BoldItalicFont=cmunbi.ttf]{cmunrm.ttf}\setmonofont[Path=/usr/share/fonts/truetype/cmu/,UprightFont=cmuntt.ttf,BoldFont=cmuntb.ttf,ItalicFont=cmunit.ttf,BoldItalicFont=cmuntx.ttf]{cmunrm.ttf}, {\ttfamily \setmainfont[Path=/usr/share/fonts/truetype/cmu/,UprightFont=cmunrm.ttf,BoldFont=cmunbx.ttf,ItalicFont=cmunti.ttf,BoldItalicFont=cmunbi.ttf]{cmuntt.ttf}\setmonofont[Path=/usr/share/fonts/truetype/cmu/,UprightFont=cmuntt.ttf,BoldFont=cmuntb.ttf,ItalicFont=cmunit.ttf,BoldItalicFont=cmuntx.ttf]{cmuntt.ttf}\ttfamily equation}\setmainfont[Path=/usr/share/fonts/truetype/cmu/,UprightFont=cmunrm.ttf,BoldFont=cmunbx.ttf,ItalicFont=cmunti.ttf,BoldItalicFont=cmunbi.ttf]{cmunrm.ttf}\setmonofont[Path=/usr/share/fonts/truetype/cmu/,UprightFont=cmuntt.ttf,BoldFont=cmuntb.ttf,ItalicFont=cmunit.ttf,BoldItalicFont=cmuntx.ttf]{cmunrm.ttf}, etc.) may be used.

Inside frames, you can use environments like {\ttfamily \setmainfont[Path=/usr/share/fonts/truetype/cmu/,UprightFont=cmunrm.ttf,BoldFont=cmunbx.ttf,ItalicFont=cmunti.ttf,BoldItalicFont=cmunbi.ttf]{cmuntt.ttf}\setmonofont[Path=/usr/share/fonts/truetype/cmu/,UprightFont=cmuntt.ttf,BoldFont=cmuntb.ttf,ItalicFont=cmunit.ttf,BoldItalicFont=cmuntx.ttf]{cmuntt.ttf}\ttfamily block}\setmainfont[Path=/usr/share/fonts/truetype/cmu/,UprightFont=cmunrm.ttf,BoldFont=cmunbx.ttf,ItalicFont=cmunti.ttf,BoldItalicFont=cmunbi.ttf]{cmunrm.ttf}\setmonofont[Path=/usr/share/fonts/truetype/cmu/,UprightFont=cmuntt.ttf,BoldFont=cmuntb.ttf,ItalicFont=cmunit.ttf,BoldItalicFont=cmuntx.ttf]{cmunrm.ttf}, {\ttfamily \setmainfont[Path=/usr/share/fonts/truetype/cmu/,UprightFont=cmunrm.ttf,BoldFont=cmunbx.ttf,ItalicFont=cmunti.ttf,BoldItalicFont=cmunbi.ttf]{cmuntt.ttf}\setmonofont[Path=/usr/share/fonts/truetype/cmu/,UprightFont=cmuntt.ttf,BoldFont=cmuntb.ttf,ItalicFont=cmunit.ttf,BoldItalicFont=cmuntx.ttf]{cmuntt.ttf}\ttfamily theorem}\setmainfont[Path=/usr/share/fonts/truetype/cmu/,UprightFont=cmunrm.ttf,BoldFont=cmunbx.ttf,ItalicFont=cmunti.ttf,BoldItalicFont=cmunbi.ttf]{cmunrm.ttf}\setmonofont[Path=/usr/share/fonts/truetype/cmu/,UprightFont=cmuntt.ttf,BoldFont=cmuntb.ttf,ItalicFont=cmunit.ttf,BoldItalicFont=cmuntx.ttf]{cmunrm.ttf}, {\ttfamily \setmainfont[Path=/usr/share/fonts/truetype/cmu/,UprightFont=cmunrm.ttf,BoldFont=cmunbx.ttf,ItalicFont=cmunti.ttf,BoldItalicFont=cmunbi.ttf]{cmuntt.ttf}\setmonofont[Path=/usr/share/fonts/truetype/cmu/,UprightFont=cmuntt.ttf,BoldFont=cmuntb.ttf,ItalicFont=cmunit.ttf,BoldItalicFont=cmuntx.ttf]{cmuntt.ttf}\ttfamily proof}\setmainfont[Path=/usr/share/fonts/truetype/cmu/,UprightFont=cmunrm.ttf,BoldFont=cmunbx.ttf,ItalicFont=cmunti.ttf,BoldItalicFont=cmunbi.ttf]{cmunrm.ttf}\setmonofont[Path=/usr/share/fonts/truetype/cmu/,UprightFont=cmuntt.ttf,BoldFont=cmuntb.ttf,ItalicFont=cmunit.ttf,BoldItalicFont=cmuntx.ttf]{cmunrm.ttf}, ... Also, {\ttfamily \setmainfont[Path=/usr/share/fonts/truetype/cmu/,UprightFont=cmunrm.ttf,BoldFont=cmunbx.ttf,ItalicFont=cmunti.ttf,BoldItalicFont=cmunbi.ttf]{cmuntt.ttf}\setmonofont[Path=/usr/share/fonts/truetype/cmu/,UprightFont=cmuntt.ttf,BoldFont=cmuntb.ttf,ItalicFont=cmunit.ttf,BoldItalicFont=cmuntx.ttf]{cmuntt.ttf}\ttfamily \textbackslash{}maketitle}{$\text{ }$}\setmainfont[Path=/usr/share/fonts/truetype/cmu/,UprightFont=cmunrm.ttf,BoldFont=cmunbx.ttf,ItalicFont=cmunti.ttf,BoldItalicFont=cmunbi.ttf]{cmunrm.ttf}\setmonofont[Path=/usr/share/fonts/truetype/cmu/,UprightFont=cmuntt.ttf,BoldFont=cmuntb.ttf,ItalicFont=cmunit.ttf,BoldItalicFont=cmuntx.ttf]{cmunrm.ttf} is possible to create the frontpage, if {\ttfamily \setmainfont[Path=/usr/share/fonts/truetype/cmu/,UprightFont=cmunrm.ttf,BoldFont=cmunbx.ttf,ItalicFont=cmunti.ttf,BoldItalicFont=cmunbi.ttf]{cmuntt.ttf}\setmonofont[Path=/usr/share/fonts/truetype/cmu/,UprightFont=cmuntt.ttf,BoldFont=cmuntb.ttf,ItalicFont=cmunit.ttf,BoldItalicFont=cmuntx.ttf]{cmuntt.ttf}\ttfamily title}{$\text{ }$}\setmainfont[Path=/usr/share/fonts/truetype/cmu/,UprightFont=cmunrm.ttf,BoldFont=cmunbx.ttf,ItalicFont=cmunti.ttf,BoldItalicFont=cmunbi.ttf]{cmunrm.ttf}\setmonofont[Path=/usr/share/fonts/truetype/cmu/,UprightFont=cmuntt.ttf,BoldFont=cmuntb.ttf,ItalicFont=cmunit.ttf,BoldItalicFont=cmuntx.ttf]{cmunrm.ttf} and {\ttfamily \setmainfont[Path=/usr/share/fonts/truetype/cmu/,UprightFont=cmunrm.ttf,BoldFont=cmunbx.ttf,ItalicFont=cmunti.ttf,BoldItalicFont=cmunbi.ttf]{cmuntt.ttf}\setmonofont[Path=/usr/share/fonts/truetype/cmu/,UprightFont=cmuntt.ttf,BoldFont=cmuntb.ttf,ItalicFont=cmunit.ttf,BoldItalicFont=cmuntx.ttf]{cmuntt.ttf}\ttfamily author}{$\text{ }$}\setmainfont[Path=/usr/share/fonts/truetype/cmu/,UprightFont=cmunrm.ttf,BoldFont=cmunbx.ttf,ItalicFont=cmunti.ttf,BoldItalicFont=cmunbi.ttf]{cmunrm.ttf}\setmonofont[Path=/usr/share/fonts/truetype/cmu/,UprightFont=cmuntt.ttf,BoldFont=cmuntb.ttf,ItalicFont=cmunit.ttf,BoldItalicFont=cmuntx.ttf]{cmunrm.ttf} are set.

Trick: Instead of using 
\begin{Shaded}
\begin{Highlighting}[]

\NormalTok{\textbackslash{}begin\{frame\}...\textbackslash{}end\{frame\}}\newline
\end{Highlighting}
\end{Shaded}
, you can also use 
\begin{Shaded}
\begin{Highlighting}[]

\NormalTok{\textbackslash{}frame\{...\}}\newline
\end{Highlighting}
\end{Shaded}
.

For the actual talk, if you can compile it with {\ttfamily \setmainfont[Path=/usr/share/fonts/truetype/cmu/,UprightFont=cmunrm.ttf,BoldFont=cmunbx.ttf,ItalicFont=cmunti.ttf,BoldItalicFont=cmunbi.ttf]{cmuntt.ttf}\setmonofont[Path=/usr/share/fonts/truetype/cmu/,UprightFont=cmuntt.ttf,BoldFont=cmuntb.ttf,ItalicFont=cmunit.ttf,BoldItalicFont=cmuntx.ttf]{cmuntt.ttf}\ttfamily pdflatex}{$\text{ }$}\setmainfont[Path=/usr/share/fonts/truetype/cmu/,UprightFont=cmunrm.ttf,BoldFont=cmunbx.ttf,ItalicFont=cmunti.ttf,BoldItalicFont=cmunbi.ttf]{cmunrm.ttf}\setmonofont[Path=/usr/share/fonts/truetype/cmu/,UprightFont=cmuntt.ttf,BoldFont=cmuntb.ttf,ItalicFont=cmunit.ttf,BoldItalicFont=cmuntx.ttf]{cmunrm.ttf} then you could use a pdf reader with a fullscreen mode, such as \myhref{https://en.wikipedia.org/wiki/Okular}{Okular}, \myhref{https://en.wikipedia.org/wiki/Evince}{Evince} or Adobe Reader.  If you want to navigate in your presentation, you can use the almost invisible links in the bottom right corner without leaving the fullscreen mode.
\subsection{Document Structure}
\label{732}
\subsubsection{Title page and information}
\label{733}

First, you give information about authors, titles and dates in the preamble.

\begin{Shaded}
\begin{Highlighting}[]

\NormalTok{\textbackslash{}title[Crisis]\ensuremath{\text{ }}}\CommentTok{\%\ensuremath{\text{ }}(optional,\ensuremath{\text{ }}only\ensuremath{\text{ }}for\ensuremath{\text{ }}long\ensuremath{\text{ }}titles)}\newline
\NormalTok{\{The\ensuremath{\text{ }}Economics\ensuremath{\text{ }}of\ensuremath{\text{ }}Financial\ensuremath{\text{ }}Crisis\}}\newline
\NormalTok{\textbackslash{}subtitle\{Evidence\ensuremath{\text{ }}from\ensuremath{\text{ }}India\}}\newline
\NormalTok{\textbackslash{}author[Author,\ensuremath{\text{ }}Anders]\ensuremath{\text{ }}}\CommentTok{\%\ensuremath{\text{ }}(optional,\ensuremath{\text{ }}for\ensuremath{\text{ }}multiple\ensuremath{\text{ }}authors)}\newline
\NormalTok{\{F.~Author\textbackslash{}inst\{1\}\ensuremath{\text{ }}\textbackslash{}and\ensuremath{\text{ }}S.~Anders\textbackslash{}inst\{2\}\}}\newline
\NormalTok{\textbackslash{}institute[Universities\ensuremath{\text{ }}Here\ensuremath{\text{ }}and\ensuremath{\text{ }}There]\ensuremath{\text{ }}}\CommentTok{\%\ensuremath{\text{ }}(optional)}\newline
\NormalTok{\{}\newline
\ensuremath{\text{ }}\ensuremath{\text{ }}\NormalTok{\textbackslash{}inst\{1\}}\CommentTok{\%}\newline
\ensuremath{\text{ }}\ensuremath{\text{ }}\NormalTok{Institute\ensuremath{\text{ }}of\ensuremath{\text{ }}Computer\ensuremath{\text{ }}Science\textbackslash{}\textbackslash{}}\newline
\ensuremath{\text{ }}\ensuremath{\text{ }}\NormalTok{University\ensuremath{\text{ }}Here}\newline
\ensuremath{\text{ }}\ensuremath{\text{ }}\NormalTok{\textbackslash{}and}\newline
\ensuremath{\text{ }}\ensuremath{\text{ }}\NormalTok{\textbackslash{}inst\{2\}}\CommentTok{\%}\newline
\ensuremath{\text{ }}\ensuremath{\text{ }}\NormalTok{Institute\ensuremath{\text{ }}of\ensuremath{\text{ }}Theoretical\ensuremath{\text{ }}Philosophy\textbackslash{}\textbackslash{}}\newline
\ensuremath{\text{ }}\ensuremath{\text{ }}\NormalTok{University\ensuremath{\text{ }}There}\newline
\NormalTok{\}}\newline
\NormalTok{\textbackslash{}date[KPT\ensuremath{\text{ }}2004]\ensuremath{\text{ }}}\CommentTok{\%\ensuremath{\text{ }}(optional)}\newline
\NormalTok{\{Conference\ensuremath{\text{ }}on\ensuremath{\text{ }}Presentation\ensuremath{\text{ }}Techniques,\ensuremath{\text{ }}2004\}}\newline
\NormalTok{\textbackslash{}subject\{Computer\ensuremath{\text{ }}Science\}}\newline
\end{Highlighting}
\end{Shaded}


Then, in the document, you add the title page : 

\begin{Shaded}
\begin{Highlighting}[]

\NormalTok{\textbackslash{}frame\{\textbackslash{}titlepage\}}\newline
\end{Highlighting}
\end{Shaded}

\subsubsection{Table of Contents}
\label{734}
The table of contents, with the current section highlighted, is displayed by:


\begin{Shaded}
\begin{Highlighting}[]

\NormalTok{\textbackslash{}begin\{frame\}}\newline
\NormalTok{\textbackslash{}frametitle\{Table\ensuremath{\text{ }}of\ensuremath{\text{ }}Contents\}}\newline
\NormalTok{\textbackslash{}tableofcontents[currentsection]}\newline
\NormalTok{\textbackslash{}end\{frame\}}\newline
\end{Highlighting}
\end{Shaded}


This can be done automatically at the beginning of each section using the following code in the preamble:


\begin{Shaded}
\begin{Highlighting}[]

\NormalTok{\textbackslash{}AtBeginSection[]}\newline
\NormalTok{\{}\newline
\ensuremath{\text{ }}\ensuremath{\text{ }}\NormalTok{\textbackslash{}begin\{frame\}}\newline
\ensuremath{\text{ }}\ensuremath{\text{ }}\ensuremath{\text{ }}\ensuremath{\text{ }}\NormalTok{\textbackslash{}frametitle\{Table\ensuremath{\text{ }}of\ensuremath{\text{ }}Contents\}}\newline
\ensuremath{\text{ }}\ensuremath{\text{ }}\ensuremath{\text{ }}\ensuremath{\text{ }}\NormalTok{\textbackslash{}tableofcontents[currentsection]}\newline
\ensuremath{\text{ }}\ensuremath{\text{ }}\NormalTok{\textbackslash{}end\{frame\}}\newline
\NormalTok{\}\ensuremath{\text{ }}}\newline
\end{Highlighting}
\end{Shaded}


Or for subsections: 


\begin{Shaded}
\begin{Highlighting}[]

\NormalTok{\textbackslash{}AtBeginSubsection[]}\newline
\NormalTok{\{}\newline
\ensuremath{\text{ }}\ensuremath{\text{ }}\NormalTok{\textbackslash{}begin\{frame\}}\newline
\ensuremath{\text{ }}\ensuremath{\text{ }}\ensuremath{\text{ }}\ensuremath{\text{ }}\NormalTok{\textbackslash{}frametitle\{Table\ensuremath{\text{ }}of\ensuremath{\text{ }}Contents\}}\newline
\ensuremath{\text{ }}\ensuremath{\text{ }}\ensuremath{\text{ }}\ensuremath{\text{ }}\NormalTok{\textbackslash{}tableofcontents[currentsection,currentsubsection]}\newline
\ensuremath{\text{ }}\ensuremath{\text{ }}\NormalTok{\textbackslash{}end\{frame\}}\newline
\NormalTok{\}}\newline
\end{Highlighting}
\end{Shaded}

\subsubsection{Sections and subsections}
\label{735}

As in all other LaTeX files, it is possible to structure the document using 

\begin{Shaded}
\begin{Highlighting}[]

\NormalTok{\textbackslash{}section[Section]\{My\ensuremath{\text{ }}section\}}\newline
\end{Highlighting}
\end{Shaded}

,

\begin{Shaded}
\begin{Highlighting}[]

\NormalTok{\textbackslash{}subsection[Subsection]\{My\ensuremath{\text{ }}subsection\}}\newline
\end{Highlighting}
\end{Shaded}

and

\begin{Shaded}
\begin{Highlighting}[]

\NormalTok{\textbackslash{}subsubsection[Subsubsection]\{My\ensuremath{\text{ }}subsubsection\}}\newline
\end{Highlighting}
\end{Shaded}


Those commands have to be put before and between frames. They will modify the Table of contents with the optional argument. The argument in brackets will be written on the slide, depending on the used theme.
\subsubsection{References (Beamer)}
\label{736}

Beamer does not officially support BibTeX.  Instead bibliography items will need to be partly set \symbol{34}by hand\symbol{34} (see \myhref{http://www.tex.ac.uk/tex-archive/macros/latex/contrib/beamer/doc/beameruserguide.pdf}{beameruserguide.pdf 3.12}). The following example shows a references slide containing two entries:


\begin{Shaded}
\begin{Highlighting}[]

\NormalTok{\textbackslash{}begin\{frame\}[allowframebreaks]}\newline
\ensuremath{\text{ }}\ensuremath{\text{ }}\NormalTok{\textbackslash{}frametitle<presentation>\{Further\ensuremath{\text{ }}Reading\}\ensuremath{\text{ }}\ensuremath{\text{ }}\ensuremath{\text{ }}\ensuremath{\text{ }}}\newline
\ensuremath{\text{ }}\ensuremath{\text{ }}\NormalTok{\textbackslash{}begin\{thebibliography\}\{10\}\ensuremath{\text{ }}\ensuremath{\text{ }}\ensuremath{\text{ }}\ensuremath{\text{ }}}\newline
\ensuremath{\text{ }}\ensuremath{\text{ }}\NormalTok{\textbackslash{}beamertemplatebookbibitems}\newline
\ensuremath{\text{ }}\ensuremath{\text{ }}\NormalTok{\textbackslash{}bibitem\{Autor1990\}}\newline
\ensuremath{\text{ }}\ensuremath{\text{ }}\ensuremath{\text{ }}\ensuremath{\text{ }}\NormalTok{A.~Autor.}\newline
\ensuremath{\text{ }}\ensuremath{\text{ }}\ensuremath{\text{ }}\ensuremath{\text{ }}\NormalTok{\textbackslash{}newblock\ensuremath{\text{ }}\{\textbackslash{}em\ensuremath{\text{ }}Introduction\ensuremath{\text{ }}to\ensuremath{\text{ }}Giving\ensuremath{\text{ }}Presentations\}.}\newline
\ensuremath{\text{ }}\ensuremath{\text{ }}\ensuremath{\text{ }}\ensuremath{\text{ }}\NormalTok{\textbackslash{}newblock\ensuremath{\text{ }}Klein-Verlag,\ensuremath{\text{ }}1990.}\newline
\ensuremath{\text{ }}\ensuremath{\text{ }}\NormalTok{\textbackslash{}beamertemplatearticlebibitems}\newline
\ensuremath{\text{ }}\ensuremath{\text{ }}\NormalTok{\textbackslash{}bibitem\{Jemand2000\}}\newline
\ensuremath{\text{ }}\ensuremath{\text{ }}\ensuremath{\text{ }}\ensuremath{\text{ }}\NormalTok{S.~Jemand.}\newline
\ensuremath{\text{ }}\ensuremath{\text{ }}\ensuremath{\text{ }}\ensuremath{\text{ }}\NormalTok{\textbackslash{}newblock\ensuremath{\text{ }}On\ensuremath{\text{ }}this\ensuremath{\text{ }}and\ensuremath{\text{ }}that.}\newline
\ensuremath{\text{ }}\ensuremath{\text{ }}\ensuremath{\text{ }}\ensuremath{\text{ }}\NormalTok{\textbackslash{}newblock\ensuremath{\text{ }}\{\textbackslash{}em\ensuremath{\text{ }}Journal\ensuremath{\text{ }}of\ensuremath{\text{ }}This\ensuremath{\text{ }}and\ensuremath{\text{ }}That\},\ensuremath{\text{ }}2(1):50--100,\ensuremath{\text{ }}2000.}\newline
\ensuremath{\text{ }}\ensuremath{\text{ }}\NormalTok{\textbackslash{}end\{thebibliography\}}\newline
\NormalTok{\textbackslash{}end\{frame\}}\newline
\end{Highlighting}
\end{Shaded}


As the reference list grows, the reference slide will divide into two slides and so on, through use of the {\ttfamily \setmainfont[Path=/usr/share/fonts/truetype/cmu/,UprightFont=cmunrm.ttf,BoldFont=cmunbx.ttf,ItalicFont=cmunti.ttf,BoldItalicFont=cmunbi.ttf]{cmuntt.ttf}\setmonofont[Path=/usr/share/fonts/truetype/cmu/,UprightFont=cmuntt.ttf,BoldFont=cmuntb.ttf,ItalicFont=cmunit.ttf,BoldItalicFont=cmuntx.ttf]{cmuntt.ttf}\ttfamily allowframebreaks}{$\text{ }$}\setmainfont[Path=/usr/share/fonts/truetype/cmu/,UprightFont=cmunrm.ttf,BoldFont=cmunbx.ttf,ItalicFont=cmunti.ttf,BoldItalicFont=cmunbi.ttf]{cmunrm.ttf}\setmonofont[Path=/usr/share/fonts/truetype/cmu/,UprightFont=cmuntt.ttf,BoldFont=cmuntb.ttf,ItalicFont=cmunit.ttf,BoldItalicFont=cmuntx.ttf]{cmunrm.ttf} option.  Individual items can be cited after adding an \textquotesingle{}optional\textquotesingle{} label to the relevant {\ttfamily \setmainfont[Path=/usr/share/fonts/truetype/cmu/,UprightFont=cmunrm.ttf,BoldFont=cmunbx.ttf,ItalicFont=cmunti.ttf,BoldItalicFont=cmunbi.ttf]{cmuntt.ttf}\setmonofont[Path=/usr/share/fonts/truetype/cmu/,UprightFont=cmuntt.ttf,BoldFont=cmuntb.ttf,ItalicFont=cmunit.ttf,BoldItalicFont=cmuntx.ttf]{cmuntt.ttf}\ttfamily bibitem}{$\text{ }$}\setmainfont[Path=/usr/share/fonts/truetype/cmu/,UprightFont=cmunrm.ttf,BoldFont=cmunbx.ttf,ItalicFont=cmunti.ttf,BoldItalicFont=cmunbi.ttf]{cmunrm.ttf}\setmonofont[Path=/usr/share/fonts/truetype/cmu/,UprightFont=cmuntt.ttf,BoldFont=cmuntb.ttf,ItalicFont=cmunit.ttf,BoldItalicFont=cmuntx.ttf]{cmunrm.ttf} stanza.  The citation call is simply 
\begin{Shaded}
\begin{Highlighting}[]

\NormalTok{\textbackslash{}cite}\newline
\end{Highlighting}
\end{Shaded}
. Beamer also supports limited customization of the way references are presented (see the manual). Those who wish to use \myhref{http://www.ctan.org/pkg/natbib/}{natbib}, for example, with Beamer may need to troubleshoot both their document setup and the relevant BibTeX style file.

The different types of referenced work are indicated with a little symbol (e.g. a book, an article, etc.). The Symbol is set with the commands {\ttfamily \setmainfont[Path=/usr/share/fonts/truetype/cmu/,UprightFont=cmunrm.ttf,BoldFont=cmunbx.ttf,ItalicFont=cmunti.ttf,BoldItalicFont=cmunbi.ttf]{cmuntt.ttf}\setmonofont[Path=/usr/share/fonts/truetype/cmu/,UprightFont=cmuntt.ttf,BoldFont=cmuntb.ttf,ItalicFont=cmunit.ttf,BoldItalicFont=cmuntx.ttf]{cmuntt.ttf}\ttfamily beamertemplatebookbibitems}{$\text{ }$}\setmainfont[Path=/usr/share/fonts/truetype/cmu/,UprightFont=cmunrm.ttf,BoldFont=cmunbx.ttf,ItalicFont=cmunti.ttf,BoldItalicFont=cmunbi.ttf]{cmunrm.ttf}\setmonofont[Path=/usr/share/fonts/truetype/cmu/,UprightFont=cmuntt.ttf,BoldFont=cmuntb.ttf,ItalicFont=cmunit.ttf,BoldItalicFont=cmuntx.ttf]{cmunrm.ttf} and {\ttfamily {$\text{ }$}\setmainfont[Path=/usr/share/fonts/truetype/cmu/,UprightFont=cmunrm.ttf,BoldFont=cmunbx.ttf,ItalicFont=cmunti.ttf,BoldItalicFont=cmunbi.ttf]{cmuntt.ttf}\setmonofont[Path=/usr/share/fonts/truetype/cmu/,UprightFont=cmuntt.ttf,BoldFont=cmuntb.ttf,ItalicFont=cmunit.ttf,BoldItalicFont=cmuntx.ttf]{cmuntt.ttf}\ttfamily  beamertemplatearticlebibitems}\setmainfont[Path=/usr/share/fonts/truetype/cmu/,UprightFont=cmunrm.ttf,BoldFont=cmunbx.ttf,ItalicFont=cmunti.ttf,BoldItalicFont=cmunbi.ttf]{cmunrm.ttf}\setmonofont[Path=/usr/share/fonts/truetype/cmu/,UprightFont=cmuntt.ttf,BoldFont=cmuntb.ttf,ItalicFont=cmunit.ttf,BoldItalicFont=cmuntx.ttf]{cmunrm.ttf}. It is also possible to use {\ttfamily \setmainfont[Path=/usr/share/fonts/truetype/cmu/,UprightFont=cmunrm.ttf,BoldFont=cmunbx.ttf,ItalicFont=cmunti.ttf,BoldItalicFont=cmunbi.ttf]{cmuntt.ttf}\setmonofont[Path=/usr/share/fonts/truetype/cmu/,UprightFont=cmuntt.ttf,BoldFont=cmuntb.ttf,ItalicFont=cmunit.ttf,BoldItalicFont=cmuntx.ttf]{cmuntt.ttf}\ttfamily setbeamertemplate}{$\text{ }$}\setmainfont[Path=/usr/share/fonts/truetype/cmu/,UprightFont=cmunrm.ttf,BoldFont=cmunbx.ttf,ItalicFont=cmunti.ttf,BoldItalicFont=cmunbi.ttf]{cmunrm.ttf}\setmonofont[Path=/usr/share/fonts/truetype/cmu/,UprightFont=cmuntt.ttf,BoldFont=cmuntb.ttf,ItalicFont=cmunit.ttf,BoldItalicFont=cmuntx.ttf]{cmunrm.ttf} directly, like so



\begin{Shaded}
\begin{Highlighting}[]

\NormalTok{\textbackslash{}begin\{frame\}[allowframebreaks]}\newline
\ensuremath{\text{ }}\ensuremath{\text{ }}\NormalTok{\textbackslash{}frametitle<presentation>\{Further\ensuremath{\text{ }}Reading\}\ensuremath{\text{ }}\ensuremath{\text{ }}\ensuremath{\text{ }}\ensuremath{\text{ }}}\newline
\ensuremath{\text{ }}\ensuremath{\text{ }}\NormalTok{\textbackslash{}begin\{thebibliography\}\{10\}\ensuremath{\text{ }}\ensuremath{\text{ }}\ensuremath{\text{ }}\ensuremath{\text{ }}}\newline
\ensuremath{\text{ }}\ensuremath{\text{ }}\NormalTok{\textbackslash{}setbeamertemplate\{bibliography\ensuremath{\text{ }}item\}[book]}\newline
\ensuremath{\text{ }}\ensuremath{\text{ }}\NormalTok{\textbackslash{}bibitem\{Autor1990\}}\newline
\ensuremath{\text{ }}\ensuremath{\text{ }}\ensuremath{\text{ }}\ensuremath{\text{ }}\NormalTok{A.~Autor.}\newline
\ensuremath{\text{ }}\ensuremath{\text{ }}\ensuremath{\text{ }}\ensuremath{\text{ }}\NormalTok{\textbackslash{}newblock\ensuremath{\text{ }}\{\textbackslash{}em\ensuremath{\text{ }}Introduction\ensuremath{\text{ }}to\ensuremath{\text{ }}Giving\ensuremath{\text{ }}Presentations\}.}\newline
\ensuremath{\text{ }}\ensuremath{\text{ }}\ensuremath{\text{ }}\ensuremath{\text{ }}\NormalTok{\textbackslash{}newblock\ensuremath{\text{ }}Klein-Verlag,\ensuremath{\text{ }}1990.}\newline
\ensuremath{\text{ }}\ensuremath{\text{ }}\NormalTok{\textbackslash{}setbeamertemplate\{bibliography\ensuremath{\text{ }}item\}[article]}\newline
\ensuremath{\text{ }}\ensuremath{\text{ }}\NormalTok{\textbackslash{}bibitem\{Jemand2000\}}\newline
\ensuremath{\text{ }}\ensuremath{\text{ }}\ensuremath{\text{ }}\ensuremath{\text{ }}\NormalTok{S.~Jemand.}\newline
\ensuremath{\text{ }}\ensuremath{\text{ }}\ensuremath{\text{ }}\ensuremath{\text{ }}\NormalTok{\textbackslash{}newblock\ensuremath{\text{ }}On\ensuremath{\text{ }}this\ensuremath{\text{ }}and\ensuremath{\text{ }}that.}\newline
\ensuremath{\text{ }}\ensuremath{\text{ }}\ensuremath{\text{ }}\ensuremath{\text{ }}\NormalTok{\textbackslash{}newblock\ensuremath{\text{ }}\{\textbackslash{}em\ensuremath{\text{ }}Journal\ensuremath{\text{ }}of\ensuremath{\text{ }}This\ensuremath{\text{ }}and\ensuremath{\text{ }}That\},\ensuremath{\text{ }}2(1):50--100,\ensuremath{\text{ }}2000.}\newline
\ensuremath{\text{ }}\ensuremath{\text{ }}\NormalTok{\textbackslash{}end\{thebibliography\}}\newline
\NormalTok{\textbackslash{}end\{frame\}}\newline
\end{Highlighting}
\end{Shaded}


Other possible types of bibliography items, besides {\ttfamily {$\text{ }$}\setmainfont[Path=/usr/share/fonts/truetype/cmu/,UprightFont=cmunrm.ttf,BoldFont=cmunbx.ttf,ItalicFont=cmunti.ttf,BoldItalicFont=cmunbi.ttf]{cmuntt.ttf}\setmonofont[Path=/usr/share/fonts/truetype/cmu/,UprightFont=cmuntt.ttf,BoldFont=cmuntb.ttf,ItalicFont=cmunit.ttf,BoldItalicFont=cmuntx.ttf]{cmuntt.ttf}\ttfamily  book}\setmainfont[Path=/usr/share/fonts/truetype/cmu/,UprightFont=cmunrm.ttf,BoldFont=cmunbx.ttf,ItalicFont=cmunti.ttf,BoldItalicFont=cmunbi.ttf]{cmunrm.ttf}\setmonofont[Path=/usr/share/fonts/truetype/cmu/,UprightFont=cmuntt.ttf,BoldFont=cmuntb.ttf,ItalicFont=cmunit.ttf,BoldItalicFont=cmuntx.ttf]{cmunrm.ttf} and {\ttfamily {$\text{ }$}\setmainfont[Path=/usr/share/fonts/truetype/cmu/,UprightFont=cmunrm.ttf,BoldFont=cmunbx.ttf,ItalicFont=cmunti.ttf,BoldItalicFont=cmunbi.ttf]{cmuntt.ttf}\setmonofont[Path=/usr/share/fonts/truetype/cmu/,UprightFont=cmuntt.ttf,BoldFont=cmuntb.ttf,ItalicFont=cmunit.ttf,BoldItalicFont=cmuntx.ttf]{cmuntt.ttf}\ttfamily  article}\setmainfont[Path=/usr/share/fonts/truetype/cmu/,UprightFont=cmunrm.ttf,BoldFont=cmunbx.ttf,ItalicFont=cmunti.ttf,BoldItalicFont=cmunbi.ttf]{cmunrm.ttf}\setmonofont[Path=/usr/share/fonts/truetype/cmu/,UprightFont=cmuntt.ttf,BoldFont=cmuntb.ttf,ItalicFont=cmunit.ttf,BoldItalicFont=cmuntx.ttf]{cmunrm.ttf}, include e.g. {\ttfamily {$\text{ }$}\setmainfont[Path=/usr/share/fonts/truetype/cmu/,UprightFont=cmunrm.ttf,BoldFont=cmunbx.ttf,ItalicFont=cmunti.ttf,BoldItalicFont=cmunbi.ttf]{cmuntt.ttf}\setmonofont[Path=/usr/share/fonts/truetype/cmu/,UprightFont=cmuntt.ttf,BoldFont=cmuntb.ttf,ItalicFont=cmunit.ttf,BoldItalicFont=cmuntx.ttf]{cmuntt.ttf}\ttfamily  online}\setmainfont[Path=/usr/share/fonts/truetype/cmu/,UprightFont=cmunrm.ttf,BoldFont=cmunbx.ttf,ItalicFont=cmunti.ttf,BoldItalicFont=cmunbi.ttf]{cmunrm.ttf}\setmonofont[Path=/usr/share/fonts/truetype/cmu/,UprightFont=cmuntt.ttf,BoldFont=cmuntb.ttf,ItalicFont=cmunit.ttf,BoldItalicFont=cmuntx.ttf]{cmunrm.ttf}, {\ttfamily {$\text{ }$}\setmainfont[Path=/usr/share/fonts/truetype/cmu/,UprightFont=cmunrm.ttf,BoldFont=cmunbx.ttf,ItalicFont=cmunti.ttf,BoldItalicFont=cmunbi.ttf]{cmuntt.ttf}\setmonofont[Path=/usr/share/fonts/truetype/cmu/,UprightFont=cmuntt.ttf,BoldFont=cmuntb.ttf,ItalicFont=cmunit.ttf,BoldItalicFont=cmuntx.ttf]{cmuntt.ttf}\ttfamily  triangle}\setmainfont[Path=/usr/share/fonts/truetype/cmu/,UprightFont=cmunrm.ttf,BoldFont=cmunbx.ttf,ItalicFont=cmunti.ttf,BoldItalicFont=cmunbi.ttf]{cmunrm.ttf}\setmonofont[Path=/usr/share/fonts/truetype/cmu/,UprightFont=cmuntt.ttf,BoldFont=cmuntb.ttf,ItalicFont=cmunit.ttf,BoldItalicFont=cmuntx.ttf]{cmunrm.ttf} and {\ttfamily {$\text{ }$}\setmainfont[Path=/usr/share/fonts/truetype/cmu/,UprightFont=cmunrm.ttf,BoldFont=cmunbx.ttf,ItalicFont=cmunti.ttf,BoldItalicFont=cmunbi.ttf]{cmuntt.ttf}\setmonofont[Path=/usr/share/fonts/truetype/cmu/,UprightFont=cmuntt.ttf,BoldFont=cmuntb.ttf,ItalicFont=cmunit.ttf,BoldItalicFont=cmuntx.ttf]{cmuntt.ttf}\ttfamily  text}\setmainfont[Path=/usr/share/fonts/truetype/cmu/,UprightFont=cmunrm.ttf,BoldFont=cmunbx.ttf,ItalicFont=cmunti.ttf,BoldItalicFont=cmunbi.ttf]{cmunrm.ttf}\setmonofont[Path=/usr/share/fonts/truetype/cmu/,UprightFont=cmuntt.ttf,BoldFont=cmuntb.ttf,ItalicFont=cmunit.ttf,BoldItalicFont=cmuntx.ttf]{cmunrm.ttf}. It is also possible to have user defined bibliography items by including a graphic.

If one wants to have full references appear as foot notes, use the 
\begin{Shaded}
\begin{Highlighting}[]

\NormalTok{\textbackslash{}footfullcite}\newline
\end{Highlighting}
\end{Shaded}
.
For example, it is possible to use

\begin{Shaded}
\begin{Highlighting}[]

\NormalTok{\textbackslash{}documentclass[10pt,handout,english]\{beamer\}}\newline
\NormalTok{\textbackslash{}usepackage[english]\{babel\}}\newline
\NormalTok{\textbackslash{}usepackage[backend=biber,style=numeric-comp,sorting=none]\{biblatex\}}\newline
\NormalTok{\textbackslash{}addbibresource\{biblio.bib\}}\newline
\ensuremath{\text{ }}\newline
\NormalTok{\textbackslash{}begin\{frame\}}\newline
\ensuremath{\text{ }}\NormalTok{\textbackslash{}frametitle\{Title\}}\newline
\ensuremath{\text{ }}\NormalTok{A\ensuremath{\text{ }}reference~\textbackslash{}footfullcite\{ref_bib\},\ensuremath{\text{ }}with\ensuremath{\text{ }}ref_bib\ensuremath{\text{ }}an\ensuremath{\text{ }}item\ensuremath{\text{ }}of\ensuremath{\text{ }}the\ensuremath{\text{ }}.bib\ensuremath{\text{ }}file.}\newline
\NormalTok{\textbackslash{}end\{frame\}}\newline
\end{Highlighting}
\end{Shaded}

\subsection{Style}
\label{737}
\subsubsection{Themes}
\label{738}

The first solution is to use a built-{}in theme such as Warsaw, Berlin, etc. The second solution is to specify colors, inner themes and outer themes.
\paragraph{The Built-{}in solution}
{$\text{ }$}\newline\label{739}

To the preamble you can add the following line:


\begin{Shaded}
\begin{Highlighting}[]

\NormalTok{\textbackslash{}usetheme\{Warsaw\}}\newline
\end{Highlighting}
\end{Shaded}


to use the \symbol{34}Warsaw\symbol{34} theme. {\ttfamily \setmainfont[Path=/usr/share/fonts/truetype/cmu/,UprightFont=cmunrm.ttf,BoldFont=cmunbx.ttf,ItalicFont=cmunti.ttf,BoldItalicFont=cmunbi.ttf]{cmuntt.ttf}\setmonofont[Path=/usr/share/fonts/truetype/cmu/,UprightFont=cmuntt.ttf,BoldFont=cmuntb.ttf,ItalicFont=cmunit.ttf,BoldItalicFont=cmuntx.ttf]{cmuntt.ttf}\ttfamily Beamer}{$\text{ }$}\setmainfont[Path=/usr/share/fonts/truetype/cmu/,UprightFont=cmunrm.ttf,BoldFont=cmunbx.ttf,ItalicFont=cmunti.ttf,BoldItalicFont=cmunbi.ttf]{cmunrm.ttf}\setmonofont[Path=/usr/share/fonts/truetype/cmu/,UprightFont=cmuntt.ttf,BoldFont=cmuntb.ttf,ItalicFont=cmunit.ttf,BoldItalicFont=cmuntx.ttf]{cmunrm.ttf} has several themes, many of which are named after cities (e.g. Frankfurt, Madrid, Berlin, etc.).

This \myhref{http://www.hartwork.org/beamer-theme-matrix/}{Theme Matrix} contains the various theme and color combinations included with {\ttfamily \setmainfont[Path=/usr/share/fonts/truetype/cmu/,UprightFont=cmunrm.ttf,BoldFont=cmunbx.ttf,ItalicFont=cmunti.ttf,BoldItalicFont=cmunbi.ttf]{cmuntt.ttf}\setmonofont[Path=/usr/share/fonts/truetype/cmu/,UprightFont=cmuntt.ttf,BoldFont=cmuntb.ttf,ItalicFont=cmunit.ttf,BoldItalicFont=cmuntx.ttf]{cmuntt.ttf}\ttfamily beamer}\setmainfont[Path=/usr/share/fonts/truetype/cmu/,UprightFont=cmunrm.ttf,BoldFont=cmunbx.ttf,ItalicFont=cmunti.ttf,BoldItalicFont=cmunbi.ttf]{cmunrm.ttf}\setmonofont[Path=/usr/share/fonts/truetype/cmu/,UprightFont=cmuntt.ttf,BoldFont=cmuntb.ttf,ItalicFont=cmunit.ttf,BoldItalicFont=cmuntx.ttf]{cmunrm.ttf}. For more customizing options, have a look to the official documentation included in your distribution of {\ttfamily \setmainfont[Path=/usr/share/fonts/truetype/cmu/,UprightFont=cmunrm.ttf,BoldFont=cmunbx.ttf,ItalicFont=cmunti.ttf,BoldItalicFont=cmunbi.ttf]{cmuntt.ttf}\setmonofont[Path=/usr/share/fonts/truetype/cmu/,UprightFont=cmuntt.ttf,BoldFont=cmuntb.ttf,ItalicFont=cmunit.ttf,BoldItalicFont=cmuntx.ttf]{cmuntt.ttf}\ttfamily beamer}\setmainfont[Path=/usr/share/fonts/truetype/cmu/,UprightFont=cmunrm.ttf,BoldFont=cmunbx.ttf,ItalicFont=cmunti.ttf,BoldItalicFont=cmunbi.ttf]{cmunrm.ttf}\setmonofont[Path=/usr/share/fonts/truetype/cmu/,UprightFont=cmuntt.ttf,BoldFont=cmuntb.ttf,ItalicFont=cmunit.ttf,BoldItalicFont=cmuntx.ttf]{cmunrm.ttf}, particularly the part {\itshape \setmainfont[Path=/usr/share/fonts/truetype/cmu/,UprightFont=cmunrm.ttf,BoldFont=cmunbx.ttf,ItalicFont=cmunti.ttf,BoldItalicFont=cmunbi.ttf]{cmunti.ttf}\setmonofont[Path=/usr/share/fonts/truetype/cmu/,UprightFont=cmuntt.ttf,BoldFont=cmuntb.ttf,ItalicFont=cmunit.ttf,BoldItalicFont=cmuntx.ttf]{cmunti.ttf}\itshape Change the way it looks}\setmainfont[Path=/usr/share/fonts/truetype/cmu/,UprightFont=cmunrm.ttf,BoldFont=cmunbx.ttf,ItalicFont=cmunti.ttf,BoldItalicFont=cmunbi.ttf]{cmunrm.ttf}\setmonofont[Path=/usr/share/fonts/truetype/cmu/,UprightFont=cmuntt.ttf,BoldFont=cmuntb.ttf,ItalicFont=cmunit.ttf,BoldItalicFont=cmuntx.ttf]{cmunrm.ttf}.

The full list of themes is:{\scriptsize{}
{\scalefont{0.83810}\begin{longtable}{>{\RaggedRight}p{0.13907\linewidth}>{\RaggedRight}p{0.13824\linewidth}>{\RaggedRight}p{0.13907\linewidth}>{\RaggedRight}p{0.14391\linewidth}>{\RaggedRight}p{0.12920\linewidth}>{\RaggedRight}p{0.13907\linewidth}} 
\hspace*{0pt}\ignorespaces{}\hspace*{0pt}\begin{myitemize}\item{}  Antibes\item{}  Bergen\item{}  Berkeley\item{}  Berlin\item{}  Copenhagen\end{myitemize}&\hspace*{0pt}\ignorespaces{}\hspace*{0pt}\begin{myitemize}\item{}  Darmstadt\item{}  Dresden\item{}  Frankfurt\item{}  Goettingen\item{}  Hannover\end{myitemize}&\hspace*{0pt}\ignorespaces{}\hspace*{0pt}\begin{myitemize}\item{}  Ilmenau\item{}  JuanLesPins\item{}  Luebeck\item{}  Madrid\item{}  Malmoe\end{myitemize}&\hspace*{0pt}\ignorespaces{}\hspace*{0pt}\begin{myitemize}\item{}  Marburg\item{}  Montpellier\item{}  PaloAlto\item{}  Pittsburgh\item{}  Rochester\end{myitemize}&\hspace*{0pt}\ignorespaces{}\hspace*{0pt}\begin{myitemize}\item{}  Singapore\item{}  Szeged\item{}  Warsaw\item{}  boxes\item{}  default\end{myitemize}&\hspace*{0pt}\ignorespaces{}\hspace*{0pt}\begin{myitemize}\item{}  CambridgeUS\item{}  \item{}  \item{}  \item{}  \end{myitemize} 
\end{longtable}
}}

Color themes, typically with animal names, can be specified with


\begin{Shaded}
\begin{Highlighting}[]

\NormalTok{\textbackslash{}usecolortheme\{beaver\}}\newline
\end{Highlighting}
\end{Shaded}


The full list of color themes is:
\begin{longtable}{>{\RaggedRight}p{0.30764\linewidth}>{\RaggedRight}p{0.27998\linewidth}>{\RaggedRight}p{0.32667\linewidth}} 
\hspace*{0pt}\ignorespaces{}\hspace*{0pt}\begin{myitemize}\item{}  default\item{}  albatross\item{}  beaver\item{}  beetle\item{}  crane\end{myitemize}&\hspace*{0pt}\ignorespaces{}\hspace*{0pt}\begin{myitemize}\item{}  dolphin\item{}  dove\item{}  fly\item{}  lily\item{}  orchid\end{myitemize}&\hspace*{0pt}\ignorespaces{}\hspace*{0pt}\begin{myitemize}\item{}  rose\item{}  seagull\item{}  seahorse\item{}  whale\item{}  wolverine\end{myitemize} 
\end{longtable}

\paragraph{The {\itshape \setmainfont[Path=/usr/share/fonts/truetype/cmu/,UprightFont=cmunrm.ttf,BoldFont=cmunbx.ttf,ItalicFont=cmunti.ttf,BoldItalicFont=cmunbi.ttf]{cmunti.ttf}\setmonofont[Path=/usr/share/fonts/truetype/cmu/,UprightFont=cmuntt.ttf,BoldFont=cmuntb.ttf,ItalicFont=cmunit.ttf,BoldItalicFont=cmuntx.ttf]{cmunti.ttf}\itshape do it yourself}{$\text{ }$}\setmainfont[Path=/usr/share/fonts/truetype/cmu/,UprightFont=cmunrm.ttf,BoldFont=cmunbx.ttf,ItalicFont=cmunti.ttf,BoldItalicFont=cmunbi.ttf]{cmunrm.ttf}\setmonofont[Path=/usr/share/fonts/truetype/cmu/,UprightFont=cmuntt.ttf,BoldFont=cmuntb.ttf,ItalicFont=cmunit.ttf,BoldItalicFont=cmuntx.ttf]{cmunrm.ttf} solution}
{$\text{ }$}\newline\label{740}

First you can specify the {\itshape \setmainfont[Path=/usr/share/fonts/truetype/cmu/,UprightFont=cmunrm.ttf,BoldFont=cmunbx.ttf,ItalicFont=cmunti.ttf,BoldItalicFont=cmunbi.ttf]{cmunti.ttf}\setmonofont[Path=/usr/share/fonts/truetype/cmu/,UprightFont=cmuntt.ttf,BoldFont=cmuntb.ttf,ItalicFont=cmunit.ttf,BoldItalicFont=cmuntx.ttf]{cmunti.ttf}\itshape outertheme}\setmainfont[Path=/usr/share/fonts/truetype/cmu/,UprightFont=cmunrm.ttf,BoldFont=cmunbx.ttf,ItalicFont=cmunti.ttf,BoldItalicFont=cmunbi.ttf]{cmunrm.ttf}\setmonofont[Path=/usr/share/fonts/truetype/cmu/,UprightFont=cmuntt.ttf,BoldFont=cmuntb.ttf,ItalicFont=cmunit.ttf,BoldItalicFont=cmuntx.ttf]{cmunrm.ttf}. The outertheme defines the head and the footline of each slide. 


\begin{Shaded}
\begin{Highlighting}[]

\NormalTok{\textbackslash{}useoutertheme\{infolines\}\ensuremath{\text{ }}}\newline
\end{Highlighting}
\end{Shaded}


Here is a list of all available outer themes:
\begin{myitemize}
\item{}  infolines 
\item{}  miniframes
\item{}  shadow
\item{}  sidebar
\item{}  smoothbars
\item{}  smoothtree
\item{}  split
\item{}  tree
\end{myitemize}


Then you can add the innertheme: 


\begin{Shaded}
\begin{Highlighting}[]

\NormalTok{\textbackslash{}useinnertheme\{rectangles\}\ensuremath{\text{ }}}\newline
\end{Highlighting}
\end{Shaded}


Here is a list of all available inner themes: 

\begin{myitemize}
\item{}  rectangles
\item{}  circles
\item{}  inmargin
\item{}  rounded
\end{myitemize}


You can define the color of every element: 


\begin{Shaded}
\begin{Highlighting}[]

\NormalTok{\textbackslash{}setbeamercolor\{alerted\ensuremath{\text{ }}text\}\{fg=orange\}}\newline
\NormalTok{\textbackslash{}setbeamercolor\{background\ensuremath{\text{ }}canvas\}\{bg=white\}}\newline
\NormalTok{\textbackslash{}setbeamercolor\{block\ensuremath{\text{ }}body\ensuremath{\text{ }}alerted\}\{bg=normal\ensuremath{\text{ }}text.bg!90!black\}}\newline
\NormalTok{\textbackslash{}setbeamercolor\{block\ensuremath{\text{ }}body\}\{bg=normal\ensuremath{\text{ }}text.bg!90!black\}}\newline
\NormalTok{\textbackslash{}setbeamercolor\{block\ensuremath{\text{ }}body\ensuremath{\text{ }}example\}\{bg=normal\ensuremath{\text{ }}text.bg!90!black\}}\newline
\NormalTok{\textbackslash{}setbeamercolor\{block\ensuremath{\text{ }}title\ensuremath{\text{ }}alerted\}\{use=\{normal\ensuremath{\text{ }}text,alerted\ensuremath{\text{ }}text\},fg=alerted}\newline
\ensuremath{\text{ }}\NormalTok{text.fg!75!normal\ensuremath{\text{ }}text.fg,bg=normal\ensuremath{\text{ }}text.bg!75!black\}}\newline
\NormalTok{\textbackslash{}setbeamercolor\{block\ensuremath{\text{ }}title\}\{bg=blue\}}\newline
\NormalTok{\textbackslash{}setbeamercolor\{block\ensuremath{\text{ }}title\ensuremath{\text{ }}example\}\{use=\{normal\ensuremath{\text{ }}text,example\ensuremath{\text{ }}text\},fg=example}\newline
\ensuremath{\text{ }}\NormalTok{text.fg!75!normal\ensuremath{\text{ }}text.fg,bg=normal\ensuremath{\text{ }}text.bg!75!black\}}\newline
\NormalTok{\textbackslash{}setbeamercolor\{fine\ensuremath{\text{ }}separation\ensuremath{\text{ }}line\}\{\}}\newline
\NormalTok{\textbackslash{}setbeamercolor\{frametitle\}\{fg=brown\}}\newline
\NormalTok{\textbackslash{}setbeamercolor\{item\ensuremath{\text{ }}projected\}\{fg=black\}}\newline
\NormalTok{\textbackslash{}setbeamercolor\{normal\ensuremath{\text{ }}text\}\{bg=black,fg=yellow\}}\newline
\NormalTok{\textbackslash{}setbeamercolor\{palette\ensuremath{\text{ }}sidebar\ensuremath{\text{ }}primary\}\{use=normal\ensuremath{\text{ }}text,fg=normal\ensuremath{\text{ }}text.fg\}}\newline
\NormalTok{\textbackslash{}setbeamercolor\{palette\ensuremath{\text{ }}sidebar\ensuremath{\text{ }}quaternary\}\{use=structure,fg=structure.fg\}}\newline
\NormalTok{\textbackslash{}setbeamercolor\{palette\ensuremath{\text{ }}sidebar\ensuremath{\text{ }}secondary\}\{use=structure,fg=structure.fg\}}\newline
\NormalTok{\textbackslash{}setbeamercolor\{palette\ensuremath{\text{ }}sidebar\ensuremath{\text{ }}tertiary\}\{use=normal\ensuremath{\text{ }}text,fg=normal\ensuremath{\text{ }}text.fg\}}\newline
\NormalTok{\textbackslash{}setbeamercolor\{section\ensuremath{\text{ }}in\ensuremath{\text{ }}sidebar\}\{fg=brown\}}\newline
\NormalTok{\textbackslash{}setbeamercolor\{section\ensuremath{\text{ }}in\ensuremath{\text{ }}sidebar\ensuremath{\text{ }}shaded\}\{fg=grey\}}\newline
\NormalTok{\textbackslash{}setbeamercolor\{separation\ensuremath{\text{ }}line\}\{\}}\newline
\NormalTok{\textbackslash{}setbeamercolor\{sidebar\}\{bg=red\}}\newline
\NormalTok{\textbackslash{}setbeamercolor\{sidebar\}\{parent=palette\ensuremath{\text{ }}primary\}}\newline
\NormalTok{\textbackslash{}setbeamercolor\{structure\}\{bg=black,\ensuremath{\text{ }}fg=green\}}\newline
\NormalTok{\textbackslash{}setbeamercolor\{subsection\ensuremath{\text{ }}in\ensuremath{\text{ }}sidebar\}\{fg=brown\}}\newline
\NormalTok{\textbackslash{}setbeamercolor\{subsection\ensuremath{\text{ }}in\ensuremath{\text{ }}sidebar\ensuremath{\text{ }}shaded\}\{fg=grey\}}\newline
\NormalTok{\textbackslash{}setbeamercolor\{title\}\{fg=brown\}}\newline
\NormalTok{\textbackslash{}setbeamercolor\{titlelike\}\{fg=brown\}}\newline
\end{Highlighting}
\end{Shaded}


Colors can be defined as usual: 


\begin{Shaded}
\begin{Highlighting}[]

\NormalTok{\textbackslash{}definecolor\{chocolate\}\{RGB\}\{33,33,33\}}\newline
\end{Highlighting}
\end{Shaded}


Block styles can also be defined: 


\begin{Shaded}
\begin{Highlighting}[]

\NormalTok{\textbackslash{}setbeamertemplate\{blocks\}[rounded][shadow=true]}\newline
\NormalTok{\textbackslash{}setbeamertemplate\{background\ensuremath{\text{ }}canvas\}[vertical}\newline
\ensuremath{\text{ }}\NormalTok{shading][bottom=white,top=structure.fg!25]}\newline
\NormalTok{\textbackslash{}setbeamertemplate\{sidebar\ensuremath{\text{ }}canvas\ensuremath{\text{ }}left\}[horizontal}\newline
\ensuremath{\text{ }}\NormalTok{shading][left=white!40!black,right=black]}\newline
\end{Highlighting}
\end{Shaded}


You can also suppress the navigation bar: 

\begin{Shaded}
\begin{Highlighting}[]

\NormalTok{\textbackslash{}beamertemplatenavigationsymbolsempty\ensuremath{\text{ }}}\newline
\end{Highlighting}
\end{Shaded}

\subsubsection{Fonts}
\label{741}

You may also change the fonts for particular elements. If you wanted the title of the presentation as rendered by 
\begin{Shaded}
\begin{Highlighting}[]

\NormalTok{\textbackslash{}frame\{\textbackslash{}titlepage\}}\newline
\end{Highlighting}
\end{Shaded}
 to occur in a serif font instead of the default sanserif, you would use:


\begin{Shaded}
\begin{Highlighting}[]

\NormalTok{\textbackslash{}setbeamerfont\{title\}\{family=\textbackslash{}rm\}}\newline
\end{Highlighting}
\end{Shaded}


You could take this a step further if you are using OpenType fonts with Xe(La)TeX and specify a serif font with increased size and oldstyle proportional alternate number glyphs:


\begin{Shaded}
\begin{Highlighting}[]

\NormalTok{\textbackslash{}setbeamerfont\{title\}\{family=\textbackslash{}rm\textbackslash{}addfontfeatures\{Scale=1.18,\ensuremath{\text{ }}Numbers=\{Lining,}\newline
\ensuremath{\text{ }}\NormalTok{Proportional\}\}\}}\newline
\end{Highlighting}
\end{Shaded}

\paragraph{Math Fonts}
{$\text{ }$}\newline\label{742}
The default settings for {\ttfamily \setmainfont[Path=/usr/share/fonts/truetype/cmu/,UprightFont=cmunrm.ttf,BoldFont=cmunbx.ttf,ItalicFont=cmunti.ttf,BoldItalicFont=cmunbi.ttf]{cmuntt.ttf}\setmonofont[Path=/usr/share/fonts/truetype/cmu/,UprightFont=cmuntt.ttf,BoldFont=cmuntb.ttf,ItalicFont=cmunit.ttf,BoldItalicFont=cmuntx.ttf]{cmuntt.ttf}\ttfamily beamer}{$\text{ }$}\setmainfont[Path=/usr/share/fonts/truetype/cmu/,UprightFont=cmunrm.ttf,BoldFont=cmunbx.ttf,ItalicFont=cmunti.ttf,BoldItalicFont=cmunbi.ttf]{cmunrm.ttf}\setmonofont[Path=/usr/share/fonts/truetype/cmu/,UprightFont=cmuntt.ttf,BoldFont=cmuntb.ttf,ItalicFont=cmunit.ttf,BoldItalicFont=cmuntx.ttf]{cmunrm.ttf} use a different set of math fonts than one would expect from creating a simple math article. One quick fix for this is to use at the beginning of the file the option {\ttfamily \setmainfont[Path=/usr/share/fonts/truetype/cmu/,UprightFont=cmunrm.ttf,BoldFont=cmunbx.ttf,ItalicFont=cmunti.ttf,BoldItalicFont=cmunbi.ttf]{cmuntt.ttf}\setmonofont[Path=/usr/share/fonts/truetype/cmu/,UprightFont=cmuntt.ttf,BoldFont=cmuntb.ttf,ItalicFont=cmunit.ttf,BoldItalicFont=cmuntx.ttf]{cmuntt.ttf}\ttfamily mathserif}\setmainfont[Path=/usr/share/fonts/truetype/cmu/,UprightFont=cmunrm.ttf,BoldFont=cmunbx.ttf,ItalicFont=cmunti.ttf,BoldItalicFont=cmunbi.ttf]{cmunrm.ttf}\setmonofont[Path=/usr/share/fonts/truetype/cmu/,UprightFont=cmuntt.ttf,BoldFont=cmuntb.ttf,ItalicFont=cmunit.ttf,BoldItalicFont=cmuntx.ttf]{cmunrm.ttf}


\begin{Shaded}
\begin{Highlighting}[]

\NormalTok{\textbackslash{}documentclass[mathserif]\{beamer\}}\newline
\end{Highlighting}
\end{Shaded}


Others have proposed to use the command


\begin{Shaded}
\begin{Highlighting}[]

\NormalTok{\textbackslash{}usefonttheme[onlymath]\{serif\}}\newline
\end{Highlighting}
\end{Shaded}


but it is not clear if this works for absolutely every math character.
\subsection{Frames Options}
\label{743}

The {\itshape \setmainfont[Path=/usr/share/fonts/truetype/cmu/,UprightFont=cmunrm.ttf,BoldFont=cmunbx.ttf,ItalicFont=cmunti.ttf,BoldItalicFont=cmunbi.ttf]{cmunti.ttf}\setmonofont[Path=/usr/share/fonts/truetype/cmu/,UprightFont=cmuntt.ttf,BoldFont=cmuntb.ttf,ItalicFont=cmunit.ttf,BoldItalicFont=cmuntx.ttf]{cmunti.ttf}\itshape plain}{$\text{ }$}\setmainfont[Path=/usr/share/fonts/truetype/cmu/,UprightFont=cmunrm.ttf,BoldFont=cmunbx.ttf,ItalicFont=cmunti.ttf,BoldItalicFont=cmunbi.ttf]{cmunrm.ttf}\setmonofont[Path=/usr/share/fonts/truetype/cmu/,UprightFont=cmuntt.ttf,BoldFont=cmuntb.ttf,ItalicFont=cmunit.ttf,BoldItalicFont=cmuntx.ttf]{cmunrm.ttf} option. Sometimes you need to include a large figure or a large table and you don\textquotesingle{}t want to have the bottom and the top off the slides. In that case, use the plain option: 


\begin{Shaded}
\begin{Highlighting}[]

\NormalTok{\textbackslash{}frame[plain]\{}\newline
\CommentTok{\%\ensuremath{\text{ }}...}\newline
\NormalTok{\}}\newline
\end{Highlighting}
\end{Shaded}


If you want to include lots of text on a slide, use the {\itshape \setmainfont[Path=/usr/share/fonts/truetype/cmu/,UprightFont=cmunrm.ttf,BoldFont=cmunbx.ttf,ItalicFont=cmunti.ttf,BoldItalicFont=cmunbi.ttf]{cmunti.ttf}\setmonofont[Path=/usr/share/fonts/truetype/cmu/,UprightFont=cmuntt.ttf,BoldFont=cmuntb.ttf,ItalicFont=cmunit.ttf,BoldItalicFont=cmuntx.ttf]{cmunti.ttf}\itshape shrink}{$\text{ }$}\setmainfont[Path=/usr/share/fonts/truetype/cmu/,UprightFont=cmunrm.ttf,BoldFont=cmunbx.ttf,ItalicFont=cmunti.ttf,BoldItalicFont=cmunbi.ttf]{cmunrm.ttf}\setmonofont[Path=/usr/share/fonts/truetype/cmu/,UprightFont=cmuntt.ttf,BoldFont=cmuntb.ttf,ItalicFont=cmunit.ttf,BoldItalicFont=cmuntx.ttf]{cmunrm.ttf} option.

\begin{Shaded}
\begin{Highlighting}[]

\NormalTok{\textbackslash{}frame[shrink]\{}\newline
\CommentTok{\%\ensuremath{\text{ }}...}\newline
\NormalTok{\}}\newline
\end{Highlighting}
\end{Shaded}


The {\itshape \setmainfont[Path=/usr/share/fonts/truetype/cmu/,UprightFont=cmunrm.ttf,BoldFont=cmunbx.ttf,ItalicFont=cmunti.ttf,BoldItalicFont=cmunbi.ttf]{cmunti.ttf}\setmonofont[Path=/usr/share/fonts/truetype/cmu/,UprightFont=cmuntt.ttf,BoldFont=cmuntb.ttf,ItalicFont=cmunit.ttf,BoldItalicFont=cmuntx.ttf]{cmunti.ttf}\itshape allowframebreaks}{$\text{ }$}\setmainfont[Path=/usr/share/fonts/truetype/cmu/,UprightFont=cmunrm.ttf,BoldFont=cmunbx.ttf,ItalicFont=cmunti.ttf,BoldItalicFont=cmunbi.ttf]{cmunrm.ttf}\setmonofont[Path=/usr/share/fonts/truetype/cmu/,UprightFont=cmuntt.ttf,BoldFont=cmuntb.ttf,ItalicFont=cmunit.ttf,BoldItalicFont=cmuntx.ttf]{cmunrm.ttf} option will auto-{}create new frames if there is too much content to be displayed on one.

\begin{Shaded}
\begin{Highlighting}[]

\NormalTok{\textbackslash{}frame[allowframebreaks]\{}\newline
\CommentTok{\%\ensuremath{\text{ }}...}\newline
\NormalTok{\}}\newline
\end{Highlighting}
\end{Shaded}



Before using any verbatim environment (like {\ttfamily \setmainfont[Path=/usr/share/fonts/truetype/cmu/,UprightFont=cmunrm.ttf,BoldFont=cmunbx.ttf,ItalicFont=cmunti.ttf,BoldItalicFont=cmunbi.ttf]{cmuntt.ttf}\setmonofont[Path=/usr/share/fonts/truetype/cmu/,UprightFont=cmuntt.ttf,BoldFont=cmuntb.ttf,ItalicFont=cmunit.ttf,BoldItalicFont=cmuntx.ttf]{cmuntt.ttf}\ttfamily listings}\setmainfont[Path=/usr/share/fonts/truetype/cmu/,UprightFont=cmunrm.ttf,BoldFont=cmunbx.ttf,ItalicFont=cmunti.ttf,BoldItalicFont=cmunbi.ttf]{cmunrm.ttf}\setmonofont[Path=/usr/share/fonts/truetype/cmu/,UprightFont=cmuntt.ttf,BoldFont=cmuntb.ttf,ItalicFont=cmunit.ttf,BoldItalicFont=cmuntx.ttf]{cmunrm.ttf}), you should pass the option {\ttfamily \setmainfont[Path=/usr/share/fonts/truetype/cmu/,UprightFont=cmunrm.ttf,BoldFont=cmunbx.ttf,ItalicFont=cmunti.ttf,BoldItalicFont=cmunbi.ttf]{cmuntt.ttf}\setmonofont[Path=/usr/share/fonts/truetype/cmu/,UprightFont=cmuntt.ttf,BoldFont=cmuntb.ttf,ItalicFont=cmunit.ttf,BoldItalicFont=cmuntx.ttf]{cmuntt.ttf}\ttfamily fragile}{$\text{ }$}\setmainfont[Path=/usr/share/fonts/truetype/cmu/,UprightFont=cmunrm.ttf,BoldFont=cmunbx.ttf,ItalicFont=cmunti.ttf,BoldItalicFont=cmunbi.ttf]{cmunrm.ttf}\setmonofont[Path=/usr/share/fonts/truetype/cmu/,UprightFont=cmuntt.ttf,BoldFont=cmuntb.ttf,ItalicFont=cmunit.ttf,BoldItalicFont=cmuntx.ttf]{cmunrm.ttf} to the 
\begin{Shaded}
\begin{Highlighting}[]

\NormalTok{frame}\newline
\end{Highlighting}
\end{Shaded}
 environment, as verbatim environments need to be typeset differently.  Usually, the form {\ttfamily \setmainfont[Path=/usr/share/fonts/truetype/cmu/,UprightFont=cmunrm.ttf,BoldFont=cmunbx.ttf,ItalicFont=cmunti.ttf,BoldItalicFont=cmunbi.ttf]{cmuntt.ttf}\setmonofont[Path=/usr/share/fonts/truetype/cmu/,UprightFont=cmuntt.ttf,BoldFont=cmuntb.ttf,ItalicFont=cmunit.ttf,BoldItalicFont=cmuntx.ttf]{cmuntt.ttf}\ttfamily fragile=singleslide}{$\text{ }$}\setmainfont[Path=/usr/share/fonts/truetype/cmu/,UprightFont=cmunrm.ttf,BoldFont=cmunbx.ttf,ItalicFont=cmunti.ttf,BoldItalicFont=cmunbi.ttf]{cmunrm.ttf}\setmonofont[Path=/usr/share/fonts/truetype/cmu/,UprightFont=cmuntt.ttf,BoldFont=cmuntb.ttf,ItalicFont=cmunit.ttf,BoldItalicFont=cmuntx.ttf]{cmunrm.ttf} is usable (for details see the manual). Note that the {\ttfamily \setmainfont[Path=/usr/share/fonts/truetype/cmu/,UprightFont=cmunrm.ttf,BoldFont=cmunbx.ttf,ItalicFont=cmunti.ttf,BoldItalicFont=cmunbi.ttf]{cmuntt.ttf}\setmonofont[Path=/usr/share/fonts/truetype/cmu/,UprightFont=cmuntt.ttf,BoldFont=cmuntb.ttf,ItalicFont=cmunit.ttf,BoldItalicFont=cmuntx.ttf]{cmuntt.ttf}\ttfamily fragile}{$\text{ }$}\setmainfont[Path=/usr/share/fonts/truetype/cmu/,UprightFont=cmunrm.ttf,BoldFont=cmunbx.ttf,ItalicFont=cmunti.ttf,BoldItalicFont=cmunbi.ttf]{cmunrm.ttf}\setmonofont[Path=/usr/share/fonts/truetype/cmu/,UprightFont=cmuntt.ttf,BoldFont=cmuntb.ttf,ItalicFont=cmunit.ttf,BoldItalicFont=cmuntx.ttf]{cmunrm.ttf} option may not be used with 
\begin{Shaded}
\begin{Highlighting}[]

\NormalTok{\textbackslash{}frame}\newline
\end{Highlighting}
\end{Shaded}
 commands since it expects to encounter a 
\begin{Shaded}
\begin{Highlighting}[]

\NormalTok{\textbackslash{}end\{frame\}}\newline
\end{Highlighting}
\end{Shaded}
, which should be alone on a single line.

\begin{Shaded}
\begin{Highlighting}[]

\NormalTok{\textbackslash{}begin\{frame\}[fragile]}\newline
\NormalTok{\textbackslash{}frametitle\{Source\ensuremath{\text{ }}code\}}\newline
\ensuremath{\text{ }}\newline
\NormalTok{\textbackslash{}begin\{lstlisting\}[caption=First\ensuremath{\text{ }}C\ensuremath{\text{ }}example]}\newline
\NormalTok{int\ensuremath{\text{ }}main()}\newline
\NormalTok{\{}\newline
\ensuremath{\text{ }}\ensuremath{\text{ }}\ensuremath{\text{ }}\ensuremath{\text{ }}\NormalTok{printf("Hello\ensuremath{\text{ }}World!");}\newline
\ensuremath{\text{ }}\ensuremath{\text{ }}\ensuremath{\text{ }}\ensuremath{\text{ }}\NormalTok{return\ensuremath{\text{ }}0;}\newline
\NormalTok{\}}\newline
\NormalTok{\textbackslash{}end\{lstlisting\}}\newline
\NormalTok{\textbackslash{}end\{frame\}}\newline
\end{Highlighting}
\end{Shaded}

\subsection{Hyperlink navigation}
\label{744}

Internal and external hyperlinks can be used in beamer to assist navigation.  Clean looking buttons can also be added.

\LaTeXNullTemplate{}

By default the beamer class adds navigation buttons in the bottom right corner. To remove them one can place 

\begin{Shaded}
\begin{Highlighting}[]

\NormalTok{\textbackslash{}beamertemplatenavigationsymbolsempty}\newline
\end{Highlighting}
\end{Shaded}

in the preamble.
\subsection{Animations}
\label{745}

The following is merely an introduction to the possibilities in beamer.  Chapter 8 of the beamer manual provides much more detail, on many more features.

Making items appear on a slide is possible by simply using the 
\begin{Shaded}
\begin{Highlighting}[]

\NormalTok{\textbackslash{}pause}\newline
\end{Highlighting}
\end{Shaded}
 statement:


\begin{Shaded}
\begin{Highlighting}[]

\NormalTok{\textbackslash{}begin\{frame\}}\newline
\NormalTok{\textbackslash{}frametitle\{Some\ensuremath{\text{ }}background\}}\newline
\NormalTok{We\ensuremath{\text{ }}start\ensuremath{\text{ }}our\ensuremath{\text{ }}discussion\ensuremath{\text{ }}with\ensuremath{\text{ }}some\ensuremath{\text{ }}concepts.}\newline
\NormalTok{\textbackslash{}pause}\newline
\NormalTok{The\ensuremath{\text{ }}first\ensuremath{\text{ }}concept\ensuremath{\text{ }}we\ensuremath{\text{ }}introduce\ensuremath{\text{ }}originates\ensuremath{\text{ }}with\ensuremath{\text{ }}Erd\textbackslash{}H\ensuremath{\text{ }}os.}\newline
\NormalTok{\textbackslash{}end\{frame\}}\newline
\end{Highlighting}
\end{Shaded}

Text or figures after 
\begin{Shaded}
\begin{Highlighting}[]

\NormalTok{\textbackslash{}pause}\newline
\end{Highlighting}
\end{Shaded}
 will display after one of the following events (which may vary between PDF viewers): pressing space, return or page down on the keyboard, or using the mouse to scroll down or click the next slide button.  Pause can be used within 
\begin{Shaded}
\begin{Highlighting}[]

\NormalTok{\textbackslash{}itemize}\newline
\end{Highlighting}
\end{Shaded}
 etc.
\subsubsection{Text animations}
\label{746}

For text animations, for example in the itemize environment, it is possible to specify appearance and disappearance of text by using 
\begin{Shaded}
\begin{Highlighting}[]

\NormalTok{<a-b>}\newline
\end{Highlighting}
\end{Shaded}
 where {\bfseries \setmainfont[Path=/usr/share/fonts/truetype/cmu/,UprightFont=cmunrm.ttf,BoldFont=cmunbx.ttf,ItalicFont=cmunti.ttf,BoldItalicFont=cmunbi.ttf]{cmunbx.ttf}\setmonofont[Path=/usr/share/fonts/truetype/cmu/,UprightFont=cmuntt.ttf,BoldFont=cmuntb.ttf,ItalicFont=cmunit.ttf,BoldItalicFont=cmuntx.ttf]{cmunbx.ttf}\bfseries a}{$\text{ }$}\setmainfont[Path=/usr/share/fonts/truetype/cmu/,UprightFont=cmunrm.ttf,BoldFont=cmunbx.ttf,ItalicFont=cmunti.ttf,BoldItalicFont=cmunbi.ttf]{cmunrm.ttf}\setmonofont[Path=/usr/share/fonts/truetype/cmu/,UprightFont=cmuntt.ttf,BoldFont=cmuntb.ttf,ItalicFont=cmunit.ttf,BoldItalicFont=cmuntx.ttf]{cmunrm.ttf} and {\bfseries \setmainfont[Path=/usr/share/fonts/truetype/cmu/,UprightFont=cmunrm.ttf,BoldFont=cmunbx.ttf,ItalicFont=cmunti.ttf,BoldItalicFont=cmunbi.ttf]{cmunbx.ttf}\setmonofont[Path=/usr/share/fonts/truetype/cmu/,UprightFont=cmuntt.ttf,BoldFont=cmuntb.ttf,ItalicFont=cmunit.ttf,BoldItalicFont=cmuntx.ttf]{cmunbx.ttf}\bfseries b}{$\text{ }$}\setmainfont[Path=/usr/share/fonts/truetype/cmu/,UprightFont=cmunrm.ttf,BoldFont=cmunbx.ttf,ItalicFont=cmunti.ttf,BoldItalicFont=cmunbi.ttf]{cmunrm.ttf}\setmonofont[Path=/usr/share/fonts/truetype/cmu/,UprightFont=cmuntt.ttf,BoldFont=cmuntb.ttf,ItalicFont=cmunit.ttf,BoldItalicFont=cmuntx.ttf]{cmunrm.ttf} are the numbers of the events the item is to be displayed for (inclusive).  For example:

\begin{Shaded}
\begin{Highlighting}[]

\NormalTok{\textbackslash{}begin\{itemize\}}\newline
\ensuremath{\text{ }}\ensuremath{\text{ }}\NormalTok{\textbackslash{}item\ensuremath{\text{ }}This\ensuremath{\text{ }}one\ensuremath{\text{ }}is\ensuremath{\text{ }}always\ensuremath{\text{ }}shown}\newline
\ensuremath{\text{ }}\ensuremath{\text{ }}\NormalTok{\textbackslash{}item<1->\ensuremath{\text{ }}The\ensuremath{\text{ }}first\ensuremath{\text{ }}time\ensuremath{\text{ }}(i.e.\ensuremath{\text{ }}as\ensuremath{\text{ }}soon\ensuremath{\text{ }}as\ensuremath{\text{ }}the\ensuremath{\text{ }}slide\ensuremath{\text{ }}loads)}\newline
\ensuremath{\text{ }}\ensuremath{\text{ }}\NormalTok{\textbackslash{}item<2->\ensuremath{\text{ }}The\ensuremath{\text{ }}second\ensuremath{\text{ }}time}\newline
\ensuremath{\text{ }}\ensuremath{\text{ }}\NormalTok{\textbackslash{}item<1->\ensuremath{\text{ }}Also\ensuremath{\text{ }}the\ensuremath{\text{ }}first\ensuremath{\text{ }}time}\newline
\ensuremath{\text{ }}\ensuremath{\text{ }}\NormalTok{\textbackslash{}only<1-1>\ensuremath{\text{ }}\{This\ensuremath{\text{ }}one\ensuremath{\text{ }}is\ensuremath{\text{ }}shown\ensuremath{\text{ }}at\ensuremath{\text{ }}the\ensuremath{\text{ }}first\ensuremath{\text{ }}time,\ensuremath{\text{ }}but\ensuremath{\text{ }}it\ensuremath{\text{ }}will\ensuremath{\text{ }}hide\ensuremath{\text{ }}soon\ensuremath{\text{ }}(on\ensuremath{\text{ }}the}\newline
\ensuremath{\text{ }}\NormalTok{next\ensuremath{\text{ }}event\ensuremath{\text{ }}after\ensuremath{\text{ }}the\ensuremath{\text{ }}slide\ensuremath{\text{ }}loads).\}}\newline
\NormalTok{\textbackslash{}end\{itemize\}}\newline
\end{Highlighting}
\end{Shaded}


A simpler approach for revealing one item per click is to use 
\begin{Shaded}
\begin{Highlighting}[]

\NormalTok{\textbackslash{}begin\{itemize\}[<+->]}\newline
\end{Highlighting}
\end{Shaded}
.


\begin{Shaded}
\begin{Highlighting}[]

\NormalTok{\textbackslash{}begin\{frame\}}\newline
	\NormalTok{\textbackslash{}frametitle\{`Hidden\ensuremath{\text{ }}higher-order\ensuremath{\text{ }}concepts?\textquotesingle{}\}}\newline
	\NormalTok{\textbackslash{}begin\{itemize\}[<+->]}\newline
	\NormalTok{\textbackslash{}item\ensuremath{\text{ }}The\ensuremath{\text{ }}truths\ensuremath{\text{ }}of\ensuremath{\text{ }}arithmetic\ensuremath{\text{ }}which\ensuremath{\text{ }}are\ensuremath{\text{ }}independent\ensuremath{\text{ }}of\ensuremath{\text{ }}PA\ensuremath{\text{ }}in\ensuremath{\text{ }}some\ensuremath{\text{ }}}\newline
	\NormalTok{sense\ensuremath{\text{ }}themselves\ensuremath{\text{ }}`\{contain\}\ensuremath{\text{ }}essentially\ensuremath{\text{ }}\{\textbackslash{}color\{blue\}\{hidden\ensuremath{\text{ }}higher-order\}\},}\newline
	\ensuremath{\text{ }}\NormalTok{or\ensuremath{\text{ }}infinitary,\ensuremath{\text{ }}concepts\textquotesingle{}???}\newline
	\NormalTok{\textbackslash{}item\ensuremath{\text{ }}`Truths\ensuremath{\text{ }}in\ensuremath{\text{ }}the\ensuremath{\text{ }}language\ensuremath{\text{ }}of\ensuremath{\text{ }}arithmetic\ensuremath{\text{ }}which\ensuremath{\text{ }}\textbackslash{}ldots}\newline
	\NormalTok{\textbackslash{}item	That\ensuremath{\text{ }}suggests\ensuremath{\text{ }}stronger\ensuremath{\text{ }}version\ensuremath{\text{ }}of\ensuremath{\text{ }}Isaacson\textquotesingle{}s\ensuremath{\text{ }}thesis.\ensuremath{\text{ }}}\newline
	\NormalTok{\textbackslash{}end\{itemize\}}\newline
\NormalTok{\textbackslash{}end\{frame\}}\newline
\end{Highlighting}
\end{Shaded}


In all these cases, pressing page up, scrolling up, or clicking the previous slide button in the navigation bar will backtrack through the sequence.
\subsection{Handout mode}
\label{747}
In beamer class, the default mode is {\itshape \setmainfont[Path=/usr/share/fonts/truetype/cmu/,UprightFont=cmunrm.ttf,BoldFont=cmunbx.ttf,ItalicFont=cmunti.ttf,BoldItalicFont=cmunbi.ttf]{cmunti.ttf}\setmonofont[Path=/usr/share/fonts/truetype/cmu/,UprightFont=cmuntt.ttf,BoldFont=cmuntb.ttf,ItalicFont=cmunit.ttf,BoldItalicFont=cmuntx.ttf]{cmunti.ttf}\itshape presentation}{$\text{ }$}\setmainfont[Path=/usr/share/fonts/truetype/cmu/,UprightFont=cmunrm.ttf,BoldFont=cmunbx.ttf,ItalicFont=cmunti.ttf,BoldItalicFont=cmunbi.ttf]{cmunrm.ttf}\setmonofont[Path=/usr/share/fonts/truetype/cmu/,UprightFont=cmuntt.ttf,BoldFont=cmuntb.ttf,ItalicFont=cmunit.ttf,BoldItalicFont=cmuntx.ttf]{cmunrm.ttf} which makes the slides. However, you can work in a different mode that is called {\itshape \setmainfont[Path=/usr/share/fonts/truetype/cmu/,UprightFont=cmunrm.ttf,BoldFont=cmunbx.ttf,ItalicFont=cmunti.ttf,BoldItalicFont=cmunbi.ttf]{cmunti.ttf}\setmonofont[Path=/usr/share/fonts/truetype/cmu/,UprightFont=cmuntt.ttf,BoldFont=cmuntb.ttf,ItalicFont=cmunit.ttf,BoldItalicFont=cmuntx.ttf]{cmunti.ttf}\itshape handout}{$\text{ }$}\setmainfont[Path=/usr/share/fonts/truetype/cmu/,UprightFont=cmunrm.ttf,BoldFont=cmunbx.ttf,ItalicFont=cmunti.ttf,BoldItalicFont=cmunbi.ttf]{cmunrm.ttf}\setmonofont[Path=/usr/share/fonts/truetype/cmu/,UprightFont=cmuntt.ttf,BoldFont=cmuntb.ttf,ItalicFont=cmunit.ttf,BoldItalicFont=cmuntx.ttf]{cmunrm.ttf} by setting this option when calling the class:

\begin{Shaded}
\begin{Highlighting}[]

\NormalTok{\textbackslash{}documentclass[12pt,handout]\{beamer\}}\newline
\end{Highlighting}
\end{Shaded}

This mode is useful to see each slide only one time with all its stuff on it, making any 
\begin{Shaded}
\begin{Highlighting}[]

\NormalTok{\textbackslash{}itemize[<+->]}\newline
\end{Highlighting}
\end{Shaded}
 environments visible all at once (for instance, printable version). Nevertheless, this makes an issue when working with the 
\begin{Shaded}
\begin{Highlighting}[]

\NormalTok{only}\newline
\end{Highlighting}
\end{Shaded}
 command, because its purpose is to have {\itshape \setmainfont[Path=/usr/share/fonts/truetype/cmu/,UprightFont=cmunrm.ttf,BoldFont=cmunbx.ttf,ItalicFont=cmunti.ttf,BoldItalicFont=cmunbi.ttf]{cmunti.ttf}\setmonofont[Path=/usr/share/fonts/truetype/cmu/,UprightFont=cmuntt.ttf,BoldFont=cmuntb.ttf,ItalicFont=cmunit.ttf,BoldItalicFont=cmuntx.ttf]{cmunti.ttf}\itshape only}{$\text{ }$}\setmainfont[Path=/usr/share/fonts/truetype/cmu/,UprightFont=cmunrm.ttf,BoldFont=cmunbx.ttf,ItalicFont=cmunti.ttf,BoldItalicFont=cmunbi.ttf]{cmunrm.ttf}\setmonofont[Path=/usr/share/fonts/truetype/cmu/,UprightFont=cmuntt.ttf,BoldFont=cmuntb.ttf,ItalicFont=cmunit.ttf,BoldItalicFont=cmuntx.ttf]{cmunrm.ttf} some text or figures at a time and not all of them together.

If you want to solve this, you can add a statement to specify precisely the behavior when dealing with 
\begin{Shaded}
\begin{Highlighting}[]

\NormalTok{only}\newline
\end{Highlighting}
\end{Shaded}
 commands in handout mode. Suppose you have a code like this

\begin{Shaded}
\begin{Highlighting}[]

\NormalTok{\textbackslash{}only<1>\{\textbackslash{}includegraphics\{pic1.eps\}\}}\newline
\NormalTok{\textbackslash{}only<2>\{\textbackslash{}includegraphics\{pic2.eps\}\}}\newline
\end{Highlighting}
\end{Shaded}

These pictures being completely different, you want them both in the handout, but they cannot be both on the same slide since they are large. The solution is to add the handout statement to have the following:

\begin{Shaded}
\begin{Highlighting}[]

\NormalTok{\textbackslash{}only<1|\ensuremath{\text{ }}handout:1>\{\textbackslash{}includegraphics\{pic1.eps\}\}}\newline
\NormalTok{\textbackslash{}only<2|\ensuremath{\text{ }}handout:2>\{\textbackslash{}includegraphics\{pic2.eps\}\}}\newline
\end{Highlighting}
\end{Shaded}

This will ensure the handout will make a slide for each picture.

Now imagine you still have your two pictures with the only statements, but the second one show the first one plus some other graphs and you don\textquotesingle{}t need the first one to appear in the handout. You can thus precise the handout mode not to include some only commands by:

\begin{Shaded}
\begin{Highlighting}[]

\NormalTok{\textbackslash{}only<1|\ensuremath{\text{ }}handout:0>\{\textbackslash{}includegraphics\{pic1.eps\}\}}\newline
\NormalTok{\textbackslash{}only<2>\{\textbackslash{}includegraphics\{pic2.eps\}\}}\newline
\end{Highlighting}
\end{Shaded}


The command can also be used to hide frames, e.g.

\begin{Shaded}
\begin{Highlighting}[]

\NormalTok{\textbackslash{}begin\{frame\}<handout:0>}\newline
\end{Highlighting}
\end{Shaded}

or even, if you have written a frame that you don\textquotesingle{}t want anymore but maybe you will need it later, you can write

\begin{Shaded}
\begin{Highlighting}[]

\NormalTok{\textbackslash{}begin\{frame\}<0|\ensuremath{\text{ }}handout:0>}\newline
\end{Highlighting}
\end{Shaded}

and this will hide your slide in both modes. (The order matters. Don\textquotesingle{}t put handout:0|beamer:0 or it won\textquotesingle{}t work.)

A last word about the handout mode is about the notes. Actually, the full syntax for a frame is 

\begin{Shaded}
\begin{Highlighting}[]

\NormalTok{\textbackslash{}begin\{frame\}}\newline
\NormalTok{...}\newline
\NormalTok{\textbackslash{}end\{frame\}}\newline
\NormalTok{\textbackslash{}note\{...\}}\newline
\NormalTok{\textbackslash{}note\{...\}}\newline
\NormalTok{...}\newline
\end{Highlighting}
\end{Shaded}

and you can write your notes about a frame in the field {\itshape \setmainfont[Path=/usr/share/fonts/truetype/cmu/,UprightFont=cmunrm.ttf,BoldFont=cmunbx.ttf,ItalicFont=cmunti.ttf,BoldItalicFont=cmunbi.ttf]{cmunti.ttf}\setmonofont[Path=/usr/share/fonts/truetype/cmu/,UprightFont=cmuntt.ttf,BoldFont=cmuntb.ttf,ItalicFont=cmunit.ttf,BoldItalicFont=cmuntx.ttf]{cmunti.ttf}\itshape note}{$\text{ }$}\setmainfont[Path=/usr/share/fonts/truetype/cmu/,UprightFont=cmunrm.ttf,BoldFont=cmunbx.ttf,ItalicFont=cmunti.ttf,BoldItalicFont=cmunbi.ttf]{cmunrm.ttf}\setmonofont[Path=/usr/share/fonts/truetype/cmu/,UprightFont=cmuntt.ttf,BoldFont=cmuntb.ttf,ItalicFont=cmunit.ttf,BoldItalicFont=cmuntx.ttf]{cmunrm.ttf} (many of them if needed). Using this, you can add an option to the class calling, either

\begin{Shaded}
\begin{Highlighting}[]

\NormalTok{\textbackslash{}documentclass[12pt,handout,notes=only]\{beamer\}}\newline
\end{Highlighting}
\end{Shaded}

or

\begin{Shaded}
\begin{Highlighting}[]

\NormalTok{\textbackslash{}documentclass[12pt,handout,notes=show]\{beamer\}}\newline
\end{Highlighting}
\end{Shaded}

The first one is useful when you make a presentation to have only the notes you need, while the second one could be given to those who have followed your presentation or those who missed it, for them to have both the slides with what you said.

Note that the \textquotesingle{}handout\textquotesingle{} option in the \textbackslash{}documentclass line suppress all the animations.

{\bfseries \setmainfont[Path=/usr/share/fonts/truetype/cmu/,UprightFont=cmunrm.ttf,BoldFont=cmunbx.ttf,ItalicFont=cmunti.ttf,BoldItalicFont=cmunbi.ttf]{cmunbx.ttf}\setmonofont[Path=/usr/share/fonts/truetype/cmu/,UprightFont=cmuntt.ttf,BoldFont=cmuntb.ttf,ItalicFont=cmunit.ttf,BoldItalicFont=cmuntx.ttf]{cmunbx.ttf}\bfseries Important:}{$\text{ }$}\setmainfont[Path=/usr/share/fonts/truetype/cmu/,UprightFont=cmunrm.ttf,BoldFont=cmunbx.ttf,ItalicFont=cmunti.ttf,BoldItalicFont=cmunbi.ttf]{cmunrm.ttf}\setmonofont[Path=/usr/share/fonts/truetype/cmu/,UprightFont=cmuntt.ttf,BoldFont=cmuntb.ttf,ItalicFont=cmunit.ttf,BoldItalicFont=cmuntx.ttf]{cmunrm.ttf} the {\itshape \setmainfont[Path=/usr/share/fonts/truetype/cmu/,UprightFont=cmunrm.ttf,BoldFont=cmunbx.ttf,ItalicFont=cmunti.ttf,BoldItalicFont=cmunbi.ttf]{cmunti.ttf}\setmonofont[Path=/usr/share/fonts/truetype/cmu/,UprightFont=cmuntt.ttf,BoldFont=cmuntb.ttf,ItalicFont=cmunit.ttf,BoldItalicFont=cmuntx.ttf]{cmunti.ttf}\itshape notes=only}{$\text{ }$}\setmainfont[Path=/usr/share/fonts/truetype/cmu/,UprightFont=cmunrm.ttf,BoldFont=cmunbx.ttf,ItalicFont=cmunti.ttf,BoldItalicFont=cmunbi.ttf]{cmunrm.ttf}\setmonofont[Path=/usr/share/fonts/truetype/cmu/,UprightFont=cmuntt.ttf,BoldFont=cmuntb.ttf,ItalicFont=cmunit.ttf,BoldItalicFont=cmuntx.ttf]{cmunrm.ttf} mode is {\bfseries \setmainfont[Path=/usr/share/fonts/truetype/cmu/,UprightFont=cmunrm.ttf,BoldFont=cmunbx.ttf,ItalicFont=cmunti.ttf,BoldItalicFont=cmunbi.ttf]{cmunbx.ttf}\setmonofont[Path=/usr/share/fonts/truetype/cmu/,UprightFont=cmuntt.ttf,BoldFont=cmuntb.ttf,ItalicFont=cmunit.ttf,BoldItalicFont=cmuntx.ttf]{cmunbx.ttf}\bfseries literally}{$\text{ }$}\setmainfont[Path=/usr/share/fonts/truetype/cmu/,UprightFont=cmunrm.ttf,BoldFont=cmunbx.ttf,ItalicFont=cmunti.ttf,BoldItalicFont=cmunbi.ttf]{cmunrm.ttf}\setmonofont[Path=/usr/share/fonts/truetype/cmu/,UprightFont=cmuntt.ttf,BoldFont=cmuntb.ttf,ItalicFont=cmunit.ttf,BoldItalicFont=cmuntx.ttf]{cmunrm.ttf} doing only the notes. This means there will be no output file but the DVI. Thus it requires you to have run the compilation in another mode before. If you use separate files for a better distinction between the modes, you may need to copy the .aux file from the handout compilation with the slides (w/o the notes).
\subsection{Columns and Blocks}
\label{748}

There are two handy environments for structuring a slide: \symbol{34}blocks\symbol{34}, which divide the slide (horizontally) into headed sections, and \symbol{34}columns\symbol{34} which divides a slide (vertically) into columns.  Blocks and columns can be used inside each other.
\subsubsection{Columns}
\label{749}

Example

\begin{Shaded}
\begin{Highlighting}[]

\NormalTok{\textbackslash{}begin\{frame\}\{Example\ensuremath{\text{ }}of\ensuremath{\text{ }}columns\ensuremath{\text{ }}1\}}\newline
\ensuremath{\text{ }}\ensuremath{\text{ }}\ensuremath{\text{ }}\ensuremath{\text{ }}\NormalTok{\textbackslash{}begin\{columns\}[c]\ensuremath{\text{ }}}\CommentTok{\%\ensuremath{\text{ }}the\ensuremath{\text{ }}"c"\ensuremath{\text{ }}option\ensuremath{\text{ }}specifies\ensuremath{\text{ }}center\ensuremath{\text{ }}vertical\ensuremath{\text{ }}alignment}\newline
\ensuremath{\text{ }}\ensuremath{\text{ }}\ensuremath{\text{ }}\ensuremath{\text{ }}\NormalTok{\textbackslash{}column\{.5\textbackslash{}textwidth\}\ensuremath{\text{ }}}\CommentTok{\%\ensuremath{\text{ }}column\ensuremath{\text{ }}designated\ensuremath{\text{ }}by\ensuremath{\text{ }}a\ensuremath{\text{ }}command}\newline
\ensuremath{\text{ }}\ensuremath{\text{ }}\ensuremath{\text{ }}\ensuremath{\text{ }}\ensuremath{\text{ }}\NormalTok{Contents\ensuremath{\text{ }}of\ensuremath{\text{ }}the\ensuremath{\text{ }}first\ensuremath{\text{ }}column}\newline
\ensuremath{\text{ }}\ensuremath{\text{ }}\ensuremath{\text{ }}\ensuremath{\text{ }}\NormalTok{\textbackslash{}column\{.5\textbackslash{}textwidth\}}\newline
\ensuremath{\text{ }}\ensuremath{\text{ }}\ensuremath{\text{ }}\ensuremath{\text{ }}\ensuremath{\text{ }}\NormalTok{Contents\ensuremath{\text{ }}split\ensuremath{\text{ }}\textbackslash{}\textbackslash{}\ensuremath{\text{ }}into\ensuremath{\text{ }}two\ensuremath{\text{ }}lines}\newline
\ensuremath{\text{ }}\ensuremath{\text{ }}\ensuremath{\text{ }}\ensuremath{\text{ }}\NormalTok{\textbackslash{}end\{columns\}}\newline
\NormalTok{\textbackslash{}end\{frame\}}\newline
\ensuremath{\text{ }}\newline
\NormalTok{\textbackslash{}begin\{frame\}\{Example\ensuremath{\text{ }}of\ensuremath{\text{ }}columns\ensuremath{\text{ }}2\}}\newline
\ensuremath{\text{ }}\ensuremath{\text{ }}\ensuremath{\text{ }}\ensuremath{\text{ }}\ensuremath{\text{ }}\NormalTok{\textbackslash{}begin\{columns\}[T]\ensuremath{\text{ }}}\CommentTok{\%\ensuremath{\text{ }}contents\ensuremath{\text{ }}are\ensuremath{\text{ }}top\ensuremath{\text{ }}vertically\ensuremath{\text{ }}aligned}\newline
\ensuremath{\text{ }}\ensuremath{\text{ }}\ensuremath{\text{ }}\ensuremath{\text{ }}\ensuremath{\text{ }}\NormalTok{\textbackslash{}begin\{column\}[T]\{5cm\}\ensuremath{\text{ }}}\CommentTok{\%\ensuremath{\text{ }}each\ensuremath{\text{ }}column\ensuremath{\text{ }}can\ensuremath{\text{ }}also\ensuremath{\text{ }}be\ensuremath{\text{ }}its\ensuremath{\text{ }}own\ensuremath{\text{ }}environment}\newline
\ensuremath{\text{ }}\ensuremath{\text{ }}\ensuremath{\text{ }}\ensuremath{\text{ }}\ensuremath{\text{ }}\NormalTok{Contents\ensuremath{\text{ }}of\ensuremath{\text{ }}first\ensuremath{\text{ }}column\ensuremath{\text{ }}\textbackslash{}\textbackslash{}\ensuremath{\text{ }}split\ensuremath{\text{ }}into\ensuremath{\text{ }}two\ensuremath{\text{ }}lines}\newline
\ensuremath{\text{ }}\ensuremath{\text{ }}\ensuremath{\text{ }}\ensuremath{\text{ }}\ensuremath{\text{ }}\NormalTok{\textbackslash{}end\{column\}}\newline
\ensuremath{\text{ }}\ensuremath{\text{ }}\ensuremath{\text{ }}\ensuremath{\text{ }}\ensuremath{\text{ }}\NormalTok{\textbackslash{}begin\{column\}[T]\{5cm\}\ensuremath{\text{ }}}\CommentTok{\%\ensuremath{\text{ }}alternative\ensuremath{\text{ }}top-align\ensuremath{\text{ }}that\textquotesingle{}s\ensuremath{\text{ }}better\ensuremath{\text{ }}for\ensuremath{\text{ }}graphics}\newline
\ensuremath{\text{ }}\ensuremath{\text{ }}\ensuremath{\text{ }}\ensuremath{\text{ }}\ensuremath{\text{ }}\ensuremath{\text{ }}\ensuremath{\text{ }}\ensuremath{\text{ }}\ensuremath{\text{ }}\ensuremath{\text{ }}\NormalTok{\textbackslash{}includegraphics[height=3cm]\{graphic.png\}}\newline
\ensuremath{\text{ }}\ensuremath{\text{ }}\ensuremath{\text{ }}\ensuremath{\text{ }}\ensuremath{\text{ }}\NormalTok{\textbackslash{}end\{column\}}\newline
\ensuremath{\text{ }}\ensuremath{\text{ }}\ensuremath{\text{ }}\ensuremath{\text{ }}\ensuremath{\text{ }}\NormalTok{\textbackslash{}end\{columns\}}\newline
\NormalTok{\textbackslash{}end\{frame\}\ensuremath{\text{ }}}\newline
\end{Highlighting}
\end{Shaded}



\begin{minipage}{1.0\linewidth}
\begin{center}
\includegraphics[width=1.0\linewidth,height=6.5in,keepaspectratio]{../images/159.png}
\end{center}
\raggedright{}\myfigurewithcaption{159}{Example of columns in Beamer}
\end{minipage}\vspace{0.75cm}


\subsubsection{Blocks}
\label{750}

Enclosing text in the {\itshape \setmainfont[Path=/usr/share/fonts/truetype/cmu/,UprightFont=cmunrm.ttf,BoldFont=cmunbx.ttf,ItalicFont=cmunti.ttf,BoldItalicFont=cmunbi.ttf]{cmunti.ttf}\setmonofont[Path=/usr/share/fonts/truetype/cmu/,UprightFont=cmuntt.ttf,BoldFont=cmuntb.ttf,ItalicFont=cmunit.ttf,BoldItalicFont=cmuntx.ttf]{cmunti.ttf}\itshape block}{$\text{ }$}\setmainfont[Path=/usr/share/fonts/truetype/cmu/,UprightFont=cmunrm.ttf,BoldFont=cmunbx.ttf,ItalicFont=cmunti.ttf,BoldItalicFont=cmunbi.ttf]{cmunrm.ttf}\setmonofont[Path=/usr/share/fonts/truetype/cmu/,UprightFont=cmuntt.ttf,BoldFont=cmuntb.ttf,ItalicFont=cmunit.ttf,BoldItalicFont=cmuntx.ttf]{cmunrm.ttf} environment creates a distinct, headed block of text (a blank heading can be used). This allows to visually distinguish parts of a slide easily. There are three basic types of block. Their formatting depends on the theme being used.

Simple

\begin{Shaded}
\begin{Highlighting}[]

\NormalTok{\textbackslash{}begin\{frame\}}\newline
\ensuremath{\text{ }}\newline
\ensuremath{\text{ }}\ensuremath{\text{ }}\ensuremath{\text{ }}\NormalTok{\textbackslash{}begin\{block\}\{This\ensuremath{\text{ }}is\ensuremath{\text{ }}a\ensuremath{\text{ }}Block\}}\newline
\ensuremath{\text{ }}\ensuremath{\text{ }}\ensuremath{\text{ }}\ensuremath{\text{ }}\ensuremath{\text{ }}\ensuremath{\text{ }}\NormalTok{This\ensuremath{\text{ }}is\ensuremath{\text{ }}important\ensuremath{\text{ }}information}\newline
\ensuremath{\text{ }}\ensuremath{\text{ }}\ensuremath{\text{ }}\NormalTok{\textbackslash{}end\{block\}}\newline
\ensuremath{\text{ }}\newline
\ensuremath{\text{ }}\ensuremath{\text{ }}\ensuremath{\text{ }}\NormalTok{\textbackslash{}begin\{alertblock\}\{This\ensuremath{\text{ }}is\ensuremath{\text{ }}an\ensuremath{\text{ }}Alert\ensuremath{\text{ }}block\}}\newline
\ensuremath{\text{ }}\ensuremath{\text{ }}\ensuremath{\text{ }}\NormalTok{This\ensuremath{\text{ }}is\ensuremath{\text{ }}an\ensuremath{\text{ }}important\ensuremath{\text{ }}alert}\newline
\ensuremath{\text{ }}\ensuremath{\text{ }}\ensuremath{\text{ }}\NormalTok{\textbackslash{}end\{alertblock\}}\newline
\ensuremath{\text{ }}\newline
\ensuremath{\text{ }}\ensuremath{\text{ }}\ensuremath{\text{ }}\NormalTok{\textbackslash{}begin\{exampleblock\}\{This\ensuremath{\text{ }}is\ensuremath{\text{ }}an\ensuremath{\text{ }}Example\ensuremath{\text{ }}block\}}\newline
\ensuremath{\text{ }}\ensuremath{\text{ }}\ensuremath{\text{ }}\NormalTok{This\ensuremath{\text{ }}is\ensuremath{\text{ }}an\ensuremath{\text{ }}example\ensuremath{\text{ }}}\newline
\ensuremath{\text{ }}\ensuremath{\text{ }}\ensuremath{\text{ }}\NormalTok{\textbackslash{}end\{exampleblock\}}\newline
\ensuremath{\text{ }}\newline
\NormalTok{\textbackslash{}end\{frame\}}\newline
\end{Highlighting}
\end{Shaded}



\begin{minipage}{1.0\linewidth}
\begin{center}
\includegraphics[width=1.0\linewidth,height=6.5in,keepaspectratio]{../images/160.png}
\end{center}
\raggedright{}\myfigurewithcaption{160}{Ejemplo de bloques en una presentación con Beamer}
\end{minipage}\vspace{0.75cm}


\subsection{PDF options}
\label{751}

You can specify the default options of your PDF.\myfootnote{Other possible values are defined in the \myfnhref{http://mirror.switch.ch/ftp/mirror/tex/macros/latex/contrib/hyperref/doc/manual.html\#TBL-7-40-1}{hyperref manual}}


\begin{Shaded}
\begin{Highlighting}[]

\NormalTok{\textbackslash{}hypersetup\{pdfstartview=\{Fit\}\}\ensuremath{\text{ }}}\CommentTok{\%\ensuremath{\text{ }}fits\ensuremath{\text{ }}the\ensuremath{\text{ }}presentation\ensuremath{\text{ }}to\ensuremath{\text{ }}the\ensuremath{\text{ }}window\ensuremath{\text{ }}when\ensuremath{\text{ }}first}\newline
\ensuremath{\text{ }}\NormalTok{displayed}\newline
\ensuremath{\text{ }}\newline
\end{Highlighting}
\end{Shaded}

\subsection{Numbering slides}
\label{752}
It is possible to number slides using this snippet:


\begin{Shaded}
\begin{Highlighting}[]

\NormalTok{\textbackslash{}insertframenumber/\textbackslash{}inserttotalframenumber}\newline
\end{Highlighting}
\end{Shaded}


However, this poses two problems for some presentation authors: the title slide is numbered as the first one, and the appendix or so-{}called \symbol{34}backup\symbol{34} (aka appendix, reserve) slides are included in the total count despite them not being intended to be public until a \symbol{34}hard\symbol{34} question is asked.\myfootnote{\myfnhref{http://web.stanford.edu/~dgleich/notebook/2009/05/appendix_slides_in_beamer_cont_1.html}{Appendix Slides in Beamer: Controlling frame numbers}} This is where two features come in:
\begin{myitemize}
\item{}  Ability to reset the frames counter at any slide. For instance, this may be inserted at the title slide to avoid counting it:
\end{myitemize}


\begin{Shaded}
\begin{Highlighting}[]

\NormalTok{\textbackslash{}addtocounter\{framenumber\}\{-1\}}\newline
\end{Highlighting}
\end{Shaded}

Or alternatively this:

\begin{Shaded}
\begin{Highlighting}[]

\NormalTok{\textbackslash{}setcounter\{framenumber\}\{0\}\ensuremath{\text{ }}or\ensuremath{\text{ }}\textbackslash{}setcounter\{framenumber\}\{1\}}\newline
\end{Highlighting}
\end{Shaded}

\begin{myitemize}
\item{}  The first of the above applies to section slides to avoid counting them.
\item{}  This stuff works around the problem of counting the backup slides:
\end{myitemize}


\begin{Shaded}
\begin{Highlighting}[]

\CommentTok{\%\ensuremath{\text{ }}(Thanks,\ensuremath{\text{ }}David\ensuremath{\text{ }}Gleich!)}\newline
\CommentTok{\%\ensuremath{\text{ }}All\ensuremath{\text{ }}your\ensuremath{\text{ }}regular\ensuremath{\text{ }}slides}\newline
\CommentTok{\%\ensuremath{\text{ }}After\ensuremath{\text{ }}your\ensuremath{\text{ }}last\ensuremath{\text{ }}numbered\ensuremath{\text{ }}slide}\newline
\NormalTok{\textbackslash{}appendix}\newline
\NormalTok{\textbackslash{}newcounter\{finalframe\}}\newline
\NormalTok{\textbackslash{}setcounter\{finalframe\}\{\textbackslash{}value\{framenumber\}\}}\newline
\CommentTok{\%\ensuremath{\text{ }}Backup\ensuremath{\text{ }}frames}\newline
\NormalTok{\textbackslash{}setcounter\{framenumber\}\{\textbackslash{}value\{finalframe\}\}}\newline
\NormalTok{\textbackslash{}end\{document\}}\newline
\end{Highlighting}
\end{Shaded}

\section{The powerdot package}
\label{753}

The powerdot package is an alternative to beamer. It is available from \myhref{http://www.ctan.org/tex-archive/macros/latex/contrib/powerdot/}{CTAN}.  The \myhref{http://mirrors.ctan.org/macros/latex/contrib/powerdot/doc/powerdot.pdf}{documentation} explains the features in great detail.

The powerdot package is loaded by calling the {\ttfamily \setmainfont[Path=/usr/share/fonts/truetype/cmu/,UprightFont=cmunrm.ttf,BoldFont=cmunbx.ttf,ItalicFont=cmunti.ttf,BoldItalicFont=cmunbi.ttf]{cmuntt.ttf}\setmonofont[Path=/usr/share/fonts/truetype/cmu/,UprightFont=cmuntt.ttf,BoldFont=cmuntb.ttf,ItalicFont=cmunit.ttf,BoldItalicFont=cmuntx.ttf]{cmuntt.ttf}\ttfamily powerdot}{$\text{ }$}\setmainfont[Path=/usr/share/fonts/truetype/cmu/,UprightFont=cmunrm.ttf,BoldFont=cmunbx.ttf,ItalicFont=cmunti.ttf,BoldItalicFont=cmunbi.ttf]{cmunrm.ttf}\setmonofont[Path=/usr/share/fonts/truetype/cmu/,UprightFont=cmuntt.ttf,BoldFont=cmuntb.ttf,ItalicFont=cmunit.ttf,BoldItalicFont=cmuntx.ttf]{cmunrm.ttf} class:

\begin{Shaded}
\begin{Highlighting}[]

\NormalTok{\textbackslash{}documentclass\{powerdot\}}\newline
\end{Highlighting}
\end{Shaded}

The usual header information may then be specified.

Inside the usual {\ttfamily \setmainfont[Path=/usr/share/fonts/truetype/cmu/,UprightFont=cmunrm.ttf,BoldFont=cmunbx.ttf,ItalicFont=cmunti.ttf,BoldItalicFont=cmunbi.ttf]{cmuntt.ttf}\setmonofont[Path=/usr/share/fonts/truetype/cmu/,UprightFont=cmuntt.ttf,BoldFont=cmuntb.ttf,ItalicFont=cmunit.ttf,BoldItalicFont=cmuntx.ttf]{cmuntt.ttf}\ttfamily document}{$\text{ }$}\setmainfont[Path=/usr/share/fonts/truetype/cmu/,UprightFont=cmunrm.ttf,BoldFont=cmunbx.ttf,ItalicFont=cmunti.ttf,BoldItalicFont=cmunbi.ttf]{cmunrm.ttf}\setmonofont[Path=/usr/share/fonts/truetype/cmu/,UprightFont=cmuntt.ttf,BoldFont=cmuntb.ttf,ItalicFont=cmunit.ttf,BoldItalicFont=cmuntx.ttf]{cmunrm.ttf} environment, multiple {\ttfamily \setmainfont[Path=/usr/share/fonts/truetype/cmu/,UprightFont=cmunrm.ttf,BoldFont=cmunbx.ttf,ItalicFont=cmunti.ttf,BoldItalicFont=cmunbi.ttf]{cmuntt.ttf}\setmonofont[Path=/usr/share/fonts/truetype/cmu/,UprightFont=cmuntt.ttf,BoldFont=cmuntb.ttf,ItalicFont=cmunit.ttf,BoldItalicFont=cmuntx.ttf]{cmuntt.ttf}\ttfamily slide}{$\text{ }$}\setmainfont[Path=/usr/share/fonts/truetype/cmu/,UprightFont=cmunrm.ttf,BoldFont=cmunbx.ttf,ItalicFont=cmunti.ttf,BoldItalicFont=cmunbi.ttf]{cmunrm.ttf}\setmonofont[Path=/usr/share/fonts/truetype/cmu/,UprightFont=cmuntt.ttf,BoldFont=cmuntb.ttf,ItalicFont=cmunit.ttf,BoldItalicFont=cmuntx.ttf]{cmunrm.ttf} environments specify the content to be put on each slide.

\begin{Shaded}
\begin{Highlighting}[]

\NormalTok{\textbackslash{}begin\{document\}}\newline
\ensuremath{\text{ }}\ensuremath{\text{ }}\NormalTok{\textbackslash{}begin\{slide\}\{This\ensuremath{\text{ }}is\ensuremath{\text{ }}the\ensuremath{\text{ }}first\ensuremath{\text{ }}slide\}}\newline
\ensuremath{\text{ }}\ensuremath{\text{ }}\ensuremath{\text{ }}\ensuremath{\text{ }}\CommentTok{\%Content\ensuremath{\text{ }}goes\ensuremath{\text{ }}here}\newline
\ensuremath{\text{ }}\ensuremath{\text{ }}\NormalTok{\textbackslash{}end\{slide\}}\newline
\ensuremath{\text{ }}\ensuremath{\text{ }}\NormalTok{\textbackslash{}begin\{slide\}\{This\ensuremath{\text{ }}is\ensuremath{\text{ }}the\ensuremath{\text{ }}second\ensuremath{\text{ }}slide\}}\newline
\ensuremath{\text{ }}\ensuremath{\text{ }}\ensuremath{\text{ }}\ensuremath{\text{ }}\CommentTok{\%More\ensuremath{\text{ }}content\ensuremath{\text{ }}goes\ensuremath{\text{ }}here}\newline
\ensuremath{\text{ }}\ensuremath{\text{ }}\NormalTok{\textbackslash{}end\{slide\}}\newline
\CommentTok{\%\ensuremath{\text{ }}etc}\newline
\NormalTok{\textbackslash{}end\{document\}}\newline
\end{Highlighting}
\end{Shaded}

\section{References}
\label{754}

\LaTeXNullTemplate{}
\section{Links}
\label{755}
\begin{myitemize}
\item{}  \myhref{https://en.wikipedia.org/wiki/Beamer\%20\%28LaTeX\%29}{Wikipedia:Beamer (LaTeX)}
\item{}  \myhref{http://www.ctan.org/tex-archive/macros/latex/contrib/beamer/doc/beameruserguide.pdf}{Beamer user guide} (pdf) from CTAN
\item{}  \myhref{http://mirrors.ctan.org/macros/latex/contrib/powerdot/doc/powerdot.pdf}{The powerdot class} (pdf) from CTAN 
\item{}  \myhref{http://www.math-linux.com/spip.php?article77}{A tutorial for creating presentations using beamer}
\end{myitemize}




\myhref{https://fr.wikibooks.org/wiki/LaTeX\%2FFaire\%20des\%20pr\%C3\%A9sentations}{fr:LaTeX/Faire des présentations}
\myhref{https://sr.wikibooks.org/wiki/LaTeX\%2F\%D0\%9F\%D1\%80\%D0\%B5\%D0\%B7\%D0\%B5\%D0\%BD\%D1\%82\%D0\%B0\%D1\%86\%D0\%B8\%D1\%98\%D0\%B5}{sr:LaTeX/Презентације}\chapter{Teacher\textquotesingle{}s Corner}

\myminitoc
\label{756}

\label{757}

\section{Intro}
\label{758}

LaTeX has specific features for teachers. We present the {\bfseries \setmainfont[Path=/usr/share/fonts/truetype/cmu/,UprightFont=cmunrm.ttf,BoldFont=cmunbx.ttf,ItalicFont=cmunti.ttf,BoldItalicFont=cmunbi.ttf]{cmunbx.ttf}\setmonofont[Path=/usr/share/fonts/truetype/cmu/,UprightFont=cmuntt.ttf,BoldFont=cmuntb.ttf,ItalicFont=cmunit.ttf,BoldItalicFont=cmuntx.ttf]{cmunbx.ttf}\bfseries exam}{$\text{ }$}\setmainfont[Path=/usr/share/fonts/truetype/cmu/,UprightFont=cmunrm.ttf,BoldFont=cmunbx.ttf,ItalicFont=cmunti.ttf,BoldItalicFont=cmunbi.ttf]{cmunrm.ttf}\setmonofont[Path=/usr/share/fonts/truetype/cmu/,UprightFont=cmuntt.ttf,BoldFont=cmuntb.ttf,ItalicFont=cmunit.ttf,BoldItalicFont=cmuntx.ttf]{cmunrm.ttf} class\myfootnote{\myfnhref{http://www-math.mit.edu/~psh/exam/examdoc.pdf}{examdoc} Using the exam document class} which is useful for designing exams and exercises with solutions. Interested people could also have a look at the {\bfseries \setmainfont[Path=/usr/share/fonts/truetype/cmu/,UprightFont=cmunrm.ttf,BoldFont=cmunbx.ttf,ItalicFont=cmunti.ttf,BoldItalicFont=cmunbi.ttf]{cmunbx.ttf}\setmonofont[Path=/usr/share/fonts/truetype/cmu/,UprightFont=cmuntt.ttf,BoldFont=cmuntb.ttf,ItalicFont=cmunit.ttf,BoldItalicFont=cmuntx.ttf]{cmunbx.ttf}\bfseries probsoln}{$\text{ }$}\setmainfont[Path=/usr/share/fonts/truetype/cmu/,UprightFont=cmunrm.ttf,BoldFont=cmunbx.ttf,ItalicFont=cmunti.ttf,BoldItalicFont=cmunbi.ttf]{cmunrm.ttf}\setmonofont[Path=/usr/share/fonts/truetype/cmu/,UprightFont=cmuntt.ttf,BoldFont=cmuntb.ttf,ItalicFont=cmunit.ttf,BoldItalicFont=cmuntx.ttf]{cmunrm.ttf} package\myfootnote{\myfnhref{http://www.tex.ac.uk/tex-archive/macros/latex/contrib/probsoln/probsoln.pdf}{Probsoln} Creating problem sheets optionally
with solutions}, the {\bfseries \setmainfont[Path=/usr/share/fonts/truetype/cmu/,UprightFont=cmunrm.ttf,BoldFont=cmunbx.ttf,ItalicFont=cmunti.ttf,BoldItalicFont=cmunbi.ttf]{cmunbx.ttf}\setmonofont[Path=/usr/share/fonts/truetype/cmu/,UprightFont=cmuntt.ttf,BoldFont=cmuntb.ttf,ItalicFont=cmunit.ttf,BoldItalicFont=cmuntx.ttf]{cmunbx.ttf}\bfseries mathexm}{$\text{ }$}\setmainfont[Path=/usr/share/fonts/truetype/cmu/,UprightFont=cmunrm.ttf,BoldFont=cmunbx.ttf,ItalicFont=cmunti.ttf,BoldItalicFont=cmunbi.ttf]{cmunrm.ttf}\setmonofont[Path=/usr/share/fonts/truetype/cmu/,UprightFont=cmuntt.ttf,BoldFont=cmuntb.ttf,ItalicFont=cmunit.ttf,BoldItalicFont=cmuntx.ttf]{cmunrm.ttf} document class\myfootnote{\myfnhref{http://mirrors.ctan.org/macros/latex/contrib/mathexam/doc/mathexam.pdf}{mathexm documentation}}, or the {\bfseries \setmainfont[Path=/usr/share/fonts/truetype/cmu/,UprightFont=cmunrm.ttf,BoldFont=cmunbx.ttf,ItalicFont=cmunti.ttf,BoldItalicFont=cmunbi.ttf]{cmunbx.ttf}\setmonofont[Path=/usr/share/fonts/truetype/cmu/,UprightFont=cmuntt.ttf,BoldFont=cmuntb.ttf,ItalicFont=cmunit.ttf,BoldItalicFont=cmuntx.ttf]{cmunbx.ttf}\bfseries exsheets}{$\text{ }$}\setmainfont[Path=/usr/share/fonts/truetype/cmu/,UprightFont=cmunrm.ttf,BoldFont=cmunbx.ttf,ItalicFont=cmunti.ttf,BoldItalicFont=cmunbi.ttf]{cmunrm.ttf}\setmonofont[Path=/usr/share/fonts/truetype/cmu/,UprightFont=cmuntt.ttf,BoldFont=cmuntb.ttf,ItalicFont=cmunit.ttf,BoldItalicFont=cmuntx.ttf]{cmunrm.ttf} package\myfootnote{\myfnhref{http://mirrors.ctan.org/macros/latex/contrib/exsheets/exsheets_en.pdf}{exsheets documentation} Create exercise sheets and exams}.
\section{The exam class}
\label{759}

We present the {\bfseries \setmainfont[Path=/usr/share/fonts/truetype/cmu/,UprightFont=cmunrm.ttf,BoldFont=cmunbx.ttf,ItalicFont=cmunti.ttf,BoldItalicFont=cmunbi.ttf]{cmunbx.ttf}\setmonofont[Path=/usr/share/fonts/truetype/cmu/,UprightFont=cmuntt.ttf,BoldFont=cmuntb.ttf,ItalicFont=cmunit.ttf,BoldItalicFont=cmuntx.ttf]{cmunbx.ttf}\bfseries exam}{$\text{ }$}\setmainfont[Path=/usr/share/fonts/truetype/cmu/,UprightFont=cmunrm.ttf,BoldFont=cmunbx.ttf,ItalicFont=cmunti.ttf,BoldItalicFont=cmunbi.ttf]{cmunrm.ttf}\setmonofont[Path=/usr/share/fonts/truetype/cmu/,UprightFont=cmuntt.ttf,BoldFont=cmuntb.ttf,ItalicFont=cmunit.ttf,BoldItalicFont=cmuntx.ttf]{cmunrm.ttf} class. 
The exam class is well suited to design exams with solutions. You just have to specify in the preamble if you want the solutions to be printed or not. You can also count the number of points. 
\subsection{Preamble}
\label{760}

In the preamble you can specify the following lines : 
\begin{Shaded}
\begin{Highlighting}[]

\NormalTok{\textbackslash{}documentclass[a4paper,11pt]\{exam\}}
\NormalTok{\textbackslash{}printanswers }\CommentTok{% If you want to print answers}
\CommentTok{% \textbackslash{}noprintanswers % If you don't want to print answers}
\NormalTok{\textbackslash{}addpoints }\CommentTok{% if you want to count the points}
\CommentTok{% \textbackslash{}noaddpoints % if you don't want to count the points}
\CommentTok{% Specifies the way question are displayed:}
\NormalTok{\textbackslash{}qformat\{\textbackslash{}textbf\{Question\textbackslash{}thequestion\}\textbackslash{}quad(\textbackslash{}thepoints)\textbackslash{}hfill\}}
\NormalTok{\textbackslash{}usepackage\{color\} }\CommentTok{% defines a new color}
\NormalTok{\textbackslash{}definecolor\{SolutionColor\}\{rgb\}\{0.8,0.9,1\} }\CommentTok{% light blue}
\NormalTok{\textbackslash{}shadedsolutions }\CommentTok{% defines the style of the solution environment}
\CommentTok{% \textbackslash{}framedsolutions % defines the style of the solution environment}
\CommentTok{% Defines the title of the solution environment:}
\NormalTok{\textbackslash{}renewcommand\{\textbackslash{}solutiontitle\}\{\textbackslash{}noindent\textbackslash{}textbf\{Solution:\}\textbackslash{}par\textbackslash{}noindent\}}
\end{Highlighting}
\end{Shaded}


You can replace the 3 first lines with the following : 
\begin{Shaded}
\begin{Highlighting}[]

\NormalTok{\textbackslash{}documentclass[a4paper,11pt,answers,addpoints]\{exam\}}
\end{Highlighting}
\end{Shaded}

\subsection{Document}
\label{761}

\begin{myitemize}
\item{}  The exam is included in the {\bfseries \setmainfont[Path=/usr/share/fonts/truetype/cmu/,UprightFont=cmunrm.ttf,BoldFont=cmunbx.ttf,ItalicFont=cmunti.ttf,BoldItalicFont=cmunbi.ttf]{cmunbx.ttf}\setmonofont[Path=/usr/share/fonts/truetype/cmu/,UprightFont=cmuntt.ttf,BoldFont=cmuntb.ttf,ItalicFont=cmunit.ttf,BoldItalicFont=cmuntx.ttf]{cmunbx.ttf}\bfseries questions}{$\text{ }$}\setmainfont[Path=/usr/share/fonts/truetype/cmu/,UprightFont=cmunrm.ttf,BoldFont=cmunbx.ttf,ItalicFont=cmunti.ttf,BoldItalicFont=cmunbi.ttf]{cmunrm.ttf}\setmonofont[Path=/usr/share/fonts/truetype/cmu/,UprightFont=cmuntt.ttf,BoldFont=cmuntb.ttf,ItalicFont=cmunit.ttf,BoldItalicFont=cmuntx.ttf]{cmunrm.ttf} environment.
\item{}  The command {\bfseries \setmainfont[Path=/usr/share/fonts/truetype/cmu/,UprightFont=cmunrm.ttf,BoldFont=cmunbx.ttf,ItalicFont=cmunti.ttf,BoldItalicFont=cmunbi.ttf]{cmunbx.ttf}\setmonofont[Path=/usr/share/fonts/truetype/cmu/,UprightFont=cmuntt.ttf,BoldFont=cmuntb.ttf,ItalicFont=cmunit.ttf,BoldItalicFont=cmuntx.ttf]{cmunbx.ttf}\bfseries \textbackslash{}question}{$\text{ }$}\setmainfont[Path=/usr/share/fonts/truetype/cmu/,UprightFont=cmunrm.ttf,BoldFont=cmunbx.ttf,ItalicFont=cmunti.ttf,BoldItalicFont=cmunbi.ttf]{cmunrm.ttf}\setmonofont[Path=/usr/share/fonts/truetype/cmu/,UprightFont=cmuntt.ttf,BoldFont=cmuntb.ttf,ItalicFont=cmunit.ttf,BoldItalicFont=cmuntx.ttf]{cmunrm.ttf} introduces a new question. 
\item{}  The number of points is specified in squared brackets.
\item{}  The solution is given in the {\bfseries \setmainfont[Path=/usr/share/fonts/truetype/cmu/,UprightFont=cmunrm.ttf,BoldFont=cmunbx.ttf,ItalicFont=cmunti.ttf,BoldItalicFont=cmunbi.ttf]{cmunbx.ttf}\setmonofont[Path=/usr/share/fonts/truetype/cmu/,UprightFont=cmuntt.ttf,BoldFont=cmuntb.ttf,ItalicFont=cmunit.ttf,BoldItalicFont=cmuntx.ttf]{cmunbx.ttf}\bfseries solution}{$\text{ }$}\setmainfont[Path=/usr/share/fonts/truetype/cmu/,UprightFont=cmunrm.ttf,BoldFont=cmunbx.ttf,ItalicFont=cmunti.ttf,BoldItalicFont=cmunbi.ttf]{cmunrm.ttf}\setmonofont[Path=/usr/share/fonts/truetype/cmu/,UprightFont=cmuntt.ttf,BoldFont=cmuntb.ttf,ItalicFont=cmunit.ttf,BoldItalicFont=cmuntx.ttf]{cmunrm.ttf} environment. It appears only if {\bfseries \setmainfont[Path=/usr/share/fonts/truetype/cmu/,UprightFont=cmunrm.ttf,BoldFont=cmunbx.ttf,ItalicFont=cmunti.ttf,BoldItalicFont=cmunbi.ttf]{cmunbx.ttf}\setmonofont[Path=/usr/share/fonts/truetype/cmu/,UprightFont=cmuntt.ttf,BoldFont=cmuntb.ttf,ItalicFont=cmunit.ttf,BoldItalicFont=cmuntx.ttf]{cmunbx.ttf}\bfseries \textbackslash{}printanswers}{$\text{ }$}\setmainfont[Path=/usr/share/fonts/truetype/cmu/,UprightFont=cmunrm.ttf,BoldFont=cmunbx.ttf,ItalicFont=cmunti.ttf,BoldItalicFont=cmunbi.ttf]{cmunrm.ttf}\setmonofont[Path=/usr/share/fonts/truetype/cmu/,UprightFont=cmuntt.ttf,BoldFont=cmuntb.ttf,ItalicFont=cmunit.ttf,BoldItalicFont=cmuntx.ttf]{cmunrm.ttf} or {\bfseries \setmainfont[Path=/usr/share/fonts/truetype/cmu/,UprightFont=cmunrm.ttf,BoldFont=cmunbx.ttf,ItalicFont=cmunti.ttf,BoldItalicFont=cmunbi.ttf]{cmunbx.ttf}\setmonofont[Path=/usr/share/fonts/truetype/cmu/,UprightFont=cmuntt.ttf,BoldFont=cmuntb.ttf,ItalicFont=cmunit.ttf,BoldItalicFont=cmuntx.ttf]{cmunbx.ttf}\bfseries answers}{$\text{ }$}\setmainfont[Path=/usr/share/fonts/truetype/cmu/,UprightFont=cmunrm.ttf,BoldFont=cmunbx.ttf,ItalicFont=cmunti.ttf,BoldItalicFont=cmunbi.ttf]{cmunrm.ttf}\setmonofont[Path=/usr/share/fonts/truetype/cmu/,UprightFont=cmuntt.ttf,BoldFont=cmuntb.ttf,ItalicFont=cmunit.ttf,BoldItalicFont=cmuntx.ttf]{cmunrm.ttf} as an option of the {\bfseries \setmainfont[Path=/usr/share/fonts/truetype/cmu/,UprightFont=cmunrm.ttf,BoldFont=cmunbx.ttf,ItalicFont=cmunti.ttf,BoldItalicFont=cmunbi.ttf]{cmunbx.ttf}\setmonofont[Path=/usr/share/fonts/truetype/cmu/,UprightFont=cmuntt.ttf,BoldFont=cmuntb.ttf,ItalicFont=cmunit.ttf,BoldItalicFont=cmuntx.ttf]{cmunbx.ttf}\bfseries \textbackslash{}documentclass}{$\text{ }$}\setmainfont[Path=/usr/share/fonts/truetype/cmu/,UprightFont=cmunrm.ttf,BoldFont=cmunbx.ttf,ItalicFont=cmunti.ttf,BoldItalicFont=cmunbi.ttf]{cmunrm.ttf}\setmonofont[Path=/usr/share/fonts/truetype/cmu/,UprightFont=cmuntt.ttf,BoldFont=cmuntb.ttf,ItalicFont=cmunit.ttf,BoldItalicFont=cmuntx.ttf]{cmunrm.ttf} are specified in the preamble.
\end{myitemize}


Here is an example : 


\begin{Shaded}
\begin{Highlighting}[]

\NormalTok{\textbackslash{}begin\{questions\}\ensuremath{\text{ }}}\CommentTok{\%\ensuremath{\text{ }}Begins\ensuremath{\text{ }}the\ensuremath{\text{ }}questions\ensuremath{\text{ }}environment}\newline
\NormalTok{\textbackslash{}question[2]\ensuremath{\text{ }}What\ensuremath{\text{ }}is\ensuremath{\text{ }}the\ensuremath{\text{ }}solution?\ensuremath{\text{ }}}\CommentTok{\%\ensuremath{\text{ }}Introduces\ensuremath{\text{ }}a\ensuremath{\text{ }}new\ensuremath{\text{ }}question\ensuremath{\text{ }}which\ensuremath{\text{ }}is\ensuremath{\text{ }}worth\ensuremath{\text{ }}2}\newline
\ensuremath{\text{ }}\NormalTok{points}\newline
\NormalTok{\textbackslash{}begin\{solution\}\ensuremath{\text{ }}}\newline
\NormalTok{Here\ensuremath{\text{ }}is\ensuremath{\text{ }}the\ensuremath{\text{ }}solution\ensuremath{\text{ }}}\newline
\NormalTok{\textbackslash{}end\{solution\}}\newline
\NormalTok{\textbackslash{}question[5]\ensuremath{\text{ }}What\ensuremath{\text{ }}is\ensuremath{\text{ }}your\ensuremath{\text{ }}opinion?}\newline
\NormalTok{\textbackslash{}begin\{solution\}}\newline
\NormalTok{This\ensuremath{\text{ }}is\ensuremath{\text{ }}my\ensuremath{\text{ }}opinion}\newline
\NormalTok{\textbackslash{}end\{solution\}}\newline
\NormalTok{\textbackslash{}end\{questions\}}\newline
\end{Highlighting}
\end{Shaded}


It is also possible to add stuff only if answers are printed using the {\bfseries \setmainfont[Path=/usr/share/fonts/truetype/cmu/,UprightFont=cmunrm.ttf,BoldFont=cmunbx.ttf,ItalicFont=cmunti.ttf,BoldItalicFont=cmunbi.ttf]{cmunbx.ttf}\setmonofont[Path=/usr/share/fonts/truetype/cmu/,UprightFont=cmuntt.ttf,BoldFont=cmuntb.ttf,ItalicFont=cmunit.ttf,BoldItalicFont=cmuntx.ttf]{cmunbx.ttf}\bfseries \textbackslash{}ifprintanswers}{$\text{ }$}\setmainfont[Path=/usr/share/fonts/truetype/cmu/,UprightFont=cmunrm.ttf,BoldFont=cmunbx.ttf,ItalicFont=cmunti.ttf,BoldItalicFont=cmunbi.ttf]{cmunrm.ttf}\setmonofont[Path=/usr/share/fonts/truetype/cmu/,UprightFont=cmuntt.ttf,BoldFont=cmuntb.ttf,ItalicFont=cmunit.ttf,BoldItalicFont=cmuntx.ttf]{cmunrm.ttf} command. 


\begin{Shaded}
\begin{Highlighting}[]

\NormalTok{\textbackslash{}ifprintanswers}\newline
\NormalTok{Only\ensuremath{\text{ }}if\ensuremath{\text{ }}answers\ensuremath{\text{ }}are\ensuremath{\text{ }}printed}\newline
\NormalTok{\textbackslash{}else}\newline
\NormalTok{Only\ensuremath{\text{ }}if\ensuremath{\text{ }}answers\ensuremath{\text{ }}are\ensuremath{\text{ }}not\ensuremath{\text{ }}printed}\newline
\NormalTok{\textbackslash{}fi}\newline
\end{Highlighting}
\end{Shaded}

\subsection{Introduction}
\label{762}

The macro {\bfseries \setmainfont[Path=/usr/share/fonts/truetype/cmu/,UprightFont=cmunrm.ttf,BoldFont=cmunbx.ttf,ItalicFont=cmunti.ttf,BoldItalicFont=cmunbi.ttf]{cmunbx.ttf}\setmonofont[Path=/usr/share/fonts/truetype/cmu/,UprightFont=cmuntt.ttf,BoldFont=cmuntb.ttf,ItalicFont=cmunit.ttf,BoldItalicFont=cmuntx.ttf]{cmunbx.ttf}\bfseries \textbackslash{}numquestions}{$\text{ }$}\setmainfont[Path=/usr/share/fonts/truetype/cmu/,UprightFont=cmunrm.ttf,BoldFont=cmunbx.ttf,ItalicFont=cmunti.ttf,BoldItalicFont=cmunbi.ttf]{cmunrm.ttf}\setmonofont[Path=/usr/share/fonts/truetype/cmu/,UprightFont=cmuntt.ttf,BoldFont=cmuntb.ttf,ItalicFont=cmunit.ttf,BoldItalicFont=cmuntx.ttf]{cmunrm.ttf} gives the total number of questions. 
The macro {\bfseries \setmainfont[Path=/usr/share/fonts/truetype/cmu/,UprightFont=cmunrm.ttf,BoldFont=cmunbx.ttf,ItalicFont=cmunti.ttf,BoldItalicFont=cmunbi.ttf]{cmunbx.ttf}\setmonofont[Path=/usr/share/fonts/truetype/cmu/,UprightFont=cmuntt.ttf,BoldFont=cmuntb.ttf,ItalicFont=cmunit.ttf,BoldItalicFont=cmuntx.ttf]{cmunbx.ttf}\bfseries \textbackslash{}numpoints}{$\text{ }$}\setmainfont[Path=/usr/share/fonts/truetype/cmu/,UprightFont=cmunrm.ttf,BoldFont=cmunbx.ttf,ItalicFont=cmunti.ttf,BoldItalicFont=cmunbi.ttf]{cmunrm.ttf}\setmonofont[Path=/usr/share/fonts/truetype/cmu/,UprightFont=cmuntt.ttf,BoldFont=cmuntb.ttf,ItalicFont=cmunit.ttf,BoldItalicFont=cmuntx.ttf]{cmunrm.ttf} gives the total number of points.


\begin{Shaded}
\begin{Highlighting}[]

\NormalTok{\textbackslash{}begin\{minipage\}\{.8\textbackslash{}textwidth\}}\newline
\NormalTok{This\ensuremath{\text{ }}exam\ensuremath{\text{ }}includes\ensuremath{\text{ }}\textbackslash{}numquestions\textbackslash{}\ensuremath{\text{ }}questions.\ensuremath{\text{ }}The\ensuremath{\text{ }}total\ensuremath{\text{ }}number\ensuremath{\text{ }}of\ensuremath{\text{ }}points\ensuremath{\text{ }}is}\newline
\ensuremath{\text{ }}\NormalTok{\textbackslash{}numpoints.}\newline
\NormalTok{\textbackslash{}end\{minipage\}}\newline
\end{Highlighting}
\end{Shaded}


The backslash after {\bfseries \setmainfont[Path=/usr/share/fonts/truetype/cmu/,UprightFont=cmunrm.ttf,BoldFont=cmunbx.ttf,ItalicFont=cmunti.ttf,BoldItalicFont=cmunbi.ttf]{cmunbx.ttf}\setmonofont[Path=/usr/share/fonts/truetype/cmu/,UprightFont=cmuntt.ttf,BoldFont=cmuntb.ttf,ItalicFont=cmunit.ttf,BoldItalicFont=cmuntx.ttf]{cmunbx.ttf}\bfseries \textbackslash{}numquestion}{$\text{ }$}\setmainfont[Path=/usr/share/fonts/truetype/cmu/,UprightFont=cmunrm.ttf,BoldFont=cmunbx.ttf,ItalicFont=cmunti.ttf,BoldItalicFont=cmunbi.ttf]{cmunrm.ttf}\setmonofont[Path=/usr/share/fonts/truetype/cmu/,UprightFont=cmuntt.ttf,BoldFont=cmuntb.ttf,ItalicFont=cmunit.ttf,BoldItalicFont=cmuntx.ttf]{cmunrm.ttf} prevents the macro from gobbling the following whitespace as it normally would.
\section{References}
\label{763}

\LaTeXNullTemplate{}

\chapter{Curriculum Vitae}

\myminitoc
\label{764}

\label{765}


A {\itshape \setmainfont[Path=/usr/share/fonts/truetype/cmu/,UprightFont=cmunrm.ttf,BoldFont=cmunbx.ttf,ItalicFont=cmunti.ttf,BoldItalicFont=cmunbi.ttf]{cmunti.ttf}\setmonofont[Path=/usr/share/fonts/truetype/cmu/,UprightFont=cmuntt.ttf,BoldFont=cmuntb.ttf,ItalicFont=cmunit.ttf,BoldItalicFont=cmuntx.ttf]{cmunti.ttf}\itshape curriculum vitæ}{$\text{ }$}\setmainfont[Path=/usr/share/fonts/truetype/cmu/,UprightFont=cmunrm.ttf,BoldFont=cmunbx.ttf,ItalicFont=cmunti.ttf,BoldItalicFont=cmunbi.ttf]{cmunrm.ttf}\setmonofont[Path=/usr/share/fonts/truetype/cmu/,UprightFont=cmuntt.ttf,BoldFont=cmuntb.ttf,ItalicFont=cmunit.ttf,BoldItalicFont=cmuntx.ttf]{cmunrm.ttf} or résumé has a universal requirement: its formatting must be flawless. This is a great example of cases where the power of LaTeX comes to the front.
Thanks to its strong typographical stance, LaTeX is definitely a document processor of choice to write a CV.

Of course you can design your own CV by hand. Otherwise, you may want to use a dedicated class for that task. This way, writing a CV in LaTeX is as simple as filling the forms, and you are done. Seeveeze makes 3 of them available (ModernCV PlasmatiCV and FriggeriCV) from a simple web form: no coding or editor required.

A full list of CV packages is available at \myhref{http://www.ctan.org/topic/cv}{CTAN}.
\section{curve}
\label{766}
\LaTeXNullTemplate{} 
\section{europecv}
\label{767}

\begin{Shaded}
\begin{Highlighting}[]

 
\NormalTok{\textbackslash{}documentclass[utf8, a4paper, 10pt, helvetica, narrow, flagWB, booktabs,}
 \NormalTok{totpages, english]\{europecv\}}
\NormalTok{\textbackslash{}usepackage\{graphicx\}                        }\CommentTok{% Required to draw the flag}
\NormalTok{\textbackslash{}usepackage[a4paper, left=3cm, right=2cm, top=2cm, bottom=2cm]\{geometry\} }
\NormalTok{\textbackslash{}usepackage\{babel\}}
 
\CommentTok{% Commands europecv}
 
\NormalTok{\textbackslash{}ecvLogoWidth\{12mm\}		             }\CommentTok{% Size logo europass}
\CommentTok{%\textbackslash{}ecvLeftColumnWidth\{4cm\}		     % Size of column and vertical line}
 \NormalTok{(different from standard)}
\CommentTok{%\textbackslash{}ecvfootnote\{footnote\}	             % Foot notes}
\NormalTok{\textbackslash{}ecvname\{\textbackslash{}textsc\{Surname\}, First Name\}}
 
\CommentTok{% Personal picture}
 
\NormalTok{\textbackslash{}ecvbeforepicture\{\textbackslash{}raggedleft\}}
\NormalTok{\textbackslash{}ecvpicture[height=1in]\{namefile_pic\}   }\CommentTok{% File picture without extension        }
         
\NormalTok{\textbackslash{}ecvafterpicture\{\textbackslash{}ecvspace\{-2.5cm\} \}}
 
\CommentTok{% Address}
 
\NormalTok{\textbackslash{}ecvaddress\{Address first line\textbackslash{}\textbackslash{}& Address second line\textbackslash{}\textbackslash{}& City, State\}}
 
\CommentTok{% Telephone }
 
\NormalTok{\textbackslash{}ecvtelephone\{+44 (0) 123 4567\}}
\CommentTok{%\textbackslash{}ecvfax\{+39 01234567\}}
 
\NormalTok{\textbackslash{}ecvemail\{john@someserver\}}
 
\CommentTok{% Other personal info}
 
\NormalTok{\textbackslash{}ecvnationality\{Nationality\}}
\NormalTok{\textbackslash{}ecvdateofbirth\{01/01/1900\}}
\NormalTok{\textbackslash{}ecvgender\{Male\}}
 
\NormalTok{\textbackslash{}begin\{document\}}
 
\CommentTok{% Begin europecv environment}
 
\NormalTok{\textbackslash{}begin\{europecv\}}
 
\NormalTok{\textbackslash{}ecvpersonalinfo		}\CommentTok{% Print personal info in preamble}
 
\NormalTok{\textbackslash{}ecvitem\{\}\{\}		}\CommentTok{% 1 free line - \textbackslash{}ecvitem\{\}\{\} adds elements to a section}
\CommentTok{%\textbackslash{}ecvsection\{\}		% \textbackslash{}ecvsection\{\} adds sections}
 
\NormalTok{\textbackslash{}ecvitem\{\textbackslash{}large\textbackslash{}textbf\{Desired employment / Occupational field\}}
 \NormalTok{\}\{\textbackslash{}Large\textbackslash{}textbf\{Dream job\} \}  }\CommentTok{% desired job}
 
\CommentTok{% Sections}
 
\CommentTok{% School}
 
\NormalTok{\textbackslash{}ecvsection\{Education and training\}}
 
\NormalTok{\textbackslash{}ecvitem\{Dates\}\{From September 1900 to August 1905\}\textbackslash{}\textbackslash{}}
\NormalTok{\textbackslash{}ecvitem\{Title of qualification awarded\}\{Name of the\textbackslash{}\textbackslash{}& degree\}\textbackslash{}\textbackslash{}}
\NormalTok{\textbackslash{}ecvitem\{Principal subjects/occupational skills covered\}\{Learned skills\}\textbackslash{}\textbackslash{}}
\NormalTok{\textbackslash{}ecvitem\{Name and type of organisation providing education and training\}\{My}
 \NormalTok{University\textbackslash{}\textbackslash{}&}
\NormalTok{Address\textbackslash{}\textbackslash{}&}
\NormalTok{City\textbackslash{}\textbackslash{}& Nation\textbackslash{}\textbackslash{}&}
\NormalTok{Post code\textbackslash{}\textbackslash{}&}
\NormalTok{Tel. +44 (0) 123 45678 23\}\textbackslash{}\textbackslash{}}
\NormalTok{\textbackslash{}ecvitem\{Level in national or international classification\}\{Level of degree\}\textbackslash{}\textbackslash{}}
 
\CommentTok{%\textbackslash{}pagebreak\{\}}
 
\CommentTok{% Single course}
 
\NormalTok{\textbackslash{}ecvitem\{Dates\}\{August 2013\}}
\NormalTok{\textbackslash{}ecvitem\{Title of qualification awarded\}\{Name of certification\}}
\NormalTok{\textbackslash{}ecvitem\{Principal subjects/occupational skills covered\}\{Skills of}
 \NormalTok{certification\}}
\NormalTok{\textbackslash{}ecvitem\{Name and type of organisation providing education and}
 \NormalTok{training\}\{Institution\}\textbackslash{}\textbackslash{}}
 
\CommentTok{% Last working experience}
 
\NormalTok{\textbackslash{}ecvsection\{Work Experience\}}
\NormalTok{\textbackslash{}ecvitem\{Dates\}\{From June 1957 to February 1987\}\textbackslash{}\textbackslash{}}
\NormalTok{\textbackslash{}ecvitem\{Occupation or position held\}\{Name of the job\}\textbackslash{}\textbackslash{}}
\NormalTok{\textbackslash{}ecvitem\{Main activities and responsibilities\}\{Activities during \textbackslash{}\textbackslash{}& this job\}}
\NormalTok{\textbackslash{}ecvitem\{Name and address of employer\}\{Name of employer\textbackslash{}\textbackslash{}&}
\NormalTok{Employer address\textbackslash{}\textbackslash{}&}
\NormalTok{Second line\textbackslash{}\textbackslash{}& City\textbackslash{}\textbackslash{}& Nation\textbackslash{}\textbackslash{}&}
\NormalTok{Tel. +39 (0) 1234 5678\}\textbackslash{}\textbackslash{}}
\NormalTok{\textbackslash{}ecvitem\{Type of business or sector\}\{Business\}\textbackslash{}\textbackslash{}}
 
\CommentTok{% Volunteer experiences}
 
\NormalTok{\textbackslash{}ecvsection\{Volunteer Experience\}}
 
\NormalTok{\textbackslash{}ecvitem\{Dates\}\{From August 2000 to present\}\textbackslash{}\textbackslash{}}
\NormalTok{\textbackslash{}ecvitem\{Occupation or position held\}\{First Aider\}\textbackslash{}\textbackslash{}}
\NormalTok{\textbackslash{}ecvitem\{Main activities and responsibilities\}\{Activities\}}
\NormalTok{\textbackslash{}ecvitem\{Name and address of employer\}\{Name\textbackslash{}\textbackslash{}&}
\NormalTok{Address\textbackslash{}\textbackslash{}&}
\NormalTok{City\textbackslash{}\textbackslash{}& Post code\textbackslash{}\textbackslash{}&}
\NormalTok{Nation\textbackslash{}\textbackslash{}&}
\NormalTok{Tel. +44 (0) 1234 7654\}\textbackslash{}\textbackslash{}}
\NormalTok{\textbackslash{}ecvitem\{Type of business or sector\}\{Business\}\textbackslash{}\textbackslash{}}
 
\CommentTok{% Personal competences}
 
\NormalTok{\textbackslash{}ecvsection\{Personal skills and competences\}}
 
\CommentTok{% Lenguages}
 
\CommentTok{% Mothertongue}
 
\NormalTok{\textbackslash{}ecvmothertongue[10pt]\{Italian\}\textbackslash{}\textbackslash{}		}\CommentTok{% 10pt leave a one-char line space}
 \NormalTok{before the text}
 
\CommentTok{% Table for common lenguage evaluation}
 
\NormalTok{\textbackslash{}ecvlanguageheader\{(*)\}}
\NormalTok{\textbackslash{}ecvlanguage\{English\}\{\textbackslash{}ecvCOne\}\{\textbackslash{}ecvCOne\}\{\textbackslash{}ecvCOne\}\{\textbackslash{}ecvCOne\}\{\textbackslash{}ecvCOne\}	     }
    \CommentTok{% second language and levels}
      \CommentTok{% Language levels A1 - A2 - B1 - B2 - C1 - C2 from basic to advanced.}
      \CommentTok
 \NormalTok{third}
\NormalTok{\textbackslash{}ecvlastlanguage\{Russian\}\{\textbackslash{}ecvAOne\}\{\textbackslash{}ecvATwo\}\{\textbackslash{}ecvBOne\}\{\textbackslash{}ecvCTwo\}\{\textbackslash{}ecvBTwo\} 	}
\CommentTok{% last language}
 
\NormalTok{\textbackslash{}ecvlanguagefooter\{(*)\}\textbackslash{}\textbackslash{}}
 
\CommentTok{% Social skills}
 
\NormalTok{\textbackslash{}ecvitem\{Social skills and competences\}\{- First social skill;\textbackslash{}\textbackslash{}& - Second social}
 \NormalTok{skill\}\textbackslash{}\textbackslash{}}
 
\CommentTok{% Technical skills}
 
\NormalTok{\textbackslash{}ecvitem\{Technical skills and competences\}\{- First technical skill;\textbackslash{}\textbackslash{}& - Second}
 \NormalTok{technical skill\}\textbackslash{}\textbackslash{}}
 
\CommentTok{% Computer skills}
 
\NormalTok{\textbackslash{}ecvitem\{Computer skills and competences\}\{- First skill;\textbackslash{}\textbackslash{}& - Second\}\textbackslash{}\textbackslash{}}
 
\CommentTok{% Other skills}
 
\NormalTok{\textbackslash{}ecvitem\{Other skills and competences\}\{- First otherskill\}\textbackslash{}\textbackslash{}}
 
\CommentTok{% Driving Licence}
 
\NormalTok{\textbackslash{}ecvitem\{Driving licence(s)\}\{Category and Type\}\textbackslash{}\textbackslash{}}
 
\CommentTok{% Annexes}
 
\NormalTok{\textbackslash{}ecvsection\{Annexes\}}
\NormalTok{\textbackslash{}ecvitem\{\}\{On request:\}}
\NormalTok{\textbackslash{}ecvitem\{\}\{Birth certificate\}}
\NormalTok{\textbackslash{}ecvitem\{\}\{Passport\}}
\NormalTok{\textbackslash{}ecvitem\{\}\{Driving licence\}}
\NormalTok{\textbackslash{}ecvitem\{\}\{Criminal record certificate\}}
\NormalTok{\textbackslash{}ecvitem\{\}\{University study plan\}}
\NormalTok{\textbackslash{}ecvitem\{\}\{\}}
 
\CommentTok{% Disclaimer}
 
\NormalTok{\textbackslash{}ecvsection\{Disclaimer\}}
\NormalTok{\textbackslash{}ecvitem\{\}\{This informations may be used for all purposes permitted by law and}
 \NormalTok{under the Data Protection Act 1998.\textbackslash{}\textbackslash{}&}
\NormalTok{Autorizzo l'utilizzo dei dati personali contenuti nel presente curriculum ai}
 \NormalTok{sensi del D.Lgs. 196/2003 e s.m.i. (Codice in materia di protezione dei dati}
 \NormalTok{personali)\}}
 
\NormalTok{\textbackslash{}end\{europecv\}}
\NormalTok{\textbackslash{}end\{document\}}
\end{Highlighting}
\end{Shaded}

\section{moderncv}
\label{768}

From CTAN:

{\itshape \setmainfont[Path=/usr/share/fonts/truetype/cmu/,UprightFont=cmunrm.ttf,BoldFont=cmunbx.ttf,ItalicFont=cmunti.ttf,BoldItalicFont=cmunbi.ttf]{cmunti.ttf}\setmonofont[Path=/usr/share/fonts/truetype/cmu/,UprightFont=cmuntt.ttf,BoldFont=cmuntb.ttf,ItalicFont=cmunit.ttf,BoldItalicFont=cmuntx.ttf]{cmunti.ttf}\itshape Moderncv pro­vides a doc­u­ment­class for type­set­ting mod­ern cur­ricu­lums vi­tae, both in a clas­sic and in a ca­sual style. It is fairly cus­tomiz­able, al­low­ing you to de­fine your own style by chang­ing the colours, the fonts, etc.}\setmainfont[Path=/usr/share/fonts/truetype/cmu/,UprightFont=cmunrm.ttf,BoldFont=cmunbx.ttf,ItalicFont=cmunti.ttf,BoldItalicFont=cmunbi.ttf]{cmunrm.ttf}\setmonofont[Path=/usr/share/fonts/truetype/cmu/,UprightFont=cmuntt.ttf,BoldFont=cmuntb.ttf,ItalicFont=cmunit.ttf,BoldItalicFont=cmuntx.ttf]{cmunrm.ttf}

The official package provides some well commented templates which may be a good start. You can find those templates in your distribution (if documentation is installed along packages) or ultimately on \myhref{http://www.ctan.org/tex-archive/macros/latex/contrib/moderncv/examples}{CTAN}.

We will not repeat the templates here, so we will only provide a crash course. You should really have a look at the templates for more details.
\subsection{First document}
\label{769}

Most commands are self-{}explanatory.
\begin{Shaded}
\begin{Highlighting}[]

\NormalTok{\textbackslash{}documentclass[11pt,a4paper,sans]\{moderncv\}}
 
\CommentTok{%% ModernCV themes}
\NormalTok{\textbackslash{}moderncvstyle\{casual\}}
\NormalTok{\textbackslash{}moderncvcolor\{blue\}}
\NormalTok{\textbackslash{}renewcommand\{\textbackslash{}familydefault\}\{\textbackslash{}sfdefault\}}
\NormalTok{\textbackslash{}nopagenumbers\{\}}
 
\CommentTok{%% Character encoding}
\NormalTok{\textbackslash{}usepackage[utf8]\{inputenc\}}
 
\CommentTok{%% Adjust the page margins}
\NormalTok{\textbackslash{}usepackage[scale=0.75]\{geometry\}}
 
\CommentTok{%% Personal data}
\NormalTok{\textbackslash{}firstname\{John\}}
\NormalTok{\textbackslash{}familyname\{Doe\}}
\NormalTok{\textbackslash{}title\{Resumé title (optional)\} }
\NormalTok{\textbackslash{}address\{street and number\}\{postcode city\} }
\NormalTok{\textbackslash{}mobile\{+1~(234)~567~890\} }
\NormalTok{\textbackslash{}phone\{+2~(345)~678~901\} }
\NormalTok{\textbackslash{}fax\{+3~(456)~789~012\} }
\NormalTok{\textbackslash{}email\{john@doe.org\} }
\NormalTok{\textbackslash{}homepage\{www.johndoe.com\} }
\NormalTok{\textbackslash{}extrainfo\{additional information\} }
\NormalTok{\textbackslash{}photo[64pt][0.4pt]\{picture\}}
\NormalTok{\textbackslash{}quote\{Some quote (optional)\}}
 
\CommentTok{%%------------------------------------------------------------------------------}
\CommentTok{%% Content}
\CommentTok{%%------------------------------------------------------------------------------}
\NormalTok{\textbackslash{}begin\{document\}}
\NormalTok{\textbackslash{}makecvtitle}
 
\NormalTok{\textbackslash{}section\{Education\}}
\NormalTok{\textbackslash{}cventry\{year--year\}\{Degree\}\{Institution\}\{City\}\{ \textbackslash{}textit\{Grade\} \}\{Description\} }
 \CommentTok{% arguments 3 to 6 can be left empty}
\NormalTok{\textbackslash{}cvitem\{title\}\{ \textbackslash{}emph\{Title\} \}}
\NormalTok{\textbackslash{}cvitemwithcomment\{Language 1\}\{Skill level\}\{Comment\}}
\NormalTok{\textbackslash{}cvdoubleitem\{category X\}\{XXX, YYY, ZZZ\}\{category Y\}\{XXX, YYY, ZZZ\}}
\NormalTok{\textbackslash{}cvlistitem\{Item 1\}}
\NormalTok{\textbackslash{}cvlistdoubleitem\{Item 2\}\{Item 3\}}
\CommentTok{%% ...}
 
\NormalTok{\textbackslash{}bibliography\{publications\}}
\NormalTok{\textbackslash{}end\{document\}}
\end{Highlighting}
\end{Shaded}

\subsection{Theme previews}
\label{770}
\begin{longtable}{>{\RaggedRight}p{0.5\linewidth}}  


\begin{minipage}{1.0\linewidth}
\begin{center}
\includegraphics[width=1.0\linewidth,height=6.5in,keepaspectratio]{../images/161.\SVGExtension}
\end{center}
\raggedright{}\myfigurewithcaption{161}{Banking black theme}
\end{minipage}\vspace{0.75cm}

\\ 


\begin{minipage}{1.0\linewidth}
\begin{center}
\includegraphics[width=1.0\linewidth,height=6.5in,keepaspectratio]{../images/162.\SVGExtension}
\end{center}
\raggedright{}\myfigurewithcaption{162}{Classic green theme}
\end{minipage}\vspace{0.75cm}

\\ 
\end{longtable}
\section{Multilingual support}
\label{771}

It is especially convenient for résumés to have only one document for several output languages, since many parts are shared among versions (personal data, structure, etc.).

LaTeX with appropriate macros provide a comfortable way to manage it. See \mylref{209}{Internationalization}.
\section{References}
\label{772}

\LaTeXNullTemplate{}



\myhref{https://sr.wikibooks.org/wiki/LaTeX\%2F\%D0\%9A\%D1\%80\%D0\%B0\%D1\%82\%D0\%BA\%D0\%B0\%20\%D0\%B1\%D0\%B8\%D0\%BE\%D0\%B3\%D1\%80\%D0\%B0\%D1\%84\%D0\%B8\%D1\%98\%D0\%B0}{sr:LaTeX/Кратка биографија}
\mypart{Creating Graphics}\chapter{Introducing Procedural Graphics}

\myminitoc
\label{773}

\label{774}


In the \mylref{336}{Importing Graphics} chapter, you learned that you can import or link graphics into LaTeX, such as graphics that you have created in another program or obtained elsewhere. In this chapter, you will learn how to create or embed graphics directly in a LaTeX document. The graphics is marked up using commands similar to those for typesetting bold text or creating mathematical formulas, as the following example of embedded graphics shows:

\begin{longtable}{p{1.0\linewidth}}
\begin{Shaded}
\begin{Highlighting}[]
\NormalTok{\textbackslash{}begin\{displaymath\}}
\NormalTok{\textbackslash{}xymatrix\{ \textbackslash{}bullet \textbackslash{}ar[r] \textbackslash{}ar@\{.>\}[r] & \textbackslash{}bullet \}}
\NormalTok{\textbackslash{}end\{displaymath\}}
\end{Highlighting}
\end{Shaded}
\\


\begin{minipage}{0.37500\textwidth}
\begin{center}
\includegraphics[width=1.0\textwidth,height=6.5in,keepaspectratio]{../images/163.png}
\end{center}
\raggedright{}\myfigurewithoutcaption{163}
\end{minipage}\vspace{0.75cm}


\end{longtable}

There are several packages supporting the creation of graphics directly in LaTeX, including \mylref{779}{{\ttfamily \setmainfont[Path=/usr/share/fonts/truetype/cmu/,UprightFont=cmunrm.ttf,BoldFont=cmunbx.ttf,ItalicFont=cmunti.ttf,BoldItalicFont=cmunbi.ttf]{cmuntt.ttf}\setmonofont[Path=/usr/share/fonts/truetype/cmu/,UprightFont=cmuntt.ttf,BoldFont=cmuntb.ttf,ItalicFont=cmunit.ttf,BoldItalicFont=cmuntx.ttf]{cmuntt.ttf}\ttfamily picture}}, \mylref{833}{{\ttfamily \setmainfont[Path=/usr/share/fonts/truetype/cmu/,UprightFont=cmunrm.ttf,BoldFont=cmunbx.ttf,ItalicFont=cmunti.ttf,BoldItalicFont=cmunbi.ttf]{cmuntt.ttf}\setmonofont[Path=/usr/share/fonts/truetype/cmu/,UprightFont=cmuntt.ttf,BoldFont=cmuntb.ttf,ItalicFont=cmunit.ttf,BoldItalicFont=cmuntx.ttf]{cmuntt.ttf}\ttfamily xy-{}Pic}} and \mylref{793}{{\ttfamily \setmainfont[Path=/usr/share/fonts/truetype/cmu/,UprightFont=cmunrm.ttf,BoldFont=cmunbx.ttf,ItalicFont=cmunti.ttf,BoldItalicFont=cmunbi.ttf]{cmuntt.ttf}\setmonofont[Path=/usr/share/fonts/truetype/cmu/,UprightFont=cmuntt.ttf,BoldFont=cmuntb.ttf,ItalicFont=cmunit.ttf,BoldItalicFont=cmuntx.ttf]{cmuntt.ttf}\ttfamily PGF/TikZ}}, described in the following sections.

Compared to WYSIWIG tools like Xfig or Inkscape, this approach is more time consuming, but leads to much better results. Furthermore, the output is flawlessly integrated to your document (no contrast in size nor fonts).

See the \mylref{336}{Importing Graphics} for more details on graphics importation and some attempts to circumvent to integration issue.
\section{Overview}
\label{775}

The {\ttfamily \setmainfont[Path=/usr/share/fonts/truetype/cmu/,UprightFont=cmunrm.ttf,BoldFont=cmunbx.ttf,ItalicFont=cmunti.ttf,BoldItalicFont=cmunbi.ttf]{cmuntt.ttf}\setmonofont[Path=/usr/share/fonts/truetype/cmu/,UprightFont=cmuntt.ttf,BoldFont=cmuntb.ttf,ItalicFont=cmunit.ttf,BoldItalicFont=cmuntx.ttf]{cmuntt.ttf}\ttfamily picture}{$\text{ }$}\setmainfont[Path=/usr/share/fonts/truetype/cmu/,UprightFont=cmunrm.ttf,BoldFont=cmunbx.ttf,ItalicFont=cmunti.ttf,BoldItalicFont=cmunbi.ttf]{cmunrm.ttf}\setmonofont[Path=/usr/share/fonts/truetype/cmu/,UprightFont=cmuntt.ttf,BoldFont=cmuntb.ttf,ItalicFont=cmunit.ttf,BoldItalicFont=cmuntx.ttf]{cmunrm.ttf} environment allows programming pictures directly in LaTeX. On the one hand, there are rather severe constraints, as the slopes of line segments as well as the radii of circles are restricted to a narrow choice of values. On the other hand, the picture environment of LaTeX2e brings with it the {\ttfamily \setmainfont[Path=/usr/share/fonts/truetype/cmu/,UprightFont=cmunrm.ttf,BoldFont=cmunbx.ttf,ItalicFont=cmunti.ttf,BoldItalicFont=cmunbi.ttf]{cmuntt.ttf}\setmonofont[Path=/usr/share/fonts/truetype/cmu/,UprightFont=cmuntt.ttf,BoldFont=cmuntb.ttf,ItalicFont=cmunit.ttf,BoldItalicFont=cmuntx.ttf]{cmuntt.ttf}\ttfamily \textbackslash{}qbezier}{$\text{ }$}\setmainfont[Path=/usr/share/fonts/truetype/cmu/,UprightFont=cmunrm.ttf,BoldFont=cmunbx.ttf,ItalicFont=cmunti.ttf,BoldItalicFont=cmunbi.ttf]{cmunrm.ttf}\setmonofont[Path=/usr/share/fonts/truetype/cmu/,UprightFont=cmuntt.ttf,BoldFont=cmuntb.ttf,ItalicFont=cmunit.ttf,BoldItalicFont=cmuntx.ttf]{cmunrm.ttf} command, \symbol{34}q\symbol{34} meaning {\itshape \setmainfont[Path=/usr/share/fonts/truetype/cmu/,UprightFont=cmunrm.ttf,BoldFont=cmunbx.ttf,ItalicFont=cmunti.ttf,BoldItalicFont=cmunbi.ttf]{cmunti.ttf}\setmonofont[Path=/usr/share/fonts/truetype/cmu/,UprightFont=cmuntt.ttf,BoldFont=cmuntb.ttf,ItalicFont=cmunit.ttf,BoldItalicFont=cmuntx.ttf]{cmunti.ttf}\itshape quadratic}\setmainfont[Path=/usr/share/fonts/truetype/cmu/,UprightFont=cmunrm.ttf,BoldFont=cmunbx.ttf,ItalicFont=cmunti.ttf,BoldItalicFont=cmunbi.ttf]{cmunrm.ttf}\setmonofont[Path=/usr/share/fonts/truetype/cmu/,UprightFont=cmuntt.ttf,BoldFont=cmuntb.ttf,ItalicFont=cmunit.ttf,BoldItalicFont=cmuntx.ttf]{cmunrm.ttf}. Many frequently-{}used curves such as circles, ellipses, and \myhref{https://en.wikipedia.org/wiki/catenary}{catenaries} can be satisfactorily approximated by quadratic Bézier curves, although this may require some mathematical toil. If a programming language like Java is used to generate {\ttfamily \setmainfont[Path=/usr/share/fonts/truetype/cmu/,UprightFont=cmunrm.ttf,BoldFont=cmunbx.ttf,ItalicFont=cmunti.ttf,BoldItalicFont=cmunbi.ttf]{cmuntt.ttf}\setmonofont[Path=/usr/share/fonts/truetype/cmu/,UprightFont=cmuntt.ttf,BoldFont=cmuntb.ttf,ItalicFont=cmunit.ttf,BoldItalicFont=cmuntx.ttf]{cmuntt.ttf}\ttfamily \textbackslash{}qbezier}{$\text{ }$}\setmainfont[Path=/usr/share/fonts/truetype/cmu/,UprightFont=cmunrm.ttf,BoldFont=cmunbx.ttf,ItalicFont=cmunti.ttf,BoldItalicFont=cmunbi.ttf]{cmunrm.ttf}\setmonofont[Path=/usr/share/fonts/truetype/cmu/,UprightFont=cmuntt.ttf,BoldFont=cmuntb.ttf,ItalicFont=cmunit.ttf,BoldItalicFont=cmuntx.ttf]{cmunrm.ttf} blocks of LaTeX input files, the picture environment becomes quite powerful.

Although programming pictures directly in LaTeX is severely restricted, and often rather tiresome, there are still reasons for doing so. The documents thus produced are \symbol{34}small\symbol{34} with respect to bytes, and there are no additional graphics files to be dragged along.

Packages like {\ttfamily \setmainfont[Path=/usr/share/fonts/truetype/cmu/,UprightFont=cmunrm.ttf,BoldFont=cmunbx.ttf,ItalicFont=cmunti.ttf,BoldItalicFont=cmunbi.ttf]{cmuntt.ttf}\setmonofont[Path=/usr/share/fonts/truetype/cmu/,UprightFont=cmuntt.ttf,BoldFont=cmuntb.ttf,ItalicFont=cmunit.ttf,BoldItalicFont=cmuntx.ttf]{cmuntt.ttf}\ttfamily epic}\setmainfont[Path=/usr/share/fonts/truetype/cmu/,UprightFont=cmunrm.ttf,BoldFont=cmunbx.ttf,ItalicFont=cmunti.ttf,BoldItalicFont=cmunbi.ttf]{cmunrm.ttf}\setmonofont[Path=/usr/share/fonts/truetype/cmu/,UprightFont=cmuntt.ttf,BoldFont=cmuntb.ttf,ItalicFont=cmunit.ttf,BoldItalicFont=cmuntx.ttf]{cmunrm.ttf}, {\ttfamily \setmainfont[Path=/usr/share/fonts/truetype/cmu/,UprightFont=cmunrm.ttf,BoldFont=cmunbx.ttf,ItalicFont=cmunti.ttf,BoldItalicFont=cmunbi.ttf]{cmuntt.ttf}\setmonofont[Path=/usr/share/fonts/truetype/cmu/,UprightFont=cmuntt.ttf,BoldFont=cmuntb.ttf,ItalicFont=cmunit.ttf,BoldItalicFont=cmuntx.ttf]{cmuntt.ttf}\ttfamily eepic}{$\text{ }$}\setmainfont[Path=/usr/share/fonts/truetype/cmu/,UprightFont=cmunrm.ttf,BoldFont=cmunbx.ttf,ItalicFont=cmunti.ttf,BoldItalicFont=cmunbi.ttf]{cmunrm.ttf}\setmonofont[Path=/usr/share/fonts/truetype/cmu/,UprightFont=cmuntt.ttf,BoldFont=cmuntb.ttf,ItalicFont=cmunit.ttf,BoldItalicFont=cmuntx.ttf]{cmunrm.ttf} or {\ttfamily \setmainfont[Path=/usr/share/fonts/truetype/cmu/,UprightFont=cmunrm.ttf,BoldFont=cmunbx.ttf,ItalicFont=cmunti.ttf,BoldItalicFont=cmunbi.ttf]{cmuntt.ttf}\setmonofont[Path=/usr/share/fonts/truetype/cmu/,UprightFont=cmuntt.ttf,BoldFont=cmuntb.ttf,ItalicFont=cmunit.ttf,BoldItalicFont=cmuntx.ttf]{cmuntt.ttf}\ttfamily pstricks}{$\text{ }$}\setmainfont[Path=/usr/share/fonts/truetype/cmu/,UprightFont=cmunrm.ttf,BoldFont=cmunbx.ttf,ItalicFont=cmunti.ttf,BoldItalicFont=cmunbi.ttf]{cmunrm.ttf}\setmonofont[Path=/usr/share/fonts/truetype/cmu/,UprightFont=cmuntt.ttf,BoldFont=cmuntb.ttf,ItalicFont=cmunit.ttf,BoldItalicFont=cmuntx.ttf]{cmunrm.ttf} enhance the original picture environment, and greatly strengthen the graphical power of LaTeX.

While the former two packages just enhance the picture environment, the {\ttfamily \setmainfont[Path=/usr/share/fonts/truetype/cmu/,UprightFont=cmunrm.ttf,BoldFont=cmunbx.ttf,ItalicFont=cmunti.ttf,BoldItalicFont=cmunbi.ttf]{cmuntt.ttf}\setmonofont[Path=/usr/share/fonts/truetype/cmu/,UprightFont=cmuntt.ttf,BoldFont=cmuntb.ttf,ItalicFont=cmunit.ttf,BoldItalicFont=cmuntx.ttf]{cmuntt.ttf}\ttfamily pstricks}{$\text{ }$}\setmainfont[Path=/usr/share/fonts/truetype/cmu/,UprightFont=cmunrm.ttf,BoldFont=cmunbx.ttf,ItalicFont=cmunti.ttf,BoldItalicFont=cmunbi.ttf]{cmunrm.ttf}\setmonofont[Path=/usr/share/fonts/truetype/cmu/,UprightFont=cmuntt.ttf,BoldFont=cmuntb.ttf,ItalicFont=cmunit.ttf,BoldItalicFont=cmuntx.ttf]{cmunrm.ttf} package has its own drawing environment, {\ttfamily \setmainfont[Path=/usr/share/fonts/truetype/cmu/,UprightFont=cmunrm.ttf,BoldFont=cmunbx.ttf,ItalicFont=cmunti.ttf,BoldItalicFont=cmunbi.ttf]{cmuntt.ttf}\setmonofont[Path=/usr/share/fonts/truetype/cmu/,UprightFont=cmuntt.ttf,BoldFont=cmuntb.ttf,ItalicFont=cmunit.ttf,BoldItalicFont=cmuntx.ttf]{cmuntt.ttf}\ttfamily pspicture}\setmainfont[Path=/usr/share/fonts/truetype/cmu/,UprightFont=cmunrm.ttf,BoldFont=cmunbx.ttf,ItalicFont=cmunti.ttf,BoldItalicFont=cmunbi.ttf]{cmunrm.ttf}\setmonofont[Path=/usr/share/fonts/truetype/cmu/,UprightFont=cmuntt.ttf,BoldFont=cmuntb.ttf,ItalicFont=cmunit.ttf,BoldItalicFont=cmuntx.ttf]{cmunrm.ttf}. The power of {\ttfamily \setmainfont[Path=/usr/share/fonts/truetype/cmu/,UprightFont=cmunrm.ttf,BoldFont=cmunbx.ttf,ItalicFont=cmunti.ttf,BoldItalicFont=cmunbi.ttf]{cmuntt.ttf}\setmonofont[Path=/usr/share/fonts/truetype/cmu/,UprightFont=cmuntt.ttf,BoldFont=cmuntb.ttf,ItalicFont=cmunit.ttf,BoldItalicFont=cmuntx.ttf]{cmuntt.ttf}\ttfamily pstricks}{$\text{ }$}\setmainfont[Path=/usr/share/fonts/truetype/cmu/,UprightFont=cmunrm.ttf,BoldFont=cmunbx.ttf,ItalicFont=cmunti.ttf,BoldItalicFont=cmunbi.ttf]{cmunrm.ttf}\setmonofont[Path=/usr/share/fonts/truetype/cmu/,UprightFont=cmuntt.ttf,BoldFont=cmuntb.ttf,ItalicFont=cmunit.ttf,BoldItalicFont=cmuntx.ttf]{cmunrm.ttf} stems from the fact that this package makes extensive use of PostScript possibilities. Unfortunately it has one big shortcoming: it doesn\textquotesingle{}t work together with pdfLaTeX, as such. To generate a PDF document from TeX source, you have to go from TeX to DVI to PDF, losing hyperlinks, metadata, and microtypographic features of pdflatex in the process.

In addition, numerous packages have been written for specific purposes. One of them is {\itshape \setmainfont[Path=/usr/share/fonts/truetype/cmu/,UprightFont=cmunrm.ttf,BoldFont=cmunbx.ttf,ItalicFont=cmunti.ttf,BoldItalicFont=cmunbi.ttf]{cmunti.ttf}\setmonofont[Path=/usr/share/fonts/truetype/cmu/,UprightFont=cmuntt.ttf,BoldFont=cmuntb.ttf,ItalicFont=cmunit.ttf,BoldItalicFont=cmuntx.ttf]{cmunti.ttf}\itshape XY-{}pic,}{$\text{ }$}\setmainfont[Path=/usr/share/fonts/truetype/cmu/,UprightFont=cmunrm.ttf,BoldFont=cmunbx.ttf,ItalicFont=cmunti.ttf,BoldItalicFont=cmunbi.ttf]{cmunrm.ttf}\setmonofont[Path=/usr/share/fonts/truetype/cmu/,UprightFont=cmuntt.ttf,BoldFont=cmuntb.ttf,ItalicFont=cmunit.ttf,BoldItalicFont=cmuntx.ttf]{cmunrm.ttf} described at the end of this chapter. A wide variety of these packages are described in detail in {\itshape \setmainfont[Path=/usr/share/fonts/truetype/cmu/,UprightFont=cmunrm.ttf,BoldFont=cmunbx.ttf,ItalicFont=cmunti.ttf,BoldItalicFont=cmunbi.ttf]{cmunti.ttf}\setmonofont[Path=/usr/share/fonts/truetype/cmu/,UprightFont=cmuntt.ttf,BoldFont=cmuntb.ttf,ItalicFont=cmunit.ttf,BoldItalicFont=cmuntx.ttf]{cmunti.ttf}\itshape The LaTeX Graphics Companion}{$\text{ }$}\setmainfont[Path=/usr/share/fonts/truetype/cmu/,UprightFont=cmunrm.ttf,BoldFont=cmunbx.ttf,ItalicFont=cmunti.ttf,BoldItalicFont=cmunbi.ttf]{cmunrm.ttf}\setmonofont[Path=/usr/share/fonts/truetype/cmu/,UprightFont=cmuntt.ttf,BoldFont=cmuntb.ttf,ItalicFont=cmunit.ttf,BoldItalicFont=cmuntx.ttf]{cmunrm.ttf} (not to be confused with {\itshape \setmainfont[Path=/usr/share/fonts/truetype/cmu/,UprightFont=cmunrm.ttf,BoldFont=cmunbx.ttf,ItalicFont=cmunti.ttf,BoldItalicFont=cmunbi.ttf]{cmunti.ttf}\setmonofont[Path=/usr/share/fonts/truetype/cmu/,UprightFont=cmuntt.ttf,BoldFont=cmuntb.ttf,ItalicFont=cmunit.ttf,BoldItalicFont=cmuntx.ttf]{cmunti.ttf}\itshape The LaTeX Companion}\setmainfont[Path=/usr/share/fonts/truetype/cmu/,UprightFont=cmunrm.ttf,BoldFont=cmunbx.ttf,ItalicFont=cmunti.ttf,BoldItalicFont=cmunbi.ttf]{cmunrm.ttf}\setmonofont[Path=/usr/share/fonts/truetype/cmu/,UprightFont=cmuntt.ttf,BoldFont=cmuntb.ttf,ItalicFont=cmunit.ttf,BoldItalicFont=cmuntx.ttf]{cmunrm.ttf}).

Perhaps the most powerful graphical tool related with LaTeX is \myhref{https://en.wikipedia.org/wiki/MetaPost}{MetaPost}, the twin of Donald E. Knuth’s \myhref{https://en.wikipedia.org/wiki/METAFONT}{METAFONT}. MetaPost has the very powerful and mathematically sophisticated programming language of METAFONT. Contrary to METAFONT, which generates bitmaps, MetaPost generates encapsulated PostScript files, which can be imported in LaTeX. For an introduction, see {\itshape \setmainfont[Path=/usr/share/fonts/truetype/cmu/,UprightFont=cmunrm.ttf,BoldFont=cmunbx.ttf,ItalicFont=cmunti.ttf,BoldItalicFont=cmunbi.ttf]{cmunti.ttf}\setmonofont[Path=/usr/share/fonts/truetype/cmu/,UprightFont=cmuntt.ttf,BoldFont=cmuntb.ttf,ItalicFont=cmunit.ttf,BoldItalicFont=cmuntx.ttf]{cmunti.ttf}\itshape A User’s Manual for MetaPost.}{$\text{ }$}\setmainfont[Path=/usr/share/fonts/truetype/cmu/,UprightFont=cmunrm.ttf,BoldFont=cmunbx.ttf,ItalicFont=cmunti.ttf,BoldItalicFont=cmunbi.ttf]{cmunrm.ttf}\setmonofont[Path=/usr/share/fonts/truetype/cmu/,UprightFont=cmuntt.ttf,BoldFont=cmuntb.ttf,ItalicFont=cmunit.ttf,BoldItalicFont=cmuntx.ttf]{cmunrm.ttf}  A very thorough discussion of LaTeX and TEX strategies for graphics (and fonts) can be found in {\itshape \setmainfont[Path=/usr/share/fonts/truetype/cmu/,UprightFont=cmunrm.ttf,BoldFont=cmunbx.ttf,ItalicFont=cmunti.ttf,BoldItalicFont=cmunbi.ttf]{cmunti.ttf}\setmonofont[Path=/usr/share/fonts/truetype/cmu/,UprightFont=cmuntt.ttf,BoldFont=cmuntb.ttf,ItalicFont=cmunit.ttf,BoldItalicFont=cmuntx.ttf]{cmunti.ttf}\itshape TEX Unbound.}\setmainfont[Path=/usr/share/fonts/truetype/cmu/,UprightFont=cmunrm.ttf,BoldFont=cmunbx.ttf,ItalicFont=cmunti.ttf,BoldItalicFont=cmunbi.ttf]{cmunrm.ttf}\setmonofont[Path=/usr/share/fonts/truetype/cmu/,UprightFont=cmuntt.ttf,BoldFont=cmuntb.ttf,ItalicFont=cmunit.ttf,BoldItalicFont=cmuntx.ttf]{cmunrm.ttf}

The last but certainly not least are the PGF/TikZ and Asymptote systems. While the previous systems ({\ttfamily \setmainfont[Path=/usr/share/fonts/truetype/cmu/,UprightFont=cmunrm.ttf,BoldFont=cmunbx.ttf,ItalicFont=cmunti.ttf,BoldItalicFont=cmunbi.ttf]{cmuntt.ttf}\setmonofont[Path=/usr/share/fonts/truetype/cmu/,UprightFont=cmuntt.ttf,BoldFont=cmuntb.ttf,ItalicFont=cmunit.ttf,BoldItalicFont=cmuntx.ttf]{cmuntt.ttf}\ttfamily picture}\setmainfont[Path=/usr/share/fonts/truetype/cmu/,UprightFont=cmunrm.ttf,BoldFont=cmunbx.ttf,ItalicFont=cmunti.ttf,BoldItalicFont=cmunbi.ttf]{cmunrm.ttf}\setmonofont[Path=/usr/share/fonts/truetype/cmu/,UprightFont=cmuntt.ttf,BoldFont=cmuntb.ttf,ItalicFont=cmunit.ttf,BoldItalicFont=cmuntx.ttf]{cmunrm.ttf}, {\ttfamily \setmainfont[Path=/usr/share/fonts/truetype/cmu/,UprightFont=cmunrm.ttf,BoldFont=cmunbx.ttf,ItalicFont=cmunti.ttf,BoldItalicFont=cmunbi.ttf]{cmuntt.ttf}\setmonofont[Path=/usr/share/fonts/truetype/cmu/,UprightFont=cmuntt.ttf,BoldFont=cmuntb.ttf,ItalicFont=cmunit.ttf,BoldItalicFont=cmuntx.ttf]{cmuntt.ttf}\ttfamily epic}\setmainfont[Path=/usr/share/fonts/truetype/cmu/,UprightFont=cmunrm.ttf,BoldFont=cmunbx.ttf,ItalicFont=cmunti.ttf,BoldItalicFont=cmunbi.ttf]{cmunrm.ttf}\setmonofont[Path=/usr/share/fonts/truetype/cmu/,UprightFont=cmuntt.ttf,BoldFont=cmuntb.ttf,ItalicFont=cmunit.ttf,BoldItalicFont=cmuntx.ttf]{cmunrm.ttf}, {\ttfamily \setmainfont[Path=/usr/share/fonts/truetype/cmu/,UprightFont=cmunrm.ttf,BoldFont=cmunbx.ttf,ItalicFont=cmunti.ttf,BoldItalicFont=cmunbi.ttf]{cmuntt.ttf}\setmonofont[Path=/usr/share/fonts/truetype/cmu/,UprightFont=cmuntt.ttf,BoldFont=cmuntb.ttf,ItalicFont=cmunit.ttf,BoldItalicFont=cmuntx.ttf]{cmuntt.ttf}\ttfamily pstricks}{$\text{ }$}\setmainfont[Path=/usr/share/fonts/truetype/cmu/,UprightFont=cmunrm.ttf,BoldFont=cmunbx.ttf,ItalicFont=cmunti.ttf,BoldItalicFont=cmunbi.ttf]{cmunrm.ttf}\setmonofont[Path=/usr/share/fonts/truetype/cmu/,UprightFont=cmuntt.ttf,BoldFont=cmuntb.ttf,ItalicFont=cmunit.ttf,BoldItalicFont=cmuntx.ttf]{cmunrm.ttf} or {\ttfamily \setmainfont[Path=/usr/share/fonts/truetype/cmu/,UprightFont=cmunrm.ttf,BoldFont=cmunbx.ttf,ItalicFont=cmunti.ttf,BoldItalicFont=cmunbi.ttf]{cmuntt.ttf}\setmonofont[Path=/usr/share/fonts/truetype/cmu/,UprightFont=cmuntt.ttf,BoldFont=cmuntb.ttf,ItalicFont=cmunit.ttf,BoldItalicFont=cmuntx.ttf]{cmuntt.ttf}\ttfamily metapost}\setmainfont[Path=/usr/share/fonts/truetype/cmu/,UprightFont=cmunrm.ttf,BoldFont=cmunbx.ttf,ItalicFont=cmunti.ttf,BoldItalicFont=cmunbi.ttf]{cmunrm.ttf}\setmonofont[Path=/usr/share/fonts/truetype/cmu/,UprightFont=cmuntt.ttf,BoldFont=cmuntb.ttf,ItalicFont=cmunit.ttf,BoldItalicFont=cmuntx.ttf]{cmunrm.ttf}) focus on the {\itshape \setmainfont[Path=/usr/share/fonts/truetype/cmu/,UprightFont=cmunrm.ttf,BoldFont=cmunbx.ttf,ItalicFont=cmunti.ttf,BoldItalicFont=cmunbi.ttf]{cmunti.ttf}\setmonofont[Path=/usr/share/fonts/truetype/cmu/,UprightFont=cmuntt.ttf,BoldFont=cmuntb.ttf,ItalicFont=cmunit.ttf,BoldItalicFont=cmuntx.ttf]{cmunti.ttf}\itshape how}{$\text{ }$}\setmainfont[Path=/usr/share/fonts/truetype/cmu/,UprightFont=cmunrm.ttf,BoldFont=cmunbx.ttf,ItalicFont=cmunti.ttf,BoldItalicFont=cmunbi.ttf]{cmunrm.ttf}\setmonofont[Path=/usr/share/fonts/truetype/cmu/,UprightFont=cmuntt.ttf,BoldFont=cmuntb.ttf,ItalicFont=cmunit.ttf,BoldItalicFont=cmuntx.ttf]{cmunrm.ttf} to draw, TikZ and Asymptote focus more on the {\itshape \setmainfont[Path=/usr/share/fonts/truetype/cmu/,UprightFont=cmunrm.ttf,BoldFont=cmunbx.ttf,ItalicFont=cmunti.ttf,BoldItalicFont=cmunbi.ttf]{cmunti.ttf}\setmonofont[Path=/usr/share/fonts/truetype/cmu/,UprightFont=cmuntt.ttf,BoldFont=cmuntb.ttf,ItalicFont=cmunit.ttf,BoldItalicFont=cmuntx.ttf]{cmunti.ttf}\itshape what}{$\text{ }$}\setmainfont[Path=/usr/share/fonts/truetype/cmu/,UprightFont=cmunrm.ttf,BoldFont=cmunbx.ttf,ItalicFont=cmunti.ttf,BoldItalicFont=cmunbi.ttf]{cmunrm.ttf}\setmonofont[Path=/usr/share/fonts/truetype/cmu/,UprightFont=cmuntt.ttf,BoldFont=cmuntb.ttf,ItalicFont=cmunit.ttf,BoldItalicFont=cmuntx.ttf]{cmunrm.ttf} to draw. One could say that TikZ and Asymptote are to drawing in LaTeX as LaTeX is to digital typesetting. It\textquotesingle{}s recommended to use one of these if your LaTeX distribution includes it. TikZ is a pure (La)TeX system, not reliant on external software, while \myhref{https://en.wikipedia.org/wiki/Asymptote\%20\%28vector\%20graphics\%20language\%29}{Asymptote} is an external system which integrates seamlessly with (La)TeX.  If using Asymptote, it is very helpful to use \myhref{http://www.ctan.org/pkg/latexmk/}{latexmk} to manage the compilation steps.

In many cases, especially for more advanced diagrams, it may be easier to draw the graphics using external vector graphics software, and then import the file into the document (see \mylref{336}{../Importing Graphics}). However most software does not support LaTeX fonts or mathematical notation, which can result in not suitable and inconsistent graphics. There are several solutions to this problem.




\myhref{https://sr.wikibooks.org/wiki/LaTeX\%2F\%D0\%9F\%D1\%80\%D0\%B5\%D0\%B4\%D1\%81\%D1\%82\%D0\%B0\%D0\%B2\%D1\%99\%D0\%B0\%D1\%9A\%D0\%B5\%20\%D0\%BF\%D1\%80\%D0\%BE\%D1\%86\%D0\%B5\%D0\%B4\%D1\%83\%D1\%80\%D0\%B0\%D0\%BB\%D0\%BD\%D0\%B8\%D1\%85\%20\%D0\%B3\%D1\%80\%D0\%B0\%D1\%84\%D0\%B8\%D0\%BA\%D0\%B0}{sr:LaTeX/Представљање процедуралних графика}\chapter{MetaPost}

\myminitoc
\label{776}

\label{777}


\LaTeXNullTemplate{}

\chapter{Picture}

\myminitoc
\label{778}

\label{779}


The \LaTeXTT{picture} environment allows programming pictures directly in LaTeX. On the one hand, there are rather severe constraints, as the slopes of line segments as well as the radii of circles are restricted to a narrow choice of values. On the other hand, the picture environment of LaTeX2e brings with it the \LaTeXTT{\textbackslash{}qbezier} command, \symbol{34}q\symbol{34} meaning {\itshape \setmainfont[Path=/usr/share/fonts/truetype/cmu/,UprightFont=cmunrm.ttf,BoldFont=cmunbx.ttf,ItalicFont=cmunti.ttf,BoldItalicFont=cmunbi.ttf]{cmunti.ttf}\setmonofont[Path=/usr/share/fonts/truetype/cmu/,UprightFont=cmuntt.ttf,BoldFont=cmuntb.ttf,ItalicFont=cmunit.ttf,BoldItalicFont=cmuntx.ttf]{cmunti.ttf}\itshape quadratic}\setmainfont[Path=/usr/share/fonts/truetype/cmu/,UprightFont=cmunrm.ttf,BoldFont=cmunbx.ttf,ItalicFont=cmunti.ttf,BoldItalicFont=cmunbi.ttf]{cmunrm.ttf}\setmonofont[Path=/usr/share/fonts/truetype/cmu/,UprightFont=cmuntt.ttf,BoldFont=cmuntb.ttf,ItalicFont=cmunit.ttf,BoldItalicFont=cmuntx.ttf]{cmunrm.ttf}. Many frequently-{}used curves such as circles, ellipses, and \myhref{https://en.wikipedia.org/wiki/catenary}{catenaries} can be satisfactorily approximated by quadratic Bézier curves, although this may require some mathematical toil. If a programming language like Java is used to generate \LaTeXTT{\textbackslash{}qbezier} blocks of LaTeX input files, the picture environment becomes quite powerful.

Although programming pictures directly in LaTeX is severely restricted, and often rather tiresome, there are still reasons for doing so. The documents thus produced are \symbol{34}small\symbol{34} with respect to bytes, and there are no additional graphics files to be dragged along.

Packages like \LaTeXTT{pict2e}, \LaTeXTT{epic}, \LaTeXTT{eepic} or \LaTeXTT{pstricks} enhance the original picture environment, and greatly strengthen the graphical power of LaTeX.
\section{Basic commands}
\label{780}
A \LaTeXTT{picture} environment is available in any LaTeX distribution, without the need of loading any external package. This environment is created with one of the two commands

\begin{Shaded}
\begin{Highlighting}[]

\NormalTok{\textbackslash{}begin\{picture\}(x, y)}
 \NormalTok{...}
\NormalTok{\textbackslash{}end\{picture\}}
\end{Highlighting}
\end{Shaded}


or

\begin{Shaded}
\begin{Highlighting}[]

\NormalTok{\textbackslash{}begin\{picture\}(x, y)(x0, y0)}
\NormalTok{...}
\NormalTok{\textbackslash{}end\{picture\}}
\end{Highlighting}
\end{Shaded}


The first pair, {$(x, y)$}, affects the reservation, within the document, of rectangular space for the picture.

The optional second pair, {$(x_0, y_0)$}, assigns arbitrary coordinates to the bottom left corner of the reserved rectangle.

The numbers {\itshape \setmainfont[Path=/usr/share/fonts/truetype/cmu/,UprightFont=cmunrm.ttf,BoldFont=cmunbx.ttf,ItalicFont=cmunti.ttf,BoldItalicFont=cmunbi.ttf]{cmunti.ttf}\setmonofont[Path=/usr/share/fonts/truetype/cmu/,UprightFont=cmuntt.ttf,BoldFont=cmuntb.ttf,ItalicFont=cmunit.ttf,BoldItalicFont=cmuntx.ttf]{cmunti.ttf}\itshape x}\setmainfont[Path=/usr/share/fonts/truetype/cmu/,UprightFont=cmunrm.ttf,BoldFont=cmunbx.ttf,ItalicFont=cmunti.ttf,BoldItalicFont=cmunbi.ttf]{cmunrm.ttf}\setmonofont[Path=/usr/share/fonts/truetype/cmu/,UprightFont=cmuntt.ttf,BoldFont=cmuntb.ttf,ItalicFont=cmunit.ttf,BoldItalicFont=cmuntx.ttf]{cmunrm.ttf}, {\itshape \setmainfont[Path=/usr/share/fonts/truetype/cmu/,UprightFont=cmunrm.ttf,BoldFont=cmunbx.ttf,ItalicFont=cmunti.ttf,BoldItalicFont=cmunbi.ttf]{cmunti.ttf}\setmonofont[Path=/usr/share/fonts/truetype/cmu/,UprightFont=cmuntt.ttf,BoldFont=cmuntb.ttf,ItalicFont=cmunit.ttf,BoldItalicFont=cmuntx.ttf]{cmunti.ttf}\itshape y}\setmainfont[Path=/usr/share/fonts/truetype/cmu/,UprightFont=cmunrm.ttf,BoldFont=cmunbx.ttf,ItalicFont=cmunti.ttf,BoldItalicFont=cmunbi.ttf]{cmunrm.ttf}\setmonofont[Path=/usr/share/fonts/truetype/cmu/,UprightFont=cmuntt.ttf,BoldFont=cmuntb.ttf,ItalicFont=cmunit.ttf,BoldItalicFont=cmuntx.ttf]{cmunrm.ttf}, {\itshape \setmainfont[Path=/usr/share/fonts/truetype/cmu/,UprightFont=cmunrm.ttf,BoldFont=cmunbx.ttf,ItalicFont=cmunti.ttf,BoldItalicFont=cmunbi.ttf]{cmunti.ttf}\setmonofont[Path=/usr/share/fonts/truetype/cmu/,UprightFont=cmuntt.ttf,BoldFont=cmuntb.ttf,ItalicFont=cmunit.ttf,BoldItalicFont=cmuntx.ttf]{cmunti.ttf}\itshape x}\setmainfont[Path=/usr/share/fonts/truetype/cmu/,UprightFont=cmunrm.ttf,BoldFont=cmunbx.ttf,ItalicFont=cmunti.ttf,BoldItalicFont=cmunbi.ttf]{cmunrm.ttf}\setmonofont[Path=/usr/share/fonts/truetype/cmu/,UprightFont=cmuntt.ttf,BoldFont=cmuntb.ttf,ItalicFont=cmunit.ttf,BoldItalicFont=cmuntx.ttf]{cmunrm.ttf}0, {\itshape \setmainfont[Path=/usr/share/fonts/truetype/cmu/,UprightFont=cmunrm.ttf,BoldFont=cmunbx.ttf,ItalicFont=cmunti.ttf,BoldItalicFont=cmunbi.ttf]{cmunti.ttf}\setmonofont[Path=/usr/share/fonts/truetype/cmu/,UprightFont=cmuntt.ttf,BoldFont=cmuntb.ttf,ItalicFont=cmunit.ttf,BoldItalicFont=cmuntx.ttf]{cmunti.ttf}\itshape y}\setmainfont[Path=/usr/share/fonts/truetype/cmu/,UprightFont=cmunrm.ttf,BoldFont=cmunbx.ttf,ItalicFont=cmunti.ttf,BoldItalicFont=cmunbi.ttf]{cmunrm.ttf}\setmonofont[Path=/usr/share/fonts/truetype/cmu/,UprightFont=cmuntt.ttf,BoldFont=cmuntb.ttf,ItalicFont=cmunit.ttf,BoldItalicFont=cmuntx.ttf]{cmunrm.ttf}0 are numbers (lengths) in the units of \LaTeXTT{\textbackslash{}unitlength}, which can be reset any time (but not within a picture environment) with a command such as

\begin{Shaded}
\begin{Highlighting}[]

\NormalTok{\textbackslash{}setlength\{\textbackslash{}unitlength\}\{1.2cm\}}
\end{Highlighting}
\end{Shaded}


The default value of \LaTeXTT{\textbackslash{}unitlength} is \LaTeXTT{1pt}.

Most drawing commands have one of the two forms

\begin{Shaded}
\begin{Highlighting}[]

\NormalTok{\textbackslash{}put(x, y)\{object\}}
\end{Highlighting}
\end{Shaded}


or

\begin{Shaded}
\begin{Highlighting}[]

\NormalTok{\textbackslash{}multiput(x, y)(dx, dy)\{n\}\{object\}}
\end{Highlighting}
\end{Shaded}


Bézier curves are an exception. They are drawn with the command

\begin{Shaded}
\begin{Highlighting}[]

\NormalTok{\textbackslash{}qbezier(x1, y1)(x2, y2)(x3, y3)}
\end{Highlighting}
\end{Shaded}


With the package \LaTeXTT{picture} absolute dimension (like 15pt) and expression are allowed, in addition to numbers relative to \LaTeXTT{\textbackslash{}unitlength}.
\section{Line segments}
\label{781}

Line segments are drawn with the command:

\begin{Shaded}
\begin{Highlighting}[]

\NormalTok{\textbackslash{}put(x, y)\{ \textbackslash{}line(x1, y1)\{length\} \}}
\end{Highlighting}
\end{Shaded}


The \LaTeXTT{\textbackslash{}line} command has two arguments:
\begin{myenumerate}
\item{}  a direction vector,
\item{}  a \symbol{34}length\symbol{34} (sort of: this argument is the vertical length in the case of a vertical line segment and in all other cases the horizontal distance of the line, rather than the length of the segment itself).
\end{myenumerate}


The components of the direction vector are restricted to the integers ({\itshape \setmainfont[Path=/usr/share/fonts/truetype/cmu/,UprightFont=cmunrm.ttf,BoldFont=cmunbx.ttf,ItalicFont=cmunti.ttf,BoldItalicFont=cmunbi.ttf]{cmunti.ttf}\setmonofont[Path=/usr/share/fonts/truetype/cmu/,UprightFont=cmuntt.ttf,BoldFont=cmuntb.ttf,ItalicFont=cmunit.ttf,BoldItalicFont=cmuntx.ttf]{cmunti.ttf}\itshape −6, −5, ... , 5, 6}\setmainfont[Path=/usr/share/fonts/truetype/cmu/,UprightFont=cmunrm.ttf,BoldFont=cmunbx.ttf,ItalicFont=cmunti.ttf,BoldItalicFont=cmunbi.ttf]{cmunrm.ttf}\setmonofont[Path=/usr/share/fonts/truetype/cmu/,UprightFont=cmuntt.ttf,BoldFont=cmuntb.ttf,ItalicFont=cmunit.ttf,BoldItalicFont=cmuntx.ttf]{cmunrm.ttf}) and they have to be coprime (no common divisor except 1). The figure below illustrates all 25 possible slope values in the first quadrant. The length is relative to \LaTeXTT{\textbackslash{}unitlength}.

\begin{longtable}{p{1.0\linewidth}}
\begin{Shaded}
\begin{Highlighting}[]

\NormalTok{\textbackslash{}setlength\{\textbackslash{}unitlength\}\{5cm\}}
\NormalTok{\textbackslash{}begin\{picture\}(1,1)}
\NormalTok{\textbackslash{}put(0,0)\{\textbackslash{}line(0,1)\{1\}\}}
\NormalTok{\textbackslash{}put(0,0)\{\textbackslash{}line(1,0)\{1\}\}}
\NormalTok{\textbackslash{}put(0,0)\{\textbackslash{}line(1,1)\{1\}\}}
\NormalTok{\textbackslash{}put(0,0)\{\textbackslash{}line(1,2)\{.5\}\}}
\NormalTok{\textbackslash{}put(0,0)\{\textbackslash{}line(1,3)\{.3333\}\}}
\NormalTok{\textbackslash{}put(0,0)\{\textbackslash{}line(1,4)\{.25\}\}}
\NormalTok{\textbackslash{}put(0,0)\{\textbackslash{}line(1,5)\{.2\}\}}
\NormalTok{\textbackslash{}put(0,0)\{\textbackslash{}line(1,6)\{.1667\}\}}
\NormalTok{\textbackslash{}put(0,0)\{\textbackslash{}line(2,1)\{1\}\}}
\NormalTok{\textbackslash{}put(0,0)\{\textbackslash{}line(2,3)\{.6667\}\}}
\NormalTok{\textbackslash{}put(0,0)\{\textbackslash{}line(2,5)\{.4\}\}}
\NormalTok{\textbackslash{}put(0,0)\{\textbackslash{}line(3,1)\{1\}\}}
\NormalTok{\textbackslash{}put(0,0)\{\textbackslash{}line(3,2)\{1\}\}}
\NormalTok{\textbackslash{}put(0,0)\{\textbackslash{}line(3,4)\{.75\}\}}
\NormalTok{\textbackslash{}put(0,0)\{\textbackslash{}line(3,5)\{.6\}\}}
\NormalTok{\textbackslash{}put(0,0)\{\textbackslash{}line(4,1)\{1\}\}}
\NormalTok{\textbackslash{}put(0,0)\{\textbackslash{}line(4,3)\{1\}\}}
\NormalTok{\textbackslash{}put(0,0)\{\textbackslash{}line(4,5)\{.8\}\}}
\NormalTok{\textbackslash{}put(0,0)\{\textbackslash{}line(5,1)\{1\}\}}
\NormalTok{\textbackslash{}put(0,0)\{\textbackslash{}line(5,2)\{1\}\}}
\NormalTok{\textbackslash{}put(0,0)\{\textbackslash{}line(5,3)\{1\}\}}
\NormalTok{\textbackslash{}put(0,0)\{\textbackslash{}line(5,4)\{1\}\}}
\NormalTok{\textbackslash{}put(0,0)\{\textbackslash{}line(5,6)\{.8333\}\}}
\NormalTok{\textbackslash{}put(0,0)\{\textbackslash{}line(6,1)\{1\}\}}
\NormalTok{\textbackslash{}put(0,0)\{\textbackslash{}line(6,5)\{1\}\}}
\NormalTok{\textbackslash{}end\{picture\}}
\end{Highlighting}
\end{Shaded}
\\



\begin{minipage}{1.0\linewidth}
\begin{center}
\includegraphics[width=1.0\linewidth,height=6.5in,keepaspectratio]{../images/164.png}
\end{center}
\raggedright{}\myfigurewithoutcaption{164}
\end{minipage}\vspace{0.75cm}



\end{longtable}
\section{Arrows}
\label{782}

Arrows are drawn with the command
\begin{Shaded}
\begin{Highlighting}[]

\NormalTok{\textbackslash{}put(x, y)\{\textbackslash{}vector(x1, y1)\{length\}\}}
\end{Highlighting}
\end{Shaded}


For arrows, the components of the direction vector are even more narrowly restricted than for line segments, namely to the integers ({\itshape \setmainfont[Path=/usr/share/fonts/truetype/cmu/,UprightFont=cmunrm.ttf,BoldFont=cmunbx.ttf,ItalicFont=cmunti.ttf,BoldItalicFont=cmunbi.ttf]{cmunti.ttf}\setmonofont[Path=/usr/share/fonts/truetype/cmu/,UprightFont=cmuntt.ttf,BoldFont=cmuntb.ttf,ItalicFont=cmunit.ttf,BoldItalicFont=cmuntx.ttf]{cmunti.ttf}\itshape −4, −3, ... , 3, 4}\setmainfont[Path=/usr/share/fonts/truetype/cmu/,UprightFont=cmunrm.ttf,BoldFont=cmunbx.ttf,ItalicFont=cmunti.ttf,BoldItalicFont=cmunbi.ttf]{cmunrm.ttf}\setmonofont[Path=/usr/share/fonts/truetype/cmu/,UprightFont=cmuntt.ttf,BoldFont=cmuntb.ttf,ItalicFont=cmunit.ttf,BoldItalicFont=cmuntx.ttf]{cmunrm.ttf}). Components also have to be coprime (no common divisor except 1). Notice the effect of the \LaTeXTT{\textbackslash{}thicklines} command on the two arrows pointing to the upper left.

\begin{longtable}{p{1.0\linewidth}}
\begin{Shaded}
\begin{Highlighting}[]

\NormalTok{\textbackslash{}setlength\{\textbackslash{}unitlength\}\{0.75mm\}}
\NormalTok{\textbackslash{}begin\{picture\}(60,40)}
\NormalTok{\textbackslash{}put(30,20)\{\textbackslash{}vector(1,0)\{30\}\}}
\NormalTok{\textbackslash{}put(30,20)\{\textbackslash{}vector(4,1)\{20\}\}}
\NormalTok{\textbackslash{}put(30,20)\{\textbackslash{}vector(3,1)\{25\}\}}
\NormalTok{\textbackslash{}put(30,20)\{\textbackslash{}vector(2,1)\{30\}\}}
\NormalTok{\textbackslash{}put(30,20)\{\textbackslash{}vector(1,2)\{10\}\}}
\NormalTok{\textbackslash{}thicklines}
\NormalTok{\textbackslash{}put(30,20)\{\textbackslash{}vector(-4,1)\{30\}\}}
\NormalTok{\textbackslash{}put(30,20)\{\textbackslash{}vector(-1,4)\{5\}\}}
\NormalTok{\textbackslash{}thinlines}
\NormalTok{\textbackslash{}put(30,20)\{\textbackslash{}vector(-1,-1)\{5\}\}}
\NormalTok{\textbackslash{}put(30,20)\{\textbackslash{}vector(-1,-4)\{5\}\}}
\NormalTok{\textbackslash{}end\{picture\}}
\end{Highlighting}
\end{Shaded}
\\



\begin{minipage}{1.0\linewidth}
\begin{center}
\includegraphics[width=1.0\linewidth,height=6.5in,keepaspectratio]{../images/165.png}
\end{center}
\raggedright{}\myfigurewithoutcaption{165}
\end{minipage}\vspace{0.75cm}



\end{longtable}
\section{Circles}
\label{783}

The command
\begin{Shaded}
\begin{Highlighting}[]

\NormalTok{\textbackslash{}put(x, y)\{\textbackslash{}circle\{diameter\}\}}
\end{Highlighting}
\end{Shaded}


draws a circle with center (x, y) and diameter (not radius) specified by {\itshape \setmainfont[Path=/usr/share/fonts/truetype/cmu/,UprightFont=cmunrm.ttf,BoldFont=cmunbx.ttf,ItalicFont=cmunti.ttf,BoldItalicFont=cmunbi.ttf]{cmunti.ttf}\setmonofont[Path=/usr/share/fonts/truetype/cmu/,UprightFont=cmuntt.ttf,BoldFont=cmuntb.ttf,ItalicFont=cmunit.ttf,BoldItalicFont=cmuntx.ttf]{cmunti.ttf}\itshape diameter}\setmainfont[Path=/usr/share/fonts/truetype/cmu/,UprightFont=cmunrm.ttf,BoldFont=cmunbx.ttf,ItalicFont=cmunti.ttf,BoldItalicFont=cmunbi.ttf]{cmunrm.ttf}\setmonofont[Path=/usr/share/fonts/truetype/cmu/,UprightFont=cmuntt.ttf,BoldFont=cmuntb.ttf,ItalicFont=cmunit.ttf,BoldItalicFont=cmuntx.ttf]{cmunrm.ttf}. The picture environment only admits diameters up to approximately 14mm, and even below this limit, not all diameters are possible. The \LaTeXTT{\textbackslash{}circle*} command produces disks (filled circles). As in the case of line segments, one may have to resort to additional packages, such as \LaTeXTT{eepic}, \LaTeXTT{pstricks}, or \LaTeXTT{tikz}.

\begin{longtable}{p{1.0\linewidth}}
\begin{Shaded}
\begin{Highlighting}[]

\NormalTok{\textbackslash{}setlength\{\textbackslash{}unitlength\}\{1mm\}}
\NormalTok{\textbackslash{}begin\{picture\}(60, 40)}
\NormalTok{\textbackslash{}put(20,30)\{\textbackslash{}circle\{1\}\}}
\NormalTok{\textbackslash{}put(20,30)\{\textbackslash{}circle\{2\}\}}
\NormalTok{\textbackslash{}put(20,30)\{\textbackslash{}circle\{4\}\}}
\NormalTok{\textbackslash{}put(20,30)\{\textbackslash{}circle\{8\}\}}
\NormalTok{\textbackslash{}put(20,30)\{\textbackslash{}circle\{16\}\}}
\NormalTok{\textbackslash{}put(20,30)\{\textbackslash{}circle\{32\}\}}
\NormalTok{\textbackslash{}put(40,30)\{\textbackslash{}circle\{1\}\}}
\NormalTok{\textbackslash{}put(40,30)\{\textbackslash{}circle\{2\}\}}
\NormalTok{\textbackslash{}put(40,30)\{\textbackslash{}circle\{3\}\}}
\NormalTok{\textbackslash{}put(40,30)\{\textbackslash{}circle\{4\}\}}
\NormalTok{\textbackslash{}put(40,30)\{\textbackslash{}circle\{5\}\}}
\NormalTok{\textbackslash{}put(40,30)\{\textbackslash{}circle\{6\}\}}
\NormalTok{\textbackslash{}put(40,30)\{\textbackslash{}circle\{7\}\}}
\NormalTok{\textbackslash{}put(40,30)\{\textbackslash{}circle\{8\}\}}
\NormalTok{\textbackslash{}put(40,30)\{\textbackslash{}circle\{9\}\}}
\NormalTok{\textbackslash{}put(40,30)\{\textbackslash{}circle\{10\}\}}
\NormalTok{\textbackslash{}put(40,30)\{\textbackslash{}circle\{11\}\}}
\NormalTok{\textbackslash{}put(40,30)\{\textbackslash{}circle\{12\}\}}
\NormalTok{\textbackslash{}put(40,30)\{\textbackslash{}circle\{13\}\}}
\NormalTok{\textbackslash{}put(40,30)\{\textbackslash{}circle\{14\}\}}
\NormalTok{\textbackslash{}put(15,10)\{\textbackslash{}circle*\{1\}\}}
\NormalTok{\textbackslash{}put(20,10)\{\textbackslash{}circle*\{2\}\}}
\NormalTok{\textbackslash{}put(25,10)\{\textbackslash{}circle*\{3\}\}}
\NormalTok{\textbackslash{}put(30,10)\{\textbackslash{}circle*\{4\}\}}
\NormalTok{\textbackslash{}put(35,10)\{\textbackslash{}circle*\{5\}\}}
\NormalTok{\textbackslash{}end\{picture\}}
\end{Highlighting}
\end{Shaded}
\\



\begin{minipage}{1.0\linewidth}
\begin{center}
\includegraphics[width=1.0\linewidth,height=6.5in,keepaspectratio]{../images/166.png}
\end{center}
\raggedright{}\myfigurewithoutcaption{166}
\end{minipage}\vspace{0.75cm}



\end{longtable}

There is another possibility within the picture environment. If one is not afraid of doing the necessary calculations (or leaving them to a program), arbitrary circles and ellipses can be patched together from quadratic Bézier curves. See {\itshape \setmainfont[Path=/usr/share/fonts/truetype/cmu/,UprightFont=cmunrm.ttf,BoldFont=cmunbx.ttf,ItalicFont=cmunti.ttf,BoldItalicFont=cmunbi.ttf]{cmunti.ttf}\setmonofont[Path=/usr/share/fonts/truetype/cmu/,UprightFont=cmuntt.ttf,BoldFont=cmuntb.ttf,ItalicFont=cmunit.ttf,BoldItalicFont=cmuntx.ttf]{cmunti.ttf}\itshape Graphics in LaTeX2e}{$\text{ }$}\setmainfont[Path=/usr/share/fonts/truetype/cmu/,UprightFont=cmunrm.ttf,BoldFont=cmunbx.ttf,ItalicFont=cmunti.ttf,BoldItalicFont=cmunbi.ttf]{cmunrm.ttf}\setmonofont[Path=/usr/share/fonts/truetype/cmu/,UprightFont=cmuntt.ttf,BoldFont=cmuntb.ttf,ItalicFont=cmunit.ttf,BoldItalicFont=cmuntx.ttf]{cmunrm.ttf} for examples and Java source files.
\section{Text and formulae}
\label{784}

As this example shows, text and formulae can be written in the environment with the \LaTeXTT{\textbackslash{}put} command in the usual way:

\begin{longtable}{p{1.0\linewidth}}
\begin{Shaded}
\begin{Highlighting}[]

\NormalTok{\textbackslash{}setlength\{\textbackslash{}unitlength\}\{0.8cm\}}
\NormalTok{\textbackslash{}begin\{picture\}(6,5)}
\NormalTok{\textbackslash{}thicklines}
\NormalTok{\textbackslash{}put(1,0.5)\{\textbackslash{}line(2,1)\{3\}\}}
\NormalTok{\textbackslash{}put(4,2)\{\textbackslash{}line(-2,1)\{2\}\}}
\NormalTok{\textbackslash{}put(2,3)\{\textbackslash{}line(-2,-5)\{1\}\}}
\NormalTok{\textbackslash{}put(0.7,0.3)\{$A$\}}
\NormalTok{\textbackslash{}put(4.05,1.9)\{$B$\}}
\NormalTok{\textbackslash{}put(1.7,2.95)\{$C$\}}
\NormalTok{\textbackslash{}put(3.1,2.5)\{$a$\}}
\NormalTok{\textbackslash{}put(1.3,1.7)\{$b$\}}
\NormalTok{\textbackslash{}put(2.5,1.05)\{$c$\}}
\NormalTok{\textbackslash{}put(0.3,4)\{$F=\textbackslash{}sqrt\{s(s-a)(s-b)(s-c)\}$\}}
\NormalTok{\textbackslash{}put(3.5,0.4)\{$\textbackslash{}displaystyle s:=\textbackslash{}frac\{a+b+c\}\{2\}$\}}
\NormalTok{\textbackslash{}end\{picture\}}
\end{Highlighting}
\end{Shaded}
\\



\begin{minipage}{1.0\linewidth}
\begin{center}
\includegraphics[width=1.0\linewidth,height=6.5in,keepaspectratio]{../images/167.png}
\end{center}
\raggedright{}\myfigurewithoutcaption{167}
\end{minipage}\vspace{0.75cm}



\end{longtable}
\section{{\itshape \setmainfont[Path=/usr/share/fonts/truetype/cmu/,UprightFont=cmunrm.ttf,BoldFont=cmunbx.ttf,ItalicFont=cmunti.ttf,BoldItalicFont=cmunbi.ttf]{cmunti.ttf}\setmonofont[Path=/usr/share/fonts/truetype/cmu/,UprightFont=cmuntt.ttf,BoldFont=cmuntb.ttf,ItalicFont=cmunit.ttf,BoldItalicFont=cmuntx.ttf]{cmunti.ttf}\itshape \textbackslash{}multiput}{$\text{ }$}\setmainfont[Path=/usr/share/fonts/truetype/cmu/,UprightFont=cmunrm.ttf,BoldFont=cmunbx.ttf,ItalicFont=cmunti.ttf,BoldItalicFont=cmunbi.ttf]{cmunrm.ttf}\setmonofont[Path=/usr/share/fonts/truetype/cmu/,UprightFont=cmuntt.ttf,BoldFont=cmuntb.ttf,ItalicFont=cmunit.ttf,BoldItalicFont=cmuntx.ttf]{cmunrm.ttf} and {\itshape \setmainfont[Path=/usr/share/fonts/truetype/cmu/,UprightFont=cmunrm.ttf,BoldFont=cmunbx.ttf,ItalicFont=cmunti.ttf,BoldItalicFont=cmunbi.ttf]{cmunti.ttf}\setmonofont[Path=/usr/share/fonts/truetype/cmu/,UprightFont=cmuntt.ttf,BoldFont=cmuntb.ttf,ItalicFont=cmunit.ttf,BoldItalicFont=cmuntx.ttf]{cmunti.ttf}\itshape \textbackslash{}linethickness}}
\label{785}\setmainfont[Path=/usr/share/fonts/truetype/cmu/,UprightFont=cmunrm.ttf,BoldFont=cmunbx.ttf,ItalicFont=cmunti.ttf,BoldItalicFont=cmunbi.ttf]{cmunrm.ttf}\setmonofont[Path=/usr/share/fonts/truetype/cmu/,UprightFont=cmuntt.ttf,BoldFont=cmuntb.ttf,ItalicFont=cmunit.ttf,BoldItalicFont=cmuntx.ttf]{cmunrm.ttf}

The command
\begin{Shaded}
\begin{Highlighting}[]

\NormalTok{\textbackslash{}multiput(x, y)(dx, dy )\{n\}\{object\}}
\end{Highlighting}
\end{Shaded}


has 4 arguments: the starting point, the translation vector from one object to the next, the number of objects, and the object to be drawn. The {\ttfamily \setmainfont[Path=/usr/share/fonts/truetype/cmu/,UprightFont=cmunrm.ttf,BoldFont=cmunbx.ttf,ItalicFont=cmunti.ttf,BoldItalicFont=cmunbi.ttf]{cmuntt.ttf}\setmonofont[Path=/usr/share/fonts/truetype/cmu/,UprightFont=cmuntt.ttf,BoldFont=cmuntb.ttf,ItalicFont=cmunit.ttf,BoldItalicFont=cmuntx.ttf]{cmuntt.ttf}\ttfamily \textbackslash{}linethickness}{$\text{ }$}\setmainfont[Path=/usr/share/fonts/truetype/cmu/,UprightFont=cmunrm.ttf,BoldFont=cmunbx.ttf,ItalicFont=cmunti.ttf,BoldItalicFont=cmunbi.ttf]{cmunrm.ttf}\setmonofont[Path=/usr/share/fonts/truetype/cmu/,UprightFont=cmuntt.ttf,BoldFont=cmuntb.ttf,ItalicFont=cmunit.ttf,BoldItalicFont=cmuntx.ttf]{cmunrm.ttf} command applies to horizontal and vertical line segments, but neither to oblique line segments, nor to circles. It does, however, apply to quadratic Bézier curves!

\begin{longtable}{p{1.0\linewidth}}
\begin{Shaded}
\begin{Highlighting}[]

\NormalTok{\textbackslash{}setlength\{\textbackslash{}unitlength\}\{2mm\}}
\NormalTok{\textbackslash{}begin\{picture\}(30,20)}
\NormalTok{\textbackslash{}linethickness\{0.075mm\}}
\NormalTok{\textbackslash{}multiput(0,0)(1,0)\{26\}}\CommentTok
\NormalTok{\{\textbackslash{}line(1,0)\{25\}\}}
\NormalTok{\textbackslash{}linethickness\{0.15mm\}}
\NormalTok{\textbackslash{}multiput(0,0)(5,0)\{6\}}\CommentTok
\NormalTok{\{\textbackslash{}line(1,0)\{25\}\}}
\NormalTok{\textbackslash{}linethickness\{0.3mm\}}
\NormalTok{\textbackslash{}multiput(5,0)(10,0)\{2\}}\CommentTok
\NormalTok{\{\textbackslash{}line(1,0)\{25\}\}}
\NormalTok{\textbackslash{}end\{picture\}}
\end{Highlighting}
\end{Shaded}
\\



\begin{minipage}{1.0\linewidth}
\begin{center}
\includegraphics[width=1.0\linewidth,height=6.5in,keepaspectratio]{../images/168.png}
\end{center}
\raggedright{}\myfigurewithoutcaption{168}
\end{minipage}\vspace{0.75cm}



\end{longtable}
\section{Ovals}
\label{786}

The command
\begin{Shaded}
\begin{Highlighting}[]

\NormalTok{\textbackslash{}put(x, y)\{\textbackslash{}oval(w, h)\}}
\end{Highlighting}
\end{Shaded}


or
\begin{Shaded}
\begin{Highlighting}[]

\NormalTok{\textbackslash{}put(x, y)\{\textbackslash{}oval(w, h)[position]\}}
\end{Highlighting}
\end{Shaded}


produces an oval centered at {\itshape \setmainfont[Path=/usr/share/fonts/truetype/cmu/,UprightFont=cmunrm.ttf,BoldFont=cmunbx.ttf,ItalicFont=cmunti.ttf,BoldItalicFont=cmunbi.ttf]{cmunti.ttf}\setmonofont[Path=/usr/share/fonts/truetype/cmu/,UprightFont=cmuntt.ttf,BoldFont=cmuntb.ttf,ItalicFont=cmunit.ttf,BoldItalicFont=cmuntx.ttf]{cmunti.ttf}\itshape (x, y)}{$\text{ }$}\setmainfont[Path=/usr/share/fonts/truetype/cmu/,UprightFont=cmunrm.ttf,BoldFont=cmunbx.ttf,ItalicFont=cmunti.ttf,BoldItalicFont=cmunbi.ttf]{cmunrm.ttf}\setmonofont[Path=/usr/share/fonts/truetype/cmu/,UprightFont=cmuntt.ttf,BoldFont=cmuntb.ttf,ItalicFont=cmunit.ttf,BoldItalicFont=cmuntx.ttf]{cmunrm.ttf} and having width {\itshape \setmainfont[Path=/usr/share/fonts/truetype/cmu/,UprightFont=cmunrm.ttf,BoldFont=cmunbx.ttf,ItalicFont=cmunti.ttf,BoldItalicFont=cmunbi.ttf]{cmunti.ttf}\setmonofont[Path=/usr/share/fonts/truetype/cmu/,UprightFont=cmuntt.ttf,BoldFont=cmuntb.ttf,ItalicFont=cmunit.ttf,BoldItalicFont=cmuntx.ttf]{cmunti.ttf}\itshape w}{$\text{ }$}\setmainfont[Path=/usr/share/fonts/truetype/cmu/,UprightFont=cmunrm.ttf,BoldFont=cmunbx.ttf,ItalicFont=cmunti.ttf,BoldItalicFont=cmunbi.ttf]{cmunrm.ttf}\setmonofont[Path=/usr/share/fonts/truetype/cmu/,UprightFont=cmuntt.ttf,BoldFont=cmuntb.ttf,ItalicFont=cmunit.ttf,BoldItalicFont=cmuntx.ttf]{cmunrm.ttf} and height {\itshape \setmainfont[Path=/usr/share/fonts/truetype/cmu/,UprightFont=cmunrm.ttf,BoldFont=cmunbx.ttf,ItalicFont=cmunti.ttf,BoldItalicFont=cmunbi.ttf]{cmunti.ttf}\setmonofont[Path=/usr/share/fonts/truetype/cmu/,UprightFont=cmuntt.ttf,BoldFont=cmuntb.ttf,ItalicFont=cmunit.ttf,BoldItalicFont=cmuntx.ttf]{cmunti.ttf}\itshape h}\setmainfont[Path=/usr/share/fonts/truetype/cmu/,UprightFont=cmunrm.ttf,BoldFont=cmunbx.ttf,ItalicFont=cmunti.ttf,BoldItalicFont=cmunbi.ttf]{cmunrm.ttf}\setmonofont[Path=/usr/share/fonts/truetype/cmu/,UprightFont=cmuntt.ttf,BoldFont=cmuntb.ttf,ItalicFont=cmunit.ttf,BoldItalicFont=cmuntx.ttf]{cmunrm.ttf}. The optional position arguments {\itshape \setmainfont[Path=/usr/share/fonts/truetype/cmu/,UprightFont=cmunrm.ttf,BoldFont=cmunbx.ttf,ItalicFont=cmunti.ttf,BoldItalicFont=cmunbi.ttf]{cmunti.ttf}\setmonofont[Path=/usr/share/fonts/truetype/cmu/,UprightFont=cmuntt.ttf,BoldFont=cmuntb.ttf,ItalicFont=cmunit.ttf,BoldItalicFont=cmuntx.ttf]{cmunti.ttf}\itshape b, t, l, r}{$\text{ }$}\setmainfont[Path=/usr/share/fonts/truetype/cmu/,UprightFont=cmunrm.ttf,BoldFont=cmunbx.ttf,ItalicFont=cmunti.ttf,BoldItalicFont=cmunbi.ttf]{cmunrm.ttf}\setmonofont[Path=/usr/share/fonts/truetype/cmu/,UprightFont=cmuntt.ttf,BoldFont=cmuntb.ttf,ItalicFont=cmunit.ttf,BoldItalicFont=cmuntx.ttf]{cmunrm.ttf} refer to \symbol{34}top\symbol{34}, \symbol{34}bottom\symbol{34}, \symbol{34}left\symbol{34}, \symbol{34}right\symbol{34}, and can be combined, as the example illustrates. Line thickness can be controlled by two kinds of commands: \LaTeXTT{\textbackslash{}linethickness\{\textquotesingle{}\textquotesingle{}length\textquotesingle{}\textquotesingle{}\}} on the one hand, \LaTeXTT{\textbackslash{}thinlines} and \LaTeXTT{\textbackslash{}thicklines} on the other. While \LaTeXTT{\textbackslash{}linethickness\{\textquotesingle{}\textquotesingle{}length\textquotesingle{}\textquotesingle{}\}} applies only to horizontal and vertical lines (and quadratic Bézier curves), \LaTeXTT{\textbackslash{}thinlines} and \LaTeXTT{\textbackslash{}thicklines} apply to oblique line segments as well as to circles and ovals.

\begin{longtable}{p{1.0\linewidth}}
\begin{Shaded}
\begin{Highlighting}[]

\NormalTok{\textbackslash{}setlength\{\textbackslash{}unitlength\}\{0.75cm\}}
\NormalTok{\textbackslash{}begin\{picture\}(6,4)}
\NormalTok{\textbackslash{}linethickness\{0.075mm\}}
\NormalTok{\textbackslash{}multiput(0,0)(1,0)\{7\}}\CommentTok
\NormalTok{\{\textbackslash{}line(1,0)\{6\}\}}
\NormalTok{\textbackslash{}thicklines}
\NormalTok{\textbackslash{}put(2,3)\{\textbackslash{}oval(3,1.8)\}}
\NormalTok{\textbackslash{}thinlines}
\NormalTok{\textbackslash{}put(3,2)\{\textbackslash{}oval(3,1.8)\}}
\NormalTok{\textbackslash{}thicklines}
\NormalTok{\textbackslash{}put(2,1)\{\textbackslash{}oval(3,1.8)[tl]\}}
\NormalTok{\textbackslash{}put(4,1)\{\textbackslash{}oval(3,1.8)[b]\}}
\NormalTok{\textbackslash{}put(4,3)\{\textbackslash{}oval(3,1.8)[r]\}}
\NormalTok{\textbackslash{}put(3,1.5)\{\textbackslash{}oval(1.8,0.4)\}}
\NormalTok{\textbackslash{}end\{picture\}}
\end{Highlighting}
\end{Shaded}
\\



\begin{minipage}{1.0\linewidth}
\begin{center}
\includegraphics[width=1.0\linewidth,height=6.5in,keepaspectratio]{../images/169.png}
\end{center}
\raggedright{}\myfigurewithoutcaption{169}
\end{minipage}\vspace{0.75cm}



\end{longtable}
\section{Multiple use of predefined picture boxes}
\label{787}

A picture box can be {\itshape \setmainfont[Path=/usr/share/fonts/truetype/cmu/,UprightFont=cmunrm.ttf,BoldFont=cmunbx.ttf,ItalicFont=cmunti.ttf,BoldItalicFont=cmunbi.ttf]{cmunti.ttf}\setmonofont[Path=/usr/share/fonts/truetype/cmu/,UprightFont=cmuntt.ttf,BoldFont=cmuntb.ttf,ItalicFont=cmunit.ttf,BoldItalicFont=cmuntx.ttf]{cmunti.ttf}\itshape declared}{$\text{ }$}\setmainfont[Path=/usr/share/fonts/truetype/cmu/,UprightFont=cmunrm.ttf,BoldFont=cmunbx.ttf,ItalicFont=cmunti.ttf,BoldItalicFont=cmunbi.ttf]{cmunrm.ttf}\setmonofont[Path=/usr/share/fonts/truetype/cmu/,UprightFont=cmuntt.ttf,BoldFont=cmuntb.ttf,ItalicFont=cmunit.ttf,BoldItalicFont=cmuntx.ttf]{cmunrm.ttf} by the command

\begin{Shaded}
\begin{Highlighting}[]

\NormalTok{\textbackslash{}newsavebox\{name\}}
\end{Highlighting}
\end{Shaded}


then {\itshape \setmainfont[Path=/usr/share/fonts/truetype/cmu/,UprightFont=cmunrm.ttf,BoldFont=cmunbx.ttf,ItalicFont=cmunti.ttf,BoldItalicFont=cmunbi.ttf]{cmunti.ttf}\setmonofont[Path=/usr/share/fonts/truetype/cmu/,UprightFont=cmuntt.ttf,BoldFont=cmuntb.ttf,ItalicFont=cmunit.ttf,BoldItalicFont=cmuntx.ttf]{cmunti.ttf}\itshape defined}{$\text{ }$}\setmainfont[Path=/usr/share/fonts/truetype/cmu/,UprightFont=cmunrm.ttf,BoldFont=cmunbx.ttf,ItalicFont=cmunti.ttf,BoldItalicFont=cmunbi.ttf]{cmunrm.ttf}\setmonofont[Path=/usr/share/fonts/truetype/cmu/,UprightFont=cmuntt.ttf,BoldFont=cmuntb.ttf,ItalicFont=cmunit.ttf,BoldItalicFont=cmuntx.ttf]{cmunrm.ttf} by

\begin{Shaded}
\begin{Highlighting}[]

\NormalTok{\textbackslash{}savebox\{name\}(width,height)[position]\{content\}}
\end{Highlighting}
\end{Shaded}


and finally arbitrarily often be {\itshape \setmainfont[Path=/usr/share/fonts/truetype/cmu/,UprightFont=cmunrm.ttf,BoldFont=cmunbx.ttf,ItalicFont=cmunti.ttf,BoldItalicFont=cmunbi.ttf]{cmunti.ttf}\setmonofont[Path=/usr/share/fonts/truetype/cmu/,UprightFont=cmuntt.ttf,BoldFont=cmuntb.ttf,ItalicFont=cmunit.ttf,BoldItalicFont=cmuntx.ttf]{cmunti.ttf}\itshape drawn}{$\text{ }$}\setmainfont[Path=/usr/share/fonts/truetype/cmu/,UprightFont=cmunrm.ttf,BoldFont=cmunbx.ttf,ItalicFont=cmunti.ttf,BoldItalicFont=cmunbi.ttf]{cmunrm.ttf}\setmonofont[Path=/usr/share/fonts/truetype/cmu/,UprightFont=cmuntt.ttf,BoldFont=cmuntb.ttf,ItalicFont=cmunit.ttf,BoldItalicFont=cmuntx.ttf]{cmunrm.ttf} by

\begin{Shaded}
\begin{Highlighting}[]

\NormalTok{\textbackslash{}put(x, y)\{\textbackslash{}usebox\{name\}\}}
\end{Highlighting}
\end{Shaded}


The optional position parameter has the effect of defining the \symbol{34}anchor point\symbol{34} of the savebox. In the example it is set to \symbol{34}bl\symbol{34} which puts the anchor point into the bottom left corner of the savebox. The other position specifiers are top and right.

The {\itshape \setmainfont[Path=/usr/share/fonts/truetype/cmu/,UprightFont=cmunrm.ttf,BoldFont=cmunbx.ttf,ItalicFont=cmunti.ttf,BoldItalicFont=cmunbi.ttf]{cmunti.ttf}\setmonofont[Path=/usr/share/fonts/truetype/cmu/,UprightFont=cmuntt.ttf,BoldFont=cmuntb.ttf,ItalicFont=cmunit.ttf,BoldItalicFont=cmuntx.ttf]{cmunti.ttf}\itshape name}{$\text{ }$}\setmainfont[Path=/usr/share/fonts/truetype/cmu/,UprightFont=cmunrm.ttf,BoldFont=cmunbx.ttf,ItalicFont=cmunti.ttf,BoldItalicFont=cmunbi.ttf]{cmunrm.ttf}\setmonofont[Path=/usr/share/fonts/truetype/cmu/,UprightFont=cmuntt.ttf,BoldFont=cmuntb.ttf,ItalicFont=cmunit.ttf,BoldItalicFont=cmuntx.ttf]{cmunrm.ttf} argument refers to a LaTeX storage bin and therefore is of a command nature (which accounts for the backslashes in the current example). Boxed pictures can be nested: In this example, \LaTeXTT{\textbackslash{}foldera} is used within the definition of \LaTeXTT{\textbackslash{}folderb}. The \LaTeXTT{\textbackslash{}oval} command had to be used as the \LaTeXTT{\textbackslash{}line} command does not work if the segment length is less than about 3 mm.

\begin{longtable}{p{1.0\linewidth}}
\begin{Shaded}
\begin{Highlighting}[]

\NormalTok{\textbackslash{}setlength\{\textbackslash{}unitlength\}\{0.5mm\}}
\NormalTok{\textbackslash{}begin\{picture\}(120,168)}
\NormalTok{\textbackslash{}newsavebox\{\textbackslash{}foldera\}}
\NormalTok{\textbackslash{}savebox\{\textbackslash{}foldera\}}
  \NormalTok{(40,32)[bl]\{}\CommentTok{% definition}
  \NormalTok{\textbackslash{}multiput(0,0)(0,28)\{2\}}
    \NormalTok{\{\textbackslash{}line(1,0)\{40\}<!---->\}}
  \NormalTok{\textbackslash{}multiput(0,0)(40,0)\{2\}}
    \NormalTok{\{\textbackslash{}line(0,1)\{28\}<!---->\}}
  \NormalTok{\textbackslash{}put(1,28)\{\textbackslash{}oval(2,2)[tl]\}}
  \NormalTok{\textbackslash{}put(1,29)\{\textbackslash{}line(1,0)\{5\}<!---->\}}
  \NormalTok{\textbackslash{}put(9,29)\{\textbackslash{}oval(6,6)[tl]\}}
  \NormalTok{\textbackslash{}put(9,32)\{\textbackslash{}line(1,0)\{8\}<!---->\}}
  \NormalTok{\textbackslash{}put(17,29)\{\textbackslash{}oval(6,6)[tr]\}}
  \NormalTok{\textbackslash{}put(20,29)\{\textbackslash{}line(1,0)\{19\}<!---->\}}
  \NormalTok{\textbackslash{}put(39,28)\{\textbackslash{}oval(2,2)[tr]\}}
\NormalTok{\}}
 
\NormalTok{\textbackslash{}newsavebox\{\textbackslash{}folderb\}}
\NormalTok{\textbackslash{}savebox\{\textbackslash{}folderb\}}
  \NormalTok{(40,32)[l]\{}\CommentTok{% definition}
  \NormalTok{\textbackslash{}put(0,14)\{\textbackslash{}line(1,0)\{8\}<!---->\}}
  \NormalTok{\textbackslash{}put(8,0)\{\textbackslash{}usebox\{\textbackslash{}foldera\}<!---->\}}
\NormalTok{\}}
 
\NormalTok{\textbackslash{}put(34,26)\{\textbackslash{}line(0,1)\{102\}\}}
\NormalTok{\textbackslash{}put(14,128)\{\textbackslash{}usebox\{\textbackslash{}foldera\}\}}
\NormalTok{\textbackslash{}multiput(34,86)(0,-37)\{3\}}
\NormalTok{\{\textbackslash{}usebox\{\textbackslash{}folderb\}\}}
\NormalTok{\textbackslash{}end\{picture\}}
\end{Highlighting}
\end{Shaded}
\\



\begin{minipage}{0.75000\textwidth}
\begin{center}
\includegraphics[width=1.0\textwidth,height=6.5in,keepaspectratio]{../images/170.png}
\end{center}
\raggedright{}\myfigurewithoutcaption{170}
\end{minipage}\vspace{0.75cm}



\end{longtable}
\section{Quadratic Bézier curves}
\label{788}

The command 

\begin{Shaded}
\begin{Highlighting}[]

\NormalTok{\textbackslash{}qbezier(x1, y1)(x, y)(x2, y2)}
\end{Highlighting}
\end{Shaded}


draws a quadratic bezier curve where {$P_1 = (x_1, y_1)$}, {$P_2 = (x_2, y_2)$} denote the end points, and {$S = (x, y)$} denotes the intermediate control point. The respective tangent slopes, {$m_1$} and {$m_2$}, can be obtained from the equations
\begin{myquote}
\item{} \begin{equation*} \begin{cases} x= \frac{m_2 x_2 - m_1 x_1  - (y_2 - y_1)}{m_2 - m_1} \\ y= y_i + m_i (x - x_i); \quad (i=1,2\, \hbox{ gives same solution}) \end{cases} \end{equation*}
\end{myquote}


See {\itshape \setmainfont[Path=/usr/share/fonts/truetype/cmu/,UprightFont=cmunrm.ttf,BoldFont=cmunbx.ttf,ItalicFont=cmunti.ttf,BoldItalicFont=cmunbi.ttf]{cmunti.ttf}\setmonofont[Path=/usr/share/fonts/truetype/cmu/,UprightFont=cmuntt.ttf,BoldFont=cmuntb.ttf,ItalicFont=cmunit.ttf,BoldItalicFont=cmuntx.ttf]{cmunti.ttf}\itshape Graphics in LaTeX2e}{$\text{ }$}\setmainfont[Path=/usr/share/fonts/truetype/cmu/,UprightFont=cmunrm.ttf,BoldFont=cmunbx.ttf,ItalicFont=cmunti.ttf,BoldItalicFont=cmunbi.ttf]{cmunrm.ttf}\setmonofont[Path=/usr/share/fonts/truetype/cmu/,UprightFont=cmuntt.ttf,BoldFont=cmuntb.ttf,ItalicFont=cmunit.ttf,BoldItalicFont=cmuntx.ttf]{cmunrm.ttf} for a Java program which generates the necessary \LaTeXTT{\textbackslash{}qbezier} command line.

\begin{longtable}{p{1.0\linewidth}}
\begin{Shaded}
\begin{Highlighting}[]

\NormalTok{\textbackslash{}setlength\{\textbackslash{}unitlength\}\{0.8cm\}}
\NormalTok{\textbackslash{}begin\{picture\}(6,4)}
\NormalTok{\textbackslash{}linethickness\{0.075mm\}}
\NormalTok{\textbackslash{}multiput(0,0)(1,0)\{7\}}
\NormalTok{\{\textbackslash{}line(0,1)\{4\}\}}
\NormalTok{\textbackslash{}multiput(0,0)(0,1)\{5\}}
\NormalTok{\{\textbackslash{}line(1,0)\{6\}\}}
\NormalTok{\textbackslash{}thicklines}
\NormalTok{\textbackslash{}put(0.5,0.5)\{\textbackslash{}line(1,5)\{0.5\}\}}
\NormalTok{\textbackslash{}put(1,3)\{\textbackslash{}line(4,1)\{2\}\}}
\NormalTok{\textbackslash{}qbezier(0.5,0.5)(1,3)(3,3.5)}
\NormalTok{\textbackslash{}thinlines}
\NormalTok{\textbackslash{}put(2.5,2)\{\textbackslash{}line(2,-1)\{3\}\}}
\NormalTok{\textbackslash{}put(5.5,0.5)\{\textbackslash{}line(-1,5)\{0.5\}\}}
\NormalTok{\textbackslash{}linethickness\{1mm\}}
\NormalTok{\textbackslash{}qbezier(2.5,2)(5.5,0.5)(5,3)}
\NormalTok{\textbackslash{}thinlines}
\NormalTok{\textbackslash{}qbezier(4,2)(4,3)(3,3)}
\NormalTok{\textbackslash{}qbezier(3,3)(2,3)(2,2)}
\NormalTok{\textbackslash{}qbezier(2,2)(2,1)(3,1)}
\NormalTok{\textbackslash{}qbezier(3,1)(4,1)(4,2)}
\NormalTok{\textbackslash{}end\{picture\}}
\end{Highlighting}
\end{Shaded}
\\



\begin{minipage}{1.0\linewidth}
\begin{center}
\includegraphics[width=1.0\linewidth,height=6.5in,keepaspectratio]{../images/171.png}
\end{center}
\raggedright{}\myfigurewithoutcaption{171}
\end{minipage}\vspace{0.75cm}



\end{longtable}

As this example illustrates, splitting up a circle into 4 quadratic Bézier curves is not satisfactory. At least 8 are needed. The figure again shows the effect of the \LaTeXTT{\textbackslash{}linethickness} command on horizontal or vertical lines, and of the \LaTeXTT{\textbackslash{}thinlines} and the \LaTeXTT{\textbackslash{}thicklines} commands on oblique line segments. It also shows that both kinds of commands affect quadratic Bézier curves, each command overriding all previous ones.
\section{Catenary}
\label{789}

\begin{longtable}{p{1.0\linewidth}}
\begin{Shaded}
\begin{Highlighting}[]

\NormalTok{\textbackslash{}setlength\{\textbackslash{}unitlength\}\{1cm\}}
\NormalTok{\textbackslash{}begin\{picture\}(4.3,3.6)(-2.5,-0.25)}
\NormalTok{\textbackslash{}put(-2,0)\{\textbackslash{}vector(1,0)\{4.4\}\}}
\NormalTok{\textbackslash{}put(2.45,-.05)\{$x$\}}
\NormalTok{\textbackslash{}put(0,0)\{\textbackslash{}vector(0,1)\{3.2\}\}}
\NormalTok{\textbackslash{}put(0,3.35)\{\textbackslash{}makebox(0,0)\{$y$\}\}}
\NormalTok{\textbackslash{}qbezier(0.0,0.0)(1.2384,0.0)}
\NormalTok{(2.0,2.7622)}
\NormalTok{\textbackslash{}qbezier(0.0,0.0)(-1.2384,0.0)}
\NormalTok{(-2.0,2.7622)}
\NormalTok{\textbackslash{}linethickness\{.075mm\}}
\NormalTok{\textbackslash{}multiput(-2,0)(1,0)\{5\}}
\NormalTok{\{\textbackslash{}line(0,1)\{3\}\}}
\NormalTok{\textbackslash{}multiput(-2,0)(0,1)\{4\}}
\NormalTok{\{\textbackslash{}line(1,0)\{4\}\}}
\NormalTok{\textbackslash{}linethickness\{.2mm\}}
\NormalTok{\textbackslash{}put( .3,.12763)\{\textbackslash{}line(1,0)\{.4\}\}}
\NormalTok{\textbackslash{}put(.5,-.07237)\{\textbackslash{}line(0,1)\{.4\}\}}
\NormalTok{\textbackslash{}put(-.7,.12763)\{\textbackslash{}line(1,0)\{.4\}\}}
\NormalTok{\textbackslash{}put(-.5,-.07237)\{\textbackslash{}line(0,1)\{.4\}\}}
\NormalTok{\textbackslash{}put(.8,.54308)\{\textbackslash{}line(1,0)\{.4\}\}}
\NormalTok{\textbackslash{}put(1,.34308)\{\textbackslash{}line(0,1)\{.4\}\}}
\NormalTok{\textbackslash{}put(-1.2,.54308)\{\textbackslash{}line(1,0)\{.4\}\}}
\NormalTok{\textbackslash{}put(-1,.34308)\{\textbackslash{}line(0,1)\{.4\}\}}
\NormalTok{\textbackslash{}put(1.3,1.35241)\{\textbackslash{}line(1,0)\{.4\}\}}
\NormalTok{\textbackslash{}put(1.5,1.15241)\{\textbackslash{}line(0,1)\{.4\}\}}
\NormalTok{\textbackslash{}put(-1.7,1.35241)\{\textbackslash{}line(1,0)\{.4\}\}}
\NormalTok{\textbackslash{}put(-1.5,1.15241)\{\textbackslash{}line(0,1)\{.4\}\}}
\NormalTok{\textbackslash{}put(-2.5,-0.25)\{\textbackslash{}circle*\{0.2\}\}}
\NormalTok{\textbackslash{}end\{picture\}}
\end{Highlighting}
\end{Shaded}
\\



\begin{minipage}{1.0\linewidth}
\begin{center}
\includegraphics[width=1.0\linewidth,height=6.5in,keepaspectratio]{../images/172.png}
\end{center}
\raggedright{}\myfigurewithoutcaption{172}
\end{minipage}\vspace{0.75cm}



\end{longtable}

In this figure, each symmetric half of the catenary {$y= \cosh x - 1$} is approximated by a quadratic Bézier curve. The right half of the curve ends in the point (2, 2.7622), the slope there having the value m = 3.6269. Using again equation (*), we can calculate the intermediate control points. They turn out to be (1.2384, 0) and (−1.2384, 0). The crosses indicate points of the real catenary. The error is barely noticeable, being less than one percent. This example points out the use of the optional argument of the \LaTeXTT{\textbackslash{}begin\{picture\}} command. The picture is defined in convenient \symbol{34}mathematical\symbol{34} coordinates, whereas by the command

\begin{Shaded}
\begin{Highlighting}[]

\NormalTok{\textbackslash{}begin\{picture\}(4.3,3.6)(-2.5,-0.25)}
\end{Highlighting}
\end{Shaded}


its lower left corner (marked by the black disk) is assigned the coordinates (−2.5,−0.25).
\section{Plotting graphs}
\label{790}

\begin{longtable}{p{1.0\linewidth}}
\begin{Shaded}
\begin{Highlighting}[]

\NormalTok{\textbackslash{}setlength\{\textbackslash{}unitlength\}\{1cm\}}
\NormalTok{\textbackslash{}begin\{picture\}(6,6)(-3,-3)}
\NormalTok{\textbackslash{}put(-1.5,0)\{\textbackslash{}vector(1,0)\{3\}\}}
\NormalTok{\textbackslash{}put(2.7,-0.1)\{$\textbackslash{}chi$\}}
\NormalTok{\textbackslash{}put(0,-1.5)\{\textbackslash{}vector(0,1)\{3\}\}}
\NormalTok{\textbackslash{}multiput(-2.5,1)(0.4,0)\{13\}}
\NormalTok{\{\textbackslash{}line(1,0)\{0.2\}\}}
\NormalTok{\textbackslash{}multiput(-2.5,-1)(0.4,0)\{13\}}
\NormalTok{\{\textbackslash{}line(1,0)\{0.2\}\}}
\NormalTok{\textbackslash{}put(0.2,1.4)}
\NormalTok{\{$\textbackslash{}beta=v/c=\textbackslash{}tanh\textbackslash{}chi$\}}
\NormalTok{\textbackslash{}qbezier(0,0)(0.8853,0.8853)}
\NormalTok{(2,0.9640)}
\NormalTok{\textbackslash{}qbezier(0,0)(-0.8853,-0.8853)}
\NormalTok{(-2,-0.9640)}
\NormalTok{\textbackslash{}put(-3,-2)\{\textbackslash{}circle*\{0.2\}\}}
\NormalTok{\textbackslash{}end\{picture\}}
\end{Highlighting}
\end{Shaded}
\\



\begin{minipage}{1.0\linewidth}
\begin{center}
\includegraphics[width=1.0\linewidth,height=6.5in,keepaspectratio]{../images/173.png}
\end{center}
\raggedright{}\myfigurewithoutcaption{173}
\end{minipage}\vspace{0.75cm}



\end{longtable}


The control points of the two Bézier curves were calculated with formulas (*). The positive branch is determined by {$P_1 = (0, 0)$}, {$m_1 = 1$} and {$P_2 = (2, \tanh 2)$}, {$m_2 = 1/ \cosh^2 2$}. Again, the picture is defined in mathematically convenient coordinates, and the lower left corner is assigned the mathematical coordinates (−3,−2) (black disk).
\section{The {\itshape \setmainfont[Path=/usr/share/fonts/truetype/cmu/,UprightFont=cmunrm.ttf,BoldFont=cmunbx.ttf,ItalicFont=cmunti.ttf,BoldItalicFont=cmunbi.ttf]{cmunti.ttf}\setmonofont[Path=/usr/share/fonts/truetype/cmu/,UprightFont=cmuntt.ttf,BoldFont=cmuntb.ttf,ItalicFont=cmunit.ttf,BoldItalicFont=cmuntx.ttf]{cmunti.ttf}\itshape picture}{$\text{ }$}\setmainfont[Path=/usr/share/fonts/truetype/cmu/,UprightFont=cmunrm.ttf,BoldFont=cmunbx.ttf,ItalicFont=cmunti.ttf,BoldItalicFont=cmunbi.ttf]{cmunrm.ttf}\setmonofont[Path=/usr/share/fonts/truetype/cmu/,UprightFont=cmuntt.ttf,BoldFont=cmuntb.ttf,ItalicFont=cmunit.ttf,BoldItalicFont=cmuntx.ttf]{cmunrm.ttf} environment and gnuplot}
\label{791}

The powerful scientific plotting package \myhref{https://en.wikipedia.org/wiki/gnuplot}{gnuplot} has the capability to output directly to a LaTeX \LaTeXTT{picture} environment. It is often far more convenient to plot directly to LaTeX, since this saves having to deal with potentially troublesome postscript files. Plotting scientific data (or, indeed, mathematical figures) this way gives much greater control, and of course typesetting ability, than is available from other means (such as postscript).
Such pictures can then be added to a document by an \LaTeXTT{\textbackslash{}include\{\}} command.

N.B. gnuplot is a powerful piece of software with a vast array of commands. A full discussion of gnuplot lies beyond the scope of this note. See \mylref{0}{\myplainurl{http://www.gnuplot.info/files/tutorial.pdf}}\myplainurl{http://} for a tutorial.



\myhref{https://sr.wikibooks.org/wiki/LaTeX\%2F\%D0\%A1\%D0\%BB\%D0\%B8\%D0\%BA\%D0\%B0}{sr:LaTeX/Слика}\chapter{PGF/TikZ}

\myminitoc
\label{792}

\label{793}



\begin{minipage}{0.62500\textwidth}
\begin{center}
\includegraphics[width=1.0\textwidth,height=6.5in,keepaspectratio]{../images/174.\SVGExtension}
\end{center}
\raggedright{}\myfigurewithcaption{174}{Example of graphics done with Ti{\itshape \setmainfont[Path=/usr/share/fonts/truetype/cmu/,UprightFont=cmunrm.ttf,BoldFont=cmunbx.ttf,ItalicFont=cmunti.ttf,BoldItalicFont=cmunbi.ttf]{cmunti.ttf}\setmonofont[Path=/usr/share/fonts/truetype/cmu/,UprightFont=cmuntt.ttf,BoldFont=cmuntb.ttf,ItalicFont=cmunit.ttf,BoldItalicFont=cmuntx.ttf]{cmunti.ttf}\itshape k}\setmainfont[Path=/usr/share/fonts/truetype/cmu/,UprightFont=cmunrm.ttf,BoldFont=cmunbx.ttf,ItalicFont=cmunti.ttf,BoldItalicFont=cmunbi.ttf]{cmunrm.ttf}\setmonofont[Path=/usr/share/fonts/truetype/cmu/,UprightFont=cmuntt.ttf,BoldFont=cmuntb.ttf,ItalicFont=cmunit.ttf,BoldItalicFont=cmuntx.ttf]{cmunrm.ttf}z. Note the slightly translucent top layer.}
\end{minipage}\vspace{0.75cm}



One way to draw graphics directly with TeX commands is \myhref{https://en.wikipedia.org/wiki/PGF\%2FTikZ}{PGF/TikZ}. Ti{\itshape \setmainfont[Path=/usr/share/fonts/truetype/cmu/,UprightFont=cmunrm.ttf,BoldFont=cmunbx.ttf,ItalicFont=cmunti.ttf,BoldItalicFont=cmunbi.ttf]{cmunti.ttf}\setmonofont[Path=/usr/share/fonts/truetype/cmu/,UprightFont=cmuntt.ttf,BoldFont=cmuntb.ttf,ItalicFont=cmunit.ttf,BoldItalicFont=cmuntx.ttf]{cmunti.ttf}\itshape k}\setmainfont[Path=/usr/share/fonts/truetype/cmu/,UprightFont=cmunrm.ttf,BoldFont=cmunbx.ttf,ItalicFont=cmunti.ttf,BoldItalicFont=cmunbi.ttf]{cmunrm.ttf}\setmonofont[Path=/usr/share/fonts/truetype/cmu/,UprightFont=cmuntt.ttf,BoldFont=cmuntb.ttf,ItalicFont=cmunit.ttf,BoldItalicFont=cmuntx.ttf]{cmunrm.ttf}Z  can produce portable graphics in both PDF and PostScript formats using either plain (pdf)TEX, (pdf)Latex or ConTEXt. It comes with very good \myhref{http://ftp.fau.de/ctan/graphics/pgf/base/doc/pgfmanual.pdf}{documentation} and an extensive collection of examples: \myplainurl{http://www.texample.net/tikz/}

PGF (\symbol{34}portable graphics format\symbol{34}) is the basic layer, providing a set of basic commands for producing graphics, and Ti{\itshape \setmainfont[Path=/usr/share/fonts/truetype/cmu/,UprightFont=cmunrm.ttf,BoldFont=cmunbx.ttf,ItalicFont=cmunti.ttf,BoldItalicFont=cmunbi.ttf]{cmunti.ttf}\setmonofont[Path=/usr/share/fonts/truetype/cmu/,UprightFont=cmuntt.ttf,BoldFont=cmuntb.ttf,ItalicFont=cmunit.ttf,BoldItalicFont=cmuntx.ttf]{cmunti.ttf}\itshape k}\setmainfont[Path=/usr/share/fonts/truetype/cmu/,UprightFont=cmunrm.ttf,BoldFont=cmunbx.ttf,ItalicFont=cmunti.ttf,BoldItalicFont=cmunbi.ttf]{cmunrm.ttf}\setmonofont[Path=/usr/share/fonts/truetype/cmu/,UprightFont=cmuntt.ttf,BoldFont=cmuntb.ttf,ItalicFont=cmunit.ttf,BoldItalicFont=cmuntx.ttf]{cmunrm.ttf}Z (\symbol{34}Ti{\itshape \setmainfont[Path=/usr/share/fonts/truetype/cmu/,UprightFont=cmunrm.ttf,BoldFont=cmunbx.ttf,ItalicFont=cmunti.ttf,BoldItalicFont=cmunbi.ttf]{cmunti.ttf}\setmonofont[Path=/usr/share/fonts/truetype/cmu/,UprightFont=cmuntt.ttf,BoldFont=cmuntb.ttf,ItalicFont=cmunit.ttf,BoldItalicFont=cmuntx.ttf]{cmunti.ttf}\itshape k}\setmainfont[Path=/usr/share/fonts/truetype/cmu/,UprightFont=cmunrm.ttf,BoldFont=cmunbx.ttf,ItalicFont=cmunti.ttf,BoldItalicFont=cmunbi.ttf]{cmunrm.ttf}\setmonofont[Path=/usr/share/fonts/truetype/cmu/,UprightFont=cmuntt.ttf,BoldFont=cmuntb.ttf,ItalicFont=cmunit.ttf,BoldItalicFont=cmuntx.ttf]{cmunrm.ttf}Z ist {\itshape \setmainfont[Path=/usr/share/fonts/truetype/cmu/,UprightFont=cmunrm.ttf,BoldFont=cmunbx.ttf,ItalicFont=cmunti.ttf,BoldItalicFont=cmunbi.ttf]{cmunti.ttf}\setmonofont[Path=/usr/share/fonts/truetype/cmu/,UprightFont=cmuntt.ttf,BoldFont=cmuntb.ttf,ItalicFont=cmunit.ttf,BoldItalicFont=cmuntx.ttf]{cmunti.ttf}\itshape kein}{$\text{ }$}\setmainfont[Path=/usr/share/fonts/truetype/cmu/,UprightFont=cmunrm.ttf,BoldFont=cmunbx.ttf,ItalicFont=cmunti.ttf,BoldItalicFont=cmunbi.ttf]{cmunrm.ttf}\setmonofont[Path=/usr/share/fonts/truetype/cmu/,UprightFont=cmuntt.ttf,BoldFont=cmuntb.ttf,ItalicFont=cmunit.ttf,BoldItalicFont=cmuntx.ttf]{cmunrm.ttf} Zeichenprogramm\symbol{34}) is the frontend layer with a special syntax, making the use of PGF easier. Ti{\itshape \setmainfont[Path=/usr/share/fonts/truetype/cmu/,UprightFont=cmunrm.ttf,BoldFont=cmunbx.ttf,ItalicFont=cmunti.ttf,BoldItalicFont=cmunbi.ttf]{cmunti.ttf}\setmonofont[Path=/usr/share/fonts/truetype/cmu/,UprightFont=cmuntt.ttf,BoldFont=cmuntb.ttf,ItalicFont=cmunit.ttf,BoldItalicFont=cmuntx.ttf]{cmunti.ttf}\itshape k}\setmainfont[Path=/usr/share/fonts/truetype/cmu/,UprightFont=cmunrm.ttf,BoldFont=cmunbx.ttf,ItalicFont=cmunti.ttf,BoldItalicFont=cmunbi.ttf]{cmunrm.ttf}\setmonofont[Path=/usr/share/fonts/truetype/cmu/,UprightFont=cmuntt.ttf,BoldFont=cmuntb.ttf,ItalicFont=cmunit.ttf,BoldItalicFont=cmuntx.ttf]{cmunrm.ttf}Z commands are prevalently similar to Metafont, the option mechanism is similar to PsTricks syntax. 

While the previous systems ({\ttfamily \setmainfont[Path=/usr/share/fonts/truetype/cmu/,UprightFont=cmunrm.ttf,BoldFont=cmunbx.ttf,ItalicFont=cmunti.ttf,BoldItalicFont=cmunbi.ttf]{cmuntt.ttf}\setmonofont[Path=/usr/share/fonts/truetype/cmu/,UprightFont=cmuntt.ttf,BoldFont=cmuntb.ttf,ItalicFont=cmunit.ttf,BoldItalicFont=cmuntx.ttf]{cmuntt.ttf}\ttfamily picture}\setmainfont[Path=/usr/share/fonts/truetype/cmu/,UprightFont=cmunrm.ttf,BoldFont=cmunbx.ttf,ItalicFont=cmunti.ttf,BoldItalicFont=cmunbi.ttf]{cmunrm.ttf}\setmonofont[Path=/usr/share/fonts/truetype/cmu/,UprightFont=cmuntt.ttf,BoldFont=cmuntb.ttf,ItalicFont=cmunit.ttf,BoldItalicFont=cmuntx.ttf]{cmunrm.ttf}, {\ttfamily \setmainfont[Path=/usr/share/fonts/truetype/cmu/,UprightFont=cmunrm.ttf,BoldFont=cmunbx.ttf,ItalicFont=cmunti.ttf,BoldItalicFont=cmunbi.ttf]{cmuntt.ttf}\setmonofont[Path=/usr/share/fonts/truetype/cmu/,UprightFont=cmuntt.ttf,BoldFont=cmuntb.ttf,ItalicFont=cmunit.ttf,BoldItalicFont=cmuntx.ttf]{cmuntt.ttf}\ttfamily epic}\setmainfont[Path=/usr/share/fonts/truetype/cmu/,UprightFont=cmunrm.ttf,BoldFont=cmunbx.ttf,ItalicFont=cmunti.ttf,BoldItalicFont=cmunbi.ttf]{cmunrm.ttf}\setmonofont[Path=/usr/share/fonts/truetype/cmu/,UprightFont=cmuntt.ttf,BoldFont=cmuntb.ttf,ItalicFont=cmunit.ttf,BoldItalicFont=cmuntx.ttf]{cmunrm.ttf}, {\ttfamily \setmainfont[Path=/usr/share/fonts/truetype/cmu/,UprightFont=cmunrm.ttf,BoldFont=cmunbx.ttf,ItalicFont=cmunti.ttf,BoldItalicFont=cmunbi.ttf]{cmuntt.ttf}\setmonofont[Path=/usr/share/fonts/truetype/cmu/,UprightFont=cmuntt.ttf,BoldFont=cmuntb.ttf,ItalicFont=cmunit.ttf,BoldItalicFont=cmuntx.ttf]{cmuntt.ttf}\ttfamily pstricks}{$\text{ }$}\setmainfont[Path=/usr/share/fonts/truetype/cmu/,UprightFont=cmunrm.ttf,BoldFont=cmunbx.ttf,ItalicFont=cmunti.ttf,BoldItalicFont=cmunbi.ttf]{cmunrm.ttf}\setmonofont[Path=/usr/share/fonts/truetype/cmu/,UprightFont=cmuntt.ttf,BoldFont=cmuntb.ttf,ItalicFont=cmunit.ttf,BoldItalicFont=cmuntx.ttf]{cmunrm.ttf} or {\ttfamily \setmainfont[Path=/usr/share/fonts/truetype/cmu/,UprightFont=cmunrm.ttf,BoldFont=cmunbx.ttf,ItalicFont=cmunti.ttf,BoldItalicFont=cmunbi.ttf]{cmuntt.ttf}\setmonofont[Path=/usr/share/fonts/truetype/cmu/,UprightFont=cmuntt.ttf,BoldFont=cmuntb.ttf,ItalicFont=cmunit.ttf,BoldItalicFont=cmuntx.ttf]{cmuntt.ttf}\ttfamily metapost}\setmainfont[Path=/usr/share/fonts/truetype/cmu/,UprightFont=cmunrm.ttf,BoldFont=cmunbx.ttf,ItalicFont=cmunti.ttf,BoldItalicFont=cmunbi.ttf]{cmunrm.ttf}\setmonofont[Path=/usr/share/fonts/truetype/cmu/,UprightFont=cmuntt.ttf,BoldFont=cmuntb.ttf,ItalicFont=cmunit.ttf,BoldItalicFont=cmuntx.ttf]{cmunrm.ttf}) focus on the {\itshape \setmainfont[Path=/usr/share/fonts/truetype/cmu/,UprightFont=cmunrm.ttf,BoldFont=cmunbx.ttf,ItalicFont=cmunti.ttf,BoldItalicFont=cmunbi.ttf]{cmunti.ttf}\setmonofont[Path=/usr/share/fonts/truetype/cmu/,UprightFont=cmuntt.ttf,BoldFont=cmuntb.ttf,ItalicFont=cmunit.ttf,BoldItalicFont=cmuntx.ttf]{cmunti.ttf}\itshape how}{$\text{ }$}\setmainfont[Path=/usr/share/fonts/truetype/cmu/,UprightFont=cmunrm.ttf,BoldFont=cmunbx.ttf,ItalicFont=cmunti.ttf,BoldItalicFont=cmunbi.ttf]{cmunrm.ttf}\setmonofont[Path=/usr/share/fonts/truetype/cmu/,UprightFont=cmuntt.ttf,BoldFont=cmuntb.ttf,ItalicFont=cmunit.ttf,BoldItalicFont=cmuntx.ttf]{cmunrm.ttf} to draw, TikZ focuses more on the {\itshape \setmainfont[Path=/usr/share/fonts/truetype/cmu/,UprightFont=cmunrm.ttf,BoldFont=cmunbx.ttf,ItalicFont=cmunti.ttf,BoldItalicFont=cmunbi.ttf]{cmunti.ttf}\setmonofont[Path=/usr/share/fonts/truetype/cmu/,UprightFont=cmuntt.ttf,BoldFont=cmuntb.ttf,ItalicFont=cmunit.ttf,BoldItalicFont=cmuntx.ttf]{cmunti.ttf}\itshape what}{$\text{ }$}\setmainfont[Path=/usr/share/fonts/truetype/cmu/,UprightFont=cmunrm.ttf,BoldFont=cmunbx.ttf,ItalicFont=cmunti.ttf,BoldItalicFont=cmunbi.ttf]{cmunrm.ttf}\setmonofont[Path=/usr/share/fonts/truetype/cmu/,UprightFont=cmuntt.ttf,BoldFont=cmuntb.ttf,ItalicFont=cmunit.ttf,BoldItalicFont=cmuntx.ttf]{cmunrm.ttf} to draw. One could say that TikZ is to drawing in LaTeX as LaTeX is to digital typesetting. It\textquotesingle{}s recommended to use it if your LaTeX distribution includes it.

Other packages building on top of Ti{\itshape \setmainfont[Path=/usr/share/fonts/truetype/cmu/,UprightFont=cmunrm.ttf,BoldFont=cmunbx.ttf,ItalicFont=cmunti.ttf,BoldItalicFont=cmunbi.ttf]{cmunti.ttf}\setmonofont[Path=/usr/share/fonts/truetype/cmu/,UprightFont=cmuntt.ttf,BoldFont=cmuntb.ttf,ItalicFont=cmunit.ttf,BoldItalicFont=cmuntx.ttf]{cmunti.ttf}\itshape k}\setmainfont[Path=/usr/share/fonts/truetype/cmu/,UprightFont=cmunrm.ttf,BoldFont=cmunbx.ttf,ItalicFont=cmunti.ttf,BoldItalicFont=cmunbi.ttf]{cmunrm.ttf}\setmonofont[Path=/usr/share/fonts/truetype/cmu/,UprightFont=cmuntt.ttf,BoldFont=cmuntb.ttf,ItalicFont=cmunit.ttf,BoldItalicFont=cmuntx.ttf]{cmunrm.ttf}Z (e.g., for drawing electrical circuits) can be found here: \myplainurl{http://ftp.dante.de/tex-archive/help/Catalogue/bytopic.html\#pgftikzsection}

In the following some basics of Ti{\itshape \setmainfont[Path=/usr/share/fonts/truetype/cmu/,UprightFont=cmunrm.ttf,BoldFont=cmunbx.ttf,ItalicFont=cmunti.ttf,BoldItalicFont=cmunbi.ttf]{cmunti.ttf}\setmonofont[Path=/usr/share/fonts/truetype/cmu/,UprightFont=cmuntt.ttf,BoldFont=cmuntb.ttf,ItalicFont=cmunit.ttf,BoldItalicFont=cmuntx.ttf]{cmunti.ttf}\itshape k}\setmainfont[Path=/usr/share/fonts/truetype/cmu/,UprightFont=cmunrm.ttf,BoldFont=cmunbx.ttf,ItalicFont=cmunti.ttf,BoldItalicFont=cmunbi.ttf]{cmunrm.ttf}\setmonofont[Path=/usr/share/fonts/truetype/cmu/,UprightFont=cmuntt.ttf,BoldFont=cmuntb.ttf,ItalicFont=cmunit.ttf,BoldItalicFont=cmuntx.ttf]{cmunrm.ttf}Z are presented.
\section{Loading Package, Libraries -{} tikzpicture environment}
\label{794}
Using Ti{\itshape \setmainfont[Path=/usr/share/fonts/truetype/cmu/,UprightFont=cmunrm.ttf,BoldFont=cmunbx.ttf,ItalicFont=cmunti.ttf,BoldItalicFont=cmunbi.ttf]{cmunti.ttf}\setmonofont[Path=/usr/share/fonts/truetype/cmu/,UprightFont=cmuntt.ttf,BoldFont=cmuntb.ttf,ItalicFont=cmunit.ttf,BoldItalicFont=cmuntx.ttf]{cmunti.ttf}\itshape k}\setmainfont[Path=/usr/share/fonts/truetype/cmu/,UprightFont=cmunrm.ttf,BoldFont=cmunbx.ttf,ItalicFont=cmunti.ttf,BoldItalicFont=cmunbi.ttf]{cmunrm.ttf}\setmonofont[Path=/usr/share/fonts/truetype/cmu/,UprightFont=cmuntt.ttf,BoldFont=cmuntb.ttf,ItalicFont=cmunit.ttf,BoldItalicFont=cmuntx.ttf]{cmunrm.ttf}Z in a LaTeX document requires loading the tikz package:

\begin{Shaded}
\begin{Highlighting}[]

\NormalTok{\textbackslash{}usepackage\{tikz\}}\newline
\end{Highlighting}
\end{Shaded}

somewhere in the preamble. This will automatically load the pgf package. To load further libraries use

\begin{Shaded}
\begin{Highlighting}[]

\NormalTok{\textbackslash{}usetikzlibrary\{\setmainfont[Path=/usr/share/fonts/truetype/freefont/,UprightFont=FreeSerif.ttf,BoldFont=FreeSerifBold.ttf,ItalicFont=FreeSerifItalic.ttf,BoldItalicFont=FreeSerifBoldItalic.ttf]{FreeSerif.ttf}\setmonofont[Path=/usr/share/fonts/truetype/freefont/,UprightFont=FreeMono.ttf,BoldFont=FreeMonoBold.ttf,ItalicFont=FreeMonoOblique.ttf,BoldItalicFont=FreeMonoBoldOblique.ttf]{FreeSerif.ttf}⟨\setmainfont[Path=/usr/share/fonts/truetype/cmu/,UprightFont=cmunrm.ttf,BoldFont=cmunbx.ttf,ItalicFont=cmunti.ttf,BoldItalicFont=cmunbi.ttf]{cmunrm.ttf}\setmonofont[Path=/usr/share/fonts/truetype/cmu/,UprightFont=cmuntt.ttf,BoldFont=cmuntb.ttf,ItalicFont=cmunit.ttf,BoldItalicFont=cmuntx.ttf]{cmunrm.ttf}list\ensuremath{\text{ }}of\ensuremath{\text{ }}libraries\ensuremath{\text{ }}separated\ensuremath{\text{ }}by\ensuremath{\text{ }}commas\setmainfont[Path=/usr/share/fonts/truetype/freefont/,UprightFont=FreeSerif.ttf,BoldFont=FreeSerifBold.ttf,ItalicFont=FreeSerifItalic.ttf,BoldItalicFont=FreeSerifBoldItalic.ttf]{FreeSerif.ttf}\setmonofont[Path=/usr/share/fonts/truetype/freefont/,UprightFont=FreeMono.ttf,BoldFont=FreeMonoBold.ttf,ItalicFont=FreeMonoOblique.ttf,BoldItalicFont=FreeMonoBoldOblique.ttf]{FreeSerif.ttf}⟩\setmainfont[Path=/usr/share/fonts/truetype/cmu/,UprightFont=cmunrm.ttf,BoldFont=cmunbx.ttf,ItalicFont=cmunti.ttf,BoldItalicFont=cmunbi.ttf]{cmunrm.ttf}\setmonofont[Path=/usr/share/fonts/truetype/cmu/,UprightFont=cmuntt.ttf,BoldFont=cmuntb.ttf,ItalicFont=cmunit.ttf,BoldItalicFont=cmuntx.ttf]{cmunrm.ttf}\}}\newline
\end{Highlighting}
\end{Shaded}

Examples for libraries are \symbol{34}{\ttfamily \setmainfont[Path=/usr/share/fonts/truetype/cmu/,UprightFont=cmunrm.ttf,BoldFont=cmunbx.ttf,ItalicFont=cmunti.ttf,BoldItalicFont=cmunbi.ttf]{cmuntt.ttf}\setmonofont[Path=/usr/share/fonts/truetype/cmu/,UprightFont=cmuntt.ttf,BoldFont=cmuntb.ttf,ItalicFont=cmunit.ttf,BoldItalicFont=cmuntx.ttf]{cmuntt.ttf}\ttfamily arrows}\setmainfont[Path=/usr/share/fonts/truetype/cmu/,UprightFont=cmunrm.ttf,BoldFont=cmunbx.ttf,ItalicFont=cmunti.ttf,BoldItalicFont=cmunbi.ttf]{cmunrm.ttf}\setmonofont[Path=/usr/share/fonts/truetype/cmu/,UprightFont=cmuntt.ttf,BoldFont=cmuntb.ttf,ItalicFont=cmunit.ttf,BoldItalicFont=cmuntx.ttf]{cmunrm.ttf}\symbol{34}, \symbol{34}{\ttfamily \setmainfont[Path=/usr/share/fonts/truetype/cmu/,UprightFont=cmunrm.ttf,BoldFont=cmunbx.ttf,ItalicFont=cmunti.ttf,BoldItalicFont=cmunbi.ttf]{cmuntt.ttf}\setmonofont[Path=/usr/share/fonts/truetype/cmu/,UprightFont=cmuntt.ttf,BoldFont=cmuntb.ttf,ItalicFont=cmunit.ttf,BoldItalicFont=cmuntx.ttf]{cmuntt.ttf}\ttfamily automata}\setmainfont[Path=/usr/share/fonts/truetype/cmu/,UprightFont=cmunrm.ttf,BoldFont=cmunbx.ttf,ItalicFont=cmunti.ttf,BoldItalicFont=cmunbi.ttf]{cmunrm.ttf}\setmonofont[Path=/usr/share/fonts/truetype/cmu/,UprightFont=cmuntt.ttf,BoldFont=cmuntb.ttf,ItalicFont=cmunit.ttf,BoldItalicFont=cmuntx.ttf]{cmunrm.ttf}\symbol{34}, \symbol{34}{\ttfamily \setmainfont[Path=/usr/share/fonts/truetype/cmu/,UprightFont=cmunrm.ttf,BoldFont=cmunbx.ttf,ItalicFont=cmunti.ttf,BoldItalicFont=cmunbi.ttf]{cmuntt.ttf}\setmonofont[Path=/usr/share/fonts/truetype/cmu/,UprightFont=cmuntt.ttf,BoldFont=cmuntb.ttf,ItalicFont=cmunit.ttf,BoldItalicFont=cmuntx.ttf]{cmuntt.ttf}\ttfamily backgrounds}\setmainfont[Path=/usr/share/fonts/truetype/cmu/,UprightFont=cmunrm.ttf,BoldFont=cmunbx.ttf,ItalicFont=cmunti.ttf,BoldItalicFont=cmunbi.ttf]{cmunrm.ttf}\setmonofont[Path=/usr/share/fonts/truetype/cmu/,UprightFont=cmuntt.ttf,BoldFont=cmuntb.ttf,ItalicFont=cmunit.ttf,BoldItalicFont=cmuntx.ttf]{cmunrm.ttf}\symbol{34}, \symbol{34}{\ttfamily \setmainfont[Path=/usr/share/fonts/truetype/cmu/,UprightFont=cmunrm.ttf,BoldFont=cmunbx.ttf,ItalicFont=cmunti.ttf,BoldItalicFont=cmunbi.ttf]{cmuntt.ttf}\setmonofont[Path=/usr/share/fonts/truetype/cmu/,UprightFont=cmuntt.ttf,BoldFont=cmuntb.ttf,ItalicFont=cmunit.ttf,BoldItalicFont=cmuntx.ttf]{cmuntt.ttf}\ttfamily calendar}\setmainfont[Path=/usr/share/fonts/truetype/cmu/,UprightFont=cmunrm.ttf,BoldFont=cmunbx.ttf,ItalicFont=cmunti.ttf,BoldItalicFont=cmunbi.ttf]{cmunrm.ttf}\setmonofont[Path=/usr/share/fonts/truetype/cmu/,UprightFont=cmuntt.ttf,BoldFont=cmuntb.ttf,ItalicFont=cmunit.ttf,BoldItalicFont=cmuntx.ttf]{cmunrm.ttf}\symbol{34}, \symbol{34}{\ttfamily \setmainfont[Path=/usr/share/fonts/truetype/cmu/,UprightFont=cmunrm.ttf,BoldFont=cmunbx.ttf,ItalicFont=cmunti.ttf,BoldItalicFont=cmunbi.ttf]{cmuntt.ttf}\setmonofont[Path=/usr/share/fonts/truetype/cmu/,UprightFont=cmuntt.ttf,BoldFont=cmuntb.ttf,ItalicFont=cmunit.ttf,BoldItalicFont=cmuntx.ttf]{cmuntt.ttf}\ttfamily chains}\setmainfont[Path=/usr/share/fonts/truetype/cmu/,UprightFont=cmunrm.ttf,BoldFont=cmunbx.ttf,ItalicFont=cmunti.ttf,BoldItalicFont=cmunbi.ttf]{cmunrm.ttf}\setmonofont[Path=/usr/share/fonts/truetype/cmu/,UprightFont=cmuntt.ttf,BoldFont=cmuntb.ttf,ItalicFont=cmunit.ttf,BoldItalicFont=cmuntx.ttf]{cmunrm.ttf}\symbol{34}, \symbol{34}{\ttfamily \setmainfont[Path=/usr/share/fonts/truetype/cmu/,UprightFont=cmunrm.ttf,BoldFont=cmunbx.ttf,ItalicFont=cmunti.ttf,BoldItalicFont=cmunbi.ttf]{cmuntt.ttf}\setmonofont[Path=/usr/share/fonts/truetype/cmu/,UprightFont=cmuntt.ttf,BoldFont=cmuntb.ttf,ItalicFont=cmunit.ttf,BoldItalicFont=cmuntx.ttf]{cmuntt.ttf}\ttfamily matrix}\setmainfont[Path=/usr/share/fonts/truetype/cmu/,UprightFont=cmunrm.ttf,BoldFont=cmunbx.ttf,ItalicFont=cmunti.ttf,BoldItalicFont=cmunbi.ttf]{cmunrm.ttf}\setmonofont[Path=/usr/share/fonts/truetype/cmu/,UprightFont=cmuntt.ttf,BoldFont=cmuntb.ttf,ItalicFont=cmunit.ttf,BoldItalicFont=cmuntx.ttf]{cmunrm.ttf}\symbol{34}, \symbol{34}{\ttfamily \setmainfont[Path=/usr/share/fonts/truetype/cmu/,UprightFont=cmunrm.ttf,BoldFont=cmunbx.ttf,ItalicFont=cmunti.ttf,BoldItalicFont=cmunbi.ttf]{cmuntt.ttf}\setmonofont[Path=/usr/share/fonts/truetype/cmu/,UprightFont=cmuntt.ttf,BoldFont=cmuntb.ttf,ItalicFont=cmunit.ttf,BoldItalicFont=cmuntx.ttf]{cmuntt.ttf}\ttfamily mindmap}\setmainfont[Path=/usr/share/fonts/truetype/cmu/,UprightFont=cmunrm.ttf,BoldFont=cmunbx.ttf,ItalicFont=cmunti.ttf,BoldItalicFont=cmunbi.ttf]{cmunrm.ttf}\setmonofont[Path=/usr/share/fonts/truetype/cmu/,UprightFont=cmuntt.ttf,BoldFont=cmuntb.ttf,ItalicFont=cmunit.ttf,BoldItalicFont=cmuntx.ttf]{cmunrm.ttf}\symbol{34}, \symbol{34}{\ttfamily \setmainfont[Path=/usr/share/fonts/truetype/cmu/,UprightFont=cmunrm.ttf,BoldFont=cmunbx.ttf,ItalicFont=cmunti.ttf,BoldItalicFont=cmunbi.ttf]{cmuntt.ttf}\setmonofont[Path=/usr/share/fonts/truetype/cmu/,UprightFont=cmuntt.ttf,BoldFont=cmuntb.ttf,ItalicFont=cmunit.ttf,BoldItalicFont=cmuntx.ttf]{cmuntt.ttf}\ttfamily patterns}\setmainfont[Path=/usr/share/fonts/truetype/cmu/,UprightFont=cmunrm.ttf,BoldFont=cmunbx.ttf,ItalicFont=cmunti.ttf,BoldItalicFont=cmunbi.ttf]{cmunrm.ttf}\setmonofont[Path=/usr/share/fonts/truetype/cmu/,UprightFont=cmuntt.ttf,BoldFont=cmuntb.ttf,ItalicFont=cmunit.ttf,BoldItalicFont=cmuntx.ttf]{cmunrm.ttf}\symbol{34}, \symbol{34}{\ttfamily \setmainfont[Path=/usr/share/fonts/truetype/cmu/,UprightFont=cmunrm.ttf,BoldFont=cmunbx.ttf,ItalicFont=cmunti.ttf,BoldItalicFont=cmunbi.ttf]{cmuntt.ttf}\setmonofont[Path=/usr/share/fonts/truetype/cmu/,UprightFont=cmuntt.ttf,BoldFont=cmuntb.ttf,ItalicFont=cmunit.ttf,BoldItalicFont=cmuntx.ttf]{cmuntt.ttf}\ttfamily petri}\setmainfont[Path=/usr/share/fonts/truetype/cmu/,UprightFont=cmunrm.ttf,BoldFont=cmunbx.ttf,ItalicFont=cmunti.ttf,BoldItalicFont=cmunbi.ttf]{cmunrm.ttf}\setmonofont[Path=/usr/share/fonts/truetype/cmu/,UprightFont=cmuntt.ttf,BoldFont=cmuntb.ttf,ItalicFont=cmunit.ttf,BoldItalicFont=cmuntx.ttf]{cmunrm.ttf}\symbol{34}, \symbol{34}{\ttfamily \setmainfont[Path=/usr/share/fonts/truetype/cmu/,UprightFont=cmunrm.ttf,BoldFont=cmunbx.ttf,ItalicFont=cmunti.ttf,BoldItalicFont=cmunbi.ttf]{cmuntt.ttf}\setmonofont[Path=/usr/share/fonts/truetype/cmu/,UprightFont=cmuntt.ttf,BoldFont=cmuntb.ttf,ItalicFont=cmunit.ttf,BoldItalicFont=cmuntx.ttf]{cmuntt.ttf}\ttfamily shadows}\setmainfont[Path=/usr/share/fonts/truetype/cmu/,UprightFont=cmunrm.ttf,BoldFont=cmunbx.ttf,ItalicFont=cmunti.ttf,BoldItalicFont=cmunbi.ttf]{cmunrm.ttf}\setmonofont[Path=/usr/share/fonts/truetype/cmu/,UprightFont=cmuntt.ttf,BoldFont=cmuntb.ttf,ItalicFont=cmunit.ttf,BoldItalicFont=cmuntx.ttf]{cmunrm.ttf}\symbol{34}, \symbol{34}{\ttfamily \setmainfont[Path=/usr/share/fonts/truetype/cmu/,UprightFont=cmunrm.ttf,BoldFont=cmunbx.ttf,ItalicFont=cmunti.ttf,BoldItalicFont=cmunbi.ttf]{cmuntt.ttf}\setmonofont[Path=/usr/share/fonts/truetype/cmu/,UprightFont=cmuntt.ttf,BoldFont=cmuntb.ttf,ItalicFont=cmunit.ttf,BoldItalicFont=cmuntx.ttf]{cmuntt.ttf}\ttfamily shapes.geometric}\setmainfont[Path=/usr/share/fonts/truetype/cmu/,UprightFont=cmunrm.ttf,BoldFont=cmunbx.ttf,ItalicFont=cmunti.ttf,BoldItalicFont=cmunbi.ttf]{cmunrm.ttf}\setmonofont[Path=/usr/share/fonts/truetype/cmu/,UprightFont=cmuntt.ttf,BoldFont=cmuntb.ttf,ItalicFont=cmunit.ttf,BoldItalicFont=cmuntx.ttf]{cmunrm.ttf}\symbol{34}, \symbol{34}{\ttfamily \setmainfont[Path=/usr/share/fonts/truetype/cmu/,UprightFont=cmunrm.ttf,BoldFont=cmunbx.ttf,ItalicFont=cmunti.ttf,BoldItalicFont=cmunbi.ttf]{cmuntt.ttf}\setmonofont[Path=/usr/share/fonts/truetype/cmu/,UprightFont=cmuntt.ttf,BoldFont=cmuntb.ttf,ItalicFont=cmunit.ttf,BoldItalicFont=cmuntx.ttf]{cmuntt.ttf}\ttfamily shapes.misc}\setmainfont[Path=/usr/share/fonts/truetype/cmu/,UprightFont=cmunrm.ttf,BoldFont=cmunbx.ttf,ItalicFont=cmunti.ttf,BoldItalicFont=cmunbi.ttf]{cmunrm.ttf}\setmonofont[Path=/usr/share/fonts/truetype/cmu/,UprightFont=cmuntt.ttf,BoldFont=cmuntb.ttf,ItalicFont=cmunit.ttf,BoldItalicFont=cmuntx.ttf]{cmunrm.ttf}\symbol{34}, \symbol{34}{\ttfamily \setmainfont[Path=/usr/share/fonts/truetype/cmu/,UprightFont=cmunrm.ttf,BoldFont=cmunbx.ttf,ItalicFont=cmunti.ttf,BoldItalicFont=cmunbi.ttf]{cmuntt.ttf}\setmonofont[Path=/usr/share/fonts/truetype/cmu/,UprightFont=cmuntt.ttf,BoldFont=cmuntb.ttf,ItalicFont=cmunit.ttf,BoldItalicFont=cmuntx.ttf]{cmuntt.ttf}\ttfamily spy}\setmainfont[Path=/usr/share/fonts/truetype/cmu/,UprightFont=cmunrm.ttf,BoldFont=cmunbx.ttf,ItalicFont=cmunti.ttf,BoldItalicFont=cmunbi.ttf]{cmunrm.ttf}\setmonofont[Path=/usr/share/fonts/truetype/cmu/,UprightFont=cmuntt.ttf,BoldFont=cmuntb.ttf,ItalicFont=cmunit.ttf,BoldItalicFont=cmuntx.ttf]{cmunrm.ttf}\symbol{34}, \symbol{34}{\ttfamily \setmainfont[Path=/usr/share/fonts/truetype/cmu/,UprightFont=cmunrm.ttf,BoldFont=cmunbx.ttf,ItalicFont=cmunti.ttf,BoldItalicFont=cmunbi.ttf]{cmuntt.ttf}\setmonofont[Path=/usr/share/fonts/truetype/cmu/,UprightFont=cmuntt.ttf,BoldFont=cmuntb.ttf,ItalicFont=cmunit.ttf,BoldItalicFont=cmuntx.ttf]{cmuntt.ttf}\ttfamily trees}\setmainfont[Path=/usr/share/fonts/truetype/cmu/,UprightFont=cmunrm.ttf,BoldFont=cmunbx.ttf,ItalicFont=cmunti.ttf,BoldItalicFont=cmunbi.ttf]{cmunrm.ttf}\setmonofont[Path=/usr/share/fonts/truetype/cmu/,UprightFont=cmuntt.ttf,BoldFont=cmuntb.ttf,ItalicFont=cmunit.ttf,BoldItalicFont=cmuntx.ttf]{cmunrm.ttf}\symbol{34}.

Drawing commands have to be enclosed in an tikzpicture environment 

\begin{Shaded}
\begin{Highlighting}[]

\NormalTok{\textbackslash{}begin\{tikzpicture\}[\setmainfont[Path=/usr/share/fonts/truetype/freefont/,UprightFont=FreeSerif.ttf,BoldFont=FreeSerifBold.ttf,ItalicFont=FreeSerifItalic.ttf,BoldItalicFont=FreeSerifBoldItalic.ttf]{FreeSerif.ttf}\setmonofont[Path=/usr/share/fonts/truetype/freefont/,UprightFont=FreeMono.ttf,BoldFont=FreeMonoBold.ttf,ItalicFont=FreeMonoOblique.ttf,BoldItalicFont=FreeMonoBoldOblique.ttf]{FreeSerif.ttf}⟨\setmainfont[Path=/usr/share/fonts/truetype/cmu/,UprightFont=cmunrm.ttf,BoldFont=cmunbx.ttf,ItalicFont=cmunti.ttf,BoldItalicFont=cmunbi.ttf]{cmunrm.ttf}\setmonofont[Path=/usr/share/fonts/truetype/cmu/,UprightFont=cmuntt.ttf,BoldFont=cmuntb.ttf,ItalicFont=cmunit.ttf,BoldItalicFont=cmuntx.ttf]{cmunrm.ttf}options\setmainfont[Path=/usr/share/fonts/truetype/freefont/,UprightFont=FreeSerif.ttf,BoldFont=FreeSerifBold.ttf,ItalicFont=FreeSerifItalic.ttf,BoldItalicFont=FreeSerifBoldItalic.ttf]{FreeSerif.ttf}\setmonofont[Path=/usr/share/fonts/truetype/freefont/,UprightFont=FreeMono.ttf,BoldFont=FreeMonoBold.ttf,ItalicFont=FreeMonoOblique.ttf,BoldItalicFont=FreeMonoBoldOblique.ttf]{FreeSerif.ttf}⟩\setmainfont[Path=/usr/share/fonts/truetype/cmu/,UprightFont=cmunrm.ttf,BoldFont=cmunbx.ttf,ItalicFont=cmunti.ttf,BoldItalicFont=cmunbi.ttf]{cmunrm.ttf}\setmonofont[Path=/usr/share/fonts/truetype/cmu/,UprightFont=cmuntt.ttf,BoldFont=cmuntb.ttf,ItalicFont=cmunit.ttf,BoldItalicFont=cmuntx.ttf]{cmunrm.ttf}]\ensuremath{\text{ }}}\newline
\ensuremath{\text{ }}\ensuremath{\text{ }}\NormalTok{\setmainfont[Path=/usr/share/fonts/truetype/freefont/,UprightFont=FreeSerif.ttf,BoldFont=FreeSerifBold.ttf,ItalicFont=FreeSerifItalic.ttf,BoldItalicFont=FreeSerifBoldItalic.ttf]{FreeSerif.ttf}\setmonofont[Path=/usr/share/fonts/truetype/freefont/,UprightFont=FreeMono.ttf,BoldFont=FreeMonoBold.ttf,ItalicFont=FreeMonoOblique.ttf,BoldItalicFont=FreeMonoBoldOblique.ttf]{FreeSerif.ttf}⟨\setmainfont[Path=/usr/share/fonts/truetype/cmu/,UprightFont=cmunrm.ttf,BoldFont=cmunbx.ttf,ItalicFont=cmunti.ttf,BoldItalicFont=cmunbi.ttf]{cmunrm.ttf}\setmonofont[Path=/usr/share/fonts/truetype/cmu/,UprightFont=cmuntt.ttf,BoldFont=cmuntb.ttf,ItalicFont=cmunit.ttf,BoldItalicFont=cmuntx.ttf]{cmunrm.ttf}tikz\ensuremath{\text{ }}commands\setmainfont[Path=/usr/share/fonts/truetype/freefont/,UprightFont=FreeSerif.ttf,BoldFont=FreeSerifBold.ttf,ItalicFont=FreeSerifItalic.ttf,BoldItalicFont=FreeSerifBoldItalic.ttf]{FreeSerif.ttf}\setmonofont[Path=/usr/share/fonts/truetype/freefont/,UprightFont=FreeMono.ttf,BoldFont=FreeMonoBold.ttf,ItalicFont=FreeMonoOblique.ttf,BoldItalicFont=FreeMonoBoldOblique.ttf]{FreeSerif.ttf}⟩\setmainfont[Path=/usr/share/fonts/truetype/cmu/,UprightFont=cmunrm.ttf,BoldFont=cmunbx.ttf,ItalicFont=cmunti.ttf,BoldItalicFont=cmunbi.ttf]{cmunrm.ttf}\setmonofont[Path=/usr/share/fonts/truetype/cmu/,UprightFont=cmuntt.ttf,BoldFont=cmuntb.ttf,ItalicFont=cmunit.ttf,BoldItalicFont=cmuntx.ttf]{cmunrm.ttf}}\newline
\NormalTok{\textbackslash{}end\{tikzpicture\}}\newline
\end{Highlighting}
\end{Shaded}

or alternatively 

\begin{Shaded}
\begin{Highlighting}[]

\NormalTok{\textbackslash{}tikz[\setmainfont[Path=/usr/share/fonts/truetype/freefont/,UprightFont=FreeSerif.ttf,BoldFont=FreeSerifBold.ttf,ItalicFont=FreeSerifItalic.ttf,BoldItalicFont=FreeSerifBoldItalic.ttf]{FreeSerif.ttf}\setmonofont[Path=/usr/share/fonts/truetype/freefont/,UprightFont=FreeMono.ttf,BoldFont=FreeMonoBold.ttf,ItalicFont=FreeMonoOblique.ttf,BoldItalicFont=FreeMonoBoldOblique.ttf]{FreeSerif.ttf}⟨\setmainfont[Path=/usr/share/fonts/truetype/cmu/,UprightFont=cmunrm.ttf,BoldFont=cmunbx.ttf,ItalicFont=cmunti.ttf,BoldItalicFont=cmunbi.ttf]{cmunrm.ttf}\setmonofont[Path=/usr/share/fonts/truetype/cmu/,UprightFont=cmuntt.ttf,BoldFont=cmuntb.ttf,ItalicFont=cmunit.ttf,BoldItalicFont=cmuntx.ttf]{cmunrm.ttf}options\setmainfont[Path=/usr/share/fonts/truetype/freefont/,UprightFont=FreeSerif.ttf,BoldFont=FreeSerifBold.ttf,ItalicFont=FreeSerifItalic.ttf,BoldItalicFont=FreeSerifBoldItalic.ttf]{FreeSerif.ttf}\setmonofont[Path=/usr/share/fonts/truetype/freefont/,UprightFont=FreeMono.ttf,BoldFont=FreeMonoBold.ttf,ItalicFont=FreeMonoOblique.ttf,BoldItalicFont=FreeMonoBoldOblique.ttf]{FreeSerif.ttf}⟩\setmainfont[Path=/usr/share/fonts/truetype/cmu/,UprightFont=cmunrm.ttf,BoldFont=cmunbx.ttf,ItalicFont=cmunti.ttf,BoldItalicFont=cmunbi.ttf]{cmunrm.ttf}\setmonofont[Path=/usr/share/fonts/truetype/cmu/,UprightFont=cmuntt.ttf,BoldFont=cmuntb.ttf,ItalicFont=cmunit.ttf,BoldItalicFont=cmuntx.ttf]{cmunrm.ttf}]\{\setmainfont[Path=/usr/share/fonts/truetype/freefont/,UprightFont=FreeSerif.ttf,BoldFont=FreeSerifBold.ttf,ItalicFont=FreeSerifItalic.ttf,BoldItalicFont=FreeSerifBoldItalic.ttf]{FreeSerif.ttf}\setmonofont[Path=/usr/share/fonts/truetype/freefont/,UprightFont=FreeMono.ttf,BoldFont=FreeMonoBold.ttf,ItalicFont=FreeMonoOblique.ttf,BoldItalicFont=FreeMonoBoldOblique.ttf]{FreeSerif.ttf}⟨\setmainfont[Path=/usr/share/fonts/truetype/cmu/,UprightFont=cmunrm.ttf,BoldFont=cmunbx.ttf,ItalicFont=cmunti.ttf,BoldItalicFont=cmunbi.ttf]{cmunrm.ttf}\setmonofont[Path=/usr/share/fonts/truetype/cmu/,UprightFont=cmuntt.ttf,BoldFont=cmuntb.ttf,ItalicFont=cmunit.ttf,BoldItalicFont=cmuntx.ttf]{cmunrm.ttf}tikz\ensuremath{\text{ }}commands\setmainfont[Path=/usr/share/fonts/truetype/freefont/,UprightFont=FreeSerif.ttf,BoldFont=FreeSerifBold.ttf,ItalicFont=FreeSerifItalic.ttf,BoldItalicFont=FreeSerifBoldItalic.ttf]{FreeSerif.ttf}\setmonofont[Path=/usr/share/fonts/truetype/freefont/,UprightFont=FreeMono.ttf,BoldFont=FreeMonoBold.ttf,ItalicFont=FreeMonoOblique.ttf,BoldItalicFont=FreeMonoBoldOblique.ttf]{FreeSerif.ttf}⟩\setmainfont[Path=/usr/share/fonts/truetype/cmu/,UprightFont=cmunrm.ttf,BoldFont=cmunbx.ttf,ItalicFont=cmunti.ttf,BoldItalicFont=cmunbi.ttf]{cmunrm.ttf}\setmonofont[Path=/usr/share/fonts/truetype/cmu/,UprightFont=cmuntt.ttf,BoldFont=cmuntb.ttf,ItalicFont=cmunit.ttf,BoldItalicFont=cmuntx.ttf]{cmunrm.ttf}\}}\newline
\end{Highlighting}
\end{Shaded}

One possible option useful for inlined graphics is
\setmainfont[Path=/usr/share/fonts/truetype/cmu/,UprightFont=cmunrm.ttf,BoldFont=cmunbx.ttf,ItalicFont=cmunti.ttf,BoldItalicFont=cmunbi.ttf]{cmunrm.ttf}\setmonofont[Path=/usr/share/fonts/truetype/cmu/,UprightFont=cmuntt.ttf,BoldFont=cmuntb.ttf,ItalicFont=cmunit.ttf,BoldItalicFont=cmuntx.ttf]{cmunrm.ttf}
\begin{Shaded}
\begin{Highlighting}[]

\NormalTok{baseline=\setmainfont[Path=/usr/share/fonts/truetype/freefont/,UprightFont=FreeSerif.ttf,BoldFont=FreeSerifBold.ttf,ItalicFont=FreeSerifItalic.ttf,BoldItalicFont=FreeSerifBoldItalic.ttf]{FreeSerif.ttf}\setmonofont[Path=/usr/share/fonts/truetype/freefont/,UprightFont=FreeMono.ttf,BoldFont=FreeMonoBold.ttf,ItalicFont=FreeMonoOblique.ttf,BoldItalicFont=FreeMonoBoldOblique.ttf]{FreeSerif.ttf}⟨\setmainfont[Path=/usr/share/fonts/truetype/cmu/,UprightFont=cmunrm.ttf,BoldFont=cmunbx.ttf,ItalicFont=cmunti.ttf,BoldItalicFont=cmunbi.ttf]{cmunrm.ttf}\setmonofont[Path=/usr/share/fonts/truetype/cmu/,UprightFont=cmuntt.ttf,BoldFont=cmuntb.ttf,ItalicFont=cmunit.ttf,BoldItalicFont=cmuntx.ttf]{cmunrm.ttf}dimension\setmainfont[Path=/usr/share/fonts/truetype/freefont/,UprightFont=FreeSerif.ttf,BoldFont=FreeSerifBold.ttf,ItalicFont=FreeSerifItalic.ttf,BoldItalicFont=FreeSerifBoldItalic.ttf]{FreeSerif.ttf}\setmonofont[Path=/usr/share/fonts/truetype/freefont/,UprightFont=FreeMono.ttf,BoldFont=FreeMonoBold.ttf,ItalicFont=FreeMonoOblique.ttf,BoldItalicFont=FreeMonoBoldOblique.ttf]{FreeSerif.ttf}⟩\setmainfont[Path=/usr/share/fonts/truetype/cmu/,UprightFont=cmunrm.ttf,BoldFont=cmunbx.ttf,ItalicFont=cmunti.ttf,BoldItalicFont=cmunbi.ttf]{cmunrm.ttf}\setmonofont[Path=/usr/share/fonts/truetype/cmu/,UprightFont=cmuntt.ttf,BoldFont=cmuntb.ttf,ItalicFont=cmunit.ttf,BoldItalicFont=cmuntx.ttf]{cmunrm.ttf}}\newline
\end{Highlighting}
\end{Shaded}

Without that option the lower end of the picture is put on the baseline of the surrounding text. Using this option, you can specify that the picture should be raised or lowered such that the height \setmainfont[Path=/usr/share/fonts/truetype/freefont/,UprightFont=FreeSerif.ttf,BoldFont=FreeSerifBold.ttf,ItalicFont=FreeSerifItalic.ttf,BoldItalicFont=FreeSerifBoldItalic.ttf]{FreeSerif.ttf}\setmonofont[Path=/usr/share/fonts/truetype/freefont/,UprightFont=FreeMono.ttf,BoldFont=FreeMonoBold.ttf,ItalicFont=FreeMonoOblique.ttf,BoldItalicFont=FreeMonoBoldOblique.ttf]{FreeSerif.ttf}⟨\setmainfont[Path=/usr/share/fonts/truetype/cmu/,UprightFont=cmunrm.ttf,BoldFont=cmunbx.ttf,ItalicFont=cmunti.ttf,BoldItalicFont=cmunbi.ttf]{cmunrm.ttf}\setmonofont[Path=/usr/share/fonts/truetype/cmu/,UprightFont=cmuntt.ttf,BoldFont=cmuntb.ttf,ItalicFont=cmunit.ttf,BoldItalicFont=cmuntx.ttf]{cmunrm.ttf}dimension\setmainfont[Path=/usr/share/fonts/truetype/freefont/,UprightFont=FreeSerif.ttf,BoldFont=FreeSerifBold.ttf,ItalicFont=FreeSerifItalic.ttf,BoldItalicFont=FreeSerifBoldItalic.ttf]{FreeSerif.ttf}\setmonofont[Path=/usr/share/fonts/truetype/freefont/,UprightFont=FreeMono.ttf,BoldFont=FreeMonoBold.ttf,ItalicFont=FreeMonoOblique.ttf,BoldItalicFont=FreeMonoBoldOblique.ttf]{FreeSerif.ttf}⟩{$\text{ }$}\setmainfont[Path=/usr/share/fonts/truetype/cmu/,UprightFont=cmunrm.ttf,BoldFont=cmunbx.ttf,ItalicFont=cmunti.ttf,BoldItalicFont=cmunbi.ttf]{cmunrm.ttf}\setmonofont[Path=/usr/share/fonts/truetype/cmu/,UprightFont=cmuntt.ttf,BoldFont=cmuntb.ttf,ItalicFont=cmunit.ttf,BoldItalicFont=cmuntx.ttf]{cmunrm.ttf} is on the baseline.

Another option to scale the entire picture is
\setmainfont[Path=/usr/share/fonts/truetype/cmu/,UprightFont=cmunrm.ttf,BoldFont=cmunbx.ttf,ItalicFont=cmunti.ttf,BoldItalicFont=cmunbi.ttf]{cmunrm.ttf}\setmonofont[Path=/usr/share/fonts/truetype/cmu/,UprightFont=cmuntt.ttf,BoldFont=cmuntb.ttf,ItalicFont=cmunit.ttf,BoldItalicFont=cmuntx.ttf]{cmunrm.ttf}
\begin{Shaded}
\begin{Highlighting}[]

\NormalTok{scale=\setmainfont[Path=/usr/share/fonts/truetype/freefont/,UprightFont=FreeSerif.ttf,BoldFont=FreeSerifBold.ttf,ItalicFont=FreeSerifItalic.ttf,BoldItalicFont=FreeSerifBoldItalic.ttf]{FreeSerif.ttf}\setmonofont[Path=/usr/share/fonts/truetype/freefont/,UprightFont=FreeMono.ttf,BoldFont=FreeMonoBold.ttf,ItalicFont=FreeMonoOblique.ttf,BoldItalicFont=FreeMonoBoldOblique.ttf]{FreeSerif.ttf}⟨\setmainfont[Path=/usr/share/fonts/truetype/cmu/,UprightFont=cmunrm.ttf,BoldFont=cmunbx.ttf,ItalicFont=cmunti.ttf,BoldItalicFont=cmunbi.ttf]{cmunrm.ttf}\setmonofont[Path=/usr/share/fonts/truetype/cmu/,UprightFont=cmuntt.ttf,BoldFont=cmuntb.ttf,ItalicFont=cmunit.ttf,BoldItalicFont=cmuntx.ttf]{cmunrm.ttf}factor\setmainfont[Path=/usr/share/fonts/truetype/freefont/,UprightFont=FreeSerif.ttf,BoldFont=FreeSerifBold.ttf,ItalicFont=FreeSerifItalic.ttf,BoldItalicFont=FreeSerifBoldItalic.ttf]{FreeSerif.ttf}\setmonofont[Path=/usr/share/fonts/truetype/freefont/,UprightFont=FreeMono.ttf,BoldFont=FreeMonoBold.ttf,ItalicFont=FreeMonoOblique.ttf,BoldItalicFont=FreeMonoBoldOblique.ttf]{FreeSerif.ttf}⟩\setmainfont[Path=/usr/share/fonts/truetype/cmu/,UprightFont=cmunrm.ttf,BoldFont=cmunbx.ttf,ItalicFont=cmunti.ttf,BoldItalicFont=cmunbi.ttf]{cmunrm.ttf}\setmonofont[Path=/usr/share/fonts/truetype/cmu/,UprightFont=cmuntt.ttf,BoldFont=cmuntb.ttf,ItalicFont=cmunit.ttf,BoldItalicFont=cmuntx.ttf]{cmunrm.ttf}}\newline
\end{Highlighting}
\end{Shaded}

or different for height and width, e.g:

\begin{Shaded}
\begin{Highlighting}[]

\NormalTok{xscale=2.5,\ensuremath{\text{ }}yscale=0.5}\newline
\end{Highlighting}
\end{Shaded}

\section{Specifying Coordinates}
\label{795}
Coordinates are specified in round brackets in an arbitrary TEX dimension either using Cartesian coordinates (comma separated), e.g. 1cm in the x direction and 2pt in the y direction

\begin{Shaded}
\begin{Highlighting}[]

\NormalTok{(1cm,2pt)}\newline
\end{Highlighting}
\end{Shaded}

or using polar coordinates (colon separated), e.g. 1cm in 30 degree direction

\begin{Shaded}
\begin{Highlighting}[]

\NormalTok{(30:1cm)}\newline
\end{Highlighting}
\end{Shaded}

Without specifying a unit {\ttfamily \setmainfont[Path=/usr/share/fonts/truetype/cmu/,UprightFont=cmunrm.ttf,BoldFont=cmunbx.ttf,ItalicFont=cmunti.ttf,BoldItalicFont=cmunbi.ttf]{cmuntt.ttf}\setmonofont[Path=/usr/share/fonts/truetype/cmu/,UprightFont=cmuntt.ttf,BoldFont=cmuntb.ttf,ItalicFont=cmunit.ttf,BoldItalicFont=cmuntx.ttf]{cmuntt.ttf}\ttfamily (1,2)}\setmainfont[Path=/usr/share/fonts/truetype/cmu/,UprightFont=cmunrm.ttf,BoldFont=cmunbx.ttf,ItalicFont=cmunti.ttf,BoldItalicFont=cmunbi.ttf]{cmunrm.ttf}\setmonofont[Path=/usr/share/fonts/truetype/cmu/,UprightFont=cmuntt.ttf,BoldFont=cmuntb.ttf,ItalicFont=cmunit.ttf,BoldItalicFont=cmuntx.ttf]{cmunrm.ttf}, the standard one is cm {\ttfamily \setmainfont[Path=/usr/share/fonts/truetype/cmu/,UprightFont=cmunrm.ttf,BoldFont=cmunbx.ttf,ItalicFont=cmunti.ttf,BoldItalicFont=cmunbi.ttf]{cmuntt.ttf}\setmonofont[Path=/usr/share/fonts/truetype/cmu/,UprightFont=cmuntt.ttf,BoldFont=cmuntb.ttf,ItalicFont=cmunit.ttf,BoldItalicFont=cmuntx.ttf]{cmuntt.ttf}\ttfamily (1cm,2cm)}\setmainfont[Path=/usr/share/fonts/truetype/cmu/,UprightFont=cmunrm.ttf,BoldFont=cmunbx.ttf,ItalicFont=cmunti.ttf,BoldItalicFont=cmunbi.ttf]{cmunrm.ttf}\setmonofont[Path=/usr/share/fonts/truetype/cmu/,UprightFont=cmuntt.ttf,BoldFont=cmuntb.ttf,ItalicFont=cmunit.ttf,BoldItalicFont=cmuntx.ttf]{cmunrm.ttf}. 

Relative coordinates to the previous given point are given by adding one or two plus signs in front of the coordinate. With \symbol{34}{\ttfamily \setmainfont[Path=/usr/share/fonts/truetype/cmu/,UprightFont=cmunrm.ttf,BoldFont=cmunbx.ttf,ItalicFont=cmunti.ttf,BoldItalicFont=cmunbi.ttf]{cmuntt.ttf}\setmonofont[Path=/usr/share/fonts/truetype/cmu/,UprightFont=cmuntt.ttf,BoldFont=cmuntb.ttf,ItalicFont=cmunit.ttf,BoldItalicFont=cmuntx.ttf]{cmuntt.ttf}\ttfamily ++}\setmainfont[Path=/usr/share/fonts/truetype/cmu/,UprightFont=cmunrm.ttf,BoldFont=cmunbx.ttf,ItalicFont=cmunti.ttf,BoldItalicFont=cmunbi.ttf]{cmunrm.ttf}\setmonofont[Path=/usr/share/fonts/truetype/cmu/,UprightFont=cmuntt.ttf,BoldFont=cmuntb.ttf,ItalicFont=cmunit.ttf,BoldItalicFont=cmuntx.ttf]{cmunrm.ttf}\symbol{34} the last point of the path becomes the current position, with \symbol{34}{\ttfamily \setmainfont[Path=/usr/share/fonts/truetype/cmu/,UprightFont=cmunrm.ttf,BoldFont=cmunbx.ttf,ItalicFont=cmunti.ttf,BoldItalicFont=cmunbi.ttf]{cmuntt.ttf}\setmonofont[Path=/usr/share/fonts/truetype/cmu/,UprightFont=cmuntt.ttf,BoldFont=cmuntb.ttf,ItalicFont=cmunit.ttf,BoldItalicFont=cmuntx.ttf]{cmuntt.ttf}\ttfamily +}\setmainfont[Path=/usr/share/fonts/truetype/cmu/,UprightFont=cmunrm.ttf,BoldFont=cmunbx.ttf,ItalicFont=cmunti.ttf,BoldItalicFont=cmunbi.ttf]{cmunrm.ttf}\setmonofont[Path=/usr/share/fonts/truetype/cmu/,UprightFont=cmuntt.ttf,BoldFont=cmuntb.ttf,ItalicFont=cmunit.ttf,BoldItalicFont=cmuntx.ttf]{cmunrm.ttf}\symbol{34} the previous point stays the current path position. Example: 2 standard units to the right of the last point used:

\begin{Shaded}
\begin{Highlighting}[]

\NormalTok{++(2,0)}\newline
\end{Highlighting}
\end{Shaded}

\section{Syntax for Paths}
\label{796}
A path is a series of straight and curved line segments(in a simplified explanation). The instruction has to end with a semicolon.

\begin{Shaded}
\begin{Highlighting}[]

\NormalTok{\textbackslash{}path[<options>]\setmainfont[Path=/usr/share/fonts/truetype/freefont/,UprightFont=FreeSerif.ttf,BoldFont=FreeSerifBold.ttf,ItalicFont=FreeSerifItalic.ttf,BoldItalicFont=FreeSerifBoldItalic.ttf]{FreeSerif.ttf}\setmonofont[Path=/usr/share/fonts/truetype/freefont/,UprightFont=FreeMono.ttf,BoldFont=FreeMonoBold.ttf,ItalicFont=FreeMonoOblique.ttf,BoldItalicFont=FreeMonoBoldOblique.ttf]{FreeSerif.ttf}⟨\setmainfont[Path=/usr/share/fonts/truetype/cmu/,UprightFont=cmunrm.ttf,BoldFont=cmunbx.ttf,ItalicFont=cmunti.ttf,BoldItalicFont=cmunbi.ttf]{cmunrm.ttf}\setmonofont[Path=/usr/share/fonts/truetype/cmu/,UprightFont=cmuntt.ttf,BoldFont=cmuntb.ttf,ItalicFont=cmunit.ttf,BoldItalicFont=cmuntx.ttf]{cmunrm.ttf}specification\setmainfont[Path=/usr/share/fonts/truetype/freefont/,UprightFont=FreeSerif.ttf,BoldFont=FreeSerifBold.ttf,ItalicFont=FreeSerifItalic.ttf,BoldItalicFont=FreeSerifBoldItalic.ttf]{FreeSerif.ttf}\setmonofont[Path=/usr/share/fonts/truetype/freefont/,UprightFont=FreeMono.ttf,BoldFont=FreeMonoBold.ttf,ItalicFont=FreeMonoOblique.ttf,BoldItalicFont=FreeMonoBoldOblique.ttf]{FreeSerif.ttf}⟩\setmainfont[Path=/usr/share/fonts/truetype/cmu/,UprightFont=cmunrm.ttf,BoldFont=cmunbx.ttf,ItalicFont=cmunti.ttf,BoldItalicFont=cmunbi.ttf]{cmunrm.ttf}\setmonofont[Path=/usr/share/fonts/truetype/cmu/,UprightFont=cmuntt.ttf,BoldFont=cmuntb.ttf,ItalicFont=cmunit.ttf,BoldItalicFont=cmuntx.ttf]{cmunrm.ttf};}\newline
\end{Highlighting}
\end{Shaded}

One instruction can spread over several lines, or several instructions can be put on one line.
\subsection{Path actions}
\label{797}

Options for path actions are e.g: \symbol{34}{\ttfamily \setmainfont[Path=/usr/share/fonts/truetype/cmu/,UprightFont=cmunrm.ttf,BoldFont=cmunbx.ttf,ItalicFont=cmunti.ttf,BoldItalicFont=cmunbi.ttf]{cmuntt.ttf}\setmonofont[Path=/usr/share/fonts/truetype/cmu/,UprightFont=cmuntt.ttf,BoldFont=cmuntb.ttf,ItalicFont=cmunit.ttf,BoldItalicFont=cmuntx.ttf]{cmuntt.ttf}\ttfamily draw}\setmainfont[Path=/usr/share/fonts/truetype/cmu/,UprightFont=cmunrm.ttf,BoldFont=cmunbx.ttf,ItalicFont=cmunti.ttf,BoldItalicFont=cmunbi.ttf]{cmunrm.ttf}\setmonofont[Path=/usr/share/fonts/truetype/cmu/,UprightFont=cmuntt.ttf,BoldFont=cmuntb.ttf,ItalicFont=cmunit.ttf,BoldItalicFont=cmuntx.ttf]{cmunrm.ttf}\symbol{34}, \symbol{34}{\ttfamily \setmainfont[Path=/usr/share/fonts/truetype/cmu/,UprightFont=cmunrm.ttf,BoldFont=cmunbx.ttf,ItalicFont=cmunti.ttf,BoldItalicFont=cmunbi.ttf]{cmuntt.ttf}\setmonofont[Path=/usr/share/fonts/truetype/cmu/,UprightFont=cmuntt.ttf,BoldFont=cmuntb.ttf,ItalicFont=cmunit.ttf,BoldItalicFont=cmuntx.ttf]{cmuntt.ttf}\ttfamily fill}\setmainfont[Path=/usr/share/fonts/truetype/cmu/,UprightFont=cmunrm.ttf,BoldFont=cmunbx.ttf,ItalicFont=cmunti.ttf,BoldItalicFont=cmunbi.ttf]{cmunrm.ttf}\setmonofont[Path=/usr/share/fonts/truetype/cmu/,UprightFont=cmuntt.ttf,BoldFont=cmuntb.ttf,ItalicFont=cmunit.ttf,BoldItalicFont=cmuntx.ttf]{cmunrm.ttf}\symbol{34}, \symbol{34}{\ttfamily \setmainfont[Path=/usr/share/fonts/truetype/cmu/,UprightFont=cmunrm.ttf,BoldFont=cmunbx.ttf,ItalicFont=cmunti.ttf,BoldItalicFont=cmunbi.ttf]{cmuntt.ttf}\setmonofont[Path=/usr/share/fonts/truetype/cmu/,UprightFont=cmuntt.ttf,BoldFont=cmuntb.ttf,ItalicFont=cmunit.ttf,BoldItalicFont=cmuntx.ttf]{cmuntt.ttf}\ttfamily pattern}\setmainfont[Path=/usr/share/fonts/truetype/cmu/,UprightFont=cmunrm.ttf,BoldFont=cmunbx.ttf,ItalicFont=cmunti.ttf,BoldItalicFont=cmunbi.ttf]{cmunrm.ttf}\setmonofont[Path=/usr/share/fonts/truetype/cmu/,UprightFont=cmuntt.ttf,BoldFont=cmuntb.ttf,ItalicFont=cmunit.ttf,BoldItalicFont=cmuntx.ttf]{cmunrm.ttf}\symbol{34}, \symbol{34}{\ttfamily \setmainfont[Path=/usr/share/fonts/truetype/cmu/,UprightFont=cmunrm.ttf,BoldFont=cmunbx.ttf,ItalicFont=cmunti.ttf,BoldItalicFont=cmunbi.ttf]{cmuntt.ttf}\setmonofont[Path=/usr/share/fonts/truetype/cmu/,UprightFont=cmuntt.ttf,BoldFont=cmuntb.ttf,ItalicFont=cmunit.ttf,BoldItalicFont=cmuntx.ttf]{cmuntt.ttf}\ttfamily shade}\setmainfont[Path=/usr/share/fonts/truetype/cmu/,UprightFont=cmunrm.ttf,BoldFont=cmunbx.ttf,ItalicFont=cmunti.ttf,BoldItalicFont=cmunbi.ttf]{cmunrm.ttf}\setmonofont[Path=/usr/share/fonts/truetype/cmu/,UprightFont=cmuntt.ttf,BoldFont=cmuntb.ttf,ItalicFont=cmunit.ttf,BoldItalicFont=cmuntx.ttf]{cmunrm.ttf}\symbol{34} (a variation on filling that changes colors smoothly from one to another), \symbol{34}{\ttfamily \setmainfont[Path=/usr/share/fonts/truetype/cmu/,UprightFont=cmunrm.ttf,BoldFont=cmunbx.ttf,ItalicFont=cmunti.ttf,BoldItalicFont=cmunbi.ttf]{cmuntt.ttf}\setmonofont[Path=/usr/share/fonts/truetype/cmu/,UprightFont=cmuntt.ttf,BoldFont=cmuntb.ttf,ItalicFont=cmunit.ttf,BoldItalicFont=cmuntx.ttf]{cmuntt.ttf}\ttfamily clip}\setmainfont[Path=/usr/share/fonts/truetype/cmu/,UprightFont=cmunrm.ttf,BoldFont=cmunbx.ttf,ItalicFont=cmunti.ttf,BoldItalicFont=cmunbi.ttf]{cmunrm.ttf}\setmonofont[Path=/usr/share/fonts/truetype/cmu/,UprightFont=cmuntt.ttf,BoldFont=cmuntb.ttf,ItalicFont=cmunit.ttf,BoldItalicFont=cmuntx.ttf]{cmunrm.ttf}\symbol{34} (all subsequent drawings up to the end of the current scope are clipped against the current path and the size of subsequent paths will not be important for the picture size), \symbol{34}{\ttfamily \setmainfont[Path=/usr/share/fonts/truetype/cmu/,UprightFont=cmunrm.ttf,BoldFont=cmunbx.ttf,ItalicFont=cmunti.ttf,BoldItalicFont=cmunbi.ttf]{cmuntt.ttf}\setmonofont[Path=/usr/share/fonts/truetype/cmu/,UprightFont=cmuntt.ttf,BoldFont=cmuntb.ttf,ItalicFont=cmunit.ttf,BoldItalicFont=cmuntx.ttf]{cmuntt.ttf}\ttfamily use as bounding box}\setmainfont[Path=/usr/share/fonts/truetype/cmu/,UprightFont=cmunrm.ttf,BoldFont=cmunbx.ttf,ItalicFont=cmunti.ttf,BoldItalicFont=cmunbi.ttf]{cmunrm.ttf}\setmonofont[Path=/usr/share/fonts/truetype/cmu/,UprightFont=cmuntt.ttf,BoldFont=cmuntb.ttf,ItalicFont=cmunit.ttf,BoldItalicFont=cmuntx.ttf]{cmunrm.ttf}\symbol{34}. The \symbol{34}{\ttfamily \setmainfont[Path=/usr/share/fonts/truetype/cmu/,UprightFont=cmunrm.ttf,BoldFont=cmunbx.ttf,ItalicFont=cmunti.ttf,BoldItalicFont=cmunbi.ttf]{cmuntt.ttf}\setmonofont[Path=/usr/share/fonts/truetype/cmu/,UprightFont=cmuntt.ttf,BoldFont=cmuntb.ttf,ItalicFont=cmunit.ttf,BoldItalicFont=cmuntx.ttf]{cmuntt.ttf}\ttfamily \textbackslash{}path}\setmainfont[Path=/usr/share/fonts/truetype/cmu/,UprightFont=cmunrm.ttf,BoldFont=cmunbx.ttf,ItalicFont=cmunti.ttf,BoldItalicFont=cmunbi.ttf]{cmunrm.ttf}\setmonofont[Path=/usr/share/fonts/truetype/cmu/,UprightFont=cmuntt.ttf,BoldFont=cmuntb.ttf,ItalicFont=cmunit.ttf,BoldItalicFont=cmuntx.ttf]{cmunrm.ttf}\symbol{34} command with these options can be combined to: \symbol{34}{\ttfamily \setmainfont[Path=/usr/share/fonts/truetype/cmu/,UprightFont=cmunrm.ttf,BoldFont=cmunbx.ttf,ItalicFont=cmunti.ttf,BoldItalicFont=cmunbi.ttf]{cmuntt.ttf}\setmonofont[Path=/usr/share/fonts/truetype/cmu/,UprightFont=cmuntt.ttf,BoldFont=cmuntb.ttf,ItalicFont=cmunit.ttf,BoldItalicFont=cmuntx.ttf]{cmuntt.ttf}\ttfamily \textbackslash{}draw}\setmainfont[Path=/usr/share/fonts/truetype/cmu/,UprightFont=cmunrm.ttf,BoldFont=cmunbx.ttf,ItalicFont=cmunti.ttf,BoldItalicFont=cmunbi.ttf]{cmunrm.ttf}\setmonofont[Path=/usr/share/fonts/truetype/cmu/,UprightFont=cmuntt.ttf,BoldFont=cmuntb.ttf,ItalicFont=cmunit.ttf,BoldItalicFont=cmuntx.ttf]{cmunrm.ttf}\symbol{34}, \symbol{34}{\ttfamily \setmainfont[Path=/usr/share/fonts/truetype/cmu/,UprightFont=cmunrm.ttf,BoldFont=cmunbx.ttf,ItalicFont=cmunti.ttf,BoldItalicFont=cmunbi.ttf]{cmuntt.ttf}\setmonofont[Path=/usr/share/fonts/truetype/cmu/,UprightFont=cmuntt.ttf,BoldFont=cmuntb.ttf,ItalicFont=cmunit.ttf,BoldItalicFont=cmuntx.ttf]{cmuntt.ttf}\ttfamily \textbackslash{}fill}\setmainfont[Path=/usr/share/fonts/truetype/cmu/,UprightFont=cmunrm.ttf,BoldFont=cmunbx.ttf,ItalicFont=cmunti.ttf,BoldItalicFont=cmunbi.ttf]{cmunrm.ttf}\setmonofont[Path=/usr/share/fonts/truetype/cmu/,UprightFont=cmuntt.ttf,BoldFont=cmuntb.ttf,ItalicFont=cmunit.ttf,BoldItalicFont=cmuntx.ttf]{cmunrm.ttf}\symbol{34}, \symbol{34}{\ttfamily \setmainfont[Path=/usr/share/fonts/truetype/cmu/,UprightFont=cmunrm.ttf,BoldFont=cmunbx.ttf,ItalicFont=cmunti.ttf,BoldItalicFont=cmunbi.ttf]{cmuntt.ttf}\setmonofont[Path=/usr/share/fonts/truetype/cmu/,UprightFont=cmuntt.ttf,BoldFont=cmuntb.ttf,ItalicFont=cmunit.ttf,BoldItalicFont=cmuntx.ttf]{cmuntt.ttf}\ttfamily \textbackslash{}filldraw}\setmainfont[Path=/usr/share/fonts/truetype/cmu/,UprightFont=cmunrm.ttf,BoldFont=cmunbx.ttf,ItalicFont=cmunti.ttf,BoldItalicFont=cmunbi.ttf]{cmunrm.ttf}\setmonofont[Path=/usr/share/fonts/truetype/cmu/,UprightFont=cmuntt.ttf,BoldFont=cmuntb.ttf,ItalicFont=cmunit.ttf,BoldItalicFont=cmuntx.ttf]{cmunrm.ttf}\symbol{34}, \symbol{34}{\ttfamily \setmainfont[Path=/usr/share/fonts/truetype/cmu/,UprightFont=cmunrm.ttf,BoldFont=cmunbx.ttf,ItalicFont=cmunti.ttf,BoldItalicFont=cmunbi.ttf]{cmuntt.ttf}\setmonofont[Path=/usr/share/fonts/truetype/cmu/,UprightFont=cmuntt.ttf,BoldFont=cmuntb.ttf,ItalicFont=cmunit.ttf,BoldItalicFont=cmuntx.ttf]{cmuntt.ttf}\ttfamily \textbackslash{}pattern}\setmainfont[Path=/usr/share/fonts/truetype/cmu/,UprightFont=cmunrm.ttf,BoldFont=cmunbx.ttf,ItalicFont=cmunti.ttf,BoldItalicFont=cmunbi.ttf]{cmunrm.ttf}\setmonofont[Path=/usr/share/fonts/truetype/cmu/,UprightFont=cmuntt.ttf,BoldFont=cmuntb.ttf,ItalicFont=cmunit.ttf,BoldItalicFont=cmuntx.ttf]{cmunrm.ttf}\symbol{34}, \symbol{34}{\ttfamily \setmainfont[Path=/usr/share/fonts/truetype/cmu/,UprightFont=cmunrm.ttf,BoldFont=cmunbx.ttf,ItalicFont=cmunti.ttf,BoldItalicFont=cmunbi.ttf]{cmuntt.ttf}\setmonofont[Path=/usr/share/fonts/truetype/cmu/,UprightFont=cmuntt.ttf,BoldFont=cmuntb.ttf,ItalicFont=cmunit.ttf,BoldItalicFont=cmuntx.ttf]{cmuntt.ttf}\ttfamily \textbackslash{}shade}\setmainfont[Path=/usr/share/fonts/truetype/cmu/,UprightFont=cmunrm.ttf,BoldFont=cmunbx.ttf,ItalicFont=cmunti.ttf,BoldItalicFont=cmunbi.ttf]{cmunrm.ttf}\setmonofont[Path=/usr/share/fonts/truetype/cmu/,UprightFont=cmuntt.ttf,BoldFont=cmuntb.ttf,ItalicFont=cmunit.ttf,BoldItalicFont=cmuntx.ttf]{cmunrm.ttf}\symbol{34}, \symbol{34}{\ttfamily \setmainfont[Path=/usr/share/fonts/truetype/cmu/,UprightFont=cmunrm.ttf,BoldFont=cmunbx.ttf,ItalicFont=cmunti.ttf,BoldItalicFont=cmunbi.ttf]{cmuntt.ttf}\setmonofont[Path=/usr/share/fonts/truetype/cmu/,UprightFont=cmuntt.ttf,BoldFont=cmuntb.ttf,ItalicFont=cmunit.ttf,BoldItalicFont=cmuntx.ttf]{cmuntt.ttf}\ttfamily \textbackslash{}shadedraw}\setmainfont[Path=/usr/share/fonts/truetype/cmu/,UprightFont=cmunrm.ttf,BoldFont=cmunbx.ttf,ItalicFont=cmunti.ttf,BoldItalicFont=cmunbi.ttf]{cmunrm.ttf}\setmonofont[Path=/usr/share/fonts/truetype/cmu/,UprightFont=cmuntt.ttf,BoldFont=cmuntb.ttf,ItalicFont=cmunit.ttf,BoldItalicFont=cmuntx.ttf]{cmunrm.ttf}\symbol{34}, \symbol{34}{\ttfamily \setmainfont[Path=/usr/share/fonts/truetype/cmu/,UprightFont=cmunrm.ttf,BoldFont=cmunbx.ttf,ItalicFont=cmunti.ttf,BoldItalicFont=cmunbi.ttf]{cmuntt.ttf}\setmonofont[Path=/usr/share/fonts/truetype/cmu/,UprightFont=cmuntt.ttf,BoldFont=cmuntb.ttf,ItalicFont=cmunit.ttf,BoldItalicFont=cmuntx.ttf]{cmuntt.ttf}\ttfamily \textbackslash{}clip}\setmainfont[Path=/usr/share/fonts/truetype/cmu/,UprightFont=cmunrm.ttf,BoldFont=cmunbx.ttf,ItalicFont=cmunti.ttf,BoldItalicFont=cmunbi.ttf]{cmunrm.ttf}\setmonofont[Path=/usr/share/fonts/truetype/cmu/,UprightFont=cmuntt.ttf,BoldFont=cmuntb.ttf,ItalicFont=cmunit.ttf,BoldItalicFont=cmuntx.ttf]{cmunrm.ttf}\symbol{34}, \symbol{34}{\ttfamily \setmainfont[Path=/usr/share/fonts/truetype/cmu/,UprightFont=cmunrm.ttf,BoldFont=cmunbx.ttf,ItalicFont=cmunti.ttf,BoldItalicFont=cmunbi.ttf]{cmuntt.ttf}\setmonofont[Path=/usr/share/fonts/truetype/cmu/,UprightFont=cmuntt.ttf,BoldFont=cmuntb.ttf,ItalicFont=cmunit.ttf,BoldItalicFont=cmuntx.ttf]{cmuntt.ttf}\ttfamily \textbackslash{}useasboundingbox}\setmainfont[Path=/usr/share/fonts/truetype/cmu/,UprightFont=cmunrm.ttf,BoldFont=cmunbx.ttf,ItalicFont=cmunti.ttf,BoldItalicFont=cmunbi.ttf]{cmunrm.ttf}\setmonofont[Path=/usr/share/fonts/truetype/cmu/,UprightFont=cmuntt.ttf,BoldFont=cmuntb.ttf,ItalicFont=cmunit.ttf,BoldItalicFont=cmuntx.ttf]{cmunrm.ttf}\symbol{34} .
\subsection{Geometric path actions}
\label{798}

Geometric path options: \symbol{34}{\ttfamily \setmainfont[Path=/usr/share/fonts/truetype/cmu/,UprightFont=cmunrm.ttf,BoldFont=cmunbx.ttf,ItalicFont=cmunti.ttf,BoldItalicFont=cmunbi.ttf]{cmuntt.ttf}\setmonofont[Path=/usr/share/fonts/truetype/cmu/,UprightFont=cmuntt.ttf,BoldFont=cmuntb.ttf,ItalicFont=cmunit.ttf,BoldItalicFont=cmuntx.ttf]{cmuntt.ttf}\ttfamily rotate=<{}angle in degree>{}}\setmainfont[Path=/usr/share/fonts/truetype/cmu/,UprightFont=cmunrm.ttf,BoldFont=cmunbx.ttf,ItalicFont=cmunti.ttf,BoldItalicFont=cmunbi.ttf]{cmunrm.ttf}\setmonofont[Path=/usr/share/fonts/truetype/cmu/,UprightFont=cmuntt.ttf,BoldFont=cmuntb.ttf,ItalicFont=cmunit.ttf,BoldItalicFont=cmuntx.ttf]{cmunrm.ttf}\symbol{34}, \symbol{34}{\ttfamily \setmainfont[Path=/usr/share/fonts/truetype/cmu/,UprightFont=cmunrm.ttf,BoldFont=cmunbx.ttf,ItalicFont=cmunti.ttf,BoldItalicFont=cmunbi.ttf]{cmuntt.ttf}\setmonofont[Path=/usr/share/fonts/truetype/cmu/,UprightFont=cmuntt.ttf,BoldFont=cmuntb.ttf,ItalicFont=cmunit.ttf,BoldItalicFont=cmuntx.ttf]{cmuntt.ttf}\ttfamily xshift=<{}length>{}}\setmainfont[Path=/usr/share/fonts/truetype/cmu/,UprightFont=cmunrm.ttf,BoldFont=cmunbx.ttf,ItalicFont=cmunti.ttf,BoldItalicFont=cmunbi.ttf]{cmunrm.ttf}\setmonofont[Path=/usr/share/fonts/truetype/cmu/,UprightFont=cmuntt.ttf,BoldFont=cmuntb.ttf,ItalicFont=cmunit.ttf,BoldItalicFont=cmuntx.ttf]{cmunrm.ttf}\symbol{34}, \symbol{34}{\ttfamily \setmainfont[Path=/usr/share/fonts/truetype/cmu/,UprightFont=cmunrm.ttf,BoldFont=cmunbx.ttf,ItalicFont=cmunti.ttf,BoldItalicFont=cmunbi.ttf]{cmuntt.ttf}\setmonofont[Path=/usr/share/fonts/truetype/cmu/,UprightFont=cmuntt.ttf,BoldFont=cmuntb.ttf,ItalicFont=cmunit.ttf,BoldItalicFont=cmuntx.ttf]{cmuntt.ttf}\ttfamily yshift=<{}length>{}}\setmainfont[Path=/usr/share/fonts/truetype/cmu/,UprightFont=cmunrm.ttf,BoldFont=cmunbx.ttf,ItalicFont=cmunti.ttf,BoldItalicFont=cmunbi.ttf]{cmunrm.ttf}\setmonofont[Path=/usr/share/fonts/truetype/cmu/,UprightFont=cmuntt.ttf,BoldFont=cmuntb.ttf,ItalicFont=cmunit.ttf,BoldItalicFont=cmuntx.ttf]{cmunrm.ttf}\symbol{34}, \symbol{34}{\ttfamily \setmainfont[Path=/usr/share/fonts/truetype/cmu/,UprightFont=cmunrm.ttf,BoldFont=cmunbx.ttf,ItalicFont=cmunti.ttf,BoldItalicFont=cmunbi.ttf]{cmuntt.ttf}\setmonofont[Path=/usr/share/fonts/truetype/cmu/,UprightFont=cmuntt.ttf,BoldFont=cmuntb.ttf,ItalicFont=cmunit.ttf,BoldItalicFont=cmuntx.ttf]{cmuntt.ttf}\ttfamily scaling=<{}factor>{}}\setmainfont[Path=/usr/share/fonts/truetype/cmu/,UprightFont=cmunrm.ttf,BoldFont=cmunbx.ttf,ItalicFont=cmunti.ttf,BoldItalicFont=cmunbi.ttf]{cmunrm.ttf}\setmonofont[Path=/usr/share/fonts/truetype/cmu/,UprightFont=cmuntt.ttf,BoldFont=cmuntb.ttf,ItalicFont=cmunit.ttf,BoldItalicFont=cmuntx.ttf]{cmunrm.ttf}\symbol{34}, \symbol{34}{\ttfamily \setmainfont[Path=/usr/share/fonts/truetype/cmu/,UprightFont=cmunrm.ttf,BoldFont=cmunbx.ttf,ItalicFont=cmunti.ttf,BoldItalicFont=cmunbi.ttf]{cmuntt.ttf}\setmonofont[Path=/usr/share/fonts/truetype/cmu/,UprightFont=cmuntt.ttf,BoldFont=cmuntb.ttf,ItalicFont=cmunit.ttf,BoldItalicFont=cmuntx.ttf]{cmuntt.ttf}\ttfamily xscale=<{}factor>{}}\setmainfont[Path=/usr/share/fonts/truetype/cmu/,UprightFont=cmunrm.ttf,BoldFont=cmunbx.ttf,ItalicFont=cmunti.ttf,BoldItalicFont=cmunbi.ttf]{cmunrm.ttf}\setmonofont[Path=/usr/share/fonts/truetype/cmu/,UprightFont=cmuntt.ttf,BoldFont=cmuntb.ttf,ItalicFont=cmunit.ttf,BoldItalicFont=cmuntx.ttf]{cmunrm.ttf}\symbol{34}, \symbol{34}{\ttfamily \setmainfont[Path=/usr/share/fonts/truetype/cmu/,UprightFont=cmunrm.ttf,BoldFont=cmunbx.ttf,ItalicFont=cmunti.ttf,BoldItalicFont=cmunbi.ttf]{cmuntt.ttf}\setmonofont[Path=/usr/share/fonts/truetype/cmu/,UprightFont=cmuntt.ttf,BoldFont=cmuntb.ttf,ItalicFont=cmunit.ttf,BoldItalicFont=cmuntx.ttf]{cmuntt.ttf}\ttfamily yscale=<{}factor>{}}\setmainfont[Path=/usr/share/fonts/truetype/cmu/,UprightFont=cmunrm.ttf,BoldFont=cmunbx.ttf,ItalicFont=cmunti.ttf,BoldItalicFont=cmunbi.ttf]{cmunrm.ttf}\setmonofont[Path=/usr/share/fonts/truetype/cmu/,UprightFont=cmuntt.ttf,BoldFont=cmuntb.ttf,ItalicFont=cmunit.ttf,BoldItalicFont=cmuntx.ttf]{cmunrm.ttf}\symbol{34}.
\subsection{Color}
\label{799}

Color options for drawing paths: \symbol{34}{\ttfamily \setmainfont[Path=/usr/share/fonts/truetype/cmu/,UprightFont=cmunrm.ttf,BoldFont=cmunbx.ttf,ItalicFont=cmunti.ttf,BoldItalicFont=cmunbi.ttf]{cmuntt.ttf}\setmonofont[Path=/usr/share/fonts/truetype/cmu/,UprightFont=cmuntt.ttf,BoldFont=cmuntb.ttf,ItalicFont=cmunit.ttf,BoldItalicFont=cmuntx.ttf]{cmuntt.ttf}\ttfamily color=<{}color name>{}}\setmainfont[Path=/usr/share/fonts/truetype/cmu/,UprightFont=cmunrm.ttf,BoldFont=cmunbx.ttf,ItalicFont=cmunti.ttf,BoldItalicFont=cmunbi.ttf]{cmunrm.ttf}\setmonofont[Path=/usr/share/fonts/truetype/cmu/,UprightFont=cmuntt.ttf,BoldFont=cmuntb.ttf,ItalicFont=cmunit.ttf,BoldItalicFont=cmuntx.ttf]{cmunrm.ttf}\symbol{34}, \symbol{34}{\ttfamily \setmainfont[Path=/usr/share/fonts/truetype/cmu/,UprightFont=cmunrm.ttf,BoldFont=cmunbx.ttf,ItalicFont=cmunti.ttf,BoldItalicFont=cmunbi.ttf]{cmuntt.ttf}\setmonofont[Path=/usr/share/fonts/truetype/cmu/,UprightFont=cmuntt.ttf,BoldFont=cmuntb.ttf,ItalicFont=cmunit.ttf,BoldItalicFont=cmuntx.ttf]{cmuntt.ttf}\ttfamily draw=<{}line color>{}}\setmainfont[Path=/usr/share/fonts/truetype/cmu/,UprightFont=cmunrm.ttf,BoldFont=cmunbx.ttf,ItalicFont=cmunti.ttf,BoldItalicFont=cmunbi.ttf]{cmunrm.ttf}\setmonofont[Path=/usr/share/fonts/truetype/cmu/,UprightFont=cmuntt.ttf,BoldFont=cmuntb.ttf,ItalicFont=cmunit.ttf,BoldItalicFont=cmuntx.ttf]{cmunrm.ttf}\symbol{34}, \symbol{34}{\ttfamily \setmainfont[Path=/usr/share/fonts/truetype/cmu/,UprightFont=cmunrm.ttf,BoldFont=cmunbx.ttf,ItalicFont=cmunti.ttf,BoldItalicFont=cmunbi.ttf]{cmuntt.ttf}\setmonofont[Path=/usr/share/fonts/truetype/cmu/,UprightFont=cmuntt.ttf,BoldFont=cmuntb.ttf,ItalicFont=cmunit.ttf,BoldItalicFont=cmuntx.ttf]{cmuntt.ttf}\ttfamily opacity=<{}factor>{}}\setmainfont[Path=/usr/share/fonts/truetype/cmu/,UprightFont=cmunrm.ttf,BoldFont=cmunbx.ttf,ItalicFont=cmunti.ttf,BoldItalicFont=cmunbi.ttf]{cmunrm.ttf}\setmonofont[Path=/usr/share/fonts/truetype/cmu/,UprightFont=cmuntt.ttf,BoldFont=cmuntb.ttf,ItalicFont=cmunit.ttf,BoldItalicFont=cmuntx.ttf]{cmunrm.ttf}\symbol{34}. Following colors are predefined: red, green, blue, cyan , magenta, yellow, black, gray, darkgray, lightgray, brown, lime, olive, orange, pink, purple, teal, violet and white.
\subsection{Line width}
\label{800}

Line width options: \symbol{34}{\ttfamily \setmainfont[Path=/usr/share/fonts/truetype/cmu/,UprightFont=cmunrm.ttf,BoldFont=cmunbx.ttf,ItalicFont=cmunti.ttf,BoldItalicFont=cmunbi.ttf]{cmuntt.ttf}\setmonofont[Path=/usr/share/fonts/truetype/cmu/,UprightFont=cmuntt.ttf,BoldFont=cmuntb.ttf,ItalicFont=cmunit.ttf,BoldItalicFont=cmuntx.ttf]{cmuntt.ttf}\ttfamily line width=<{}dimension>{}}\setmainfont[Path=/usr/share/fonts/truetype/cmu/,UprightFont=cmunrm.ttf,BoldFont=cmunbx.ttf,ItalicFont=cmunti.ttf,BoldItalicFont=cmunbi.ttf]{cmunrm.ttf}\setmonofont[Path=/usr/share/fonts/truetype/cmu/,UprightFont=cmuntt.ttf,BoldFont=cmuntb.ttf,ItalicFont=cmunit.ttf,BoldItalicFont=cmuntx.ttf]{cmunrm.ttf}\symbol{34}, and abbreviations \symbol{34}{\ttfamily \setmainfont[Path=/usr/share/fonts/truetype/cmu/,UprightFont=cmunrm.ttf,BoldFont=cmunbx.ttf,ItalicFont=cmunti.ttf,BoldItalicFont=cmunbi.ttf]{cmuntt.ttf}\setmonofont[Path=/usr/share/fonts/truetype/cmu/,UprightFont=cmuntt.ttf,BoldFont=cmuntb.ttf,ItalicFont=cmunit.ttf,BoldItalicFont=cmuntx.ttf]{cmuntt.ttf}\ttfamily ultra thin}\setmainfont[Path=/usr/share/fonts/truetype/cmu/,UprightFont=cmunrm.ttf,BoldFont=cmunbx.ttf,ItalicFont=cmunti.ttf,BoldItalicFont=cmunbi.ttf]{cmunrm.ttf}\setmonofont[Path=/usr/share/fonts/truetype/cmu/,UprightFont=cmuntt.ttf,BoldFont=cmuntb.ttf,ItalicFont=cmunit.ttf,BoldItalicFont=cmuntx.ttf]{cmunrm.ttf}\symbol{34} for 0.1pt, \symbol{34}{\ttfamily \setmainfont[Path=/usr/share/fonts/truetype/cmu/,UprightFont=cmunrm.ttf,BoldFont=cmunbx.ttf,ItalicFont=cmunti.ttf,BoldItalicFont=cmunbi.ttf]{cmuntt.ttf}\setmonofont[Path=/usr/share/fonts/truetype/cmu/,UprightFont=cmuntt.ttf,BoldFont=cmuntb.ttf,ItalicFont=cmunit.ttf,BoldItalicFont=cmuntx.ttf]{cmuntt.ttf}\ttfamily very thin}\setmainfont[Path=/usr/share/fonts/truetype/cmu/,UprightFont=cmunrm.ttf,BoldFont=cmunbx.ttf,ItalicFont=cmunti.ttf,BoldItalicFont=cmunbi.ttf]{cmunrm.ttf}\setmonofont[Path=/usr/share/fonts/truetype/cmu/,UprightFont=cmuntt.ttf,BoldFont=cmuntb.ttf,ItalicFont=cmunit.ttf,BoldItalicFont=cmuntx.ttf]{cmunrm.ttf}\symbol{34} for 0.2pt, \symbol{34}{\ttfamily \setmainfont[Path=/usr/share/fonts/truetype/cmu/,UprightFont=cmunrm.ttf,BoldFont=cmunbx.ttf,ItalicFont=cmunti.ttf,BoldItalicFont=cmunbi.ttf]{cmuntt.ttf}\setmonofont[Path=/usr/share/fonts/truetype/cmu/,UprightFont=cmuntt.ttf,BoldFont=cmuntb.ttf,ItalicFont=cmunit.ttf,BoldItalicFont=cmuntx.ttf]{cmuntt.ttf}\ttfamily thin}\setmainfont[Path=/usr/share/fonts/truetype/cmu/,UprightFont=cmunrm.ttf,BoldFont=cmunbx.ttf,ItalicFont=cmunti.ttf,BoldItalicFont=cmunbi.ttf]{cmunrm.ttf}\setmonofont[Path=/usr/share/fonts/truetype/cmu/,UprightFont=cmuntt.ttf,BoldFont=cmuntb.ttf,ItalicFont=cmunit.ttf,BoldItalicFont=cmuntx.ttf]{cmunrm.ttf}\symbol{34} for 0.4pt (the default width), \symbol{34}{\ttfamily \setmainfont[Path=/usr/share/fonts/truetype/cmu/,UprightFont=cmunrm.ttf,BoldFont=cmunbx.ttf,ItalicFont=cmunti.ttf,BoldItalicFont=cmunbi.ttf]{cmuntt.ttf}\setmonofont[Path=/usr/share/fonts/truetype/cmu/,UprightFont=cmuntt.ttf,BoldFont=cmuntb.ttf,ItalicFont=cmunit.ttf,BoldItalicFont=cmuntx.ttf]{cmuntt.ttf}\ttfamily semithick}\setmainfont[Path=/usr/share/fonts/truetype/cmu/,UprightFont=cmunrm.ttf,BoldFont=cmunbx.ttf,ItalicFont=cmunti.ttf,BoldItalicFont=cmunbi.ttf]{cmunrm.ttf}\setmonofont[Path=/usr/share/fonts/truetype/cmu/,UprightFont=cmuntt.ttf,BoldFont=cmuntb.ttf,ItalicFont=cmunit.ttf,BoldItalicFont=cmuntx.ttf]{cmunrm.ttf}\symbol{34} for 0.6pt, \symbol{34}{\ttfamily \setmainfont[Path=/usr/share/fonts/truetype/cmu/,UprightFont=cmunrm.ttf,BoldFont=cmunbx.ttf,ItalicFont=cmunti.ttf,BoldItalicFont=cmunbi.ttf]{cmuntt.ttf}\setmonofont[Path=/usr/share/fonts/truetype/cmu/,UprightFont=cmuntt.ttf,BoldFont=cmuntb.ttf,ItalicFont=cmunit.ttf,BoldItalicFont=cmuntx.ttf]{cmuntt.ttf}\ttfamily thick}\setmainfont[Path=/usr/share/fonts/truetype/cmu/,UprightFont=cmunrm.ttf,BoldFont=cmunbx.ttf,ItalicFont=cmunti.ttf,BoldItalicFont=cmunbi.ttf]{cmunrm.ttf}\setmonofont[Path=/usr/share/fonts/truetype/cmu/,UprightFont=cmuntt.ttf,BoldFont=cmuntb.ttf,ItalicFont=cmunit.ttf,BoldItalicFont=cmuntx.ttf]{cmunrm.ttf}\symbol{34} for 0.8pt, \symbol{34}{\ttfamily \setmainfont[Path=/usr/share/fonts/truetype/cmu/,UprightFont=cmunrm.ttf,BoldFont=cmunbx.ttf,ItalicFont=cmunti.ttf,BoldItalicFont=cmunbi.ttf]{cmuntt.ttf}\setmonofont[Path=/usr/share/fonts/truetype/cmu/,UprightFont=cmuntt.ttf,BoldFont=cmuntb.ttf,ItalicFont=cmunit.ttf,BoldItalicFont=cmuntx.ttf]{cmuntt.ttf}\ttfamily very thick}\setmainfont[Path=/usr/share/fonts/truetype/cmu/,UprightFont=cmunrm.ttf,BoldFont=cmunbx.ttf,ItalicFont=cmunti.ttf,BoldItalicFont=cmunbi.ttf]{cmunrm.ttf}\setmonofont[Path=/usr/share/fonts/truetype/cmu/,UprightFont=cmuntt.ttf,BoldFont=cmuntb.ttf,ItalicFont=cmunit.ttf,BoldItalicFont=cmuntx.ttf]{cmunrm.ttf}\symbol{34} for 1.2pt, \symbol{34}{\ttfamily \setmainfont[Path=/usr/share/fonts/truetype/cmu/,UprightFont=cmunrm.ttf,BoldFont=cmunbx.ttf,ItalicFont=cmunti.ttf,BoldItalicFont=cmunbi.ttf]{cmuntt.ttf}\setmonofont[Path=/usr/share/fonts/truetype/cmu/,UprightFont=cmuntt.ttf,BoldFont=cmuntb.ttf,ItalicFont=cmunit.ttf,BoldItalicFont=cmuntx.ttf]{cmuntt.ttf}\ttfamily ultra thick}\setmainfont[Path=/usr/share/fonts/truetype/cmu/,UprightFont=cmunrm.ttf,BoldFont=cmunbx.ttf,ItalicFont=cmunti.ttf,BoldItalicFont=cmunbi.ttf]{cmunrm.ttf}\setmonofont[Path=/usr/share/fonts/truetype/cmu/,UprightFont=cmuntt.ttf,BoldFont=cmuntb.ttf,ItalicFont=cmunit.ttf,BoldItalicFont=cmuntx.ttf]{cmunrm.ttf}\symbol{34} for 1.6pt. 
\subsection{Line end}
\label{801}

Line end, line join options: \symbol{34}{\ttfamily \setmainfont[Path=/usr/share/fonts/truetype/cmu/,UprightFont=cmunrm.ttf,BoldFont=cmunbx.ttf,ItalicFont=cmunti.ttf,BoldItalicFont=cmunbi.ttf]{cmuntt.ttf}\setmonofont[Path=/usr/share/fonts/truetype/cmu/,UprightFont=cmuntt.ttf,BoldFont=cmuntb.ttf,ItalicFont=cmunit.ttf,BoldItalicFont=cmuntx.ttf]{cmuntt.ttf}\ttfamily line cap=<{}type: round, rect, or butt>{}}\setmainfont[Path=/usr/share/fonts/truetype/cmu/,UprightFont=cmunrm.ttf,BoldFont=cmunbx.ttf,ItalicFont=cmunti.ttf,BoldItalicFont=cmunbi.ttf]{cmunrm.ttf}\setmonofont[Path=/usr/share/fonts/truetype/cmu/,UprightFont=cmuntt.ttf,BoldFont=cmuntb.ttf,ItalicFont=cmunit.ttf,BoldItalicFont=cmuntx.ttf]{cmunrm.ttf}\symbol{34}, \symbol{34}{\ttfamily \setmainfont[Path=/usr/share/fonts/truetype/cmu/,UprightFont=cmunrm.ttf,BoldFont=cmunbx.ttf,ItalicFont=cmunti.ttf,BoldItalicFont=cmunbi.ttf]{cmuntt.ttf}\setmonofont[Path=/usr/share/fonts/truetype/cmu/,UprightFont=cmuntt.ttf,BoldFont=cmuntb.ttf,ItalicFont=cmunit.ttf,BoldItalicFont=cmuntx.ttf]{cmuntt.ttf}\ttfamily arrows=<{}start arrow kind>{}-{}<{}end arrow kind>{}}\setmainfont[Path=/usr/share/fonts/truetype/cmu/,UprightFont=cmunrm.ttf,BoldFont=cmunbx.ttf,ItalicFont=cmunti.ttf,BoldItalicFont=cmunbi.ttf]{cmunrm.ttf}\setmonofont[Path=/usr/share/fonts/truetype/cmu/,UprightFont=cmuntt.ttf,BoldFont=cmuntb.ttf,ItalicFont=cmunit.ttf,BoldItalicFont=cmuntx.ttf]{cmunrm.ttf}\symbol{34}, \symbol{34}{\ttfamily \setmainfont[Path=/usr/share/fonts/truetype/cmu/,UprightFont=cmunrm.ttf,BoldFont=cmunbx.ttf,ItalicFont=cmunti.ttf,BoldItalicFont=cmunbi.ttf]{cmuntt.ttf}\setmonofont[Path=/usr/share/fonts/truetype/cmu/,UprightFont=cmuntt.ttf,BoldFont=cmuntb.ttf,ItalicFont=cmunit.ttf,BoldItalicFont=cmuntx.ttf]{cmuntt.ttf}\ttfamily rounded corners}\setmainfont[Path=/usr/share/fonts/truetype/cmu/,UprightFont=cmunrm.ttf,BoldFont=cmunbx.ttf,ItalicFont=cmunti.ttf,BoldItalicFont=cmunbi.ttf]{cmunrm.ttf}\setmonofont[Path=/usr/share/fonts/truetype/cmu/,UprightFont=cmuntt.ttf,BoldFont=cmuntb.ttf,ItalicFont=cmunit.ttf,BoldItalicFont=cmuntx.ttf]{cmunrm.ttf}\symbol{34}, \symbol{34}{\ttfamily \setmainfont[Path=/usr/share/fonts/truetype/cmu/,UprightFont=cmunrm.ttf,BoldFont=cmunbx.ttf,ItalicFont=cmunti.ttf,BoldItalicFont=cmunbi.ttf]{cmuntt.ttf}\setmonofont[Path=/usr/share/fonts/truetype/cmu/,UprightFont=cmuntt.ttf,BoldFont=cmuntb.ttf,ItalicFont=cmunit.ttf,BoldItalicFont=cmuntx.ttf]{cmuntt.ttf}\ttfamily rounded corners=<{}size>{}}\setmainfont[Path=/usr/share/fonts/truetype/cmu/,UprightFont=cmunrm.ttf,BoldFont=cmunbx.ttf,ItalicFont=cmunti.ttf,BoldItalicFont=cmunbi.ttf]{cmunrm.ttf}\setmonofont[Path=/usr/share/fonts/truetype/cmu/,UprightFont=cmuntt.ttf,BoldFont=cmuntb.ttf,ItalicFont=cmunit.ttf,BoldItalicFont=cmuntx.ttf]{cmunrm.ttf}\symbol{34}, \symbol{34}{\ttfamily \setmainfont[Path=/usr/share/fonts/truetype/cmu/,UprightFont=cmunrm.ttf,BoldFont=cmunbx.ttf,ItalicFont=cmunti.ttf,BoldItalicFont=cmunbi.ttf]{cmuntt.ttf}\setmonofont[Path=/usr/share/fonts/truetype/cmu/,UprightFont=cmuntt.ttf,BoldFont=cmuntb.ttf,ItalicFont=cmunit.ttf,BoldItalicFont=cmuntx.ttf]{cmuntt.ttf}\ttfamily line join=<{}type: round, bevel, or miter>{}}\setmainfont[Path=/usr/share/fonts/truetype/cmu/,UprightFont=cmunrm.ttf,BoldFont=cmunbx.ttf,ItalicFont=cmunti.ttf,BoldItalicFont=cmunbi.ttf]{cmunrm.ttf}\setmonofont[Path=/usr/share/fonts/truetype/cmu/,UprightFont=cmuntt.ttf,BoldFont=cmuntb.ttf,ItalicFont=cmunit.ttf,BoldItalicFont=cmuntx.ttf]{cmunrm.ttf}\symbol{34}.
\subsection{Line pattern}
\label{802}

Line pattern options: \symbol{34}{\ttfamily \setmainfont[Path=/usr/share/fonts/truetype/cmu/,UprightFont=cmunrm.ttf,BoldFont=cmunbx.ttf,ItalicFont=cmunti.ttf,BoldItalicFont=cmunbi.ttf]{cmuntt.ttf}\setmonofont[Path=/usr/share/fonts/truetype/cmu/,UprightFont=cmuntt.ttf,BoldFont=cmuntb.ttf,ItalicFont=cmunit.ttf,BoldItalicFont=cmuntx.ttf]{cmuntt.ttf}\ttfamily dash pattern=<{}dash pattern>{}}\setmainfont[Path=/usr/share/fonts/truetype/cmu/,UprightFont=cmunrm.ttf,BoldFont=cmunbx.ttf,ItalicFont=cmunti.ttf,BoldItalicFont=cmunbi.ttf]{cmunrm.ttf}\setmonofont[Path=/usr/share/fonts/truetype/cmu/,UprightFont=cmuntt.ttf,BoldFont=cmuntb.ttf,ItalicFont=cmunit.ttf,BoldItalicFont=cmuntx.ttf]{cmunrm.ttf}\symbol{34} (e.g. \symbol{34}{\ttfamily \setmainfont[Path=/usr/share/fonts/truetype/cmu/,UprightFont=cmunrm.ttf,BoldFont=cmunbx.ttf,ItalicFont=cmunti.ttf,BoldItalicFont=cmunbi.ttf]{cmuntt.ttf}\setmonofont[Path=/usr/share/fonts/truetype/cmu/,UprightFont=cmuntt.ttf,BoldFont=cmuntb.ttf,ItalicFont=cmunit.ttf,BoldItalicFont=cmuntx.ttf]{cmuntt.ttf}\ttfamily dash pattern=on 2pt off 3pt on 4pt off 4pt}\setmainfont[Path=/usr/share/fonts/truetype/cmu/,UprightFont=cmunrm.ttf,BoldFont=cmunbx.ttf,ItalicFont=cmunti.ttf,BoldItalicFont=cmunbi.ttf]{cmunrm.ttf}\setmonofont[Path=/usr/share/fonts/truetype/cmu/,UprightFont=cmuntt.ttf,BoldFont=cmuntb.ttf,ItalicFont=cmunit.ttf,BoldItalicFont=cmuntx.ttf]{cmunrm.ttf}\symbol{34}), \symbol{34}{\ttfamily \setmainfont[Path=/usr/share/fonts/truetype/cmu/,UprightFont=cmunrm.ttf,BoldFont=cmunbx.ttf,ItalicFont=cmunti.ttf,BoldItalicFont=cmunbi.ttf]{cmuntt.ttf}\setmonofont[Path=/usr/share/fonts/truetype/cmu/,UprightFont=cmuntt.ttf,BoldFont=cmuntb.ttf,ItalicFont=cmunit.ttf,BoldItalicFont=cmuntx.ttf]{cmuntt.ttf}\ttfamily dash phase=\setmainfont[Path=/usr/share/fonts/truetype/freefont/,UprightFont=FreeSerif.ttf,BoldFont=FreeSerifBold.ttf,ItalicFont=FreeSerifItalic.ttf,BoldItalicFont=FreeSerifBoldItalic.ttf]{FreeMono.ttf}\setmonofont[Path=/usr/share/fonts/truetype/freefont/,UprightFont=FreeMono.ttf,BoldFont=FreeMonoBold.ttf,ItalicFont=FreeMonoOblique.ttf,BoldItalicFont=FreeMonoBoldOblique.ttf]{FreeMono.ttf}\ttfamily ⟨\setmainfont[Path=/usr/share/fonts/truetype/cmu/,UprightFont=cmunrm.ttf,BoldFont=cmunbx.ttf,ItalicFont=cmunti.ttf,BoldItalicFont=cmunbi.ttf]{cmuntt.ttf}\setmonofont[Path=/usr/share/fonts/truetype/cmu/,UprightFont=cmuntt.ttf,BoldFont=cmuntb.ttf,ItalicFont=cmunit.ttf,BoldItalicFont=cmuntx.ttf]{cmuntt.ttf}\ttfamily dash phase\setmainfont[Path=/usr/share/fonts/truetype/freefont/,UprightFont=FreeSerif.ttf,BoldFont=FreeSerifBold.ttf,ItalicFont=FreeSerifItalic.ttf,BoldItalicFont=FreeSerifBoldItalic.ttf]{FreeMono.ttf}\setmonofont[Path=/usr/share/fonts/truetype/freefont/,UprightFont=FreeMono.ttf,BoldFont=FreeMonoBold.ttf,ItalicFont=FreeMonoOblique.ttf,BoldItalicFont=FreeMonoBoldOblique.ttf]{FreeMono.ttf}\ttfamily ⟩}\setmainfont[Path=/usr/share/fonts/truetype/cmu/,UprightFont=cmunrm.ttf,BoldFont=cmunbx.ttf,ItalicFont=cmunti.ttf,BoldItalicFont=cmunbi.ttf]{cmunrm.ttf}\setmonofont[Path=/usr/share/fonts/truetype/cmu/,UprightFont=cmuntt.ttf,BoldFont=cmuntb.ttf,ItalicFont=cmunit.ttf,BoldItalicFont=cmuntx.ttf]{cmunrm.ttf}\symbol{34}, \symbol{34}{\ttfamily \setmainfont[Path=/usr/share/fonts/truetype/cmu/,UprightFont=cmunrm.ttf,BoldFont=cmunbx.ttf,ItalicFont=cmunti.ttf,BoldItalicFont=cmunbi.ttf]{cmuntt.ttf}\setmonofont[Path=/usr/share/fonts/truetype/cmu/,UprightFont=cmuntt.ttf,BoldFont=cmuntb.ttf,ItalicFont=cmunit.ttf,BoldItalicFont=cmuntx.ttf]{cmuntt.ttf}\ttfamily solid}\setmainfont[Path=/usr/share/fonts/truetype/cmu/,UprightFont=cmunrm.ttf,BoldFont=cmunbx.ttf,ItalicFont=cmunti.ttf,BoldItalicFont=cmunbi.ttf]{cmunrm.ttf}\setmonofont[Path=/usr/share/fonts/truetype/cmu/,UprightFont=cmuntt.ttf,BoldFont=cmuntb.ttf,ItalicFont=cmunit.ttf,BoldItalicFont=cmuntx.ttf]{cmunrm.ttf}\symbol{34}, \symbol{34}{\ttfamily \setmainfont[Path=/usr/share/fonts/truetype/cmu/,UprightFont=cmunrm.ttf,BoldFont=cmunbx.ttf,ItalicFont=cmunti.ttf,BoldItalicFont=cmunbi.ttf]{cmuntt.ttf}\setmonofont[Path=/usr/share/fonts/truetype/cmu/,UprightFont=cmuntt.ttf,BoldFont=cmuntb.ttf,ItalicFont=cmunit.ttf,BoldItalicFont=cmuntx.ttf]{cmuntt.ttf}\ttfamily dashed}\setmainfont[Path=/usr/share/fonts/truetype/cmu/,UprightFont=cmunrm.ttf,BoldFont=cmunbx.ttf,ItalicFont=cmunti.ttf,BoldItalicFont=cmunbi.ttf]{cmunrm.ttf}\setmonofont[Path=/usr/share/fonts/truetype/cmu/,UprightFont=cmuntt.ttf,BoldFont=cmuntb.ttf,ItalicFont=cmunit.ttf,BoldItalicFont=cmuntx.ttf]{cmunrm.ttf}\symbol{34}, \symbol{34}{\ttfamily \setmainfont[Path=/usr/share/fonts/truetype/cmu/,UprightFont=cmunrm.ttf,BoldFont=cmunbx.ttf,ItalicFont=cmunti.ttf,BoldItalicFont=cmunbi.ttf]{cmuntt.ttf}\setmonofont[Path=/usr/share/fonts/truetype/cmu/,UprightFont=cmuntt.ttf,BoldFont=cmuntb.ttf,ItalicFont=cmunit.ttf,BoldItalicFont=cmuntx.ttf]{cmuntt.ttf}\ttfamily dotted}\setmainfont[Path=/usr/share/fonts/truetype/cmu/,UprightFont=cmunrm.ttf,BoldFont=cmunbx.ttf,ItalicFont=cmunti.ttf,BoldItalicFont=cmunbi.ttf]{cmunrm.ttf}\setmonofont[Path=/usr/share/fonts/truetype/cmu/,UprightFont=cmuntt.ttf,BoldFont=cmuntb.ttf,ItalicFont=cmunit.ttf,BoldItalicFont=cmuntx.ttf]{cmunrm.ttf}\symbol{34}, \symbol{34}{\ttfamily \setmainfont[Path=/usr/share/fonts/truetype/cmu/,UprightFont=cmunrm.ttf,BoldFont=cmunbx.ttf,ItalicFont=cmunti.ttf,BoldItalicFont=cmunbi.ttf]{cmuntt.ttf}\setmonofont[Path=/usr/share/fonts/truetype/cmu/,UprightFont=cmuntt.ttf,BoldFont=cmuntb.ttf,ItalicFont=cmunit.ttf,BoldItalicFont=cmuntx.ttf]{cmuntt.ttf}\ttfamily dashdotted}\setmainfont[Path=/usr/share/fonts/truetype/cmu/,UprightFont=cmunrm.ttf,BoldFont=cmunbx.ttf,ItalicFont=cmunti.ttf,BoldItalicFont=cmunbi.ttf]{cmunrm.ttf}\setmonofont[Path=/usr/share/fonts/truetype/cmu/,UprightFont=cmuntt.ttf,BoldFont=cmuntb.ttf,ItalicFont=cmunit.ttf,BoldItalicFont=cmuntx.ttf]{cmunrm.ttf}\symbol{34}, \symbol{34}{\ttfamily \setmainfont[Path=/usr/share/fonts/truetype/cmu/,UprightFont=cmunrm.ttf,BoldFont=cmunbx.ttf,ItalicFont=cmunti.ttf,BoldItalicFont=cmunbi.ttf]{cmuntt.ttf}\setmonofont[Path=/usr/share/fonts/truetype/cmu/,UprightFont=cmuntt.ttf,BoldFont=cmuntb.ttf,ItalicFont=cmunit.ttf,BoldItalicFont=cmuntx.ttf]{cmuntt.ttf}\ttfamily densely dotted}\setmainfont[Path=/usr/share/fonts/truetype/cmu/,UprightFont=cmunrm.ttf,BoldFont=cmunbx.ttf,ItalicFont=cmunti.ttf,BoldItalicFont=cmunbi.ttf]{cmunrm.ttf}\setmonofont[Path=/usr/share/fonts/truetype/cmu/,UprightFont=cmuntt.ttf,BoldFont=cmuntb.ttf,ItalicFont=cmunit.ttf,BoldItalicFont=cmuntx.ttf]{cmunrm.ttf}\symbol{34}, \symbol{34}{\ttfamily \setmainfont[Path=/usr/share/fonts/truetype/cmu/,UprightFont=cmunrm.ttf,BoldFont=cmunbx.ttf,ItalicFont=cmunti.ttf,BoldItalicFont=cmunbi.ttf]{cmuntt.ttf}\setmonofont[Path=/usr/share/fonts/truetype/cmu/,UprightFont=cmuntt.ttf,BoldFont=cmuntb.ttf,ItalicFont=cmunit.ttf,BoldItalicFont=cmuntx.ttf]{cmuntt.ttf}\ttfamily loosely dotted}\setmainfont[Path=/usr/share/fonts/truetype/cmu/,UprightFont=cmunrm.ttf,BoldFont=cmunbx.ttf,ItalicFont=cmunti.ttf,BoldItalicFont=cmunbi.ttf]{cmunrm.ttf}\setmonofont[Path=/usr/share/fonts/truetype/cmu/,UprightFont=cmuntt.ttf,BoldFont=cmuntb.ttf,ItalicFont=cmunit.ttf,BoldItalicFont=cmuntx.ttf]{cmunrm.ttf}\symbol{34}, \symbol{34}{\ttfamily \setmainfont[Path=/usr/share/fonts/truetype/cmu/,UprightFont=cmunrm.ttf,BoldFont=cmunbx.ttf,ItalicFont=cmunti.ttf,BoldItalicFont=cmunbi.ttf]{cmuntt.ttf}\setmonofont[Path=/usr/share/fonts/truetype/cmu/,UprightFont=cmuntt.ttf,BoldFont=cmuntb.ttf,ItalicFont=cmunit.ttf,BoldItalicFont=cmuntx.ttf]{cmuntt.ttf}\ttfamily double}\setmainfont[Path=/usr/share/fonts/truetype/cmu/,UprightFont=cmunrm.ttf,BoldFont=cmunbx.ttf,ItalicFont=cmunti.ttf,BoldItalicFont=cmunbi.ttf]{cmunrm.ttf}\setmonofont[Path=/usr/share/fonts/truetype/cmu/,UprightFont=cmuntt.ttf,BoldFont=cmuntb.ttf,ItalicFont=cmunit.ttf,BoldItalicFont=cmuntx.ttf]{cmunrm.ttf}\symbol{34}.

Options for filling paths are e.g. \symbol{34}{\ttfamily \setmainfont[Path=/usr/share/fonts/truetype/cmu/,UprightFont=cmunrm.ttf,BoldFont=cmunbx.ttf,ItalicFont=cmunti.ttf,BoldItalicFont=cmunbi.ttf]{cmuntt.ttf}\setmonofont[Path=/usr/share/fonts/truetype/cmu/,UprightFont=cmuntt.ttf,BoldFont=cmuntb.ttf,ItalicFont=cmunit.ttf,BoldItalicFont=cmuntx.ttf]{cmuntt.ttf}\ttfamily fill=<{}fill color>{}}\setmainfont[Path=/usr/share/fonts/truetype/cmu/,UprightFont=cmunrm.ttf,BoldFont=cmunbx.ttf,ItalicFont=cmunti.ttf,BoldItalicFont=cmunbi.ttf]{cmunrm.ttf}\setmonofont[Path=/usr/share/fonts/truetype/cmu/,UprightFont=cmuntt.ttf,BoldFont=cmuntb.ttf,ItalicFont=cmunit.ttf,BoldItalicFont=cmuntx.ttf]{cmunrm.ttf}\symbol{34}, \symbol{34}{\ttfamily \setmainfont[Path=/usr/share/fonts/truetype/cmu/,UprightFont=cmunrm.ttf,BoldFont=cmunbx.ttf,ItalicFont=cmunti.ttf,BoldItalicFont=cmunbi.ttf]{cmuntt.ttf}\setmonofont[Path=/usr/share/fonts/truetype/cmu/,UprightFont=cmuntt.ttf,BoldFont=cmuntb.ttf,ItalicFont=cmunit.ttf,BoldItalicFont=cmuntx.ttf]{cmuntt.ttf}\ttfamily pattern=<{}name>{}}\setmainfont[Path=/usr/share/fonts/truetype/cmu/,UprightFont=cmunrm.ttf,BoldFont=cmunbx.ttf,ItalicFont=cmunti.ttf,BoldItalicFont=cmunbi.ttf]{cmunrm.ttf}\setmonofont[Path=/usr/share/fonts/truetype/cmu/,UprightFont=cmuntt.ttf,BoldFont=cmuntb.ttf,ItalicFont=cmunit.ttf,BoldItalicFont=cmuntx.ttf]{cmunrm.ttf}\symbol{34}, \symbol{34}{\ttfamily \setmainfont[Path=/usr/share/fonts/truetype/cmu/,UprightFont=cmunrm.ttf,BoldFont=cmunbx.ttf,ItalicFont=cmunti.ttf,BoldItalicFont=cmunbi.ttf]{cmuntt.ttf}\setmonofont[Path=/usr/share/fonts/truetype/cmu/,UprightFont=cmuntt.ttf,BoldFont=cmuntb.ttf,ItalicFont=cmunit.ttf,BoldItalicFont=cmuntx.ttf]{cmuntt.ttf}\ttfamily pattern color=<{}color>{}}\setmainfont[Path=/usr/share/fonts/truetype/cmu/,UprightFont=cmunrm.ttf,BoldFont=cmunbx.ttf,ItalicFont=cmunti.ttf,BoldItalicFont=cmunbi.ttf]{cmunrm.ttf}\setmonofont[Path=/usr/share/fonts/truetype/cmu/,UprightFont=cmuntt.ttf,BoldFont=cmuntb.ttf,ItalicFont=cmunit.ttf,BoldItalicFont=cmuntx.ttf]{cmunrm.ttf}\symbol{34}
\section{Drawing straight lines}
\label{803}

Straight lines are given by coordinates separated by a double minus, 
\begin{longtable}{p{1.0\linewidth}}
\begin{Shaded}
\begin{Highlighting}[]

\NormalTok{\textbackslash{}draw (1,0) -- (0,0) -- (0,1);}
\end{Highlighting}
\end{Shaded}
\\



\begin{minipage}{1.0\linewidth}
\begin{center}
\includegraphics[width=1.0\linewidth,height=6.5in,keepaspectratio]{../images/175.\SVGExtension}
\end{center}
\raggedright{}\myfigurewithoutcaption{175}
\end{minipage}\vspace{0.75cm}



\end{longtable}
The first coordinate represents a move-{}to operation. This is followed by a series of “path extension operations”, like \symbol{34}{\ttfamily \setmainfont[Path=/usr/share/fonts/truetype/cmu/,UprightFont=cmunrm.ttf,BoldFont=cmunbx.ttf,ItalicFont=cmunti.ttf,BoldItalicFont=cmunbi.ttf]{cmuntt.ttf}\setmonofont[Path=/usr/share/fonts/truetype/cmu/,UprightFont=cmuntt.ttf,BoldFont=cmuntb.ttf,ItalicFont=cmunit.ttf,BoldItalicFont=cmuntx.ttf]{cmuntt.ttf}\ttfamily -{}-{} (coordinates)}\setmainfont[Path=/usr/share/fonts/truetype/cmu/,UprightFont=cmunrm.ttf,BoldFont=cmunbx.ttf,ItalicFont=cmunti.ttf,BoldItalicFont=cmunbi.ttf]{cmunrm.ttf}\setmonofont[Path=/usr/share/fonts/truetype/cmu/,UprightFont=cmuntt.ttf,BoldFont=cmuntb.ttf,ItalicFont=cmunit.ttf,BoldItalicFont=cmuntx.ttf]{cmunrm.ttf}\symbol{34}.

The same path with some drawing options:
\begin{longtable}{p{1.0\linewidth}}
\begin{Shaded}
\begin{Highlighting}[]

\NormalTok{\textbackslash{}draw[red, dashed, very thick, rotate=30] (1,0) -- (0,0) -- (0,1);}
\end{Highlighting}
\end{Shaded}
\\



\begin{minipage}{1.0\linewidth}
\begin{center}
\includegraphics[width=1.0\linewidth,height=6.5in,keepaspectratio]{../images/176.\SVGExtension}
\end{center}
\raggedright{}\myfigurewithoutcaption{176}
\end{minipage}\vspace{0.75cm}



\end{longtable}

A connected path can be closed using the \symbol{34}{\ttfamily \setmainfont[Path=/usr/share/fonts/truetype/cmu/,UprightFont=cmunrm.ttf,BoldFont=cmunbx.ttf,ItalicFont=cmunti.ttf,BoldItalicFont=cmunbi.ttf]{cmuntt.ttf}\setmonofont[Path=/usr/share/fonts/truetype/cmu/,UprightFont=cmuntt.ttf,BoldFont=cmuntb.ttf,ItalicFont=cmunit.ttf,BoldItalicFont=cmuntx.ttf]{cmuntt.ttf}\ttfamily -{}-{}cycle}\setmainfont[Path=/usr/share/fonts/truetype/cmu/,UprightFont=cmunrm.ttf,BoldFont=cmunbx.ttf,ItalicFont=cmunti.ttf,BoldItalicFont=cmunbi.ttf]{cmunrm.ttf}\setmonofont[Path=/usr/share/fonts/truetype/cmu/,UprightFont=cmuntt.ttf,BoldFont=cmuntb.ttf,ItalicFont=cmunit.ttf,BoldItalicFont=cmuntx.ttf]{cmunrm.ttf}\symbol{34} operation:
\begin{longtable}{p{1.0\linewidth}}
\begin{Shaded}
\begin{Highlighting}[]

\NormalTok{\textbackslash{}draw (1,0) -- (0,0) -- (0,1) -- cycle;}
\end{Highlighting}
\end{Shaded}
\\



\begin{minipage}{1.0\linewidth}
\begin{center}
\includegraphics[width=1.0\linewidth,height=6.5in,keepaspectratio]{../images/177.\SVGExtension}
\end{center}
\raggedright{}\myfigurewithoutcaption{177}
\end{minipage}\vspace{0.75cm}



\end{longtable}

A further move-{}to operation in an existing path starts a new part of the path, which is not connected to the previous part of the path. Here: Move to (0,0) straight line to (2,0), move to (0,1) straight line to (2,1):
\begin{longtable}{p{1.0\linewidth}}
\begin{Shaded}
\begin{Highlighting}[]

\NormalTok{\textbackslash{}draw (0,0) -- (2,0) (0,1) -- (2,1);}
\end{Highlighting}
\end{Shaded}
\\



\begin{minipage}{1.0\linewidth}
\begin{center}
\includegraphics[width=1.0\linewidth,height=6.5in,keepaspectratio]{../images/178.\SVGExtension}
\end{center}
\raggedright{}\myfigurewithoutcaption{178}
\end{minipage}\vspace{0.75cm}



\end{longtable}

Two points can be connected by straight lines that are only horizontal and vertical.  For a connection that is first horizontal and then vertical, use
\begin{longtable}{p{1.0\linewidth}}
\begin{Shaded}
\begin{Highlighting}[]

\NormalTok{\textbackslash{}draw (0,0) - (1,1);}
\end{Highlighting}
\end{Shaded}
\\



\begin{minipage}{1.0\linewidth}
\begin{center}
\includegraphics[width=1.0\linewidth,height=6.5in,keepaspectratio]{../images/179.\SVGExtension}
\end{center}
\raggedright{}\myfigurewithoutcaption{179}
\end{minipage}\vspace{0.75cm}



\end{longtable}
or first vertical then horizontal, use
\begin{longtable}{p{1.0\linewidth}}
\begin{Shaded}
\begin{Highlighting}[]

\NormalTok{\textbackslash{}draw (0,0) - (1,1);}
\end{Highlighting}
\end{Shaded}
\\



\begin{minipage}{1.0\linewidth}
\begin{center}
\includegraphics[width=1.0\linewidth,height=6.5in,keepaspectratio]{../images/180.\SVGExtension}
\end{center}
\raggedright{}\myfigurewithoutcaption{180}
\end{minipage}\vspace{0.75cm}



\end{longtable}
\section{Drawing curved paths}
\label{804}

Curved paths using a Bezier curve can be created using the \symbol{34}{\ttfamily \setmainfont[Path=/usr/share/fonts/truetype/cmu/,UprightFont=cmunrm.ttf,BoldFont=cmunbx.ttf,ItalicFont=cmunti.ttf,BoldItalicFont=cmunbi.ttf]{cmuntt.ttf}\setmonofont[Path=/usr/share/fonts/truetype/cmu/,UprightFont=cmuntt.ttf,BoldFont=cmuntb.ttf,ItalicFont=cmunit.ttf,BoldItalicFont=cmuntx.ttf]{cmuntt.ttf}\ttfamily ..controls() ..()}\setmainfont[Path=/usr/share/fonts/truetype/cmu/,UprightFont=cmunrm.ttf,BoldFont=cmunbx.ttf,ItalicFont=cmunti.ttf,BoldItalicFont=cmunbi.ttf]{cmunrm.ttf}\setmonofont[Path=/usr/share/fonts/truetype/cmu/,UprightFont=cmuntt.ttf,BoldFont=cmuntb.ttf,ItalicFont=cmunit.ttf,BoldItalicFont=cmuntx.ttf]{cmunrm.ttf}\symbol{34} command, with one or two control points.
\begin{longtable}{p{1.0\linewidth}}
\begin{Shaded}
\begin{Highlighting}[]

\NormalTok{\textbackslash{}draw (0,0) .. controls (1,1) .. (4,0)}
      \NormalTok{(5,0) .. controls (6,0) and (6,1) .. (5,2);}
\end{Highlighting}
\end{Shaded}
\\



\begin{minipage}{1.0\linewidth}
\begin{center}
\includegraphics[width=1.0\linewidth,height=6.5in,keepaspectratio]{../images/181.\SVGExtension}
\end{center}
\raggedright{}\myfigurewithoutcaption{181}
\end{minipage}\vspace{0.75cm}



\end{longtable}
\section{User-{}defined paths}
\label{805}

User-{}defined paths can be created using the \symbol{34}{\ttfamily \setmainfont[Path=/usr/share/fonts/truetype/cmu/,UprightFont=cmunrm.ttf,BoldFont=cmunbx.ttf,ItalicFont=cmunti.ttf,BoldItalicFont=cmunbi.ttf]{cmuntt.ttf}\setmonofont[Path=/usr/share/fonts/truetype/cmu/,UprightFont=cmuntt.ttf,BoldFont=cmuntb.ttf,ItalicFont=cmunit.ttf,BoldItalicFont=cmuntx.ttf]{cmuntt.ttf}\ttfamily to}\setmainfont[Path=/usr/share/fonts/truetype/cmu/,UprightFont=cmunrm.ttf,BoldFont=cmunbx.ttf,ItalicFont=cmunti.ttf,BoldItalicFont=cmunbi.ttf]{cmunrm.ttf}\setmonofont[Path=/usr/share/fonts/truetype/cmu/,UprightFont=cmuntt.ttf,BoldFont=cmuntb.ttf,ItalicFont=cmunit.ttf,BoldItalicFont=cmuntx.ttf]{cmunrm.ttf}\symbol{34} operation. Without an option it corresponds to a straight line, exactly like the double minus command. Using the \symbol{34}{\ttfamily \setmainfont[Path=/usr/share/fonts/truetype/cmu/,UprightFont=cmunrm.ttf,BoldFont=cmunbx.ttf,ItalicFont=cmunti.ttf,BoldItalicFont=cmunbi.ttf]{cmuntt.ttf}\setmonofont[Path=/usr/share/fonts/truetype/cmu/,UprightFont=cmuntt.ttf,BoldFont=cmuntb.ttf,ItalicFont=cmunit.ttf,BoldItalicFont=cmuntx.ttf]{cmuntt.ttf}\ttfamily out}\setmainfont[Path=/usr/share/fonts/truetype/cmu/,UprightFont=cmunrm.ttf,BoldFont=cmunbx.ttf,ItalicFont=cmunti.ttf,BoldItalicFont=cmunbi.ttf]{cmunrm.ttf}\setmonofont[Path=/usr/share/fonts/truetype/cmu/,UprightFont=cmuntt.ttf,BoldFont=cmuntb.ttf,ItalicFont=cmunit.ttf,BoldItalicFont=cmuntx.ttf]{cmunrm.ttf}\symbol{34} and \symbol{34}{\ttfamily \setmainfont[Path=/usr/share/fonts/truetype/cmu/,UprightFont=cmunrm.ttf,BoldFont=cmunbx.ttf,ItalicFont=cmunti.ttf,BoldItalicFont=cmunbi.ttf]{cmuntt.ttf}\setmonofont[Path=/usr/share/fonts/truetype/cmu/,UprightFont=cmuntt.ttf,BoldFont=cmuntb.ttf,ItalicFont=cmunit.ttf,BoldItalicFont=cmuntx.ttf]{cmuntt.ttf}\ttfamily in}\setmainfont[Path=/usr/share/fonts/truetype/cmu/,UprightFont=cmunrm.ttf,BoldFont=cmunbx.ttf,ItalicFont=cmunti.ttf,BoldItalicFont=cmunbi.ttf]{cmunrm.ttf}\setmonofont[Path=/usr/share/fonts/truetype/cmu/,UprightFont=cmuntt.ttf,BoldFont=cmuntb.ttf,ItalicFont=cmunit.ttf,BoldItalicFont=cmuntx.ttf]{cmunrm.ttf}\symbol{34} option a curved path can created. E.g. \symbol{34}{\ttfamily \setmainfont[Path=/usr/share/fonts/truetype/cmu/,UprightFont=cmunrm.ttf,BoldFont=cmunbx.ttf,ItalicFont=cmunti.ttf,BoldItalicFont=cmunbi.ttf]{cmuntt.ttf}\setmonofont[Path=/usr/share/fonts/truetype/cmu/,UprightFont=cmuntt.ttf,BoldFont=cmuntb.ttf,ItalicFont=cmunit.ttf,BoldItalicFont=cmuntx.ttf]{cmuntt.ttf}\ttfamily {$\text{[}$}out=135,in=45{$\text{]}$}}\setmainfont[Path=/usr/share/fonts/truetype/cmu/,UprightFont=cmunrm.ttf,BoldFont=cmunbx.ttf,ItalicFont=cmunti.ttf,BoldItalicFont=cmunbi.ttf]{cmunrm.ttf}\setmonofont[Path=/usr/share/fonts/truetype/cmu/,UprightFont=cmuntt.ttf,BoldFont=cmuntb.ttf,ItalicFont=cmunit.ttf,BoldItalicFont=cmuntx.ttf]{cmunrm.ttf}\symbol{34} causes the path to leave at an angle of 135 degree at the first coordinate and arrive at an angle of 45 degree at the second coordinate.
\begin{longtable}{p{1.0\linewidth}}
\begin{Shaded}
\begin{Highlighting}[]

\NormalTok{\textbackslash{}draw (0,0) to (3,2);}
\NormalTok{\textbackslash{}draw (0,0) to[out=90,in=180] (3,2);}
\NormalTok{\textbackslash{}draw (0,0) to[bend right] (3,2);}
\end{Highlighting}
\end{Shaded}
\\



\begin{minipage}{1.0\linewidth}
\begin{center}
\includegraphics[width=1.0\linewidth,height=6.5in,keepaspectratio]{../images/182.\SVGExtension}
\end{center}
\raggedright{}\myfigurewithoutcaption{182}
\end{minipage}\vspace{0.75cm}



\end{longtable}
(The syntax for a bend to the right may seem a little counter-{}intuitive.  Think of it as an instruction to veer to the right at the beginning of the path and then smoothly curve to the end point, not as saying that the path curves to the right throughout its length.)

For rectangles a special syntax exist. Use a move-{}to operation to one corner and after \symbol{34}{\ttfamily \setmainfont[Path=/usr/share/fonts/truetype/cmu/,UprightFont=cmunrm.ttf,BoldFont=cmunbx.ttf,ItalicFont=cmunti.ttf,BoldItalicFont=cmunbi.ttf]{cmuntt.ttf}\setmonofont[Path=/usr/share/fonts/truetype/cmu/,UprightFont=cmuntt.ttf,BoldFont=cmuntb.ttf,ItalicFont=cmunit.ttf,BoldItalicFont=cmuntx.ttf]{cmuntt.ttf}\ttfamily rectangle}\setmainfont[Path=/usr/share/fonts/truetype/cmu/,UprightFont=cmunrm.ttf,BoldFont=cmunbx.ttf,ItalicFont=cmunti.ttf,BoldItalicFont=cmunbi.ttf]{cmunrm.ttf}\setmonofont[Path=/usr/share/fonts/truetype/cmu/,UprightFont=cmuntt.ttf,BoldFont=cmuntb.ttf,ItalicFont=cmunit.ttf,BoldItalicFont=cmuntx.ttf]{cmunrm.ttf}\symbol{34} the coordinates of the diagonal corner. The last one becomes the new current point.
\begin{longtable}{p{1.0\linewidth}}
\begin{Shaded}
\begin{Highlighting}[]

\NormalTok{\textbackslash{}draw (0,0) rectangle (1,1);}
\NormalTok{\textbackslash{}shade[top color=yellow, bottom color=black] (0,0) rectangle (2,-1);}
\NormalTok{\textbackslash{}filldraw[fill=green!20!white, draw=green!40!black] (0,0) rectangle (2,1);}
\end{Highlighting}
\end{Shaded}
\\



\begin{minipage}{1.0\linewidth}
\begin{center}
\includegraphics[width=1.0\linewidth,height=6.5in,keepaspectratio]{../images/183.\SVGExtension}
\end{center}
\raggedright{}\myfigurewithoutcaption{183}
\end{minipage}\vspace{0.75cm}



\end{longtable}
The fill color \symbol{34}{\ttfamily \setmainfont[Path=/usr/share/fonts/truetype/cmu/,UprightFont=cmunrm.ttf,BoldFont=cmunbx.ttf,ItalicFont=cmunti.ttf,BoldItalicFont=cmunbi.ttf]{cmuntt.ttf}\setmonofont[Path=/usr/share/fonts/truetype/cmu/,UprightFont=cmuntt.ttf,BoldFont=cmuntb.ttf,ItalicFont=cmunit.ttf,BoldItalicFont=cmuntx.ttf]{cmuntt.ttf}\ttfamily green!20!white}\setmainfont[Path=/usr/share/fonts/truetype/cmu/,UprightFont=cmunrm.ttf,BoldFont=cmunbx.ttf,ItalicFont=cmunti.ttf,BoldItalicFont=cmunbi.ttf]{cmunrm.ttf}\setmonofont[Path=/usr/share/fonts/truetype/cmu/,UprightFont=cmuntt.ttf,BoldFont=cmuntb.ttf,ItalicFont=cmunit.ttf,BoldItalicFont=cmuntx.ttf]{cmunrm.ttf}\symbol{34} means 20\% green and 80\% white mixed together.
\section{Circles, ellipses}
\label{806}
Circles and ellipses paths are defined beginning with their center then using the \symbol{34}{\ttfamily \setmainfont[Path=/usr/share/fonts/truetype/cmu/,UprightFont=cmunrm.ttf,BoldFont=cmunbx.ttf,ItalicFont=cmunti.ttf,BoldItalicFont=cmunbi.ttf]{cmuntt.ttf}\setmonofont[Path=/usr/share/fonts/truetype/cmu/,UprightFont=cmuntt.ttf,BoldFont=cmuntb.ttf,ItalicFont=cmunit.ttf,BoldItalicFont=cmuntx.ttf]{cmuntt.ttf}\ttfamily circle command}\setmainfont[Path=/usr/share/fonts/truetype/cmu/,UprightFont=cmunrm.ttf,BoldFont=cmunbx.ttf,ItalicFont=cmunti.ttf,BoldItalicFont=cmunbi.ttf]{cmunrm.ttf}\setmonofont[Path=/usr/share/fonts/truetype/cmu/,UprightFont=cmuntt.ttf,BoldFont=cmuntb.ttf,ItalicFont=cmunit.ttf,BoldItalicFont=cmuntx.ttf]{cmunrm.ttf}\symbol{34} either with one length as radius of a circle or with two lengths as semi-{}axes of an ellipse.
\begin{longtable}{p{1.0\linewidth}}
\begin{Shaded}
\begin{Highlighting}[]

\NormalTok{\textbackslash{}draw (0,0) circle [radius=1.5];}
\NormalTok{\textbackslash{}draw (0,0) circle (2cm); }\CommentTok{% old syntax}
\NormalTok{\textbackslash{}draw (0,0) circle [x radius=1.5cm, y radius=10mm];}
\NormalTok{\textbackslash{}draw (0,0) circle (1.2cm and 8mm); }\CommentTok{% old syntax}
\NormalTok{\textbackslash{}draw (0,0) circle [x radius=1cm, y radius=5mm, rotate=30];}
\NormalTok{\textbackslash{}draw[rotate=30] (0,0) ellipse (20pt and 10pt);  }\CommentTok{% old syntax}
\end{Highlighting}
\end{Shaded}
\\



\begin{minipage}{1.0\linewidth}
\begin{center}
\includegraphics[width=1.0\linewidth,height=6.5in,keepaspectratio]{../images/184.\SVGExtension}
\end{center}
\raggedright{}\myfigurewithoutcaption{184}
\end{minipage}\vspace{0.75cm}



\end{longtable}
\section{Arcs}
\label{807}

The command \symbol{34}{\ttfamily \setmainfont[Path=/usr/share/fonts/truetype/cmu/,UprightFont=cmunrm.ttf,BoldFont=cmunbx.ttf,ItalicFont=cmunti.ttf,BoldItalicFont=cmunbi.ttf]{cmuntt.ttf}\setmonofont[Path=/usr/share/fonts/truetype/cmu/,UprightFont=cmuntt.ttf,BoldFont=cmuntb.ttf,ItalicFont=cmunit.ttf,BoldItalicFont=cmuntx.ttf]{cmuntt.ttf}\ttfamily arc}\setmainfont[Path=/usr/share/fonts/truetype/cmu/,UprightFont=cmunrm.ttf,BoldFont=cmunbx.ttf,ItalicFont=cmunti.ttf,BoldItalicFont=cmunbi.ttf]{cmunrm.ttf}\setmonofont[Path=/usr/share/fonts/truetype/cmu/,UprightFont=cmuntt.ttf,BoldFont=cmuntb.ttf,ItalicFont=cmunit.ttf,BoldItalicFont=cmuntx.ttf]{cmunrm.ttf}\symbol{34} creates a part of a circle or an ellipse:
\begin{longtable}{p{1.0\linewidth}}
\begin{Shaded}
\begin{Highlighting}[]

\NormalTok{\textbackslash{}draw (0,0) arc (0:270:8mm);}
\NormalTok{\textbackslash{}draw (0,0) arc (0:315:1.75cm and 1cm);}
\NormalTok{\textbackslash{}filldraw[fill=cyan, draw=blue] (0,0) -- (12mm,0mm) arc (0:30:12mm) -- (0,0);}
\end{Highlighting}
\end{Shaded}
\\



\begin{minipage}{1.0\linewidth}
\begin{center}
\includegraphics[width=1.0\linewidth,height=6.5in,keepaspectratio]{../images/185.\SVGExtension}
\end{center}
\raggedright{}\myfigurewithoutcaption{185}
\end{minipage}\vspace{0.75cm}



\end{longtable}

Or in an alternative syntax:


\begin{Shaded}
\begin{Highlighting}[]

\NormalTok{\textbackslash{}draw\ensuremath{\text{ }}(0,0)\ensuremath{\text{ }}\ensuremath{\text{ }}arc[radius\ensuremath{\text{ }}=\ensuremath{\text{ }}8mm,\ensuremath{\text{ }}start\ensuremath{\text{ }}angle=\ensuremath{\text{ }}0,\ensuremath{\text{ }}end\ensuremath{\text{ }}angle=\ensuremath{\text{ }}270];}\newline
\NormalTok{\textbackslash{}draw\ensuremath{\text{ }}(0,0)\ensuremath{\text{ }}\ensuremath{\text{ }}arc[x\ensuremath{\text{ }}radius\ensuremath{\text{ }}=\ensuremath{\text{ }}1.75cm,\ensuremath{\text{ }}y\ensuremath{\text{ }}radius\ensuremath{\text{ }}=\ensuremath{\text{ }}1cm,\ensuremath{\text{ }}start\ensuremath{\text{ }}angle=\ensuremath{\text{ }}0,\ensuremath{\text{ }}end\ensuremath{\text{ }}angle=}\newline
\ensuremath{\text{ }}\NormalTok{315];}\newline
\end{Highlighting}
\end{Shaded}

\section{Special curves}
\label{808}

There are many more predefined commands for special paths, like \symbol{34}{\ttfamily \setmainfont[Path=/usr/share/fonts/truetype/cmu/,UprightFont=cmunrm.ttf,BoldFont=cmunbx.ttf,ItalicFont=cmunti.ttf,BoldItalicFont=cmunbi.ttf]{cmuntt.ttf}\setmonofont[Path=/usr/share/fonts/truetype/cmu/,UprightFont=cmuntt.ttf,BoldFont=cmuntb.ttf,ItalicFont=cmunit.ttf,BoldItalicFont=cmuntx.ttf]{cmuntt.ttf}\ttfamily grid}\setmainfont[Path=/usr/share/fonts/truetype/cmu/,UprightFont=cmunrm.ttf,BoldFont=cmunbx.ttf,ItalicFont=cmunti.ttf,BoldItalicFont=cmunbi.ttf]{cmunrm.ttf}\setmonofont[Path=/usr/share/fonts/truetype/cmu/,UprightFont=cmuntt.ttf,BoldFont=cmuntb.ttf,ItalicFont=cmunit.ttf,BoldItalicFont=cmuntx.ttf]{cmunrm.ttf}\symbol{34}, \symbol{34}{\ttfamily \setmainfont[Path=/usr/share/fonts/truetype/cmu/,UprightFont=cmunrm.ttf,BoldFont=cmunbx.ttf,ItalicFont=cmunti.ttf,BoldItalicFont=cmunbi.ttf]{cmuntt.ttf}\setmonofont[Path=/usr/share/fonts/truetype/cmu/,UprightFont=cmuntt.ttf,BoldFont=cmuntb.ttf,ItalicFont=cmunit.ttf,BoldItalicFont=cmuntx.ttf]{cmuntt.ttf}\ttfamily parabola}\setmainfont[Path=/usr/share/fonts/truetype/cmu/,UprightFont=cmunrm.ttf,BoldFont=cmunbx.ttf,ItalicFont=cmunti.ttf,BoldItalicFont=cmunbi.ttf]{cmunrm.ttf}\setmonofont[Path=/usr/share/fonts/truetype/cmu/,UprightFont=cmuntt.ttf,BoldFont=cmuntb.ttf,ItalicFont=cmunit.ttf,BoldItalicFont=cmuntx.ttf]{cmunrm.ttf}\symbol{34}, \symbol{34}{\ttfamily \setmainfont[Path=/usr/share/fonts/truetype/cmu/,UprightFont=cmunrm.ttf,BoldFont=cmunbx.ttf,ItalicFont=cmunti.ttf,BoldItalicFont=cmunbi.ttf]{cmuntt.ttf}\setmonofont[Path=/usr/share/fonts/truetype/cmu/,UprightFont=cmuntt.ttf,BoldFont=cmuntb.ttf,ItalicFont=cmunit.ttf,BoldItalicFont=cmuntx.ttf]{cmuntt.ttf}\ttfamily sin}\setmainfont[Path=/usr/share/fonts/truetype/cmu/,UprightFont=cmunrm.ttf,BoldFont=cmunbx.ttf,ItalicFont=cmunti.ttf,BoldItalicFont=cmunbi.ttf]{cmunrm.ttf}\setmonofont[Path=/usr/share/fonts/truetype/cmu/,UprightFont=cmuntt.ttf,BoldFont=cmuntb.ttf,ItalicFont=cmunit.ttf,BoldItalicFont=cmuntx.ttf]{cmunrm.ttf}\symbol{34}, \symbol{34}{\ttfamily \setmainfont[Path=/usr/share/fonts/truetype/cmu/,UprightFont=cmunrm.ttf,BoldFont=cmunbx.ttf,ItalicFont=cmunti.ttf,BoldItalicFont=cmunbi.ttf]{cmuntt.ttf}\setmonofont[Path=/usr/share/fonts/truetype/cmu/,UprightFont=cmuntt.ttf,BoldFont=cmuntb.ttf,ItalicFont=cmunit.ttf,BoldItalicFont=cmuntx.ttf]{cmuntt.ttf}\ttfamily cos}\setmainfont[Path=/usr/share/fonts/truetype/cmu/,UprightFont=cmunrm.ttf,BoldFont=cmunbx.ttf,ItalicFont=cmunti.ttf,BoldItalicFont=cmunbi.ttf]{cmunrm.ttf}\setmonofont[Path=/usr/share/fonts/truetype/cmu/,UprightFont=cmuntt.ttf,BoldFont=cmuntb.ttf,ItalicFont=cmunit.ttf,BoldItalicFont=cmuntx.ttf]{cmunrm.ttf}\symbol{34} (sine or cosine curve in the interval {$\text{[}$}0,π/2{$\text{]}$}).
\begin{longtable}{p{1.0\linewidth}}
\begin{Shaded}
\begin{Highlighting}[]

\NormalTok{\textbackslash{}draw[help lines] (0,0) grid (2,3);}
\NormalTok{\textbackslash{}draw[step=0.5, gray, very thin] (-1.4,-1.4) grid (1.4,1.4);}
\NormalTok{\textbackslash{}draw (0,0) parabola (1,1.5) parabola[bend at end] (2,0);}
\NormalTok{\textbackslash{}draw (0,0) sin (1,1) cos (2,0) sin (3,-1) cos (4,0) sin (5,1);}
\end{Highlighting}
\end{Shaded}
\\



\begin{minipage}{1.0\linewidth}
\begin{center}
\includegraphics[width=1.0\linewidth,height=6.5in,keepaspectratio]{../images/186.\SVGExtension}
\end{center}
\raggedright{}\myfigurewithoutcaption{186}
\end{minipage}\vspace{0.75cm}



\end{longtable}
The option \symbol{34}help lines\symbol{34} denotes \symbol{34}fine gray\symbol{34}.

To add arrow tips there are simple options for the drawing command:
\begin{longtable}{p{1.0\linewidth}}
\begin{Shaded}
\begin{Highlighting}[]

\NormalTok{\textbackslash{}draw [->] (0,0) -- (30:20pt); }
\NormalTok{\textbackslash{}draw [<->] (1,0) arc (180:30:10pt); }
\NormalTok{\textbackslash{}draw [<<->] (2,0) -- ++(0.5,10pt) -- ++(0.5,-10pt) -- ++(0.5,10pt);}
\end{Highlighting}
\end{Shaded}
\\



\begin{minipage}{1.0\linewidth}
\begin{center}
\includegraphics[width=1.0\linewidth,height=6.5in,keepaspectratio]{../images/187.\SVGExtension}
\end{center}
\raggedright{}\myfigurewithoutcaption{187}
\end{minipage}\vspace{0.75cm}



\end{longtable}

A loop can be realized by \symbol{34}{\ttfamily \setmainfont[Path=/usr/share/fonts/truetype/cmu/,UprightFont=cmunrm.ttf,BoldFont=cmunbx.ttf,ItalicFont=cmunti.ttf,BoldItalicFont=cmunbi.ttf]{cmuntt.ttf}\setmonofont[Path=/usr/share/fonts/truetype/cmu/,UprightFont=cmuntt.ttf,BoldFont=cmuntb.ttf,ItalicFont=cmunit.ttf,BoldItalicFont=cmuntx.ttf]{cmuntt.ttf}\ttfamily \textbackslash{}foreach \setmainfont[Path=/usr/share/fonts/truetype/freefont/,UprightFont=FreeSerif.ttf,BoldFont=FreeSerifBold.ttf,ItalicFont=FreeSerifItalic.ttf,BoldItalicFont=FreeSerifBoldItalic.ttf]{FreeMono.ttf}\setmonofont[Path=/usr/share/fonts/truetype/freefont/,UprightFont=FreeMono.ttf,BoldFont=FreeMonoBold.ttf,ItalicFont=FreeMonoOblique.ttf,BoldItalicFont=FreeMonoBoldOblique.ttf]{FreeMono.ttf}\ttfamily ⟨\setmainfont[Path=/usr/share/fonts/truetype/cmu/,UprightFont=cmunrm.ttf,BoldFont=cmunbx.ttf,ItalicFont=cmunti.ttf,BoldItalicFont=cmunbi.ttf]{cmuntt.ttf}\setmonofont[Path=/usr/share/fonts/truetype/cmu/,UprightFont=cmuntt.ttf,BoldFont=cmuntb.ttf,ItalicFont=cmunit.ttf,BoldItalicFont=cmuntx.ttf]{cmuntt.ttf}\ttfamily variable\setmainfont[Path=/usr/share/fonts/truetype/freefont/,UprightFont=FreeSerif.ttf,BoldFont=FreeSerifBold.ttf,ItalicFont=FreeSerifItalic.ttf,BoldItalicFont=FreeSerifBoldItalic.ttf]{FreeMono.ttf}\setmonofont[Path=/usr/share/fonts/truetype/freefont/,UprightFont=FreeMono.ttf,BoldFont=FreeMonoBold.ttf,ItalicFont=FreeMonoOblique.ttf,BoldItalicFont=FreeMonoBoldOblique.ttf]{FreeMono.ttf}\ttfamily ⟩{$\text{ }$}\setmainfont[Path=/usr/share/fonts/truetype/cmu/,UprightFont=cmunrm.ttf,BoldFont=cmunbx.ttf,ItalicFont=cmunti.ttf,BoldItalicFont=cmunbi.ttf]{cmuntt.ttf}\setmonofont[Path=/usr/share/fonts/truetype/cmu/,UprightFont=cmuntt.ttf,BoldFont=cmuntb.ttf,ItalicFont=cmunit.ttf,BoldItalicFont=cmuntx.ttf]{cmuntt.ttf}\ttfamily  in \{\setmainfont[Path=/usr/share/fonts/truetype/freefont/,UprightFont=FreeSerif.ttf,BoldFont=FreeSerifBold.ttf,ItalicFont=FreeSerifItalic.ttf,BoldItalicFont=FreeSerifBoldItalic.ttf]{FreeMono.ttf}\setmonofont[Path=/usr/share/fonts/truetype/freefont/,UprightFont=FreeMono.ttf,BoldFont=FreeMonoBold.ttf,ItalicFont=FreeMonoOblique.ttf,BoldItalicFont=FreeMonoBoldOblique.ttf]{FreeMono.ttf}\ttfamily ⟨\setmainfont[Path=/usr/share/fonts/truetype/cmu/,UprightFont=cmunrm.ttf,BoldFont=cmunbx.ttf,ItalicFont=cmunti.ttf,BoldItalicFont=cmunbi.ttf]{cmuntt.ttf}\setmonofont[Path=/usr/share/fonts/truetype/cmu/,UprightFont=cmuntt.ttf,BoldFont=cmuntb.ttf,ItalicFont=cmunit.ttf,BoldItalicFont=cmuntx.ttf]{cmuntt.ttf}\ttfamily list of values\setmainfont[Path=/usr/share/fonts/truetype/freefont/,UprightFont=FreeSerif.ttf,BoldFont=FreeSerifBold.ttf,ItalicFont=FreeSerifItalic.ttf,BoldItalicFont=FreeSerifBoldItalic.ttf]{FreeMono.ttf}\setmonofont[Path=/usr/share/fonts/truetype/freefont/,UprightFont=FreeMono.ttf,BoldFont=FreeMonoBold.ttf,ItalicFont=FreeMonoOblique.ttf,BoldItalicFont=FreeMonoBoldOblique.ttf]{FreeMono.ttf}\ttfamily ⟩\setmainfont[Path=/usr/share/fonts/truetype/cmu/,UprightFont=cmunrm.ttf,BoldFont=cmunbx.ttf,ItalicFont=cmunti.ttf,BoldItalicFont=cmunbi.ttf]{cmuntt.ttf}\setmonofont[Path=/usr/share/fonts/truetype/cmu/,UprightFont=cmuntt.ttf,BoldFont=cmuntb.ttf,ItalicFont=cmunit.ttf,BoldItalicFont=cmuntx.ttf]{cmuntt.ttf}\ttfamily \} \setmainfont[Path=/usr/share/fonts/truetype/freefont/,UprightFont=FreeSerif.ttf,BoldFont=FreeSerifBold.ttf,ItalicFont=FreeSerifItalic.ttf,BoldItalicFont=FreeSerifBoldItalic.ttf]{FreeMono.ttf}\setmonofont[Path=/usr/share/fonts/truetype/freefont/,UprightFont=FreeMono.ttf,BoldFont=FreeMonoBold.ttf,ItalicFont=FreeMonoOblique.ttf,BoldItalicFont=FreeMonoBoldOblique.ttf]{FreeMono.ttf}\ttfamily ⟨\setmainfont[Path=/usr/share/fonts/truetype/cmu/,UprightFont=cmunrm.ttf,BoldFont=cmunbx.ttf,ItalicFont=cmunti.ttf,BoldItalicFont=cmunbi.ttf]{cmuntt.ttf}\setmonofont[Path=/usr/share/fonts/truetype/cmu/,UprightFont=cmuntt.ttf,BoldFont=cmuntb.ttf,ItalicFont=cmunit.ttf,BoldItalicFont=cmuntx.ttf]{cmuntt.ttf}\ttfamily commands\setmainfont[Path=/usr/share/fonts/truetype/freefont/,UprightFont=FreeSerif.ttf,BoldFont=FreeSerifBold.ttf,ItalicFont=FreeSerifItalic.ttf,BoldItalicFont=FreeSerifBoldItalic.ttf]{FreeMono.ttf}\setmonofont[Path=/usr/share/fonts/truetype/freefont/,UprightFont=FreeMono.ttf,BoldFont=FreeMonoBold.ttf,ItalicFont=FreeMonoOblique.ttf,BoldItalicFont=FreeMonoBoldOblique.ttf]{FreeMono.ttf}\ttfamily ⟩}\setmainfont[Path=/usr/share/fonts/truetype/cmu/,UprightFont=cmunrm.ttf,BoldFont=cmunbx.ttf,ItalicFont=cmunti.ttf,BoldItalicFont=cmunbi.ttf]{cmunrm.ttf}\setmonofont[Path=/usr/share/fonts/truetype/cmu/,UprightFont=cmuntt.ttf,BoldFont=cmuntb.ttf,ItalicFont=cmunit.ttf,BoldItalicFont=cmuntx.ttf]{cmunrm.ttf}\symbol{34}.
\begin{longtable}{p{1.0\linewidth}}
\begin{Shaded}
\begin{Highlighting}[]

\NormalTok{\textbackslash{}foreach \textbackslash{}x in \{0,...,9\} }
  \NormalTok{\textbackslash{}draw (\textbackslash{}x,0) circle (0.4);}
\end{Highlighting}
\end{Shaded}
\\



\begin{minipage}{1.0\linewidth}
\begin{center}
\includegraphics[width=1.0\linewidth,height=6.5in,keepaspectratio]{../images/188.\SVGExtension}
\end{center}
\raggedright{}\myfigurewithoutcaption{188}
\end{minipage}\vspace{0.75cm}



\end{longtable}

PGF also has a math engine which enables you to plot functions:

\begin{Shaded}
\begin{Highlighting}[]

\NormalTok{\textbackslash{}draw\ensuremath{\text{ }}[domain=<xmin>:<xmax>]\ensuremath{\text{ }}plot\ensuremath{\text{ }}(\textbackslash{}x,\ensuremath{\text{ }}\{function\});}\newline
\end{Highlighting}
\end{Shaded}

Many functions are possible, including factorial(\textbackslash{}x), sqrt(\textbackslash{}x), pow(\textbackslash{}x,y), exp(\textbackslash{}x), ln(\textbackslash{}x), log10(\textbackslash{}x), log2(\textbackslash{}x), abs(\textbackslash{}x), mod(\textbackslash{}x,y), round(\textbackslash{}x), floor(\textbackslash{}x), ceil(\textbackslash{}x), sin(\textbackslash{}x), cos(\textbackslash{}x), tan(x), min(\textbackslash{}x,y,), and max(\textbackslash{}x,y). The trigonometric functions assume that x is in degrees; to express x in radians follow it with the notation \symbol{34}r\symbol{34}, e.g., sin(\textbackslash{}x r). Two useful constants are e, which is equal to 2.718281828, and pi, which is equal to 3.141592654.

An example with two functions:
\begin{longtable}{p{1.0\linewidth}}
\begin{Shaded}
\begin{Highlighting}[]

\NormalTok{\textbackslash{}draw [help lines] (-2,0) grid (2,4); }
\NormalTok{\textbackslash{}draw [->] (-2.2,0) -- (2.2,0); }
\NormalTok{\textbackslash{}draw [->] (0,0) -- (0,4.2); }
\NormalTok{\textbackslash{}draw [green, thick, domain=-2:2] plot (\textbackslash{}x, \{4-\textbackslash{}x*\textbackslash{}x\}); }
\NormalTok{\textbackslash{}draw [domain=-2:2, samples=50] plot (\textbackslash{}x, \{1+cos(pi*\textbackslash{}x r\});}
\end{Highlighting}
\end{Shaded}
\\



\begin{minipage}{1.0\linewidth}
\begin{center}
\includegraphics[width=1.0\linewidth,height=6.5in,keepaspectratio]{../images/189.\SVGExtension}
\end{center}
\raggedright{}\myfigurewithoutcaption{189}
\end{minipage}\vspace{0.75cm}



\end{longtable}
\section{Nodes}
\label{809}

A node is typically a rectangle or circle or another simple shape with some text on it. In the simplest case, a node is just some text that is placed at some coordinate.
Nodes are not part of the path itself, they are added to the picture after the path has been drawn.

Inside a path operation use the following syntax after a given coordinate:

\begin{Shaded}
\begin{Highlighting}[]

\NormalTok{node[<options>](<name>)\{<text>\}}\newline
\end{Highlighting}
\end{Shaded}

The \symbol{34}{\ttfamily \setmainfont[Path=/usr/share/fonts/truetype/cmu/,UprightFont=cmunrm.ttf,BoldFont=cmunbx.ttf,ItalicFont=cmunti.ttf,BoldItalicFont=cmunbi.ttf]{cmuntt.ttf}\setmonofont[Path=/usr/share/fonts/truetype/cmu/,UprightFont=cmuntt.ttf,BoldFont=cmuntb.ttf,ItalicFont=cmunit.ttf,BoldItalicFont=cmuntx.ttf]{cmuntt.ttf}\ttfamily (<{}name>{})}\setmainfont[Path=/usr/share/fonts/truetype/cmu/,UprightFont=cmunrm.ttf,BoldFont=cmunbx.ttf,ItalicFont=cmunti.ttf,BoldItalicFont=cmunbi.ttf]{cmunrm.ttf}\setmonofont[Path=/usr/share/fonts/truetype/cmu/,UprightFont=cmuntt.ttf,BoldFont=cmuntb.ttf,ItalicFont=cmunit.ttf,BoldItalicFont=cmuntx.ttf]{cmunrm.ttf}\symbol{34} is a name for later reference and it is optional. 
If you only want to name a certain position without writing text there are two possibilities:

\begin{Shaded}
\begin{Highlighting}[]

\NormalTok{node[<options>](<name>)\{\}}\newline
\NormalTok{coordinate[<options>](<name>)}\newline
\end{Highlighting}
\end{Shaded}


Writing text along a given path using the node command is shown as a simple example:
\begin{longtable}{p{1.0\linewidth}}
\begin{Shaded}
\begin{Highlighting}[]

\NormalTok{\textbackslash{}draw[dotted]}
    \NormalTok{(0,0) node \{1st node\}}
 \NormalTok{-- (1,1) node \{2nd node\}}
 \NormalTok{-- (0,2) node \{3rd node\}}
 \NormalTok{-- cycle;}
\end{Highlighting}
\end{Shaded}
\\



\begin{minipage}{1.0\linewidth}
\begin{center}
\includegraphics[width=1.0\linewidth,height=6.5in,keepaspectratio]{../images/190.\SVGExtension}
\end{center}
\raggedright{}\myfigurewithoutcaption{190}
\end{minipage}\vspace{0.75cm}



\end{longtable}

Possible options for the node command are e.g. \symbol{34}{\ttfamily \setmainfont[Path=/usr/share/fonts/truetype/cmu/,UprightFont=cmunrm.ttf,BoldFont=cmunbx.ttf,ItalicFont=cmunti.ttf,BoldItalicFont=cmunbi.ttf]{cmuntt.ttf}\setmonofont[Path=/usr/share/fonts/truetype/cmu/,UprightFont=cmuntt.ttf,BoldFont=cmuntb.ttf,ItalicFont=cmunit.ttf,BoldItalicFont=cmuntx.ttf]{cmuntt.ttf}\ttfamily inner sep=<{}dimension>{}}\setmainfont[Path=/usr/share/fonts/truetype/cmu/,UprightFont=cmunrm.ttf,BoldFont=cmunbx.ttf,ItalicFont=cmunti.ttf,BoldItalicFont=cmunbi.ttf]{cmunrm.ttf}\setmonofont[Path=/usr/share/fonts/truetype/cmu/,UprightFont=cmuntt.ttf,BoldFont=cmuntb.ttf,ItalicFont=cmunit.ttf,BoldItalicFont=cmuntx.ttf]{cmunrm.ttf}\symbol{34}, \symbol{34}{\ttfamily \setmainfont[Path=/usr/share/fonts/truetype/cmu/,UprightFont=cmunrm.ttf,BoldFont=cmunbx.ttf,ItalicFont=cmunti.ttf,BoldItalicFont=cmunbi.ttf]{cmuntt.ttf}\setmonofont[Path=/usr/share/fonts/truetype/cmu/,UprightFont=cmuntt.ttf,BoldFont=cmuntb.ttf,ItalicFont=cmunit.ttf,BoldItalicFont=cmuntx.ttf]{cmuntt.ttf}\ttfamily outer sep=<{}dimension>{}}\setmainfont[Path=/usr/share/fonts/truetype/cmu/,UprightFont=cmunrm.ttf,BoldFont=cmunbx.ttf,ItalicFont=cmunti.ttf,BoldItalicFont=cmunbi.ttf]{cmunrm.ttf}\setmonofont[Path=/usr/share/fonts/truetype/cmu/,UprightFont=cmuntt.ttf,BoldFont=cmuntb.ttf,ItalicFont=cmunit.ttf,BoldItalicFont=cmuntx.ttf]{cmunrm.ttf}\symbol{34}, \symbol{34}{\ttfamily \setmainfont[Path=/usr/share/fonts/truetype/cmu/,UprightFont=cmunrm.ttf,BoldFont=cmunbx.ttf,ItalicFont=cmunti.ttf,BoldItalicFont=cmunbi.ttf]{cmuntt.ttf}\setmonofont[Path=/usr/share/fonts/truetype/cmu/,UprightFont=cmuntt.ttf,BoldFont=cmuntb.ttf,ItalicFont=cmunit.ttf,BoldItalicFont=cmuntx.ttf]{cmuntt.ttf}\ttfamily minimum size=<{}dimension>{}}\setmainfont[Path=/usr/share/fonts/truetype/cmu/,UprightFont=cmunrm.ttf,BoldFont=cmunbx.ttf,ItalicFont=cmunti.ttf,BoldItalicFont=cmunbi.ttf]{cmunrm.ttf}\setmonofont[Path=/usr/share/fonts/truetype/cmu/,UprightFont=cmuntt.ttf,BoldFont=cmuntb.ttf,ItalicFont=cmunit.ttf,BoldItalicFont=cmuntx.ttf]{cmunrm.ttf}\symbol{34}, \symbol{34}{\ttfamily \setmainfont[Path=/usr/share/fonts/truetype/cmu/,UprightFont=cmunrm.ttf,BoldFont=cmunbx.ttf,ItalicFont=cmunti.ttf,BoldItalicFont=cmunbi.ttf]{cmuntt.ttf}\setmonofont[Path=/usr/share/fonts/truetype/cmu/,UprightFont=cmuntt.ttf,BoldFont=cmuntb.ttf,ItalicFont=cmunit.ttf,BoldItalicFont=cmuntx.ttf]{cmuntt.ttf}\ttfamily shape aspect=<{}aspect ratio>{}}\setmainfont[Path=/usr/share/fonts/truetype/cmu/,UprightFont=cmunrm.ttf,BoldFont=cmunbx.ttf,ItalicFont=cmunti.ttf,BoldItalicFont=cmunbi.ttf]{cmunrm.ttf}\setmonofont[Path=/usr/share/fonts/truetype/cmu/,UprightFont=cmuntt.ttf,BoldFont=cmuntb.ttf,ItalicFont=cmunit.ttf,BoldItalicFont=cmuntx.ttf]{cmunrm.ttf}\symbol{34}, \symbol{34}{\ttfamily \setmainfont[Path=/usr/share/fonts/truetype/cmu/,UprightFont=cmunrm.ttf,BoldFont=cmunbx.ttf,ItalicFont=cmunti.ttf,BoldItalicFont=cmunbi.ttf]{cmuntt.ttf}\setmonofont[Path=/usr/share/fonts/truetype/cmu/,UprightFont=cmuntt.ttf,BoldFont=cmuntb.ttf,ItalicFont=cmunit.ttf,BoldItalicFont=cmuntx.ttf]{cmuntt.ttf}\ttfamily text=<{}color>{}}\setmainfont[Path=/usr/share/fonts/truetype/cmu/,UprightFont=cmunrm.ttf,BoldFont=cmunbx.ttf,ItalicFont=cmunti.ttf,BoldItalicFont=cmunbi.ttf]{cmunrm.ttf}\setmonofont[Path=/usr/share/fonts/truetype/cmu/,UprightFont=cmuntt.ttf,BoldFont=cmuntb.ttf,ItalicFont=cmunit.ttf,BoldItalicFont=cmuntx.ttf]{cmunrm.ttf}\symbol{34}, \symbol{34}{\ttfamily \setmainfont[Path=/usr/share/fonts/truetype/cmu/,UprightFont=cmunrm.ttf,BoldFont=cmunbx.ttf,ItalicFont=cmunti.ttf,BoldItalicFont=cmunbi.ttf]{cmuntt.ttf}\setmonofont[Path=/usr/share/fonts/truetype/cmu/,UprightFont=cmuntt.ttf,BoldFont=cmuntb.ttf,ItalicFont=cmunit.ttf,BoldItalicFont=cmuntx.ttf]{cmuntt.ttf}\ttfamily font=}\setmainfont[Path=/usr/share/fonts/truetype/cmu/,UprightFont=cmunrm.ttf,BoldFont=cmunbx.ttf,ItalicFont=cmunti.ttf,BoldItalicFont=cmunbi.ttf]{cmunrm.ttf}\setmonofont[Path=/usr/share/fonts/truetype/cmu/,UprightFont=cmuntt.ttf,BoldFont=cmuntb.ttf,ItalicFont=cmunit.ttf,BoldItalicFont=cmuntx.ttf]{cmunrm.ttf}\symbol{34}, \symbol{34}{\ttfamily \setmainfont[Path=/usr/share/fonts/truetype/cmu/,UprightFont=cmunrm.ttf,BoldFont=cmunbx.ttf,ItalicFont=cmunti.ttf,BoldItalicFont=cmunbi.ttf]{cmuntt.ttf}\setmonofont[Path=/usr/share/fonts/truetype/cmu/,UprightFont=cmuntt.ttf,BoldFont=cmuntb.ttf,ItalicFont=cmunit.ttf,BoldItalicFont=cmuntx.ttf]{cmuntt.ttf}\ttfamily align=<{}left\_right\_center>{}}\setmainfont[Path=/usr/share/fonts/truetype/cmu/,UprightFont=cmunrm.ttf,BoldFont=cmunbx.ttf,ItalicFont=cmunti.ttf,BoldItalicFont=cmunbi.ttf]{cmunrm.ttf}\setmonofont[Path=/usr/share/fonts/truetype/cmu/,UprightFont=cmuntt.ttf,BoldFont=cmuntb.ttf,ItalicFont=cmunit.ttf,BoldItalicFont=cmuntx.ttf]{cmunrm.ttf}\symbol{34}.

A node is centered at the current coordinate by default. Often it would be better to have the node placed beside the actual coordinate: Right (\symbol{34}{\ttfamily \setmainfont[Path=/usr/share/fonts/truetype/cmu/,UprightFont=cmunrm.ttf,BoldFont=cmunbx.ttf,ItalicFont=cmunti.ttf,BoldItalicFont=cmunbi.ttf]{cmuntt.ttf}\setmonofont[Path=/usr/share/fonts/truetype/cmu/,UprightFont=cmuntt.ttf,BoldFont=cmuntb.ttf,ItalicFont=cmunit.ttf,BoldItalicFont=cmuntx.ttf]{cmuntt.ttf}\ttfamily right}\setmainfont[Path=/usr/share/fonts/truetype/cmu/,UprightFont=cmunrm.ttf,BoldFont=cmunbx.ttf,ItalicFont=cmunti.ttf,BoldItalicFont=cmunbi.ttf]{cmunrm.ttf}\setmonofont[Path=/usr/share/fonts/truetype/cmu/,UprightFont=cmuntt.ttf,BoldFont=cmuntb.ttf,ItalicFont=cmunit.ttf,BoldItalicFont=cmuntx.ttf]{cmunrm.ttf}\symbol{34} or \symbol{34}{\ttfamily \setmainfont[Path=/usr/share/fonts/truetype/cmu/,UprightFont=cmunrm.ttf,BoldFont=cmunbx.ttf,ItalicFont=cmunti.ttf,BoldItalicFont=cmunbi.ttf]{cmuntt.ttf}\setmonofont[Path=/usr/share/fonts/truetype/cmu/,UprightFont=cmuntt.ttf,BoldFont=cmuntb.ttf,ItalicFont=cmunit.ttf,BoldItalicFont=cmuntx.ttf]{cmuntt.ttf}\ttfamily anchor=west}\setmainfont[Path=/usr/share/fonts/truetype/cmu/,UprightFont=cmunrm.ttf,BoldFont=cmunbx.ttf,ItalicFont=cmunti.ttf,BoldItalicFont=cmunbi.ttf]{cmunrm.ttf}\setmonofont[Path=/usr/share/fonts/truetype/cmu/,UprightFont=cmuntt.ttf,BoldFont=cmuntb.ttf,ItalicFont=cmunit.ttf,BoldItalicFont=cmuntx.ttf]{cmunrm.ttf}\symbol{34}), left (\symbol{34}{\ttfamily \setmainfont[Path=/usr/share/fonts/truetype/cmu/,UprightFont=cmunrm.ttf,BoldFont=cmunbx.ttf,ItalicFont=cmunti.ttf,BoldItalicFont=cmunbi.ttf]{cmuntt.ttf}\setmonofont[Path=/usr/share/fonts/truetype/cmu/,UprightFont=cmuntt.ttf,BoldFont=cmuntb.ttf,ItalicFont=cmunit.ttf,BoldItalicFont=cmuntx.ttf]{cmuntt.ttf}\ttfamily left}\setmainfont[Path=/usr/share/fonts/truetype/cmu/,UprightFont=cmunrm.ttf,BoldFont=cmunbx.ttf,ItalicFont=cmunti.ttf,BoldItalicFont=cmunbi.ttf]{cmunrm.ttf}\setmonofont[Path=/usr/share/fonts/truetype/cmu/,UprightFont=cmuntt.ttf,BoldFont=cmuntb.ttf,ItalicFont=cmunit.ttf,BoldItalicFont=cmuntx.ttf]{cmunrm.ttf}\symbol{34} or \symbol{34}{\ttfamily \setmainfont[Path=/usr/share/fonts/truetype/cmu/,UprightFont=cmunrm.ttf,BoldFont=cmunbx.ttf,ItalicFont=cmunti.ttf,BoldItalicFont=cmunbi.ttf]{cmuntt.ttf}\setmonofont[Path=/usr/share/fonts/truetype/cmu/,UprightFont=cmuntt.ttf,BoldFont=cmuntb.ttf,ItalicFont=cmunit.ttf,BoldItalicFont=cmuntx.ttf]{cmuntt.ttf}\ttfamily anchor=east}\setmainfont[Path=/usr/share/fonts/truetype/cmu/,UprightFont=cmunrm.ttf,BoldFont=cmunbx.ttf,ItalicFont=cmunti.ttf,BoldItalicFont=cmunbi.ttf]{cmunrm.ttf}\setmonofont[Path=/usr/share/fonts/truetype/cmu/,UprightFont=cmuntt.ttf,BoldFont=cmuntb.ttf,ItalicFont=cmunit.ttf,BoldItalicFont=cmuntx.ttf]{cmunrm.ttf}\symbol{34}), above (\symbol{34}{\ttfamily \setmainfont[Path=/usr/share/fonts/truetype/cmu/,UprightFont=cmunrm.ttf,BoldFont=cmunbx.ttf,ItalicFont=cmunti.ttf,BoldItalicFont=cmunbi.ttf]{cmuntt.ttf}\setmonofont[Path=/usr/share/fonts/truetype/cmu/,UprightFont=cmuntt.ttf,BoldFont=cmuntb.ttf,ItalicFont=cmunit.ttf,BoldItalicFont=cmuntx.ttf]{cmuntt.ttf}\ttfamily above}\setmainfont[Path=/usr/share/fonts/truetype/cmu/,UprightFont=cmunrm.ttf,BoldFont=cmunbx.ttf,ItalicFont=cmunti.ttf,BoldItalicFont=cmunbi.ttf]{cmunrm.ttf}\setmonofont[Path=/usr/share/fonts/truetype/cmu/,UprightFont=cmuntt.ttf,BoldFont=cmuntb.ttf,ItalicFont=cmunit.ttf,BoldItalicFont=cmuntx.ttf]{cmunrm.ttf}\symbol{34} or \symbol{34}{\ttfamily \setmainfont[Path=/usr/share/fonts/truetype/cmu/,UprightFont=cmunrm.ttf,BoldFont=cmunbx.ttf,ItalicFont=cmunti.ttf,BoldItalicFont=cmunbi.ttf]{cmuntt.ttf}\setmonofont[Path=/usr/share/fonts/truetype/cmu/,UprightFont=cmuntt.ttf,BoldFont=cmuntb.ttf,ItalicFont=cmunit.ttf,BoldItalicFont=cmuntx.ttf]{cmuntt.ttf}\ttfamily anchor=south}\setmainfont[Path=/usr/share/fonts/truetype/cmu/,UprightFont=cmunrm.ttf,BoldFont=cmunbx.ttf,ItalicFont=cmunti.ttf,BoldItalicFont=cmunbi.ttf]{cmunrm.ttf}\setmonofont[Path=/usr/share/fonts/truetype/cmu/,UprightFont=cmuntt.ttf,BoldFont=cmuntb.ttf,ItalicFont=cmunit.ttf,BoldItalicFont=cmuntx.ttf]{cmunrm.ttf}\symbol{34}), below (\symbol{34}{\ttfamily \setmainfont[Path=/usr/share/fonts/truetype/cmu/,UprightFont=cmunrm.ttf,BoldFont=cmunbx.ttf,ItalicFont=cmunti.ttf,BoldItalicFont=cmunbi.ttf]{cmuntt.ttf}\setmonofont[Path=/usr/share/fonts/truetype/cmu/,UprightFont=cmuntt.ttf,BoldFont=cmuntb.ttf,ItalicFont=cmunit.ttf,BoldItalicFont=cmuntx.ttf]{cmuntt.ttf}\ttfamily below}\setmainfont[Path=/usr/share/fonts/truetype/cmu/,UprightFont=cmunrm.ttf,BoldFont=cmunbx.ttf,ItalicFont=cmunti.ttf,BoldItalicFont=cmunbi.ttf]{cmunrm.ttf}\setmonofont[Path=/usr/share/fonts/truetype/cmu/,UprightFont=cmuntt.ttf,BoldFont=cmuntb.ttf,ItalicFont=cmunit.ttf,BoldItalicFont=cmuntx.ttf]{cmunrm.ttf}\symbol{34} or \symbol{34}{\ttfamily \setmainfont[Path=/usr/share/fonts/truetype/cmu/,UprightFont=cmunrm.ttf,BoldFont=cmunbx.ttf,ItalicFont=cmunti.ttf,BoldItalicFont=cmunbi.ttf]{cmuntt.ttf}\setmonofont[Path=/usr/share/fonts/truetype/cmu/,UprightFont=cmuntt.ttf,BoldFont=cmuntb.ttf,ItalicFont=cmunit.ttf,BoldItalicFont=cmuntx.ttf]{cmuntt.ttf}\ttfamily anchor=north}\setmainfont[Path=/usr/share/fonts/truetype/cmu/,UprightFont=cmunrm.ttf,BoldFont=cmunbx.ttf,ItalicFont=cmunti.ttf,BoldItalicFont=cmunbi.ttf]{cmunrm.ttf}\setmonofont[Path=/usr/share/fonts/truetype/cmu/,UprightFont=cmuntt.ttf,BoldFont=cmuntb.ttf,ItalicFont=cmunit.ttf,BoldItalicFont=cmuntx.ttf]{cmunrm.ttf}\symbol{34}). Combinations are also possible, like \symbol{34}{\ttfamily \setmainfont[Path=/usr/share/fonts/truetype/cmu/,UprightFont=cmunrm.ttf,BoldFont=cmunbx.ttf,ItalicFont=cmunti.ttf,BoldItalicFont=cmunbi.ttf]{cmuntt.ttf}\setmonofont[Path=/usr/share/fonts/truetype/cmu/,UprightFont=cmuntt.ttf,BoldFont=cmuntb.ttf,ItalicFont=cmunit.ttf,BoldItalicFont=cmuntx.ttf]{cmuntt.ttf}\ttfamily anchor=north east}\setmainfont[Path=/usr/share/fonts/truetype/cmu/,UprightFont=cmunrm.ttf,BoldFont=cmunbx.ttf,ItalicFont=cmunti.ttf,BoldItalicFont=cmunbi.ttf]{cmunrm.ttf}\setmonofont[Path=/usr/share/fonts/truetype/cmu/,UprightFont=cmuntt.ttf,BoldFont=cmuntb.ttf,ItalicFont=cmunit.ttf,BoldItalicFont=cmuntx.ttf]{cmunrm.ttf}\symbol{34} or \symbol{34}{\ttfamily \setmainfont[Path=/usr/share/fonts/truetype/cmu/,UprightFont=cmunrm.ttf,BoldFont=cmunbx.ttf,ItalicFont=cmunti.ttf,BoldItalicFont=cmunbi.ttf]{cmuntt.ttf}\setmonofont[Path=/usr/share/fonts/truetype/cmu/,UprightFont=cmuntt.ttf,BoldFont=cmuntb.ttf,ItalicFont=cmunit.ttf,BoldItalicFont=cmuntx.ttf]{cmuntt.ttf}\ttfamily below left}\setmainfont[Path=/usr/share/fonts/truetype/cmu/,UprightFont=cmunrm.ttf,BoldFont=cmunbx.ttf,ItalicFont=cmunti.ttf,BoldItalicFont=cmunbi.ttf]{cmunrm.ttf}\setmonofont[Path=/usr/share/fonts/truetype/cmu/,UprightFont=cmuntt.ttf,BoldFont=cmuntb.ttf,ItalicFont=cmunit.ttf,BoldItalicFont=cmuntx.ttf]{cmunrm.ttf}\symbol{34}.

\begin{longtable}{p{1.0\linewidth}}
\begin{Shaded}
\begin{Highlighting}[]

\NormalTok{\textbackslash{}fill[fill=yellow]}
    \NormalTok{(0,0) node \{1st node\}}
 \NormalTok{-- (1,1) node[circle,inner sep=0pt,draw] \{2nd node\}}
 \NormalTok{-- (0,2) node[fill=red!20,draw,double,rounded corners] \{3rd node\};}
\end{Highlighting}
\end{Shaded}
\\



\begin{minipage}{1.0\linewidth}
\begin{center}
\includegraphics[width=1.0\linewidth,height=6.5in,keepaspectratio]{../images/191.\SVGExtension}
\end{center}
\raggedright{}\myfigurewithoutcaption{191}
\end{minipage}\vspace{0.75cm}



\end{longtable}

To place nodes on a line or a curve use the \symbol{34}{\ttfamily \setmainfont[Path=/usr/share/fonts/truetype/cmu/,UprightFont=cmunrm.ttf,BoldFont=cmunbx.ttf,ItalicFont=cmunti.ttf,BoldItalicFont=cmunbi.ttf]{cmuntt.ttf}\setmonofont[Path=/usr/share/fonts/truetype/cmu/,UprightFont=cmuntt.ttf,BoldFont=cmuntb.ttf,ItalicFont=cmunit.ttf,BoldItalicFont=cmuntx.ttf]{cmuntt.ttf}\ttfamily pos=<{}fraction>{}}\setmainfont[Path=/usr/share/fonts/truetype/cmu/,UprightFont=cmunrm.ttf,BoldFont=cmunbx.ttf,ItalicFont=cmunti.ttf,BoldItalicFont=cmunbi.ttf]{cmunrm.ttf}\setmonofont[Path=/usr/share/fonts/truetype/cmu/,UprightFont=cmuntt.ttf,BoldFont=cmuntb.ttf,ItalicFont=cmunit.ttf,BoldItalicFont=cmuntx.ttf]{cmunrm.ttf}\symbol{34} option, where fraction is a floating point number between 0 representing the previous coordinate and 1 representing the current coordinate.
\begin{longtable}{p{1.0\linewidth}}
\begin{Shaded}
\begin{Highlighting}[]

\NormalTok{\textbackslash{}draw (0,0) -- (3,1)}
  \NormalTok{node[pos=0]\{0\} node[pos=0.5]\{1/2\} node[pos=0.9]\{9/10\};}
\end{Highlighting}
\end{Shaded}
\\



\begin{minipage}{1.0\linewidth}
\begin{center}
\includegraphics[width=1.0\linewidth,height=6.5in,keepaspectratio]{../images/192.\SVGExtension}
\end{center}
\raggedright{}\myfigurewithoutcaption{192}
\end{minipage}\vspace{0.75cm}



\end{longtable}
There exist some abbreviations: \symbol{34}{\ttfamily \setmainfont[Path=/usr/share/fonts/truetype/cmu/,UprightFont=cmunrm.ttf,BoldFont=cmunbx.ttf,ItalicFont=cmunti.ttf,BoldItalicFont=cmunbi.ttf]{cmuntt.ttf}\setmonofont[Path=/usr/share/fonts/truetype/cmu/,UprightFont=cmuntt.ttf,BoldFont=cmuntb.ttf,ItalicFont=cmunit.ttf,BoldItalicFont=cmuntx.ttf]{cmuntt.ttf}\ttfamily at start}\setmainfont[Path=/usr/share/fonts/truetype/cmu/,UprightFont=cmunrm.ttf,BoldFont=cmunbx.ttf,ItalicFont=cmunti.ttf,BoldItalicFont=cmunbi.ttf]{cmunrm.ttf}\setmonofont[Path=/usr/share/fonts/truetype/cmu/,UprightFont=cmuntt.ttf,BoldFont=cmuntb.ttf,ItalicFont=cmunit.ttf,BoldItalicFont=cmuntx.ttf]{cmunrm.ttf}\symbol{34} for \symbol{34}{\ttfamily \setmainfont[Path=/usr/share/fonts/truetype/cmu/,UprightFont=cmunrm.ttf,BoldFont=cmunbx.ttf,ItalicFont=cmunti.ttf,BoldItalicFont=cmunbi.ttf]{cmuntt.ttf}\setmonofont[Path=/usr/share/fonts/truetype/cmu/,UprightFont=cmuntt.ttf,BoldFont=cmuntb.ttf,ItalicFont=cmunit.ttf,BoldItalicFont=cmuntx.ttf]{cmuntt.ttf}\ttfamily pos=0}\setmainfont[Path=/usr/share/fonts/truetype/cmu/,UprightFont=cmunrm.ttf,BoldFont=cmunbx.ttf,ItalicFont=cmunti.ttf,BoldItalicFont=cmunbi.ttf]{cmunrm.ttf}\setmonofont[Path=/usr/share/fonts/truetype/cmu/,UprightFont=cmuntt.ttf,BoldFont=cmuntb.ttf,ItalicFont=cmunit.ttf,BoldItalicFont=cmuntx.ttf]{cmunrm.ttf}\symbol{34}, \symbol{34}{\ttfamily \setmainfont[Path=/usr/share/fonts/truetype/cmu/,UprightFont=cmunrm.ttf,BoldFont=cmunbx.ttf,ItalicFont=cmunti.ttf,BoldItalicFont=cmunbi.ttf]{cmuntt.ttf}\setmonofont[Path=/usr/share/fonts/truetype/cmu/,UprightFont=cmuntt.ttf,BoldFont=cmuntb.ttf,ItalicFont=cmunit.ttf,BoldItalicFont=cmuntx.ttf]{cmuntt.ttf}\ttfamily very near start}\setmainfont[Path=/usr/share/fonts/truetype/cmu/,UprightFont=cmunrm.ttf,BoldFont=cmunbx.ttf,ItalicFont=cmunti.ttf,BoldItalicFont=cmunbi.ttf]{cmunrm.ttf}\setmonofont[Path=/usr/share/fonts/truetype/cmu/,UprightFont=cmuntt.ttf,BoldFont=cmuntb.ttf,ItalicFont=cmunit.ttf,BoldItalicFont=cmuntx.ttf]{cmunrm.ttf}\symbol{34} for \symbol{34}{\ttfamily \setmainfont[Path=/usr/share/fonts/truetype/cmu/,UprightFont=cmunrm.ttf,BoldFont=cmunbx.ttf,ItalicFont=cmunti.ttf,BoldItalicFont=cmunbi.ttf]{cmuntt.ttf}\setmonofont[Path=/usr/share/fonts/truetype/cmu/,UprightFont=cmuntt.ttf,BoldFont=cmuntb.ttf,ItalicFont=cmunit.ttf,BoldItalicFont=cmuntx.ttf]{cmuntt.ttf}\ttfamily pos=0.125}\setmainfont[Path=/usr/share/fonts/truetype/cmu/,UprightFont=cmunrm.ttf,BoldFont=cmunbx.ttf,ItalicFont=cmunti.ttf,BoldItalicFont=cmunbi.ttf]{cmunrm.ttf}\setmonofont[Path=/usr/share/fonts/truetype/cmu/,UprightFont=cmuntt.ttf,BoldFont=cmuntb.ttf,ItalicFont=cmunit.ttf,BoldItalicFont=cmuntx.ttf]{cmunrm.ttf}\symbol{34}, \symbol{34}{\ttfamily \setmainfont[Path=/usr/share/fonts/truetype/cmu/,UprightFont=cmunrm.ttf,BoldFont=cmunbx.ttf,ItalicFont=cmunti.ttf,BoldItalicFont=cmunbi.ttf]{cmuntt.ttf}\setmonofont[Path=/usr/share/fonts/truetype/cmu/,UprightFont=cmuntt.ttf,BoldFont=cmuntb.ttf,ItalicFont=cmunit.ttf,BoldItalicFont=cmuntx.ttf]{cmuntt.ttf}\ttfamily near start}\setmainfont[Path=/usr/share/fonts/truetype/cmu/,UprightFont=cmunrm.ttf,BoldFont=cmunbx.ttf,ItalicFont=cmunti.ttf,BoldItalicFont=cmunbi.ttf]{cmunrm.ttf}\setmonofont[Path=/usr/share/fonts/truetype/cmu/,UprightFont=cmuntt.ttf,BoldFont=cmuntb.ttf,ItalicFont=cmunit.ttf,BoldItalicFont=cmuntx.ttf]{cmunrm.ttf}\symbol{34} for \symbol{34}{\ttfamily \setmainfont[Path=/usr/share/fonts/truetype/cmu/,UprightFont=cmunrm.ttf,BoldFont=cmunbx.ttf,ItalicFont=cmunti.ttf,BoldItalicFont=cmunbi.ttf]{cmuntt.ttf}\setmonofont[Path=/usr/share/fonts/truetype/cmu/,UprightFont=cmuntt.ttf,BoldFont=cmuntb.ttf,ItalicFont=cmunit.ttf,BoldItalicFont=cmuntx.ttf]{cmuntt.ttf}\ttfamily pos=0.25}\setmainfont[Path=/usr/share/fonts/truetype/cmu/,UprightFont=cmunrm.ttf,BoldFont=cmunbx.ttf,ItalicFont=cmunti.ttf,BoldItalicFont=cmunbi.ttf]{cmunrm.ttf}\setmonofont[Path=/usr/share/fonts/truetype/cmu/,UprightFont=cmuntt.ttf,BoldFont=cmuntb.ttf,ItalicFont=cmunit.ttf,BoldItalicFont=cmuntx.ttf]{cmunrm.ttf}\symbol{34}, \symbol{34}{\ttfamily \setmainfont[Path=/usr/share/fonts/truetype/cmu/,UprightFont=cmunrm.ttf,BoldFont=cmunbx.ttf,ItalicFont=cmunti.ttf,BoldItalicFont=cmunbi.ttf]{cmuntt.ttf}\setmonofont[Path=/usr/share/fonts/truetype/cmu/,UprightFont=cmuntt.ttf,BoldFont=cmuntb.ttf,ItalicFont=cmunit.ttf,BoldItalicFont=cmuntx.ttf]{cmuntt.ttf}\ttfamily midway}\setmainfont[Path=/usr/share/fonts/truetype/cmu/,UprightFont=cmunrm.ttf,BoldFont=cmunbx.ttf,ItalicFont=cmunti.ttf,BoldItalicFont=cmunbi.ttf]{cmunrm.ttf}\setmonofont[Path=/usr/share/fonts/truetype/cmu/,UprightFont=cmuntt.ttf,BoldFont=cmuntb.ttf,ItalicFont=cmunit.ttf,BoldItalicFont=cmuntx.ttf]{cmunrm.ttf}\symbol{34} for \symbol{34}{\ttfamily \setmainfont[Path=/usr/share/fonts/truetype/cmu/,UprightFont=cmunrm.ttf,BoldFont=cmunbx.ttf,ItalicFont=cmunti.ttf,BoldItalicFont=cmunbi.ttf]{cmuntt.ttf}\setmonofont[Path=/usr/share/fonts/truetype/cmu/,UprightFont=cmuntt.ttf,BoldFont=cmuntb.ttf,ItalicFont=cmunit.ttf,BoldItalicFont=cmuntx.ttf]{cmuntt.ttf}\ttfamily pos=0.5}\setmainfont[Path=/usr/share/fonts/truetype/cmu/,UprightFont=cmunrm.ttf,BoldFont=cmunbx.ttf,ItalicFont=cmunti.ttf,BoldItalicFont=cmunbi.ttf]{cmunrm.ttf}\setmonofont[Path=/usr/share/fonts/truetype/cmu/,UprightFont=cmuntt.ttf,BoldFont=cmuntb.ttf,ItalicFont=cmunit.ttf,BoldItalicFont=cmuntx.ttf]{cmunrm.ttf}\symbol{34}, \symbol{34}{\ttfamily \setmainfont[Path=/usr/share/fonts/truetype/cmu/,UprightFont=cmunrm.ttf,BoldFont=cmunbx.ttf,ItalicFont=cmunti.ttf,BoldItalicFont=cmunbi.ttf]{cmuntt.ttf}\setmonofont[Path=/usr/share/fonts/truetype/cmu/,UprightFont=cmuntt.ttf,BoldFont=cmuntb.ttf,ItalicFont=cmunit.ttf,BoldItalicFont=cmuntx.ttf]{cmuntt.ttf}\ttfamily near end}\setmainfont[Path=/usr/share/fonts/truetype/cmu/,UprightFont=cmunrm.ttf,BoldFont=cmunbx.ttf,ItalicFont=cmunti.ttf,BoldItalicFont=cmunbi.ttf]{cmunrm.ttf}\setmonofont[Path=/usr/share/fonts/truetype/cmu/,UprightFont=cmuntt.ttf,BoldFont=cmuntb.ttf,ItalicFont=cmunit.ttf,BoldItalicFont=cmuntx.ttf]{cmunrm.ttf}\symbol{34} for \symbol{34}{\ttfamily \setmainfont[Path=/usr/share/fonts/truetype/cmu/,UprightFont=cmunrm.ttf,BoldFont=cmunbx.ttf,ItalicFont=cmunti.ttf,BoldItalicFont=cmunbi.ttf]{cmuntt.ttf}\setmonofont[Path=/usr/share/fonts/truetype/cmu/,UprightFont=cmuntt.ttf,BoldFont=cmuntb.ttf,ItalicFont=cmunit.ttf,BoldItalicFont=cmuntx.ttf]{cmuntt.ttf}\ttfamily pos=0.75}\setmainfont[Path=/usr/share/fonts/truetype/cmu/,UprightFont=cmunrm.ttf,BoldFont=cmunbx.ttf,ItalicFont=cmunti.ttf,BoldItalicFont=cmunbi.ttf]{cmunrm.ttf}\setmonofont[Path=/usr/share/fonts/truetype/cmu/,UprightFont=cmuntt.ttf,BoldFont=cmuntb.ttf,ItalicFont=cmunit.ttf,BoldItalicFont=cmuntx.ttf]{cmunrm.ttf}\symbol{34}, \symbol{34}{\ttfamily \setmainfont[Path=/usr/share/fonts/truetype/cmu/,UprightFont=cmunrm.ttf,BoldFont=cmunbx.ttf,ItalicFont=cmunti.ttf,BoldItalicFont=cmunbi.ttf]{cmuntt.ttf}\setmonofont[Path=/usr/share/fonts/truetype/cmu/,UprightFont=cmuntt.ttf,BoldFont=cmuntb.ttf,ItalicFont=cmunit.ttf,BoldItalicFont=cmuntx.ttf]{cmuntt.ttf}\ttfamily very near end}\setmainfont[Path=/usr/share/fonts/truetype/cmu/,UprightFont=cmunrm.ttf,BoldFont=cmunbx.ttf,ItalicFont=cmunti.ttf,BoldItalicFont=cmunbi.ttf]{cmunrm.ttf}\setmonofont[Path=/usr/share/fonts/truetype/cmu/,UprightFont=cmuntt.ttf,BoldFont=cmuntb.ttf,ItalicFont=cmunit.ttf,BoldItalicFont=cmuntx.ttf]{cmunrm.ttf}\symbol{34} for \symbol{34}{\ttfamily \setmainfont[Path=/usr/share/fonts/truetype/cmu/,UprightFont=cmunrm.ttf,BoldFont=cmunbx.ttf,ItalicFont=cmunti.ttf,BoldItalicFont=cmunbi.ttf]{cmuntt.ttf}\setmonofont[Path=/usr/share/fonts/truetype/cmu/,UprightFont=cmuntt.ttf,BoldFont=cmuntb.ttf,ItalicFont=cmunit.ttf,BoldItalicFont=cmuntx.ttf]{cmuntt.ttf}\ttfamily pos=0.875}\setmainfont[Path=/usr/share/fonts/truetype/cmu/,UprightFont=cmunrm.ttf,BoldFont=cmunbx.ttf,ItalicFont=cmunti.ttf,BoldItalicFont=cmunbi.ttf]{cmunrm.ttf}\setmonofont[Path=/usr/share/fonts/truetype/cmu/,UprightFont=cmuntt.ttf,BoldFont=cmuntb.ttf,ItalicFont=cmunit.ttf,BoldItalicFont=cmuntx.ttf]{cmunrm.ttf}\symbol{34}, \symbol{34}{\ttfamily \setmainfont[Path=/usr/share/fonts/truetype/cmu/,UprightFont=cmunrm.ttf,BoldFont=cmunbx.ttf,ItalicFont=cmunti.ttf,BoldItalicFont=cmunbi.ttf]{cmuntt.ttf}\setmonofont[Path=/usr/share/fonts/truetype/cmu/,UprightFont=cmuntt.ttf,BoldFont=cmuntb.ttf,ItalicFont=cmunit.ttf,BoldItalicFont=cmuntx.ttf]{cmuntt.ttf}\ttfamily at end}\setmainfont[Path=/usr/share/fonts/truetype/cmu/,UprightFont=cmunrm.ttf,BoldFont=cmunbx.ttf,ItalicFont=cmunti.ttf,BoldItalicFont=cmunbi.ttf]{cmunrm.ttf}\setmonofont[Path=/usr/share/fonts/truetype/cmu/,UprightFont=cmuntt.ttf,BoldFont=cmuntb.ttf,ItalicFont=cmunit.ttf,BoldItalicFont=cmuntx.ttf]{cmunrm.ttf}\symbol{34} for \symbol{34}{\ttfamily \setmainfont[Path=/usr/share/fonts/truetype/cmu/,UprightFont=cmunrm.ttf,BoldFont=cmunbx.ttf,ItalicFont=cmunti.ttf,BoldItalicFont=cmunbi.ttf]{cmuntt.ttf}\setmonofont[Path=/usr/share/fonts/truetype/cmu/,UprightFont=cmuntt.ttf,BoldFont=cmuntb.ttf,ItalicFont=cmunit.ttf,BoldItalicFont=cmuntx.ttf]{cmuntt.ttf}\ttfamily pos=1}\setmainfont[Path=/usr/share/fonts/truetype/cmu/,UprightFont=cmunrm.ttf,BoldFont=cmunbx.ttf,ItalicFont=cmunti.ttf,BoldItalicFont=cmunbi.ttf]{cmunrm.ttf}\setmonofont[Path=/usr/share/fonts/truetype/cmu/,UprightFont=cmuntt.ttf,BoldFont=cmuntb.ttf,ItalicFont=cmunit.ttf,BoldItalicFont=cmuntx.ttf]{cmunrm.ttf}\symbol{34}.

The \symbol{34}{\ttfamily \setmainfont[Path=/usr/share/fonts/truetype/cmu/,UprightFont=cmunrm.ttf,BoldFont=cmunbx.ttf,ItalicFont=cmunti.ttf,BoldItalicFont=cmunbi.ttf]{cmuntt.ttf}\setmonofont[Path=/usr/share/fonts/truetype/cmu/,UprightFont=cmuntt.ttf,BoldFont=cmuntb.ttf,ItalicFont=cmunit.ttf,BoldItalicFont=cmuntx.ttf]{cmuntt.ttf}\ttfamily sloped}\setmainfont[Path=/usr/share/fonts/truetype/cmu/,UprightFont=cmunrm.ttf,BoldFont=cmunbx.ttf,ItalicFont=cmunti.ttf,BoldItalicFont=cmunbi.ttf]{cmunrm.ttf}\setmonofont[Path=/usr/share/fonts/truetype/cmu/,UprightFont=cmuntt.ttf,BoldFont=cmuntb.ttf,ItalicFont=cmunit.ttf,BoldItalicFont=cmuntx.ttf]{cmunrm.ttf}\symbol{34} option causes the node to be rotated to become a tangent to the curve.

Since nodes are often the only path operation on paths, there are special commands for creating
paths containing only a node, the first with text ouput, the second without:

\begin{Shaded}
\begin{Highlighting}[]

\NormalTok{\textbackslash{}node[<options>](<name>)\ensuremath{\text{ }}at\ensuremath{\text{ }}(<coordinate>)\{<text>\};}\newline
\NormalTok{\textbackslash{}coordinate[<options>](<name>)\ensuremath{\text{ }}at\ensuremath{\text{ }}(<coordinate>);}\newline
\end{Highlighting}
\end{Shaded}


One can connect nodes using the nodes\textquotesingle{} labels as coordinates. Having \symbol{34}{\ttfamily \setmainfont[Path=/usr/share/fonts/truetype/cmu/,UprightFont=cmunrm.ttf,BoldFont=cmunbx.ttf,ItalicFont=cmunti.ttf,BoldItalicFont=cmunbi.ttf]{cmuntt.ttf}\setmonofont[Path=/usr/share/fonts/truetype/cmu/,UprightFont=cmuntt.ttf,BoldFont=cmuntb.ttf,ItalicFont=cmunit.ttf,BoldItalicFont=cmuntx.ttf]{cmuntt.ttf}\ttfamily \textbackslash{}path(0,0) node(x) \{\} (3,1) node(y) \{\};}\setmainfont[Path=/usr/share/fonts/truetype/cmu/,UprightFont=cmunrm.ttf,BoldFont=cmunbx.ttf,ItalicFont=cmunti.ttf,BoldItalicFont=cmunbi.ttf]{cmunrm.ttf}\setmonofont[Path=/usr/share/fonts/truetype/cmu/,UprightFont=cmuntt.ttf,BoldFont=cmuntb.ttf,ItalicFont=cmunit.ttf,BoldItalicFont=cmuntx.ttf]{cmunrm.ttf}\symbol{34} defined, the node at (0,0) got the name \symbol{34}{\ttfamily \setmainfont[Path=/usr/share/fonts/truetype/cmu/,UprightFont=cmunrm.ttf,BoldFont=cmunbx.ttf,ItalicFont=cmunti.ttf,BoldItalicFont=cmunbi.ttf]{cmuntt.ttf}\setmonofont[Path=/usr/share/fonts/truetype/cmu/,UprightFont=cmuntt.ttf,BoldFont=cmuntb.ttf,ItalicFont=cmunit.ttf,BoldItalicFont=cmuntx.ttf]{cmuntt.ttf}\ttfamily (x)}\setmainfont[Path=/usr/share/fonts/truetype/cmu/,UprightFont=cmunrm.ttf,BoldFont=cmunbx.ttf,ItalicFont=cmunti.ttf,BoldItalicFont=cmunbi.ttf]{cmunrm.ttf}\setmonofont[Path=/usr/share/fonts/truetype/cmu/,UprightFont=cmuntt.ttf,BoldFont=cmuntb.ttf,ItalicFont=cmunit.ttf,BoldItalicFont=cmuntx.ttf]{cmunrm.ttf}\symbol{34} and the one at (3,1) got the name \symbol{34}{\ttfamily \setmainfont[Path=/usr/share/fonts/truetype/cmu/,UprightFont=cmunrm.ttf,BoldFont=cmunbx.ttf,ItalicFont=cmunti.ttf,BoldItalicFont=cmunbi.ttf]{cmuntt.ttf}\setmonofont[Path=/usr/share/fonts/truetype/cmu/,UprightFont=cmuntt.ttf,BoldFont=cmuntb.ttf,ItalicFont=cmunit.ttf,BoldItalicFont=cmuntx.ttf]{cmuntt.ttf}\ttfamily (y)}\setmainfont[Path=/usr/share/fonts/truetype/cmu/,UprightFont=cmunrm.ttf,BoldFont=cmunbx.ttf,ItalicFont=cmunti.ttf,BoldItalicFont=cmunbi.ttf]{cmunrm.ttf}\setmonofont[Path=/usr/share/fonts/truetype/cmu/,UprightFont=cmuntt.ttf,BoldFont=cmuntb.ttf,ItalicFont=cmunit.ttf,BoldItalicFont=cmuntx.ttf]{cmunrm.ttf}\symbol{34}. 
\begin{longtable}{p{1.0\linewidth}}
\begin{Shaded}
\begin{Highlighting}[]

\NormalTok{\textbackslash{}path (0,0) node(x) \{\} }
      \NormalTok{(3,1) node(y) \{\};}
\NormalTok{\textbackslash{}draw (x) -- (y);}
\end{Highlighting}
\end{Shaded}
\\



\begin{minipage}{1.0\linewidth}
\begin{center}
\includegraphics[width=1.0\linewidth,height=6.5in,keepaspectratio]{../images/193.\SVGExtension}
\end{center}
\raggedright{}\myfigurewithoutcaption{193}
\end{minipage}\vspace{0.75cm}



\end{longtable}

Equivalent to

\begin{Shaded}
\begin{Highlighting}[]

\NormalTok{\textbackslash{}coordinate\ensuremath{\text{ }}(x)\ensuremath{\text{ }}at\ensuremath{\text{ }}(0,0);\ensuremath{\text{ }}}\newline
\NormalTok{\textbackslash{}coordinate\ensuremath{\text{ }}(y)\ensuremath{\text{ }}at\ensuremath{\text{ }}(3,1);}\newline
\NormalTok{\textbackslash{}draw\ensuremath{\text{ }}(x)\ensuremath{\text{ }}--\ensuremath{\text{ }}(y);}\newline
\end{Highlighting}
\end{Shaded}


Multiline text can be included inside a node. A new line is indicated by double backslash \symbol{34}\textbackslash{}\textbackslash{}\symbol{34}, but additionally you have to specify the alignment using the node option \symbol{34}align=\symbol{34}. Here an example:

\begin{longtable}{p{1.0\linewidth}}
\begin{Shaded}
\begin{Highlighting}[]

\NormalTok{\textbackslash{}filldraw }
\NormalTok{(0,0) circle (2pt) node[align=left,   below] \{test 1\textbackslash{}\textbackslash{}is aligned left\} --}
\NormalTok{(4,0) circle (2pt) node[align=center, below] \{test 2\textbackslash{}\textbackslash{}is centered\}     -- }
\NormalTok{(8,0) circle (2pt) node[align=right,  below] \{test 3\textbackslash{}\textbackslash{}is right aligned\};}
\end{Highlighting}
\end{Shaded}
\\



\begin{minipage}{1.0\linewidth}
\begin{center}
\includegraphics[width=1.0\linewidth,height=6.5in,keepaspectratio]{../images/194.\SVGExtension}
\end{center}
\raggedright{}\myfigurewithoutcaption{194}
\end{minipage}\vspace{0.75cm}



\end{longtable}


Path construction operations try to be clever, such that the path starts at the border of the node\textquotesingle{}s shape and not from the node\textquotesingle{}s center.
\begin{longtable}{p{1.0\linewidth}}
\begin{Shaded}
\begin{Highlighting}[]

\NormalTok{\textbackslash{}path (0,0) node(x) \{Hello World!\}}
\NormalTok{(3,1) node[circle,draw](y) \{$\textbackslash{}int_1^2 x \textbackslash{}mathrm d x$\};}
\NormalTok{\textbackslash{}draw[->,blue] (x) -- (y);}
\NormalTok{\textbackslash{}draw[->,red] (x) - node[near start,below] \{label\} (y);}
\NormalTok{\textbackslash{}draw[->,orange] (x) .. controls +(up:1cm) and +(left:1cm) .. node[above,sloped]}
 \NormalTok{\{label\} (y);}
\end{Highlighting}
\end{Shaded}
\\



\begin{minipage}{1.0\linewidth}
\begin{center}
\includegraphics[width=1.0\linewidth,height=6.5in,keepaspectratio]{../images/195.\SVGExtension}
\end{center}
\raggedright{}\myfigurewithoutcaption{195}
\end{minipage}\vspace{0.75cm}



\end{longtable}

Once the node x has been defined, you can use anchors as defined above relative to (x) as \symbol{34}{\ttfamily \setmainfont[Path=/usr/share/fonts/truetype/cmu/,UprightFont=cmunrm.ttf,BoldFont=cmunbx.ttf,ItalicFont=cmunti.ttf,BoldItalicFont=cmunbi.ttf]{cmuntt.ttf}\setmonofont[Path=/usr/share/fonts/truetype/cmu/,UprightFont=cmuntt.ttf,BoldFont=cmuntb.ttf,ItalicFont=cmunit.ttf,BoldItalicFont=cmuntx.ttf]{cmuntt.ttf}\ttfamily (x.<{}anchor>{})}\setmainfont[Path=/usr/share/fonts/truetype/cmu/,UprightFont=cmunrm.ttf,BoldFont=cmunbx.ttf,ItalicFont=cmunti.ttf,BoldItalicFont=cmunbi.ttf]{cmunrm.ttf}\setmonofont[Path=/usr/share/fonts/truetype/cmu/,UprightFont=cmuntt.ttf,BoldFont=cmuntb.ttf,ItalicFont=cmunit.ttf,BoldItalicFont=cmuntx.ttf]{cmunrm.ttf}\symbol{34}, like \symbol{34}{\ttfamily \setmainfont[Path=/usr/share/fonts/truetype/cmu/,UprightFont=cmunrm.ttf,BoldFont=cmunbx.ttf,ItalicFont=cmunti.ttf,BoldItalicFont=cmunbi.ttf]{cmuntt.ttf}\setmonofont[Path=/usr/share/fonts/truetype/cmu/,UprightFont=cmuntt.ttf,BoldFont=cmuntb.ttf,ItalicFont=cmunit.ttf,BoldItalicFont=cmuntx.ttf]{cmuntt.ttf}\ttfamily (x.north)}\setmainfont[Path=/usr/share/fonts/truetype/cmu/,UprightFont=cmunrm.ttf,BoldFont=cmunbx.ttf,ItalicFont=cmunti.ttf,BoldItalicFont=cmunbi.ttf]{cmunrm.ttf}\setmonofont[Path=/usr/share/fonts/truetype/cmu/,UprightFont=cmuntt.ttf,BoldFont=cmuntb.ttf,ItalicFont=cmunit.ttf,BoldItalicFont=cmuntx.ttf]{cmunrm.ttf}\symbol{34}.
\section{Examples}
\label{810}\subsection{Example 1}
\label{811}
\begin{longtable}{p{1.0\linewidth}}
\begin{Shaded}
\begin{Highlighting}[]

\NormalTok{\textbackslash{}documentclass\{article\}}
\NormalTok{\textbackslash{}usepackage\{tikz\}}
\NormalTok{\textbackslash{}begin\{document\}}
  \NormalTok{\textbackslash{}begin\{tikzpicture\}}
  \NormalTok{\textbackslash{}draw[thick,rounded corners=8pt] (0,0) -- (0,2) -- (1,3.25) }
   \NormalTok{-- (2,2) -- (2,0) -- (0,2) -- (2,2) -- (0,0) -- (2,0);}
  \NormalTok{\textbackslash{}end\{tikzpicture\}}
\NormalTok{\textbackslash{}end\{document\}}
\end{Highlighting}
\end{Shaded}
\\



\begin{minipage}{1.0\linewidth}
\begin{center}
\includegraphics[width=1.0\linewidth,height=6.5in,keepaspectratio]{../images/196.\SVGExtension}
\end{center}
\raggedright{}\myfigurewithoutcaption{196}
\end{minipage}\vspace{0.75cm}



\end{longtable}\subsection{Example 2}
\label{812}
\begin{longtable}{p{1.0\linewidth}}
\begin{Shaded}
\begin{Highlighting}[]

\NormalTok{\textbackslash{}documentclass\{article\}}
\NormalTok{\textbackslash{}usepackage\{tikz\}}
\NormalTok{\textbackslash{}begin\{document\}}
 \NormalTok{\textbackslash{}begin\{tikzpicture\}[scale=3]}
 \NormalTok{\textbackslash{}draw[step=.5cm, gray, very thin] (-1.2,-1.2) grid (1.2,1.2); }
 \NormalTok{\textbackslash{}filldraw[fill=green!20,draw=green!50!black] (0,0) -- (3mm,0mm) arc (0:30:3mm) -- cycle; }
 \NormalTok{\textbackslash{}draw[->] (-1.25,0) -- (1.25,0) coordinate (x axis);}
 \NormalTok{\textbackslash{}draw[->] (0,-1.25) -- (0,1.25) coordinate (y axis);}
 \NormalTok{\textbackslash{}draw (0,0) circle (1cm);}
 \NormalTok{\textbackslash{}draw[very thick,red] (30:1cm) -- node[left,fill=white] \{$\textbackslash{}sin \textbackslash{}alpha$\} (30:1cm - x axis);}
 \NormalTok{\textbackslash{}draw[very thick,blue] (30:1cm - x axis) -- node[below=2pt,fill=white] \{$\textbackslash{}cos \textbackslash{}alpha$\} (0,0);}
 \NormalTok{\textbackslash{}draw (0,0) -- (30:1cm);}
 \NormalTok{\textbackslash{}foreach \textbackslash{}x/\textbackslash{}xtext in \{-1, -0.5/-\textbackslash{}frac\{1\}\{2\}, 1\} }
   \NormalTok{\textbackslash{}draw (\textbackslash{}x cm,1pt) -- (\textbackslash{}x cm,-1pt) node[anchor=north,fill=white] \{$\textbackslash{}xtext$\};}
 \NormalTok{\textbackslash{}foreach \textbackslash{}y/\textbackslash{}ytext in \{-1, -0.5/-\textbackslash{}frac\{1\}\{2\}, 0.5/\textbackslash{}frac\{1\}\{2\}, 1\} }
   \NormalTok{\textbackslash{}draw (1pt,\textbackslash{}y cm) -- (-1pt,\textbackslash{}y cm) node[anchor=east,fill=white] \{$\textbackslash{}ytext$\};}
 \NormalTok{\textbackslash{}end\{tikzpicture\}}
\NormalTok{\textbackslash{}end\{document\}}
\end{Highlighting}
\end{Shaded}
\\



\begin{minipage}{1.0\linewidth}
\begin{center}
\includegraphics[width=1.0\linewidth,height=6.5in,keepaspectratio]{../images/197.\SVGExtension}
\end{center}
\raggedright{}\myfigurewithoutcaption{197}
\end{minipage}\vspace{0.75cm}



\end{longtable}
\subsection{Example 3: A Torus}
\label{813}
\begin{longtable}{p{1.0\linewidth}}
\begin{Shaded}
\begin{Highlighting}[]

\NormalTok{\textbackslash{}documentclass\{article\}}
\NormalTok{\textbackslash{}usepackage\{tikz\}}
\NormalTok{\textbackslash{}begin\{document\}}
 \NormalTok{\textbackslash{}begin\{tikzpicture\}}
  \NormalTok{\textbackslash{}draw (-1,0) to[bend left] (1,0);}
  \NormalTok{\textbackslash{}draw (-1.2,.1) to[bend right] (1.2,.1);}
  \NormalTok{\textbackslash{}draw[rotate=0] (0,0) ellipse (100pt and 50pt);}
\NormalTok{\textbackslash{}end\{tikzpicture\}}
\NormalTok{\textbackslash{}end\{document\}}
\end{Highlighting}
\end{Shaded}
\\



\begin{minipage}{1.0\linewidth}
\begin{center}
\includegraphics[width=1.0\linewidth,height=6.5in,keepaspectratio]{../images/198.\SVGExtension}
\end{center}
\raggedright{}\myfigurewithoutcaption{198}
\end{minipage}\vspace{0.75cm}



\end{longtable}\subsection{Example 4: Some functions}
\label{814}
\begin{longtable}{p{1.0\linewidth}}
\begin{Shaded}
\begin{Highlighting}[]

\NormalTok{\textbackslash{}documentclass\{article\}}
\NormalTok{\textbackslash{}usepackage\{tikz\}}
\NormalTok{\textbackslash{}begin\{document\}}
  \NormalTok{\textbackslash{}begin\{tikzpicture\}[domain=0:4] }
    \NormalTok{\textbackslash{}draw[very thin,color=gray] (-0.1,-1.1) grid (3.9,3.9);}
    \NormalTok{\textbackslash{}draw[->] (-0.2,0) -- (4.2,0) node[right] \{$x$\}; }
    \NormalTok{\textbackslash{}draw[->] (0,-1.2) -- (0,4.2) node[above] \{$f(x)$\};}
    \NormalTok{\textbackslash{}draw[color=red]    plot (\textbackslash{}x,\textbackslash{}x)             node[right] \{$f(x) =x$\}; }
    \NormalTok{\textbackslash{}draw[color=blue]   plot (\textbackslash{}x,\{sin(\textbackslash{}x r)\})    node[right] \{$f(x) = \textbackslash{}sin x$\}; }
    \NormalTok{\textbackslash{}draw[color=orange] plot (\textbackslash{}x,\{0.05*exp(\textbackslash{}x)\}) node[right] \{$f(x) = \textbackslash{}frac\{1\}\{20\} \textbackslash{}mathrm e^x$\};}
  \NormalTok{\textbackslash{}end\{tikzpicture\}}
\NormalTok{\textbackslash{}end\{document\}}
\end{Highlighting}
\end{Shaded}
\\



\begin{minipage}{1.0\linewidth}
\begin{center}
\includegraphics[width=1.0\linewidth,height=6.5in,keepaspectratio]{../images/199.\SVGExtension}
\end{center}
\raggedright{}\myfigurewithoutcaption{199}
\end{minipage}\vspace{0.75cm}



\end{longtable}
\chapter{PSTricks}

\myminitoc
\label{815}

\label{816}


PSTricks is a set of extensions. The base package is {\ttfamily \setmainfont[Path=/usr/share/fonts/truetype/cmu/,UprightFont=cmunrm.ttf,BoldFont=cmunbx.ttf,ItalicFont=cmunti.ttf,BoldItalicFont=cmunbi.ttf]{cmuntt.ttf}\setmonofont[Path=/usr/share/fonts/truetype/cmu/,UprightFont=cmuntt.ttf,BoldFont=cmuntb.ttf,ItalicFont=cmunit.ttf,BoldItalicFont=cmuntx.ttf]{cmuntt.ttf}\ttfamily pstricks}\setmainfont[Path=/usr/share/fonts/truetype/cmu/,UprightFont=cmunrm.ttf,BoldFont=cmunbx.ttf,ItalicFont=cmunti.ttf,BoldItalicFont=cmunbi.ttf]{cmunrm.ttf}\setmonofont[Path=/usr/share/fonts/truetype/cmu/,UprightFont=cmuntt.ttf,BoldFont=cmuntb.ttf,ItalicFont=cmunit.ttf,BoldItalicFont=cmuntx.ttf]{cmunrm.ttf}, other packages may be loaded when required.

The {\ttfamily \setmainfont[Path=/usr/share/fonts/truetype/cmu/,UprightFont=cmunrm.ttf,BoldFont=cmunbx.ttf,ItalicFont=cmunti.ttf,BoldItalicFont=cmunbi.ttf]{cmuntt.ttf}\setmonofont[Path=/usr/share/fonts/truetype/cmu/,UprightFont=cmuntt.ttf,BoldFont=cmuntb.ttf,ItalicFont=cmunit.ttf,BoldItalicFont=cmuntx.ttf]{cmuntt.ttf}\ttfamily xcolor}{$\text{ }$}\setmainfont[Path=/usr/share/fonts/truetype/cmu/,UprightFont=cmunrm.ttf,BoldFont=cmunbx.ttf,ItalicFont=cmunti.ttf,BoldItalicFont=cmunbi.ttf]{cmunrm.ttf}\setmonofont[Path=/usr/share/fonts/truetype/cmu/,UprightFont=cmuntt.ttf,BoldFont=cmuntb.ttf,ItalicFont=cmunit.ttf,BoldItalicFont=cmuntx.ttf]{cmunrm.ttf} extension gets loaded along PSTricks, so there is no need to load it manually.

PSTricks has one technical specification: it uses PostScript internally, hence the name. Thus you cannot use the {\ttfamily \setmainfont[Path=/usr/share/fonts/truetype/cmu/,UprightFont=cmunrm.ttf,BoldFont=cmunbx.ttf,ItalicFont=cmunti.ttf,BoldItalicFont=cmunbi.ttf]{cmuntt.ttf}\setmonofont[Path=/usr/share/fonts/truetype/cmu/,UprightFont=cmuntt.ttf,BoldFont=cmuntb.ttf,ItalicFont=cmunit.ttf,BoldItalicFont=cmuntx.ttf]{cmuntt.ttf}\ttfamily pdftex}{$\text{ }$}\setmainfont[Path=/usr/share/fonts/truetype/cmu/,UprightFont=cmunrm.ttf,BoldFont=cmunbx.ttf,ItalicFont=cmunti.ttf,BoldItalicFont=cmunbi.ttf]{cmunrm.ttf}\setmonofont[Path=/usr/share/fonts/truetype/cmu/,UprightFont=cmuntt.ttf,BoldFont=cmuntb.ttf,ItalicFont=cmunit.ttf,BoldItalicFont=cmuntx.ttf]{cmunrm.ttf} or {\ttfamily \setmainfont[Path=/usr/share/fonts/truetype/cmu/,UprightFont=cmunrm.ttf,BoldFont=cmunbx.ttf,ItalicFont=cmunti.ttf,BoldItalicFont=cmunbi.ttf]{cmuntt.ttf}\setmonofont[Path=/usr/share/fonts/truetype/cmu/,UprightFont=cmuntt.ttf,BoldFont=cmuntb.ttf,ItalicFont=cmunit.ttf,BoldItalicFont=cmuntx.ttf]{cmuntt.ttf}\ttfamily pdflatex}{$\text{ }$}\setmainfont[Path=/usr/share/fonts/truetype/cmu/,UprightFont=cmunrm.ttf,BoldFont=cmunbx.ttf,ItalicFont=cmunti.ttf,BoldItalicFont=cmunbi.ttf]{cmunrm.ttf}\setmonofont[Path=/usr/share/fonts/truetype/cmu/,UprightFont=cmuntt.ttf,BoldFont=cmuntb.ttf,ItalicFont=cmunit.ttf,BoldItalicFont=cmuntx.ttf]{cmunrm.ttf} compilers, you will need to use {\ttfamily \setmainfont[Path=/usr/share/fonts/truetype/cmu/,UprightFont=cmunrm.ttf,BoldFont=cmunbx.ttf,ItalicFont=cmunti.ttf,BoldItalicFont=cmunbi.ttf]{cmuntt.ttf}\setmonofont[Path=/usr/share/fonts/truetype/cmu/,UprightFont=cmuntt.ttf,BoldFont=cmuntb.ttf,ItalicFont=cmunit.ttf,BoldItalicFont=cmuntx.ttf]{cmuntt.ttf}\ttfamily dvips}{$\text{ }$}\setmainfont[Path=/usr/share/fonts/truetype/cmu/,UprightFont=cmunrm.ttf,BoldFont=cmunbx.ttf,ItalicFont=cmunti.ttf,BoldItalicFont=cmunbi.ttf]{cmunrm.ttf}\setmonofont[Path=/usr/share/fonts/truetype/cmu/,UprightFont=cmuntt.ttf,BoldFont=cmuntb.ttf,ItalicFont=cmunit.ttf,BoldItalicFont=cmuntx.ttf]{cmunrm.ttf} to get your proper document. It is still possible to get PDF from PS files thanks to {\ttfamily \setmainfont[Path=/usr/share/fonts/truetype/cmu/,UprightFont=cmunrm.ttf,BoldFont=cmunbx.ttf,ItalicFont=cmunti.ttf,BoldItalicFont=cmunbi.ttf]{cmuntt.ttf}\setmonofont[Path=/usr/share/fonts/truetype/cmu/,UprightFont=cmuntt.ttf,BoldFont=cmuntb.ttf,ItalicFont=cmunit.ttf,BoldItalicFont=cmuntx.ttf]{cmuntt.ttf}\ttfamily ps2pdf}\setmainfont[Path=/usr/share/fonts/truetype/cmu/,UprightFont=cmunrm.ttf,BoldFont=cmunbx.ttf,ItalicFont=cmunti.ttf,BoldItalicFont=cmunbi.ttf]{cmunrm.ttf}\setmonofont[Path=/usr/share/fonts/truetype/cmu/,UprightFont=cmuntt.ttf,BoldFont=cmuntb.ttf,ItalicFont=cmunit.ttf,BoldItalicFont=cmuntx.ttf]{cmunrm.ttf}. There is also the possibility to use the PDFTricks extension, which makes it feasible to use {\ttfamily \setmainfont[Path=/usr/share/fonts/truetype/cmu/,UprightFont=cmunrm.ttf,BoldFont=cmunbx.ttf,ItalicFont=cmunti.ttf,BoldItalicFont=cmunbi.ttf]{cmuntt.ttf}\setmonofont[Path=/usr/share/fonts/truetype/cmu/,UprightFont=cmuntt.ttf,BoldFont=cmuntb.ttf,ItalicFont=cmunit.ttf,BoldItalicFont=cmuntx.ttf]{cmuntt.ttf}\ttfamily pdflatex}{$\text{ }$}\setmainfont[Path=/usr/share/fonts/truetype/cmu/,UprightFont=cmunrm.ttf,BoldFont=cmunbx.ttf,ItalicFont=cmunti.ttf,BoldItalicFont=cmunbi.ttf]{cmunrm.ttf}\setmonofont[Path=/usr/share/fonts/truetype/cmu/,UprightFont=cmuntt.ttf,BoldFont=cmuntb.ttf,ItalicFont=cmunit.ttf,BoldItalicFont=cmuntx.ttf]{cmunrm.ttf} together with PSTricks commands.

However, if you have installed the package {\ttfamily \setmainfont[Path=/usr/share/fonts/truetype/cmu/,UprightFont=cmunrm.ttf,BoldFont=cmunbx.ttf,ItalicFont=cmunti.ttf,BoldItalicFont=cmunbi.ttf]{cmuntt.ttf}\setmonofont[Path=/usr/share/fonts/truetype/cmu/,UprightFont=cmuntt.ttf,BoldFont=cmuntb.ttf,ItalicFont=cmunit.ttf,BoldItalicFont=cmuntx.ttf]{cmuntt.ttf}\ttfamily xetex-{}pstricks}\setmainfont[Path=/usr/share/fonts/truetype/cmu/,UprightFont=cmunrm.ttf,BoldFont=cmunbx.ttf,ItalicFont=cmunti.ttf,BoldItalicFont=cmunbi.ttf]{cmunrm.ttf}\setmonofont[Path=/usr/share/fonts/truetype/cmu/,UprightFont=cmuntt.ttf,BoldFont=cmuntb.ttf,ItalicFont=cmunit.ttf,BoldItalicFont=cmuntx.ttf]{cmunrm.ttf}, you can use {\ttfamily \setmainfont[Path=/usr/share/fonts/truetype/cmu/,UprightFont=cmunrm.ttf,BoldFont=cmunbx.ttf,ItalicFont=cmunti.ttf,BoldItalicFont=cmunbi.ttf]{cmuntt.ttf}\setmonofont[Path=/usr/share/fonts/truetype/cmu/,UprightFont=cmuntt.ttf,BoldFont=cmuntb.ttf,ItalicFont=cmunit.ttf,BoldItalicFont=cmuntx.ttf]{cmuntt.ttf}\ttfamily pstricks}{$\text{ }$}\setmainfont[Path=/usr/share/fonts/truetype/cmu/,UprightFont=cmunrm.ttf,BoldFont=cmunbx.ttf,ItalicFont=cmunti.ttf,BoldItalicFont=cmunbi.ttf]{cmunrm.ttf}\setmonofont[Path=/usr/share/fonts/truetype/cmu/,UprightFont=cmuntt.ttf,BoldFont=cmuntb.ttf,ItalicFont=cmunit.ttf,BoldItalicFont=cmuntx.ttf]{cmunrm.ttf} with {\ttfamily \setmainfont[Path=/usr/share/fonts/truetype/cmu/,UprightFont=cmunrm.ttf,BoldFont=cmunbx.ttf,ItalicFont=cmunti.ttf,BoldItalicFont=cmunbi.ttf]{cmuntt.ttf}\setmonofont[Path=/usr/share/fonts/truetype/cmu/,UprightFont=cmuntt.ttf,BoldFont=cmuntb.ttf,ItalicFont=cmunit.ttf,BoldItalicFont=cmuntx.ttf]{cmuntt.ttf}\ttfamily xetex}{$\text{ }$}\setmainfont[Path=/usr/share/fonts/truetype/cmu/,UprightFont=cmunrm.ttf,BoldFont=cmunbx.ttf,ItalicFont=cmunti.ttf,BoldItalicFont=cmunbi.ttf]{cmunrm.ttf}\setmonofont[Path=/usr/share/fonts/truetype/cmu/,UprightFont=cmuntt.ttf,BoldFont=cmuntb.ttf,ItalicFont=cmunit.ttf,BoldItalicFont=cmuntx.ttf]{cmunrm.ttf} or {\ttfamily \setmainfont[Path=/usr/share/fonts/truetype/cmu/,UprightFont=cmunrm.ttf,BoldFont=cmunbx.ttf,ItalicFont=cmunti.ttf,BoldItalicFont=cmunbi.ttf]{cmuntt.ttf}\setmonofont[Path=/usr/share/fonts/truetype/cmu/,UprightFont=cmuntt.ttf,BoldFont=cmuntb.ttf,ItalicFont=cmunit.ttf,BoldItalicFont=cmuntx.ttf]{cmuntt.ttf}\ttfamily xelatex}{$\text{ }$}\setmainfont[Path=/usr/share/fonts/truetype/cmu/,UprightFont=cmunrm.ttf,BoldFont=cmunbx.ttf,ItalicFont=cmunti.ttf,BoldItalicFont=cmunbi.ttf]{cmunrm.ttf}\setmonofont[Path=/usr/share/fonts/truetype/cmu/,UprightFont=cmuntt.ttf,BoldFont=cmuntb.ttf,ItalicFont=cmunit.ttf,BoldItalicFont=cmuntx.ttf]{cmunrm.ttf} without modification of source file.
\section{The {\ttfamily \setmainfont[Path=/usr/share/fonts/truetype/cmu/,UprightFont=cmunrm.ttf,BoldFont=cmunbx.ttf,ItalicFont=cmunti.ttf,BoldItalicFont=cmunbi.ttf]{cmuntt.ttf}\setmonofont[Path=/usr/share/fonts/truetype/cmu/,UprightFont=cmuntt.ttf,BoldFont=cmuntb.ttf,ItalicFont=cmunit.ttf,BoldItalicFont=cmuntx.ttf]{cmuntt.ttf}\ttfamily pspicture}{$\text{ }$}\setmainfont[Path=/usr/share/fonts/truetype/cmu/,UprightFont=cmunrm.ttf,BoldFont=cmunbx.ttf,ItalicFont=cmunti.ttf,BoldItalicFont=cmunbi.ttf]{cmunrm.ttf}\setmonofont[Path=/usr/share/fonts/truetype/cmu/,UprightFont=cmuntt.ttf,BoldFont=cmuntb.ttf,ItalicFont=cmunit.ttf,BoldItalicFont=cmuntx.ttf]{cmunrm.ttf} environment}
\label{817}

PSTricks commands are usually placed in a 
\begin{Shaded}
\begin{Highlighting}[]

\NormalTok{pspicture}\newline
\end{Highlighting}
\end{Shaded}
 environment.


\begin{Shaded}
\begin{Highlighting}[]

\NormalTok{\textbackslash{}begin\{pspicture\}(x1,y1)}\newline
\CommentTok{\%\ensuremath{\text{ }}...}\newline
\NormalTok{\textbackslash{}end\{pspicture\}}\newline
\end{Highlighting}
\end{Shaded}


The first argument between parentheses specifies the coordinates of the upper-{}right corner of the picture. The bottom-{}left corner is at (0,0) and is placed at the reference point of the next character in the LaTeX document.

It is also possible to specify the coordinates (x0,y0) of the bottom-{}left corner:

\begin{Shaded}
\begin{Highlighting}[]

\NormalTok{\textbackslash{}begin\{pspicture\}(x0,y0)(x1,y1)}\newline
\CommentTok{\%\ensuremath{\text{ }}...}\newline
\NormalTok{\textbackslash{}end\{pspicture\}}\newline
\end{Highlighting}
\end{Shaded}

Thus the size of the picture is {\itshape \setmainfont[Path=/usr/share/fonts/truetype/cmu/,UprightFont=cmunrm.ttf,BoldFont=cmunbx.ttf,ItalicFont=cmunti.ttf,BoldItalicFont=cmunbi.ttf]{cmunti.ttf}\setmonofont[Path=/usr/share/fonts/truetype/cmu/,UprightFont=cmuntt.ttf,BoldFont=cmuntb.ttf,ItalicFont=cmunit.ttf,BoldItalicFont=cmuntx.ttf]{cmunti.ttf}\itshape (x1-{}x0)x(y1-{}y0)}\setmainfont[Path=/usr/share/fonts/truetype/cmu/,UprightFont=cmunrm.ttf,BoldFont=cmunbx.ttf,ItalicFont=cmunti.ttf,BoldItalicFont=cmunbi.ttf]{cmunrm.ttf}\setmonofont[Path=/usr/share/fonts/truetype/cmu/,UprightFont=cmuntt.ttf,BoldFont=cmuntb.ttf,ItalicFont=cmunit.ttf,BoldItalicFont=cmuntx.ttf]{cmunrm.ttf}.  The default unit for coordinates is centimeters (cm); this can be changed with 
\begin{Shaded}
\begin{Highlighting}[]

\NormalTok{\textbackslash{}psset}\newline
\end{Highlighting}
\end{Shaded}
, as in 

\begin{Shaded}
\begin{Highlighting}[]

\NormalTok{\textbackslash{}psset\{unit=1bp\}}\newline
\end{Highlighting}
\end{Shaded}
.  Any TeX dimension is allowed.
\section{Fundamental objects}
\label{818}
\subsection{Lines and polylines}
\label{819}

A simple line gets printed with

\begin{Shaded}
\begin{Highlighting}[]

\NormalTok{\textbackslash{}psline(x0,y0)(x1,y1)}\newline
\end{Highlighting}
\end{Shaded}


To get a vector, add an arrow as parameter:

\begin{Shaded}
\begin{Highlighting}[]

\NormalTok{\textbackslash{}psline\{->\}(x0,y0)(x1,y1)}\newline
\end{Highlighting}
\end{Shaded}


You can add as many points as you want to get a polyline:

\begin{Shaded}
\begin{Highlighting}[]

\NormalTok{\textbackslash{}psline(x0,y0)(x1,y1)(x2,y3)…(xn,yn)}\newline
\end{Highlighting}
\end{Shaded}


To get rounded corners, add the following option:

\begin{Shaded}
\begin{Highlighting}[]

\NormalTok{\textbackslash{}psline[linearc=0.2]\{->\}(0,0)(2,1)(1,1)}\newline
\end{Highlighting}
\end{Shaded}

or

\begin{Shaded}
\begin{Highlighting}[]

\NormalTok{\textbackslash{}psline[linearc=0.2,arrows=->](0,0)(2,1)(1,1)}\newline
\end{Highlighting}
\end{Shaded}

\subsection{Rectangles}
\label{820}


\begin{Shaded}
\begin{Highlighting}[]

\NormalTok{\textbackslash{}psframe(x0,y0)(x1,y1)}\newline
\NormalTok{\textbackslash{}psframe*(x0,y0)(x1,y1)}\newline
\end{Highlighting}
\end{Shaded}


The starred version prints a filled rectangle.
Use the following parameter to get rounded corners:

\begin{Shaded}
\begin{Highlighting}[]

\NormalTok{\textbackslash{}psframe[framearc=0.2](x0,y0)(x1,y1)}\newline
\end{Highlighting}
\end{Shaded}

\subsection{Polygons}
\label{821}

Polygons are always closed. The syntax is the same as for 
\begin{Shaded}
\begin{Highlighting}[]

\NormalTok{\textbackslash{}psline}\newline
\end{Highlighting}
\end{Shaded}
:

\begin{Shaded}
\begin{Highlighting}[]

\NormalTok{\textbackslash{}pspolygon(x0,y0)(x1,y1)(x2,y2)...(xn,yn)}\newline
\end{Highlighting}
\end{Shaded}


As for rectangles, the starred version prints a filled polygon. And the 
\begin{Shaded}
\begin{Highlighting}[]

\NormalTok{linearc=0.2}\newline
\end{Highlighting}
\end{Shaded}
 option will print rounded corners.
\subsection{Circles, arc and ellipses}
\label{822}

Starred version fills the shape.

For circles, you need to provide center coordinates and radius:

\begin{Shaded}
\begin{Highlighting}[]

\NormalTok{\textbackslash{}pscircle(x,y)\{r\}}\newline
\end{Highlighting}
\end{Shaded}


To restrict the drawing to an arc, append the starting and ending angles in trigonometric notation:

\begin{Shaded}
\begin{Highlighting}[]

\NormalTok{\textbackslash{}psarc(x,y)\{r\}\{angle1\}\{angle2\}}\newline
\end{Highlighting}
\end{Shaded}


Finally, ellipses:

\begin{Shaded}
\begin{Highlighting}[]

\NormalTok{\textbackslash{}psellipse(x,y)(horizontal_axis,vertical_axis)}\newline
\end{Highlighting}
\end{Shaded}


\subsection{Curves}
\label{823}


\begin{Shaded}
\begin{Highlighting}[]

\NormalTok{\textbackslash{}psparabola(x0,y0)(x1,y1)}\newline
\end{Highlighting}
\end{Shaded}

will print a symmetric parabola with vertical asymptote, vertex {\itshape \setmainfont[Path=/usr/share/fonts/truetype/cmu/,UprightFont=cmunrm.ttf,BoldFont=cmunbx.ttf,ItalicFont=cmunti.ttf,BoldItalicFont=cmunbi.ttf]{cmunti.ttf}\setmonofont[Path=/usr/share/fonts/truetype/cmu/,UprightFont=cmuntt.ttf,BoldFont=cmuntb.ttf,ItalicFont=cmunit.ttf,BoldItalicFont=cmuntx.ttf]{cmunti.ttf}\itshape (x1,y1)}{$\text{ }$}\setmainfont[Path=/usr/share/fonts/truetype/cmu/,UprightFont=cmunrm.ttf,BoldFont=cmunbx.ttf,ItalicFont=cmunti.ttf,BoldItalicFont=cmunbi.ttf]{cmunrm.ttf}\setmonofont[Path=/usr/share/fonts/truetype/cmu/,UprightFont=cmuntt.ttf,BoldFont=cmuntb.ttf,ItalicFont=cmunit.ttf,BoldItalicFont=cmuntx.ttf]{cmunrm.ttf} and ending at (x0,y0).

Use 
\begin{Shaded}
\begin{Highlighting}[]

\NormalTok{\textbackslash{}psbezier}\newline
\end{Highlighting}
\end{Shaded}
 to print a Bézier curve with an arbitrary number of control points. Arcs have at most 4 control points. Use the 
\begin{Shaded}
\begin{Highlighting}[]

\NormalTok{showpoints=true}\newline
\end{Highlighting}
\end{Shaded}
 option to print the control points and the tangents.

Use 
\begin{Shaded}
\begin{Highlighting}[]

\NormalTok{\textbackslash{}pscurve}\newline
\end{Highlighting}
\end{Shaded}
 to print the interpolation of the given points.
The 
\begin{Shaded}
\begin{Highlighting}[]

\NormalTok{\textbackslash{}psecurve}\newline
\end{Highlighting}
\end{Shaded}
 command omits the first and the last arcs.
\section{Text}
\label{824}
Use

\begin{Shaded}
\begin{Highlighting}[]

\NormalTok{\textbackslash{}rput(x,y)\{text\}}\newline
\end{Highlighting}
\end{Shaded}

to print text. Provide an angle to rotate the text.

\begin{Shaded}
\begin{Highlighting}[]

\NormalTok{\textbackslash{}rput\{angle\}(x,y)\{text\}}\newline
\end{Highlighting}
\end{Shaded}


You can provide the anchor of the text which will be at the specified coordinate.

\begin{Shaded}
\begin{Highlighting}[]

\NormalTok{\textbackslash{}rput[t]\{45\}(5,5)\{text\}}\newline
\end{Highlighting}
\end{Shaded}


Available anchors:
\begin{myitemize}
\item{}  B, Bl, Br: baseline center, left and right.
\item{}  t, tl, tr: top center, left and right.
\item{}  b, bl, br: bottom center, left and right.
\end{myitemize}


There is also the 
\begin{Shaded}
\begin{Highlighting}[]

\NormalTok{\textbackslash{}uput}\newline
\end{Highlighting}
\end{Shaded}
 command with further options:

\begin{Shaded}
\begin{Highlighting}[]

\NormalTok{\textbackslash{}uput\{distance\}[angle](x,y)\{text\}}\newline
\end{Highlighting}
\end{Shaded}

The 
\begin{Shaded}
\begin{Highlighting}[]

\NormalTok{distance}\newline
\end{Highlighting}
\end{Shaded}
 parameter is the distance from the coordinate.

PSTricks features several frame style for text.
\begin{myitemize}
\item{}  \textbackslash{}psframebox\{text\}: rectangle.
\item{}  \textbackslash{}psdblframebox\{text\}: double rectangle.
\item{}  \textbackslash{}psshadowbox\{text\}: shaded rectangle.
\item{}  \textbackslash{}pscirclebox\{text\}: circle.
\item{}  \textbackslash{}psovalbox\{text\}: oval.
\item{}  \textbackslash{}psdiabox\{text\}: diamond.
\item{}  \textbackslash{}pstribox\{text\}: triangle.
\end{myitemize}


Example:

\begin{Shaded}
\begin{Highlighting}[]

\NormalTok{\textbackslash{}rput(5,5)\{\textbackslash{}psdiabox*[fillcolor=green]\{text\}\}}\newline
\end{Highlighting}
\end{Shaded}



Using the 
\begin{Shaded}
\begin{Highlighting}[]

\NormalTok{pst-text}\newline
\end{Highlighting}
\end{Shaded}
 extension, it is possible to draw a text path.

\begin{Shaded}
\begin{Highlighting}[]

\NormalTok{\textbackslash{}pstextpath\{shape\}\{text\}}\newline
\end{Highlighting}
\end{Shaded}


To print a text following a path without printing the path, you need to use 
\begin{Shaded}
\begin{Highlighting}[]

\NormalTok{\textbackslash{}psset\{linestyle=none\}}\newline
\end{Highlighting}
\end{Shaded}
.

Example:

\begin{Shaded}
\begin{Highlighting}[]

\NormalTok{\textbackslash{}usepackage\{pst-text\}}\newline
\ensuremath{\text{ }}\newline
\CommentTok{\%\ensuremath{\text{ }}...}\newline
\NormalTok{\textbackslash{}begin\{pspicture\}(5,5)}\newline
\NormalTok{\textbackslash{}psset\{linestyle=none\}}\newline
\NormalTok{\textbackslash{}pstextpath\{\textbackslash{}psline(0,0)(1,1)(2,0)\}\{triangle\ensuremath{\text{ }}text\}}\newline
\NormalTok{\textbackslash{}end\{pspicture\}}\newline
\end{Highlighting}
\end{Shaded}

\section{Grids}
\label{825}

Without any parameter, the 
\begin{Shaded}
\begin{Highlighting}[]

\NormalTok{\textbackslash{}psgrid}\newline
\end{Highlighting}
\end{Shaded}
 command will print a grid all over the {\itshape \setmainfont[Path=/usr/share/fonts/truetype/cmu/,UprightFont=cmunrm.ttf,BoldFont=cmunbx.ttf,ItalicFont=cmunti.ttf,BoldItalicFont=cmunbi.ttf]{cmunti.ttf}\setmonofont[Path=/usr/share/fonts/truetype/cmu/,UprightFont=cmuntt.ttf,BoldFont=cmuntb.ttf,ItalicFont=cmunit.ttf,BoldItalicFont=cmuntx.ttf]{cmunti.ttf}\itshape pspicture}\setmainfont[Path=/usr/share/fonts/truetype/cmu/,UprightFont=cmunrm.ttf,BoldFont=cmunbx.ttf,ItalicFont=cmunti.ttf,BoldItalicFont=cmunbi.ttf]{cmunrm.ttf}\setmonofont[Path=/usr/share/fonts/truetype/cmu/,UprightFont=cmuntt.ttf,BoldFont=cmuntb.ttf,ItalicFont=cmunit.ttf,BoldItalicFont=cmuntx.ttf]{cmunrm.ttf}, with a spacing of 0.2 ({\itshape \setmainfont[Path=/usr/share/fonts/truetype/cmu/,UprightFont=cmunrm.ttf,BoldFont=cmunbx.ttf,ItalicFont=cmunti.ttf,BoldItalicFont=cmunbi.ttf]{cmunti.ttf}\setmonofont[Path=/usr/share/fonts/truetype/cmu/,UprightFont=cmuntt.ttf,BoldFont=cmuntb.ttf,ItalicFont=cmunit.ttf,BoldItalicFont=cmuntx.ttf]{cmunti.ttf}\itshape i.e.}{$\text{ }$}\setmainfont[Path=/usr/share/fonts/truetype/cmu/,UprightFont=cmunrm.ttf,BoldFont=cmunbx.ttf,ItalicFont=cmunti.ttf,BoldItalicFont=cmunbi.ttf]{cmunrm.ttf}\setmonofont[Path=/usr/share/fonts/truetype/cmu/,UprightFont=cmuntt.ttf,BoldFont=cmuntb.ttf,ItalicFont=cmunit.ttf,BoldItalicFont=cmuntx.ttf]{cmunrm.ttf} 2mm). You can specify parameters:

\begin{myitemize}
\item{}  
\begin{Shaded}
\begin{Highlighting}[]

\NormalTok{\textbackslash{}psgrid(xmax,ymax)}\newline
\end{Highlighting}
\end{Shaded}
: prints a grid from {\itshape \setmainfont[Path=/usr/share/fonts/truetype/cmu/,UprightFont=cmunrm.ttf,BoldFont=cmunbx.ttf,ItalicFont=cmunti.ttf,BoldItalicFont=cmunbi.ttf]{cmunti.ttf}\setmonofont[Path=/usr/share/fonts/truetype/cmu/,UprightFont=cmuntt.ttf,BoldFont=cmuntb.ttf,ItalicFont=cmunit.ttf,BoldItalicFont=cmuntx.ttf]{cmunti.ttf}\itshape (0,0)}{$\text{ }$}\setmainfont[Path=/usr/share/fonts/truetype/cmu/,UprightFont=cmunrm.ttf,BoldFont=cmunbx.ttf,ItalicFont=cmunti.ttf,BoldItalicFont=cmunbi.ttf]{cmunrm.ttf}\setmonofont[Path=/usr/share/fonts/truetype/cmu/,UprightFont=cmuntt.ttf,BoldFont=cmuntb.ttf,ItalicFont=cmunit.ttf,BoldItalicFont=cmuntx.ttf]{cmunrm.ttf} to {\itshape \setmainfont[Path=/usr/share/fonts/truetype/cmu/,UprightFont=cmunrm.ttf,BoldFont=cmunbx.ttf,ItalicFont=cmunti.ttf,BoldItalicFont=cmunbi.ttf]{cmunti.ttf}\setmonofont[Path=/usr/share/fonts/truetype/cmu/,UprightFont=cmuntt.ttf,BoldFont=cmuntb.ttf,ItalicFont=cmunit.ttf,BoldItalicFont=cmuntx.ttf]{cmunti.ttf}\itshape (xmax,ymax)}\setmainfont[Path=/usr/share/fonts/truetype/cmu/,UprightFont=cmunrm.ttf,BoldFont=cmunbx.ttf,ItalicFont=cmunti.ttf,BoldItalicFont=cmunbi.ttf]{cmunrm.ttf}\setmonofont[Path=/usr/share/fonts/truetype/cmu/,UprightFont=cmuntt.ttf,BoldFont=cmuntb.ttf,ItalicFont=cmunit.ttf,BoldItalicFont=cmuntx.ttf]{cmunrm.ttf}.
\item{}  
\begin{Shaded}
\begin{Highlighting}[]

\NormalTok{\textbackslash{}psgrid(xmin,ymin)(xmax,ymax)}\newline
\end{Highlighting}
\end{Shaded}
: prints a grid from {\itshape \setmainfont[Path=/usr/share/fonts/truetype/cmu/,UprightFont=cmunrm.ttf,BoldFont=cmunbx.ttf,ItalicFont=cmunti.ttf,BoldItalicFont=cmunbi.ttf]{cmunti.ttf}\setmonofont[Path=/usr/share/fonts/truetype/cmu/,UprightFont=cmuntt.ttf,BoldFont=cmuntb.ttf,ItalicFont=cmunit.ttf,BoldItalicFont=cmuntx.ttf]{cmunti.ttf}\itshape (xmin,ymin)}{$\text{ }$}\setmainfont[Path=/usr/share/fonts/truetype/cmu/,UprightFont=cmunrm.ttf,BoldFont=cmunbx.ttf,ItalicFont=cmunti.ttf,BoldItalicFont=cmunbi.ttf]{cmunrm.ttf}\setmonofont[Path=/usr/share/fonts/truetype/cmu/,UprightFont=cmuntt.ttf,BoldFont=cmuntb.ttf,ItalicFont=cmunit.ttf,BoldItalicFont=cmuntx.ttf]{cmunrm.ttf} to {\itshape \setmainfont[Path=/usr/share/fonts/truetype/cmu/,UprightFont=cmunrm.ttf,BoldFont=cmunbx.ttf,ItalicFont=cmunti.ttf,BoldItalicFont=cmunbi.ttf]{cmunti.ttf}\setmonofont[Path=/usr/share/fonts/truetype/cmu/,UprightFont=cmuntt.ttf,BoldFont=cmuntb.ttf,ItalicFont=cmunit.ttf,BoldItalicFont=cmuntx.ttf]{cmunti.ttf}\itshape (xmax,ymax)}\setmainfont[Path=/usr/share/fonts/truetype/cmu/,UprightFont=cmunrm.ttf,BoldFont=cmunbx.ttf,ItalicFont=cmunti.ttf,BoldItalicFont=cmunbi.ttf]{cmunrm.ttf}\setmonofont[Path=/usr/share/fonts/truetype/cmu/,UprightFont=cmuntt.ttf,BoldFont=cmuntb.ttf,ItalicFont=cmunit.ttf,BoldItalicFont=cmuntx.ttf]{cmunrm.ttf}.
\item{}  
\begin{Shaded}
\begin{Highlighting}[]

\NormalTok{\textbackslash{}psgrid(x0,y0)(xmin,ymin)(xmax,ymax)}\newline
\end{Highlighting}
\end{Shaded}
: prints a grid from {\itshape \setmainfont[Path=/usr/share/fonts/truetype/cmu/,UprightFont=cmunrm.ttf,BoldFont=cmunbx.ttf,ItalicFont=cmunti.ttf,BoldItalicFont=cmunbi.ttf]{cmunti.ttf}\setmonofont[Path=/usr/share/fonts/truetype/cmu/,UprightFont=cmuntt.ttf,BoldFont=cmuntb.ttf,ItalicFont=cmunit.ttf,BoldItalicFont=cmuntx.ttf]{cmunti.ttf}\itshape (xmin,ymin)}{$\text{ }$}\setmainfont[Path=/usr/share/fonts/truetype/cmu/,UprightFont=cmunrm.ttf,BoldFont=cmunbx.ttf,ItalicFont=cmunti.ttf,BoldItalicFont=cmunbi.ttf]{cmunrm.ttf}\setmonofont[Path=/usr/share/fonts/truetype/cmu/,UprightFont=cmuntt.ttf,BoldFont=cmuntb.ttf,ItalicFont=cmunit.ttf,BoldItalicFont=cmuntx.ttf]{cmunrm.ttf} to {\itshape \setmainfont[Path=/usr/share/fonts/truetype/cmu/,UprightFont=cmunrm.ttf,BoldFont=cmunbx.ttf,ItalicFont=cmunti.ttf,BoldItalicFont=cmunbi.ttf]{cmunti.ttf}\setmonofont[Path=/usr/share/fonts/truetype/cmu/,UprightFont=cmuntt.ttf,BoldFont=cmuntb.ttf,ItalicFont=cmunit.ttf,BoldItalicFont=cmuntx.ttf]{cmunti.ttf}\itshape (xmax,ymax)}\setmainfont[Path=/usr/share/fonts/truetype/cmu/,UprightFont=cmunrm.ttf,BoldFont=cmunbx.ttf,ItalicFont=cmunti.ttf,BoldItalicFont=cmunbi.ttf]{cmunrm.ttf}\setmonofont[Path=/usr/share/fonts/truetype/cmu/,UprightFont=cmuntt.ttf,BoldFont=cmuntb.ttf,ItalicFont=cmunit.ttf,BoldItalicFont=cmuntx.ttf]{cmunrm.ttf}, one of the node is at {\itshape \setmainfont[Path=/usr/share/fonts/truetype/cmu/,UprightFont=cmunrm.ttf,BoldFont=cmunbx.ttf,ItalicFont=cmunti.ttf,BoldItalicFont=cmunbi.ttf]{cmunti.ttf}\setmonofont[Path=/usr/share/fonts/truetype/cmu/,UprightFont=cmuntt.ttf,BoldFont=cmuntb.ttf,ItalicFont=cmunit.ttf,BoldItalicFont=cmuntx.ttf]{cmunti.ttf}\itshape (x0,y0)}\setmainfont[Path=/usr/share/fonts/truetype/cmu/,UprightFont=cmunrm.ttf,BoldFont=cmunbx.ttf,ItalicFont=cmunti.ttf,BoldItalicFont=cmunbi.ttf]{cmunrm.ttf}\setmonofont[Path=/usr/share/fonts/truetype/cmu/,UprightFont=cmuntt.ttf,BoldFont=cmuntb.ttf,ItalicFont=cmunit.ttf,BoldItalicFont=cmuntx.ttf]{cmunrm.ttf}.
\item{}  
\begin{Shaded}
\begin{Highlighting}[]

\NormalTok{griddots=value}\newline
\end{Highlighting}
\end{Shaded}
: the full line of the main graduations is replaced by a dotted line. The {\itshape \setmainfont[Path=/usr/share/fonts/truetype/cmu/,UprightFont=cmunrm.ttf,BoldFont=cmunbx.ttf,ItalicFont=cmunti.ttf,BoldItalicFont=cmunbi.ttf]{cmunti.ttf}\setmonofont[Path=/usr/share/fonts/truetype/cmu/,UprightFont=cmuntt.ttf,BoldFont=cmuntb.ttf,ItalicFont=cmunit.ttf,BoldItalicFont=cmuntx.ttf]{cmunti.ttf}\itshape value}{$\text{ }$}\setmainfont[Path=/usr/share/fonts/truetype/cmu/,UprightFont=cmunrm.ttf,BoldFont=cmunbx.ttf,ItalicFont=cmunti.ttf,BoldItalicFont=cmunbi.ttf]{cmunrm.ttf}\setmonofont[Path=/usr/share/fonts/truetype/cmu/,UprightFont=cmuntt.ttf,BoldFont=cmuntb.ttf,ItalicFont=cmunit.ttf,BoldItalicFont=cmuntx.ttf]{cmunrm.ttf} is the number of dots per graduation.
\item{}  
\begin{Shaded}
\begin{Highlighting}[]

\NormalTok{subgriddots=value}\newline
\end{Highlighting}
\end{Shaded}
: same as 
\begin{Shaded}
\begin{Highlighting}[]

\NormalTok{griddots}\newline
\end{Highlighting}
\end{Shaded}
 but for sub-{}graduations.
\item{}  
\begin{Shaded}
\begin{Highlighting}[]

\NormalTok{gridcolor=color,subgridcolor=color}\newline
\end{Highlighting}
\end{Shaded}
: color of graduations and sub-{}graduations.
\item{}  
\begin{Shaded}
\begin{Highlighting}[]

\NormalTok{gridwidth=value,subgridwidth=value}\newline
\end{Highlighting}
\end{Shaded}
: width of the lines.
\item{}  
\begin{Shaded}
\begin{Highlighting}[]

\NormalTok{subgriddiv=value}\newline
\end{Highlighting}
\end{Shaded}
: number of subgraduations between two main graduations.
\item{}  
\begin{Shaded}
\begin{Highlighting}[]

\NormalTok{gridlabels=value}\newline
\end{Highlighting}
\end{Shaded}
: size of the label numbers.
\item{}  
\begin{Shaded}
\begin{Highlighting}[]

\NormalTok{ticksize=value}\newline
\end{Highlighting}
\end{Shaded}
: self-{}explanatory.
\item{}  
\begin{Shaded}
\begin{Highlighting}[]

\NormalTok{ticksize=valueneg\ensuremath{\text{ }}valuepos}\newline
\end{Highlighting}
\end{Shaded}
: same as above, but {\itshape \setmainfont[Path=/usr/share/fonts/truetype/cmu/,UprightFont=cmunrm.ttf,BoldFont=cmunbx.ttf,ItalicFont=cmunti.ttf,BoldItalicFont=cmunbi.ttf]{cmunti.ttf}\setmonofont[Path=/usr/share/fonts/truetype/cmu/,UprightFont=cmuntt.ttf,BoldFont=cmuntb.ttf,ItalicFont=cmunit.ttf,BoldItalicFont=cmuntx.ttf]{cmunti.ttf}\itshape valueneg}{$\text{ }$}\setmainfont[Path=/usr/share/fonts/truetype/cmu/,UprightFont=cmunrm.ttf,BoldFont=cmunbx.ttf,ItalicFont=cmunti.ttf,BoldItalicFont=cmunbi.ttf]{cmunrm.ttf}\setmonofont[Path=/usr/share/fonts/truetype/cmu/,UprightFont=cmuntt.ttf,BoldFont=cmuntb.ttf,ItalicFont=cmunit.ttf,BoldItalicFont=cmuntx.ttf]{cmunrm.ttf} specifies the size for negative coordinates, {\itshape \setmainfont[Path=/usr/share/fonts/truetype/cmu/,UprightFont=cmunrm.ttf,BoldFont=cmunbx.ttf,ItalicFont=cmunti.ttf,BoldItalicFont=cmunbi.ttf]{cmunti.ttf}\setmonofont[Path=/usr/share/fonts/truetype/cmu/,UprightFont=cmuntt.ttf,BoldFont=cmuntb.ttf,ItalicFont=cmunit.ttf,BoldItalicFont=cmuntx.ttf]{cmunti.ttf}\itshape valuepos}{$\text{ }$}\setmainfont[Path=/usr/share/fonts/truetype/cmu/,UprightFont=cmunrm.ttf,BoldFont=cmunbx.ttf,ItalicFont=cmunti.ttf,BoldItalicFont=cmunbi.ttf]{cmunrm.ttf}\setmonofont[Path=/usr/share/fonts/truetype/cmu/,UprightFont=cmuntt.ttf,BoldFont=cmuntb.ttf,ItalicFont=cmunit.ttf,BoldItalicFont=cmuntx.ttf]{cmunrm.ttf} for positive coordinates.
\item{}  
\begin{Shaded}
\begin{Highlighting}[]

\NormalTok{ticklinestyle=value}\newline
\end{Highlighting}
\end{Shaded}
: self-{}explanatory. {\itshape \setmainfont[Path=/usr/share/fonts/truetype/cmu/,UprightFont=cmunrm.ttf,BoldFont=cmunbx.ttf,ItalicFont=cmunti.ttf,BoldItalicFont=cmunbi.ttf]{cmunti.ttf}\setmonofont[Path=/usr/share/fonts/truetype/cmu/,UprightFont=cmuntt.ttf,BoldFont=cmuntb.ttf,ItalicFont=cmunit.ttf,BoldItalicFont=cmuntx.ttf]{cmunti.ttf}\itshape value}{$\text{ }$}\setmainfont[Path=/usr/share/fonts/truetype/cmu/,UprightFont=cmunrm.ttf,BoldFont=cmunbx.ttf,ItalicFont=cmunti.ttf,BoldItalicFont=cmunbi.ttf]{cmunrm.ttf}\setmonofont[Path=/usr/share/fonts/truetype/cmu/,UprightFont=cmuntt.ttf,BoldFont=cmuntb.ttf,ItalicFont=cmunit.ttf,BoldItalicFont=cmuntx.ttf]{cmunrm.ttf} may be one of 
\begin{Shaded}
\begin{Highlighting}[]

\NormalTok{solid,\ensuremath{\text{ }}dashed,\ensuremath{\text{ }}dotted}\newline
\end{Highlighting}
\end{Shaded}
. This is useful for huge graduations ({\itshape \setmainfont[Path=/usr/share/fonts/truetype/cmu/,UprightFont=cmunrm.ttf,BoldFont=cmunbx.ttf,ItalicFont=cmunti.ttf,BoldItalicFont=cmunbi.ttf]{cmunti.ttf}\setmonofont[Path=/usr/share/fonts/truetype/cmu/,UprightFont=cmuntt.ttf,BoldFont=cmuntb.ttf,ItalicFont=cmunit.ttf,BoldItalicFont=cmuntx.ttf]{cmunti.ttf}\itshape i.e.}{$\text{ }$}\setmainfont[Path=/usr/share/fonts/truetype/cmu/,UprightFont=cmunrm.ttf,BoldFont=cmunbx.ttf,ItalicFont=cmunti.ttf,BoldItalicFont=cmunbi.ttf]{cmunrm.ttf}\setmonofont[Path=/usr/share/fonts/truetype/cmu/,UprightFont=cmuntt.ttf,BoldFont=cmuntb.ttf,ItalicFont=cmunit.ttf,BoldItalicFont=cmuntx.ttf]{cmunrm.ttf} 
\begin{Shaded}
\begin{Highlighting}[]

\NormalTok{ticksize}\newline
\end{Highlighting}
\end{Shaded}
 is high).
\end{myitemize}

{\bfseries
\begin{mydescription}Example
\end{mydescription}
}


\begin{Shaded}
\begin{Highlighting}[]

\NormalTok{\textbackslash{}psgrid[griddots=5,\ensuremath{\text{ }}subgriddiv=0,\ensuremath{\text{ }}gridlabels=0pt](-1,-1)(5,5)}\newline
\end{Highlighting}
\end{Shaded}

{\bfseries
\begin{mydescription}Axis
\end{mydescription}
}

If you want to add axes, use the {\ttfamily \setmainfont[Path=/usr/share/fonts/truetype/cmu/,UprightFont=cmunrm.ttf,BoldFont=cmunbx.ttf,ItalicFont=cmunti.ttf,BoldItalicFont=cmunbi.ttf]{cmuntt.ttf}\setmonofont[Path=/usr/share/fonts/truetype/cmu/,UprightFont=cmuntt.ttf,BoldFont=cmuntb.ttf,ItalicFont=cmunit.ttf,BoldItalicFont=cmuntx.ttf]{cmuntt.ttf}\ttfamily pstricks-{}add}{$\text{ }$}\setmainfont[Path=/usr/share/fonts/truetype/cmu/,UprightFont=cmunrm.ttf,BoldFont=cmunbx.ttf,ItalicFont=cmunti.ttf,BoldItalicFont=cmunbi.ttf]{cmunrm.ttf}\setmonofont[Path=/usr/share/fonts/truetype/cmu/,UprightFont=cmuntt.ttf,BoldFont=cmuntb.ttf,ItalicFont=cmunit.ttf,BoldItalicFont=cmuntx.ttf]{cmunrm.ttf} extension with the following commands:

\begin{Shaded}
\begin{Highlighting}[]

\NormalTok{\textbackslash{}psaxes(xmin,ymin)(xmax,ymax)}\newline
\NormalTok{\textbackslash{}psaxes(x0,y0)(xmin,ymin)(xmax,ymax)}\newline
\end{Highlighting}
\end{Shaded}

{\itshape \setmainfont[Path=/usr/share/fonts/truetype/cmu/,UprightFont=cmunrm.ttf,BoldFont=cmunbx.ttf,ItalicFont=cmunti.ttf,BoldItalicFont=cmunbi.ttf]{cmunti.ttf}\setmonofont[Path=/usr/share/fonts/truetype/cmu/,UprightFont=cmuntt.ttf,BoldFont=cmuntb.ttf,ItalicFont=cmunit.ttf,BoldItalicFont=cmuntx.ttf]{cmunti.ttf}\itshape (xmin,ymin)}{$\text{ }$}\setmainfont[Path=/usr/share/fonts/truetype/cmu/,UprightFont=cmunrm.ttf,BoldFont=cmunbx.ttf,ItalicFont=cmunti.ttf,BoldItalicFont=cmunbi.ttf]{cmunrm.ttf}\setmonofont[Path=/usr/share/fonts/truetype/cmu/,UprightFont=cmuntt.ttf,BoldFont=cmuntb.ttf,ItalicFont=cmunit.ttf,BoldItalicFont=cmuntx.ttf]{cmunrm.ttf} and {\itshape \setmainfont[Path=/usr/share/fonts/truetype/cmu/,UprightFont=cmunrm.ttf,BoldFont=cmunbx.ttf,ItalicFont=cmunti.ttf,BoldItalicFont=cmunbi.ttf]{cmunti.ttf}\setmonofont[Path=/usr/share/fonts/truetype/cmu/,UprightFont=cmuntt.ttf,BoldFont=cmuntb.ttf,ItalicFont=cmunit.ttf,BoldItalicFont=cmuntx.ttf]{cmunti.ttf}\itshape (xmax,ymax)}{$\text{ }$}\setmainfont[Path=/usr/share/fonts/truetype/cmu/,UprightFont=cmunrm.ttf,BoldFont=cmunbx.ttf,ItalicFont=cmunti.ttf,BoldItalicFont=cmunbi.ttf]{cmunrm.ttf}\setmonofont[Path=/usr/share/fonts/truetype/cmu/,UprightFont=cmuntt.ttf,BoldFont=cmuntb.ttf,ItalicFont=cmunit.ttf,BoldItalicFont=cmuntx.ttf]{cmunrm.ttf} being the extreme, {\itshape \setmainfont[Path=/usr/share/fonts/truetype/cmu/,UprightFont=cmunrm.ttf,BoldFont=cmunbx.ttf,ItalicFont=cmunti.ttf,BoldItalicFont=cmunbi.ttf]{cmunti.ttf}\setmonofont[Path=/usr/share/fonts/truetype/cmu/,UprightFont=cmuntt.ttf,BoldFont=cmuntb.ttf,ItalicFont=cmunit.ttf,BoldItalicFont=cmuntx.ttf]{cmunti.ttf}\itshape (x0,y0)}{$\text{ }$}\setmainfont[Path=/usr/share/fonts/truetype/cmu/,UprightFont=cmunrm.ttf,BoldFont=cmunbx.ttf,ItalicFont=cmunti.ttf,BoldItalicFont=cmunbi.ttf]{cmunrm.ttf}\setmonofont[Path=/usr/share/fonts/truetype/cmu/,UprightFont=cmuntt.ttf,BoldFont=cmuntb.ttf,ItalicFont=cmunit.ttf,BoldItalicFont=cmuntx.ttf]{cmunrm.ttf} being the intersection.
{\bfseries
\begin{mydescription}Options
\end{mydescription}
}

\begin{myitemize}
\item{}  
\begin{Shaded}
\begin{Highlighting}[]

\NormalTok{Dx=value}\newline
\end{Highlighting}
\end{Shaded}
 and 
\begin{Shaded}
\begin{Highlighting}[]

\NormalTok{Dy=value}\newline
\end{Highlighting}
\end{Shaded}
 defines the spacing between graduations.
\item{}  
\begin{Shaded}
\begin{Highlighting}[]

\NormalTok{comma}\newline
\end{Highlighting}
\end{Shaded}
 lets you use the comma as decimal separator.
\item{}  As for lines, 
\begin{Shaded}
\begin{Highlighting}[]

\NormalTok{\{->\}}\newline
\end{Highlighting}
\end{Shaded}
 adds arrows on axes.
\end{myitemize}

{\bfseries
\begin{mydescription}Example
\end{mydescription}
}



\begin{Shaded}
\begin{Highlighting}[]

\NormalTok{\textbackslash{}usepackage\{pstricks-add\}}\newline
\CommentTok{\%\ensuremath{\text{ }}...}\newline
\NormalTok{\textbackslash{}begin\{pspicture\}(-1,-1)(5,5)}\newline
\NormalTok{\textbackslash{}psaxes[comma,Dx=0.5,Dy=0.5]\{->\}(0,0)(3,3)}\newline
\NormalTok{\textbackslash{}end\{pspicture\}}\newline
\end{Highlighting}
\end{Shaded}

\section{Generic parameters}
\label{826}
\subsection{All shapes}
\label{827}

These are to be added between square brackets.

\begin{myitemize}
\item{}  
\begin{Shaded}
\begin{Highlighting}[]

\NormalTok{linewidth=value}\newline
\end{Highlighting}
\end{Shaded}
: if {\itshape \setmainfont[Path=/usr/share/fonts/truetype/cmu/,UprightFont=cmunrm.ttf,BoldFont=cmunbx.ttf,ItalicFont=cmunti.ttf,BoldItalicFont=cmunbi.ttf]{cmunti.ttf}\setmonofont[Path=/usr/share/fonts/truetype/cmu/,UprightFont=cmuntt.ttf,BoldFont=cmuntb.ttf,ItalicFont=cmunit.ttf,BoldItalicFont=cmuntx.ttf]{cmunti.ttf}\itshape value}{$\text{ }$}\setmainfont[Path=/usr/share/fonts/truetype/cmu/,UprightFont=cmunrm.ttf,BoldFont=cmunbx.ttf,ItalicFont=cmunti.ttf,BoldItalicFont=cmunbi.ttf]{cmunrm.ttf}\setmonofont[Path=/usr/share/fonts/truetype/cmu/,UprightFont=cmuntt.ttf,BoldFont=cmuntb.ttf,ItalicFont=cmunit.ttf,BoldItalicFont=cmuntx.ttf]{cmunrm.ttf} is without unit, then the default unit is used.
\item{}  
\begin{Shaded}
\begin{Highlighting}[]

\NormalTok{linecolor=color}\newline
\end{Highlighting}
\end{Shaded}
: {\itshape \setmainfont[Path=/usr/share/fonts/truetype/cmu/,UprightFont=cmunrm.ttf,BoldFont=cmunbx.ttf,ItalicFont=cmunti.ttf,BoldItalicFont=cmunbi.ttf]{cmunti.ttf}\setmonofont[Path=/usr/share/fonts/truetype/cmu/,UprightFont=cmuntt.ttf,BoldFont=cmuntb.ttf,ItalicFont=cmunit.ttf,BoldItalicFont=cmuntx.ttf]{cmunti.ttf}\itshape color}{$\text{ }$}\setmainfont[Path=/usr/share/fonts/truetype/cmu/,UprightFont=cmunrm.ttf,BoldFont=cmunbx.ttf,ItalicFont=cmunti.ttf,BoldItalicFont=cmunbi.ttf]{cmunrm.ttf}\setmonofont[Path=/usr/share/fonts/truetype/cmu/,UprightFont=cmuntt.ttf,BoldFont=cmuntb.ttf,ItalicFont=cmunit.ttf,BoldItalicFont=cmuntx.ttf]{cmunrm.ttf} is as defined by the {\ttfamily \setmainfont[Path=/usr/share/fonts/truetype/cmu/,UprightFont=cmunrm.ttf,BoldFont=cmunbx.ttf,ItalicFont=cmunti.ttf,BoldItalicFont=cmunbi.ttf]{cmuntt.ttf}\setmonofont[Path=/usr/share/fonts/truetype/cmu/,UprightFont=cmuntt.ttf,BoldFont=cmuntb.ttf,ItalicFont=cmunit.ttf,BoldItalicFont=cmuntx.ttf]{cmuntt.ttf}\ttfamily xcolor}\setmainfont[Path=/usr/share/fonts/truetype/cmu/,UprightFont=cmunrm.ttf,BoldFont=cmunbx.ttf,ItalicFont=cmunti.ttf,BoldItalicFont=cmunbi.ttf]{cmunrm.ttf}\setmonofont[Path=/usr/share/fonts/truetype/cmu/,UprightFont=cmuntt.ttf,BoldFont=cmuntb.ttf,ItalicFont=cmunit.ttf,BoldItalicFont=cmuntx.ttf]{cmunrm.ttf}package.
\item{}  
\begin{Shaded}
\begin{Highlighting}[]

\NormalTok{linestyle=value}\newline
\end{Highlighting}
\end{Shaded}
: {\itshape \setmainfont[Path=/usr/share/fonts/truetype/cmu/,UprightFont=cmunrm.ttf,BoldFont=cmunbx.ttf,ItalicFont=cmunti.ttf,BoldItalicFont=cmunbi.ttf]{cmunti.ttf}\setmonofont[Path=/usr/share/fonts/truetype/cmu/,UprightFont=cmuntt.ttf,BoldFont=cmuntb.ttf,ItalicFont=cmunit.ttf,BoldItalicFont=cmuntx.ttf]{cmunti.ttf}\itshape value}{$\text{ }$}\setmainfont[Path=/usr/share/fonts/truetype/cmu/,UprightFont=cmunrm.ttf,BoldFont=cmunbx.ttf,ItalicFont=cmunti.ttf,BoldItalicFont=cmunbi.ttf]{cmunrm.ttf}\setmonofont[Path=/usr/share/fonts/truetype/cmu/,UprightFont=cmuntt.ttf,BoldFont=cmuntb.ttf,ItalicFont=cmunit.ttf,BoldItalicFont=cmuntx.ttf]{cmunrm.ttf} is one of 
\begin{Shaded}
\begin{Highlighting}[]

\NormalTok{dashed,dotted}\newline
\end{Highlighting}
\end{Shaded}
.
\item{}  
\begin{Shaded}
\begin{Highlighting}[]

\NormalTok{doubleline=true}\newline
\end{Highlighting}
\end{Shaded}
.
\item{}  
\begin{Shaded}
\begin{Highlighting}[]

\NormalTok{showpoints=true}\newline
\end{Highlighting}
\end{Shaded}
: highlights points.
\begin{myitemize}
\item{}  
\begin{Shaded}
\begin{Highlighting}[]

\NormalTok{dotscale=value}\newline
\end{Highlighting}
\end{Shaded}
 specifies the size of the points.
\item{}  
\begin{Shaded}
\begin{Highlighting}[]

\NormalTok{dotstyle=value}\newline
\end{Highlighting}
\end{Shaded}
 where {\itshape \setmainfont[Path=/usr/share/fonts/truetype/cmu/,UprightFont=cmunrm.ttf,BoldFont=cmunbx.ttf,ItalicFont=cmunti.ttf,BoldItalicFont=cmunbi.ttf]{cmunti.ttf}\setmonofont[Path=/usr/share/fonts/truetype/cmu/,UprightFont=cmuntt.ttf,BoldFont=cmuntb.ttf,ItalicFont=cmunit.ttf,BoldItalicFont=cmuntx.ttf]{cmunti.ttf}\itshape value}{$\text{ }$}\setmainfont[Path=/usr/share/fonts/truetype/cmu/,UprightFont=cmunrm.ttf,BoldFont=cmunbx.ttf,ItalicFont=cmunti.ttf,BoldItalicFont=cmunbi.ttf]{cmunrm.ttf}\setmonofont[Path=/usr/share/fonts/truetype/cmu/,UprightFont=cmuntt.ttf,BoldFont=cmuntb.ttf,ItalicFont=cmunit.ttf,BoldItalicFont=cmuntx.ttf]{cmunrm.ttf} is among:
\begin{myitemize}
\item{}  *: disc
\item{}  o: circle
\item{}  +,x: cross
\item{}  square, squarre*: starred version is filled.
\item{}  diamond, diamond*
\item{}  triangle, triangle*
\item{}  etc.
\end{myitemize}

\end{myitemize}

\end{myitemize}


For example

\begin{Shaded}
\begin{Highlighting}[]

\NormalTok{\textbackslash{}pscircle[linewidth=0.2,linestyle=dashed,linecolor=blue](0,0)\{1\}}\newline
\end{Highlighting}
\end{Shaded}


To apply parameters globally:

\begin{Shaded}
\begin{Highlighting}[]

\NormalTok{\textbackslash{}psset\{linewidth=0.2,linestyle=dashed,linecolor=blue\}}\newline
\NormalTok{\textbackslash{}pscircle(0,0)\{1\}}\newline
\end{Highlighting}
\end{Shaded}


This command also lets you change the default unit for lengths.
\begin{myitemize}
\item{}  
\begin{Shaded}
\begin{Highlighting}[]

\NormalTok{unit=value}\newline
\end{Highlighting}
\end{Shaded}

\item{}  
\begin{Shaded}
\begin{Highlighting}[]

\NormalTok{xunit=value}\newline
\end{Highlighting}
\end{Shaded}
 and 
\begin{Shaded}
\begin{Highlighting}[]

\NormalTok{yunit=value}\newline
\end{Highlighting}
\end{Shaded}

\end{myitemize}


{\itshape \setmainfont[Path=/usr/share/fonts/truetype/cmu/,UprightFont=cmunrm.ttf,BoldFont=cmunbx.ttf,ItalicFont=cmunti.ttf,BoldItalicFont=cmunbi.ttf]{cmunti.ttf}\setmonofont[Path=/usr/share/fonts/truetype/cmu/,UprightFont=cmuntt.ttf,BoldFont=cmuntb.ttf,ItalicFont=cmunit.ttf,BoldItalicFont=cmuntx.ttf]{cmunti.ttf}\itshape value}{$\text{ }$}\setmainfont[Path=/usr/share/fonts/truetype/cmu/,UprightFont=cmunrm.ttf,BoldFont=cmunbx.ttf,ItalicFont=cmunti.ttf,BoldItalicFont=cmunbi.ttf]{cmunrm.ttf}\setmonofont[Path=/usr/share/fonts/truetype/cmu/,UprightFont=cmuntt.ttf,BoldFont=cmuntb.ttf,ItalicFont=cmunit.ttf,BoldItalicFont=cmuntx.ttf]{cmunrm.ttf} is a number with or without unit. This changes the scale of the drawings, but will not change the width of lines.
\subsection{Open shapes}
\label{828}

You can define the extreme of an open shape (line, polyline, arc, etc.) with an optional parameter 
\begin{Shaded}
\begin{Highlighting}[]

\NormalTok{\{symbol1-symbol2\}}\newline
\end{Highlighting}
\end{Shaded}
. There is a decent list of available symbols.
\begin{myitemize}
\item{}  <{} or >{}: arrow.
\item{}  <{}<{} or >{}>{}: double arrow.
\item{}  |: bar.
\item{}  |*: centered bar.
\item{}  oo: circle.
\item{}  o: centered circle.
\item{}  **: disk.
\item{}  *: centered disk.
\item{}  |<{} or >{}|: arrow plus bar.
\item{}  cc: rounded extreme.
\item{}  c: centered rounded extreme.
\end{myitemize}


Example:

\begin{Shaded}
\begin{Highlighting}[]

\NormalTok{\textbackslash{}psline\{|->>\}(x0,y0)(x1,y1)}\newline
\end{Highlighting}
\end{Shaded}

\subsection{Closed shapes}
\label{829}

For closed shape you may define the fillstyle.

\begin{myitemize}
\item{}  
\begin{Shaded}
\begin{Highlighting}[]

\NormalTok{fillstyle=value}\newline
\end{Highlighting}
\end{Shaded}
: pattern. Possible values: 
\begin{Shaded}
\begin{Highlighting}[]

\NormalTok{crosshatch,\ensuremath{\text{ }}crosshatch*,\ensuremath{\text{ }}vlines,\ensuremath{\text{ }}vlines*,\ensuremath{\text{ }}hlines,\ensuremath{\text{ }}hlines*,\ensuremath{\text{ }}solid}\newline
\end{Highlighting}
\end{Shaded}
.
\item{}  
\begin{Shaded}
\begin{Highlighting}[]

\NormalTok{fillcolor=color}\newline
\end{Highlighting}
\end{Shaded}
.
\item{}  
\begin{Shaded}
\begin{Highlighting}[]

\NormalTok{hatchcolor=color}\newline
\end{Highlighting}
\end{Shaded}
.
\item{}  
\begin{Shaded}
\begin{Highlighting}[]

\NormalTok{hatchwidth=value}\newline
\end{Highlighting}
\end{Shaded}
.
\item{}  
\begin{Shaded}
\begin{Highlighting}[]

\NormalTok{hatchsep=value}\newline
\end{Highlighting}
\end{Shaded}
.
\item{}  
\begin{Shaded}
\begin{Highlighting}[]

\NormalTok{hatchangle=value}\newline
\end{Highlighting}
\end{Shaded}
.
\end{myitemize}


Example:

\begin{Shaded}
\begin{Highlighting}[]

\NormalTok{\textbackslash{}pscircle[hatchcolor=blue,fillstyle=vlines](0,0)\{1\}}\newline
\end{Highlighting}
\end{Shaded}

\section{Object location}
\label{830}

The 
\begin{Shaded}
\begin{Highlighting}[]

\NormalTok{\textbackslash{}rput}\newline
\end{Highlighting}
\end{Shaded}
 and 
\begin{Shaded}
\begin{Highlighting}[]

\NormalTok{\textbackslash{}uput}\newline
\end{Highlighting}
\end{Shaded}
 commands can be used to move any object.
{\bfseries
\begin{mydescription}Example
\end{mydescription}
}


\begin{Shaded}
\begin{Highlighting}[]

\NormalTok{\textbackslash{}begin\{pspicture\}(5,5)}\newline
\NormalTok{\textbackslash{}psline\{->\}(0,0)(1,1)}\newline
\NormalTok{\textbackslash{}rput(1,1)\{\textbackslash{}psline\{->\}(0,0)(1,1)\}}\newline
\NormalTok{\textbackslash{}end\{pspicture\}}\newline
\end{Highlighting}
\end{Shaded}

or

\begin{Shaded}
\begin{Highlighting}[]

\NormalTok{\textbackslash{}begin\{pspicture\}(5,5)}\newline
\NormalTok{\textbackslash{}psline\{->\}(0,0)(1,1)}\newline
\NormalTok{\textbackslash{}psline\{->\}(1,1)(2,2)}\newline
\NormalTok{\textbackslash{}end\{pspicture\}}\newline
\end{Highlighting}
\end{Shaded}

You can repeat the operation with 
\begin{Shaded}
\begin{Highlighting}[]

\NormalTok{\textbackslash{}multirput}\newline
\end{Highlighting}
\end{Shaded}
:

\begin{Shaded}
\begin{Highlighting}[]

\NormalTok{\textbackslash{}multirput(x0,y0)(xoffset,\ensuremath{\text{ }}yoffset)\{times\}\{object\}}\newline
\end{Highlighting}
\end{Shaded}

You can use the same options as for 
\begin{Shaded}
\begin{Highlighting}[]

\NormalTok{\textbackslash{}rput}\newline
\end{Highlighting}
\end{Shaded}
:

\begin{Shaded}
\begin{Highlighting}[]

\NormalTok{\textbackslash{}multirput[reference]\{angle\}(x0,y0)(xoffset,\ensuremath{\text{ }}yoffset)\{times\}\{object\}}\newline
\end{Highlighting}
\end{Shaded}


With no text but with graphics only, you can use the 
\begin{Shaded}
\begin{Highlighting}[]

\NormalTok{\textbackslash{}multips}\newline
\end{Highlighting}
\end{Shaded}
 command:

\begin{Shaded}
\begin{Highlighting}[]

\NormalTok{\textbackslash{}multips(x0,y0)(xoffset,\ensuremath{\text{ }}yoffset)\{times\}\{object\}}\newline
\NormalTok{\textbackslash{}multips\{angle\}(x0,y0)(xoffset,yoffset)\{times\}\{object\}}\newline
\end{Highlighting}
\end{Shaded}

\section{The {\ttfamily \setmainfont[Path=/usr/share/fonts/truetype/cmu/,UprightFont=cmunrm.ttf,BoldFont=cmunbx.ttf,ItalicFont=cmunti.ttf,BoldItalicFont=cmunbi.ttf]{cmuntt.ttf}\setmonofont[Path=/usr/share/fonts/truetype/cmu/,UprightFont=cmuntt.ttf,BoldFont=cmuntb.ttf,ItalicFont=cmunit.ttf,BoldItalicFont=cmuntx.ttf]{cmuntt.ttf}\ttfamily PDFTricks}{$\text{ }$}\setmainfont[Path=/usr/share/fonts/truetype/cmu/,UprightFont=cmunrm.ttf,BoldFont=cmunbx.ttf,ItalicFont=cmunti.ttf,BoldItalicFont=cmunbi.ttf]{cmunrm.ttf}\setmonofont[Path=/usr/share/fonts/truetype/cmu/,UprightFont=cmuntt.ttf,BoldFont=cmuntb.ttf,ItalicFont=cmunit.ttf,BoldItalicFont=cmuntx.ttf]{cmunrm.ttf} extension}
\label{831}
The original {\ttfamily \setmainfont[Path=/usr/share/fonts/truetype/cmu/,UprightFont=cmunrm.ttf,BoldFont=cmunbx.ttf,ItalicFont=cmunti.ttf,BoldItalicFont=cmunbi.ttf]{cmuntt.ttf}\setmonofont[Path=/usr/share/fonts/truetype/cmu/,UprightFont=cmuntt.ttf,BoldFont=cmuntb.ttf,ItalicFont=cmunit.ttf,BoldItalicFont=cmuntx.ttf]{cmuntt.ttf}\ttfamily PSTricks}{$\text{ }$}\setmainfont[Path=/usr/share/fonts/truetype/cmu/,UprightFont=cmunrm.ttf,BoldFont=cmunbx.ttf,ItalicFont=cmunti.ttf,BoldItalicFont=cmunbi.ttf]{cmunrm.ttf}\setmonofont[Path=/usr/share/fonts/truetype/cmu/,UprightFont=cmuntt.ttf,BoldFont=cmuntb.ttf,ItalicFont=cmunit.ttf,BoldItalicFont=cmuntx.ttf]{cmunrm.ttf} package does not work with {\ttfamily \setmainfont[Path=/usr/share/fonts/truetype/cmu/,UprightFont=cmunrm.ttf,BoldFont=cmunbx.ttf,ItalicFont=cmunti.ttf,BoldItalicFont=cmunbi.ttf]{cmuntt.ttf}\setmonofont[Path=/usr/share/fonts/truetype/cmu/,UprightFont=cmuntt.ttf,BoldFont=cmuntb.ttf,ItalicFont=cmunit.ttf,BoldItalicFont=cmuntx.ttf]{cmuntt.ttf}\ttfamily pdflatex}\setmainfont[Path=/usr/share/fonts/truetype/cmu/,UprightFont=cmunrm.ttf,BoldFont=cmunbx.ttf,ItalicFont=cmunti.ttf,BoldItalicFont=cmunbi.ttf]{cmunrm.ttf}\setmonofont[Path=/usr/share/fonts/truetype/cmu/,UprightFont=cmuntt.ttf,BoldFont=cmuntb.ttf,ItalicFont=cmunit.ttf,BoldItalicFont=cmuntx.ttf]{cmunrm.ttf}, but thankfully {\ttfamily \setmainfont[Path=/usr/share/fonts/truetype/cmu/,UprightFont=cmunrm.ttf,BoldFont=cmunbx.ttf,ItalicFont=cmunti.ttf,BoldItalicFont=cmunbi.ttf]{cmuntt.ttf}\setmonofont[Path=/usr/share/fonts/truetype/cmu/,UprightFont=cmuntt.ttf,BoldFont=cmuntb.ttf,ItalicFont=cmunit.ttf,BoldItalicFont=cmuntx.ttf]{cmuntt.ttf}\ttfamily PDFTricks}{$\text{ }$}\setmainfont[Path=/usr/share/fonts/truetype/cmu/,UprightFont=cmunrm.ttf,BoldFont=cmunbx.ttf,ItalicFont=cmunti.ttf,BoldItalicFont=cmunbi.ttf]{cmunrm.ttf}\setmonofont[Path=/usr/share/fonts/truetype/cmu/,UprightFont=cmuntt.ttf,BoldFont=cmuntb.ttf,ItalicFont=cmunit.ttf,BoldItalicFont=cmuntx.ttf]{cmunrm.ttf} allows us to bypass this limitation.
{\bfseries
\begin{mydescription}Usage
\end{mydescription}
}

\begin{myitemize}
\item{}  Declare the {\ttfamily \setmainfont[Path=/usr/share/fonts/truetype/cmu/,UprightFont=cmunrm.ttf,BoldFont=cmunbx.ttf,ItalicFont=cmunti.ttf,BoldItalicFont=cmunbi.ttf]{cmuntt.ttf}\setmonofont[Path=/usr/share/fonts/truetype/cmu/,UprightFont=cmuntt.ttf,BoldFont=cmuntb.ttf,ItalicFont=cmunit.ttf,BoldItalicFont=cmuntx.ttf]{cmuntt.ttf}\ttfamily PDFTricks}{$\text{ }$}\setmainfont[Path=/usr/share/fonts/truetype/cmu/,UprightFont=cmunrm.ttf,BoldFont=cmunbx.ttf,ItalicFont=cmunti.ttf,BoldItalicFont=cmunbi.ttf]{cmunrm.ttf}\setmonofont[Path=/usr/share/fonts/truetype/cmu/,UprightFont=cmuntt.ttf,BoldFont=cmuntb.ttf,ItalicFont=cmunit.ttf,BoldItalicFont=cmuntx.ttf]{cmunrm.ttf} packages in the preamble.
\item{}  Place all {\ttfamily \setmainfont[Path=/usr/share/fonts/truetype/cmu/,UprightFont=cmunrm.ttf,BoldFont=cmunbx.ttf,ItalicFont=cmunti.ttf,BoldItalicFont=cmunbi.ttf]{cmuntt.ttf}\setmonofont[Path=/usr/share/fonts/truetype/cmu/,UprightFont=cmuntt.ttf,BoldFont=cmuntb.ttf,ItalicFont=cmunit.ttf,BoldItalicFont=cmuntx.ttf]{cmuntt.ttf}\ttfamily PSTricks}{$\text{ }$}\setmainfont[Path=/usr/share/fonts/truetype/cmu/,UprightFont=cmunrm.ttf,BoldFont=cmunbx.ttf,ItalicFont=cmunti.ttf,BoldItalicFont=cmunbi.ttf]{cmunrm.ttf}\setmonofont[Path=/usr/share/fonts/truetype/cmu/,UprightFont=cmuntt.ttf,BoldFont=cmuntb.ttf,ItalicFont=cmunit.ttf,BoldItalicFont=cmuntx.ttf]{cmunrm.ttf} extensions in a 
\begin{Shaded}
\begin{Highlighting}[]

\NormalTok{psinputs}\newline
\end{Highlighting}
\end{Shaded}
 environment; place all {\ttfamily \setmainfont[Path=/usr/share/fonts/truetype/cmu/,UprightFont=cmunrm.ttf,BoldFont=cmunbx.ttf,ItalicFont=cmunti.ttf,BoldItalicFont=cmunbi.ttf]{cmuntt.ttf}\setmonofont[Path=/usr/share/fonts/truetype/cmu/,UprightFont=cmuntt.ttf,BoldFont=cmuntb.ttf,ItalicFont=cmunit.ttf,BoldItalicFont=cmuntx.ttf]{cmuntt.ttf}\ttfamily PSTricks}{$\text{ }$}\setmainfont[Path=/usr/share/fonts/truetype/cmu/,UprightFont=cmunrm.ttf,BoldFont=cmunbx.ttf,ItalicFont=cmunti.ttf,BoldItalicFont=cmunbi.ttf]{cmunrm.ttf}\setmonofont[Path=/usr/share/fonts/truetype/cmu/,UprightFont=cmuntt.ttf,BoldFont=cmuntb.ttf,ItalicFont=cmunit.ttf,BoldItalicFont=cmuntx.ttf]{cmunrm.ttf} commands in a 
\begin{Shaded}
\begin{Highlighting}[]

\NormalTok{pdfpic}\newline
\end{Highlighting}
\end{Shaded}
 environment.
\item{}  Compile with {\ttfamily \setmainfont[Path=/usr/share/fonts/truetype/cmu/,UprightFont=cmunrm.ttf,BoldFont=cmunbx.ttf,ItalicFont=cmunti.ttf,BoldItalicFont=cmunbi.ttf]{cmuntt.ttf}\setmonofont[Path=/usr/share/fonts/truetype/cmu/,UprightFont=cmuntt.ttf,BoldFont=cmuntb.ttf,ItalicFont=cmunit.ttf,BoldItalicFont=cmuntx.ttf]{cmuntt.ttf}\ttfamily pdflatex -{}shell-{}escape <{}file>{}}\setmainfont[Path=/usr/share/fonts/truetype/cmu/,UprightFont=cmunrm.ttf,BoldFont=cmunbx.ttf,ItalicFont=cmunti.ttf,BoldItalicFont=cmunbi.ttf]{cmunrm.ttf}\setmonofont[Path=/usr/share/fonts/truetype/cmu/,UprightFont=cmuntt.ttf,BoldFont=cmuntb.ttf,ItalicFont=cmunit.ttf,BoldItalicFont=cmuntx.ttf]{cmunrm.ttf}.
\end{myitemize}


The {\ttfamily \setmainfont[Path=/usr/share/fonts/truetype/cmu/,UprightFont=cmunrm.ttf,BoldFont=cmunbx.ttf,ItalicFont=cmunti.ttf,BoldItalicFont=cmunbi.ttf]{cmuntt.ttf}\setmonofont[Path=/usr/share/fonts/truetype/cmu/,UprightFont=cmuntt.ttf,BoldFont=cmuntb.ttf,ItalicFont=cmunit.ttf,BoldItalicFont=cmuntx.ttf]{cmuntt.ttf}\ttfamily -{}shell-{}escape}{$\text{ }$}\setmainfont[Path=/usr/share/fonts/truetype/cmu/,UprightFont=cmunrm.ttf,BoldFont=cmunbx.ttf,ItalicFont=cmunti.ttf,BoldItalicFont=cmunbi.ttf]{cmunrm.ttf}\setmonofont[Path=/usr/share/fonts/truetype/cmu/,UprightFont=cmuntt.ttf,BoldFont=cmuntb.ttf,ItalicFont=cmunit.ttf,BoldItalicFont=cmuntx.ttf]{cmunrm.ttf} parameter enables shell command calls. It is required for {\ttfamily \setmainfont[Path=/usr/share/fonts/truetype/cmu/,UprightFont=cmunrm.ttf,BoldFont=cmunbx.ttf,ItalicFont=cmunti.ttf,BoldItalicFont=cmunbi.ttf]{cmuntt.ttf}\setmonofont[Path=/usr/share/fonts/truetype/cmu/,UprightFont=cmuntt.ttf,BoldFont=cmuntb.ttf,ItalicFont=cmunit.ttf,BoldItalicFont=cmuntx.ttf]{cmuntt.ttf}\ttfamily PDFTricks}{$\text{ }$}\setmainfont[Path=/usr/share/fonts/truetype/cmu/,UprightFont=cmunrm.ttf,BoldFont=cmunbx.ttf,ItalicFont=cmunti.ttf,BoldItalicFont=cmunbi.ttf]{cmunrm.ttf}\setmonofont[Path=/usr/share/fonts/truetype/cmu/,UprightFont=cmuntt.ttf,BoldFont=cmuntb.ttf,ItalicFont=cmunit.ttf,BoldItalicFont=cmuntx.ttf]{cmunrm.ttf} to run.
{\bfseries
\begin{mydescription}Example
\end{mydescription}
}


\begin{Shaded}
\begin{Highlighting}[]

\NormalTok{\textbackslash{}documentclass\{article\}}\newline
\ensuremath{\text{ }}\newline
\NormalTok{\textbackslash{}usepackage\{pdftricks\}}\newline
\NormalTok{\textbackslash{}begin\{psinputs\}}\newline
\ensuremath{\text{ }}\ensuremath{\text{ }}\ensuremath{\text{ }}\NormalTok{\textbackslash{}usepackage\{pstricks\}}\newline
\ensuremath{\text{ }}\ensuremath{\text{ }}\ensuremath{\text{ }}\NormalTok{\textbackslash{}usepackage\{multido\}}\newline
\NormalTok{\textbackslash{}end\{psinputs\}}\newline
\ensuremath{\text{ }}\newline
\CommentTok{\%\ensuremath{\text{ }}...}\newline
\ensuremath{\text{ }}\newline
\NormalTok{\textbackslash{}begin\{document\}}\newline
\ensuremath{\text{ }}\newline
\CommentTok{\%\ensuremath{\text{ }}...}\newline
\ensuremath{\text{ }}\newline
\NormalTok{\textbackslash{}begin\{pdfpic\}}\newline
\ensuremath{\text{ }}\ensuremath{\text{ }}\ensuremath{\text{ }}\NormalTok{\textbackslash{}psset\{unit=\textbackslash{}linewidth\}}\newline
\ensuremath{\text{ }}\ensuremath{\text{ }}\ensuremath{\text{ }}\NormalTok{\textbackslash{}begin\{pspicture\}(0,0)(10,10)}\newline
\ensuremath{\text{ }}\ensuremath{\text{ }}\ensuremath{\text{ }}\ensuremath{\text{ }}\ensuremath{\text{ }}\ensuremath{\text{ }}\NormalTok{[...]}\newline
\ensuremath{\text{ }}\ensuremath{\text{ }}\ensuremath{\text{ }}\NormalTok{\textbackslash{}end\{pspicture\}}\newline
\NormalTok{\textbackslash{}end\{pdfpic\}}\newline
\ensuremath{\text{ }}\newline
\CommentTok{\%\ensuremath{\text{ }}...}\newline
\ensuremath{\text{ }}\newline
\NormalTok{\textbackslash{}end\{document\}}\newline
\end{Highlighting}
\end{Shaded}


Another way to use {\ttfamily \setmainfont[Path=/usr/share/fonts/truetype/cmu/,UprightFont=cmunrm.ttf,BoldFont=cmunbx.ttf,ItalicFont=cmunti.ttf,BoldItalicFont=cmunbi.ttf]{cmuntt.ttf}\setmonofont[Path=/usr/share/fonts/truetype/cmu/,UprightFont=cmuntt.ttf,BoldFont=cmuntb.ttf,ItalicFont=cmunit.ttf,BoldItalicFont=cmuntx.ttf]{cmuntt.ttf}\ttfamily PSTricks}{$\text{ }$}\setmainfont[Path=/usr/share/fonts/truetype/cmu/,UprightFont=cmunrm.ttf,BoldFont=cmunbx.ttf,ItalicFont=cmunti.ttf,BoldItalicFont=cmunbi.ttf]{cmunrm.ttf}\setmonofont[Path=/usr/share/fonts/truetype/cmu/,UprightFont=cmuntt.ttf,BoldFont=cmuntb.ttf,ItalicFont=cmunit.ttf,BoldItalicFont=cmuntx.ttf]{cmunrm.ttf} with {\ttfamily \setmainfont[Path=/usr/share/fonts/truetype/cmu/,UprightFont=cmunrm.ttf,BoldFont=cmunbx.ttf,ItalicFont=cmunti.ttf,BoldItalicFont=cmunbi.ttf]{cmuntt.ttf}\setmonofont[Path=/usr/share/fonts/truetype/cmu/,UprightFont=cmuntt.ttf,BoldFont=cmuntb.ttf,ItalicFont=cmunit.ttf,BoldItalicFont=cmuntx.ttf]{cmuntt.ttf}\ttfamily pdflatex}{$\text{ }$}\setmainfont[Path=/usr/share/fonts/truetype/cmu/,UprightFont=cmunrm.ttf,BoldFont=cmunbx.ttf,ItalicFont=cmunti.ttf,BoldItalicFont=cmunbi.ttf]{cmunrm.ttf}\setmonofont[Path=/usr/share/fonts/truetype/cmu/,UprightFont=cmuntt.ttf,BoldFont=cmuntb.ttf,ItalicFont=cmunit.ttf,BoldItalicFont=cmuntx.ttf]{cmunrm.ttf} is the {\ttfamily \setmainfont[Path=/usr/share/fonts/truetype/cmu/,UprightFont=cmunrm.ttf,BoldFont=cmunbx.ttf,ItalicFont=cmunti.ttf,BoldItalicFont=cmunbi.ttf]{cmuntt.ttf}\setmonofont[Path=/usr/share/fonts/truetype/cmu/,UprightFont=cmuntt.ttf,BoldFont=cmuntb.ttf,ItalicFont=cmunit.ttf,BoldItalicFont=cmuntx.ttf]{cmuntt.ttf}\ttfamily pst-{}pdf}{$\text{ }$}\setmainfont[Path=/usr/share/fonts/truetype/cmu/,UprightFont=cmunrm.ttf,BoldFont=cmunbx.ttf,ItalicFont=cmunti.ttf,BoldItalicFont=cmunbi.ttf]{cmunrm.ttf}\setmonofont[Path=/usr/share/fonts/truetype/cmu/,UprightFont=cmuntt.ttf,BoldFont=cmuntb.ttf,ItalicFont=cmunit.ttf,BoldItalicFont=cmuntx.ttf]{cmunrm.ttf} package.



\myhref{https://sr.wikibooks.org/wiki/LaTeX\%2FPSTricks}{sr:LaTeX/PSTricks}\chapter{Xy-{}pic}

\myminitoc
\label{832}

\label{833}



{\ttfamily \setmainfont[Path=/usr/share/fonts/truetype/cmu/,UprightFont=cmunrm.ttf,BoldFont=cmunbx.ttf,ItalicFont=cmunti.ttf,BoldItalicFont=cmunbi.ttf]{cmuntt.ttf}\setmonofont[Path=/usr/share/fonts/truetype/cmu/,UprightFont=cmuntt.ttf,BoldFont=cmuntb.ttf,ItalicFont=cmunit.ttf,BoldItalicFont=cmuntx.ttf]{cmuntt.ttf}\ttfamily xy}{$\text{ }$}\setmainfont[Path=/usr/share/fonts/truetype/cmu/,UprightFont=cmunrm.ttf,BoldFont=cmunbx.ttf,ItalicFont=cmunti.ttf,BoldItalicFont=cmunbi.ttf]{cmunrm.ttf}\setmonofont[Path=/usr/share/fonts/truetype/cmu/,UprightFont=cmuntt.ttf,BoldFont=cmuntb.ttf,ItalicFont=cmunit.ttf,BoldItalicFont=cmuntx.ttf]{cmunrm.ttf} is a special package for drawing diagrams. To use it, simply add the
following line to the preamble of your document:

\begin{Shaded}
\begin{Highlighting}[]

\NormalTok{\textbackslash{}usepackage[all]\{xy\}}\newline
\end{Highlighting}
\end{Shaded}

where \symbol{34}all\symbol{34} means you want to load a large standard set of functions from {\itshape \setmainfont[Path=/usr/share/fonts/truetype/cmu/,UprightFont=cmunrm.ttf,BoldFont=cmunbx.ttf,ItalicFont=cmunti.ttf,BoldItalicFont=cmunbi.ttf]{cmunti.ttf}\setmonofont[Path=/usr/share/fonts/truetype/cmu/,UprightFont=cmuntt.ttf,BoldFont=cmuntb.ttf,ItalicFont=cmunit.ttf,BoldItalicFont=cmuntx.ttf]{cmunti.ttf}\itshape Xy-{}pic}\setmainfont[Path=/usr/share/fonts/truetype/cmu/,UprightFont=cmunrm.ttf,BoldFont=cmunbx.ttf,ItalicFont=cmunti.ttf,BoldItalicFont=cmunbi.ttf]{cmunrm.ttf}\setmonofont[Path=/usr/share/fonts/truetype/cmu/,UprightFont=cmuntt.ttf,BoldFont=cmuntb.ttf,ItalicFont=cmunit.ttf,BoldItalicFont=cmuntx.ttf]{cmunrm.ttf}, suitable for developing the kind of diagrams discussed here.

The primary way to draw {\itshape \setmainfont[Path=/usr/share/fonts/truetype/cmu/,UprightFont=cmunrm.ttf,BoldFont=cmunbx.ttf,ItalicFont=cmunti.ttf,BoldItalicFont=cmunbi.ttf]{cmunti.ttf}\setmonofont[Path=/usr/share/fonts/truetype/cmu/,UprightFont=cmuntt.ttf,BoldFont=cmuntb.ttf,ItalicFont=cmunit.ttf,BoldItalicFont=cmuntx.ttf]{cmunti.ttf}\itshape Xy-{}pic}{$\text{ }$}\setmainfont[Path=/usr/share/fonts/truetype/cmu/,UprightFont=cmunrm.ttf,BoldFont=cmunbx.ttf,ItalicFont=cmunti.ttf,BoldItalicFont=cmunbi.ttf]{cmunrm.ttf}\setmonofont[Path=/usr/share/fonts/truetype/cmu/,UprightFont=cmuntt.ttf,BoldFont=cmuntb.ttf,ItalicFont=cmunit.ttf,BoldItalicFont=cmuntx.ttf]{cmunrm.ttf} diagrams is over a matrix-{}oriented canvas, where each diagram element is placed in a matrix slot:

\begin{longtable}{p{1.0\linewidth}}
\begin{Shaded}
\begin{Highlighting}[]
\NormalTok{\textbackslash{}begin\{displaymath\}}
    \NormalTok{\textbackslash{}xymatrix\{A & B \textbackslash{}\textbackslash{}}
              \NormalTok{C & D \}}
\NormalTok{\textbackslash{}end\{displaymath\}}
\end{Highlighting}
\end{Shaded}
\\


\begin{minipage}{0.37500\textwidth}
\begin{center}
\includegraphics[width=1.0\textwidth,height=6.5in,keepaspectratio]{../images/200.png}
\end{center}
\raggedright{}\myfigurewithoutcaption{200}
\end{minipage}\vspace{0.75cm}


\end{longtable}

The {\ttfamily \setmainfont[Path=/usr/share/fonts/truetype/cmu/,UprightFont=cmunrm.ttf,BoldFont=cmunbx.ttf,ItalicFont=cmunti.ttf,BoldItalicFont=cmunbi.ttf]{cmuntt.ttf}\setmonofont[Path=/usr/share/fonts/truetype/cmu/,UprightFont=cmuntt.ttf,BoldFont=cmuntb.ttf,ItalicFont=cmunit.ttf,BoldItalicFont=cmuntx.ttf]{cmuntt.ttf}\ttfamily \textbackslash{}xymatrix}{$\text{ }$}\setmainfont[Path=/usr/share/fonts/truetype/cmu/,UprightFont=cmunrm.ttf,BoldFont=cmunbx.ttf,ItalicFont=cmunti.ttf,BoldItalicFont=cmunbi.ttf]{cmunrm.ttf}\setmonofont[Path=/usr/share/fonts/truetype/cmu/,UprightFont=cmuntt.ttf,BoldFont=cmuntb.ttf,ItalicFont=cmunit.ttf,BoldItalicFont=cmuntx.ttf]{cmunrm.ttf} command must be used in math mode. Here, we specified two lines and two columns. To make this matrix a diagram we just add directed arrows using the {\ttfamily \setmainfont[Path=/usr/share/fonts/truetype/cmu/,UprightFont=cmunrm.ttf,BoldFont=cmunbx.ttf,ItalicFont=cmunti.ttf,BoldItalicFont=cmunbi.ttf]{cmuntt.ttf}\setmonofont[Path=/usr/share/fonts/truetype/cmu/,UprightFont=cmuntt.ttf,BoldFont=cmuntb.ttf,ItalicFont=cmunit.ttf,BoldItalicFont=cmuntx.ttf]{cmuntt.ttf}\ttfamily \textbackslash{}ar}{$\text{ }$}\setmainfont[Path=/usr/share/fonts/truetype/cmu/,UprightFont=cmunrm.ttf,BoldFont=cmunbx.ttf,ItalicFont=cmunti.ttf,BoldItalicFont=cmunbi.ttf]{cmunrm.ttf}\setmonofont[Path=/usr/share/fonts/truetype/cmu/,UprightFont=cmuntt.ttf,BoldFont=cmuntb.ttf,ItalicFont=cmunit.ttf,BoldItalicFont=cmuntx.ttf]{cmunrm.ttf} command.

\begin{longtable}{p{1.0\linewidth}}
\begin{Shaded}
\begin{Highlighting}[]
\NormalTok{\textbackslash{}begin\{displaymath\}}
    \NormalTok{\textbackslash{}xymatrix\{ A \textbackslash{}ar[r] & B \textbackslash{}ar[d] \textbackslash{}\textbackslash{}}
               \NormalTok{D \textbackslash{}ar[u] & C \textbackslash{}ar[l] \}}
\NormalTok{\textbackslash{}end\{displaymath\}}
\end{Highlighting}
\end{Shaded}
\\


\begin{minipage}{0.37500\textwidth}
\begin{center}
\includegraphics[width=1.0\textwidth,height=6.5in,keepaspectratio]{../images/201.png}
\end{center}
\raggedright{}\myfigurewithoutcaption{201}
\end{minipage}\vspace{0.75cm}


\end{longtable}

The arrow command is placed on the origin cell for the arrow. The arguments are the direction the arrow should point to (up, down, right and left).

\begin{longtable}{p{1.0\linewidth}}
\begin{Shaded}
\begin{Highlighting}[]
\NormalTok{\textbackslash{}begin\{displaymath\}}
    \NormalTok{\textbackslash{}xymatrix\{}
        \NormalTok{A \textbackslash{}ar[d] \textbackslash{}ar[dr] \textbackslash{}ar[r] & B \textbackslash{}\textbackslash{}}
        \NormalTok{D                       & C \}}
\NormalTok{\textbackslash{}end\{displaymath\}}
\end{Highlighting}
\end{Shaded}
\\


\begin{minipage}{0.37500\textwidth}
\begin{center}
\includegraphics[width=1.0\textwidth,height=6.5in,keepaspectratio]{../images/202.png}
\end{center}
\raggedright{}\myfigurewithoutcaption{202}
\end{minipage}\vspace{0.75cm}


\end{longtable}

To make diagonals, just use more than one direction. In fact, you can repeat directions to make bigger arrows.


\begin{longtable}{p{1.0\linewidth}}
\begin{Shaded}
\begin{Highlighting}[]
\NormalTok{\textbackslash{}begin\{displaymath\}}
    \NormalTok{\textbackslash{}xymatrix\{}
        \NormalTok{A \textbackslash{}ar[d] \textbackslash{}ar[dr] \textbackslash{}ar[drr] &   &   \textbackslash{}\textbackslash{}}
        \NormalTok{B                         & C & D \}}
\NormalTok{\textbackslash{}end\{displaymath\}}
\end{Highlighting}
\end{Shaded}
\\


\begin{minipage}{0.50000\textwidth}
\begin{center}
\includegraphics[width=1.0\textwidth,height=6.5in,keepaspectratio]{../images/203.png}
\end{center}
\raggedright{}\myfigurewithoutcaption{203}
\end{minipage}\vspace{0.75cm}


\end{longtable}

We can draw even more interesting diagrams by adding labels to the arrows. To do this, we use the common superscript and subscript operators.

\begin{longtable}{p{1.0\linewidth}}
\begin{Shaded}
\begin{Highlighting}[]
\NormalTok{\textbackslash{}begin\{displaymath\}}
    \NormalTok{\textbackslash{}xymatrix\{}
        \NormalTok{A \textbackslash{}ar[r]^f \textbackslash{}ar[d]_g & B \textbackslash{}ar[d]^\{g'\} \textbackslash{}\textbackslash{}}
        \NormalTok{D \textbackslash{}ar[r]_\{f'\}       & C \}}
\NormalTok{\textbackslash{}end\{displaymath\}}
\end{Highlighting}
\end{Shaded}
\\


\begin{minipage}{0.37500\textwidth}
\begin{center}
\includegraphics[width=1.0\textwidth,height=6.5in,keepaspectratio]{../images/204.png}
\end{center}
\raggedright{}\myfigurewithoutcaption{204}
\end{minipage}\vspace{0.75cm}


\end{longtable}

As shown, you use these operators as in math mode. The only difference is that that superscript means \symbol{34}on top of the arrow\symbol{34}, and subscript means \symbol{34}under the arrow\symbol{34}. There is a third operator, the vertical bar: | It causes text to be placed in the arrow.

\begin{longtable}{p{1.0\linewidth}}
\begin{Shaded}
\begin{Highlighting}[]
\NormalTok{\textbackslash{}begin\{displaymath\}}
    \NormalTok{\textbackslash{}xymatrix\{}
        \NormalTok{A \textbackslash{}ar[r]f \textbackslash{}ar[d]g & B \textbackslash{}ar[d]\{g'\} \textbackslash{}\textbackslash{}}
        \NormalTok{D \textbackslash{}ar[r]\{f'\}       & C \}}
\NormalTok{\textbackslash{}end\{displaymath\}}
\end{Highlighting}
\end{Shaded}
\\


\begin{minipage}{0.37500\textwidth}
\begin{center}
\includegraphics[width=1.0\textwidth,height=6.5in,keepaspectratio]{../images/205.png}
\end{center}
\raggedright{}\myfigurewithoutcaption{205}
\end{minipage}\vspace{0.75cm}


\end{longtable}

To draw an arrow with a hole in it, use {\ttfamily \setmainfont[Path=/usr/share/fonts/truetype/cmu/,UprightFont=cmunrm.ttf,BoldFont=cmunbx.ttf,ItalicFont=cmunti.ttf,BoldItalicFont=cmunbi.ttf]{cmuntt.ttf}\setmonofont[Path=/usr/share/fonts/truetype/cmu/,UprightFont=cmuntt.ttf,BoldFont=cmuntb.ttf,ItalicFont=cmunit.ttf,BoldItalicFont=cmuntx.ttf]{cmuntt.ttf}\ttfamily \textbackslash{}ar{$\text{[}$}...{$\text{]}$}|\textbackslash{}hole}\setmainfont[Path=/usr/share/fonts/truetype/cmu/,UprightFont=cmunrm.ttf,BoldFont=cmunbx.ttf,ItalicFont=cmunti.ttf,BoldItalicFont=cmunbi.ttf]{cmunrm.ttf}\setmonofont[Path=/usr/share/fonts/truetype/cmu/,UprightFont=cmuntt.ttf,BoldFont=cmuntb.ttf,ItalicFont=cmunit.ttf,BoldItalicFont=cmuntx.ttf]{cmunrm.ttf}. In some situations, it is important to distinguish between different types of arrows. This can be done by putting labels on them, or changing their appearance

\begin{longtable}{p{1.0\linewidth}}
\begin{Shaded}
\begin{Highlighting}[]

\NormalTok{\textbackslash{}begin\{displaymath\}}
    \NormalTok{\textbackslash{}xymatrix\{}
        \NormalTok{\textbackslash{}bullet\textbackslash{}ar@\{->\}[rr]     && \textbackslash{}bullet\textbackslash{}\textbackslash{}}
        \NormalTok{\textbackslash{}bullet\textbackslash{}ar@\{.<\}[rr]     && \textbackslash{}bullet\textbackslash{}\textbackslash{}}
        \NormalTok{\textbackslash{}bullet\textbackslash{}ar@\{~)\}[rr]     && \textbackslash{}bullet\textbackslash{}\textbackslash{}}
        \NormalTok{\textbackslash{}bullet\textbackslash{}ar@\{=(\}[rr]     && \textbackslash{}bullet\textbackslash{}\textbackslash{}}
        \NormalTok{\textbackslash{}bullet\textbackslash{}ar@\{~/\}[rr]     && \textbackslash{}bullet\textbackslash{}\textbackslash{}}
        \NormalTok{\textbackslash{}bullet\textbackslash{}ar@\{^\{(\}->\}[rr] && \textbackslash{}bullet\textbackslash{}\textbackslash{}}
        \NormalTok{\textbackslash{}bullet\textbackslash{}ar@2\{->\}[rr]    && \textbackslash{}bullet\textbackslash{}\textbackslash{}}
        \NormalTok{\textbackslash{}bullet\textbackslash{}ar@3\{->\}[rr]    && \textbackslash{}bullet\textbackslash{}\textbackslash{}}
        \NormalTok{\textbackslash{}bullet\textbackslash{}ar@\{=+\}[rr]     && \textbackslash{}bullet \}}
\NormalTok{\textbackslash{}end\{displaymath\}}
\end{Highlighting}
\end{Shaded}
\\


\begin{minipage}{0.35000\textwidth}
\begin{center}
\includegraphics[width=1.0\textwidth,height=6.5in,keepaspectratio]{../images/206.png}
\end{center}
\raggedright{}\myfigurewithoutcaption{206}
\end{minipage}\vspace{0.75cm}


\end{longtable}

Notice the difference between the following two diagrams:

\begin{longtable}{p{1.0\linewidth}}
\begin{Shaded}
\begin{Highlighting}[]
\NormalTok{\textbackslash{}begin\{displaymath\}}
    \NormalTok{\textbackslash{}xymatrix\{ \textbackslash{}bullet \textbackslash{}ar[r] \textbackslash{}ar@\{.>\}[r] & \textbackslash{}bullet \}}
\NormalTok{\textbackslash{}end\{displaymath\}}
\end{Highlighting}
\end{Shaded}
\\


\begin{minipage}{0.37500\textwidth}
\begin{center}
\includegraphics[width=1.0\textwidth,height=6.5in,keepaspectratio]{../images/207.png}
\end{center}
\raggedright{}\myfigurewithoutcaption{207}
\end{minipage}\vspace{0.75cm}


\end{longtable}

\begin{longtable}{p{1.0\linewidth}}
\begin{Shaded}
\begin{Highlighting}[]
\NormalTok{\textbackslash{}begin\{displaymath\}}
    \NormalTok{\textbackslash{}xymatrix\{}
        \NormalTok{\textbackslash{}bullet \textbackslash{}ar@/^/[r]}
        \NormalTok{\textbackslash{}ar@/_/@\{.>\}[r] &}
        \NormalTok{\textbackslash{}bullet \}}
\NormalTok{\textbackslash{}end\{displaymath\}}
\end{Highlighting}
\end{Shaded}
\\


\begin{minipage}{0.37500\textwidth}
\begin{center}
\includegraphics[width=1.0\textwidth,height=6.5in,keepaspectratio]{../images/208.png}
\end{center}
\raggedright{}\myfigurewithoutcaption{208}
\end{minipage}\vspace{0.75cm}


\end{longtable}

The modifiers between the slashes define how the curves are drawn. {\itshape \setmainfont[Path=/usr/share/fonts/truetype/cmu/,UprightFont=cmunrm.ttf,BoldFont=cmunbx.ttf,ItalicFont=cmunti.ttf,BoldItalicFont=cmunbi.ttf]{cmunti.ttf}\setmonofont[Path=/usr/share/fonts/truetype/cmu/,UprightFont=cmuntt.ttf,BoldFont=cmuntb.ttf,ItalicFont=cmunit.ttf,BoldItalicFont=cmuntx.ttf]{cmunti.ttf}\itshape Xy-{}pic}{$\text{ }$}\setmainfont[Path=/usr/share/fonts/truetype/cmu/,UprightFont=cmunrm.ttf,BoldFont=cmunbx.ttf,ItalicFont=cmunti.ttf,BoldItalicFont=cmunbi.ttf]{cmunrm.ttf}\setmonofont[Path=/usr/share/fonts/truetype/cmu/,UprightFont=cmuntt.ttf,BoldFont=cmuntb.ttf,ItalicFont=cmunit.ttf,BoldItalicFont=cmuntx.ttf]{cmunrm.ttf} offers many ways to influence the drawing of curves; for more information, check the {\itshape \setmainfont[Path=/usr/share/fonts/truetype/cmu/,UprightFont=cmunrm.ttf,BoldFont=cmunbx.ttf,ItalicFont=cmunti.ttf,BoldItalicFont=cmunbi.ttf]{cmunti.ttf}\setmonofont[Path=/usr/share/fonts/truetype/cmu/,UprightFont=cmuntt.ttf,BoldFont=cmuntb.ttf,ItalicFont=cmunit.ttf,BoldItalicFont=cmuntx.ttf]{cmunti.ttf}\itshape Xy-{}pic}{$\text{ }$}\setmainfont[Path=/usr/share/fonts/truetype/cmu/,UprightFont=cmunrm.ttf,BoldFont=cmunbx.ttf,ItalicFont=cmunti.ttf,BoldItalicFont=cmunbi.ttf]{cmunrm.ttf}\setmonofont[Path=/usr/share/fonts/truetype/cmu/,UprightFont=cmuntt.ttf,BoldFont=cmuntb.ttf,ItalicFont=cmunit.ttf,BoldItalicFont=cmuntx.ttf]{cmunrm.ttf} documentation.

If you are interested in a more thorough introduction then consult the \myhref{http://xy-pic.sourceforge.net}{Xy-{}pic Home Page}, which contains links to several other tutorials as well as the reference documentation.




\myhref{https://sr.wikibooks.org/wiki/LaTeX\%2FXy-pic}{sr:LaTeX/Xy-{}pic}\chapter{Creating 3D graphics}

\myminitoc
\label{834}

\label{835}


\LaTeXNullTemplate{}

For creating three-{}dimensional graphics, there is basic functionality in the \mylref{793}{PGF/TikZ} package, although drawing 3D graphics with PGF/TikZ is very non-{}flexible, mainly because it lacks functionality for identifying the surfaces that are covered by other surfaces and should be excluded from the rendered image.

A package that can handle this correctly is the \myhref{http://www.ctan.org/pkg/pst-solides3d}{{\ttfamily \setmainfont[Path=/usr/share/fonts/truetype/cmu/,UprightFont=cmunrm.ttf,BoldFont=cmunbx.ttf,ItalicFont=cmunti.ttf,BoldItalicFont=cmunbi.ttf]{cmuntt.ttf}\setmonofont[Path=/usr/share/fonts/truetype/cmu/,UprightFont=cmuntt.ttf,BoldFont=cmuntb.ttf,ItalicFont=cmunit.ttf,BoldItalicFont=cmuntx.ttf]{cmuntt.ttf}\ttfamily pst-{}solides3d}} package.

Another way to create 3D graphics is to use \myhref{https://en.wikipedia.org/wiki/Asymptote\%20\%28vector\%20graphics\%20language\%29}{Asymptote}.

Yet another way to create 3D graphics is to use \myhref{http://sketch4latex.sourceforge.net/}{Sketch}.

\LaTeXNullTemplate{}

\myhref{https://sr.wikibooks.org/wiki/LaTeX\%2F\%D0\%9F\%D1\%80\%D0\%B0\%D0\%B2\%D1\%99\%D0\%B5\%D1\%9A\%D0\%B5\%203D\%20\%D0\%B3\%D1\%80\%D0\%B0\%D1\%84\%D0\%B8\%D0\%BAa}{sr:LaTeX/Прављење 3D графикa}
\mypart{Programming}\chapter{Macros}

\myminitoc
\label{836}

\label{837}


Documents produced with the commands you have learned up to this point will look acceptable to a large audience. While they are not fancy-{}looking, they obey all the established rules of good typesetting, which will make them easy to read and pleasant to look at. However, there are situations where LaTeX does not provide a command or environment that matches your needs, or the output produced by some existing command may not meet your requirements.

In this chapter, we will try to give some hints on how to teach LaTeX new tricks and how to make it produce output that looks different from what is provided by default.

LaTeX is a fairly high-{}level language compared to Plain TeX and thus is more limited. The next \mylref{852}{chapter} will focus on Plain TeX and will explain advanced techniques for programming.
\section{New commands}
\label{838}

To add your own commands, use the
\begin{Shaded}
\begin{Highlighting}[]

\NormalTok{\textbackslash{}newcommand\{name\}[num]\{definition\}}
\end{Highlighting}
\end{Shaded}

command. Basically, the command requires two arguments: the {\itshape \setmainfont[Path=/usr/share/fonts/truetype/cmu/,UprightFont=cmunrm.ttf,BoldFont=cmunbx.ttf,ItalicFont=cmunti.ttf,BoldItalicFont=cmunbi.ttf]{cmunti.ttf}\setmonofont[Path=/usr/share/fonts/truetype/cmu/,UprightFont=cmuntt.ttf,BoldFont=cmuntb.ttf,ItalicFont=cmunit.ttf,BoldItalicFont=cmuntx.ttf]{cmunti.ttf}\itshape name}{$\text{ }$}\setmainfont[Path=/usr/share/fonts/truetype/cmu/,UprightFont=cmunrm.ttf,BoldFont=cmunbx.ttf,ItalicFont=cmunti.ttf,BoldItalicFont=cmunbi.ttf]{cmunrm.ttf}\setmonofont[Path=/usr/share/fonts/truetype/cmu/,UprightFont=cmuntt.ttf,BoldFont=cmuntb.ttf,ItalicFont=cmunit.ttf,BoldItalicFont=cmuntx.ttf]{cmunrm.ttf} of the command you want to create, and the {\itshape \setmainfont[Path=/usr/share/fonts/truetype/cmu/,UprightFont=cmunrm.ttf,BoldFont=cmunbx.ttf,ItalicFont=cmunti.ttf,BoldItalicFont=cmunbi.ttf]{cmunti.ttf}\setmonofont[Path=/usr/share/fonts/truetype/cmu/,UprightFont=cmuntt.ttf,BoldFont=cmuntb.ttf,ItalicFont=cmunit.ttf,BoldItalicFont=cmuntx.ttf]{cmunti.ttf}\itshape definition}{$\text{ }$}\setmainfont[Path=/usr/share/fonts/truetype/cmu/,UprightFont=cmunrm.ttf,BoldFont=cmunbx.ttf,ItalicFont=cmunti.ttf,BoldItalicFont=cmunbi.ttf]{cmunrm.ttf}\setmonofont[Path=/usr/share/fonts/truetype/cmu/,UprightFont=cmuntt.ttf,BoldFont=cmuntb.ttf,ItalicFont=cmunit.ttf,BoldItalicFont=cmuntx.ttf]{cmunrm.ttf} of the command. Note that the command {\itshape \setmainfont[Path=/usr/share/fonts/truetype/cmu/,UprightFont=cmunrm.ttf,BoldFont=cmunbx.ttf,ItalicFont=cmunti.ttf,BoldItalicFont=cmunbi.ttf]{cmunti.ttf}\setmonofont[Path=/usr/share/fonts/truetype/cmu/,UprightFont=cmuntt.ttf,BoldFont=cmuntb.ttf,ItalicFont=cmunit.ttf,BoldItalicFont=cmuntx.ttf]{cmunti.ttf}\itshape name}{$\text{ }$}\setmainfont[Path=/usr/share/fonts/truetype/cmu/,UprightFont=cmunrm.ttf,BoldFont=cmunbx.ttf,ItalicFont=cmunti.ttf,BoldItalicFont=cmunbi.ttf]{cmunrm.ttf}\setmonofont[Path=/usr/share/fonts/truetype/cmu/,UprightFont=cmuntt.ttf,BoldFont=cmuntb.ttf,ItalicFont=cmunit.ttf,BoldItalicFont=cmuntx.ttf]{cmunrm.ttf} can but need not be enclosed in braces, as you like. The {\itshape \setmainfont[Path=/usr/share/fonts/truetype/cmu/,UprightFont=cmunrm.ttf,BoldFont=cmunbx.ttf,ItalicFont=cmunti.ttf,BoldItalicFont=cmunbi.ttf]{cmunti.ttf}\setmonofont[Path=/usr/share/fonts/truetype/cmu/,UprightFont=cmuntt.ttf,BoldFont=cmuntb.ttf,ItalicFont=cmunit.ttf,BoldItalicFont=cmuntx.ttf]{cmunti.ttf}\itshape num}{$\text{ }$}\setmainfont[Path=/usr/share/fonts/truetype/cmu/,UprightFont=cmunrm.ttf,BoldFont=cmunbx.ttf,ItalicFont=cmunti.ttf,BoldItalicFont=cmunbi.ttf]{cmunrm.ttf}\setmonofont[Path=/usr/share/fonts/truetype/cmu/,UprightFont=cmuntt.ttf,BoldFont=cmuntb.ttf,ItalicFont=cmunit.ttf,BoldItalicFont=cmuntx.ttf]{cmunrm.ttf} argument in square brackets is optional and specifies the number of arguments the new command takes (up to 9 are possible). If missing it defaults to 0, i.e. no argument allowed.

The following two examples should help you to get the idea. The first example defines a new command called {\ttfamily \setmainfont[Path=/usr/share/fonts/truetype/cmu/,UprightFont=cmunrm.ttf,BoldFont=cmunbx.ttf,ItalicFont=cmunti.ttf,BoldItalicFont=cmunbi.ttf]{cmuntt.ttf}\setmonofont[Path=/usr/share/fonts/truetype/cmu/,UprightFont=cmuntt.ttf,BoldFont=cmuntb.ttf,ItalicFont=cmunit.ttf,BoldItalicFont=cmuntx.ttf]{cmuntt.ttf}\ttfamily \textbackslash{}wbal}{$\text{ }$}\setmainfont[Path=/usr/share/fonts/truetype/cmu/,UprightFont=cmunrm.ttf,BoldFont=cmunbx.ttf,ItalicFont=cmunti.ttf,BoldItalicFont=cmunbi.ttf]{cmunrm.ttf}\setmonofont[Path=/usr/share/fonts/truetype/cmu/,UprightFont=cmuntt.ttf,BoldFont=cmuntb.ttf,ItalicFont=cmunit.ttf,BoldItalicFont=cmuntx.ttf]{cmunrm.ttf} that will print “The Wikibook about LaTeX”. Such a command could come in handy if you had to write the title of this book over and over again.

\begin{longtable}{p{1.0\linewidth}}
\begin{Shaded}
\begin{Highlighting}[]

\NormalTok{\textbackslash{}newcommand\{\textbackslash{}wbal\}\{The Wikibook about \textbackslash{}LaTeX\}}
\NormalTok{This is ‘‘\textbackslash{}wbal}
\end{Highlighting}
\end{Shaded}
\\

This is “The Wikibook about LaTeX” … “The Wikibook about LaTeX”

\end{longtable}

The next example illustrates how to define a new command that takes one argument. The \LaTeXTT{\#1} tag gets replaced by the argument you specify. If you wanted to use more than one argument, use \LaTeXTT{\#2} and so on, these arguments are added in an extra set of brackets.

\begin{longtable}{p{1.0\linewidth}}
\begin{Shaded}
\begin{Highlighting}[]

\NormalTok{\textbackslash{}newcommand\{\textbackslash{}wbalsup\}[1] \{}
  \NormalTok{This is the Wikibook about LaTeX }
  \NormalTok{supported by #1\}}
\NormalTok{\textbackslash{}newcommand\{\textbackslash{}wbalTwo\}[2] \{}
  \NormalTok{This is the Wikibook about LaTeX}
  \NormalTok{supported by #1 and #2\}}
\CommentTok{% in the document body:}
\NormalTok{\textbackslash{}begin\{itemize\}}
\NormalTok{\textbackslash{}item \textbackslash{}wbalsup\{Wikimedia\}}
\NormalTok{\textbackslash{}item \textbackslash{}wbalsup\{lots of users!\}}
\NormalTok{\textbackslash{}item \textbackslash{}wbalTwo\{John Doe\}\{Anthea Smith\}}
\NormalTok{\textbackslash{}end\{itemize\}}
\end{Highlighting}
\end{Shaded}
\\

\begin{myitemize}
\item{} This is the Wikibook about LaTeX supported by Wikimedia
\item{} This is the Wikibook about LaTeX supported by lots of users!
\item{} This is the Wikibook about LaTeX supported by John Doe and Anthea Smith

\end{myitemize}

\end{longtable}

Name your new command \LaTeXTT{\textbackslash{}wbalTwo} and not \LaTeXTT{\textbackslash{}wbal2} as digits cannot be used to name macros {\mbox{$\text{---}$}} invalid characters will error out at compile-{}time.

LaTeX will not allow you to create a new command that would overwrite an existing one. But there is a special command in case you explicitly want this: \LaTeXTT{\textbackslash{}renewcommand}. It uses the same syntax as the \LaTeXTT{\textbackslash{}newcommand} command.

In certain cases you might also want to use the \LaTeXTT{\textbackslash{}providecommand} command. It works like \LaTeXTT{\textbackslash{}newcommand}, but if the command is already defined, LaTeX will silently ignore the new command.

With LaTex2e, it is also possible to add a default parameter to a command with the following syntax:

\begin{Shaded}
\begin{Highlighting}[]

\NormalTok{\textbackslash{}newcommand\{name\}[num][default]\{definition\}}
\end{Highlighting}
\end{Shaded}


If the default parameter of \LaTeXTT{\textbackslash{}newcommand} is present, then the first of the number of arguments specified by \LaTeXTT{num} is optional with a default value of \LaTeXTT{default}; if absent, then all of the arguments are required.

\begin{longtable}{p{1.0\linewidth}}
\begin{Shaded}
\begin{Highlighting}[]

\NormalTok{\textbackslash{}newcommand\{\textbackslash{}wbalTwo\}[2][Wikimedia]\{}
  \NormalTok{This is the Wikibook about LaTeX}
  \NormalTok{supported by \{#1\} and \{#2\}!\}}
\CommentTok{% in the document body:}
\NormalTok{\textbackslash{}begin\{itemize\}}
\NormalTok{\textbackslash{}item \textbackslash{}wbalTwo\{John Doe\}}
\NormalTok{\textbackslash{}item \textbackslash{}wbalTwo[lots of users]\{John Doe\}}
\NormalTok{\textbackslash{}end\{itemize\}}
\end{Highlighting}
\end{Shaded}
\\

\begin{myitemize}
\item{} This is the Wikibook about LaTeX supported by Wikimedia and John Doe!
\item{} This is the Wikibook about LaTeX supported by lots of users and John Doe!

\end{myitemize}

\end{longtable}
{\bfseries
\begin{mydescription}Note
\end{mydescription}
}
\begin{myquote}\item{} When the command is used with an explicit first parameter it is given enclosed with brackets (here \symbol{34}\LaTeXTT{{$\text{[}$}lots of users{$\text{]}$}}\symbol{34}).
\end{myquote}


Here is a common example: if you are writing a book about Mathematics and you have to use vectors, you have to decide how they will look. There are several different standards, used in many books. If {\itshape \setmainfont[Path=/usr/share/fonts/truetype/cmu/,UprightFont=cmunrm.ttf,BoldFont=cmunbx.ttf,ItalicFont=cmunti.ttf,BoldItalicFont=cmunbi.ttf]{cmunti.ttf}\setmonofont[Path=/usr/share/fonts/truetype/cmu/,UprightFont=cmuntt.ttf,BoldFont=cmuntb.ttf,ItalicFont=cmunit.ttf,BoldItalicFont=cmuntx.ttf]{cmunti.ttf}\itshape a}{$\text{ }$}\setmainfont[Path=/usr/share/fonts/truetype/cmu/,UprightFont=cmunrm.ttf,BoldFont=cmunbx.ttf,ItalicFont=cmunti.ttf,BoldItalicFont=cmunbi.ttf]{cmunrm.ttf}\setmonofont[Path=/usr/share/fonts/truetype/cmu/,UprightFont=cmuntt.ttf,BoldFont=cmuntb.ttf,ItalicFont=cmunit.ttf,BoldItalicFont=cmuntx.ttf]{cmunrm.ttf} is a vector, some people like to add an arrow over it ({$\vec{a}$}), other people write it underlined ({\itshape \uline{\setmainfont[Path=/usr/share/fonts/truetype/cmu/,UprightFont=cmunrm.ttf,BoldFont=cmunbx.ttf,ItalicFont=cmunti.ttf,BoldItalicFont=cmunbi.ttf]{cmunti.ttf}\setmonofont[Path=/usr/share/fonts/truetype/cmu/,UprightFont=cmuntt.ttf,BoldFont=cmuntb.ttf,ItalicFont=cmunit.ttf,BoldItalicFont=cmuntx.ttf]{cmunti.ttf}\itshape a}}\setmainfont[Path=/usr/share/fonts/truetype/cmu/,UprightFont=cmunrm.ttf,BoldFont=cmunbx.ttf,ItalicFont=cmunti.ttf,BoldItalicFont=cmunbi.ttf]{cmunrm.ttf}\setmonofont[Path=/usr/share/fonts/truetype/cmu/,UprightFont=cmuntt.ttf,BoldFont=cmuntb.ttf,ItalicFont=cmunit.ttf,BoldItalicFont=cmuntx.ttf]{cmunrm.ttf}); another common version is to write it bold ({\bfseries \setmainfont[Path=/usr/share/fonts/truetype/cmu/,UprightFont=cmunrm.ttf,BoldFont=cmunbx.ttf,ItalicFont=cmunti.ttf,BoldItalicFont=cmunbi.ttf]{cmunbx.ttf}\setmonofont[Path=/usr/share/fonts/truetype/cmu/,UprightFont=cmuntt.ttf,BoldFont=cmuntb.ttf,ItalicFont=cmunit.ttf,BoldItalicFont=cmuntx.ttf]{cmunbx.ttf}\bfseries a}\setmainfont[Path=/usr/share/fonts/truetype/cmu/,UprightFont=cmunrm.ttf,BoldFont=cmunbx.ttf,ItalicFont=cmunti.ttf,BoldItalicFont=cmunbi.ttf]{cmunrm.ttf}\setmonofont[Path=/usr/share/fonts/truetype/cmu/,UprightFont=cmuntt.ttf,BoldFont=cmuntb.ttf,ItalicFont=cmunit.ttf,BoldItalicFont=cmuntx.ttf]{cmunrm.ttf}). Let us assume you want to write your vectors with an arrow over them; then add the following line in your {\ttfamily \setmainfont[Path=/usr/share/fonts/truetype/cmu/,UprightFont=cmunrm.ttf,BoldFont=cmunbx.ttf,ItalicFont=cmunti.ttf,BoldItalicFont=cmunbi.ttf]{cmuntt.ttf}\setmonofont[Path=/usr/share/fonts/truetype/cmu/,UprightFont=cmuntt.ttf,BoldFont=cmuntb.ttf,ItalicFont=cmunit.ttf,BoldItalicFont=cmuntx.ttf]{cmuntt.ttf}\ttfamily mystyle.sty}\setmainfont[Path=/usr/share/fonts/truetype/cmu/,UprightFont=cmunrm.ttf,BoldFont=cmunbx.ttf,ItalicFont=cmunti.ttf,BoldItalicFont=cmunbi.ttf]{cmunrm.ttf}\setmonofont[Path=/usr/share/fonts/truetype/cmu/,UprightFont=cmuntt.ttf,BoldFont=cmuntb.ttf,ItalicFont=cmunit.ttf,BoldItalicFont=cmuntx.ttf]{cmunrm.ttf}.

\begin{Shaded}
\begin{Highlighting}[]

\NormalTok{\textbackslash{}newcommand\{\textbackslash{}myvec\}[1]\{\textbackslash{}vec\{#1\}\}}
\end{Highlighting}
\end{Shaded}


and write your vectors inside the new \LaTeXTT{\textbackslash{}myvec\{...\}} command. You can call it as you wish, but you\textquotesingle{}d better choose a short name because you will probably write it very often. Then, if you change your mind and you want your vectors to look differently you just have to change the definition of your \LaTeXTT{\textbackslash{}myvec\{...\}}. Use this approach whenever you can: this will save you a lot of time and increase the consistency of your document.
\subsection{DeclareRobustCommand}
\label{839}
Some commands are {\itshape \setmainfont[Path=/usr/share/fonts/truetype/cmu/,UprightFont=cmunrm.ttf,BoldFont=cmunbx.ttf,ItalicFont=cmunti.ttf,BoldItalicFont=cmunbi.ttf]{cmunti.ttf}\setmonofont[Path=/usr/share/fonts/truetype/cmu/,UprightFont=cmuntt.ttf,BoldFont=cmuntb.ttf,ItalicFont=cmunit.ttf,BoldItalicFont=cmuntx.ttf]{cmunti.ttf}\itshape fragile}\setmainfont[Path=/usr/share/fonts/truetype/cmu/,UprightFont=cmunrm.ttf,BoldFont=cmunbx.ttf,ItalicFont=cmunti.ttf,BoldItalicFont=cmunbi.ttf]{cmunrm.ttf}\setmonofont[Path=/usr/share/fonts/truetype/cmu/,UprightFont=cmuntt.ttf,BoldFont=cmuntb.ttf,ItalicFont=cmunit.ttf,BoldItalicFont=cmuntx.ttf]{cmunrm.ttf}, that is they fail in some environments.  If a macro works in body text but not in (for example) a figure caption, it\textquotesingle{}s worth trying to replace the \LaTeXTT{\textbackslash{}newcommand\{\textbackslash{}MyCommand\}...} declaration with \LaTeXTT{\textbackslash{}DeclareRobustCommand\{\textbackslash{}MyCommand\}...} in the preamble.  This is especially true for macros which, when expanded, produce text that is written to a {\ttfamily \setmainfont[Path=/usr/share/fonts/truetype/cmu/,UprightFont=cmunrm.ttf,BoldFont=cmunbx.ttf,ItalicFont=cmunti.ttf,BoldItalicFont=cmunbi.ttf]{cmuntt.ttf}\setmonofont[Path=/usr/share/fonts/truetype/cmu/,UprightFont=cmuntt.ttf,BoldFont=cmuntb.ttf,ItalicFont=cmunit.ttf,BoldItalicFont=cmuntx.ttf]{cmuntt.ttf}\ttfamily .aux}{$\text{ }$}\setmainfont[Path=/usr/share/fonts/truetype/cmu/,UprightFont=cmunrm.ttf,BoldFont=cmunbx.ttf,ItalicFont=cmunti.ttf,BoldItalicFont=cmunbi.ttf]{cmunrm.ttf}\setmonofont[Path=/usr/share/fonts/truetype/cmu/,UprightFont=cmuntt.ttf,BoldFont=cmuntb.ttf,ItalicFont=cmunit.ttf,BoldItalicFont=cmuntx.ttf]{cmunrm.ttf} file.
\section{New environments}
\label{840}

Just as with the \LaTeXTT{\textbackslash{}newcommand} command, there is a command to create your own environments. The \LaTeXTT{\textbackslash{}newenvironment} command uses the following syntax:

\begin{Shaded}
\begin{Highlighting}[]

\NormalTok{\textbackslash{}newenvironment\{name\}[num]\{before\}\{after\}}
\end{Highlighting}
\end{Shaded}


Again \LaTeXTT{\textbackslash{}newenvironment} can have an optional argument. When the \LaTeXTT{\textbackslash{}begin\{name\}} command (which starts the environment) is encountered, the material specified in the \LaTeXTT{before} argument is processed before the text in the environment gets processed. The material in the \LaTeXTT{after} argument gets processed when the \LaTeXTT{\textbackslash{}end\{name\}} command (which ends the environment) is encountered.

The \LaTeXTT{num} argument is used the same way as in the \LaTeXTT{\textbackslash{}newcommand} command. LaTeX makes sure that you do not define an environment that already exists. If you ever want to change an existing environment, you can use the \LaTeXTT{\textbackslash{}renewenvironment} command. It uses the same syntax as the \LaTeXTT{\textbackslash{}newenvironment} command.

The example below illustrates the usage of the \LaTeXTT{\textbackslash{}newenvironment} command:

\begin{longtable}{p{1.0\linewidth}}
\begin{Shaded}
\begin{Highlighting}[]

\NormalTok{\textbackslash{}newenvironment\{king\}}
\NormalTok{\{ \textbackslash{}rule\{1ex\}\{1ex\}\textbackslash{}hspace\{\textbackslash{}stretch\{1\}\} \}}
\NormalTok{\{ \textbackslash{}hspace\{\textbackslash{}stretch\{1\}\}\textbackslash{}rule\{1ex\}\{1ex\} \}}
 
\NormalTok{\textbackslash{}begin\{king\}}
\NormalTok{My humble subjects \textbackslash{}ldots}
\NormalTok{\textbackslash{}end\{king\}}
\end{Highlighting}
\end{Shaded}
\\



\begin{minipage}{0.75000\textwidth}
\begin{center}
\includegraphics[width=1.0\textwidth,height=6.5in,keepaspectratio]{../images/209.png}
\end{center}
\raggedright{}\myfigurewithoutcaption{209}
\end{minipage}\vspace{0.75cm}



\end{longtable}
\subsection{Extra space}
\label{841}

When creating a new environment you may easily get bitten by extra spaces
creeping in, which can potentially have fatal effects. For example when you
want to create a title environment which suppresses its own indentation as well
as the one on the following paragraph. The \LaTeXTT{\textbackslash{}ignorespaces} command in the
begin block of the environment will make it ignore any space after executing
the begin block. The end block is a bit more tricky as special processing
occurs at the end of an environment. With the \LaTeXTT{\textbackslash{}ignorespacesafterend}
LaTeX will issue an \LaTeXTT{\textbackslash{}ignorespaces} after the special ‘end’ processing has
occurred.

\begin{longtable}{p{1.0\linewidth}}
\begin{Shaded}
\begin{Highlighting}[]

\NormalTok{\textbackslash{}newenvironment\{simple\}}\CommentTok
\NormalTok{\{\textbackslash{}par\textbackslash{}noindent\}}
 
\NormalTok{\textbackslash{}begin\{simple\}}
\NormalTok{See the space\textbackslash{}\textbackslash{}to the left.}
\NormalTok{\textbackslash{}end\{simple\}}
\NormalTok{Same\textbackslash{}\textbackslash{}here.}
\end{Highlighting}
\end{Shaded}
\\

\TemplatePreformat{$\text{ }$\newline{}
$\text{ }${}$\text{ }${}See$\text{ }${}the$\text{ }${}space$\text{ }$\newline{}
to$\text{ }${}the$\text{ }${}left.$\text{ }$\newline{}
$\text{ }${}$\text{ }$\newline{}
$\text{ }${}$\text{ }${}Same$\text{ }$\newline{}
here.$\text{ }$\newline{}
}

\end{longtable}


\begin{longtable}{p{1.0\linewidth}}
\begin{Shaded}
\begin{Highlighting}[]

\NormalTok{\textbackslash{}newenvironment\{correct\}}\CommentTok
\NormalTok{\{\textbackslash{}par\textbackslash{}noindent}\CommentTok{%}
\NormalTok{\textbackslash{}ignorespacesafterend\}}
 
\NormalTok{\textbackslash{}begin\{correct\}}
\NormalTok{No space\textbackslash{}\textbackslash{}to the left.}
\NormalTok{\textbackslash{}end\{correct\}}
\NormalTok{Same\textbackslash{}\textbackslash{}here.}
\end{Highlighting}
\end{Shaded}
\\

\TemplatePreformat{$\text{ }$\newline{}
No$\text{ }${}space$\text{ }$\newline{}
to$\text{ }${}the$\text{ }${}left.$\text{ }$\newline{}
$\text{ }${}$\text{ }$\newline{}
Same$\text{ }$\newline{}
here.$\text{ }$\newline{}
}

\end{longtable}

Also, if you\textquotesingle{}re still having problems with extra space being appended at the end of your environment when using the {\ttfamily \setmainfont[Path=/usr/share/fonts/truetype/cmu/,UprightFont=cmunrm.ttf,BoldFont=cmunbx.ttf,ItalicFont=cmunti.ttf,BoldItalicFont=cmunbi.ttf]{cmuntt.ttf}\setmonofont[Path=/usr/share/fonts/truetype/cmu/,UprightFont=cmuntt.ttf,BoldFont=cmuntb.ttf,ItalicFont=cmunit.ttf,BoldItalicFont=cmuntx.ttf]{cmuntt.ttf}\ttfamily \textbackslash{}input}{$\text{ }$}\setmainfont[Path=/usr/share/fonts/truetype/cmu/,UprightFont=cmunrm.ttf,BoldFont=cmunbx.ttf,ItalicFont=cmunti.ttf,BoldItalicFont=cmunbi.ttf]{cmunrm.ttf}\setmonofont[Path=/usr/share/fonts/truetype/cmu/,UprightFont=cmuntt.ttf,BoldFont=cmuntb.ttf,ItalicFont=cmunit.ttf,BoldItalicFont=cmuntx.ttf]{cmunrm.ttf} for external source, make sure there is no space between the beginning, sourcing, and end of the environment, such as:

\begin{Shaded}
\begin{Highlighting}[]

\NormalTok{\textbackslash{}begin\{correct\}\textbackslash{}input\{somefile.tex\}\textbackslash{}end\{correct\}}
\end{Highlighting}
\end{Shaded}


or

\begin{Shaded}
\begin{Highlighting}[]

\NormalTok{\textbackslash{}begin\{correct\}}\CommentTok
\NormalTok{\textbackslash{}end\{correct\}}
\end{Highlighting}
\end{Shaded}

\section{Declare commands within new environment}
\label{842}
New commands can be declared within newenvironment.  
Commands declared within the newenvironment refer to their arguments by doubling the \# character.
In the following example, a new environment is declared along with a nested command:

\begin{Shaded}
\begin{Highlighting}[]

\NormalTok{\textbackslash{}newenvironment\{topics\}\{}
\NormalTok{\textbackslash{}newcommand\{\textbackslash{}topic\}[2]\{ \textbackslash{}item\{##1 / ##2\textbackslash{}\} \}}
\NormalTok{Topics:}
\NormalTok{\textbackslash{}begin\{itemize\}}
\NormalTok{\}}
\NormalTok{\{}
\NormalTok{\textbackslash{}end\{itemize\}}
\NormalTok{\}}
\end{Highlighting}
\end{Shaded}


If, by mistake, the arguments passed to the \textbackslash{}topics macro are defined with a single \# character, the following error message will be thrown:
\\

\TemplateSpaceIndent{$\text{ }${}$\text{ }${}$\text{ }${}$\text{ }${}!$\text{ }${}Illegal$\text{ }${}parameter$\text{ }${}number$\text{ }${}in$\text{ }${}definition$\text{ }${}of$\text{ }${}\textbackslash{}topics.}

\section{Extending the number of arguments}
\label{843}

The \LaTeXTT{xkeyval} packages will let you define key/value options for commands.
\begin{Shaded}
\begin{Highlighting}[]

\NormalTok{\textbackslash{}mycommand[key1=value1, key3=value3]\{some text\}}
\end{Highlighting}
\end{Shaded}


The package is quite complete and documentation is exhaustive. We recommend that package developers read it. \myplainurl{http://www.ctan.org/pkg/xkeyval}

Let\textquotesingle{}s provide a simple example\myfootnote{\myfnhref{http://tex.stackexchange.com/questions/13270/a-package-template-using-xkeyval}{tex.stackexchange.com}}:

\begin{Shaded}
\begin{Highlighting}[]

\NormalTok{\textbackslash{}usepackage\{xkeyval\}}
\CommentTok{% ...}
 
\NormalTok{\textbackslash{}makeatletter}
\NormalTok{\textbackslash{}def\textbackslash{}my@emphstyle#1\{\textbackslash{}csname my@style@#1\textbackslash{}endcsname\}}
\CommentTok{%% Predefined styles}
\NormalTok{\textbackslash{}providecommand\textbackslash{}my@style@default\{\textbackslash{}em\}}
\NormalTok{\textbackslash{}providecommand\textbackslash{}my@style@bold\{\textbackslash{}bfseries\}}
 
\NormalTok{\textbackslash{}define@key\{myemph\}\{code\}\{}\CommentTok
  \NormalTok{\textbackslash{}def\textbackslash{}my@emphstyle\{\textbackslash{}csname my@style@#1\textbackslash{}endcsname\}}
\NormalTok{\}}
\NormalTok{\textbackslash{}newcommand\textbackslash{}setemph[1]\{}\CommentTok
  \NormalTok{\{\textbackslash{}my@emphstyle #1\}}
\NormalTok{\}}
 
\NormalTok{\textbackslash{}makeatother}
 
\NormalTok{Something \textbackslash{}emph\{important\}}
 
\NormalTok{\textbackslash{}setemph\{style=bold\}}
\NormalTok{Something \textbackslash{}emph\{important\}}
 
\NormalTok{\textbackslash{}setemph\{code=\textbackslash{}Large\textbackslash{}sffamily\}}
\NormalTok{Something \textbackslash{}emph\{important\}}
\end{Highlighting}
\end{Shaded}

\section{Arithmetic}
\label{844}



UNKNOWN TEMPLATE  
Expand

{}



LaTeX can manipulate numbers.

The \LaTeXTT{calc} package provides the common infix notation.

\begin{Shaded}
\begin{Highlighting}[]

\NormalTok{\textbackslash{}usepackage\{calc\}}
\CommentTok{% ...}
\NormalTok{\textbackslash{}newcounter\{mine\}}
\NormalTok{\textbackslash{}setcounter\{mine\}\{2*17\}}
\NormalTok{\textbackslash{}themine}
\end{Highlighting}
\end{Shaded}


For high-{}precision computations, you can use the \LaTeXTT{fp}\myfootnote{\myfnhref{http://ctan.mackichan.com/macros/latex/contrib/fp/README}{ctan.mackichan.com}} package.

\begin{Shaded}
\begin{Highlighting}[]

\NormalTok{\textbackslash{}usepackage\{fp\}}
 
\CommentTok{% Clip}
\NormalTok{\textbackslash{}[}
\NormalTok{\textbackslash{}FPmul\textbackslash{}result\{2\}\{7\}}
\NormalTok{\textbackslash{}FPclip\textbackslash{}result\textbackslash{}result}
\NormalTok{2*7 = \textbackslash{}result}
\NormalTok{\textbackslash{}]}
 
\CommentTok{% Infix}
\NormalTok{\textbackslash{}[}
\NormalTok{\textbackslash{}newcommand\textbackslash{}result\{11\}}
\NormalTok{\textbackslash{}sqrt\{\textbackslash{}sin(2+\textbackslash{}result)\} \textbackslash{}approx}
\NormalTok{\textbackslash{}FPeval\textbackslash{}result\{round(root(2,sin(result + 2.5)),2)\}}
\NormalTok{\textbackslash{}result}
\NormalTok{\textbackslash{}]}
 
\CommentTok{% Postfix}
\NormalTok{\textbackslash{}[}
\NormalTok{\textbackslash{}FPupn\textbackslash{}result\{17 2.5 + 17.5 swap - 2 1 + * 2 swap /\} }\CommentTok{% or \textbackslash{}FPupn\textbackslash{}result\{2 17.5}
 \NormalTok{17 2.5 + - 2 1 + * /\}}
\NormalTok{\textbackslash{}FPclip\textbackslash{}result\textbackslash{}result}
\NormalTok{(17+2.5 - 17.5) * (2+1) / 2  = \textbackslash{}result}
\NormalTok{\textbackslash{}]}
 
\CommentTok{% High precision}
\NormalTok{\textbackslash{}[}
\NormalTok{\textbackslash{}FPdiv\textbackslash{}result\{17\}\{7\}}
\NormalTok{\textbackslash{}frac\{17\}\{7\} \textbackslash{}approx \textbackslash{}FPtrunc\textbackslash{}result\textbackslash{}result\{3\}}
\NormalTok{\textbackslash{}result}
\NormalTok{\textbackslash{}]}
\end{Highlighting}
\end{Shaded}

\section{Conditionals}
\label{845}

LaTeX can use conditionals thanks to the \LaTeXTT{ifthen} package.

\begin{Shaded}
\begin{Highlighting}[]

\NormalTok{\textbackslash{}usepackage\{ifthen\}}
\CommentTok{% ...}
 
\NormalTok{\textbackslash{}ifthenelse\{ \textbackslash{}equal\{\textbackslash{}myvar\}\{true\} \}\{}
  \NormalTok{This is true.}
\NormalTok{\}\{}
  \NormalTok{This is false.}
\NormalTok{\}}
\end{Highlighting}
\end{Shaded}

\section{Loops}
\label{846}

The \LaTeXTT{PGF/TikZ} extension provides the \LaTeXTT{\textbackslash{}foreach} command.

\begin{Shaded}
\begin{Highlighting}[]

\NormalTok{\textbackslash{}usepackage\{tikz\}}
\CommentTok{% ...}
 
\NormalTok{\textbackslash{}foreach \textbackslash{}i/\textbackslash{}q in \{wheat/50g, water/1L, yeast/2g\}\{}
  \NormalTok{\textbackslash{}noindent\textbackslash{}i\textbackslash{}dotfill\textbackslash{}q\textbackslash{}\textbackslash{}}
\NormalTok{\}}
\end{Highlighting}
\end{Shaded}


If you are only using  \LaTeXTT{\textbackslash{}foreach} and not drawing graphics, you may instead use the {\ttfamily \setmainfont[Path=/usr/share/fonts/truetype/cmu/,UprightFont=cmunrm.ttf,BoldFont=cmunbx.ttf,ItalicFont=cmunti.ttf,BoldItalicFont=cmunbi.ttf]{cmuntt.ttf}\setmonofont[Path=/usr/share/fonts/truetype/cmu/,UprightFont=cmuntt.ttf,BoldFont=cmuntb.ttf,ItalicFont=cmunit.ttf,BoldItalicFont=cmuntx.ttf]{cmuntt.ttf}\ttfamily pgffor}{$\text{ }$}\setmainfont[Path=/usr/share/fonts/truetype/cmu/,UprightFont=cmunrm.ttf,BoldFont=cmunbx.ttf,ItalicFont=cmunti.ttf,BoldItalicFont=cmunbi.ttf]{cmunrm.ttf}\setmonofont[Path=/usr/share/fonts/truetype/cmu/,UprightFont=cmuntt.ttf,BoldFont=cmuntb.ttf,ItalicFont=cmunit.ttf,BoldItalicFont=cmuntx.ttf]{cmunrm.ttf} package directly.

Alternatively you can check out the \LaTeXTT{multido} package.
\section{Strings}
\label{847}

\LaTeXTT{xstring} provides a lot of features. From CTAN:
\begin{myitemize}
\item{}  testing a string’s contents
\item{}  extracting substrings
\item{}  substitution of substrings
\item{}  string length
\item{}  position of a substring
\item{}  number of recurrences of a substring
\end{myitemize}


Examples:
\begin{Shaded}
\begin{Highlighting}[]

\NormalTok{\textbackslash{}usepackage\{xstring\}}
\CommentTok{% ...}
 
\NormalTok{\textbackslash{}newcommand\textbackslash{}mystr\{Hello World!\}}
 
\NormalTok{The string ``\textbackslash{}mystr'' has \textbackslash{}StrLen\{\textbackslash{}mystr\}\{\} characters.}
 
\NormalTok{Predicate ``\textbackslash{}mystr\{\} contains the word Hello'' is}
 \NormalTok{\textbackslash{}IfSubStr\{\textbackslash{}mystr\}\{Hello\}\{true\}\{false\}.}
\end{Highlighting}
\end{Shaded}

\section{LaTeX Hooks}
\label{848}

LaTeX provide two hooks:
\begin{myitemize}
\item{}  \LaTeXTT{\textbackslash{}AtBeginDocument} will let you specify a set of commands that will be executed when \LaTeXTT{\textbackslash{}begin\{document\}} is met.
\item{}  \LaTeXTT{\textbackslash{}AtEndDocument} does the same for \LaTeXTT{\textbackslash{}end\{document\}}.
\end{myitemize}


This gives you some more flexiblity for macros. It can be useful to override settings that get executed after the preamble. These hooks can be called several times. The commands will be executed in the order they were set.

For instance, let\textquotesingle{}s replace the page numbers with oldstylenums:
\begin{Shaded}
\begin{Highlighting}[]

\NormalTok{\textbackslash{}usepackage\{textcomp\}}
 
\NormalTok{\textbackslash{}AtBeginDocument\{}\CommentTok{%}
  \CommentTok{% Make the page numbers in text figures}
  \NormalTok{\textbackslash{}let\textbackslash{}myThePage\textbackslash{}thepage}
  \NormalTok{\textbackslash{}renewcommand\{\textbackslash{}thepage\}\{ \textbackslash{}oldstylenums\{\textbackslash{}myThePage\} \}}
\NormalTok{\}}
\end{Highlighting}
\end{Shaded}


There are also hooks for classes and packages. See \mylref{881}{Creating Packages}.
\section{Command-{}line LaTeX}
\label{849}

If you work on a Unix-{}like OS, you might be using Makefiles or any kind of script to build your LaTeX projects. In that connection it might be interesting to produce different versions of the same document by calling LaTeX with command-{}line parameters. If you add the following structure to your document:

\begin{Shaded}
\begin{Highlighting}[]

\NormalTok{\textbackslash{}usepackage\{ifthen\}}
\CommentTok{%...}
 
\CommentTok{% default value.}
\NormalTok{\textbackslash{}providecommand\textbackslash{}blackandwhite\{false\}}
\CommentTok{%...}
 
\NormalTok{\textbackslash{}ifthenelse\{ \textbackslash{}equal\{\textbackslash{}blackandwhite\}\{true\} \}\{}
\CommentTok{% "black and white" mode; do something..}
\NormalTok{\}\{}
\CommentTok{% "color" mode; do something different..}
\NormalTok{\}}
\end{Highlighting}
\end{Shaded}


Now you can call LaTeX like this:\\

\TemplateSpaceIndent{$\text{ }${}latex$\text{ }${}\textquotesingle{}\textbackslash{}providecommand\{\textbackslash{}blackandwhite\}\{true\}\textbackslash{}input\{test.tex\}\textquotesingle{}}


First the command \LaTeXTT{\textbackslash{}blackandwhite} gets defined and then the actual file is read with input. By setting \LaTeXTT{\textbackslash{}blackandwhite} to false the color version of the document would be produced.
\section{Notes and References}
\label{850}
\LaTeXNullTemplate{}

\chapter{Plain TeX}

\myminitoc
\label{851}

\label{852}


While you play with LaTeX macros, you will notice that it is quite limited. You may wonder how all these packages you are using every day have been implemented with so little. In fact, LaTeX is a set of Plain TeX macros and most packages use Plain TeX code. Plain TeX is much more low-{}level, it has much more capabilities at the cost of a steep learning curve and complex programming.

Up to a few exceptions, you can use the full Plain TeX language within a valid LaTeX document whereas the opposite is false.
\section{Vocabulary}
\label{853}

To avoid confusion it seems necessary to explain some terms.
\begin{myitemize}
\item{}  A {\itshape \setmainfont[Path=/usr/share/fonts/truetype/cmu/,UprightFont=cmunrm.ttf,BoldFont=cmunbx.ttf,ItalicFont=cmunti.ttf,BoldItalicFont=cmunbi.ttf]{cmunti.ttf}\setmonofont[Path=/usr/share/fonts/truetype/cmu/,UprightFont=cmuntt.ttf,BoldFont=cmuntb.ttf,ItalicFont=cmunit.ttf,BoldItalicFont=cmuntx.ttf]{cmunti.ttf}\itshape group}{$\text{ }$}\setmainfont[Path=/usr/share/fonts/truetype/cmu/,UprightFont=cmunrm.ttf,BoldFont=cmunbx.ttf,ItalicFont=cmunti.ttf,BoldItalicFont=cmunbi.ttf]{cmunrm.ttf}\setmonofont[Path=/usr/share/fonts/truetype/cmu/,UprightFont=cmuntt.ttf,BoldFont=cmuntb.ttf,ItalicFont=cmunit.ttf,BoldItalicFont=cmuntx.ttf]{cmunrm.ttf} is everything after an opening brace and before the matching closing brace.
\item{}  A {\itshape \setmainfont[Path=/usr/share/fonts/truetype/cmu/,UprightFont=cmunrm.ttf,BoldFont=cmunbx.ttf,ItalicFont=cmunti.ttf,BoldItalicFont=cmunbi.ttf]{cmunti.ttf}\setmonofont[Path=/usr/share/fonts/truetype/cmu/,UprightFont=cmuntt.ttf,BoldFont=cmuntb.ttf,ItalicFont=cmunit.ttf,BoldItalicFont=cmuntx.ttf]{cmunti.ttf}\itshape token}{$\text{ }$}\setmainfont[Path=/usr/share/fonts/truetype/cmu/,UprightFont=cmunrm.ttf,BoldFont=cmunbx.ttf,ItalicFont=cmunti.ttf,BoldItalicFont=cmunbi.ttf]{cmunrm.ttf}\setmonofont[Path=/usr/share/fonts/truetype/cmu/,UprightFont=cmuntt.ttf,BoldFont=cmuntb.ttf,ItalicFont=cmunit.ttf,BoldItalicFont=cmuntx.ttf]{cmunrm.ttf} is a character, a control sequence, or a group.
\item{}  A {\itshape \setmainfont[Path=/usr/share/fonts/truetype/cmu/,UprightFont=cmunrm.ttf,BoldFont=cmunbx.ttf,ItalicFont=cmunti.ttf,BoldItalicFont=cmunbi.ttf]{cmunti.ttf}\setmonofont[Path=/usr/share/fonts/truetype/cmu/,UprightFont=cmuntt.ttf,BoldFont=cmuntb.ttf,ItalicFont=cmunit.ttf,BoldItalicFont=cmuntx.ttf]{cmunti.ttf}\itshape control sequence}{$\text{ }$}\setmainfont[Path=/usr/share/fonts/truetype/cmu/,UprightFont=cmunrm.ttf,BoldFont=cmunbx.ttf,ItalicFont=cmunti.ttf,BoldItalicFont=cmunbi.ttf]{cmunrm.ttf}\setmonofont[Path=/usr/share/fonts/truetype/cmu/,UprightFont=cmuntt.ttf,BoldFont=cmuntb.ttf,ItalicFont=cmunit.ttf,BoldItalicFont=cmuntx.ttf]{cmunrm.ttf} is anything that begins with a \LaTeXTT{\textbackslash{}}. It is not printed as is, it is expanded by the TeX engine according to its type.
\item{}  A {\itshape \setmainfont[Path=/usr/share/fonts/truetype/cmu/,UprightFont=cmunrm.ttf,BoldFont=cmunbx.ttf,ItalicFont=cmunti.ttf,BoldItalicFont=cmunbi.ttf]{cmunti.ttf}\setmonofont[Path=/usr/share/fonts/truetype/cmu/,UprightFont=cmuntt.ttf,BoldFont=cmuntb.ttf,ItalicFont=cmunit.ttf,BoldItalicFont=cmuntx.ttf]{cmunti.ttf}\itshape command}{$\text{ }$}\setmainfont[Path=/usr/share/fonts/truetype/cmu/,UprightFont=cmunrm.ttf,BoldFont=cmunbx.ttf,ItalicFont=cmunti.ttf,BoldItalicFont=cmunbi.ttf]{cmunrm.ttf}\setmonofont[Path=/usr/share/fonts/truetype/cmu/,UprightFont=cmuntt.ttf,BoldFont=cmuntb.ttf,ItalicFont=cmunit.ttf,BoldItalicFont=cmuntx.ttf]{cmunrm.ttf} (or {\itshape \setmainfont[Path=/usr/share/fonts/truetype/cmu/,UprightFont=cmunrm.ttf,BoldFont=cmunbx.ttf,ItalicFont=cmunti.ttf,BoldItalicFont=cmunbi.ttf]{cmunti.ttf}\setmonofont[Path=/usr/share/fonts/truetype/cmu/,UprightFont=cmuntt.ttf,BoldFont=cmuntb.ttf,ItalicFont=cmunit.ttf,BoldItalicFont=cmuntx.ttf]{cmunti.ttf}\itshape function}{$\text{ }$}\setmainfont[Path=/usr/share/fonts/truetype/cmu/,UprightFont=cmunrm.ttf,BoldFont=cmunbx.ttf,ItalicFont=cmunti.ttf,BoldItalicFont=cmunbi.ttf]{cmunrm.ttf}\setmonofont[Path=/usr/share/fonts/truetype/cmu/,UprightFont=cmuntt.ttf,BoldFont=cmuntb.ttf,ItalicFont=cmunit.ttf,BoldItalicFont=cmuntx.ttf]{cmunrm.ttf} or {\itshape \setmainfont[Path=/usr/share/fonts/truetype/cmu/,UprightFont=cmunrm.ttf,BoldFont=cmunbx.ttf,ItalicFont=cmunti.ttf,BoldItalicFont=cmunbi.ttf]{cmunti.ttf}\setmonofont[Path=/usr/share/fonts/truetype/cmu/,UprightFont=cmuntt.ttf,BoldFont=cmuntb.ttf,ItalicFont=cmunit.ttf,BoldItalicFont=cmuntx.ttf]{cmunti.ttf}\itshape macro}\setmainfont[Path=/usr/share/fonts/truetype/cmu/,UprightFont=cmunrm.ttf,BoldFont=cmunbx.ttf,ItalicFont=cmunti.ttf,BoldItalicFont=cmunbi.ttf]{cmunrm.ttf}\setmonofont[Path=/usr/share/fonts/truetype/cmu/,UprightFont=cmuntt.ttf,BoldFont=cmuntb.ttf,ItalicFont=cmunit.ttf,BoldItalicFont=cmuntx.ttf]{cmunrm.ttf}) is a control sequence that may expand to text, to (re)definition of control sequences, etc.
\item{}  A {\itshape \setmainfont[Path=/usr/share/fonts/truetype/cmu/,UprightFont=cmunrm.ttf,BoldFont=cmunbx.ttf,ItalicFont=cmunti.ttf,BoldItalicFont=cmunbi.ttf]{cmunti.ttf}\setmonofont[Path=/usr/share/fonts/truetype/cmu/,UprightFont=cmuntt.ttf,BoldFont=cmuntb.ttf,ItalicFont=cmunit.ttf,BoldItalicFont=cmuntx.ttf]{cmunti.ttf}\itshape primitive}{$\text{ }$}\setmainfont[Path=/usr/share/fonts/truetype/cmu/,UprightFont=cmunrm.ttf,BoldFont=cmunbx.ttf,ItalicFont=cmunti.ttf,BoldItalicFont=cmunbi.ttf]{cmunrm.ttf}\setmonofont[Path=/usr/share/fonts/truetype/cmu/,UprightFont=cmuntt.ttf,BoldFont=cmuntb.ttf,ItalicFont=cmunit.ttf,BoldItalicFont=cmuntx.ttf]{cmunrm.ttf} is a command that is hard coded in the TeX engine, {\itshape \setmainfont[Path=/usr/share/fonts/truetype/cmu/,UprightFont=cmunrm.ttf,BoldFont=cmunbx.ttf,ItalicFont=cmunti.ttf,BoldItalicFont=cmunbi.ttf]{cmunti.ttf}\setmonofont[Path=/usr/share/fonts/truetype/cmu/,UprightFont=cmuntt.ttf,BoldFont=cmuntb.ttf,ItalicFont=cmunit.ttf,BoldItalicFont=cmuntx.ttf]{cmunti.ttf}\itshape i.e.}{$\text{ }$}\setmainfont[Path=/usr/share/fonts/truetype/cmu/,UprightFont=cmunrm.ttf,BoldFont=cmunbx.ttf,ItalicFont=cmunti.ttf,BoldItalicFont=cmunbi.ttf]{cmunrm.ttf}\setmonofont[Path=/usr/share/fonts/truetype/cmu/,UprightFont=cmuntt.ttf,BoldFont=cmuntb.ttf,ItalicFont=cmunit.ttf,BoldItalicFont=cmuntx.ttf]{cmunrm.ttf} it is not written in Plain TeX.
\item{}  A {\itshape \setmainfont[Path=/usr/share/fonts/truetype/cmu/,UprightFont=cmunrm.ttf,BoldFont=cmunbx.ttf,ItalicFont=cmunti.ttf,BoldItalicFont=cmunbi.ttf]{cmunti.ttf}\setmonofont[Path=/usr/share/fonts/truetype/cmu/,UprightFont=cmuntt.ttf,BoldFont=cmuntb.ttf,ItalicFont=cmunit.ttf,BoldItalicFont=cmuntx.ttf]{cmunti.ttf}\itshape register}{$\text{ }$}\setmainfont[Path=/usr/share/fonts/truetype/cmu/,UprightFont=cmunrm.ttf,BoldFont=cmunbx.ttf,ItalicFont=cmunti.ttf,BoldItalicFont=cmunbi.ttf]{cmunrm.ttf}\setmonofont[Path=/usr/share/fonts/truetype/cmu/,UprightFont=cmuntt.ttf,BoldFont=cmuntb.ttf,ItalicFont=cmunit.ttf,BoldItalicFont=cmuntx.ttf]{cmunrm.ttf} is the TeX way to handle variables. They are limited in numbers (256 for each type of register in classic TeX, 32767 in e-{}TeX).
\item{}  A {\itshape \setmainfont[Path=/usr/share/fonts/truetype/cmu/,UprightFont=cmunrm.ttf,BoldFont=cmunbx.ttf,ItalicFont=cmunti.ttf,BoldItalicFont=cmunbi.ttf]{cmunti.ttf}\setmonofont[Path=/usr/share/fonts/truetype/cmu/,UprightFont=cmuntt.ttf,BoldFont=cmuntb.ttf,ItalicFont=cmunit.ttf,BoldItalicFont=cmuntx.ttf]{cmunti.ttf}\itshape length}{$\text{ }$}\setmainfont[Path=/usr/share/fonts/truetype/cmu/,UprightFont=cmunrm.ttf,BoldFont=cmunbx.ttf,ItalicFont=cmunti.ttf,BoldItalicFont=cmunbi.ttf]{cmunrm.ttf}\setmonofont[Path=/usr/share/fonts/truetype/cmu/,UprightFont=cmuntt.ttf,BoldFont=cmuntb.ttf,ItalicFont=cmunit.ttf,BoldItalicFont=cmuntx.ttf]{cmunrm.ttf} is a control sequence that contains a length (a number followed by a unit). See \mylref{456}{Lengths}.
\item{}  A {\itshape \setmainfont[Path=/usr/share/fonts/truetype/cmu/,UprightFont=cmunrm.ttf,BoldFont=cmunbx.ttf,ItalicFont=cmunti.ttf,BoldItalicFont=cmunbi.ttf]{cmunti.ttf}\setmonofont[Path=/usr/share/fonts/truetype/cmu/,UprightFont=cmuntt.ttf,BoldFont=cmuntb.ttf,ItalicFont=cmunit.ttf,BoldItalicFont=cmuntx.ttf]{cmunti.ttf}\itshape font}{$\text{ }$}\setmainfont[Path=/usr/share/fonts/truetype/cmu/,UprightFont=cmunrm.ttf,BoldFont=cmunbx.ttf,ItalicFont=cmunti.ttf,BoldItalicFont=cmunbi.ttf]{cmunrm.ttf}\setmonofont[Path=/usr/share/fonts/truetype/cmu/,UprightFont=cmuntt.ttf,BoldFont=cmuntb.ttf,ItalicFont=cmunit.ttf,BoldItalicFont=cmuntx.ttf]{cmunrm.ttf} is a control sequence that refers to a font file. See \mylref{163}{Fonts}.
\item{}  A {\itshape \setmainfont[Path=/usr/share/fonts/truetype/cmu/,UprightFont=cmunrm.ttf,BoldFont=cmunbx.ttf,ItalicFont=cmunti.ttf,BoldItalicFont=cmunbi.ttf]{cmunti.ttf}\setmonofont[Path=/usr/share/fonts/truetype/cmu/,UprightFont=cmuntt.ttf,BoldFont=cmuntb.ttf,ItalicFont=cmunit.ttf,BoldItalicFont=cmuntx.ttf]{cmunti.ttf}\itshape box}{$\text{ }$}\setmainfont[Path=/usr/share/fonts/truetype/cmu/,UprightFont=cmunrm.ttf,BoldFont=cmunbx.ttf,ItalicFont=cmunti.ttf,BoldItalicFont=cmunbi.ttf]{cmunrm.ttf}\setmonofont[Path=/usr/share/fonts/truetype/cmu/,UprightFont=cmuntt.ttf,BoldFont=cmuntb.ttf,ItalicFont=cmunit.ttf,BoldItalicFont=cmuntx.ttf]{cmunrm.ttf} is an object that is made for printing. Anything that ends on the paper is a box: letters, paragraphs, pages... See \mylref{478}{Boxes}.
\item{}  A {\itshape \setmainfont[Path=/usr/share/fonts/truetype/cmu/,UprightFont=cmunrm.ttf,BoldFont=cmunbx.ttf,ItalicFont=cmunti.ttf,BoldItalicFont=cmunbi.ttf]{cmunti.ttf}\setmonofont[Path=/usr/share/fonts/truetype/cmu/,UprightFont=cmuntt.ttf,BoldFont=cmuntb.ttf,ItalicFont=cmunit.ttf,BoldItalicFont=cmuntx.ttf]{cmunti.ttf}\itshape glue}{$\text{ }$}\setmainfont[Path=/usr/share/fonts/truetype/cmu/,UprightFont=cmunrm.ttf,BoldFont=cmunbx.ttf,ItalicFont=cmunti.ttf,BoldItalicFont=cmunbi.ttf]{cmunrm.ttf}\setmonofont[Path=/usr/share/fonts/truetype/cmu/,UprightFont=cmuntt.ttf,BoldFont=cmuntb.ttf,ItalicFont=cmunit.ttf,BoldItalicFont=cmuntx.ttf]{cmunrm.ttf} is a certain amount of space that is put between boxes when they are being concatenated.
\item{}  A {\itshape \setmainfont[Path=/usr/share/fonts/truetype/cmu/,UprightFont=cmunrm.ttf,BoldFont=cmunbx.ttf,ItalicFont=cmunti.ttf,BoldItalicFont=cmunbi.ttf]{cmunti.ttf}\setmonofont[Path=/usr/share/fonts/truetype/cmu/,UprightFont=cmuntt.ttf,BoldFont=cmuntb.ttf,ItalicFont=cmunit.ttf,BoldItalicFont=cmuntx.ttf]{cmunti.ttf}\itshape counter}{$\text{ }$}\setmainfont[Path=/usr/share/fonts/truetype/cmu/,UprightFont=cmunrm.ttf,BoldFont=cmunbx.ttf,ItalicFont=cmunti.ttf,BoldItalicFont=cmunbi.ttf]{cmunrm.ttf}\setmonofont[Path=/usr/share/fonts/truetype/cmu/,UprightFont=cmuntt.ttf,BoldFont=cmuntb.ttf,ItalicFont=cmunit.ttf,BoldItalicFont=cmuntx.ttf]{cmunrm.ttf} is a register containing a number. See \mylref{469}{Counters}.
\end{myitemize}


There may be more terms, but we hope that it will do it for now.
\section{Catcodes}
\label{854}

In TeX some characters have a special meaning that is not to print the associated glyph.
For example, \LaTeXTT{\textbackslash{}} is used to introduce a control sequence, and will not print a backslash by default.

To distinguish between different meanings of the characters, TeX split them into {\itshape \setmainfont[Path=/usr/share/fonts/truetype/cmu/,UprightFont=cmunrm.ttf,BoldFont=cmunbx.ttf,ItalicFont=cmunti.ttf,BoldItalicFont=cmunbi.ttf]{cmunti.ttf}\setmonofont[Path=/usr/share/fonts/truetype/cmu/,UprightFont=cmuntt.ttf,BoldFont=cmuntb.ttf,ItalicFont=cmunit.ttf,BoldItalicFont=cmuntx.ttf]{cmunti.ttf}\itshape category codes}\setmainfont[Path=/usr/share/fonts/truetype/cmu/,UprightFont=cmunrm.ttf,BoldFont=cmunbx.ttf,ItalicFont=cmunti.ttf,BoldItalicFont=cmunbi.ttf]{cmunrm.ttf}\setmonofont[Path=/usr/share/fonts/truetype/cmu/,UprightFont=cmuntt.ttf,BoldFont=cmuntb.ttf,ItalicFont=cmunit.ttf,BoldItalicFont=cmuntx.ttf]{cmunrm.ttf}, or {\itshape \setmainfont[Path=/usr/share/fonts/truetype/cmu/,UprightFont=cmunrm.ttf,BoldFont=cmunbx.ttf,ItalicFont=cmunti.ttf,BoldItalicFont=cmunbi.ttf]{cmunti.ttf}\setmonofont[Path=/usr/share/fonts/truetype/cmu/,UprightFont=cmuntt.ttf,BoldFont=cmuntb.ttf,ItalicFont=cmunit.ttf,BoldItalicFont=cmuntx.ttf]{cmunti.ttf}\itshape catcodes}{$\text{ }$}\setmainfont[Path=/usr/share/fonts/truetype/cmu/,UprightFont=cmunrm.ttf,BoldFont=cmunbx.ttf,ItalicFont=cmunti.ttf,BoldItalicFont=cmunbi.ttf]{cmunrm.ttf}\setmonofont[Path=/usr/share/fonts/truetype/cmu/,UprightFont=cmuntt.ttf,BoldFont=cmuntb.ttf,ItalicFont=cmunit.ttf,BoldItalicFont=cmuntx.ttf]{cmunrm.ttf} for short. There are 16 category codes in TeX.

A powerful feature of TeX is its ability to redefine the language itself, since there is a \LaTeXTT{\textbackslash{}catcode} function that will let you change the category code of any characters.

However, this is not recommended, as it can make code difficult to read. Should you redefine any catcode in a class or in a style file, make sure to revert it back at the end of your file.

If you redefine catcodes in your document, make sure to do it after the preamble to prevent clashes with package loading.

\begin{longtable}{|>{\RaggedRight}p{0.07985\linewidth}|>{\RaggedRight}p{0.39949\linewidth}|>{\RaggedRight}p{0.43495\linewidth}|} \hline 
{\bfseries \hspace*{0pt}\ignorespaces{}\hspace*{0pt} Code}&{\bfseries \hspace*{0pt}\ignorespaces{}\hspace*{0pt} Description}&{\bfseries \hspace*{0pt}\ignorespaces{}\hspace*{0pt} Default set}\endhead  \hline \hspace*{0pt}\ignorespaces{}\hspace*{0pt} 0  &\hspace*{0pt}\ignorespaces{}\hspace*{0pt} Escape character and control sequences &\hspace*{0pt}\ignorespaces{}\hspace*{0pt} \LaTeXTT{\textbackslash{}} \\ \hline \hspace*{0pt}\ignorespaces{}\hspace*{0pt} 1  &\hspace*{0pt}\ignorespaces{}\hspace*{0pt} Beginning of group &\hspace*{0pt}\ignorespaces{}\hspace*{0pt} \LaTeXTT{\{} \\ \hline \hspace*{0pt}\ignorespaces{}\hspace*{0pt} 2  &\hspace*{0pt}\ignorespaces{}\hspace*{0pt} End of group &\hspace*{0pt}\ignorespaces{}\hspace*{0pt} \LaTeXTT{\}}\\ \hline \hspace*{0pt}\ignorespaces{}\hspace*{0pt} 3  &\hspace*{0pt}\ignorespaces{}\hspace*{0pt} Math shift &\hspace*{0pt}\ignorespaces{}\hspace*{0pt} \LaTeXTT{\${}}\\ \hline \hspace*{0pt}\ignorespaces{}\hspace*{0pt} 4  &\hspace*{0pt}\ignorespaces{}\hspace*{0pt} Alignment tab &\hspace*{0pt}\ignorespaces{}\hspace*{0pt} \LaTeXTT{\&}\\ \hline \hspace*{0pt}\ignorespaces{}\hspace*{0pt} 5  &\hspace*{0pt}\ignorespaces{}\hspace*{0pt} End of line &\hspace*{0pt}\ignorespaces{}\hspace*{0pt} \LaTeXTT{\^{}\^{}M} (ASCII return)\\ \hline \hspace*{0pt}\ignorespaces{}\hspace*{0pt} 6  &\hspace*{0pt}\ignorespaces{}\hspace*{0pt} Macro parameter &\hspace*{0pt}\ignorespaces{}\hspace*{0pt} \LaTeXTT{\#}\\ \hline \hspace*{0pt}\ignorespaces{}\hspace*{0pt} 7  &\hspace*{0pt}\ignorespaces{}\hspace*{0pt} Superscript &\hspace*{0pt}\ignorespaces{}\hspace*{0pt} \LaTeXTT{\^{}} and \LaTeXTT{\^{}\^{}K}\\ \hline \hspace*{0pt}\ignorespaces{}\hspace*{0pt} 8  &\hspace*{0pt}\ignorespaces{}\hspace*{0pt} Subscript &\hspace*{0pt}\ignorespaces{}\hspace*{0pt} \LaTeXTT{\_} and \LaTeXTT{\^{}\^{}A}\\ \hline \hspace*{0pt}\ignorespaces{}\hspace*{0pt} 9  &\hspace*{0pt}\ignorespaces{}\hspace*{0pt} Ignored character &\hspace*{0pt}\ignorespaces{}\hspace*{0pt} \LaTeXTT{\^{}\^{}@} (ASCII null)\\ \hline \hspace*{0pt}\ignorespaces{}\hspace*{0pt} 10 &\hspace*{0pt}\ignorespaces{}\hspace*{0pt} Space &\hspace*{0pt}\ignorespaces{}\hspace*{0pt} \LaTeXTT{␣} and \LaTeXTT{\^{}\^{}I} (ASCII horizontal tab)\\ \hline \hspace*{0pt}\ignorespaces{}\hspace*{0pt} 11 &\hspace*{0pt}\ignorespaces{}\hspace*{0pt} Letter &\hspace*{0pt}\ignorespaces{}\hspace*{0pt} \LaTeXTT{A...Z} and \LaTeXTT{a...z}\\ \hline \hspace*{0pt}\ignorespaces{}\hspace*{0pt} 12 &\hspace*{0pt}\ignorespaces{}\hspace*{0pt} Other character &\hspace*{0pt}\ignorespaces{}\hspace*{0pt} everything not listed in the other catcodes. Most notably, @.\\ \hline \hspace*{0pt}\ignorespaces{}\hspace*{0pt} 13 &\hspace*{0pt}\ignorespaces{}\hspace*{0pt} Active character &\hspace*{0pt}\ignorespaces{}\hspace*{0pt} \LaTeXTT{\~{}} and \LaTeXTT{\^{}\^{}L} (ASCII form feed)\\ \hline \hspace*{0pt}\ignorespaces{}\hspace*{0pt} 14 &\hspace*{0pt}\ignorespaces{}\hspace*{0pt} Comment character &\hspace*{0pt}\ignorespaces{}\hspace*{0pt} \LaTeXTT{\%}\\ \hline \hspace*{0pt}\ignorespaces{}\hspace*{0pt} 15 &\hspace*{0pt}\ignorespaces{}\hspace*{0pt} Invalid character &\hspace*{0pt}\ignorespaces{}\hspace*{0pt} \LaTeXTT{\^{}\^{}?} (ASCII delete)\\ \hline 
\end{longtable}

\subsection{Active characters}
\label{855}

Active characters resemble macros: they are single characters that will expand before any other command.
\begin{longtable}{p{1.0\linewidth}}
\begin{Shaded}
\begin{Highlighting}[]

\NormalTok{\textbackslash{}catcode` = 13}
\NormalTok{\textbackslash{}def\{\textbackslash{}TeX\}}
\NormalTok{...}
\NormalTok{This is a stupid example of .}
\end{Highlighting}
\end{Shaded}
\\

This is a stupid example of TeX.

\end{longtable}
Note that an active character needs to be directly followed by a definition, otherwise the compilation will fail.
\subsection{Examples}
\label{856}
{\bfseries
\begin{mydescription}Texinfo
\end{mydescription}
}

\myhref{https://en.wikipedia.org/wiki/Texinfo}{Texinfo} uses a syntax similar to TeX with one major difference: all functions are introduced with a @ instead of a \LaTeXTT{\textbackslash{}}. This is not by chance: it actually uses TeX to print the PDF version of the files.
What it basically does is inputting {\ttfamily \setmainfont[Path=/usr/share/fonts/truetype/cmu/,UprightFont=cmunrm.ttf,BoldFont=cmunbx.ttf,ItalicFont=cmunti.ttf,BoldItalicFont=cmunbi.ttf]{cmuntt.ttf}\setmonofont[Path=/usr/share/fonts/truetype/cmu/,UprightFont=cmuntt.ttf,BoldFont=cmuntb.ttf,ItalicFont=cmunit.ttf,BoldItalicFont=cmuntx.ttf]{cmuntt.ttf}\ttfamily texinfo.tex}{$\text{ }$}\setmainfont[Path=/usr/share/fonts/truetype/cmu/,UprightFont=cmunrm.ttf,BoldFont=cmunbx.ttf,ItalicFont=cmunti.ttf,BoldItalicFont=cmunbi.ttf]{cmunrm.ttf}\setmonofont[Path=/usr/share/fonts/truetype/cmu/,UprightFont=cmuntt.ttf,BoldFont=cmuntb.ttf,ItalicFont=cmunit.ttf,BoldItalicFont=cmuntx.ttf]{cmunrm.ttf} which redefines the control sequence character. Possible implementation:

\begin{longtable}{p{1.0\linewidth}}
\begin{Shaded}
\begin{Highlighting}[]

\NormalTok{\textbackslash{}catcode`\textbackslash{}@=0}
\NormalTok{@def@@\{@char64\} }\CommentTok{% To write '@' character.}
\NormalTok{\textbackslash{}catcode`\textbackslash{}\textbackslash{}=13 @def\textbackslash{}\{\{@tt @char92\}\}}
 
\NormalTok{The @TeX command was previously written '\textbackslash{}TeX'. It is now written '@@TeX'.}
\end{Highlighting}
\end{Shaded}
\\

The TeX command was previously written \textquotesingle{}\textbackslash{}TeX\textquotesingle{}.
It is now written \textquotesingle{}@TeX\textquotesingle{}.

\end{longtable}
With this redefinition, the \textquotesingle{}@\textquotesingle{} should now introduce every command, while the \textquotesingle{}\textbackslash{}\textquotesingle{} will actually print a backslash character.
{\bfseries
\begin{mydescription}Itemize
\end{mydescription}
}

Some may find the LaTeX syntax of list environments a bit cumbersome. Here is a quick way to define a wiki-{}like itemize:

\begin{Shaded}
\begin{Highlighting}[]

\NormalTok{\textbackslash{}catcode` = 13}
\NormalTok{\textbackslash{}def\{\textbackslash{}item \{--\}\}}
\NormalTok{\textbackslash{}def\textbackslash{}itemize#1\{\{\textbackslash{}leftskip = 40 pt #1 \textbackslash{}par\}\}}
 
\NormalTok{\textbackslash{}itemize\{}
 \NormalTok{First item}
 \NormalTok{Second item}
\NormalTok{\}}
\end{Highlighting}
\end{Shaded}

{\bfseries
\begin{mydescription}Dollar and math
\end{mydescription}
}

If you have many \textquotesingle{}dollar\textquotesingle{} symbols to print, you may be better off to change the math shift character.
\begin{Shaded}
\begin{Highlighting}[]

\NormalTok{\textbackslash{}catcode`$ = 11}
\NormalTok{\textbackslash{}catcode` = 3}
 
\NormalTok{It costs $100.}
\NormalTok{Let's do the math: 50+50=100. Let's highlight it:}
\NormalTok{50+50=100}
\end{Highlighting}
\end{Shaded}

\subsection{{\itshape \setmainfont[Path=/usr/share/fonts/truetype/cmu/,UprightFont=cmunrm.ttf,BoldFont=cmunbx.ttf,ItalicFont=cmunti.ttf,BoldItalicFont=cmunbi.ttf]{cmunti.ttf}\setmonofont[Path=/usr/share/fonts/truetype/cmu/,UprightFont=cmuntt.ttf,BoldFont=cmuntb.ttf,ItalicFont=cmunit.ttf,BoldItalicFont=cmuntx.ttf]{cmunti.ttf}\itshape \textbackslash{}makeatletter}{$\text{ }$}\setmainfont[Path=/usr/share/fonts/truetype/cmu/,UprightFont=cmunrm.ttf,BoldFont=cmunbx.ttf,ItalicFont=cmunti.ttf,BoldItalicFont=cmunbi.ttf]{cmunrm.ttf}\setmonofont[Path=/usr/share/fonts/truetype/cmu/,UprightFont=cmuntt.ttf,BoldFont=cmuntb.ttf,ItalicFont=cmunit.ttf,BoldItalicFont=cmuntx.ttf]{cmunrm.ttf} and {\itshape \setmainfont[Path=/usr/share/fonts/truetype/cmu/,UprightFont=cmunrm.ttf,BoldFont=cmunbx.ttf,ItalicFont=cmunti.ttf,BoldItalicFont=cmunbi.ttf]{cmunti.ttf}\setmonofont[Path=/usr/share/fonts/truetype/cmu/,UprightFont=cmuntt.ttf,BoldFont=cmuntb.ttf,ItalicFont=cmunit.ttf,BoldItalicFont=cmuntx.ttf]{cmunti.ttf}\itshape \textbackslash{}makeatother}}
\label{857}\setmainfont[Path=/usr/share/fonts/truetype/cmu/,UprightFont=cmunrm.ttf,BoldFont=cmunbx.ttf,ItalicFont=cmunti.ttf,BoldItalicFont=cmunbi.ttf]{cmunrm.ttf}\setmonofont[Path=/usr/share/fonts/truetype/cmu/,UprightFont=cmuntt.ttf,BoldFont=cmuntb.ttf,ItalicFont=cmunit.ttf,BoldItalicFont=cmuntx.ttf]{cmunrm.ttf}

If you have done a bit of LaTeX hacking, you must have encountered those two commands, \LaTeXTT{\textbackslash{}makeatletter} and \LaTeXTT{\textbackslash{}makeatother}.

In TeX the \textquotesingle{}@\textquotesingle{} characters belongs to catcode 11 {\itshape \setmainfont[Path=/usr/share/fonts/truetype/cmu/,UprightFont=cmunrm.ttf,BoldFont=cmunbx.ttf,ItalicFont=cmunti.ttf,BoldItalicFont=cmunbi.ttf]{cmunti.ttf}\setmonofont[Path=/usr/share/fonts/truetype/cmu/,UprightFont=cmuntt.ttf,BoldFont=cmuntb.ttf,ItalicFont=cmunit.ttf,BoldItalicFont=cmuntx.ttf]{cmunti.ttf}\itshape letters}{$\text{ }$}\setmainfont[Path=/usr/share/fonts/truetype/cmu/,UprightFont=cmunrm.ttf,BoldFont=cmunbx.ttf,ItalicFont=cmunti.ttf,BoldItalicFont=cmunbi.ttf]{cmunrm.ttf}\setmonofont[Path=/usr/share/fonts/truetype/cmu/,UprightFont=cmuntt.ttf,BoldFont=cmuntb.ttf,ItalicFont=cmunit.ttf,BoldItalicFont=cmuntx.ttf]{cmunrm.ttf} by default. It means you can use it for macro names.
LaTeX makes use of the catcode to specify a rule: all non-{}public, internal macros that are not supposed to be accessed by the end-{}user contains at least one \textquotesingle{}@\textquotesingle{} character in their name.
In the document, LaTeX changes the catcode of \textquotesingle{}@\textquotesingle{} to 12, {\itshape \setmainfont[Path=/usr/share/fonts/truetype/cmu/,UprightFont=cmunrm.ttf,BoldFont=cmunbx.ttf,ItalicFont=cmunti.ttf,BoldItalicFont=cmunbi.ttf]{cmunti.ttf}\setmonofont[Path=/usr/share/fonts/truetype/cmu/,UprightFont=cmuntt.ttf,BoldFont=cmuntb.ttf,ItalicFont=cmunit.ttf,BoldItalicFont=cmuntx.ttf]{cmunti.ttf}\itshape others}\setmainfont[Path=/usr/share/fonts/truetype/cmu/,UprightFont=cmunrm.ttf,BoldFont=cmunbx.ttf,ItalicFont=cmunti.ttf,BoldItalicFont=cmunbi.ttf]{cmunrm.ttf}\setmonofont[Path=/usr/share/fonts/truetype/cmu/,UprightFont=cmuntt.ttf,BoldFont=cmuntb.ttf,ItalicFont=cmunit.ttf,BoldItalicFont=cmuntx.ttf]{cmunrm.ttf}.

That\textquotesingle{}s why when you need to access LaTeX internals, you must enclose all the commands accessing private functions with \LaTeXTT{\textbackslash{}makeatletter} and \LaTeXTT{\textbackslash{}makeatother}. All they do is just changing the catcode:

\begin{Shaded}
\begin{Highlighting}[]

\NormalTok{\textbackslash{}def\textbackslash{}makeatletter\{\textbackslash{}catcode`@ = 11\}}
\NormalTok{\textbackslash{}def\textbackslash{}makeatother\{\textbackslash{}catcode`@ = 12\}}
\end{Highlighting}
\end{Shaded}

\section{Plain TeX macros}
\label{858}

\LaTeXTT{\textbackslash{}newcommand} and \LaTeXTT{\textbackslash{}renewcommand} are LaTeX-{}specific control sequences. They check that no existing command gets shadowed by the new definition.

In Plain TeX, the primitives for macro definition make no check on possible shadowing. It\textquotesingle{}s up to you to make sure you are not breaking anything.

The syntax is
\begin{Shaded}
\begin{Highlighting}[]

\NormalTok{\textbackslash{}def<macroname>#1<sep1>#2<sep2>\{macro content, use of argument #1, blah, #2 ...\}}
\end{Highlighting}
\end{Shaded}


You can use (almost) any sequence of character between arguments. For instance let\textquotesingle{}s write a simple macro that will convert the decimal separator from point  to comma. First try:

\begin{Shaded}
\begin{Highlighting}[]

\NormalTok{\textbackslash{}def\textbackslash{}pointtocomma #1.#2\{(#1,#2)\}}
\CommentTok{%%...}
 
\NormalTok{\textbackslash{}pointtocomma 123.456}
\end{Highlighting}
\end{Shaded}


This will print {\itshape \setmainfont[Path=/usr/share/fonts/truetype/cmu/,UprightFont=cmunrm.ttf,BoldFont=cmunbx.ttf,ItalicFont=cmunti.ttf,BoldItalicFont=cmunbi.ttf]{cmunti.ttf}\setmonofont[Path=/usr/share/fonts/truetype/cmu/,UprightFont=cmuntt.ttf,BoldFont=cmuntb.ttf,ItalicFont=cmunit.ttf,BoldItalicFont=cmuntx.ttf]{cmunti.ttf}\itshape (123,4)56}\setmainfont[Path=/usr/share/fonts/truetype/cmu/,UprightFont=cmunrm.ttf,BoldFont=cmunbx.ttf,ItalicFont=cmunti.ttf,BoldItalicFont=cmunbi.ttf]{cmunrm.ttf}\setmonofont[Path=/usr/share/fonts/truetype/cmu/,UprightFont=cmuntt.ttf,BoldFont=cmuntb.ttf,ItalicFont=cmunit.ttf,BoldItalicFont=cmuntx.ttf]{cmunrm.ttf}. We added the parentheses just to highlight the issue here. Each parameter is the shortest possible input sequence that matches the macro definition, separators included. Thus \LaTeXTT{\#1} matches all characters up to the first point, and \LaTeXTT{\#2} matches the first token only, {\itshape \setmainfont[Path=/usr/share/fonts/truetype/cmu/,UprightFont=cmunrm.ttf,BoldFont=cmunbx.ttf,ItalicFont=cmunti.ttf,BoldItalicFont=cmunbi.ttf]{cmunti.ttf}\setmonofont[Path=/usr/share/fonts/truetype/cmu/,UprightFont=cmuntt.ttf,BoldFont=cmuntb.ttf,ItalicFont=cmunit.ttf,BoldItalicFont=cmuntx.ttf]{cmunti.ttf}\itshape i.e.}{$\text{ }$}\setmainfont[Path=/usr/share/fonts/truetype/cmu/,UprightFont=cmunrm.ttf,BoldFont=cmunbx.ttf,ItalicFont=cmunti.ttf,BoldItalicFont=cmunbi.ttf]{cmunrm.ttf}\setmonofont[Path=/usr/share/fonts/truetype/cmu/,UprightFont=cmuntt.ttf,BoldFont=cmuntb.ttf,ItalicFont=cmunit.ttf,BoldItalicFont=cmuntx.ttf]{cmunrm.ttf} the first character, since there is no separator after it.

Solution: add a second separator. A space may seem convenient:
\begin{Shaded}
\begin{Highlighting}[]

\NormalTok{\textbackslash{}def\textbackslash{}pointtocomma #1.#2 \{(#1,#2)\}}
\end{Highlighting}
\end{Shaded}


As a general rule, everytime you expect several parameters with specific separators, think out the last separator. If you do not want to play with separators, then Plain TeX macros are used just as LaTeX macros (without default parameter):

\begin{Shaded}
\begin{Highlighting}[]

\NormalTok{\textbackslash{}def\textbackslash{}mymacro#1#2#3\{\{\textbackslash{}bf #1\}#2\{\textbackslash{}bf #3\}\}}
\CommentTok{%% ...}
\NormalTok{\textbackslash{}mymacro\{word1\}\{word2 word3\}\{!!!\}}
\end{Highlighting}
\end{Shaded}

\subsection{Expanded definitions}
\label{859}

TeX has another definition command: \LaTeXTT{\textbackslash{}edef}, which stands for {\itshape \setmainfont[Path=/usr/share/fonts/truetype/cmu/,UprightFont=cmunrm.ttf,BoldFont=cmunbx.ttf,ItalicFont=cmunti.ttf,BoldItalicFont=cmunbi.ttf]{cmunti.ttf}\setmonofont[Path=/usr/share/fonts/truetype/cmu/,UprightFont=cmuntt.ttf,BoldFont=cmuntb.ttf,ItalicFont=cmunit.ttf,BoldItalicFont=cmuntx.ttf]{cmunti.ttf}\itshape expanded def}\setmainfont[Path=/usr/share/fonts/truetype/cmu/,UprightFont=cmunrm.ttf,BoldFont=cmunbx.ttf,ItalicFont=cmunti.ttf,BoldItalicFont=cmunbi.ttf]{cmunrm.ttf}\setmonofont[Path=/usr/share/fonts/truetype/cmu/,UprightFont=cmuntt.ttf,BoldFont=cmuntb.ttf,ItalicFont=cmunit.ttf,BoldItalicFont=cmuntx.ttf]{cmunrm.ttf}. The syntax remains the same:
\begin{Shaded}
\begin{Highlighting}[]

\NormalTok{\textbackslash{}edef<macroname><argumentslist>\{<expanded content>\}}
\end{Highlighting}
\end{Shaded}

The content gets expanded (but not executed, {\itshape \setmainfont[Path=/usr/share/fonts/truetype/cmu/,UprightFont=cmunrm.ttf,BoldFont=cmunbx.ttf,ItalicFont=cmunti.ttf,BoldItalicFont=cmunbi.ttf]{cmunti.ttf}\setmonofont[Path=/usr/share/fonts/truetype/cmu/,UprightFont=cmuntt.ttf,BoldFont=cmuntb.ttf,ItalicFont=cmunit.ttf,BoldItalicFont=cmuntx.ttf]{cmunti.ttf}\itshape i.e.}{$\text{ }$}\setmainfont[Path=/usr/share/fonts/truetype/cmu/,UprightFont=cmunrm.ttf,BoldFont=cmunbx.ttf,ItalicFont=cmunti.ttf,BoldItalicFont=cmunbi.ttf]{cmunrm.ttf}\setmonofont[Path=/usr/share/fonts/truetype/cmu/,UprightFont=cmuntt.ttf,BoldFont=cmuntb.ttf,ItalicFont=cmunit.ttf,BoldItalicFont=cmuntx.ttf]{cmunrm.ttf} printed) at the point where \LaTeXTT{\textbackslash{}edef} is used, instead of where the defined macro is used. Macro expansion is not always obvious...

Example:
\begin{Shaded}
\begin{Highlighting}[]

\NormalTok{\textbackslash{}def\textbackslash{}intro\{Example\}}
\NormalTok{\textbackslash{}edef\textbackslash{}example#1\{\textbackslash{}intro~---~#1\}}
\NormalTok{\textbackslash{}def\textbackslash{}intro\{Exercise\}}
 
\NormalTok{\textbackslash{}example\{This is an example\}}
\end{Highlighting}
\end{Shaded}


Here the redefinition of \LaTeXTT{\textbackslash{}intro} will have no effect on \LaTeXTT{\textbackslash{}example}.
\subsection{Global definitions}
\label{860}

Definitions are limited to their scope. However it might be convenient sometimes to define a macro inside a group that remain valid outside the group, and until the end of the document. This is what we call {\itshape \setmainfont[Path=/usr/share/fonts/truetype/cmu/,UprightFont=cmunrm.ttf,BoldFont=cmunbx.ttf,ItalicFont=cmunti.ttf,BoldItalicFont=cmunbi.ttf]{cmunti.ttf}\setmonofont[Path=/usr/share/fonts/truetype/cmu/,UprightFont=cmuntt.ttf,BoldFont=cmuntb.ttf,ItalicFont=cmunit.ttf,BoldItalicFont=cmuntx.ttf]{cmunti.ttf}\itshape global definitions}\setmainfont[Path=/usr/share/fonts/truetype/cmu/,UprightFont=cmunrm.ttf,BoldFont=cmunbx.ttf,ItalicFont=cmunti.ttf,BoldItalicFont=cmunbi.ttf]{cmunrm.ttf}\setmonofont[Path=/usr/share/fonts/truetype/cmu/,UprightFont=cmuntt.ttf,BoldFont=cmuntb.ttf,ItalicFont=cmunit.ttf,BoldItalicFont=cmuntx.ttf]{cmunrm.ttf}.

\begin{Shaded}
\begin{Highlighting}[]

\NormalTok{\{}
\NormalTok{\textbackslash{}def\textbackslash{}LocalTeX\{Local\textbackslash{}TeX\}}
\NormalTok{\textbackslash{}global\textbackslash{}def\textbackslash{}GlobalTeX\{Global\textbackslash{}TeX\}}
\NormalTok{\}}
\NormalTok{I can still access the \textbackslash{}GlobalTeX\{\} macro here.}
\end{Highlighting}
\end{Shaded}


You can also use the \LaTeXTT{\textbackslash{}global} command with \LaTeXTT{\textbackslash{}edef}.

Both commands have a shortcut:
\begin{myitemize}
\item{}  \LaTeXTT{\textbackslash{}gdef} for \LaTeXTT{\textbackslash{}global\textbackslash{}def}
\item{}  \LaTeXTT{\textbackslash{}xdef} for \LaTeXTT{\textbackslash{}global\textbackslash{}edef}
\end{myitemize}

\subsection{Long definitions}
\label{861}

The previous definition commands would not allow you to use them over multiple paragraphs, {\itshape \setmainfont[Path=/usr/share/fonts/truetype/cmu/,UprightFont=cmunrm.ttf,BoldFont=cmunbx.ttf,ItalicFont=cmunti.ttf,BoldItalicFont=cmunbi.ttf]{cmunti.ttf}\setmonofont[Path=/usr/share/fonts/truetype/cmu/,UprightFont=cmuntt.ttf,BoldFont=cmuntb.ttf,ItalicFont=cmunit.ttf,BoldItalicFont=cmuntx.ttf]{cmunti.ttf}\itshape i.e.}{$\text{ }$}\setmainfont[Path=/usr/share/fonts/truetype/cmu/,UprightFont=cmunrm.ttf,BoldFont=cmunbx.ttf,ItalicFont=cmunti.ttf,BoldItalicFont=cmunbi.ttf]{cmunrm.ttf}\setmonofont[Path=/usr/share/fonts/truetype/cmu/,UprightFont=cmuntt.ttf,BoldFont=cmuntb.ttf,ItalicFont=cmunit.ttf,BoldItalicFont=cmuntx.ttf]{cmunrm.ttf} text containing the \LaTeXTT{\textbackslash{}par} command -{}-{} or double line breaks.

You can prefix the definition with the \LaTeXTT{\textbackslash{}long} command to allow multi-{}paragraph arguments.

Example:
\begin{Shaded}
\begin{Highlighting}[]

\NormalTok{\textbackslash{}long\textbackslash{}def\textbackslash{}dummy#1\{#1\}}
\NormalTok{\textbackslash{}dummy\{First paragraph\textbackslash{}par Second paragraph\}}
\end{Highlighting}
\end{Shaded}

\subsection{Outer definitions}
\label{862}

This prefix macro prevent definitions from being used in some context. It is
useful to consolidate macros and make them less error-{}prone because of bad
contexts. {\itshape \setmainfont[Path=/usr/share/fonts/truetype/cmu/,UprightFont=cmunrm.ttf,BoldFont=cmunbx.ttf,ItalicFont=cmunti.ttf,BoldItalicFont=cmunbi.ttf]{cmunti.ttf}\setmonofont[Path=/usr/share/fonts/truetype/cmu/,UprightFont=cmuntt.ttf,BoldFont=cmuntb.ttf,ItalicFont=cmunit.ttf,BoldItalicFont=cmuntx.ttf]{cmunti.ttf}\itshape Outer macros}{$\text{ }$}\setmainfont[Path=/usr/share/fonts/truetype/cmu/,UprightFont=cmunrm.ttf,BoldFont=cmunbx.ttf,ItalicFont=cmunti.ttf,BoldItalicFont=cmunbi.ttf]{cmunrm.ttf}\setmonofont[Path=/usr/share/fonts/truetype/cmu/,UprightFont=cmuntt.ttf,BoldFont=cmuntb.ttf,ItalicFont=cmunit.ttf,BoldItalicFont=cmuntx.ttf]{cmunrm.ttf} are meant to be used outside of any context, hence
the name.

For instance the following code will fail:
\begin{Shaded}
\begin{Highlighting}[]

\NormalTok{\textbackslash{}outer\textbackslash{}def\textbackslash{}test\{a test\}}
\NormalTok{\textbackslash{}def\textbackslash{}failure\{\textbackslash{}test\}}
\end{Highlighting}
\end{Shaded}


Outer macros are not allowed to appear in:
\begin{myitemize}
\item{}  macro parameters
\item{}  skipped conditional
\item{}  ...
\end{myitemize}

\subsection{{\itshape \setmainfont[Path=/usr/share/fonts/truetype/cmu/,UprightFont=cmunrm.ttf,BoldFont=cmunbx.ttf,ItalicFont=cmunti.ttf,BoldItalicFont=cmunbi.ttf]{cmunti.ttf}\setmonofont[Path=/usr/share/fonts/truetype/cmu/,UprightFont=cmuntt.ttf,BoldFont=cmuntb.ttf,ItalicFont=cmunit.ttf,BoldItalicFont=cmuntx.ttf]{cmunti.ttf}\itshape let}{$\text{ }$}\setmainfont[Path=/usr/share/fonts/truetype/cmu/,UprightFont=cmunrm.ttf,BoldFont=cmunbx.ttf,ItalicFont=cmunti.ttf,BoldItalicFont=cmunbi.ttf]{cmunrm.ttf}\setmonofont[Path=/usr/share/fonts/truetype/cmu/,UprightFont=cmuntt.ttf,BoldFont=cmuntb.ttf,ItalicFont=cmunit.ttf,BoldItalicFont=cmuntx.ttf]{cmunrm.ttf} and {\itshape \setmainfont[Path=/usr/share/fonts/truetype/cmu/,UprightFont=cmunrm.ttf,BoldFont=cmunbx.ttf,ItalicFont=cmunti.ttf,BoldItalicFont=cmunbi.ttf]{cmunti.ttf}\setmonofont[Path=/usr/share/fonts/truetype/cmu/,UprightFont=cmuntt.ttf,BoldFont=cmuntb.ttf,ItalicFont=cmunit.ttf,BoldItalicFont=cmuntx.ttf]{cmunti.ttf}\itshape futurelet}}
\label{863}\setmainfont[Path=/usr/share/fonts/truetype/cmu/,UprightFont=cmunrm.ttf,BoldFont=cmunbx.ttf,ItalicFont=cmunti.ttf,BoldItalicFont=cmunbi.ttf]{cmunrm.ttf}\setmonofont[Path=/usr/share/fonts/truetype/cmu/,UprightFont=cmuntt.ttf,BoldFont=cmuntb.ttf,ItalicFont=cmunit.ttf,BoldItalicFont=cmuntx.ttf]{cmunrm.ttf}

\LaTeXTT{\textbackslash{}let<{}csname>{}<{}token>{}} is the same as \LaTeXTT{\textbackslash{}expandafter\textbackslash{}def\textbackslash{}expandafter<{}csname>{}\textbackslash{}expandafter\{<{}content>{}\}}. It defines a new control sequence name which is equivalent to the specified \LaTeXTT{token}. The \LaTeXTT{token} is usually another control sequence.

Note that \LaTeXTT{\textbackslash{}let} will expand the \LaTeXTT{token} one time only, contrary to \LaTeXTT{\textbackslash{}edef} which will expand recursively until no further expansion is possible.

Example\myfootnote{From \myfnhref{http://tex.stackexchange.com/questions/8163/what-is-the-difference-between-let-and-edef}{tex.stackexchange.com}: What is the difference between \textbackslash{}let and \textbackslash{}edef?}:
\begin{Shaded}
\begin{Highlighting}[]

\NormalTok{Using let:\textbackslash{}par}
\NormalTok{\textbackslash{}def\textbackslash{}txt\{a\}}
\NormalTok{\textbackslash{}def\textbackslash{}foo\{\textbackslash{}txt\}}
\NormalTok{\textbackslash{}let\textbackslash{}bar\textbackslash{}foo}
\NormalTok{\textbackslash{}bar }\CommentTok{% Prints a}
\NormalTok{\textbackslash{}def\textbackslash{}txt\{b\}}
\NormalTok{\textbackslash{}bar }\CommentTok{% Prints b}
 
\NormalTok{Using edef:\textbackslash{}par}
\NormalTok{\textbackslash{}def\textbackslash{}txt\{a\}}
\NormalTok{\textbackslash{}def\textbackslash{}foo\{\textbackslash{}txt\}}
\NormalTok{\textbackslash{}edef\textbackslash{}bar\{\textbackslash{}foo\}}
\NormalTok{\textbackslash{}bar }\CommentTok{% Prints a}
\NormalTok{\textbackslash{}def\textbackslash{}txt\{b\}}
\NormalTok{\textbackslash{}bar }\CommentTok{% Prints a}
\end{Highlighting}
\end{Shaded}


\LaTeXTT{\textbackslash{}futurelet<{}csname>{}<{}token1>{}<{}token2>{}...} works a bit differently. \LaTeXTT{token2} is assigned to \LaTeXTT{csname}; after that TeX processes the \LaTeXTT{<{}token1>{}<{}token2>{}...} sequence. So \LaTeXTT{\textbackslash{}futurelet} allows you to assign a token while using it right after.
\subsection{Special control sequence name}
\label{864}

Some macros may have a name that is not directly writable as is. This is the case of macros whose name is made up of macro names.
Example:
\begin{Shaded}
\begin{Highlighting}[]

\NormalTok{\textbackslash{}def\textbackslash{}status\{full\}}
\NormalTok{\textbackslash{}def\textbackslash{}varempty\{This is empty\}}
\NormalTok{\textbackslash{}def\textbackslash{}varfull\{This is full\}}
 
\NormalTok{\textbackslash{}csname var\textbackslash{}status \textbackslash{}endcsname}
\end{Highlighting}
\end{Shaded}

The last line will print a sentence depending on the \LaTeXTT{\textbackslash{}status}.

This command actually does the opposite of \LaTeXTT{\textbackslash{}string} which prints a control sequence name without expanding it:
\begin{longtable}{p{1.0\linewidth}}
\begin{Shaded}
\begin{Highlighting}[]

\NormalTok{\{\textbackslash{}tt \textbackslash{}string\textbackslash{}TeX\}}
\end{Highlighting}
\end{Shaded}
\\

{\ttfamily \setmainfont[Path=/usr/share/fonts/truetype/cmu/,UprightFont=cmunrm.ttf,BoldFont=cmunbx.ttf,ItalicFont=cmunti.ttf,BoldItalicFont=cmunbi.ttf]{cmuntt.ttf}\setmonofont[Path=/usr/share/fonts/truetype/cmu/,UprightFont=cmuntt.ttf,BoldFont=cmuntb.ttf,ItalicFont=cmunit.ttf,BoldItalicFont=cmuntx.ttf]{cmuntt.ttf}\ttfamily \textbackslash{}TeX}\setmainfont[Path=/usr/share/fonts/truetype/cmu/,UprightFont=cmunrm.ttf,BoldFont=cmunbx.ttf,ItalicFont=cmunti.ttf,BoldItalicFont=cmunbi.ttf]{cmunrm.ttf}\setmonofont[Path=/usr/share/fonts/truetype/cmu/,UprightFont=cmuntt.ttf,BoldFont=cmuntb.ttf,ItalicFont=cmunit.ttf,BoldItalicFont=cmuntx.ttf]{cmunrm.ttf}

\end{longtable}
\subsection{Controlling expansion}
\label{865}

\LaTeXTT{\textbackslash{}expandafter\{token1\}\{token2\}} will expand \LaTeXTT{token2} before \LaTeXTT{token1}. It is sometimes needed when \LaTeXTT{token2} expansion is desired but cannot happen because of \LaTeXTT{token1}.

\begin{longtable}{p{1.0\linewidth}}
\begin{Shaded}
\begin{Highlighting}[]

\NormalTok{\{\textbackslash{}tt \textbackslash{}expandafter\textbackslash{}string\textbackslash{}csname TeX\textbackslash{}endcsname\}}
\end{Highlighting}
\end{Shaded}
\\

{\ttfamily \setmainfont[Path=/usr/share/fonts/truetype/cmu/,UprightFont=cmunrm.ttf,BoldFont=cmunbx.ttf,ItalicFont=cmunti.ttf,BoldItalicFont=cmunbi.ttf]{cmuntt.ttf}\setmonofont[Path=/usr/share/fonts/truetype/cmu/,UprightFont=cmuntt.ttf,BoldFont=cmuntb.ttf,ItalicFont=cmunit.ttf,BoldItalicFont=cmuntx.ttf]{cmuntt.ttf}\ttfamily \textbackslash{}TeX}\setmainfont[Path=/usr/share/fonts/truetype/cmu/,UprightFont=cmunrm.ttf,BoldFont=cmunbx.ttf,ItalicFont=cmunti.ttf,BoldItalicFont=cmunbi.ttf]{cmunrm.ttf}\setmonofont[Path=/usr/share/fonts/truetype/cmu/,UprightFont=cmuntt.ttf,BoldFont=cmuntb.ttf,ItalicFont=cmunit.ttf,BoldItalicFont=cmuntx.ttf]{cmunrm.ttf}

\end{longtable}

\LaTeXTT{\textbackslash{}noexpand} is useful to have fine grained control over what gets expanded in an \LaTeXTT{\textbackslash{}edef}. Example:
\begin{Shaded}
\begin{Highlighting}[]

\NormalTok{\textbackslash{}def\textbackslash{}intro\{Example\}}
\NormalTok{\textbackslash{}def\textbackslash{}separator\{~---~\}}
\NormalTok{\textbackslash{}edef\textbackslash{}example#1\{\textbackslash{}intro\textbackslash{}noexpand\textbackslash{}separator#1\} }
 
\NormalTok{\textbackslash{}example\{no expand makes the separator dynamic in an \{\textbackslash{}tt \textbackslash{}string\textbackslash{}edef}\AlertTok{\}}\NormalTok{.\}}
 
\NormalTok{\textbackslash{}def\textbackslash{}intro\{For instance\}}
\NormalTok{\textbackslash{}def\textbackslash{}separator\{~:~\}}
 
\NormalTok{\textbackslash{}example\{the separator changed, but not the first word.\}}
\end{Highlighting}
\end{Shaded}



\LaTeXTT{\textbackslash{}the} control sequence will let you see the content of various TeX types:
\begin{myitemize}
\item{}  catcodes
\item{}  chardef
\item{}  font parameters
\item{}  internal parameters
\item{}  lengths
\item{}  registers
\item{}  ...
\end{myitemize}


Example:
\begin{Shaded}
\begin{Highlighting}[]

\NormalTok{Text dimensions: $ \textbackslash{}the\textbackslash{}hsize \textbackslash{}times \textbackslash{}the\textbackslash{}vsize $}
\end{Highlighting}
\end{Shaded}

\section{Registers}
\label{866}

Registers are kind of typed variables. They are limited in numbers, ranging from 0 to 255.
There are 6 different types:
\begin{longtable}{|>{\RaggedRight}p{0.28512\linewidth}|>{\RaggedRight}p{0.65774\linewidth}|} \hline 
{\bfseries \hspace*{0pt}\ignorespaces{}\hspace*{0pt} Type}&{\bfseries \hspace*{0pt}\ignorespaces{}\hspace*{0pt} Description}\endhead  \hline \hspace*{0pt}\ignorespaces{}\hspace*{0pt} box    &\hspace*{0pt}\ignorespaces{}\hspace*{0pt} one box\\ \hline \hspace*{0pt}\ignorespaces{}\hspace*{0pt} count  &\hspace*{0pt}\ignorespaces{}\hspace*{0pt} an integer\\ \hline \hspace*{0pt}\ignorespaces{}\hspace*{0pt} dimen  &\hspace*{0pt}\ignorespaces{}\hspace*{0pt} a length\\ \hline \hspace*{0pt}\ignorespaces{}\hspace*{0pt} muskip &\hspace*{0pt}\ignorespaces{}\hspace*{0pt} a glue (in mu unit)\\ \hline \hspace*{0pt}\ignorespaces{}\hspace*{0pt} skip   &\hspace*{0pt}\ignorespaces{}\hspace*{0pt} a glue\\ \hline \hspace*{0pt}\ignorespaces{}\hspace*{0pt} toks   &\hspace*{0pt}\ignorespaces{}\hspace*{0pt} a sequence of tokens\\ \hline 
\end{longtable}


TeX uses some registers internally, so you would be better off not using them.

List of reserved registers:
\begin{myitemize}
\item{}  \textbackslash{}box255 is used for the contents of a page 
\item{}  \textbackslash{}count0-{}\textbackslash{}count9 are used for page numbering
\end{myitemize}


Scratch registers (freely available):
\begin{myitemize}
\item{}  \textbackslash{}box0-{}\textbackslash{}box254
\item{}  \textbackslash{}count255
\item{}  \textbackslash{}dimen0-{}\textbackslash{}dimen9
\item{}  \textbackslash{}muskip0-{}\textbackslash{}muskip9
\item{}  \textbackslash{}skip0-{}\textbackslash{}skip9
\end{myitemize}


Assign register using the \textquotesingle{}=\textquotesingle{} control character. For box registers, use the \LaTeXTT{\textbackslash{}setbox} command instead.
\begin{Shaded}
\begin{Highlighting}[]

\NormalTok{\textbackslash{}count255=17}
\NormalTok{\textbackslash{}setbox\textbackslash{}mybox=\textbackslash{}hbox\{blah\}}
\end{Highlighting}
\end{Shaded}


You may use one of the following reservation macro to prevent any clash:
\begin{Shaded}
\begin{Highlighting}[]

\NormalTok{\textbackslash{}newbox}
\NormalTok{\textbackslash{}newcount}
\NormalTok{\textbackslash{}newdimen}
\NormalTok{\textbackslash{}newmuskip}
\NormalTok{\textbackslash{}newskip}
\NormalTok{\textbackslash{}newtoks}
\end{Highlighting}
\end{Shaded}


These macros use the following syntax: \LaTeXTT{\textbackslash{}new*<{}csname>{}}.
Example:
\begin{Shaded}
\begin{Highlighting}[]

\NormalTok{\textbackslash{}newbox\textbackslash{}mybox}
\NormalTok{\textbackslash{}setbox\textbackslash{}mybox=\textbackslash{}hbox\{blah\}}
\end{Highlighting}
\end{Shaded}


These commands can not be used inside macros, otherwise every call to the macro would reserve another register.

You can print a register using the \LaTeXTT{\textbackslash{}the} command. For counters use the \LaTeXTT{\textbackslash{}number} command instead. For boxes use the \LaTeXTT{\textbackslash{}box} command.
\begin{Shaded}
\begin{Highlighting}[]

\NormalTok{\textbackslash{}the\textbackslash{}hsize}
\NormalTok{\textbackslash{}number\textbackslash{}count255}
\NormalTok{\textbackslash{}box\textbackslash{}mybox}
\end{Highlighting}
\end{Shaded}

\section{Arithmetic}
\label{867}

The arithmetic capabilities of TeX are very limited, although this base suffice to extend it to some interesting features.
The three main functions:
\begin{Shaded}
\begin{Highlighting}[]

\NormalTok{\textbackslash{}advance <register> by <number>}
\NormalTok{\textbackslash{}multiply <register> by <number>}
\NormalTok{\textbackslash{}divide <register> by <number>}
\end{Highlighting}
\end{Shaded}


\LaTeXTT{register} may be of type count, dimen, muskip or skip. It does not make sense for box nor toks.
\section{Conditionals}
\label{868}

The base syntax is
\begin{Shaded}
\begin{Highlighting}[]

\NormalTok{\textbackslash{}if* <test><true action>\textbackslash{}fi}
\NormalTok{\textbackslash{}if* <test><true action>\textbackslash{}else<false action>\textbackslash{}fi}
\end{Highlighting}
\end{Shaded}

where \LaTeXTT{\textbackslash{}if*} is one command among the following.
\begin{longtable}{|>{\RaggedRight}p{0.28673\linewidth}|>{\RaggedRight}p{0.65613\linewidth}|} \hline 
{\bfseries \hspace*{0pt}\ignorespaces{}\hspace*{0pt} Control sequence}&{\bfseries \hspace*{0pt}\ignorespaces{}\hspace*{0pt} Description}\endhead  \hline \hspace*{0pt}\ignorespaces{}\hspace*{0pt} \LaTeXTT{\textbackslash{}if <{}a>{}<{}b>{}}&\hspace*{0pt}\ignorespaces{}\hspace*{0pt} True if two character codes are equal.\\ \hline \hspace*{0pt}\ignorespaces{}\hspace*{0pt} \LaTeXTT{\textbackslash{}ifcat <{}a>{}<{}b>{}}&\hspace*{0pt}\ignorespaces{}\hspace*{0pt} True if two category codes are equal.\\ \hline \hspace*{0pt}\ignorespaces{}\hspace*{0pt} \LaTeXTT{\textbackslash{}ifdim <{}a>{}<{}rel>{}<{}b>{}}&\hspace*{0pt}\ignorespaces{}\hspace*{0pt} Dimension relation, either {\ttfamily \setmainfont[Path=/usr/share/fonts/truetype/cmu/,UprightFont=cmunrm.ttf,BoldFont=cmunbx.ttf,ItalicFont=cmunti.ttf,BoldItalicFont=cmunbi.ttf]{cmuntt.ttf}\setmonofont[Path=/usr/share/fonts/truetype/cmu/,UprightFont=cmuntt.ttf,BoldFont=cmuntb.ttf,ItalicFont=cmunit.ttf,BoldItalicFont=cmuntx.ttf]{cmuntt.ttf}\ttfamily <{}}\setmainfont[Path=/usr/share/fonts/truetype/cmu/,UprightFont=cmunrm.ttf,BoldFont=cmunbx.ttf,ItalicFont=cmunti.ttf,BoldItalicFont=cmunbi.ttf]{cmunrm.ttf}\setmonofont[Path=/usr/share/fonts/truetype/cmu/,UprightFont=cmuntt.ttf,BoldFont=cmuntb.ttf,ItalicFont=cmunit.ttf,BoldItalicFont=cmuntx.ttf]{cmunrm.ttf}, {\ttfamily \setmainfont[Path=/usr/share/fonts/truetype/cmu/,UprightFont=cmunrm.ttf,BoldFont=cmunbx.ttf,ItalicFont=cmunti.ttf,BoldItalicFont=cmunbi.ttf]{cmuntt.ttf}\setmonofont[Path=/usr/share/fonts/truetype/cmu/,UprightFont=cmuntt.ttf,BoldFont=cmuntb.ttf,ItalicFont=cmunit.ttf,BoldItalicFont=cmuntx.ttf]{cmuntt.ttf}\ttfamily >{}}{$\text{ }$}\setmainfont[Path=/usr/share/fonts/truetype/cmu/,UprightFont=cmunrm.ttf,BoldFont=cmunbx.ttf,ItalicFont=cmunti.ttf,BoldItalicFont=cmunbi.ttf]{cmunrm.ttf}\setmonofont[Path=/usr/share/fonts/truetype/cmu/,UprightFont=cmuntt.ttf,BoldFont=cmuntb.ttf,ItalicFont=cmunit.ttf,BoldItalicFont=cmuntx.ttf]{cmunrm.ttf} or {\ttfamily \setmainfont[Path=/usr/share/fonts/truetype/cmu/,UprightFont=cmunrm.ttf,BoldFont=cmunbx.ttf,ItalicFont=cmunti.ttf,BoldItalicFont=cmunbi.ttf]{cmuntt.ttf}\setmonofont[Path=/usr/share/fonts/truetype/cmu/,UprightFont=cmuntt.ttf,BoldFont=cmuntb.ttf,ItalicFont=cmunit.ttf,BoldItalicFont=cmuntx.ttf]{cmuntt.ttf}\ttfamily =}\setmainfont[Path=/usr/share/fonts/truetype/cmu/,UprightFont=cmunrm.ttf,BoldFont=cmunbx.ttf,ItalicFont=cmunti.ttf,BoldItalicFont=cmunbi.ttf]{cmunrm.ttf}\setmonofont[Path=/usr/share/fonts/truetype/cmu/,UprightFont=cmuntt.ttf,BoldFont=cmuntb.ttf,ItalicFont=cmunit.ttf,BoldItalicFont=cmuntx.ttf]{cmunrm.ttf}.\\ \hline \hspace*{0pt}\ignorespaces{}\hspace*{0pt} \LaTeXTT{\textbackslash{}ifeof}&\hspace*{0pt}\ignorespaces{}\hspace*{0pt} True if End-{}Of-{}File or non-{}existent file.\\ \hline \hspace*{0pt}\ignorespaces{}\hspace*{0pt} \LaTeXTT{\textbackslash{}iffalse}&\hspace*{0pt}\ignorespaces{}\hspace*{0pt} Always false.\\ \hline \hspace*{0pt}\ignorespaces{}\hspace*{0pt} \LaTeXTT{\textbackslash{}ifhbox <{}reg>{}}&\hspace*{0pt}\ignorespaces{}\hspace*{0pt} True if box register contains a horizontal box.\\ \hline \hspace*{0pt}\ignorespaces{}\hspace*{0pt} \LaTeXTT{\textbackslash{}ifhmode}&\hspace*{0pt}\ignorespaces{}\hspace*{0pt} True if in horizontal mode.\\ \hline \hspace*{0pt}\ignorespaces{}\hspace*{0pt} \LaTeXTT{\textbackslash{}ifinner}&\hspace*{0pt}\ignorespaces{}\hspace*{0pt} True if in internal mode.\\ \hline \hspace*{0pt}\ignorespaces{}\hspace*{0pt} \LaTeXTT{\textbackslash{}ifmmode}&\hspace*{0pt}\ignorespaces{}\hspace*{0pt} True if in math mode.\\ \hline \hspace*{0pt}\ignorespaces{}\hspace*{0pt} \LaTeXTT{\textbackslash{}ifnum <{}a>{}<{}rel>{}<{}b>{}}&\hspace*{0pt}\ignorespaces{}\hspace*{0pt} Number relation, either {\ttfamily \setmainfont[Path=/usr/share/fonts/truetype/cmu/,UprightFont=cmunrm.ttf,BoldFont=cmunbx.ttf,ItalicFont=cmunti.ttf,BoldItalicFont=cmunbi.ttf]{cmuntt.ttf}\setmonofont[Path=/usr/share/fonts/truetype/cmu/,UprightFont=cmuntt.ttf,BoldFont=cmuntb.ttf,ItalicFont=cmunit.ttf,BoldItalicFont=cmuntx.ttf]{cmuntt.ttf}\ttfamily <{}}\setmainfont[Path=/usr/share/fonts/truetype/cmu/,UprightFont=cmunrm.ttf,BoldFont=cmunbx.ttf,ItalicFont=cmunti.ttf,BoldItalicFont=cmunbi.ttf]{cmunrm.ttf}\setmonofont[Path=/usr/share/fonts/truetype/cmu/,UprightFont=cmuntt.ttf,BoldFont=cmuntb.ttf,ItalicFont=cmunit.ttf,BoldItalicFont=cmuntx.ttf]{cmunrm.ttf}, {\ttfamily \setmainfont[Path=/usr/share/fonts/truetype/cmu/,UprightFont=cmunrm.ttf,BoldFont=cmunbx.ttf,ItalicFont=cmunti.ttf,BoldItalicFont=cmunbi.ttf]{cmuntt.ttf}\setmonofont[Path=/usr/share/fonts/truetype/cmu/,UprightFont=cmuntt.ttf,BoldFont=cmuntb.ttf,ItalicFont=cmunit.ttf,BoldItalicFont=cmuntx.ttf]{cmuntt.ttf}\ttfamily >{}}{$\text{ }$}\setmainfont[Path=/usr/share/fonts/truetype/cmu/,UprightFont=cmunrm.ttf,BoldFont=cmunbx.ttf,ItalicFont=cmunti.ttf,BoldItalicFont=cmunbi.ttf]{cmunrm.ttf}\setmonofont[Path=/usr/share/fonts/truetype/cmu/,UprightFont=cmuntt.ttf,BoldFont=cmuntb.ttf,ItalicFont=cmunit.ttf,BoldItalicFont=cmuntx.ttf]{cmunrm.ttf} or {\ttfamily \setmainfont[Path=/usr/share/fonts/truetype/cmu/,UprightFont=cmunrm.ttf,BoldFont=cmunbx.ttf,ItalicFont=cmunti.ttf,BoldItalicFont=cmunbi.ttf]{cmuntt.ttf}\setmonofont[Path=/usr/share/fonts/truetype/cmu/,UprightFont=cmuntt.ttf,BoldFont=cmuntb.ttf,ItalicFont=cmunit.ttf,BoldItalicFont=cmuntx.ttf]{cmuntt.ttf}\ttfamily =}\setmainfont[Path=/usr/share/fonts/truetype/cmu/,UprightFont=cmunrm.ttf,BoldFont=cmunbx.ttf,ItalicFont=cmunti.ttf,BoldItalicFont=cmunbi.ttf]{cmunrm.ttf}\setmonofont[Path=/usr/share/fonts/truetype/cmu/,UprightFont=cmuntt.ttf,BoldFont=cmuntb.ttf,ItalicFont=cmunit.ttf,BoldItalicFont=cmuntx.ttf]{cmunrm.ttf}.\\ \hline \hspace*{0pt}\ignorespaces{}\hspace*{0pt} \LaTeXTT{\textbackslash{}ifodd <{}num>{}}&\hspace*{0pt}\ignorespaces{}\hspace*{0pt} True if number is odd.\\ \hline \hspace*{0pt}\ignorespaces{}\hspace*{0pt} \LaTeXTT{\textbackslash{}iftrue}&\hspace*{0pt}\ignorespaces{}\hspace*{0pt} Always true.\\ \hline \hspace*{0pt}\ignorespaces{}\hspace*{0pt} \LaTeXTT{\textbackslash{}ifvbox <{}reg>{}}&\hspace*{0pt}\ignorespaces{}\hspace*{0pt} True if box register contains a vertical box.\\ \hline \hspace*{0pt}\ignorespaces{}\hspace*{0pt} \LaTeXTT{\textbackslash{}ifvmode}&\hspace*{0pt}\ignorespaces{}\hspace*{0pt} True if in vertical mode.\\ \hline \hspace*{0pt}\ignorespaces{}\hspace*{0pt} \LaTeXTT{\textbackslash{}ifvoid <{}reg>{}}&\hspace*{0pt}\ignorespaces{}\hspace*{0pt} True if box register is empty.\\ \hline \hspace*{0pt}\ignorespaces{}\hspace*{0pt} \LaTeXTT{\textbackslash{}ifx <{}a>{}<{}b>{}}&\hspace*{0pt}\ignorespaces{}\hspace*{0pt} True if two macros expands to the same, or if two character codes are equal, or if two category codes are equal.\\ \hline 
\end{longtable}


Example:
\begin{longtable}{p{1.0\linewidth}}
\begin{Shaded}
\begin{Highlighting}[]

\NormalTok{\textbackslash{}ifnum 5>6}
\NormalTok{This is true}
\NormalTok{\textbackslash{}else}
\NormalTok{This is false}
\NormalTok{\textbackslash{}fi}
\end{Highlighting}
\end{Shaded}
\\

This is false

\end{longtable}

\subsection{Self defined conditionals}
\label{869}

You can create new conditionals (as a kind of {\itshape \setmainfont[Path=/usr/share/fonts/truetype/cmu/,UprightFont=cmunrm.ttf,BoldFont=cmunbx.ttf,ItalicFont=cmunti.ttf,BoldItalicFont=cmunbi.ttf]{cmunti.ttf}\setmonofont[Path=/usr/share/fonts/truetype/cmu/,UprightFont=cmuntt.ttf,BoldFont=cmuntb.ttf,ItalicFont=cmunit.ttf,BoldItalicFont=cmuntx.ttf]{cmunti.ttf}\itshape boolean variables}\setmainfont[Path=/usr/share/fonts/truetype/cmu/,UprightFont=cmunrm.ttf,BoldFont=cmunbx.ttf,ItalicFont=cmunti.ttf,BoldItalicFont=cmunbi.ttf]{cmunrm.ttf}\setmonofont[Path=/usr/share/fonts/truetype/cmu/,UprightFont=cmuntt.ttf,BoldFont=cmuntb.ttf,ItalicFont=cmunit.ttf,BoldItalicFont=cmuntx.ttf]{cmunrm.ttf}) with the \LaTeXTT{\textbackslash{}newif} command. With this self defined conditionals you can control the output of your code in an elegant way.
The best way to illustrate the use of conditionals is through an example.

Two versions of a document must be generated. One version for group A the other one for the rest of people (i.e. not belonging to group A):

1. We use \LaTeXTT{\textbackslash{}newif} to define our conditional (i.e. boolean variable).
\begin{Shaded}
\begin{Highlighting}[]
 \NormalTok{\textbackslash{}newif\textbackslash{}ifgroupA }
\end{Highlighting}
\end{Shaded}


2. In the following way we set a value (true or false) for our conditional 
\begin{Shaded}
\begin{Highlighting}[]
 
\NormalTok{\textbackslash{}groupAtrue }\CommentTok{% or}
\NormalTok{\textbackslash{}groupAfalse}
\end{Highlighting}
\end{Shaded}

that is: \begin{Shaded}
\begin{Highlighting}[]

\NormalTok{\textbackslash{}<conditionalsname>true}
\NormalTok{\textbackslash{}<conditionalsname>false}
\end{Highlighting}
\end{Shaded}
 depending on which value we want to set in our conditional.

3. Now we can use our conditional anywhere after in an {\itshape \setmainfont[Path=/usr/share/fonts/truetype/cmu/,UprightFont=cmunrm.ttf,BoldFont=cmunbx.ttf,ItalicFont=cmunti.ttf,BoldItalicFont=cmunbi.ttf]{cmunti.ttf}\setmonofont[Path=/usr/share/fonts/truetype/cmu/,UprightFont=cmuntt.ttf,BoldFont=cmuntb.ttf,ItalicFont=cmunit.ttf,BoldItalicFont=cmuntx.ttf]{cmunti.ttf}\itshape if control structure}\setmainfont[Path=/usr/share/fonts/truetype/cmu/,UprightFont=cmunrm.ttf,BoldFont=cmunbx.ttf,ItalicFont=cmunti.ttf,BoldItalicFont=cmunbi.ttf]{cmunrm.ttf}\setmonofont[Path=/usr/share/fonts/truetype/cmu/,UprightFont=cmuntt.ttf,BoldFont=cmuntb.ttf,ItalicFont=cmunit.ttf,BoldItalicFont=cmuntx.ttf]{cmunrm.ttf}.
\begin{Shaded}
\begin{Highlighting}[]

\NormalTok{\textbackslash{}ifgroupA}
  \CommentTok{% Here we write the code of the document that is}
  \CommentTok{% intended for the group A}
\NormalTok{\textbackslash{}else}
  \CommentTok{% Here we write the code of the document that is }
  \CommentTok{% intended for the rest of the people}
\NormalTok{\textbackslash{}fi}
\end{Highlighting}
\end{Shaded}


A full example is:
\begin{longtable}{p{1.0\linewidth}}
\begin{Shaded}
\begin{Highlighting}[]

\NormalTok{\textbackslash{}newif\textbackslash{}ifdirector }
 
\CommentTok{%I set the conditional to false}
\NormalTok{\textbackslash{}directorfalse}
 
\NormalTok{\textbackslash{}ifdirector}
 \NormalTok{I write something for the director.}
\NormalTok{\textbackslash{}else}
 \NormalTok{I write something for common people.}
\NormalTok{\textbackslash{}fi }
\end{Highlighting}
\end{Shaded}
\\
 I write something for common people.

\end{longtable}
\subsection{Case statement}
\label{870}

The syntax is \LaTeXTT{\textbackslash{}ifcase <{}number>{}<{}case0>{}\textbackslash{}or<{}case1>{}\textbackslash{}or...\textbackslash{}else<{}defaultcase>{}\textbackslash{}fi}. If \LaTeXTT{number} is equal to the case number, its content will be printed. Note that it starts at 0.

\begin{longtable}{p{1.0\linewidth}}
\begin{Shaded}
\begin{Highlighting}[]

\NormalTok{\textbackslash{}ifcase 2 a\textbackslash{}or b\textbackslash{}or c\textbackslash{}or d\textbackslash{}else e\textbackslash{}fi}
\end{Highlighting}
\end{Shaded}
\\

c

\end{longtable}

\LaTeXTT{\textbackslash{}else} is used to specify the default case (whenever none of the previous cases have matched).
\section{Loops}
\label{871}

The base syntax is
\begin{Shaded}
\begin{Highlighting}[]

\NormalTok{\textbackslash{}loop <content> \textbackslash{}if*<condition><true action>\textbackslash{}repeat}
\end{Highlighting}
\end{Shaded}

As always, \LaTeXTT{content} and \LaTeXTT{true action} are arbitrary TeX contents. \LaTeXTT{\textbackslash{}if*} refers to any of the \mylref{868}{conditionals}. Note that there is no \LaTeXTT{false action}, you cannot put an \LaTeXTT{\textbackslash{}else} between \LaTeXTT{\textbackslash{}if*} and \LaTeXTT{\textbackslash{}repeat}. In some case this will be the opposite of what you want; you have to change the condition or to define a new conditional using \LaTeXTT{\textbackslash{}newif}.
Example:

\begin{Shaded}
\begin{Highlighting}[]

\NormalTok{\textbackslash{}count255 = 1}
\NormalTok{\textbackslash{}loop}
  \NormalTok{\textbackslash{}TeX}
\NormalTok{\textbackslash{}ifnum\textbackslash{}count255 < 10}
\NormalTok{\textbackslash{}advance\textbackslash{}count255 by 1}
\NormalTok{\textbackslash{}repeat}
\end{Highlighting}
\end{Shaded}

The above code will print TeX ten times.
\section{Doing nothing}
\label{872}
Sometimes it may be useful to tell TeX that you want to do nothing.
There is two commands for that: \LaTeXTT{\textbackslash{}relax} and \LaTeXTT{\textbackslash{}empty}.

Classic example:
\begin{Shaded}
\begin{Highlighting}[]

\NormalTok{\textbackslash{}def\textbackslash{}myspace\{\textbackslash{}hskip 25pt\textbackslash{}relax\}}
\NormalTok{\textbackslash{}myspace\{\} plus 10pt}
\end{Highlighting}
\end{Shaded}

The \LaTeXTT{\textbackslash{}relax} prevents undesired behaviour if a \LaTeXTT{plus} or a \LaTeXTT{minus} is encounter after the command.

The difference between \LaTeXTT{\textbackslash{}empty} and  \LaTeXTT{\textbackslash{}relax} lies in the expansion: \LaTeXTT{\textbackslash{}empty} disappears after macro expansion.
\section{TeX characters}
\label{873}\subsection{{\itshape \setmainfont[Path=/usr/share/fonts/truetype/cmu/,UprightFont=cmunrm.ttf,BoldFont=cmunbx.ttf,ItalicFont=cmunti.ttf,BoldItalicFont=cmunbi.ttf]{cmunti.ttf}\setmonofont[Path=/usr/share/fonts/truetype/cmu/,UprightFont=cmuntt.ttf,BoldFont=cmuntb.ttf,ItalicFont=cmunit.ttf,BoldItalicFont=cmuntx.ttf]{cmunti.ttf}\itshape char}}
\label{874}\setmainfont[Path=/usr/share/fonts/truetype/cmu/,UprightFont=cmunrm.ttf,BoldFont=cmunbx.ttf,ItalicFont=cmunti.ttf,BoldItalicFont=cmunbi.ttf]{cmunrm.ttf}\setmonofont[Path=/usr/share/fonts/truetype/cmu/,UprightFont=cmuntt.ttf,BoldFont=cmuntb.ttf,ItalicFont=cmunit.ttf,BoldItalicFont=cmuntx.ttf]{cmunrm.ttf}

We can print all characters using the \LaTeXTT{\textbackslash{}char \{charcode\}} command. The charcode is actually the byte value.
For example
\begin{Shaded}
\begin{Highlighting}[]

\NormalTok{\textbackslash{}char65 = \textbackslash{}char `A = \textbackslash{}char `\textbackslash{}A}
\end{Highlighting}
\end{Shaded}


Most characters correspond to the ASCII value (e.g. A-{}Za-{}z), some replace the non-{}printable characters from ASCII.
\subsection{{\itshape \setmainfont[Path=/usr/share/fonts/truetype/cmu/,UprightFont=cmunrm.ttf,BoldFont=cmunbx.ttf,ItalicFont=cmunti.ttf,BoldItalicFont=cmunbi.ttf]{cmunti.ttf}\setmonofont[Path=/usr/share/fonts/truetype/cmu/,UprightFont=cmuntt.ttf,BoldFont=cmuntb.ttf,ItalicFont=cmunit.ttf,BoldItalicFont=cmuntx.ttf]{cmunti.ttf}\itshape chardef}{$\text{ }$}\setmainfont[Path=/usr/share/fonts/truetype/cmu/,UprightFont=cmunrm.ttf,BoldFont=cmunbx.ttf,ItalicFont=cmunti.ttf,BoldItalicFont=cmunbi.ttf]{cmunrm.ttf}\setmonofont[Path=/usr/share/fonts/truetype/cmu/,UprightFont=cmuntt.ttf,BoldFont=cmuntb.ttf,ItalicFont=cmunit.ttf,BoldItalicFont=cmuntx.ttf]{cmunrm.ttf} and {\itshape \setmainfont[Path=/usr/share/fonts/truetype/cmu/,UprightFont=cmunrm.ttf,BoldFont=cmunbx.ttf,ItalicFont=cmunti.ttf,BoldItalicFont=cmunbi.ttf]{cmunti.ttf}\setmonofont[Path=/usr/share/fonts/truetype/cmu/,UprightFont=cmuntt.ttf,BoldFont=cmuntb.ttf,ItalicFont=cmunit.ttf,BoldItalicFont=cmuntx.ttf]{cmunti.ttf}\itshape mathchardef}}
\label{875}\setmainfont[Path=/usr/share/fonts/truetype/cmu/,UprightFont=cmunrm.ttf,BoldFont=cmunbx.ttf,ItalicFont=cmunti.ttf,BoldItalicFont=cmunbi.ttf]{cmunrm.ttf}\setmonofont[Path=/usr/share/fonts/truetype/cmu/,UprightFont=cmuntt.ttf,BoldFont=cmuntb.ttf,ItalicFont=cmunit.ttf,BoldItalicFont=cmuntx.ttf]{cmunrm.ttf}
You can define control sequence to expand to a specific char. The syntax is \LaTeXTT{\textbackslash{}chardef<{}control sequence>{}=<{}charcode>{}}.
The following sequences do the same thing.
\begin{Shaded}
\begin{Highlighting}[]

\NormalTok{\textbackslash{}chardef\textbackslash{}myA=65}
\NormalTok{\textbackslash{}chardef\textbackslash{}myA=`A}
\NormalTok{\textbackslash{}chardef\textbackslash{}myA=`\textbackslash{}A}
\end{Highlighting}
\end{Shaded}


Example:
\begin{Shaded}
\begin{Highlighting}[]

\NormalTok{\textbackslash{}mathchardef\textbackslash{}alphachar = "010B}
\NormalTok{$\textbackslash{}alphachar$}
\end{Highlighting}
\end{Shaded}

\subsection{Font encoding map}
\label{876}

We can use the above primitive to print the font encoding map.
\begin{Shaded}
\begin{Highlighting}[]

\NormalTok{\textbackslash{}count255 = 0}
\NormalTok{\textbackslash{}loop}
  \NormalTok{[\textbackslash{}number\textbackslash{}count255 =\textbackslash{}char\textbackslash{}number\textbackslash{}count255]}
\NormalTok{\textbackslash{}ifnum\textbackslash{}count255 < 127}
\NormalTok{\textbackslash{}advance\textbackslash{}count255 by 1}
\NormalTok{\textbackslash{}repeat}
\end{Highlighting}
\end{Shaded}


Another version, with different fonts, one entry per line:
\begin{Shaded}
\begin{Highlighting}[]

\NormalTok{\textbackslash{}count255 = 0}
\NormalTok{\textbackslash{}loop}
  \NormalTok{[\textbackslash{}number\textbackslash{}count255 =}
    \NormalTok{\textbackslash{}char\textbackslash{}number\textbackslash{}count255 \textbackslash{} }
    \NormalTok{\{\textbackslash{}tt \textbackslash{}char\textbackslash{}number\textbackslash{}count255\}}
    \NormalTok{\{\textbackslash{}it \textbackslash{}char\textbackslash{}number\textbackslash{}count255\}}
  \NormalTok{]}
  \NormalTok{\textbackslash{}hfil\textbackslash{}break}
\NormalTok{\textbackslash{}ifnum\textbackslash{}count255 < 127}
\NormalTok{\textbackslash{}advance\textbackslash{}count255 by 1}
\NormalTok{\textbackslash{}repeat}
\end{Highlighting}
\end{Shaded}

\section{Verbatim lines and spaces}
\label{877}

It is rather confusing to discover (La)TeX treats all whitespace as the same type of spacing glue. Plain TeX provides some commands to preserve the spacing and newlines as you wrote it:
\begin{Shaded}
\begin{Highlighting}[]

\NormalTok{\textbackslash{}begingroup}
\NormalTok{\textbackslash{}obeylines}
\NormalTok{\textbackslash{}obeyspaces}
\NormalTok{Relevant text here}
\NormalTok{\textbackslash{}endgroup}
\end{Highlighting}
\end{Shaded}


which means that you will probably need to combine your own verbatim environment, and your command:

\begin{Shaded}
\begin{Highlighting}[]

\NormalTok{\textbackslash{}newenvironment\{myverbatim\}\{\textbackslash{}begingroup \textbackslash{}obeylines \textbackslash{}obeyspaces\}\{\textbackslash{}endgroup\}}
\NormalTok{\textbackslash{}newcommand\{\textbackslash{}mycommand\}[n]\{do something with #1 .. #n\}}
\end{Highlighting}
\end{Shaded}


and then in your tex file:
\begin{Shaded}
\begin{Highlighting}[]

\NormalTok{\textbackslash{}begin\{myverbatim\}}
\NormalTok{\textbackslash{}mycommand\{}
\NormalTok{whichever text it is important you}
\NormalTok{preserve the spacing and newslines}
\NormalTok{for, like when you want to generate}
\NormalTok{a verbatim block later on.}
\NormalTok{\}}
\NormalTok{\textbackslash{}end\{myverbatim\}}
\end{Highlighting}
\end{Shaded}

\section{Macros defining macros}
\label{878}

This is useful in some case, for example to define language commands as explained in \mylref{212}{Multilingual versions}, where the end user can write
\begin{Shaded}
\begin{Highlighting}[]

\NormalTok{\textbackslash{}en\{some english text\}}
\NormalTok{\textbackslash{}de\{etwas deutscher Text\}}
\end{Highlighting}
\end{Shaded}

and make sure it switches to the appropriate Babel language.

Let\textquotesingle{}s define a macros that will define language commands for instance. These commands are simple: if the argument is the value of the \LaTeXTT{\textbackslash{}locale} variable, then the corresponding macro prints its content directly. Otherwise, it does nothing.

Basicly, what we want to do is extremely simple: define a bunch of macros like this:
\begin{Shaded}
\begin{Highlighting}[]

\NormalTok{\textbackslash{}newcommand\{\textbackslash{}de\}[1]\{#1\}}
\NormalTok{\textbackslash{}newcommand\{\textbackslash{}en\}[1]\{\}}
\NormalTok{\textbackslash{}newcommand\{\textbackslash{}fr\}[1]\{\}}
\end{Highlighting}
\end{Shaded}


In the previous snippet of code, only the \LaTeXTT{\textbackslash{}de} command in going to output its content, \LaTeXTT{\textbackslash{}en} and \LaTeXTT{\textbackslash{}fr} will print nothing at all. That\textquotesingle{}s what we want. The problem arises when you want to automate the task, or if you have a lot of languages, and you want to change the language selection. You just have to move the \LaTeXTT{\#1}, but that\textquotesingle{}s not convenient and it makes it impossible to choose the Babel language from command line. Think this out...

What we are going to do is to define the language commands dynamically following the value of the \LaTeXTT{\textbackslash{}locale} variable (or any variable of your choice). Hence the use of the \LaTeXTT{\textbackslash{}equal} command from the \LaTeXTT{ifthen} package.

Since it is hardly possible to write it in LaTeX, we will use some Plain TeX.

\begin{Shaded}
\begin{Highlighting}[]

\NormalTok{\textbackslash{}def\textbackslash{}locale\{de\}}
 
\NormalTok{\textbackslash{}def\textbackslash{}localedef#1\{}
  \NormalTok{\textbackslash{}ifthenelse\{ \textbackslash{}equal\{\textbackslash{}locale\}\{#1\} \}\{}
    \CommentTok{%% Set the Babel language.}
    \CommentTok{%% Define the command to print the content.}
  \NormalTok{\}\{}
    \CommentTok{%% Define the command to print nothing.}
  \NormalTok{\}}
\NormalTok{\}}
\end{Highlighting}
\end{Shaded}


Another problem arises: how to define a command whose name is a variable? In most programming languages that\textquotesingle{}s not possible at all. What we could be tempted to write is

\begin{Shaded}
\begin{Highlighting}[]

\NormalTok{\textbackslash{}def\textbackslash{}#1 #1\{#1\}}
\end{Highlighting}
\end{Shaded}


It will fail for two reasons.

\begin{myenumerate}
\item{}  The two last \textquotesingle{}\#1\textquotesingle{} are supposed to refer to the arguments of the {\itshape \setmainfont[Path=/usr/share/fonts/truetype/cmu/,UprightFont=cmunrm.ttf,BoldFont=cmunbx.ttf,ItalicFont=cmunti.ttf,BoldItalicFont=cmunbi.ttf]{cmunti.ttf}\setmonofont[Path=/usr/share/fonts/truetype/cmu/,UprightFont=cmuntt.ttf,BoldFont=cmuntb.ttf,ItalicFont=cmunit.ttf,BoldItalicFont=cmuntx.ttf]{cmunti.ttf}\itshape new}{$\text{ }$}\setmainfont[Path=/usr/share/fonts/truetype/cmu/,UprightFont=cmunrm.ttf,BoldFont=cmunbx.ttf,ItalicFont=cmunti.ttf,BoldItalicFont=cmunbi.ttf]{cmunrm.ttf}\setmonofont[Path=/usr/share/fonts/truetype/cmu/,UprightFont=cmuntt.ttf,BoldFont=cmuntb.ttf,ItalicFont=cmunit.ttf,BoldItalicFont=cmuntx.ttf]{cmunrm.ttf} macro, but they get expanded to the \LaTeXTT{\textbackslash{}localedef} macro first argument because they are in the body of that macro.
\item{}  \LaTeXTT{\textbackslash{}\#1} gets expanded to two tokens: \textquotesingle{}\#\textquotesingle{} and \textquotesingle{}1\textquotesingle{}, and the \LaTeXTT{\textbackslash{}def} command will fail as it requires a valid control sequence name.
\end{myenumerate}


The solution to problem 1 is simple: use \textquotesingle{}\#\#1\textquotesingle{}, which will expand to \textquotesingle{}\#1\textquotesingle{} when the macro is executed.

For problem 2, it is a little bit tricky. It is possible to tell {\ttfamily \setmainfont[Path=/usr/share/fonts/truetype/cmu/,UprightFont=cmunrm.ttf,BoldFont=cmunbx.ttf,ItalicFont=cmunti.ttf,BoldItalicFont=cmunbi.ttf]{cmuntt.ttf}\setmonofont[Path=/usr/share/fonts/truetype/cmu/,UprightFont=cmuntt.ttf,BoldFont=cmuntb.ttf,ItalicFont=cmunit.ttf,BoldItalicFont=cmuntx.ttf]{cmuntt.ttf}\ttfamily tex}{$\text{ }$}\setmainfont[Path=/usr/share/fonts/truetype/cmu/,UprightFont=cmunrm.ttf,BoldFont=cmunbx.ttf,ItalicFont=cmunti.ttf,BoldItalicFont=cmunbi.ttf]{cmunrm.ttf}\setmonofont[Path=/usr/share/fonts/truetype/cmu/,UprightFont=cmuntt.ttf,BoldFont=cmuntb.ttf,ItalicFont=cmunit.ttf,BoldItalicFont=cmuntx.ttf]{cmunrm.ttf} that a specific token is a control sequence. This is what the \LaTeXTT{\textbackslash{}csname...\textbackslash{}endcsname} is used for. However

\begin{Shaded}
\begin{Highlighting}[]

\NormalTok{\textbackslash{}def\textbackslash{}csname#1\textbackslash{}endcsname ##1\{##1\}}
\end{Highlighting}
\end{Shaded}


will fail because it will redefine \LaTeXTT{\textbackslash{}csname} to \textquotesingle{}\#1\textquotesingle{}, which is not what we want, then {\ttfamily \setmainfont[Path=/usr/share/fonts/truetype/cmu/,UprightFont=cmunrm.ttf,BoldFont=cmunbx.ttf,ItalicFont=cmunti.ttf,BoldItalicFont=cmunbi.ttf]{cmuntt.ttf}\setmonofont[Path=/usr/share/fonts/truetype/cmu/,UprightFont=cmuntt.ttf,BoldFont=cmuntb.ttf,ItalicFont=cmunit.ttf,BoldItalicFont=cmuntx.ttf]{cmuntt.ttf}\ttfamily tex}{$\text{ }$}\setmainfont[Path=/usr/share/fonts/truetype/cmu/,UprightFont=cmunrm.ttf,BoldFont=cmunbx.ttf,ItalicFont=cmunti.ttf,BoldItalicFont=cmunbi.ttf]{cmunrm.ttf}\setmonofont[Path=/usr/share/fonts/truetype/cmu/,UprightFont=cmuntt.ttf,BoldFont=cmuntb.ttf,ItalicFont=cmunit.ttf,BoldItalicFont=cmuntx.ttf]{cmunrm.ttf} will encounter \LaTeXTT{\textbackslash{}endcsname}, which will result in an error.

We need to delay the expansion of \LaTeXTT{\textbackslash{}def}, {\itshape \setmainfont[Path=/usr/share/fonts/truetype/cmu/,UprightFont=cmunrm.ttf,BoldFont=cmunbx.ttf,ItalicFont=cmunti.ttf,BoldItalicFont=cmunbi.ttf]{cmunti.ttf}\setmonofont[Path=/usr/share/fonts/truetype/cmu/,UprightFont=cmuntt.ttf,BoldFont=cmuntb.ttf,ItalicFont=cmunit.ttf,BoldItalicFont=cmuntx.ttf]{cmunti.ttf}\itshape i.e.}{$\text{ }$}\setmainfont[Path=/usr/share/fonts/truetype/cmu/,UprightFont=cmunrm.ttf,BoldFont=cmunbx.ttf,ItalicFont=cmunti.ttf,BoldItalicFont=cmunbi.ttf]{cmunrm.ttf}\setmonofont[Path=/usr/share/fonts/truetype/cmu/,UprightFont=cmuntt.ttf,BoldFont=cmuntb.ttf,ItalicFont=cmunit.ttf,BoldItalicFont=cmuntx.ttf]{cmunrm.ttf} to tell {\ttfamily \setmainfont[Path=/usr/share/fonts/truetype/cmu/,UprightFont=cmunrm.ttf,BoldFont=cmunbx.ttf,ItalicFont=cmunti.ttf,BoldItalicFont=cmunbi.ttf]{cmuntt.ttf}\setmonofont[Path=/usr/share/fonts/truetype/cmu/,UprightFont=cmuntt.ttf,BoldFont=cmuntb.ttf,ItalicFont=cmunit.ttf,BoldItalicFont=cmuntx.ttf]{cmuntt.ttf}\ttfamily tex}{$\text{ }$}\setmainfont[Path=/usr/share/fonts/truetype/cmu/,UprightFont=cmunrm.ttf,BoldFont=cmunbx.ttf,ItalicFont=cmunti.ttf,BoldItalicFont=cmunbi.ttf]{cmunrm.ttf}\setmonofont[Path=/usr/share/fonts/truetype/cmu/,UprightFont=cmuntt.ttf,BoldFont=cmuntb.ttf,ItalicFont=cmunit.ttf,BoldItalicFont=cmuntx.ttf]{cmunrm.ttf} to expand the \LaTeXTT{\textbackslash{}csname} stuff first, then to apply \LaTeXTT{\textbackslash{}def} on it. There is a command for that: \LaTeXTT{\textbackslash{}expandafter\{token1\}\{token2\}}. It will expand \LaTeXTT{\{token2\}} before \LaTeXTT{\{token1\}}.

Finally if we want to set language from command line, we must be able to set the \LaTeXTT{\textbackslash{}locale} variable so that the one in the source code is the default value that can be overridden by the one in the command line. This can be done with \LaTeXTT{\textbackslash{}provdecommand}:

\begin{Shaded}
\begin{Highlighting}[]

\NormalTok{\textbackslash{}providecommand\textbackslash{}locale\{fr\}}
\end{Highlighting}
\end{Shaded}


The final code is
\begin{Shaded}
\begin{Highlighting}[]

\CommentTok{%% Required package.}
\NormalTok{\textbackslash{}usepackage\{ifthen\}}
 
\CommentTok{%% TeX function that generates the language commands.}
\NormalTok{\textbackslash{}def\textbackslash{}localedef#1#2\{}
  \NormalTok{\textbackslash{}ifthenelse\{ \textbackslash{}equal\{\textbackslash{}locale\}\{#1\} \}\{}
    \NormalTok{\textbackslash{}selectlanguage\{#2\}}
    \NormalTok{\textbackslash{}expandafter\textbackslash{}def\textbackslash{}csname#1\textbackslash{}endcsname ##1\{##1\}}
  \NormalTok{\}\{}
    \NormalTok{\textbackslash{}expandafter\textbackslash{}def\textbackslash{}csname#1\textbackslash{}endcsname ##1\{\}}
  \NormalTok{\}}
\NormalTok{\}}
 
\CommentTok{%% Selected language. Can be placed anywhere before the language commands.}
\NormalTok{\textbackslash{}providecommand\textbackslash{}locale\{fr\}}
 
\CommentTok{%% Language commands.}
\NormalTok{\textbackslash{}localedef\{de\}\{ngerman\}}
\NormalTok{\textbackslash{}localedef\{en\}\{english\}}
\NormalTok{\textbackslash{}localedef\{fr\}\{frenchb\}}
\CommentTok{%% ...}
\end{Highlighting}
\end{Shaded}


And you can compile with\\

\TemplateSpaceIndent{$\text{ }${}latex$\text{ }${}\textquotesingle{}\textbackslash{}providecommand\textbackslash{}locale\{en\}\textbackslash{}input\{mydocument.tex\}\textquotesingle{}}

\section{Notes and References}
\label{879}
\LaTeXNullTemplate{}
{\bfseries
\begin{mydescription}Further reading
\end{mydescription}
}

\begin{myitemize}
\item{}  \myhref{http://www.ctan.org/pkg/texbook}{The TeXbook}, {\itshape \setmainfont[Path=/usr/share/fonts/truetype/cmu/,UprightFont=cmunrm.ttf,BoldFont=cmunbx.ttf,ItalicFont=cmunti.ttf,BoldItalicFont=cmunbi.ttf]{cmunti.ttf}\setmonofont[Path=/usr/share/fonts/truetype/cmu/,UprightFont=cmuntt.ttf,BoldFont=cmuntb.ttf,ItalicFont=cmunit.ttf,BoldItalicFont=cmuntx.ttf]{cmunti.ttf}\itshape Donald Knuth}
\item{} {$\text{ }$}\setmainfont[Path=/usr/share/fonts/truetype/cmu/,UprightFont=cmunrm.ttf,BoldFont=cmunbx.ttf,ItalicFont=cmunti.ttf,BoldItalicFont=cmunbi.ttf]{cmunrm.ttf}\setmonofont[Path=/usr/share/fonts/truetype/cmu/,UprightFont=cmuntt.ttf,BoldFont=cmuntb.ttf,ItalicFont=cmunit.ttf,BoldItalicFont=cmuntx.ttf]{cmunrm.ttf} \myhref{http://www.ctan.org/pkg/texbytopic}{TeX by Topic}, {\itshape \setmainfont[Path=/usr/share/fonts/truetype/cmu/,UprightFont=cmunrm.ttf,BoldFont=cmunbx.ttf,ItalicFont=cmunti.ttf,BoldItalicFont=cmunbi.ttf]{cmunti.ttf}\setmonofont[Path=/usr/share/fonts/truetype/cmu/,UprightFont=cmuntt.ttf,BoldFont=cmuntb.ttf,ItalicFont=cmunit.ttf,BoldItalicFont=cmuntx.ttf]{cmunti.ttf}\itshape Victor Eijkhout}
\item{} {$\text{ }$}\setmainfont[Path=/usr/share/fonts/truetype/cmu/,UprightFont=cmunrm.ttf,BoldFont=cmunbx.ttf,ItalicFont=cmunti.ttf,BoldItalicFont=cmunbi.ttf]{cmunrm.ttf}\setmonofont[Path=/usr/share/fonts/truetype/cmu/,UprightFont=cmuntt.ttf,BoldFont=cmuntb.ttf,ItalicFont=cmunit.ttf,BoldItalicFont=cmuntx.ttf]{cmunrm.ttf} \myhref{http://www.ctan.org/pkg/impatient}{TeX for the Impatient}, {\itshape \setmainfont[Path=/usr/share/fonts/truetype/cmu/,UprightFont=cmunrm.ttf,BoldFont=cmunbx.ttf,ItalicFont=cmunti.ttf,BoldItalicFont=cmunbi.ttf]{cmunti.ttf}\setmonofont[Path=/usr/share/fonts/truetype/cmu/,UprightFont=cmuntt.ttf,BoldFont=cmuntb.ttf,ItalicFont=cmunit.ttf,BoldItalicFont=cmuntx.ttf]{cmunti.ttf}\itshape Paul W. Abrahams, Karl Berry and Kathryn A. Hargreaves}
\end{myitemize}
\setmainfont[Path=/usr/share/fonts/truetype/cmu/,UprightFont=cmunrm.ttf,BoldFont=cmunbx.ttf,ItalicFont=cmunti.ttf,BoldItalicFont=cmunbi.ttf]{cmunrm.ttf}\setmonofont[Path=/usr/share/fonts/truetype/cmu/,UprightFont=cmuntt.ttf,BoldFont=cmuntb.ttf,ItalicFont=cmunit.ttf,BoldItalicFont=cmuntx.ttf]{cmunrm.ttf}

\chapter{Creating Packages}

\myminitoc
\label{880}

\label{881}


If you define a lot of new environments and commands, the preamble of your document will get quite long. In this situation, it is a good idea to create a LaTeX package or class containing all your command and environment definitions. It can be made dynamic enough to fit to all your future documents.

Classes are {\ttfamily \setmainfont[Path=/usr/share/fonts/truetype/cmu/,UprightFont=cmunrm.ttf,BoldFont=cmunbx.ttf,ItalicFont=cmunti.ttf,BoldItalicFont=cmunbi.ttf]{cmuntt.ttf}\setmonofont[Path=/usr/share/fonts/truetype/cmu/,UprightFont=cmuntt.ttf,BoldFont=cmuntb.ttf,ItalicFont=cmunit.ttf,BoldItalicFont=cmuntx.ttf]{cmuntt.ttf}\ttfamily .cls}{$\text{ }$}\setmainfont[Path=/usr/share/fonts/truetype/cmu/,UprightFont=cmunrm.ttf,BoldFont=cmunbx.ttf,ItalicFont=cmunti.ttf,BoldItalicFont=cmunbi.ttf]{cmunrm.ttf}\setmonofont[Path=/usr/share/fonts/truetype/cmu/,UprightFont=cmuntt.ttf,BoldFont=cmuntb.ttf,ItalicFont=cmunit.ttf,BoldItalicFont=cmuntx.ttf]{cmunrm.ttf} files, packages are stored in {\ttfamily \setmainfont[Path=/usr/share/fonts/truetype/cmu/,UprightFont=cmunrm.ttf,BoldFont=cmunbx.ttf,ItalicFont=cmunti.ttf,BoldItalicFont=cmunbi.ttf]{cmuntt.ttf}\setmonofont[Path=/usr/share/fonts/truetype/cmu/,UprightFont=cmuntt.ttf,BoldFont=cmuntb.ttf,ItalicFont=cmunit.ttf,BoldItalicFont=cmuntx.ttf]{cmuntt.ttf}\ttfamily .sty}{$\text{ }$}\setmainfont[Path=/usr/share/fonts/truetype/cmu/,UprightFont=cmunrm.ttf,BoldFont=cmunbx.ttf,ItalicFont=cmunti.ttf,BoldItalicFont=cmunbi.ttf]{cmunrm.ttf}\setmonofont[Path=/usr/share/fonts/truetype/cmu/,UprightFont=cmuntt.ttf,BoldFont=cmuntb.ttf,ItalicFont=cmunit.ttf,BoldItalicFont=cmuntx.ttf]{cmunrm.ttf} files. They are very similar, the main difference being that you can load only one class per document.

After deciding to create an own package or class, you should think 
about which license the package/class has. A license is of great importance,
either to protect your file, or to make it available for others.

\section{{\ttfamily \setmainfont[Path=/usr/share/fonts/truetype/cmu/,UprightFont=cmunrm.ttf,BoldFont=cmunbx.ttf,ItalicFont=cmunti.ttf,BoldItalicFont=cmunbi.ttf]{cmuntt.ttf}\setmonofont[Path=/usr/share/fonts/truetype/cmu/,UprightFont=cmuntt.ttf,BoldFont=cmuntb.ttf,ItalicFont=cmunit.ttf,BoldItalicFont=cmuntx.ttf]{cmuntt.ttf}\ttfamily makeatletter}{$\text{ }$}\setmainfont[Path=/usr/share/fonts/truetype/cmu/,UprightFont=cmunrm.ttf,BoldFont=cmunbx.ttf,ItalicFont=cmunti.ttf,BoldItalicFont=cmunbi.ttf]{cmunrm.ttf}\setmonofont[Path=/usr/share/fonts/truetype/cmu/,UprightFont=cmuntt.ttf,BoldFont=cmuntb.ttf,ItalicFont=cmunit.ttf,BoldItalicFont=cmuntx.ttf]{cmunrm.ttf} and {\ttfamily \setmainfont[Path=/usr/share/fonts/truetype/cmu/,UprightFont=cmunrm.ttf,BoldFont=cmunbx.ttf,ItalicFont=cmunti.ttf,BoldItalicFont=cmunbi.ttf]{cmuntt.ttf}\setmonofont[Path=/usr/share/fonts/truetype/cmu/,UprightFont=cmuntt.ttf,BoldFont=cmuntb.ttf,ItalicFont=cmunit.ttf,BoldItalicFont=cmuntx.ttf]{cmuntt.ttf}\ttfamily makeatother}{$\text{ }$}\setmainfont[Path=/usr/share/fonts/truetype/cmu/,UprightFont=cmunrm.ttf,BoldFont=cmunbx.ttf,ItalicFont=cmunti.ttf,BoldItalicFont=cmunbi.ttf]{cmunrm.ttf}\setmonofont[Path=/usr/share/fonts/truetype/cmu/,UprightFont=cmuntt.ttf,BoldFont=cmuntb.ttf,ItalicFont=cmunit.ttf,BoldItalicFont=cmuntx.ttf]{cmunrm.ttf}}
\label{882}

By default, LaTeX will allow the use of the \textquotesingle{}@\textquotesingle{} characters for control sequences from within package and class files, but not from within an end-{}user document. This way it is possible to {\itshape \setmainfont[Path=/usr/share/fonts/truetype/cmu/,UprightFont=cmunrm.ttf,BoldFont=cmunbx.ttf,ItalicFont=cmunti.ttf,BoldItalicFont=cmunbi.ttf]{cmunti.ttf}\setmonofont[Path=/usr/share/fonts/truetype/cmu/,UprightFont=cmuntt.ttf,BoldFont=cmuntb.ttf,ItalicFont=cmunit.ttf,BoldItalicFont=cmuntx.ttf]{cmunti.ttf}\itshape protect}{$\text{ }$}\setmainfont[Path=/usr/share/fonts/truetype/cmu/,UprightFont=cmunrm.ttf,BoldFont=cmunbx.ttf,ItalicFont=cmunti.ttf,BoldItalicFont=cmunbi.ttf]{cmunrm.ttf}\setmonofont[Path=/usr/share/fonts/truetype/cmu/,UprightFont=cmuntt.ttf,BoldFont=cmuntb.ttf,ItalicFont=cmunit.ttf,BoldItalicFont=cmuntx.ttf]{cmunrm.ttf} commands, {\itshape \setmainfont[Path=/usr/share/fonts/truetype/cmu/,UprightFont=cmunrm.ttf,BoldFont=cmunbx.ttf,ItalicFont=cmunti.ttf,BoldItalicFont=cmunbi.ttf]{cmunti.ttf}\setmonofont[Path=/usr/share/fonts/truetype/cmu/,UprightFont=cmuntt.ttf,BoldFont=cmuntb.ttf,ItalicFont=cmunit.ttf,BoldItalicFont=cmuntx.ttf]{cmunti.ttf}\itshape i.e.}{$\text{ }$}\setmainfont[Path=/usr/share/fonts/truetype/cmu/,UprightFont=cmunrm.ttf,BoldFont=cmunbx.ttf,ItalicFont=cmunti.ttf,BoldItalicFont=cmunbi.ttf]{cmunrm.ttf}\setmonofont[Path=/usr/share/fonts/truetype/cmu/,UprightFont=cmuntt.ttf,BoldFont=cmuntb.ttf,ItalicFont=cmunit.ttf,BoldItalicFont=cmuntx.ttf]{cmunrm.ttf} to make them accessible from packages only.

However it is possible to override this security with the duo \LaTeXTT{\textbackslash{}makeatletter} and \LaTeXTT{\textbackslash{}makeatother}. These commands only make sense in a regular document, they are not needed in package or class files.

\begin{Shaded}
\begin{Highlighting}[]

\NormalTok{\textbackslash{}documentclass\{...\}}
\CommentTok{%...}
 
\NormalTok{\textbackslash{}begin\{document\}}
 
\NormalTok{\textbackslash{}makeatletter}
\NormalTok{\textbackslash{}@author}
\NormalTok{\textbackslash{}makeatother}
 
\NormalTok{\textbackslash{}end\{document\}}
\end{Highlighting}
\end{Shaded}

\section{Creating your own package}
\label{883}

Your package can be made available in your document just like any other package: using the \LaTeXTT{\textbackslash{}usepackage} command. Writing a package basically consists of copying the contents of your document preamble into a separate file with a name ending in {\ttfamily \setmainfont[Path=/usr/share/fonts/truetype/cmu/,UprightFont=cmunrm.ttf,BoldFont=cmunbx.ttf,ItalicFont=cmunti.ttf,BoldItalicFont=cmunbi.ttf]{cmuntt.ttf}\setmonofont[Path=/usr/share/fonts/truetype/cmu/,UprightFont=cmuntt.ttf,BoldFont=cmuntb.ttf,ItalicFont=cmunit.ttf,BoldItalicFont=cmuntx.ttf]{cmuntt.ttf}\ttfamily .sty}\setmainfont[Path=/usr/share/fonts/truetype/cmu/,UprightFont=cmunrm.ttf,BoldFont=cmunbx.ttf,ItalicFont=cmunti.ttf,BoldItalicFont=cmunbi.ttf]{cmunrm.ttf}\setmonofont[Path=/usr/share/fonts/truetype/cmu/,UprightFont=cmuntt.ttf,BoldFont=cmuntb.ttf,ItalicFont=cmunit.ttf,BoldItalicFont=cmuntx.ttf]{cmunrm.ttf}.

Let\textquotesingle{}s write a first {\ttfamily \setmainfont[Path=/usr/share/fonts/truetype/cmu/,UprightFont=cmunrm.ttf,BoldFont=cmunbx.ttf,ItalicFont=cmunti.ttf,BoldItalicFont=cmunbi.ttf]{cmuntt.ttf}\setmonofont[Path=/usr/share/fonts/truetype/cmu/,UprightFont=cmuntt.ttf,BoldFont=cmuntb.ttf,ItalicFont=cmunit.ttf,BoldItalicFont=cmuntx.ttf]{cmuntt.ttf}\ttfamily custom.sty}{$\text{ }$}\setmainfont[Path=/usr/share/fonts/truetype/cmu/,UprightFont=cmunrm.ttf,BoldFont=cmunbx.ttf,ItalicFont=cmunti.ttf,BoldItalicFont=cmunbi.ttf]{cmunrm.ttf}\setmonofont[Path=/usr/share/fonts/truetype/cmu/,UprightFont=cmuntt.ttf,BoldFont=cmuntb.ttf,ItalicFont=cmunit.ttf,BoldItalicFont=cmuntx.ttf]{cmunrm.ttf} file as an example package:
\begin{Shaded}
\begin{Highlighting}[]

\NormalTok{\textbackslash{}NeedsTeXFormat\{LaTeX2e\}[1994/06/01]}
\NormalTok{\textbackslash{}ProvidesPackage\{custom\}[2013/01/13 Custom Package]}
 
\NormalTok{\textbackslash{}RequirePackage\{lmodern\}}
 
\CommentTok{%% 'sans serif' option}
\NormalTok{\textbackslash{}DeclareOption\{sans\}\{}
  \NormalTok{\textbackslash{}renewcommand\{\textbackslash{}familydefault\}\{\textbackslash{}sfdefault\}}
\NormalTok{\}}
 
\CommentTok{%% 'roman' option}
\NormalTok{\textbackslash{}DeclareOption\{roman\}\{}
  \NormalTok{\textbackslash{}renewcommand\{\textbackslash{}familydefault\}\{\textbackslash{}rmdefault\}}
\NormalTok{\}}
 
\CommentTok{%% Global indentation option}
\NormalTok{\textbackslash{}newif\textbackslash{}if@neverindent\textbackslash{}@neverindentfalse}
\NormalTok{\textbackslash{}DeclareOption\{neverindent\}\{}
  \NormalTok{\textbackslash{}@neverindenttrue}
\NormalTok{\}}
 
\NormalTok{\textbackslash{}ExecuteOptions\{roman\}}
 
\NormalTok{\textbackslash{}ProcessOptions\textbackslash{}relax}
 
\CommentTok{%% Traditional LaTeX or TeX follows...}
\CommentTok{% ...}
 
\NormalTok{\textbackslash{}newlength\{\textbackslash{}pardefault\}}
\NormalTok{\textbackslash{}setlength\{\textbackslash{}pardefault\}\{\textbackslash{}parindent\}}
\NormalTok{\textbackslash{}newcommand\{\textbackslash{}neverindent\}\{ \textbackslash{}setlength\{\textbackslash{}parindent\}\{0pt\} \}}
\NormalTok{\textbackslash{}newcommand\{\textbackslash{}autoindent\}\{ \textbackslash{}setlength\{\textbackslash{}parindent\}\{\textbackslash{}pardefault\} \}}
 
\NormalTok{\textbackslash{}if@neverindent}
\NormalTok{\textbackslash{}neverindent}
\NormalTok{\textbackslash{}fi}
 
\CommentTok{% ...}
 
\NormalTok{\textbackslash{}endinput}
\end{Highlighting}
\end{Shaded}


\begin{myitemize}
\item{}  \LaTeXTT{\textbackslash{}NeedsTeXFormat\{...\}} specifies  which version of TeX or LaTeX is required  at least to run your package. The optional date may be used to specify the version more precisely.
\end{myitemize}


\begin{myitemize}
\item{}  \LaTeXTT{\textbackslash{}ProvidesPackage\{<{}name>{}\}{$\text{[}$}<{}version>{}{$\text{]}$}} A package introduces itself using this command. {\itshape \setmainfont[Path=/usr/share/fonts/truetype/cmu/,UprightFont=cmunrm.ttf,BoldFont=cmunbx.ttf,ItalicFont=cmunti.ttf,BoldItalicFont=cmunbi.ttf]{cmunti.ttf}\setmonofont[Path=/usr/share/fonts/truetype/cmu/,UprightFont=cmuntt.ttf,BoldFont=cmuntb.ttf,ItalicFont=cmunit.ttf,BoldItalicFont=cmuntx.ttf]{cmunti.ttf}\itshape <{}name>{}}{$\text{ }$}\setmainfont[Path=/usr/share/fonts/truetype/cmu/,UprightFont=cmunrm.ttf,BoldFont=cmunbx.ttf,ItalicFont=cmunti.ttf,BoldItalicFont=cmunbi.ttf]{cmunrm.ttf}\setmonofont[Path=/usr/share/fonts/truetype/cmu/,UprightFont=cmuntt.ttf,BoldFont=cmuntb.ttf,ItalicFont=cmunit.ttf,BoldItalicFont=cmuntx.ttf]{cmunrm.ttf} should be identical to the {\itshape \setmainfont[Path=/usr/share/fonts/truetype/cmu/,UprightFont=cmunrm.ttf,BoldFont=cmunbx.ttf,ItalicFont=cmunti.ttf,BoldItalicFont=cmunbi.ttf]{cmunti.ttf}\setmonofont[Path=/usr/share/fonts/truetype/cmu/,UprightFont=cmuntt.ttf,BoldFont=cmuntb.ttf,ItalicFont=cmunit.ttf,BoldItalicFont=cmuntx.ttf]{cmunti.ttf}\itshape basename}{$\text{ }$}\setmainfont[Path=/usr/share/fonts/truetype/cmu/,UprightFont=cmunrm.ttf,BoldFont=cmunbx.ttf,ItalicFont=cmunti.ttf,BoldItalicFont=cmunbi.ttf]{cmunrm.ttf}\setmonofont[Path=/usr/share/fonts/truetype/cmu/,UprightFont=cmuntt.ttf,BoldFont=cmuntb.ttf,ItalicFont=cmunit.ttf,BoldItalicFont=cmuntx.ttf]{cmunrm.ttf} of the file itself. The {\itshape \setmainfont[Path=/usr/share/fonts/truetype/cmu/,UprightFont=cmunrm.ttf,BoldFont=cmunbx.ttf,ItalicFont=cmunti.ttf,BoldItalicFont=cmunbi.ttf]{cmunti.ttf}\setmonofont[Path=/usr/share/fonts/truetype/cmu/,UprightFont=cmuntt.ttf,BoldFont=cmuntb.ttf,ItalicFont=cmunit.ttf,BoldItalicFont=cmuntx.ttf]{cmunti.ttf}\itshape <{}version>{}}{$\text{ }$}\setmainfont[Path=/usr/share/fonts/truetype/cmu/,UprightFont=cmunrm.ttf,BoldFont=cmunbx.ttf,ItalicFont=cmunti.ttf,BoldItalicFont=cmunbi.ttf]{cmunrm.ttf}\setmonofont[Path=/usr/share/fonts/truetype/cmu/,UprightFont=cmuntt.ttf,BoldFont=cmuntb.ttf,ItalicFont=cmunit.ttf,BoldItalicFont=cmuntx.ttf]{cmunrm.ttf} should begin with a date in the format {\itshape \setmainfont[Path=/usr/share/fonts/truetype/cmu/,UprightFont=cmunrm.ttf,BoldFont=cmunbx.ttf,ItalicFont=cmunti.ttf,BoldItalicFont=cmunbi.ttf]{cmunti.ttf}\setmonofont[Path=/usr/share/fonts/truetype/cmu/,UprightFont=cmuntt.ttf,BoldFont=cmuntb.ttf,ItalicFont=cmunit.ttf,BoldItalicFont=cmuntx.ttf]{cmunti.ttf}\itshape YYYY/MM/DD}\setmainfont[Path=/usr/share/fonts/truetype/cmu/,UprightFont=cmunrm.ttf,BoldFont=cmunbx.ttf,ItalicFont=cmunti.ttf,BoldItalicFont=cmunbi.ttf]{cmunrm.ttf}\setmonofont[Path=/usr/share/fonts/truetype/cmu/,UprightFont=cmuntt.ttf,BoldFont=cmuntb.ttf,ItalicFont=cmunit.ttf,BoldItalicFont=cmuntx.ttf]{cmunrm.ttf}. Version information should be kept updated while developing a package.
\item{}  Next you may write some TeX or LaTeX code like loading package, but write only the bare minimum needed for the package options set below.
\item{}  \LaTeXTT{\textbackslash{}RequirePackage} is equivalent to \LaTeXTT{\textbackslash{}usepackage}.
\item{}  \LaTeXTT{\textbackslash{}DeclareOptions} are end-{}user parameters. Each option is declared by one such command.
\item{}  \LaTeXTT{\textbackslash{}ExecuteOptions\{...\}} tells which are the default.
\item{}  \LaTeXTT{\textbackslash{}ProcessOptions\textbackslash{}relax} terminates the option processing.
\item{}  Write whatever you want in it using all the LaTeX commands you know. Normally you should define new commands or import other packages.
\item{}  \LaTeXTT{\textbackslash{}endinput}: this {\itshape \setmainfont[Path=/usr/share/fonts/truetype/cmu/,UprightFont=cmunrm.ttf,BoldFont=cmunbx.ttf,ItalicFont=cmunti.ttf,BoldItalicFont=cmunbi.ttf]{cmunti.ttf}\setmonofont[Path=/usr/share/fonts/truetype/cmu/,UprightFont=cmuntt.ttf,BoldFont=cmuntb.ttf,ItalicFont=cmunit.ttf,BoldItalicFont=cmuntx.ttf]{cmunti.ttf}\itshape must}{$\text{ }$}\setmainfont[Path=/usr/share/fonts/truetype/cmu/,UprightFont=cmunrm.ttf,BoldFont=cmunbx.ttf,ItalicFont=cmunti.ttf,BoldItalicFont=cmunbi.ttf]{cmunrm.ttf}\setmonofont[Path=/usr/share/fonts/truetype/cmu/,UprightFont=cmuntt.ttf,BoldFont=cmuntb.ttf,ItalicFont=cmunit.ttf,BoldItalicFont=cmuntx.ttf]{cmunrm.ttf} be the last command.
\end{myitemize}


Once your package is ready, we can use it in any document. Import your new package with the known command \LaTeXTT{\textbackslash{}usepackage\{mypack\}}.
The file {\ttfamily \setmainfont[Path=/usr/share/fonts/truetype/cmu/,UprightFont=cmunrm.ttf,BoldFont=cmunbx.ttf,ItalicFont=cmunti.ttf,BoldItalicFont=cmunbi.ttf]{cmuntt.ttf}\setmonofont[Path=/usr/share/fonts/truetype/cmu/,UprightFont=cmuntt.ttf,BoldFont=cmuntb.ttf,ItalicFont=cmunit.ttf,BoldItalicFont=cmuntx.ttf]{cmuntt.ttf}\ttfamily custom.sty}{$\text{ }$}\setmainfont[Path=/usr/share/fonts/truetype/cmu/,UprightFont=cmunrm.ttf,BoldFont=cmunbx.ttf,ItalicFont=cmunti.ttf,BoldItalicFont=cmunbi.ttf]{cmunrm.ttf}\setmonofont[Path=/usr/share/fonts/truetype/cmu/,UprightFont=cmuntt.ttf,BoldFont=cmuntb.ttf,ItalicFont=cmunit.ttf,BoldItalicFont=cmuntx.ttf]{cmunrm.ttf} and the LaTeX source you are compiling must be in the same directory.

\begin{Shaded}
\begin{Highlighting}[]

\NormalTok{\textbackslash{}documentclass\{...\}}
\NormalTok{\textbackslash{}usepackage[neverindent,sans]\{custom\}}
\CommentTok{%...}
 
\NormalTok{\textbackslash{}begin\{document\}}
 
\NormalTok{Blah...}
 
\NormalTok{\textbackslash{}end\{document\}}
\end{Highlighting}
\end{Shaded}



For a more convenient use, it is possible to place the package within {\ttfamily \setmainfont[Path=/usr/share/fonts/truetype/cmu/,UprightFont=cmunrm.ttf,BoldFont=cmunbx.ttf,ItalicFont=cmunti.ttf,BoldItalicFont=cmunbi.ttf]{cmuntt.ttf}\setmonofont[Path=/usr/share/fonts/truetype/cmu/,UprightFont=cmuntt.ttf,BoldFont=cmuntb.ttf,ItalicFont=cmunit.ttf,BoldItalicFont=cmuntx.ttf]{cmuntt.ttf}\ttfamily \${}TEXMFHOME}{$\text{ }$}\setmainfont[Path=/usr/share/fonts/truetype/cmu/,UprightFont=cmunrm.ttf,BoldFont=cmunbx.ttf,ItalicFont=cmunti.ttf,BoldItalicFont=cmunbi.ttf]{cmunrm.ttf}\setmonofont[Path=/usr/share/fonts/truetype/cmu/,UprightFont=cmuntt.ttf,BoldFont=cmuntb.ttf,ItalicFont=cmunit.ttf,BoldItalicFont=cmuntx.ttf]{cmunrm.ttf} (which is {\ttfamily \setmainfont[Path=/usr/share/fonts/truetype/cmu/,UprightFont=cmunrm.ttf,BoldFont=cmunbx.ttf,ItalicFont=cmunti.ttf,BoldItalicFont=cmunbi.ttf]{cmuntt.ttf}\setmonofont[Path=/usr/share/fonts/truetype/cmu/,UprightFont=cmuntt.ttf,BoldFont=cmuntb.ttf,ItalicFont=cmunit.ttf,BoldItalicFont=cmuntx.ttf]{cmuntt.ttf}\ttfamily \~{}/texmf}{$\text{ }$}\setmainfont[Path=/usr/share/fonts/truetype/cmu/,UprightFont=cmunrm.ttf,BoldFont=cmunbx.ttf,ItalicFont=cmunti.ttf,BoldItalicFont=cmunbi.ttf]{cmunrm.ttf}\setmonofont[Path=/usr/share/fonts/truetype/cmu/,UprightFont=cmuntt.ttf,BoldFont=cmuntb.ttf,ItalicFont=cmunit.ttf,BoldItalicFont=cmuntx.ttf]{cmunrm.ttf} by default) according to the TeX Directory Structure (TDS). That would be
\\

\TemplateSpaceIndent{$\text{ }${}\${}TEXMFHOME/tex/latex/custom/custom.sty}


On Windows \textquotesingle{}\~{}\textquotesingle{} is often {\ttfamily \setmainfont[Path=/usr/share/fonts/truetype/cmu/,UprightFont=cmunrm.ttf,BoldFont=cmunbx.ttf,ItalicFont=cmunti.ttf,BoldItalicFont=cmunbi.ttf]{cmuntt.ttf}\setmonofont[Path=/usr/share/fonts/truetype/cmu/,UprightFont=cmuntt.ttf,BoldFont=cmuntb.ttf,ItalicFont=cmunit.ttf,BoldItalicFont=cmuntx.ttf]{cmuntt.ttf}\ttfamily C:\textbackslash{}Users\textbackslash{}username}\setmainfont[Path=/usr/share/fonts/truetype/cmu/,UprightFont=cmunrm.ttf,BoldFont=cmunbx.ttf,ItalicFont=cmunti.ttf,BoldItalicFont=cmunbi.ttf]{cmunrm.ttf}\setmonofont[Path=/usr/share/fonts/truetype/cmu/,UprightFont=cmuntt.ttf,BoldFont=cmuntb.ttf,ItalicFont=cmunit.ttf,BoldItalicFont=cmuntx.ttf]{cmunrm.ttf}.

You may have to run {\ttfamily \setmainfont[Path=/usr/share/fonts/truetype/cmu/,UprightFont=cmunrm.ttf,BoldFont=cmunbx.ttf,ItalicFont=cmunti.ttf,BoldItalicFont=cmunbi.ttf]{cmuntt.ttf}\setmonofont[Path=/usr/share/fonts/truetype/cmu/,UprightFont=cmuntt.ttf,BoldFont=cmuntb.ttf,ItalicFont=cmunit.ttf,BoldItalicFont=cmuntx.ttf]{cmuntt.ttf}\ttfamily texhash}{$\text{ }$}\setmainfont[Path=/usr/share/fonts/truetype/cmu/,UprightFont=cmunrm.ttf,BoldFont=cmunbx.ttf,ItalicFont=cmunti.ttf,BoldItalicFont=cmunbi.ttf]{cmunrm.ttf}\setmonofont[Path=/usr/share/fonts/truetype/cmu/,UprightFont=cmuntt.ttf,BoldFont=cmuntb.ttf,ItalicFont=cmunit.ttf,BoldItalicFont=cmuntx.ttf]{cmunrm.ttf} (or equivalent) to make your TeX distribution index the new file, thus making it available for use for any document. It will allow you to use your package as detailed above, but without it needing to be in the same directory as your document.
\section{Creating your own class}
\label{884}

It is also possible to create your own class file. The process is similar to the creation of your own package, you can call your own style file in the preamble of any document by the command:
\begin{Shaded}
\begin{Highlighting}[]

\NormalTok{\textbackslash{}documentclass\{myclass\}}
\end{Highlighting}
\end{Shaded}


The name of the class file is then {\ttfamily \setmainfont[Path=/usr/share/fonts/truetype/cmu/,UprightFont=cmunrm.ttf,BoldFont=cmunbx.ttf,ItalicFont=cmunti.ttf,BoldItalicFont=cmunbi.ttf]{cmuntt.ttf}\setmonofont[Path=/usr/share/fonts/truetype/cmu/,UprightFont=cmuntt.ttf,BoldFont=cmuntb.ttf,ItalicFont=cmunit.ttf,BoldItalicFont=cmuntx.ttf]{cmuntt.ttf}\ttfamily myclass.cls}\setmainfont[Path=/usr/share/fonts/truetype/cmu/,UprightFont=cmunrm.ttf,BoldFont=cmunbx.ttf,ItalicFont=cmunti.ttf,BoldItalicFont=cmunbi.ttf]{cmunrm.ttf}\setmonofont[Path=/usr/share/fonts/truetype/cmu/,UprightFont=cmuntt.ttf,BoldFont=cmuntb.ttf,ItalicFont=cmunit.ttf,BoldItalicFont=cmuntx.ttf]{cmunrm.ttf}. Let\textquotesingle{}s write a simple example:

\begin{Shaded}
\begin{Highlighting}[]

\NormalTok{\textbackslash{}NeedsTeXFormat\{LaTeX2e\}}
\NormalTok{\textbackslash{}ProvidesClass\{myclass\}[2011/12/23 My Class]}
 
\CommentTok{%% Article options}
\NormalTok{\textbackslash{}DeclareOption\{10pt\}\{}
  \NormalTok{\textbackslash{}PassOptionsToClass\{\textbackslash{}CurrentOption\}\{article\}}
\NormalTok{\}}
 
\CommentTok{%% Custom package options}
\NormalTok{\textbackslash{}DeclareOption\{sansserif\}\{}
  \NormalTok{\textbackslash{}PassOptionsToPackage\{\textbackslash{}CurrentOption\}\{paxcommands\}}
\NormalTok{\}}
\NormalTok{\textbackslash{}DeclareOption\{neverindent\}\{}
  \NormalTok{\textbackslash{}PassOptionsToPackage\{\textbackslash{}CurrentOption\}\{paxcommands\}}
\NormalTok{\}}
 
\CommentTok{%% Fallback}
\NormalTok{\textbackslash{}DeclareOption*\{}
  \NormalTok{\textbackslash{}ClassWarning\{myclass\}\{Unknown option '\textbackslash{}CurrentOption'\}}
\NormalTok{\}}
 
\CommentTok{%% Execute default options}
\NormalTok{\textbackslash{}ExecuteOptions\{10pt\}}
 
\CommentTok{%% Process given options}
\NormalTok{\textbackslash{}ProcessOptions\textbackslash{}relax}
 
\CommentTok{%% Load base}
\NormalTok{\textbackslash{}LoadClass[a4paper]\{article\}}
 
\CommentTok{%% Load additional packages and commands.}
\NormalTok{\textbackslash{}RequirePackage\{custom\}}
 
\CommentTok{%% Additional TeX/LaTeX code...}
 
\NormalTok{\textbackslash{}endinput}
\end{Highlighting}
\end{Shaded}



\begin{myitemize}
\item{}  \LaTeXTT{\textbackslash{}ProvidesClass} is the counterpart of \LaTeXTT{\textbackslash{}ProvidesPackage}.
\item{}  \LaTeXTT{\textbackslash{}PassOptionsToClass} and \LaTeXTT{\textbackslash{}PassOptionsToPackage} are used to automatically invoke the corresponding options when the class or the package is loaded.
\item{}  \LaTeXTT{\textbackslash{}DeclareOption*}: the starred version lets you handle non-{}implemented options.
\item{}  \LaTeXTT{\textbackslash{}ClassWarning} will show the corresponding message in the TeX compiler output.
\item{}  \LaTeXTT{\textbackslash{}LoadClass} specifies the unique parent class, if any.
\end{myitemize}

\section{Hooks}
\label{885}

There are also hooks for classes and packages.
\begin{myitemize}
\item{}  \LaTeXTT{\textbackslash{}AtEndOfPackage}
\item{}  \LaTeXTT{\textbackslash{}AtEndOfClass}
\end{myitemize}


They behave as the document hooks. See \mylref{848}{LaTeX Hooks}.

\chapter{Themes}

\myminitoc
\label{886}

\label{887}


Newcomers to LaTeX often feel disappointed by the lack of visual customization offered by the system.
Actually this is done on purpose: the LaTeX philosophy takes a point at doing the formatting while the writer focuses on the content.

In this chapter, we will show what we can achieve with some efforts.
\section{Introduction}
\label{888}

In the following we will write the theme, a package that will only change the appearance of the document, so that our document will work with or without the theme.

Note that if it may look eye-{}candy, this is absolutely not a model of typography. You should not use such theme for serious publications. This is more a technogical example to exhibit LaTeX capabilities.

\begin{longtable}{>{\RaggedRight}p{0.5\linewidth}}  


\begin{minipage}{1.0\linewidth}
\begin{center}
\includegraphics[width=1.0\linewidth,height=6.5in,keepaspectratio]{../images/210.\SVGExtension}
\end{center}
\raggedright{}\myfigurewithcaption{210}{Custom theme (TOC)}
\end{minipage}\vspace{0.75cm}

\\ 


\begin{minipage}{1.0\linewidth}
\begin{center}
\includegraphics[width=1.0\linewidth,height=6.5in,keepaspectratio]{../images/211.\SVGExtension}
\end{center}
\raggedright{}\myfigurewithcaption{211}{Custom theme}
\end{minipage}\vspace{0.75cm}

\\ 


\begin{minipage}{1.0\linewidth}
\begin{center}
\includegraphics[width=1.0\linewidth,height=6.5in,keepaspectratio]{../images/212.\SVGExtension}
\end{center}
\raggedright{}\myfigurewithcaption{212}{Custom theme (red)}
\end{minipage}\vspace{0.75cm}

\\ 
\end{longtable}
\section{Package configuration}
\label{889}

Nothing much to say here. This is a direct application of the \mylref{881}{Creating Packages} chapter.

We load the required packages.
\begin{myitemize}
\item{}  \LaTeXTT{needspace} is used to prevent page break right after a sectioning command.
\item{}  \LaTeXTT{tikz} is used to draw the fancy material.
\end{myitemize}


We define a color option, you may use as much as you want. Defining colors with specific names makes it very flexible. We also use an option to toggle the fancy reflection effect which might be a little {\itshape \setmainfont[Path=/usr/share/fonts/truetype/cmu/,UprightFont=cmunrm.ttf,BoldFont=cmunbx.ttf,ItalicFont=cmunti.ttf,BoldItalicFont=cmunbi.ttf]{cmunti.ttf}\setmonofont[Path=/usr/share/fonts/truetype/cmu/,UprightFont=cmuntt.ttf,BoldFont=cmuntb.ttf,ItalicFont=cmunit.ttf,BoldItalicFont=cmuntx.ttf]{cmunti.ttf}\itshape too much}\setmainfont[Path=/usr/share/fonts/truetype/cmu/,UprightFont=cmunrm.ttf,BoldFont=cmunbx.ttf,ItalicFont=cmunti.ttf,BoldItalicFont=cmunbi.ttf]{cmunrm.ttf}\setmonofont[Path=/usr/share/fonts/truetype/cmu/,UprightFont=cmuntt.ttf,BoldFont=cmuntb.ttf,ItalicFont=cmunit.ttf,BoldItalicFont=cmuntx.ttf]{cmunrm.ttf}!

\begin{Shaded}
\begin{Highlighting}[]

\NormalTok{\textbackslash{}NeedsTeXFormat\{LaTeX2e\}}
\NormalTok{\textbackslash{}ProvidesPackage\{theme-fancy\}[2013/01/13 v1.0 fancy theme]}
 
\CommentTok{%%%%%%%%%%%%%%%%%%%%%%%%%%%%%%%%%%%%%%%%%%%%%%%%%%%%%%%%%%%%%%%%%%%%%%%%%%%%%%%%}
\CommentTok{%% Packages}
 
\NormalTok{\textbackslash{}RequirePackage\{geometry\}}
\NormalTok{\textbackslash{}RequirePackage\{needspace\}}
\NormalTok{\textbackslash{}RequirePackage[svgnames]\{xcolor\}}
 
\NormalTok{\textbackslash{}RequirePackage\{hyperref\}}
\NormalTok{\textbackslash{}hypersetup\{colorlinks=true\}}
 
\NormalTok{\textbackslash{}RequirePackage\{fancyhdr\}}
 
\NormalTok{\textbackslash{}RequirePackage\{tikz\}}
\NormalTok{\textbackslash{}usetikzlibrary\{}
  \NormalTok{calc,}
  \NormalTok{decorations.pathmorphing,}
  \NormalTok{fadings,}
  \NormalTok{shadows,}
  \NormalTok{shapes.geometric,}
  \NormalTok{shapes.misc,}
\NormalTok{\}}
 
\CommentTok{%%%%%%%%%%%%%%%%%%%%%%%%%%%%%%%%%%%%%%%%%%%%%%%%%%%%%%%%%%%%%%%%%%%%%%%%%%%%%%%%}
\CommentTok{%% Options}
 
\CommentTok{%% Toggle reflection.}
\NormalTok{\textbackslash{}newif\textbackslash{}if@mirrors\textbackslash{}@mirrorsfalse}
\NormalTok{\textbackslash{}DeclareOption\{mirrors\}\{}
  \NormalTok{\textbackslash{}@mirrorstrue}
\NormalTok{\}}
 
\CommentTok{%% Colors.}
\NormalTok{\textbackslash{}newif\textbackslash{}if@red\textbackslash{}@redfalse}
\NormalTok{\textbackslash{}DeclareOption\{red\}\{}
  \NormalTok{\textbackslash{}@redtrue}
\NormalTok{\}}
 
\NormalTok{\textbackslash{}ExecuteOptions\{\}}
\NormalTok{\textbackslash{}ProcessOptions\textbackslash{}relax}
 
\CommentTok{%%%%%%%%%%%%%%%%%%%%%%%%%%%%%%%%%%%%%%%%%%%%%%%%%%%%%%%%%%%%%%%%%%%%%%%%%%%%%%%%}
\CommentTok{%% Configuration}
 
\NormalTok{\textbackslash{}renewcommand\{\textbackslash{}familydefault\}\{\textbackslash{}sfdefault\}}
\NormalTok{\textbackslash{}setlength\{\textbackslash{}parskip\}\{0.5\textbackslash{}baselineskip\}}
 
\CommentTok{%% Colors}
\NormalTok{\textbackslash{}colorlet\{toctitle\}\{DarkGray!50!black\}}
\NormalTok{\textbackslash{}colorlet\{titlebg\}\{MidnightBlue\}}
\NormalTok{\textbackslash{}colorlet\{titlefg\}\{LightBlue\}}
\NormalTok{\textbackslash{}colorlet\{titletxt\}\{MidnightBlue\}}
\NormalTok{\textbackslash{}colorlet\{sectionfg\}\{MidnightBlue\}}
\NormalTok{\textbackslash{}colorlet\{subsectionfg\}\{SteelBlue\}}
\NormalTok{\textbackslash{}colorlet\{subsubsectionfg\}\{LightSteelBlue!60!black\}}
 
\NormalTok{\textbackslash{}if@red}
\NormalTok{\textbackslash{}colorlet\{toctitle\}\{DarkGray!50!black\}}
\NormalTok{\textbackslash{}colorlet\{titlebg\}\{DarkRed\}}
\NormalTok{\textbackslash{}colorlet\{titlefg\}\{FireBrick!50\}}
\NormalTok{\textbackslash{}colorlet\{titletxt\}\{DarkRed\}}
\NormalTok{\textbackslash{}colorlet\{sectionfg\}\{DarkRed\}}
\NormalTok{\textbackslash{}colorlet\{subsectionfg\}\{Crimson!50!black\}}
\NormalTok{\textbackslash{}colorlet\{subsubsectionfg\}\{LightSteelBlue!60!black\}}
\NormalTok{\textbackslash{}fi}
\end{Highlighting}
\end{Shaded}

\section{Header and  footer}
\label{890}

We use TikZ to draw a filled semicircle.

\LaTeXTT{fancyhdr} is used to set header and footer.  We take care of using the \LaTeXTT{fancy} style and to start from scratch by erasing the previous header and footer with \LaTeXTT{\textbackslash{}fancyhf\{\}}.

\begin{Shaded}
\begin{Highlighting}[]

\CommentTok{%%%%%%%%%%%%%%%%%%%%%%%%%%%%%%%%%%%%%%%%%%%%%%%%%%%%%%%%%%%%%%%%%%%%%%%%%%%%%%%%}
\CommentTok{%% Header and Footer}
 
\NormalTok{\textbackslash{}tikzstyle\{foliostyle\}=[fill=Lavender, text=MidnightBlue, inner sep=5pt,}
 \NormalTok{semicircle]}
 
\NormalTok{\textbackslash{}pagestyle\{fancy\}}
\NormalTok{\textbackslash{}fancyhf\{\}}
\NormalTok{\textbackslash{}fancyfoot[C]\{}
  \NormalTok{\textbackslash{}vskip 3pt}
  \NormalTok{\textbackslash{}begin\{tikzpicture\}}
    \NormalTok{\textbackslash{}node[foliostyle]}
    \NormalTok{\{\textbackslash{}bfseries\textbackslash{}thepage\};}
  \NormalTok{\textbackslash{}end\{tikzpicture\}}
\NormalTok{\}}
 
\NormalTok{\textbackslash{}renewcommand\{\textbackslash{}headrulewidth\}\{0.8pt\}}
\NormalTok{\textbackslash{}addtolength\{\textbackslash{}headheight\}\{\textbackslash{}baselineskip\} }
\NormalTok{\textbackslash{}renewcommand\{\textbackslash{}headrule\}\{\textbackslash{}color\{LightGray\}\textbackslash{}hrule\}}
\NormalTok{\textbackslash{}fancyhead[LE]\{ \textbackslash{}textcolor\{gray\}\{\textbackslash{}slshape \textbackslash{}rightmark\} \} }
\NormalTok{\textbackslash{}fancyhead[RO]\{ \textbackslash{}textcolor\{gray\}\{\textbackslash{}slshape \textbackslash{}leftmark\} \}}
\end{Highlighting}
\end{Shaded}

\section{Table of contents}
\label{891}

We redefine commands used by \LaTeXTT{\textbackslash{}tableofcontents}.

\begin{Shaded}
\begin{Highlighting}[]

\CommentTok{%%%%%%%%%%%%%%%%%%%%%%%%%%%%%%%%%%%%%%%%%%%%%%%%%%%%%%%%%%%%%%%%%%%%%%%%%%%%%%%%}
\CommentTok{%% Table of contents}
 
\NormalTok{\textbackslash{}newcommand\{\textbackslash{}sectiontoccolor\}\{sectionfg\}}
\NormalTok{\textbackslash{}newcommand\{\textbackslash{}subsectiontoccolor\}\{subsectionfg\}}
\NormalTok{\textbackslash{}newcommand\{\textbackslash{}subsubsectiontoccolor\}\{subsubsectionfg\}}
 
\NormalTok{\textbackslash{}renewcommand*\textbackslash{}l@section\{\textbackslash{}color\{\textbackslash{}sec}
\NormalTok{tiontoccolor\}\textbackslash{}def\textbackslash{}@linkcolor\{\textbackslash{}sectiontoccolor\}\textbackslash{}@dottedtocline\{1\}\{1.5em\}\{2.3em\}\}}
\NormalTok{\textbackslash{}renewcommand*\textbackslash{}l@subsection\{\textbackslash{}color\{\textbackslash{}subsectio}
\NormalTok{ntoccolor\}\textbackslash{}def\textbackslash{}@linkcolor\{\textbackslash{}subsectiontoccolor\}\textbackslash{}@dottedtocline\{1\}\{2.3em\}\{3.1em\}\}}
\NormalTok{\textbackslash{}renewcommand*\textbackslash{}l@subsubsection\{\textbackslash{}color\{\textbackslash{}subsubsectionto}
\NormalTok{ccolor\}\textbackslash{}def\textbackslash{}@linkcolor\{\textbackslash{}subsubsectiontoccolor\}\textbackslash{}@dottedtocline\{1\}\{3.1em\}\{3.9em\}\}}
\NormalTok{\textbackslash{}def\textbackslash{}contentsline#1#2#3#4\{}\CommentTok
  \NormalTok{\textbackslash{}csname l@#1\textbackslash{}endcsname\{#2\}\{#3\}}\CommentTok
    \NormalTok{\textbackslash{}hyper@linkstart\{link\}\{#4\}\{#3\}\textbackslash{}hyper@linkend}
  \NormalTok{\}}\CommentTok{%}
  \NormalTok{\textbackslash{}fi}
\NormalTok{\}}
 
\CommentTok{%% New title format -- 'section' is used by default.}
\NormalTok{\textbackslash{}newcommand\{\textbackslash{}tocformat\}[1]\{\{\textbackslash{}Huge\textbackslash{}bf#1\}\}}
 
\NormalTok{\textbackslash{}renewcommand\textbackslash{}tableofcontents\{}\CommentTok
  \NormalTok{\textbackslash{}@starttoc\{toc\}}\CommentTok{%}
\NormalTok{\}}
\end{Highlighting}
\end{Shaded}

\section{Sectioning}
\label{892}

This is definitely the most complex part. It is not that hard since the code is almost the same for \LaTeXTT{\textbackslash{}section}, \LaTeXTT{\textbackslash{}subsection} and \LaTeXTT{\textbackslash{}subsubsection}.

We use \LaTeXTT{\textbackslash{}needspace} to make sure there is no line break right after a sectioning command. We enclose the command in a group where we set a font size since the space we need is \LaTeXTT{\textbackslash{}baselineskip} which depends on the font size.

Starred commands will not set the counters (LaTeX detault behaviour). You can choose to handle starred command differently by resetting the counters for instance.

We preceed the section printing by a \LaTeXTT{\textbackslash{}noindent}.
We make sure to end the section printing by a \LaTeXTT{\textbackslash{}par} command to make sure following text gets printed properly.

For \LaTeXTT{\textbackslash{}subsection} we make use of the \LaTeXTT{mirrors} option to change the appearance accordingly.

To handle the PDF bookmarks properly we need the following lines at the end of the definitions.
\begin{Shaded}
\begin{Highlighting}[]

\NormalTok{\textbackslash{}phantomsection}
\NormalTok{\textbackslash{}addcontentsline\{toc\}\{section\}\{\textbackslash{}thesection~#1\}}
\end{Highlighting}
\end{Shaded}


Finally, for \LaTeXTT{\textbackslash{}section} only, we want it to print in the header, so we call the \LaTeXTT{\textbackslash{}sectionmark} command. Here we changed the behaviour of the starred command over the original LaTeX version, since we define and use the \LaTeXTT{\textbackslash{}sectionmarkstar} command.


\begin{Shaded}
\begin{Highlighting}[]

\CommentTok{%%%%%%%%%%%%%%%%%%%%%%%%%%%%%%%%%%%%%%%%%%%%%%%%%%%%%%%%%%%%%%%%%%%%%%%%%%%%%%%%}
\CommentTok{%% Section style}
 
\NormalTok{\textbackslash{}renewcommand\textbackslash{}section\{}
  \NormalTok{\textbackslash{}@ifstar}
  \NormalTok{\textbackslash{}my@sectionstar}
  \NormalTok{\textbackslash{}my@section}
\NormalTok{\}}
 
\CommentTok{%% Note: to justify, text width must be set to \textbackslash{}textwidth - 2*(inner sep).}
\NormalTok{\textbackslash{}tikzstyle\{sectionstyle\}=[}
  \NormalTok{inner sep=5pt,}
  \NormalTok{text width=\textbackslash{}textwidth-10pt,}
  \NormalTok{left color=sectionfg!100!white,}
  \NormalTok{right color=sectionfg!50!white,}
  \NormalTok{rounded corners,}
  \NormalTok{text=Ivory,}
  \NormalTok{rectangle}
\NormalTok{]}
 
\NormalTok{\textbackslash{}newcommand\textbackslash{}my@section[1]\{}
  \NormalTok{\textbackslash{}stepcounter\{section\}}
  \NormalTok{\{\textbackslash{}Large\textbackslash{}needspace\{\textbackslash{}baselineskip\}\}}
  \NormalTok{\textbackslash{}noindent}
  \NormalTok{\textbackslash{}begin\{tikzpicture\}}
    \NormalTok{\textbackslash{}node[sectionstyle] \{\textbackslash{}bfseries\textbackslash{}Large\textbackslash{}thesection\textbackslash{}quad#1\};}
  \NormalTok{\textbackslash{}end\{tikzpicture\}}
  \NormalTok{\textbackslash{}par}
  \NormalTok{\textbackslash{}phantomsection}
  \NormalTok{\textbackslash{}addcontentsline\{toc\}\{section\}\{\textbackslash{}thesection~#1\}}
  \NormalTok{\textbackslash{}sectionmark\{#1\}}
\NormalTok{\}}
 
\NormalTok{\textbackslash{}newcommand\{\textbackslash{}sectionmarkstar\}[1]\{\textbackslash{}markboth\{\textbackslash{}MakeUppercase\{#1\}\}\{\}\}}
 
\NormalTok{\textbackslash{}newcommand\textbackslash{}my@sectionstar[1]\{}
  \NormalTok{\{\textbackslash{}Large\textbackslash{}needspace\{\textbackslash{}baselineskip\}\}}
  \NormalTok{\textbackslash{}noindent}
  \NormalTok{\textbackslash{}begin\{tikzpicture\}}
    \NormalTok{\textbackslash{}node[sectionstyle] \{\textbackslash{}bfseries\textbackslash{}Large#1\};}
  \NormalTok{\textbackslash{}end\{tikzpicture\}}
  \NormalTok{\textbackslash{}par}
  \NormalTok{\textbackslash{}phantomsection}
  \NormalTok{\textbackslash{}addcontentsline\{toc\}\{section\}\{#1\}}
  \NormalTok{\textbackslash{}sectionmarkstar\{#1\}}
\NormalTok{\}}
 
 
\CommentTok{%%%%%%%%%%%%%%%%%%%%%%%%%%%%%%%%%%%%%%%%%%%%%%%%%%%%%%%%%%%%%%%%%%%%%%%%%%%%%%%%}
\CommentTok{%% Subsection style}
 
\NormalTok{\textbackslash{}renewcommand\textbackslash{}subsection\{}
  \NormalTok{\textbackslash{}@ifstar}
  \NormalTok{\textbackslash{}my@subsectionstar}
  \NormalTok{\textbackslash{}my@subsection}
\NormalTok{\}}
 
\NormalTok{\textbackslash{}tikzstyle\{subsectionstyle\}=[}
  \NormalTok{left color=subsectionfg!50!white,}
  \NormalTok{right color=subsectionfg!100!white,}
  \NormalTok{text=Ivory,}
  \NormalTok{ellipse,}
  \NormalTok{inner sep=5pt}
\NormalTok{]}
 
\NormalTok{\textbackslash{}newcommand\textbackslash{}my@subsection[1]\{}
  \NormalTok{\textbackslash{}stepcounter\{subsection\}}
  \NormalTok{\{\textbackslash{}Large\textbackslash{}needspace\{\textbackslash{}baselineskip\}\}}
  \NormalTok{\textbackslash{}noindent  }
  \NormalTok{\textbackslash{}begin\{tikzpicture\}}
    \NormalTok{\textbackslash{}node[subsectionstyle,anchor=west] (number) at (0,0)}
 \NormalTok{\{\textbackslash{}bfseries\textbackslash{}Large\textbackslash{}thesubsection\};}
    \NormalTok{\textbackslash{}if@mirrors}
    \NormalTok{\textbackslash{}node[above right,subsectionfg,anchor=south west] at}
 \NormalTok{($(number.east)+(0.1,-0.1)$) \{\textbackslash{}large\textbackslash{}bfseries#1\};}
    \NormalTok{\textbackslash{}node[yscale=-1, scope fading=south, opacity=0.4, above, anchor=south west,}
 \NormalTok{subsectionfg] at ($(number.east)+(0.1,0.1)$) \{\textbackslash{}large\textbackslash{}bfseries#1\};}
    \NormalTok{\textbackslash{}else}
    \NormalTok{\textbackslash{}node[above right,subsectionfg,anchor=west] at ($(number.east)+(0.1,0)$)}
 \NormalTok{\{\textbackslash{}large\textbackslash{}bfseries#1\};}
    \NormalTok{\textbackslash{}fi}
  \NormalTok{\textbackslash{}end\{tikzpicture\}}
  \NormalTok{\textbackslash{}par}
  \NormalTok{\textbackslash{}phantomsection}
  \NormalTok{\textbackslash{}addcontentsline\{toc\}\{subsection\}\{\textbackslash{}thesubsection~#1\}}
\NormalTok{\}}
 
\NormalTok{\textbackslash{}newcommand\textbackslash{}my@subsectionstar[1]\{}
  \NormalTok{\{\textbackslash{}Large\textbackslash{}needspace\{\textbackslash{}baselineskip\}\}}
  \NormalTok{\textbackslash{}noindent}
  \NormalTok{\textbackslash{}begin\{tikzpicture\}}
    \NormalTok{\textbackslash{}node[subsectionstyle,anchor=west] (number) at (0,0)}
 \NormalTok{\{\textbackslash{}bfseries\textbackslash{}Large\textbackslash{}phantom\{1\}\};}
    \CommentTok{% }
    \NormalTok{\textbackslash{}if@mirrors}
    \NormalTok{\textbackslash{}node[above right,subsectionfg,anchor=south west] at}
 \NormalTok{($(number.east)+(0.1,-0.1)$) \{\textbackslash{}large\textbackslash{}bfseries#1\};}
    \NormalTok{\textbackslash{}node[yscale=-1, scope fading=south, opacity=0.4, above, anchor=south west,}
 \NormalTok{subsectionfg] at ($(number.east)+(0.1,0.1)$) \{\textbackslash{}large\textbackslash{}bfseries#1\};}
    \NormalTok{\textbackslash{}else}
    \NormalTok{\textbackslash{}node[above right,subsectionfg,anchor=west] at ($(number.east)+(0.1,0)$)}
 \NormalTok{\{\textbackslash{}large\textbackslash{}bfseries#1\};}
    \NormalTok{\textbackslash{}fi}
  \NormalTok{\textbackslash{}end\{tikzpicture\}}
  \NormalTok{\textbackslash{}par}
  \NormalTok{\textbackslash{}phantomsection}
  \NormalTok{\textbackslash{}addcontentsline\{toc\}\{subsection\}\{#1\}}
\NormalTok{\}}
 
\CommentTok{%%%%%%%%%%%%%%%%%%%%%%%%%%%%%%%%%%%%%%%%%%%%%%%%%%%%%%%%%%%%%%%%%%%%%%%%%%%%%%%%}
\CommentTok{%% Subsubsection style}
 
\NormalTok{\textbackslash{}renewcommand\textbackslash{}subsubsection\{}
  \NormalTok{\textbackslash{}@ifstar}
  \NormalTok{\textbackslash{}my@subsubsectionstar}
  \NormalTok{\textbackslash{}my@subsubsection}
\NormalTok{\}}
 
\NormalTok{\textbackslash{}tikzstyle\{subsubsectionstyle\}=[}
  \NormalTok{left color=subsubsectionfg!50!white,}
  \NormalTok{right color=subsubsectionfg!100!white,}
  \NormalTok{text=Ivory,}
  \NormalTok{shape=trapezium,}
  \NormalTok{inner sep=5pt}
\NormalTok{]}
 
\NormalTok{\textbackslash{}newcommand\textbackslash{}my@subsubsection[1]\{}
  \NormalTok{\textbackslash{}stepcounter\{subsubsection\}}
  \NormalTok{\textbackslash{}noindent  }
  \NormalTok{\textbackslash{}begin\{tikzpicture\}}
	\NormalTok{\textbackslash{}node[subsubsectionstyle] (number) \{\textbackslash{}bfseries\textbackslash{}large\textbackslash{}thesubsubsection\};}
	\NormalTok{\textbackslash{}node[subsubsectionfg, right of=number, anchor=west] \{\textbackslash{}large\textbackslash{}bfseries#1\};}
  \NormalTok{\textbackslash{}end\{tikzpicture\}}
  \NormalTok{\textbackslash{}par}
  \NormalTok{\textbackslash{}phantomsection}
  \NormalTok{\textbackslash{}addcontentsline\{toc\}\{subsubsection\}\{\textbackslash{}thesubsubsection~#1\}}
\NormalTok{\}}
 
\NormalTok{\textbackslash{}newcommand\textbackslash{}my@subsubsectionstar[1]\{}
  \NormalTok{\textbackslash{}noindent}
  \NormalTok{\textbackslash{}begin\{tikzpicture\}}
	\NormalTok{\textbackslash{}node[subsubsectionstyle] (number) \{\textbackslash{}bfseries\textbackslash{}large\textbackslash{}vphantom\{1\}\};}
	\NormalTok{\textbackslash{}node[subsubsectionfg, right of=number, anchor=west] \{\textbackslash{}large\textbackslash{}bfseries#1\};}
  \NormalTok{\textbackslash{}end\{tikzpicture\}}
  \NormalTok{\textbackslash{}par}
  \NormalTok{\textbackslash{}phantomsection}
  \NormalTok{\textbackslash{}addcontentsline\{toc\}\{subsubsection\}\{#1\}}
\NormalTok{\}}
 
\NormalTok{\textbackslash{}endinput}
\end{Highlighting}
\end{Shaded}


\section{Notes and References}
\label{893}
\LaTeXNullTemplate{}


\mypart{Miscellaneous}\chapter{Modular Documents}

\myminitoc
\label{894}

\label{895}


During this guide we have seen what is possible to do and how this can be achieved, but the question is: I want to write a proper text with LaTeX, what to do then? Where should I start from? This is a short step-{}by-{}step guide about how to start a document properly, keeping a good high-{}level structure. This is all about organizing your files using the modular capabilities of LaTeX. This way it will be very easy to make modifications even when the document is almost finished. These are all just suggestions, but you might take inspiration from that to create your own document.
\section{Project structure}
\label{896}
Create a clear structure of the whole project this way:
\begin{myenumerate}
\item{}  create a directory only for the project. We\textquotesingle{}ll refer to that in the following parts as the {\itshape \setmainfont[Path=/usr/share/fonts/truetype/cmu/,UprightFont=cmunrm.ttf,BoldFont=cmunbx.ttf,ItalicFont=cmunti.ttf,BoldItalicFont=cmunbi.ttf]{cmunti.ttf}\setmonofont[Path=/usr/share/fonts/truetype/cmu/,UprightFont=cmuntt.ttf,BoldFont=cmuntb.ttf,ItalicFont=cmunit.ttf,BoldItalicFont=cmuntx.ttf]{cmunti.ttf}\itshape root directory}
\item{} {$\text{ }$}\setmainfont[Path=/usr/share/fonts/truetype/cmu/,UprightFont=cmunrm.ttf,BoldFont=cmunbx.ttf,ItalicFont=cmunti.ttf,BoldItalicFont=cmunbi.ttf]{cmunrm.ttf}\setmonofont[Path=/usr/share/fonts/truetype/cmu/,UprightFont=cmuntt.ttf,BoldFont=cmuntb.ttf,ItalicFont=cmunit.ttf,BoldItalicFont=cmuntx.ttf]{cmunrm.ttf} create two other directories inside the root, one for LaTeX documents, the other one for images. Since you\textquotesingle{}ll have to write their name quite often, choose short names. A suggestion would be simply {\itshape \setmainfont[Path=/usr/share/fonts/truetype/cmu/,UprightFont=cmunrm.ttf,BoldFont=cmunbx.ttf,ItalicFont=cmunti.ttf,BoldItalicFont=cmunbi.ttf]{cmunti.ttf}\setmonofont[Path=/usr/share/fonts/truetype/cmu/,UprightFont=cmuntt.ttf,BoldFont=cmuntb.ttf,ItalicFont=cmunit.ttf,BoldItalicFont=cmuntx.ttf]{cmunti.ttf}\itshape tex}{$\text{ }$}\setmainfont[Path=/usr/share/fonts/truetype/cmu/,UprightFont=cmunrm.ttf,BoldFont=cmunbx.ttf,ItalicFont=cmunti.ttf,BoldItalicFont=cmunbi.ttf]{cmunrm.ttf}\setmonofont[Path=/usr/share/fonts/truetype/cmu/,UprightFont=cmuntt.ttf,BoldFont=cmuntb.ttf,ItalicFont=cmunit.ttf,BoldItalicFont=cmuntx.ttf]{cmunrm.ttf} and {\itshape \setmainfont[Path=/usr/share/fonts/truetype/cmu/,UprightFont=cmunrm.ttf,BoldFont=cmunbx.ttf,ItalicFont=cmunti.ttf,BoldItalicFont=cmunbi.ttf]{cmunti.ttf}\setmonofont[Path=/usr/share/fonts/truetype/cmu/,UprightFont=cmuntt.ttf,BoldFont=cmuntb.ttf,ItalicFont=cmunit.ttf,BoldItalicFont=cmuntx.ttf]{cmunti.ttf}\itshape img}\setmainfont[Path=/usr/share/fonts/truetype/cmu/,UprightFont=cmunrm.ttf,BoldFont=cmunbx.ttf,ItalicFont=cmunti.ttf,BoldItalicFont=cmunbi.ttf]{cmunrm.ttf}\setmonofont[Path=/usr/share/fonts/truetype/cmu/,UprightFont=cmuntt.ttf,BoldFont=cmuntb.ttf,ItalicFont=cmunit.ttf,BoldItalicFont=cmuntx.ttf]{cmunrm.ttf}.
\item{}  create your document (we\textquotesingle{}ll call it document.tex, but you can use the name you prefer) and your own package (for example {\itshape \setmainfont[Path=/usr/share/fonts/truetype/cmu/,UprightFont=cmunrm.ttf,BoldFont=cmunbx.ttf,ItalicFont=cmunti.ttf,BoldItalicFont=cmunbi.ttf]{cmunti.ttf}\setmonofont[Path=/usr/share/fonts/truetype/cmu/,UprightFont=cmuntt.ttf,BoldFont=cmuntb.ttf,ItalicFont=cmunit.ttf,BoldItalicFont=cmuntx.ttf]{cmunti.ttf}\itshape mystyle.sty}\setmainfont[Path=/usr/share/fonts/truetype/cmu/,UprightFont=cmunrm.ttf,BoldFont=cmunbx.ttf,ItalicFont=cmunti.ttf,BoldItalicFont=cmunbi.ttf]{cmunrm.ttf}\setmonofont[Path=/usr/share/fonts/truetype/cmu/,UprightFont=cmuntt.ttf,BoldFont=cmuntb.ttf,ItalicFont=cmunit.ttf,BoldItalicFont=cmuntx.ttf]{cmunrm.ttf}); this second file will help you to keep the code cleaner.
\end{myenumerate}


If you followed all those steps, these files should be in your root directory, using \symbol{34}/\symbol{34} for each directory:\\

\TemplateSpaceIndent{$\text{ }${}./document.tex$\text{ }$\newline{}
$\text{ }${}./mystyle.sty$\text{ }$\newline{}
$\text{ }${}./tex/$\text{ }$\newline{}
$\text{ }${}./img/}

nothing else.
\section{Getting LaTeX to process multiple files}
\label{897}
As your work grows, your LaTeX file can become unwieldy and confusing, especially if you are writing a long article with substantial, discrete sections, or a full-{}length book. In such cases it is good practice to split your work into several files. For example, if you are writing a book, it makes a lot of sense to write each chapter in a separate {\ttfamily \setmainfont[Path=/usr/share/fonts/truetype/cmu/,UprightFont=cmunrm.ttf,BoldFont=cmunbx.ttf,ItalicFont=cmunti.ttf,BoldItalicFont=cmunbi.ttf]{cmuntt.ttf}\setmonofont[Path=/usr/share/fonts/truetype/cmu/,UprightFont=cmuntt.ttf,BoldFont=cmuntb.ttf,ItalicFont=cmunit.ttf,BoldItalicFont=cmuntx.ttf]{cmuntt.ttf}\ttfamily .tex}{$\text{ }$}\setmainfont[Path=/usr/share/fonts/truetype/cmu/,UprightFont=cmunrm.ttf,BoldFont=cmunbx.ttf,ItalicFont=cmunti.ttf,BoldItalicFont=cmunbi.ttf]{cmunrm.ttf}\setmonofont[Path=/usr/share/fonts/truetype/cmu/,UprightFont=cmuntt.ttf,BoldFont=cmuntb.ttf,ItalicFont=cmunit.ttf,BoldItalicFont=cmuntx.ttf]{cmunrm.ttf} file. LaTeX makes this very easy thanks to two commands:

\begin{Shaded}
\begin{Highlighting}[]

\NormalTok{\textbackslash{}input\{filename\}}\newline
\end{Highlighting}
\end{Shaded}

and

\begin{Shaded}
\begin{Highlighting}[]

\NormalTok{\textbackslash{}include\{filename\}}\newline
\end{Highlighting}
\end{Shaded}

\subsection{Comparing the methods: input vs include}
\label{898}
The differences between these two ways to include files is explained below. What they have in common is that they process the contents of {\ttfamily \setmainfont[Path=/usr/share/fonts/truetype/cmu/,UprightFont=cmunrm.ttf,BoldFont=cmunbx.ttf,ItalicFont=cmunti.ttf,BoldItalicFont=cmunbi.ttf]{cmuntt.ttf}\setmonofont[Path=/usr/share/fonts/truetype/cmu/,UprightFont=cmuntt.ttf,BoldFont=cmuntb.ttf,ItalicFont=cmunit.ttf,BoldItalicFont=cmuntx.ttf]{cmuntt.ttf}\ttfamily filename.tex}{$\text{ }$}\setmainfont[Path=/usr/share/fonts/truetype/cmu/,UprightFont=cmunrm.ttf,BoldFont=cmunbx.ttf,ItalicFont=cmunti.ttf,BoldItalicFont=cmunbi.ttf]{cmunrm.ttf}\setmonofont[Path=/usr/share/fonts/truetype/cmu/,UprightFont=cmuntt.ttf,BoldFont=cmuntb.ttf,ItalicFont=cmunit.ttf,BoldItalicFont=cmuntx.ttf]{cmunrm.ttf} before continuing with the rest of the base file (the file that contains these statements).
When the compiler processes your base file and reaches one of the commands \LaTeXTT{\textbackslash{}input} or \LaTeXTT{\textbackslash{}include}, it reads {\ttfamily \setmainfont[Path=/usr/share/fonts/truetype/cmu/,UprightFont=cmunrm.ttf,BoldFont=cmunbx.ttf,ItalicFont=cmunti.ttf,BoldItalicFont=cmunbi.ttf]{cmuntt.ttf}\setmonofont[Path=/usr/share/fonts/truetype/cmu/,UprightFont=cmuntt.ttf,BoldFont=cmuntb.ttf,ItalicFont=cmunit.ttf,BoldItalicFont=cmuntx.ttf]{cmuntt.ttf}\ttfamily filename.tex}{$\text{ }$}\setmainfont[Path=/usr/share/fonts/truetype/cmu/,UprightFont=cmunrm.ttf,BoldFont=cmunbx.ttf,ItalicFont=cmunti.ttf,BoldItalicFont=cmunbi.ttf]{cmunrm.ttf}\setmonofont[Path=/usr/share/fonts/truetype/cmu/,UprightFont=cmuntt.ttf,BoldFont=cmuntb.ttf,ItalicFont=cmunit.ttf,BoldItalicFont=cmuntx.ttf]{cmunrm.ttf} and processes its content in accordance with the formatting commands specified in the base file. This way you can put all the formatting options in your base file and write the contents using \LaTeXTT{\textbackslash{}input} or \LaTeXTT{\textbackslash{}include} in the files which contain the actual content of your work. This means that the important part of your working process, i.e. writing, is kept largely separate from formatting choices. This is one of the main reasons why LaTeX is so good for serious writing! You will thus be dealing solely with text and very basic commands such as \LaTeXTT{\textbackslash{}section}, \LaTeXTT{\textbackslash{}emph} etc. Your document will be uncluttered and much easier to work with.

The second method of including a file, \LaTeXTT{\textbackslash{}include\{filename\}}, differs from the first in some important ways. You cannot nest \LaTeXTT{\textbackslash{}include} statements within a file added via \LaTeXTT{\textbackslash{}include}, whereas \LaTeXTT{\textbackslash{}input}, on the other hand, allows you to call files which themselves call other files, ad infinitum (well, nearly!). You can, however, \LaTeXTT{\textbackslash{}include} a file which contains one or more \LaTeXTT{\textbackslash{}input} commands. Please resist the temptation to nest files in this way simply because the system can do it: you will end up with just another kind of complexity!

Another important difference is that using \LaTeXTT{\textbackslash{}include} will force a page break (which makes it ideal for a book\textquotesingle{}s chapters), whereas the \LaTeXTT{\textbackslash{}input} command does not (which in turn makes it ideal for use within, say, a long article with discrete sections, which of course are not normally set on a new page).  

In either case the {\ttfamily \setmainfont[Path=/usr/share/fonts/truetype/cmu/,UprightFont=cmunrm.ttf,BoldFont=cmunbx.ttf,ItalicFont=cmunti.ttf,BoldItalicFont=cmunbi.ttf]{cmuntt.ttf}\setmonofont[Path=/usr/share/fonts/truetype/cmu/,UprightFont=cmuntt.ttf,BoldFont=cmuntb.ttf,ItalicFont=cmunit.ttf,BoldItalicFont=cmuntx.ttf]{cmuntt.ttf}\ttfamily .tex}{$\text{ }$}\setmainfont[Path=/usr/share/fonts/truetype/cmu/,UprightFont=cmunrm.ttf,BoldFont=cmunbx.ttf,ItalicFont=cmunti.ttf,BoldItalicFont=cmunbi.ttf]{cmunrm.ttf}\setmonofont[Path=/usr/share/fonts/truetype/cmu/,UprightFont=cmuntt.ttf,BoldFont=cmuntb.ttf,ItalicFont=cmunit.ttf,BoldItalicFont=cmuntx.ttf]{cmunrm.ttf} filename extension is optional.

Working on discrete parts of your documents has consequences for how the base file is compiled; these will be dealt with below.
\subsection{Using different paths}
\label{899}

When the LaTeX compiler finds a reference to an external file in the base file, it will look for it in the same directory. However, you can in principle refer to any file on your system, using both absolute and relative paths.

An {\itshape \setmainfont[Path=/usr/share/fonts/truetype/cmu/,UprightFont=cmunrm.ttf,BoldFont=cmunbx.ttf,ItalicFont=cmunti.ttf,BoldItalicFont=cmunbi.ttf]{cmunti.ttf}\setmonofont[Path=/usr/share/fonts/truetype/cmu/,UprightFont=cmuntt.ttf,BoldFont=cmuntb.ttf,ItalicFont=cmunit.ttf,BoldItalicFont=cmuntx.ttf]{cmunti.ttf}\itshape absolute}{$\text{ }$}\setmainfont[Path=/usr/share/fonts/truetype/cmu/,UprightFont=cmunrm.ttf,BoldFont=cmunbx.ttf,ItalicFont=cmunti.ttf,BoldItalicFont=cmunbi.ttf]{cmunrm.ttf}\setmonofont[Path=/usr/share/fonts/truetype/cmu/,UprightFont=cmuntt.ttf,BoldFont=cmuntb.ttf,ItalicFont=cmunit.ttf,BoldItalicFont=cmuntx.ttf]{cmunrm.ttf} path is a full path-{} and filename with every element specified. So, {\ttfamily \setmainfont[Path=/usr/share/fonts/truetype/cmu/,UprightFont=cmunrm.ttf,BoldFont=cmunbx.ttf,ItalicFont=cmunti.ttf,BoldItalicFont=cmunbi.ttf]{cmuntt.ttf}\setmonofont[Path=/usr/share/fonts/truetype/cmu/,UprightFont=cmuntt.ttf,BoldFont=cmuntb.ttf,ItalicFont=cmunit.ttf,BoldItalicFont=cmuntx.ttf]{cmuntt.ttf}\ttfamily filename.tex}{$\text{ }$}\setmainfont[Path=/usr/share/fonts/truetype/cmu/,UprightFont=cmunrm.ttf,BoldFont=cmunbx.ttf,ItalicFont=cmunti.ttf,BoldItalicFont=cmunbi.ttf]{cmunrm.ttf}\setmonofont[Path=/usr/share/fonts/truetype/cmu/,UprightFont=cmuntt.ttf,BoldFont=cmuntb.ttf,ItalicFont=cmunit.ttf,BoldItalicFont=cmuntx.ttf]{cmunrm.ttf} might have the full path,

\begin{Shaded}
\begin{Highlighting}[]

\NormalTok{\textbackslash{}input\{/home/user/texfiles/filename.tex\}}\newline
\end{Highlighting}
\end{Shaded}

If you had created the directory {\ttfamily \setmainfont[Path=/usr/share/fonts/truetype/cmu/,UprightFont=cmunrm.ttf,BoldFont=cmunbx.ttf,ItalicFont=cmunti.ttf,BoldItalicFont=cmunbi.ttf]{cmuntt.ttf}\setmonofont[Path=/usr/share/fonts/truetype/cmu/,UprightFont=cmuntt.ttf,BoldFont=cmuntb.ttf,ItalicFont=cmunit.ttf,BoldItalicFont=cmuntx.ttf]{cmuntt.ttf}\ttfamily myfiles}{$\text{ }$}\setmainfont[Path=/usr/share/fonts/truetype/cmu/,UprightFont=cmunrm.ttf,BoldFont=cmunbx.ttf,ItalicFont=cmunti.ttf,BoldItalicFont=cmunbi.ttf]{cmunrm.ttf}\setmonofont[Path=/usr/share/fonts/truetype/cmu/,UprightFont=cmuntt.ttf,BoldFont=cmuntb.ttf,ItalicFont=cmunit.ttf,BoldItalicFont=cmuntx.ttf]{cmunrm.ttf} for your writing project, in your {\ttfamily \setmainfont[Path=/usr/share/fonts/truetype/cmu/,UprightFont=cmunrm.ttf,BoldFont=cmunbx.ttf,ItalicFont=cmunti.ttf,BoldItalicFont=cmunbi.ttf]{cmuntt.ttf}\setmonofont[Path=/usr/share/fonts/truetype/cmu/,UprightFont=cmuntt.ttf,BoldFont=cmuntb.ttf,ItalicFont=cmunit.ttf,BoldItalicFont=cmuntx.ttf]{cmuntt.ttf}\ttfamily texfiles}{$\text{ }$}\setmainfont[Path=/usr/share/fonts/truetype/cmu/,UprightFont=cmunrm.ttf,BoldFont=cmunbx.ttf,ItalicFont=cmunti.ttf,BoldItalicFont=cmunbi.ttf]{cmunrm.ttf}\setmonofont[Path=/usr/share/fonts/truetype/cmu/,UprightFont=cmuntt.ttf,BoldFont=cmuntb.ttf,ItalicFont=cmunit.ttf,BoldItalicFont=cmuntx.ttf]{cmunrm.ttf} directory, its full path would be,

\begin{Shaded}
\begin{Highlighting}[]

\NormalTok{\textbackslash{}input\{/home/user/texfiles/myfiles/filename.tex\}}\newline
\end{Highlighting}
\end{Shaded}

Obviously, using absolute paths is inefficient if you are referring to a file in the current directory. If, however, you need to include a file which is always kept at a specific place in your system, you may refer to it with an absolute path, for example,

\begin{Shaded}
\begin{Highlighting}[]

\NormalTok{\textbackslash{}input\{/home/user/documents/useful/foo.tex\}}\newline
\end{Highlighting}
\end{Shaded}

In practice, an absolute file path is generally used when one has to refer to a file which is quite some way away in the file system (or perhaps even on a different server!). One word of warning: do not leave empty spaces in the filenames, they can cause ambiguous behaviour. Either leave no spaces or use underscores {\bfseries \setmainfont[Path=/usr/share/fonts/truetype/cmu/,UprightFont=cmunrm.ttf,BoldFont=cmunbx.ttf,ItalicFont=cmunti.ttf,BoldItalicFont=cmunbi.ttf]{cmunbx.ttf}\setmonofont[Path=/usr/share/fonts/truetype/cmu/,UprightFont=cmuntt.ttf,BoldFont=cmuntb.ttf,ItalicFont=cmunit.ttf,BoldItalicFont=cmuntx.ttf]{cmunbx.ttf}\bfseries \_}{$\text{ }$}\setmainfont[Path=/usr/share/fonts/truetype/cmu/,UprightFont=cmunrm.ttf,BoldFont=cmunbx.ttf,ItalicFont=cmunti.ttf,BoldItalicFont=cmunbi.ttf]{cmunrm.ttf}\setmonofont[Path=/usr/share/fonts/truetype/cmu/,UprightFont=cmuntt.ttf,BoldFont=cmuntb.ttf,ItalicFont=cmunit.ttf,BoldItalicFont=cmuntx.ttf]{cmunrm.ttf} instead. 

You may, however, need to make your source portable (to another computer or to a different location of your harddisk), in which case {\itshape \setmainfont[Path=/usr/share/fonts/truetype/cmu/,UprightFont=cmunrm.ttf,BoldFont=cmunbx.ttf,ItalicFont=cmunti.ttf,BoldItalicFont=cmunbi.ttf]{cmunti.ttf}\setmonofont[Path=/usr/share/fonts/truetype/cmu/,UprightFont=cmuntt.ttf,BoldFont=cmuntb.ttf,ItalicFont=cmunit.ttf,BoldItalicFont=cmuntx.ttf]{cmunti.ttf}\itshape relative}{$\text{ }$}\setmainfont[Path=/usr/share/fonts/truetype/cmu/,UprightFont=cmunrm.ttf,BoldFont=cmunbx.ttf,ItalicFont=cmunti.ttf,BoldItalicFont=cmunbi.ttf]{cmunrm.ttf}\setmonofont[Path=/usr/share/fonts/truetype/cmu/,UprightFont=cmuntt.ttf,BoldFont=cmuntb.ttf,ItalicFont=cmunit.ttf,BoldItalicFont=cmuntx.ttf]{cmunrm.ttf} paths should be used if you wish to avoid unnecessary rewriting of path names. Or, a relative path may simply be a more efficient and elegant way of referring to a file. A relative path is one which is defined in relation to the current directory, in our case the one which contains the base file. LaTeX uses the standard UNIX notation: with a simple dot {\bfseries \setmainfont[Path=/usr/share/fonts/truetype/cmu/,UprightFont=cmunrm.ttf,BoldFont=cmunbx.ttf,ItalicFont=cmunti.ttf,BoldItalicFont=cmunbi.ttf]{cmunbx.ttf}\setmonofont[Path=/usr/share/fonts/truetype/cmu/,UprightFont=cmuntt.ttf,BoldFont=cmuntb.ttf,ItalicFont=cmunit.ttf,BoldItalicFont=cmuntx.ttf]{cmunbx.ttf}\bfseries .}{$\text{ }$}\setmainfont[Path=/usr/share/fonts/truetype/cmu/,UprightFont=cmunrm.ttf,BoldFont=cmunbx.ttf,ItalicFont=cmunti.ttf,BoldItalicFont=cmunbi.ttf]{cmunrm.ttf}\setmonofont[Path=/usr/share/fonts/truetype/cmu/,UprightFont=cmuntt.ttf,BoldFont=cmuntb.ttf,ItalicFont=cmunit.ttf,BoldItalicFont=cmuntx.ttf]{cmunrm.ttf} you refer to the current directory, and by two dots {\bfseries \setmainfont[Path=/usr/share/fonts/truetype/cmu/,UprightFont=cmunrm.ttf,BoldFont=cmunbx.ttf,ItalicFont=cmunti.ttf,BoldItalicFont=cmunbi.ttf]{cmunbx.ttf}\setmonofont[Path=/usr/share/fonts/truetype/cmu/,UprightFont=cmuntt.ttf,BoldFont=cmuntb.ttf,ItalicFont=cmunit.ttf,BoldItalicFont=cmuntx.ttf]{cmunbx.ttf}\bfseries ..}{$\text{ }$}\setmainfont[Path=/usr/share/fonts/truetype/cmu/,UprightFont=cmunrm.ttf,BoldFont=cmunbx.ttf,ItalicFont=cmunti.ttf,BoldItalicFont=cmunbi.ttf]{cmunrm.ttf}\setmonofont[Path=/usr/share/fonts/truetype/cmu/,UprightFont=cmuntt.ttf,BoldFont=cmuntb.ttf,ItalicFont=cmunit.ttf,BoldItalicFont=cmuntx.ttf]{cmunrm.ttf} you refer to the previous directory, that is the one above the current directory in the file system tree. The slash {\bfseries \setmainfont[Path=/usr/share/fonts/truetype/cmu/,UprightFont=cmunrm.ttf,BoldFont=cmunbx.ttf,ItalicFont=cmunti.ttf,BoldItalicFont=cmunbi.ttf]{cmunbx.ttf}\setmonofont[Path=/usr/share/fonts/truetype/cmu/,UprightFont=cmuntt.ttf,BoldFont=cmuntb.ttf,ItalicFont=cmunit.ttf,BoldItalicFont=cmuntx.ttf]{cmunbx.ttf}\bfseries /}{$\text{ }$}\setmainfont[Path=/usr/share/fonts/truetype/cmu/,UprightFont=cmunrm.ttf,BoldFont=cmunbx.ttf,ItalicFont=cmunti.ttf,BoldItalicFont=cmunbi.ttf]{cmunrm.ttf}\setmonofont[Path=/usr/share/fonts/truetype/cmu/,UprightFont=cmuntt.ttf,BoldFont=cmuntb.ttf,ItalicFont=cmunit.ttf,BoldItalicFont=cmuntx.ttf]{cmunrm.ttf} is used to separate the different components of a pathname: directories and filenames. So by {\bfseries \setmainfont[Path=/usr/share/fonts/truetype/cmu/,UprightFont=cmunrm.ttf,BoldFont=cmunbx.ttf,ItalicFont=cmunti.ttf,BoldItalicFont=cmunbi.ttf]{cmunbx.ttf}\setmonofont[Path=/usr/share/fonts/truetype/cmu/,UprightFont=cmuntt.ttf,BoldFont=cmuntb.ttf,ItalicFont=cmunit.ttf,BoldItalicFont=cmuntx.ttf]{cmunbx.ttf}\bfseries ./}{$\text{ }$}\setmainfont[Path=/usr/share/fonts/truetype/cmu/,UprightFont=cmunrm.ttf,BoldFont=cmunbx.ttf,ItalicFont=cmunti.ttf,BoldItalicFont=cmunbi.ttf]{cmunrm.ttf}\setmonofont[Path=/usr/share/fonts/truetype/cmu/,UprightFont=cmuntt.ttf,BoldFont=cmuntb.ttf,ItalicFont=cmunit.ttf,BoldItalicFont=cmuntx.ttf]{cmunrm.ttf} you refer to the current directory, by {\bfseries \setmainfont[Path=/usr/share/fonts/truetype/cmu/,UprightFont=cmunrm.ttf,BoldFont=cmunbx.ttf,ItalicFont=cmunti.ttf,BoldItalicFont=cmunbi.ttf]{cmunbx.ttf}\setmonofont[Path=/usr/share/fonts/truetype/cmu/,UprightFont=cmuntt.ttf,BoldFont=cmuntb.ttf,ItalicFont=cmunit.ttf,BoldItalicFont=cmuntx.ttf]{cmunbx.ttf}\bfseries ../}{$\text{ }$}\setmainfont[Path=/usr/share/fonts/truetype/cmu/,UprightFont=cmunrm.ttf,BoldFont=cmunbx.ttf,ItalicFont=cmunti.ttf,BoldItalicFont=cmunbi.ttf]{cmunrm.ttf}\setmonofont[Path=/usr/share/fonts/truetype/cmu/,UprightFont=cmuntt.ttf,BoldFont=cmuntb.ttf,ItalicFont=cmunit.ttf,BoldItalicFont=cmuntx.ttf]{cmunrm.ttf} you refer to the previous directory, by {\bfseries \setmainfont[Path=/usr/share/fonts/truetype/cmu/,UprightFont=cmunrm.ttf,BoldFont=cmunbx.ttf,ItalicFont=cmunti.ttf,BoldItalicFont=cmunbi.ttf]{cmunbx.ttf}\setmonofont[Path=/usr/share/fonts/truetype/cmu/,UprightFont=cmuntt.ttf,BoldFont=cmuntb.ttf,ItalicFont=cmunit.ttf,BoldItalicFont=cmuntx.ttf]{cmunbx.ttf}\bfseries ../../}{$\text{ }$}\setmainfont[Path=/usr/share/fonts/truetype/cmu/,UprightFont=cmunrm.ttf,BoldFont=cmunbx.ttf,ItalicFont=cmunti.ttf,BoldItalicFont=cmunbi.ttf]{cmunrm.ttf}\setmonofont[Path=/usr/share/fonts/truetype/cmu/,UprightFont=cmuntt.ttf,BoldFont=cmuntb.ttf,ItalicFont=cmunit.ttf,BoldItalicFont=cmuntx.ttf]{cmunrm.ttf} you refer to a directory which is two steps upwards in the filesystem tree.
Writing

\begin{Shaded}
\begin{Highlighting}[]

\NormalTok{\textbackslash{}input\{./filename.tex\}}\newline
\end{Highlighting}
\end{Shaded}

will have {\itshape \setmainfont[Path=/usr/share/fonts/truetype/cmu/,UprightFont=cmunrm.ttf,BoldFont=cmunbx.ttf,ItalicFont=cmunti.ttf,BoldItalicFont=cmunbi.ttf]{cmunti.ttf}\setmonofont[Path=/usr/share/fonts/truetype/cmu/,UprightFont=cmuntt.ttf,BoldFont=cmuntb.ttf,ItalicFont=cmunit.ttf,BoldItalicFont=cmuntx.ttf]{cmunti.ttf}\itshape exactly}{$\text{ }$}\setmainfont[Path=/usr/share/fonts/truetype/cmu/,UprightFont=cmunrm.ttf,BoldFont=cmunbx.ttf,ItalicFont=cmunti.ttf,BoldItalicFont=cmunbi.ttf]{cmunrm.ttf}\setmonofont[Path=/usr/share/fonts/truetype/cmu/,UprightFont=cmuntt.ttf,BoldFont=cmuntb.ttf,ItalicFont=cmunit.ttf,BoldItalicFont=cmuntx.ttf]{cmunrm.ttf} the same effect as writing

\begin{Shaded}
\begin{Highlighting}[]

\NormalTok{\textbackslash{}input\{filename.tex\}}\newline
\end{Highlighting}
\end{Shaded}

but if you found it more convenient to put all your files in a sub-{}directory of your current directory, called {\ttfamily \setmainfont[Path=/usr/share/fonts/truetype/cmu/,UprightFont=cmunrm.ttf,BoldFont=cmunbx.ttf,ItalicFont=cmunti.ttf,BoldItalicFont=cmunbi.ttf]{cmuntt.ttf}\setmonofont[Path=/usr/share/fonts/truetype/cmu/,UprightFont=cmuntt.ttf,BoldFont=cmuntb.ttf,ItalicFont=cmunit.ttf,BoldItalicFont=cmuntx.ttf]{cmuntt.ttf}\ttfamily myfiles}\setmainfont[Path=/usr/share/fonts/truetype/cmu/,UprightFont=cmunrm.ttf,BoldFont=cmunbx.ttf,ItalicFont=cmunti.ttf,BoldItalicFont=cmunbi.ttf]{cmunrm.ttf}\setmonofont[Path=/usr/share/fonts/truetype/cmu/,UprightFont=cmuntt.ttf,BoldFont=cmuntb.ttf,ItalicFont=cmunit.ttf,BoldItalicFont=cmuntx.ttf]{cmunrm.ttf}, you would refer to that file by specifying

\begin{Shaded}
\begin{Highlighting}[]

\NormalTok{\textbackslash{}input\{./myfiles/filename.tex\}}\newline
\end{Highlighting}
\end{Shaded}

Indeed, in our example of the absolute path above, you could refer to that file relatively, too:

\begin{Shaded}
\begin{Highlighting}[]

\NormalTok{\textbackslash{}input\{../../documents/useful/foo.tex\}}\newline
\end{Highlighting}
\end{Shaded}

Of course, all commonly used file systems – Linux, Mac OS X and Windows – also feature the UNIX {\bfseries \setmainfont[Path=/usr/share/fonts/truetype/cmu/,UprightFont=cmunrm.ttf,BoldFont=cmunbx.ttf,ItalicFont=cmunti.ttf,BoldItalicFont=cmunbi.ttf]{cmunbx.ttf}\setmonofont[Path=/usr/share/fonts/truetype/cmu/,UprightFont=cmuntt.ttf,BoldFont=cmuntb.ttf,ItalicFont=cmunit.ttf,BoldItalicFont=cmuntx.ttf]{cmunbx.ttf}\bfseries ./}\setmainfont[Path=/usr/share/fonts/truetype/cmu/,UprightFont=cmunrm.ttf,BoldFont=cmunbx.ttf,ItalicFont=cmunti.ttf,BoldItalicFont=cmunbi.ttf]{cmunrm.ttf}\setmonofont[Path=/usr/share/fonts/truetype/cmu/,UprightFont=cmuntt.ttf,BoldFont=cmuntb.ttf,ItalicFont=cmunit.ttf,BoldItalicFont=cmuntx.ttf]{cmunrm.ttf}, {\bfseries \setmainfont[Path=/usr/share/fonts/truetype/cmu/,UprightFont=cmunrm.ttf,BoldFont=cmunbx.ttf,ItalicFont=cmunti.ttf,BoldItalicFont=cmunbi.ttf]{cmunbx.ttf}\setmonofont[Path=/usr/share/fonts/truetype/cmu/,UprightFont=cmuntt.ttf,BoldFont=cmuntb.ttf,ItalicFont=cmunit.ttf,BoldItalicFont=cmuntx.ttf]{cmunbx.ttf}\bfseries ../}{$\text{ }$}\setmainfont[Path=/usr/share/fonts/truetype/cmu/,UprightFont=cmunrm.ttf,BoldFont=cmunbx.ttf,ItalicFont=cmunti.ttf,BoldItalicFont=cmunbi.ttf]{cmunrm.ttf}\setmonofont[Path=/usr/share/fonts/truetype/cmu/,UprightFont=cmuntt.ttf,BoldFont=cmuntb.ttf,ItalicFont=cmunit.ttf,BoldItalicFont=cmuntx.ttf]{cmunrm.ttf} facility outlined above. Do note, however, that LaTeX uses forward slashes {\bfseries \setmainfont[Path=/usr/share/fonts/truetype/cmu/,UprightFont=cmunrm.ttf,BoldFont=cmunbx.ttf,ItalicFont=cmunti.ttf,BoldItalicFont=cmunbi.ttf]{cmunbx.ttf}\setmonofont[Path=/usr/share/fonts/truetype/cmu/,UprightFont=cmuntt.ttf,BoldFont=cmuntb.ttf,ItalicFont=cmunit.ttf,BoldItalicFont=cmuntx.ttf]{cmunbx.ttf}\bfseries /}{$\text{ }$}\setmainfont[Path=/usr/share/fonts/truetype/cmu/,UprightFont=cmunrm.ttf,BoldFont=cmunbx.ttf,ItalicFont=cmunti.ttf,BoldItalicFont=cmunbi.ttf]{cmunrm.ttf}\setmonofont[Path=/usr/share/fonts/truetype/cmu/,UprightFont=cmuntt.ttf,BoldFont=cmuntb.ttf,ItalicFont=cmunit.ttf,BoldItalicFont=cmuntx.ttf]{cmunrm.ttf} even on Microsoft Windows platforms, which use backslashes {\bfseries \setmainfont[Path=/usr/share/fonts/truetype/cmu/,UprightFont=cmunrm.ttf,BoldFont=cmunbx.ttf,ItalicFont=cmunti.ttf,BoldItalicFont=cmunbi.ttf]{cmunbx.ttf}\setmonofont[Path=/usr/share/fonts/truetype/cmu/,UprightFont=cmuntt.ttf,BoldFont=cmuntb.ttf,ItalicFont=cmunit.ttf,BoldItalicFont=cmuntx.ttf]{cmunbx.ttf}\bfseries \textbackslash{}}{$\text{ }$}\setmainfont[Path=/usr/share/fonts/truetype/cmu/,UprightFont=cmunrm.ttf,BoldFont=cmunbx.ttf,ItalicFont=cmunti.ttf,BoldItalicFont=cmunbi.ttf]{cmunrm.ttf}\setmonofont[Path=/usr/share/fonts/truetype/cmu/,UprightFont=cmuntt.ttf,BoldFont=cmuntb.ttf,ItalicFont=cmunit.ttf,BoldItalicFont=cmuntx.ttf]{cmunrm.ttf} in pathnames. LaTeX implementations for Windows systems perform this conversion for you, which ensures that your document will be valid across all installations.

This flexibility, inherent in the way in which LaTeX is integrated with modern file systems, lets you input files in a way which suits your particular set-{}up.

When using relative paths within a LaTeX file imported by {\ttfamily \setmainfont[Path=/usr/share/fonts/truetype/cmu/,UprightFont=cmunrm.ttf,BoldFont=cmunbx.ttf,ItalicFont=cmunti.ttf,BoldItalicFont=cmunbi.ttf]{cmuntt.ttf}\setmonofont[Path=/usr/share/fonts/truetype/cmu/,UprightFont=cmuntt.ttf,BoldFont=cmuntb.ttf,ItalicFont=cmunit.ttf,BoldItalicFont=cmuntx.ttf]{cmuntt.ttf}\ttfamily \textbackslash{}input}{$\text{ }$}\setmainfont[Path=/usr/share/fonts/truetype/cmu/,UprightFont=cmunrm.ttf,BoldFont=cmunbx.ttf,ItalicFont=cmunti.ttf,BoldItalicFont=cmunbi.ttf]{cmunrm.ttf}\setmonofont[Path=/usr/share/fonts/truetype/cmu/,UprightFont=cmuntt.ttf,BoldFont=cmuntb.ttf,ItalicFont=cmunit.ttf,BoldItalicFont=cmuntx.ttf]{cmunrm.ttf} or {\ttfamily \setmainfont[Path=/usr/share/fonts/truetype/cmu/,UprightFont=cmunrm.ttf,BoldFont=cmunbx.ttf,ItalicFont=cmunti.ttf,BoldItalicFont=cmunbi.ttf]{cmuntt.ttf}\setmonofont[Path=/usr/share/fonts/truetype/cmu/,UprightFont=cmuntt.ttf,BoldFont=cmuntb.ttf,ItalicFont=cmunit.ttf,BoldItalicFont=cmuntx.ttf]{cmuntt.ttf}\ttfamily \textbackslash{}include}\setmainfont[Path=/usr/share/fonts/truetype/cmu/,UprightFont=cmunrm.ttf,BoldFont=cmunbx.ttf,ItalicFont=cmunti.ttf,BoldItalicFont=cmunbi.ttf]{cmunrm.ttf}\setmonofont[Path=/usr/share/fonts/truetype/cmu/,UprightFont=cmuntt.ttf,BoldFont=cmuntb.ttf,ItalicFont=cmunit.ttf,BoldItalicFont=cmuntx.ttf]{cmunrm.ttf}, it is important to note that the paths are relative to the directory in which the main .tex file resides, not to the directory in which the included (or input) file is found.  This is likely to be an issue if using a folder per chapter, with the figures in each chapter\textquotesingle{}s folder, and using \textbackslash{}include to read the chapter source into a main LaTeX file in a parent folder.
\subsection{Compiling the base file}
\label{900}

When you compile your document, page references and the like will change according to your use of the {\ttfamily \setmainfont[Path=/usr/share/fonts/truetype/cmu/,UprightFont=cmunrm.ttf,BoldFont=cmunbx.ttf,ItalicFont=cmunti.ttf,BoldItalicFont=cmunbi.ttf]{cmuntt.ttf}\setmonofont[Path=/usr/share/fonts/truetype/cmu/,UprightFont=cmuntt.ttf,BoldFont=cmuntb.ttf,ItalicFont=cmunit.ttf,BoldItalicFont=cmuntx.ttf]{cmuntt.ttf}\ttfamily \textbackslash{}input}{$\text{ }$}\setmainfont[Path=/usr/share/fonts/truetype/cmu/,UprightFont=cmunrm.ttf,BoldFont=cmunbx.ttf,ItalicFont=cmunti.ttf,BoldItalicFont=cmunbi.ttf]{cmunrm.ttf}\setmonofont[Path=/usr/share/fonts/truetype/cmu/,UprightFont=cmuntt.ttf,BoldFont=cmuntb.ttf,ItalicFont=cmunit.ttf,BoldItalicFont=cmuntx.ttf]{cmunrm.ttf} and {\ttfamily \setmainfont[Path=/usr/share/fonts/truetype/cmu/,UprightFont=cmunrm.ttf,BoldFont=cmunbx.ttf,ItalicFont=cmunti.ttf,BoldItalicFont=cmunbi.ttf]{cmuntt.ttf}\setmonofont[Path=/usr/share/fonts/truetype/cmu/,UprightFont=cmuntt.ttf,BoldFont=cmuntb.ttf,ItalicFont=cmunit.ttf,BoldItalicFont=cmuntx.ttf]{cmuntt.ttf}\ttfamily \textbackslash{}include}{$\text{ }$}\setmainfont[Path=/usr/share/fonts/truetype/cmu/,UprightFont=cmunrm.ttf,BoldFont=cmunbx.ttf,ItalicFont=cmunti.ttf,BoldItalicFont=cmunbi.ttf]{cmunrm.ttf}\setmonofont[Path=/usr/share/fonts/truetype/cmu/,UprightFont=cmuntt.ttf,BoldFont=cmuntb.ttf,ItalicFont=cmunit.ttf,BoldItalicFont=cmuntx.ttf]{cmunrm.ttf} commands. Normally LaTeX users only run the compiler on parts of the document to check that an individual chapter is syntactically correct and looks as the writer intended. A full run is generally only performed for producing a full draft or the final version. In such cases, it is invariably necessary to run LaTeX twice or more to resolve all the page numbers, references, etc. (especially if you are using bibliographic software such as BiBTeX, too).

The simplest way to check that one or more of the various components of your work is syntactically robust, is to comment out the command with a percentage sign, for example:


\begin{Shaded}
\begin{Highlighting}[]

\NormalTok{\textbackslash{}documentclass\{article\}}\newline
\NormalTok{\textbackslash{}begin\{document\}}\newline
\CommentTok{\%\textbackslash{}input\{Section_1\}}\newline
\CommentTok{\%\textbackslash{}input\{Section_2\}}\newline
\CommentTok{\%\textbackslash{}input\{Section_3\}}\newline
\NormalTok{\textbackslash{}input\{Section_4\}}\newline
\CommentTok{\%\textbackslash{}input\{Section_5\}}\newline
\NormalTok{\textbackslash{}end\{document\}}\newline
\end{Highlighting}
\end{Shaded}

This code will process your base file with the {\ttfamily \setmainfont[Path=/usr/share/fonts/truetype/cmu/,UprightFont=cmunrm.ttf,BoldFont=cmunbx.ttf,ItalicFont=cmunti.ttf,BoldItalicFont=cmunbi.ttf]{cmuntt.ttf}\setmonofont[Path=/usr/share/fonts/truetype/cmu/,UprightFont=cmuntt.ttf,BoldFont=cmuntb.ttf,ItalicFont=cmunit.ttf,BoldItalicFont=cmuntx.ttf]{cmuntt.ttf}\ttfamily article}{$\text{ }$}\setmainfont[Path=/usr/share/fonts/truetype/cmu/,UprightFont=cmunrm.ttf,BoldFont=cmunbx.ttf,ItalicFont=cmunti.ttf,BoldItalicFont=cmunbi.ttf]{cmunrm.ttf}\setmonofont[Path=/usr/share/fonts/truetype/cmu/,UprightFont=cmuntt.ttf,BoldFont=cmuntb.ttf,ItalicFont=cmunit.ttf,BoldItalicFont=cmuntx.ttf]{cmunrm.ttf} conventions but only the material in the file {\ttfamily \setmainfont[Path=/usr/share/fonts/truetype/cmu/,UprightFont=cmunrm.ttf,BoldFont=cmunbx.ttf,ItalicFont=cmunti.ttf,BoldItalicFont=cmunbi.ttf]{cmuntt.ttf}\setmonofont[Path=/usr/share/fonts/truetype/cmu/,UprightFont=cmuntt.ttf,BoldFont=cmuntb.ttf,ItalicFont=cmunit.ttf,BoldItalicFont=cmuntx.ttf]{cmuntt.ttf}\ttfamily Section\_4.tex}{$\text{ }$}\setmainfont[Path=/usr/share/fonts/truetype/cmu/,UprightFont=cmunrm.ttf,BoldFont=cmunbx.ttf,ItalicFont=cmunti.ttf,BoldItalicFont=cmunbi.ttf]{cmunrm.ttf}\setmonofont[Path=/usr/share/fonts/truetype/cmu/,UprightFont=cmuntt.ttf,BoldFont=cmuntb.ttf,ItalicFont=cmunit.ttf,BoldItalicFont=cmuntx.ttf]{cmunrm.ttf} will be processed. If that was, say, the last thing you needed to check before sending off to that major journal, you would then simply remove all the percentage signs and rerun LaTeX, repeating the compiling process as necessary to resolve all references, page numbers and so on.
\subsection{Using \textbackslash{}includeonly}
\label{901}

Using this command provides more complex, and hence more useful possibilities. If you include the following command in your preamble, i.e. before \textbackslash{}begin\{document\},

\begin{Shaded}
\begin{Highlighting}[]

\NormalTok{\textbackslash{}includeonly\{filename1,filename2,...\}}\newline
\end{Highlighting}
\end{Shaded}

only the files specified between the curly braces will be included. Note that you can have one or more files as the argument to this command: separate them with a comma, no spaces. If you are using absolute or relative paths to the files, type in the complete reference.

This requires that there are {\ttfamily \setmainfont[Path=/usr/share/fonts/truetype/cmu/,UprightFont=cmunrm.ttf,BoldFont=cmunbx.ttf,ItalicFont=cmunti.ttf,BoldItalicFont=cmunbi.ttf]{cmuntt.ttf}\setmonofont[Path=/usr/share/fonts/truetype/cmu/,UprightFont=cmuntt.ttf,BoldFont=cmuntb.ttf,ItalicFont=cmunit.ttf,BoldItalicFont=cmuntx.ttf]{cmuntt.ttf}\ttfamily \textbackslash{}include}{$\text{ }$}\setmainfont[Path=/usr/share/fonts/truetype/cmu/,UprightFont=cmunrm.ttf,BoldFont=cmunbx.ttf,ItalicFont=cmunti.ttf,BoldItalicFont=cmunbi.ttf]{cmunrm.ttf}\setmonofont[Path=/usr/share/fonts/truetype/cmu/,UprightFont=cmuntt.ttf,BoldFont=cmuntb.ttf,ItalicFont=cmunit.ttf,BoldItalicFont=cmuntx.ttf]{cmunrm.ttf} commands in the document which specify these files. The filename should be written without the {\ttfamily \setmainfont[Path=/usr/share/fonts/truetype/cmu/,UprightFont=cmunrm.ttf,BoldFont=cmunbx.ttf,ItalicFont=cmunti.ttf,BoldItalicFont=cmunbi.ttf]{cmuntt.ttf}\setmonofont[Path=/usr/share/fonts/truetype/cmu/,UprightFont=cmuntt.ttf,BoldFont=cmuntb.ttf,ItalicFont=cmunit.ttf,BoldItalicFont=cmuntx.ttf]{cmuntt.ttf}\ttfamily .tex}{$\text{ }$}\setmainfont[Path=/usr/share/fonts/truetype/cmu/,UprightFont=cmunrm.ttf,BoldFont=cmunbx.ttf,ItalicFont=cmunti.ttf,BoldItalicFont=cmunbi.ttf]{cmunrm.ttf}\setmonofont[Path=/usr/share/fonts/truetype/cmu/,UprightFont=cmuntt.ttf,BoldFont=cmuntb.ttf,ItalicFont=cmunit.ttf,BoldItalicFont=cmuntx.ttf]{cmunrm.ttf} file extension:


\begin{Shaded}
\begin{Highlighting}[]

\NormalTok{\textbackslash{}documentclass\{book\}}\newline
\NormalTok{\textbackslash{}includeonly\{Chapter_1,Chapter_4\}\ensuremath{\text{ }}\ensuremath{\text{ }}}\CommentTok{\%\ensuremath{\text{ }}compile\ensuremath{\text{ }}just\ensuremath{\text{ }}chapters\ensuremath{\text{ }}1\ensuremath{\text{ }}and\ensuremath{\text{ }}4,\ensuremath{\text{ }}space}\newline
\ensuremath{\text{ }}\NormalTok{characters\ensuremath{\text{ }}not\ensuremath{\text{ }}permitted}\newline
\NormalTok{\textbackslash{}begin\{document\}}\newline
\NormalTok{\textbackslash{}include\{Chapter_1\}\ensuremath{\text{ }}\ensuremath{\text{ }}\ensuremath{\text{ }}\ensuremath{\text{ }}\ensuremath{\text{ }}\ensuremath{\text{ }}\ensuremath{\text{ }}\ensuremath{\text{ }}\ensuremath{\text{ }}\ensuremath{\text{ }}\ensuremath{\text{ }}\ensuremath{\text{ }}\ensuremath{\text{ }}\ensuremath{\text{ }}\ensuremath{\text{ }}\ensuremath{\text{ }}}\CommentTok{\%\ensuremath{\text{ }}omit\ensuremath{\text{ }}the\ensuremath{\text{ }}\textquotesingle{}.tex\textquotesingle{}\ensuremath{\text{ }}extension}\newline
\NormalTok{\textbackslash{}include\{Chapter_2\}}\newline
\NormalTok{\textbackslash{}include\{Chapter_3\}}\newline
\NormalTok{\textbackslash{}include\{Chapter_4\}}\newline
\NormalTok{\textbackslash{}end\{document\}}\newline
\end{Highlighting}
\end{Shaded}


This code would process the base file but only include the content of the author\textquotesingle{}s first and fourth chapters ({\ttfamily \setmainfont[Path=/usr/share/fonts/truetype/cmu/,UprightFont=cmunrm.ttf,BoldFont=cmunbx.ttf,ItalicFont=cmunti.ttf,BoldItalicFont=cmunbi.ttf]{cmuntt.ttf}\setmonofont[Path=/usr/share/fonts/truetype/cmu/,UprightFont=cmuntt.ttf,BoldFont=cmuntb.ttf,ItalicFont=cmunit.ttf,BoldItalicFont=cmuntx.ttf]{cmuntt.ttf}\ttfamily Chapter\_1.tex}{$\text{ }$}\setmainfont[Path=/usr/share/fonts/truetype/cmu/,UprightFont=cmunrm.ttf,BoldFont=cmunbx.ttf,ItalicFont=cmunti.ttf,BoldItalicFont=cmunbi.ttf]{cmunrm.ttf}\setmonofont[Path=/usr/share/fonts/truetype/cmu/,UprightFont=cmuntt.ttf,BoldFont=cmuntb.ttf,ItalicFont=cmunit.ttf,BoldItalicFont=cmuntx.ttf]{cmunrm.ttf} and {\ttfamily \setmainfont[Path=/usr/share/fonts/truetype/cmu/,UprightFont=cmunrm.ttf,BoldFont=cmunbx.ttf,ItalicFont=cmunti.ttf,BoldItalicFont=cmunbi.ttf]{cmuntt.ttf}\setmonofont[Path=/usr/share/fonts/truetype/cmu/,UprightFont=cmuntt.ttf,BoldFont=cmuntb.ttf,ItalicFont=cmunit.ttf,BoldItalicFont=cmuntx.ttf]{cmuntt.ttf}\ttfamily Chapter\_4.tex}\setmainfont[Path=/usr/share/fonts/truetype/cmu/,UprightFont=cmunrm.ttf,BoldFont=cmunbx.ttf,ItalicFont=cmunti.ttf,BoldItalicFont=cmunbi.ttf]{cmunrm.ttf}\setmonofont[Path=/usr/share/fonts/truetype/cmu/,UprightFont=cmuntt.ttf,BoldFont=cmuntb.ttf,ItalicFont=cmunit.ttf,BoldItalicFont=cmuntx.ttf]{cmunrm.ttf}). Importantly, this alternative retains as much of the {\ttfamily \setmainfont[Path=/usr/share/fonts/truetype/cmu/,UprightFont=cmunrm.ttf,BoldFont=cmunbx.ttf,ItalicFont=cmunti.ttf,BoldItalicFont=cmunbi.ttf]{cmuntt.ttf}\setmonofont[Path=/usr/share/fonts/truetype/cmu/,UprightFont=cmuntt.ttf,BoldFont=cmuntb.ttf,ItalicFont=cmunit.ttf,BoldItalicFont=cmuntx.ttf]{cmuntt.ttf}\ttfamily .aux}{$\text{ }$}\setmainfont[Path=/usr/share/fonts/truetype/cmu/,UprightFont=cmunrm.ttf,BoldFont=cmunbx.ttf,ItalicFont=cmunti.ttf,BoldItalicFont=cmunbi.ttf]{cmunrm.ttf}\setmonofont[Path=/usr/share/fonts/truetype/cmu/,UprightFont=cmuntt.ttf,BoldFont=cmuntb.ttf,ItalicFont=cmunit.ttf,BoldItalicFont=cmuntx.ttf]{cmunrm.ttf} information as possible from the previous run, so messes up your cross-{}references much less than the makeshift suggestion above.
\subsection{Separate compilation of child documents}
\label{902}
A disadvantage of solely using {\ttfamily \setmainfont[Path=/usr/share/fonts/truetype/cmu/,UprightFont=cmunrm.ttf,BoldFont=cmunbx.ttf,ItalicFont=cmunti.ttf,BoldItalicFont=cmunbi.ttf]{cmuntt.ttf}\setmonofont[Path=/usr/share/fonts/truetype/cmu/,UprightFont=cmuntt.ttf,BoldFont=cmuntb.ttf,ItalicFont=cmunit.ttf,BoldItalicFont=cmuntx.ttf]{cmuntt.ttf}\ttfamily \textbackslash{}input}{$\text{ }$}\setmainfont[Path=/usr/share/fonts/truetype/cmu/,UprightFont=cmunrm.ttf,BoldFont=cmunbx.ttf,ItalicFont=cmunti.ttf,BoldItalicFont=cmunbi.ttf]{cmunrm.ttf}\setmonofont[Path=/usr/share/fonts/truetype/cmu/,UprightFont=cmuntt.ttf,BoldFont=cmuntb.ttf,ItalicFont=cmunit.ttf,BoldItalicFont=cmuntx.ttf]{cmunrm.ttf} and {\ttfamily \setmainfont[Path=/usr/share/fonts/truetype/cmu/,UprightFont=cmunrm.ttf,BoldFont=cmunbx.ttf,ItalicFont=cmunti.ttf,BoldItalicFont=cmunbi.ttf]{cmuntt.ttf}\setmonofont[Path=/usr/share/fonts/truetype/cmu/,UprightFont=cmuntt.ttf,BoldFont=cmuntb.ttf,ItalicFont=cmunit.ttf,BoldItalicFont=cmuntx.ttf]{cmuntt.ttf}\ttfamily \textbackslash{}include}{$\text{ }$}\setmainfont[Path=/usr/share/fonts/truetype/cmu/,UprightFont=cmunrm.ttf,BoldFont=cmunbx.ttf,ItalicFont=cmunti.ttf,BoldItalicFont=cmunbi.ttf]{cmunrm.ttf}\setmonofont[Path=/usr/share/fonts/truetype/cmu/,UprightFont=cmuntt.ttf,BoldFont=cmuntb.ttf,ItalicFont=cmunit.ttf,BoldItalicFont=cmuntx.ttf]{cmunrm.ttf} is that only the base document can be compiled. However, you may decide that you work better on individual sections of text and wish to edit and compile those separate from the main file. There are a few packages available to address this problem.
\subsubsection{Subfiles}
\label{903}
The \myhref{http://www.ctan.org/pkg/subfiles}{subfiles package} provides a way to compile sections of a document using the same preamble as the main document.

In the main document, the package must be loaded as:

\begin{Shaded}
\begin{Highlighting}[]

\NormalTok{\textbackslash{}usepackage\{subfiles\}}\newline
\end{Highlighting}
\end{Shaded}


Instead of using {\ttfamily \setmainfont[Path=/usr/share/fonts/truetype/cmu/,UprightFont=cmunrm.ttf,BoldFont=cmunbx.ttf,ItalicFont=cmunti.ttf,BoldItalicFont=cmunbi.ttf]{cmuntt.ttf}\setmonofont[Path=/usr/share/fonts/truetype/cmu/,UprightFont=cmuntt.ttf,BoldFont=cmuntb.ttf,ItalicFont=cmunit.ttf,BoldItalicFont=cmuntx.ttf]{cmuntt.ttf}\ttfamily \textbackslash{}input}{$\text{ }$}\setmainfont[Path=/usr/share/fonts/truetype/cmu/,UprightFont=cmunrm.ttf,BoldFont=cmunbx.ttf,ItalicFont=cmunti.ttf,BoldItalicFont=cmunbi.ttf]{cmunrm.ttf}\setmonofont[Path=/usr/share/fonts/truetype/cmu/,UprightFont=cmuntt.ttf,BoldFont=cmuntb.ttf,ItalicFont=cmunit.ttf,BoldItalicFont=cmuntx.ttf]{cmunrm.ttf} and {\ttfamily \setmainfont[Path=/usr/share/fonts/truetype/cmu/,UprightFont=cmunrm.ttf,BoldFont=cmunbx.ttf,ItalicFont=cmunti.ttf,BoldItalicFont=cmunbi.ttf]{cmuntt.ttf}\setmonofont[Path=/usr/share/fonts/truetype/cmu/,UprightFont=cmuntt.ttf,BoldFont=cmuntb.ttf,ItalicFont=cmunit.ttf,BoldItalicFont=cmuntx.ttf]{cmuntt.ttf}\ttfamily \textbackslash{}include}\setmainfont[Path=/usr/share/fonts/truetype/cmu/,UprightFont=cmunrm.ttf,BoldFont=cmunbx.ttf,ItalicFont=cmunti.ttf,BoldItalicFont=cmunbi.ttf]{cmunrm.ttf}\setmonofont[Path=/usr/share/fonts/truetype/cmu/,UprightFont=cmuntt.ttf,BoldFont=cmuntb.ttf,ItalicFont=cmunit.ttf,BoldItalicFont=cmuntx.ttf]{cmunrm.ttf}, child documents must be loaded as follows:

\begin{Shaded}
\begin{Highlighting}[]

\NormalTok{\textbackslash{}subfile\{filename\}}\newline
\end{Highlighting}
\end{Shaded}


The child documents must start with the following statements:

\begin{Shaded}
\begin{Highlighting}[]

\NormalTok{\textbackslash{}documentclass[main.tex]\{subfiles\}}\newline
\NormalTok{\textbackslash{}begin\{document\}}\newline
\end{Highlighting}
\end{Shaded}

and end with:

\begin{Shaded}
\begin{Highlighting}[]

\NormalTok{\textbackslash{}end\{document\}}\newline
\end{Highlighting}
\end{Shaded}


It is possible to add parts that will only be applied if the child document is compiled by its own, by defining an \symbol{34}identity\symbol{34} command {\ttfamily \setmainfont[Path=/usr/share/fonts/truetype/cmu/,UprightFont=cmunrm.ttf,BoldFont=cmunbx.ttf,ItalicFont=cmunti.ttf,BoldItalicFont=cmunbi.ttf]{cmuntt.ttf}\setmonofont[Path=/usr/share/fonts/truetype/cmu/,UprightFont=cmuntt.ttf,BoldFont=cmuntb.ttf,ItalicFont=cmunit.ttf,BoldItalicFont=cmuntx.ttf]{cmuntt.ttf}\ttfamily \textbackslash{}newcommand\{\textbackslash{}onlyinsubfile\}{$\text{[}$}1{$\text{]}$}\{\#1\}}{$\text{ }$}\setmainfont[Path=/usr/share/fonts/truetype/cmu/,UprightFont=cmunrm.ttf,BoldFont=cmunbx.ttf,ItalicFont=cmunti.ttf,BoldItalicFont=cmunbi.ttf]{cmunrm.ttf}\setmonofont[Path=/usr/share/fonts/truetype/cmu/,UprightFont=cmuntt.ttf,BoldFont=cmuntb.ttf,ItalicFont=cmunit.ttf,BoldItalicFont=cmuntx.ttf]{cmunrm.ttf} in the main document and then overwriting it after {\ttfamily \setmainfont[Path=/usr/share/fonts/truetype/cmu/,UprightFont=cmunrm.ttf,BoldFont=cmunbx.ttf,ItalicFont=cmunti.ttf,BoldItalicFont=cmunbi.ttf]{cmuntt.ttf}\setmonofont[Path=/usr/share/fonts/truetype/cmu/,UprightFont=cmuntt.ttf,BoldFont=cmuntb.ttf,ItalicFont=cmunit.ttf,BoldItalicFont=cmuntx.ttf]{cmuntt.ttf}\ttfamily \textbackslash{}begin\{document\}}{$\text{ }$}\setmainfont[Path=/usr/share/fonts/truetype/cmu/,UprightFont=cmunrm.ttf,BoldFont=cmunbx.ttf,ItalicFont=cmunti.ttf,BoldItalicFont=cmunbi.ttf]{cmunrm.ttf}\setmonofont[Path=/usr/share/fonts/truetype/cmu/,UprightFont=cmuntt.ttf,BoldFont=cmuntb.ttf,ItalicFont=cmunit.ttf,BoldItalicFont=cmuntx.ttf]{cmunrm.ttf} using {\ttfamily \setmainfont[Path=/usr/share/fonts/truetype/cmu/,UprightFont=cmunrm.ttf,BoldFont=cmunbx.ttf,ItalicFont=cmunti.ttf,BoldItalicFont=cmunbi.ttf]{cmuntt.ttf}\setmonofont[Path=/usr/share/fonts/truetype/cmu/,UprightFont=cmuntt.ttf,BoldFont=cmuntb.ttf,ItalicFont=cmunit.ttf,BoldItalicFont=cmuntx.ttf]{cmuntt.ttf}\ttfamily \textbackslash{}renewcommand\{\textbackslash{}onlyinsubfile\}{$\text{[}$}1{$\text{]}$}\{\}}\setmainfont[Path=/usr/share/fonts/truetype/cmu/,UprightFont=cmunrm.ttf,BoldFont=cmunbx.ttf,ItalicFont=cmunti.ttf,BoldItalicFont=cmunbi.ttf]{cmunrm.ttf}\setmonofont[Path=/usr/share/fonts/truetype/cmu/,UprightFont=cmuntt.ttf,BoldFont=cmuntb.ttf,ItalicFont=cmunit.ttf,BoldItalicFont=cmuntx.ttf]{cmunrm.ttf}. Similarly, the same can be done for parts to appear only if compiled by the main document.

In summary, the base document ({\ttfamily \setmainfont[Path=/usr/share/fonts/truetype/cmu/,UprightFont=cmunrm.ttf,BoldFont=cmunbx.ttf,ItalicFont=cmunti.ttf,BoldItalicFont=cmunbi.ttf]{cmuntt.ttf}\setmonofont[Path=/usr/share/fonts/truetype/cmu/,UprightFont=cmuntt.ttf,BoldFont=cmuntb.ttf,ItalicFont=cmunit.ttf,BoldItalicFont=cmuntx.ttf]{cmuntt.ttf}\ttfamily main.tex}\setmainfont[Path=/usr/share/fonts/truetype/cmu/,UprightFont=cmunrm.ttf,BoldFont=cmunbx.ttf,ItalicFont=cmunti.ttf,BoldItalicFont=cmunbi.ttf]{cmunrm.ttf}\setmonofont[Path=/usr/share/fonts/truetype/cmu/,UprightFont=cmuntt.ttf,BoldFont=cmuntb.ttf,ItalicFont=cmunit.ttf,BoldItalicFont=cmuntx.ttf]{cmunrm.ttf}) looks like:

\begin{Shaded}
\begin{Highlighting}[]

\NormalTok{\textbackslash{}documentclass\{book\}}\newline
\NormalTok{\textbackslash{}usepackage\{subfiles\}}\newline
\NormalTok{\textbackslash{}newcommand\{\textbackslash{}onlyinsubfile\}[1]\{#1\}}\newline
\NormalTok{\textbackslash{}newcommand\{\textbackslash{}notinsubfile\}[1]\{\}}\newline
\ensuremath{\text{ }}\newline
\NormalTok{\textbackslash{}begin\{document\}}\newline
\NormalTok{\textbackslash{}renewcommand\{\textbackslash{}onlyinsubfile\}[1]\{\}}\newline
\NormalTok{\textbackslash{}renewcommand\{\textbackslash{}notinsubfile\}[1]\{#1\}}\newline
\CommentTok{\%\%\ensuremath{\text{ }}my\ensuremath{\text{ }}document\ensuremath{\text{ }}content}\newline
\NormalTok{\textbackslash{}subfile\{chapter1\}}\newline
\CommentTok{\%\%\ensuremath{\text{ }}more\ensuremath{\text{ }}of\ensuremath{\text{ }}my\ensuremath{\text{ }}document\ensuremath{\text{ }}content}\newline
\NormalTok{\textbackslash{}end\{document\}}\newline
\end{Highlighting}
\end{Shaded}


and Chapter 1 ({\ttfamily \setmainfont[Path=/usr/share/fonts/truetype/cmu/,UprightFont=cmunrm.ttf,BoldFont=cmunbx.ttf,ItalicFont=cmunti.ttf,BoldItalicFont=cmunbi.ttf]{cmuntt.ttf}\setmonofont[Path=/usr/share/fonts/truetype/cmu/,UprightFont=cmuntt.ttf,BoldFont=cmuntb.ttf,ItalicFont=cmunit.ttf,BoldItalicFont=cmuntx.ttf]{cmuntt.ttf}\ttfamily chapter1.tex}\setmainfont[Path=/usr/share/fonts/truetype/cmu/,UprightFont=cmunrm.ttf,BoldFont=cmunbx.ttf,ItalicFont=cmunti.ttf,BoldItalicFont=cmunbi.ttf]{cmunrm.ttf}\setmonofont[Path=/usr/share/fonts/truetype/cmu/,UprightFont=cmuntt.ttf,BoldFont=cmuntb.ttf,ItalicFont=cmunit.ttf,BoldItalicFont=cmuntx.ttf]{cmunrm.ttf}) looks like:


\begin{Shaded}
\begin{Highlighting}[]

\NormalTok{\textbackslash{}documentclass[main.tex]\{subfiles\}}\newline
\NormalTok{\textbackslash{}begin\{document\}}\newline
\CommentTok{\%\%\ensuremath{\text{ }}my\ensuremath{\text{ }}chapter\ensuremath{\text{ }}1\ensuremath{\text{ }}content}\newline
\NormalTok{\textbackslash{}onlyinsubfile\{this\ensuremath{\text{ }}only\ensuremath{\text{ }}appears\ensuremath{\text{ }}if\ensuremath{\text{ }}chapter1.tex\ensuremath{\text{ }}is\ensuremath{\text{ }}compiled\ensuremath{\text{ }}(not\ensuremath{\text{ }}when\ensuremath{\text{ }}main.tex}\newline
\ensuremath{\text{ }}\NormalTok{is\ensuremath{\text{ }}compiled)\}}\newline
\NormalTok{\textbackslash{}notinsubfile\{this\ensuremath{\text{ }}only\ensuremath{\text{ }}appears\ensuremath{\text{ }}if\ensuremath{\text{ }}main.tex\ensuremath{\text{ }}is\ensuremath{\text{ }}compiled\ensuremath{\text{ }}(not\ensuremath{\text{ }}when\ensuremath{\text{ }}chapter1.tex}\newline
\ensuremath{\text{ }}\NormalTok{is\ensuremath{\text{ }}compiled)\}}\newline
\CommentTok{\%\%\ensuremath{\text{ }}more\ensuremath{\text{ }}of\ensuremath{\text{ }}my\ensuremath{\text{ }}chapter\ensuremath{\text{ }}1\ensuremath{\text{ }}content}\newline
\CommentTok{\%\%\ensuremath{\text{ }}}\newline
\NormalTok{\textbackslash{}end\{document\}}\newline
\end{Highlighting}
\end{Shaded}


Some linux distributions don\textquotesingle{}t have subfiles package in their latex distributions, since it was not included until TeXLive 2012. You can download \myhref{http://mirrors.ctan.org/install/macros/latex/contrib/subfiles.tds.zip}{subfiles.tds.zip} from CTAN. This package will contain two files {\ttfamily \setmainfont[Path=/usr/share/fonts/truetype/cmu/,UprightFont=cmunrm.ttf,BoldFont=cmunbx.ttf,ItalicFont=cmunti.ttf,BoldItalicFont=cmunbi.ttf]{cmuntt.ttf}\setmonofont[Path=/usr/share/fonts/truetype/cmu/,UprightFont=cmuntt.ttf,BoldFont=cmuntb.ttf,ItalicFont=cmunit.ttf,BoldItalicFont=cmuntx.ttf]{cmuntt.ttf}\ttfamily subfiles.cls}{$\text{ }$}\setmainfont[Path=/usr/share/fonts/truetype/cmu/,UprightFont=cmunrm.ttf,BoldFont=cmunbx.ttf,ItalicFont=cmunti.ttf,BoldItalicFont=cmunbi.ttf]{cmunrm.ttf}\setmonofont[Path=/usr/share/fonts/truetype/cmu/,UprightFont=cmuntt.ttf,BoldFont=cmuntb.ttf,ItalicFont=cmunit.ttf,BoldItalicFont=cmuntx.ttf]{cmunrm.ttf} and {\ttfamily \setmainfont[Path=/usr/share/fonts/truetype/cmu/,UprightFont=cmunrm.ttf,BoldFont=cmunbx.ttf,ItalicFont=cmunti.ttf,BoldItalicFont=cmunbi.ttf]{cmuntt.ttf}\setmonofont[Path=/usr/share/fonts/truetype/cmu/,UprightFont=cmuntt.ttf,BoldFont=cmuntb.ttf,ItalicFont=cmunit.ttf,BoldItalicFont=cmuntx.ttf]{cmuntt.ttf}\ttfamily subfiles.sty}\setmainfont[Path=/usr/share/fonts/truetype/cmu/,UprightFont=cmunrm.ttf,BoldFont=cmunbx.ttf,ItalicFont=cmunti.ttf,BoldItalicFont=cmunbi.ttf]{cmunrm.ttf}\setmonofont[Path=/usr/share/fonts/truetype/cmu/,UprightFont=cmuntt.ttf,BoldFont=cmuntb.ttf,ItalicFont=cmunit.ttf,BoldItalicFont=cmuntx.ttf]{cmunrm.ttf}. Move these files to a directory under the name {\ttfamily \setmainfont[Path=/usr/share/fonts/truetype/cmu/,UprightFont=cmunrm.ttf,BoldFont=cmunbx.ttf,ItalicFont=cmunti.ttf,BoldItalicFont=cmunbi.ttf]{cmuntt.ttf}\setmonofont[Path=/usr/share/fonts/truetype/cmu/,UprightFont=cmuntt.ttf,BoldFont=cmuntb.ttf,ItalicFont=cmunit.ttf,BoldItalicFont=cmuntx.ttf]{cmuntt.ttf}\ttfamily subfiles}{$\text{ }$}\setmainfont[Path=/usr/share/fonts/truetype/cmu/,UprightFont=cmunrm.ttf,BoldFont=cmunbx.ttf,ItalicFont=cmunti.ttf,BoldItalicFont=cmunbi.ttf]{cmunrm.ttf}\setmonofont[Path=/usr/share/fonts/truetype/cmu/,UprightFont=cmuntt.ttf,BoldFont=cmuntb.ttf,ItalicFont=cmunit.ttf,BoldItalicFont=cmuntx.ttf]{cmunrm.ttf} in the path {\ttfamily \setmainfont[Path=/usr/share/fonts/truetype/cmu/,UprightFont=cmunrm.ttf,BoldFont=cmunbx.ttf,ItalicFont=cmunti.ttf,BoldItalicFont=cmunbi.ttf]{cmuntt.ttf}\setmonofont[Path=/usr/share/fonts/truetype/cmu/,UprightFont=cmuntt.ttf,BoldFont=cmuntb.ttf,ItalicFont=cmunit.ttf,BoldItalicFont=cmuntx.ttf]{cmuntt.ttf}\ttfamily /usr/share/texmf/tex/latex}\setmainfont[Path=/usr/share/fonts/truetype/cmu/,UprightFont=cmunrm.ttf,BoldFont=cmunbx.ttf,ItalicFont=cmunti.ttf,BoldItalicFont=cmunbi.ttf]{cmunrm.ttf}\setmonofont[Path=/usr/share/fonts/truetype/cmu/,UprightFont=cmuntt.ttf,BoldFont=cmuntb.ttf,ItalicFont=cmunit.ttf,BoldItalicFont=cmuntx.ttf]{cmunrm.ttf}. This still won\textquotesingle{}t make the package available; the {\ttfamily \setmainfont[Path=/usr/share/fonts/truetype/cmu/,UprightFont=cmunrm.ttf,BoldFont=cmunbx.ttf,ItalicFont=cmunti.ttf,BoldItalicFont=cmunbi.ttf]{cmuntt.ttf}\setmonofont[Path=/usr/share/fonts/truetype/cmu/,UprightFont=cmuntt.ttf,BoldFont=cmuntb.ttf,ItalicFont=cmunit.ttf,BoldItalicFont=cmuntx.ttf]{cmuntt.ttf}\ttfamily texhash}{$\text{ }$}\setmainfont[Path=/usr/share/fonts/truetype/cmu/,UprightFont=cmunrm.ttf,BoldFont=cmunbx.ttf,ItalicFont=cmunti.ttf,BoldItalicFont=cmunbi.ttf]{cmunrm.ttf}\setmonofont[Path=/usr/share/fonts/truetype/cmu/,UprightFont=cmuntt.ttf,BoldFont=cmuntb.ttf,ItalicFont=cmunit.ttf,BoldItalicFont=cmuntx.ttf]{cmunrm.ttf} program must be executed first. Now you are good to go!
\subsubsection{Standalone}
\label{904}
The \myhref{http://www.ctan.org/pkg/standalone}{standalone package} is designed for moving more of the opposite direction than subfiles. It provides a means for importing the preamble of child documents into the main document, allowing for a flexible way to include text or images in multiple documents (e.g. an article and a \mylref{729}{presentation}).

In the main document, the package must be loaded as:

\begin{Shaded}
\begin{Highlighting}[]

\NormalTok{\textbackslash{}usepackage\{standalone\}}\newline
\end{Highlighting}
\end{Shaded}


Child documents are loaded using {\ttfamily \setmainfont[Path=/usr/share/fonts/truetype/cmu/,UprightFont=cmunrm.ttf,BoldFont=cmunbx.ttf,ItalicFont=cmunti.ttf,BoldItalicFont=cmunbi.ttf]{cmuntt.ttf}\setmonofont[Path=/usr/share/fonts/truetype/cmu/,UprightFont=cmuntt.ttf,BoldFont=cmuntb.ttf,ItalicFont=cmunit.ttf,BoldItalicFont=cmuntx.ttf]{cmuntt.ttf}\ttfamily \textbackslash{}input}{$\text{ }$}\setmainfont[Path=/usr/share/fonts/truetype/cmu/,UprightFont=cmunrm.ttf,BoldFont=cmunbx.ttf,ItalicFont=cmunti.ttf,BoldItalicFont=cmunbi.ttf]{cmunrm.ttf}\setmonofont[Path=/usr/share/fonts/truetype/cmu/,UprightFont=cmuntt.ttf,BoldFont=cmuntb.ttf,ItalicFont=cmunit.ttf,BoldItalicFont=cmuntx.ttf]{cmunrm.ttf} or {\ttfamily \setmainfont[Path=/usr/share/fonts/truetype/cmu/,UprightFont=cmunrm.ttf,BoldFont=cmunbx.ttf,ItalicFont=cmunti.ttf,BoldItalicFont=cmunbi.ttf]{cmuntt.ttf}\setmonofont[Path=/usr/share/fonts/truetype/cmu/,UprightFont=cmuntt.ttf,BoldFont=cmuntb.ttf,ItalicFont=cmunit.ttf,BoldItalicFont=cmuntx.ttf]{cmuntt.ttf}\ttfamily \textbackslash{}include}\setmainfont[Path=/usr/share/fonts/truetype/cmu/,UprightFont=cmunrm.ttf,BoldFont=cmunbx.ttf,ItalicFont=cmunti.ttf,BoldItalicFont=cmunbi.ttf]{cmunrm.ttf}\setmonofont[Path=/usr/share/fonts/truetype/cmu/,UprightFont=cmuntt.ttf,BoldFont=cmuntb.ttf,ItalicFont=cmunit.ttf,BoldItalicFont=cmuntx.ttf]{cmunrm.ttf}.

The child documents contain, for example, the following statements:

\begin{Shaded}
\begin{Highlighting}[]

\NormalTok{\textbackslash{}documentclass\{standalone\}}\newline
\CommentTok{\%\ensuremath{\text{ }}Load\ensuremath{\text{ }}any\ensuremath{\text{ }}packages\ensuremath{\text{ }}needed\ensuremath{\text{ }}for\ensuremath{\text{ }}this\ensuremath{\text{ }}document}\newline
\NormalTok{\textbackslash{}begin\{document\}}\newline
\CommentTok{\%\ensuremath{\text{ }}Your\ensuremath{\text{ }}document\ensuremath{\text{ }}or\ensuremath{\text{ }}picture}\newline
\NormalTok{\textbackslash{}end\{document\}}\newline
\end{Highlighting}
\end{Shaded}


In summary, the base document ({\ttfamily \setmainfont[Path=/usr/share/fonts/truetype/cmu/,UprightFont=cmunrm.ttf,BoldFont=cmunbx.ttf,ItalicFont=cmunti.ttf,BoldItalicFont=cmunbi.ttf]{cmuntt.ttf}\setmonofont[Path=/usr/share/fonts/truetype/cmu/,UprightFont=cmuntt.ttf,BoldFont=cmuntb.ttf,ItalicFont=cmunit.ttf,BoldItalicFont=cmuntx.ttf]{cmuntt.ttf}\ttfamily main.tex}\setmainfont[Path=/usr/share/fonts/truetype/cmu/,UprightFont=cmunrm.ttf,BoldFont=cmunbx.ttf,ItalicFont=cmunti.ttf,BoldItalicFont=cmunbi.ttf]{cmunrm.ttf}\setmonofont[Path=/usr/share/fonts/truetype/cmu/,UprightFont=cmuntt.ttf,BoldFont=cmuntb.ttf,ItalicFont=cmunit.ttf,BoldItalicFont=cmuntx.ttf]{cmunrm.ttf}) looks like:

\begin{Shaded}
\begin{Highlighting}[]

\NormalTok{\textbackslash{}documentclass\{book\}}\newline
\NormalTok{\textbackslash{}usepackage\{standalone\}}\newline
\NormalTok{\textbackslash{}begin\{document\}}\newline
\CommentTok{\%\%\ensuremath{\text{ }}my\ensuremath{\text{ }}document\ensuremath{\text{ }}content}\newline
\NormalTok{\textbackslash{}input\{chapter1\}}\newline
\CommentTok{\%\%\ensuremath{\text{ }}more\ensuremath{\text{ }}of\ensuremath{\text{ }}my\ensuremath{\text{ }}document\ensuremath{\text{ }}content}\newline
\NormalTok{\textbackslash{}end\{document\}}\newline
\end{Highlighting}
\end{Shaded}


and Chapter 1 ({\ttfamily \setmainfont[Path=/usr/share/fonts/truetype/cmu/,UprightFont=cmunrm.ttf,BoldFont=cmunbx.ttf,ItalicFont=cmunti.ttf,BoldItalicFont=cmunbi.ttf]{cmuntt.ttf}\setmonofont[Path=/usr/share/fonts/truetype/cmu/,UprightFont=cmuntt.ttf,BoldFont=cmuntb.ttf,ItalicFont=cmunit.ttf,BoldItalicFont=cmuntx.ttf]{cmuntt.ttf}\ttfamily chapter1.tex}\setmainfont[Path=/usr/share/fonts/truetype/cmu/,UprightFont=cmunrm.ttf,BoldFont=cmunbx.ttf,ItalicFont=cmunti.ttf,BoldItalicFont=cmunbi.ttf]{cmunrm.ttf}\setmonofont[Path=/usr/share/fonts/truetype/cmu/,UprightFont=cmuntt.ttf,BoldFont=cmuntb.ttf,ItalicFont=cmunit.ttf,BoldItalicFont=cmuntx.ttf]{cmunrm.ttf}) looks like:


\begin{Shaded}
\begin{Highlighting}[]

\NormalTok{\textbackslash{}documentclass\{standalone\}}\newline
\CommentTok{\%\ensuremath{\text{ }}Preamble}\newline
\NormalTok{\textbackslash{}begin\{document\}}\newline
\CommentTok{\%\%\ensuremath{\text{ }}my\ensuremath{\text{ }}chapter\ensuremath{\text{ }}1\ensuremath{\text{ }}content}\newline
\CommentTok{\%\%}\newline
\CommentTok{\%\%\ensuremath{\text{ }}more\ensuremath{\text{ }}of\ensuremath{\text{ }}my\ensuremath{\text{ }}chapter\ensuremath{\text{ }}1\ensuremath{\text{ }}content}\newline
\NormalTok{\textbackslash{}end\{document\}}\newline
\end{Highlighting}
\end{Shaded}

\subsubsection{Import}
\label{905}
The \myhref{http://ctan.org/pkg/import}{import package} allows for relative directories. While subfiles fails to have a way of a subfile itself having references relative to its own directory, the {\ttfamily \setmainfont[Path=/usr/share/fonts/truetype/cmu/,UprightFont=cmunrm.ttf,BoldFont=cmunbx.ttf,ItalicFont=cmunti.ttf,BoldItalicFont=cmunbi.ttf]{cmuntt.ttf}\setmonofont[Path=/usr/share/fonts/truetype/cmu/,UprightFont=cmuntt.ttf,BoldFont=cmuntb.ttf,ItalicFont=cmunit.ttf,BoldItalicFont=cmuntx.ttf]{cmuntt.ttf}\ttfamily \textbackslash{}subimport}{$\text{ }$}\setmainfont[Path=/usr/share/fonts/truetype/cmu/,UprightFont=cmunrm.ttf,BoldFont=cmunbx.ttf,ItalicFont=cmunti.ttf,BoldItalicFont=cmunbi.ttf]{cmunrm.ttf}\setmonofont[Path=/usr/share/fonts/truetype/cmu/,UprightFont=cmuntt.ttf,BoldFont=cmuntb.ttf,ItalicFont=cmunit.ttf,BoldItalicFont=cmuntx.ttf]{cmunrm.ttf} command provides this functionality.
\subsection{Inserting PDF files}
\label{906}

If you need to insert an existing, possibly multi-{}page, PDF file into your LaTeX document, whether or not the included PDF was compiled with LaTeX or another tool, consider using the \myhref{http://www.ctan.org/tex-archive/macros/latex/contrib/pdfpages/}{pdfpages package}. In the preamble, include the package:


\begin{Shaded}
\begin{Highlighting}[]

\NormalTok{\textbackslash{}usepackage[final]\{pdfpages\}}\newline
\end{Highlighting}
\end{Shaded}


This package also allows you to specify which pages you wish to include: for example, to insert pages 3 to 6 from some file {\ttfamily \setmainfont[Path=/usr/share/fonts/truetype/cmu/,UprightFont=cmunrm.ttf,BoldFont=cmunbx.ttf,ItalicFont=cmunti.ttf,BoldItalicFont=cmunbi.ttf]{cmuntt.ttf}\setmonofont[Path=/usr/share/fonts/truetype/cmu/,UprightFont=cmuntt.ttf,BoldFont=cmuntb.ttf,ItalicFont=cmunit.ttf,BoldItalicFont=cmuntx.ttf]{cmuntt.ttf}\ttfamily insertme.pdf}\setmainfont[Path=/usr/share/fonts/truetype/cmu/,UprightFont=cmunrm.ttf,BoldFont=cmunbx.ttf,ItalicFont=cmunti.ttf,BoldItalicFont=cmunbi.ttf]{cmunrm.ttf}\setmonofont[Path=/usr/share/fonts/truetype/cmu/,UprightFont=cmuntt.ttf,BoldFont=cmuntb.ttf,ItalicFont=cmunit.ttf,BoldItalicFont=cmuntx.ttf]{cmunrm.ttf}, use:


\begin{Shaded}
\begin{Highlighting}[]

\NormalTok{\textbackslash{}includepdf[pages=3-6]\{insertme.pdf\}}\newline
\end{Highlighting}
\end{Shaded}


To insert the whole of {\ttfamily \setmainfont[Path=/usr/share/fonts/truetype/cmu/,UprightFont=cmunrm.ttf,BoldFont=cmunbx.ttf,ItalicFont=cmunti.ttf,BoldItalicFont=cmunbi.ttf]{cmuntt.ttf}\setmonofont[Path=/usr/share/fonts/truetype/cmu/,UprightFont=cmuntt.ttf,BoldFont=cmuntb.ttf,ItalicFont=cmunit.ttf,BoldItalicFont=cmuntx.ttf]{cmuntt.ttf}\ttfamily insertme.pdf}\setmainfont[Path=/usr/share/fonts/truetype/cmu/,UprightFont=cmunrm.ttf,BoldFont=cmunbx.ttf,ItalicFont=cmunti.ttf,BoldItalicFont=cmunbi.ttf]{cmunrm.ttf}\setmonofont[Path=/usr/share/fonts/truetype/cmu/,UprightFont=cmuntt.ttf,BoldFont=cmuntb.ttf,ItalicFont=cmunit.ttf,BoldItalicFont=cmuntx.ttf]{cmunrm.ttf}:


\begin{Shaded}
\begin{Highlighting}[]

\NormalTok{\textbackslash{}includepdf[pages=-]\{insertme.pdf\}}\newline
\end{Highlighting}
\end{Shaded}


For full functionality, compile the output with {\ttfamily \setmainfont[Path=/usr/share/fonts/truetype/cmu/,UprightFont=cmunrm.ttf,BoldFont=cmunbx.ttf,ItalicFont=cmunti.ttf,BoldItalicFont=cmunbi.ttf]{cmuntt.ttf}\setmonofont[Path=/usr/share/fonts/truetype/cmu/,UprightFont=cmuntt.ttf,BoldFont=cmuntb.ttf,ItalicFont=cmunit.ttf,BoldItalicFont=cmuntx.ttf]{cmuntt.ttf}\ttfamily pdflatex}\setmainfont[Path=/usr/share/fonts/truetype/cmu/,UprightFont=cmunrm.ttf,BoldFont=cmunbx.ttf,ItalicFont=cmunti.ttf,BoldItalicFont=cmunbi.ttf]{cmunrm.ttf}\setmonofont[Path=/usr/share/fonts/truetype/cmu/,UprightFont=cmuntt.ttf,BoldFont=cmuntb.ttf,ItalicFont=cmunit.ttf,BoldItalicFont=cmuntx.ttf]{cmunrm.ttf}.

Additional information can be found in the chapter \mylref{934}{Export To Other Formats}.
\section{The file {\ttfamily \setmainfont[Path=/usr/share/fonts/truetype/cmu/,UprightFont=cmunrm.ttf,BoldFont=cmunbx.ttf,ItalicFont=cmunti.ttf,BoldItalicFont=cmunbi.ttf]{cmuntt.ttf}\setmonofont[Path=/usr/share/fonts/truetype/cmu/,UprightFont=cmuntt.ttf,BoldFont=cmuntb.ttf,ItalicFont=cmunit.ttf,BoldItalicFont=cmuntx.ttf]{cmuntt.ttf}\ttfamily mystyle.sty}}
\label{907}\setmainfont[Path=/usr/share/fonts/truetype/cmu/,UprightFont=cmunrm.ttf,BoldFont=cmunbx.ttf,ItalicFont=cmunti.ttf,BoldItalicFont=cmunbi.ttf]{cmunrm.ttf}\setmonofont[Path=/usr/share/fonts/truetype/cmu/,UprightFont=cmuntt.ttf,BoldFont=cmuntb.ttf,ItalicFont=cmunit.ttf,BoldItalicFont=cmuntx.ttf]{cmunrm.ttf}

Instead of putting all the packages you need at the beginning of your document as you could, the best way is to load all the packages you need inside another dummy package called {\itshape \setmainfont[Path=/usr/share/fonts/truetype/cmu/,UprightFont=cmunrm.ttf,BoldFont=cmunbx.ttf,ItalicFont=cmunti.ttf,BoldItalicFont=cmunbi.ttf]{cmunti.ttf}\setmonofont[Path=/usr/share/fonts/truetype/cmu/,UprightFont=cmuntt.ttf,BoldFont=cmuntb.ttf,ItalicFont=cmunit.ttf,BoldItalicFont=cmuntx.ttf]{cmunti.ttf}\itshape mystyle}{$\text{ }$}\setmainfont[Path=/usr/share/fonts/truetype/cmu/,UprightFont=cmunrm.ttf,BoldFont=cmunbx.ttf,ItalicFont=cmunti.ttf,BoldItalicFont=cmunbi.ttf]{cmunrm.ttf}\setmonofont[Path=/usr/share/fonts/truetype/cmu/,UprightFont=cmuntt.ttf,BoldFont=cmuntb.ttf,ItalicFont=cmunit.ttf,BoldItalicFont=cmuntx.ttf]{cmunrm.ttf} you will create just for your document. The good point of doing this is that you will just have to add one single {\ttfamily \setmainfont[Path=/usr/share/fonts/truetype/cmu/,UprightFont=cmunrm.ttf,BoldFont=cmunbx.ttf,ItalicFont=cmunti.ttf,BoldItalicFont=cmunbi.ttf]{cmuntt.ttf}\setmonofont[Path=/usr/share/fonts/truetype/cmu/,UprightFont=cmuntt.ttf,BoldFont=cmuntb.ttf,ItalicFont=cmunit.ttf,BoldItalicFont=cmuntx.ttf]{cmuntt.ttf}\ttfamily \textbackslash{}usepackage}{$\text{ }$}\setmainfont[Path=/usr/share/fonts/truetype/cmu/,UprightFont=cmunrm.ttf,BoldFont=cmunbx.ttf,ItalicFont=cmunti.ttf,BoldItalicFont=cmunbi.ttf]{cmunrm.ttf}\setmonofont[Path=/usr/share/fonts/truetype/cmu/,UprightFont=cmuntt.ttf,BoldFont=cmuntb.ttf,ItalicFont=cmunit.ttf,BoldItalicFont=cmuntx.ttf]{cmunrm.ttf} in your {\ttfamily \setmainfont[Path=/usr/share/fonts/truetype/cmu/,UprightFont=cmunrm.ttf,BoldFont=cmunbx.ttf,ItalicFont=cmunti.ttf,BoldItalicFont=cmunbi.ttf]{cmuntt.ttf}\setmonofont[Path=/usr/share/fonts/truetype/cmu/,UprightFont=cmuntt.ttf,BoldFont=cmuntb.ttf,ItalicFont=cmunit.ttf,BoldItalicFont=cmuntx.ttf]{cmuntt.ttf}\ttfamily main.tex}{$\text{ }$}\setmainfont[Path=/usr/share/fonts/truetype/cmu/,UprightFont=cmunrm.ttf,BoldFont=cmunbx.ttf,ItalicFont=cmunti.ttf,BoldItalicFont=cmunbi.ttf]{cmunrm.ttf}\setmonofont[Path=/usr/share/fonts/truetype/cmu/,UprightFont=cmuntt.ttf,BoldFont=cmuntb.ttf,ItalicFont=cmunit.ttf,BoldItalicFont=cmuntx.ttf]{cmunrm.ttf} document, keeping your code much cleaner. Moreover, all the info about your style will be within one file, so when you will start another document you\textquotesingle{}ll just have to copy that file and include it properly, so you\textquotesingle{}ll have exactly the same style you have used.

Creating your own style is very simple: create a file called {\ttfamily \setmainfont[Path=/usr/share/fonts/truetype/cmu/,UprightFont=cmunrm.ttf,BoldFont=cmunbx.ttf,ItalicFont=cmunti.ttf,BoldItalicFont=cmunbi.ttf]{cmuntt.ttf}\setmonofont[Path=/usr/share/fonts/truetype/cmu/,UprightFont=cmuntt.ttf,BoldFont=cmuntb.ttf,ItalicFont=cmunit.ttf,BoldItalicFont=cmuntx.ttf]{cmuntt.ttf}\ttfamily mystyle.sty}{$\text{ }$}\setmainfont[Path=/usr/share/fonts/truetype/cmu/,UprightFont=cmunrm.ttf,BoldFont=cmunbx.ttf,ItalicFont=cmunti.ttf,BoldItalicFont=cmunbi.ttf]{cmunrm.ttf}\setmonofont[Path=/usr/share/fonts/truetype/cmu/,UprightFont=cmuntt.ttf,BoldFont=cmuntb.ttf,ItalicFont=cmunit.ttf,BoldItalicFont=cmuntx.ttf]{cmunrm.ttf} (you could name it as you wish, but it has to end with \symbol{34}.sty\symbol{34}). Write at the beginning of the {\ttfamily \setmainfont[Path=/usr/share/fonts/truetype/cmu/,UprightFont=cmunrm.ttf,BoldFont=cmunbx.ttf,ItalicFont=cmunti.ttf,BoldItalicFont=cmunbi.ttf]{cmuntt.ttf}\setmonofont[Path=/usr/share/fonts/truetype/cmu/,UprightFont=cmuntt.ttf,BoldFont=cmuntb.ttf,ItalicFont=cmunit.ttf,BoldItalicFont=cmuntx.ttf]{cmuntt.ttf}\ttfamily mystyle.sty}{$\text{ }$}\setmainfont[Path=/usr/share/fonts/truetype/cmu/,UprightFont=cmunrm.ttf,BoldFont=cmunbx.ttf,ItalicFont=cmunti.ttf,BoldItalicFont=cmunbi.ttf]{cmunrm.ttf}\setmonofont[Path=/usr/share/fonts/truetype/cmu/,UprightFont=cmuntt.ttf,BoldFont=cmuntb.ttf,ItalicFont=cmunit.ttf,BoldItalicFont=cmuntx.ttf]{cmunrm.ttf} file:

\begin{Shaded}
\begin{Highlighting}[]

\NormalTok{\textbackslash{}ProvidesPackage\{mystyle\}}\newline
\end{Highlighting}
\end{Shaded}

Then add all the packages you want with the standard command {\ttfamily \setmainfont[Path=/usr/share/fonts/truetype/cmu/,UprightFont=cmunrm.ttf,BoldFont=cmunbx.ttf,ItalicFont=cmunti.ttf,BoldItalicFont=cmunbi.ttf]{cmuntt.ttf}\setmonofont[Path=/usr/share/fonts/truetype/cmu/,UprightFont=cmuntt.ttf,BoldFont=cmuntb.ttf,ItalicFont=cmunit.ttf,BoldItalicFont=cmuntx.ttf]{cmuntt.ttf}\ttfamily \textbackslash{}usepackage\{...\}}{$\text{ }$}\setmainfont[Path=/usr/share/fonts/truetype/cmu/,UprightFont=cmunrm.ttf,BoldFont=cmunbx.ttf,ItalicFont=cmunti.ttf,BoldItalicFont=cmunbi.ttf]{cmunrm.ttf}\setmonofont[Path=/usr/share/fonts/truetype/cmu/,UprightFont=cmuntt.ttf,BoldFont=cmuntb.ttf,ItalicFont=cmunit.ttf,BoldItalicFont=cmuntx.ttf]{cmunrm.ttf} as you would do normally, change the value of all the variables you want, etc. It will work like the code you put here would be copied and pasted within your document. 

While writing, whenever you have to take a decision about formatting, define your own command for it and add it to your {\ttfamily \setmainfont[Path=/usr/share/fonts/truetype/cmu/,UprightFont=cmunrm.ttf,BoldFont=cmunbx.ttf,ItalicFont=cmunti.ttf,BoldItalicFont=cmunbi.ttf]{cmuntt.ttf}\setmonofont[Path=/usr/share/fonts/truetype/cmu/,UprightFont=cmuntt.ttf,BoldFont=cmuntb.ttf,ItalicFont=cmunit.ttf,BoldItalicFont=cmuntx.ttf]{cmuntt.ttf}\ttfamily mystyle.sty}\setmainfont[Path=/usr/share/fonts/truetype/cmu/,UprightFont=cmunrm.ttf,BoldFont=cmunbx.ttf,ItalicFont=cmunti.ttf,BoldItalicFont=cmunbi.ttf]{cmunrm.ttf}\setmonofont[Path=/usr/share/fonts/truetype/cmu/,UprightFont=cmuntt.ttf,BoldFont=cmuntb.ttf,ItalicFont=cmunit.ttf,BoldItalicFont=cmuntx.ttf]{cmunrm.ttf}:let LaTeX work for you. If you do so, it will be very easy to change it if you change your mind.

This is actually the beginning of the process of writing a package. See \mylref{837}{LaTeX/Macros} for more details.

For a list of several packages you can use, see  the \myhref{https://en.wikibooks.org/wiki/LaTeX\%2FPackages\%23Packages_list}{List of Packages} section.
\section{The main document {\ttfamily \setmainfont[Path=/usr/share/fonts/truetype/cmu/,UprightFont=cmunrm.ttf,BoldFont=cmunbx.ttf,ItalicFont=cmunti.ttf,BoldItalicFont=cmunbi.ttf]{cmuntt.ttf}\setmonofont[Path=/usr/share/fonts/truetype/cmu/,UprightFont=cmuntt.ttf,BoldFont=cmuntb.ttf,ItalicFont=cmunit.ttf,BoldItalicFont=cmuntx.ttf]{cmuntt.ttf}\ttfamily document.tex}}
\label{908}\setmainfont[Path=/usr/share/fonts/truetype/cmu/,UprightFont=cmunrm.ttf,BoldFont=cmunbx.ttf,ItalicFont=cmunti.ttf,BoldItalicFont=cmunbi.ttf]{cmunrm.ttf}\setmonofont[Path=/usr/share/fonts/truetype/cmu/,UprightFont=cmuntt.ttf,BoldFont=cmuntb.ttf,ItalicFont=cmunit.ttf,BoldItalicFont=cmuntx.ttf]{cmunrm.ttf}

Then create a file called {\ttfamily \setmainfont[Path=/usr/share/fonts/truetype/cmu/,UprightFont=cmunrm.ttf,BoldFont=cmunbx.ttf,ItalicFont=cmunti.ttf,BoldItalicFont=cmunbi.ttf]{cmuntt.ttf}\setmonofont[Path=/usr/share/fonts/truetype/cmu/,UprightFont=cmuntt.ttf,BoldFont=cmuntb.ttf,ItalicFont=cmunit.ttf,BoldItalicFont=cmuntx.ttf]{cmuntt.ttf}\ttfamily document.tex}\setmainfont[Path=/usr/share/fonts/truetype/cmu/,UprightFont=cmunrm.ttf,BoldFont=cmunbx.ttf,ItalicFont=cmunti.ttf,BoldItalicFont=cmunbi.ttf]{cmunrm.ttf}\setmonofont[Path=/usr/share/fonts/truetype/cmu/,UprightFont=cmuntt.ttf,BoldFont=cmuntb.ttf,ItalicFont=cmunit.ttf,BoldItalicFont=cmuntx.ttf]{cmunrm.ttf}; this will be the main file, the one you will compile, even if you shouldn\textquotesingle{}t need to edit it very often because you will be working on other files. It should look like this (it\textquotesingle{}s the sample code for a {\itshape \setmainfont[Path=/usr/share/fonts/truetype/cmu/,UprightFont=cmunrm.ttf,BoldFont=cmunbx.ttf,ItalicFont=cmunti.ttf,BoldItalicFont=cmunbi.ttf]{cmunti.ttf}\setmonofont[Path=/usr/share/fonts/truetype/cmu/,UprightFont=cmuntt.ttf,BoldFont=cmuntb.ttf,ItalicFont=cmunit.ttf,BoldItalicFont=cmuntx.ttf]{cmunti.ttf}\itshape report}\setmainfont[Path=/usr/share/fonts/truetype/cmu/,UprightFont=cmunrm.ttf,BoldFont=cmunbx.ttf,ItalicFont=cmunti.ttf,BoldItalicFont=cmunbi.ttf]{cmunrm.ttf}\setmonofont[Path=/usr/share/fonts/truetype/cmu/,UprightFont=cmuntt.ttf,BoldFont=cmuntb.ttf,ItalicFont=cmunit.ttf,BoldItalicFont=cmuntx.ttf]{cmunrm.ttf}, but you might easily change it to {\itshape \setmainfont[Path=/usr/share/fonts/truetype/cmu/,UprightFont=cmunrm.ttf,BoldFont=cmunbx.ttf,ItalicFont=cmunti.ttf,BoldItalicFont=cmunbi.ttf]{cmunti.ttf}\setmonofont[Path=/usr/share/fonts/truetype/cmu/,UprightFont=cmuntt.ttf,BoldFont=cmuntb.ttf,ItalicFont=cmunit.ttf,BoldItalicFont=cmuntx.ttf]{cmunti.ttf}\itshape article}{$\text{ }$}\setmainfont[Path=/usr/share/fonts/truetype/cmu/,UprightFont=cmunrm.ttf,BoldFont=cmunbx.ttf,ItalicFont=cmunti.ttf,BoldItalicFont=cmunbi.ttf]{cmunrm.ttf}\setmonofont[Path=/usr/share/fonts/truetype/cmu/,UprightFont=cmuntt.ttf,BoldFont=cmuntb.ttf,ItalicFont=cmunit.ttf,BoldItalicFont=cmuntx.ttf]{cmunrm.ttf} or whatever else):

\begin{Shaded}
\begin{Highlighting}[]

\NormalTok{\textbackslash{}documentclass[12pt,a4paper]\{report\}}\newline
\NormalTok{\textbackslash{}usepackage\{graphicx\}}\newline
\NormalTok{\textbackslash{}usepackage\{ifpdf\}}\newline
\NormalTok{\textbackslash{}ifpdf}\newline
\ensuremath{\text{ }}\ensuremath{\text{ }}\ensuremath{\text{ }}\CommentTok{\%\ensuremath{\text{ }}put\ensuremath{\text{ }}here\ensuremath{\text{ }}packages\ensuremath{\text{ }}only\ensuremath{\text{ }}for\ensuremath{\text{ }}the\ensuremath{\text{ }}PDF:}\newline
\ensuremath{\text{ }}\ensuremath{\text{ }}\ensuremath{\text{ }}\NormalTok{\textbackslash{}DeclareGraphicsExtensions\{.pdf,.png,.jpg,.mps\}}\newline
\ensuremath{\text{ }}\ensuremath{\text{ }}\ensuremath{\text{ }}\NormalTok{\textbackslash{}usepackage\{hyperref\}\ensuremath{\text{ }}}\newline
\NormalTok{\textbackslash{}else}\newline
\ensuremath{\text{ }}\ensuremath{\text{ }}\ensuremath{\text{ }}\CommentTok{\%\ensuremath{\text{ }}put\ensuremath{\text{ }}here\ensuremath{\text{ }}packages\ensuremath{\text{ }}only\ensuremath{\text{ }}for\ensuremath{\text{ }}the\ensuremath{\text{ }}DVI:}\newline
\NormalTok{\textbackslash{}fi}\newline
\ensuremath{\text{ }}\newline
\CommentTok{\%\ensuremath{\text{ }}put\ensuremath{\text{ }}all\ensuremath{\text{ }}the\ensuremath{\text{ }}other\ensuremath{\text{ }}packages\ensuremath{\text{ }}here:}\newline
\ensuremath{\text{ }}\newline
\NormalTok{\textbackslash{}usepackage\{mystyle\}}\newline
\ensuremath{\text{ }}\newline
\NormalTok{\textbackslash{}begin\{document\}}\newline
\ensuremath{\text{ }}\newline
\NormalTok{\textbackslash{}input\{./tex/title.tex\}}\newline
\CommentTok{\%\textbackslash{}maketitle}\newline
\NormalTok{\textbackslash{}tableofcontents}\newline
\NormalTok{\textbackslash{}listoffigures}\newline
\NormalTok{\textbackslash{}listoftables}\newline
\ensuremath{\text{ }}\newline
\NormalTok{\textbackslash{}input\{./tex/intro.tex\}}\newline
\NormalTok{\textbackslash{}input\{./tex/main_part.tex\}}\newline
\NormalTok{\textbackslash{}input\{./tex/conclusions.tex\}}\newline
\ensuremath{\text{ }}\newline
\NormalTok{\textbackslash{}appendix}\newline
\NormalTok{\textbackslash{}input\{./tex/myappendix.tex\}}\newline
\ensuremath{\text{ }}\newline
\ensuremath{\text{ }}\newline
\CommentTok{\%\ensuremath{\text{ }}Bibliography:}\newline
\NormalTok{\textbackslash{}clearpage}\newline
\NormalTok{\textbackslash{}addcontentsline\{toc\}\{chapter\}\{Bibliography\}}\newline
\NormalTok{\textbackslash{}input\{./tex/mybibliography.tex\}}\newline
\ensuremath{\text{ }}\newline
\NormalTok{\textbackslash{}end\{document\}}\newline
\end{Highlighting}
\end{Shaded}

Here a lot of code expressed in previous sections has been used. At the beginning there is the header discussed in the \mylref{968}{Tips \& Tricks} section, so you will be able to compile in both DVI and PDF. Then you import the only package you need, that is your {\itshape \setmainfont[Path=/usr/share/fonts/truetype/cmu/,UprightFont=cmunrm.ttf,BoldFont=cmunbx.ttf,ItalicFont=cmunti.ttf,BoldItalicFont=cmunbi.ttf]{cmunti.ttf}\setmonofont[Path=/usr/share/fonts/truetype/cmu/,UprightFont=cmuntt.ttf,BoldFont=cmuntb.ttf,ItalicFont=cmunit.ttf,BoldItalicFont=cmuntx.ttf]{cmunti.ttf}\itshape mystyle.sty}{$\text{ }$}\setmainfont[Path=/usr/share/fonts/truetype/cmu/,UprightFont=cmunrm.ttf,BoldFont=cmunbx.ttf,ItalicFont=cmunti.ttf,BoldItalicFont=cmunbi.ttf]{cmunrm.ttf}\setmonofont[Path=/usr/share/fonts/truetype/cmu/,UprightFont=cmuntt.ttf,BoldFont=cmuntb.ttf,ItalicFont=cmunit.ttf,BoldItalicFont=cmuntx.ttf]{cmunrm.ttf} (note that in the code it has to be imported without the extension), then your document starts. Then it inserts the title: we don\textquotesingle{}t like the output of {\ttfamily \setmainfont[Path=/usr/share/fonts/truetype/cmu/,UprightFont=cmunrm.ttf,BoldFont=cmunbx.ttf,ItalicFont=cmunti.ttf,BoldItalicFont=cmunbi.ttf]{cmuntt.ttf}\setmonofont[Path=/usr/share/fonts/truetype/cmu/,UprightFont=cmuntt.ttf,BoldFont=cmuntb.ttf,ItalicFont=cmunit.ttf,BoldItalicFont=cmuntx.ttf]{cmuntt.ttf}\ttfamily \textbackslash{}maketitle}{$\text{ }$}\setmainfont[Path=/usr/share/fonts/truetype/cmu/,UprightFont=cmunrm.ttf,BoldFont=cmunbx.ttf,ItalicFont=cmunti.ttf,BoldItalicFont=cmunbi.ttf]{cmunrm.ttf}\setmonofont[Path=/usr/share/fonts/truetype/cmu/,UprightFont=cmuntt.ttf,BoldFont=cmuntb.ttf,ItalicFont=cmunit.ttf,BoldItalicFont=cmuntx.ttf]{cmunrm.ttf} so we created our own, the code for it will be in a file called {\ttfamily \setmainfont[Path=/usr/share/fonts/truetype/cmu/,UprightFont=cmunrm.ttf,BoldFont=cmunbx.ttf,ItalicFont=cmunti.ttf,BoldItalicFont=cmunbi.ttf]{cmuntt.ttf}\setmonofont[Path=/usr/share/fonts/truetype/cmu/,UprightFont=cmuntt.ttf,BoldFont=cmuntb.ttf,ItalicFont=cmunit.ttf,BoldItalicFont=cmuntx.ttf]{cmuntt.ttf}\ttfamily title.tex}{$\text{ }$}\setmainfont[Path=/usr/share/fonts/truetype/cmu/,UprightFont=cmunrm.ttf,BoldFont=cmunbx.ttf,ItalicFont=cmunti.ttf,BoldItalicFont=cmunbi.ttf]{cmunrm.ttf}\setmonofont[Path=/usr/share/fonts/truetype/cmu/,UprightFont=cmuntt.ttf,BoldFont=cmuntb.ttf,ItalicFont=cmunit.ttf,BoldItalicFont=cmuntx.ttf]{cmunrm.ttf} in the folder called {\ttfamily \setmainfont[Path=/usr/share/fonts/truetype/cmu/,UprightFont=cmunrm.ttf,BoldFont=cmunbx.ttf,ItalicFont=cmunti.ttf,BoldItalicFont=cmunbi.ttf]{cmuntt.ttf}\setmonofont[Path=/usr/share/fonts/truetype/cmu/,UprightFont=cmuntt.ttf,BoldFont=cmuntb.ttf,ItalicFont=cmunit.ttf,BoldItalicFont=cmuntx.ttf]{cmuntt.ttf}\ttfamily tex}{$\text{ }$}\setmainfont[Path=/usr/share/fonts/truetype/cmu/,UprightFont=cmunrm.ttf,BoldFont=cmunbx.ttf,ItalicFont=cmunti.ttf,BoldItalicFont=cmunbi.ttf]{cmunrm.ttf}\setmonofont[Path=/usr/share/fonts/truetype/cmu/,UprightFont=cmuntt.ttf,BoldFont=cmuntb.ttf,ItalicFont=cmunit.ttf,BoldItalicFont=cmuntx.ttf]{cmunrm.ttf} we created before. How to write it is explained in the \mylref{293}{Title Creation} section. Then tables of contents, figure and tables are inserted. If you don\textquotesingle{}t want them, just comment out those lines. Then the main part of the document is inserted. As you can see, there is no text in {\ttfamily \setmainfont[Path=/usr/share/fonts/truetype/cmu/,UprightFont=cmunrm.ttf,BoldFont=cmunbx.ttf,ItalicFont=cmunti.ttf,BoldItalicFont=cmunbi.ttf]{cmuntt.ttf}\setmonofont[Path=/usr/share/fonts/truetype/cmu/,UprightFont=cmuntt.ttf,BoldFont=cmuntb.ttf,ItalicFont=cmunit.ttf,BoldItalicFont=cmuntx.ttf]{cmuntt.ttf}\ttfamily document.tex}\setmainfont[Path=/usr/share/fonts/truetype/cmu/,UprightFont=cmunrm.ttf,BoldFont=cmunbx.ttf,ItalicFont=cmunti.ttf,BoldItalicFont=cmunbi.ttf]{cmunrm.ttf}\setmonofont[Path=/usr/share/fonts/truetype/cmu/,UprightFont=cmuntt.ttf,BoldFont=cmuntb.ttf,ItalicFont=cmunit.ttf,BoldItalicFont=cmuntx.ttf]{cmunrm.ttf}: everything is in other files in the {\ttfamily \setmainfont[Path=/usr/share/fonts/truetype/cmu/,UprightFont=cmunrm.ttf,BoldFont=cmunbx.ttf,ItalicFont=cmunti.ttf,BoldItalicFont=cmunbi.ttf]{cmuntt.ttf}\setmonofont[Path=/usr/share/fonts/truetype/cmu/,UprightFont=cmuntt.ttf,BoldFont=cmuntb.ttf,ItalicFont=cmunit.ttf,BoldItalicFont=cmuntx.ttf]{cmuntt.ttf}\ttfamily tex}{$\text{ }$}\setmainfont[Path=/usr/share/fonts/truetype/cmu/,UprightFont=cmunrm.ttf,BoldFont=cmunbx.ttf,ItalicFont=cmunti.ttf,BoldItalicFont=cmunbi.ttf]{cmunrm.ttf}\setmonofont[Path=/usr/share/fonts/truetype/cmu/,UprightFont=cmuntt.ttf,BoldFont=cmuntb.ttf,ItalicFont=cmunit.ttf,BoldItalicFont=cmuntx.ttf]{cmunrm.ttf} directory so that you can easily edit them. We are separating our text from the structural code, so we are improving the \symbol{34}What You See is What You Mean\symbol{34} nature of LaTeX. Then we can see the appendix and finally the Bibliography. It is in a separate file and it is manually added to the table of contents using a tip suggested in the \mylref{968}{Tips \& Tricks}.

Once you have created your {\ttfamily \setmainfont[Path=/usr/share/fonts/truetype/cmu/,UprightFont=cmunrm.ttf,BoldFont=cmunbx.ttf,ItalicFont=cmunti.ttf,BoldItalicFont=cmunbi.ttf]{cmuntt.ttf}\setmonofont[Path=/usr/share/fonts/truetype/cmu/,UprightFont=cmuntt.ttf,BoldFont=cmuntb.ttf,ItalicFont=cmunit.ttf,BoldItalicFont=cmuntx.ttf]{cmuntt.ttf}\ttfamily document.tex}{$\text{ }$}\setmainfont[Path=/usr/share/fonts/truetype/cmu/,UprightFont=cmunrm.ttf,BoldFont=cmunbx.ttf,ItalicFont=cmunti.ttf,BoldItalicFont=cmunbi.ttf]{cmunrm.ttf}\setmonofont[Path=/usr/share/fonts/truetype/cmu/,UprightFont=cmuntt.ttf,BoldFont=cmuntb.ttf,ItalicFont=cmunit.ttf,BoldItalicFont=cmuntx.ttf]{cmunrm.ttf} you won\textquotesingle{}t need to edit it anymore, unless you want to add other files in the {\ttfamily \setmainfont[Path=/usr/share/fonts/truetype/cmu/,UprightFont=cmunrm.ttf,BoldFont=cmunbx.ttf,ItalicFont=cmunti.ttf,BoldItalicFont=cmunbi.ttf]{cmuntt.ttf}\setmonofont[Path=/usr/share/fonts/truetype/cmu/,UprightFont=cmuntt.ttf,BoldFont=cmuntb.ttf,ItalicFont=cmunit.ttf,BoldItalicFont=cmuntx.ttf]{cmuntt.ttf}\ttfamily tex}{$\text{ }$}\setmainfont[Path=/usr/share/fonts/truetype/cmu/,UprightFont=cmunrm.ttf,BoldFont=cmunbx.ttf,ItalicFont=cmunti.ttf,BoldItalicFont=cmunbi.ttf]{cmunrm.ttf}\setmonofont[Path=/usr/share/fonts/truetype/cmu/,UprightFont=cmuntt.ttf,BoldFont=cmuntb.ttf,ItalicFont=cmunit.ttf,BoldItalicFont=cmuntx.ttf]{cmunrm.ttf} directory, but this is not going to happen very often. Now you can write your document, separating it into as many files as you want and adding many pictures without getting confused: thanks to the rigid structure you gave to the project, you will be able to keep track of all your edits clearly.

A suggestion: do not give your files names like \symbol{34}chapter\_01.tex\symbol{34} or \symbol{34}figure\_03.png\symbol{34}, i.e. try to avoid using numbers in file-{}names: if the numbering LaTeX gives them automatically, is different from the one you gave (and this will likely happen) you will get really confused. When naming a file, stop for a second, think about a short name that can fully explain what is inside the file without being ambiguous, it will let you save a lot of time as soon as the document gets larger.
\section{External Links}
\label{909}
\begin{myitemize}
\item{}  \myhref{http://mirrors.ctan.org/macros/latex/contrib/subfiles/subfiles.pdf}{Subfiles package documentation}
\item{}  \myhref{http://mirrors.ctan.org/macros/latex/contrib/standalone/standalone.pdf}{Standalone package documentation}
\item{}  \myhref{http://mirrors.ctan.org/macros/latex/contrib/pdfpages/pdfpages.pdf}{pdfpages package documentation}
\end{myitemize}


\chapter{Collaborative Writing of LaTeX Documents}

\myminitoc
\label{910}

\label{911}

\LaTeXNullTemplate{}


{\bfseries \setmainfont[Path=/usr/share/fonts/truetype/cmu/,UprightFont=cmunrm.ttf,BoldFont=cmunbx.ttf,ItalicFont=cmunti.ttf,BoldItalicFont=cmunbi.ttf]{cmunbx.ttf}\setmonofont[Path=/usr/share/fonts/truetype/cmu/,UprightFont=cmuntt.ttf,BoldFont=cmuntb.ttf,ItalicFont=cmunit.ttf,BoldItalicFont=cmuntx.ttf]{cmunbx.ttf}\bfseries Note:}{$\text{ }$}\setmainfont[Path=/usr/share/fonts/truetype/cmu/,UprightFont=cmunrm.ttf,BoldFont=cmunbx.ttf,ItalicFont=cmunti.ttf,BoldItalicFont=cmunbi.ttf]{cmunrm.ttf}\setmonofont[Path=/usr/share/fonts/truetype/cmu/,UprightFont=cmuntt.ttf,BoldFont=cmuntb.ttf,ItalicFont=cmunit.ttf,BoldItalicFont=cmuntx.ttf]{cmunrm.ttf} 
This Wikibook is based on the article 
\myhref{http://tug.org/pracjourn/2007-3/henningsen/}{{\itshape \setmainfont[Path=/usr/share/fonts/truetype/cmu/,UprightFont=cmunrm.ttf,BoldFont=cmunbx.ttf,ItalicFont=cmunti.ttf,BoldItalicFont=cmunbi.ttf]{cmunti.ttf}\setmonofont[Path=/usr/share/fonts/truetype/cmu/,UprightFont=cmuntt.ttf,BoldFont=cmuntb.ttf,ItalicFont=cmunit.ttf,BoldItalicFont=cmuntx.ttf]{cmunti.ttf}\itshape Tools for Collaborative Writing of Scientific LaTeX Documents}}
by \myhref{https://en.wikibooks.org/wiki/User\%3AArnehe}{Arne Henningsen}
that is published in {\itshape \setmainfont[Path=/usr/share/fonts/truetype/cmu/,UprightFont=cmunrm.ttf,BoldFont=cmunbx.ttf,ItalicFont=cmunti.ttf,BoldItalicFont=cmunbi.ttf]{cmunti.ttf}\setmonofont[Path=/usr/share/fonts/truetype/cmu/,UprightFont=cmuntt.ttf,BoldFont=cmuntb.ttf,ItalicFont=cmunit.ttf,BoldItalicFont=cmuntx.ttf]{cmunti.ttf}\itshape The PracTeX Journal}{$\text{ }$}\setmainfont[Path=/usr/share/fonts/truetype/cmu/,UprightFont=cmunrm.ttf,BoldFont=cmunbx.ttf,ItalicFont=cmunti.ttf,BoldItalicFont=cmunbi.ttf]{cmunrm.ttf}\setmonofont[Path=/usr/share/fonts/truetype/cmu/,UprightFont=cmuntt.ttf,BoldFont=cmuntb.ttf,ItalicFont=cmunit.ttf,BoldItalicFont=cmuntx.ttf]{cmunrm.ttf} 2007, number 3 
(\myplainurl{http://www.tug.org/pracjourn/).}

\section{Abstract}
\label{912}

Collaborative writing of documents requires a strong synchronisation among authors. This Wikibook describes a possible way to organise the collaborative preparation of LaTeX documents. The presented solution is primarily based on the version control system {\itshape \setmainfont[Path=/usr/share/fonts/truetype/cmu/,UprightFont=cmunrm.ttf,BoldFont=cmunbx.ttf,ItalicFont=cmunti.ttf,BoldItalicFont=cmunbi.ttf]{cmunti.ttf}\setmonofont[Path=/usr/share/fonts/truetype/cmu/,UprightFont=cmuntt.ttf,BoldFont=cmuntb.ttf,ItalicFont=cmunit.ttf,BoldItalicFont=cmuntx.ttf]{cmunti.ttf}\itshape Subversion}{$\text{ }$}\setmainfont[Path=/usr/share/fonts/truetype/cmu/,UprightFont=cmunrm.ttf,BoldFont=cmunbx.ttf,ItalicFont=cmunti.ttf,BoldItalicFont=cmunbi.ttf]{cmunrm.ttf}\setmonofont[Path=/usr/share/fonts/truetype/cmu/,UprightFont=cmuntt.ttf,BoldFont=cmuntb.ttf,ItalicFont=cmunit.ttf,BoldItalicFont=cmuntx.ttf]{cmunrm.ttf} (\myplainurl{http://subversion.apache.org/).} The Wikibook describes how {\itshape \setmainfont[Path=/usr/share/fonts/truetype/cmu/,UprightFont=cmunrm.ttf,BoldFont=cmunbx.ttf,ItalicFont=cmunti.ttf,BoldItalicFont=cmunbi.ttf]{cmunti.ttf}\setmonofont[Path=/usr/share/fonts/truetype/cmu/,UprightFont=cmuntt.ttf,BoldFont=cmuntb.ttf,ItalicFont=cmunit.ttf,BoldItalicFont=cmuntx.ttf]{cmunti.ttf}\itshape Subversion}{$\text{ }$}\setmainfont[Path=/usr/share/fonts/truetype/cmu/,UprightFont=cmunrm.ttf,BoldFont=cmunbx.ttf,ItalicFont=cmunti.ttf,BoldItalicFont=cmunbi.ttf]{cmunrm.ttf}\setmonofont[Path=/usr/share/fonts/truetype/cmu/,UprightFont=cmuntt.ttf,BoldFont=cmuntb.ttf,ItalicFont=cmunit.ttf,BoldItalicFont=cmuntx.ttf]{cmunrm.ttf} can be used together with several other software tools and LaTeX packages to organise the collaborative preparation of LaTeX documents.
\subsection{Other  Methods}
\label{913}

\begin{myitemize}
\item{}  You can use one of the online solutions listed in the \mylref{10}{Installation} chapter. Most of them have collaboration features.
\end{myitemize}


\begin{myitemize}
\item{}  Another option for collaboration is \myhref{http://www.getdropbox.com}{dropbox}. It has 2 GB free storage and versioning system. Works like SVN, but more automated and therefore especially useful for beginning LaTeX users. However, Dropbox is not a true versioning control system, and as such it does not allow you to roll the article back to previous versions.
\end{myitemize}


\begin{myitemize}
\item{}  You can use an online collaborative tool built on top of a versioning control system, such as \myhref{https://authorea.com/}{Authorea} or \myhref{https://www.sharelatex.com/}{ShareLatex}. Authorea performs most of the actions described in this document, but in the background (it is built on Git). It allows authors to enter LaTeX or Markdown via a GUI with mathematical notation, figures, d3.js plots, IPython notebooks, data, and tables. All content is rendered to HTML5. Authorea  also features a commenting system and article-{}based chat to ease collaboration and review.
\end{myitemize}


\begin{myitemize}
\item{}  As the LaTeX system uses plain text, you can use synchronous collaborative editors like \myhref{https://en.wikipedia.org/wiki/Gobby}{Gobby}. In Gobby you can write your documents in collaboration with anyone in real time.  It is strongly recommended that you use utf8 encoding (especially if there are users on multiple operating systems collaborating) and a stable network (typically wired networks).  
\end{myitemize}


\begin{myitemize}
\item{}  \myhref{http://titanpad.com}{TitanPad} (or other \myhref{http://etherpad.org/etherpadsites.html}{clones} of \myhref{https://en.wikipedia.org/wiki/EtherPad}{EtherPad}). To compile use the command:$\text{ }$\newline{}
{\ttfamily \setmainfont[Path=/usr/share/fonts/truetype/cmu/,UprightFont=cmunrm.ttf,BoldFont=cmunbx.ttf,ItalicFont=cmunti.ttf,BoldItalicFont=cmunbi.ttf]{cmuntt.ttf}\setmonofont[Path=/usr/share/fonts/truetype/cmu/,UprightFont=cmuntt.ttf,BoldFont=cmuntb.ttf,ItalicFont=cmunit.ttf,BoldItalicFont=cmuntx.ttf]{cmuntt.ttf}\ttfamily wget -{}O filename.tex \symbol{34}http://titanpad.com/ep/pad/export/xxxx/latest?format=txt\symbol{34} \&\& (latex filename.tex)}$\text{ }$\newline{}
{$\text{ }$}\setmainfont[Path=/usr/share/fonts/truetype/cmu/,UprightFont=cmunrm.ttf,BoldFont=cmunbx.ttf,ItalicFont=cmunti.ttf,BoldItalicFont=cmunbi.ttf]{cmunrm.ttf}\setmonofont[Path=/usr/share/fonts/truetype/cmu/,UprightFont=cmuntt.ttf,BoldFont=cmuntb.ttf,ItalicFont=cmunit.ttf,BoldItalicFont=cmuntx.ttf]{cmunrm.ttf} where \textquotesingle{}xxxx\textquotesingle{} should be replaced by the pad number (something like \textquotesingle{}z7rSrfrYcH\textquotesingle{}).
\end{myitemize}


\begin{myitemize}
\item{}  With a dedicated Linux box with LaTeX \& Dropbox it\textquotesingle{}s possible to use Google docs and \myhref{https://gist.github.com/1995648}{some scripting} to get automatically generated PDFs on Dropbox from updates on Google Docs.
\end{myitemize}


\begin{myitemize}
\item{}  You can use a \myhref{https://en.wikipedia.org/wiki/Distributed_revision_control}{distributed version control system} such as \myhref{https://en.wikipedia.org/wiki/Mercurial}{Mercurial} or \myhref{https://en.wikibooks.org/wiki/Git}{Git}. This is the definitive solution for users looking for control and advanced features like branch and merge. The learning curve will be steeper than that for a web-{}based solution.
\end{myitemize}

\section{Introduction}
\label{914}

The collaborative preparation of documents
requires a considerable amount of coordination among the authors.
This coordination can be organised in many different ways,
where the best way depends on the specific circumstances.

In this Wikibook, I describe 
how the collaborative writing of LaTeX documents is organised
at our department 
(Division of Agricultural Policy, Department of Agricultural Economics, University of Kiel, Germany).
I present our software tools, and describe how we use them.
Thus, this Wikibook provides some ideas and hints
that will be useful for other LaTeX users
who prepare documents together with their co-{}authors.

\section{Interchanging Documents}
\label{915}

There are many ways to interchange documents among authors.
One possibility is to compose documents by interchanging e-{}mail messages.
This method has the advantage
that common users generally do not have to install and learn the usage of
any extra software,
because virtually all authors have an e-{}mail account.
Furthermore, the author who has modified the document
can easily attach the document and explain the changes by e-{}mail as well.
Unfortunately, there is a problem when two or more authors are working
at the same time on the same document.
So, how can authors synchronise these files?

A second possibility is to provide the document on a common file server,
which is available in most departments.
The risk of overwriting each others\textquotesingle{} modifications
can be eliminated by locking files that are currently edited.
However, generally the file server can be only accessed from within a department.
Hence, authors who are out of the building 
cannot use this method to update/commit their changes.
In this case,
they will have to use another way to overcome this problem.
So, how can authors access these files?

A third possibility is to use a version control system.
A comprehensive list of version control systems can be found at
\myhref{http://en.wikipedia.org/wiki/List_of_revision_control_software}{Wikipedia}.
Version control systems keep track of all changes in files in a project.
If many authors modify a document at the same time,
the version control system tries to merge all modifications automatically.
However, if multiple authors have modified the same line,
the modifications cannot be merged automatically,
and the user has to resolve the conflict by deciding manually
which of the changes should be kept.
Authors can also comment their modifications
so that the co-{}authors can easily understand the workflow of this file.
As version control systems generally communicate over the internet
(e.g. through TCP/IP connections),
they can be used from different computers with internet connections.
A restrictive firewall policy might prevent
the version control system from connecting to the internet.
In this case, the network administrator has to be asked
to open the appropriate port.
The internet is only used for synchronising the files.
Hence, a permanent internet connection is not required.
The only drawback of a version control system could be that it has to be installed and configured.

Moreover, a version control system is useful
even if a single user is working on a project.
First, the user can track (and possibly revoke) all previous modifications.
Second, this is a convenient way to have a backup of the files
on other computers (e.g. on the version control server).
Third, this allows the user to easily switch between different computers
(e.g. office, laptop, home).
\section{The Version Control System {\itshape \setmainfont[Path=/usr/share/fonts/truetype/cmu/,UprightFont=cmunrm.ttf,BoldFont=cmunbx.ttf,ItalicFont=cmunti.ttf,BoldItalicFont=cmunbi.ttf]{cmunti.ttf}\setmonofont[Path=/usr/share/fonts/truetype/cmu/,UprightFont=cmuntt.ttf,BoldFont=cmuntb.ttf,ItalicFont=cmunit.ttf,BoldItalicFont=cmuntx.ttf]{cmunti.ttf}\itshape Subversion}{$\text{ }$}\setmainfont[Path=/usr/share/fonts/truetype/cmu/,UprightFont=cmunrm.ttf,BoldFont=cmunbx.ttf,ItalicFont=cmunti.ttf,BoldItalicFont=cmunbi.ttf]{cmunrm.ttf}\setmonofont[Path=/usr/share/fonts/truetype/cmu/,UprightFont=cmuntt.ttf,BoldFont=cmuntb.ttf,ItalicFont=cmunit.ttf,BoldItalicFont=cmuntx.ttf]{cmunrm.ttf}}
\label{916}

\myhref{http://subversion.apache.org/}{Subversion (SVN)} comes as a successor to the popular version control system CVS. SVN operates on a client-{}server model in which a central server hosts a project repository that users copy and modify locally. A repository functions similarly to a library in that it permits users to check out the current project, make changes, and then check it back in. The server records all changes a user checks in (usually with a message summarizing what changes the user made) so that other users can easily apply those changes to their own local files. 

Each user has a local {\itshape \setmainfont[Path=/usr/share/fonts/truetype/cmu/,UprightFont=cmunrm.ttf,BoldFont=cmunbx.ttf,ItalicFont=cmunti.ttf,BoldItalicFont=cmunbi.ttf]{cmunti.ttf}\setmonofont[Path=/usr/share/fonts/truetype/cmu/,UprightFont=cmuntt.ttf,BoldFont=cmuntb.ttf,ItalicFont=cmunit.ttf,BoldItalicFont=cmuntx.ttf]{cmunti.ttf}\itshape working copy}{$\text{ }$}\setmainfont[Path=/usr/share/fonts/truetype/cmu/,UprightFont=cmunrm.ttf,BoldFont=cmunbx.ttf,ItalicFont=cmunti.ttf,BoldItalicFont=cmunbi.ttf]{cmunrm.ttf}\setmonofont[Path=/usr/share/fonts/truetype/cmu/,UprightFont=cmuntt.ttf,BoldFont=cmuntb.ttf,ItalicFont=cmunit.ttf,BoldItalicFont=cmuntx.ttf]{cmunrm.ttf} of a remote {\itshape \setmainfont[Path=/usr/share/fonts/truetype/cmu/,UprightFont=cmunrm.ttf,BoldFont=cmunbx.ttf,ItalicFont=cmunti.ttf,BoldItalicFont=cmunbi.ttf]{cmunti.ttf}\setmonofont[Path=/usr/share/fonts/truetype/cmu/,UprightFont=cmuntt.ttf,BoldFont=cmuntb.ttf,ItalicFont=cmunit.ttf,BoldItalicFont=cmuntx.ttf]{cmunti.ttf}\itshape repository}\setmainfont[Path=/usr/share/fonts/truetype/cmu/,UprightFont=cmunrm.ttf,BoldFont=cmunbx.ttf,ItalicFont=cmunti.ttf,BoldItalicFont=cmunbi.ttf]{cmunrm.ttf}\setmonofont[Path=/usr/share/fonts/truetype/cmu/,UprightFont=cmuntt.ttf,BoldFont=cmuntb.ttf,ItalicFont=cmunit.ttf,BoldItalicFont=cmuntx.ttf]{cmunrm.ttf}. For instance, users can {\itshape \setmainfont[Path=/usr/share/fonts/truetype/cmu/,UprightFont=cmunrm.ttf,BoldFont=cmunbx.ttf,ItalicFont=cmunti.ttf,BoldItalicFont=cmunbi.ttf]{cmunti.ttf}\setmonofont[Path=/usr/share/fonts/truetype/cmu/,UprightFont=cmuntt.ttf,BoldFont=cmuntb.ttf,ItalicFont=cmunit.ttf,BoldItalicFont=cmuntx.ttf]{cmunti.ttf}\itshape update}{$\text{ }$}\setmainfont[Path=/usr/share/fonts/truetype/cmu/,UprightFont=cmunrm.ttf,BoldFont=cmunbx.ttf,ItalicFont=cmunti.ttf,BoldItalicFont=cmunbi.ttf]{cmunrm.ttf}\setmonofont[Path=/usr/share/fonts/truetype/cmu/,UprightFont=cmuntt.ttf,BoldFont=cmuntb.ttf,ItalicFont=cmunit.ttf,BoldItalicFont=cmuntx.ttf]{cmunrm.ttf} changes from the repository to their working copy, {\itshape \setmainfont[Path=/usr/share/fonts/truetype/cmu/,UprightFont=cmunrm.ttf,BoldFont=cmunbx.ttf,ItalicFont=cmunti.ttf,BoldItalicFont=cmunbi.ttf]{cmunti.ttf}\setmonofont[Path=/usr/share/fonts/truetype/cmu/,UprightFont=cmuntt.ttf,BoldFont=cmuntb.ttf,ItalicFont=cmunit.ttf,BoldItalicFont=cmuntx.ttf]{cmunti.ttf}\itshape commit}{$\text{ }$}\setmainfont[Path=/usr/share/fonts/truetype/cmu/,UprightFont=cmunrm.ttf,BoldFont=cmunbx.ttf,ItalicFont=cmunti.ttf,BoldItalicFont=cmunbi.ttf]{cmunrm.ttf}\setmonofont[Path=/usr/share/fonts/truetype/cmu/,UprightFont=cmuntt.ttf,BoldFont=cmuntb.ttf,ItalicFont=cmunit.ttf,BoldItalicFont=cmuntx.ttf]{cmunrm.ttf} changes from their own working copy to the repository, or (re)view the differences between working copy and repository.

To set up a SVN version control system, the SVN server software has to be installed on a (single) computer with permanent internet access. (If this computer has no static IP address, one can use a service like \myhref{http://www.dyndns.com/}{DynDNS} to be able to access the server with a static hostname.) It can run on many Unix, modern MS Windows, and Mac OS X platforms.

Users do not have to install the SVN server software, but a SVN \symbol{34}client\symbol{34} software. This is the unique way to access the repositories on the server. Besides the basic SVN command-{}line client, there are several Graphical User Interface Tools (GUIs) and plug-{}ins for accessing the SVN server (see \myplainurl{http://subversion.tigris.org/links.html).} Additionally, there are very good manuals about SVN freely available on the internet (e.g. \myplainurl{http://svnbook.red-bean.com).}

At our department, we run the SVN server on a GNU-{}Linux system, because most Linux distributions include it. In this sense, installing, configuring, and maintaining SVN is a very simple task.

Most MS Windows users access the SVN server by the \myhref{http://tortoisesvn.tigris.org/}{TortoiseSVN} client, because it provides the most usual interface for common users. Linux users usually use SVN utilities from the command-{}line, or \myhref{http://zoneit.free.fr/esvn/}{eSvn}-{}-{}a GUI frontend-{}-{}with \myhref{http://kdiff3.sourceforge.net/}{KDiff3} for showing complex differences.
\section{Hosting LaTeX files in {\itshape \setmainfont[Path=/usr/share/fonts/truetype/cmu/,UprightFont=cmunrm.ttf,BoldFont=cmunbx.ttf,ItalicFont=cmunti.ttf,BoldItalicFont=cmunbi.ttf]{cmunti.ttf}\setmonofont[Path=/usr/share/fonts/truetype/cmu/,UprightFont=cmuntt.ttf,BoldFont=cmuntb.ttf,ItalicFont=cmunit.ttf,BoldItalicFont=cmuntx.ttf]{cmunti.ttf}\itshape Subversion}{$\text{ }$}\setmainfont[Path=/usr/share/fonts/truetype/cmu/,UprightFont=cmunrm.ttf,BoldFont=cmunbx.ttf,ItalicFont=cmunti.ttf,BoldItalicFont=cmunbi.ttf]{cmunrm.ttf}\setmonofont[Path=/usr/share/fonts/truetype/cmu/,UprightFont=cmuntt.ttf,BoldFont=cmuntb.ttf,ItalicFont=cmunit.ttf,BoldItalicFont=cmuntx.ttf]{cmunrm.ttf}}
\label{917}



\begin{minipage}{1.0\linewidth}
\begin{center}
\includegraphics[width=1.0\linewidth,height=6.5in,keepaspectratio]{../images/213.png}
\end{center}
\raggedright{}\myfigurewithcaption{213}{Figure 1: 
Common {\ttfamily \setmainfont[Path=/usr/share/fonts/truetype/cmu/,UprightFont=cmunrm.ttf,BoldFont=cmunbx.ttf,ItalicFont=cmunti.ttf,BoldItalicFont=cmunbi.ttf]{cmuntt.ttf}\setmonofont[Path=/usr/share/fonts/truetype/cmu/,UprightFont=cmuntt.ttf,BoldFont=cmuntb.ttf,ItalicFont=cmunit.ttf,BoldItalicFont=cmuntx.ttf]{cmuntt.ttf}\ttfamily texmf}{$\text{ }$}\setmainfont[Path=/usr/share/fonts/truetype/cmu/,UprightFont=cmunrm.ttf,BoldFont=cmunbx.ttf,ItalicFont=cmunti.ttf,BoldItalicFont=cmunbi.ttf]{cmunrm.ttf}\setmonofont[Path=/usr/share/fonts/truetype/cmu/,UprightFont=cmuntt.ttf,BoldFont=cmuntb.ttf,ItalicFont=cmunit.ttf,BoldItalicFont=cmuntx.ttf]{cmunrm.ttf} tree shown in {\itshape \setmainfont[Path=/usr/share/fonts/truetype/cmu/,UprightFont=cmunrm.ttf,BoldFont=cmunbx.ttf,ItalicFont=cmunti.ttf,BoldItalicFont=cmunbi.ttf]{cmunti.ttf}\setmonofont[Path=/usr/share/fonts/truetype/cmu/,UprightFont=cmuntt.ttf,BoldFont=cmuntb.ttf,ItalicFont=cmunit.ttf,BoldItalicFont=cmuntx.ttf]{cmunti.ttf}\itshape eSvn{\bfseries \setmainfont[Path=/usr/share/fonts/truetype/cmu/,UprightFont=cmunrm.ttf,BoldFont=cmunbx.ttf,ItalicFont=cmunti.ttf,BoldItalicFont=cmunbi.ttf]{cmunbi.ttf}\setmonofont[Path=/usr/share/fonts/truetype/cmu/,UprightFont=cmuntt.ttf,BoldFont=cmuntb.ttf,ItalicFont=cmunit.ttf,BoldItalicFont=cmuntx.ttf]{cmunbi.ttf}\bfseries \itshape s Repository Browser}}}
\end{minipage}\vspace{0.75cm}

{\itshape {\bfseries }}
On our {\itshape \setmainfont[Path=/usr/share/fonts/truetype/cmu/,UprightFont=cmunrm.ttf,BoldFont=cmunbx.ttf,ItalicFont=cmunti.ttf,BoldItalicFont=cmunbi.ttf]{cmunti.ttf}\setmonofont[Path=/usr/share/fonts/truetype/cmu/,UprightFont=cmuntt.ttf,BoldFont=cmuntb.ttf,ItalicFont=cmunit.ttf,BoldItalicFont=cmuntx.ttf]{cmunti.ttf}\itshape Subversion}{$\text{ }$}\setmainfont[Path=/usr/share/fonts/truetype/cmu/,UprightFont=cmunrm.ttf,BoldFont=cmunbx.ttf,ItalicFont=cmunti.ttf,BoldItalicFont=cmunbi.ttf]{cmunrm.ttf}\setmonofont[Path=/usr/share/fonts/truetype/cmu/,UprightFont=cmuntt.ttf,BoldFont=cmuntb.ttf,ItalicFont=cmunit.ttf,BoldItalicFont=cmuntx.ttf]{cmunrm.ttf} server,
we have one repository for a common {\ttfamily \setmainfont[Path=/usr/share/fonts/truetype/cmu/,UprightFont=cmunrm.ttf,BoldFont=cmunbx.ttf,ItalicFont=cmunti.ttf,BoldItalicFont=cmunbi.ttf]{cmuntt.ttf}\setmonofont[Path=/usr/share/fonts/truetype/cmu/,UprightFont=cmuntt.ttf,BoldFont=cmuntb.ttf,ItalicFont=cmunit.ttf,BoldItalicFont=cmuntx.ttf]{cmuntt.ttf}\ttfamily texmf}{$\text{ }$}\setmainfont[Path=/usr/share/fonts/truetype/cmu/,UprightFont=cmunrm.ttf,BoldFont=cmunbx.ttf,ItalicFont=cmunti.ttf,BoldItalicFont=cmunbi.ttf]{cmunrm.ttf}\setmonofont[Path=/usr/share/fonts/truetype/cmu/,UprightFont=cmuntt.ttf,BoldFont=cmuntb.ttf,ItalicFont=cmunit.ttf,BoldItalicFont=cmuntx.ttf]{cmunrm.ttf} tree.
Its structure complies with the
{\bfseries \setmainfont[Path=/usr/share/fonts/truetype/cmu/,UprightFont=cmunrm.ttf,BoldFont=cmunbx.ttf,ItalicFont=cmunti.ttf,BoldItalicFont=cmunbi.ttf]{cmunbx.ttf}\setmonofont[Path=/usr/share/fonts/truetype/cmu/,UprightFont=cmuntt.ttf,BoldFont=cmuntb.ttf,ItalicFont=cmunit.ttf,BoldItalicFont=cmuntx.ttf]{cmunbx.ttf}\bfseries TeX Directory Structure}{$\text{ }$}\setmainfont[Path=/usr/share/fonts/truetype/cmu/,UprightFont=cmunrm.ttf,BoldFont=cmunbx.ttf,ItalicFont=cmunti.ttf,BoldItalicFont=cmunbi.ttf]{cmunrm.ttf}\setmonofont[Path=/usr/share/fonts/truetype/cmu/,UprightFont=cmuntt.ttf,BoldFont=cmuntb.ttf,ItalicFont=cmunit.ttf,BoldItalicFont=cmuntx.ttf]{cmunrm.ttf} guidelines
(TDS, \myplainurl{http://www.tug.org/tds/tds.html,}
see figure 1).
This repository provides LaTeX classes, LaTeX styles, and BibTeX styles
that are not available in the LaTeX distributions of the users,
e.g. because they were bought
or developed for the internal use at our department.
All users have a working copy of this repository
and have configured LaTeX to use this as their
personal {\ttfamily \setmainfont[Path=/usr/share/fonts/truetype/cmu/,UprightFont=cmunrm.ttf,BoldFont=cmunbx.ttf,ItalicFont=cmunti.ttf,BoldItalicFont=cmunbi.ttf]{cmuntt.ttf}\setmonofont[Path=/usr/share/fonts/truetype/cmu/,UprightFont=cmuntt.ttf,BoldFont=cmuntb.ttf,ItalicFont=cmunit.ttf,BoldItalicFont=cmuntx.ttf]{cmuntt.ttf}\ttfamily texmf}{$\text{ }$}\setmainfont[Path=/usr/share/fonts/truetype/cmu/,UprightFont=cmunrm.ttf,BoldFont=cmunbx.ttf,ItalicFont=cmunti.ttf,BoldItalicFont=cmunbi.ttf]{cmunrm.ttf}\setmonofont[Path=/usr/share/fonts/truetype/cmu/,UprightFont=cmuntt.ttf,BoldFont=cmuntb.ttf,ItalicFont=cmunit.ttf,BoldItalicFont=cmuntx.ttf]{cmunrm.ttf} tree.
For instance, teTeX (\myplainurl{http://www.tug.org/tetex/)} users can edit their TeX configuration file
(e.g. {\ttfamily \setmainfont[Path=/usr/share/fonts/truetype/cmu/,UprightFont=cmunrm.ttf,BoldFont=cmunbx.ttf,ItalicFont=cmunti.ttf,BoldItalicFont=cmunbi.ttf]{cmuntt.ttf}\setmonofont[Path=/usr/share/fonts/truetype/cmu/,UprightFont=cmuntt.ttf,BoldFont=cmuntb.ttf,ItalicFont=cmunit.ttf,BoldItalicFont=cmuntx.ttf]{cmuntt.ttf}\ttfamily /etc/texmf/web2c/texmf.cnf}\setmainfont[Path=/usr/share/fonts/truetype/cmu/,UprightFont=cmunrm.ttf,BoldFont=cmunbx.ttf,ItalicFont=cmunti.ttf,BoldItalicFont=cmunbi.ttf]{cmunrm.ttf}\setmonofont[Path=/usr/share/fonts/truetype/cmu/,UprightFont=cmuntt.ttf,BoldFont=cmuntb.ttf,ItalicFont=cmunit.ttf,BoldItalicFont=cmuntx.ttf]{cmunrm.ttf})
and set the variable {\ttfamily \setmainfont[Path=/usr/share/fonts/truetype/cmu/,UprightFont=cmunrm.ttf,BoldFont=cmunbx.ttf,ItalicFont=cmunti.ttf,BoldItalicFont=cmunbi.ttf]{cmuntt.ttf}\setmonofont[Path=/usr/share/fonts/truetype/cmu/,UprightFont=cmuntt.ttf,BoldFont=cmuntb.ttf,ItalicFont=cmunit.ttf,BoldItalicFont=cmuntx.ttf]{cmuntt.ttf}\ttfamily TEXMFHOME}\setmainfont[Path=/usr/share/fonts/truetype/cmu/,UprightFont=cmunrm.ttf,BoldFont=cmunbx.ttf,ItalicFont=cmunti.ttf,BoldItalicFont=cmunbi.ttf]{cmunrm.ttf}\setmonofont[Path=/usr/share/fonts/truetype/cmu/,UprightFont=cmuntt.ttf,BoldFont=cmuntb.ttf,ItalicFont=cmunit.ttf,BoldItalicFont=cmuntx.ttf]{cmunrm.ttf}
to the path of the working copy of the common {\ttfamily \setmainfont[Path=/usr/share/fonts/truetype/cmu/,UprightFont=cmunrm.ttf,BoldFont=cmunbx.ttf,ItalicFont=cmunti.ttf,BoldItalicFont=cmunbi.ttf]{cmuntt.ttf}\setmonofont[Path=/usr/share/fonts/truetype/cmu/,UprightFont=cmuntt.ttf,BoldFont=cmuntb.ttf,ItalicFont=cmunit.ttf,BoldItalicFont=cmuntx.ttf]{cmuntt.ttf}\ttfamily texmf}{$\text{ }$}\setmainfont[Path=/usr/share/fonts/truetype/cmu/,UprightFont=cmunrm.ttf,BoldFont=cmunbx.ttf,ItalicFont=cmunti.ttf,BoldItalicFont=cmunbi.ttf]{cmunrm.ttf}\setmonofont[Path=/usr/share/fonts/truetype/cmu/,UprightFont=cmuntt.ttf,BoldFont=cmuntb.ttf,ItalicFont=cmunit.ttf,BoldItalicFont=cmuntx.ttf]{cmunrm.ttf} tree
(e.g. by {\ttfamily \setmainfont[Path=/usr/share/fonts/truetype/cmu/,UprightFont=cmunrm.ttf,BoldFont=cmunbx.ttf,ItalicFont=cmunti.ttf,BoldItalicFont=cmunbi.ttf]{cmuntt.ttf}\setmonofont[Path=/usr/share/fonts/truetype/cmu/,UprightFont=cmuntt.ttf,BoldFont=cmuntb.ttf,ItalicFont=cmunit.ttf,BoldItalicFont=cmuntx.ttf]{cmuntt.ttf}\ttfamily TEXMFHOME = \${}HOME/texmf}\setmainfont[Path=/usr/share/fonts/truetype/cmu/,UprightFont=cmunrm.ttf,BoldFont=cmunbx.ttf,ItalicFont=cmunti.ttf,BoldItalicFont=cmunbi.ttf]{cmunrm.ttf}\setmonofont[Path=/usr/share/fonts/truetype/cmu/,UprightFont=cmuntt.ttf,BoldFont=cmuntb.ttf,ItalicFont=cmunit.ttf,BoldItalicFont=cmuntx.ttf]{cmunrm.ttf});
MiKTeX (\myplainurl{http://www.miktex.org/)} users can add
the path of the working copy of the common {\ttfamily \setmainfont[Path=/usr/share/fonts/truetype/cmu/,UprightFont=cmunrm.ttf,BoldFont=cmunbx.ttf,ItalicFont=cmunti.ttf,BoldItalicFont=cmunbi.ttf]{cmuntt.ttf}\setmonofont[Path=/usr/share/fonts/truetype/cmu/,UprightFont=cmuntt.ttf,BoldFont=cmuntb.ttf,ItalicFont=cmunit.ttf,BoldItalicFont=cmuntx.ttf]{cmuntt.ttf}\ttfamily texmf}{$\text{ }$}\setmainfont[Path=/usr/share/fonts/truetype/cmu/,UprightFont=cmunrm.ttf,BoldFont=cmunbx.ttf,ItalicFont=cmunti.ttf,BoldItalicFont=cmunbi.ttf]{cmunrm.ttf}\setmonofont[Path=/usr/share/fonts/truetype/cmu/,UprightFont=cmuntt.ttf,BoldFont=cmuntb.ttf,ItalicFont=cmunit.ttf,BoldItalicFont=cmuntx.ttf]{cmunrm.ttf} tree
in the \textquotesingle{}Roots\textquotesingle{} tab of the MiKTeX Options.

If a new class or style file has been added
(but not if these files have been modified),
the users have to update their \textquotesingle{}file name data base\textquotesingle{} (FNDB)
before they can use these classes and styles.
For instance, teTeX users have to execute {\ttfamily \setmainfont[Path=/usr/share/fonts/truetype/cmu/,UprightFont=cmunrm.ttf,BoldFont=cmunbx.ttf,ItalicFont=cmunti.ttf,BoldItalicFont=cmunbi.ttf]{cmuntt.ttf}\setmonofont[Path=/usr/share/fonts/truetype/cmu/,UprightFont=cmuntt.ttf,BoldFont=cmuntb.ttf,ItalicFont=cmunit.ttf,BoldItalicFont=cmuntx.ttf]{cmuntt.ttf}\ttfamily texhash}\setmainfont[Path=/usr/share/fonts/truetype/cmu/,UprightFont=cmunrm.ttf,BoldFont=cmunbx.ttf,ItalicFont=cmunti.ttf,BoldItalicFont=cmunbi.ttf]{cmunrm.ttf}\setmonofont[Path=/usr/share/fonts/truetype/cmu/,UprightFont=cmuntt.ttf,BoldFont=cmuntb.ttf,ItalicFont=cmunit.ttf,BoldItalicFont=cmuntx.ttf]{cmunrm.ttf};
MiKTeX users have to click on the button \textquotesingle{}Refresh FNDB\textquotesingle{}
in the \textquotesingle{}General\textquotesingle{} tab of the MiKTeX Options.

Furthermore, the repository contains manuals
explaining the specific LaTeX software solution
at our department (e.g. this document).

The {\itshape \setmainfont[Path=/usr/share/fonts/truetype/cmu/,UprightFont=cmunrm.ttf,BoldFont=cmunbx.ttf,ItalicFont=cmunti.ttf,BoldItalicFont=cmunbi.ttf]{cmunti.ttf}\setmonofont[Path=/usr/share/fonts/truetype/cmu/,UprightFont=cmuntt.ttf,BoldFont=cmuntb.ttf,ItalicFont=cmunit.ttf,BoldItalicFont=cmuntx.ttf]{cmunti.ttf}\itshape Subversion}{$\text{ }$}\setmainfont[Path=/usr/share/fonts/truetype/cmu/,UprightFont=cmunrm.ttf,BoldFont=cmunbx.ttf,ItalicFont=cmunti.ttf,BoldItalicFont=cmunbi.ttf]{cmunrm.ttf}\setmonofont[Path=/usr/share/fonts/truetype/cmu/,UprightFont=cmuntt.ttf,BoldFont=cmuntb.ttf,ItalicFont=cmunit.ttf,BoldItalicFont=cmuntx.ttf]{cmunrm.ttf} server hosts a separate repository
for each project of our department.
Although branching, merging, and tagging is less important
for writing text documents than for writing source code for software,
our repository layouts follow the recommendations of 
the \textquotesingle{}Subversion book\textquotesingle{} (\myplainurl{http://svnbook.red-bean.com).}
In this sense, each repository has the three directories
{\ttfamily \setmainfont[Path=/usr/share/fonts/truetype/cmu/,UprightFont=cmunrm.ttf,BoldFont=cmunbx.ttf,ItalicFont=cmunti.ttf,BoldItalicFont=cmunbi.ttf]{cmuntt.ttf}\setmonofont[Path=/usr/share/fonts/truetype/cmu/,UprightFont=cmuntt.ttf,BoldFont=cmuntb.ttf,ItalicFont=cmunit.ttf,BoldItalicFont=cmuntx.ttf]{cmuntt.ttf}\ttfamily /trunk}\setmainfont[Path=/usr/share/fonts/truetype/cmu/,UprightFont=cmunrm.ttf,BoldFont=cmunbx.ttf,ItalicFont=cmunti.ttf,BoldItalicFont=cmunbi.ttf]{cmunrm.ttf}\setmonofont[Path=/usr/share/fonts/truetype/cmu/,UprightFont=cmuntt.ttf,BoldFont=cmuntb.ttf,ItalicFont=cmunit.ttf,BoldItalicFont=cmuntx.ttf]{cmunrm.ttf}, {\ttfamily \setmainfont[Path=/usr/share/fonts/truetype/cmu/,UprightFont=cmunrm.ttf,BoldFont=cmunbx.ttf,ItalicFont=cmunti.ttf,BoldItalicFont=cmunbi.ttf]{cmuntt.ttf}\setmonofont[Path=/usr/share/fonts/truetype/cmu/,UprightFont=cmuntt.ttf,BoldFont=cmuntb.ttf,ItalicFont=cmunit.ttf,BoldItalicFont=cmuntx.ttf]{cmuntt.ttf}\ttfamily /branches}\setmainfont[Path=/usr/share/fonts/truetype/cmu/,UprightFont=cmunrm.ttf,BoldFont=cmunbx.ttf,ItalicFont=cmunti.ttf,BoldItalicFont=cmunbi.ttf]{cmunrm.ttf}\setmonofont[Path=/usr/share/fonts/truetype/cmu/,UprightFont=cmuntt.ttf,BoldFont=cmuntb.ttf,ItalicFont=cmunit.ttf,BoldItalicFont=cmuntx.ttf]{cmunrm.ttf}, and {\ttfamily \setmainfont[Path=/usr/share/fonts/truetype/cmu/,UprightFont=cmunrm.ttf,BoldFont=cmunbx.ttf,ItalicFont=cmunti.ttf,BoldItalicFont=cmunbi.ttf]{cmuntt.ttf}\setmonofont[Path=/usr/share/fonts/truetype/cmu/,UprightFont=cmuntt.ttf,BoldFont=cmuntb.ttf,ItalicFont=cmunit.ttf,BoldItalicFont=cmuntx.ttf]{cmuntt.ttf}\ttfamily /tags}\setmainfont[Path=/usr/share/fonts/truetype/cmu/,UprightFont=cmunrm.ttf,BoldFont=cmunbx.ttf,ItalicFont=cmunti.ttf,BoldItalicFont=cmunbi.ttf]{cmunrm.ttf}\setmonofont[Path=/usr/share/fonts/truetype/cmu/,UprightFont=cmuntt.ttf,BoldFont=cmuntb.ttf,ItalicFont=cmunit.ttf,BoldItalicFont=cmuntx.ttf]{cmunrm.ttf}.

The most important directory is {\ttfamily \setmainfont[Path=/usr/share/fonts/truetype/cmu/,UprightFont=cmunrm.ttf,BoldFont=cmunbx.ttf,ItalicFont=cmunti.ttf,BoldItalicFont=cmunbi.ttf]{cmuntt.ttf}\setmonofont[Path=/usr/share/fonts/truetype/cmu/,UprightFont=cmuntt.ttf,BoldFont=cmuntb.ttf,ItalicFont=cmunit.ttf,BoldItalicFont=cmuntx.ttf]{cmuntt.ttf}\ttfamily /trunk}\setmainfont[Path=/usr/share/fonts/truetype/cmu/,UprightFont=cmunrm.ttf,BoldFont=cmunbx.ttf,ItalicFont=cmunti.ttf,BoldItalicFont=cmunbi.ttf]{cmunrm.ttf}\setmonofont[Path=/usr/share/fonts/truetype/cmu/,UprightFont=cmuntt.ttf,BoldFont=cmuntb.ttf,ItalicFont=cmunit.ttf,BoldItalicFont=cmuntx.ttf]{cmunrm.ttf}.
If a single text document belongs to the project,
all files and subdirectories of this text document are in
{\ttfamily \setmainfont[Path=/usr/share/fonts/truetype/cmu/,UprightFont=cmunrm.ttf,BoldFont=cmunbx.ttf,ItalicFont=cmunti.ttf,BoldItalicFont=cmunbi.ttf]{cmuntt.ttf}\setmonofont[Path=/usr/share/fonts/truetype/cmu/,UprightFont=cmuntt.ttf,BoldFont=cmuntb.ttf,ItalicFont=cmunit.ttf,BoldItalicFont=cmuntx.ttf]{cmuntt.ttf}\ttfamily /trunk}\setmainfont[Path=/usr/share/fonts/truetype/cmu/,UprightFont=cmunrm.ttf,BoldFont=cmunbx.ttf,ItalicFont=cmunti.ttf,BoldItalicFont=cmunbi.ttf]{cmunrm.ttf}\setmonofont[Path=/usr/share/fonts/truetype/cmu/,UprightFont=cmuntt.ttf,BoldFont=cmuntb.ttf,ItalicFont=cmunit.ttf,BoldItalicFont=cmuntx.ttf]{cmunrm.ttf}.
If the project yields two or more different text documents,
{\ttfamily \setmainfont[Path=/usr/share/fonts/truetype/cmu/,UprightFont=cmunrm.ttf,BoldFont=cmunbx.ttf,ItalicFont=cmunti.ttf,BoldItalicFont=cmunbi.ttf]{cmuntt.ttf}\setmonofont[Path=/usr/share/fonts/truetype/cmu/,UprightFont=cmuntt.ttf,BoldFont=cmuntb.ttf,ItalicFont=cmunit.ttf,BoldItalicFont=cmuntx.ttf]{cmuntt.ttf}\ttfamily /trunk}{$\text{ }$}\setmainfont[Path=/usr/share/fonts/truetype/cmu/,UprightFont=cmunrm.ttf,BoldFont=cmunbx.ttf,ItalicFont=cmunti.ttf,BoldItalicFont=cmunbi.ttf]{cmunrm.ttf}\setmonofont[Path=/usr/share/fonts/truetype/cmu/,UprightFont=cmuntt.ttf,BoldFont=cmuntb.ttf,ItalicFont=cmunit.ttf,BoldItalicFont=cmuntx.ttf]{cmunrm.ttf} contains a subdirectory for each text document.
A slightly different version (a {\bfseries \setmainfont[Path=/usr/share/fonts/truetype/cmu/,UprightFont=cmunrm.ttf,BoldFont=cmunbx.ttf,ItalicFont=cmunti.ttf,BoldItalicFont=cmunbi.ttf]{cmunbx.ttf}\setmonofont[Path=/usr/share/fonts/truetype/cmu/,UprightFont=cmuntt.ttf,BoldFont=cmuntb.ttf,ItalicFont=cmunit.ttf,BoldItalicFont=cmuntx.ttf]{cmunbx.ttf}\bfseries branch}\setmainfont[Path=/usr/share/fonts/truetype/cmu/,UprightFont=cmunrm.ttf,BoldFont=cmunbx.ttf,ItalicFont=cmunti.ttf,BoldItalicFont=cmunbi.ttf]{cmunrm.ttf}\setmonofont[Path=/usr/share/fonts/truetype/cmu/,UprightFont=cmuntt.ttf,BoldFont=cmuntb.ttf,ItalicFont=cmunit.ttf,BoldItalicFont=cmuntx.ttf]{cmunrm.ttf}) of a text document
(e.g. for presentation at a conference)
can be prepared either in an additional subdirectory of {\ttfamily \setmainfont[Path=/usr/share/fonts/truetype/cmu/,UprightFont=cmunrm.ttf,BoldFont=cmunbx.ttf,ItalicFont=cmunti.ttf,BoldItalicFont=cmunbi.ttf]{cmuntt.ttf}\setmonofont[Path=/usr/share/fonts/truetype/cmu/,UprightFont=cmuntt.ttf,BoldFont=cmuntb.ttf,ItalicFont=cmunit.ttf,BoldItalicFont=cmuntx.ttf]{cmuntt.ttf}\ttfamily /trunk}\setmainfont[Path=/usr/share/fonts/truetype/cmu/,UprightFont=cmunrm.ttf,BoldFont=cmunbx.ttf,ItalicFont=cmunti.ttf,BoldItalicFont=cmunbi.ttf]{cmunrm.ttf}\setmonofont[Path=/usr/share/fonts/truetype/cmu/,UprightFont=cmuntt.ttf,BoldFont=cmuntb.ttf,ItalicFont=cmunit.ttf,BoldItalicFont=cmuntx.ttf]{cmunrm.ttf}
or in a new subdirectory of {\ttfamily \setmainfont[Path=/usr/share/fonts/truetype/cmu/,UprightFont=cmunrm.ttf,BoldFont=cmunbx.ttf,ItalicFont=cmunti.ttf,BoldItalicFont=cmunbi.ttf]{cmuntt.ttf}\setmonofont[Path=/usr/share/fonts/truetype/cmu/,UprightFont=cmuntt.ttf,BoldFont=cmuntb.ttf,ItalicFont=cmunit.ttf,BoldItalicFont=cmuntx.ttf]{cmuntt.ttf}\ttfamily /branches}\setmainfont[Path=/usr/share/fonts/truetype/cmu/,UprightFont=cmunrm.ttf,BoldFont=cmunbx.ttf,ItalicFont=cmunti.ttf,BoldItalicFont=cmunbi.ttf]{cmunrm.ttf}\setmonofont[Path=/usr/share/fonts/truetype/cmu/,UprightFont=cmuntt.ttf,BoldFont=cmuntb.ttf,ItalicFont=cmunit.ttf,BoldItalicFont=cmuntx.ttf]{cmunrm.ttf}.
When a text document is submitted to a journal or a conference,
we create a {\bfseries \setmainfont[Path=/usr/share/fonts/truetype/cmu/,UprightFont=cmunrm.ttf,BoldFont=cmunbx.ttf,ItalicFont=cmunti.ttf,BoldItalicFont=cmunbi.ttf]{cmunbx.ttf}\setmonofont[Path=/usr/share/fonts/truetype/cmu/,UprightFont=cmuntt.ttf,BoldFont=cmuntb.ttf,ItalicFont=cmunit.ttf,BoldItalicFont=cmuntx.ttf]{cmunbx.ttf}\bfseries tag}{$\text{ }$}\setmainfont[Path=/usr/share/fonts/truetype/cmu/,UprightFont=cmunrm.ttf,BoldFont=cmunbx.ttf,ItalicFont=cmunti.ttf,BoldItalicFont=cmunbi.ttf]{cmunrm.ttf}\setmonofont[Path=/usr/share/fonts/truetype/cmu/,UprightFont=cmuntt.ttf,BoldFont=cmuntb.ttf,ItalicFont=cmunit.ttf,BoldItalicFont=cmuntx.ttf]{cmunrm.ttf} in the directory {\ttfamily \setmainfont[Path=/usr/share/fonts/truetype/cmu/,UprightFont=cmunrm.ttf,BoldFont=cmunbx.ttf,ItalicFont=cmunti.ttf,BoldItalicFont=cmunbi.ttf]{cmuntt.ttf}\setmonofont[Path=/usr/share/fonts/truetype/cmu/,UprightFont=cmuntt.ttf,BoldFont=cmuntb.ttf,ItalicFont=cmunit.ttf,BoldItalicFont=cmuntx.ttf]{cmuntt.ttf}\ttfamily /tags}\setmainfont[Path=/usr/share/fonts/truetype/cmu/,UprightFont=cmunrm.ttf,BoldFont=cmunbx.ttf,ItalicFont=cmunti.ttf,BoldItalicFont=cmunbi.ttf]{cmunrm.ttf}\setmonofont[Path=/usr/share/fonts/truetype/cmu/,UprightFont=cmuntt.ttf,BoldFont=cmuntb.ttf,ItalicFont=cmunit.ttf,BoldItalicFont=cmuntx.ttf]{cmunrm.ttf}
so that it is easy to identify the submitted version
of the document at a later date.
This feature has been proven very useful.
When creating branches and tags,
it is important always to use
the {\itshape \setmainfont[Path=/usr/share/fonts/truetype/cmu/,UprightFont=cmunrm.ttf,BoldFont=cmunbx.ttf,ItalicFont=cmunti.ttf,BoldItalicFont=cmunbi.ttf]{cmunti.ttf}\setmonofont[Path=/usr/share/fonts/truetype/cmu/,UprightFont=cmuntt.ttf,BoldFont=cmuntb.ttf,ItalicFont=cmunit.ttf,BoldItalicFont=cmuntx.ttf]{cmunti.ttf}\itshape Subversion}{$\text{ }$}\setmainfont[Path=/usr/share/fonts/truetype/cmu/,UprightFont=cmunrm.ttf,BoldFont=cmunbx.ttf,ItalicFont=cmunti.ttf,BoldItalicFont=cmunbi.ttf]{cmunrm.ttf}\setmonofont[Path=/usr/share/fonts/truetype/cmu/,UprightFont=cmuntt.ttf,BoldFont=cmuntb.ttf,ItalicFont=cmunit.ttf,BoldItalicFont=cmuntx.ttf]{cmunrm.ttf} client (and not the tools of the local file system)
for these actions,
because this saves disk space on the server
and it preserves information about the same history of these documents.

Often the question arises,
which files should be put under version control.
Generally, all files that are directly modified by the user
and that are necessary for compiling the document
should be included in the version control system.
Typically, these are the LaTeX source code ({\ttfamily \setmainfont[Path=/usr/share/fonts/truetype/cmu/,UprightFont=cmunrm.ttf,BoldFont=cmunbx.ttf,ItalicFont=cmunti.ttf,BoldItalicFont=cmunbi.ttf]{cmuntt.ttf}\setmonofont[Path=/usr/share/fonts/truetype/cmu/,UprightFont=cmuntt.ttf,BoldFont=cmuntb.ttf,ItalicFont=cmunit.ttf,BoldItalicFont=cmuntx.ttf]{cmuntt.ttf}\ttfamily *.tex}\setmainfont[Path=/usr/share/fonts/truetype/cmu/,UprightFont=cmunrm.ttf,BoldFont=cmunbx.ttf,ItalicFont=cmunti.ttf,BoldItalicFont=cmunbi.ttf]{cmunrm.ttf}\setmonofont[Path=/usr/share/fonts/truetype/cmu/,UprightFont=cmuntt.ttf,BoldFont=cmuntb.ttf,ItalicFont=cmunit.ttf,BoldItalicFont=cmuntx.ttf]{cmunrm.ttf}) files
(the main document and possibly some subdocuments)
and all pictures that are inserted in the document
({\ttfamily \setmainfont[Path=/usr/share/fonts/truetype/cmu/,UprightFont=cmunrm.ttf,BoldFont=cmunbx.ttf,ItalicFont=cmunti.ttf,BoldItalicFont=cmunbi.ttf]{cmuntt.ttf}\setmonofont[Path=/usr/share/fonts/truetype/cmu/,UprightFont=cmuntt.ttf,BoldFont=cmuntb.ttf,ItalicFont=cmunit.ttf,BoldItalicFont=cmuntx.ttf]{cmuntt.ttf}\ttfamily *.eps}\setmainfont[Path=/usr/share/fonts/truetype/cmu/,UprightFont=cmunrm.ttf,BoldFont=cmunbx.ttf,ItalicFont=cmunti.ttf,BoldItalicFont=cmunbi.ttf]{cmunrm.ttf}\setmonofont[Path=/usr/share/fonts/truetype/cmu/,UprightFont=cmuntt.ttf,BoldFont=cmuntb.ttf,ItalicFont=cmunit.ttf,BoldItalicFont=cmuntx.ttf]{cmunrm.ttf}, {\ttfamily \setmainfont[Path=/usr/share/fonts/truetype/cmu/,UprightFont=cmunrm.ttf,BoldFont=cmunbx.ttf,ItalicFont=cmunti.ttf,BoldItalicFont=cmunbi.ttf]{cmuntt.ttf}\setmonofont[Path=/usr/share/fonts/truetype/cmu/,UprightFont=cmuntt.ttf,BoldFont=cmuntb.ttf,ItalicFont=cmunit.ttf,BoldItalicFont=cmuntx.ttf]{cmuntt.ttf}\ttfamily *.jpg}\setmainfont[Path=/usr/share/fonts/truetype/cmu/,UprightFont=cmunrm.ttf,BoldFont=cmunbx.ttf,ItalicFont=cmunti.ttf,BoldItalicFont=cmunbi.ttf]{cmunrm.ttf}\setmonofont[Path=/usr/share/fonts/truetype/cmu/,UprightFont=cmuntt.ttf,BoldFont=cmuntb.ttf,ItalicFont=cmunit.ttf,BoldItalicFont=cmuntx.ttf]{cmunrm.ttf}, {\ttfamily \setmainfont[Path=/usr/share/fonts/truetype/cmu/,UprightFont=cmunrm.ttf,BoldFont=cmunbx.ttf,ItalicFont=cmunti.ttf,BoldItalicFont=cmunbi.ttf]{cmuntt.ttf}\setmonofont[Path=/usr/share/fonts/truetype/cmu/,UprightFont=cmuntt.ttf,BoldFont=cmuntb.ttf,ItalicFont=cmunit.ttf,BoldItalicFont=cmuntx.ttf]{cmuntt.ttf}\ttfamily *.png}\setmainfont[Path=/usr/share/fonts/truetype/cmu/,UprightFont=cmunrm.ttf,BoldFont=cmunbx.ttf,ItalicFont=cmunti.ttf,BoldItalicFont=cmunbi.ttf]{cmunrm.ttf}\setmonofont[Path=/usr/share/fonts/truetype/cmu/,UprightFont=cmuntt.ttf,BoldFont=cmuntb.ttf,ItalicFont=cmunit.ttf,BoldItalicFont=cmuntx.ttf]{cmunrm.ttf}, and {\ttfamily \setmainfont[Path=/usr/share/fonts/truetype/cmu/,UprightFont=cmunrm.ttf,BoldFont=cmunbx.ttf,ItalicFont=cmunti.ttf,BoldItalicFont=cmunbi.ttf]{cmuntt.ttf}\setmonofont[Path=/usr/share/fonts/truetype/cmu/,UprightFont=cmuntt.ttf,BoldFont=cmuntb.ttf,ItalicFont=cmunit.ttf,BoldItalicFont=cmuntx.ttf]{cmuntt.ttf}\ttfamily *.pdf}{$\text{ }$}\setmainfont[Path=/usr/share/fonts/truetype/cmu/,UprightFont=cmunrm.ttf,BoldFont=cmunbx.ttf,ItalicFont=cmunti.ttf,BoldItalicFont=cmunbi.ttf]{cmunrm.ttf}\setmonofont[Path=/usr/share/fonts/truetype/cmu/,UprightFont=cmuntt.ttf,BoldFont=cmuntb.ttf,ItalicFont=cmunit.ttf,BoldItalicFont=cmuntx.ttf]{cmunrm.ttf} files).
All LaTeX classes ({\ttfamily \setmainfont[Path=/usr/share/fonts/truetype/cmu/,UprightFont=cmunrm.ttf,BoldFont=cmunbx.ttf,ItalicFont=cmunti.ttf,BoldItalicFont=cmunbi.ttf]{cmuntt.ttf}\setmonofont[Path=/usr/share/fonts/truetype/cmu/,UprightFont=cmuntt.ttf,BoldFont=cmuntb.ttf,ItalicFont=cmunit.ttf,BoldItalicFont=cmuntx.ttf]{cmuntt.ttf}\ttfamily *.cls}\setmainfont[Path=/usr/share/fonts/truetype/cmu/,UprightFont=cmunrm.ttf,BoldFont=cmunbx.ttf,ItalicFont=cmunti.ttf,BoldItalicFont=cmunbi.ttf]{cmunrm.ttf}\setmonofont[Path=/usr/share/fonts/truetype/cmu/,UprightFont=cmuntt.ttf,BoldFont=cmuntb.ttf,ItalicFont=cmunit.ttf,BoldItalicFont=cmuntx.ttf]{cmunrm.ttf}), LaTeX styles ({\ttfamily \setmainfont[Path=/usr/share/fonts/truetype/cmu/,UprightFont=cmunrm.ttf,BoldFont=cmunbx.ttf,ItalicFont=cmunti.ttf,BoldItalicFont=cmunbi.ttf]{cmuntt.ttf}\setmonofont[Path=/usr/share/fonts/truetype/cmu/,UprightFont=cmuntt.ttf,BoldFont=cmuntb.ttf,ItalicFont=cmunit.ttf,BoldItalicFont=cmuntx.ttf]{cmuntt.ttf}\ttfamily *.sty}\setmainfont[Path=/usr/share/fonts/truetype/cmu/,UprightFont=cmunrm.ttf,BoldFont=cmunbx.ttf,ItalicFont=cmunti.ttf,BoldItalicFont=cmunbi.ttf]{cmunrm.ttf}\setmonofont[Path=/usr/share/fonts/truetype/cmu/,UprightFont=cmuntt.ttf,BoldFont=cmuntb.ttf,ItalicFont=cmunit.ttf,BoldItalicFont=cmuntx.ttf]{cmunrm.ttf}),
BibTeX data bases ({\ttfamily \setmainfont[Path=/usr/share/fonts/truetype/cmu/,UprightFont=cmunrm.ttf,BoldFont=cmunbx.ttf,ItalicFont=cmunti.ttf,BoldItalicFont=cmunbi.ttf]{cmuntt.ttf}\setmonofont[Path=/usr/share/fonts/truetype/cmu/,UprightFont=cmuntt.ttf,BoldFont=cmuntb.ttf,ItalicFont=cmunit.ttf,BoldItalicFont=cmuntx.ttf]{cmuntt.ttf}\ttfamily *.bib}\setmainfont[Path=/usr/share/fonts/truetype/cmu/,UprightFont=cmunrm.ttf,BoldFont=cmunbx.ttf,ItalicFont=cmunti.ttf,BoldItalicFont=cmunbi.ttf]{cmunrm.ttf}\setmonofont[Path=/usr/share/fonts/truetype/cmu/,UprightFont=cmuntt.ttf,BoldFont=cmuntb.ttf,ItalicFont=cmunit.ttf,BoldItalicFont=cmuntx.ttf]{cmunrm.ttf}), and BibTeX styles ({\ttfamily \setmainfont[Path=/usr/share/fonts/truetype/cmu/,UprightFont=cmunrm.ttf,BoldFont=cmunbx.ttf,ItalicFont=cmunti.ttf,BoldItalicFont=cmunbi.ttf]{cmuntt.ttf}\setmonofont[Path=/usr/share/fonts/truetype/cmu/,UprightFont=cmuntt.ttf,BoldFont=cmuntb.ttf,ItalicFont=cmunit.ttf,BoldItalicFont=cmuntx.ttf]{cmuntt.ttf}\ttfamily *.bst}\setmainfont[Path=/usr/share/fonts/truetype/cmu/,UprightFont=cmunrm.ttf,BoldFont=cmunbx.ttf,ItalicFont=cmunti.ttf,BoldItalicFont=cmunbi.ttf]{cmunrm.ttf}\setmonofont[Path=/usr/share/fonts/truetype/cmu/,UprightFont=cmuntt.ttf,BoldFont=cmuntb.ttf,ItalicFont=cmunit.ttf,BoldItalicFont=cmuntx.ttf]{cmunrm.ttf})
generally should be hosted
in the repository of the common {\ttfamily \setmainfont[Path=/usr/share/fonts/truetype/cmu/,UprightFont=cmunrm.ttf,BoldFont=cmunbx.ttf,ItalicFont=cmunti.ttf,BoldItalicFont=cmunbi.ttf]{cmuntt.ttf}\setmonofont[Path=/usr/share/fonts/truetype/cmu/,UprightFont=cmuntt.ttf,BoldFont=cmuntb.ttf,ItalicFont=cmunit.ttf,BoldItalicFont=cmuntx.ttf]{cmuntt.ttf}\ttfamily texmf}{$\text{ }$}\setmainfont[Path=/usr/share/fonts/truetype/cmu/,UprightFont=cmunrm.ttf,BoldFont=cmunbx.ttf,ItalicFont=cmunti.ttf,BoldItalicFont=cmunbi.ttf]{cmunrm.ttf}\setmonofont[Path=/usr/share/fonts/truetype/cmu/,UprightFont=cmuntt.ttf,BoldFont=cmuntb.ttf,ItalicFont=cmunit.ttf,BoldItalicFont=cmuntx.ttf]{cmunrm.ttf} tree,
but they could be included in the respective repository,
if some (external) co-{}authors do not have access
to the common {\ttfamily \setmainfont[Path=/usr/share/fonts/truetype/cmu/,UprightFont=cmunrm.ttf,BoldFont=cmunbx.ttf,ItalicFont=cmunti.ttf,BoldItalicFont=cmunbi.ttf]{cmuntt.ttf}\setmonofont[Path=/usr/share/fonts/truetype/cmu/,UprightFont=cmuntt.ttf,BoldFont=cmuntb.ttf,ItalicFont=cmunit.ttf,BoldItalicFont=cmuntx.ttf]{cmuntt.ttf}\ttfamily texmf}{$\text{ }$}\setmainfont[Path=/usr/share/fonts/truetype/cmu/,UprightFont=cmunrm.ttf,BoldFont=cmunbx.ttf,ItalicFont=cmunti.ttf,BoldItalicFont=cmunbi.ttf]{cmunrm.ttf}\setmonofont[Path=/usr/share/fonts/truetype/cmu/,UprightFont=cmuntt.ttf,BoldFont=cmuntb.ttf,ItalicFont=cmunit.ttf,BoldItalicFont=cmuntx.ttf]{cmunrm.ttf} tree.
On the other hand,
all files that are automatically created or modified during
the compilation process (e.g.
{\ttfamily \setmainfont[Path=/usr/share/fonts/truetype/cmu/,UprightFont=cmunrm.ttf,BoldFont=cmunbx.ttf,ItalicFont=cmunti.ttf,BoldItalicFont=cmunbi.ttf]{cmuntt.ttf}\setmonofont[Path=/usr/share/fonts/truetype/cmu/,UprightFont=cmuntt.ttf,BoldFont=cmuntb.ttf,ItalicFont=cmunit.ttf,BoldItalicFont=cmuntx.ttf]{cmuntt.ttf}\ttfamily *.aut}\setmainfont[Path=/usr/share/fonts/truetype/cmu/,UprightFont=cmunrm.ttf,BoldFont=cmunbx.ttf,ItalicFont=cmunti.ttf,BoldItalicFont=cmunbi.ttf]{cmunrm.ttf}\setmonofont[Path=/usr/share/fonts/truetype/cmu/,UprightFont=cmuntt.ttf,BoldFont=cmuntb.ttf,ItalicFont=cmunit.ttf,BoldItalicFont=cmuntx.ttf]{cmunrm.ttf}, {\ttfamily \setmainfont[Path=/usr/share/fonts/truetype/cmu/,UprightFont=cmunrm.ttf,BoldFont=cmunbx.ttf,ItalicFont=cmunti.ttf,BoldItalicFont=cmunbi.ttf]{cmuntt.ttf}\setmonofont[Path=/usr/share/fonts/truetype/cmu/,UprightFont=cmuntt.ttf,BoldFont=cmuntb.ttf,ItalicFont=cmunit.ttf,BoldItalicFont=cmuntx.ttf]{cmuntt.ttf}\ttfamily *.aux}\setmainfont[Path=/usr/share/fonts/truetype/cmu/,UprightFont=cmunrm.ttf,BoldFont=cmunbx.ttf,ItalicFont=cmunti.ttf,BoldItalicFont=cmunbi.ttf]{cmunrm.ttf}\setmonofont[Path=/usr/share/fonts/truetype/cmu/,UprightFont=cmuntt.ttf,BoldFont=cmuntb.ttf,ItalicFont=cmunit.ttf,BoldItalicFont=cmuntx.ttf]{cmunrm.ttf}, {\ttfamily \setmainfont[Path=/usr/share/fonts/truetype/cmu/,UprightFont=cmunrm.ttf,BoldFont=cmunbx.ttf,ItalicFont=cmunti.ttf,BoldItalicFont=cmunbi.ttf]{cmuntt.ttf}\setmonofont[Path=/usr/share/fonts/truetype/cmu/,UprightFont=cmuntt.ttf,BoldFont=cmuntb.ttf,ItalicFont=cmunit.ttf,BoldItalicFont=cmuntx.ttf]{cmuntt.ttf}\ttfamily *.bbl}\setmainfont[Path=/usr/share/fonts/truetype/cmu/,UprightFont=cmunrm.ttf,BoldFont=cmunbx.ttf,ItalicFont=cmunti.ttf,BoldItalicFont=cmunbi.ttf]{cmunrm.ttf}\setmonofont[Path=/usr/share/fonts/truetype/cmu/,UprightFont=cmuntt.ttf,BoldFont=cmuntb.ttf,ItalicFont=cmunit.ttf,BoldItalicFont=cmuntx.ttf]{cmunrm.ttf},
{\ttfamily \setmainfont[Path=/usr/share/fonts/truetype/cmu/,UprightFont=cmunrm.ttf,BoldFont=cmunbx.ttf,ItalicFont=cmunti.ttf,BoldItalicFont=cmunbi.ttf]{cmuntt.ttf}\setmonofont[Path=/usr/share/fonts/truetype/cmu/,UprightFont=cmuntt.ttf,BoldFont=cmuntb.ttf,ItalicFont=cmunit.ttf,BoldItalicFont=cmuntx.ttf]{cmuntt.ttf}\ttfamily *.bix}\setmainfont[Path=/usr/share/fonts/truetype/cmu/,UprightFont=cmunrm.ttf,BoldFont=cmunbx.ttf,ItalicFont=cmunti.ttf,BoldItalicFont=cmunbi.ttf]{cmunrm.ttf}\setmonofont[Path=/usr/share/fonts/truetype/cmu/,UprightFont=cmuntt.ttf,BoldFont=cmuntb.ttf,ItalicFont=cmunit.ttf,BoldItalicFont=cmuntx.ttf]{cmunrm.ttf}, {\ttfamily \setmainfont[Path=/usr/share/fonts/truetype/cmu/,UprightFont=cmunrm.ttf,BoldFont=cmunbx.ttf,ItalicFont=cmunti.ttf,BoldItalicFont=cmunbi.ttf]{cmuntt.ttf}\setmonofont[Path=/usr/share/fonts/truetype/cmu/,UprightFont=cmuntt.ttf,BoldFont=cmuntb.ttf,ItalicFont=cmunit.ttf,BoldItalicFont=cmuntx.ttf]{cmuntt.ttf}\ttfamily *.blg}\setmainfont[Path=/usr/share/fonts/truetype/cmu/,UprightFont=cmunrm.ttf,BoldFont=cmunbx.ttf,ItalicFont=cmunti.ttf,BoldItalicFont=cmunbi.ttf]{cmunrm.ttf}\setmonofont[Path=/usr/share/fonts/truetype/cmu/,UprightFont=cmuntt.ttf,BoldFont=cmuntb.ttf,ItalicFont=cmunit.ttf,BoldItalicFont=cmuntx.ttf]{cmunrm.ttf},
{\ttfamily \setmainfont[Path=/usr/share/fonts/truetype/cmu/,UprightFont=cmunrm.ttf,BoldFont=cmunbx.ttf,ItalicFont=cmunti.ttf,BoldItalicFont=cmunbi.ttf]{cmuntt.ttf}\setmonofont[Path=/usr/share/fonts/truetype/cmu/,UprightFont=cmuntt.ttf,BoldFont=cmuntb.ttf,ItalicFont=cmunit.ttf,BoldItalicFont=cmuntx.ttf]{cmuntt.ttf}\ttfamily *.dvi}\setmainfont[Path=/usr/share/fonts/truetype/cmu/,UprightFont=cmunrm.ttf,BoldFont=cmunbx.ttf,ItalicFont=cmunti.ttf,BoldItalicFont=cmunbi.ttf]{cmunrm.ttf}\setmonofont[Path=/usr/share/fonts/truetype/cmu/,UprightFont=cmuntt.ttf,BoldFont=cmuntb.ttf,ItalicFont=cmunit.ttf,BoldItalicFont=cmuntx.ttf]{cmunrm.ttf}, {\ttfamily \setmainfont[Path=/usr/share/fonts/truetype/cmu/,UprightFont=cmunrm.ttf,BoldFont=cmunbx.ttf,ItalicFont=cmunti.ttf,BoldItalicFont=cmunbi.ttf]{cmuntt.ttf}\setmonofont[Path=/usr/share/fonts/truetype/cmu/,UprightFont=cmuntt.ttf,BoldFont=cmuntb.ttf,ItalicFont=cmunit.ttf,BoldItalicFont=cmuntx.ttf]{cmuntt.ttf}\ttfamily *.glo}\setmainfont[Path=/usr/share/fonts/truetype/cmu/,UprightFont=cmunrm.ttf,BoldFont=cmunbx.ttf,ItalicFont=cmunti.ttf,BoldItalicFont=cmunbi.ttf]{cmunrm.ttf}\setmonofont[Path=/usr/share/fonts/truetype/cmu/,UprightFont=cmuntt.ttf,BoldFont=cmuntb.ttf,ItalicFont=cmunit.ttf,BoldItalicFont=cmuntx.ttf]{cmunrm.ttf}, {\ttfamily \setmainfont[Path=/usr/share/fonts/truetype/cmu/,UprightFont=cmunrm.ttf,BoldFont=cmunbx.ttf,ItalicFont=cmunti.ttf,BoldItalicFont=cmunbi.ttf]{cmuntt.ttf}\setmonofont[Path=/usr/share/fonts/truetype/cmu/,UprightFont=cmuntt.ttf,BoldFont=cmuntb.ttf,ItalicFont=cmunit.ttf,BoldItalicFont=cmuntx.ttf]{cmuntt.ttf}\ttfamily *.gls}\setmainfont[Path=/usr/share/fonts/truetype/cmu/,UprightFont=cmunrm.ttf,BoldFont=cmunbx.ttf,ItalicFont=cmunti.ttf,BoldItalicFont=cmunbi.ttf]{cmunrm.ttf}\setmonofont[Path=/usr/share/fonts/truetype/cmu/,UprightFont=cmuntt.ttf,BoldFont=cmuntb.ttf,ItalicFont=cmunit.ttf,BoldItalicFont=cmuntx.ttf]{cmunrm.ttf}, {\ttfamily \setmainfont[Path=/usr/share/fonts/truetype/cmu/,UprightFont=cmunrm.ttf,BoldFont=cmunbx.ttf,ItalicFont=cmunti.ttf,BoldItalicFont=cmunbi.ttf]{cmuntt.ttf}\setmonofont[Path=/usr/share/fonts/truetype/cmu/,UprightFont=cmuntt.ttf,BoldFont=cmuntb.ttf,ItalicFont=cmunit.ttf,BoldItalicFont=cmuntx.ttf]{cmuntt.ttf}\ttfamily *.idx}\setmainfont[Path=/usr/share/fonts/truetype/cmu/,UprightFont=cmunrm.ttf,BoldFont=cmunbx.ttf,ItalicFont=cmunti.ttf,BoldItalicFont=cmunbi.ttf]{cmunrm.ttf}\setmonofont[Path=/usr/share/fonts/truetype/cmu/,UprightFont=cmuntt.ttf,BoldFont=cmuntb.ttf,ItalicFont=cmunit.ttf,BoldItalicFont=cmuntx.ttf]{cmunrm.ttf},
{\ttfamily \setmainfont[Path=/usr/share/fonts/truetype/cmu/,UprightFont=cmunrm.ttf,BoldFont=cmunbx.ttf,ItalicFont=cmunti.ttf,BoldItalicFont=cmunbi.ttf]{cmuntt.ttf}\setmonofont[Path=/usr/share/fonts/truetype/cmu/,UprightFont=cmuntt.ttf,BoldFont=cmuntb.ttf,ItalicFont=cmunit.ttf,BoldItalicFont=cmuntx.ttf]{cmuntt.ttf}\ttfamily *.ilg}\setmainfont[Path=/usr/share/fonts/truetype/cmu/,UprightFont=cmunrm.ttf,BoldFont=cmunbx.ttf,ItalicFont=cmunti.ttf,BoldItalicFont=cmunbi.ttf]{cmunrm.ttf}\setmonofont[Path=/usr/share/fonts/truetype/cmu/,UprightFont=cmuntt.ttf,BoldFont=cmuntb.ttf,ItalicFont=cmunit.ttf,BoldItalicFont=cmuntx.ttf]{cmunrm.ttf}, {\ttfamily \setmainfont[Path=/usr/share/fonts/truetype/cmu/,UprightFont=cmunrm.ttf,BoldFont=cmunbx.ttf,ItalicFont=cmunti.ttf,BoldItalicFont=cmunbi.ttf]{cmuntt.ttf}\setmonofont[Path=/usr/share/fonts/truetype/cmu/,UprightFont=cmuntt.ttf,BoldFont=cmuntb.ttf,ItalicFont=cmunit.ttf,BoldItalicFont=cmuntx.ttf]{cmuntt.ttf}\ttfamily *.ind}\setmainfont[Path=/usr/share/fonts/truetype/cmu/,UprightFont=cmunrm.ttf,BoldFont=cmunbx.ttf,ItalicFont=cmunti.ttf,BoldItalicFont=cmunbi.ttf]{cmunrm.ttf}\setmonofont[Path=/usr/share/fonts/truetype/cmu/,UprightFont=cmuntt.ttf,BoldFont=cmuntb.ttf,ItalicFont=cmunit.ttf,BoldItalicFont=cmuntx.ttf]{cmunrm.ttf}, {\ttfamily \setmainfont[Path=/usr/share/fonts/truetype/cmu/,UprightFont=cmunrm.ttf,BoldFont=cmunbx.ttf,ItalicFont=cmunti.ttf,BoldItalicFont=cmunbi.ttf]{cmuntt.ttf}\setmonofont[Path=/usr/share/fonts/truetype/cmu/,UprightFont=cmuntt.ttf,BoldFont=cmuntb.ttf,ItalicFont=cmunit.ttf,BoldItalicFont=cmuntx.ttf]{cmuntt.ttf}\ttfamily *.ist}\setmainfont[Path=/usr/share/fonts/truetype/cmu/,UprightFont=cmunrm.ttf,BoldFont=cmunbx.ttf,ItalicFont=cmunti.ttf,BoldItalicFont=cmunbi.ttf]{cmunrm.ttf}\setmonofont[Path=/usr/share/fonts/truetype/cmu/,UprightFont=cmuntt.ttf,BoldFont=cmuntb.ttf,ItalicFont=cmunit.ttf,BoldItalicFont=cmuntx.ttf]{cmunrm.ttf},
{\ttfamily \setmainfont[Path=/usr/share/fonts/truetype/cmu/,UprightFont=cmunrm.ttf,BoldFont=cmunbx.ttf,ItalicFont=cmunti.ttf,BoldItalicFont=cmunbi.ttf]{cmuntt.ttf}\setmonofont[Path=/usr/share/fonts/truetype/cmu/,UprightFont=cmuntt.ttf,BoldFont=cmuntb.ttf,ItalicFont=cmunit.ttf,BoldItalicFont=cmuntx.ttf]{cmuntt.ttf}\ttfamily *.lof}\setmainfont[Path=/usr/share/fonts/truetype/cmu/,UprightFont=cmunrm.ttf,BoldFont=cmunbx.ttf,ItalicFont=cmunti.ttf,BoldItalicFont=cmunbi.ttf]{cmunrm.ttf}\setmonofont[Path=/usr/share/fonts/truetype/cmu/,UprightFont=cmuntt.ttf,BoldFont=cmuntb.ttf,ItalicFont=cmunit.ttf,BoldItalicFont=cmuntx.ttf]{cmunrm.ttf}, {\ttfamily \setmainfont[Path=/usr/share/fonts/truetype/cmu/,UprightFont=cmunrm.ttf,BoldFont=cmunbx.ttf,ItalicFont=cmunti.ttf,BoldItalicFont=cmunbi.ttf]{cmuntt.ttf}\setmonofont[Path=/usr/share/fonts/truetype/cmu/,UprightFont=cmuntt.ttf,BoldFont=cmuntb.ttf,ItalicFont=cmunit.ttf,BoldItalicFont=cmuntx.ttf]{cmuntt.ttf}\ttfamily *.log}\setmainfont[Path=/usr/share/fonts/truetype/cmu/,UprightFont=cmunrm.ttf,BoldFont=cmunbx.ttf,ItalicFont=cmunti.ttf,BoldItalicFont=cmunbi.ttf]{cmunrm.ttf}\setmonofont[Path=/usr/share/fonts/truetype/cmu/,UprightFont=cmuntt.ttf,BoldFont=cmuntb.ttf,ItalicFont=cmunit.ttf,BoldItalicFont=cmuntx.ttf]{cmunrm.ttf}, {\ttfamily \setmainfont[Path=/usr/share/fonts/truetype/cmu/,UprightFont=cmunrm.ttf,BoldFont=cmunbx.ttf,ItalicFont=cmunti.ttf,BoldItalicFont=cmunbi.ttf]{cmuntt.ttf}\setmonofont[Path=/usr/share/fonts/truetype/cmu/,UprightFont=cmuntt.ttf,BoldFont=cmuntb.ttf,ItalicFont=cmunit.ttf,BoldItalicFont=cmuntx.ttf]{cmuntt.ttf}\ttfamily *.lot}\setmainfont[Path=/usr/share/fonts/truetype/cmu/,UprightFont=cmunrm.ttf,BoldFont=cmunbx.ttf,ItalicFont=cmunti.ttf,BoldItalicFont=cmunbi.ttf]{cmunrm.ttf}\setmonofont[Path=/usr/share/fonts/truetype/cmu/,UprightFont=cmuntt.ttf,BoldFont=cmuntb.ttf,ItalicFont=cmunit.ttf,BoldItalicFont=cmuntx.ttf]{cmunrm.ttf}, {\ttfamily \setmainfont[Path=/usr/share/fonts/truetype/cmu/,UprightFont=cmunrm.ttf,BoldFont=cmunbx.ttf,ItalicFont=cmunti.ttf,BoldItalicFont=cmunbi.ttf]{cmuntt.ttf}\setmonofont[Path=/usr/share/fonts/truetype/cmu/,UprightFont=cmuntt.ttf,BoldFont=cmuntb.ttf,ItalicFont=cmunit.ttf,BoldItalicFont=cmuntx.ttf]{cmuntt.ttf}\ttfamily *.nav}\setmainfont[Path=/usr/share/fonts/truetype/cmu/,UprightFont=cmunrm.ttf,BoldFont=cmunbx.ttf,ItalicFont=cmunti.ttf,BoldItalicFont=cmunbi.ttf]{cmunrm.ttf}\setmonofont[Path=/usr/share/fonts/truetype/cmu/,UprightFont=cmuntt.ttf,BoldFont=cmuntb.ttf,ItalicFont=cmunit.ttf,BoldItalicFont=cmuntx.ttf]{cmunrm.ttf}, {\ttfamily \setmainfont[Path=/usr/share/fonts/truetype/cmu/,UprightFont=cmunrm.ttf,BoldFont=cmunbx.ttf,ItalicFont=cmunti.ttf,BoldItalicFont=cmunbi.ttf]{cmuntt.ttf}\setmonofont[Path=/usr/share/fonts/truetype/cmu/,UprightFont=cmuntt.ttf,BoldFont=cmuntb.ttf,ItalicFont=cmunit.ttf,BoldItalicFont=cmuntx.ttf]{cmuntt.ttf}\ttfamily *.nlo}\setmainfont[Path=/usr/share/fonts/truetype/cmu/,UprightFont=cmunrm.ttf,BoldFont=cmunbx.ttf,ItalicFont=cmunti.ttf,BoldItalicFont=cmunbi.ttf]{cmunrm.ttf}\setmonofont[Path=/usr/share/fonts/truetype/cmu/,UprightFont=cmuntt.ttf,BoldFont=cmuntb.ttf,ItalicFont=cmunit.ttf,BoldItalicFont=cmuntx.ttf]{cmunrm.ttf},
{\ttfamily \setmainfont[Path=/usr/share/fonts/truetype/cmu/,UprightFont=cmunrm.ttf,BoldFont=cmunbx.ttf,ItalicFont=cmunti.ttf,BoldItalicFont=cmunbi.ttf]{cmuntt.ttf}\setmonofont[Path=/usr/share/fonts/truetype/cmu/,UprightFont=cmuntt.ttf,BoldFont=cmuntb.ttf,ItalicFont=cmunit.ttf,BoldItalicFont=cmuntx.ttf]{cmuntt.ttf}\ttfamily *.out}\setmainfont[Path=/usr/share/fonts/truetype/cmu/,UprightFont=cmunrm.ttf,BoldFont=cmunbx.ttf,ItalicFont=cmunti.ttf,BoldItalicFont=cmunbi.ttf]{cmunrm.ttf}\setmonofont[Path=/usr/share/fonts/truetype/cmu/,UprightFont=cmuntt.ttf,BoldFont=cmuntb.ttf,ItalicFont=cmunit.ttf,BoldItalicFont=cmuntx.ttf]{cmunrm.ttf}, {\ttfamily \setmainfont[Path=/usr/share/fonts/truetype/cmu/,UprightFont=cmunrm.ttf,BoldFont=cmunbx.ttf,ItalicFont=cmunti.ttf,BoldItalicFont=cmunbi.ttf]{cmuntt.ttf}\setmonofont[Path=/usr/share/fonts/truetype/cmu/,UprightFont=cmuntt.ttf,BoldFont=cmuntb.ttf,ItalicFont=cmunit.ttf,BoldItalicFont=cmuntx.ttf]{cmuntt.ttf}\ttfamily *.pdf}\setmainfont[Path=/usr/share/fonts/truetype/cmu/,UprightFont=cmunrm.ttf,BoldFont=cmunbx.ttf,ItalicFont=cmunti.ttf,BoldItalicFont=cmunbi.ttf]{cmunrm.ttf}\setmonofont[Path=/usr/share/fonts/truetype/cmu/,UprightFont=cmuntt.ttf,BoldFont=cmuntb.ttf,ItalicFont=cmunit.ttf,BoldItalicFont=cmuntx.ttf]{cmunrm.ttf}, {\ttfamily \setmainfont[Path=/usr/share/fonts/truetype/cmu/,UprightFont=cmunrm.ttf,BoldFont=cmunbx.ttf,ItalicFont=cmunti.ttf,BoldItalicFont=cmunbi.ttf]{cmuntt.ttf}\setmonofont[Path=/usr/share/fonts/truetype/cmu/,UprightFont=cmuntt.ttf,BoldFont=cmuntb.ttf,ItalicFont=cmunit.ttf,BoldItalicFont=cmuntx.ttf]{cmuntt.ttf}\ttfamily *.ps}\setmainfont[Path=/usr/share/fonts/truetype/cmu/,UprightFont=cmunrm.ttf,BoldFont=cmunbx.ttf,ItalicFont=cmunti.ttf,BoldItalicFont=cmunbi.ttf]{cmunrm.ttf}\setmonofont[Path=/usr/share/fonts/truetype/cmu/,UprightFont=cmuntt.ttf,BoldFont=cmuntb.ttf,ItalicFont=cmunit.ttf,BoldItalicFont=cmuntx.ttf]{cmunrm.ttf},
{\ttfamily \setmainfont[Path=/usr/share/fonts/truetype/cmu/,UprightFont=cmunrm.ttf,BoldFont=cmunbx.ttf,ItalicFont=cmunti.ttf,BoldItalicFont=cmunbi.ttf]{cmuntt.ttf}\setmonofont[Path=/usr/share/fonts/truetype/cmu/,UprightFont=cmuntt.ttf,BoldFont=cmuntb.ttf,ItalicFont=cmunit.ttf,BoldItalicFont=cmuntx.ttf]{cmuntt.ttf}\ttfamily *.snm}\setmainfont[Path=/usr/share/fonts/truetype/cmu/,UprightFont=cmunrm.ttf,BoldFont=cmunbx.ttf,ItalicFont=cmunti.ttf,BoldItalicFont=cmunbi.ttf]{cmunrm.ttf}\setmonofont[Path=/usr/share/fonts/truetype/cmu/,UprightFont=cmuntt.ttf,BoldFont=cmuntb.ttf,ItalicFont=cmunit.ttf,BoldItalicFont=cmuntx.ttf]{cmunrm.ttf},
and {\ttfamily \setmainfont[Path=/usr/share/fonts/truetype/cmu/,UprightFont=cmunrm.ttf,BoldFont=cmunbx.ttf,ItalicFont=cmunti.ttf,BoldItalicFont=cmunbi.ttf]{cmuntt.ttf}\setmonofont[Path=/usr/share/fonts/truetype/cmu/,UprightFont=cmuntt.ttf,BoldFont=cmuntb.ttf,ItalicFont=cmunit.ttf,BoldItalicFont=cmuntx.ttf]{cmuntt.ttf}\ttfamily *.toc}{$\text{ }$}\setmainfont[Path=/usr/share/fonts/truetype/cmu/,UprightFont=cmunrm.ttf,BoldFont=cmunbx.ttf,ItalicFont=cmunti.ttf,BoldItalicFont=cmunbi.ttf]{cmunrm.ttf}\setmonofont[Path=/usr/share/fonts/truetype/cmu/,UprightFont=cmuntt.ttf,BoldFont=cmuntb.ttf,ItalicFont=cmunit.ttf,BoldItalicFont=cmuntx.ttf]{cmunrm.ttf} files)
or by the (LaTeX or BibTeX) editor (e.g.
{\ttfamily \setmainfont[Path=/usr/share/fonts/truetype/cmu/,UprightFont=cmunrm.ttf,BoldFont=cmunbx.ttf,ItalicFont=cmunti.ttf,BoldItalicFont=cmunbi.ttf]{cmuntt.ttf}\setmonofont[Path=/usr/share/fonts/truetype/cmu/,UprightFont=cmuntt.ttf,BoldFont=cmuntb.ttf,ItalicFont=cmunit.ttf,BoldItalicFont=cmuntx.ttf]{cmuntt.ttf}\ttfamily *.bak}\setmainfont[Path=/usr/share/fonts/truetype/cmu/,UprightFont=cmunrm.ttf,BoldFont=cmunbx.ttf,ItalicFont=cmunti.ttf,BoldItalicFont=cmunbi.ttf]{cmunrm.ttf}\setmonofont[Path=/usr/share/fonts/truetype/cmu/,UprightFont=cmuntt.ttf,BoldFont=cmuntb.ttf,ItalicFont=cmunit.ttf,BoldItalicFont=cmuntx.ttf]{cmunrm.ttf}, {\ttfamily \setmainfont[Path=/usr/share/fonts/truetype/cmu/,UprightFont=cmunrm.ttf,BoldFont=cmunbx.ttf,ItalicFont=cmunti.ttf,BoldItalicFont=cmunbi.ttf]{cmuntt.ttf}\setmonofont[Path=/usr/share/fonts/truetype/cmu/,UprightFont=cmuntt.ttf,BoldFont=cmuntb.ttf,ItalicFont=cmunit.ttf,BoldItalicFont=cmuntx.ttf]{cmuntt.ttf}\ttfamily *.bib\~{}}\setmainfont[Path=/usr/share/fonts/truetype/cmu/,UprightFont=cmunrm.ttf,BoldFont=cmunbx.ttf,ItalicFont=cmunti.ttf,BoldItalicFont=cmunbi.ttf]{cmunrm.ttf}\setmonofont[Path=/usr/share/fonts/truetype/cmu/,UprightFont=cmuntt.ttf,BoldFont=cmuntb.ttf,ItalicFont=cmunit.ttf,BoldItalicFont=cmuntx.ttf]{cmunrm.ttf}, {\ttfamily \setmainfont[Path=/usr/share/fonts/truetype/cmu/,UprightFont=cmunrm.ttf,BoldFont=cmunbx.ttf,ItalicFont=cmunti.ttf,BoldItalicFont=cmunbi.ttf]{cmuntt.ttf}\setmonofont[Path=/usr/share/fonts/truetype/cmu/,UprightFont=cmuntt.ttf,BoldFont=cmuntb.ttf,ItalicFont=cmunit.ttf,BoldItalicFont=cmuntx.ttf]{cmuntt.ttf}\ttfamily *.kilepr}\setmainfont[Path=/usr/share/fonts/truetype/cmu/,UprightFont=cmunrm.ttf,BoldFont=cmunbx.ttf,ItalicFont=cmunti.ttf,BoldItalicFont=cmunbi.ttf]{cmunrm.ttf}\setmonofont[Path=/usr/share/fonts/truetype/cmu/,UprightFont=cmuntt.ttf,BoldFont=cmuntb.ttf,ItalicFont=cmunit.ttf,BoldItalicFont=cmuntx.ttf]{cmunrm.ttf}, {\ttfamily \setmainfont[Path=/usr/share/fonts/truetype/cmu/,UprightFont=cmunrm.ttf,BoldFont=cmunbx.ttf,ItalicFont=cmunti.ttf,BoldItalicFont=cmunbi.ttf]{cmuntt.ttf}\setmonofont[Path=/usr/share/fonts/truetype/cmu/,UprightFont=cmuntt.ttf,BoldFont=cmuntb.ttf,ItalicFont=cmunit.ttf,BoldItalicFont=cmuntx.ttf]{cmuntt.ttf}\ttfamily *.prj}\setmainfont[Path=/usr/share/fonts/truetype/cmu/,UprightFont=cmunrm.ttf,BoldFont=cmunbx.ttf,ItalicFont=cmunti.ttf,BoldItalicFont=cmunbi.ttf]{cmunrm.ttf}\setmonofont[Path=/usr/share/fonts/truetype/cmu/,UprightFont=cmuntt.ttf,BoldFont=cmuntb.ttf,ItalicFont=cmunit.ttf,BoldItalicFont=cmuntx.ttf]{cmunrm.ttf},
{\ttfamily \setmainfont[Path=/usr/share/fonts/truetype/cmu/,UprightFont=cmunrm.ttf,BoldFont=cmunbx.ttf,ItalicFont=cmunti.ttf,BoldItalicFont=cmunbi.ttf]{cmuntt.ttf}\setmonofont[Path=/usr/share/fonts/truetype/cmu/,UprightFont=cmuntt.ttf,BoldFont=cmuntb.ttf,ItalicFont=cmunit.ttf,BoldItalicFont=cmuntx.ttf]{cmuntt.ttf}\ttfamily *.sav}\setmainfont[Path=/usr/share/fonts/truetype/cmu/,UprightFont=cmunrm.ttf,BoldFont=cmunbx.ttf,ItalicFont=cmunti.ttf,BoldItalicFont=cmunbi.ttf]{cmunrm.ttf}\setmonofont[Path=/usr/share/fonts/truetype/cmu/,UprightFont=cmuntt.ttf,BoldFont=cmuntb.ttf,ItalicFont=cmunit.ttf,BoldItalicFont=cmuntx.ttf]{cmunrm.ttf}, {\ttfamily \setmainfont[Path=/usr/share/fonts/truetype/cmu/,UprightFont=cmunrm.ttf,BoldFont=cmunbx.ttf,ItalicFont=cmunti.ttf,BoldItalicFont=cmunbi.ttf]{cmuntt.ttf}\setmonofont[Path=/usr/share/fonts/truetype/cmu/,UprightFont=cmuntt.ttf,BoldFont=cmuntb.ttf,ItalicFont=cmunit.ttf,BoldItalicFont=cmuntx.ttf]{cmuntt.ttf}\ttfamily *.tcp}\setmainfont[Path=/usr/share/fonts/truetype/cmu/,UprightFont=cmunrm.ttf,BoldFont=cmunbx.ttf,ItalicFont=cmunti.ttf,BoldItalicFont=cmunbi.ttf]{cmunrm.ttf}\setmonofont[Path=/usr/share/fonts/truetype/cmu/,UprightFont=cmuntt.ttf,BoldFont=cmuntb.ttf,ItalicFont=cmunit.ttf,BoldItalicFont=cmuntx.ttf]{cmunrm.ttf}, {\ttfamily \setmainfont[Path=/usr/share/fonts/truetype/cmu/,UprightFont=cmunrm.ttf,BoldFont=cmunbx.ttf,ItalicFont=cmunti.ttf,BoldItalicFont=cmunbi.ttf]{cmuntt.ttf}\setmonofont[Path=/usr/share/fonts/truetype/cmu/,UprightFont=cmuntt.ttf,BoldFont=cmuntb.ttf,ItalicFont=cmunit.ttf,BoldItalicFont=cmuntx.ttf]{cmuntt.ttf}\ttfamily *.tmp}\setmainfont[Path=/usr/share/fonts/truetype/cmu/,UprightFont=cmunrm.ttf,BoldFont=cmunbx.ttf,ItalicFont=cmunti.ttf,BoldItalicFont=cmunbi.ttf]{cmunrm.ttf}\setmonofont[Path=/usr/share/fonts/truetype/cmu/,UprightFont=cmuntt.ttf,BoldFont=cmuntb.ttf,ItalicFont=cmunit.ttf,BoldItalicFont=cmuntx.ttf]{cmunrm.ttf}, {\ttfamily \setmainfont[Path=/usr/share/fonts/truetype/cmu/,UprightFont=cmunrm.ttf,BoldFont=cmunbx.ttf,ItalicFont=cmunti.ttf,BoldItalicFont=cmunbi.ttf]{cmuntt.ttf}\setmonofont[Path=/usr/share/fonts/truetype/cmu/,UprightFont=cmuntt.ttf,BoldFont=cmuntb.ttf,ItalicFont=cmunit.ttf,BoldItalicFont=cmuntx.ttf]{cmuntt.ttf}\ttfamily *.tps}\setmainfont[Path=/usr/share/fonts/truetype/cmu/,UprightFont=cmunrm.ttf,BoldFont=cmunbx.ttf,ItalicFont=cmunti.ttf,BoldItalicFont=cmunbi.ttf]{cmunrm.ttf}\setmonofont[Path=/usr/share/fonts/truetype/cmu/,UprightFont=cmuntt.ttf,BoldFont=cmuntb.ttf,ItalicFont=cmunit.ttf,BoldItalicFont=cmuntx.ttf]{cmunrm.ttf},
and {\ttfamily \setmainfont[Path=/usr/share/fonts/truetype/cmu/,UprightFont=cmunrm.ttf,BoldFont=cmunbx.ttf,ItalicFont=cmunti.ttf,BoldItalicFont=cmunbi.ttf]{cmuntt.ttf}\setmonofont[Path=/usr/share/fonts/truetype/cmu/,UprightFont=cmuntt.ttf,BoldFont=cmuntb.ttf,ItalicFont=cmunit.ttf,BoldItalicFont=cmuntx.ttf]{cmuntt.ttf}\ttfamily *.tex\~{}}{$\text{ }$}\setmainfont[Path=/usr/share/fonts/truetype/cmu/,UprightFont=cmunrm.ttf,BoldFont=cmunbx.ttf,ItalicFont=cmunti.ttf,BoldItalicFont=cmunbi.ttf]{cmunrm.ttf}\setmonofont[Path=/usr/share/fonts/truetype/cmu/,UprightFont=cmuntt.ttf,BoldFont=cmuntb.ttf,ItalicFont=cmunit.ttf,BoldItalicFont=cmuntx.ttf]{cmunrm.ttf} files)
generally should be {\bfseries \setmainfont[Path=/usr/share/fonts/truetype/cmu/,UprightFont=cmunrm.ttf,BoldFont=cmunbx.ttf,ItalicFont=cmunti.ttf,BoldItalicFont=cmunbi.ttf]{cmunbx.ttf}\setmonofont[Path=/usr/share/fonts/truetype/cmu/,UprightFont=cmuntt.ttf,BoldFont=cmuntb.ttf,ItalicFont=cmunit.ttf,BoldItalicFont=cmuntx.ttf]{cmunbx.ttf}\bfseries not}{$\text{ }$}\setmainfont[Path=/usr/share/fonts/truetype/cmu/,UprightFont=cmunrm.ttf,BoldFont=cmunbx.ttf,ItalicFont=cmunti.ttf,BoldItalicFont=cmunbi.ttf]{cmunrm.ttf}\setmonofont[Path=/usr/share/fonts/truetype/cmu/,UprightFont=cmuntt.ttf,BoldFont=cmuntb.ttf,ItalicFont=cmunit.ttf,BoldItalicFont=cmuntx.ttf]{cmunrm.ttf} under version control,
because these files are not necessary for compilation
and generally do not include additional information.
Furthermore, these files are regularly modified
so that conflicts are very likely.
\section{{\itshape \setmainfont[Path=/usr/share/fonts/truetype/cmu/,UprightFont=cmunrm.ttf,BoldFont=cmunbx.ttf,ItalicFont=cmunti.ttf,BoldItalicFont=cmunbi.ttf]{cmunti.ttf}\setmonofont[Path=/usr/share/fonts/truetype/cmu/,UprightFont=cmuntt.ttf,BoldFont=cmuntb.ttf,ItalicFont=cmunit.ttf,BoldItalicFont=cmuntx.ttf]{cmunti.ttf}\itshape Subversion}{$\text{ }$}\setmainfont[Path=/usr/share/fonts/truetype/cmu/,UprightFont=cmunrm.ttf,BoldFont=cmunbx.ttf,ItalicFont=cmunti.ttf,BoldItalicFont=cmunbi.ttf]{cmunrm.ttf}\setmonofont[Path=/usr/share/fonts/truetype/cmu/,UprightFont=cmuntt.ttf,BoldFont=cmuntb.ttf,ItalicFont=cmunit.ttf,BoldItalicFont=cmuntx.ttf]{cmunrm.ttf} really makes the {\bfseries \setmainfont[Path=/usr/share/fonts/truetype/cmu/,UprightFont=cmunrm.ttf,BoldFont=cmunbx.ttf,ItalicFont=cmunti.ttf,BoldItalicFont=cmunbi.ttf]{cmunbx.ttf}\setmonofont[Path=/usr/share/fonts/truetype/cmu/,UprightFont=cmuntt.ttf,BoldFont=cmuntb.ttf,ItalicFont=cmunit.ttf,BoldItalicFont=cmuntx.ttf]{cmunbx.ttf}\bfseries diff}\setmainfont[Path=/usr/share/fonts/truetype/cmu/,UprightFont=cmunrm.ttf,BoldFont=cmunbx.ttf,ItalicFont=cmunti.ttf,BoldItalicFont=cmunbi.ttf]{cmunrm.ttf}\setmonofont[Path=/usr/share/fonts/truetype/cmu/,UprightFont=cmuntt.ttf,BoldFont=cmuntb.ttf,ItalicFont=cmunit.ttf,BoldItalicFont=cmuntx.ttf]{cmunrm.ttf}erence}
\label{918}

A great feature of a version control system is
that all authors can easily trace the workflow of a project
by viewing the differences between arbitrary versions of the files.
Authors are primarily interested in \textquotesingle{}effective\textquotesingle{} modifications
of the source code
that change the compiled document,
but not in \textquotesingle{}ineffective\textquotesingle{} modifications
that have no impact on the compiled document
(e.g. the position of line breaks).
Software tools for comparing text documents (\textquotesingle{}diff tools\textquotesingle{})
generally cannot differentiate
between \textquotesingle{}effective\textquotesingle{} and \textquotesingle{}ineffective\textquotesingle{} modifications;
they highlight both types of modifications.
This considerably increases the effort
to find and review the \textquotesingle{}effective\textquotesingle{} modifications.
Therefore, \textquotesingle{}ineffective\textquotesingle{} modifications should be avoided.

In this sense, it is very important
not to change the positions of line breaks without cause.
Hence, automatic line wrapping of the users\textquotesingle{} LaTeX editors
should be turned off
and line breaks should be added manually.
Otherwise, if a single word in the beginning of a paragraph is added
or removed,
all line breaks of this paragraph might change
so that most diff tools indicate the entire paragraph as modified,
because they compare the files line by line.
The diff tools {\itshape \setmainfont[Path=/usr/share/fonts/truetype/cmu/,UprightFont=cmunrm.ttf,BoldFont=cmunbx.ttf,ItalicFont=cmunti.ttf,BoldItalicFont=cmunbi.ttf]{cmunti.ttf}\setmonofont[Path=/usr/share/fonts/truetype/cmu/,UprightFont=cmuntt.ttf,BoldFont=cmuntb.ttf,ItalicFont=cmunit.ttf,BoldItalicFont=cmuntx.ttf]{cmunti.ttf}\itshape wdiff}{$\text{ }$}\setmainfont[Path=/usr/share/fonts/truetype/cmu/,UprightFont=cmunrm.ttf,BoldFont=cmunbx.ttf,ItalicFont=cmunti.ttf,BoldItalicFont=cmunbi.ttf]{cmunrm.ttf}\setmonofont[Path=/usr/share/fonts/truetype/cmu/,UprightFont=cmuntt.ttf,BoldFont=cmuntb.ttf,ItalicFont=cmunit.ttf,BoldItalicFont=cmuntx.ttf]{cmunrm.ttf} (\myplainurl{http://www.gnu.org/software/wdiff/)}
and {\itshape \setmainfont[Path=/usr/share/fonts/truetype/cmu/,UprightFont=cmunrm.ttf,BoldFont=cmunbx.ttf,ItalicFont=cmunti.ttf,BoldItalicFont=cmunbi.ttf]{cmunti.ttf}\setmonofont[Path=/usr/share/fonts/truetype/cmu/,UprightFont=cmuntt.ttf,BoldFont=cmuntb.ttf,ItalicFont=cmunit.ttf,BoldItalicFont=cmuntx.ttf]{cmunti.ttf}\itshape dwdiff}{$\text{ }$}\setmainfont[Path=/usr/share/fonts/truetype/cmu/,UprightFont=cmunrm.ttf,BoldFont=cmunbx.ttf,ItalicFont=cmunti.ttf,BoldItalicFont=cmunbi.ttf]{cmunrm.ttf}\setmonofont[Path=/usr/share/fonts/truetype/cmu/,UprightFont=cmuntt.ttf,BoldFont=cmuntb.ttf,ItalicFont=cmunit.ttf,BoldItalicFont=cmuntx.ttf]{cmunrm.ttf} (\myplainurl{http://os.ghalkes.nl/dwdiff.html)}
are not affected by the positions of line breaks,
because they compare documents word by word.
However, their output is less clear
so that modifications are more difficult to track.
Moreover, these tools cannot be used directly with the {\itshape \setmainfont[Path=/usr/share/fonts/truetype/cmu/,UprightFont=cmunrm.ttf,BoldFont=cmunbx.ttf,ItalicFont=cmunti.ttf,BoldItalicFont=cmunbi.ttf]{cmunti.ttf}\setmonofont[Path=/usr/share/fonts/truetype/cmu/,UprightFont=cmuntt.ttf,BoldFont=cmuntb.ttf,ItalicFont=cmunit.ttf,BoldItalicFont=cmuntx.ttf]{cmunti.ttf}\itshape Subversion}\setmainfont[Path=/usr/share/fonts/truetype/cmu/,UprightFont=cmunrm.ttf,BoldFont=cmunbx.ttf,ItalicFont=cmunti.ttf,BoldItalicFont=cmunbi.ttf]{cmunrm.ttf}\setmonofont[Path=/usr/share/fonts/truetype/cmu/,UprightFont=cmuntt.ttf,BoldFont=cmuntb.ttf,ItalicFont=cmunit.ttf,BoldItalicFont=cmuntx.ttf]{cmunrm.ttf}
command-{}line switch {\ttfamily \setmainfont[Path=/usr/share/fonts/truetype/cmu/,UprightFont=cmunrm.ttf,BoldFont=cmunbx.ttf,ItalicFont=cmunti.ttf,BoldItalicFont=cmunbi.ttf]{cmuntt.ttf}\setmonofont[Path=/usr/share/fonts/truetype/cmu/,UprightFont=cmuntt.ttf,BoldFont=cmuntb.ttf,ItalicFont=cmunit.ttf,BoldItalicFont=cmuntx.ttf]{cmuntt.ttf}\ttfamily -{}-{}diff-{}cmd}\setmainfont[Path=/usr/share/fonts/truetype/cmu/,UprightFont=cmunrm.ttf,BoldFont=cmunbx.ttf,ItalicFont=cmunti.ttf,BoldItalicFont=cmunbi.ttf]{cmunrm.ttf}\setmonofont[Path=/usr/share/fonts/truetype/cmu/,UprightFont=cmuntt.ttf,BoldFont=cmuntb.ttf,ItalicFont=cmunit.ttf,BoldItalicFont=cmuntx.ttf]{cmunrm.ttf},
but a small wrapper script has to be used
(\myplainurl{http://textsnippets.com/posts/show/1033).}

A reasonable convention is to add a line break after each sentence
and start each new sentence in a new line.
Note that this has an advantage also beyond version control:
if you want to find a sentence in your LaTeX code
that you have seen in a compiled (DVI, PS, or PDF) file
or on a printout,
you can easily identify the first few words of this sentence
and screen for these words on the left border of your editor window.

Furthermore,
we split long sentences into several lines
so that each line has at most 80 characters,
because it is rather inconvenient to search for (small) differences
in long lines.
(Note:
For instance, the LaTeX editor {\itshape \setmainfont[Path=/usr/share/fonts/truetype/cmu/,UprightFont=cmunrm.ttf,BoldFont=cmunbx.ttf,ItalicFont=cmunti.ttf,BoldItalicFont=cmunbi.ttf]{cmunti.ttf}\setmonofont[Path=/usr/share/fonts/truetype/cmu/,UprightFont=cmuntt.ttf,BoldFont=cmuntb.ttf,ItalicFont=cmunit.ttf,BoldItalicFont=cmuntx.ttf]{cmunti.ttf}\itshape Kile}\setmainfont[Path=/usr/share/fonts/truetype/cmu/,UprightFont=cmunrm.ttf,BoldFont=cmunbx.ttf,ItalicFont=cmunti.ttf,BoldItalicFont=cmunbi.ttf]{cmunrm.ttf}\setmonofont[Path=/usr/share/fonts/truetype/cmu/,UprightFont=cmuntt.ttf,BoldFont=cmuntb.ttf,ItalicFont=cmunit.ttf,BoldItalicFont=cmuntx.ttf]{cmunrm.ttf}
(\myplainurl{http://kile.sourceforge.net/)}
can assist the user in this task
when it is configured to add a vertical line
that marks the 80th column.)
We find it very useful to introduce
the additional line breaks at logical breaks of the sentence,
e.g. before a relative clause
or a new part of the sentence starts.
An example LaTeX code
that is formatted according to these guidelines
is the source code of the article 
{\itshape \setmainfont[Path=/usr/share/fonts/truetype/cmu/,UprightFont=cmunrm.ttf,BoldFont=cmunbx.ttf,ItalicFont=cmunti.ttf,BoldItalicFont=cmunbi.ttf]{cmunti.ttf}\setmonofont[Path=/usr/share/fonts/truetype/cmu/,UprightFont=cmuntt.ttf,BoldFont=cmuntb.ttf,ItalicFont=cmunit.ttf,BoldItalicFont=cmuntx.ttf]{cmunti.ttf}\itshape Tools for Collaborative Writing of Scientific LaTeX Documents}\setmainfont[Path=/usr/share/fonts/truetype/cmu/,UprightFont=cmunrm.ttf,BoldFont=cmunbx.ttf,ItalicFont=cmunti.ttf,BoldItalicFont=cmunbi.ttf]{cmunrm.ttf}\setmonofont[Path=/usr/share/fonts/truetype/cmu/,UprightFont=cmuntt.ttf,BoldFont=cmuntb.ttf,ItalicFont=cmunit.ttf,BoldItalicFont=cmuntx.ttf]{cmunrm.ttf}
by \myhref{https://en.wikibooks.org/wiki/User\%3AArnehe}{Arne Henningsen}
that is published (including the source code) 
in {\itshape \setmainfont[Path=/usr/share/fonts/truetype/cmu/,UprightFont=cmunrm.ttf,BoldFont=cmunbx.ttf,ItalicFont=cmunti.ttf,BoldItalicFont=cmunbi.ttf]{cmunti.ttf}\setmonofont[Path=/usr/share/fonts/truetype/cmu/,UprightFont=cmuntt.ttf,BoldFont=cmuntb.ttf,ItalicFont=cmunit.ttf,BoldItalicFont=cmuntx.ttf]{cmunti.ttf}\itshape The PracTeX Journal}{$\text{ }$}\setmainfont[Path=/usr/share/fonts/truetype/cmu/,UprightFont=cmunrm.ttf,BoldFont=cmunbx.ttf,ItalicFont=cmunti.ttf,BoldItalicFont=cmunbi.ttf]{cmunrm.ttf}\setmonofont[Path=/usr/share/fonts/truetype/cmu/,UprightFont=cmuntt.ttf,BoldFont=cmuntb.ttf,ItalicFont=cmunit.ttf,BoldItalicFont=cmuntx.ttf]{cmunrm.ttf} 2007, Number 3 (\myplainurl{http://www.tug.org/pracjourn/2007-3/henningsen/).}

If the authors work on different operating systems,
their LaTeX editors will probably save the files with different newline (end-{}of-{}line) characters
(\myplainurl{http://en.wikipedia.org/wiki/Newline).}
To avoid this type of \textquotesingle{}ineffective\textquotesingle{} modifications,
all users can agree on a specific newline character
and configure their editor to use this newline character.
Another alternative is to add the subversion property \textquotesingle{}svn:eol-{}style\textquotesingle{} and set it to \textquotesingle{}native\textquotesingle{}.
In this case, {\itshape \setmainfont[Path=/usr/share/fonts/truetype/cmu/,UprightFont=cmunrm.ttf,BoldFont=cmunbx.ttf,ItalicFont=cmunti.ttf,BoldItalicFont=cmunbi.ttf]{cmunti.ttf}\setmonofont[Path=/usr/share/fonts/truetype/cmu/,UprightFont=cmuntt.ttf,BoldFont=cmuntb.ttf,ItalicFont=cmunit.ttf,BoldItalicFont=cmuntx.ttf]{cmunti.ttf}\itshape Subversion}{$\text{ }$}\setmainfont[Path=/usr/share/fonts/truetype/cmu/,UprightFont=cmunrm.ttf,BoldFont=cmunbx.ttf,ItalicFont=cmunti.ttf,BoldItalicFont=cmunbi.ttf]{cmunrm.ttf}\setmonofont[Path=/usr/share/fonts/truetype/cmu/,UprightFont=cmuntt.ttf,BoldFont=cmuntb.ttf,ItalicFont=cmunit.ttf,BoldItalicFont=cmuntx.ttf]{cmunrm.ttf} automatically converts all newline characters of this file 
to the native newline character of the author\textquotesingle{}s operating system
(\myplainurl{http://svnbook.red-bean.com/en/1.4/svn.advanced.props.file-portability.html\#svn.advanced.props.special.eol-style).} 

There is also another important reason
for reducing the number of \textquotesingle{}ineffective\textquotesingle{} modifications:
if several authors work on the same file,
the probability that the same line is modified by two or more authors
at the same time increases with the number of modified lines.
Hence, \textquotesingle{}ineffective\textquotesingle{} modifications unnecessarily increase
the risk of conflicts (see section 
\mylref{915}{Interchanging Documents}).



\begin{minipage}{1.0\linewidth}
\begin{center}
\includegraphics[width=1.0\linewidth,height=6.5in,keepaspectratio]{../images/214.png}
\end{center}
\raggedright{}\myfigurewithcaption{214}{Figure 2: Reviewing modifications in {\itshape \setmainfont[Path=/usr/share/fonts/truetype/cmu/,UprightFont=cmunrm.ttf,BoldFont=cmunbx.ttf,ItalicFont=cmunti.ttf,BoldItalicFont=cmunbi.ttf]{cmunti.ttf}\setmonofont[Path=/usr/share/fonts/truetype/cmu/,UprightFont=cmuntt.ttf,BoldFont=cmuntb.ttf,ItalicFont=cmunit.ttf,BoldItalicFont=cmuntx.ttf]{cmunti.ttf}\itshape KDiff3}}
\end{minipage}\vspace{0.75cm}


Furthermore, version control systems
allow a very effective quality assurance measure:
all authors should critically review their own modifications
before they commit them to the repository
(see figure 2).
The differences between the user\textquotesingle{}s working copy and the repository
can be easily inspected with a single {\itshape \setmainfont[Path=/usr/share/fonts/truetype/cmu/,UprightFont=cmunrm.ttf,BoldFont=cmunbx.ttf,ItalicFont=cmunti.ttf,BoldItalicFont=cmunbi.ttf]{cmunti.ttf}\setmonofont[Path=/usr/share/fonts/truetype/cmu/,UprightFont=cmuntt.ttf,BoldFont=cmuntb.ttf,ItalicFont=cmunit.ttf,BoldItalicFont=cmuntx.ttf]{cmunti.ttf}\itshape Subversion}{$\text{ }$}\setmainfont[Path=/usr/share/fonts/truetype/cmu/,UprightFont=cmunrm.ttf,BoldFont=cmunbx.ttf,ItalicFont=cmunti.ttf,BoldItalicFont=cmunbi.ttf]{cmunrm.ttf}\setmonofont[Path=/usr/share/fonts/truetype/cmu/,UprightFont=cmuntt.ttf,BoldFont=cmuntb.ttf,ItalicFont=cmunit.ttf,BoldItalicFont=cmuntx.ttf]{cmunrm.ttf} command
or with one or two clicks in a graphical {\itshape \setmainfont[Path=/usr/share/fonts/truetype/cmu/,UprightFont=cmunrm.ttf,BoldFont=cmunbx.ttf,ItalicFont=cmunti.ttf,BoldItalicFont=cmunbi.ttf]{cmunti.ttf}\setmonofont[Path=/usr/share/fonts/truetype/cmu/,UprightFont=cmuntt.ttf,BoldFont=cmuntb.ttf,ItalicFont=cmunit.ttf,BoldItalicFont=cmuntx.ttf]{cmunti.ttf}\itshape Subversion}{$\text{ }$}\setmainfont[Path=/usr/share/fonts/truetype/cmu/,UprightFont=cmunrm.ttf,BoldFont=cmunbx.ttf,ItalicFont=cmunti.ttf,BoldItalicFont=cmunbi.ttf]{cmunrm.ttf}\setmonofont[Path=/usr/share/fonts/truetype/cmu/,UprightFont=cmuntt.ttf,BoldFont=cmuntb.ttf,ItalicFont=cmunit.ttf,BoldItalicFont=cmuntx.ttf]{cmunrm.ttf} client.
Furthermore, authors should verify
that their code can be compiled flawlessly
before they commit their modifications to the repository.
Otherwise, the co-{}authors have to pay for these mistakes
when they want to compile the document.
However, this directive is not only reasonable for version control systems
but also for all other ways to interchange documents among authors.

{\itshape \setmainfont[Path=/usr/share/fonts/truetype/cmu/,UprightFont=cmunrm.ttf,BoldFont=cmunbx.ttf,ItalicFont=cmunti.ttf,BoldItalicFont=cmunbi.ttf]{cmunti.ttf}\setmonofont[Path=/usr/share/fonts/truetype/cmu/,UprightFont=cmuntt.ttf,BoldFont=cmuntb.ttf,ItalicFont=cmunit.ttf,BoldItalicFont=cmuntx.ttf]{cmunti.ttf}\itshape Subversion}{$\text{ }$}\setmainfont[Path=/usr/share/fonts/truetype/cmu/,UprightFont=cmunrm.ttf,BoldFont=cmunbx.ttf,ItalicFont=cmunti.ttf,BoldItalicFont=cmunbi.ttf]{cmunrm.ttf}\setmonofont[Path=/usr/share/fonts/truetype/cmu/,UprightFont=cmuntt.ttf,BoldFont=cmuntb.ttf,ItalicFont=cmunit.ttf,BoldItalicFont=cmuntx.ttf]{cmunrm.ttf} has a feature called \textquotesingle{}Keyword Substitution\textquotesingle{}
that includes dynamic version information about a file
(e.g. the revision number or the last author)
into the contents of the file itself
(see e.g. \myplainurl{http://svnbook.red-bean.com,} chapter 3).
Sometimes, it is useful to include these information
not only as a comment in the LaTeX source code,
but also in the (compiled) DVI, PS, or PDF document.
This can be achieved with the LaTeX packages
{\itshape \setmainfont[Path=/usr/share/fonts/truetype/cmu/,UprightFont=cmunrm.ttf,BoldFont=cmunbx.ttf,ItalicFont=cmunti.ttf,BoldItalicFont=cmunbi.ttf]{cmunti.ttf}\setmonofont[Path=/usr/share/fonts/truetype/cmu/,UprightFont=cmuntt.ttf,BoldFont=cmuntb.ttf,ItalicFont=cmunit.ttf,BoldItalicFont=cmuntx.ttf]{cmunti.ttf}\itshape svn}{$\text{ }$}\setmainfont[Path=/usr/share/fonts/truetype/cmu/,UprightFont=cmunrm.ttf,BoldFont=cmunbx.ttf,ItalicFont=cmunti.ttf,BoldItalicFont=cmunbi.ttf]{cmunrm.ttf}\setmonofont[Path=/usr/share/fonts/truetype/cmu/,UprightFont=cmuntt.ttf,BoldFont=cmuntb.ttf,ItalicFont=cmunit.ttf,BoldItalicFont=cmuntx.ttf]{cmunrm.ttf} (\myplainurl{http://www.ctan.org/tex-archive/macros/latex/contrib/svn/),}
{\itshape \setmainfont[Path=/usr/share/fonts/truetype/cmu/,UprightFont=cmunrm.ttf,BoldFont=cmunbx.ttf,ItalicFont=cmunti.ttf,BoldItalicFont=cmunbi.ttf]{cmunti.ttf}\setmonofont[Path=/usr/share/fonts/truetype/cmu/,UprightFont=cmuntt.ttf,BoldFont=cmuntb.ttf,ItalicFont=cmunit.ttf,BoldItalicFont=cmuntx.ttf]{cmunti.ttf}\itshape svninfo}{$\text{ }$}\setmainfont[Path=/usr/share/fonts/truetype/cmu/,UprightFont=cmunrm.ttf,BoldFont=cmunbx.ttf,ItalicFont=cmunti.ttf,BoldItalicFont=cmunbi.ttf]{cmunrm.ttf}\setmonofont[Path=/usr/share/fonts/truetype/cmu/,UprightFont=cmuntt.ttf,BoldFont=cmuntb.ttf,ItalicFont=cmunit.ttf,BoldItalicFont=cmuntx.ttf]{cmunrm.ttf} (\myplainurl{http://www.ctan.org/tex-archive/macros/latex/contrib/svninfo/),}
or (preferably) {\itshape \setmainfont[Path=/usr/share/fonts/truetype/cmu/,UprightFont=cmunrm.ttf,BoldFont=cmunbx.ttf,ItalicFont=cmunti.ttf,BoldItalicFont=cmunbi.ttf]{cmunti.ttf}\setmonofont[Path=/usr/share/fonts/truetype/cmu/,UprightFont=cmuntt.ttf,BoldFont=cmuntb.ttf,ItalicFont=cmunit.ttf,BoldItalicFont=cmuntx.ttf]{cmunti.ttf}\itshape svn-{}multi}{$\text{ }$}\setmainfont[Path=/usr/share/fonts/truetype/cmu/,UprightFont=cmunrm.ttf,BoldFont=cmunbx.ttf,ItalicFont=cmunti.ttf,BoldItalicFont=cmunbi.ttf]{cmunrm.ttf}\setmonofont[Path=/usr/share/fonts/truetype/cmu/,UprightFont=cmuntt.ttf,BoldFont=cmuntb.ttf,ItalicFont=cmunit.ttf,BoldItalicFont=cmuntx.ttf]{cmunrm.ttf} (\myplainurl{http://www.ctan.org/tex-archive/macros/latex/contrib/svn-multi/).}

The most important directives for collaborative writing of LaTeX documents
with version control systems are summarised in the following box.


{\bfseries \setmainfont[Path=/usr/share/fonts/truetype/cmu/,UprightFont=cmunrm.ttf,BoldFont=cmunbx.ttf,ItalicFont=cmunti.ttf,BoldItalicFont=cmunbi.ttf]{cmunbx.ttf}\setmonofont[Path=/usr/share/fonts/truetype/cmu/,UprightFont=cmuntt.ttf,BoldFont=cmuntb.ttf,ItalicFont=cmunit.ttf,BoldItalicFont=cmuntx.ttf]{cmunbx.ttf}\bfseries Directives for using LaTeX with version control systems}
\begin{myenumerate}
\item{} {$\text{ }$}\setmainfont[Path=/usr/share/fonts/truetype/cmu/,UprightFont=cmunrm.ttf,BoldFont=cmunbx.ttf,ItalicFont=cmunti.ttf,BoldItalicFont=cmunbi.ttf]{cmunrm.ttf}\setmonofont[Path=/usr/share/fonts/truetype/cmu/,UprightFont=cmuntt.ttf,BoldFont=cmuntb.ttf,ItalicFont=cmunit.ttf,BoldItalicFont=cmuntx.ttf]{cmunrm.ttf} Avoid \textquotesingle{}ineffective\textquotesingle{} modifications.
\item{}  Do not change line breaks without good reason.
\item{}  Turn off automatic line wrapping of your LaTeX editor.
\item{}  Start each new sentence in a new line.
\item{}  Split long sentences into several lines so that each line has at most 80 characters.
\item{}  Put only those files under version control that are directly modified by the user.
\item{}  Verify that your code can be compiled flawlessly before committing your modifications to the repository.
\item{}  Use {\itshape \setmainfont[Path=/usr/share/fonts/truetype/cmu/,UprightFont=cmunrm.ttf,BoldFont=cmunbx.ttf,ItalicFont=cmunti.ttf,BoldItalicFont=cmunbi.ttf]{cmunti.ttf}\setmonofont[Path=/usr/share/fonts/truetype/cmu/,UprightFont=cmuntt.ttf,BoldFont=cmuntb.ttf,ItalicFont=cmunit.ttf,BoldItalicFont=cmuntx.ttf]{cmunti.ttf}\itshape Subversion{\bfseries \setmainfont[Path=/usr/share/fonts/truetype/cmu/,UprightFont=cmunrm.ttf,BoldFont=cmunbx.ttf,ItalicFont=cmunti.ttf,BoldItalicFont=cmunbi.ttf]{cmunbi.ttf}\setmonofont[Path=/usr/share/fonts/truetype/cmu/,UprightFont=cmuntt.ttf,BoldFont=cmuntb.ttf,ItalicFont=cmunit.ttf,BoldItalicFont=cmuntx.ttf]{cmunbi.ttf}\bfseries \itshape s diff feature to critically review your modifications before committing them to the repository.}}
\item{} {$\text{ }$}\setmainfont[Path=/usr/share/fonts/truetype/cmu/,UprightFont=cmunrm.ttf,BoldFont=cmunbx.ttf,ItalicFont=cmunti.ttf,BoldItalicFont=cmunbi.ttf]{cmunrm.ttf}\setmonofont[Path=/usr/share/fonts/truetype/cmu/,UprightFont=cmuntt.ttf,BoldFont=cmuntb.ttf,ItalicFont=cmunit.ttf,BoldItalicFont=cmuntx.ttf]{cmunrm.ttf} Add a meaningful and descriptive comment when committing your modifications to the repository.
\item{}  Use the {\itshape \setmainfont[Path=/usr/share/fonts/truetype/cmu/,UprightFont=cmunrm.ttf,BoldFont=cmunbx.ttf,ItalicFont=cmunti.ttf,BoldItalicFont=cmunbi.ttf]{cmunti.ttf}\setmonofont[Path=/usr/share/fonts/truetype/cmu/,UprightFont=cmuntt.ttf,BoldFont=cmuntb.ttf,ItalicFont=cmunit.ttf,BoldItalicFont=cmuntx.ttf]{cmunti.ttf}\itshape Subversion}{$\text{ }$}\setmainfont[Path=/usr/share/fonts/truetype/cmu/,UprightFont=cmunrm.ttf,BoldFont=cmunbx.ttf,ItalicFont=cmunti.ttf,BoldItalicFont=cmunbi.ttf]{cmunrm.ttf}\setmonofont[Path=/usr/share/fonts/truetype/cmu/,UprightFont=cmuntt.ttf,BoldFont=cmuntb.ttf,ItalicFont=cmunit.ttf,BoldItalicFont=cmuntx.ttf]{cmunrm.ttf} client for copying, moving, or renaming files and folders that are under revision control.
\end{myenumerate}



If the users are willing to let go of the built-{}in {\itshape \setmainfont[Path=/usr/share/fonts/truetype/cmu/,UprightFont=cmunrm.ttf,BoldFont=cmunbx.ttf,ItalicFont=cmunti.ttf,BoldItalicFont=cmunbi.ttf]{cmunti.ttf}\setmonofont[Path=/usr/share/fonts/truetype/cmu/,UprightFont=cmuntt.ttf,BoldFont=cmuntb.ttf,ItalicFont=cmunit.ttf,BoldItalicFont=cmuntx.ttf]{cmunti.ttf}\itshape diff}{$\text{ }$}\setmainfont[Path=/usr/share/fonts/truetype/cmu/,UprightFont=cmunrm.ttf,BoldFont=cmunbx.ttf,ItalicFont=cmunti.ttf,BoldItalicFont=cmunbi.ttf]{cmunrm.ttf}\setmonofont[Path=/usr/share/fonts/truetype/cmu/,UprightFont=cmuntt.ttf,BoldFont=cmuntb.ttf,ItalicFont=cmunit.ttf,BoldItalicFont=cmuntx.ttf]{cmunrm.ttf} utility of SVN and use {\itshape \setmainfont[Path=/usr/share/fonts/truetype/cmu/,UprightFont=cmunrm.ttf,BoldFont=cmunbx.ttf,ItalicFont=cmunti.ttf,BoldItalicFont=cmunbi.ttf]{cmunti.ttf}\setmonofont[Path=/usr/share/fonts/truetype/cmu/,UprightFont=cmuntt.ttf,BoldFont=cmuntb.ttf,ItalicFont=cmunit.ttf,BoldItalicFont=cmuntx.ttf]{cmunti.ttf}\itshape diff}{$\text{ }$}\setmainfont[Path=/usr/share/fonts/truetype/cmu/,UprightFont=cmunrm.ttf,BoldFont=cmunbx.ttf,ItalicFont=cmunti.ttf,BoldItalicFont=cmunbi.ttf]{cmunrm.ttf}\setmonofont[Path=/usr/share/fonts/truetype/cmu/,UprightFont=cmuntt.ttf,BoldFont=cmuntb.ttf,ItalicFont=cmunit.ttf,BoldItalicFont=cmuntx.ttf]{cmunrm.ttf} tools that are local on their workstations, they can put to use such tools that are more tailored to text documents. The {\itshape \setmainfont[Path=/usr/share/fonts/truetype/cmu/,UprightFont=cmunrm.ttf,BoldFont=cmunbx.ttf,ItalicFont=cmunti.ttf,BoldItalicFont=cmunbi.ttf]{cmunti.ttf}\setmonofont[Path=/usr/share/fonts/truetype/cmu/,UprightFont=cmuntt.ttf,BoldFont=cmuntb.ttf,ItalicFont=cmunit.ttf,BoldItalicFont=cmuntx.ttf]{cmunti.ttf}\itshape diff}{$\text{ }$}\setmainfont[Path=/usr/share/fonts/truetype/cmu/,UprightFont=cmunrm.ttf,BoldFont=cmunbx.ttf,ItalicFont=cmunti.ttf,BoldItalicFont=cmunbi.ttf]{cmunrm.ttf}\setmonofont[Path=/usr/share/fonts/truetype/cmu/,UprightFont=cmuntt.ttf,BoldFont=cmuntb.ttf,ItalicFont=cmunit.ttf,BoldItalicFont=cmuntx.ttf]{cmunrm.ttf} tool that comes with SVN was designed with source code in mind. As such, it is built to be more useful for files of short lines. Other tools, such as {\bfseries \setmainfont[Path=/usr/share/fonts/truetype/cmu/,UprightFont=cmunrm.ttf,BoldFont=cmunbx.ttf,ItalicFont=cmunti.ttf,BoldItalicFont=cmunbi.ttf]{cmunbx.ttf}\setmonofont[Path=/usr/share/fonts/truetype/cmu/,UprightFont=cmuntt.ttf,BoldFont=cmuntb.ttf,ItalicFont=cmunit.ttf,BoldItalicFont=cmuntx.ttf]{cmunbx.ttf}\bfseries Compare It!}{$\text{ }$}\setmainfont[Path=/usr/share/fonts/truetype/cmu/,UprightFont=cmunrm.ttf,BoldFont=cmunbx.ttf,ItalicFont=cmunti.ttf,BoldItalicFont=cmunbi.ttf]{cmunrm.ttf}\setmonofont[Path=/usr/share/fonts/truetype/cmu/,UprightFont=cmuntt.ttf,BoldFont=cmuntb.ttf,ItalicFont=cmunit.ttf,BoldItalicFont=cmuntx.ttf]{cmunrm.ttf} allows to conveniently compare text files where each line can span hundreds of characters (such as when each line represents a paragraph). When using a {\itshape \setmainfont[Path=/usr/share/fonts/truetype/cmu/,UprightFont=cmunrm.ttf,BoldFont=cmunbx.ttf,ItalicFont=cmunti.ttf,BoldItalicFont=cmunbi.ttf]{cmunti.ttf}\setmonofont[Path=/usr/share/fonts/truetype/cmu/,UprightFont=cmuntt.ttf,BoldFont=cmuntb.ttf,ItalicFont=cmunit.ttf,BoldItalicFont=cmuntx.ttf]{cmunti.ttf}\itshape diff}{$\text{ }$}\setmainfont[Path=/usr/share/fonts/truetype/cmu/,UprightFont=cmunrm.ttf,BoldFont=cmunbx.ttf,ItalicFont=cmunti.ttf,BoldItalicFont=cmunbi.ttf]{cmunrm.ttf}\setmonofont[Path=/usr/share/fonts/truetype/cmu/,UprightFont=cmuntt.ttf,BoldFont=cmuntb.ttf,ItalicFont=cmunit.ttf,BoldItalicFont=cmuntx.ttf]{cmunrm.ttf} tool that allows convenient views of files with long lines, the users can author the TeX files without a strict line-{}breaking policy.
\subsection{Visualizing {\itshape \setmainfont[Path=/usr/share/fonts/truetype/cmu/,UprightFont=cmunrm.ttf,BoldFont=cmunbx.ttf,ItalicFont=cmunti.ttf,BoldItalicFont=cmunbi.ttf]{cmunti.ttf}\setmonofont[Path=/usr/share/fonts/truetype/cmu/,UprightFont=cmuntt.ttf,BoldFont=cmuntb.ttf,ItalicFont=cmunit.ttf,BoldItalicFont=cmuntx.ttf]{cmunti.ttf}\itshape diffs}{$\text{ }$}\setmainfont[Path=/usr/share/fonts/truetype/cmu/,UprightFont=cmunrm.ttf,BoldFont=cmunbx.ttf,ItalicFont=cmunti.ttf,BoldItalicFont=cmunbi.ttf]{cmunrm.ttf}\setmonofont[Path=/usr/share/fonts/truetype/cmu/,UprightFont=cmuntt.ttf,BoldFont=cmuntb.ttf,ItalicFont=cmunit.ttf,BoldItalicFont=cmuntx.ttf]{cmunrm.ttf} in LaTeX: latexdiff and changebar}
\label{919}

The tools \myhref{http://www.ctan.org/tex-archive/support/latexdiff/}{latexdiff} and \myhref{http://www.ctan.org/tex-archive/macros/latex/contrib/changebar/}{changebar} can visualize differences of two LaTeX files inside a generated document. This makes it easier to see impact of certain changes or discuss changes with people not custom to LaTeX. Changebar comes with a script {\ttfamily \setmainfont[Path=/usr/share/fonts/truetype/cmu/,UprightFont=cmunrm.ttf,BoldFont=cmunbx.ttf,ItalicFont=cmunti.ttf,BoldItalicFont=cmunbi.ttf]{cmuntt.ttf}\setmonofont[Path=/usr/share/fonts/truetype/cmu/,UprightFont=cmuntt.ttf,BoldFont=cmuntb.ttf,ItalicFont=cmunit.ttf,BoldItalicFont=cmuntx.ttf]{cmuntt.ttf}\ttfamily chbar.sh}{$\text{ }$}\setmainfont[Path=/usr/share/fonts/truetype/cmu/,UprightFont=cmunrm.ttf,BoldFont=cmunbx.ttf,ItalicFont=cmunti.ttf,BoldItalicFont=cmunbi.ttf]{cmunrm.ttf}\setmonofont[Path=/usr/share/fonts/truetype/cmu/,UprightFont=cmuntt.ttf,BoldFont=cmuntb.ttf,ItalicFont=cmunit.ttf,BoldItalicFont=cmuntx.ttf]{cmunrm.ttf} which inserts a bar in the margin indicating parts that have changed. Latexdiff allows different styles of visualization. The default is that discarded text is marked as red and added text is marked as blue. It also supports a mode similar to Changebar which adds a bar in the margin. Latexdiff comes with a script {\ttfamily \setmainfont[Path=/usr/share/fonts/truetype/cmu/,UprightFont=cmunrm.ttf,BoldFont=cmunbx.ttf,ItalicFont=cmunti.ttf,BoldItalicFont=cmunbi.ttf]{cmuntt.ttf}\setmonofont[Path=/usr/share/fonts/truetype/cmu/,UprightFont=cmuntt.ttf,BoldFont=cmuntb.ttf,ItalicFont=cmunit.ttf,BoldItalicFont=cmuntx.ttf]{cmuntt.ttf}\ttfamily latexrevise}{$\text{ }$}\setmainfont[Path=/usr/share/fonts/truetype/cmu/,UprightFont=cmunrm.ttf,BoldFont=cmunbx.ttf,ItalicFont=cmunti.ttf,BoldItalicFont=cmunbi.ttf]{cmunrm.ttf}\setmonofont[Path=/usr/share/fonts/truetype/cmu/,UprightFont=cmuntt.ttf,BoldFont=cmuntb.ttf,ItalicFont=cmunit.ttf,BoldItalicFont=cmuntx.ttf]{cmunrm.ttf} which can be used to accept or decline changes. It also has a wrapper script to support version control systems such as the discussed Subversion.

An example on how to use Latexdiff in the Terminal.
\\

\TemplateSpaceIndent{$\text{ }${}$\text{ }${}$\text{ }${}$\text{ }${}latexdiff$\text{ }${}old.tex$\text{ }${}new.tex$\text{ }${}>{}$\text{ }${}diff.tex$\text{ }${}$\text{ }${}$\text{ }${}$\text{ }${}$\text{ }${}$\text{ }${}$\text{ }${}$\text{ }${}$\text{ }${}$\text{ }${}$\text{ }${}$\text{ }${}$\text{ }${}$\text{ }${}$\text{ }${}\#$\text{ }${}Files$\text{ }${}old.tex$\text{ }${}and$\text{ }$\newline{}
$\text{ }${}new.tex$\text{ }${}are$\text{ }${}compared$\text{ }${}and$\text{ }${}the$\text{ }${}file$\text{ }${}visualizing$\text{ }${}the$\text{ }${}changes$\text{ }${}is$\text{ }${}written$\text{ }${}to$\text{ }${}diff.tex$\text{ }$\newline{}
$\text{ }${}$\text{ }${}$\text{ }${}$\text{ }${}pdflatex$\text{ }${}diff.tex$\text{ }${}$\text{ }${}$\text{ }${}$\text{ }${}$\text{ }${}$\text{ }${}$\text{ }${}$\text{ }${}$\text{ }${}$\text{ }${}$\text{ }${}$\text{ }${}$\text{ }${}$\text{ }${}$\text{ }${}$\text{ }${}$\text{ }${}$\text{ }${}$\text{ }${}$\text{ }${}$\text{ }${}$\text{ }${}$\text{ }${}$\text{ }${}$\text{ }${}$\text{ }${}$\text{ }${}$\text{ }${}$\text{ }${}$\text{ }${}$\text{ }${}$\text{ }${}$\text{ }${}$\text{ }${}\#$\text{ }${}Create$\text{ }${}a$\text{ }${}PDF$\text{ }${}showing$\text{ }$\newline{}
$\text{ }${}the$\text{ }${}changes}


The program \myhref{http://www.qtrac.eu/diffpdf.html}{DiffPDF} can be used to compare two existing PDFs visually. There is also a command line tool \myhref{http://www.qtrac.eu/comparepdf.html}{comparepdf} based on DiffPDF.
\section{Managing collaborative bibliographies}
\label{920}

Writing of scientific articles, reports, and books requires
the citation of all relevant sources.
BibTeX is an excellent tool for citing references and creating bibliographies
(Markey 2005, Fenn 2006).
Many different BibTeX styles can be found
on CTAN (\myplainurl{http://www.ctan.org)}
and on the LaTeX Bibliography Styles Database
(\myplainurl{http://jo.irisson.free.fr/bstdatabase/).}
If no suitable BibTeX style can be found,
most desired styles can be conveniently assembled with
{\itshape \setmainfont[Path=/usr/share/fonts/truetype/cmu/,UprightFont=cmunrm.ttf,BoldFont=cmunbx.ttf,ItalicFont=cmunti.ttf,BoldItalicFont=cmunbi.ttf]{cmunti.ttf}\setmonofont[Path=/usr/share/fonts/truetype/cmu/,UprightFont=cmuntt.ttf,BoldFont=cmuntb.ttf,ItalicFont=cmunit.ttf,BoldItalicFont=cmuntx.ttf]{cmunti.ttf}\itshape custombib}\setmainfont[Path=/usr/share/fonts/truetype/cmu/,UprightFont=cmunrm.ttf,BoldFont=cmunbx.ttf,ItalicFont=cmunti.ttf,BoldItalicFont=cmunbi.ttf]{cmunrm.ttf}\setmonofont[Path=/usr/share/fonts/truetype/cmu/,UprightFont=cmuntt.ttf,BoldFont=cmuntb.ttf,ItalicFont=cmunit.ttf,BoldItalicFont=cmuntx.ttf]{cmunrm.ttf}/{\itshape \setmainfont[Path=/usr/share/fonts/truetype/cmu/,UprightFont=cmunrm.ttf,BoldFont=cmunbx.ttf,ItalicFont=cmunti.ttf,BoldItalicFont=cmunbi.ttf]{cmunti.ttf}\setmonofont[Path=/usr/share/fonts/truetype/cmu/,UprightFont=cmuntt.ttf,BoldFont=cmuntb.ttf,ItalicFont=cmunit.ttf,BoldItalicFont=cmuntx.ttf]{cmunti.ttf}\itshape makebst}{$\text{ }$}\setmainfont[Path=/usr/share/fonts/truetype/cmu/,UprightFont=cmunrm.ttf,BoldFont=cmunbx.ttf,ItalicFont=cmunti.ttf,BoldItalicFont=cmunbi.ttf]{cmunrm.ttf}\setmonofont[Path=/usr/share/fonts/truetype/cmu/,UprightFont=cmuntt.ttf,BoldFont=cmuntb.ttf,ItalicFont=cmunit.ttf,BoldItalicFont=cmuntx.ttf]{cmunrm.ttf} 
(\myplainurl{http://www.ctan.org/tex-archive/macros/latex/contrib/custom-bib/).}
Furthermore, BibTeX style files can be created or modified manually;
however this action requires knowledge of the (unnamed) postfix stack language
that is used in BibTeX style files
(Patashnik 1988).

At our department, we have a common bibliographic data base
in the BibTeX format (.bib file).
It resides in our common {\ttfamily \setmainfont[Path=/usr/share/fonts/truetype/cmu/,UprightFont=cmunrm.ttf,BoldFont=cmunbx.ttf,ItalicFont=cmunti.ttf,BoldItalicFont=cmunbi.ttf]{cmuntt.ttf}\setmonofont[Path=/usr/share/fonts/truetype/cmu/,UprightFont=cmuntt.ttf,BoldFont=cmuntb.ttf,ItalicFont=cmunit.ttf,BoldItalicFont=cmuntx.ttf]{cmuntt.ttf}\ttfamily texmf}{$\text{ }$}\setmainfont[Path=/usr/share/fonts/truetype/cmu/,UprightFont=cmunrm.ttf,BoldFont=cmunbx.ttf,ItalicFont=cmunti.ttf,BoldItalicFont=cmunbi.ttf]{cmunrm.ttf}\setmonofont[Path=/usr/share/fonts/truetype/cmu/,UprightFont=cmuntt.ttf,BoldFont=cmuntb.ttf,ItalicFont=cmunit.ttf,BoldItalicFont=cmuntx.ttf]{cmunrm.ttf} tree
(see section \textquotesingle{}Hosting LaTeX files in {\itshape \setmainfont[Path=/usr/share/fonts/truetype/cmu/,UprightFont=cmunrm.ttf,BoldFont=cmunbx.ttf,ItalicFont=cmunti.ttf,BoldItalicFont=cmunbi.ttf]{cmunti.ttf}\setmonofont[Path=/usr/share/fonts/truetype/cmu/,UprightFont=cmuntt.ttf,BoldFont=cmuntb.ttf,ItalicFont=cmunit.ttf,BoldItalicFont=cmuntx.ttf]{cmunti.ttf}\itshape Subversion{\bfseries \setmainfont[Path=/usr/share/fonts/truetype/cmu/,UprightFont=cmunrm.ttf,BoldFont=cmunbx.ttf,ItalicFont=cmunti.ttf,BoldItalicFont=cmunbi.ttf]{cmunbi.ttf}\setmonofont[Path=/usr/share/fonts/truetype/cmu/,UprightFont=cmuntt.ttf,BoldFont=cmuntb.ttf,ItalicFont=cmunit.ttf,BoldItalicFont=cmuntx.ttf]{cmunbi.ttf}\bfseries \itshape )}}\setmainfont[Path=/usr/share/fonts/truetype/cmu/,UprightFont=cmunrm.ttf,BoldFont=cmunbx.ttf,ItalicFont=cmunti.ttf,BoldItalicFont=cmunbi.ttf]{cmunrm.ttf}\setmonofont[Path=/usr/share/fonts/truetype/cmu/,UprightFont=cmuntt.ttf,BoldFont=cmuntb.ttf,ItalicFont=cmunit.ttf,BoldItalicFont=cmuntx.ttf]{cmunrm.ttf}
in the subdirectory {\ttfamily \setmainfont[Path=/usr/share/fonts/truetype/cmu/,UprightFont=cmunrm.ttf,BoldFont=cmunbx.ttf,ItalicFont=cmunti.ttf,BoldItalicFont=cmunbi.ttf]{cmuntt.ttf}\setmonofont[Path=/usr/share/fonts/truetype/cmu/,UprightFont=cmuntt.ttf,BoldFont=cmuntb.ttf,ItalicFont=cmunit.ttf,BoldItalicFont=cmuntx.ttf]{cmuntt.ttf}\ttfamily /bibtex/bib/}\setmainfont[Path=/usr/share/fonts/truetype/cmu/,UprightFont=cmunrm.ttf,BoldFont=cmunbx.ttf,ItalicFont=cmunti.ttf,BoldItalicFont=cmunbi.ttf]{cmunrm.ttf}\setmonofont[Path=/usr/share/fonts/truetype/cmu/,UprightFont=cmuntt.ttf,BoldFont=cmuntb.ttf,ItalicFont=cmunit.ttf,BoldItalicFont=cmuntx.ttf]{cmunrm.ttf}
(see figure 1).
Hence, all users can specify this bibliography by only using the file name 
(without the full path) -{}-{}-{}
no matter where the user\textquotesingle{}s working copy of the common {\ttfamily \setmainfont[Path=/usr/share/fonts/truetype/cmu/,UprightFont=cmunrm.ttf,BoldFont=cmunbx.ttf,ItalicFont=cmunti.ttf,BoldItalicFont=cmunbi.ttf]{cmuntt.ttf}\setmonofont[Path=/usr/share/fonts/truetype/cmu/,UprightFont=cmuntt.ttf,BoldFont=cmuntb.ttf,ItalicFont=cmunit.ttf,BoldItalicFont=cmuntx.ttf]{cmuntt.ttf}\ttfamily texmf}{$\text{ }$}\setmainfont[Path=/usr/share/fonts/truetype/cmu/,UprightFont=cmunrm.ttf,BoldFont=cmunbx.ttf,ItalicFont=cmunti.ttf,BoldItalicFont=cmunbi.ttf]{cmunrm.ttf}\setmonofont[Path=/usr/share/fonts/truetype/cmu/,UprightFont=cmuntt.ttf,BoldFont=cmuntb.ttf,ItalicFont=cmunit.ttf,BoldItalicFont=cmuntx.ttf]{cmunrm.ttf} tree
is located.

All users edit our bibliographic data base
with the graphical BibTeX editor {\itshape \setmainfont[Path=/usr/share/fonts/truetype/cmu/,UprightFont=cmunrm.ttf,BoldFont=cmunbx.ttf,ItalicFont=cmunti.ttf,BoldItalicFont=cmunbi.ttf]{cmunti.ttf}\setmonofont[Path=/usr/share/fonts/truetype/cmu/,UprightFont=cmuntt.ttf,BoldFont=cmuntb.ttf,ItalicFont=cmunit.ttf,BoldItalicFont=cmuntx.ttf]{cmunti.ttf}\itshape JabRef}\setmainfont[Path=/usr/share/fonts/truetype/cmu/,UprightFont=cmunrm.ttf,BoldFont=cmunbx.ttf,ItalicFont=cmunti.ttf,BoldItalicFont=cmunbi.ttf]{cmunrm.ttf}\setmonofont[Path=/usr/share/fonts/truetype/cmu/,UprightFont=cmuntt.ttf,BoldFont=cmuntb.ttf,ItalicFont=cmunit.ttf,BoldItalicFont=cmuntx.ttf]{cmunrm.ttf}
(\myplainurl{http://jabref.sourceforge.net/).}
As {\itshape \setmainfont[Path=/usr/share/fonts/truetype/cmu/,UprightFont=cmunrm.ttf,BoldFont=cmunbx.ttf,ItalicFont=cmunti.ttf,BoldItalicFont=cmunbi.ttf]{cmunti.ttf}\setmonofont[Path=/usr/share/fonts/truetype/cmu/,UprightFont=cmuntt.ttf,BoldFont=cmuntb.ttf,ItalicFont=cmunit.ttf,BoldItalicFont=cmuntx.ttf]{cmunti.ttf}\itshape JabRef}{$\text{ }$}\setmainfont[Path=/usr/share/fonts/truetype/cmu/,UprightFont=cmunrm.ttf,BoldFont=cmunbx.ttf,ItalicFont=cmunti.ttf,BoldItalicFont=cmunbi.ttf]{cmunrm.ttf}\setmonofont[Path=/usr/share/fonts/truetype/cmu/,UprightFont=cmuntt.ttf,BoldFont=cmuntb.ttf,ItalicFont=cmunit.ttf,BoldItalicFont=cmuntx.ttf]{cmunrm.ttf} is written in {\itshape \setmainfont[Path=/usr/share/fonts/truetype/cmu/,UprightFont=cmunrm.ttf,BoldFont=cmunbx.ttf,ItalicFont=cmunti.ttf,BoldItalicFont=cmunbi.ttf]{cmunti.ttf}\setmonofont[Path=/usr/share/fonts/truetype/cmu/,UprightFont=cmuntt.ttf,BoldFont=cmuntb.ttf,ItalicFont=cmunit.ttf,BoldItalicFont=cmuntx.ttf]{cmunti.ttf}\itshape Java}\setmainfont[Path=/usr/share/fonts/truetype/cmu/,UprightFont=cmunrm.ttf,BoldFont=cmunbx.ttf,ItalicFont=cmunti.ttf,BoldItalicFont=cmunbi.ttf]{cmunrm.ttf}\setmonofont[Path=/usr/share/fonts/truetype/cmu/,UprightFont=cmuntt.ttf,BoldFont=cmuntb.ttf,ItalicFont=cmunit.ttf,BoldItalicFont=cmuntx.ttf]{cmunrm.ttf},
it runs on all major operating systems.
As different versions of {\itshape \setmainfont[Path=/usr/share/fonts/truetype/cmu/,UprightFont=cmunrm.ttf,BoldFont=cmunbx.ttf,ItalicFont=cmunti.ttf,BoldItalicFont=cmunbi.ttf]{cmunti.ttf}\setmonofont[Path=/usr/share/fonts/truetype/cmu/,UprightFont=cmuntt.ttf,BoldFont=cmuntb.ttf,ItalicFont=cmunit.ttf,BoldItalicFont=cmuntx.ttf]{cmunti.ttf}\itshape JabRef}{$\text{ }$}\setmainfont[Path=/usr/share/fonts/truetype/cmu/,UprightFont=cmunrm.ttf,BoldFont=cmunbx.ttf,ItalicFont=cmunti.ttf,BoldItalicFont=cmunbi.ttf]{cmunrm.ttf}\setmonofont[Path=/usr/share/fonts/truetype/cmu/,UprightFont=cmuntt.ttf,BoldFont=cmuntb.ttf,ItalicFont=cmunit.ttf,BoldItalicFont=cmuntx.ttf]{cmunrm.ttf} generally
save files in a slightly different way
(e.g. by introducing line breaks at different positions),
all users should use the same (e.g. last stable) version of {\itshape \setmainfont[Path=/usr/share/fonts/truetype/cmu/,UprightFont=cmunrm.ttf,BoldFont=cmunbx.ttf,ItalicFont=cmunti.ttf,BoldItalicFont=cmunbi.ttf]{cmunti.ttf}\setmonofont[Path=/usr/share/fonts/truetype/cmu/,UprightFont=cmuntt.ttf,BoldFont=cmuntb.ttf,ItalicFont=cmunit.ttf,BoldItalicFont=cmuntx.ttf]{cmunti.ttf}\itshape JabRef}\setmainfont[Path=/usr/share/fonts/truetype/cmu/,UprightFont=cmunrm.ttf,BoldFont=cmunbx.ttf,ItalicFont=cmunti.ttf,BoldItalicFont=cmunbi.ttf]{cmunrm.ttf}\setmonofont[Path=/usr/share/fonts/truetype/cmu/,UprightFont=cmuntt.ttf,BoldFont=cmuntb.ttf,ItalicFont=cmunit.ttf,BoldItalicFont=cmuntx.ttf]{cmunrm.ttf}.
Otherwise, there would be many differences
between different versions of {\ttfamily \setmainfont[Path=/usr/share/fonts/truetype/cmu/,UprightFont=cmunrm.ttf,BoldFont=cmunbx.ttf,ItalicFont=cmunti.ttf,BoldItalicFont=cmunbi.ttf]{cmuntt.ttf}\setmonofont[Path=/usr/share/fonts/truetype/cmu/,UprightFont=cmuntt.ttf,BoldFont=cmuntb.ttf,ItalicFont=cmunit.ttf,BoldItalicFont=cmuntx.ttf]{cmuntt.ttf}\ttfamily .bib}{$\text{ }$}\setmainfont[Path=/usr/share/fonts/truetype/cmu/,UprightFont=cmunrm.ttf,BoldFont=cmunbx.ttf,ItalicFont=cmunti.ttf,BoldItalicFont=cmunbi.ttf]{cmunrm.ttf}\setmonofont[Path=/usr/share/fonts/truetype/cmu/,UprightFont=cmuntt.ttf,BoldFont=cmuntb.ttf,ItalicFont=cmunit.ttf,BoldItalicFont=cmuntx.ttf]{cmunrm.ttf} files
that solely originate from using different version of {\itshape \setmainfont[Path=/usr/share/fonts/truetype/cmu/,UprightFont=cmunrm.ttf,BoldFont=cmunbx.ttf,ItalicFont=cmunti.ttf,BoldItalicFont=cmunbi.ttf]{cmunti.ttf}\setmonofont[Path=/usr/share/fonts/truetype/cmu/,UprightFont=cmuntt.ttf,BoldFont=cmuntb.ttf,ItalicFont=cmunit.ttf,BoldItalicFont=cmuntx.ttf]{cmunti.ttf}\itshape JabRef}\setmainfont[Path=/usr/share/fonts/truetype/cmu/,UprightFont=cmunrm.ttf,BoldFont=cmunbx.ttf,ItalicFont=cmunti.ttf,BoldItalicFont=cmunbi.ttf]{cmunrm.ttf}\setmonofont[Path=/usr/share/fonts/truetype/cmu/,UprightFont=cmuntt.ttf,BoldFont=cmuntb.ttf,ItalicFont=cmunit.ttf,BoldItalicFont=cmuntx.ttf]{cmunrm.ttf}.
Hence, it would be hard to find the real differences
between the compared documents.
Furthermore, the probability of conflicts would be much higher
(see section \textquotesingle{}Subversion really makes the difference\textquotesingle{}).
As {\itshape \setmainfont[Path=/usr/share/fonts/truetype/cmu/,UprightFont=cmunrm.ttf,BoldFont=cmunbx.ttf,ItalicFont=cmunti.ttf,BoldItalicFont=cmunbi.ttf]{cmunti.ttf}\setmonofont[Path=/usr/share/fonts/truetype/cmu/,UprightFont=cmuntt.ttf,BoldFont=cmuntb.ttf,ItalicFont=cmunit.ttf,BoldItalicFont=cmuntx.ttf]{cmunti.ttf}\itshape JabRef}{$\text{ }$}\setmainfont[Path=/usr/share/fonts/truetype/cmu/,UprightFont=cmunrm.ttf,BoldFont=cmunbx.ttf,ItalicFont=cmunti.ttf,BoldItalicFont=cmunbi.ttf]{cmunrm.ttf}\setmonofont[Path=/usr/share/fonts/truetype/cmu/,UprightFont=cmuntt.ttf,BoldFont=cmuntb.ttf,ItalicFont=cmunit.ttf,BoldItalicFont=cmuntx.ttf]{cmunrm.ttf} saves the BibTeX data base with the native newline character 
of the author\textquotesingle{}s operating system,
it is recommended to add the {\itshape \setmainfont[Path=/usr/share/fonts/truetype/cmu/,UprightFont=cmunrm.ttf,BoldFont=cmunbx.ttf,ItalicFont=cmunti.ttf,BoldItalicFont=cmunbi.ttf]{cmunti.ttf}\setmonofont[Path=/usr/share/fonts/truetype/cmu/,UprightFont=cmuntt.ttf,BoldFont=cmuntb.ttf,ItalicFont=cmunit.ttf,BoldItalicFont=cmuntx.ttf]{cmunti.ttf}\itshape Subversion}{$\text{ }$}\setmainfont[Path=/usr/share/fonts/truetype/cmu/,UprightFont=cmunrm.ttf,BoldFont=cmunbx.ttf,ItalicFont=cmunti.ttf,BoldItalicFont=cmunbi.ttf]{cmunrm.ttf}\setmonofont[Path=/usr/share/fonts/truetype/cmu/,UprightFont=cmuntt.ttf,BoldFont=cmuntb.ttf,ItalicFont=cmunit.ttf,BoldItalicFont=cmuntx.ttf]{cmunrm.ttf} property \textquotesingle{}svn:eol-{}style\textquotesingle{} and set it to \textquotesingle{}native\textquotesingle{}
(see section \textquotesingle{}Subversion really makes the difference\textquotesingle{}).



\begin{minipage}{1.0\linewidth}
\begin{center}
\includegraphics[width=1.0\linewidth,height=6.5in,keepaspectratio]{../images/215.png}
\end{center}
\raggedright{}\myfigurewithcaption{215}{Figure 3:
Specify default key pattern in {\itshape \setmainfont[Path=/usr/share/fonts/truetype/cmu/,UprightFont=cmunrm.ttf,BoldFont=cmunbx.ttf,ItalicFont=cmunti.ttf,BoldItalicFont=cmunbi.ttf]{cmunti.ttf}\setmonofont[Path=/usr/share/fonts/truetype/cmu/,UprightFont=cmuntt.ttf,BoldFont=cmuntb.ttf,ItalicFont=cmunit.ttf,BoldItalicFont=cmuntx.ttf]{cmunti.ttf}\itshape JabRef}}
\end{minipage}\vspace{0.75cm}


{\itshape \setmainfont[Path=/usr/share/fonts/truetype/cmu/,UprightFont=cmunrm.ttf,BoldFont=cmunbx.ttf,ItalicFont=cmunti.ttf,BoldItalicFont=cmunbi.ttf]{cmunti.ttf}\setmonofont[Path=/usr/share/fonts/truetype/cmu/,UprightFont=cmuntt.ttf,BoldFont=cmuntb.ttf,ItalicFont=cmunit.ttf,BoldItalicFont=cmuntx.ttf]{cmunti.ttf}\itshape JabRef}{$\text{ }$}\setmainfont[Path=/usr/share/fonts/truetype/cmu/,UprightFont=cmunrm.ttf,BoldFont=cmunbx.ttf,ItalicFont=cmunti.ttf,BoldItalicFont=cmunbi.ttf]{cmunrm.ttf}\setmonofont[Path=/usr/share/fonts/truetype/cmu/,UprightFont=cmuntt.ttf,BoldFont=cmuntb.ttf,ItalicFont=cmunit.ttf,BoldItalicFont=cmuntx.ttf]{cmunrm.ttf} is highly flexible
and can be configured in many details.
We make the following changes to the default configuration
of {\itshape \setmainfont[Path=/usr/share/fonts/truetype/cmu/,UprightFont=cmunrm.ttf,BoldFont=cmunbx.ttf,ItalicFont=cmunti.ttf,BoldItalicFont=cmunbi.ttf]{cmunti.ttf}\setmonofont[Path=/usr/share/fonts/truetype/cmu/,UprightFont=cmuntt.ttf,BoldFont=cmuntb.ttf,ItalicFont=cmunit.ttf,BoldItalicFont=cmuntx.ttf]{cmunti.ttf}\itshape JabRef}{$\text{ }$}\setmainfont[Path=/usr/share/fonts/truetype/cmu/,UprightFont=cmunrm.ttf,BoldFont=cmunbx.ttf,ItalicFont=cmunti.ttf,BoldItalicFont=cmunbi.ttf]{cmunrm.ttf}\setmonofont[Path=/usr/share/fonts/truetype/cmu/,UprightFont=cmuntt.ttf,BoldFont=cmuntb.ttf,ItalicFont=cmunit.ttf,BoldItalicFont=cmuntx.ttf]{cmunrm.ttf} to simplify our work.
First, we specify the default pattern for BibTeX keys
so that {\itshape \setmainfont[Path=/usr/share/fonts/truetype/cmu/,UprightFont=cmunrm.ttf,BoldFont=cmunbx.ttf,ItalicFont=cmunti.ttf,BoldItalicFont=cmunbi.ttf]{cmunti.ttf}\setmonofont[Path=/usr/share/fonts/truetype/cmu/,UprightFont=cmuntt.ttf,BoldFont=cmuntb.ttf,ItalicFont=cmunit.ttf,BoldItalicFont=cmuntx.ttf]{cmunti.ttf}\itshape JabRef}{$\text{ }$}\setmainfont[Path=/usr/share/fonts/truetype/cmu/,UprightFont=cmunrm.ttf,BoldFont=cmunbx.ttf,ItalicFont=cmunti.ttf,BoldItalicFont=cmunbi.ttf]{cmunrm.ttf}\setmonofont[Path=/usr/share/fonts/truetype/cmu/,UprightFont=cmuntt.ttf,BoldFont=cmuntb.ttf,ItalicFont=cmunit.ttf,BoldItalicFont=cmuntx.ttf]{cmunrm.ttf} can automatically generate keys
in our desired format.
This can be done by selecting
{\ttfamily \setmainfont[Path=/usr/share/fonts/truetype/cmu/,UprightFont=cmunrm.ttf,BoldFont=cmunbx.ttf,ItalicFont=cmunti.ttf,BoldItalicFont=cmunbi.ttf]{cmuntt.ttf}\setmonofont[Path=/usr/share/fonts/truetype/cmu/,UprightFont=cmuntt.ttf,BoldFont=cmuntb.ttf,ItalicFont=cmunit.ttf,BoldItalicFont=cmuntx.ttf]{cmuntt.ttf}\ttfamily Options}\setmainfont[Path=/usr/share/fonts/truetype/cmu/,UprightFont=cmunrm.ttf,BoldFont=cmunbx.ttf,ItalicFont=cmunti.ttf,BoldItalicFont=cmunbi.ttf]{cmunrm.ttf}\setmonofont[Path=/usr/share/fonts/truetype/cmu/,UprightFont=cmuntt.ttf,BoldFont=cmuntb.ttf,ItalicFont=cmunit.ttf,BoldItalicFont=cmuntx.ttf]{cmunrm.ttf}
{\mbox{$\rightarrow$}} {\ttfamily \setmainfont[Path=/usr/share/fonts/truetype/cmu/,UprightFont=cmunrm.ttf,BoldFont=cmunbx.ttf,ItalicFont=cmunti.ttf,BoldItalicFont=cmunbi.ttf]{cmuntt.ttf}\setmonofont[Path=/usr/share/fonts/truetype/cmu/,UprightFont=cmuntt.ttf,BoldFont=cmuntb.ttf,ItalicFont=cmunit.ttf,BoldItalicFont=cmuntx.ttf]{cmuntt.ttf}\ttfamily Preferences}\setmainfont[Path=/usr/share/fonts/truetype/cmu/,UprightFont=cmunrm.ttf,BoldFont=cmunbx.ttf,ItalicFont=cmunti.ttf,BoldItalicFont=cmunbi.ttf]{cmunrm.ttf}\setmonofont[Path=/usr/share/fonts/truetype/cmu/,UprightFont=cmuntt.ttf,BoldFont=cmuntb.ttf,ItalicFont=cmunit.ttf,BoldItalicFont=cmuntx.ttf]{cmunrm.ttf}
{\mbox{$\rightarrow$}} {\ttfamily \setmainfont[Path=/usr/share/fonts/truetype/cmu/,UprightFont=cmunrm.ttf,BoldFont=cmunbx.ttf,ItalicFont=cmunti.ttf,BoldItalicFont=cmunbi.ttf]{cmuntt.ttf}\setmonofont[Path=/usr/share/fonts/truetype/cmu/,UprightFont=cmuntt.ttf,BoldFont=cmuntb.ttf,ItalicFont=cmunit.ttf,BoldItalicFont=cmuntx.ttf]{cmuntt.ttf}\ttfamily Key pattern}\setmainfont[Path=/usr/share/fonts/truetype/cmu/,UprightFont=cmunrm.ttf,BoldFont=cmunbx.ttf,ItalicFont=cmunti.ttf,BoldItalicFont=cmunbi.ttf]{cmunrm.ttf}\setmonofont[Path=/usr/share/fonts/truetype/cmu/,UprightFont=cmuntt.ttf,BoldFont=cmuntb.ttf,ItalicFont=cmunit.ttf,BoldItalicFont=cmuntx.ttf]{cmunrm.ttf}
and modifying the desired pattern in the field
{\ttfamily \setmainfont[Path=/usr/share/fonts/truetype/cmu/,UprightFont=cmunrm.ttf,BoldFont=cmunbx.ttf,ItalicFont=cmunti.ttf,BoldItalicFont=cmunbi.ttf]{cmuntt.ttf}\setmonofont[Path=/usr/share/fonts/truetype/cmu/,UprightFont=cmuntt.ttf,BoldFont=cmuntb.ttf,ItalicFont=cmunit.ttf,BoldItalicFont=cmuntx.ttf]{cmuntt.ttf}\ttfamily Default pattern}\setmainfont[Path=/usr/share/fonts/truetype/cmu/,UprightFont=cmunrm.ttf,BoldFont=cmunbx.ttf,ItalicFont=cmunti.ttf,BoldItalicFont=cmunbi.ttf]{cmunrm.ttf}\setmonofont[Path=/usr/share/fonts/truetype/cmu/,UprightFont=cmuntt.ttf,BoldFont=cmuntb.ttf,ItalicFont=cmunit.ttf,BoldItalicFont=cmuntx.ttf]{cmunrm.ttf}.
For instance, we use {\ttfamily \setmainfont[Path=/usr/share/fonts/truetype/cmu/,UprightFont=cmunrm.ttf,BoldFont=cmunbx.ttf,ItalicFont=cmunti.ttf,BoldItalicFont=cmunbi.ttf]{cmuntt.ttf}\setmonofont[Path=/usr/share/fonts/truetype/cmu/,UprightFont=cmuntt.ttf,BoldFont=cmuntb.ttf,ItalicFont=cmunit.ttf,BoldItalicFont=cmuntx.ttf]{cmuntt.ttf}\ttfamily {$\text{[}$}auth:lower{$\text{]}$}{$\text{[}$}shortyear{$\text{]}$}}\setmainfont[Path=/usr/share/fonts/truetype/cmu/,UprightFont=cmunrm.ttf,BoldFont=cmunbx.ttf,ItalicFont=cmunti.ttf,BoldItalicFont=cmunbi.ttf]{cmunrm.ttf}\setmonofont[Path=/usr/share/fonts/truetype/cmu/,UprightFont=cmuntt.ttf,BoldFont=cmuntb.ttf,ItalicFont=cmunit.ttf,BoldItalicFont=cmuntx.ttf]{cmunrm.ttf}
to get the last name of the first author in lower case
and the last two digits of the year of the publication
(see figure 3).



\begin{minipage}{1.0\linewidth}
\begin{center}
\includegraphics[width=1.0\linewidth,height=6.5in,keepaspectratio]{../images/216.png}
\end{center}
\raggedright{}\myfigurewithcaption{216}{Figure 4:
Set up general fields in {\itshape \setmainfont[Path=/usr/share/fonts/truetype/cmu/,UprightFont=cmunrm.ttf,BoldFont=cmunbx.ttf,ItalicFont=cmunti.ttf,BoldItalicFont=cmunbi.ttf]{cmunti.ttf}\setmonofont[Path=/usr/share/fonts/truetype/cmu/,UprightFont=cmuntt.ttf,BoldFont=cmuntb.ttf,ItalicFont=cmunit.ttf,BoldItalicFont=cmuntx.ttf]{cmunti.ttf}\itshape JabRef}}
\end{minipage}\vspace{0.75cm}


Second, we add the BibTeX field {\ttfamily \setmainfont[Path=/usr/share/fonts/truetype/cmu/,UprightFont=cmunrm.ttf,BoldFont=cmunbx.ttf,ItalicFont=cmunti.ttf,BoldItalicFont=cmunbi.ttf]{cmuntt.ttf}\setmonofont[Path=/usr/share/fonts/truetype/cmu/,UprightFont=cmuntt.ttf,BoldFont=cmuntb.ttf,ItalicFont=cmunit.ttf,BoldItalicFont=cmuntx.ttf]{cmuntt.ttf}\ttfamily location}{$\text{ }$}\setmainfont[Path=/usr/share/fonts/truetype/cmu/,UprightFont=cmunrm.ttf,BoldFont=cmunbx.ttf,ItalicFont=cmunti.ttf,BoldItalicFont=cmunbi.ttf]{cmunrm.ttf}\setmonofont[Path=/usr/share/fonts/truetype/cmu/,UprightFont=cmuntt.ttf,BoldFont=cmuntb.ttf,ItalicFont=cmunit.ttf,BoldItalicFont=cmuntx.ttf]{cmunrm.ttf} for information
about the location,
where the publication is available as hard copy
(e.g. a book or a copy of an article).
This field can contain the name of the user who has the hard copy
and where he has it or the name of a library and the shelf-{}mark.
This field can be added in {\itshape \setmainfont[Path=/usr/share/fonts/truetype/cmu/,UprightFont=cmunrm.ttf,BoldFont=cmunbx.ttf,ItalicFont=cmunti.ttf,BoldItalicFont=cmunbi.ttf]{cmunti.ttf}\setmonofont[Path=/usr/share/fonts/truetype/cmu/,UprightFont=cmuntt.ttf,BoldFont=cmuntb.ttf,ItalicFont=cmunit.ttf,BoldItalicFont=cmuntx.ttf]{cmunti.ttf}\itshape JabRef}{$\text{ }$}\setmainfont[Path=/usr/share/fonts/truetype/cmu/,UprightFont=cmunrm.ttf,BoldFont=cmunbx.ttf,ItalicFont=cmunti.ttf,BoldItalicFont=cmunbi.ttf]{cmunrm.ttf}\setmonofont[Path=/usr/share/fonts/truetype/cmu/,UprightFont=cmuntt.ttf,BoldFont=cmuntb.ttf,ItalicFont=cmunit.ttf,BoldItalicFont=cmuntx.ttf]{cmunrm.ttf} by selecting
{\ttfamily \setmainfont[Path=/usr/share/fonts/truetype/cmu/,UprightFont=cmunrm.ttf,BoldFont=cmunbx.ttf,ItalicFont=cmunti.ttf,BoldItalicFont=cmunbi.ttf]{cmuntt.ttf}\setmonofont[Path=/usr/share/fonts/truetype/cmu/,UprightFont=cmuntt.ttf,BoldFont=cmuntb.ttf,ItalicFont=cmunit.ttf,BoldItalicFont=cmuntx.ttf]{cmuntt.ttf}\ttfamily Options}\setmainfont[Path=/usr/share/fonts/truetype/cmu/,UprightFont=cmunrm.ttf,BoldFont=cmunbx.ttf,ItalicFont=cmunti.ttf,BoldItalicFont=cmunbi.ttf]{cmunrm.ttf}\setmonofont[Path=/usr/share/fonts/truetype/cmu/,UprightFont=cmuntt.ttf,BoldFont=cmuntb.ttf,ItalicFont=cmunit.ttf,BoldItalicFont=cmuntx.ttf]{cmunrm.ttf}
{\mbox{$\rightarrow$}} {\ttfamily \setmainfont[Path=/usr/share/fonts/truetype/cmu/,UprightFont=cmunrm.ttf,BoldFont=cmunbx.ttf,ItalicFont=cmunti.ttf,BoldItalicFont=cmunbi.ttf]{cmuntt.ttf}\setmonofont[Path=/usr/share/fonts/truetype/cmu/,UprightFont=cmuntt.ttf,BoldFont=cmuntb.ttf,ItalicFont=cmunit.ttf,BoldItalicFont=cmuntx.ttf]{cmuntt.ttf}\ttfamily Set up general fields}\setmainfont[Path=/usr/share/fonts/truetype/cmu/,UprightFont=cmunrm.ttf,BoldFont=cmunbx.ttf,ItalicFont=cmunti.ttf,BoldItalicFont=cmunbi.ttf]{cmunrm.ttf}\setmonofont[Path=/usr/share/fonts/truetype/cmu/,UprightFont=cmuntt.ttf,BoldFont=cmuntb.ttf,ItalicFont=cmunit.ttf,BoldItalicFont=cmuntx.ttf]{cmunrm.ttf}
and adding the word {\ttfamily \setmainfont[Path=/usr/share/fonts/truetype/cmu/,UprightFont=cmunrm.ttf,BoldFont=cmunbx.ttf,ItalicFont=cmunti.ttf,BoldItalicFont=cmunbi.ttf]{cmuntt.ttf}\setmonofont[Path=/usr/share/fonts/truetype/cmu/,UprightFont=cmuntt.ttf,BoldFont=cmuntb.ttf,ItalicFont=cmunit.ttf,BoldItalicFont=cmuntx.ttf]{cmuntt.ttf}\ttfamily location}\setmainfont[Path=/usr/share/fonts/truetype/cmu/,UprightFont=cmunrm.ttf,BoldFont=cmunbx.ttf,ItalicFont=cmunti.ttf,BoldItalicFont=cmunbi.ttf]{cmunrm.ttf}\setmonofont[Path=/usr/share/fonts/truetype/cmu/,UprightFont=cmuntt.ttf,BoldFont=cmuntb.ttf,ItalicFont=cmunit.ttf,BoldItalicFont=cmuntx.ttf]{cmunrm.ttf}
(using the semicolon ({\ttfamily \setmainfont[Path=/usr/share/fonts/truetype/cmu/,UprightFont=cmunrm.ttf,BoldFont=cmunbx.ttf,ItalicFont=cmunti.ttf,BoldItalicFont=cmunbi.ttf]{cmuntt.ttf}\setmonofont[Path=/usr/share/fonts/truetype/cmu/,UprightFont=cmuntt.ttf,BoldFont=cmuntb.ttf,ItalicFont=cmunit.ttf,BoldItalicFont=cmuntx.ttf]{cmuntt.ttf}\ttfamily ;}\setmainfont[Path=/usr/share/fonts/truetype/cmu/,UprightFont=cmunrm.ttf,BoldFont=cmunbx.ttf,ItalicFont=cmunti.ttf,BoldItalicFont=cmunbi.ttf]{cmunrm.ttf}\setmonofont[Path=/usr/share/fonts/truetype/cmu/,UprightFont=cmuntt.ttf,BoldFont=cmuntb.ttf,ItalicFont=cmunit.ttf,BoldItalicFont=cmuntx.ttf]{cmunrm.ttf}) as delimiter)
somewhere in the line that starts with {\ttfamily \setmainfont[Path=/usr/share/fonts/truetype/cmu/,UprightFont=cmunrm.ttf,BoldFont=cmunbx.ttf,ItalicFont=cmunti.ttf,BoldItalicFont=cmunbi.ttf]{cmuntt.ttf}\setmonofont[Path=/usr/share/fonts/truetype/cmu/,UprightFont=cmuntt.ttf,BoldFont=cmuntb.ttf,ItalicFont=cmunit.ttf,BoldItalicFont=cmuntx.ttf]{cmuntt.ttf}\ttfamily General:}\setmainfont[Path=/usr/share/fonts/truetype/cmu/,UprightFont=cmunrm.ttf,BoldFont=cmunbx.ttf,ItalicFont=cmunti.ttf,BoldItalicFont=cmunbi.ttf]{cmunrm.ttf}\setmonofont[Path=/usr/share/fonts/truetype/cmu/,UprightFont=cmuntt.ttf,BoldFont=cmuntb.ttf,ItalicFont=cmunit.ttf,BoldItalicFont=cmuntx.ttf]{cmunrm.ttf}
(see figure 4).



\begin{minipage}{1.0\linewidth}
\begin{center}
\includegraphics[width=1.0\linewidth,height=6.5in,keepaspectratio]{../images/217.png}
\end{center}
\raggedright{}\myfigurewithcaption{217}{Figure 5:
Specify \textquotesingle{}Main PDF directory\textquotesingle{} in {\itshape \setmainfont[Path=/usr/share/fonts/truetype/cmu/,UprightFont=cmunrm.ttf,BoldFont=cmunbx.ttf,ItalicFont=cmunti.ttf,BoldItalicFont=cmunbi.ttf]{cmunti.ttf}\setmonofont[Path=/usr/share/fonts/truetype/cmu/,UprightFont=cmuntt.ttf,BoldFont=cmuntb.ttf,ItalicFont=cmunit.ttf,BoldItalicFont=cmuntx.ttf]{cmunti.ttf}\itshape JabRef}}
\end{minipage}\vspace{0.75cm}


Third, we put all PDF files of publications in a specific
subdirectory in our file server,
where we use the BibTeX key as file name.
We inform {\itshape \setmainfont[Path=/usr/share/fonts/truetype/cmu/,UprightFont=cmunrm.ttf,BoldFont=cmunbx.ttf,ItalicFont=cmunti.ttf,BoldItalicFont=cmunbi.ttf]{cmunti.ttf}\setmonofont[Path=/usr/share/fonts/truetype/cmu/,UprightFont=cmuntt.ttf,BoldFont=cmuntb.ttf,ItalicFont=cmunit.ttf,BoldItalicFont=cmuntx.ttf]{cmunti.ttf}\itshape JabRef}{$\text{ }$}\setmainfont[Path=/usr/share/fonts/truetype/cmu/,UprightFont=cmunrm.ttf,BoldFont=cmunbx.ttf,ItalicFont=cmunti.ttf,BoldItalicFont=cmunbi.ttf]{cmunrm.ttf}\setmonofont[Path=/usr/share/fonts/truetype/cmu/,UprightFont=cmuntt.ttf,BoldFont=cmuntb.ttf,ItalicFont=cmunit.ttf,BoldItalicFont=cmuntx.ttf]{cmunrm.ttf} about this subdirectory
by selecting
{\ttfamily \setmainfont[Path=/usr/share/fonts/truetype/cmu/,UprightFont=cmunrm.ttf,BoldFont=cmunbx.ttf,ItalicFont=cmunti.ttf,BoldItalicFont=cmunbi.ttf]{cmuntt.ttf}\setmonofont[Path=/usr/share/fonts/truetype/cmu/,UprightFont=cmuntt.ttf,BoldFont=cmuntb.ttf,ItalicFont=cmunit.ttf,BoldItalicFont=cmuntx.ttf]{cmuntt.ttf}\ttfamily Options}\setmainfont[Path=/usr/share/fonts/truetype/cmu/,UprightFont=cmunrm.ttf,BoldFont=cmunbx.ttf,ItalicFont=cmunti.ttf,BoldItalicFont=cmunbi.ttf]{cmunrm.ttf}\setmonofont[Path=/usr/share/fonts/truetype/cmu/,UprightFont=cmuntt.ttf,BoldFont=cmuntb.ttf,ItalicFont=cmunit.ttf,BoldItalicFont=cmuntx.ttf]{cmunrm.ttf}
{\mbox{$\rightarrow$}} {\ttfamily \setmainfont[Path=/usr/share/fonts/truetype/cmu/,UprightFont=cmunrm.ttf,BoldFont=cmunbx.ttf,ItalicFont=cmunti.ttf,BoldItalicFont=cmunbi.ttf]{cmuntt.ttf}\setmonofont[Path=/usr/share/fonts/truetype/cmu/,UprightFont=cmuntt.ttf,BoldFont=cmuntb.ttf,ItalicFont=cmunit.ttf,BoldItalicFont=cmuntx.ttf]{cmuntt.ttf}\ttfamily Preferences}\setmainfont[Path=/usr/share/fonts/truetype/cmu/,UprightFont=cmunrm.ttf,BoldFont=cmunbx.ttf,ItalicFont=cmunti.ttf,BoldItalicFont=cmunbi.ttf]{cmunrm.ttf}\setmonofont[Path=/usr/share/fonts/truetype/cmu/,UprightFont=cmuntt.ttf,BoldFont=cmuntb.ttf,ItalicFont=cmunit.ttf,BoldItalicFont=cmuntx.ttf]{cmunrm.ttf}
{\mbox{$\rightarrow$}} {\ttfamily \setmainfont[Path=/usr/share/fonts/truetype/cmu/,UprightFont=cmunrm.ttf,BoldFont=cmunbx.ttf,ItalicFont=cmunti.ttf,BoldItalicFont=cmunbi.ttf]{cmuntt.ttf}\setmonofont[Path=/usr/share/fonts/truetype/cmu/,UprightFont=cmuntt.ttf,BoldFont=cmuntb.ttf,ItalicFont=cmunit.ttf,BoldItalicFont=cmuntx.ttf]{cmuntt.ttf}\ttfamily External programs}\setmainfont[Path=/usr/share/fonts/truetype/cmu/,UprightFont=cmunrm.ttf,BoldFont=cmunbx.ttf,ItalicFont=cmunti.ttf,BoldItalicFont=cmunbi.ttf]{cmunrm.ttf}\setmonofont[Path=/usr/share/fonts/truetype/cmu/,UprightFont=cmuntt.ttf,BoldFont=cmuntb.ttf,ItalicFont=cmunit.ttf,BoldItalicFont=cmuntx.ttf]{cmunrm.ttf}
and adding the path of the this subdirectory
in the field {\ttfamily \setmainfont[Path=/usr/share/fonts/truetype/cmu/,UprightFont=cmunrm.ttf,BoldFont=cmunbx.ttf,ItalicFont=cmunti.ttf,BoldItalicFont=cmunbi.ttf]{cmuntt.ttf}\setmonofont[Path=/usr/share/fonts/truetype/cmu/,UprightFont=cmuntt.ttf,BoldFont=cmuntb.ttf,ItalicFont=cmunit.ttf,BoldItalicFont=cmuntx.ttf]{cmuntt.ttf}\ttfamily Main PDF directory}\setmainfont[Path=/usr/share/fonts/truetype/cmu/,UprightFont=cmunrm.ttf,BoldFont=cmunbx.ttf,ItalicFont=cmunti.ttf,BoldItalicFont=cmunbi.ttf]{cmunrm.ttf}\setmonofont[Path=/usr/share/fonts/truetype/cmu/,UprightFont=cmuntt.ttf,BoldFont=cmuntb.ttf,ItalicFont=cmunit.ttf,BoldItalicFont=cmuntx.ttf]{cmunrm.ttf}
(see figure 5).
If a PDF file of a publication is available,
the user can push the {\ttfamily \setmainfont[Path=/usr/share/fonts/truetype/cmu/,UprightFont=cmunrm.ttf,BoldFont=cmunbx.ttf,ItalicFont=cmunti.ttf,BoldItalicFont=cmunbi.ttf]{cmuntt.ttf}\setmonofont[Path=/usr/share/fonts/truetype/cmu/,UprightFont=cmuntt.ttf,BoldFont=cmuntb.ttf,ItalicFont=cmunit.ttf,BoldItalicFont=cmuntx.ttf]{cmuntt.ttf}\ttfamily Auto}{$\text{ }$}\setmainfont[Path=/usr/share/fonts/truetype/cmu/,UprightFont=cmunrm.ttf,BoldFont=cmunbx.ttf,ItalicFont=cmunti.ttf,BoldItalicFont=cmunbi.ttf]{cmunrm.ttf}\setmonofont[Path=/usr/share/fonts/truetype/cmu/,UprightFont=cmuntt.ttf,BoldFont=cmuntb.ttf,ItalicFont=cmunit.ttf,BoldItalicFont=cmuntx.ttf]{cmunrm.ttf} button
left of {\itshape \setmainfont[Path=/usr/share/fonts/truetype/cmu/,UprightFont=cmunrm.ttf,BoldFont=cmunbx.ttf,ItalicFont=cmunti.ttf,BoldItalicFont=cmunbi.ttf]{cmunti.ttf}\setmonofont[Path=/usr/share/fonts/truetype/cmu/,UprightFont=cmuntt.ttf,BoldFont=cmuntb.ttf,ItalicFont=cmunit.ttf,BoldItalicFont=cmuntx.ttf]{cmunti.ttf}\itshape JabRef{\bfseries \setmainfont[Path=/usr/share/fonts/truetype/cmu/,UprightFont=cmunrm.ttf,BoldFont=cmunbx.ttf,ItalicFont=cmunti.ttf,BoldItalicFont=cmunbi.ttf]{cmunbi.ttf}\setmonofont[Path=/usr/share/fonts/truetype/cmu/,UprightFont=cmuntt.ttf,BoldFont=cmuntb.ttf,ItalicFont=cmunit.ttf,BoldItalicFont=cmuntx.ttf]{cmunbi.ttf}\bfseries \itshape s {\ttfamily \setmainfont[Path=/usr/share/fonts/truetype/cmu/,UprightFont=cmunrm.ttf,BoldFont=cmunbx.ttf,ItalicFont=cmunti.ttf,BoldItalicFont=cmunbi.ttf]{cmuntx.ttf}\setmonofont[Path=/usr/share/fonts/truetype/cmu/,UprightFont=cmuntt.ttf,BoldFont=cmuntb.ttf,ItalicFont=cmunit.ttf,BoldItalicFont=cmuntx.ttf]{cmuntx.ttf}\ttfamily \bfseries \itshape Pdf}{$\text{ }$}\setmainfont[Path=/usr/share/fonts/truetype/cmu/,UprightFont=cmunrm.ttf,BoldFont=cmunbx.ttf,ItalicFont=cmunti.ttf,BoldItalicFont=cmunbi.ttf]{cmunbi.ttf}\setmonofont[Path=/usr/share/fonts/truetype/cmu/,UprightFont=cmuntt.ttf,BoldFont=cmuntb.ttf,ItalicFont=cmunit.ttf,BoldItalicFont=cmuntx.ttf]{cmunbi.ttf}\bfseries \itshape  field}}\setmainfont[Path=/usr/share/fonts/truetype/cmu/,UprightFont=cmunrm.ttf,BoldFont=cmunbx.ttf,ItalicFont=cmunti.ttf,BoldItalicFont=cmunbi.ttf]{cmunrm.ttf}\setmonofont[Path=/usr/share/fonts/truetype/cmu/,UprightFont=cmuntt.ttf,BoldFont=cmuntb.ttf,ItalicFont=cmunit.ttf,BoldItalicFont=cmuntx.ttf]{cmunrm.ttf}
to automatically add the file name of the PDF file.
Now, all users who have access to the file server
can open the PDF file of a publication
by simply clicking on {\itshape \setmainfont[Path=/usr/share/fonts/truetype/cmu/,UprightFont=cmunrm.ttf,BoldFont=cmunbx.ttf,ItalicFont=cmunti.ttf,BoldItalicFont=cmunbi.ttf]{cmunti.ttf}\setmonofont[Path=/usr/share/fonts/truetype/cmu/,UprightFont=cmuntt.ttf,BoldFont=cmuntb.ttf,ItalicFont=cmunit.ttf,BoldItalicFont=cmuntx.ttf]{cmunti.ttf}\itshape JabRef{\bfseries \setmainfont[Path=/usr/share/fonts/truetype/cmu/,UprightFont=cmunrm.ttf,BoldFont=cmunbx.ttf,ItalicFont=cmunti.ttf,BoldItalicFont=cmunbi.ttf]{cmunbi.ttf}\setmonofont[Path=/usr/share/fonts/truetype/cmu/,UprightFont=cmuntt.ttf,BoldFont=cmuntb.ttf,ItalicFont=cmunit.ttf,BoldItalicFont=cmuntx.ttf]{cmunbi.ttf}\bfseries \itshape s PDF icon.}}\setmainfont[Path=/usr/share/fonts/truetype/cmu/,UprightFont=cmunrm.ttf,BoldFont=cmunbx.ttf,ItalicFont=cmunti.ttf,BoldItalicFont=cmunbi.ttf]{cmunrm.ttf}\setmonofont[Path=/usr/share/fonts/truetype/cmu/,UprightFont=cmuntt.ttf,BoldFont=cmuntb.ttf,ItalicFont=cmunit.ttf,BoldItalicFont=cmuntx.ttf]{cmunrm.ttf}

If we send the LaTeX source code of a project
to a journal, publisher, or somebody else
who has no access to our common {\ttfamily \setmainfont[Path=/usr/share/fonts/truetype/cmu/,UprightFont=cmunrm.ttf,BoldFont=cmunbx.ttf,ItalicFont=cmunti.ttf,BoldItalicFont=cmunbi.ttf]{cmuntt.ttf}\setmonofont[Path=/usr/share/fonts/truetype/cmu/,UprightFont=cmuntt.ttf,BoldFont=cmuntb.ttf,ItalicFont=cmunit.ttf,BoldItalicFont=cmuntx.ttf]{cmuntt.ttf}\ttfamily texmf}{$\text{ }$}\setmainfont[Path=/usr/share/fonts/truetype/cmu/,UprightFont=cmunrm.ttf,BoldFont=cmunbx.ttf,ItalicFont=cmunti.ttf,BoldItalicFont=cmunbi.ttf]{cmunrm.ttf}\setmonofont[Path=/usr/share/fonts/truetype/cmu/,UprightFont=cmuntt.ttf,BoldFont=cmuntb.ttf,ItalicFont=cmunit.ttf,BoldItalicFont=cmuntx.ttf]{cmunrm.ttf} tree,
we do not include our entire bibliographic data base,
but extract the relevant entries with
the Perl script {\itshape \setmainfont[Path=/usr/share/fonts/truetype/cmu/,UprightFont=cmunrm.ttf,BoldFont=cmunbx.ttf,ItalicFont=cmunti.ttf,BoldItalicFont=cmunbi.ttf]{cmunti.ttf}\setmonofont[Path=/usr/share/fonts/truetype/cmu/,UprightFont=cmuntt.ttf,BoldFont=cmuntb.ttf,ItalicFont=cmunit.ttf,BoldItalicFont=cmuntx.ttf]{cmunti.ttf}\itshape aux2bib}{$\text{ }$}\setmainfont[Path=/usr/share/fonts/truetype/cmu/,UprightFont=cmunrm.ttf,BoldFont=cmunbx.ttf,ItalicFont=cmunti.ttf,BoldItalicFont=cmunbi.ttf]{cmunrm.ttf}\setmonofont[Path=/usr/share/fonts/truetype/cmu/,UprightFont=cmuntt.ttf,BoldFont=cmuntb.ttf,ItalicFont=cmunit.ttf,BoldItalicFont=cmuntx.ttf]{cmunrm.ttf} (\myplainurl{http://www.ctan.org/tex-archive/biblio/bibtex/utils/bibtools/aux2bib).}
\section{Conclusion}
\label{921}

This wikibook describes a possible way
to efficiently organise the collaborative preparation
of LaTeX documents.
The presented solution is based on the
{\itshape \setmainfont[Path=/usr/share/fonts/truetype/cmu/,UprightFont=cmunrm.ttf,BoldFont=cmunbx.ttf,ItalicFont=cmunti.ttf,BoldItalicFont=cmunbi.ttf]{cmunti.ttf}\setmonofont[Path=/usr/share/fonts/truetype/cmu/,UprightFont=cmuntt.ttf,BoldFont=cmuntb.ttf,ItalicFont=cmunit.ttf,BoldItalicFont=cmuntx.ttf]{cmunti.ttf}\itshape Subversion}{$\text{ }$}\setmainfont[Path=/usr/share/fonts/truetype/cmu/,UprightFont=cmunrm.ttf,BoldFont=cmunbx.ttf,ItalicFont=cmunti.ttf,BoldItalicFont=cmunbi.ttf]{cmunrm.ttf}\setmonofont[Path=/usr/share/fonts/truetype/cmu/,UprightFont=cmuntt.ttf,BoldFont=cmuntb.ttf,ItalicFont=cmunit.ttf,BoldItalicFont=cmuntx.ttf]{cmunrm.ttf} version control system
and several other software tools and LaTeX packages.
However, there are still a few issues
that can be improved.

First, we plan that all users install
the same LaTeX distribution.
As the {\itshape \setmainfont[Path=/usr/share/fonts/truetype/cmu/,UprightFont=cmunrm.ttf,BoldFont=cmunbx.ttf,ItalicFont=cmunti.ttf,BoldItalicFont=cmunbi.ttf]{cmunti.ttf}\setmonofont[Path=/usr/share/fonts/truetype/cmu/,UprightFont=cmuntt.ttf,BoldFont=cmuntb.ttf,ItalicFont=cmunit.ttf,BoldItalicFont=cmuntx.ttf]{cmunti.ttf}\itshape TeX Live}{$\text{ }$}\setmainfont[Path=/usr/share/fonts/truetype/cmu/,UprightFont=cmunrm.ttf,BoldFont=cmunbx.ttf,ItalicFont=cmunti.ttf,BoldItalicFont=cmunbi.ttf]{cmunrm.ttf}\setmonofont[Path=/usr/share/fonts/truetype/cmu/,UprightFont=cmuntt.ttf,BoldFont=cmuntb.ttf,ItalicFont=cmunit.ttf,BoldItalicFont=cmuntx.ttf]{cmunrm.ttf} distribution (\myplainurl{http://www.tug.org/texlive/)}
is available both for Unix and MS Windows operating systems,
we might recommend our users to switch to this LaTeX distribution
in the future.
(Currently, our users have different LaTeX distributions
that provide a different selection of LaTeX packages
and different versions of some packages.
We solve this problem by providing some packages on
our common {\ttfamily \setmainfont[Path=/usr/share/fonts/truetype/cmu/,UprightFont=cmunrm.ttf,BoldFont=cmunbx.ttf,ItalicFont=cmunti.ttf,BoldItalicFont=cmunbi.ttf]{cmuntt.ttf}\setmonofont[Path=/usr/share/fonts/truetype/cmu/,UprightFont=cmuntt.ttf,BoldFont=cmuntb.ttf,ItalicFont=cmunit.ttf,BoldItalicFont=cmuntx.ttf]{cmuntt.ttf}\ttfamily texmf}{$\text{ }$}\setmainfont[Path=/usr/share/fonts/truetype/cmu/,UprightFont=cmunrm.ttf,BoldFont=cmunbx.ttf,ItalicFont=cmunti.ttf,BoldItalicFont=cmunbi.ttf]{cmunrm.ttf}\setmonofont[Path=/usr/share/fonts/truetype/cmu/,UprightFont=cmuntt.ttf,BoldFont=cmuntb.ttf,ItalicFont=cmunit.ttf,BoldItalicFont=cmuntx.ttf]{cmunrm.ttf} tree.)

Second, we consider to simplify
the solution for a common bibliographic data base.
Currently it is based on the
version control system {\itshape \setmainfont[Path=/usr/share/fonts/truetype/cmu/,UprightFont=cmunrm.ttf,BoldFont=cmunbx.ttf,ItalicFont=cmunti.ttf,BoldItalicFont=cmunbi.ttf]{cmunti.ttf}\setmonofont[Path=/usr/share/fonts/truetype/cmu/,UprightFont=cmuntt.ttf,BoldFont=cmuntb.ttf,ItalicFont=cmunit.ttf,BoldItalicFont=cmuntx.ttf]{cmunti.ttf}\itshape Subversion}\setmainfont[Path=/usr/share/fonts/truetype/cmu/,UprightFont=cmunrm.ttf,BoldFont=cmunbx.ttf,ItalicFont=cmunti.ttf,BoldItalicFont=cmunbi.ttf]{cmunrm.ttf}\setmonofont[Path=/usr/share/fonts/truetype/cmu/,UprightFont=cmuntt.ttf,BoldFont=cmuntb.ttf,ItalicFont=cmunit.ttf,BoldItalicFont=cmuntx.ttf]{cmunrm.ttf},
the graphical BibTeX editor {\itshape \setmainfont[Path=/usr/share/fonts/truetype/cmu/,UprightFont=cmunrm.ttf,BoldFont=cmunbx.ttf,ItalicFont=cmunti.ttf,BoldItalicFont=cmunbi.ttf]{cmunti.ttf}\setmonofont[Path=/usr/share/fonts/truetype/cmu/,UprightFont=cmuntt.ttf,BoldFont=cmuntb.ttf,ItalicFont=cmunit.ttf,BoldItalicFont=cmuntx.ttf]{cmunti.ttf}\itshape JabRef}\setmainfont[Path=/usr/share/fonts/truetype/cmu/,UprightFont=cmunrm.ttf,BoldFont=cmunbx.ttf,ItalicFont=cmunti.ttf,BoldItalicFont=cmunbi.ttf]{cmunrm.ttf}\setmonofont[Path=/usr/share/fonts/truetype/cmu/,UprightFont=cmuntt.ttf,BoldFont=cmuntb.ttf,ItalicFont=cmunit.ttf,BoldItalicFont=cmuntx.ttf]{cmunrm.ttf},
and a file server for the PDF files of publications in the data base.
The usage of three different tools for one task
is rather challenging for infrequent users
and users that are not familiar with these tools.
Furthermore, the file server can be only accessed by local users.
Therefore, we consider to implement an integrated server solution
like {\itshape \setmainfont[Path=/usr/share/fonts/truetype/cmu/,UprightFont=cmunrm.ttf,BoldFont=cmunbx.ttf,ItalicFont=cmunti.ttf,BoldItalicFont=cmunbi.ttf]{cmunti.ttf}\setmonofont[Path=/usr/share/fonts/truetype/cmu/,UprightFont=cmuntt.ttf,BoldFont=cmuntb.ttf,ItalicFont=cmunit.ttf,BoldItalicFont=cmuntx.ttf]{cmunti.ttf}\itshape WIKINDX}{$\text{ }$}\setmainfont[Path=/usr/share/fonts/truetype/cmu/,UprightFont=cmunrm.ttf,BoldFont=cmunbx.ttf,ItalicFont=cmunti.ttf,BoldItalicFont=cmunbi.ttf]{cmunrm.ttf}\setmonofont[Path=/usr/share/fonts/truetype/cmu/,UprightFont=cmuntt.ttf,BoldFont=cmuntb.ttf,ItalicFont=cmunit.ttf,BoldItalicFont=cmuntx.ttf]{cmunrm.ttf} (\myplainurl{http://wikindx.sourceforge.net/),}
{\itshape \setmainfont[Path=/usr/share/fonts/truetype/cmu/,UprightFont=cmunrm.ttf,BoldFont=cmunbx.ttf,ItalicFont=cmunti.ttf,BoldItalicFont=cmunbi.ttf]{cmunti.ttf}\setmonofont[Path=/usr/share/fonts/truetype/cmu/,UprightFont=cmuntt.ttf,BoldFont=cmuntb.ttf,ItalicFont=cmunit.ttf,BoldItalicFont=cmuntx.ttf]{cmunti.ttf}\itshape Aigaion}{$\text{ }$}\setmainfont[Path=/usr/share/fonts/truetype/cmu/,UprightFont=cmunrm.ttf,BoldFont=cmunbx.ttf,ItalicFont=cmunti.ttf,BoldItalicFont=cmunbi.ttf]{cmunrm.ttf}\setmonofont[Path=/usr/share/fonts/truetype/cmu/,UprightFont=cmuntt.ttf,BoldFont=cmuntb.ttf,ItalicFont=cmunit.ttf,BoldItalicFont=cmuntx.ttf]{cmunrm.ttf} (\myplainurl{http://www.aigaion.nl/),}
or {\itshape \setmainfont[Path=/usr/share/fonts/truetype/cmu/,UprightFont=cmunrm.ttf,BoldFont=cmunbx.ttf,ItalicFont=cmunti.ttf,BoldItalicFont=cmunbi.ttf]{cmunti.ttf}\setmonofont[Path=/usr/share/fonts/truetype/cmu/,UprightFont=cmuntt.ttf,BoldFont=cmuntb.ttf,ItalicFont=cmunit.ttf,BoldItalicFont=cmuntx.ttf]{cmunti.ttf}\itshape refBASE}{$\text{ }$}\setmainfont[Path=/usr/share/fonts/truetype/cmu/,UprightFont=cmunrm.ttf,BoldFont=cmunbx.ttf,ItalicFont=cmunti.ttf,BoldItalicFont=cmunbi.ttf]{cmunrm.ttf}\setmonofont[Path=/usr/share/fonts/truetype/cmu/,UprightFont=cmuntt.ttf,BoldFont=cmuntb.ttf,ItalicFont=cmunit.ttf,BoldItalicFont=cmuntx.ttf]{cmunrm.ttf} (\myplainurl{http://refbase.sourceforge.net/).}
Using this solution only requires a computer with internet access
and a web browser,
which makes the usage of our data base considerably easier
for infrequent users.
Moreover, the stored PDF files are available
not only from within the department,
but throughout the world.
(Depending on the copy rights of the stored PDF files,
the access to the server
-{}-{}-{} or least the access to the PDF files -{}-{}-{}
has to be restricted to members of the department.)
Even Non-{}LaTeX users of our department might benefit
from a server-{}based solution,
because it should be easier to use this bibliographic data base
in (other) word processing software packages,
because these servers provide the data not only in BibTeX format,
but also in other formats.

All readers are encouraged to contribute to this wikibook
by adding further hints or ideas
or by providing further solutions to the problem
of collaborative writing of LaTeX documents.

\section{Acknowledgements}
\label{922}

Arne Henningsen thanks Francisco Reinaldo and Géraldine Henningsen
for comments and suggestions
that helped him to improve and clarify this paper,
Karsten Heymann for many hints and advices regarding LaTeX, BibTeX,
and {\itshape \setmainfont[Path=/usr/share/fonts/truetype/cmu/,UprightFont=cmunrm.ttf,BoldFont=cmunbx.ttf,ItalicFont=cmunti.ttf,BoldItalicFont=cmunbi.ttf]{cmunti.ttf}\setmonofont[Path=/usr/share/fonts/truetype/cmu/,UprightFont=cmuntt.ttf,BoldFont=cmuntb.ttf,ItalicFont=cmunit.ttf,BoldItalicFont=cmuntx.ttf]{cmunti.ttf}\itshape Subversion}\setmainfont[Path=/usr/share/fonts/truetype/cmu/,UprightFont=cmunrm.ttf,BoldFont=cmunbx.ttf,ItalicFont=cmunti.ttf,BoldItalicFont=cmunbi.ttf]{cmunrm.ttf}\setmonofont[Path=/usr/share/fonts/truetype/cmu/,UprightFont=cmuntt.ttf,BoldFont=cmuntb.ttf,ItalicFont=cmunit.ttf,BoldItalicFont=cmuntx.ttf]{cmunrm.ttf},
and Christian Henning as well as his colleagues
for supporting his intention to establish LaTeX
and {\itshape \setmainfont[Path=/usr/share/fonts/truetype/cmu/,UprightFont=cmunrm.ttf,BoldFont=cmunbx.ttf,ItalicFont=cmunti.ttf,BoldItalicFont=cmunbi.ttf]{cmunti.ttf}\setmonofont[Path=/usr/share/fonts/truetype/cmu/,UprightFont=cmuntt.ttf,BoldFont=cmuntb.ttf,ItalicFont=cmunit.ttf,BoldItalicFont=cmuntx.ttf]{cmunti.ttf}\itshape Subversion}{$\text{ }$}\setmainfont[Path=/usr/share/fonts/truetype/cmu/,UprightFont=cmunrm.ttf,BoldFont=cmunbx.ttf,ItalicFont=cmunti.ttf,BoldItalicFont=cmunbi.ttf]{cmunrm.ttf}\setmonofont[Path=/usr/share/fonts/truetype/cmu/,UprightFont=cmuntt.ttf,BoldFont=cmuntb.ttf,ItalicFont=cmunit.ttf,BoldItalicFont=cmuntx.ttf]{cmunrm.ttf} at their department.

\section{References}
\label{923}
\begin{myitemize}
\item{} Fenn, Jürgen (2006): Managing citations and your bibliography with BibTeX. The PracTEX Journal, 4. \myplainurl{http://www.tug.org/pracjourn/2006-4/fenn/.}
\end{myitemize}


\begin{myitemize}
\item{} Markey, Nicolas (2005): Tame the BeaST. The B to X of BibTeX. \myplainurl{http://www.ctan.org/tex-archive/info/bibtex/tamethebeast/ttb_en.pdf.} Version 1.3.
\end{myitemize}


\begin{myitemize}
\item{} Oren Patashnik. Designing BibTeX styles. \myplainurl{http://www.ctan.org/tex-archive/info/biblio/bibtex/contrib/doc/btxhak.pdf.}
\end{myitemize}


\begin{myitemize}
\item{} \myhref{http://mathoverflow.net/questions/3044/tools-for-collaborative-paper-writing}{Tools for collaborative paper-{}writing}
\end{myitemize}



\chapter{Export To Other Formats}

\myminitoc
\label{924}

\label{925}


Strictly speaking, LaTeX source can be used to directly generate two formats:
\begin{myitemize}
\item{}  DVI using {\ttfamily \setmainfont[Path=/usr/share/fonts/truetype/cmu/,UprightFont=cmunrm.ttf,BoldFont=cmunbx.ttf,ItalicFont=cmunti.ttf,BoldItalicFont=cmunbi.ttf]{cmuntt.ttf}\setmonofont[Path=/usr/share/fonts/truetype/cmu/,UprightFont=cmuntt.ttf,BoldFont=cmuntb.ttf,ItalicFont=cmunit.ttf,BoldItalicFont=cmuntx.ttf]{cmuntt.ttf}\ttfamily latex}\setmainfont[Path=/usr/share/fonts/truetype/cmu/,UprightFont=cmunrm.ttf,BoldFont=cmunbx.ttf,ItalicFont=cmunti.ttf,BoldItalicFont=cmunbi.ttf]{cmunrm.ttf}\setmonofont[Path=/usr/share/fonts/truetype/cmu/,UprightFont=cmuntt.ttf,BoldFont=cmuntb.ttf,ItalicFont=cmunit.ttf,BoldItalicFont=cmuntx.ttf]{cmunrm.ttf}, the first one to be supported;
\item{}  PDF using \LaTeXTT{pdflatex}, more recent.
\end{myitemize}

Using other software freely available on Internet, you can easily convert DVI and PDF to other document formats. In particular, you can obtain the PostScript version using software which is included in your LaTeX distribution. Some LaTeX IDE will give you the possibility to generate the PostScript version directly (even if it uses internally a DVI mid-{}step, e.g. LaTeX → DVI → PS). It is also possible to create PDF from DVI and vice versa. It doesn\textquotesingle{}t seem logical to create a file with two steps when you can create it straight away, but some users might need it because, as you remember from the first chapters, the format you can generate depends upon the formats of the images you want to include (EPS for DVI, PNG and JPG for PDF). Here you will find sections about different formats with description about how to get it.

Other formats can be produced, such as RTF (which can be used in Microsoft Word) and HTML. However, these documents are produced from software that parses and interprets the LaTeX files, and do not implement all the features available for the primary DVI and PDF outputs. Nonetheless, they do work, and can be crucial tools for collaboration with colleagues who do not edit documents with LaTeX.
\section{Tools installation}
\label{926}

This chapter features a lot of third-{}party tools; most of them are installed independently of your TeX distribution.

Some tools are Unix-{}specific (*BSD, GNU/Linux and Mac OS X), but it may be possible to make them work on Windows. If you have the choice, it is often easier with Unix systems for command line tools.

Some tools may already be installed. For instance, you can check if dvipng is installed and ready to use (Unix only):\\

\TemplateSpaceIndent{$\text{ }${}type$\text{ }${}dvipng}


Most of these tools are installable using your package manager or portage tree (Unix only).
\section{Preview mode}
\label{927}

This section describes how to generate a screenshot of a LaTeX page or of a specific part of the page using the LaTeX package \LaTeXTT{preview}. Screenshots are useful, for example, if you want to include a LaTeX generated formula on a presentation using you favorite slideware like Powerpoint, Keynote or LibreOffice Impress. First, start by making sure you have \LaTeXTT{preview}. See \mylref{53}{Installing Extra Packages}.

Say you want to take a screenshot of 
\begin{myquote}
\item{} \begin{equation*} \pi = \sqrt{12}\sum^\infty_{k=0} \frac{(-3)^{-k}}{2k+1}. \end{equation*}
\end{myquote}

Write this formula in the \LaTeXTT{preview} environment:

\begin{Shaded}
\begin{Highlighting}[]

\NormalTok{\textbackslash{}documentclass\{article\}}
\NormalTok{\textbackslash{}usepackage[active]\{preview\}}
\NormalTok{\textbackslash{}begin\{document\}}
\NormalTok{\textbackslash{}begin\{preview\}}
\NormalTok{\textbackslash{}[}
\NormalTok{\textbackslash{}pi = \textbackslash{}sqrt\{12\}\textbackslash{}sum^\textbackslash{}infty_\{k=0\} \textbackslash{}frac\{ (-3)^\{-k\} \}\{ 2k+1 \}}
\NormalTok{\textbackslash{}]}
\NormalTok{\textbackslash{}end\{preview\}}
\NormalTok{\textbackslash{}end\{document\}}
\end{Highlighting}
\end{Shaded}


Note the \LaTeXTT{active} option in the package declaration and the \LaTeXTT{preview} environment around the equation\textquotesingle{}s code. Without any of these two, you won\textquotesingle{}t get any output.

This package is also very useful to export specific parts to other format, or to produce graphics ({\itshape \setmainfont[Path=/usr/share/fonts/truetype/cmu/,UprightFont=cmunrm.ttf,BoldFont=cmunbx.ttf,ItalicFont=cmunti.ttf,BoldItalicFont=cmunbi.ttf]{cmunti.ttf}\setmonofont[Path=/usr/share/fonts/truetype/cmu/,UprightFont=cmuntt.ttf,BoldFont=cmuntb.ttf,ItalicFont=cmunit.ttf,BoldItalicFont=cmuntx.ttf]{cmunti.ttf}\itshape e.g.}{$\text{ }$}\setmainfont[Path=/usr/share/fonts/truetype/cmu/,UprightFont=cmunrm.ttf,BoldFont=cmunbx.ttf,ItalicFont=cmunti.ttf,BoldItalicFont=cmunbi.ttf]{cmunrm.ttf}\setmonofont[Path=/usr/share/fonts/truetype/cmu/,UprightFont=cmuntt.ttf,BoldFont=cmuntb.ttf,ItalicFont=cmunit.ttf,BoldItalicFont=cmuntx.ttf]{cmunrm.ttf} using \mylref{793}{PGF/TikZ}) and then including them in other documents. You can also automate the previewing of specific environments:

\begin{Shaded}
\begin{Highlighting}[]

\NormalTok{\textbackslash{}usepackage[active,tightpage]\{preview\}}
\NormalTok{\textbackslash{}PreviewEnvironment\{lstlisting\}}
\NormalTok{\textbackslash{}setlength\{\textbackslash{}PreviewBorder\}\{10pt\}}\CommentTok{%}
 
\CommentTok{% ...}
 
\NormalTok{\textbackslash{}begin\{lstlisting\}}
\NormalTok{int main()}
\NormalTok{\{}
        \NormalTok{/* ... */}
\NormalTok{\}}
\NormalTok{\textbackslash{}end\{lstlisting\}}
\end{Highlighting}
\end{Shaded}


This will produce a PDF containing only the listing content, the {\itshape \setmainfont[Path=/usr/share/fonts/truetype/cmu/,UprightFont=cmunrm.ttf,BoldFont=cmunbx.ttf,ItalicFont=cmunti.ttf,BoldItalicFont=cmunbi.ttf]{cmunti.ttf}\setmonofont[Path=/usr/share/fonts/truetype/cmu/,UprightFont=cmuntt.ttf,BoldFont=cmuntb.ttf,ItalicFont=cmunit.ttf,BoldItalicFont=cmuntx.ttf]{cmunti.ttf}\itshape page}{$\text{ }$}\setmainfont[Path=/usr/share/fonts/truetype/cmu/,UprightFont=cmunrm.ttf,BoldFont=cmunbx.ttf,ItalicFont=cmunti.ttf,BoldItalicFont=cmunbi.ttf]{cmunrm.ttf}\setmonofont[Path=/usr/share/fonts/truetype/cmu/,UprightFont=cmuntt.ttf,BoldFont=cmuntb.ttf,ItalicFont=cmunit.ttf,BoldItalicFont=cmuntx.ttf]{cmunrm.ttf} layout will depend on the shape of the source code.
\section{Convert to PDF}
\label{928}\subsection{Directly}
\label{929}\\

\TemplateSpaceIndent{$\text{ }${}pdflatex$\text{ }${}my\_file}

\subsection{DVI to PDF}
\label{930}
\TemplatePreformat{$\text{ }$\newline{}
dvipdfm$\text{ }${}my\_file.dvi$\text{ }$\newline{}
}
will create {\ttfamily \setmainfont[Path=/usr/share/fonts/truetype/cmu/,UprightFont=cmunrm.ttf,BoldFont=cmunbx.ttf,ItalicFont=cmunti.ttf,BoldItalicFont=cmunbi.ttf]{cmuntt.ttf}\setmonofont[Path=/usr/share/fonts/truetype/cmu/,UprightFont=cmuntt.ttf,BoldFont=cmuntb.ttf,ItalicFont=cmunit.ttf,BoldItalicFont=cmuntx.ttf]{cmuntt.ttf}\ttfamily my\_file.pdf}\setmainfont[Path=/usr/share/fonts/truetype/cmu/,UprightFont=cmunrm.ttf,BoldFont=cmunbx.ttf,ItalicFont=cmunti.ttf,BoldItalicFont=cmunbi.ttf]{cmunrm.ttf}\setmonofont[Path=/usr/share/fonts/truetype/cmu/,UprightFont=cmuntt.ttf,BoldFont=cmuntb.ttf,ItalicFont=cmunit.ttf,BoldItalicFont=cmuntx.ttf]{cmunrm.ttf}. Another way is to pass through PS generation:
\TemplatePreformat{$\text{ }$\newline{}
dvi2ps$\text{ }${}myfile.dvi$\text{ }$\newline{}
ps2pdf$\text{ }${}myfile.ps$\text{ }$\newline{}
}
you will get also a file called {\itshape \setmainfont[Path=/usr/share/fonts/truetype/cmu/,UprightFont=cmunrm.ttf,BoldFont=cmunbx.ttf,ItalicFont=cmunti.ttf,BoldItalicFont=cmunbi.ttf]{cmunti.ttf}\setmonofont[Path=/usr/share/fonts/truetype/cmu/,UprightFont=cmuntt.ttf,BoldFont=cmuntb.ttf,ItalicFont=cmunit.ttf,BoldItalicFont=cmuntx.ttf]{cmunti.ttf}\itshape my\_file.ps}{$\text{ }$}\setmainfont[Path=/usr/share/fonts/truetype/cmu/,UprightFont=cmunrm.ttf,BoldFont=cmunbx.ttf,ItalicFont=cmunti.ttf,BoldItalicFont=cmunbi.ttf]{cmunrm.ttf}\setmonofont[Path=/usr/share/fonts/truetype/cmu/,UprightFont=cmuntt.ttf,BoldFont=cmuntb.ttf,ItalicFont=cmunit.ttf,BoldItalicFont=cmuntx.ttf]{cmunrm.ttf} that you can delete.
\subsection{Merging PDF}
\label{931}

If you have created different PDF documents and you want to merge them into one single PDF file you can use the following command-{}line command. You need to have Ghostscript installed:
\subsubsection{Using Windows}
\label{932}
\\

\TemplateSpaceIndent{$\text{ }${}gswin32$\text{ }${}-{}dNOPAUSE$\text{ }${}-{}sDEVICE=pdfwrite$\text{ }${}-{}sOUTPUTFILE=Merged.pdf$\text{ }${}-{}dBATCH$\text{ }${}1.pdf$\text{ }${}2.pdf$\text{ }$\newline{}
$\text{ }${}3.pdf}

\subsubsection{Using Linux}
\label{933}
\\

\TemplateSpaceIndent{$\text{ }${}gs$\text{ }${}-{}dNOPAUSE$\text{ }${}-{}sDEVICE=pdfwrite$\text{ }${}-{}sOUTPUTFILE=Merged.pdf$\text{ }${}-{}dBATCH$\text{ }${}1.pdf$\text{ }${}2.pdf$\text{ }$\newline{}
$\text{ }${}3.pdf}


Alternatively, \myhref{http://pdfshuffler.sourceforge.net/}{PDF-{}Shuffler} is a small python-{}gtk application, which helps the user to merge or split pdf documents and rotate, crop and rearrange their pages using an interactive and intuitive graphical interface. This program may be avaliable in your Linux distribution\textquotesingle{}s repository.

Another option to check out is \myhref{http://www.accesspdf.com/}{pdftk} (or PDF toolkit), which is a command-{}line tool that can manipulate PDFs in many ways. To merge one or more files, use:
\\

\TemplateSpaceIndent{$\text{ }${}pdftk$\text{ }${}1.pdf$\text{ }${}2.pdf$\text{ }${}3.pdf$\text{ }${}cat$\text{ }${}output$\text{ }${}123.pdf}

\subsubsection{Using pdfLaTeX}
\label{934}

{\itshape \setmainfont[Path=/usr/share/fonts/truetype/cmu/,UprightFont=cmunrm.ttf,BoldFont=cmunbx.ttf,ItalicFont=cmunti.ttf,BoldItalicFont=cmunbi.ttf]{cmunti.ttf}\setmonofont[Path=/usr/share/fonts/truetype/cmu/,UprightFont=cmuntt.ttf,BoldFont=cmuntb.ttf,ItalicFont=cmunit.ttf,BoldItalicFont=cmuntx.ttf]{cmunti.ttf}\itshape Note:}{$\text{ }$}\setmainfont[Path=/usr/share/fonts/truetype/cmu/,UprightFont=cmunrm.ttf,BoldFont=cmunbx.ttf,ItalicFont=cmunti.ttf,BoldItalicFont=cmunbi.ttf]{cmunrm.ttf}\setmonofont[Path=/usr/share/fonts/truetype/cmu/,UprightFont=cmuntt.ttf,BoldFont=cmuntb.ttf,ItalicFont=cmunit.ttf,BoldItalicFont=cmuntx.ttf]{cmunrm.ttf} If you are merging external PDF documents into a LaTeX document which is compiled with {\ttfamily \setmainfont[Path=/usr/share/fonts/truetype/cmu/,UprightFont=cmunrm.ttf,BoldFont=cmunbx.ttf,ItalicFont=cmunti.ttf,BoldItalicFont=cmunbi.ttf]{cmuntt.ttf}\setmonofont[Path=/usr/share/fonts/truetype/cmu/,UprightFont=cmuntt.ttf,BoldFont=cmuntb.ttf,ItalicFont=cmunit.ttf,BoldItalicFont=cmuntx.ttf]{cmuntt.ttf}\ttfamily pdflatex}\setmainfont[Path=/usr/share/fonts/truetype/cmu/,UprightFont=cmunrm.ttf,BoldFont=cmunbx.ttf,ItalicFont=cmunti.ttf,BoldItalicFont=cmunbi.ttf]{cmunrm.ttf}\setmonofont[Path=/usr/share/fonts/truetype/cmu/,UprightFont=cmuntt.ttf,BoldFont=cmuntb.ttf,ItalicFont=cmunit.ttf,BoldItalicFont=cmuntx.ttf]{cmunrm.ttf}, a much simpler option is to use the \LaTeXTT{pdfpages} package, {\itshape \setmainfont[Path=/usr/share/fonts/truetype/cmu/,UprightFont=cmunrm.ttf,BoldFont=cmunbx.ttf,ItalicFont=cmunti.ttf,BoldItalicFont=cmunbi.ttf]{cmunti.ttf}\setmonofont[Path=/usr/share/fonts/truetype/cmu/,UprightFont=cmuntt.ttf,BoldFont=cmuntb.ttf,ItalicFont=cmunit.ttf,BoldItalicFont=cmuntx.ttf]{cmunti.ttf}\itshape e.g.}\setmainfont[Path=/usr/share/fonts/truetype/cmu/,UprightFont=cmunrm.ttf,BoldFont=cmunbx.ttf,ItalicFont=cmunti.ttf,BoldItalicFont=cmunbi.ttf]{cmunrm.ttf}\setmonofont[Path=/usr/share/fonts/truetype/cmu/,UprightFont=cmuntt.ttf,BoldFont=cmuntb.ttf,ItalicFont=cmunit.ttf,BoldItalicFont=cmuntx.ttf]{cmunrm.ttf}:

\begin{Shaded}
\begin{Highlighting}[]

\NormalTok{\textbackslash{}usepackage\{pdfpages\}}
\NormalTok{...}
\NormalTok{\textbackslash{}includepdf[pages=-]\{Document1.pdf\}}
\NormalTok{\textbackslash{}includepdf[pages=-]\{Document2.pdf\}}
\NormalTok{...}
\end{Highlighting}
\end{Shaded}


Three simple \myhref{https://en.wikipedia.org/wiki/Shell\%20\%28computing\%29}{shell} scripts using the \LaTeXTT{pdfpages} package are provided in the \myhref{http://www2.warwick.ac.uk/fac/sci/statistics/staff/academic/firth/software/pdfjam}{pdfjam bundle} by D. Firth. They include options to merge several pdf files (pdfjoin), put several pages in one physical sheet (pdfnup) and rotate pages (pdf90).

See also \mylref{906}{Modular Documents}
\subsection{XeTeX}
\label{935}
You can also use XeTeX (or, more precisely, XeLaTeX), which works in the same way as {\ttfamily \setmainfont[Path=/usr/share/fonts/truetype/cmu/,UprightFont=cmunrm.ttf,BoldFont=cmunbx.ttf,ItalicFont=cmunti.ttf,BoldItalicFont=cmunbi.ttf]{cmuntt.ttf}\setmonofont[Path=/usr/share/fonts/truetype/cmu/,UprightFont=cmuntt.ttf,BoldFont=cmuntb.ttf,ItalicFont=cmunit.ttf,BoldItalicFont=cmuntx.ttf]{cmuntt.ttf}\ttfamily pdflatex}\setmainfont[Path=/usr/share/fonts/truetype/cmu/,UprightFont=cmunrm.ttf,BoldFont=cmunbx.ttf,ItalicFont=cmunti.ttf,BoldItalicFont=cmunbi.ttf]{cmunrm.ttf}\setmonofont[Path=/usr/share/fonts/truetype/cmu/,UprightFont=cmuntt.ttf,BoldFont=cmuntb.ttf,ItalicFont=cmunit.ttf,BoldItalicFont=cmuntx.ttf]{cmunrm.ttf}: it creates a PDF file directly from LaTeX source. One advantage of XeTeX over standard LaTeX is support for Unicode and modern typography. See \myhref{https://en.wikipedia.org/wiki/XeTeX}{its Wikipedia entry} for more details.

Customization of PDF output in XeTeX (setting document title, author, keywords etc.) is done using the configuration of \mylref{403}{hyperref} package.
\section{Convert to PostScript}
\label{936}
{\bfseries
\begin{mydescription}from PDF
\end{mydescription}
}

\TemplatePreformat{$\text{ }$\newline{}
pdf2ps$\text{ }${}my\_file.pdf$\text{ }$\newline{}
}
{\bfseries
\begin{mydescription}from DVI
\end{mydescription}
}

\TemplatePreformat{$\text{ }$\newline{}
dvi2ps$\text{ }${}my\_file.dvi$\text{ }$\newline{}
}
\section{Convert to RTF}
\label{937}

LaTeX can be converted into an RTF file, which in turn can be opened by a word processor such as \myhref{https://en.wikipedia.org/wiki/LibreOffice\%20Writer}{LibreOffice Writer} or \myhref{https://en.wikipedia.org/wiki/Microsoft\%20Word}{Microsoft Word}. This conversion is done through \myhref{http://latex2rtf.sourceforge.net/}{latex2rtf}, which may run on any computer platform, however is only actively supported on Windows, Linux and BSD, with the last mac update being from 2001. The program operates by reading the LaTeX source, and mimicking the behaviour of the LaTeX program. {\ttfamily \setmainfont[Path=/usr/share/fonts/truetype/cmu/,UprightFont=cmunrm.ttf,BoldFont=cmunbx.ttf,ItalicFont=cmunti.ttf,BoldItalicFont=cmunbi.ttf]{cmuntt.ttf}\setmonofont[Path=/usr/share/fonts/truetype/cmu/,UprightFont=cmuntt.ttf,BoldFont=cmuntb.ttf,ItalicFont=cmunit.ttf,BoldItalicFont=cmuntx.ttf]{cmuntt.ttf}\ttfamily latex2rtf}{$\text{ }$}\setmainfont[Path=/usr/share/fonts/truetype/cmu/,UprightFont=cmunrm.ttf,BoldFont=cmunbx.ttf,ItalicFont=cmunti.ttf,BoldItalicFont=cmunbi.ttf]{cmunrm.ttf}\setmonofont[Path=/usr/share/fonts/truetype/cmu/,UprightFont=cmuntt.ttf,BoldFont=cmuntb.ttf,ItalicFont=cmunit.ttf,BoldItalicFont=cmuntx.ttf]{cmunrm.ttf} supports most of the standard implementations of LaTeX, such as standard formatting, some math typesetting, inclusion of EPS, PNG or JPG graphics, and tables. As well, it has some limited support for packages, such as varioref, and natbib. However, many other packages are not supported.

{\ttfamily \setmainfont[Path=/usr/share/fonts/truetype/cmu/,UprightFont=cmunrm.ttf,BoldFont=cmunbx.ttf,ItalicFont=cmunti.ttf,BoldItalicFont=cmunbi.ttf]{cmuntt.ttf}\setmonofont[Path=/usr/share/fonts/truetype/cmu/,UprightFont=cmuntt.ttf,BoldFont=cmuntb.ttf,ItalicFont=cmunit.ttf,BoldItalicFont=cmuntx.ttf]{cmuntt.ttf}\ttfamily latex2rtf}{$\text{ }$}\setmainfont[Path=/usr/share/fonts/truetype/cmu/,UprightFont=cmunrm.ttf,BoldFont=cmunbx.ttf,ItalicFont=cmunti.ttf,BoldItalicFont=cmunbi.ttf]{cmunrm.ttf}\setmonofont[Path=/usr/share/fonts/truetype/cmu/,UprightFont=cmuntt.ttf,BoldFont=cmuntb.ttf,ItalicFont=cmunit.ttf,BoldItalicFont=cmuntx.ttf]{cmunrm.ttf} is simple to use. The Windows version has a GUI ({\ttfamily \setmainfont[Path=/usr/share/fonts/truetype/cmu/,UprightFont=cmunrm.ttf,BoldFont=cmunbx.ttf,ItalicFont=cmunti.ttf,BoldItalicFont=cmunbi.ttf]{cmuntt.ttf}\setmonofont[Path=/usr/share/fonts/truetype/cmu/,UprightFont=cmuntt.ttf,BoldFont=cmuntb.ttf,ItalicFont=cmunit.ttf,BoldItalicFont=cmuntx.ttf]{cmuntt.ttf}\ttfamily l2rshell.exe}\setmainfont[Path=/usr/share/fonts/truetype/cmu/,UprightFont=cmunrm.ttf,BoldFont=cmunbx.ttf,ItalicFont=cmunti.ttf,BoldItalicFont=cmunbi.ttf]{cmunrm.ttf}\setmonofont[Path=/usr/share/fonts/truetype/cmu/,UprightFont=cmuntt.ttf,BoldFont=cmuntb.ttf,ItalicFont=cmunit.ttf,BoldItalicFont=cmuntx.ttf]{cmunrm.ttf}), which is straightforward to use. The command-{}line version is offered for all platforms, and can be used on an example {\ttfamily \setmainfont[Path=/usr/share/fonts/truetype/cmu/,UprightFont=cmunrm.ttf,BoldFont=cmunbx.ttf,ItalicFont=cmunti.ttf,BoldItalicFont=cmunbi.ttf]{cmuntt.ttf}\setmonofont[Path=/usr/share/fonts/truetype/cmu/,UprightFont=cmuntt.ttf,BoldFont=cmuntb.ttf,ItalicFont=cmunit.ttf,BoldItalicFont=cmuntx.ttf]{cmuntt.ttf}\ttfamily mypaper.tex}{$\text{ }$}\setmainfont[Path=/usr/share/fonts/truetype/cmu/,UprightFont=cmunrm.ttf,BoldFont=cmunbx.ttf,ItalicFont=cmunti.ttf,BoldItalicFont=cmunbi.ttf]{cmunrm.ttf}\setmonofont[Path=/usr/share/fonts/truetype/cmu/,UprightFont=cmuntt.ttf,BoldFont=cmuntb.ttf,ItalicFont=cmunit.ttf,BoldItalicFont=cmuntx.ttf]{cmunrm.ttf} file:
\\

\TemplateSpaceIndent{$\text{ }${}latex$\text{ }${}mypaper$\text{ }$\newline{}
$\text{ }${}bibtex$\text{ }${}mypaper$\text{ }${}\#$\text{ }${}if$\text{ }${}you$\text{ }${}use$\text{ }${}bibtex$\text{ }$\newline{}
$\text{ }${}latex2rtf$\text{ }${}mypaper}


Both {\ttfamily \setmainfont[Path=/usr/share/fonts/truetype/cmu/,UprightFont=cmunrm.ttf,BoldFont=cmunbx.ttf,ItalicFont=cmunti.ttf,BoldItalicFont=cmunbi.ttf]{cmuntt.ttf}\setmonofont[Path=/usr/share/fonts/truetype/cmu/,UprightFont=cmuntt.ttf,BoldFont=cmuntb.ttf,ItalicFont=cmunit.ttf,BoldItalicFont=cmuntx.ttf]{cmuntt.ttf}\ttfamily latex}{$\text{ }$}\setmainfont[Path=/usr/share/fonts/truetype/cmu/,UprightFont=cmunrm.ttf,BoldFont=cmunbx.ttf,ItalicFont=cmunti.ttf,BoldItalicFont=cmunbi.ttf]{cmunrm.ttf}\setmonofont[Path=/usr/share/fonts/truetype/cmu/,UprightFont=cmuntt.ttf,BoldFont=cmuntb.ttf,ItalicFont=cmunit.ttf,BoldItalicFont=cmuntx.ttf]{cmunrm.ttf} and (if needed) {\ttfamily \setmainfont[Path=/usr/share/fonts/truetype/cmu/,UprightFont=cmunrm.ttf,BoldFont=cmunbx.ttf,ItalicFont=cmunti.ttf,BoldItalicFont=cmunbi.ttf]{cmuntt.ttf}\setmonofont[Path=/usr/share/fonts/truetype/cmu/,UprightFont=cmuntt.ttf,BoldFont=cmuntb.ttf,ItalicFont=cmunit.ttf,BoldItalicFont=cmuntx.ttf]{cmuntt.ttf}\ttfamily bibtex}{$\text{ }$}\setmainfont[Path=/usr/share/fonts/truetype/cmu/,UprightFont=cmunrm.ttf,BoldFont=cmunbx.ttf,ItalicFont=cmunti.ttf,BoldItalicFont=cmunbi.ttf]{cmunrm.ttf}\setmonofont[Path=/usr/share/fonts/truetype/cmu/,UprightFont=cmuntt.ttf,BoldFont=cmuntb.ttf,ItalicFont=cmunit.ttf,BoldItalicFont=cmuntx.ttf]{cmunrm.ttf} commands need to be run {\itshape \setmainfont[Path=/usr/share/fonts/truetype/cmu/,UprightFont=cmunrm.ttf,BoldFont=cmunbx.ttf,ItalicFont=cmunti.ttf,BoldItalicFont=cmunbi.ttf]{cmunti.ttf}\setmonofont[Path=/usr/share/fonts/truetype/cmu/,UprightFont=cmuntt.ttf,BoldFont=cmuntb.ttf,ItalicFont=cmunit.ttf,BoldItalicFont=cmuntx.ttf]{cmunti.ttf}\itshape before}{$\text{ }$}\setmainfont[Path=/usr/share/fonts/truetype/cmu/,UprightFont=cmunrm.ttf,BoldFont=cmunbx.ttf,ItalicFont=cmunti.ttf,BoldItalicFont=cmunbi.ttf]{cmunrm.ttf}\setmonofont[Path=/usr/share/fonts/truetype/cmu/,UprightFont=cmuntt.ttf,BoldFont=cmuntb.ttf,ItalicFont=cmunit.ttf,BoldItalicFont=cmuntx.ttf]{cmunrm.ttf} {\ttfamily \setmainfont[Path=/usr/share/fonts/truetype/cmu/,UprightFont=cmunrm.ttf,BoldFont=cmunbx.ttf,ItalicFont=cmunti.ttf,BoldItalicFont=cmunbi.ttf]{cmuntt.ttf}\setmonofont[Path=/usr/share/fonts/truetype/cmu/,UprightFont=cmuntt.ttf,BoldFont=cmuntb.ttf,ItalicFont=cmunit.ttf,BoldItalicFont=cmuntx.ttf]{cmuntt.ttf}\ttfamily latex2rtf}\setmainfont[Path=/usr/share/fonts/truetype/cmu/,UprightFont=cmunrm.ttf,BoldFont=cmunbx.ttf,ItalicFont=cmunti.ttf,BoldItalicFont=cmunbi.ttf]{cmunrm.ttf}\setmonofont[Path=/usr/share/fonts/truetype/cmu/,UprightFont=cmuntt.ttf,BoldFont=cmuntb.ttf,ItalicFont=cmunit.ttf,BoldItalicFont=cmuntx.ttf]{cmunrm.ttf}, because the {\ttfamily \setmainfont[Path=/usr/share/fonts/truetype/cmu/,UprightFont=cmunrm.ttf,BoldFont=cmunbx.ttf,ItalicFont=cmunti.ttf,BoldItalicFont=cmunbi.ttf]{cmuntt.ttf}\setmonofont[Path=/usr/share/fonts/truetype/cmu/,UprightFont=cmuntt.ttf,BoldFont=cmuntb.ttf,ItalicFont=cmunit.ttf,BoldItalicFont=cmuntx.ttf]{cmuntt.ttf}\ttfamily .aux}{$\text{ }$}\setmainfont[Path=/usr/share/fonts/truetype/cmu/,UprightFont=cmunrm.ttf,BoldFont=cmunbx.ttf,ItalicFont=cmunti.ttf,BoldItalicFont=cmunbi.ttf]{cmunrm.ttf}\setmonofont[Path=/usr/share/fonts/truetype/cmu/,UprightFont=cmuntt.ttf,BoldFont=cmuntb.ttf,ItalicFont=cmunit.ttf,BoldItalicFont=cmuntx.ttf]{cmunrm.ttf} and {\ttfamily \setmainfont[Path=/usr/share/fonts/truetype/cmu/,UprightFont=cmunrm.ttf,BoldFont=cmunbx.ttf,ItalicFont=cmunti.ttf,BoldItalicFont=cmunbi.ttf]{cmuntt.ttf}\setmonofont[Path=/usr/share/fonts/truetype/cmu/,UprightFont=cmuntt.ttf,BoldFont=cmuntb.ttf,ItalicFont=cmunit.ttf,BoldItalicFont=cmuntx.ttf]{cmuntt.ttf}\ttfamily .bbl}{$\text{ }$}\setmainfont[Path=/usr/share/fonts/truetype/cmu/,UprightFont=cmunrm.ttf,BoldFont=cmunbx.ttf,ItalicFont=cmunti.ttf,BoldItalicFont=cmunbi.ttf]{cmunrm.ttf}\setmonofont[Path=/usr/share/fonts/truetype/cmu/,UprightFont=cmuntt.ttf,BoldFont=cmuntb.ttf,ItalicFont=cmunit.ttf,BoldItalicFont=cmuntx.ttf]{cmunrm.ttf} files are needed to produce the proper output. The result of this conversion will create {\ttfamily \setmainfont[Path=/usr/share/fonts/truetype/cmu/,UprightFont=cmunrm.ttf,BoldFont=cmunbx.ttf,ItalicFont=cmunti.ttf,BoldItalicFont=cmunbi.ttf]{cmuntt.ttf}\setmonofont[Path=/usr/share/fonts/truetype/cmu/,UprightFont=cmuntt.ttf,BoldFont=cmuntb.ttf,ItalicFont=cmunit.ttf,BoldItalicFont=cmuntx.ttf]{cmuntt.ttf}\ttfamily myfile.rtf}\setmainfont[Path=/usr/share/fonts/truetype/cmu/,UprightFont=cmunrm.ttf,BoldFont=cmunbx.ttf,ItalicFont=cmunti.ttf,BoldItalicFont=cmunbi.ttf]{cmunrm.ttf}\setmonofont[Path=/usr/share/fonts/truetype/cmu/,UprightFont=cmuntt.ttf,BoldFont=cmuntb.ttf,ItalicFont=cmunit.ttf,BoldItalicFont=cmuntx.ttf]{cmunrm.ttf}, which you may open in many word processors such as Microsoft Word or LibreOffice.
\section{Convert to HTML}
\label{938}

There are many converters to HTML.{\bfseries
\begin{mydescription}\myhref{http://hevea.inria.fr}{HEVEA}
\end{mydescription}
}
\\

\TemplateSpaceIndent{$\text{ }${}hevea$\text{ }${}mylatexfile}

{\bfseries
\begin{mydescription}latex2html
\end{mydescription}
}
\\

\TemplateSpaceIndent{$\text{ }${}latex2html$\text{ }${}-{}html\_version$\text{ }${}4.0,latin1,unicode$\text{ }${}-{}split$\text{ }${}1$\text{ }${}-{}nonavigation$\text{ }${}-{}noinfo$\text{ }$\newline{}
$\text{ }${}-{}title$\text{ }${}\symbol{34}MyDocument\symbol{34}$\text{ }${}MyDocument.tex}

{\bfseries
\begin{mydescription}\myhref{https://en.wikipedia.org/wiki/LaTeXML}{LaTeXML}
\end{mydescription}
}
\\

\TemplateSpaceIndent{$\text{ }${}latexmlc$\text{ }${}paper.tex$\text{ }${}-{}-{}destination=paper.html}

{\bfseries
\begin{mydescription}\myhref{https://github.com/coolwanglu/pdf2htmlEX}{pdf2htmlEX}
\end{mydescription}
}
\\

\TemplateSpaceIndent{$\text{ }${}pdf2htmlEX$\text{ }${}{$\text{[}$}options{$\text{]}$}$\text{ }${}<{}input.pdf>{}$\text{ }${}{$\text{[}$}<{}output.html>{}{$\text{]}$}}

pdf2htmlEX can convert PDF to HTML without losing text or format. It is designed as a general PDF to HTML converter, not only restricted to the PDF generated by LaTeX source. LaTeX users can compile the LaTeX source code to PDF, and then convert the PDF to HTML via pdf2htmlEX. Some introductions of pdf2htmlEX can be found on its own \myhref{https://github.com/coolwanglu/pdf2htmlEX/wiki}{wiki page}. More technical details can be found on the paper published on TUGboat: {\itshape \setmainfont[Path=/usr/share/fonts/truetype/cmu/,UprightFont=cmunrm.ttf,BoldFont=cmunbx.ttf,ItalicFont=cmunti.ttf,BoldItalicFont=cmunbi.ttf]{cmunti.ttf}\setmonofont[Path=/usr/share/fonts/truetype/cmu/,UprightFont=cmuntt.ttf,BoldFont=cmuntb.ttf,ItalicFont=cmunit.ttf,BoldItalicFont=cmuntx.ttf]{cmunti.ttf}\itshape Online publishing via pdf2htmlEX}{$\text{ }$}\setmainfont[Path=/usr/share/fonts/truetype/cmu/,UprightFont=cmunrm.ttf,BoldFont=cmunbx.ttf,ItalicFont=cmunti.ttf,BoldItalicFont=cmunbi.ttf]{cmunrm.ttf}\setmonofont[Path=/usr/share/fonts/truetype/cmu/,UprightFont=cmuntt.ttf,BoldFont=cmuntb.ttf,ItalicFont=cmunit.ttf,BoldItalicFont=cmuntx.ttf]{cmunrm.ttf} \myhref{http://coolwanglu.github.io/pdf2htmlEX/doc/tb108wang.html}{HTML} / \myhref{http://coolwanglu.github.io/pdf2htmlEX/doc/tb108wang.pdf}{PDF}. The Figure 3 of the paper gives different work-{}flows of publishing HTML online.
{\bfseries
\begin{mydescription}TeX4ht
\end{mydescription}
}

\myhref{http://www.cse.ohio-state.edu/~gurari/TeX4ht/}{TeX4ht} is a very powerful conversion program, but its configuration is not straightforward. Basically a configuration file has to be prepared, and then the program is called.
{\bfseries
\begin{mydescription}bibtex2html
\end{mydescription}
}

For BibTeX.\\

\TemplateSpaceIndent{$\text{ }${}bibtex2html$\text{ }${}mybibtexfile}

\section{Convert to image formats}
\label{939}

It is sometimes useful to convert LaTeX output to image formats for use in systems that do not support DVI nor PDF files, such as Wikipedia.

There are two families of graphics:
\begin{myitemize}
\item{}  Vector graphics can be scaled to any size, thus do not suffer from quality loss. \myhref{https://en.wikipedia.org/wiki/Scalable\%20Vector\%20Graphics}{SVG} is a vector format.
\item{}  Raster graphics define every pixel explicitly. \myhref{https://en.wikipedia.org/wiki/PNG}{PNG} is a raster format.
\end{myitemize}


So vector graphics are usually preferred. There is still some cases where raster graphics are used:
\begin{myitemize}
\item{}  The target system does not handle vector graphics, only raster graphics are supported.
\item{}  SVG can not embed fonts. So either the font will be rendered using a local .ttf or .otf font (which will mostly change the output), or  all characters must be turned to vector graphics. This last method makes the SVG big and slow. If the input LaTeX file contains a lot of text which formatting must be preserved, SVG is not that great.
\end{myitemize}


So SVG is great for drawings and a small amount of text.
JPG is a well known raster formats, however it is usually not as good as PNG for text.

In some cases it may be sufficient to simply copy a region of a PDF (or PS) file using the tools available in a PDF viewer (for example using LaTeX to typeset a formula for pasting into a presentation).  This however will not generally have sufficient resolution for whole pages or large areas. 
\subsection{Multiple formats}
\label{940}
{\bfseries
\begin{mydescription}pdftocairo
\end{mydescription}
}

There is {\ttfamily \setmainfont[Path=/usr/share/fonts/truetype/cmu/,UprightFont=cmunrm.ttf,BoldFont=cmunbx.ttf,ItalicFont=cmunti.ttf,BoldItalicFont=cmunbi.ttf]{cmuntt.ttf}\setmonofont[Path=/usr/share/fonts/truetype/cmu/,UprightFont=cmuntt.ttf,BoldFont=cmuntb.ttf,ItalicFont=cmunit.ttf,BoldItalicFont=cmuntx.ttf]{cmuntt.ttf}\ttfamily pdftocairo}{$\text{ }$}\setmainfont[Path=/usr/share/fonts/truetype/cmu/,UprightFont=cmunrm.ttf,BoldFont=cmunbx.ttf,ItalicFont=cmunti.ttf,BoldItalicFont=cmunbi.ttf]{cmunrm.ttf}\setmonofont[Path=/usr/share/fonts/truetype/cmu/,UprightFont=cmuntt.ttf,BoldFont=cmuntb.ttf,ItalicFont=cmunit.ttf,BoldItalicFont=cmuntx.ttf]{cmunrm.ttf} featured in the poppler toolset.\\

\TemplateSpaceIndent{$\text{ }${}pdftocairo$\text{ }${}-{}svg$\text{ }${}latexdoc.pdf$\text{ }${}output.svg}

{\ttfamily \setmainfont[Path=/usr/share/fonts/truetype/cmu/,UprightFont=cmunrm.ttf,BoldFont=cmunbx.ttf,ItalicFont=cmunti.ttf,BoldItalicFont=cmunbi.ttf]{cmuntt.ttf}\setmonofont[Path=/usr/share/fonts/truetype/cmu/,UprightFont=cmuntt.ttf,BoldFont=cmuntb.ttf,ItalicFont=cmunit.ttf,BoldItalicFont=cmuntx.ttf]{cmuntt.ttf}\ttfamily pdftocairo}{$\text{ }$}\setmainfont[Path=/usr/share/fonts/truetype/cmu/,UprightFont=cmunrm.ttf,BoldFont=cmunbx.ttf,ItalicFont=cmunti.ttf,BoldItalicFont=cmunbi.ttf]{cmunrm.ttf}\setmonofont[Path=/usr/share/fonts/truetype/cmu/,UprightFont=cmuntt.ttf,BoldFont=cmuntb.ttf,ItalicFont=cmunit.ttf,BoldItalicFont=cmuntx.ttf]{cmunrm.ttf} also supports various raster graphic formats.
\subsection{Vector graphics}
\label{941}
{\bfseries
\begin{mydescription}pdf2svg
\end{mydescription}
}

Direct conversion from PDF to SVG can be done using the command line tool \myhref{http://www.cityinthesky.co.uk/opensource/pdf2svg/}{pdf2svg}.\\

\TemplateSpaceIndent{$\text{ }${}pdf2svg$\text{ }${}file.pdf$\text{ }${}file.svg}

{\bfseries
\begin{mydescription}ps2svg
\end{mydescription}
}

Alternatively DVI or PDF can be converted to PS as described before, then the bash script \myhref{http://en.wikipedia.org/wiki/Wikipedia:WikiProject_Electronics/Ps2svg.sh}{ps2svg.sh} can be used (as all the software used by this script is multiplatform, this is also possible in Windows, a step-{}by-{}step guide could be written).
{\bfseries
\begin{mydescription}dvisvgm
\end{mydescription}
}

One can also use \myhref{http://dvisvgm.sourceforge.net/}{dvisvgm}, an open source utility that converts from DVI to SVG.\\

\TemplateSpaceIndent{$\text{ }${}dvisvgm$\text{ }${}-{}n$\text{ }${}file.dvi}

{\bfseries
\begin{mydescription}Inkscape
\end{mydescription}
}

Inkscape is able to convert to SVG, PDF, EPS, and other vector graphic formats.\\

\TemplateSpaceIndent{$\text{ }${}inkscape$\text{ }${}-{}-{}export-{}area-{}drawing$\text{ }${}-{}-{}export-{}ps=OUTPUT$\text{ }${}INPUT$\text{ }$\newline{}
$\text{ }${}inkscape$\text{ }${}-{}-{}export-{}area-{}page$\text{ }${}-{}-{}export-{}plain-{}svg=OUTPUT$\text{ }${}INPUT}

\subsection{Raster graphics}
\label{942}
{\bfseries
\begin{mydescription}JPEG
\end{mydescription}
}

Run {\ttfamily \setmainfont[Path=/usr/share/fonts/truetype/cmu/,UprightFont=cmunrm.ttf,BoldFont=cmunbx.ttf,ItalicFont=cmunti.ttf,BoldItalicFont=cmunbi.ttf]{cmuntt.ttf}\setmonofont[Path=/usr/share/fonts/truetype/cmu/,UprightFont=cmuntt.ttf,BoldFont=cmuntb.ttf,ItalicFont=cmunit.ttf,BoldItalicFont=cmuntx.ttf]{cmuntt.ttf}\ttfamily ghostscript}{$\text{ }$}\setmainfont[Path=/usr/share/fonts/truetype/cmu/,UprightFont=cmunrm.ttf,BoldFont=cmunbx.ttf,ItalicFont=cmunti.ttf,BoldItalicFont=cmunbi.ttf]{cmunrm.ttf}\setmonofont[Path=/usr/share/fonts/truetype/cmu/,UprightFont=cmuntt.ttf,BoldFont=cmuntb.ttf,ItalicFont=cmunit.ttf,BoldItalicFont=cmuntx.ttf]{cmunrm.ttf} on the PostScript file created by {\ttfamily \setmainfont[Path=/usr/share/fonts/truetype/cmu/,UprightFont=cmunrm.ttf,BoldFont=cmunbx.ttf,ItalicFont=cmunti.ttf,BoldItalicFont=cmunbi.ttf]{cmuntt.ttf}\setmonofont[Path=/usr/share/fonts/truetype/cmu/,UprightFont=cmuntt.ttf,BoldFont=cmuntb.ttf,ItalicFont=cmunit.ttf,BoldItalicFont=cmuntx.ttf]{cmuntt.ttf}\ttfamily pdf2ps}{$\text{ }$}\setmainfont[Path=/usr/share/fonts/truetype/cmu/,UprightFont=cmunrm.ttf,BoldFont=cmunbx.ttf,ItalicFont=cmunti.ttf,BoldItalicFont=cmunbi.ttf]{cmunrm.ttf}\setmonofont[Path=/usr/share/fonts/truetype/cmu/,UprightFont=cmuntt.ttf,BoldFont=cmuntb.ttf,ItalicFont=cmunit.ttf,BoldItalicFont=cmuntx.ttf]{cmunrm.ttf} as follows:
\\

\TemplateSpaceIndent{$\text{ }${}echo$\text{ }${}\symbol{34}quit\symbol{34}$\text{ }${}|$\text{ }${}gs$\text{ }${}-{}sDEVICE=jpeg$\text{ }${}-{}sOutputFile=document.jpg$\text{ }${}-{}r300$\text{ }${}document.ps}

{\bfseries
\begin{mydescription}GIMP
\end{mydescription}
}

Open your file with \myhref{https://en.wikibooks.org/wiki/GIMP}{GIMP}. It will ask you which page you want to convert, whether you want to use anti-{}aliasing (choose {\itshape \setmainfont[Path=/usr/share/fonts/truetype/cmu/,UprightFont=cmunrm.ttf,BoldFont=cmunbx.ttf,ItalicFont=cmunti.ttf,BoldItalicFont=cmunbi.ttf]{cmunti.ttf}\setmonofont[Path=/usr/share/fonts/truetype/cmu/,UprightFont=cmuntt.ttf,BoldFont=cmuntb.ttf,ItalicFont=cmunit.ttf,BoldItalicFont=cmuntx.ttf]{cmunti.ttf}\itshape strong}{$\text{ }$}\setmainfont[Path=/usr/share/fonts/truetype/cmu/,UprightFont=cmunrm.ttf,BoldFont=cmunbx.ttf,ItalicFont=cmunti.ttf,BoldItalicFont=cmunbi.ttf]{cmunrm.ttf}\setmonofont[Path=/usr/share/fonts/truetype/cmu/,UprightFont=cmuntt.ttf,BoldFont=cmuntb.ttf,ItalicFont=cmunit.ttf,BoldItalicFont=cmuntx.ttf]{cmunrm.ttf} if you want to get something similar to what you see on the screen). Try different resolutions to fit your needs, but 100 dpi should be enough. Once you have the image within GIMP, you can post-{}process it as you like and save it to any format supported by GIMP, as PNG for example.
{\bfseries
\begin{mydescription}dvipng
\end{mydescription}
}

A method for DVI files is \myhref{http://savannah.nongnu.org/projects/dvipng/}{dvipng}. Usage is the same as {\ttfamily \setmainfont[Path=/usr/share/fonts/truetype/cmu/,UprightFont=cmunrm.ttf,BoldFont=cmunbx.ttf,ItalicFont=cmunti.ttf,BoldItalicFont=cmunbi.ttf]{cmuntt.ttf}\setmonofont[Path=/usr/share/fonts/truetype/cmu/,UprightFont=cmuntt.ttf,BoldFont=cmuntb.ttf,ItalicFont=cmunit.ttf,BoldItalicFont=cmuntx.ttf]{cmuntt.ttf}\ttfamily dvipdfm}\setmainfont[Path=/usr/share/fonts/truetype/cmu/,UprightFont=cmunrm.ttf,BoldFont=cmunbx.ttf,ItalicFont=cmunti.ttf,BoldItalicFont=cmunbi.ttf]{cmunrm.ttf}\setmonofont[Path=/usr/share/fonts/truetype/cmu/,UprightFont=cmuntt.ttf,BoldFont=cmuntb.ttf,ItalicFont=cmunit.ttf,BoldItalicFont=cmuntx.ttf]{cmunrm.ttf}.

Run {\ttfamily \setmainfont[Path=/usr/share/fonts/truetype/cmu/,UprightFont=cmunrm.ttf,BoldFont=cmunbx.ttf,ItalicFont=cmunti.ttf,BoldItalicFont=cmunbi.ttf]{cmuntt.ttf}\setmonofont[Path=/usr/share/fonts/truetype/cmu/,UprightFont=cmuntt.ttf,BoldFont=cmuntb.ttf,ItalicFont=cmunit.ttf,BoldItalicFont=cmuntx.ttf]{cmuntt.ttf}\ttfamily latex}{$\text{ }$}\setmainfont[Path=/usr/share/fonts/truetype/cmu/,UprightFont=cmunrm.ttf,BoldFont=cmunbx.ttf,ItalicFont=cmunti.ttf,BoldItalicFont=cmunbi.ttf]{cmunrm.ttf}\setmonofont[Path=/usr/share/fonts/truetype/cmu/,UprightFont=cmuntt.ttf,BoldFont=cmuntb.ttf,ItalicFont=cmunit.ttf,BoldItalicFont=cmuntx.ttf]{cmunrm.ttf} as usual to generate the dvi file. Now, we want an X font size formula, where X is measure in pixels. You need to convert this, to dots per inch (dpi). The formula is: {\ttfamily \setmainfont[Path=/usr/share/fonts/truetype/cmu/,UprightFont=cmunrm.ttf,BoldFont=cmunbx.ttf,ItalicFont=cmunti.ttf,BoldItalicFont=cmunbi.ttf]{cmuntt.ttf}\setmonofont[Path=/usr/share/fonts/truetype/cmu/,UprightFont=cmuntt.ttf,BoldFont=cmuntb.ttf,ItalicFont=cmunit.ttf,BoldItalicFont=cmuntx.ttf]{cmuntt.ttf}\ttfamily <{}dpi>{} = <{}font\_px>{}*72.27/10}\setmainfont[Path=/usr/share/fonts/truetype/cmu/,UprightFont=cmunrm.ttf,BoldFont=cmunbx.ttf,ItalicFont=cmunti.ttf,BoldItalicFont=cmunbi.ttf]{cmunrm.ttf}\setmonofont[Path=/usr/share/fonts/truetype/cmu/,UprightFont=cmuntt.ttf,BoldFont=cmuntb.ttf,ItalicFont=cmunit.ttf,BoldItalicFont=cmuntx.ttf]{cmunrm.ttf}. If you want, for instance, X = 32, then the size in dpi corresponds to 231.26. This value will be passed to {\ttfamily \setmainfont[Path=/usr/share/fonts/truetype/cmu/,UprightFont=cmunrm.ttf,BoldFont=cmunbx.ttf,ItalicFont=cmunti.ttf,BoldItalicFont=cmunbi.ttf]{cmuntt.ttf}\setmonofont[Path=/usr/share/fonts/truetype/cmu/,UprightFont=cmuntt.ttf,BoldFont=cmuntb.ttf,ItalicFont=cmunit.ttf,BoldItalicFont=cmuntx.ttf]{cmuntt.ttf}\ttfamily dvipng}{$\text{ }$}\setmainfont[Path=/usr/share/fonts/truetype/cmu/,UprightFont=cmunrm.ttf,BoldFont=cmunbx.ttf,ItalicFont=cmunti.ttf,BoldItalicFont=cmunbi.ttf]{cmunrm.ttf}\setmonofont[Path=/usr/share/fonts/truetype/cmu/,UprightFont=cmuntt.ttf,BoldFont=cmuntb.ttf,ItalicFont=cmunit.ttf,BoldItalicFont=cmuntx.ttf]{cmunrm.ttf} using the flag {\ttfamily \setmainfont[Path=/usr/share/fonts/truetype/cmu/,UprightFont=cmunrm.ttf,BoldFont=cmunbx.ttf,ItalicFont=cmunti.ttf,BoldItalicFont=cmunbi.ttf]{cmuntt.ttf}\setmonofont[Path=/usr/share/fonts/truetype/cmu/,UprightFont=cmuntt.ttf,BoldFont=cmuntb.ttf,ItalicFont=cmunit.ttf,BoldItalicFont=cmuntx.ttf]{cmuntt.ttf}\ttfamily -{}D}\setmainfont[Path=/usr/share/fonts/truetype/cmu/,UprightFont=cmunrm.ttf,BoldFont=cmunbx.ttf,ItalicFont=cmunti.ttf,BoldItalicFont=cmunbi.ttf]{cmunrm.ttf}\setmonofont[Path=/usr/share/fonts/truetype/cmu/,UprightFont=cmuntt.ttf,BoldFont=cmuntb.ttf,ItalicFont=cmunit.ttf,BoldItalicFont=cmuntx.ttf]{cmunrm.ttf}. To generate the desired png file run the command as follows: 
\\

\TemplateSpaceIndent{$\text{ }${}dvipng$\text{ }${}-{}T$\text{ }${}tight$\text{ }${}-{}D$\text{ }${}231.26$\text{ }${}-{}o$\text{ }${}foo.png$\text{ }${}foo.dvi}


The flag {\ttfamily \setmainfont[Path=/usr/share/fonts/truetype/cmu/,UprightFont=cmunrm.ttf,BoldFont=cmunbx.ttf,ItalicFont=cmunti.ttf,BoldItalicFont=cmunbi.ttf]{cmuntt.ttf}\setmonofont[Path=/usr/share/fonts/truetype/cmu/,UprightFont=cmuntt.ttf,BoldFont=cmuntb.ttf,ItalicFont=cmunit.ttf,BoldItalicFont=cmuntx.ttf]{cmuntt.ttf}\ttfamily -{}T}{$\text{ }$}\setmainfont[Path=/usr/share/fonts/truetype/cmu/,UprightFont=cmunrm.ttf,BoldFont=cmunbx.ttf,ItalicFont=cmunti.ttf,BoldItalicFont=cmunbi.ttf]{cmunrm.ttf}\setmonofont[Path=/usr/share/fonts/truetype/cmu/,UprightFont=cmuntt.ttf,BoldFont=cmuntb.ttf,ItalicFont=cmunit.ttf,BoldItalicFont=cmuntx.ttf]{cmunrm.ttf} sets the size of the image. The option {\ttfamily \setmainfont[Path=/usr/share/fonts/truetype/cmu/,UprightFont=cmunrm.ttf,BoldFont=cmunbx.ttf,ItalicFont=cmunti.ttf,BoldItalicFont=cmunbi.ttf]{cmuntt.ttf}\setmonofont[Path=/usr/share/fonts/truetype/cmu/,UprightFont=cmuntt.ttf,BoldFont=cmuntb.ttf,ItalicFont=cmunit.ttf,BoldItalicFont=cmuntx.ttf]{cmuntt.ttf}\ttfamily tight}{$\text{ }$}\setmainfont[Path=/usr/share/fonts/truetype/cmu/,UprightFont=cmunrm.ttf,BoldFont=cmunbx.ttf,ItalicFont=cmunti.ttf,BoldItalicFont=cmunbi.ttf]{cmunrm.ttf}\setmonofont[Path=/usr/share/fonts/truetype/cmu/,UprightFont=cmuntt.ttf,BoldFont=cmuntb.ttf,ItalicFont=cmunit.ttf,BoldItalicFont=cmuntx.ttf]{cmunrm.ttf} will only include all ink put on the page. The option {\ttfamily \setmainfont[Path=/usr/share/fonts/truetype/cmu/,UprightFont=cmunrm.ttf,BoldFont=cmunbx.ttf,ItalicFont=cmunti.ttf,BoldItalicFont=cmunbi.ttf]{cmuntt.ttf}\setmonofont[Path=/usr/share/fonts/truetype/cmu/,UprightFont=cmuntt.ttf,BoldFont=cmuntb.ttf,ItalicFont=cmunit.ttf,BoldItalicFont=cmuntx.ttf]{cmuntt.ttf}\ttfamily -{}o}{$\text{ }$}\setmainfont[Path=/usr/share/fonts/truetype/cmu/,UprightFont=cmunrm.ttf,BoldFont=cmunbx.ttf,ItalicFont=cmunti.ttf,BoldItalicFont=cmunbi.ttf]{cmunrm.ttf}\setmonofont[Path=/usr/share/fonts/truetype/cmu/,UprightFont=cmuntt.ttf,BoldFont=cmuntb.ttf,ItalicFont=cmunit.ttf,BoldItalicFont=cmuntx.ttf]{cmunrm.ttf} sends the output to the file name {\ttfamily \setmainfont[Path=/usr/share/fonts/truetype/cmu/,UprightFont=cmunrm.ttf,BoldFont=cmunbx.ttf,ItalicFont=cmunti.ttf,BoldItalicFont=cmunbi.ttf]{cmuntt.ttf}\setmonofont[Path=/usr/share/fonts/truetype/cmu/,UprightFont=cmuntt.ttf,BoldFont=cmuntb.ttf,ItalicFont=cmunit.ttf,BoldItalicFont=cmuntx.ttf]{cmuntt.ttf}\ttfamily foo.png}\setmainfont[Path=/usr/share/fonts/truetype/cmu/,UprightFont=cmunrm.ttf,BoldFont=cmunbx.ttf,ItalicFont=cmunti.ttf,BoldItalicFont=cmunbi.ttf]{cmunrm.ttf}\setmonofont[Path=/usr/share/fonts/truetype/cmu/,UprightFont=cmuntt.ttf,BoldFont=cmuntb.ttf,ItalicFont=cmunit.ttf,BoldItalicFont=cmuntx.ttf]{cmunrm.ttf}.
{\bfseries
\begin{mydescription}ImageMagick
\end{mydescription}
}

The {\ttfamily \setmainfont[Path=/usr/share/fonts/truetype/cmu/,UprightFont=cmunrm.ttf,BoldFont=cmunbx.ttf,ItalicFont=cmunti.ttf,BoldItalicFont=cmunbi.ttf]{cmuntt.ttf}\setmonofont[Path=/usr/share/fonts/truetype/cmu/,UprightFont=cmuntt.ttf,BoldFont=cmuntb.ttf,ItalicFont=cmunit.ttf,BoldItalicFont=cmuntx.ttf]{cmuntt.ttf}\ttfamily convert}{$\text{ }$}\setmainfont[Path=/usr/share/fonts/truetype/cmu/,UprightFont=cmunrm.ttf,BoldFont=cmunbx.ttf,ItalicFont=cmunti.ttf,BoldItalicFont=cmunbi.ttf]{cmunrm.ttf}\setmonofont[Path=/usr/share/fonts/truetype/cmu/,UprightFont=cmuntt.ttf,BoldFont=cmuntb.ttf,ItalicFont=cmunit.ttf,BoldItalicFont=cmuntx.ttf]{cmunrm.ttf} command from the \myhref{http://www.imagemagick.org/}{ImageMagick} suite can convert both DVI and PDF files to PNG.\\

\TemplateSpaceIndent{$\text{ }${}convert$\text{ }${}input.pdf$\text{ }${}output.png}

{\bfseries
\begin{mydescription}optipng
\end{mydescription}
}

You can optimize the resulting image using \myhref{http://optipng.sourceforge.net/}{optipng} so that it will take up less space.
\section{Convert to plain text}
\label{943}

If you are thinking of converting to plain text for spell-{}checking or to count words, there may be an easier way -{}-{} read \mylref{976}{Tips and Tricks} first.

Most LaTeX distributions come with {\ttfamily \setmainfont[Path=/usr/share/fonts/truetype/cmu/,UprightFont=cmunrm.ttf,BoldFont=cmunbx.ttf,ItalicFont=cmunti.ttf,BoldItalicFont=cmunbi.ttf]{cmuntt.ttf}\setmonofont[Path=/usr/share/fonts/truetype/cmu/,UprightFont=cmuntt.ttf,BoldFont=cmuntb.ttf,ItalicFont=cmunit.ttf,BoldItalicFont=cmuntx.ttf]{cmuntt.ttf}\ttfamily detex}{$\text{ }$}\setmainfont[Path=/usr/share/fonts/truetype/cmu/,UprightFont=cmunrm.ttf,BoldFont=cmunbx.ttf,ItalicFont=cmunti.ttf,BoldItalicFont=cmunbi.ttf]{cmunrm.ttf}\setmonofont[Path=/usr/share/fonts/truetype/cmu/,UprightFont=cmuntt.ttf,BoldFont=cmuntb.ttf,ItalicFont=cmunit.ttf,BoldItalicFont=cmuntx.ttf]{cmunrm.ttf} program, which strips LaTeX commands. It can handle multi-{}file projects, so all you need is to give one command:\\

\TemplateSpaceIndent{$\text{ }${}detex$\text{ }${}yourfile}

(note the omission of .tex extension). This will output result to standard output. If you want the plain text go to a file, use\\

\TemplateSpaceIndent{$\text{ }${}detex$\text{ }${}yourfile$\text{ }${}>{}$\text{ }${}yourfile.txt}


If the output from {\ttfamily \setmainfont[Path=/usr/share/fonts/truetype/cmu/,UprightFont=cmunrm.ttf,BoldFont=cmunbx.ttf,ItalicFont=cmunti.ttf,BoldItalicFont=cmunbi.ttf]{cmuntt.ttf}\setmonofont[Path=/usr/share/fonts/truetype/cmu/,UprightFont=cmuntt.ttf,BoldFont=cmuntb.ttf,ItalicFont=cmunit.ttf,BoldItalicFont=cmuntx.ttf]{cmuntt.ttf}\ttfamily detex}{$\text{ }$}\setmainfont[Path=/usr/share/fonts/truetype/cmu/,UprightFont=cmunrm.ttf,BoldFont=cmunbx.ttf,ItalicFont=cmunti.ttf,BoldItalicFont=cmunbi.ttf]{cmunrm.ttf}\setmonofont[Path=/usr/share/fonts/truetype/cmu/,UprightFont=cmuntt.ttf,BoldFont=cmuntb.ttf,ItalicFont=cmunit.ttf,BoldItalicFont=cmuntx.ttf]{cmunrm.ttf} does not satisfy you, you can try a newer version available on \myhref{http://code.google.com/p/opendetex/}{Google Code}, or use HTML conversion first and then copy text from your browser.

If you want to keep the formating, you can use a {\itshape \setmainfont[Path=/usr/share/fonts/truetype/cmu/,UprightFont=cmunrm.ttf,BoldFont=cmunbx.ttf,ItalicFont=cmunti.ttf,BoldItalicFont=cmunbi.ttf]{cmunti.ttf}\setmonofont[Path=/usr/share/fonts/truetype/cmu/,UprightFont=cmuntt.ttf,BoldFont=cmuntb.ttf,ItalicFont=cmunit.ttf,BoldItalicFont=cmuntx.ttf]{cmunti.ttf}\itshape DVI-{}to-{}plain text}{$\text{ }$}\setmainfont[Path=/usr/share/fonts/truetype/cmu/,UprightFont=cmunrm.ttf,BoldFont=cmunbx.ttf,ItalicFont=cmunti.ttf,BoldItalicFont=cmunbi.ttf]{cmunrm.ttf}\setmonofont[Path=/usr/share/fonts/truetype/cmu/,UprightFont=cmuntt.ttf,BoldFont=cmuntb.ttf,ItalicFont=cmunit.ttf,BoldItalicFont=cmuntx.ttf]{cmunrm.ttf} converter, like {\ttfamily \setmainfont[Path=/usr/share/fonts/truetype/cmu/,UprightFont=cmunrm.ttf,BoldFont=cmunbx.ttf,ItalicFont=cmunti.ttf,BoldItalicFont=cmunbi.ttf]{cmuntt.ttf}\setmonofont[Path=/usr/share/fonts/truetype/cmu/,UprightFont=cmuntt.ttf,BoldFont=cmuntb.ttf,ItalicFont=cmunit.ttf,BoldItalicFont=cmuntx.ttf]{cmuntt.ttf}\ttfamily catdvi}\setmainfont[Path=/usr/share/fonts/truetype/cmu/,UprightFont=cmunrm.ttf,BoldFont=cmunbx.ttf,ItalicFont=cmunti.ttf,BoldItalicFont=cmunbi.ttf]{cmunrm.ttf}\setmonofont[Path=/usr/share/fonts/truetype/cmu/,UprightFont=cmuntt.ttf,BoldFont=cmuntb.ttf,ItalicFont=cmunit.ttf,BoldItalicFont=cmuntx.ttf]{cmunrm.ttf}. Example:\\

\TemplateSpaceIndent{$\text{ }${}catdvi$\text{ }${}yourfile.dvi$\text{ }${}|$\text{ }${}fmt$\text{ }${}-{}u}

The use of {\itshape \setmainfont[Path=/usr/share/fonts/truetype/cmu/,UprightFont=cmunrm.ttf,BoldFont=cmunbx.ttf,ItalicFont=cmunti.ttf,BoldItalicFont=cmunbi.ttf]{cmunti.ttf}\setmonofont[Path=/usr/share/fonts/truetype/cmu/,UprightFont=cmuntt.ttf,BoldFont=cmuntb.ttf,ItalicFont=cmunit.ttf,BoldItalicFont=cmuntx.ttf]{cmunti.ttf}\itshape fmt -{}u}{$\text{ }$}\setmainfont[Path=/usr/share/fonts/truetype/cmu/,UprightFont=cmunrm.ttf,BoldFont=cmunbx.ttf,ItalicFont=cmunti.ttf,BoldItalicFont=cmunbi.ttf]{cmunrm.ttf}\setmonofont[Path=/usr/share/fonts/truetype/cmu/,UprightFont=cmuntt.ttf,BoldFont=cmuntb.ttf,ItalicFont=cmunit.ttf,BoldItalicFont=cmuntx.ttf]{cmunrm.ttf} (available on most Unices) will remove the justification.



\myhref{https://sr.wikibooks.org/wiki/LaTeX\%2F\%D0\%9E\%D1\%82\%D0\%BF\%D1\%80\%D0\%B5\%D0\%BC\%D0\%B0\%D1\%9A\%D0\%B5\%20\%D1\%83\%20\%D0\%B4\%D1\%80\%D1\%83\%D0\%B3\%D0\%B5\%20\%D1\%84\%D0\%BE\%D1\%80\%D0\%BC\%D0\%B0\%D1\%82\%D0\%B5}{sr:LaTeX/Отпремање у друге формате}
\mypart{Help and Recommendations}\chapter{FAQ}

\myminitoc
\label{944}

\label{945}

\section{Margins are too wide}
\label{946}

LaTeX\textquotesingle{}s default margins may seem too large. In most cases, this is a preferred default and improves readability.

If you still disagree, you can easily change them with
\begin{Shaded}
\begin{Highlighting}[]

\NormalTok{\textbackslash{}usepackage\{geometry\}}
\CommentTok{% or}
\NormalTok{\textbackslash{}usepackage[margin=1.5in]\{geometry\}}
\end{Highlighting}
\end{Shaded}


See \mylref{303}{Page Layout}.
\section{Avoid excessive double line breaks in source code}
\label{947}

Too many paragraphs of one line or two do not look very good.

Remember the TeX rule:
\begin{myitemize}
\item{}  If two or more consecutive line breaks are found, TeX starts a new paragraph.
\item{}  If only one linebreak is found, TeX inserts a space if there is no space directly before or after it.
\end{myitemize}


You might be tempted to put blank lines all the time to improve the readability of your source code, but this will have an impact on formatting. The solution is simple: put a comment at the very beginning of the blank lines. This will prevent TeX from seeing another line break—all characters up to and including the next line break after a comment are ignored.

Example:
\begin{Shaded}
\begin{Highlighting}[]

\NormalTok{We are in the first paragraph here.}
\CommentTok{%}
\NormalTok{We are still in the first paragraph.}
 
\NormalTok{This time, this is another paragraph.}
\end{Highlighting}
\end{Shaded}

\section{Simplified special character input}
\label{948}

So long as your computing environment supports UTF-{}8, you can enter special characters directly rather than entering the TeX commands for diacritics and other extended characters.  E.g., \begin{Shaded}
\begin{Highlighting}[]

\NormalTok{R\textbackslash{}'esum\textbackslash{}'e can also be written résumé.}
\end{Highlighting}
\end{Shaded}


This requires that:
\begin{myitemize}
\item{}  your text editor supports and is set to save your file in UTF-{}8;
\item{}  you add the \LaTeXTT{\textbackslash{}usepackage{$\text{[}$}utf8{$\text{]}$}\{inputenc\}} line in the preamble.
\end{myitemize}


Avoid using \LaTeXTT{latin1}. See \mylref{192}{Special Characters}.
\section{Writing the euro symbol directly}
\label{949}

Add the following lines in your preamble:
\begin{Shaded}
\begin{Highlighting}[]

\NormalTok{\textbackslash{}usepackage[utf8]\{inputenc\}}
\NormalTok{\textbackslash{}usepackage\{marvosym\}}
\NormalTok{\textbackslash{}DeclareUnicodeCharacter\{20AC\}\{\textbackslash{}EUR\{\}\}}
\end{Highlighting}
\end{Shaded}

\section{LaTeX paragraph headings have title and content on the same line}
\label{950}

Some people do not like the way \LaTeXTT{\textbackslash{}paragraph\{...\}} writes the title on the same line as the content. This is actually fairly common in a lot of documents and not as weird as it may seem at first.

There are ways to get around the default behavior, however; see \mylref{140}{\textbackslash{}paragraph line break} for more information.
\section{{\itshape \setmainfont[Path=/usr/share/fonts/truetype/cmu/,UprightFont=cmunrm.ttf,BoldFont=cmunbx.ttf,ItalicFont=cmunti.ttf,BoldItalicFont=cmunbi.ttf]{cmunti.ttf}\setmonofont[Path=/usr/share/fonts/truetype/cmu/,UprightFont=cmuntt.ttf,BoldFont=cmuntb.ttf,ItalicFont=cmunit.ttf,BoldItalicFont=cmuntx.ttf]{cmunti.ttf}\itshape Fonts are ugly/jagged/bitmaps}{$\text{ }$}\setmainfont[Path=/usr/share/fonts/truetype/cmu/,UprightFont=cmunrm.ttf,BoldFont=cmunbx.ttf,ItalicFont=cmunti.ttf,BoldItalicFont=cmunbi.ttf]{cmunrm.ttf}\setmonofont[Path=/usr/share/fonts/truetype/cmu/,UprightFont=cmuntt.ttf,BoldFont=cmuntb.ttf,ItalicFont=cmunit.ttf,BoldItalicFont=cmuntx.ttf]{cmunrm.ttf} or {\itshape \setmainfont[Path=/usr/share/fonts/truetype/cmu/,UprightFont=cmunrm.ttf,BoldFont=cmunbx.ttf,ItalicFont=cmunti.ttf,BoldItalicFont=cmunbi.ttf]{cmunti.ttf}\setmonofont[Path=/usr/share/fonts/truetype/cmu/,UprightFont=cmuntt.ttf,BoldFont=cmuntb.ttf,ItalicFont=cmunit.ttf,BoldItalicFont=cmuntx.ttf]{cmunti.ttf}\itshape PDF search fails}{$\text{ }$}\setmainfont[Path=/usr/share/fonts/truetype/cmu/,UprightFont=cmunrm.ttf,BoldFont=cmunbx.ttf,ItalicFont=cmunti.ttf,BoldItalicFont=cmunbi.ttf]{cmunrm.ttf}\setmonofont[Path=/usr/share/fonts/truetype/cmu/,UprightFont=cmuntt.ttf,BoldFont=cmuntb.ttf,ItalicFont=cmunit.ttf,BoldItalicFont=cmuntx.ttf]{cmunrm.ttf} or {\itshape \setmainfont[Path=/usr/share/fonts/truetype/cmu/,UprightFont=cmunrm.ttf,BoldFont=cmunbx.ttf,ItalicFont=cmunti.ttf,BoldItalicFont=cmunbi.ttf]{cmunti.ttf}\setmonofont[Path=/usr/share/fonts/truetype/cmu/,UprightFont=cmuntt.ttf,BoldFont=cmuntb.ttf,ItalicFont=cmunit.ttf,BoldItalicFont=cmuntx.ttf]{cmunti.ttf}\itshape Copy/paste from PDF is messy}}
\label{951}\setmainfont[Path=/usr/share/fonts/truetype/cmu/,UprightFont=cmunrm.ttf,BoldFont=cmunbx.ttf,ItalicFont=cmunti.ttf,BoldItalicFont=cmunbi.ttf]{cmunrm.ttf}\setmonofont[Path=/usr/share/fonts/truetype/cmu/,UprightFont=cmuntt.ttf,BoldFont=cmuntb.ttf,ItalicFont=cmunit.ttf,BoldItalicFont=cmuntx.ttf]{cmunrm.ttf}

You must be using diacritics ({\itshape \setmainfont[Path=/usr/share/fonts/truetype/cmu/,UprightFont=cmunrm.ttf,BoldFont=cmunbx.ttf,ItalicFont=cmunti.ttf,BoldItalicFont=cmunbi.ttf]{cmunti.ttf}\setmonofont[Path=/usr/share/fonts/truetype/cmu/,UprightFont=cmuntt.ttf,BoldFont=cmuntb.ttf,ItalicFont=cmunit.ttf,BoldItalicFont=cmuntx.ttf]{cmunti.ttf}\itshape e.g.}{$\text{ }$}\setmainfont[Path=/usr/share/fonts/truetype/cmu/,UprightFont=cmunrm.ttf,BoldFont=cmunbx.ttf,ItalicFont=cmunti.ttf,BoldItalicFont=cmunbi.ttf]{cmunrm.ttf}\setmonofont[Path=/usr/share/fonts/truetype/cmu/,UprightFont=cmuntt.ttf,BoldFont=cmuntb.ttf,ItalicFont=cmunit.ttf,BoldItalicFont=cmuntx.ttf]{cmunrm.ttf} accents) with OT1 encoding (the default).
Switch to T1 encoding:
\begin{Shaded}
\begin{Highlighting}[]

\NormalTok{\textbackslash{}usepackage[T1]\{fontenc\}}
\end{Highlighting}
\end{Shaded}


If you have ugly jagged fonts after the font encoding change, then you have no Type1 compatible fonts available. Install \LaTeXTT{Computer Modern Super} or \LaTeXTT{Latin Modern} (package name may be \LaTeXTT{lm}). To use \LaTeXTT{Latin Modern} you need to include the package:
\begin{Shaded}
\begin{Highlighting}[]

\NormalTok{\textbackslash{}usepackage\{lmodern\}}
\end{Highlighting}
\end{Shaded}


See \mylref{173}{Fonts} for an explanation.
\section{Manual formatting: use of line breaks and page breaks}
\label{952}

You should {\itshape \setmainfont[Path=/usr/share/fonts/truetype/cmu/,UprightFont=cmunrm.ttf,BoldFont=cmunbx.ttf,ItalicFont=cmunti.ttf,BoldItalicFont=cmunbi.ttf]{cmunti.ttf}\setmonofont[Path=/usr/share/fonts/truetype/cmu/,UprightFont=cmuntt.ttf,BoldFont=cmuntb.ttf,ItalicFont=cmunit.ttf,BoldItalicFont=cmuntx.ttf]{cmunti.ttf}\itshape really}{$\text{ }$}\setmainfont[Path=/usr/share/fonts/truetype/cmu/,UprightFont=cmunrm.ttf,BoldFont=cmunbx.ttf,ItalicFont=cmunti.ttf,BoldItalicFont=cmunbi.ttf]{cmunrm.ttf}\setmonofont[Path=/usr/share/fonts/truetype/cmu/,UprightFont=cmuntt.ttf,BoldFont=cmuntb.ttf,ItalicFont=cmunit.ttf,BoldItalicFont=cmuntx.ttf]{cmunrm.ttf} avoid breaking lines and pages manually. The TeX engine is in charge of that. The big problem with manual formatting is that it is not dynamic. Even if it looks right the first time, the content is likely to render really badly if you change anything before the point you manually formatted.

The only place where page breaks are recommended is at the upper level of sectioning in your documents, {\itshape \setmainfont[Path=/usr/share/fonts/truetype/cmu/,UprightFont=cmunrm.ttf,BoldFont=cmunbx.ttf,ItalicFont=cmunti.ttf,BoldItalicFont=cmunbi.ttf]{cmunti.ttf}\setmonofont[Path=/usr/share/fonts/truetype/cmu/,UprightFont=cmuntt.ttf,BoldFont=cmuntb.ttf,ItalicFont=cmunit.ttf,BoldItalicFont=cmuntx.ttf]{cmunti.ttf}\itshape e.g.}{$\text{ }$}\setmainfont[Path=/usr/share/fonts/truetype/cmu/,UprightFont=cmunrm.ttf,BoldFont=cmunbx.ttf,ItalicFont=cmunti.ttf,BoldItalicFont=cmunbi.ttf]{cmunrm.ttf}\setmonofont[Path=/usr/share/fonts/truetype/cmu/,UprightFont=cmuntt.ttf,BoldFont=cmuntb.ttf,ItalicFont=cmunit.ttf,BoldItalicFont=cmuntx.ttf]{cmunrm.ttf} parts or chapters (although when you start a new part or chapter, LaTeX will ordinarily do this for you). When you do manually insert a page break, you should use \LaTeXTT{\textbackslash{}clearpage} or \LaTeXTT{\textbackslash{}cleardoublepage} which print currently floating figures before starting a new page.

If you absolutely have to insert line or page breaks manually, you should do it after you are sure you have completed your document otherwise, so that you don\textquotesingle{}t later have to come back and update it.
\section{Always finish commands with {\itshape \setmainfont[Path=/usr/share/fonts/truetype/cmu/,UprightFont=cmunrm.ttf,BoldFont=cmunbx.ttf,ItalicFont=cmunti.ttf,BoldItalicFont=cmunbi.ttf]{cmunti.ttf}\setmonofont[Path=/usr/share/fonts/truetype/cmu/,UprightFont=cmuntt.ttf,BoldFont=cmuntb.ttf,ItalicFont=cmunit.ttf,BoldItalicFont=cmuntx.ttf]{cmunti.ttf}\itshape \{\}}}
\label{953}\setmainfont[Path=/usr/share/fonts/truetype/cmu/,UprightFont=cmunrm.ttf,BoldFont=cmunbx.ttf,ItalicFont=cmunti.ttf,BoldItalicFont=cmunbi.ttf]{cmunrm.ttf}\setmonofont[Path=/usr/share/fonts/truetype/cmu/,UprightFont=cmuntt.ttf,BoldFont=cmuntb.ttf,ItalicFont=cmunit.ttf,BoldItalicFont=cmuntx.ttf]{cmunrm.ttf}

TeX has an unintuitive rule that if a control sequence (a command) is not followed by a pair of braces (with a parameter in between or not), then the following space character(s) are ignored. LaTeX will not print {\itshape \setmainfont[Path=/usr/share/fonts/truetype/cmu/,UprightFont=cmunrm.ttf,BoldFont=cmunbx.ttf,ItalicFont=cmunti.ttf,BoldItalicFont=cmunbi.ttf]{cmunti.ttf}\setmonofont[Path=/usr/share/fonts/truetype/cmu/,UprightFont=cmuntt.ttf,BoldFont=cmuntb.ttf,ItalicFont=cmunit.ttf,BoldItalicFont=cmuntx.ttf]{cmunti.ttf}\itshape any}{$\text{ }$}\setmainfont[Path=/usr/share/fonts/truetype/cmu/,UprightFont=cmunrm.ttf,BoldFont=cmunbx.ttf,ItalicFont=cmunti.ttf,BoldItalicFont=cmunbi.ttf]{cmunrm.ttf}\setmonofont[Path=/usr/share/fonts/truetype/cmu/,UprightFont=cmuntt.ttf,BoldFont=cmuntb.ttf,ItalicFont=cmunit.ttf,BoldItalicFont=cmuntx.ttf]{cmunrm.ttf} space, and the command (say, the TeX or LaTeX logos) are run together with the following word.

To fix this, use a pair of braces after the command, even if there are no parameters.
Example:
\begin{Shaded}
\begin{Highlighting}[]

\NormalTok{\textbackslash{}LaTeX is great. }\CommentTok{% BAD !}
\NormalTok{\textbackslash{}LaTeX\{\} is great. }\CommentTok{% GOOD !}
\end{Highlighting}
\end{Shaded}


(Technical explanation: a control sequence name can only be composed of characters with catcode 11, that is A-{}Z and a-{}z by default. TeX knows where the control sequence name start thanks to the backslash, and it knows where it ends when it encounters the first token which is not of catcode 11. This character is then skipped. Since consecutive spaces have been concatenated into one single space, no space is taken into account.)

It is possible to define macros that will insert a space dynamically by using the \LaTeXTT{xspace} package.
\begin{myitemize}
\item{}  If there is no brace and a space following the command, an extra space will be appended.
\item{}  If there are braces, no extra space will be printed.
\end{myitemize}


Example:
\begin{Shaded}
\begin{Highlighting}[]

\NormalTok{\textbackslash{}usepackage\{xspace\}}
\NormalTok{\textbackslash{}let\textbackslash{}latexold\textbackslash{}LaTeX}
\NormalTok{\textbackslash{}renewcommand\{\textbackslash{}LaTeX\}\{\textbackslash{}textrm\{\textbackslash{}latexold\}\textbackslash{}xspace\}}
\NormalTok{...}
\NormalTok{\textbackslash{}LaTeX is followed by a space.}
\NormalTok{\textbackslash{}LaTeX\{\} is followed by a space.}
\NormalTok{\textbackslash{}LaTeX\{\}is not followed by a space.}
\end{Highlighting}
\end{Shaded}

\section{Avoid bold and underline}
\label{954}

Typographically speaking, it is usually poor practice to use bold or underline formats in the middle of a paragraph. This has become a common habit for users of traditional word processors because these two functions are very easily accessible (along with italics).

However, bold and underline tend to overweight the text and distract the reader. When you start reading a paragraph with a bold word in the middle, you often read the emphasized part first, thus spoiling the content and breaking the order of the ideas. Italics are less obvious and do not have more weight than normal characters, so they are usually a better choice for emphasizing small amounts of text.

The original and more appropriate use of bold and underline is for special parts, such as headers, the index, glossaries, and so on. (Actually, underlining is rarely used in professional environments.)

LaTeX has a macro \LaTeXTT{\textbackslash{}emph\{...\}} for emphasizing text using italics. It should be preferred to \LaTeXTT{\textbackslash{}textit\{...\}} because \LaTeXTT{\textbackslash{}emph\{...\}} will correctly print emphasized text inside other italic text in the regular font.
\section{The proper way to use figures}
\label{955}

Users used to WYSIWYG document processors like Microsoft Word or LibreOffice often get frustrated with figures. The answer is simple: a figure is {\itshape \setmainfont[Path=/usr/share/fonts/truetype/cmu/,UprightFont=cmunrm.ttf,BoldFont=cmunbx.ttf,ItalicFont=cmunti.ttf,BoldItalicFont=cmunbi.ttf]{cmunti.ttf}\setmonofont[Path=/usr/share/fonts/truetype/cmu/,UprightFont=cmuntt.ttf,BoldFont=cmuntb.ttf,ItalicFont=cmunit.ttf,BoldItalicFont=cmuntx.ttf]{cmunti.ttf}\itshape not}{$\text{ }$}\setmainfont[Path=/usr/share/fonts/truetype/cmu/,UprightFont=cmunrm.ttf,BoldFont=cmunbx.ttf,ItalicFont=cmunti.ttf,BoldItalicFont=cmunbi.ttf]{cmunrm.ttf}\setmonofont[Path=/usr/share/fonts/truetype/cmu/,UprightFont=cmuntt.ttf,BoldFont=cmuntb.ttf,ItalicFont=cmunit.ttf,BoldItalicFont=cmuntx.ttf]{cmunrm.ttf} a picture! 

If you use \LaTeXTT{\textbackslash{}includegraphics} without enclosing it in a \LaTeXTT{figure} environment, it will behave just as in a word processor, placing the picture right at the spot where it was placed in the source.

{\itshape \setmainfont[Path=/usr/share/fonts/truetype/cmu/,UprightFont=cmunrm.ttf,BoldFont=cmunbx.ttf,ItalicFont=cmunti.ttf,BoldItalicFont=cmunbi.ttf]{cmunti.ttf}\setmonofont[Path=/usr/share/fonts/truetype/cmu/,UprightFont=cmuntt.ttf,BoldFont=cmuntb.ttf,ItalicFont=cmunit.ttf,BoldItalicFont=cmuntx.ttf]{cmunti.ttf}\itshape Figures}{$\text{ }$}\setmainfont[Path=/usr/share/fonts/truetype/cmu/,UprightFont=cmunrm.ttf,BoldFont=cmunbx.ttf,ItalicFont=cmunti.ttf,BoldItalicFont=cmunbi.ttf]{cmunrm.ttf}\setmonofont[Path=/usr/share/fonts/truetype/cmu/,UprightFont=cmuntt.ttf,BoldFont=cmuntb.ttf,ItalicFont=cmunit.ttf,BoldItalicFont=cmuntx.ttf]{cmunrm.ttf} are a type of float, which is a virtual object that LaTeX can put in places other than the exact location it was created, which helps to prevent cluttering your text with pictures and tables.

See \mylref{336}{Importing Graphics} and \mylref{362}{Floats, Figures and Captions} for more details.
\section{Text stops justifying}
\label{956}

Most likely you have used
\LaTeXTT{\textbackslash{}raggedleft},
\LaTeXTT{\textbackslash{}raggedright} or
\LaTeXTT{\textbackslash{}centering}
at some point and forgotten to switch it off. These commands are switches—they remain active until the end of the scope, or until the end of the document if there is no scope. See \mylref{132}{Paragraph Alignment} for more information.
\section{Rules of punctuation and spacing}
\label{957}

LaTeX does some work for you, but not everything. Especially regarding punctuation, you are pretty free to do what you want. Punctuation rules are different for each language. In English there is {\itshape \setmainfont[Path=/usr/share/fonts/truetype/cmu/,UprightFont=cmunrm.ttf,BoldFont=cmunbx.ttf,ItalicFont=cmunti.ttf,BoldItalicFont=cmunbi.ttf]{cmunti.ttf}\setmonofont[Path=/usr/share/fonts/truetype/cmu/,UprightFont=cmuntt.ttf,BoldFont=cmuntb.ttf,ItalicFont=cmunit.ttf,BoldItalicFont=cmuntx.ttf]{cmunti.ttf}\itshape no}{$\text{ }$}\setmainfont[Path=/usr/share/fonts/truetype/cmu/,UprightFont=cmunrm.ttf,BoldFont=cmunbx.ttf,ItalicFont=cmunti.ttf,BoldItalicFont=cmunbi.ttf]{cmunrm.ttf}\setmonofont[Path=/usr/share/fonts/truetype/cmu/,UprightFont=cmuntt.ttf,BoldFont=cmuntb.ttf,ItalicFont=cmunit.ttf,BoldItalicFont=cmuntx.ttf]{cmunrm.ttf} space before a punctuation mark and one space after it.

There are a lot of rules, but you can have a quick look at \myhref{https://en.wikipedia.org/wiki/Punctuation\%23Conventional_styles_of_English_punctuation}{Wikipedia}.
\section{Compilation fails after a Babel language change}
\label{958}
This is a limitation of Babel. Delete the {\ttfamily \setmainfont[Path=/usr/share/fonts/truetype/cmu/,UprightFont=cmunrm.ttf,BoldFont=cmunbx.ttf,ItalicFont=cmunti.ttf,BoldItalicFont=cmunbi.ttf]{cmuntt.ttf}\setmonofont[Path=/usr/share/fonts/truetype/cmu/,UprightFont=cmuntt.ttf,BoldFont=cmuntb.ttf,ItalicFont=cmunit.ttf,BoldItalicFont=cmuntx.ttf]{cmuntt.ttf}\ttfamily .aux}{$\text{ }$}\setmainfont[Path=/usr/share/fonts/truetype/cmu/,UprightFont=cmunrm.ttf,BoldFont=cmunbx.ttf,ItalicFont=cmunti.ttf,BoldItalicFont=cmunbi.ttf]{cmunrm.ttf}\setmonofont[Path=/usr/share/fonts/truetype/cmu/,UprightFont=cmuntt.ttf,BoldFont=cmuntb.ttf,ItalicFont=cmunit.ttf,BoldItalicFont=cmuntx.ttf]{cmunrm.ttf} file (or clean the project), then try compiling again.
\section{Learning LaTeX quickly or correctly}
\label{959}

Nowadays it is very common to “learn” on the web by using a search engine and copying and pasting things here and there. As with every programming language, this is generally a poor method which will lead to lack of control, unexpected results, and a lot of frustration. Really learning LaTeX is not that difficult and does not take that long. Most chapters in this book are dedicated to a specific usage, so the basics are actually covered very quickly.

If you are getting frustrated with a specific package, make sure you read its official documentation, which is usually the best source of information. Content found on the web, even in this book, is rarely as accurate as the official documentation. Inaccurate information might result in causing mistakes without you understanding why.

The time you spend learning is worth it, and it quickly makes up for the time you would lose if you don\textquotesingle{}t learn things properly and end up stuck all the time.
\section{Non-{}breaking spaces}
\label{960}

This useful feature is unknown to most newcomers, although it is available on most WYSIWYG document processors. A non-{}breaking space between two tokens (e.g. words, punctuation marks) prevents processors from inserting a line break between them. It is very important for consistent reading. LaTeX uses the \textquotesingle{}\~{}\textquotesingle{} symbol as a non-{}breaking space.

You usually use non-{}breaking spaces for punctuation marks in some languages, for units and currencies, for initials, etc.

For example, in French typography, you put a non-{}breaking space before all two-{}parts punctuation marks.
Example:

\begin{Shaded}
\begin{Highlighting}[]

\NormalTok{Il répondit~: «~Ce pain coûte-t-il 2~€~?~»}
\end{Highlighting}
\end{Shaded}


Note that writing French like this might get really painful. Thankfully, Babel with the \LaTeXTT{frenchb} option will take care of the non-{}breaking spaces for all punctuation marks. In the above example, only the non-{}breaking space for the euro symbol must remain.
\section{Smart mathematics}
\label{961}

All virtual objects designated by letters, variables or others should use a dedicated formatting. For math and a lot of other fields, the LaTeX math formatting is perfect. For instance, if you want to refer to an object {\itshape \setmainfont[Path=/usr/share/fonts/truetype/cmu/,UprightFont=cmunrm.ttf,BoldFont=cmunbx.ttf,ItalicFont=cmunti.ttf,BoldItalicFont=cmunbi.ttf]{cmunti.ttf}\setmonofont[Path=/usr/share/fonts/truetype/cmu/,UprightFont=cmuntt.ttf,BoldFont=cmuntb.ttf,ItalicFont=cmunit.ttf,BoldItalicFont=cmuntx.ttf]{cmunti.ttf}\itshape A}\setmainfont[Path=/usr/share/fonts/truetype/cmu/,UprightFont=cmunrm.ttf,BoldFont=cmunbx.ttf,ItalicFont=cmunti.ttf,BoldItalicFont=cmunbi.ttf]{cmunrm.ttf}\setmonofont[Path=/usr/share/fonts/truetype/cmu/,UprightFont=cmuntt.ttf,BoldFont=cmuntb.ttf,ItalicFont=cmunit.ttf,BoldItalicFont=cmuntx.ttf]{cmunrm.ttf}, write

\begin{Shaded}
\begin{Highlighting}[]

\NormalTok{Speaking of $A$, let's say...}
\end{Highlighting}
\end{Shaded}


If you want to refer to several objects in a sentence, it is the same.
\begin{Shaded}
\begin{Highlighting}[]

\NormalTok{Speaking of $A$, $B$ and $C$...}
\end{Highlighting}
\end{Shaded}


If you refer to a set of objects, you can still use the math notation.
\begin{Shaded}
\begin{Highlighting}[]

\NormalTok{The family $(A, B, C)$ is...}
\end{Highlighting}
\end{Shaded}


Note that this is different from usual text parentheses.
\begin{Shaded}
\begin{Highlighting}[]

\NormalTok{A sentence. ($A$, $B$, and $C$ are not concerned, but we do not mean the $(A, B,}
 \NormalTok{C)$ family.)}
\end{Highlighting}
\end{Shaded}

\section{Use vector graphics rather than raster images}
\label{962}

Raster (bitmap) graphics scale poorly and often create jagged or low-{}quality results which clash with the document quality, particularly when printed.

Using vector (line-{}oriented) graphics instead, either through LaTeX\textquotesingle{}s native diagramming tools or by exporting vector formats from your drawing or diagramming tools, will produce much higher quality results. When possible, you should prefer PDF, EPS, or SVG graphics over PNG or JPG.
\section{Stretching tables}
\label{963}

Trying to stretch tables with the default \LaTeXTT{tabular} environment will often lead to unexpected results.
The nice \LaTeXTT{tabu} package will do what you want and even much more. Alternatively if you cannot use the \LaTeXTT{tabu} package you may try \LaTeXTT{tabularx} or \LaTeXTT{tabulary} packages See \mylref{248}{Tables}.
\section{Tables are easier than you think}
\label{964}

Even though the \mylref{248}{Tables} chapter is quite long, it is worth reading. In the end, you only need to know a few things about the environment of your choice.

Some LaTeX editors feature table assistants. Also, many spreadsheet applications have a LaTeX export feature (or plugin). Again, see  \mylref{248}{Tables} for more details.
\section{Relieving cumbersome code (lists and long command names)}
\label{965}

LaTeX is sometimes cumbersome to write, especially if you are not using an adequate editor. See \mylref{22}{Editors} for some interesting choices.

You can define aliases to shorten some commands:
\begin{Shaded}
\begin{Highlighting}[]

\NormalTok{\textbackslash{}usepackage\{xspace\}}
\NormalTok{\textbackslash{}newcommand\textbackslash{}tss[1]\{\textbackslash{}textsuperscript\{#1\}\}}
\NormalTok{\textbackslash{}newcommand\textbackslash{}tbs[1]\{\textbackslash{}textbackslash\textbackslash{}xspace\}}
\end{Highlighting}
\end{Shaded}

Here the \LaTeXTT{xspace} package comes in handy to avoid swallowed spaces.

For lists you may want to try the \LaTeXTT{easylist} package. Now writing a list is as simple as
\begin{Shaded}
\begin{Highlighting}[]

\NormalTok{\textbackslash{}usepackage[ampersand]\{easylist\}}
\CommentTok{% ...}
 
\NormalTok{\textbackslash{}begin\{easylist\}}
\NormalTok{& Item 1}
\NormalTok{& Item 2}
\NormalTok{&& Subitem 1}
\NormalTok{&&& Subsubitem 1}
\NormalTok{& Item 3}
\NormalTok{&& Subitem 1}
\NormalTok{\textbackslash{}end\{easylist\}}
\end{Highlighting}
\end{Shaded}

\section{Reducing the size of your LaTeX installation}
\label{966}

The \mylref{10}{Installation} article explains in detail how to manually install a fully functional TeX environment,  including LaTeX and other features, in under 100 MB.


\chapter{Tips and Tricks}

\myminitoc
\label{967}

\label{968}

\section{Always writing LaTeX in roman}
\label{969}
If you insert the {\ttfamily \setmainfont[Path=/usr/share/fonts/truetype/cmu/,UprightFont=cmunrm.ttf,BoldFont=cmunbx.ttf,ItalicFont=cmunti.ttf,BoldItalicFont=cmunbi.ttf]{cmuntt.ttf}\setmonofont[Path=/usr/share/fonts/truetype/cmu/,UprightFont=cmuntt.ttf,BoldFont=cmuntb.ttf,ItalicFont=cmunit.ttf,BoldItalicFont=cmuntx.ttf]{cmuntt.ttf}\ttfamily \textbackslash{}LaTeX}{$\text{ }$}\setmainfont[Path=/usr/share/fonts/truetype/cmu/,UprightFont=cmunrm.ttf,BoldFont=cmunbx.ttf,ItalicFont=cmunti.ttf,BoldItalicFont=cmunbi.ttf]{cmunrm.ttf}\setmonofont[Path=/usr/share/fonts/truetype/cmu/,UprightFont=cmuntt.ttf,BoldFont=cmuntb.ttf,ItalicFont=cmunit.ttf,BoldItalicFont=cmuntx.ttf]{cmunrm.ttf} command in an area with a non-{}default font, it will be formatted accordingly. If you want to keep LaTeX written in Computer Modern roman shape, you must redefine the function.
However, the naive

\begin{Shaded}
\begin{Highlighting}[]

\NormalTok{\textbackslash{}renewcommand\{\textbackslash{}LaTeX\}\{\{\textbackslash{}rm\ensuremath{\text{ }}\textbackslash{}LaTeX\}\}}\newline
\end{Highlighting}
\end{Shaded}

will output:

\begin{Shaded}
\begin{Highlighting}[]

\NormalTok{TeX\ensuremath{\text{ }}capacity\ensuremath{\text{ }}exceeded\ensuremath{\text{ }},\ensuremath{\text{ }}sorry\ensuremath{\text{ }}[\ensuremath{\text{ }}grouping\ensuremath{\text{ }}levels\ensuremath{\text{ }}=255].}\newline
\end{Highlighting}
\end{Shaded}

So you need to create a temporary variable.

Sadly,

\begin{Shaded}
\begin{Highlighting}[]

\NormalTok{\textbackslash{}newcommand\{\textbackslash{}LaTeXtemp\}\{\textbackslash{}LaTeX\}}\newline
\NormalTok{\textbackslash{}renewcommand\{\textbackslash{}LaTeX\}\{\{\textbackslash{}rm\ensuremath{\text{ }}\textbackslash{}LaTeXtemp\}\}}\newline
\end{Highlighting}
\end{Shaded}

does not work as well.

We must use the TeX primitive {\ttfamily \setmainfont[Path=/usr/share/fonts/truetype/cmu/,UprightFont=cmunrm.ttf,BoldFont=cmunbx.ttf,ItalicFont=cmunti.ttf,BoldItalicFont=cmunbi.ttf]{cmuntt.ttf}\setmonofont[Path=/usr/share/fonts/truetype/cmu/,UprightFont=cmuntt.ttf,BoldFont=cmuntb.ttf,ItalicFont=cmunit.ttf,BoldItalicFont=cmuntx.ttf]{cmuntt.ttf}\ttfamily \textbackslash{}let}{$\text{ }$}\setmainfont[Path=/usr/share/fonts/truetype/cmu/,UprightFont=cmunrm.ttf,BoldFont=cmunbx.ttf,ItalicFont=cmunti.ttf,BoldItalicFont=cmunbi.ttf]{cmunrm.ttf}\setmonofont[Path=/usr/share/fonts/truetype/cmu/,UprightFont=cmuntt.ttf,BoldFont=cmuntb.ttf,ItalicFont=cmunit.ttf,BoldItalicFont=cmuntx.ttf]{cmunrm.ttf} instead.

\begin{Shaded}
\begin{Highlighting}[]

\NormalTok{\textbackslash{}let\textbackslash{}LaTeXtemp\textbackslash{}LaTeX}\newline
\NormalTok{\textbackslash{}renewcommand\{\textbackslash{}LaTeX\}\{\{\textbackslash{}rm\ensuremath{\text{ }}\textbackslash{}LaTeXtemp\ensuremath{\text{ }}\}\}}\newline
\end{Highlighting}
\end{Shaded}

\section{{\itshape \setmainfont[Path=/usr/share/fonts/truetype/cmu/,UprightFont=cmunrm.ttf,BoldFont=cmunbx.ttf,ItalicFont=cmunti.ttf,BoldItalicFont=cmunbi.ttf]{cmunti.ttf}\setmonofont[Path=/usr/share/fonts/truetype/cmu/,UprightFont=cmuntt.ttf,BoldFont=cmuntb.ttf,ItalicFont=cmunit.ttf,BoldItalicFont=cmuntx.ttf]{cmunti.ttf}\itshape id est}{$\text{ }$}\setmainfont[Path=/usr/share/fonts/truetype/cmu/,UprightFont=cmunrm.ttf,BoldFont=cmunbx.ttf,ItalicFont=cmunti.ttf,BoldItalicFont=cmunbi.ttf]{cmunrm.ttf}\setmonofont[Path=/usr/share/fonts/truetype/cmu/,UprightFont=cmuntt.ttf,BoldFont=cmuntb.ttf,ItalicFont=cmunit.ttf,BoldItalicFont=cmuntx.ttf]{cmunrm.ttf} and {\itshape \setmainfont[Path=/usr/share/fonts/truetype/cmu/,UprightFont=cmunrm.ttf,BoldFont=cmunbx.ttf,ItalicFont=cmunti.ttf,BoldItalicFont=cmunbi.ttf]{cmunti.ttf}\setmonofont[Path=/usr/share/fonts/truetype/cmu/,UprightFont=cmuntt.ttf,BoldFont=cmuntb.ttf,ItalicFont=cmunit.ttf,BoldItalicFont=cmuntx.ttf]{cmunti.ttf}\itshape exempli gratia}{$\text{ }$}\setmainfont[Path=/usr/share/fonts/truetype/cmu/,UprightFont=cmunrm.ttf,BoldFont=cmunbx.ttf,ItalicFont=cmunti.ttf,BoldItalicFont=cmunbi.ttf]{cmunrm.ttf}\setmonofont[Path=/usr/share/fonts/truetype/cmu/,UprightFont=cmuntt.ttf,BoldFont=cmuntb.ttf,ItalicFont=cmunit.ttf,BoldItalicFont=cmuntx.ttf]{cmunrm.ttf} (i.e. and e.g.)}
\label{970}
If you simply use the forms \symbol{34}{\ttfamily \setmainfont[Path=/usr/share/fonts/truetype/cmu/,UprightFont=cmunrm.ttf,BoldFont=cmunbx.ttf,ItalicFont=cmunti.ttf,BoldItalicFont=cmunbi.ttf]{cmuntt.ttf}\setmonofont[Path=/usr/share/fonts/truetype/cmu/,UprightFont=cmuntt.ttf,BoldFont=cmuntb.ttf,ItalicFont=cmunit.ttf,BoldItalicFont=cmuntx.ttf]{cmuntt.ttf}\ttfamily i.e.}\setmainfont[Path=/usr/share/fonts/truetype/cmu/,UprightFont=cmunrm.ttf,BoldFont=cmunbx.ttf,ItalicFont=cmunti.ttf,BoldItalicFont=cmunbi.ttf]{cmunrm.ttf}\setmonofont[Path=/usr/share/fonts/truetype/cmu/,UprightFont=cmuntt.ttf,BoldFont=cmuntb.ttf,ItalicFont=cmunit.ttf,BoldItalicFont=cmuntx.ttf]{cmunrm.ttf}\symbol{34} or \symbol{34}{\ttfamily \setmainfont[Path=/usr/share/fonts/truetype/cmu/,UprightFont=cmunrm.ttf,BoldFont=cmunbx.ttf,ItalicFont=cmunti.ttf,BoldItalicFont=cmunbi.ttf]{cmuntt.ttf}\setmonofont[Path=/usr/share/fonts/truetype/cmu/,UprightFont=cmuntt.ttf,BoldFont=cmuntb.ttf,ItalicFont=cmunit.ttf,BoldItalicFont=cmuntx.ttf]{cmuntt.ttf}\ttfamily e.g.}\setmainfont[Path=/usr/share/fonts/truetype/cmu/,UprightFont=cmunrm.ttf,BoldFont=cmunbx.ttf,ItalicFont=cmunti.ttf,BoldItalicFont=cmunbi.ttf]{cmunrm.ttf}\setmonofont[Path=/usr/share/fonts/truetype/cmu/,UprightFont=cmuntt.ttf,BoldFont=cmuntb.ttf,ItalicFont=cmunit.ttf,BoldItalicFont=cmuntx.ttf]{cmunrm.ttf}\symbol{34}, LaTeX will treat the periods as end of sentence periods (i.e. \myhref{https://en.wikipedia.org/wiki/Full\%20stop}{full stop}) since they are followed by a space, and add more space before the next \symbol{34}sentence\symbol{34}. To prevent LaTeX from adding space after the last period, the correct syntax is either \symbol{34}{\ttfamily \setmainfont[Path=/usr/share/fonts/truetype/cmu/,UprightFont=cmunrm.ttf,BoldFont=cmunbx.ttf,ItalicFont=cmunti.ttf,BoldItalicFont=cmunbi.ttf]{cmuntt.ttf}\setmonofont[Path=/usr/share/fonts/truetype/cmu/,UprightFont=cmuntt.ttf,BoldFont=cmuntb.ttf,ItalicFont=cmunit.ttf,BoldItalicFont=cmuntx.ttf]{cmuntt.ttf}\ttfamily i.e.\textbackslash{}}\setmainfont[Path=/usr/share/fonts/truetype/cmu/,UprightFont=cmunrm.ttf,BoldFont=cmunbx.ttf,ItalicFont=cmunti.ttf,BoldItalicFont=cmunbi.ttf]{cmunrm.ttf}\setmonofont[Path=/usr/share/fonts/truetype/cmu/,UprightFont=cmuntt.ttf,BoldFont=cmuntb.ttf,ItalicFont=cmunit.ttf,BoldItalicFont=cmuntx.ttf]{cmunrm.ttf}\symbol{34} or \symbol{34}{\ttfamily \setmainfont[Path=/usr/share/fonts/truetype/cmu/,UprightFont=cmunrm.ttf,BoldFont=cmunbx.ttf,ItalicFont=cmunti.ttf,BoldItalicFont=cmunbi.ttf]{cmuntt.ttf}\setmonofont[Path=/usr/share/fonts/truetype/cmu/,UprightFont=cmuntt.ttf,BoldFont=cmuntb.ttf,ItalicFont=cmunit.ttf,BoldItalicFont=cmuntx.ttf]{cmuntt.ttf}\ttfamily e.g.\textbackslash{}}\setmainfont[Path=/usr/share/fonts/truetype/cmu/,UprightFont=cmunrm.ttf,BoldFont=cmunbx.ttf,ItalicFont=cmunti.ttf,BoldItalicFont=cmunbi.ttf]{cmunrm.ttf}\setmonofont[Path=/usr/share/fonts/truetype/cmu/,UprightFont=cmuntt.ttf,BoldFont=cmuntb.ttf,ItalicFont=cmunit.ttf,BoldItalicFont=cmuntx.ttf]{cmunrm.ttf}\symbol{34}.

Depending on style (e.g., {\itshape \myhref{https://en.wikipedia.org/wiki/The\%20Chicago\%20Manual\%20of\%20Style}{\setmainfont[Path=/usr/share/fonts/truetype/cmu/,UprightFont=cmunrm.ttf,BoldFont=cmunbx.ttf,ItalicFont=cmunti.ttf,BoldItalicFont=cmunbi.ttf]{cmunti.ttf}\setmonofont[Path=/usr/share/fonts/truetype/cmu/,UprightFont=cmuntt.ttf,BoldFont=cmuntb.ttf,ItalicFont=cmunit.ttf,BoldItalicFont=cmuntx.ttf]{cmunti.ttf}\itshape The Chicago Manual of Style}}), a comma can be used afterwards, which is interpreted by LaTeX as part of a sentence, since the period is not followed by any space. In this case, \symbol{34}{\ttfamily \setmainfont[Path=/usr/share/fonts/truetype/cmu/,UprightFont=cmunrm.ttf,BoldFont=cmunbx.ttf,ItalicFont=cmunti.ttf,BoldItalicFont=cmunbi.ttf]{cmuntt.ttf}\setmonofont[Path=/usr/share/fonts/truetype/cmu/,UprightFont=cmuntt.ttf,BoldFont=cmuntb.ttf,ItalicFont=cmunit.ttf,BoldItalicFont=cmuntx.ttf]{cmuntt.ttf}\ttfamily i.e.,}\setmainfont[Path=/usr/share/fonts/truetype/cmu/,UprightFont=cmunrm.ttf,BoldFont=cmunbx.ttf,ItalicFont=cmunti.ttf,BoldItalicFont=cmunbi.ttf]{cmunrm.ttf}\setmonofont[Path=/usr/share/fonts/truetype/cmu/,UprightFont=cmuntt.ttf,BoldFont=cmuntb.ttf,ItalicFont=cmunit.ttf,BoldItalicFont=cmuntx.ttf]{cmunrm.ttf}\symbol{34} and \symbol{34}{\ttfamily \setmainfont[Path=/usr/share/fonts/truetype/cmu/,UprightFont=cmunrm.ttf,BoldFont=cmunbx.ttf,ItalicFont=cmunti.ttf,BoldItalicFont=cmunbi.ttf]{cmuntt.ttf}\setmonofont[Path=/usr/share/fonts/truetype/cmu/,UprightFont=cmuntt.ttf,BoldFont=cmuntb.ttf,ItalicFont=cmunit.ttf,BoldItalicFont=cmuntx.ttf]{cmuntt.ttf}\ttfamily e.g.,}\setmainfont[Path=/usr/share/fonts/truetype/cmu/,UprightFont=cmunrm.ttf,BoldFont=cmunbx.ttf,ItalicFont=cmunti.ttf,BoldItalicFont=cmunbi.ttf]{cmunrm.ttf}\setmonofont[Path=/usr/share/fonts/truetype/cmu/,UprightFont=cmuntt.ttf,BoldFont=cmuntb.ttf,ItalicFont=cmunit.ttf,BoldItalicFont=cmuntx.ttf]{cmunrm.ttf}\symbol{34} do not need any special attention.

If the command {\ttfamily \setmainfont[Path=/usr/share/fonts/truetype/cmu/,UprightFont=cmunrm.ttf,BoldFont=cmunbx.ttf,ItalicFont=cmunti.ttf,BoldItalicFont=cmunbi.ttf]{cmuntt.ttf}\setmonofont[Path=/usr/share/fonts/truetype/cmu/,UprightFont=cmuntt.ttf,BoldFont=cmuntb.ttf,ItalicFont=cmunit.ttf,BoldItalicFont=cmuntx.ttf]{cmuntt.ttf}\ttfamily \textbackslash{}frenchspacing}{$\text{ }$}\setmainfont[Path=/usr/share/fonts/truetype/cmu/,UprightFont=cmunrm.ttf,BoldFont=cmunbx.ttf,ItalicFont=cmunti.ttf,BoldItalicFont=cmunbi.ttf]{cmunrm.ttf}\setmonofont[Path=/usr/share/fonts/truetype/cmu/,UprightFont=cmuntt.ttf,BoldFont=cmuntb.ttf,ItalicFont=cmunit.ttf,BoldItalicFont=cmuntx.ttf]{cmunrm.ttf} is used in the preamble, the space between sentences is always consistent.
\section{Grouping Figure/Equation Numbering by Section}
\label{971}

For long documents the numbering can become cumbersome as the numbers reach into double and triple digits. To reset the counters at the start of each section and prefix the numbers by the section number, include the following in the preamble.


\begin{Shaded}
\begin{Highlighting}[]

\NormalTok{\textbackslash{}usepackage\{amsmath\}}\newline
\NormalTok{\textbackslash{}numberwithin\{equation\}\{section\}}\newline
\NormalTok{\textbackslash{}numberwithin\{figure\}\{section\}}\newline
\end{Highlighting}
\end{Shaded}


The same can be done with similar counter types and document units such as \symbol{34}subsection\symbol{34}.


\section{Graphics and Graph editors}
\label{972}
\subsection{Vector image editors with LaTeX support}
\label{973}
It is often preferable to use the same font and font size in your images as in the document. Moreover, for scientific images, you may need mathematical formulae or special characters (such as Greek letters). Both things can be achieved easily if the image editor allows you to use LaTeX code in your image. Most vector image editors do not offer this option. There are, however, a few exceptions.

In early days, LaTeX users used \myhref{https://en.wikipedia.org/wiki/Xfig}{Xfig} for their drawings. The editor is still used by quite a few people nowadays because it has special \textquotesingle{}export to LaTeX\textquotesingle{} features. It also gives you some very basic ways of encapsulating LaTeX text and math in the image (setting the text\textquotesingle{}s \textquotesingle{}special flag\textquotesingle{} to \textquotesingle{}special\textquotesingle{} instead of \textquotesingle{}normal\textquotesingle{}). When exporting, all LaTeX text will be put in a .tex-{}file, separately from the rest of the image (which is put in a .ps file). 

A newer and easier-{}to-{}use vector image editor specially tailored to LaTeX use is \myhref{https://en.wikipedia.org/wiki/Ipe_\%28program\%29}{IPE}. It allows any LaTeX command, including but not limited to mathematical formulae in the image. The program saves its files as editable .eps or .pdf files, which eliminates the need of exporting your image each time you have edited it. 

A very versatile vector image editor is \myhref{https://en.wikipedia.org/wiki/Inkscape}{Inkscape}. It does not support LaTeX text by itself, but you can use the plugin \myhref{http://pav.iki.fi/software/textext/}{Textext} for that. This allows you to put any block of LaTeX code in your image. Additionally since version 0.48 you can export to vectorgraphics with texts separated in a .tex file. Using this way text is rendered by the latex compiler itself.

LaTeXDraw is a free and open source graphical PSTricks generator and editor. It allows you to draw basic geometric objects and save the result in a variety of formats including .jpg, .png, .eps, .bmp as well as .tex. In the last case the saved file contains PSTricks/LaTeX code only. Owing to that you can include any possible LaTeX code in the picture, since the file is rendered by your LaTeX environment directly.

Another way to generate vectorgraphics is using the \myhref{https://en.wikipedia.org/wiki/Asymptote_\%28vector_graphics_language\%29}{Asymptote} language. It is a programming language which produces vector images in encapsulated postscript format and supports LaTeX syntax in any textlabels.
\subsection{Graphs with gnuplot}
\label{974}
A simple method to include graphs and charts in LaTeX documents is to create it within a common spreadsheet software (OpenOffice Calc or MS Office Excel etc.) and include it in the document as a cropped screenshot. However, this produces poor quality rasterized images. Calc also allows you to copy-{}paste the charts into OpenOffice Draw and save them as PDF files.

Using Microsoft Excel 2010, charts can be copied directly to Microsoft Expression Design 4, where they can be saved as PDF files. These PDF files can be included in LaTeX. This method produces high quality vectorized images.

An excellent method to render graphs is through {\bfseries \myhref{https://en.wikipedia.org/wiki/gnuplot}{\setmainfont[Path=/usr/share/fonts/truetype/cmu/,UprightFont=cmunrm.ttf,BoldFont=cmunbx.ttf,ItalicFont=cmunti.ttf,BoldItalicFont=cmunbi.ttf]{cmunbx.ttf}\setmonofont[Path=/usr/share/fonts/truetype/cmu/,UprightFont=cmuntt.ttf,BoldFont=cmuntb.ttf,ItalicFont=cmunit.ttf,BoldItalicFont=cmuntx.ttf]{cmunbx.ttf}\bfseries gnuplot}}, a free and versatile plotting software, that has a special output filter directly for exporting files to LaTeX. We assume, that the data is in a CSV file (comma separated text) in the first and third column. A simple gnuplot script to plot the data can look like this:


\begin{minipage}{1.0\linewidth}
\begin{center}
\includegraphics[width=1.0\linewidth,height=6.5in,keepaspectratio]{../images/218.\SVGExtension}
\end{center}
\raggedright{}\myfigurewithcaption{218}{gnuplot can plot various numerical data, functions, error distribution as well as 3D graphs and surfaces}
\end{minipage}\vspace{0.75cm}

\\

\TemplateSpaceIndent{$\text{ }${}set$\text{ }${}format$\text{ }${}\symbol{34}\${}\%g\${}\symbol{34}$\text{ }$\newline{}
$\text{ }${}set$\text{ }${}title$\text{ }${}\symbol{34}Graph$\text{ }${}3:$\text{ }${}Dependence$\text{ }${}of$\text{ }${}\${}V\_p\${}$\text{ }${}on$\text{ }${}\${}R\_0\${}\symbol{34}$\text{ }$\newline{}
$\text{ }${}set$\text{ }${}xlabel$\text{ }${}\symbol{34}Resistance$\text{ }${}\${}R\_0\${}$\text{ }${}{$\text{[}$}\${}\textbackslash{}Omega\${}{$\text{]}$}\symbol{34}$\text{ }$\newline{}
$\text{ }${}set$\text{ }${}ylabel$\text{ }${}\symbol{34}Voltage$\text{ }${}\${}V\_p\${}$\text{ }${}{$\text{[}$}V{$\text{]}$}\symbol{34}$\text{ }$\newline{}
$\text{ }${}set$\text{ }${}border$\text{ }${}3$\text{ }$\newline{}
$\text{ }${}set$\text{ }${}xtics$\text{ }${}nomirror$\text{ }$\newline{}
$\text{ }${}set$\text{ }${}ytics$\text{ }${}nomirror$\text{ }$\newline{}
$\text{ }${}set$\text{ }${}terminal$\text{ }${}epslatex$\text{ }$\newline{}
$\text{ }${}set$\text{ }${}output$\text{ }${}\symbol{34}graph1.eps\symbol{34}$\text{ }$\newline{}
$\text{ }${}plot$\text{ }${}\symbol{34}graph1.csv\symbol{34}$\text{ }${}using$\text{ }${}1:3$\text{ }${}$\text{ }${}$\text{ }${}\#Plot$\text{ }${}the$\text{ }${}data}


Now gnuplot produces two files: the graph drawing in {\ttfamily \setmainfont[Path=/usr/share/fonts/truetype/cmu/,UprightFont=cmunrm.ttf,BoldFont=cmunbx.ttf,ItalicFont=cmunti.ttf,BoldItalicFont=cmunbi.ttf]{cmuntt.ttf}\setmonofont[Path=/usr/share/fonts/truetype/cmu/,UprightFont=cmuntt.ttf,BoldFont=cmuntb.ttf,ItalicFont=cmunit.ttf,BoldItalicFont=cmuntx.ttf]{cmuntt.ttf}\ttfamily graph.eps}{$\text{ }$}\setmainfont[Path=/usr/share/fonts/truetype/cmu/,UprightFont=cmunrm.ttf,BoldFont=cmunbx.ttf,ItalicFont=cmunti.ttf,BoldItalicFont=cmunbi.ttf]{cmunrm.ttf}\setmonofont[Path=/usr/share/fonts/truetype/cmu/,UprightFont=cmuntt.ttf,BoldFont=cmuntb.ttf,ItalicFont=cmunit.ttf,BoldItalicFont=cmuntx.ttf]{cmunrm.ttf} and the text in {\ttfamily \setmainfont[Path=/usr/share/fonts/truetype/cmu/,UprightFont=cmunrm.ttf,BoldFont=cmunbx.ttf,ItalicFont=cmunti.ttf,BoldItalicFont=cmunbi.ttf]{cmuntt.ttf}\setmonofont[Path=/usr/share/fonts/truetype/cmu/,UprightFont=cmuntt.ttf,BoldFont=cmuntb.ttf,ItalicFont=cmunit.ttf,BoldItalicFont=cmuntx.ttf]{cmuntt.ttf}\ttfamily graph.tex}\setmainfont[Path=/usr/share/fonts/truetype/cmu/,UprightFont=cmunrm.ttf,BoldFont=cmunbx.ttf,ItalicFont=cmunti.ttf,BoldItalicFont=cmunbi.ttf]{cmunrm.ttf}\setmonofont[Path=/usr/share/fonts/truetype/cmu/,UprightFont=cmuntt.ttf,BoldFont=cmuntb.ttf,ItalicFont=cmunit.ttf,BoldItalicFont=cmuntx.ttf]{cmunrm.ttf}. The second includes the EPS image, so that we only need to include the file graph.tex in our document:
\\

\TemplateSpaceIndent{$\text{ }${}\textbackslash{}input\{graph1.tex\}}


The above steps can be automated by the package gnuplottex. By placing gnuplot commands inside \textbackslash{}begin\{gnuplot\}\textbackslash{}end\{gnuplot\}, and compiling with latex -{}shell-{}escape, the graphs are created and added into your document.

Failure to access gnuplot from latex for Windows can be solved by making file title only in one word. Don\textquotesingle{}t type {\bfseries \setmainfont[Path=/usr/share/fonts/truetype/cmu/,UprightFont=cmunrm.ttf,BoldFont=cmunbx.ttf,ItalicFont=cmunti.ttf,BoldItalicFont=cmunbi.ttf]{cmunbx.ttf}\setmonofont[Path=/usr/share/fonts/truetype/cmu/,UprightFont=cmuntt.ttf,BoldFont=cmuntb.ttf,ItalicFont=cmunit.ttf,BoldItalicFont=cmuntx.ttf]{cmunbx.ttf}\bfseries my report.tex}{$\text{ }$}\setmainfont[Path=/usr/share/fonts/truetype/cmu/,UprightFont=cmunrm.ttf,BoldFont=cmunbx.ttf,ItalicFont=cmunti.ttf,BoldItalicFont=cmunbi.ttf]{cmunrm.ttf}\setmonofont[Path=/usr/share/fonts/truetype/cmu/,UprightFont=cmuntt.ttf,BoldFont=cmuntb.ttf,ItalicFont=cmunit.ttf,BoldItalicFont=cmuntx.ttf]{cmunrm.ttf} for your title file, but do {\bfseries \setmainfont[Path=/usr/share/fonts/truetype/cmu/,UprightFont=cmunrm.ttf,BoldFont=cmunbx.ttf,ItalicFont=cmunti.ttf,BoldItalicFont=cmunbi.ttf]{cmunbx.ttf}\setmonofont[Path=/usr/share/fonts/truetype/cmu/,UprightFont=cmuntt.ttf,BoldFont=cmuntb.ttf,ItalicFont=cmunit.ttf,BoldItalicFont=cmuntx.ttf]{cmunbx.ttf}\bfseries myreport.tex}{$\text{ }$}\setmainfont[Path=/usr/share/fonts/truetype/cmu/,UprightFont=cmunrm.ttf,BoldFont=cmunbx.ttf,ItalicFont=cmunti.ttf,BoldItalicFont=cmunbi.ttf]{cmunrm.ttf}\setmonofont[Path=/usr/share/fonts/truetype/cmu/,UprightFont=cmuntt.ttf,BoldFont=cmuntb.ttf,ItalicFont=cmunit.ttf,BoldItalicFont=cmuntx.ttf]{cmunrm.ttf}  .

When you are using gnuplottex it is also possible to directly pass the terminal settings as an argument to the environment\\

\TemplateSpaceIndent{$\text{ }${}$\text{ }${}\textbackslash{}begin\{gnuplot\}{$\text{[}$}terminal=epslatex,$\text{ }${}terminaloptions=color,$\text{ }${}scale=0.9,$\text{ }$\newline{}
$\text{ }${}linewidth=2$\text{ }${}{$\text{]}$}$\text{ }$\newline{}
$\text{ }${}$\text{ }${}...$\text{ }$\newline{}
$\text{ }${}$\text{ }${}\textbackslash{}end\{gnuplot\}}


Using gnuplottex can cause fraudulent text-{}highlighting in some editors when using algebraic functions on imported data, such as: \\

\TemplateSpaceIndent{$\text{ }${}$\text{ }${}(2*(\${}1)):2}

Some editors will think of all following text as part of a formula and highlight it as such (because of the \textquotesingle{}\${}\textquotesingle{} that is interpreted as part of the latex code). This can be avoided by ending with:\\

\TemplateSpaceIndent{$\text{ }${}\#\${}$\text{ }$\newline{}
$\text{ }${}\textbackslash{}end\{gnuplot\}}

As it uncomments the dollar sign for the gnuplot interpreter, but is not affecting the interpretation of the .tex by the editor.

When using pdfLaTeX instead of simple LaTeX, we must convert the EPS image to PDF and to substitute the name in the {\ttfamily \setmainfont[Path=/usr/share/fonts/truetype/cmu/,UprightFont=cmunrm.ttf,BoldFont=cmunbx.ttf,ItalicFont=cmunti.ttf,BoldItalicFont=cmunbi.ttf]{cmuntt.ttf}\setmonofont[Path=/usr/share/fonts/truetype/cmu/,UprightFont=cmuntt.ttf,BoldFont=cmuntb.ttf,ItalicFont=cmunit.ttf,BoldItalicFont=cmuntx.ttf]{cmuntt.ttf}\ttfamily graph1.tex}{$\text{ }$}\setmainfont[Path=/usr/share/fonts/truetype/cmu/,UprightFont=cmunrm.ttf,BoldFont=cmunbx.ttf,ItalicFont=cmunti.ttf,BoldItalicFont=cmunbi.ttf]{cmunrm.ttf}\setmonofont[Path=/usr/share/fonts/truetype/cmu/,UprightFont=cmuntt.ttf,BoldFont=cmuntb.ttf,ItalicFont=cmunit.ttf,BoldItalicFont=cmuntx.ttf]{cmunrm.ttf} file.  If we are working with a Unix-{}like shell, it is simply done using:
\\

\TemplateSpaceIndent{$\text{ }${}eps2pdf$\text{ }${}graph1.eps$\text{ }$\newline{}
$\text{ }${}sed$\text{ }${}-{}i$\text{ }${}s/\symbol{34}.eps\symbol{34}/\symbol{34}.pdf\symbol{34}/g$\text{ }${}graph1.tex}


With the included tex file we can work as with an ordinary image.

Instead of calling {\ttfamily \setmainfont[Path=/usr/share/fonts/truetype/cmu/,UprightFont=cmunrm.ttf,BoldFont=cmunbx.ttf,ItalicFont=cmunti.ttf,BoldItalicFont=cmunbi.ttf]{cmuntt.ttf}\setmonofont[Path=/usr/share/fonts/truetype/cmu/,UprightFont=cmuntt.ttf,BoldFont=cmuntb.ttf,ItalicFont=cmunit.ttf,BoldItalicFont=cmuntx.ttf]{cmuntt.ttf}\ttfamily eps2pdf}{$\text{ }$}\setmainfont[Path=/usr/share/fonts/truetype/cmu/,UprightFont=cmunrm.ttf,BoldFont=cmunbx.ttf,ItalicFont=cmunti.ttf,BoldItalicFont=cmunbi.ttf]{cmunrm.ttf}\setmonofont[Path=/usr/share/fonts/truetype/cmu/,UprightFont=cmuntt.ttf,BoldFont=cmuntb.ttf,ItalicFont=cmunit.ttf,BoldItalicFont=cmuntx.ttf]{cmunrm.ttf} directly, we can also include the {\ttfamily \setmainfont[Path=/usr/share/fonts/truetype/cmu/,UprightFont=cmunrm.ttf,BoldFont=cmunbx.ttf,ItalicFont=cmunti.ttf,BoldItalicFont=cmunbi.ttf]{cmuntt.ttf}\setmonofont[Path=/usr/share/fonts/truetype/cmu/,UprightFont=cmuntt.ttf,BoldFont=cmuntb.ttf,ItalicFont=cmunit.ttf,BoldItalicFont=cmuntx.ttf]{cmuntt.ttf}\ttfamily epstopdf}{$\text{ }$}\setmainfont[Path=/usr/share/fonts/truetype/cmu/,UprightFont=cmunrm.ttf,BoldFont=cmunbx.ttf,ItalicFont=cmunti.ttf,BoldItalicFont=cmunbi.ttf]{cmunrm.ttf}\setmonofont[Path=/usr/share/fonts/truetype/cmu/,UprightFont=cmuntt.ttf,BoldFont=cmuntb.ttf,ItalicFont=cmunit.ttf,BoldItalicFont=cmuntx.ttf]{cmunrm.ttf} package that automates the process. If we include a graphics now and leave out the file extension, {\ttfamily \setmainfont[Path=/usr/share/fonts/truetype/cmu/,UprightFont=cmunrm.ttf,BoldFont=cmunbx.ttf,ItalicFont=cmunti.ttf,BoldItalicFont=cmunbi.ttf]{cmuntt.ttf}\setmonofont[Path=/usr/share/fonts/truetype/cmu/,UprightFont=cmuntt.ttf,BoldFont=cmuntb.ttf,ItalicFont=cmunit.ttf,BoldItalicFont=cmuntx.ttf]{cmuntt.ttf}\ttfamily epstopdf}{$\text{ }$}\setmainfont[Path=/usr/share/fonts/truetype/cmu/,UprightFont=cmunrm.ttf,BoldFont=cmunbx.ttf,ItalicFont=cmunti.ttf,BoldItalicFont=cmunbi.ttf]{cmunrm.ttf}\setmonofont[Path=/usr/share/fonts/truetype/cmu/,UprightFont=cmuntt.ttf,BoldFont=cmuntb.ttf,ItalicFont=cmunit.ttf,BoldItalicFont=cmuntx.ttf]{cmunrm.ttf} will automatically transform the .eps-{}file to PDF and insert it in the text.
\\

\TemplateSpaceIndent{$\text{ }${}\textbackslash{}includegraphics\{graph1\}}


This way, if we choose to output to PS or DVI, the EPS version is used and if we output to PDF directly, the converted PDF graphics is used. Please note that usage of {\ttfamily \setmainfont[Path=/usr/share/fonts/truetype/cmu/,UprightFont=cmunrm.ttf,BoldFont=cmunbx.ttf,ItalicFont=cmunti.ttf,BoldItalicFont=cmunbi.ttf]{cmuntt.ttf}\setmonofont[Path=/usr/share/fonts/truetype/cmu/,UprightFont=cmuntt.ttf,BoldFont=cmuntb.ttf,ItalicFont=cmunit.ttf,BoldItalicFont=cmuntx.ttf]{cmuntt.ttf}\ttfamily epstopdf}{$\text{ }$}\setmainfont[Path=/usr/share/fonts/truetype/cmu/,UprightFont=cmunrm.ttf,BoldFont=cmunbx.ttf,ItalicFont=cmunti.ttf,BoldItalicFont=cmunbi.ttf]{cmunrm.ttf}\setmonofont[Path=/usr/share/fonts/truetype/cmu/,UprightFont=cmuntt.ttf,BoldFont=cmuntb.ttf,ItalicFont=cmunit.ttf,BoldItalicFont=cmuntx.ttf]{cmunrm.ttf} requires compiling with latex -{}shell-{}escape.

Note: Emacs AucTex users might want to check out \myhref{http://cars9.uchicago.edu/~ravel/software/gnuplot-mode.html}{Gnuplot-{}mode}.
\subsection{Generate png screenshots}
\label{975}

See \mylref{925}{Export To Other Formats}.
\section{Spell-{}checking and Word Counting}
\label{976}
If you want to spell-{}check your document, you can use the command-{}line {\ttfamily \setmainfont[Path=/usr/share/fonts/truetype/cmu/,UprightFont=cmunrm.ttf,BoldFont=cmunbx.ttf,ItalicFont=cmunti.ttf,BoldItalicFont=cmunbi.ttf]{cmuntt.ttf}\setmonofont[Path=/usr/share/fonts/truetype/cmu/,UprightFont=cmuntt.ttf,BoldFont=cmuntb.ttf,ItalicFont=cmunit.ttf,BoldItalicFont=cmuntx.ttf]{cmuntt.ttf}\ttfamily aspell}\setmainfont[Path=/usr/share/fonts/truetype/cmu/,UprightFont=cmunrm.ttf,BoldFont=cmunbx.ttf,ItalicFont=cmunti.ttf,BoldItalicFont=cmunbi.ttf]{cmunrm.ttf}\setmonofont[Path=/usr/share/fonts/truetype/cmu/,UprightFont=cmuntt.ttf,BoldFont=cmuntb.ttf,ItalicFont=cmunit.ttf,BoldItalicFont=cmuntx.ttf]{cmunrm.ttf}, {\ttfamily \setmainfont[Path=/usr/share/fonts/truetype/cmu/,UprightFont=cmunrm.ttf,BoldFont=cmunbx.ttf,ItalicFont=cmunti.ttf,BoldItalicFont=cmunbi.ttf]{cmuntt.ttf}\setmonofont[Path=/usr/share/fonts/truetype/cmu/,UprightFont=cmuntt.ttf,BoldFont=cmuntb.ttf,ItalicFont=cmunit.ttf,BoldItalicFont=cmuntx.ttf]{cmuntt.ttf}\ttfamily hunspell}{$\text{ }$}\setmainfont[Path=/usr/share/fonts/truetype/cmu/,UprightFont=cmunrm.ttf,BoldFont=cmunbx.ttf,ItalicFont=cmunti.ttf,BoldItalicFont=cmunbi.ttf]{cmunrm.ttf}\setmonofont[Path=/usr/share/fonts/truetype/cmu/,UprightFont=cmuntt.ttf,BoldFont=cmuntb.ttf,ItalicFont=cmunit.ttf,BoldItalicFont=cmuntx.ttf]{cmunrm.ttf} (preferably), or {\ttfamily \setmainfont[Path=/usr/share/fonts/truetype/cmu/,UprightFont=cmunrm.ttf,BoldFont=cmunbx.ttf,ItalicFont=cmunti.ttf,BoldItalicFont=cmunbi.ttf]{cmuntt.ttf}\setmonofont[Path=/usr/share/fonts/truetype/cmu/,UprightFont=cmuntt.ttf,BoldFont=cmuntb.ttf,ItalicFont=cmunit.ttf,BoldItalicFont=cmuntx.ttf]{cmuntt.ttf}\ttfamily ispell}{$\text{ }$}\setmainfont[Path=/usr/share/fonts/truetype/cmu/,UprightFont=cmunrm.ttf,BoldFont=cmunbx.ttf,ItalicFont=cmunti.ttf,BoldItalicFont=cmunbi.ttf]{cmunrm.ttf}\setmonofont[Path=/usr/share/fonts/truetype/cmu/,UprightFont=cmuntt.ttf,BoldFont=cmuntb.ttf,ItalicFont=cmunit.ttf,BoldItalicFont=cmuntx.ttf]{cmunrm.ttf} programs.\\

\TemplateSpaceIndent{$\text{ }${}ispell$\text{ }${}yourfile.tex$\text{ }$\newline{}
$\text{ }${}aspell$\text{ }${}-{}-{}mode=tex$\text{ }${}-{}c$\text{ }${}yourfile.tex$\text{ }$\newline{}
$\text{ }${}hunspell$\text{ }${}-{}l$\text{ }${}-{}t$\text{ }${}-{}i$\text{ }${}utf-{}8$\text{ }${}yourfile.tex}


All three understand LaTeX and will skip LaTeX commands. You can also use a LaTeX editor with built-{}in spell checking, such as \myhref{https://en.wikipedia.org/wiki/LyX}{LyX}, \myhref{https://en.wikipedia.org/wiki/Kile}{Kile}, or \myhref{https://en.wikipedia.org/wiki/Emacs}{Emacs}. Last another option is to \mylref{943}{convert LaTeX source to plain text} and open resulting file in a word processor like OpenOffice.org or KOffice.

If you want to count words you can, again, use LyX or convert your LaTeX source to plain text and use, for example, UNIX {\ttfamily \setmainfont[Path=/usr/share/fonts/truetype/cmu/,UprightFont=cmunrm.ttf,BoldFont=cmunbx.ttf,ItalicFont=cmunti.ttf,BoldItalicFont=cmunbi.ttf]{cmuntt.ttf}\setmonofont[Path=/usr/share/fonts/truetype/cmu/,UprightFont=cmuntt.ttf,BoldFont=cmuntb.ttf,ItalicFont=cmunit.ttf,BoldItalicFont=cmuntx.ttf]{cmuntt.ttf}\ttfamily wc}{$\text{ }$}\setmainfont[Path=/usr/share/fonts/truetype/cmu/,UprightFont=cmunrm.ttf,BoldFont=cmunbx.ttf,ItalicFont=cmunti.ttf,BoldItalicFont=cmunbi.ttf]{cmunrm.ttf}\setmonofont[Path=/usr/share/fonts/truetype/cmu/,UprightFont=cmuntt.ttf,BoldFont=cmuntb.ttf,ItalicFont=cmunit.ttf,BoldItalicFont=cmuntx.ttf]{cmunrm.ttf} command:\\

\TemplateSpaceIndent{$\text{ }${}detex$\text{ }${}yourfile$\text{ }${}|$\text{ }${}wc}


An alternative to the {\ttfamily \setmainfont[Path=/usr/share/fonts/truetype/cmu/,UprightFont=cmunrm.ttf,BoldFont=cmunbx.ttf,ItalicFont=cmunti.ttf,BoldItalicFont=cmunbi.ttf]{cmuntt.ttf}\setmonofont[Path=/usr/share/fonts/truetype/cmu/,UprightFont=cmuntt.ttf,BoldFont=cmuntb.ttf,ItalicFont=cmunit.ttf,BoldItalicFont=cmuntx.ttf]{cmuntt.ttf}\ttfamily detex}{$\text{ }$}\setmainfont[Path=/usr/share/fonts/truetype/cmu/,UprightFont=cmunrm.ttf,BoldFont=cmunbx.ttf,ItalicFont=cmunti.ttf,BoldItalicFont=cmunbi.ttf]{cmunrm.ttf}\setmonofont[Path=/usr/share/fonts/truetype/cmu/,UprightFont=cmuntt.ttf,BoldFont=cmuntb.ttf,ItalicFont=cmunit.ttf,BoldItalicFont=cmuntx.ttf]{cmunrm.ttf} command is the {\ttfamily \setmainfont[Path=/usr/share/fonts/truetype/cmu/,UprightFont=cmunrm.ttf,BoldFont=cmunbx.ttf,ItalicFont=cmunti.ttf,BoldItalicFont=cmunbi.ttf]{cmuntt.ttf}\setmonofont[Path=/usr/share/fonts/truetype/cmu/,UprightFont=cmuntt.ttf,BoldFont=cmuntb.ttf,ItalicFont=cmunit.ttf,BoldItalicFont=cmuntx.ttf]{cmuntt.ttf}\ttfamily pdftotext}{$\text{ }$}\setmainfont[Path=/usr/share/fonts/truetype/cmu/,UprightFont=cmunrm.ttf,BoldFont=cmunbx.ttf,ItalicFont=cmunti.ttf,BoldItalicFont=cmunbi.ttf]{cmunrm.ttf}\setmonofont[Path=/usr/share/fonts/truetype/cmu/,UprightFont=cmuntt.ttf,BoldFont=cmuntb.ttf,ItalicFont=cmunit.ttf,BoldItalicFont=cmuntx.ttf]{cmunrm.ttf} command which extracts an ASCII text file from PDF:
\\

\TemplateSpaceIndent{$\text{ }${}1.$\text{ }${}pdflatex$\text{ }${}yourfile.tex$\text{ }$\newline{}
$\text{ }${}2.$\text{ }${}pdftotext$\text{ }${}yourfile.pdf$\text{ }$\newline{}
$\text{ }${}3.$\text{ }${}wc$\text{ }${}yourfile.txt}

\section{New even page}
\label{977}

In the twoside-{}mode you have the ability to get a new odd-{}side page by: 

\begin{Shaded}
\begin{Highlighting}[]

\NormalTok{\textbackslash{}cleardoublepage}\newline
\end{Highlighting}
\end{Shaded}


However, LaTeX doesn\textquotesingle{}t give you the ability to get a new even-{}side page. The following method opens up this;

The following must be put in your document preamble:


\begin{Shaded}
\begin{Highlighting}[]

\NormalTok{\textbackslash{}usepackage\{ifthen\}}\newline
\ensuremath{\text{ }}\newline
\NormalTok{\textbackslash{}newcommand\{\textbackslash{}newevenside\}\{}\newline
	\NormalTok{\textbackslash{}ifthenelse\{\textbackslash{}isodd\{\textbackslash{}thepage\}\}\{\textbackslash{}newpage\}\{}\newline
	\NormalTok{\textbackslash{}newpage}\newline
\ensuremath{\text{ }}\ensuremath{\text{ }}\ensuremath{\text{ }}\ensuremath{\text{ }}\ensuremath{\text{ }}\ensuremath{\text{ }}\ensuremath{\text{ }}\ensuremath{\text{ }}\NormalTok{\textbackslash{}phantom\{placeholder\}\ensuremath{\text{ }}}\CommentTok{\%\ensuremath{\text{ }}doesn\textquotesingle{}t\ensuremath{\text{ }}appear\ensuremath{\text{ }}on\ensuremath{\text{ }}page}\newline
	\NormalTok{\textbackslash{}thispagestyle\{empty\}\ensuremath{\text{ }}}\CommentTok{\%\ensuremath{\text{ }}if\ensuremath{\text{ }}want\ensuremath{\text{ }}no\ensuremath{\text{ }}header/footer}\newline
	\NormalTok{\textbackslash{}newpage}\newline
	\NormalTok{\}}\newline
\NormalTok{\}}\newline
\end{Highlighting}
\end{Shaded}


To active the new even-{}side page, type the following where you want the new even-{}side:


\begin{Shaded}
\begin{Highlighting}[]

\NormalTok{\textbackslash{}newevenside}\newline
\end{Highlighting}
\end{Shaded}


If the given page is an odd-{}side page, the next new page is subsequently an even-{}side page, and LaTeX will do nothing more than a regular \textbackslash{}newpage. However, if the given page is an even page, LaTeX will make a new (odd) page, put in a placeholder, and make another new (even) page. A crude but effective method.

\section{Sidebar with information}
\label{978}
If you want to put a sidebar with information like copyright and author, you might want to use the {\ttfamily \setmainfont[Path=/usr/share/fonts/truetype/cmu/,UprightFont=cmunrm.ttf,BoldFont=cmunbx.ttf,ItalicFont=cmunti.ttf,BoldItalicFont=cmunbi.ttf]{cmuntt.ttf}\setmonofont[Path=/usr/share/fonts/truetype/cmu/,UprightFont=cmuntt.ttf,BoldFont=cmuntb.ttf,ItalicFont=cmunit.ttf,BoldItalicFont=cmuntx.ttf]{cmuntt.ttf}\ttfamily eso-{}pic}{$\text{ }$}\setmainfont[Path=/usr/share/fonts/truetype/cmu/,UprightFont=cmunrm.ttf,BoldFont=cmunbx.ttf,ItalicFont=cmunti.ttf,BoldItalicFont=cmunbi.ttf]{cmunrm.ttf}\setmonofont[Path=/usr/share/fonts/truetype/cmu/,UprightFont=cmuntt.ttf,BoldFont=cmuntb.ttf,ItalicFont=cmunit.ttf,BoldItalicFont=cmuntx.ttf]{cmunrm.ttf} package.
Example:

\begin{Shaded}
\begin{Highlighting}[]

\NormalTok{\textbackslash{}usepackage\{eso-pic\}}\newline
\NormalTok{...}\newline
\NormalTok{\textbackslash{}AddToShipoutPicture\{}\CommentTok{\%}\newline
\ensuremath{\text{ }}\ensuremath{\text{ }}\NormalTok{\textbackslash{}AtPageLowerLeft\{}\CommentTok{\%}\newline
\ensuremath{\text{ }}\ensuremath{\text{ }}\ensuremath{\text{ }}\ensuremath{\text{ }}\NormalTok{\textbackslash{}rotatebox\{90\}\{}\CommentTok{\%}\newline
\ensuremath{\text{ }}\ensuremath{\text{ }}\ensuremath{\text{ }}\ensuremath{\text{ }}\ensuremath{\text{ }}\ensuremath{\text{ }}\ensuremath{\text{ }}\ensuremath{\text{ }}\NormalTok{\textbackslash{}begin\{minipage\}\{\textbackslash{}paperheight\}}\newline
\ensuremath{\text{ }}\ensuremath{\text{ }}\ensuremath{\text{ }}\ensuremath{\text{ }}\ensuremath{\text{ }}\ensuremath{\text{ }}\ensuremath{\text{ }}\ensuremath{\text{ }}\ensuremath{\text{ }}\ensuremath{\text{ }}\NormalTok{\textbackslash{}centering\textbackslash{}textcopyright~\textbackslash{}today\{\}\ensuremath{\text{ }}Humble\ensuremath{\text{ }}me}\newline
\ensuremath{\text{ }}\ensuremath{\text{ }}\ensuremath{\text{ }}\ensuremath{\text{ }}\ensuremath{\text{ }}\ensuremath{\text{ }}\ensuremath{\text{ }}\ensuremath{\text{ }}\NormalTok{\textbackslash{}end\{minipage\}\ensuremath{\text{ }}}\CommentTok{\%}\newline
\ensuremath{\text{ }}\ensuremath{\text{ }}\ensuremath{\text{ }}\ensuremath{\text{ }}\ensuremath{\text{ }}\ensuremath{\text{ }}\NormalTok{\}}\newline
\ensuremath{\text{ }}\ensuremath{\text{ }}\ensuremath{\text{ }}\ensuremath{\text{ }}\NormalTok{\}\ensuremath{\text{ }}}\CommentTok{\%}\newline
\ensuremath{\text{ }}\ensuremath{\text{ }}\NormalTok{\}}\CommentTok{\%}\newline
\end{Highlighting}
\end{Shaded}


If you want it on one page only, use the starred version of the {\itshape \setmainfont[Path=/usr/share/fonts/truetype/cmu/,UprightFont=cmunrm.ttf,BoldFont=cmunbx.ttf,ItalicFont=cmunti.ttf,BoldItalicFont=cmunbi.ttf]{cmunti.ttf}\setmonofont[Path=/usr/share/fonts/truetype/cmu/,UprightFont=cmuntt.ttf,BoldFont=cmuntb.ttf,ItalicFont=cmunit.ttf,BoldItalicFont=cmuntx.ttf]{cmunti.ttf}\itshape AddToShipoutPicture}{$\text{ }$}\setmainfont[Path=/usr/share/fonts/truetype/cmu/,UprightFont=cmunrm.ttf,BoldFont=cmunbx.ttf,ItalicFont=cmunti.ttf,BoldItalicFont=cmunbi.ttf]{cmunrm.ttf}\setmonofont[Path=/usr/share/fonts/truetype/cmu/,UprightFont=cmuntt.ttf,BoldFont=cmuntb.ttf,ItalicFont=cmunit.ttf,BoldItalicFont=cmuntx.ttf]{cmunrm.ttf} command at the page you want it. ({\ttfamily \setmainfont[Path=/usr/share/fonts/truetype/cmu/,UprightFont=cmunrm.ttf,BoldFont=cmunbx.ttf,ItalicFont=cmunti.ttf,BoldItalicFont=cmunbi.ttf]{cmuntt.ttf}\setmonofont[Path=/usr/share/fonts/truetype/cmu/,UprightFont=cmuntt.ttf,BoldFont=cmuntb.ttf,ItalicFont=cmunit.ttf,BoldItalicFont=cmuntx.ttf]{cmuntt.ttf}\ttfamily \textbackslash{}AddToShipoutPicture*\{...\}}\setmainfont[Path=/usr/share/fonts/truetype/cmu/,UprightFont=cmunrm.ttf,BoldFont=cmunbx.ttf,ItalicFont=cmunti.ttf,BoldItalicFont=cmunbi.ttf]{cmunrm.ttf}\setmonofont[Path=/usr/share/fonts/truetype/cmu/,UprightFont=cmuntt.ttf,BoldFont=cmuntb.ttf,ItalicFont=cmunit.ttf,BoldItalicFont=cmuntx.ttf]{cmunrm.ttf})
\section{Hide auxiliary files}
\label{979}

If you\textquotesingle{}re using pdflatex you can create a folder in which all the output files will be stored, so your top directory looks cleaner.


\begin{Shaded}
\begin{Highlighting}[]

\KeywordTok{pdflatex}\ensuremath{\text{ }}\NormalTok{-output-directory\ensuremath{\text{ }}tmp}\newline
\end{Highlighting}
\end{Shaded}


Please note that the folder tmp should exist. However if you\textquotesingle{}re using a Unix-{}based system you can do something like this:

\begin{Shaded}
\begin{Highlighting}[]

\KeywordTok{alias}\ensuremath{\text{ }}\NormalTok{pdflatex=}\StringTok{\textquotesingle{}mkdir\ensuremath{\text{ }}-p\ensuremath{\text{ }}tmp;\ensuremath{\text{ }}pdflatex\ensuremath{\text{ }}-output-directory\ensuremath{\text{ }}tmp\textquotesingle{}}\newline
\end{Highlighting}
\end{Shaded}


Or for vim modify your .vimrc:

\begin{Shaded}
\begin{Highlighting}[]

\StringTok{"\ensuremath{\text{ }}use\ensuremath{\text{ }}pdflatex}\newline
\StringTok{let\ensuremath{\text{ }}g:Tex_DefaultTargetFormat=\textquotesingle{}pdf\textquotesingle{}}\newline
\StringTok{let\ensuremath{\text{ }}g:Tex_MultipleCompileFormats=\textquotesingle{}pdf,dvi\textquotesingle{}}\newline
\StringTok{let\ensuremath{\text{ }}g:Tex_CompileRule_pdf\ensuremath{\text{ }}=\ensuremath{\text{ }}\textquotesingle{}mkdir\ensuremath{\text{ }}-p\ensuremath{\text{ }}tmp;\ensuremath{\text{ }}pdflatex\ensuremath{\text{ }}-output-directory\ensuremath{\text{ }}tmp\ensuremath{\text{ }}}\newline
\StringTok{-interaction=nonstopmode\ensuremath{\text{ }}}\OtherTok{\$*}\StringTok{;\ensuremath{\text{ }}cp\ensuremath{\text{ }}tmp/*.pdf\ensuremath{\text{ }}.\textquotesingle{}}\newline
\end{Highlighting}
\end{Shaded}



\mypart{Appendices}\chapter{Authors}

\myminitoc
\label{980}

\label{981}

\section{Included books}
\label{982}

The following books have been included in this wikibook (or we are working on it!), with permission of the author:

\begin{myitemize}
\item{}  Andy Roberts\textquotesingle{} \myhref{http://www.andy-roberts.net/misc/latex/index.html}{Getting to grips with Latex}.
\item{}  \myhref{http://www.ctan.org/tex-archive/info/lshort/english/lshort.pdf}{Not So Short Introduction to LaTex2e} by Tobias Oetiker, Hubert Partl and Irene Hyna. We have contacted the authors by email asking for permission: they allowed us to use their material, but they never edited directly this wikibook. That book is released under the GPL, that is not compatible with the GFDL used here in Wikibooks. Anyway, we have the permission of the authors to use their work. You can freely copy text from that guide to here. If you find text on both the original book and here on Wikibooks, then that text is double licensed under GPL and GFDL. For more information about Tobias Oetiker and Hubert Partl, their websites are \myplainurl{http://it.oetiker.ch/} and \myplainurl{http://homepage.boku.ac.at/partl/} respectively.
\item{}  \myhref{http://sarovar.org/projects/ltxprimer/}{LaTeX Primer} from the Indian TeX Users Group. Their document is released under the {\itshape \setmainfont[Path=/usr/share/fonts/truetype/cmu/,UprightFont=cmunrm.ttf,BoldFont=cmunbx.ttf,ItalicFont=cmunti.ttf,BoldItalicFont=cmunbi.ttf]{cmunti.ttf}\setmonofont[Path=/usr/share/fonts/truetype/cmu/,UprightFont=cmuntt.ttf,BoldFont=cmuntb.ttf,ItalicFont=cmunit.ttf,BoldItalicFont=cmuntx.ttf]{cmunti.ttf}\itshape GNU Free Documentation License}\setmainfont[Path=/usr/share/fonts/truetype/cmu/,UprightFont=cmunrm.ttf,BoldFont=cmunbx.ttf,ItalicFont=cmunti.ttf,BoldItalicFont=cmunbi.ttf]{cmunrm.ttf}\setmonofont[Path=/usr/share/fonts/truetype/cmu/,UprightFont=cmuntt.ttf,BoldFont=cmuntb.ttf,ItalicFont=cmunit.ttf,BoldItalicFont=cmuntx.ttf]{cmunrm.ttf}, the same as Wikibooks, so we can include parts of their document as we wish. In any case, we have contacted Indian TeX Users Group and they allowed us to do it.
\item{}  David Wilkins\textquotesingle{} \myhref{http://www.maths.tcd.ie/~dwilkins/LaTeXPrimer/}{Getting started with LaTeX}. The book is not released under any free license, but we have contacted the author asking him for the permission to use parts of his book on Wikibooks. He agreed: his work is still protected but you are allowed to copy the parts you want on this Wikibook. If you see text on both the original work and here, then that part (and only that part) is released under the terms of GFDL, like any other text here on Wikibooks.
\end{myitemize}

{\bfseries
\begin{mydescription}In progress
\end{mydescription}
}

\begin{myitemize}
\item{}  Peter Flynn\textquotesingle{}s \myhref{http://www.ctan.org/tex-archive/info/beginlatex/beginlatex-3.6.pdf}{Formatting information, a beginner\textquotesingle{}s guide to typesetting with LaTeX}. We have contacted him by email asking for permission to use his work. The original book is released under the {\itshape \setmainfont[Path=/usr/share/fonts/truetype/cmu/,UprightFont=cmunrm.ttf,BoldFont=cmunbx.ttf,ItalicFont=cmunti.ttf,BoldItalicFont=cmunbi.ttf]{cmunti.ttf}\setmonofont[Path=/usr/share/fonts/truetype/cmu/,UprightFont=cmuntt.ttf,BoldFont=cmuntb.ttf,ItalicFont=cmunit.ttf,BoldItalicFont=cmuntx.ttf]{cmunti.ttf}\itshape GNU Free Documentation License}\setmainfont[Path=/usr/share/fonts/truetype/cmu/,UprightFont=cmunrm.ttf,BoldFont=cmunbx.ttf,ItalicFont=cmunti.ttf,BoldItalicFont=cmunbi.ttf]{cmunrm.ttf}\setmonofont[Path=/usr/share/fonts/truetype/cmu/,UprightFont=cmuntt.ttf,BoldFont=cmuntb.ttf,ItalicFont=cmunit.ttf,BoldItalicFont=cmuntx.ttf]{cmunrm.ttf}, the same as Wikibooks. For more information, his personal website is \myplainurl{http://silmaril.ie/cgi-bin/blog.}
\end{myitemize}

\section{Wiki users}
\label{983}

Major contributors to the book on Wikibooks are:
\begin{myitemize}
\item{}  \myhref{https://en.wikibooks.org/wiki/User\%3AAlejo2083}{Alessio Damato}
\item{}  \myhref{https://en.wikibooks.org/wiki/User\%3AJtwdog}{Jtwdog}
\item{}  \myhref{https://en.wikibooks.org/wiki/User\%3AAmbrevar}{Pierre Neidhardt}
\end{myitemize}


\chapter{Links}

\myminitoc
\label{984}

\label{985}


\myhref{https://en.wikipedia.org/wiki/TeX}{w:TeX}
\myhref{https://en.wikipedia.org/wiki/LaTeX}{w:LaTeX}

Here are some other online resources available:\subsection{Community}
\label{986}
\begin{myitemize}
\item{}  \myhref{http://www.tug.org/}{The TeX Users Group} Includes links to free versions of (La)TeX for many kinds of computers.
\item{}  \myhref{http://uk.tug.org/}{UK-{}TUG} The UK TeX Users\textquotesingle{} Group
\item{}  \myhref{http://www.tug.org.in/}{TUGIndia} The Indian TeX Users Group
\item{}  {$\text{[}$}news:comp.text.tex comp.text.tex{$\text{]}$} Newsgroup for (La)TeX related questions
\item{}  \myhref{http://www.ctan.org/}{CTAN} hundreds of add-{}on packages and programs
\end{myitemize}

\subsection{Tutorials/FAQs}
\label{987}
\begin{myitemize}
\item{}  Tobias Oetiker\textquotesingle{}s Not So Short Introduction to LaTex2e:$\text{ }$\newline{}
\myplainurl{http://www.ctan.org/tex-archive/info/lshort/english/lshort.pdf} also at$\text{ }$\newline{}
\myplainurl{http://web.archive.org/web/20010603070337/http://people.ee.ethz.ch/~oetiker/lshort/lshort.pdf}
\item{}  Vel\textquotesingle{}s introduction to LaTeX: What is it, why should you use it, who should use it and how to get started: $\text{ }$\newline{}
\myplainurl{http://www.vel.co.nz/vel.co.nz/Blog/Entries/2009/11/4_LaTeX_Document_Preparation_System.html}
\item{}  Peter Flynn\textquotesingle{}s beginner\textquotesingle{}s guide (formatting):$\text{ }$\newline{}
\myplainurl{http://www.ctan.org/tex-archive/info/beginlatex/beginlatex-3.6.pdf}
\item{}  The AMS Short Math Guide for LaTeX, a concise summary of math formula typesetting features $\text{ }$\newline{}
\myplainurl{http://www.ams.org/tex/amslatex.html} 
\item{}  amsmath users guide (PDF) and related files:$\text{ }$\newline{}
\myplainurl{http://www.ctan.org/tex-archive/macros/latex/required/amslatex/math/}
\item{}  LaTeX Primer from the Indian TeX Users Group:$\text{ }$\newline{}
\myplainurl{http://sarovar.org/projects/ltxprimer/}
\item{}  LaTeX Primer $\text{ }$\newline{}
\myplainurl{http://www.maths.tcd.ie/~dwilkins/LaTeXPrimer/}
\item{}  PSTricks-{}-{}fancy graphics exploiting PDF capabilities$\text{ }$\newline{}
\myplainurl{http://sarovar.org/projects/pstricks/}
\item{}  PDFScreen-{}-{}create LaTeX PDF files that have navigation buttons used for presentations:$\text{ }$\newline{}
\myplainurl{http://sarovar.org/projects/pdfscreen/}
\item{}  David Bausum\textquotesingle{}s list of TeX primitives (these are the fundamental commands used in TeX):$\text{ }$\newline{}
\myplainurl{http://www.tug.org/utilities/plain/cseq.html}
\item{}  Leslie Lamport\textquotesingle{}s manual for the commands that are unique to LaTeX (commands not used in plain TeX):$\text{ }$\newline{}
\myplainurl{http://www.tex.uniyar.ac.ru/doc/latex2e.pdf}
\item{}  The UK TeX FAQ List of questions and answers that are frequently posted at comp.text.tex$\text{ }$\newline{}
\myplainurl{http://www.tex.ac.uk/faq}
\item{}  TeX on Mac OS X: Guide to using TeX and LaTeX on a Mac $\text{ }$\newline{}
\myplainurl{http://www.rna.nl/tex.html}
\item{}  Text Processing using LaTeX $\text{ }$\newline{}
\myplainurl{http://www-h.eng.cam.ac.uk/help/tpl/textprocessing/}
\item{}  The (La)TeX encyclopaedia $\text{ }$\newline{}
\myplainurl{http://tex.loria.fr/index.html}
\item{}  Hypertext Help with LaTeX $\text{ }$\newline{}
\myplainurl{http://www.giss.nasa.gov/tools/latex/index.html}
\item{}  EpsLatex: a very comprehensive guide to images, figures and graphics$\text{ }$\newline{}
\myplainurl{http://www.ctan.org/tex-archive/info/epslatex.pdf} 
\item{}  The Comprehensive LaTeX Symbol List (in PDF) $\text{ }$\newline{}
\myplainurl{http://www.ctan.org/tex-archive/info/symbols/comprehensive/symbols-a4.pdf}
\item{}  Getting to Grips with LaTeX (HTML) Collection of Latex tutorials taking you from the very basics towards more advanced topics $\text{ }$\newline{}
\myplainurl{http://www.andy-roberts.net/misc/latex/index.html}
\item{}  Chapter 8 (about typesetting mathematics) of the {\itshape \setmainfont[Path=/usr/share/fonts/truetype/cmu/,UprightFont=cmunrm.ttf,BoldFont=cmunbx.ttf,ItalicFont=cmunti.ttf,BoldItalicFont=cmunbi.ttf]{cmunti.ttf}\setmonofont[Path=/usr/share/fonts/truetype/cmu/,UprightFont=cmuntt.ttf,BoldFont=cmuntb.ttf,ItalicFont=cmunit.ttf,BoldItalicFont=cmuntx.ttf]{cmunti.ttf}\itshape LaTeX companion}{$\text{ }$}\setmainfont[Path=/usr/share/fonts/truetype/cmu/,UprightFont=cmunrm.ttf,BoldFont=cmunbx.ttf,ItalicFont=cmunti.ttf,BoldItalicFont=cmunbi.ttf]{cmunrm.ttf}\setmonofont[Path=/usr/share/fonts/truetype/cmu/,UprightFont=cmuntt.ttf,BoldFont=cmuntb.ttf,ItalicFont=cmunit.ttf,BoldItalicFont=cmuntx.ttf]{cmunrm.ttf} $\text{ }$\newline{}
\myplainurl{http://www.macrotex.net/texbooks/latexcomp-ch8.pdf}
\end{myitemize}

\subsection{Reference}
\label{988}
\begin{myitemize}
\item{}  \myhref{http://www.latex-project.org/}{LaTeX Project Site} 
\item{}  \myhref{http://www.ctan.org}{The Comprehensive TeX Archive Network} Latest (La)TeX-{}related packages and software
\item{}  \myhref{http://www.tug.org/tds/}{TeX Directory Structure}, used by many (La)TeX distributions
\item{}  \myhref{http://www.math.missouri.edu/~stephen/naturalmath/}{Natural Math} converts natural language math formulas to LaTeX representation
\item{}  \myhref{http://www.ctan.org/tex-archive/info/l2tabu/english/l2tabuen.pdf}{Obsolete packages and commands}
\item{}  Lamport\textquotesingle{}s book {\itshape \setmainfont[Path=/usr/share/fonts/truetype/cmu/,UprightFont=cmunrm.ttf,BoldFont=cmunbx.ttf,ItalicFont=cmunti.ttf,BoldItalicFont=cmunbi.ttf]{cmunti.ttf}\setmonofont[Path=/usr/share/fonts/truetype/cmu/,UprightFont=cmuntt.ttf,BoldFont=cmuntb.ttf,ItalicFont=cmunit.ttf,BoldItalicFont=cmuntx.ttf]{cmunti.ttf}\itshape LaTeX: A Document Preparation System}
\end{myitemize}
\setmainfont[Path=/usr/share/fonts/truetype/cmu/,UprightFont=cmunrm.ttf,BoldFont=cmunbx.ttf,ItalicFont=cmunti.ttf,BoldItalicFont=cmunbi.ttf]{cmunrm.ttf}\setmonofont[Path=/usr/share/fonts/truetype/cmu/,UprightFont=cmuntt.ttf,BoldFont=cmuntb.ttf,ItalicFont=cmunit.ttf,BoldItalicFont=cmuntx.ttf]{cmunrm.ttf}
\subsection{Templates}
\label{989}
\begin{myitemize}
\item{}  \myhref{http://www.LaTeXTemplates.com}{A resource for free high quality LaTeX templates for a variety of applications}
\item{}  \myhref{http://openwetware.org/wiki/LaTeX_template_for_PhD_thesis}{LaTeX template for writing PhD thesis}, 2007
\item{}  \myhref{http://www.cs.ucl.ac.uk/students/mphil_phd/resources_for_research_students/latex_for_research_thesis}{UCL computer department thesis template}
\item{}  \myhref{http://www.cs.utexas.edu/users/jbednar/latex/}{UT thesis template}, 2006
\item{}  \myhref{http://code.google.com/p/latex-template}{A template that supports an easy conversion to *.odt (*.doc), *.pdf and *.html in one run}, 2009
\end{myitemize}


\chapter{Package Reference}

\myminitoc
\label{990}

\label{991}


This is an incomplete list of useful packages that can be used for a wide range of different kind of documents. Each package has a short description next to it and, when available, there is a link to a section describing such package in detail. All of them (unless stated) should be included in your LaTeX distribution as {\itshape \setmainfont[Path=/usr/share/fonts/truetype/cmu/,UprightFont=cmunrm.ttf,BoldFont=cmunbx.ttf,ItalicFont=cmunti.ttf,BoldItalicFont=cmunbi.ttf]{cmunti.ttf}\setmonofont[Path=/usr/share/fonts/truetype/cmu/,UprightFont=cmuntt.ttf,BoldFont=cmuntb.ttf,ItalicFont=cmunit.ttf,BoldItalicFont=cmuntx.ttf]{cmunti.ttf}\itshape package\_name.sty}\setmainfont[Path=/usr/share/fonts/truetype/cmu/,UprightFont=cmunrm.ttf,BoldFont=cmunbx.ttf,ItalicFont=cmunti.ttf,BoldItalicFont=cmunbi.ttf]{cmunrm.ttf}\setmonofont[Path=/usr/share/fonts/truetype/cmu/,UprightFont=cmuntt.ttf,BoldFont=cmuntb.ttf,ItalicFont=cmunit.ttf,BoldItalicFont=cmuntx.ttf]{cmunrm.ttf}. For more information, refer to the documentation of the single packages, as described in \mylref{53}{Installing Extra Packages}.

The list is in alphabetical order.

{\scalefont{0.52741}\begin{longtable}{|>{\RaggedRight}p{0.47143\linewidth}|>{\RaggedRight}p{0.47143\linewidth}|} \hline 
\hspace*{0pt}\ignorespaces{}\hspace*{0pt}{\bfseries \setmainfont[Path=/usr/share/fonts/truetype/cmu/,UprightFont=cmunrm.ttf,BoldFont=cmunbx.ttf,ItalicFont=cmunti.ttf,BoldItalicFont=cmunbi.ttf]{cmunbx.ttf}\setmonofont[Path=/usr/share/fonts/truetype/cmu/,UprightFont=cmuntt.ttf,BoldFont=cmuntb.ttf,ItalicFont=cmunit.ttf,BoldItalicFont=cmuntx.ttf]{cmunbx.ttf}\bfseries amsmath}&\hspace*{0pt}\ignorespaces{}\hspace*{0pt}{$\text{ }$}\setmainfont[Path=/usr/share/fonts/truetype/cmu/,UprightFont=cmunrm.ttf,BoldFont=cmunbx.ttf,ItalicFont=cmunti.ttf,BoldItalicFont=cmunbi.ttf]{cmunrm.ttf}\setmonofont[Path=/usr/share/fonts/truetype/cmu/,UprightFont=cmuntt.ttf,BoldFont=cmuntb.ttf,ItalicFont=cmunit.ttf,BoldItalicFont=cmuntx.ttf]{cmunrm.ttf} It contains the advanced math extensions for LaTeX. The complete documentation should be in your LaTeX distribution; the file is called {\itshape \setmainfont[Path=/usr/share/fonts/truetype/cmu/,UprightFont=cmunrm.ttf,BoldFont=cmunbx.ttf,ItalicFont=cmunti.ttf,BoldItalicFont=cmunbi.ttf]{cmunti.ttf}\setmonofont[Path=/usr/share/fonts/truetype/cmu/,UprightFont=cmuntt.ttf,BoldFont=cmuntb.ttf,ItalicFont=cmunit.ttf,BoldItalicFont=cmuntx.ttf]{cmunti.ttf}\itshape amsdoc}\setmainfont[Path=/usr/share/fonts/truetype/cmu/,UprightFont=cmunrm.ttf,BoldFont=cmunbx.ttf,ItalicFont=cmunti.ttf,BoldItalicFont=cmunbi.ttf]{cmunrm.ttf}\setmonofont[Path=/usr/share/fonts/truetype/cmu/,UprightFont=cmuntt.ttf,BoldFont=cmuntb.ttf,ItalicFont=cmunit.ttf,BoldItalicFont=cmuntx.ttf]{cmunrm.ttf}, and can be {\itshape \setmainfont[Path=/usr/share/fonts/truetype/cmu/,UprightFont=cmunrm.ttf,BoldFont=cmunbx.ttf,ItalicFont=cmunti.ttf,BoldItalicFont=cmunbi.ttf]{cmunti.ttf}\setmonofont[Path=/usr/share/fonts/truetype/cmu/,UprightFont=cmuntt.ttf,BoldFont=cmuntb.ttf,ItalicFont=cmunit.ttf,BoldItalicFont=cmuntx.ttf]{cmunti.ttf}\itshape dvi}{$\text{ }$}\setmainfont[Path=/usr/share/fonts/truetype/cmu/,UprightFont=cmunrm.ttf,BoldFont=cmunbx.ttf,ItalicFont=cmunti.ttf,BoldItalicFont=cmunbi.ttf]{cmunrm.ttf}\setmonofont[Path=/usr/share/fonts/truetype/cmu/,UprightFont=cmuntt.ttf,BoldFont=cmuntb.ttf,ItalicFont=cmunit.ttf,BoldItalicFont=cmuntx.ttf]{cmunrm.ttf} or {\itshape \setmainfont[Path=/usr/share/fonts/truetype/cmu/,UprightFont=cmunrm.ttf,BoldFont=cmunbx.ttf,ItalicFont=cmunti.ttf,BoldItalicFont=cmunbi.ttf]{cmunti.ttf}\setmonofont[Path=/usr/share/fonts/truetype/cmu/,UprightFont=cmuntt.ttf,BoldFont=cmuntb.ttf,ItalicFont=cmunit.ttf,BoldItalicFont=cmuntx.ttf]{cmunti.ttf}\itshape pdf}\setmainfont[Path=/usr/share/fonts/truetype/cmu/,UprightFont=cmunrm.ttf,BoldFont=cmunbx.ttf,ItalicFont=cmunti.ttf,BoldItalicFont=cmunbi.ttf]{cmunrm.ttf}\setmonofont[Path=/usr/share/fonts/truetype/cmu/,UprightFont=cmuntt.ttf,BoldFont=cmuntb.ttf,ItalicFont=cmunit.ttf,BoldItalicFont=cmuntx.ttf]{cmunrm.ttf}. For more information, see the chapter about \mylref{497}{Mathematics}.\\ \hline \hspace*{0pt}\ignorespaces{}\hspace*{0pt}{\bfseries \setmainfont[Path=/usr/share/fonts/truetype/cmu/,UprightFont=cmunrm.ttf,BoldFont=cmunbx.ttf,ItalicFont=cmunti.ttf,BoldItalicFont=cmunbi.ttf]{cmunbx.ttf}\setmonofont[Path=/usr/share/fonts/truetype/cmu/,UprightFont=cmuntt.ttf,BoldFont=cmuntb.ttf,ItalicFont=cmunit.ttf,BoldItalicFont=cmuntx.ttf]{cmunbx.ttf}\bfseries amssymb}&\hspace*{0pt}\ignorespaces{}\hspace*{0pt}{$\text{ }$}\setmainfont[Path=/usr/share/fonts/truetype/cmu/,UprightFont=cmunrm.ttf,BoldFont=cmunbx.ttf,ItalicFont=cmunti.ttf,BoldItalicFont=cmunbi.ttf]{cmunrm.ttf}\setmonofont[Path=/usr/share/fonts/truetype/cmu/,UprightFont=cmuntt.ttf,BoldFont=cmuntb.ttf,ItalicFont=cmunit.ttf,BoldItalicFont=cmuntx.ttf]{cmunrm.ttf} It adds new symbols in to be used in math mode.\\ \hline \hspace*{0pt}\ignorespaces{}\hspace*{0pt}{\bfseries \setmainfont[Path=/usr/share/fonts/truetype/cmu/,UprightFont=cmunrm.ttf,BoldFont=cmunbx.ttf,ItalicFont=cmunti.ttf,BoldItalicFont=cmunbi.ttf]{cmunbx.ttf}\setmonofont[Path=/usr/share/fonts/truetype/cmu/,UprightFont=cmuntt.ttf,BoldFont=cmuntb.ttf,ItalicFont=cmunit.ttf,BoldItalicFont=cmuntx.ttf]{cmunbx.ttf}\bfseries amsthm}&\hspace*{0pt}\ignorespaces{}\hspace*{0pt}{$\text{ }$}\setmainfont[Path=/usr/share/fonts/truetype/cmu/,UprightFont=cmunrm.ttf,BoldFont=cmunbx.ttf,ItalicFont=cmunti.ttf,BoldItalicFont=cmunbi.ttf]{cmunrm.ttf}\setmonofont[Path=/usr/share/fonts/truetype/cmu/,UprightFont=cmuntt.ttf,BoldFont=cmuntb.ttf,ItalicFont=cmunit.ttf,BoldItalicFont=cmuntx.ttf]{cmunrm.ttf} It introduces the \LaTeXTT{proof} environment and the \LaTeXTT{\textbackslash{}theoremstyle} command. For more information see the \mylref{557}{Theorems} section.\\ \hline \hspace*{0pt}\ignorespaces{}\hspace*{0pt}{\bfseries \setmainfont[Path=/usr/share/fonts/truetype/cmu/,UprightFont=cmunrm.ttf,BoldFont=cmunbx.ttf,ItalicFont=cmunti.ttf,BoldItalicFont=cmunbi.ttf]{cmunbx.ttf}\setmonofont[Path=/usr/share/fonts/truetype/cmu/,UprightFont=cmuntt.ttf,BoldFont=cmuntb.ttf,ItalicFont=cmunit.ttf,BoldItalicFont=cmuntx.ttf]{cmunbx.ttf}\bfseries array}&\hspace*{0pt}\ignorespaces{}\hspace*{0pt}{$\text{ }$}\setmainfont[Path=/usr/share/fonts/truetype/cmu/,UprightFont=cmunrm.ttf,BoldFont=cmunbx.ttf,ItalicFont=cmunti.ttf,BoldItalicFont=cmunbi.ttf]{cmunrm.ttf}\setmonofont[Path=/usr/share/fonts/truetype/cmu/,UprightFont=cmuntt.ttf,BoldFont=cmuntb.ttf,ItalicFont=cmunit.ttf,BoldItalicFont=cmuntx.ttf]{cmunrm.ttf} It extends the possibility of LaTeX to handle tables, fixing some bugs and adding new features. Using it, you can create very complicated and customized tables. For more information, see the \mylref{248}{Tables} section.\\ \hline \hspace*{0pt}\ignorespaces{}\hspace*{0pt}{\bfseries \setmainfont[Path=/usr/share/fonts/truetype/cmu/,UprightFont=cmunrm.ttf,BoldFont=cmunbx.ttf,ItalicFont=cmunti.ttf,BoldItalicFont=cmunbi.ttf]{cmunbx.ttf}\setmonofont[Path=/usr/share/fonts/truetype/cmu/,UprightFont=cmuntt.ttf,BoldFont=cmuntb.ttf,ItalicFont=cmunit.ttf,BoldItalicFont=cmuntx.ttf]{cmunbx.ttf}\bfseries babel}&\hspace*{0pt}\ignorespaces{}\hspace*{0pt}{$\text{ }$}\setmainfont[Path=/usr/share/fonts/truetype/cmu/,UprightFont=cmunrm.ttf,BoldFont=cmunbx.ttf,ItalicFont=cmunti.ttf,BoldItalicFont=cmunbi.ttf]{cmunrm.ttf}\setmonofont[Path=/usr/share/fonts/truetype/cmu/,UprightFont=cmuntt.ttf,BoldFont=cmuntb.ttf,ItalicFont=cmunit.ttf,BoldItalicFont=cmuntx.ttf]{cmunrm.ttf} It provides the internationalization of LaTeX. It has to be loaded in any document, and you have to give as an option the main language you are going to use in the document. For more information see the \mylref{209}{Internationalization} section.\\ \hline \hspace*{0pt}\ignorespaces{}\hspace*{0pt}{\bfseries \setmainfont[Path=/usr/share/fonts/truetype/cmu/,UprightFont=cmunrm.ttf,BoldFont=cmunbx.ttf,ItalicFont=cmunti.ttf,BoldItalicFont=cmunbi.ttf]{cmunbx.ttf}\setmonofont[Path=/usr/share/fonts/truetype/cmu/,UprightFont=cmuntt.ttf,BoldFont=cmuntb.ttf,ItalicFont=cmunit.ttf,BoldItalicFont=cmuntx.ttf]{cmunbx.ttf}\bfseries bm}&\hspace*{0pt}\ignorespaces{}\hspace*{0pt}{$\text{ }$}\setmainfont[Path=/usr/share/fonts/truetype/cmu/,UprightFont=cmunrm.ttf,BoldFont=cmunbx.ttf,ItalicFont=cmunti.ttf,BoldItalicFont=cmunbi.ttf]{cmunrm.ttf}\setmonofont[Path=/usr/share/fonts/truetype/cmu/,UprightFont=cmuntt.ttf,BoldFont=cmuntb.ttf,ItalicFont=cmunit.ttf,BoldItalicFont=cmuntx.ttf]{cmunrm.ttf} Allows use of bold greek letters in math mode using the \LaTeXTT{\textbackslash{}bm\{...\}} command. This supersedes the \LaTeXTT{amsbsy} package.\\ \hline \hspace*{0pt}\ignorespaces{}\hspace*{0pt}{\bfseries \setmainfont[Path=/usr/share/fonts/truetype/cmu/,UprightFont=cmunrm.ttf,BoldFont=cmunbx.ttf,ItalicFont=cmunti.ttf,BoldItalicFont=cmunbi.ttf]{cmunbx.ttf}\setmonofont[Path=/usr/share/fonts/truetype/cmu/,UprightFont=cmuntt.ttf,BoldFont=cmuntb.ttf,ItalicFont=cmunit.ttf,BoldItalicFont=cmuntx.ttf]{cmunbx.ttf}\bfseries booktabs}&\hspace*{0pt}\ignorespaces{}\hspace*{0pt}{$\text{ }$}\setmainfont[Path=/usr/share/fonts/truetype/cmu/,UprightFont=cmunrm.ttf,BoldFont=cmunbx.ttf,ItalicFont=cmunti.ttf,BoldItalicFont=cmunbi.ttf]{cmunrm.ttf}\setmonofont[Path=/usr/share/fonts/truetype/cmu/,UprightFont=cmuntt.ttf,BoldFont=cmuntb.ttf,ItalicFont=cmunit.ttf,BoldItalicFont=cmuntx.ttf]{cmunrm.ttf} provides ex­tra com­mands as well as be­hind-{}the-{}scenes op­ti­mi­sa­tion for producing tables. Guide­lines are given as to what con­sti­tutes a good ta­ble in the package documentation.\\ \hline \hspace*{0pt}\ignorespaces{}\hspace*{0pt}{\bfseries \setmainfont[Path=/usr/share/fonts/truetype/cmu/,UprightFont=cmunrm.ttf,BoldFont=cmunbx.ttf,ItalicFont=cmunti.ttf,BoldItalicFont=cmunbi.ttf]{cmunbx.ttf}\setmonofont[Path=/usr/share/fonts/truetype/cmu/,UprightFont=cmuntt.ttf,BoldFont=cmuntb.ttf,ItalicFont=cmunit.ttf,BoldItalicFont=cmuntx.ttf]{cmunbx.ttf}\bfseries boxedminipage}&\hspace*{0pt}\ignorespaces{}\hspace*{0pt}{$\text{ }$}\setmainfont[Path=/usr/share/fonts/truetype/cmu/,UprightFont=cmunrm.ttf,BoldFont=cmunbx.ttf,ItalicFont=cmunti.ttf,BoldItalicFont=cmunbi.ttf]{cmunrm.ttf}\setmonofont[Path=/usr/share/fonts/truetype/cmu/,UprightFont=cmuntt.ttf,BoldFont=cmuntb.ttf,ItalicFont=cmunit.ttf,BoldItalicFont=cmuntx.ttf]{cmunrm.ttf} It introduces the \LaTeXTT{boxedminipage} environment, that works exactly like \LaTeXTT{minipage} but adds a frame around it.\\ \hline \hspace*{0pt}\ignorespaces{}\hspace*{0pt}{\bfseries \setmainfont[Path=/usr/share/fonts/truetype/cmu/,UprightFont=cmunrm.ttf,BoldFont=cmunbx.ttf,ItalicFont=cmunti.ttf,BoldItalicFont=cmunbi.ttf]{cmunbx.ttf}\setmonofont[Path=/usr/share/fonts/truetype/cmu/,UprightFont=cmuntt.ttf,BoldFont=cmuntb.ttf,ItalicFont=cmunit.ttf,BoldItalicFont=cmuntx.ttf]{cmunbx.ttf}\bfseries caption}&\hspace*{0pt}\ignorespaces{}\hspace*{0pt}{$\text{ }$}\setmainfont[Path=/usr/share/fonts/truetype/cmu/,UprightFont=cmunrm.ttf,BoldFont=cmunbx.ttf,ItalicFont=cmunti.ttf,BoldItalicFont=cmunbi.ttf]{cmunrm.ttf}\setmonofont[Path=/usr/share/fonts/truetype/cmu/,UprightFont=cmuntt.ttf,BoldFont=cmuntb.ttf,ItalicFont=cmunit.ttf,BoldItalicFont=cmuntx.ttf]{cmunrm.ttf} Allows customization of appearance and placement of captions for figures, tables, etc.\\ \hline \hspace*{0pt}\ignorespaces{}\hspace*{0pt}{\bfseries \setmainfont[Path=/usr/share/fonts/truetype/cmu/,UprightFont=cmunrm.ttf,BoldFont=cmunbx.ttf,ItalicFont=cmunti.ttf,BoldItalicFont=cmunbi.ttf]{cmunbx.ttf}\setmonofont[Path=/usr/share/fonts/truetype/cmu/,UprightFont=cmuntt.ttf,BoldFont=cmuntb.ttf,ItalicFont=cmunit.ttf,BoldItalicFont=cmuntx.ttf]{cmunbx.ttf}\bfseries cancel}&\hspace*{0pt}\ignorespaces{}\hspace*{0pt}{$\text{ }$}\setmainfont[Path=/usr/share/fonts/truetype/cmu/,UprightFont=cmunrm.ttf,BoldFont=cmunbx.ttf,ItalicFont=cmunti.ttf,BoldItalicFont=cmunbi.ttf]{cmunrm.ttf}\setmonofont[Path=/usr/share/fonts/truetype/cmu/,UprightFont=cmuntt.ttf,BoldFont=cmuntb.ttf,ItalicFont=cmunit.ttf,BoldItalicFont=cmuntx.ttf]{cmunrm.ttf} Provides commands for striking out mathematical expressions. The syntax is\LaTeXTT{\textbackslash{}cancel\{x\}}or\LaTeXTT{\textbackslash{}cancelto\{0\}\{x\}}\\ \hline \hspace*{0pt}\ignorespaces{}\hspace*{0pt}{\bfseries \setmainfont[Path=/usr/share/fonts/truetype/cmu/,UprightFont=cmunrm.ttf,BoldFont=cmunbx.ttf,ItalicFont=cmunti.ttf,BoldItalicFont=cmunbi.ttf]{cmunbx.ttf}\setmonofont[Path=/usr/share/fonts/truetype/cmu/,UprightFont=cmuntt.ttf,BoldFont=cmuntb.ttf,ItalicFont=cmunit.ttf,BoldItalicFont=cmuntx.ttf]{cmunbx.ttf}\bfseries changepage}&\hspace*{0pt}\ignorespaces{}\hspace*{0pt}{$\text{ }$}\setmainfont[Path=/usr/share/fonts/truetype/cmu/,UprightFont=cmunrm.ttf,BoldFont=cmunbx.ttf,ItalicFont=cmunti.ttf,BoldItalicFont=cmunbi.ttf]{cmunrm.ttf}\setmonofont[Path=/usr/share/fonts/truetype/cmu/,UprightFont=cmuntt.ttf,BoldFont=cmuntb.ttf,ItalicFont=cmunit.ttf,BoldItalicFont=cmuntx.ttf]{cmunrm.ttf} To easily change the margins of pages. The syntax is\TemplateSource{\newline{}\textbackslash{}changepage\{textheight\}\{textwidth\}\%\newline{}$\text{ }${}$\text{ }${}\{evensidemargin\}\{oddsidemargin\}\%\newline{}$\text{ }${}$\text{ }${}\{columnsep\}\{topmargin\}\%\newline{}$\text{ }${}$\text{ }${}\{headheight\}\{headsep\}\%\newline{}$\text{ }${}$\text{ }${}\{footskip\}\newline{}}All the arguments can be both positive and negative numbers; they will be added (keeping the sign) to the relative variable.\\ \hline \hspace*{0pt}\ignorespaces{}\hspace*{0pt}{\bfseries \setmainfont[Path=/usr/share/fonts/truetype/cmu/,UprightFont=cmunrm.ttf,BoldFont=cmunbx.ttf,ItalicFont=cmunti.ttf,BoldItalicFont=cmunbi.ttf]{cmunbx.ttf}\setmonofont[Path=/usr/share/fonts/truetype/cmu/,UprightFont=cmuntt.ttf,BoldFont=cmuntb.ttf,ItalicFont=cmunit.ttf,BoldItalicFont=cmuntx.ttf]{cmunbx.ttf}\bfseries cite}&\hspace*{0pt}\ignorespaces{}\hspace*{0pt}{$\text{ }$}\setmainfont[Path=/usr/share/fonts/truetype/cmu/,UprightFont=cmunrm.ttf,BoldFont=cmunbx.ttf,ItalicFont=cmunti.ttf,BoldItalicFont=cmunbi.ttf]{cmunrm.ttf}\setmonofont[Path=/usr/share/fonts/truetype/cmu/,UprightFont=cmuntt.ttf,BoldFont=cmuntb.ttf,ItalicFont=cmunit.ttf,BoldItalicFont=cmuntx.ttf]{cmunrm.ttf} Sup­ports com­pressed, sorted lists of nu­mer­i­cal ci­ta­tions, and also deals with var­i­ous punc­tu­a­tion and other is­sues of rep­re­sen­ta­tion, in­clud­ing com­pre­hen­sive man­age­ment of break points.\\ \hline \hspace*{0pt}\ignorespaces{}\hspace*{0pt}{\bfseries \setmainfont[Path=/usr/share/fonts/truetype/cmu/,UprightFont=cmunrm.ttf,BoldFont=cmunbx.ttf,ItalicFont=cmunti.ttf,BoldItalicFont=cmunbi.ttf]{cmunbx.ttf}\setmonofont[Path=/usr/share/fonts/truetype/cmu/,UprightFont=cmuntt.ttf,BoldFont=cmuntb.ttf,ItalicFont=cmunit.ttf,BoldItalicFont=cmuntx.ttf]{cmunbx.ttf}\bfseries color}&\hspace*{0pt}\ignorespaces{}\hspace*{0pt}{$\text{ }$}\setmainfont[Path=/usr/share/fonts/truetype/cmu/,UprightFont=cmunrm.ttf,BoldFont=cmunbx.ttf,ItalicFont=cmunti.ttf,BoldItalicFont=cmunbi.ttf]{cmunrm.ttf}\setmonofont[Path=/usr/share/fonts/truetype/cmu/,UprightFont=cmuntt.ttf,BoldFont=cmuntb.ttf,ItalicFont=cmunit.ttf,BoldItalicFont=cmuntx.ttf]{cmunrm.ttf} It adds support for colored text. For more information, see the \mylref{147}{relevant section}.\\ \hline \hspace*{0pt}\ignorespaces{}\hspace*{0pt}{\bfseries \setmainfont[Path=/usr/share/fonts/truetype/cmu/,UprightFont=cmunrm.ttf,BoldFont=cmunbx.ttf,ItalicFont=cmunti.ttf,BoldItalicFont=cmunbi.ttf]{cmunbx.ttf}\setmonofont[Path=/usr/share/fonts/truetype/cmu/,UprightFont=cmuntt.ttf,BoldFont=cmuntb.ttf,ItalicFont=cmunit.ttf,BoldItalicFont=cmuntx.ttf]{cmunbx.ttf}\bfseries easylist}&\hspace*{0pt}\ignorespaces{}\hspace*{0pt}{$\text{ }$}\setmainfont[Path=/usr/share/fonts/truetype/cmu/,UprightFont=cmunrm.ttf,BoldFont=cmunbx.ttf,ItalicFont=cmunti.ttf,BoldItalicFont=cmunbi.ttf]{cmunrm.ttf}\setmonofont[Path=/usr/share/fonts/truetype/cmu/,UprightFont=cmuntt.ttf,BoldFont=cmuntb.ttf,ItalicFont=cmunit.ttf,BoldItalicFont=cmuntx.ttf]{cmunrm.ttf} Adds support for arbitrarily-{}deep nested lists (useful for outlines). See \mylref{186}{List Structures}.\\ \hline \hspace*{0pt}\ignorespaces{}\hspace*{0pt}{\bfseries \setmainfont[Path=/usr/share/fonts/truetype/cmu/,UprightFont=cmunrm.ttf,BoldFont=cmunbx.ttf,ItalicFont=cmunti.ttf,BoldItalicFont=cmunbi.ttf]{cmunbx.ttf}\setmonofont[Path=/usr/share/fonts/truetype/cmu/,UprightFont=cmuntt.ttf,BoldFont=cmuntb.ttf,ItalicFont=cmunit.ttf,BoldItalicFont=cmuntx.ttf]{cmunbx.ttf}\bfseries esint}&\hspace*{0pt}\ignorespaces{}\hspace*{0pt}{$\text{ }$}\setmainfont[Path=/usr/share/fonts/truetype/cmu/,UprightFont=cmunrm.ttf,BoldFont=cmunbx.ttf,ItalicFont=cmunti.ttf,BoldItalicFont=cmunbi.ttf]{cmunrm.ttf}\setmonofont[Path=/usr/share/fonts/truetype/cmu/,UprightFont=cmuntt.ttf,BoldFont=cmuntb.ttf,ItalicFont=cmunit.ttf,BoldItalicFont=cmuntx.ttf]{cmunrm.ttf} Adds additional integral symbols, for integrals over squares, clockwise integrals over sets, etc.\\ \hline \hspace*{0pt}\ignorespaces{}\hspace*{0pt}{\bfseries \setmainfont[Path=/usr/share/fonts/truetype/cmu/,UprightFont=cmunrm.ttf,BoldFont=cmunbx.ttf,ItalicFont=cmunti.ttf,BoldItalicFont=cmunbi.ttf]{cmunbx.ttf}\setmonofont[Path=/usr/share/fonts/truetype/cmu/,UprightFont=cmuntt.ttf,BoldFont=cmuntb.ttf,ItalicFont=cmunit.ttf,BoldItalicFont=cmuntx.ttf]{cmunbx.ttf}\bfseries eucal}&\hspace*{0pt}\ignorespaces{}\hspace*{0pt}{$\text{ }$}\setmainfont[Path=/usr/share/fonts/truetype/cmu/,UprightFont=cmunrm.ttf,BoldFont=cmunbx.ttf,ItalicFont=cmunti.ttf,BoldItalicFont=cmunbi.ttf]{cmunrm.ttf}\setmonofont[Path=/usr/share/fonts/truetype/cmu/,UprightFont=cmuntt.ttf,BoldFont=cmuntb.ttf,ItalicFont=cmunit.ttf,BoldItalicFont=cmuntx.ttf]{cmunrm.ttf} Other mathematical symbols.\\ \hline \hspace*{0pt}\ignorespaces{}\hspace*{0pt}{\bfseries \setmainfont[Path=/usr/share/fonts/truetype/cmu/,UprightFont=cmunrm.ttf,BoldFont=cmunbx.ttf,ItalicFont=cmunti.ttf,BoldItalicFont=cmunbi.ttf]{cmunbx.ttf}\setmonofont[Path=/usr/share/fonts/truetype/cmu/,UprightFont=cmuntt.ttf,BoldFont=cmuntb.ttf,ItalicFont=cmunit.ttf,BoldItalicFont=cmuntx.ttf]{cmunbx.ttf}\bfseries fancyhdr}&\hspace*{0pt}\ignorespaces{}\hspace*{0pt}{$\text{ }$}\setmainfont[Path=/usr/share/fonts/truetype/cmu/,UprightFont=cmunrm.ttf,BoldFont=cmunbx.ttf,ItalicFont=cmunti.ttf,BoldItalicFont=cmunbi.ttf]{cmunrm.ttf}\setmonofont[Path=/usr/share/fonts/truetype/cmu/,UprightFont=cmuntt.ttf,BoldFont=cmuntb.ttf,ItalicFont=cmunit.ttf,BoldItalicFont=cmuntx.ttf]{cmunrm.ttf} To change header and footer of any page of the document. It is described in the \mylref{303}{Page Layout} section.\\ \hline \hspace*{0pt}\ignorespaces{}\hspace*{0pt}{\bfseries \setmainfont[Path=/usr/share/fonts/truetype/cmu/,UprightFont=cmunrm.ttf,BoldFont=cmunbx.ttf,ItalicFont=cmunti.ttf,BoldItalicFont=cmunbi.ttf]{cmunbx.ttf}\setmonofont[Path=/usr/share/fonts/truetype/cmu/,UprightFont=cmuntt.ttf,BoldFont=cmuntb.ttf,ItalicFont=cmunit.ttf,BoldItalicFont=cmuntx.ttf]{cmunbx.ttf}\bfseries fontenc}&\hspace*{0pt}\ignorespaces{}\hspace*{0pt}{$\text{ }$}\setmainfont[Path=/usr/share/fonts/truetype/cmu/,UprightFont=cmunrm.ttf,BoldFont=cmunbx.ttf,ItalicFont=cmunti.ttf,BoldItalicFont=cmunbi.ttf]{cmunrm.ttf}\setmonofont[Path=/usr/share/fonts/truetype/cmu/,UprightFont=cmuntt.ttf,BoldFont=cmuntb.ttf,ItalicFont=cmunit.ttf,BoldItalicFont=cmuntx.ttf]{cmunrm.ttf} To choose the font encoding of the output text. You might need it if you are writing documents in a language other than English. Check in the \mylref{163}{Fonts} section.\\ \hline \hspace*{0pt}\ignorespaces{}\hspace*{0pt}{\bfseries \setmainfont[Path=/usr/share/fonts/truetype/cmu/,UprightFont=cmunrm.ttf,BoldFont=cmunbx.ttf,ItalicFont=cmunti.ttf,BoldItalicFont=cmunbi.ttf]{cmunbx.ttf}\setmonofont[Path=/usr/share/fonts/truetype/cmu/,UprightFont=cmuntt.ttf,BoldFont=cmuntb.ttf,ItalicFont=cmunit.ttf,BoldItalicFont=cmuntx.ttf]{cmunbx.ttf}\bfseries geometry}&\hspace*{0pt}\ignorespaces{}\hspace*{0pt}{$\text{ }$}\setmainfont[Path=/usr/share/fonts/truetype/cmu/,UprightFont=cmunrm.ttf,BoldFont=cmunbx.ttf,ItalicFont=cmunti.ttf,BoldItalicFont=cmunbi.ttf]{cmunrm.ttf}\setmonofont[Path=/usr/share/fonts/truetype/cmu/,UprightFont=cmuntt.ttf,BoldFont=cmuntb.ttf,ItalicFont=cmunit.ttf,BoldItalicFont=cmuntx.ttf]{cmunrm.ttf} For easy management of document margins and the document page size. See \mylref{303}{Page Layout}.\\ \hline \hspace*{0pt}\ignorespaces{}\hspace*{0pt}{\bfseries \setmainfont[Path=/usr/share/fonts/truetype/cmu/,UprightFont=cmunrm.ttf,BoldFont=cmunbx.ttf,ItalicFont=cmunti.ttf,BoldItalicFont=cmunbi.ttf]{cmunbx.ttf}\setmonofont[Path=/usr/share/fonts/truetype/cmu/,UprightFont=cmuntt.ttf,BoldFont=cmuntb.ttf,ItalicFont=cmunit.ttf,BoldItalicFont=cmuntx.ttf]{cmunbx.ttf}\bfseries glossaries}&\hspace*{0pt}\ignorespaces{}\hspace*{0pt}{$\text{ }$}\setmainfont[Path=/usr/share/fonts/truetype/cmu/,UprightFont=cmunrm.ttf,BoldFont=cmunbx.ttf,ItalicFont=cmunti.ttf,BoldItalicFont=cmunbi.ttf]{cmunrm.ttf}\setmonofont[Path=/usr/share/fonts/truetype/cmu/,UprightFont=cmuntt.ttf,BoldFont=cmuntb.ttf,ItalicFont=cmunit.ttf,BoldItalicFont=cmuntx.ttf]{cmunrm.ttf} For creation of glossaries and list of acronyms. For more information, see the \mylref{643}{relevant chapter}.\\ \hline \hspace*{0pt}\ignorespaces{}\hspace*{0pt}{\bfseries \setmainfont[Path=/usr/share/fonts/truetype/cmu/,UprightFont=cmunrm.ttf,BoldFont=cmunbx.ttf,ItalicFont=cmunti.ttf,BoldItalicFont=cmunbi.ttf]{cmunbx.ttf}\setmonofont[Path=/usr/share/fonts/truetype/cmu/,UprightFont=cmuntt.ttf,BoldFont=cmuntb.ttf,ItalicFont=cmunit.ttf,BoldItalicFont=cmuntx.ttf]{cmunbx.ttf}\bfseries graphicx}&\hspace*{0pt}\ignorespaces{}\hspace*{0pt}{$\text{ }$}\setmainfont[Path=/usr/share/fonts/truetype/cmu/,UprightFont=cmunrm.ttf,BoldFont=cmunbx.ttf,ItalicFont=cmunti.ttf,BoldItalicFont=cmunbi.ttf]{cmunrm.ttf}\setmonofont[Path=/usr/share/fonts/truetype/cmu/,UprightFont=cmuntt.ttf,BoldFont=cmuntb.ttf,ItalicFont=cmunit.ttf,BoldItalicFont=cmuntx.ttf]{cmunrm.ttf} allows you to insert graphic files within a document.\\ \hline \hspace*{0pt}\ignorespaces{}\hspace*{0pt}{\bfseries \setmainfont[Path=/usr/share/fonts/truetype/cmu/,UprightFont=cmunrm.ttf,BoldFont=cmunbx.ttf,ItalicFont=cmunti.ttf,BoldItalicFont=cmunbi.ttf]{cmunbx.ttf}\setmonofont[Path=/usr/share/fonts/truetype/cmu/,UprightFont=cmuntt.ttf,BoldFont=cmuntb.ttf,ItalicFont=cmunit.ttf,BoldItalicFont=cmuntx.ttf]{cmunbx.ttf}\bfseries hyperref}&\hspace*{0pt}\ignorespaces{}\hspace*{0pt}{$\text{ }$}\setmainfont[Path=/usr/share/fonts/truetype/cmu/,UprightFont=cmunrm.ttf,BoldFont=cmunbx.ttf,ItalicFont=cmunti.ttf,BoldItalicFont=cmunbi.ttf]{cmunrm.ttf}\setmonofont[Path=/usr/share/fonts/truetype/cmu/,UprightFont=cmuntt.ttf,BoldFont=cmuntb.ttf,ItalicFont=cmunit.ttf,BoldItalicFont=cmuntx.ttf]{cmunrm.ttf} It gives LaTeX the possibility to manage links within the document or to any URL when you compile in PDF. For more information, see the \mylref{392}{relevant section}.\\ \hline \hspace*{0pt}\ignorespaces{}\hspace*{0pt}{\bfseries \setmainfont[Path=/usr/share/fonts/truetype/cmu/,UprightFont=cmunrm.ttf,BoldFont=cmunbx.ttf,ItalicFont=cmunti.ttf,BoldItalicFont=cmunbi.ttf]{cmunbx.ttf}\setmonofont[Path=/usr/share/fonts/truetype/cmu/,UprightFont=cmuntt.ttf,BoldFont=cmuntb.ttf,ItalicFont=cmunit.ttf,BoldItalicFont=cmuntx.ttf]{cmunbx.ttf}\bfseries indentfirst}&\hspace*{0pt}\ignorespaces{}\hspace*{0pt}{$\text{ }$}\setmainfont[Path=/usr/share/fonts/truetype/cmu/,UprightFont=cmunrm.ttf,BoldFont=cmunbx.ttf,ItalicFont=cmunti.ttf,BoldItalicFont=cmunbi.ttf]{cmunrm.ttf}\setmonofont[Path=/usr/share/fonts/truetype/cmu/,UprightFont=cmuntt.ttf,BoldFont=cmuntb.ttf,ItalicFont=cmunit.ttf,BoldItalicFont=cmuntx.ttf]{cmunrm.ttf} Once loaded, the beginning of any chapter/section is indented by the usual paragraph indentation.\\ \hline \hspace*{0pt}\ignorespaces{}\hspace*{0pt}{\bfseries \setmainfont[Path=/usr/share/fonts/truetype/cmu/,UprightFont=cmunrm.ttf,BoldFont=cmunbx.ttf,ItalicFont=cmunti.ttf,BoldItalicFont=cmunbi.ttf]{cmunbx.ttf}\setmonofont[Path=/usr/share/fonts/truetype/cmu/,UprightFont=cmuntt.ttf,BoldFont=cmuntb.ttf,ItalicFont=cmunit.ttf,BoldItalicFont=cmuntx.ttf]{cmunbx.ttf}\bfseries inputenc}&\hspace*{0pt}\ignorespaces{}\hspace*{0pt}{$\text{ }$}\setmainfont[Path=/usr/share/fonts/truetype/cmu/,UprightFont=cmunrm.ttf,BoldFont=cmunbx.ttf,ItalicFont=cmunti.ttf,BoldItalicFont=cmunbi.ttf]{cmunrm.ttf}\setmonofont[Path=/usr/share/fonts/truetype/cmu/,UprightFont=cmuntt.ttf,BoldFont=cmuntb.ttf,ItalicFont=cmunit.ttf,BoldItalicFont=cmuntx.ttf]{cmunrm.ttf} To choose the encoding of the input text. You might need it if you are writing documents in a language other than English. Check in the \mylref{192}{Special Characters} section.\\ \hline \hspace*{0pt}\ignorespaces{}\hspace*{0pt}{\bfseries \setmainfont[Path=/usr/share/fonts/truetype/cmu/,UprightFont=cmunrm.ttf,BoldFont=cmunbx.ttf,ItalicFont=cmunti.ttf,BoldItalicFont=cmunbi.ttf]{cmunbx.ttf}\setmonofont[Path=/usr/share/fonts/truetype/cmu/,UprightFont=cmuntt.ttf,BoldFont=cmuntb.ttf,ItalicFont=cmunit.ttf,BoldItalicFont=cmuntx.ttf]{cmunbx.ttf}\bfseries latexsym}&\hspace*{0pt}\ignorespaces{}\hspace*{0pt}{$\text{ }$}\setmainfont[Path=/usr/share/fonts/truetype/cmu/,UprightFont=cmunrm.ttf,BoldFont=cmunbx.ttf,ItalicFont=cmunti.ttf,BoldItalicFont=cmunbi.ttf]{cmunrm.ttf}\setmonofont[Path=/usr/share/fonts/truetype/cmu/,UprightFont=cmuntt.ttf,BoldFont=cmuntb.ttf,ItalicFont=cmunit.ttf,BoldItalicFont=cmuntx.ttf]{cmunrm.ttf} Other mathematical symbols.\\ \hline \hspace*{0pt}\ignorespaces{}\hspace*{0pt}{\bfseries \setmainfont[Path=/usr/share/fonts/truetype/cmu/,UprightFont=cmunrm.ttf,BoldFont=cmunbx.ttf,ItalicFont=cmunti.ttf,BoldItalicFont=cmunbi.ttf]{cmunbx.ttf}\setmonofont[Path=/usr/share/fonts/truetype/cmu/,UprightFont=cmuntt.ttf,BoldFont=cmuntb.ttf,ItalicFont=cmunit.ttf,BoldItalicFont=cmuntx.ttf]{cmunbx.ttf}\bfseries listings}&\hspace*{0pt}\ignorespaces{}\hspace*{0pt}{$\text{ }$}\setmainfont[Path=/usr/share/fonts/truetype/cmu/,UprightFont=cmunrm.ttf,BoldFont=cmunbx.ttf,ItalicFont=cmunti.ttf,BoldItalicFont=cmunbi.ttf]{cmunrm.ttf}\setmonofont[Path=/usr/share/fonts/truetype/cmu/,UprightFont=cmuntt.ttf,BoldFont=cmuntb.ttf,ItalicFont=cmunit.ttf,BoldItalicFont=cmuntx.ttf]{cmunrm.ttf} To insert programming code within the document. Many languages are supported and the output can be customized. For more information, see the \mylref{593}{Source Code Listings}.\\ \hline \hspace*{0pt}\ignorespaces{}\hspace*{0pt}{\bfseries \setmainfont[Path=/usr/share/fonts/truetype/cmu/,UprightFont=cmunrm.ttf,BoldFont=cmunbx.ttf,ItalicFont=cmunti.ttf,BoldItalicFont=cmunbi.ttf]{cmunbx.ttf}\setmonofont[Path=/usr/share/fonts/truetype/cmu/,UprightFont=cmuntt.ttf,BoldFont=cmuntb.ttf,ItalicFont=cmunit.ttf,BoldItalicFont=cmuntx.ttf]{cmunbx.ttf}\bfseries mathptmx}&\hspace*{0pt}\ignorespaces{}\hspace*{0pt}{$\text{ }$}\setmainfont[Path=/usr/share/fonts/truetype/cmu/,UprightFont=cmunrm.ttf,BoldFont=cmunbx.ttf,ItalicFont=cmunti.ttf,BoldItalicFont=cmunbi.ttf]{cmunrm.ttf}\setmonofont[Path=/usr/share/fonts/truetype/cmu/,UprightFont=cmuntt.ttf,BoldFont=cmuntb.ttf,ItalicFont=cmunit.ttf,BoldItalicFont=cmuntx.ttf]{cmunrm.ttf} Sets the default font of the entire document (including math formulae) to Times New Roman, which is a more familiar font, and useful in saving space when fighting against page limits.\\ \hline \hspace*{0pt}\ignorespaces{}\hspace*{0pt}{\bfseries \setmainfont[Path=/usr/share/fonts/truetype/cmu/,UprightFont=cmunrm.ttf,BoldFont=cmunbx.ttf,ItalicFont=cmunti.ttf,BoldItalicFont=cmunbi.ttf]{cmunbx.ttf}\setmonofont[Path=/usr/share/fonts/truetype/cmu/,UprightFont=cmuntt.ttf,BoldFont=cmuntb.ttf,ItalicFont=cmunit.ttf,BoldItalicFont=cmuntx.ttf]{cmunbx.ttf}\bfseries mathrsfs}&\hspace*{0pt}\ignorespaces{}\hspace*{0pt}{$\text{ }$}\setmainfont[Path=/usr/share/fonts/truetype/cmu/,UprightFont=cmunrm.ttf,BoldFont=cmunbx.ttf,ItalicFont=cmunti.ttf,BoldItalicFont=cmunbi.ttf]{cmunrm.ttf}\setmonofont[Path=/usr/share/fonts/truetype/cmu/,UprightFont=cmuntt.ttf,BoldFont=cmuntb.ttf,ItalicFont=cmunit.ttf,BoldItalicFont=cmuntx.ttf]{cmunrm.ttf} Other mathematical symbols.\\ \hline \hspace*{0pt}\ignorespaces{}\hspace*{0pt}{\bfseries \setmainfont[Path=/usr/share/fonts/truetype/cmu/,UprightFont=cmunrm.ttf,BoldFont=cmunbx.ttf,ItalicFont=cmunti.ttf,BoldItalicFont=cmunbi.ttf]{cmunbx.ttf}\setmonofont[Path=/usr/share/fonts/truetype/cmu/,UprightFont=cmuntt.ttf,BoldFont=cmuntb.ttf,ItalicFont=cmunit.ttf,BoldItalicFont=cmuntx.ttf]{cmunbx.ttf}\bfseries mhchem}&\hspace*{0pt}\ignorespaces{}\hspace*{0pt}{$\text{ }$}\setmainfont[Path=/usr/share/fonts/truetype/cmu/,UprightFont=cmunrm.ttf,BoldFont=cmunbx.ttf,ItalicFont=cmunti.ttf,BoldItalicFont=cmunbi.ttf]{cmunrm.ttf}\setmonofont[Path=/usr/share/fonts/truetype/cmu/,UprightFont=cmuntt.ttf,BoldFont=cmuntb.ttf,ItalicFont=cmunit.ttf,BoldItalicFont=cmuntx.ttf]{cmunrm.ttf} allows you to easily type chemical species and equations. It automatically formats chemical species so you don\textquotesingle{}t have to use subscript commands. It also Allows you to draw chemical formulas. \\ \hline \hspace*{0pt}\ignorespaces{}\hspace*{0pt}{\bfseries \setmainfont[Path=/usr/share/fonts/truetype/cmu/,UprightFont=cmunrm.ttf,BoldFont=cmunbx.ttf,ItalicFont=cmunti.ttf,BoldItalicFont=cmunbi.ttf]{cmunbx.ttf}\setmonofont[Path=/usr/share/fonts/truetype/cmu/,UprightFont=cmuntt.ttf,BoldFont=cmuntb.ttf,ItalicFont=cmunit.ttf,BoldItalicFont=cmuntx.ttf]{cmunbx.ttf}\bfseries microtype}&\hspace*{0pt}\ignorespaces{}\hspace*{0pt}{$\text{ }$}\setmainfont[Path=/usr/share/fonts/truetype/cmu/,UprightFont=cmunrm.ttf,BoldFont=cmunbx.ttf,ItalicFont=cmunti.ttf,BoldItalicFont=cmunbi.ttf]{cmunrm.ttf}\setmonofont[Path=/usr/share/fonts/truetype/cmu/,UprightFont=cmuntt.ttf,BoldFont=cmuntb.ttf,ItalicFont=cmunit.ttf,BoldItalicFont=cmuntx.ttf]{cmunrm.ttf} It provides an improvement to LaTeX\textquotesingle{}s default ty­po­graphic ex­ten­sions, improvements in such areas as char­ac­ter pro­tru­sion and font ex­pan­sion, in­ter­word spac­ing and ad­di­tional kern­ing, and hy­phen­at­able letter-{}spacing\\ \hline \hspace*{0pt}\ignorespaces{}\hspace*{0pt}{\bfseries \setmainfont[Path=/usr/share/fonts/truetype/cmu/,UprightFont=cmunrm.ttf,BoldFont=cmunbx.ttf,ItalicFont=cmunti.ttf,BoldItalicFont=cmunbi.ttf]{cmunbx.ttf}\setmonofont[Path=/usr/share/fonts/truetype/cmu/,UprightFont=cmuntt.ttf,BoldFont=cmuntb.ttf,ItalicFont=cmunit.ttf,BoldItalicFont=cmuntx.ttf]{cmunbx.ttf}\bfseries multicol}&\hspace*{0pt}\ignorespaces{}\hspace*{0pt}{$\text{ }$}\setmainfont[Path=/usr/share/fonts/truetype/cmu/,UprightFont=cmunrm.ttf,BoldFont=cmunbx.ttf,ItalicFont=cmunti.ttf,BoldItalicFont=cmunbi.ttf]{cmunrm.ttf}\setmonofont[Path=/usr/share/fonts/truetype/cmu/,UprightFont=cmuntt.ttf,BoldFont=cmuntb.ttf,ItalicFont=cmunit.ttf,BoldItalicFont=cmuntx.ttf]{cmunrm.ttf} provides the mul­ti­cols environment which typesets text into multiple columns.\\ \hline \hspace*{0pt}\ignorespaces{}\hspace*{0pt}{\bfseries \setmainfont[Path=/usr/share/fonts/truetype/cmu/,UprightFont=cmunrm.ttf,BoldFont=cmunbx.ttf,ItalicFont=cmunti.ttf,BoldItalicFont=cmunbi.ttf]{cmunbx.ttf}\setmonofont[Path=/usr/share/fonts/truetype/cmu/,UprightFont=cmuntt.ttf,BoldFont=cmuntb.ttf,ItalicFont=cmunit.ttf,BoldItalicFont=cmuntx.ttf]{cmunbx.ttf}\bfseries natbib}&\hspace*{0pt}\ignorespaces{}\hspace*{0pt}{$\text{ }$}\setmainfont[Path=/usr/share/fonts/truetype/cmu/,UprightFont=cmunrm.ttf,BoldFont=cmunbx.ttf,ItalicFont=cmunti.ttf,BoldItalicFont=cmunbi.ttf]{cmunrm.ttf}\setmonofont[Path=/usr/share/fonts/truetype/cmu/,UprightFont=cmuntt.ttf,BoldFont=cmuntb.ttf,ItalicFont=cmunit.ttf,BoldItalicFont=cmuntx.ttf]{cmunrm.ttf} Gives additional citation options and styles.\\ \hline \hspace*{0pt}\ignorespaces{}\hspace*{0pt}{\bfseries \setmainfont[Path=/usr/share/fonts/truetype/cmu/,UprightFont=cmunrm.ttf,BoldFont=cmunbx.ttf,ItalicFont=cmunti.ttf,BoldItalicFont=cmunbi.ttf]{cmunbx.ttf}\setmonofont[Path=/usr/share/fonts/truetype/cmu/,UprightFont=cmuntt.ttf,BoldFont=cmuntb.ttf,ItalicFont=cmunit.ttf,BoldItalicFont=cmuntx.ttf]{cmunbx.ttf}\bfseries paralist}&\hspace*{0pt}\ignorespaces{}\hspace*{0pt}{$\text{ }$}\setmainfont[Path=/usr/share/fonts/truetype/cmu/,UprightFont=cmunrm.ttf,BoldFont=cmunbx.ttf,ItalicFont=cmunti.ttf,BoldItalicFont=cmunbi.ttf]{cmunrm.ttf}\setmonofont[Path=/usr/share/fonts/truetype/cmu/,UprightFont=cmuntt.ttf,BoldFont=cmuntb.ttf,ItalicFont=cmunit.ttf,BoldItalicFont=cmuntx.ttf]{cmunrm.ttf} provides compactitem environment which typesets list items much more closely than LaTeX\textquotesingle{}s default.\\ \hline \hspace*{0pt}\ignorespaces{}\hspace*{0pt}{\bfseries \setmainfont[Path=/usr/share/fonts/truetype/cmu/,UprightFont=cmunrm.ttf,BoldFont=cmunbx.ttf,ItalicFont=cmunti.ttf,BoldItalicFont=cmunbi.ttf]{cmunbx.ttf}\setmonofont[Path=/usr/share/fonts/truetype/cmu/,UprightFont=cmuntt.ttf,BoldFont=cmuntb.ttf,ItalicFont=cmunit.ttf,BoldItalicFont=cmuntx.ttf]{cmunbx.ttf}\bfseries pdfpages}&\hspace*{0pt}\ignorespaces{}\hspace*{0pt}{$\text{ }$}\setmainfont[Path=/usr/share/fonts/truetype/cmu/,UprightFont=cmunrm.ttf,BoldFont=cmunbx.ttf,ItalicFont=cmunti.ttf,BoldItalicFont=cmunbi.ttf]{cmunrm.ttf}\setmonofont[Path=/usr/share/fonts/truetype/cmu/,UprightFont=cmuntt.ttf,BoldFont=cmuntb.ttf,ItalicFont=cmunit.ttf,BoldItalicFont=cmuntx.ttf]{cmunrm.ttf} This package simplifies the insertion of external multi-{}page PDF or PS documents.\\ \hline \hspace*{0pt}\ignorespaces{}\hspace*{0pt}{\bfseries \setmainfont[Path=/usr/share/fonts/truetype/cmu/,UprightFont=cmunrm.ttf,BoldFont=cmunbx.ttf,ItalicFont=cmunti.ttf,BoldItalicFont=cmunbi.ttf]{cmunbx.ttf}\setmonofont[Path=/usr/share/fonts/truetype/cmu/,UprightFont=cmuntt.ttf,BoldFont=cmuntb.ttf,ItalicFont=cmunit.ttf,BoldItalicFont=cmuntx.ttf]{cmunbx.ttf}\bfseries rotating}&\hspace*{0pt}\ignorespaces{}\hspace*{0pt}{$\text{ }$}\setmainfont[Path=/usr/share/fonts/truetype/cmu/,UprightFont=cmunrm.ttf,BoldFont=cmunbx.ttf,ItalicFont=cmunti.ttf,BoldItalicFont=cmunbi.ttf]{cmunrm.ttf}\setmonofont[Path=/usr/share/fonts/truetype/cmu/,UprightFont=cmuntt.ttf,BoldFont=cmuntb.ttf,ItalicFont=cmunit.ttf,BoldItalicFont=cmuntx.ttf]{cmunrm.ttf} It lets you rotate any kind of object. It is particularly useful for rotating tables. For more information, see the \mylref{243}{relevant section}.\\ \hline \hspace*{0pt}\ignorespaces{}\hspace*{0pt}{\bfseries \setmainfont[Path=/usr/share/fonts/truetype/cmu/,UprightFont=cmunrm.ttf,BoldFont=cmunbx.ttf,ItalicFont=cmunti.ttf,BoldItalicFont=cmunbi.ttf]{cmunbx.ttf}\setmonofont[Path=/usr/share/fonts/truetype/cmu/,UprightFont=cmuntt.ttf,BoldFont=cmuntb.ttf,ItalicFont=cmunit.ttf,BoldItalicFont=cmuntx.ttf]{cmunbx.ttf}\bfseries setspace}&\hspace*{0pt}\ignorespaces{}\hspace*{0pt}{$\text{ }$}\setmainfont[Path=/usr/share/fonts/truetype/cmu/,UprightFont=cmunrm.ttf,BoldFont=cmunbx.ttf,ItalicFont=cmunti.ttf,BoldItalicFont=cmunbi.ttf]{cmunrm.ttf}\setmonofont[Path=/usr/share/fonts/truetype/cmu/,UprightFont=cmuntt.ttf,BoldFont=cmuntb.ttf,ItalicFont=cmunit.ttf,BoldItalicFont=cmuntx.ttf]{cmunrm.ttf} Lets you change line spacing, e.g. provides the \LaTeXTT{\textbackslash{}doublespacing} command for making double spaced documents. For more information, see the \mylref{109}{relevant section}.\\ \hline \hspace*{0pt}\ignorespaces{}\hspace*{0pt}{\bfseries \setmainfont[Path=/usr/share/fonts/truetype/cmu/,UprightFont=cmunrm.ttf,BoldFont=cmunbx.ttf,ItalicFont=cmunti.ttf,BoldItalicFont=cmunbi.ttf]{cmunbx.ttf}\setmonofont[Path=/usr/share/fonts/truetype/cmu/,UprightFont=cmuntt.ttf,BoldFont=cmuntb.ttf,ItalicFont=cmunit.ttf,BoldItalicFont=cmuntx.ttf]{cmunbx.ttf}\bfseries showkeys}&\hspace*{0pt}\ignorespaces{}\hspace*{0pt}{$\text{ }$}\setmainfont[Path=/usr/share/fonts/truetype/cmu/,UprightFont=cmunrm.ttf,BoldFont=cmunbx.ttf,ItalicFont=cmunti.ttf,BoldItalicFont=cmunbi.ttf]{cmunrm.ttf}\setmonofont[Path=/usr/share/fonts/truetype/cmu/,UprightFont=cmuntt.ttf,BoldFont=cmuntb.ttf,ItalicFont=cmunit.ttf,BoldItalicFont=cmuntx.ttf]{cmunrm.ttf} A useful package related to referencing. If you wish to reference an image or formula, you have to give it a name using {\ttfamily \setmainfont[Path=/usr/share/fonts/truetype/cmu/,UprightFont=cmunrm.ttf,BoldFont=cmunbx.ttf,ItalicFont=cmunti.ttf,BoldItalicFont=cmunbi.ttf]{cmuntt.ttf}\setmonofont[Path=/usr/share/fonts/truetype/cmu/,UprightFont=cmuntt.ttf,BoldFont=cmuntb.ttf,ItalicFont=cmunit.ttf,BoldItalicFont=cmuntx.ttf]{cmuntt.ttf}\ttfamily \textbackslash{}label\{...\}}{$\text{ }$}\setmainfont[Path=/usr/share/fonts/truetype/cmu/,UprightFont=cmunrm.ttf,BoldFont=cmunbx.ttf,ItalicFont=cmunti.ttf,BoldItalicFont=cmunbi.ttf]{cmunrm.ttf}\setmonofont[Path=/usr/share/fonts/truetype/cmu/,UprightFont=cmuntt.ttf,BoldFont=cmuntb.ttf,ItalicFont=cmunit.ttf,BoldItalicFont=cmuntx.ttf]{cmunrm.ttf} and then you can recall it using \LaTeXTT{\textbackslash{}ref\{...\}}. When you compile the document these will be replaced only with numbers, and you can\textquotesingle{}t know which label you had used unless you take a look at the source. If you have loaded the \LaTeXTT{showkeys} package, you will see the label just next or above the relevant number in the compiled version. An example of a reference to a section is \begin{minipage}{1.0\linewidth}\begin{center}\includegraphics[width=1.0\linewidth,height=6.5in,keepaspectratio]{../images/219.png}\end{center}\myfigurewithoutcaption{219}\end{minipage}. This way you can easily keep track of the labels you add or use, simply looking at the preview (both {\itshape \setmainfont[Path=/usr/share/fonts/truetype/cmu/,UprightFont=cmunrm.ttf,BoldFont=cmunbx.ttf,ItalicFont=cmunti.ttf,BoldItalicFont=cmunbi.ttf]{cmunti.ttf}\setmonofont[Path=/usr/share/fonts/truetype/cmu/,UprightFont=cmuntt.ttf,BoldFont=cmuntb.ttf,ItalicFont=cmunit.ttf,BoldItalicFont=cmuntx.ttf]{cmunti.ttf}\itshape dvi}{$\text{ }$}\setmainfont[Path=/usr/share/fonts/truetype/cmu/,UprightFont=cmunrm.ttf,BoldFont=cmunbx.ttf,ItalicFont=cmunti.ttf,BoldItalicFont=cmunbi.ttf]{cmunrm.ttf}\setmonofont[Path=/usr/share/fonts/truetype/cmu/,UprightFont=cmuntt.ttf,BoldFont=cmuntb.ttf,ItalicFont=cmunit.ttf,BoldItalicFont=cmuntx.ttf]{cmunrm.ttf} or {\itshape \setmainfont[Path=/usr/share/fonts/truetype/cmu/,UprightFont=cmunrm.ttf,BoldFont=cmunbx.ttf,ItalicFont=cmunti.ttf,BoldItalicFont=cmunbi.ttf]{cmunti.ttf}\setmonofont[Path=/usr/share/fonts/truetype/cmu/,UprightFont=cmuntt.ttf,BoldFont=cmuntb.ttf,ItalicFont=cmunit.ttf,BoldItalicFont=cmuntx.ttf]{cmunti.ttf}\itshape pdf}\setmainfont[Path=/usr/share/fonts/truetype/cmu/,UprightFont=cmunrm.ttf,BoldFont=cmunbx.ttf,ItalicFont=cmunti.ttf,BoldItalicFont=cmunbi.ttf]{cmunrm.ttf}\setmonofont[Path=/usr/share/fonts/truetype/cmu/,UprightFont=cmuntt.ttf,BoldFont=cmuntb.ttf,ItalicFont=cmunit.ttf,BoldItalicFont=cmuntx.ttf]{cmunrm.ttf}). Just before the final version, remove it.\\ \hline \hspace*{0pt}\ignorespaces{}\hspace*{0pt}{\bfseries \setmainfont[Path=/usr/share/fonts/truetype/cmu/,UprightFont=cmunrm.ttf,BoldFont=cmunbx.ttf,ItalicFont=cmunti.ttf,BoldItalicFont=cmunbi.ttf]{cmunbx.ttf}\setmonofont[Path=/usr/share/fonts/truetype/cmu/,UprightFont=cmuntt.ttf,BoldFont=cmuntb.ttf,ItalicFont=cmunit.ttf,BoldItalicFont=cmuntx.ttf]{cmunbx.ttf}\bfseries showidx}&\hspace*{0pt}\ignorespaces{}\hspace*{0pt}{$\text{ }$}\setmainfont[Path=/usr/share/fonts/truetype/cmu/,UprightFont=cmunrm.ttf,BoldFont=cmunbx.ttf,ItalicFont=cmunti.ttf,BoldItalicFont=cmunbi.ttf]{cmunrm.ttf}\setmonofont[Path=/usr/share/fonts/truetype/cmu/,UprightFont=cmuntt.ttf,BoldFont=cmuntb.ttf,ItalicFont=cmunit.ttf,BoldItalicFont=cmuntx.ttf]{cmunrm.ttf} It prints out all index entries in the left margin of the text. This is quite useful for proofreading a document and verifying the index. For more information, see the \mylref{626}{Indexing} section.\\ \hline \hspace*{0pt}\ignorespaces{}\hspace*{0pt}{\bfseries \setmainfont[Path=/usr/share/fonts/truetype/cmu/,UprightFont=cmunrm.ttf,BoldFont=cmunbx.ttf,ItalicFont=cmunti.ttf,BoldItalicFont=cmunbi.ttf]{cmunbx.ttf}\setmonofont[Path=/usr/share/fonts/truetype/cmu/,UprightFont=cmuntt.ttf,BoldFont=cmuntb.ttf,ItalicFont=cmunit.ttf,BoldItalicFont=cmuntx.ttf]{cmunbx.ttf}\bfseries subfiles}&\hspace*{0pt}\ignorespaces{}\hspace*{0pt}{$\text{ }$}\setmainfont[Path=/usr/share/fonts/truetype/cmu/,UprightFont=cmunrm.ttf,BoldFont=cmunbx.ttf,ItalicFont=cmunti.ttf,BoldItalicFont=cmunbi.ttf]{cmunrm.ttf}\setmonofont[Path=/usr/share/fonts/truetype/cmu/,UprightFont=cmuntt.ttf,BoldFont=cmuntb.ttf,ItalicFont=cmunit.ttf,BoldItalicFont=cmuntx.ttf]{cmunrm.ttf} The \symbol{34}root\symbol{34} and \symbol{34}child\symbol{34} document can be compiled at the same time without making changes to the \symbol{34}child\symbol{34} document. For more information, see the \mylref{895}{Modular Documents} section.\\ \hline \hspace*{0pt}\ignorespaces{}\hspace*{0pt}{\bfseries \setmainfont[Path=/usr/share/fonts/truetype/cmu/,UprightFont=cmunrm.ttf,BoldFont=cmunbx.ttf,ItalicFont=cmunti.ttf,BoldItalicFont=cmunbi.ttf]{cmunbx.ttf}\setmonofont[Path=/usr/share/fonts/truetype/cmu/,UprightFont=cmuntt.ttf,BoldFont=cmuntb.ttf,ItalicFont=cmunit.ttf,BoldItalicFont=cmuntx.ttf]{cmunbx.ttf}\bfseries subcaption}&\hspace*{0pt}\ignorespaces{}\hspace*{0pt}{$\text{ }$}\setmainfont[Path=/usr/share/fonts/truetype/cmu/,UprightFont=cmunrm.ttf,BoldFont=cmunbx.ttf,ItalicFont=cmunti.ttf,BoldItalicFont=cmunbi.ttf]{cmunrm.ttf}\setmonofont[Path=/usr/share/fonts/truetype/cmu/,UprightFont=cmuntt.ttf,BoldFont=cmuntb.ttf,ItalicFont=cmunit.ttf,BoldItalicFont=cmuntx.ttf]{cmunrm.ttf} It allows to define multiple floats (figures, tables) within one environment giving individual captions and labels in the form 1a, 1b.\\ \hline \hspace*{0pt}\ignorespaces{}\hspace*{0pt}{\bfseries \setmainfont[Path=/usr/share/fonts/truetype/cmu/,UprightFont=cmunrm.ttf,BoldFont=cmunbx.ttf,ItalicFont=cmunti.ttf,BoldItalicFont=cmunbi.ttf]{cmunbx.ttf}\setmonofont[Path=/usr/share/fonts/truetype/cmu/,UprightFont=cmuntt.ttf,BoldFont=cmuntb.ttf,ItalicFont=cmunit.ttf,BoldItalicFont=cmuntx.ttf]{cmunbx.ttf}\bfseries syntonly}&\hspace*{0pt}\ignorespaces{}\hspace*{0pt}{$\text{ }$}\setmainfont[Path=/usr/share/fonts/truetype/cmu/,UprightFont=cmunrm.ttf,BoldFont=cmunbx.ttf,ItalicFont=cmunti.ttf,BoldItalicFont=cmunbi.ttf]{cmunrm.ttf}\setmonofont[Path=/usr/share/fonts/truetype/cmu/,UprightFont=cmuntt.ttf,BoldFont=cmuntb.ttf,ItalicFont=cmunit.ttf,BoldItalicFont=cmuntx.ttf]{cmunrm.ttf} If you add the following code in your preamble:\TemplateSource{\newline{}\textbackslash{}usepackage\{syntonly\}\newline{}\textbackslash{}syntaxonly\newline{}}LaTeX skims through your document only checking for proper syntax and usage of the commands, but doesn’t produce any (DVI or PDF) output. As LaTeX runs faster in this mode you may save yourself valuable time. If you want to get the output, you can simply comment out the second line.\\ \hline \hspace*{0pt}\ignorespaces{}\hspace*{0pt}{\bfseries \setmainfont[Path=/usr/share/fonts/truetype/cmu/,UprightFont=cmunrm.ttf,BoldFont=cmunbx.ttf,ItalicFont=cmunti.ttf,BoldItalicFont=cmunbi.ttf]{cmunbx.ttf}\setmonofont[Path=/usr/share/fonts/truetype/cmu/,UprightFont=cmuntt.ttf,BoldFont=cmuntb.ttf,ItalicFont=cmunit.ttf,BoldItalicFont=cmuntx.ttf]{cmunbx.ttf}\bfseries textcomp}&\hspace*{0pt}\ignorespaces{}\hspace*{0pt}{$\text{ }$}\setmainfont[Path=/usr/share/fonts/truetype/cmu/,UprightFont=cmunrm.ttf,BoldFont=cmunbx.ttf,ItalicFont=cmunti.ttf,BoldItalicFont=cmunbi.ttf]{cmunrm.ttf}\setmonofont[Path=/usr/share/fonts/truetype/cmu/,UprightFont=cmuntt.ttf,BoldFont=cmuntb.ttf,ItalicFont=cmunit.ttf,BoldItalicFont=cmuntx.ttf]{cmunrm.ttf} Provides extra symbols, e.g. arrows like \LaTeXTT{\textbackslash{}textrightarrow}, various currencies (\LaTeXTT{\textbackslash{}texteuro},...), things like \LaTeXTT{\textbackslash{}textcelsius} and many others.\\ \hline \hspace*{0pt}\ignorespaces{}\hspace*{0pt}{\bfseries \setmainfont[Path=/usr/share/fonts/truetype/cmu/,UprightFont=cmunrm.ttf,BoldFont=cmunbx.ttf,ItalicFont=cmunti.ttf,BoldItalicFont=cmunbi.ttf]{cmunbx.ttf}\setmonofont[Path=/usr/share/fonts/truetype/cmu/,UprightFont=cmuntt.ttf,BoldFont=cmuntb.ttf,ItalicFont=cmunit.ttf,BoldItalicFont=cmuntx.ttf]{cmunbx.ttf}\bfseries theorem}&\hspace*{0pt}\ignorespaces{}\hspace*{0pt}{$\text{ }$}\setmainfont[Path=/usr/share/fonts/truetype/cmu/,UprightFont=cmunrm.ttf,BoldFont=cmunbx.ttf,ItalicFont=cmunti.ttf,BoldItalicFont=cmunbi.ttf]{cmunrm.ttf}\setmonofont[Path=/usr/share/fonts/truetype/cmu/,UprightFont=cmuntt.ttf,BoldFont=cmuntb.ttf,ItalicFont=cmunit.ttf,BoldItalicFont=cmuntx.ttf]{cmunrm.ttf} You can change the style of newly defined theorems. For more information see the \mylref{557}{Theorems} section.\\ \hline \hspace*{0pt}\ignorespaces{}\hspace*{0pt}{\bfseries \setmainfont[Path=/usr/share/fonts/truetype/cmu/,UprightFont=cmunrm.ttf,BoldFont=cmunbx.ttf,ItalicFont=cmunti.ttf,BoldItalicFont=cmunbi.ttf]{cmunbx.ttf}\setmonofont[Path=/usr/share/fonts/truetype/cmu/,UprightFont=cmuntt.ttf,BoldFont=cmuntb.ttf,ItalicFont=cmunit.ttf,BoldItalicFont=cmuntx.ttf]{cmunbx.ttf}\bfseries todonotes}&\hspace*{0pt}\ignorespaces{}\hspace*{0pt}{$\text{ }$}\setmainfont[Path=/usr/share/fonts/truetype/cmu/,UprightFont=cmunrm.ttf,BoldFont=cmunbx.ttf,ItalicFont=cmunti.ttf,BoldItalicFont=cmunbi.ttf]{cmunrm.ttf}\setmonofont[Path=/usr/share/fonts/truetype/cmu/,UprightFont=cmuntt.ttf,BoldFont=cmuntb.ttf,ItalicFont=cmunit.ttf,BoldItalicFont=cmuntx.ttf]{cmunrm.ttf} Lets you insert notes of stuff to do with the syntax \LaTeXTT{\textbackslash{}todo\{Add details.\}}.\\ \hline \hspace*{0pt}\ignorespaces{}\hspace*{0pt}\myhref{http://ctan.org/tex-archive/macros/latex/contrib/siunitx}{{\bfseries \setmainfont[Path=/usr/share/fonts/truetype/cmu/,UprightFont=cmunrm.ttf,BoldFont=cmunbx.ttf,ItalicFont=cmunti.ttf,BoldItalicFont=cmunbi.ttf]{cmunbx.ttf}\setmonofont[Path=/usr/share/fonts/truetype/cmu/,UprightFont=cmuntt.ttf,BoldFont=cmuntb.ttf,ItalicFont=cmunit.ttf,BoldItalicFont=cmuntx.ttf]{cmunbx.ttf}\bfseries siunitx}}&\hspace*{0pt}\ignorespaces{}\hspace*{0pt} Helps you typeset of SI-{}units correctly. For example \LaTeXTT{\textbackslash{}SI\{12\}\{\textbackslash{}mega\textbackslash{}hertz\}}. Automatically handles the correct spacing between the number and the unit. Note that even non-{}SI-{}units are set, like dB, rad, ...\\ \hline \hspace*{0pt}\ignorespaces{}\hspace*{0pt}{\bfseries \setmainfont[Path=/usr/share/fonts/truetype/cmu/,UprightFont=cmunrm.ttf,BoldFont=cmunbx.ttf,ItalicFont=cmunti.ttf,BoldItalicFont=cmunbi.ttf]{cmunbx.ttf}\setmonofont[Path=/usr/share/fonts/truetype/cmu/,UprightFont=cmuntt.ttf,BoldFont=cmuntb.ttf,ItalicFont=cmunit.ttf,BoldItalicFont=cmuntx.ttf]{cmunbx.ttf}\bfseries ulem}&\hspace*{0pt}\ignorespaces{}\hspace*{0pt}{$\text{ }$}\setmainfont[Path=/usr/share/fonts/truetype/cmu/,UprightFont=cmunrm.ttf,BoldFont=cmunbx.ttf,ItalicFont=cmunti.ttf,BoldItalicFont=cmunbi.ttf]{cmunrm.ttf}\setmonofont[Path=/usr/share/fonts/truetype/cmu/,UprightFont=cmuntt.ttf,BoldFont=cmuntb.ttf,ItalicFont=cmunit.ttf,BoldItalicFont=cmuntx.ttf]{cmunrm.ttf} It allows to underline text (either with straight or wavy line). Few examples of usage are added to the \mylref{163}{Fonts} chapter.\\ \hline \hspace*{0pt}\ignorespaces{}\hspace*{0pt}{\bfseries \setmainfont[Path=/usr/share/fonts/truetype/cmu/,UprightFont=cmunrm.ttf,BoldFont=cmunbx.ttf,ItalicFont=cmunti.ttf,BoldItalicFont=cmunbi.ttf]{cmunbx.ttf}\setmonofont[Path=/usr/share/fonts/truetype/cmu/,UprightFont=cmuntt.ttf,BoldFont=cmuntb.ttf,ItalicFont=cmunit.ttf,BoldItalicFont=cmuntx.ttf]{cmunbx.ttf}\bfseries url}&\hspace*{0pt}\ignorespaces{}\hspace*{0pt}{$\text{ }$}\setmainfont[Path=/usr/share/fonts/truetype/cmu/,UprightFont=cmunrm.ttf,BoldFont=cmunbx.ttf,ItalicFont=cmunti.ttf,BoldItalicFont=cmunbi.ttf]{cmunrm.ttf}\setmonofont[Path=/usr/share/fonts/truetype/cmu/,UprightFont=cmuntt.ttf,BoldFont=cmuntb.ttf,ItalicFont=cmunit.ttf,BoldItalicFont=cmuntx.ttf]{cmunrm.ttf} It defines the \LaTeXTT{\textbackslash{}url\{...\}} command. URLs often contain special character such as \textquotesingle{}\_\textquotesingle{} and \textquotesingle{}\&\textquotesingle{}, in order to write them you should {\itshape \setmainfont[Path=/usr/share/fonts/truetype/cmu/,UprightFont=cmunrm.ttf,BoldFont=cmunbx.ttf,ItalicFont=cmunti.ttf,BoldItalicFont=cmunbi.ttf]{cmunti.ttf}\setmonofont[Path=/usr/share/fonts/truetype/cmu/,UprightFont=cmuntt.ttf,BoldFont=cmuntb.ttf,ItalicFont=cmunit.ttf,BoldItalicFont=cmuntx.ttf]{cmunti.ttf}\itshape escape}{$\text{ }$}\setmainfont[Path=/usr/share/fonts/truetype/cmu/,UprightFont=cmunrm.ttf,BoldFont=cmunbx.ttf,ItalicFont=cmunti.ttf,BoldItalicFont=cmunbi.ttf]{cmunrm.ttf}\setmonofont[Path=/usr/share/fonts/truetype/cmu/,UprightFont=cmuntt.ttf,BoldFont=cmuntb.ttf,ItalicFont=cmunit.ttf,BoldItalicFont=cmuntx.ttf]{cmunrm.ttf} them inserting a backslash, but if you write them as an argument of \LaTeXTT{\textbackslash{}url\{...\}}, you don\textquotesingle{}t need to escape any special character and it will take care of proper formatting for you. If you are using \LaTeXTT{hyperref}, you don\textquotesingle{}t need to load \LaTeXTT{url} because it already provides the \LaTeXTT{\textbackslash{}url\{...\}} command.\\ \hline \hspace*{0pt}\ignorespaces{}\hspace*{0pt}{\bfseries \setmainfont[Path=/usr/share/fonts/truetype/cmu/,UprightFont=cmunrm.ttf,BoldFont=cmunbx.ttf,ItalicFont=cmunti.ttf,BoldItalicFont=cmunbi.ttf]{cmunbx.ttf}\setmonofont[Path=/usr/share/fonts/truetype/cmu/,UprightFont=cmuntt.ttf,BoldFont=cmuntb.ttf,ItalicFont=cmunit.ttf,BoldItalicFont=cmuntx.ttf]{cmunbx.ttf}\bfseries verbatim}&\hspace*{0pt}\ignorespaces{}\hspace*{0pt}{$\text{ }$}\setmainfont[Path=/usr/share/fonts/truetype/cmu/,UprightFont=cmunrm.ttf,BoldFont=cmunbx.ttf,ItalicFont=cmunti.ttf,BoldItalicFont=cmunbi.ttf]{cmunrm.ttf}\setmonofont[Path=/usr/share/fonts/truetype/cmu/,UprightFont=cmuntt.ttf,BoldFont=cmuntb.ttf,ItalicFont=cmunit.ttf,BoldItalicFont=cmuntx.ttf]{cmunrm.ttf} It improves the \LaTeXTT{verbatim} environment, fixing some bugs. Moreover, it provides the \LaTeXTT{comment} environment, that lets you add multiple-{}line comments or easily comment out big parts of the code.\\ \hline \hspace*{0pt}\ignorespaces{}\hspace*{0pt}{\bfseries \setmainfont[Path=/usr/share/fonts/truetype/cmu/,UprightFont=cmunrm.ttf,BoldFont=cmunbx.ttf,ItalicFont=cmunti.ttf,BoldItalicFont=cmunbi.ttf]{cmunbx.ttf}\setmonofont[Path=/usr/share/fonts/truetype/cmu/,UprightFont=cmuntt.ttf,BoldFont=cmuntb.ttf,ItalicFont=cmunit.ttf,BoldItalicFont=cmuntx.ttf]{cmunbx.ttf}\bfseries wrapfig}&\hspace*{0pt}\ignorespaces{}\hspace*{0pt}{$\text{ }$}\setmainfont[Path=/usr/share/fonts/truetype/cmu/,UprightFont=cmunrm.ttf,BoldFont=cmunbx.ttf,ItalicFont=cmunti.ttf,BoldItalicFont=cmunbi.ttf]{cmunrm.ttf}\setmonofont[Path=/usr/share/fonts/truetype/cmu/,UprightFont=cmuntt.ttf,BoldFont=cmuntb.ttf,ItalicFont=cmunit.ttf,BoldItalicFont=cmuntx.ttf]{cmunrm.ttf} To insert images surrounded by text. It was discussed in section \mylref{362}{Floats, Figures and Captions}.\\ \hline \hspace*{0pt}\ignorespaces{}\hspace*{0pt}{\bfseries \setmainfont[Path=/usr/share/fonts/truetype/cmu/,UprightFont=cmunrm.ttf,BoldFont=cmunbx.ttf,ItalicFont=cmunti.ttf,BoldItalicFont=cmunbi.ttf]{cmunbx.ttf}\setmonofont[Path=/usr/share/fonts/truetype/cmu/,UprightFont=cmuntt.ttf,BoldFont=cmuntb.ttf,ItalicFont=cmunit.ttf,BoldItalicFont=cmuntx.ttf]{cmunbx.ttf}\bfseries xypic}&\hspace*{0pt}\ignorespaces{}\hspace*{0pt}{$\text{ }$}\setmainfont[Path=/usr/share/fonts/truetype/cmu/,UprightFont=cmunrm.ttf,BoldFont=cmunbx.ttf,ItalicFont=cmunti.ttf,BoldItalicFont=cmunbi.ttf]{cmunrm.ttf}\setmonofont[Path=/usr/share/fonts/truetype/cmu/,UprightFont=cmuntt.ttf,BoldFont=cmuntb.ttf,ItalicFont=cmunit.ttf,BoldItalicFont=cmuntx.ttf]{cmunrm.ttf} It is used to create trees, graphs, (commutative) diagrams, and similar things. See \mylref{833}{Xy-{}pic}.\\ \hline 
\end{longtable}
}

\chapter{Sample LaTeX documents}

\myminitoc
\label{992}

\label{993}


The easiest way to learn how to use latex is to look at how other people use it. Here is a list of {\itshape \setmainfont[Path=/usr/share/fonts/truetype/cmu/,UprightFont=cmunrm.ttf,BoldFont=cmunbx.ttf,ItalicFont=cmunti.ttf,BoldItalicFont=cmunbi.ttf]{cmunti.ttf}\setmonofont[Path=/usr/share/fonts/truetype/cmu/,UprightFont=cmuntt.ttf,BoldFont=cmuntb.ttf,ItalicFont=cmunit.ttf,BoldItalicFont=cmuntx.ttf]{cmunti.ttf}\itshape real world}{$\text{ }$}\setmainfont[Path=/usr/share/fonts/truetype/cmu/,UprightFont=cmunrm.ttf,BoldFont=cmunbx.ttf,ItalicFont=cmunti.ttf,BoldItalicFont=cmunbi.ttf]{cmunrm.ttf}\setmonofont[Path=/usr/share/fonts/truetype/cmu/,UprightFont=cmuntt.ttf,BoldFont=cmuntb.ttf,ItalicFont=cmunit.ttf,BoldItalicFont=cmuntx.ttf]{cmunrm.ttf} latex sources that are freely available on the internet. The information here is sorted by application area, so that it is grouped by the scientific communities that use similar notation and LaTeX constructs.
\section{General examples}
\label{994}
Tutorial examples, books, and real world uses of LaTeX.
\begin{myitemize}
\item{}  \myhref{https://en.wikibooks.org/wiki/LaTeX\%2Fcaption.tex}{caption.tex}, \myhref{https://en.wikibooks.org/wiki/LaTeX\%2Fsimple.tex}{simple.tex}, \myhref{https://en.wikibooks.org/wiki/LaTeX\%2Fwrapped.tex}{wrapped.tex}
\item{}  {$\text{[}$}ftp://tug.ctan.org/tex-{}archive/macros/latex/base/small2e.tex small2e.tex{$\text{]}$} and {$\text{[}$}ftp://tug.ctan.org/tex-{}archive/macros/latex/base/sample2e.tex sample2e.tex{$\text{]}$}. The \symbol{34}official\symbol{34} sample documents...
\item{}  \myhref{http://www.tug.org/pracjourn/2006-2/eglen/}{A short example of how to use LaTeX for scientific reports} by Stephen J. Eglen.
\item{}  \myhref{http://www.ctan.org/tex-archive/info/lshort/english/}{The not so Short Introduction to LaTeX} by Tobias Oetiker is distributed with full latex sources.
\end{myitemize}

\section{Semantics of Programming Languages}
\label{995}

Articles on programming language research, from syntax to semantics, including source code listings, type rules, proof trees, and even some category theory. A good place to start is \myhref{http://www.ccs.neu.edu/course/csg264/latex/}{Mitchell Wand\textquotesingle{}s Latex Resources}, including a sample file that also demonstrates Didier Remy\textquotesingle{}s \myhref{http://cristal.inria.fr/~remy/latex/}{mathpartir} package. The following are latex sources of some articles, books, or presentations from this field:
\begin{myitemize}
\item{}  \myhref{http://svn.openfoundry.org/pugs/docs/talks/hw2005.tex}{Pugs: Bootstrapping Perl 6 with Haskell}. This paper by Audrey Tang contains nice examples on configuring the \myhref{https://en.wikibooks.org/wiki/LaTeX\%2FPackages\%2FListings}{listings package} to format source code.
\end{myitemize}


\chapter{Index}

\myminitoc
\label{996}

\label{997}


This is an alphabetical index of the book.

\LaTeXNullTemplate{}
\section{A}
\label{998}
\begin{myitemize}
\item{}  \mylref{63}{Absolute Beginners}
\item{}  \mylref{144}{Abstract}
\item{}  \mylref{140}{Accents}
\item{}  \mylref{581}{Algorithms}
\item{}  \mylref{513}{Arrays}
\item{}  \mylref{981}{Authors}
\end{myitemize}

\section{B}
\label{999}
\begin{myitemize}
\item{}  \mylref{209}{{\ttfamily \setmainfont[Path=/usr/share/fonts/truetype/cmu/,UprightFont=cmunrm.ttf,BoldFont=cmunbx.ttf,ItalicFont=cmunti.ttf,BoldItalicFont=cmunbi.ttf]{cmuntt.ttf}\setmonofont[Path=/usr/share/fonts/truetype/cmu/,UprightFont=cmuntt.ttf,BoldFont=cmuntb.ttf,ItalicFont=cmunit.ttf,BoldItalicFont=cmuntx.ttf]{cmuntt.ttf}\ttfamily babel}}
\item{}  \mylref{63}{Basics}
\item{}  \mylref{729}{{\ttfamily \setmainfont[Path=/usr/share/fonts/truetype/cmu/,UprightFont=cmunrm.ttf,BoldFont=cmunbx.ttf,ItalicFont=cmunti.ttf,BoldItalicFont=cmunbi.ttf]{cmuntt.ttf}\setmonofont[Path=/usr/share/fonts/truetype/cmu/,UprightFont=cmuntt.ttf,BoldFont=cmuntb.ttf,ItalicFont=cmunit.ttf,BoldItalicFont=cmuntx.ttf]{cmuntt.ttf}\ttfamily beamer}{$\text{ }$}\setmainfont[Path=/usr/share/fonts/truetype/cmu/,UprightFont=cmunrm.ttf,BoldFont=cmunbx.ttf,ItalicFont=cmunti.ttf,BoldItalicFont=cmunbi.ttf]{cmunrm.ttf}\setmonofont[Path=/usr/share/fonts/truetype/cmu/,UprightFont=cmuntt.ttf,BoldFont=cmuntb.ttf,ItalicFont=cmunit.ttf,BoldItalicFont=cmuntx.ttf]{cmunrm.ttf} package}
\item{}  \mylref{667}{Bibliography Management}
\item{}  \mylref{667}{BibTeX}
\item{}  \mylref{140}{Bold}
\item{}  \mylref{186}{Bullets}
\item{}  \mylref{186}{Bullet points}
\end{myitemize}

\section{C}
\label{1000}
\begin{myitemize}
\item{}  \mylref{362}{Captions}
\item{}  \mylref{911}{Collaborative Writing of LaTeX Documents}
\item{}  \mylref{147}{Color}
\item{}  \mylref{147}{{\ttfamily \setmainfont[Path=/usr/share/fonts/truetype/cmu/,UprightFont=cmunrm.ttf,BoldFont=cmunbx.ttf,ItalicFont=cmunti.ttf,BoldItalicFont=cmunbi.ttf]{cmuntt.ttf}\setmonofont[Path=/usr/share/fonts/truetype/cmu/,UprightFont=cmuntt.ttf,BoldFont=cmuntb.ttf,ItalicFont=cmunit.ttf,BoldItalicFont=cmuntx.ttf]{cmuntt.ttf}\ttfamily color}{$\text{ }$}\setmainfont[Path=/usr/share/fonts/truetype/cmu/,UprightFont=cmunrm.ttf,BoldFont=cmunbx.ttf,ItalicFont=cmunti.ttf,BoldItalicFont=cmunbi.ttf]{cmunrm.ttf}\setmonofont[Path=/usr/share/fonts/truetype/cmu/,UprightFont=cmuntt.ttf,BoldFont=cmuntb.ttf,ItalicFont=cmunit.ttf,BoldItalicFont=cmuntx.ttf]{cmunrm.ttf} package}
\item{}  Columns, see \mylref{328}{Multi-{}column Pages}
\item{}  \mylref{417}{Cross-{}referencing}
\item{}  \mylref{837}{Customizing LaTeX}
\end{myitemize}

\section{D}
\label{1001}
\begin{myitemize}
\item{}  \mylref{140}{Dashes}
\item{}  \mylref{140}{description environment}
\item{}  \mylref{140}{Diactrical marks}
\item{}  \mylref{87}{Document Classes}
\item{}  \mylref{89}{Document Structure}
\item{}  \mylref{774}{Drawings}
\end{myitemize}

\section{E}
\label{1002}
\begin{myitemize}
\item{}  \mylref{970}{e.g. (exempli gratia)}
\item{}  \mylref{140}{Ellipsis}
\item{}  \mylref{140}{em-{}dash}
\item{}  \mylref{140}{en-{}dash}
\item{}  \mylref{140}{enumerate}
\item{}  \mylref{439}{Errors and Warnings}
\item{}  \mylref{140}{Euro currency symbol}
\item{}  \mylref{925}{Export To Other Formats}
\end{myitemize}

\section{F}
\label{1003}
\begin{myitemize}
\item{}  \mylref{362}{Figures}
\item{}  \mylref{362}{Floats}
\item{}  \mylref{163}{Fonts}
\item{}  \mylref{303}{Footer, Page}
\item{}  \mylref{140}{Footnotes}
\item{}  \mylref{131}{Formatting}
\end{myitemize}

\section{G}
\label{1004}
\begin{myitemize}
\item{}  \mylref{895}{General Guidelines}
\item{}  Graphics
\begin{myitemize}
\item{}  \mylref{774}{Creating}
\item{}  \mylref{336}{Embedding}
\item{}  \mylref{336}{Importing}
\end{myitemize}

\item{}  \mylref{336}{{\ttfamily \setmainfont[Path=/usr/share/fonts/truetype/cmu/,UprightFont=cmunrm.ttf,BoldFont=cmunbx.ttf,ItalicFont=cmunti.ttf,BoldItalicFont=cmunbi.ttf]{cmuntt.ttf}\setmonofont[Path=/usr/share/fonts/truetype/cmu/,UprightFont=cmuntt.ttf,BoldFont=cmuntb.ttf,ItalicFont=cmunit.ttf,BoldItalicFont=cmuntx.ttf]{cmuntt.ttf}\ttfamily graphicx}{$\text{ }$}\setmainfont[Path=/usr/share/fonts/truetype/cmu/,UprightFont=cmunrm.ttf,BoldFont=cmunbx.ttf,ItalicFont=cmunti.ttf,BoldItalicFont=cmunbi.ttf]{cmunrm.ttf}\setmonofont[Path=/usr/share/fonts/truetype/cmu/,UprightFont=cmuntt.ttf,BoldFont=cmuntb.ttf,ItalicFont=cmunit.ttf,BoldItalicFont=cmuntx.ttf]{cmunrm.ttf} package}
\end{myitemize}

\section{H}
\label{1005}
\begin{myitemize}
\item{}  \mylref{303}{Header, Page}
\item{}  \mylref{938}{HTML output}
\item{}  \mylref{392}{Hyperlinks}
\item{}  \mylref{392}{{\ttfamily \setmainfont[Path=/usr/share/fonts/truetype/cmu/,UprightFont=cmunrm.ttf,BoldFont=cmunbx.ttf,ItalicFont=cmunti.ttf,BoldItalicFont=cmunbi.ttf]{cmuntt.ttf}\setmonofont[Path=/usr/share/fonts/truetype/cmu/,UprightFont=cmuntt.ttf,BoldFont=cmuntb.ttf,ItalicFont=cmunit.ttf,BoldItalicFont=cmuntx.ttf]{cmuntt.ttf}\ttfamily hyperref}{$\text{ }$}\setmainfont[Path=/usr/share/fonts/truetype/cmu/,UprightFont=cmunrm.ttf,BoldFont=cmunbx.ttf,ItalicFont=cmunti.ttf,BoldItalicFont=cmunbi.ttf]{cmunrm.ttf}\setmonofont[Path=/usr/share/fonts/truetype/cmu/,UprightFont=cmuntt.ttf,BoldFont=cmuntb.ttf,ItalicFont=cmunit.ttf,BoldItalicFont=cmuntx.ttf]{cmunrm.ttf} package}
\item{}  \mylref{140}{hyphen}
\item{}  \mylref{140}{Hyphenation}
\end{myitemize}

\section{I}
\label{1006}
\begin{myitemize}
\item{}  \mylref{970}{i.e. (id est)}
\item{}  \mylref{336}{Images}
\item{}  \mylref{336}{Importing Graphics}
\item{}  \mylref{626}{Indexing}
\item{}  \mylref{209}{Internationalization}
\item{}  \mylref{1}{Introduction}
\item{}  \mylref{140}{Italics}
\item{}  \mylref{140}{itemize}
\end{myitemize}

\section{L}
\label{1007}
\begin{myitemize}
\item{}  \mylref{417}{Labels}
\item{}  \mylref{719}{Letters}
\item{}  \mylref{985}{Links}
\item{}  \mylref{186}{Lists}
\end{myitemize}

\section{M}
\label{1008}
\begin{myitemize}
\item{}  \mylref{626}{{\ttfamily \setmainfont[Path=/usr/share/fonts/truetype/cmu/,UprightFont=cmunrm.ttf,BoldFont=cmunbx.ttf,ItalicFont=cmunti.ttf,BoldItalicFont=cmunbi.ttf]{cmuntt.ttf}\setmonofont[Path=/usr/share/fonts/truetype/cmu/,UprightFont=cmuntt.ttf,BoldFont=cmuntb.ttf,ItalicFont=cmunit.ttf,BoldItalicFont=cmuntx.ttf]{cmuntt.ttf}\ttfamily makeidx}{$\text{ }$}\setmainfont[Path=/usr/share/fonts/truetype/cmu/,UprightFont=cmunrm.ttf,BoldFont=cmunbx.ttf,ItalicFont=cmunti.ttf,BoldItalicFont=cmunbi.ttf]{cmunrm.ttf}\setmonofont[Path=/usr/share/fonts/truetype/cmu/,UprightFont=cmuntt.ttf,BoldFont=cmuntb.ttf,ItalicFont=cmunit.ttf,BoldItalicFont=cmuntx.ttf]{cmunrm.ttf} package}
\item{}  \mylref{95}{{\ttfamily \setmainfont[Path=/usr/share/fonts/truetype/cmu/,UprightFont=cmunrm.ttf,BoldFont=cmunbx.ttf,ItalicFont=cmunti.ttf,BoldItalicFont=cmunbi.ttf]{cmuntt.ttf}\setmonofont[Path=/usr/share/fonts/truetype/cmu/,UprightFont=cmuntt.ttf,BoldFont=cmuntb.ttf,ItalicFont=cmunit.ttf,BoldItalicFont=cmuntx.ttf]{cmuntt.ttf}\ttfamily \textbackslash{}maketitle}}
\item{}  \mylref{140}{Margin Notes}
\item{}  \mylref{774}{Creating Graphics}
\item{}  \mylref{497}{Mathematics}
\item{}  \mylref{513}{Matrices}
\item{}  \mylref{293}{Minipage environment example}
\item{}  \mylref{328}{Multi-{}column Pages}
\end{myitemize}

\section{P}
\label{1009}

\begin{myitemize}
\item{}  \mylref{991}{Packages}
\begin{myitemize}
\item{}  \mylref{837}{Creating 1}
\end{myitemize}

\item{}  \mylref{303}{Page Layout}
\item{}  \mylref{935}{PDF output}
\item{}  \mylref{774}{{\ttfamily \setmainfont[Path=/usr/share/fonts/truetype/cmu/,UprightFont=cmunrm.ttf,BoldFont=cmunbx.ttf,ItalicFont=cmunti.ttf,BoldItalicFont=cmunbi.ttf]{cmuntt.ttf}\setmonofont[Path=/usr/share/fonts/truetype/cmu/,UprightFont=cmuntt.ttf,BoldFont=cmuntb.ttf,ItalicFont=cmunit.ttf,BoldItalicFont=cmuntx.ttf]{cmuntt.ttf}\ttfamily picture}}
\item{}  \mylref{336}{Pictures}
\item{}  \mylref{935}{PNG output}
\item{}  \mylref{729}{Presentations}
\item{}  \mylref{581}{Pseudocode}
\end{myitemize}

\section{Q}
\label{1010}
\begin{myitemize}
\item{}  \mylref{143}{LaTeX/Paragraph Formatting\#Quoting\_text}
\end{myitemize}

\section{R}
\label{1011}
\begin{myitemize}
\item{}  \mylref{417}{References}
\item{}  \mylref{935}{RTF output}
\end{myitemize}

\section{S}
\label{1012}
\begin{myitemize}
\item{}  \mylref{113}{Sentences}
\item{}  \mylref{140}{Small Capitals}
\item{}  \mylref{140}{Source Code Listings}
\item{}  \mylref{140}{Space Between Words}
\item{}  \mylref{976}{Spell-{}checking}
\item{}  \mylref{503}{Superscript and subscript: powers and indices}
\item{}  \mylref{124}{Superscript and subscript: text mode}
\item{}  \mylref{935}{SVG output}
\end{myitemize}

\section{T}
\label{1013}
\begin{myitemize}
\item{}  \mylref{101}{Table of contents}
\item{}  \mylref{248}{Tables}
\item{}  \mylref{140}{Teletype text}
\item{}  \mylref{293}{Text Size}
\item{}  \mylref{557}{Theorems}
\item{}  \mylref{968}{Tips and Tricks}
\item{}  \mylref{293}{Title Creation}
\end{myitemize}

\section{U}
\label{1014}
\begin{myitemize}
\item{}  \mylref{139}{URLs}
\end{myitemize}

\section{V}
\label{1015}
\begin{myitemize}
\item{}  \mylref{138}{Verbatim Text}
\end{myitemize}

\section{W}
\label{1016}
\begin{myitemize}
\item{}  \mylref{976}{Word Counting}
\end{myitemize}

\section{X}
\label{1017}
\begin{myitemize}
\item{}  \mylref{180}{XeTeX}
\item{}  \mylref{833}{{\ttfamily \setmainfont[Path=/usr/share/fonts/truetype/cmu/,UprightFont=cmunrm.ttf,BoldFont=cmunbx.ttf,ItalicFont=cmunti.ttf,BoldItalicFont=cmunbi.ttf]{cmuntt.ttf}\setmonofont[Path=/usr/share/fonts/truetype/cmu/,UprightFont=cmuntt.ttf,BoldFont=cmuntb.ttf,ItalicFont=cmunit.ttf,BoldItalicFont=cmuntx.ttf]{cmuntt.ttf}\ttfamily XY-{}pic}{$\text{ }$}\setmainfont[Path=/usr/share/fonts/truetype/cmu/,UprightFont=cmunrm.ttf,BoldFont=cmunbx.ttf,ItalicFont=cmunti.ttf,BoldItalicFont=cmunbi.ttf]{cmunrm.ttf}\setmonofont[Path=/usr/share/fonts/truetype/cmu/,UprightFont=cmuntt.ttf,BoldFont=cmuntb.ttf,ItalicFont=cmunit.ttf,BoldItalicFont=cmuntx.ttf]{cmunrm.ttf} package}
\item{}  \mylref{833}{{\ttfamily \setmainfont[Path=/usr/share/fonts/truetype/cmu/,UprightFont=cmunrm.ttf,BoldFont=cmunbx.ttf,ItalicFont=cmunti.ttf,BoldItalicFont=cmunbi.ttf]{cmuntt.ttf}\setmonofont[Path=/usr/share/fonts/truetype/cmu/,UprightFont=cmuntt.ttf,BoldFont=cmuntb.ttf,ItalicFont=cmunit.ttf,BoldItalicFont=cmuntx.ttf]{cmuntt.ttf}\ttfamily xy}{$\text{ }$}\setmainfont[Path=/usr/share/fonts/truetype/cmu/,UprightFont=cmunrm.ttf,BoldFont=cmunbx.ttf,ItalicFont=cmunti.ttf,BoldItalicFont=cmunbi.ttf]{cmunrm.ttf}\setmonofont[Path=/usr/share/fonts/truetype/cmu/,UprightFont=cmuntt.ttf,BoldFont=cmuntb.ttf,ItalicFont=cmunit.ttf,BoldItalicFont=cmuntx.ttf]{cmunrm.ttf} package}
\end{myitemize}


\chapter{Command Glossary}

\myminitoc
\label{1018}

\label{1019}


This is a glossary of LaTeX commands—an alphabetical listing of LaTeX commands with the summaries of their effects. (Brackets \symbol{34}{$\text{[}$}{$\text{]}$}\symbol{34} are optional arguments and braces \symbol{34}\{\}\symbol{34} are required arguments.)

\LaTeXNullTemplate{}
\section{\#}
\label{1020}{\bfseries
\begin{mydescription} / 
\end{mydescription}
}
\begin{myquote}\item{} see \mylref{140}{slash marks}
\end{myquote}
{\bfseries
\begin{mydescription} \textbackslash{}@ 
\end{mydescription}
}
\begin{myquote}\item{} following period ends sentence
\end{myquote}
{\bfseries
\begin{mydescription} \textbackslash{}\textbackslash{}{$\text{[}$}*{$\text{]}$}{$\text{[}$}extra-{}space{$\text{]}$} 
\end{mydescription}
}
\begin{myquote}\item{} new line
\end{myquote}
{\bfseries
\begin{mydescription} \textbackslash{}, 
\end{mydescription}
}
\begin{myquote}\item{} thin space, math and text mode
\end{myquote}
{\bfseries
\begin{mydescription} \textbackslash{}; 
\end{mydescription}
}
\begin{myquote}\item{} thick space, math mode
\end{myquote}
{\bfseries
\begin{mydescription} \textbackslash{}: 
\end{mydescription}
}
\begin{myquote}\item{} medium space, math mode
\end{myquote}
{\bfseries
\begin{mydescription} \textbackslash{}! 
\end{mydescription}
}
\begin{myquote}\item{} negative thin space, math mode
\end{myquote}
{\bfseries
\begin{mydescription} \textbackslash{}-{} 
\end{mydescription}
}
\begin{myquote}\item{} hyphenation; tabbing
\end{myquote}
{\bfseries
\begin{mydescription} \textbackslash{}= 
\end{mydescription}
}
\begin{myquote}\item{} set tab, see tabbing
\end{myquote}
{\bfseries
\begin{mydescription} \textbackslash{}>{} 
\end{mydescription}
}
\begin{myquote}\item{} tab, see tabbing
\end{myquote}
{\bfseries
\begin{mydescription} \textbackslash{}<{} 
\end{mydescription}
}
\begin{myquote}\item{} back tab, see tabbing
\end{myquote}
{\bfseries
\begin{mydescription} \textbackslash{}+ 
\end{mydescription}
}
\begin{myquote}\item{} see tabbing
\end{myquote}
{\bfseries
\begin{mydescription} \textbackslash{}\textquotesingle{} 
\end{mydescription}
}
\begin{myquote}\item{} accent or tabbing
\end{myquote}
{\bfseries
\begin{mydescription} \textbackslash{}` 
\end{mydescription}
}
\begin{myquote}\item{} accent or tabbing
\end{myquote}
{\bfseries
\begin{mydescription} \textbackslash{}| 
\end{mydescription}
}
\begin{myquote}\item{} double vertical lines, math mode
\end{myquote}
{\bfseries
\begin{mydescription} \textbackslash{}( 
\end{mydescription}
}
\begin{myquote}\item{} start \mylref{497}{math environment}
\end{myquote}
{\bfseries
\begin{mydescription} \textbackslash{}) 
\end{mydescription}
}
\begin{myquote}\item{} end math environment
\end{myquote}
{\bfseries
\begin{mydescription} \textbackslash{}{$\text{[}$} 
\end{mydescription}
}
\begin{myquote}\item{} begin displaymath environment
\end{myquote}
{\bfseries
\begin{mydescription} \textbackslash{}{$\text{]}$} 
\end{mydescription}
}
\begin{myquote}\item{} end displaymath environment
\end{myquote}

\section{A}
\label{1021}{\bfseries
\begin{mydescription} \textbackslash{}addcontentsline\{file\}\{sec\_unit\}\{entry\} 
\end{mydescription}
}
\begin{myquote}\item{} adds an entry to the specified list or table
\end{myquote}
{\bfseries
\begin{mydescription} \textbackslash{}addtocontents\{file\}\{text\} 
\end{mydescription}
}
\begin{myquote}\item{}  adds text (or formatting commands) directly to the file that generates the specified list or table
\end{myquote}
{\bfseries
\begin{mydescription} \textbackslash{}addtocounter\{counter\}\{value\} 
\end{mydescription}
}
\begin{myquote}\item{} increments the counter 
\end{myquote}
{\bfseries
\begin{mydescription} \textbackslash{}address\{Return address\}
\end{mydescription}
}
{\bfseries
\begin{mydescription} \textbackslash{}addtolength\{len-{}cmd\}\{len\} 
\end{mydescription}
}
\begin{myquote}\item{}  increments a length command, see \mylref{456}{Length}
\end{myquote}
{\bfseries
\begin{mydescription} \textbackslash{}addvspace 
\end{mydescription}
}
\begin{myquote}\item{} adds a vertical space of a specified height
\end{myquote}
{\bfseries
\begin{mydescription} \textbackslash{}alph 
\end{mydescription}
}
\begin{myquote}\item{} causes the current value of a specified counter to be printed in alphabetic characters
\end{myquote}
{\bfseries
\begin{mydescription} \textbackslash{}appendix 
\end{mydescription}
}
\begin{myquote}\item{} changes the way sectional units are numbered so that information after the command is considered part of the appendix
\end{myquote}
{\bfseries
\begin{mydescription} \textbackslash{}arabic 
\end{mydescription}
}
\begin{myquote}\item{} causes the current value of a specified counter to be printed in Arabic numbers
\end{myquote}
{\bfseries
\begin{mydescription} \textbackslash{}author 
\end{mydescription}
}
\begin{myquote}\item{} declares the author(s). See \mylref{95}{Document Structure}
\end{myquote}

\section{B}
\label{1022}{\bfseries
\begin{mydescription} \textbackslash{}backslash 
\end{mydescription}
}
\begin{myquote}\item{} prints a backslash
\end{myquote}
{\bfseries
\begin{mydescription} \textbackslash{}baselineskip 
\end{mydescription}
}
\begin{myquote}\item{} a length command (see \mylref{456}{Lengths}), which specifies the minimum space between the bottom of two successive lines in a paragraph
\end{myquote}
{\bfseries
\begin{mydescription} \textbackslash{}baselinestretch 
\end{mydescription}
}
\begin{myquote}\item{} scales the value of \textbackslash{}baselineskip
\end{myquote}
{\bfseries
\begin{mydescription} \textbackslash{}bf 
\end{mydescription}
}
\begin{myquote}\item{} Boldface typeface
\end{myquote}
{\bfseries
\begin{mydescription} \textbackslash{}bibitem 
\end{mydescription}
}
\begin{myquote}\item{} generates a labeled entry for \mylref{95}{the bibliography}
\end{myquote}
{\bfseries
\begin{mydescription} \textbackslash{}bigskipamount
\end{mydescription}
}
{\bfseries
\begin{mydescription} \textbackslash{}bigskip 
\end{mydescription}
}
\begin{myquote}\item{} equivalent to \textbackslash{}vspace\{\textbackslash{}bigskipamount\}
\end{myquote}
{\bfseries
\begin{mydescription} \textbackslash{}boldmath 
\end{mydescription}
}
\begin{myquote}\item{} bold font in math mode
\end{myquote}
{\bfseries
\begin{mydescription} \textbackslash{}boldsymbol 
\end{mydescription}
}
\begin{myquote}\item{} bold font for symbols
\end{myquote}

\section{C}
\label{1023}{\bfseries
\begin{mydescription} \textbackslash{}cal 
\end{mydescription}
}
\begin{myquote}\item{} Calligraphic style in math mode
\end{myquote}
{\bfseries
\begin{mydescription} \textbackslash{}caption 
\end{mydescription}
}
\begin{myquote}\item{} generate caption for figures and tables
\end{myquote}
{\bfseries
\begin{mydescription} \textbackslash{}cdots 
\end{mydescription}
}
\begin{myquote}\item{} Centered dots
\end{myquote}
{\bfseries
\begin{mydescription} \textbackslash{}centering 
\end{mydescription}
}
\begin{myquote}\item{} Used to center align LaTeX environments
\end{myquote}
{\bfseries
\begin{mydescription} \textbackslash{}chapter 
\end{mydescription}
}
\begin{myquote}\item{} Starts a new chapter. See \mylref{97}{Document Structure}.
\end{myquote}
{\bfseries
\begin{mydescription} \textbackslash{}circle 
\end{mydescription}
}
{\bfseries
\begin{mydescription} \textbackslash{}cite 
\end{mydescription}
}
\begin{myquote}\item{} Used to \mylref{669}{make citations} from the provided bibliography
\end{myquote}
{\bfseries
\begin{mydescription} \textbackslash{}cleardoublepage 
\end{mydescription}
}
{\bfseries
\begin{mydescription} \textbackslash{}clearpage 
\end{mydescription}
}
\begin{myquote}\item{} Ends the current page and causes any floats to be printed. See \mylref{331}{Page Layout}.
\end{myquote}
{\bfseries
\begin{mydescription} \textbackslash{}cline 
\end{mydescription}
}
\begin{myquote}\item{} Adds horizontal line in a table that spans only to a range of cells. See \mylref{1028}{\textbackslash{}hline} and \mylref{248}{../Tables/} chapter.
\end{myquote}
{\bfseries
\begin{mydescription} \textbackslash{}closing 
\end{mydescription}
}
\begin{myquote}\item{} Inserts a closing phrase (e.g. \textbackslash{}closing\{yours sincerely\}), leaves space for a  handwritten signature and inserts a signature specified by \textbackslash{}signature\{\}.  Used in the {\itshape \setmainfont[Path=/usr/share/fonts/truetype/cmu/,UprightFont=cmunrm.ttf,BoldFont=cmunbx.ttf,ItalicFont=cmunti.ttf,BoldItalicFont=cmunbi.ttf]{cmunti.ttf}\setmonofont[Path=/usr/share/fonts/truetype/cmu/,UprightFont=cmuntt.ttf,BoldFont=cmuntb.ttf,ItalicFont=cmunit.ttf,BoldItalicFont=cmuntx.ttf]{cmunti.ttf}\itshape Letter}{$\text{ }$}\setmainfont[Path=/usr/share/fonts/truetype/cmu/,UprightFont=cmunrm.ttf,BoldFont=cmunbx.ttf,ItalicFont=cmunti.ttf,BoldItalicFont=cmunbi.ttf]{cmunrm.ttf}\setmonofont[Path=/usr/share/fonts/truetype/cmu/,UprightFont=cmuntt.ttf,BoldFont=cmuntb.ttf,ItalicFont=cmunit.ttf,BoldItalicFont=cmuntx.ttf]{cmunrm.ttf} class.
\end{myquote}
{\bfseries
\begin{mydescription} \textbackslash{}color 
\end{mydescription}
}
\begin{myquote}\item{} Specifies color of the text. \mylref{147}{../Colors} 
\end{myquote}
{\bfseries
\begin{mydescription} \textbackslash{}copyright 
\end{mydescription}
}
\begin{myquote}\item{} makes {\mbox{$\copyright$}} sign. See \mylref{140}{Formatting}.
\end{myquote}

\section{D}
\label{1024}{\bfseries
\begin{mydescription} \textbackslash{}dashbox 
\end{mydescription}
}
{\bfseries
\begin{mydescription} \textbackslash{}date 
\end{mydescription}
}
{\bfseries
\begin{mydescription} \textbackslash{}ddots 
\end{mydescription}
}
\begin{myquote}\item{} Inserts a diagonal ellipsis (3 diagonal dots) in math mode 
\end{myquote}
{\bfseries
\begin{mydescription} \textbackslash{}documentclass{$\text{[}$}options{$\text{]}$}\{style\} 
\end{mydescription}
}
\begin{myquote}\item{} Used to begin a latex document
\end{myquote}
{\bfseries
\begin{mydescription} \textbackslash{}dotfill
\end{mydescription}
}

\section{E}
\label{1025}{\bfseries
\begin{mydescription} \textbackslash{}em 
\end{mydescription}
}
\begin{myquote}\item{} Toggles italics on/off for the text inside curly braces with the command. Such as \{\textbackslash{}em This is in italics \textbackslash{}em but this isn\textquotesingle{}t \textbackslash{}em and this is again\}. This command allows nesting.
\end{myquote}
{\bfseries
\begin{mydescription} \textbackslash{}emph 
\end{mydescription}
}
\begin{myquote}\item{} Toggles italics on/off for the text in curly braces following the command e.g. \textbackslash{}emph\{This is in italics \textbackslash{}emph\{but this isn\textquotesingle{}t\} and this is again\}.
\end{myquote}
{\bfseries
\begin{mydescription} \textbackslash{}ensuremath (LaTeX2e) 
\end{mydescription}
}
\begin{myquote}\item{} Treats everything inside the curly braces as if it were in a math environment. Useful when creating commands in the preamble as they will work inside or out of math environments.
\end{myquote}
{\bfseries
\begin{mydescription} \textbackslash{}epigraph 
\end{mydescription}
}
\begin{myquote}\item{} Adds an epigraph. Requires {\ttfamily \setmainfont[Path=/usr/share/fonts/truetype/cmu/,UprightFont=cmunrm.ttf,BoldFont=cmunbx.ttf,ItalicFont=cmunti.ttf,BoldItalicFont=cmunbi.ttf]{cmuntt.ttf}\setmonofont[Path=/usr/share/fonts/truetype/cmu/,UprightFont=cmuntt.ttf,BoldFont=cmuntb.ttf,ItalicFont=cmunit.ttf,BoldItalicFont=cmuntx.ttf]{cmuntt.ttf}\ttfamily epigraph}{$\text{ }$}\setmainfont[Path=/usr/share/fonts/truetype/cmu/,UprightFont=cmunrm.ttf,BoldFont=cmunbx.ttf,ItalicFont=cmunti.ttf,BoldItalicFont=cmunbi.ttf]{cmunrm.ttf}\setmonofont[Path=/usr/share/fonts/truetype/cmu/,UprightFont=cmuntt.ttf,BoldFont=cmuntb.ttf,ItalicFont=cmunit.ttf,BoldItalicFont=cmuntx.ttf]{cmunrm.ttf} package.
\end{myquote}
{\bfseries
\begin{mydescription} \textbackslash{}euro 
\end{mydescription}
}
\begin{myquote}\item{} Prints euro {\mbox{$\text{\EUR}$}} symbol. Requires {\ttfamily \setmainfont[Path=/usr/share/fonts/truetype/cmu/,UprightFont=cmunrm.ttf,BoldFont=cmunbx.ttf,ItalicFont=cmunti.ttf,BoldItalicFont=cmunbi.ttf]{cmuntt.ttf}\setmonofont[Path=/usr/share/fonts/truetype/cmu/,UprightFont=cmuntt.ttf,BoldFont=cmuntb.ttf,ItalicFont=cmunit.ttf,BoldItalicFont=cmuntx.ttf]{cmuntt.ttf}\ttfamily eurosym}{$\text{ }$}\setmainfont[Path=/usr/share/fonts/truetype/cmu/,UprightFont=cmunrm.ttf,BoldFont=cmunbx.ttf,ItalicFont=cmunti.ttf,BoldItalicFont=cmunbi.ttf]{cmunrm.ttf}\setmonofont[Path=/usr/share/fonts/truetype/cmu/,UprightFont=cmuntt.ttf,BoldFont=cmuntb.ttf,ItalicFont=cmunit.ttf,BoldItalicFont=cmuntx.ttf]{cmunrm.ttf} package.
\end{myquote}

\section{F}
\label{1026}{\bfseries
\begin{mydescription} \textbackslash{}fbox 
\end{mydescription}
}
{\bfseries
\begin{mydescription} \textbackslash{}flushbottom 
\end{mydescription}
}
{\bfseries
\begin{mydescription} \textbackslash{}fnsymbol 
\end{mydescription}
}
{\bfseries
\begin{mydescription} \textbackslash{}footnote 
\end{mydescription}
}
\begin{myquote}\item{} Creates a \mylref{385}{footnote}.
\end{myquote}
{\bfseries
\begin{mydescription} \textbackslash{}footnotemark 
\end{mydescription}
}
{\bfseries
\begin{mydescription} \textbackslash{}footnotesize 
\end{mydescription}
}
\begin{myquote}\item{} Sets font size. See \mylref{124}{Text Formatting}.
\end{myquote}
{\bfseries
\begin{mydescription} \textbackslash{}footnotetext 
\end{mydescription}
}
{\bfseries
\begin{mydescription} \textbackslash{}frac 
\end{mydescription}
}
\begin{myquote}\item{} inserts a fraction in mathematics mode. The usage is \textbackslash{}frac\{numerator\}\{denominator\}.
\end{myquote}
{\bfseries
\begin{mydescription} \textbackslash{}frame 
\end{mydescription}
}
{\bfseries
\begin{mydescription} \textbackslash{}framebox 
\end{mydescription}
}
\begin{myquote}\item{} Like \textbackslash{}makebox but creates a frame around the box. See \mylref{478}{Boxes}.
\end{myquote}
{\bfseries
\begin{mydescription} \textbackslash{}frenchspacing 
\end{mydescription}
}
\begin{myquote}\item{} Instructs LaTex to abstain from inserting more space after a period (´.´) than is the case for an ordinary character. In order to untoggle this functionality resort to the command \mylref{1033}{\textbackslash{}nonfrenchspacing}.
\end{myquote}

\section{G}
\label{1027}
\section{H}
\label{1028}{\bfseries
\begin{mydescription} \textbackslash{}hfill 
\end{mydescription}
}
\begin{myquote}\item{} Abbreviation for \textbackslash{}hspace\{\textbackslash{}fill\}.
\end{myquote}
{\bfseries
\begin{mydescription} \textbackslash{}hline 
\end{mydescription}
}
\begin{myquote}\item{} adds a horizontal line in a tabular environment. See also \mylref{1023}{\textbackslash{}cline}, \mylref{248}{Tables} chapter.
\end{myquote}
{\bfseries
\begin{mydescription} \textbackslash{}href 
\end{mydescription}
}
\begin{myquote}\item{} Add a link, or an anchor. See \mylref{392}{Hyperlinks}
\end{myquote}
{\bfseries
\begin{mydescription} \textbackslash{}hrulefill 
\end{mydescription}
}
{\bfseries
\begin{mydescription} \textbackslash{}hspace 
\end{mydescription}
}
\begin{myquote}\item{} Produces horizontal space. 
\end{myquote}
{\bfseries
\begin{mydescription} \textbackslash{}huge 
\end{mydescription}
}
\begin{myquote}\item{} Sets font size. See \mylref{124}{Text Formatting}.
\end{myquote}
{\bfseries
\begin{mydescription} \textbackslash{}Huge 
\end{mydescription}
}
\begin{myquote}\item{} Sets font size. See \mylref{124}{Text Formatting}.
\end{myquote}
{\bfseries
\begin{mydescription} \textbackslash{}hyphenation\{word list\}
\end{mydescription}
}
\begin{myquote}\item{} Overrides default hyphenation algorithm for specified words.  See \mylref{116}{Hyphenation}
\end{myquote}

\section{I}
\label{1029}{\bfseries
\begin{mydescription} \textbackslash{}include 
\end{mydescription}
}
\begin{myquote}\item{} This command is different from {\ttfamily \setmainfont[Path=/usr/share/fonts/truetype/cmu/,UprightFont=cmunrm.ttf,BoldFont=cmunbx.ttf,ItalicFont=cmunti.ttf,BoldItalicFont=cmunbi.ttf]{cmuntt.ttf}\setmonofont[Path=/usr/share/fonts/truetype/cmu/,UprightFont=cmuntt.ttf,BoldFont=cmuntb.ttf,ItalicFont=cmunit.ttf,BoldItalicFont=cmuntx.ttf]{cmuntt.ttf}\ttfamily \textbackslash{}input}{$\text{ }$}\setmainfont[Path=/usr/share/fonts/truetype/cmu/,UprightFont=cmunrm.ttf,BoldFont=cmunbx.ttf,ItalicFont=cmunti.ttf,BoldItalicFont=cmunbi.ttf]{cmunrm.ttf}\setmonofont[Path=/usr/share/fonts/truetype/cmu/,UprightFont=cmuntt.ttf,BoldFont=cmuntb.ttf,ItalicFont=cmunit.ttf,BoldItalicFont=cmuntx.ttf]{cmunrm.ttf} in that it\textquotesingle{}s the output that is added instead of the commands from the other files. For more see \mylref{87}{LaTex/Basics}
\end{myquote}
{\bfseries
\begin{mydescription} \textbackslash{}includegraphics 
\end{mydescription}
}
\begin{myquote}\item{} Inserts an \mylref{356}{image}. Requires {\ttfamily \setmainfont[Path=/usr/share/fonts/truetype/cmu/,UprightFont=cmunrm.ttf,BoldFont=cmunbx.ttf,ItalicFont=cmunti.ttf,BoldItalicFont=cmunbi.ttf]{cmuntt.ttf}\setmonofont[Path=/usr/share/fonts/truetype/cmu/,UprightFont=cmuntt.ttf,BoldFont=cmuntb.ttf,ItalicFont=cmunit.ttf,BoldItalicFont=cmuntx.ttf]{cmuntt.ttf}\ttfamily graphicx}{$\text{ }$}\setmainfont[Path=/usr/share/fonts/truetype/cmu/,UprightFont=cmunrm.ttf,BoldFont=cmunbx.ttf,ItalicFont=cmunti.ttf,BoldItalicFont=cmunbi.ttf]{cmunrm.ttf}\setmonofont[Path=/usr/share/fonts/truetype/cmu/,UprightFont=cmuntt.ttf,BoldFont=cmuntb.ttf,ItalicFont=cmunit.ttf,BoldItalicFont=cmuntx.ttf]{cmunrm.ttf} package.
\end{myquote}
{\bfseries
\begin{mydescription} \textbackslash{}includeonly 
\end{mydescription}
}
{\bfseries
\begin{mydescription} \textbackslash{}indent 
\end{mydescription}
}
{\bfseries
\begin{mydescription} \textbackslash{}input 
\end{mydescription}
}
\begin{myquote}\item{} Used to read in LaTex files. For more see \mylref{87}{LaTex/Basics}.
\end{myquote}
{\bfseries
\begin{mydescription} \textbackslash{}it 
\end{mydescription}
}
\begin{myquote}\item{} Italicizes the text which is inside curly braces with the command. Such as \{\textbackslash{}it This is in italics\}. \textbackslash{}em is generally preferred since this allows nesting.
\end{myquote}
{\bfseries
\begin{mydescription} \textbackslash{}item 
\end{mydescription}
}
\begin{myquote}\item{} Creates an item in a list. Used in \mylref{186}{list structures}.
\end{myquote}

\section{K}
\label{1030}{\bfseries
\begin{mydescription} \textbackslash{}kill 
\end{mydescription}
}
\begin{myquote}\item{} Prevent a line in the tabbing environment from being printed.
\end{myquote}

\section{L}
\label{1031} {\bfseries
\begin{mydescription} \textbackslash{}label 
\end{mydescription}
}
\begin{myquote}\item{} Used to create label which can be later referenced with {\ttfamily \setmainfont[Path=/usr/share/fonts/truetype/cmu/,UprightFont=cmunrm.ttf,BoldFont=cmunbx.ttf,ItalicFont=cmunti.ttf,BoldItalicFont=cmunbi.ttf]{cmuntt.ttf}\setmonofont[Path=/usr/share/fonts/truetype/cmu/,UprightFont=cmuntt.ttf,BoldFont=cmuntb.ttf,ItalicFont=cmunit.ttf,BoldItalicFont=cmuntx.ttf]{cmuntt.ttf}\ttfamily \textbackslash{}ref}\setmainfont[Path=/usr/share/fonts/truetype/cmu/,UprightFont=cmunrm.ttf,BoldFont=cmunbx.ttf,ItalicFont=cmunti.ttf,BoldItalicFont=cmunbi.ttf]{cmunrm.ttf}\setmonofont[Path=/usr/share/fonts/truetype/cmu/,UprightFont=cmuntt.ttf,BoldFont=cmuntb.ttf,ItalicFont=cmunit.ttf,BoldItalicFont=cmuntx.ttf]{cmunrm.ttf}. See \mylref{417}{Labels and Cross-{}referencing}.
\end{myquote}
{\bfseries
\begin{mydescription} \textbackslash{}large 
\end{mydescription}
}
\begin{myquote}\item{} Sets font size. See \mylref{124}{Text Formatting}.
\end{myquote}
{\bfseries
\begin{mydescription} \textbackslash{}Large 
\end{mydescription}
}
\begin{myquote}\item{} Sets font size. See \mylref{124}{Text Formatting}.
\end{myquote}
{\bfseries
\begin{mydescription} \textbackslash{}LARGE 
\end{mydescription}
}
\begin{myquote}\item{} Sets font size. See \mylref{124}{Text Formatting}.
\end{myquote}
{\bfseries
\begin{mydescription} \textbackslash{}LaTeX 
\end{mydescription}
}
\begin{myquote}\item{} Prints LaTeX logo. See \mylref{128}{Formatting}.
\end{myquote}
{\bfseries
\begin{mydescription} \textbackslash{}LaTeXe 
\end{mydescription}
}
\begin{myquote}\item{} Prints current LaTeX version logo. See \mylref{128}{Formatting}.
\end{myquote}
{\bfseries
\begin{mydescription} \textbackslash{}ldots 
\end{mydescription}
}
\begin{myquote}\item{} Prints sequence of three dots. See \mylref{124}{Formatting}.
\end{myquote}
{\bfseries
\begin{mydescription} \textbackslash{}left 
\end{mydescription}
}
{\bfseries
\begin{mydescription} \textbackslash{}lefteqn 
\end{mydescription}
}
{\bfseries
\begin{mydescription} \textbackslash{}line 
\end{mydescription}
}
{\bfseries
\begin{mydescription} \textbackslash{}linebreak 
\end{mydescription}
}
\begin{myquote}\item{} Suggests LaTeX to break line in this place. See \mylref{331}{Page Layout}.
\end{myquote}
{\bfseries
\begin{mydescription} \textbackslash{}linethickness 
\end{mydescription}
}
{\bfseries
\begin{mydescription} \textbackslash{}linewidth 
\end{mydescription}
}
{\bfseries
\begin{mydescription} \textbackslash{}listoffigures 
\end{mydescription}
}
\begin{myquote}\item{} Inserts a list of the figures in the document.  Similar to \mylref{101}{TOC}
\end{myquote}
{\bfseries
\begin{mydescription} \textbackslash{}listoftables 
\end{mydescription}
}
\begin{myquote}\item{} Inserts a list of the tables in the document.  Similar to \mylref{101}{TOC}
\end{myquote}
{\bfseries
\begin{mydescription} \textbackslash{}location
\end{mydescription}
}

\section{M}
\label{1032}{\bfseries
\begin{mydescription} \textbackslash{}makebox 
\end{mydescription}
}
\begin{myquote}\item{} Defines a box that has a specified width, independent from its content. See \mylref{478}{Boxes}.
\end{myquote}
{\bfseries
\begin{mydescription} \textbackslash{}maketitle 
\end{mydescription}
}
\begin{myquote}\item{} Causes the title page to be typeset, using information provided by commands such as \textbackslash{}title\{\} and \textbackslash{}author\{\}.
\end{myquote}
{\bfseries
\begin{mydescription} \textbackslash{}markboth \textbackslash{}markright 
\end{mydescription}
}
{\bfseries
\begin{mydescription} \textbackslash{}mathcal 
\end{mydescription}
}
{\bfseries
\begin{mydescription} \textbackslash{}mathop 
\end{mydescription}
}
{\bfseries
\begin{mydescription} \textbackslash{}mbox 
\end{mydescription}
}
\begin{myquote}\item{} Write a text in roman font inside a math part
\end{myquote}
{\bfseries
\begin{mydescription} \textbackslash{}medskip 
\end{mydescription}
}
{\bfseries
\begin{mydescription} \textbackslash{}multicolumn 
\end{mydescription}
}
{\bfseries
\begin{mydescription} \textbackslash{}multiput
\end{mydescription}
}

\section{N}
\label{1033}{\bfseries
\begin{mydescription} \textbackslash{}newcommand 
\end{mydescription}
}
\begin{myquote}\item{} Defines a new command. See \mylref{838}{New Commands}.
\end{myquote}
{\bfseries
\begin{mydescription} \textbackslash{}newcolumntype 
\end{mydescription}
}
\begin{myquote}\item{} Defines a new type of column to be used with tables. See \mylref{248}{Tables}.
\end{myquote}
{\bfseries
\begin{mydescription} \textbackslash{}newcounter 
\end{mydescription}
}
{\bfseries
\begin{mydescription} \textbackslash{}newenvironment 
\end{mydescription}
}
\begin{myquote}\item{} Defines a new environment. See \mylref{840}{New Environments}.
\end{myquote}
{\bfseries
\begin{mydescription} \textbackslash{}newfont 
\end{mydescription}
}
{\bfseries
\begin{mydescription} \textbackslash{}newlength 
\end{mydescription}
}
{\bfseries
\begin{mydescription} \textbackslash{}newline 
\end{mydescription}
}
\begin{myquote}\item{} Ends current line and starts a new one. See \mylref{331}{Page Layout}.
\end{myquote}
{\bfseries
\begin{mydescription} \textbackslash{}newpage 
\end{mydescription}
}
\begin{myquote}\item{} Ends current page and starts a new one. See \mylref{331}{Page Layout}.
\end{myquote}
{\bfseries
\begin{mydescription} \textbackslash{}newsavebox 
\end{mydescription}
}
{\bfseries
\begin{mydescription} \textbackslash{}newtheorem 
\end{mydescription}
}
{\bfseries
\begin{mydescription} \textbackslash{}nocite 
\end{mydescription}
}
\begin{myquote}\item{} Adds a reference to the bibliography without an inline citation.  \textbackslash{}nocite\{*\} causes all entries in a bibtex database to be added to the bibliography.
\end{myquote}
{\bfseries
\begin{mydescription} \textbackslash{}noindent 
\end{mydescription}
}
{\bfseries
\begin{mydescription} \textbackslash{}nolinebreak 
\end{mydescription}
}
{\bfseries
\begin{mydescription} \textbackslash{}nonfrenchspacing 
\end{mydescription}
}
\begin{myquote}\item{} Setting the command untoggles the command \mylref{1026}{\textbackslash{}frenchspacing} and activates LaTeX standards to insert more space after a period (´.´) than after an ordinary character. 
\end{myquote}
{\bfseries
\begin{mydescription} \textbackslash{}normalsize 
\end{mydescription}
}
\begin{myquote}\item{} Sets default font size. See \mylref{124}{Text Formatting}.
\end{myquote}
{\bfseries
\begin{mydescription} \textbackslash{}nopagebreak 
\end{mydescription}
}
\begin{myquote}\item{} Suggests LaTeX not to break page in this place. See \mylref{331}{Page Layout}.
\end{myquote}
{\bfseries
\begin{mydescription} \textbackslash{}not
\end{mydescription}
}

\section{O}
\label{1034}{\bfseries
\begin{mydescription} \textbackslash{}onecolumn 
\end{mydescription}
}
{\bfseries
\begin{mydescription} \textbackslash{}opening 
\end{mydescription}
}
\begin{myquote}\item{} Inserts an opening phrase when using the {\itshape \setmainfont[Path=/usr/share/fonts/truetype/cmu/,UprightFont=cmunrm.ttf,BoldFont=cmunbx.ttf,ItalicFont=cmunti.ttf,BoldItalicFont=cmunbi.ttf]{cmunti.ttf}\setmonofont[Path=/usr/share/fonts/truetype/cmu/,UprightFont=cmuntt.ttf,BoldFont=cmuntb.ttf,ItalicFont=cmunit.ttf,BoldItalicFont=cmuntx.ttf]{cmunti.ttf}\itshape letter}{$\text{ }$}\setmainfont[Path=/usr/share/fonts/truetype/cmu/,UprightFont=cmunrm.ttf,BoldFont=cmunbx.ttf,ItalicFont=cmunti.ttf,BoldItalicFont=cmunbi.ttf]{cmunrm.ttf}\setmonofont[Path=/usr/share/fonts/truetype/cmu/,UprightFont=cmuntt.ttf,BoldFont=cmuntb.ttf,ItalicFont=cmunit.ttf,BoldItalicFont=cmuntx.ttf]{cmunrm.ttf} class, for example \textbackslash{}opening\{Dear Sir\}
\end{myquote}
{\bfseries
\begin{mydescription} \textbackslash{}oval 
\end{mydescription}
}
{\bfseries
\begin{mydescription} \textbackslash{}overbrace 
\end{mydescription}
}
\begin{myquote}\item{} Draws a brace over the argument. Can be used in displaystyle with superscript to label formulae. See \mylref{537}{Advanced Mathematics}.
\end{myquote}
{\bfseries
\begin{mydescription} \textbackslash{}overline 
\end{mydescription}
}
\begin{myquote}\item{} Draws a line over the argument.
\end{myquote}

\section{P}
\label{1035}{\bfseries
\begin{mydescription} \textbackslash{}pagebreak 
\end{mydescription}
}
\begin{myquote}\item{} Suggests LaTeX breaking page in this place. See \mylref{331}{Page Layout}.
\end{myquote}
{\bfseries
\begin{mydescription} \textbackslash{}pagenumbering 
\end{mydescription}
}
\begin{myquote}\item{} Defines the type of characters used for the page numbers. Options : arabic, roman, Roman, alph, Alph, gobble (invisible).
\end{myquote}
{\bfseries
\begin{mydescription} \textbackslash{}pageref 
\end{mydescription}
}
\begin{myquote}\item{} Used to reference to number of page where a previously declared {\ttfamily \setmainfont[Path=/usr/share/fonts/truetype/cmu/,UprightFont=cmunrm.ttf,BoldFont=cmunbx.ttf,ItalicFont=cmunti.ttf,BoldItalicFont=cmunbi.ttf]{cmuntt.ttf}\setmonofont[Path=/usr/share/fonts/truetype/cmu/,UprightFont=cmuntt.ttf,BoldFont=cmuntb.ttf,ItalicFont=cmunit.ttf,BoldItalicFont=cmuntx.ttf]{cmuntt.ttf}\ttfamily \textbackslash{}label}{$\text{ }$}\setmainfont[Path=/usr/share/fonts/truetype/cmu/,UprightFont=cmunrm.ttf,BoldFont=cmunbx.ttf,ItalicFont=cmunti.ttf,BoldItalicFont=cmunbi.ttf]{cmunrm.ttf}\setmonofont[Path=/usr/share/fonts/truetype/cmu/,UprightFont=cmuntt.ttf,BoldFont=cmuntb.ttf,ItalicFont=cmunit.ttf,BoldItalicFont=cmuntx.ttf]{cmunrm.ttf} is located.  See \mylref{373}{Floats, Figures and Captions}.
\end{myquote}
{\bfseries
\begin{mydescription} \textbackslash{}pagestyle 
\end{mydescription}
}
\begin{myquote}\item{} See \mylref{319}{Page Layout}.
\end{myquote}
{\bfseries
\begin{mydescription} \textbackslash{}par 
\end{mydescription}
}
\begin{myquote}\item{} Starts a new paragraph
\end{myquote}
{\bfseries
\begin{mydescription} \textbackslash{}paragraph 
\end{mydescription}
}
\begin{myquote}\item{} Starts a new paragraph. See \mylref{97}{Document Structure}.
\end{myquote}
{\bfseries
\begin{mydescription} \textbackslash{}parbox 
\end{mydescription}
}
\begin{myquote}\item{} Defines a box whose contents are created in paragraph mode. See \mylref{478}{Boxes}.
\end{myquote}
{\bfseries
\begin{mydescription} \textbackslash{}parindent 
\end{mydescription}
}
\begin{myquote}\item{} Normal paragraph indentation. See \mylref{456}{Lengths}.
\end{myquote}
{\bfseries
\begin{mydescription} \textbackslash{}parskip 
\end{mydescription}
}
{\bfseries
\begin{mydescription} \textbackslash{}part 
\end{mydescription}
}
\begin{myquote}\item{} Starts a new part of a book. See \mylref{97}{Document Structure}.
\end{myquote}
{\bfseries
\begin{mydescription} \textbackslash{}protect 
\end{mydescription}
}
{\bfseries
\begin{mydescription} \textbackslash{}providecommand (LaTeX2e) 
\end{mydescription}
}
\begin{myquote}\item{} See \mylref{838}{Macros}.
\end{myquote}
{\bfseries
\begin{mydescription} \textbackslash{}put
\end{mydescription}
}

\section{Q}
\label{1036}{\bfseries
\begin{mydescription} \textbackslash{}quad 
\end{mydescription}
}
\begin{myquote}\item{} Similar to space, but with the size of a capital M
\end{myquote}
{\bfseries
\begin{mydescription} \textbackslash{}qquad 
\end{mydescription}
}
\begin{myquote}\item{} double \textbackslash{}quad
\end{myquote}

\section{R}
\label{1037}{\bfseries
\begin{mydescription} \textbackslash{}raggedbottom 
\end{mydescription}
}
\begin{myquote}\item{} Command used for top justified within other environments. 
\end{myquote}
{\bfseries
\begin{mydescription} \textbackslash{}raggedleft 
\end{mydescription}
}
\begin{myquote}\item{} Command used for right justified within other environments.
\end{myquote}
{\bfseries
\begin{mydescription} \textbackslash{}raggedright 
\end{mydescription}
}
\begin{myquote}\item{} Command used for left justified within other environments.
\end{myquote}
{\bfseries
\begin{mydescription} \textbackslash{}raisebox 
\end{mydescription}
}
\begin{myquote}\item{} Creates a box and raises its content. See \mylref{478}{LaTeX/Boxes}.
\end{myquote}
{\bfseries
\begin{mydescription} \textbackslash{}ref 
\end{mydescription}
}
\begin{myquote}\item{}  Used to reference to number of previously declared {\ttfamily \setmainfont[Path=/usr/share/fonts/truetype/cmu/,UprightFont=cmunrm.ttf,BoldFont=cmunbx.ttf,ItalicFont=cmunti.ttf,BoldItalicFont=cmunbi.ttf]{cmuntt.ttf}\setmonofont[Path=/usr/share/fonts/truetype/cmu/,UprightFont=cmuntt.ttf,BoldFont=cmuntb.ttf,ItalicFont=cmunit.ttf,BoldItalicFont=cmuntx.ttf]{cmuntt.ttf}\ttfamily \textbackslash{}label}\setmainfont[Path=/usr/share/fonts/truetype/cmu/,UprightFont=cmunrm.ttf,BoldFont=cmunbx.ttf,ItalicFont=cmunti.ttf,BoldItalicFont=cmunbi.ttf]{cmunrm.ttf}\setmonofont[Path=/usr/share/fonts/truetype/cmu/,UprightFont=cmuntt.ttf,BoldFont=cmuntb.ttf,ItalicFont=cmunit.ttf,BoldItalicFont=cmuntx.ttf]{cmunrm.ttf}. See \mylref{417}{Labels and Cross-{}referencing}.
\end{myquote}
{\bfseries
\begin{mydescription} \textbackslash{}renewcommand 
\end{mydescription}
}
{\bfseries
\begin{mydescription} \textbackslash{}right 
\end{mydescription}
}
{\bfseries
\begin{mydescription} \textbackslash{}rm 
\end{mydescription}
}
\begin{myquote}\item{} Roman typeface.
\end{myquote}
{\bfseries
\begin{mydescription} \textbackslash{}roman 
\end{mydescription}
}
\begin{myquote}\item{} Causes a counter to be printed in roman numerals.
\end{myquote}
{\bfseries
\begin{mydescription} \textbackslash{}rule 
\end{mydescription}
}
\begin{myquote}\item{} Creates a line of specified width and height. See \mylref{492}{LaTeX/Rules and Struts}.
\end{myquote}

\section{S}
\label{1038}{\bfseries
\begin{mydescription} \textbackslash{}savebox 
\end{mydescription}
}
\begin{myquote}\item{} Makes a box and saves it in a named storage bin.
\end{myquote}
{\bfseries
\begin{mydescription} \textbackslash{}sbox 
\end{mydescription}
}
\begin{myquote}\item{} The short form of \textbackslash{}savebox with no optional arguments.
\end{myquote}
{\bfseries
\begin{mydescription} \textbackslash{}sc 
\end{mydescription}
}
\begin{myquote}\item{} Small caps.
\end{myquote}
{\bfseries
\begin{mydescription} \textbackslash{}scriptsize 
\end{mydescription}
}
\begin{myquote}\item{} Sets font size. See \mylref{124}{Text Formatting}.
\end{myquote}
{\bfseries
\begin{mydescription} \textbackslash{}section 
\end{mydescription}
}
\begin{myquote}\item{} Starts a new section. See \mylref{97}{Document Structure}.
\end{myquote}
{\bfseries
\begin{mydescription} \textbackslash{}setcounter 
\end{mydescription}
}
{\bfseries
\begin{mydescription} \textbackslash{}setlength 
\end{mydescription}
}
{\bfseries
\begin{mydescription} \textbackslash{}settowidth 
\end{mydescription}
}
{\bfseries
\begin{mydescription} \textbackslash{}sf 
\end{mydescription}
}
\begin{myquote}\item{} Sans serif.
\end{myquote}
{\bfseries
\begin{mydescription} \textbackslash{}shortstack 
\end{mydescription}
}
{\bfseries
\begin{mydescription} \textbackslash{}signature 
\end{mydescription}
}
\begin{myquote}\item{} In the {\itshape \setmainfont[Path=/usr/share/fonts/truetype/cmu/,UprightFont=cmunrm.ttf,BoldFont=cmunbx.ttf,ItalicFont=cmunti.ttf,BoldItalicFont=cmunbi.ttf]{cmunti.ttf}\setmonofont[Path=/usr/share/fonts/truetype/cmu/,UprightFont=cmuntt.ttf,BoldFont=cmuntb.ttf,ItalicFont=cmunit.ttf,BoldItalicFont=cmuntx.ttf]{cmunti.ttf}\itshape Letter}{$\text{ }$}\setmainfont[Path=/usr/share/fonts/truetype/cmu/,UprightFont=cmunrm.ttf,BoldFont=cmunbx.ttf,ItalicFont=cmunti.ttf,BoldItalicFont=cmunbi.ttf]{cmunrm.ttf}\setmonofont[Path=/usr/share/fonts/truetype/cmu/,UprightFont=cmuntt.ttf,BoldFont=cmuntb.ttf,ItalicFont=cmunit.ttf,BoldItalicFont=cmuntx.ttf]{cmunrm.ttf} class, specifies a signature for later insertion by \textbackslash{}closing.
\end{myquote}
{\bfseries
\begin{mydescription} \textbackslash{}sl 
\end{mydescription}
}
\begin{myquote}\item{} Slanted.
\end{myquote}
{\bfseries
\begin{mydescription} \textbackslash{}slash 
\end{mydescription}
}
\begin{myquote}\item{} See \mylref{140}{slash marks}
\end{myquote}
{\bfseries
\begin{mydescription} \textbackslash{}small 
\end{mydescription}
}
\begin{myquote}\item{} Sets font size. See \mylref{124}{Text Formatting}.
\end{myquote}
{\bfseries
\begin{mydescription} \textbackslash{}smallskip 
\end{mydescription}
}
{\bfseries
\begin{mydescription} \textbackslash{}sout 
\end{mydescription}
}
\begin{myquote}\item{} Strikes out text.  Requires {\ttfamily \setmainfont[Path=/usr/share/fonts/truetype/cmu/,UprightFont=cmunrm.ttf,BoldFont=cmunbx.ttf,ItalicFont=cmunti.ttf,BoldItalicFont=cmunbi.ttf]{cmuntt.ttf}\setmonofont[Path=/usr/share/fonts/truetype/cmu/,UprightFont=cmuntt.ttf,BoldFont=cmuntb.ttf,ItalicFont=cmunit.ttf,BoldItalicFont=cmuntx.ttf]{cmuntt.ttf}\ttfamily ulem}{$\text{ }$}\setmainfont[Path=/usr/share/fonts/truetype/cmu/,UprightFont=cmunrm.ttf,BoldFont=cmunbx.ttf,ItalicFont=cmunti.ttf,BoldItalicFont=cmunbi.ttf]{cmunrm.ttf}\setmonofont[Path=/usr/share/fonts/truetype/cmu/,UprightFont=cmuntt.ttf,BoldFont=cmuntb.ttf,ItalicFont=cmunit.ttf,BoldItalicFont=cmuntx.ttf]{cmunrm.ttf} package. See \mylref{109}{Text Formatting}.
\end{myquote}
{\bfseries
\begin{mydescription} \textbackslash{}space 
\end{mydescription}
}
\begin{myquote}\item{} force ordinary space
\end{myquote}
{\bfseries
\begin{mydescription} \textbackslash{}sqrt 
\end{mydescription}
}
\begin{myquote}\item{} Creates a \mylref{507}{root} (default square, but magnitude can be given as an optional parameter).
\end{myquote}
{\bfseries
\begin{mydescription} \textbackslash{}stackrel 
\end{mydescription}
}
\begin{myquote}\item{} Takes two arguments and stacks the first on top of the second.
\end{myquote}
{\bfseries
\begin{mydescription} \textbackslash{}stepcounter 
\end{mydescription}
}
\begin{myquote}\item{} Increase the counter.
\end{myquote}
{\bfseries
\begin{mydescription} \textbackslash{}subparagraph 
\end{mydescription}
}
\begin{myquote}\item{} Starts a new subparagraph. See \mylref{97}{Document Structure}.
\end{myquote}
{\bfseries
\begin{mydescription} \textbackslash{}subsection 
\end{mydescription}
}
\begin{myquote}\item{} Starts a new subsection. See \mylref{97}{Document Structure}.
\end{myquote}
{\bfseries
\begin{mydescription} \textbackslash{}subsubsection 
\end{mydescription}
}
\begin{myquote}\item{} Starts a new sub-{}subsection. See \mylref{97}{Document Structure}.
\end{myquote}

\section{T}
\label{1039}{\bfseries
\begin{mydescription} \textbackslash{}tableofcontents 
\end{mydescription}
}
\begin{myquote}\item{} Inserts a table of contents (based on section headings) at the point where the command appears.
\end{myquote}
{\bfseries
\begin{mydescription} \textbackslash{}telephone 
\end{mydescription}
}
\begin{myquote}\item{} In the {\itshape \setmainfont[Path=/usr/share/fonts/truetype/cmu/,UprightFont=cmunrm.ttf,BoldFont=cmunbx.ttf,ItalicFont=cmunti.ttf,BoldItalicFont=cmunbi.ttf]{cmunti.ttf}\setmonofont[Path=/usr/share/fonts/truetype/cmu/,UprightFont=cmuntt.ttf,BoldFont=cmuntb.ttf,ItalicFont=cmunit.ttf,BoldItalicFont=cmuntx.ttf]{cmunti.ttf}\itshape letter}{$\text{ }$}\setmainfont[Path=/usr/share/fonts/truetype/cmu/,UprightFont=cmunrm.ttf,BoldFont=cmunbx.ttf,ItalicFont=cmunti.ttf,BoldItalicFont=cmunbi.ttf]{cmunrm.ttf}\setmonofont[Path=/usr/share/fonts/truetype/cmu/,UprightFont=cmuntt.ttf,BoldFont=cmuntb.ttf,ItalicFont=cmunit.ttf,BoldItalicFont=cmuntx.ttf]{cmunrm.ttf} class, specifies the sender\textquotesingle{}s telephone number.
\end{myquote}
{\bfseries
\begin{mydescription} \textbackslash{}TeX 
\end{mydescription}
}
\begin{myquote}\item{} Prints TeX logo. See \mylref{128}{Text Formatting}.
\end{myquote}
{\bfseries
\begin{mydescription} \textbackslash{}textbf\{\} 
\end{mydescription}
}
\begin{myquote}\item{}  Sets bold font style. See \mylref{124}{Text Formatting}.
\end{myquote}
{\bfseries
\begin{mydescription} \textbackslash{}textcolor\{\}\{\} 
\end{mydescription}
}
\begin{myquote}\item{} Creates colored text. See \mylref{149}{Entering colored text}.
\end{myquote}
{\bfseries
\begin{mydescription} \textbackslash{}textit\{\} 
\end{mydescription}
}
\begin{myquote}\item{}  Sets italic font style. See \mylref{124}{Text Formatting}.
\end{myquote}
{\bfseries
\begin{mydescription} \textbackslash{}textmd\{\} 
\end{mydescription}
}
\begin{myquote}\item{}  Sets medium weight of a font. See \mylref{124}{Text Formatting}.
\end{myquote}
{\bfseries
\begin{mydescription} \textbackslash{}textnormal\{\} 
\end{mydescription}
}
\begin{myquote}\item{}  Sets normal font. See \mylref{124}{Text Formatting}.
\end{myquote}
{\bfseries
\begin{mydescription} \textbackslash{}textrm\{\} 
\end{mydescription}
}
\begin{myquote}\item{}  Sets roman font family. See \mylref{124}{Text Formatting}.
\end{myquote}
{\bfseries
\begin{mydescription} \textbackslash{}textsc\{\} 
\end{mydescription}
}
\begin{myquote}\item{}  Sets font style to small caps. See \mylref{124}{Text Formatting}.
\end{myquote}
{\bfseries
\begin{mydescription} \textbackslash{}textsf\{\} 
\end{mydescription}
}
\begin{myquote}\item{}  Sets sans serif font family. See \mylref{124}{Text Formatting}.
\end{myquote}
{\bfseries
\begin{mydescription} \textbackslash{}textsl\{\} 
\end{mydescription}
}
\begin{myquote}\item{}  Sets slanted font style. See \mylref{124}{Text Formatting}.
\end{myquote}
{\bfseries
\begin{mydescription} \textbackslash{}texttt\{\} 
\end{mydescription}
}
\begin{myquote}\item{}  Sets typewriter font family. See \mylref{124}{Text Formatting}.
\end{myquote}
{\bfseries
\begin{mydescription} \textbackslash{}textup\{\} 
\end{mydescription}
}
\begin{myquote}\item{}  Sets upright shape of a font. See \mylref{124}{Text Formatting}.
\end{myquote}
{\bfseries
\begin{mydescription} \textbackslash{}textwidth 
\end{mydescription}
}
{\bfseries
\begin{mydescription} \textbackslash{}textheight 
\end{mydescription}
}
{\bfseries
\begin{mydescription} \textbackslash{}thanks 
\end{mydescription}
}
{\bfseries
\begin{mydescription} \textbackslash{}thispagestyle 
\end{mydescription}
}
{\bfseries
\begin{mydescription} \textbackslash{}tiny 
\end{mydescription}
}
\begin{myquote}\item{} Sets font size. See \mylref{124}{Text Formatting}.
\end{myquote}
{\bfseries
\begin{mydescription} \textbackslash{}title 
\end{mydescription}
}
{\bfseries
\begin{mydescription} \textbackslash{}today 
\end{mydescription}
}
\begin{myquote}\item{} Writes current day. See \mylref{128}{Text Formatting}.
\end{myquote}
{\bfseries
\begin{mydescription} \textbackslash{}tt 
\end{mydescription}
}
{\bfseries
\begin{mydescription} \textbackslash{}twocolumn 
\end{mydescription}
}
{\bfseries
\begin{mydescription} \textbackslash{}typeout 
\end{mydescription}
}
{\bfseries
\begin{mydescription} \textbackslash{}typein
\end{mydescription}
}

\section{U}
\label{1040}{\bfseries
\begin{mydescription} \textbackslash{}uline 
\end{mydescription}
}
\begin{myquote}\item{} Underlines text. Requires {\ttfamily \setmainfont[Path=/usr/share/fonts/truetype/cmu/,UprightFont=cmunrm.ttf,BoldFont=cmunbx.ttf,ItalicFont=cmunti.ttf,BoldItalicFont=cmunbi.ttf]{cmuntt.ttf}\setmonofont[Path=/usr/share/fonts/truetype/cmu/,UprightFont=cmuntt.ttf,BoldFont=cmuntb.ttf,ItalicFont=cmunit.ttf,BoldItalicFont=cmuntx.ttf]{cmuntt.ttf}\ttfamily ulem}{$\text{ }$}\setmainfont[Path=/usr/share/fonts/truetype/cmu/,UprightFont=cmunrm.ttf,BoldFont=cmunbx.ttf,ItalicFont=cmunti.ttf,BoldItalicFont=cmunbi.ttf]{cmunrm.ttf}\setmonofont[Path=/usr/share/fonts/truetype/cmu/,UprightFont=cmuntt.ttf,BoldFont=cmuntb.ttf,ItalicFont=cmunit.ttf,BoldItalicFont=cmuntx.ttf]{cmunrm.ttf} package. See \mylref{140}{Formatting}.
\end{myquote}
{\bfseries
\begin{mydescription} \textbackslash{}underbrace 
\end{mydescription}
}
{\bfseries
\begin{mydescription} \textbackslash{}underline 
\end{mydescription}
}
{\bfseries
\begin{mydescription} \textbackslash{}unitlength 
\end{mydescription}
}
{\bfseries
\begin{mydescription} \textbackslash{}usebox 
\end{mydescription}
}
{\bfseries
\begin{mydescription} \textbackslash{}usecounter 
\end{mydescription}
}
{\bfseries
\begin{mydescription} \textbackslash{}uwave 
\end{mydescription}
}
\begin{myquote}\item{} Creates wavy underline. Requires {\ttfamily \setmainfont[Path=/usr/share/fonts/truetype/cmu/,UprightFont=cmunrm.ttf,BoldFont=cmunbx.ttf,ItalicFont=cmunti.ttf,BoldItalicFont=cmunbi.ttf]{cmuntt.ttf}\setmonofont[Path=/usr/share/fonts/truetype/cmu/,UprightFont=cmuntt.ttf,BoldFont=cmuntb.ttf,ItalicFont=cmunit.ttf,BoldItalicFont=cmuntx.ttf]{cmuntt.ttf}\ttfamily ulem}{$\text{ }$}\setmainfont[Path=/usr/share/fonts/truetype/cmu/,UprightFont=cmunrm.ttf,BoldFont=cmunbx.ttf,ItalicFont=cmunti.ttf,BoldItalicFont=cmunbi.ttf]{cmunrm.ttf}\setmonofont[Path=/usr/share/fonts/truetype/cmu/,UprightFont=cmuntt.ttf,BoldFont=cmuntb.ttf,ItalicFont=cmunit.ttf,BoldItalicFont=cmuntx.ttf]{cmunrm.ttf} package. See \mylref{140}{Formatting}.
\end{myquote}

\section{V}
\label{1041}{\bfseries
\begin{mydescription} \textbackslash{}value 
\end{mydescription}
}
{\bfseries
\begin{mydescription} \textbackslash{}vbox\{text\} 
\end{mydescription}
}
\begin{myquote}\item{} Encloses a paragraph\textquotesingle{}s text to prevent it from running over a page break
\end{myquote}
{\bfseries
\begin{mydescription} \textbackslash{}vcenter
\end{mydescription}
}
{\bfseries
\begin{mydescription} \textbackslash{}vdots 
\end{mydescription}
}
\begin{myquote}\item{} Creates vertical dots. See \mylref{525}{Mathematics}.
\end{myquote}
{\bfseries
\begin{mydescription} \textbackslash{}vector 
\end{mydescription}
}
{\bfseries
\begin{mydescription} \textbackslash{}verb 
\end{mydescription}
}
\begin{myquote}\item{} Creates inline verbatim text. See \mylref{138}{Formatting}.
\end{myquote}
{\bfseries
\begin{mydescription} \textbackslash{}vfill 
\end{mydescription}
}
{\bfseries
\begin{mydescription} \textbackslash{}vline 
\end{mydescription}
}
{\bfseries
\begin{mydescription} \textbackslash{}vphantom
\end{mydescription}
}
{\bfseries
\begin{mydescription} \textbackslash{}vspace
\end{mydescription}
}



\SGreen{}

\end{myitemize}
\chapter{Contributors}
\label{Contributors}
\begin{longtable}{rp{0.6\linewidth}}
\textbf{Edits}&\textbf{User}\\
1& \myhref{https://en.wikibooks.org/wiki/User:2147483647}{2147483647}\\
140& \myhref{https://en.wikibooks.org/wiki/User:3mta3}{3mta3}\\
2& \myhref{https://en.wikibooks.org/wiki/User:ABCD}{ABCD}\\
1& \myhref{https://en.wikibooks.org/w/index.php\%3ftitle=User:ATC2~enwikibooks\&action=edit\&redlink=1}{ATC2\~{}enwikibooks}\\
1& \myhref{https://en.wikibooks.org/w/index.php\%3ftitle=User:Aadornellesf\&action=edit\&redlink=1}{Aadornellesf}\\
4& \myhref{https://en.wikibooks.org/w/index.php\%3ftitle=User:Abalenkm\&action=edit\&redlink=1}{Abalenkm}\\
2& \myhref{https://en.wikibooks.org/w/index.php\%3ftitle=User:Abonnema\&action=edit\&redlink=1}{Abonnema}\\
1& \myhref{https://en.wikibooks.org/w/index.php\%3ftitle=User:Abramsky\&action=edit\&redlink=1}{Abramsky}\\
1& \myhref{https://en.wikibooks.org/w/index.php\%3ftitle=User:Abustany\&action=edit\&redlink=1}{Abustany}\\
1& \myhref{https://en.wikibooks.org/wiki/User:Adam_majewski}{Adam majewski}\\
3& \myhref{https://en.wikibooks.org/w/index.php\%3ftitle=User:Adelphious\&action=edit\&redlink=1}{Adelphious}\\
1& \myhref{https://en.wikibooks.org/w/index.php\%3ftitle=User:AdhillA97\&action=edit\&redlink=1}{AdhillA97}\\
1& \myhref{https://en.wikibooks.org/w/index.php\%3ftitle=User:Adouglass\&action=edit\&redlink=1}{Adouglass}\\
1& \myhref{https://en.wikibooks.org/w/index.php\%3ftitle=User:Adrianwn\&action=edit\&redlink=1}{Adrianwn}\\
46& \myhref{https://en.wikibooks.org/wiki/User:Adrignola}{Adrignola}\\
1& \myhref{https://en.wikibooks.org/w/index.php\%3ftitle=User:Aeonblue158\&action=edit\&redlink=1}{Aeonblue158}\\
1& \myhref{https://en.wikibooks.org/wiki/User:Ah3kal}{Ah3kal}\\
1& \myhref{https://en.wikibooks.org/w/index.php\%3ftitle=User:Ajmath62\&action=edit\&redlink=1}{Ajmath62}\\
1& \myhref{https://en.wikibooks.org/w/index.php\%3ftitle=User:Akim_Demaille\&action=edit\&redlink=1}{Akim Demaille}\\
1& \myhref{https://en.wikibooks.org/wiki/User:AlanBarrett}{AlanBarrett}\\
2& \myhref{https://en.wikibooks.org/w/index.php\%3ftitle=User:Alansandiego\&action=edit\&redlink=1}{Alansandiego}\\
121& \myhref{https://en.wikibooks.org/wiki/User:Alejo2083}{Alejo2083}\\
3& \myhref{https://en.wikibooks.org/wiki/User:AllenZh}{AllenZh}\\
9& \myhref{https://en.wikibooks.org/w/index.php\%3ftitle=User:Alzahrawi\&action=edit\&redlink=1}{Alzahrawi}\\
1& \myhref{https://en.wikibooks.org/w/index.php\%3ftitle=User:Amamory\&action=edit\&redlink=1}{Amamory}\\
998& \myhref{https://en.wikibooks.org/wiki/User:Ambrevar}{Ambrevar}\\
1& \myhref{https://en.wikibooks.org/w/index.php\%3ftitle=User:Anamma06\&action=edit\&redlink=1}{Anamma06}\\
2& \myhref{https://en.wikibooks.org/w/index.php\%3ftitle=User:Anarchyboy\&action=edit\&redlink=1}{Anarchyboy}\\
1& \myhref{https://en.wikibooks.org/w/index.php\%3ftitle=User:Anareth\&action=edit\&redlink=1}{Anareth}\\
2& \myhref{https://en.wikibooks.org/w/index.php\%3ftitle=User:AndreKR\&action=edit\&redlink=1}{AndreKR}\\
1& \myhref{https://en.wikibooks.org/w/index.php\%3ftitle=User:Andrea09falco\&action=edit\&redlink=1}{Andrea09falco}\\
6& \myhref{https://en.wikibooks.org/w/index.php\%3ftitle=User:Andyr~enwikibooks\&action=edit\&redlink=1}{Andyr\~{}enwikibooks}\\
1& \myhref{https://en.wikibooks.org/wiki/User:Angus}{Angus}\\
5& \myhref{https://en.wikibooks.org/wiki/User:Ans}{Ans}\\
2& \myhref{https://en.wikibooks.org/w/index.php\%3ftitle=User:Anthony_Deschamps\&action=edit\&redlink=1}{Anthony Deschamps}\\
2& \myhref{https://en.wikibooks.org/wiki/User:Anubhab91}{Anubhab91}\\
1& \myhref{https://en.wikibooks.org/wiki/User:Arbitrarily0}{Arbitrarily0}\\
3& \myhref{https://en.wikibooks.org/wiki/User:Arided}{Arided}\\
25& \myhref{https://en.wikibooks.org/wiki/User:Arnehe}{Arnehe}\\
1& \myhref{https://en.wikibooks.org/w/index.php\%3ftitle=User:Artemisfowl3rd\&action=edit\&redlink=1}{Artemisfowl3rd}\\
4& \myhref{https://en.wikibooks.org/w/index.php\%3ftitle=User:Arthurchy\&action=edit\&redlink=1}{Arthurchy}\\
1& \myhref{https://en.wikibooks.org/wiki/User:Arthurvogel}{Arthurvogel}\\
1& \myhref{https://en.wikibooks.org/wiki/User:Arunib}{Arunib}\\
1& \myhref{https://en.wikibooks.org/w/index.php\%3ftitle=User:Asmeurer\&action=edit\&redlink=1}{Asmeurer}\\
2& \myhref{https://en.wikibooks.org/w/index.php\%3ftitle=User:AsphyxiateDrake\&action=edit\&redlink=1}{AsphyxiateDrake}\\
1& \myhref{https://en.wikibooks.org/w/index.php\%3ftitle=User:Astrophizz~enwikibooks\&action=edit\&redlink=1}{Astrophizz\~{}enwikibooks}\\
1& \myhref{https://en.wikibooks.org/wiki/User:Atallcostsky}{Atallcostsky}\\
5& \myhref{https://en.wikibooks.org/wiki/User:Atcovi}{Atcovi}\\
2& \myhref{https://en.wikibooks.org/w/index.php\%3ftitle=User:AthanasiusOfAlex\&action=edit\&redlink=1}{AthanasiusOfAlex}\\
1& \myhref{https://en.wikibooks.org/wiki/User:Atiq_ur_Rehman}{Atiq ur Rehman}\\
1& \myhref{https://en.wikibooks.org/w/index.php\%3ftitle=User:Atulya1988\&action=edit\&redlink=1}{Atulya1988}\\
1& \myhref{https://en.wikibooks.org/w/index.php\%3ftitle=User:Austinmohr\&action=edit\&redlink=1}{Austinmohr}\\
13& \myhref{https://en.wikibooks.org/w/index.php\%3ftitle=User:Avila.gas\&action=edit\&redlink=1}{Avila.gas}\\
4& \myhref{https://en.wikibooks.org/wiki/User:Az1568}{Az1568}\\
2& \myhref{https://en.wikibooks.org/w/index.php\%3ftitle=User:BYIST\&action=edit\&redlink=1}{BYIST}\\
1& \myhref{https://en.wikibooks.org/w/index.php\%3ftitle=User:Bajrangkhichi96\&action=edit\&redlink=1}{Bajrangkhichi96}\\
1& \myhref{https://en.wikibooks.org/w/index.php\%3ftitle=User:Bakken\&action=edit\&redlink=1}{Bakken}\\
1& \myhref{https://en.wikibooks.org/w/index.php\%3ftitle=User:Bamgooly\&action=edit\&redlink=1}{Bamgooly}\\
1& \myhref{https://en.wikibooks.org/w/index.php\%3ftitle=User:Bamgooly~enwikibooks\&action=edit\&redlink=1}{Bamgooly\~{}enwikibooks}\\
4& \myhref{https://en.wikibooks.org/wiki/User:Basenga}{Basenga}\\
3& \myhref{https://en.wikibooks.org/w/index.php\%3ftitle=User:BbcNkl\&action=edit\&redlink=1}{BbcNkl}\\
1& \myhref{https://en.wikibooks.org/w/index.php\%3ftitle=User:Bcmpinc\&action=edit\&redlink=1}{Bcmpinc}\\
1& \myhref{https://en.wikibooks.org/w/index.php\%3ftitle=User:Belteshazzar\&action=edit\&redlink=1}{Belteshazzar}\\
1& \myhref{https://en.wikibooks.org/w/index.php\%3ftitle=User:Ben9243\&action=edit\&redlink=1}{Ben9243}\\
2& \myhref{https://en.wikibooks.org/wiki/User:Benjaminevans82~enwikibooks}{Benjaminevans82\~{}enwikibooks}\\
1& \myhref{https://en.wikibooks.org/w/index.php\%3ftitle=User:Benregn\&action=edit\&redlink=1}{Benregn}\\
1& \myhref{https://en.wikibooks.org/wiki/User:Benson_Muite}{Benson Muite}\\
2& \myhref{https://en.wikibooks.org/w/index.php\%3ftitle=User:Berettag\&action=edit\&redlink=1}{Berettag}\\
2& \myhref{https://en.wikibooks.org/w/index.php\%3ftitle=User:Bgeron\&action=edit\&redlink=1}{Bgeron}\\
1& \myhref{https://en.wikibooks.org/w/index.php\%3ftitle=User:Bhanuvrat\&action=edit\&redlink=1}{Bhanuvrat}\\
26& \myhref{https://en.wikibooks.org/wiki/User:BiT}{BiT}\\
1& \myhref{https://en.wikibooks.org/w/index.php\%3ftitle=User:Bianbum\&action=edit\&redlink=1}{Bianbum}\\
1& \myhref{https://en.wikibooks.org/w/index.php\%3ftitle=User:Bibi6\&action=edit\&redlink=1}{Bibi6}\\
1& \myhref{https://en.wikibooks.org/w/index.php\%3ftitle=User:Bigwyrm~enwikibooks\&action=edit\&redlink=1}{Bigwyrm\~{}enwikibooks}\\
2& \myhref{https://en.wikibooks.org/w/index.php\%3ftitle=User:Bilbo1507\&action=edit\&redlink=1}{Bilbo1507}\\
1& \myhref{https://en.wikibooks.org/w/index.php\%3ftitle=User:Billy_the_Goat_II\&action=edit\&redlink=1}{Billy the Goat II}\\
2& \myhref{https://en.wikibooks.org/w/index.php\%3ftitle=User:BlackMagic1943\&action=edit\&redlink=1}{BlackMagic1943}\\
1& \myhref{https://en.wikibooks.org/w/index.php\%3ftitle=User:Blacktrumpeter\&action=edit\&redlink=1}{Blacktrumpeter}\\
4& \myhref{https://en.wikibooks.org/w/index.php\%3ftitle=User:Blaisorblade\&action=edit\&redlink=1}{Blaisorblade}\\
1& \myhref{https://en.wikibooks.org/w/index.php\%3ftitle=User:Bombcar\&action=edit\&redlink=1}{Bombcar}\\
3& \myhref{https://en.wikibooks.org/w/index.php\%3ftitle=User:Bonuama\&action=edit\&redlink=1}{Bonuama}\\
1& \myhref{https://en.wikibooks.org/w/index.php\%3ftitle=User:Borgg\&action=edit\&redlink=1}{Borgg}\\
1& \myhref{https://en.wikibooks.org/wiki/User:Born2bgratis}{Born2bgratis}\\
1& \myhref{https://en.wikibooks.org/w/index.php\%3ftitle=User:Bpsullivan~enwikibooks\&action=edit\&redlink=1}{Bpsullivan\~{}enwikibooks}\\
1& \myhref{https://en.wikibooks.org/w/index.php\%3ftitle=User:Braindrain0000\&action=edit\&redlink=1}{Braindrain0000}\\
1& \myhref{https://en.wikibooks.org/wiki/User:Brammers}{Brammers}\\
1& \myhref{https://en.wikibooks.org/w/index.php\%3ftitle=User:Brendanarnold\&action=edit\&redlink=1}{Brendanarnold}\\
1& \myhref{https://en.wikibooks.org/w/index.php\%3ftitle=User:BrettMontgomery\&action=edit\&redlink=1}{BrettMontgomery}\\
1& \myhref{https://en.wikibooks.org/w/index.php\%3ftitle=User:Brevity\&action=edit\&redlink=1}{Brevity}\\
1& \myhref{https://en.wikibooks.org/w/index.php\%3ftitle=User:Briancricks\&action=edit\&redlink=1}{Briancricks}\\
5& \myhref{https://en.wikibooks.org/w/index.php\%3ftitle=User:Bro4\&action=edit\&redlink=1}{Bro4}\\
1& \myhref{https://en.wikibooks.org/w/index.php\%3ftitle=User:Bryant1410\&action=edit\&redlink=1}{Bryant1410}\\
1& \myhref{https://en.wikibooks.org/w/index.php\%3ftitle=User:Bsander~enwikibooks\&action=edit\&redlink=1}{Bsander\~{}enwikibooks}\\
8& \myhref{https://en.wikibooks.org/wiki/User:Bumbulski}{Bumbulski}\\
1& \myhref{https://en.wikibooks.org/w/index.php\%3ftitle=User:Bunyk\&action=edit\&redlink=1}{Bunyk}\\
1& \myhref{https://en.wikibooks.org/w/index.php\%3ftitle=User:BwDraco\&action=edit\&redlink=1}{BwDraco}\\
1& \myhref{https://en.wikibooks.org/w/index.php\%3ftitle=User:Byassine52\&action=edit\&redlink=1}{Byassine52}\\
1& \myhref{https://en.wikibooks.org/w/index.php\%3ftitle=User:Bytecrook\&action=edit\&redlink=1}{Bytecrook}\\
1& \myhref{https://en.wikibooks.org/w/index.php\%3ftitle=User:C3l\&action=edit\&redlink=1}{C3l}\\
7& \myhref{https://en.wikibooks.org/w/index.php\%3ftitle=User:CD-Stevens\&action=edit\&redlink=1}{CD-{}Stevens}\\
1& \myhref{https://en.wikibooks.org/w/index.php\%3ftitle=User:Caesura\&action=edit\&redlink=1}{Caesura}\\
1& \myhref{https://en.wikibooks.org/wiki/User:Calimo}{Calimo}\\
1& \myhref{https://en.wikibooks.org/wiki/User:CallumPoole}{CallumPoole}\\
5& \myhref{https://en.wikibooks.org/w/index.php\%3ftitle=User:Cameronc\&action=edit\&redlink=1}{Cameronc}\\
3& \myhref{https://en.wikibooks.org/w/index.php\%3ftitle=User:Cameronc~enwikibooks\&action=edit\&redlink=1}{Cameronc\~{}enwikibooks}\\
1& \myhref{https://en.wikibooks.org/w/index.php\%3ftitle=User:Canageek\&action=edit\&redlink=1}{Canageek}\\
2& \myhref{https://en.wikibooks.org/wiki/User:CarsracBot}{CarsracBot}\\
1& \myhref{https://en.wikibooks.org/w/index.php\%3ftitle=User:Cdecoro\&action=edit\&redlink=1}{Cdecoro}\\
3& \myhref{https://en.wikibooks.org/w/index.php\%3ftitle=User:Cengique\&action=edit\&redlink=1}{Cengique}\\
5& \myhref{https://en.wikibooks.org/wiki/User:Cerniagigante}{Cerniagigante}\\
1& \myhref{https://en.wikibooks.org/w/index.php\%3ftitle=User:Cfailde~enwikibooks\&action=edit\&redlink=1}{Cfailde\~{}enwikibooks}\\
1& \myhref{https://en.wikibooks.org/w/index.php\%3ftitle=User:Ch605852\&action=edit\&redlink=1}{Ch605852}\\
1& \myhref{https://en.wikibooks.org/w/index.php\%3ftitle=User:Chafe66\&action=edit\&redlink=1}{Chafe66}\\
2& \myhref{https://en.wikibooks.org/wiki/User:Chaojoker}{Chaojoker}\\
6& \myhref{https://en.wikibooks.org/wiki/User:Chazz}{Chazz}\\
1& \myhref{https://en.wikibooks.org/w/index.php\%3ftitle=User:Chbarts\&action=edit\&redlink=1}{Chbarts}\\
2& \myhref{https://en.wikibooks.org/w/index.php\%3ftitle=User:Chisophugis\&action=edit\&redlink=1}{Chisophugis}\\
142& \myhref{https://en.wikibooks.org/wiki/User:ChrisHodgesUK}{ChrisHodgesUK}\\
1& \myhref{https://en.wikibooks.org/w/index.php\%3ftitle=User:ChristianGruen\&action=edit\&redlink=1}{ChristianGruen}\\
1& \myhref{https://en.wikibooks.org/w/index.php\%3ftitle=User:Chuaprap\&action=edit\&redlink=1}{Chuaprap}\\
1& \myhref{https://en.wikibooks.org/wiki/User:Chuckhoffmann}{Chuckhoffmann}\\
2& \myhref{https://en.wikibooks.org/wiki/User:Clapsus}{Clapsus}\\
1& \myhref{https://en.wikibooks.org/wiki/User:Clebell}{Clebell}\\
1& \myhref{https://en.wikibooks.org/w/index.php\%3ftitle=User:Codairem\&action=edit\&redlink=1}{Codairem}\\
1& \myhref{https://en.wikibooks.org/w/index.php\%3ftitle=User:ColeLoki\&action=edit\&redlink=1}{ColeLoki}\\
1& \myhref{https://en.wikibooks.org/w/index.php\%3ftitle=User:Collinpark\&action=edit\&redlink=1}{Collinpark}\\
1& \myhref{https://en.wikibooks.org/w/index.php\%3ftitle=User:Colonel486\&action=edit\&redlink=1}{Colonel486}\\
1& \myhref{https://en.wikibooks.org/wiki/User:Columbus240}{Columbus240}\\
4& \myhref{https://en.wikibooks.org/wiki/User:CommonsDelinker}{CommonsDelinker}\\
1& \myhref{https://en.wikibooks.org/w/index.php\%3ftitle=User:Comput2h\&action=edit\&redlink=1}{Comput2h}\\
2& \myhref{https://en.wikibooks.org/wiki/User:Computermacgyver}{Computermacgyver}\\
5& \myhref{https://en.wikibooks.org/w/index.php\%3ftitle=User:ConditionalZenith~enwikibooks\&action=edit\&redlink=1}{ConditionalZenith\~{}enwikibooks}\\
1& \myhref{https://en.wikibooks.org/w/index.php\%3ftitle=User:Conighion\&action=edit\&redlink=1}{Conighion}\\
2& \myhref{https://en.wikibooks.org/wiki/User:Conrad.Irwin}{Conrad.Irwin}\\
1& \myhref{https://en.wikibooks.org/w/index.php\%3ftitle=User:Constantine\&action=edit\&redlink=1}{Constantine}\\
1& \myhref{https://en.wikibooks.org/w/index.php\%3ftitle=User:Control.valve\&action=edit\&redlink=1}{Control.valve}\\
1& \myhref{https://en.wikibooks.org/wiki/User:Courcelles}{Courcelles}\\
2& \myhref{https://en.wikibooks.org/w/index.php\%3ftitle=User:CptGeek\&action=edit\&redlink=1}{CptGeek}\\
1& \myhref{https://en.wikibooks.org/w/index.php\%3ftitle=User:Crabpot8\&action=edit\&redlink=1}{Crabpot8}\\
1& \myhref{https://en.wikibooks.org/w/index.php\%3ftitle=User:Crasic~enwikibooks\&action=edit\&redlink=1}{Crasic\~{}enwikibooks}\\
1& \myhref{https://en.wikibooks.org/wiki/User:Crasshopper}{Crasshopper}\\
2& \myhref{https://en.wikibooks.org/w/index.php\%3ftitle=User:CrazyTerabyte\&action=edit\&redlink=1}{CrazyTerabyte}\\
10& \myhref{https://en.wikibooks.org/w/index.php\%3ftitle=User:Crissov\&action=edit\&redlink=1}{Crissov}\\
1& \myhref{https://en.wikibooks.org/w/index.php\%3ftitle=User:CtrlAltCarrot\&action=edit\&redlink=1}{CtrlAltCarrot}\\
2& \myhref{https://en.wikibooks.org/wiki/User:C\%25C3\%25ADcero}{Cícero}\\
1& \myhref{https://en.wikibooks.org/w/index.php\%3ftitle=User:DagErlingSm\%25C3\%25B8rgrav\&action=edit\&redlink=1}{DagErlingSmørgrav}\\
77& \myhref{https://en.wikibooks.org/wiki/User:Dan_Polansky}{Dan Polansky}\\
1& \myhref{https://en.wikibooks.org/wiki/User:Daniel_Mietchen}{Daniel Mietchen}\\
1& \myhref{https://en.wikibooks.org/w/index.php\%3ftitle=User:Danielstrong52\&action=edit\&redlink=1}{Danielstrong52}\\
1& \myhref{https://en.wikibooks.org/w/index.php\%3ftitle=User:Danroa\&action=edit\&redlink=1}{Danroa}\\
1& \myhref{https://en.wikibooks.org/w/index.php\%3ftitle=User:DarkSheep\&action=edit\&redlink=1}{DarkSheep}\\
1& \myhref{https://en.wikibooks.org/wiki/User:Darklama}{Darklama}\\
1& \myhref{https://en.wikibooks.org/w/index.php\%3ftitle=User:Daveturnr\&action=edit\&redlink=1}{Daveturnr}\\
1& \myhref{https://en.wikibooks.org/w/index.php\%3ftitle=User:David_Gaudet\&action=edit\&redlink=1}{David Gaudet}\\
1& \myhref{https://en.wikibooks.org/w/index.php\%3ftitle=User:David_Ollodart\&action=edit\&redlink=1}{David Ollodart}\\
1& \myhref{https://en.wikibooks.org/wiki/User:David.s.hollman}{David.s.hollman}\\
5& \myhref{https://en.wikibooks.org/wiki/User:DavidMcKenzie}{DavidMcKenzie}\\
1& \myhref{https://en.wikibooks.org/w/index.php\%3ftitle=User:Daviewales\&action=edit\&redlink=1}{Daviewales}\\
1& \myhref{https://en.wikibooks.org/w/index.php\%3ftitle=User:Debejyo\&action=edit\&redlink=1}{Debejyo}\\
2& \myhref{https://en.wikibooks.org/wiki/User:Defender}{Defender}\\
1& \myhref{https://en.wikibooks.org/w/index.php\%3ftitle=User:Deltasun\&action=edit\&redlink=1}{Deltasun}\\
1& \myhref{https://en.wikibooks.org/w/index.php\%3ftitle=User:Dendik\&action=edit\&redlink=1}{Dendik}\\
102& \myhref{https://en.wikibooks.org/wiki/User:Derbeth}{Derbeth}\\
1& \myhref{https://en.wikibooks.org/w/index.php\%3ftitle=User:Derwaldrandfoerster\&action=edit\&redlink=1}{Derwaldrandfoerster}\\
1& \myhref{https://en.wikibooks.org/w/index.php\%3ftitle=User:Dieguico\&action=edit\&redlink=1}{Dieguico}\\
6& \myhref{https://en.wikibooks.org/wiki/User:Dilaudid}{Dilaudid}\\
1& \myhref{https://en.wikibooks.org/w/index.php\%3ftitle=User:Dingens5\&action=edit\&redlink=1}{Dingens5}\\
3& \myhref{https://en.wikibooks.org/wiki/User:Diomidis_Spinellis}{Diomidis Spinellis}\\
143& \myhref{https://en.wikibooks.org/wiki/User:Dirk_H\%25C3\%25BCnniger}{Dirk Hünniger}\\
2& \myhref{https://en.wikibooks.org/wiki/User:Dlituiev}{Dlituiev}\\
3& \myhref{https://en.wikibooks.org/wiki/User:Dmb}{Dmb}\\
2& \myhref{https://en.wikibooks.org/wiki/User:DmitriyZotikov}{DmitriyZotikov}\\
3& \myhref{https://en.wikibooks.org/w/index.php\%3ftitle=User:Dncarley\&action=edit\&redlink=1}{Dncarley}\\
11& \myhref{https://en.wikibooks.org/w/index.php\%3ftitle=User:DoVlaNik993\&action=edit\&redlink=1}{DoVlaNik993}\\
10& \myhref{https://en.wikibooks.org/w/index.php\%3ftitle=User:Doking23\&action=edit\&redlink=1}{Doking23}\\
4& \myhref{https://en.wikibooks.org/w/index.php\%3ftitle=User:Donok11\&action=edit\&redlink=1}{Donok11}\\
12& \myhref{https://en.wikibooks.org/w/index.php\%3ftitle=User:Dplaza000\&action=edit\&redlink=1}{Dplaza000}\\
1& \myhref{https://en.wikibooks.org/w/index.php\%3ftitle=User:Dporter\&action=edit\&redlink=1}{Dporter}\\
1& \myhref{https://en.wikibooks.org/w/index.php\%3ftitle=User:Dr0pi\&action=edit\&redlink=1}{Dr0pi}\\
1& \myhref{https://en.wikibooks.org/w/index.php\%3ftitle=User:DrSarah_Calumet\&action=edit\&redlink=1}{DrSarah Calumet}\\
2& \myhref{https://en.wikibooks.org/w/index.php\%3ftitle=User:DrTobbe\&action=edit\&redlink=1}{DrTobbe}\\
5& \myhref{https://en.wikibooks.org/w/index.php\%3ftitle=User:Drasquared\&action=edit\&redlink=1}{Drasquared}\\
3& \myhref{https://en.wikibooks.org/w/index.php\%3ftitle=User:Dreaven3\&action=edit\&redlink=1}{Dreaven3}\\
12& \myhref{https://en.wikibooks.org/w/index.php\%3ftitle=User:Dredmorbius\&action=edit\&redlink=1}{Dredmorbius}\\
7& \myhref{https://en.wikibooks.org/w/index.php\%3ftitle=User:Drevicko\&action=edit\&redlink=1}{Drevicko}\\
3& \myhref{https://en.wikibooks.org/wiki/User:Drewbie}{Drewbie}\\
1& \myhref{https://en.wikibooks.org/w/index.php\%3ftitle=User:Dubbaluga~enwikibooks\&action=edit\&redlink=1}{Dubbaluga\~{}enwikibooks}\\
3& \myhref{https://en.wikibooks.org/w/index.php\%3ftitle=User:E.lewis1\&action=edit\&redlink=1}{E.lewis1}\\
1& \myhref{https://en.wikibooks.org/w/index.php\%3ftitle=User:Echeban\&action=edit\&redlink=1}{Echeban}\\
13& \myhref{https://en.wikibooks.org/w/index.php\%3ftitle=User:Ediahist\&action=edit\&redlink=1}{Ediahist}\\
1& \myhref{https://en.wikibooks.org/wiki/User:Edudobay}{Edudobay}\\
1& \myhref{https://en.wikibooks.org/wiki/User:Efex3}{Efex3}\\
1& \myhref{https://en.wikibooks.org/w/index.php\%3ftitle=User:Einjohn\&action=edit\&redlink=1}{Einjohn}\\
3& \myhref{https://en.wikibooks.org/w/index.php\%3ftitle=User:Elliptic1\&action=edit\&redlink=1}{Elliptic1}\\
1& \myhref{https://en.wikibooks.org/w/index.php\%3ftitle=User:Elwikipedista~enwikibooks\&action=edit\&redlink=1}{Elwikipedista\~{}enwikibooks}\\
1& \myhref{https://en.wikibooks.org/w/index.php\%3ftitle=User:Empirical_bayesian\&action=edit\&redlink=1}{Empirical bayesian}\\
1& \myhref{https://en.wikibooks.org/w/index.php\%3ftitle=User:Emreg00\&action=edit\&redlink=1}{Emreg00}\\
1& \myhref{https://en.wikibooks.org/w/index.php\%3ftitle=User:Epic_Wink\&action=edit\&redlink=1}{Epic Wink}\\
1& \myhref{https://en.wikibooks.org/w/index.php\%3ftitle=User:ErickChacon\&action=edit\&redlink=1}{ErickChacon}\\
1& \myhref{https://en.wikibooks.org/w/index.php\%3ftitle=User:Erp\&action=edit\&redlink=1}{Erp}\\
16& \myhref{https://en.wikibooks.org/w/index.php\%3ftitle=User:Erylaos\&action=edit\&redlink=1}{Erylaos}\\
3& \myhref{https://en.wikibooks.org/w/index.php\%3ftitle=User:Escalator~enwikibooks\&action=edit\&redlink=1}{Escalator\~{}enwikibooks}\\
1& \myhref{https://en.wikibooks.org/w/index.php\%3ftitle=User:Eselmeister\&action=edit\&redlink=1}{Eselmeister}\\
2& \myhref{https://en.wikibooks.org/w/index.php\%3ftitle=User:Espinozahg\&action=edit\&redlink=1}{Espinozahg}\\
1& \myhref{https://en.wikibooks.org/wiki/User:Ethefor}{Ethefor}\\
1& \myhref{https://en.wikibooks.org/w/index.php\%3ftitle=User:Etoombs\&action=edit\&redlink=1}{Etoombs}\\
1& \myhref{https://en.wikibooks.org/w/index.php\%3ftitle=User:Eudoxos~enwikibooks\&action=edit\&redlink=1}{Eudoxos\~{}enwikibooks}\\
2& \myhref{https://en.wikibooks.org/w/index.php\%3ftitle=User:EvanKroske\&action=edit\&redlink=1}{EvanKroske}\\
1& \myhref{https://en.wikibooks.org/w/index.php\%3ftitle=User:Evin~enwikibooks\&action=edit\&redlink=1}{Evin\~{}enwikibooks}\\
5& \myhref{https://en.wikibooks.org/w/index.php\%3ftitle=User:Eyliu~enwikibooks\&action=edit\&redlink=1}{Eyliu\~{}enwikibooks}\\
1& \myhref{https://en.wikibooks.org/w/index.php\%3ftitle=User:Felipecarres\&action=edit\&redlink=1}{Felipecarres}\\
3& \myhref{https://en.wikibooks.org/w/index.php\%3ftitle=User:Ffangs\&action=edit\&redlink=1}{Ffangs}\\
1& \myhref{https://en.wikibooks.org/w/index.php\%3ftitle=User:Ffavela\&action=edit\&redlink=1}{Ffavela}\\
2& \myhref{https://en.wikibooks.org/wiki/User:Filip_Dominec}{Filip Dominec}\\
1& \myhref{https://en.wikibooks.org/w/index.php\%3ftitle=User:Fishix\&action=edit\&redlink=1}{Fishix}\\
2& \myhref{https://en.wikibooks.org/w/index.php\%3ftitle=User:Fishpi\&action=edit\&redlink=1}{Fishpi}\\
1& \myhref{https://en.wikibooks.org/w/index.php\%3ftitle=User:Flal\&action=edit\&redlink=1}{Flal}\\
1& \myhref{https://en.wikibooks.org/wiki/User:Flamenco108}{Flamenco108}\\
1& \myhref{https://en.wikibooks.org/w/index.php\%3ftitle=User:FlashSheridan\&action=edit\&redlink=1}{FlashSheridan}\\
1& \myhref{https://en.wikibooks.org/w/index.php\%3ftitle=User:Flip\&action=edit\&redlink=1}{Flip}\\
1& \myhref{https://en.wikibooks.org/wiki/User:Fmccown}{Fmccown}\\
1& \myhref{https://en.wikibooks.org/w/index.php\%3ftitle=User:Frakturfreund\&action=edit\&redlink=1}{Frakturfreund}\\
2& \myhref{https://en.wikibooks.org/wiki/User:Franklin_Yu}{Franklin Yu}\\
2& \myhref{https://en.wikibooks.org/w/index.php\%3ftitle=User:FranklyMyDear...\&action=edit\&redlink=1}{FranklyMyDear...}\\
6& \myhref{https://en.wikibooks.org/w/index.php\%3ftitle=User:Franzl_aus_tirol\&action=edit\&redlink=1}{Franzl aus tirol}\\
1& \myhref{https://en.wikibooks.org/w/index.php\%3ftitle=User:Frap\&action=edit\&redlink=1}{Frap}\\
1& \myhref{https://en.wikibooks.org/w/index.php\%3ftitle=User:Fredmaranhao\&action=edit\&redlink=1}{Fredmaranhao}\\
1& \myhref{https://en.wikibooks.org/w/index.php\%3ftitle=User:FredrikMeyer\&action=edit\&redlink=1}{FredrikMeyer}\\
1& \myhref{https://en.wikibooks.org/w/index.php\%3ftitle=User:Froskoy\&action=edit\&redlink=1}{Froskoy}\\
1& \myhref{https://en.wikibooks.org/w/index.php\%3ftitle=User:Fsart\&action=edit\&redlink=1}{Fsart}\\
1& \myhref{https://en.wikibooks.org/w/index.php\%3ftitle=User:Ftravers\&action=edit\&redlink=1}{Ftravers}\\
1& \myhref{https://en.wikibooks.org/w/index.php\%3ftitle=User:Funkenstern\&action=edit\&redlink=1}{Funkenstern}\\
1& \myhref{https://en.wikibooks.org/wiki/User:GPHemsley}{GPHemsley}\\
1& \myhref{https://en.wikibooks.org/w/index.php\%3ftitle=User:Gallen01\&action=edit\&redlink=1}{Gallen01}\\
1& \myhref{https://en.wikibooks.org/w/index.php\%3ftitle=User:Garfl\&action=edit\&redlink=1}{Garfl}\\
1& \myhref{https://en.wikibooks.org/w/index.php\%3ftitle=User:Garoth~enwikibooks\&action=edit\&redlink=1}{Garoth\~{}enwikibooks}\\
1& \myhref{https://en.wikibooks.org/wiki/User:GavinMcGimpsey}{GavinMcGimpsey}\\
3& \myhref{https://en.wikibooks.org/w/index.php\%3ftitle=User:Geetha_nitc\&action=edit\&redlink=1}{Geetha nitc}\\
2& \myhref{https://en.wikibooks.org/w/index.php\%3ftitle=User:Gelbukh\&action=edit\&redlink=1}{Gelbukh}\\
1& \myhref{https://en.wikibooks.org/w/index.php\%3ftitle=User:Geminatea~enwikibooks\&action=edit\&redlink=1}{Geminatea\~{}enwikibooks}\\
1& \myhref{https://en.wikibooks.org/w/index.php\%3ftitle=User:Genethecist\&action=edit\&redlink=1}{Genethecist}\\
1& \myhref{https://en.wikibooks.org/w/index.php\%3ftitle=User:Germanzs\&action=edit\&redlink=1}{Germanzs}\\
1& \myhref{https://en.wikibooks.org/w/index.php\%3ftitle=User:Ghostofkendo\&action=edit\&redlink=1}{Ghostofkendo}\\
6& \myhref{https://en.wikibooks.org/w/index.php\%3ftitle=User:Ghoti\&action=edit\&redlink=1}{Ghoti}\\
1& \myhref{https://en.wikibooks.org/w/index.php\%3ftitle=User:Gibravo\&action=edit\&redlink=1}{Gibravo}\\
1& \myhref{https://en.wikibooks.org/w/index.php\%3ftitle=User:Gillespie09\&action=edit\&redlink=1}{Gillespie09}\\
4& \myhref{https://en.wikibooks.org/w/index.php\%3ftitle=User:Gkc~enwikibooks\&action=edit\&redlink=1}{Gkc\~{}enwikibooks}\\
1& \myhref{https://en.wikibooks.org/w/index.php\%3ftitle=User:Gladiool\&action=edit\&redlink=1}{Gladiool}\\
1& \myhref{https://en.wikibooks.org/w/index.php\%3ftitle=User:Glad~enwikibooks\&action=edit\&redlink=1}{Glad\~{}enwikibooks}\\
3& \myhref{https://en.wikibooks.org/w/index.php\%3ftitle=User:Glosser.ca\&action=edit\&redlink=1}{Glosser.ca}\\
3& \myhref{https://en.wikibooks.org/w/index.php\%3ftitle=User:Gmacar\&action=edit\&redlink=1}{Gmacar}\\
1& \myhref{https://en.wikibooks.org/w/index.php\%3ftitle=User:Gmh04~enwikibooks\&action=edit\&redlink=1}{Gmh04\~{}enwikibooks}\\
7& \myhref{https://en.wikibooks.org/w/index.php\%3ftitle=User:Gms\&action=edit\&redlink=1}{Gms}\\
1& \myhref{https://en.wikibooks.org/w/index.php\%3ftitle=User:Go.pbam.\&action=edit\&redlink=1}{Go.pbam.}\\
1& \myhref{https://en.wikibooks.org/w/index.php\%3ftitle=User:Goldkatze\&action=edit\&redlink=1}{Goldkatze}\\
1& \myhref{https://en.wikibooks.org/wiki/User:GorillaWarfare}{GorillaWarfare}\\
1& \myhref{https://en.wikibooks.org/w/index.php\%3ftitle=User:Graemeg~enwikibooks\&action=edit\&redlink=1}{Graemeg\~{}enwikibooks}\\
1& \myhref{https://en.wikibooks.org/w/index.php\%3ftitle=User:Graf_Westerholt\&action=edit\&redlink=1}{Graf Westerholt}\\
1& \myhref{https://en.wikibooks.org/wiki/User:Greenbreen}{Greenbreen}\\
1& \myhref{https://en.wikibooks.org/w/index.php\%3ftitle=User:Grenouille~enwikibooks\&action=edit\&redlink=1}{Grenouille\~{}enwikibooks}\\
1& \myhref{https://en.wikibooks.org/w/index.php\%3ftitle=User:Grj23\&action=edit\&redlink=1}{Grj23}\\
2& \myhref{https://en.wikibooks.org/wiki/User:Gronau~enwikibooks}{Gronau\~{}enwikibooks}\\
2& \myhref{https://en.wikibooks.org/wiki/User:Gryllida}{Gryllida}\\
1& \myhref{https://en.wikibooks.org/w/index.php\%3ftitle=User:Guyrobbie\&action=edit\&redlink=1}{Guyrobbie}\\
2& \myhref{https://en.wikibooks.org/w/index.php\%3ftitle=User:Guzo\&action=edit\&redlink=1}{Guzo}\\
1& \myhref{https://en.wikibooks.org/w/index.php\%3ftitle=User:Gwpl\&action=edit\&redlink=1}{Gwpl}\\
1& \myhref{https://en.wikibooks.org/w/index.php\%3ftitle=User:Gyro_Copter\&action=edit\&redlink=1}{Gyro Copter}\\
1& \myhref{https://en.wikibooks.org/w/index.php\%3ftitle=User:G\%25C3\%25B6tz\&action=edit\&redlink=1}{Götz}\\
1& \myhref{https://en.wikibooks.org/w/index.php\%3ftitle=User:Habil_zare\&action=edit\&redlink=1}{Habil zare}\\
1& \myhref{https://en.wikibooks.org/wiki/User:Hagindaz}{Hagindaz}\\
5& \myhref{https://en.wikibooks.org/w/index.php\%3ftitle=User:Halilsen\&action=edit\&redlink=1}{Halilsen}\\
7& \myhref{https://en.wikibooks.org/w/index.php\%3ftitle=User:Hankjones\&action=edit\&redlink=1}{Hankjones}\\
1& \myhref{https://en.wikibooks.org/wiki/User:Hankwang}{Hankwang}\\
1& \myhref{https://en.wikibooks.org/wiki/User:Hannes_R\%25C3\%25B6st}{Hannes Röst}\\
1& \myhref{https://en.wikibooks.org/w/index.php\%3ftitle=User:HansCronau\&action=edit\&redlink=1}{HansCronau}\\
2& \myhref{https://en.wikibooks.org/w/index.php\%3ftitle=User:Hansfn\&action=edit\&redlink=1}{Hansfn}\\
5& \myhref{https://en.wikibooks.org/w/index.php\%3ftitle=User:Hapli\&action=edit\&redlink=1}{Hapli}\\
3& \myhref{https://en.wikibooks.org/w/index.php\%3ftitle=User:Harish_victory\&action=edit\&redlink=1}{Harish victory}\\
1& \myhref{https://en.wikibooks.org/wiki/User:Harp}{Harp}\\
3& \myhref{https://en.wikibooks.org/w/index.php\%3ftitle=User:Harrikoo\&action=edit\&redlink=1}{Harrikoo}\\
1& \myhref{https://en.wikibooks.org/w/index.php\%3ftitle=User:Harrywt\&action=edit\&redlink=1}{Harrywt}\\
1& \myhref{https://en.wikibooks.org/w/index.php\%3ftitle=User:Hdankowski\&action=edit\&redlink=1}{Hdankowski}\\
4& \myhref{https://en.wikibooks.org/wiki/User:He7d3r}{He7d3r}\\
1& \myhref{https://en.wikibooks.org/w/index.php\%3ftitle=User:Hello71\&action=edit\&redlink=1}{Hello71}\\
5& \myhref{https://en.wikibooks.org/w/index.php\%3ftitle=User:Helptry\&action=edit\&redlink=1}{Helptry}\\
1& \myhref{https://en.wikibooks.org/w/index.php\%3ftitle=User:Hendiadyon\&action=edit\&redlink=1}{Hendiadyon}\\
1& \myhref{https://en.wikibooks.org/w/index.php\%3ftitle=User:Henridv\&action=edit\&redlink=1}{Henridv}\\
2& \myhref{https://en.wikibooks.org/w/index.php\%3ftitle=User:HenrikMidtiby~enwikibooks\&action=edit\&redlink=1}{HenrikMidtiby\~{}enwikibooks}\\
4& \myhref{https://en.wikibooks.org/wiki/User:Henry_Tallboys}{Henry Tallboys}\\
5& \myhref{https://en.wikibooks.org/w/index.php\%3ftitle=User:Henrybissonnette\&action=edit\&redlink=1}{Henrybissonnette}\\
1& \myhref{https://en.wikibooks.org/wiki/User:Herbythyme}{Herbythyme}\\
2& \myhref{https://en.wikibooks.org/w/index.php\%3ftitle=User:Hermine_potter\&action=edit\&redlink=1}{Hermine potter}\\
1& \myhref{https://en.wikibooks.org/w/index.php\%3ftitle=User:Hippasus\&action=edit\&redlink=1}{Hippasus}\\
1& \myhref{https://en.wikibooks.org/w/index.php\%3ftitle=User:Hjsb\&action=edit\&redlink=1}{Hjsb}\\
1& \myhref{https://en.wikibooks.org/w/index.php\%3ftitle=User:Hokiehead~enwikibooks\&action=edit\&redlink=1}{Hokiehead\~{}enwikibooks}\\
1& \myhref{https://en.wikibooks.org/w/index.php\%3ftitle=User:Honza889\&action=edit\&redlink=1}{Honza889}\\
1& \myhref{https://en.wikibooks.org/w/index.php\%3ftitle=User:Hops_Splurt\&action=edit\&redlink=1}{Hops Splurt}\\
6& \myhref{https://en.wikibooks.org/w/index.php\%3ftitle=User:Hosszuka\&action=edit\&redlink=1}{Hosszuka}\\
3& \myhref{https://en.wikibooks.org/w/index.php\%3ftitle=User:Hroobjartr\&action=edit\&redlink=1}{Hroobjartr}\\
1& \myhref{https://en.wikibooks.org/w/index.php\%3ftitle=User:Hsmyers~enwikibooks\&action=edit\&redlink=1}{Hsmyers\~{}enwikibooks}\\
1& \myhref{https://en.wikibooks.org/w/index.php\%3ftitle=User:Hulten\&action=edit\&redlink=1}{Hulten}\\
1& \myhref{https://en.wikibooks.org/wiki/User:ILubeMyCucumbers20}{ILubeMyCucumbers20}\\
1& \myhref{https://en.wikibooks.org/w/index.php\%3ftitle=User:IMneme\&action=edit\&redlink=1}{IMneme}\\
1& \myhref{https://en.wikibooks.org/w/index.php\%3ftitle=User:Icc97\&action=edit\&redlink=1}{Icc97}\\
26& \myhref{https://en.wikibooks.org/w/index.php\%3ftitle=User:Igjimh\&action=edit\&redlink=1}{Igjimh}\\
1& \myhref{https://en.wikibooks.org/w/index.php\%3ftitle=User:Immae\&action=edit\&redlink=1}{Immae}\\
2& \myhref{https://en.wikibooks.org/w/index.php\%3ftitle=User:Incognito668\&action=edit\&redlink=1}{Incognito668}\\
1& \myhref{https://en.wikibooks.org/wiki/User:Inductiveload}{Inductiveload}\\
5& \myhref{https://en.wikibooks.org/w/index.php\%3ftitle=User:Infenwe\&action=edit\&redlink=1}{Infenwe}\\
3& \myhref{https://en.wikibooks.org/wiki/User:Infinite0694}{Infinite0694}\\
2& \myhref{https://en.wikibooks.org/w/index.php\%3ftitle=User:Insaneinside\&action=edit\&redlink=1}{Insaneinside}\\
5& \myhref{https://en.wikibooks.org/w/index.php\%3ftitle=User:InverseHypercube\&action=edit\&redlink=1}{InverseHypercube}\\
2& \myhref{https://en.wikibooks.org/w/index.php\%3ftitle=User:Irenas996\&action=edit\&redlink=1}{Irenas996}\\
1& \myhref{https://en.wikibooks.org/wiki/User:IrfanAli}{IrfanAli}\\
23& \myhref{https://en.wikibooks.org/wiki/User:Ish_ishwar~enwikibooks}{Ish ishwar\~{}enwikibooks}\\
1& \myhref{https://en.wikibooks.org/wiki/User:Itai}{Itai}\\
4& \myhref{https://en.wikibooks.org/wiki/User:JECompton~enwikibooks}{JECompton\~{}enwikibooks}\\
5& \myhref{https://en.wikibooks.org/w/index.php\%3ftitle=User:JV~enwikibooks\&action=edit\&redlink=1}{JV\~{}enwikibooks}\\
1& \myhref{https://en.wikibooks.org/w/index.php\%3ftitle=User:JW_00000\&action=edit\&redlink=1}{JW 00000}\\
6& \myhref{https://en.wikibooks.org/w/index.php\%3ftitle=User:Jacho\&action=edit\&redlink=1}{Jacho}\\
1& \myhref{https://en.wikibooks.org/wiki/User:JackPotte}{JackPotte}\\
1& \myhref{https://en.wikibooks.org/w/index.php\%3ftitle=User:Jacobrothstein\&action=edit\&redlink=1}{Jacobrothstein}\\
1& \myhref{https://en.wikibooks.org/wiki/User:Jafeluv}{Jafeluv}\\
2& \myhref{https://en.wikibooks.org/w/index.php\%3ftitle=User:Jaleks\&action=edit\&redlink=1}{Jaleks}\\
2& \myhref{https://en.wikibooks.org/w/index.php\%3ftitle=User:Jamoroch~enwikibooks\&action=edit\&redlink=1}{Jamoroch\~{}enwikibooks}\\
1& \myhref{https://en.wikibooks.org/w/index.php\%3ftitle=User:Jan_Winnicki\&action=edit\&redlink=1}{Jan Winnicki}\\
9& \myhref{https://en.wikibooks.org/w/index.php\%3ftitle=User:Janltx\&action=edit\&redlink=1}{Janltx}\\
1& \myhref{https://en.wikibooks.org/w/index.php\%3ftitle=User:Janskalicky\&action=edit\&redlink=1}{Janskalicky}\\
2& \myhref{https://en.wikibooks.org/w/index.php\%3ftitle=User:Jason_barrington~enwikibooks\&action=edit\&redlink=1}{Jason barrington\~{}enwikibooks}\\
1& \myhref{https://en.wikibooks.org/w/index.php\%3ftitle=User:Jasu\&action=edit\&redlink=1}{Jasu}\\
1& \myhref{https://en.wikibooks.org/wiki/User:Jayk~enwikibooks}{Jayk\~{}enwikibooks}\\
1& \myhref{https://en.wikibooks.org/w/index.php\%3ftitle=User:Jbsnyder\&action=edit\&redlink=1}{Jbsnyder}\\
1& \myhref{https://en.wikibooks.org/wiki/User:Jdgilbey}{Jdgilbey}\\
1& \myhref{https://en.wikibooks.org/w/index.php\%3ftitle=User:Je_ne_d\%25C3\%25A9tiens_pas_la_v\%25C3\%25A9rit\%25C3\%25A9_universelle\&action=edit\&redlink=1}{Je ne détiens pas la vérité universelle}\\
1& \myhref{https://en.wikibooks.org/wiki/User:Jeff_G.}{Jeff G.}\\
2& \myhref{https://en.wikibooks.org/w/index.php\%3ftitle=User:JenVan\&action=edit\&redlink=1}{JenVan}\\
1& \myhref{https://en.wikibooks.org/w/index.php\%3ftitle=User:Jer789\&action=edit\&redlink=1}{Jer789}\\
2& \myhref{https://en.wikibooks.org/w/index.php\%3ftitle=User:Jerome.dequeker\&action=edit\&redlink=1}{Jerome.dequeker}\\
1& \myhref{https://en.wikibooks.org/w/index.php\%3ftitle=User:Jessevanassen\&action=edit\&redlink=1}{Jessevanassen}\\
3& \myhref{https://en.wikibooks.org/wiki/User:Jevon}{Jevon}\\
5& \myhref{https://en.wikibooks.org/w/index.php\%3ftitle=User:Jflycn\&action=edit\&redlink=1}{Jflycn}\\
2& \myhref{https://en.wikibooks.org/wiki/User:Jguk}{Jguk}\\
1& \myhref{https://en.wikibooks.org/wiki/User:Jianhui67}{Jianhui67}\\
51& \myhref{https://en.wikibooks.org/wiki/User:Jimbotyson}{Jimbotyson}\\
1& \myhref{https://en.wikibooks.org/w/index.php\%3ftitle=User:Jimmaykeepsitreal\&action=edit\&redlink=1}{Jimmaykeepsitreal}\\
1& \myhref{https://en.wikibooks.org/w/index.php\%3ftitle=User:Jld\&action=edit\&redlink=1}{Jld}\\
1& \myhref{https://en.wikibooks.org/w/index.php\%3ftitle=User:Jlrn\&action=edit\&redlink=1}{Jlrn}\\
3& \myhref{https://en.wikibooks.org/wiki/User:Jluttine}{Jluttine}\\
1& \myhref{https://en.wikibooks.org/w/index.php\%3ftitle=User:Jmahler1\&action=edit\&redlink=1}{Jmahler1}\\
1& \myhref{https://en.wikibooks.org/w/index.php\%3ftitle=User:Jmcdon10\&action=edit\&redlink=1}{Jmcdon10}\\
1& \myhref{https://en.wikibooks.org/wiki/User:Joaospam}{Joaospam}\\
1& \myhref{https://en.wikibooks.org/wiki/User:Jodi.a.schneider}{Jodi.a.schneider}\\
3& \myhref{https://en.wikibooks.org/wiki/User:Joe_Schmedley}{Joe Schmedley}\\
1& \myhref{https://en.wikibooks.org/w/index.php\%3ftitle=User:Joeyboi\&action=edit\&redlink=1}{Joeyboi}\\
36& \myhref{https://en.wikibooks.org/wiki/User:Johannes_Bo}{Johannes Bo}\\
1& \myhref{https://en.wikibooks.org/w/index.php\%3ftitle=User:John1923\&action=edit\&redlink=1}{John1923}\\
14& \myhref{https://en.wikibooks.org/wiki/User:Jomegat}{Jomegat}\\
30& \myhref{https://en.wikibooks.org/wiki/User:Jonathan_Webley}{Jonathan Webley}\\
2& \myhref{https://en.wikibooks.org/wiki/User:JonnyJD}{JonnyJD}\\
1& \myhref{https://en.wikibooks.org/wiki/User:Jotomicron}{Jotomicron}\\
1& \myhref{https://en.wikibooks.org/w/index.php\%3ftitle=User:Jpoosterhuis\&action=edit\&redlink=1}{Jpoosterhuis}\\
1& \myhref{https://en.wikibooks.org/w/index.php\%3ftitle=User:Jraregris\&action=edit\&redlink=1}{Jraregris}\\
1& \myhref{https://en.wikibooks.org/w/index.php\%3ftitle=User:Jstein\&action=edit\&redlink=1}{Jstein}\\
26& \myhref{https://en.wikibooks.org/w/index.php\%3ftitle=User:Jtwdog~enwikibooks\&action=edit\&redlink=1}{Jtwdog\~{}enwikibooks}\\
2& \myhref{https://en.wikibooks.org/w/index.php\%3ftitle=User:Juliabackhausen\&action=edit\&redlink=1}{Juliabackhausen}\\
5& \myhref{https://en.wikibooks.org/wiki/User:Juliusross}{Juliusross}\\
1& \myhref{https://en.wikibooks.org/w/index.php\%3ftitle=User:Justin_W_Smith\&action=edit\&redlink=1}{Justin W Smith}\\
1& \myhref{https://en.wikibooks.org/wiki/User:Jwchong}{Jwchong}\\
1& \myhref{https://en.wikibooks.org/w/index.php\%3ftitle=User:K.Nevelsteen\&action=edit\&redlink=1}{K.Nevelsteen}\\
1& \myhref{https://en.wikibooks.org/w/index.php\%3ftitle=User:Kamarain\&action=edit\&redlink=1}{Kamarain}\\
1& \myhref{https://en.wikibooks.org/w/index.php\%3ftitle=User:Karategeek6\&action=edit\&redlink=1}{Karategeek6}\\
2& \myhref{https://en.wikibooks.org/w/index.php\%3ftitle=User:Karcih\&action=edit\&redlink=1}{Karcih}\\
5& \myhref{https://en.wikibooks.org/w/index.php\%3ftitle=User:Karlberry\&action=edit\&redlink=1}{Karlberry}\\
2& \myhref{https://en.wikibooks.org/w/index.php\%3ftitle=User:Karper\&action=edit\&redlink=1}{Karper}\\
1& \myhref{https://en.wikibooks.org/w/index.php\%3ftitle=User:Karthicknainar~enwikibooks\&action=edit\&redlink=1}{Karthicknainar\~{}enwikibooks}\\
1& \myhref{https://en.wikibooks.org/wiki/User:Kayau}{Kayau}\\
7& \myhref{https://en.wikibooks.org/wiki/User:Kazkaskazkasako}{Kazkaskazkasako}\\
1& \myhref{https://en.wikibooks.org/w/index.php\%3ftitle=User:Kcho\&action=edit\&redlink=1}{Kcho}\\
1& \myhref{https://en.wikibooks.org/w/index.php\%3ftitle=User:Kdonavin\&action=edit\&redlink=1}{Kdonavin}\\
4& \myhref{https://en.wikibooks.org/w/index.php\%3ftitle=User:Kejia\&action=edit\&redlink=1}{Kejia}\\
1& \myhref{https://en.wikibooks.org/wiki/User:Kenyon}{Kenyon}\\
7& \myhref{https://en.wikibooks.org/w/index.php\%3ftitle=User:Keplerspeed\&action=edit\&redlink=1}{Keplerspeed}\\
1& \myhref{https://en.wikibooks.org/wiki/User:Kernigh}{Kernigh}\\
2& \myhref{https://en.wikibooks.org/w/index.php\%3ftitle=User:Kevang\&action=edit\&redlink=1}{Kevang}\\
3& \myhref{https://en.wikibooks.org/w/index.php\%3ftitle=User:Kevin_Ryde\&action=edit\&redlink=1}{Kevin Ryde}\\
1& \myhref{https://en.wikibooks.org/w/index.php\%3ftitle=User:Kevinfiesta\&action=edit\&redlink=1}{Kevinfiesta}\\
12& \myhref{https://en.wikibooks.org/wiki/User:KlasN}{KlasN}\\
1& \myhref{https://en.wikibooks.org/w/index.php\%3ftitle=User:KlausFoehl\&action=edit\&redlink=1}{KlausFoehl}\\
1& \myhref{https://en.wikibooks.org/w/index.php\%3ftitle=User:Klusinyan\&action=edit\&redlink=1}{Klusinyan}\\
2& \myhref{https://en.wikibooks.org/wiki/User:Koavf}{Koavf}\\
1& \myhref{https://en.wikibooks.org/w/index.php\%3ftitle=User:Komputerwiz\&action=edit\&redlink=1}{Komputerwiz}\\
1& \myhref{https://en.wikibooks.org/w/index.php\%3ftitle=User:Konteki\&action=edit\&redlink=1}{Konteki}\\
1& \myhref{https://en.wikibooks.org/w/index.php\%3ftitle=User:Kop\&action=edit\&redlink=1}{Kop}\\
13& \myhref{https://en.wikibooks.org/w/index.php\%3ftitle=User:Kovianyo\&action=edit\&redlink=1}{Kovianyo}\\
2& \myhref{https://en.wikibooks.org/w/index.php\%3ftitle=User:Kpym\&action=edit\&redlink=1}{Kpym}\\
27& \myhref{https://en.wikibooks.org/wiki/User:Kri}{Kri}\\
1& \myhref{https://en.wikibooks.org/wiki/User:Krischik}{Krischik}\\
5& \myhref{https://en.wikibooks.org/wiki/User:Krishnachandranvn}{Krishnachandranvn}\\
1& \myhref{https://en.wikibooks.org/wiki/User:Krishnavedala}{Krishnavedala}\\
3& \myhref{https://en.wikibooks.org/wiki/User:Krisrose~enwikibooks}{Krisrose\~{}enwikibooks}\\
2& \myhref{https://en.wikibooks.org/wiki/User:Kroolik}{Kroolik}\\
4& \myhref{https://en.wikibooks.org/w/index.php\%3ftitle=User:Krst\&action=edit\&redlink=1}{Krst}\\
1& \myhref{https://en.wikibooks.org/w/index.php\%3ftitle=User:Kubieziel\&action=edit\&redlink=1}{Kubieziel}\\
1& \myhref{https://en.wikibooks.org/w/index.php\%3ftitle=User:Kuer.gee\&action=edit\&redlink=1}{Kuer.gee}\\
1& \myhref{https://en.wikibooks.org/w/index.php\%3ftitle=User:Kundor\&action=edit\&redlink=1}{Kundor}\\
6& \myhref{https://en.wikibooks.org/w/index.php\%3ftitle=User:Kurlovitsch\&action=edit\&redlink=1}{Kurlovitsch}\\
2& \myhref{https://en.wikibooks.org/w/index.php\%3ftitle=User:Kw_CUACS.TOPS\&action=edit\&redlink=1}{Kw CUACS.TOPS}\\
1& \myhref{https://en.wikibooks.org/w/index.php\%3ftitle=User:Kwetal\&action=edit\&redlink=1}{Kwetal}\\
1& \myhref{https://en.wikibooks.org/w/index.php\%3ftitle=User:Kwpolska\&action=edit\&redlink=1}{Kwpolska}\\
1& \myhref{https://en.wikibooks.org/w/index.php\%3ftitle=User:LQST\&action=edit\&redlink=1}{LQST}\\
4& \myhref{https://en.wikibooks.org/w/index.php\%3ftitle=User:LR~enwikibooks\&action=edit\&redlink=1}{LR\~{}enwikibooks}\\
1& \myhref{https://en.wikibooks.org/wiki/User:LaTeX~enwikibooks}{LaTeX\~{}enwikibooks}\\
1& \myhref{https://en.wikibooks.org/w/index.php\%3ftitle=User:Lancioni\&action=edit\&redlink=1}{Lancioni}\\
1& \myhref{https://en.wikibooks.org/w/index.php\%3ftitle=User:Lanoxx\&action=edit\&redlink=1}{Lanoxx}\\
2& \myhref{https://en.wikibooks.org/w/index.php\%3ftitle=User:Latexing\&action=edit\&redlink=1}{Latexing}\\
5& \myhref{https://en.wikibooks.org/wiki/User:Lavaka}{Lavaka}\\
1& \myhref{https://en.wikibooks.org/w/index.php\%3ftitle=User:Lbailey45\&action=edit\&redlink=1}{Lbailey45}\\
6& \myhref{https://en.wikibooks.org/w/index.php\%3ftitle=User:Leaderboard\&action=edit\&redlink=1}{Leaderboard}\\
6& \myhref{https://en.wikibooks.org/w/index.php\%3ftitle=User:Leal26\&action=edit\&redlink=1}{Leal26}\\
1& \myhref{https://en.wikibooks.org/wiki/User:Leyo}{Leyo}\\
1& \myhref{https://en.wikibooks.org/w/index.php\%3ftitle=User:Liiiii\&action=edit\&redlink=1}{Liiiii}\\
1& \myhref{https://en.wikibooks.org/w/index.php\%3ftitle=User:Limpato\&action=edit\&redlink=1}{Limpato}\\
4& \myhref{https://en.wikibooks.org/w/index.php\%3ftitle=User:Lindhe94\&action=edit\&redlink=1}{Lindhe94}\\
1& \myhref{https://en.wikibooks.org/w/index.php\%3ftitle=User:Lindhea\&action=edit\&redlink=1}{Lindhea}\\
1& \myhref{https://en.wikibooks.org/w/index.php\%3ftitle=User:LinuxChristian~enwikibooks\&action=edit\&redlink=1}{LinuxChristian\~{}enwikibooks}\\
1& \myhref{https://en.wikibooks.org/w/index.php\%3ftitle=User:Linzhongpeng\&action=edit\&redlink=1}{Linzhongpeng}\\
2& \myhref{https://en.wikibooks.org/w/index.php\%3ftitle=User:Listdata\&action=edit\&redlink=1}{Listdata}\\
1& \myhref{https://en.wikibooks.org/w/index.php\%3ftitle=User:Literaturgenerator\&action=edit\&redlink=1}{Literaturgenerator}\\
1& \myhref{https://en.wikibooks.org/wiki/User:LivingShadow}{LivingShadow}\\
2& \myhref{https://en.wikibooks.org/w/index.php\%3ftitle=User:Liwangyan\&action=edit\&redlink=1}{Liwangyan}\\
2& \myhref{https://en.wikibooks.org/wiki/User:LlamaAl}{LlamaAl}\\
1& \myhref{https://en.wikibooks.org/w/index.php\%3ftitle=User:Lnkbuildingservices4u\&action=edit\&redlink=1}{Lnkbuildingservices4u}\\
2& \myhref{https://en.wikibooks.org/w/index.php\%3ftitle=User:Lobaluna\&action=edit\&redlink=1}{Lobaluna}\\
3& \myhref{https://en.wikibooks.org/w/index.php\%3ftitle=User:Lotus_noir\&action=edit\&redlink=1}{Lotus noir}\\
1& \myhref{https://en.wikibooks.org/w/index.php\%3ftitle=User:Louabill\&action=edit\&redlink=1}{Louabill}\\
3& \myhref{https://en.wikibooks.org/w/index.php\%3ftitle=User:Louisix\&action=edit\&redlink=1}{Louisix}\\
1& \myhref{https://en.wikibooks.org/w/index.php\%3ftitle=User:Lovibond\&action=edit\&redlink=1}{Lovibond}\\
4& \myhref{https://en.wikibooks.org/w/index.php\%3ftitle=User:Lteu\&action=edit\&redlink=1}{Lteu}\\
4& \myhref{https://en.wikibooks.org/w/index.php\%3ftitle=User:Lucasreddinger\&action=edit\&redlink=1}{Lucasreddinger}\\
1& \myhref{https://en.wikibooks.org/wiki/User:MER-C}{MER-{}C}\\
1& \myhref{https://en.wikibooks.org/w/index.php\%3ftitle=User:MQ978\&action=edit\&redlink=1}{MQ978}\\
1& \myhref{https://en.wikibooks.org/w/index.php\%3ftitle=User:MaBoehm\&action=edit\&redlink=1}{MaBoehm}\\
3& \myhref{https://en.wikibooks.org/w/index.php\%3ftitle=User:Maartenweyn\&action=edit\&redlink=1}{Maartenweyn}\\
1& \myhref{https://en.wikibooks.org/wiki/User:Mabdul}{Mabdul}\\
1& \myhref{https://en.wikibooks.org/w/index.php\%3ftitle=User:Madskaddie\&action=edit\&redlink=1}{Madskaddie}\\
2& \myhref{https://en.wikibooks.org/w/index.php\%3ftitle=User:MagnusPI~enwikibooks\&action=edit\&redlink=1}{MagnusPI\~{}enwikibooks}\\
1& \myhref{https://en.wikibooks.org/w/index.php\%3ftitle=User:Mandriver\&action=edit\&redlink=1}{Mandriver}\\
1& \myhref{https://en.wikibooks.org/w/index.php\%3ftitle=User:MarSraM~enwikibooks\&action=edit\&redlink=1}{MarSraM\~{}enwikibooks}\\
1& \myhref{https://en.wikibooks.org/w/index.php\%3ftitle=User:Maratonda\&action=edit\&redlink=1}{Maratonda}\\
1& \myhref{https://en.wikibooks.org/wiki/User:Marcus_Cyron}{Marcus Cyron}\\
2& \myhref{https://en.wikibooks.org/w/index.php\%3ftitle=User:Mariafenrinha\&action=edit\&redlink=1}{Mariafenrinha}\\
4& \myhref{https://en.wikibooks.org/wiki/User:Marozols}{Marozols}\\
2& \myhref{https://en.wikibooks.org/w/index.php\%3ftitle=User:Marra\&action=edit\&redlink=1}{Marra}\\
1& \myhref{https://en.wikibooks.org/w/index.php\%3ftitle=User:Martin_scharrer\&action=edit\&redlink=1}{Martin scharrer}\\
1& \myhref{https://en.wikibooks.org/wiki/User:Martin_von_Wittich}{Martin von Wittich}\\
5& \myhref{https://en.wikibooks.org/w/index.php\%3ftitle=User:MartinSpacek\&action=edit\&redlink=1}{MartinSpacek}\\
1& \myhref{https://en.wikibooks.org/w/index.php\%3ftitle=User:Martinkunev\&action=edit\&redlink=1}{Martinkunev}\\
2& \myhref{https://en.wikibooks.org/w/index.php\%3ftitle=User:Maschen\&action=edit\&redlink=1}{Maschen}\\
2& \myhref{https://en.wikibooks.org/w/index.php\%3ftitle=User:Masterpiga\&action=edit\&redlink=1}{Masterpiga}\\
1& \myhref{https://en.wikibooks.org/w/index.php\%3ftitle=User:Matej.korvas\&action=edit\&redlink=1}{Matej.korvas}\\
1& \myhref{https://en.wikibooks.org/w/index.php\%3ftitle=User:Mateo.longo\&action=edit\&redlink=1}{Mateo.longo}\\
2& \myhref{https://en.wikibooks.org/w/index.php\%3ftitle=User:Mathieu_Perrin\&action=edit\&redlink=1}{Mathieu Perrin}\\
2& \myhref{https://en.wikibooks.org/wiki/User:Maths314}{Maths314}\\
1& \myhref{https://en.wikibooks.org/wiki/User:Matthias_M.}{Matthias M.}\\
1& \myhref{https://en.wikibooks.org/wiki/User:Mat\%25C4\%259Bj_Grabovsk\%25C3\%25BD}{Matěj Grabovský}\\
1& \myhref{https://en.wikibooks.org/w/index.php\%3ftitle=User:McSaks\&action=edit\&redlink=1}{McSaks}\\
1& \myhref{https://en.wikibooks.org/wiki/User:Mckay}{Mckay}\\
19& \myhref{https://en.wikibooks.org/wiki/User:Mcld}{Mcld}\\
1& \myhref{https://en.wikibooks.org/w/index.php\%3ftitle=User:Mdpacer\&action=edit\&redlink=1}{Mdpacer}\\
4& \myhref{https://en.wikibooks.org/wiki/User:Mecanismo}{Mecanismo}\\
1& \myhref{https://en.wikibooks.org/w/index.php\%3ftitle=User:Merciadriluca\&action=edit\&redlink=1}{Merciadriluca}\\
1& \myhref{https://en.wikibooks.org/w/index.php\%3ftitle=User:Metis_spawn\&action=edit\&redlink=1}{Metis spawn}\\
1& \myhref{https://en.wikibooks.org/w/index.php\%3ftitle=User:Mezzaluna\&action=edit\&redlink=1}{Mezzaluna}\\
1& \myhref{https://en.wikibooks.org/w/index.php\%3ftitle=User:MfR\&action=edit\&redlink=1}{MfR}\\
4& \myhref{https://en.wikibooks.org/w/index.php\%3ftitle=User:Mfwitten\&action=edit\&redlink=1}{Mfwitten}\\
2& \myhref{https://en.wikibooks.org/w/index.php\%3ftitle=User:Mgkrupa\&action=edit\&redlink=1}{Mgkrupa}\\
1& \myhref{https://en.wikibooks.org/w/index.php\%3ftitle=User:Mhartl\&action=edit\&redlink=1}{Mhartl}\\
4& \myhref{https://en.wikibooks.org/w/index.php\%3ftitle=User:Mhue\&action=edit\&redlink=1}{Mhue}\\
1& \myhref{https://en.wikibooks.org/w/index.php\%3ftitle=User:Michael_M_Hackett\&action=edit\&redlink=1}{Michael M Hackett}\\
1& \myhref{https://en.wikibooks.org/wiki/User:MichaelBillington}{MichaelBillington}\\
1& \myhref{https://en.wikibooks.org/w/index.php\%3ftitle=User:MichaelBueker\&action=edit\&redlink=1}{MichaelBueker}\\
2& \myhref{https://en.wikibooks.org/wiki/User:MichaelSchoenitzer}{MichaelSchoenitzer}\\
1& \myhref{https://en.wikibooks.org/w/index.php\%3ftitle=User:Migueldvb\&action=edit\&redlink=1}{Migueldvb}\\
3& \myhref{https://en.wikibooks.org/w/index.php\%3ftitle=User:Mihai_Capot\%25C4\%2583\&action=edit\&redlink=1}{Mihai Capotă}\\
1& \myhref{https://en.wikibooks.org/w/index.php\%3ftitle=User:Mijikenda\&action=edit\&redlink=1}{Mijikenda}\\
1& \myhref{https://en.wikibooks.org/wiki/User:Mike.lifeguard}{Mike.lifeguard}\\
3& \myhref{https://en.wikibooks.org/w/index.php\%3ftitle=User:Mikhail_Ryazanov\&action=edit\&redlink=1}{Mikhail Ryazanov}\\
2& \myhref{https://en.wikibooks.org/w/index.php\%3ftitle=User:Mimo~enwikibooks\&action=edit\&redlink=1}{Mimo\~{}enwikibooks}\\
1& \myhref{https://en.wikibooks.org/w/index.php\%3ftitle=User:MoMaT\&action=edit\&redlink=1}{MoMaT}\\
1& \myhref{https://en.wikibooks.org/wiki/User:Modest_Genius}{Modest Genius}\\
3& \myhref{https://en.wikibooks.org/w/index.php\%3ftitle=User:Morelight~enwikibooks\&action=edit\&redlink=1}{Morelight\~{}enwikibooks}\\
1& \myhref{https://en.wikibooks.org/wiki/User:Mouselb}{Mouselb}\\
1& \myhref{https://en.wikibooks.org/w/index.php\%3ftitle=User:Mpvharmelen\&action=edit\&redlink=1}{Mpvharmelen}\\
1& \myhref{https://en.wikibooks.org/wiki/User:Mr._Stradivarius}{Mr. Stradivarius}\\
1& \myhref{https://en.wikibooks.org/w/index.php\%3ftitle=User:Mrt_doulaty\&action=edit\&redlink=1}{Mrt doulaty}\\
1& \myhref{https://en.wikibooks.org/w/index.php\%3ftitle=User:Mrwil222\&action=edit\&redlink=1}{Mrwil222}\\
1& \myhref{https://en.wikibooks.org/wiki/User:Ms2ger}{Ms2ger}\\
83& \myhref{https://en.wikibooks.org/wiki/User:Mwtoews}{Mwtoews}\\
2& \myhref{https://en.wikibooks.org/w/index.php\%3ftitle=User:Naught101\&action=edit\&redlink=1}{Naught101}\\
2& \myhref{https://en.wikibooks.org/wiki/User:NavarroJ}{NavarroJ}\\
3& \myhref{https://en.wikibooks.org/w/index.php\%3ftitle=User:Nbrouard\&action=edit\&redlink=1}{Nbrouard}\\
1& \myhref{https://en.wikibooks.org/w/index.php\%3ftitle=User:Neatnate\&action=edit\&redlink=1}{Neatnate}\\
18& \myhref{https://en.wikibooks.org/w/index.php\%3ftitle=User:Neet\&action=edit\&redlink=1}{Neet}\\
1& \myhref{https://en.wikibooks.org/wiki/User:Negative24}{Negative24}\\
1& \myhref{https://en.wikibooks.org/w/index.php\%3ftitle=User:Nemoniac\&action=edit\&redlink=1}{Nemoniac}\\
1& \myhref{https://en.wikibooks.org/w/index.php\%3ftitle=User:Nemti\&action=edit\&redlink=1}{Nemti}\\
2& \myhref{https://en.wikibooks.org/wiki/User:Neoptolemus}{Neoptolemus}\\
13& \myhref{https://en.wikibooks.org/w/index.php\%3ftitle=User:Neoriddle\&action=edit\&redlink=1}{Neoriddle}\\
3& \myhref{https://en.wikibooks.org/w/index.php\%3ftitle=User:Netheril96\&action=edit\&redlink=1}{Netheril96}\\
3& \myhref{https://en.wikibooks.org/wiki/User:Ngoclong19}{Ngoclong19}\\
1& \myhref{https://en.wikibooks.org/w/index.php\%3ftitle=User:Nicfalco\&action=edit\&redlink=1}{Nicfalco}\\
3& \myhref{https://en.wikibooks.org/w/index.php\%3ftitle=User:Nicolas_Perrault_III\&action=edit\&redlink=1}{Nicolas Perrault III}\\
1& \myhref{https://en.wikibooks.org/w/index.php\%3ftitle=User:Nicolasbock\&action=edit\&redlink=1}{Nicolasbock}\\
1& \myhref{https://en.wikibooks.org/w/index.php\%3ftitle=User:Niel.Bowerman\&action=edit\&redlink=1}{Niel.Bowerman}\\
1& \myhref{https://en.wikibooks.org/w/index.php\%3ftitle=User:Nigels~enwikibooks\&action=edit\&redlink=1}{Nigels\~{}enwikibooks}\\
7& \myhref{https://en.wikibooks.org/w/index.php\%3ftitle=User:Nixphoeni\&action=edit\&redlink=1}{Nixphoeni}\\
1& \myhref{https://en.wikibooks.org/w/index.php\%3ftitle=User:Niy\&action=edit\&redlink=1}{Niy}\\
1& \myhref{https://en.wikibooks.org/w/index.php\%3ftitle=User:Nkour\&action=edit\&redlink=1}{Nkour}\\
8& \myhref{https://en.wikibooks.org/wiki/User:Nobelium}{Nobelium}\\
1& \myhref{https://en.wikibooks.org/w/index.php\%3ftitle=User:Norbert.beckers\&action=edit\&redlink=1}{Norbert.beckers}\\
1& \myhref{https://en.wikibooks.org/w/index.php\%3ftitle=User:Nothing1212\&action=edit\&redlink=1}{Nothing1212}\\
1& \myhref{https://en.wikibooks.org/w/index.php\%3ftitle=User:NqpZ\&action=edit\&redlink=1}{NqpZ}\\
6& \myhref{https://en.wikibooks.org/w/index.php\%3ftitle=User:Nsda\&action=edit\&redlink=1}{Nsda}\\
4& \myhref{https://en.wikibooks.org/w/index.php\%3ftitle=User:Nsuwan\&action=edit\&redlink=1}{Nsuwan}\\
4& \myhref{https://en.wikibooks.org/w/index.php\%3ftitle=User:Ntypanski\&action=edit\&redlink=1}{Ntypanski}\\
1& \myhref{https://en.wikibooks.org/wiki/User:Nux}{Nux}\\
1& \myhref{https://en.wikibooks.org/w/index.php\%3ftitle=User:Obelyaev\&action=edit\&redlink=1}{Obelyaev}\\
1& \myhref{https://en.wikibooks.org/w/index.php\%3ftitle=User:Oderbolz\&action=edit\&redlink=1}{Oderbolz}\\
2& \myhref{https://en.wikibooks.org/w/index.php\%3ftitle=User:Ojan\&action=edit\&redlink=1}{Ojan}\\
1& \myhref{https://en.wikibooks.org/w/index.php\%3ftitle=User:Olaf3142\&action=edit\&redlink=1}{Olaf3142}\\
2& \myhref{https://en.wikibooks.org/w/index.php\%3ftitle=User:Olesh\&action=edit\&redlink=1}{Olesh}\\
1& \myhref{https://en.wikibooks.org/w/index.php\%3ftitle=User:Olivier.descout\&action=edit\&redlink=1}{Olivier.descout}\\
2& \myhref{https://en.wikibooks.org/w/index.php\%3ftitle=User:OlivierMehani\&action=edit\&redlink=1}{OlivierMehani}\\
4& \myhref{https://en.wikibooks.org/w/index.php\%3ftitle=User:Ollydbg\&action=edit\&redlink=1}{Ollydbg}\\
21& \myhref{https://en.wikibooks.org/wiki/User:Orderud}{Orderud}\\
1& \myhref{https://en.wikibooks.org/w/index.php\%3ftitle=User:Otec_Stochastik\&action=edit\&redlink=1}{Otec Stochastik}\\
29& \myhref{https://en.wikibooks.org/wiki/User:PAC}{PAC}\\
32& \myhref{https://en.wikibooks.org/wiki/User:PAC2}{PAC2}\\
1& \myhref{https://en.wikibooks.org/wiki/User:Pamputt}{Pamputt}\\
1& \myhref{https://en.wikibooks.org/w/index.php\%3ftitle=User:Pandora85\&action=edit\&redlink=1}{Pandora85}\\
6& \myhref{https://en.wikibooks.org/wiki/User:Panic2k4}{Panic2k4}\\
1& \myhref{https://en.wikibooks.org/w/index.php\%3ftitle=User:Panoramedia\&action=edit\&redlink=1}{Panoramedia}\\
1& \myhref{https://en.wikibooks.org/w/index.php\%3ftitle=User:Pater_Christophorus\&action=edit\&redlink=1}{Pater Christophorus}\\
3& \myhref{https://en.wikibooks.org/w/index.php\%3ftitle=User:PatrickDevlin21\&action=edit\&redlink=1}{PatrickDevlin21}\\
1& \myhref{https://en.wikibooks.org/w/index.php\%3ftitle=User:PatrickGalyon\&action=edit\&redlink=1}{PatrickGalyon}\\
1& \myhref{https://en.wikibooks.org/w/index.php\%3ftitle=User:Patuck\&action=edit\&redlink=1}{Patuck}\\
2& \myhref{https://en.wikibooks.org/w/index.php\%3ftitle=User:Paul2520\&action=edit\&redlink=1}{Paul2520}\\
3& \myhref{https://en.wikibooks.org/w/index.php\%3ftitle=User:Paulgush\&action=edit\&redlink=1}{Paulgush}\\
3& \myhref{https://en.wikibooks.org/wiki/User:Paxinum}{Paxinum}\\
2& \myhref{https://en.wikibooks.org/w/index.php\%3ftitle=User:Pdelong\&action=edit\&redlink=1}{Pdelong}\\
1& \myhref{https://en.wikibooks.org/w/index.php\%3ftitle=User:PeterAllen\&action=edit\&redlink=1}{PeterAllen}\\
2& \myhref{https://en.wikibooks.org/w/index.php\%3ftitle=User:Petter_Strandmark\&action=edit\&redlink=1}{Petter Strandmark}\\
2& \myhref{https://en.wikibooks.org/w/index.php\%3ftitle=User:PhilJohnG\&action=edit\&redlink=1}{PhilJohnG}\\
120& \myhref{https://en.wikibooks.org/wiki/User:Pi_zero}{Pi zero}\\
1& \myhref{https://en.wikibooks.org/wiki/User:PiRSquared17}{PiRSquared17}\\
11& \myhref{https://en.wikibooks.org/wiki/User:Piksi}{Piksi}\\
1& \myhref{https://en.wikibooks.org/w/index.php\%3ftitle=User:Pilosa.Folivora\&action=edit\&redlink=1}{Pilosa.Folivora}\\
1& \myhref{https://en.wikibooks.org/w/index.php\%3ftitle=User:Pirround\&action=edit\&redlink=1}{Pirround}\\
13& \myhref{https://en.wikibooks.org/w/index.php\%3ftitle=User:Pmillerrhodes\&action=edit\&redlink=1}{Pmillerrhodes}\\
1& \myhref{https://en.wikibooks.org/wiki/User:Pmlineditor}{Pmlineditor}\\
1& \myhref{https://en.wikibooks.org/w/index.php\%3ftitle=User:Polytropos_Technikos\&action=edit\&redlink=1}{Polytropos Technikos}\\
1& \myhref{https://en.wikibooks.org/w/index.php\%3ftitle=User:Ppadmapriya\&action=edit\&redlink=1}{Ppadmapriya}\\
1& \myhref{https://en.wikibooks.org/w/index.php\%3ftitle=User:Prawojazdy\&action=edit\&redlink=1}{Prawojazdy}\\
1& \myhref{https://en.wikibooks.org/w/index.php\%3ftitle=User:Prispartlow\&action=edit\&redlink=1}{Prispartlow}\\
7& \myhref{https://en.wikibooks.org/w/index.php\%3ftitle=User:Pstar\&action=edit\&redlink=1}{Pstar}\\
1& \myhref{https://en.wikibooks.org/wiki/User:PsyberS}{PsyberS}\\
3& \myhref{https://en.wikibooks.org/wiki/User:QUBot}{QUBot}\\
1& \myhref{https://en.wikibooks.org/w/index.php\%3ftitle=User:Qeny\&action=edit\&redlink=1}{Qeny}\\
1& \myhref{https://en.wikibooks.org/w/index.php\%3ftitle=User:QuantumEleven\&action=edit\&redlink=1}{QuantumEleven}\\
1& \myhref{https://en.wikibooks.org/w/index.php\%3ftitle=User:Quaristice\&action=edit\&redlink=1}{Quaristice}\\
29& \myhref{https://en.wikibooks.org/wiki/User:QuiteUnusual}{QuiteUnusual}\\
5& \myhref{https://en.wikibooks.org/wiki/User:Qwertyus}{Qwertyus}\\
1& \myhref{https://en.wikibooks.org/w/index.php\%3ftitle=User:Qzxpqbp\&action=edit\&redlink=1}{Qzxpqbp}\\
1& \myhref{https://en.wikibooks.org/wiki/User:RP87}{RP87}\\
1& \myhref{https://en.wikibooks.org/w/index.php\%3ftitle=User:RTPK\&action=edit\&redlink=1}{RTPK}\\
1& \myhref{https://en.wikibooks.org/w/index.php\%3ftitle=User:Rafaelgr\&action=edit\&redlink=1}{Rafaelgr}\\
1& \myhref{https://en.wikibooks.org/w/index.php\%3ftitle=User:Rafopar\&action=edit\&redlink=1}{Rafopar}\\
1& \myhref{https://en.wikibooks.org/wiki/User:RainCity471}{RainCity471}\\
3& \myhref{https://en.wikibooks.org/w/index.php\%3ftitle=User:Rajkiran_g\&action=edit\&redlink=1}{Rajkiran g}\\
3& \myhref{https://en.wikibooks.org/wiki/User:Ramac}{Ramac}\\
2& \myhref{https://en.wikibooks.org/w/index.php\%3ftitle=User:RasmusWriedtLarsen\&action=edit\&redlink=1}{RasmusWriedtLarsen}\\
1& \myhref{https://en.wikibooks.org/wiki/User:Raylu}{Raylu}\\
1& \myhref{https://en.wikibooks.org/wiki/User:RaymondSutanto}{RaymondSutanto}\\
2& \myhref{https://en.wikibooks.org/wiki/User:Razr_Nation}{Razr Nation}\\
3& \myhref{https://en.wikibooks.org/w/index.php\%3ftitle=User:Rbonvall\&action=edit\&redlink=1}{Rbonvall}\\
1& \myhref{https://en.wikibooks.org/w/index.php\%3ftitle=User:Rdg_nz\&action=edit\&redlink=1}{Rdg nz}\\
17& \myhref{https://en.wikibooks.org/w/index.php\%3ftitle=User:RealSebix\&action=edit\&redlink=1}{RealSebix}\\
10& \myhref{https://en.wikibooks.org/wiki/User:Recent_Runes}{Recent Runes}\\
6& \myhref{https://en.wikibooks.org/wiki/User:Reddraggone9}{Reddraggone9}\\
1& \myhref{https://en.wikibooks.org/wiki/User:Redirect_fixer}{Redirect fixer}\\
4& \myhref{https://en.wikibooks.org/w/index.php\%3ftitle=User:Rehoot\&action=edit\&redlink=1}{Rehoot}\\
1& \myhref{https://en.wikibooks.org/w/index.php\%3ftitle=User:Reim\&action=edit\&redlink=1}{Reim}\\
2& \myhref{https://en.wikibooks.org/w/index.php\%3ftitle=User:Remsirems\&action=edit\&redlink=1}{Remsirems}\\
1& \myhref{https://en.wikibooks.org/wiki/User:Reyk}{Reyk}\\
1& \myhref{https://en.wikibooks.org/w/index.php\%3ftitle=User:Rhalah\&action=edit\&redlink=1}{Rhalah}\\
1& \myhref{https://en.wikibooks.org/wiki/User:Ricordisamoa}{Ricordisamoa}\\
1& \myhref{https://en.wikibooks.org/wiki/User:Risk}{Risk}\\
1& \myhref{https://en.wikibooks.org/wiki/User:Rnddim}{Rnddim}\\
1& \myhref{https://en.wikibooks.org/w/index.php\%3ftitle=User:Roarbakk\&action=edit\&redlink=1}{Roarbakk}\\
84& \myhref{https://en.wikibooks.org/wiki/User:Robbiemorrison}{Robbiemorrison}\\
1& \myhref{https://en.wikibooks.org/wiki/User:Robert_Borkowski}{Robert Borkowski}\\
4& \myhref{https://en.wikibooks.org/wiki/User:Robert_Horning}{Robert Horning}\\
3& \myhref{https://en.wikibooks.org/wiki/User:Robin~enwikibooks}{Robin\~{}enwikibooks}\\
1& \myhref{https://en.wikibooks.org/w/index.php\%3ftitle=User:Rogal~enwikibooks\&action=edit\&redlink=1}{Rogal\~{}enwikibooks}\\
1& \myhref{https://en.wikibooks.org/w/index.php\%3ftitle=User:Rogerbrent\&action=edit\&redlink=1}{Rogerbrent}\\
4& \myhref{https://en.wikibooks.org/w/index.php\%3ftitle=User:Rondenaranja~enwikibooks\&action=edit\&redlink=1}{Rondenaranja\~{}enwikibooks}\\
2& \myhref{https://en.wikibooks.org/w/index.php\%3ftitle=User:Rossdub\&action=edit\&redlink=1}{Rossdub}\\
2& \myhref{https://en.wikibooks.org/w/index.php\%3ftitle=User:Rotlink\&action=edit\&redlink=1}{Rotlink}\\
1& \myhref{https://en.wikibooks.org/w/index.php\%3ftitle=User:Royote\&action=edit\&redlink=1}{Royote}\\
1& \myhref{https://en.wikibooks.org/w/index.php\%3ftitle=User:Rror\&action=edit\&redlink=1}{Rror}\\
1& \myhref{https://en.wikibooks.org/w/index.php\%3ftitle=User:RubensMatos\&action=edit\&redlink=1}{RubensMatos}\\
1& \myhref{https://en.wikibooks.org/w/index.php\%3ftitle=User:Russell208\&action=edit\&redlink=1}{Russell208}\\
4& \myhref{https://en.wikibooks.org/w/index.php\%3ftitle=User:Sabalka~enwikibooks\&action=edit\&redlink=1}{Sabalka\~{}enwikibooks}\\
1& \myhref{https://en.wikibooks.org/w/index.php\%3ftitle=User:Saehrimnir\&action=edit\&redlink=1}{Saehrimnir}\\
2& \myhref{https://en.wikibooks.org/w/index.php\%3ftitle=User:Saippuakauppias\&action=edit\&redlink=1}{Saippuakauppias}\\
33& \myhref{https://en.wikibooks.org/w/index.php\%3ftitle=User:SamuelLB\&action=edit\&redlink=1}{SamuelLB}\\
2& \myhref{https://en.wikibooks.org/wiki/User:Sandbergja}{Sandbergja}\\
4& \myhref{https://en.wikibooks.org/wiki/User:Sanderd17}{Sanderd17}\\
1& \myhref{https://en.wikibooks.org/w/index.php\%3ftitle=User:Sandman10000\&action=edit\&redlink=1}{Sandman10000}\\
1& \myhref{https://en.wikibooks.org/wiki/User:Sandrobt}{Sandrobt}\\
13& \myhref{https://en.wikibooks.org/w/index.php\%3ftitle=User:Sargas~enwikibooks\&action=edit\&redlink=1}{Sargas\~{}enwikibooks}\\
3& \myhref{https://en.wikibooks.org/w/index.php\%3ftitle=User:Savicivas\&action=edit\&redlink=1}{Savicivas}\\
1& \myhref{https://en.wikibooks.org/w/index.php\%3ftitle=User:Sbeyer\&action=edit\&redlink=1}{Sbeyer}\\
2& \myhref{https://en.wikibooks.org/w/index.php\%3ftitle=User:Schaber\&action=edit\&redlink=1}{Schaber}\\
1& \myhref{https://en.wikibooks.org/wiki/User:SciYann}{SciYann}\\
7& \myhref{https://en.wikibooks.org/w/index.php\%3ftitle=User:Scientific29\&action=edit\&redlink=1}{Scientific29}\\
1& \myhref{https://en.wikibooks.org/w/index.php\%3ftitle=User:Scorwin\&action=edit\&redlink=1}{Scorwin}\\
1& \myhref{https://en.wikibooks.org/wiki/User:Scruss}{Scruss}\\
6& \myhref{https://en.wikibooks.org/w/index.php\%3ftitle=User:Selfworm\&action=edit\&redlink=1}{Selfworm}\\
2& \myhref{https://en.wikibooks.org/w/index.php\%3ftitle=User:Semperos\&action=edit\&redlink=1}{Semperos}\\
1& \myhref{https://en.wikibooks.org/w/index.php\%3ftitle=User:Sgenier~enwikibooks\&action=edit\&redlink=1}{Sgenier\~{}enwikibooks}\\
1& \myhref{https://en.wikibooks.org/w/index.php\%3ftitle=User:Shahbaz_Youssefi\&action=edit\&redlink=1}{Shahbaz Youssefi}\\
2& \myhref{https://en.wikibooks.org/w/index.php\%3ftitle=User:Sheep0x\&action=edit\&redlink=1}{Sheep0x}\\
4& \myhref{https://en.wikibooks.org/w/index.php\%3ftitle=User:Silca678\&action=edit\&redlink=1}{Silca678}\\
1& \myhref{https://en.wikibooks.org/w/index.php\%3ftitle=User:Silverpie\&action=edit\&redlink=1}{Silverpie}\\
1& \myhref{https://en.wikibooks.org/wiki/User:Simeon}{Simeon}\\
2& \myhref{https://en.wikibooks.org/w/index.php\%3ftitle=User:Simonjtyler\&action=edit\&redlink=1}{Simonjtyler}\\
1& \myhref{https://en.wikibooks.org/w/index.php\%3ftitle=User:Simplelatex\&action=edit\&redlink=1}{Simplelatex}\\
1& \myhref{https://en.wikibooks.org/w/index.php\%3ftitle=User:SiriusB\&action=edit\&redlink=1}{SiriusB}\\
2& \myhref{https://en.wikibooks.org/wiki/User:Sjlegg}{Sjlegg}\\
1& \myhref{https://en.wikibooks.org/w/index.php\%3ftitle=User:Skarakoleva\&action=edit\&redlink=1}{Skarakoleva}\\
1& \myhref{https://en.wikibooks.org/w/index.php\%3ftitle=User:Skim\&action=edit\&redlink=1}{Skim}\\
1& \myhref{https://en.wikibooks.org/w/index.php\%3ftitle=User:Skou~enwikibooks\&action=edit\&redlink=1}{Skou\~{}enwikibooks}\\
1& \myhref{https://en.wikibooks.org/w/index.php\%3ftitle=User:Smobbl_Bobbl\&action=edit\&redlink=1}{Smobbl Bobbl}\\
5& \myhref{https://en.wikibooks.org/w/index.php\%3ftitle=User:Snaxe920~enwikibooks\&action=edit\&redlink=1}{Snaxe920\~{}enwikibooks}\\
1& \myhref{https://en.wikibooks.org/w/index.php\%3ftitle=User:Snoopy67\&action=edit\&redlink=1}{Snoopy67}\\
4& \myhref{https://en.wikibooks.org/w/index.php\%3ftitle=User:Sobjornstad\&action=edit\&redlink=1}{Sobjornstad}\\
1& \myhref{https://en.wikibooks.org/w/index.php\%3ftitle=User:Solid_Frog\&action=edit\&redlink=1}{Solid Frog}\\
1& \myhref{https://en.wikibooks.org/w/index.php\%3ftitle=User:Sonic_the_goliath\&action=edit\&redlink=1}{Sonic the goliath}\\
4& \myhref{https://en.wikibooks.org/w/index.php\%3ftitle=User:Spag85~enwikibooks\&action=edit\&redlink=1}{Spag85\~{}enwikibooks}\\
11& \myhref{https://en.wikibooks.org/w/index.php\%3ftitle=User:Spelemann~enwikibooks\&action=edit\&redlink=1}{Spelemann\~{}enwikibooks}\\
1& \myhref{https://en.wikibooks.org/wiki/User:Speravir}{Speravir}\\
4& \myhref{https://en.wikibooks.org/wiki/User:Spirosdenaxas}{Spirosdenaxas}\\
1& \myhref{https://en.wikibooks.org/w/index.php\%3ftitle=User:Spook~enwikibooks\&action=edit\&redlink=1}{Spook\~{}enwikibooks}\\
1& \myhref{https://en.wikibooks.org/w/index.php\%3ftitle=User:Springthyme\&action=edit\&redlink=1}{Springthyme}\\
2& \myhref{https://en.wikibooks.org/w/index.php\%3ftitle=User:Squigish\&action=edit\&redlink=1}{Squigish}\\
1& \myhref{https://en.wikibooks.org/wiki/User:Staticshakedown}{Staticshakedown}\\
4& \myhref{https://en.wikibooks.org/w/index.php\%3ftitle=User:Steelangel\&action=edit\&redlink=1}{Steelangel}\\
2& \myhref{https://en.wikibooks.org/w/index.php\%3ftitle=User:Stefan.qn~enwikibooks\&action=edit\&redlink=1}{Stefan.qn\~{}enwikibooks}\\
1& \myhref{https://en.wikibooks.org/w/index.php\%3ftitle=User:Stefantauner\&action=edit\&redlink=1}{Stefantauner}\\
1& \myhref{https://en.wikibooks.org/w/index.php\%3ftitle=User:Steindani\&action=edit\&redlink=1}{Steindani}\\
3& \myhref{https://en.wikibooks.org/wiki/User:Stephan_Schneider}{Stephan Schneider}\\
1& \myhref{https://en.wikibooks.org/w/index.php\%3ftitle=User:SteveM82\&action=edit\&redlink=1}{SteveM82}\\
1& \myhref{https://en.wikibooks.org/w/index.php\%3ftitle=User:StevenJohnston\&action=edit\&redlink=1}{StevenJohnston}\\
7& \myhref{https://en.wikibooks.org/w/index.php\%3ftitle=User:Stoettner\&action=edit\&redlink=1}{Stoettner}\\
2& \myhref{https://en.wikibooks.org/w/index.php\%3ftitle=User:Strpeter\&action=edit\&redlink=1}{Strpeter}\\
1& \myhref{https://en.wikibooks.org/w/index.php\%3ftitle=User:Stuples\&action=edit\&redlink=1}{Stuples}\\
1& \myhref{https://en.wikibooks.org/w/index.php\%3ftitle=User:Sulhan\&action=edit\&redlink=1}{Sulhan}\\
2& \myhref{https://en.wikibooks.org/w/index.php\%3ftitle=User:Svick\&action=edit\&redlink=1}{Svick}\\
1& \myhref{https://en.wikibooks.org/wiki/User:Swift}{Swift}\\
6& \myhref{https://en.wikibooks.org/wiki/User:SynConlanger}{SynConlanger}\\
1& \myhref{https://en.wikibooks.org/w/index.php\%3ftitle=User:Syockit\&action=edit\&redlink=1}{Syockit}\\
2& \myhref{https://en.wikibooks.org/wiki/User:Syum90}{Syum90}\\
1& \myhref{https://en.wikibooks.org/w/index.php\%3ftitle=User:Szellmann\&action=edit\&redlink=1}{Szellmann}\\
1& \myhref{https://en.wikibooks.org/wiki/User:TWiStErRob}{TWiStErRob}\\
1& \myhref{https://en.wikibooks.org/w/index.php\%3ftitle=User:Tanzaho\&action=edit\&redlink=1}{Tanzaho}\\
1& \myhref{https://en.wikibooks.org/w/index.php\%3ftitle=User:Tau_Lambda\&action=edit\&redlink=1}{Tau Lambda}\\
1& \myhref{https://en.wikibooks.org/wiki/User:Tauriel-1}{Tauriel-{}1}\\
1& \myhref{https://en.wikibooks.org/wiki/User:Taweetham}{Taweetham}\\
3& \myhref{https://en.wikibooks.org/w/index.php\%3ftitle=User:Tazquebec\&action=edit\&redlink=1}{Tazquebec}\\
1& \myhref{https://en.wikibooks.org/wiki/User:Tdomhan}{Tdomhan}\\
1& \myhref{https://en.wikibooks.org/wiki/User:Teles}{Teles}\\
1& \myhref{https://en.wikibooks.org/w/index.php\%3ftitle=User:Tentotwo\&action=edit\&redlink=1}{Tentotwo}\\
1& \myhref{https://en.wikibooks.org/w/index.php\%3ftitle=User:Tgwizard\&action=edit\&redlink=1}{Tgwizard}\\
1& \myhref{https://en.wikibooks.org/w/index.php\%3ftitle=User:TheAnarcat\&action=edit\&redlink=1}{TheAnarcat}\\
1& \myhref{https://en.wikibooks.org/w/index.php\%3ftitle=User:Theemathas\&action=edit\&redlink=1}{Theemathas}\\
5& \myhref{https://en.wikibooks.org/w/index.php\%3ftitle=User:Thefrankinator\&action=edit\&redlink=1}{Thefrankinator}\\
21& \myhref{https://en.wikibooks.org/wiki/User:Thenub314}{Thenub314}\\
2& \myhref{https://en.wikibooks.org/w/index.php\%3ftitle=User:Thietkeweb~enwikibooks\&action=edit\&redlink=1}{Thietkeweb\~{}enwikibooks}\\
1& \myhref{https://en.wikibooks.org/wiki/User:This,_that_and_the_other}{This, that and the other}\\
1& \myhref{https://en.wikibooks.org/w/index.php\%3ftitle=User:Tia_s\%25C3\%25A1ng_m\%25E1\%25BA\%25B7t_tr\%25E1\%25BB\%259Di\&action=edit\&redlink=1}{Tia sáng mặt trời}\\
1& \myhref{https://en.wikibooks.org/wiki/User:Tim_Parenti}{Tim Parenti}\\
2& \myhref{https://en.wikibooks.org/w/index.php\%3ftitle=User:TinyTimZamboni\&action=edit\&redlink=1}{TinyTimZamboni}\\
2& \myhref{https://en.wikibooks.org/w/index.php\%3ftitle=User:Tisep\&action=edit\&redlink=1}{Tisep}\\
9& \myhref{https://en.wikibooks.org/w/index.php\%3ftitle=User:Tlinnet\&action=edit\&redlink=1}{Tlinnet}\\
1& \myhref{https://en.wikibooks.org/w/index.php\%3ftitle=User:ToematoeAdmn\&action=edit\&redlink=1}{ToematoeAdmn}\\
1& \myhref{https://en.wikibooks.org/wiki/User:Tom_Morris}{Tom Morris}\\
3& \myhref{https://en.wikibooks.org/w/index.php\%3ftitle=User:Tom.marcik\&action=edit\&redlink=1}{Tom.marcik}\\
120& \myhref{https://en.wikibooks.org/wiki/User:Tomato86}{Tomato86}\\
2& \myhref{https://en.wikibooks.org/w/index.php\%3ftitle=User:Tommypkeane\&action=edit\&redlink=1}{Tommypkeane}\\
6& \myhref{https://en.wikibooks.org/w/index.php\%3ftitle=User:Tomxlawson\&action=edit\&redlink=1}{Tomxlawson}\\
11& \myhref{https://en.wikibooks.org/w/index.php\%3ftitle=User:TomyDuby\&action=edit\&redlink=1}{TomyDuby}\\
1& \myhref{https://en.wikibooks.org/w/index.php\%3ftitle=User:Tonda\&action=edit\&redlink=1}{Tonda}\\
1& \myhref{https://en.wikibooks.org/w/index.php\%3ftitle=User:Toogley\&action=edit\&redlink=1}{Toogley}\\
1& \myhref{https://en.wikibooks.org/w/index.php\%3ftitle=User:Toothbrushbrush\&action=edit\&redlink=1}{Toothbrushbrush}\\
3& \myhref{https://en.wikibooks.org/wiki/User:Topodelapradera}{Topodelapradera}\\
5& \myhref{https://en.wikibooks.org/wiki/User:Torbjorn_T.}{Torbjorn T.}\\
5& \myhref{https://en.wikibooks.org/w/index.php\%3ftitle=User:TorfusPolymorphus\&action=edit\&redlink=1}{TorfusPolymorphus}\\
4& \myhref{https://en.wikibooks.org/w/index.php\%3ftitle=User:Tork73\&action=edit\&redlink=1}{Tork73}\\
2& \myhref{https://en.wikibooks.org/w/index.php\%3ftitle=User:TortoiseWrath\&action=edit\&redlink=1}{TortoiseWrath}\\
14& \myhref{https://en.wikibooks.org/wiki/User:Tosha}{Tosha}\\
2& \myhref{https://en.wikibooks.org/wiki/User:Towsonu2003~enwikibooks}{Towsonu2003\~{}enwikibooks}\\
2& \myhref{https://en.wikibooks.org/w/index.php\%3ftitle=User:Tpapastylianou\&action=edit\&redlink=1}{Tpapastylianou}\\
1& \myhref{https://en.wikibooks.org/w/index.php\%3ftitle=User:Tpr~enwikibooks\&action=edit\&redlink=1}{Tpr\~{}enwikibooks}\\
1& \myhref{https://en.wikibooks.org/wiki/User:Trace}{Trace}\\
1& \myhref{https://en.wikibooks.org/wiki/User:Tualha}{Tualha}\\
1& \myhref{https://en.wikibooks.org/w/index.php\%3ftitle=User:Tuetschek\&action=edit\&redlink=1}{Tuetschek}\\
1& \myhref{https://en.wikibooks.org/w/index.php\%3ftitle=User:Tuka~enwikibooks\&action=edit\&redlink=1}{Tuka\~{}enwikibooks}\\
7& \myhref{https://en.wikibooks.org/w/index.php\%3ftitle=User:Tully~enwikibooks\&action=edit\&redlink=1}{Tully\~{}enwikibooks}\\
1& \myhref{https://en.wikibooks.org/wiki/User:Tweenk}{Tweenk}\\
1& \myhref{https://en.wikibooks.org/w/index.php\%3ftitle=User:Uluboz\&action=edit\&redlink=1}{Uluboz}\\
1& \myhref{https://en.wikibooks.org/w/index.php\%3ftitle=User:Unbitwise\&action=edit\&redlink=1}{Unbitwise}\\
1& \myhref{https://en.wikibooks.org/w/index.php\%3ftitle=User:Unco\&action=edit\&redlink=1}{Unco}\\
1& \myhref{https://en.wikibooks.org/w/index.php\%3ftitle=User:Unlikelyuser\&action=edit\&redlink=1}{Unlikelyuser}\\
3& \myhref{https://en.wikibooks.org/wiki/User:Urhixidur}{Urhixidur}\\
1& \myhref{https://en.wikibooks.org/w/index.php\%3ftitle=User:User000name\&action=edit\&redlink=1}{User000name}\\
1& \myhref{https://en.wikibooks.org/w/index.php\%3ftitle=User:Uwe_Hartwig\&action=edit\&redlink=1}{Uwe Hartwig}\\
6& \myhref{https://en.wikibooks.org/wiki/User:Vadik_wiki}{Vadik wiki}\\
3& \myhref{https://en.wikibooks.org/w/index.php\%3ftitle=User:Vaffelkake\&action=edit\&redlink=1}{Vaffelkake}\\
1& \myhref{https://en.wikibooks.org/w/index.php\%3ftitle=User:Vanjanssen\&action=edit\&redlink=1}{Vanjanssen}\\
1& \myhref{https://en.wikibooks.org/w/index.php\%3ftitle=User:Vaucouleur\&action=edit\&redlink=1}{Vaucouleur}\\
3& \myhref{https://en.wikibooks.org/w/index.php\%3ftitle=User:Velociostrich\&action=edit\&redlink=1}{Velociostrich}\\
1& \myhref{https://en.wikibooks.org/w/index.php\%3ftitle=User:Vermiculus~enwikibooks\&action=edit\&redlink=1}{Vermiculus\~{}enwikibooks}\\
10& \myhref{https://en.wikibooks.org/w/index.php\%3ftitle=User:Vesal\&action=edit\&redlink=1}{Vesal}\\
1& \myhref{https://en.wikibooks.org/w/index.php\%3ftitle=User:Vinaisundaram\&action=edit\&redlink=1}{Vinaisundaram}\\
3& \myhref{https://en.wikibooks.org/w/index.php\%3ftitle=User:Vioricavinersan\&action=edit\&redlink=1}{Vioricavinersan}\\
2& \myhref{https://en.wikibooks.org/wiki/User:VitoFrancisco}{VitoFrancisco}\\
3& \myhref{https://en.wikibooks.org/w/index.php\%3ftitle=User:Vog2\&action=edit\&redlink=1}{Vog2}\\
3& \myhref{https://en.wikibooks.org/w/index.php\%3ftitle=User:Volvens\&action=edit\&redlink=1}{Volvens}\\
77& \myhref{https://en.wikibooks.org/wiki/User:Waldir}{Waldir}\\
1& \myhref{https://en.wikibooks.org/w/index.php\%3ftitle=User:WardMuylaert\&action=edit\&redlink=1}{WardMuylaert}\\
1& \myhref{https://en.wikibooks.org/wiki/User:Waylesange}{Waylesange}\\
4& \myhref{https://en.wikibooks.org/w/index.php\%3ftitle=User:Wdcf\&action=edit\&redlink=1}{Wdcf}\\
1& \myhref{https://en.wikibooks.org/w/index.php\%3ftitle=User:Webinn\&action=edit\&redlink=1}{Webinn}\\
2& \myhref{https://en.wikibooks.org/w/index.php\%3ftitle=User:Wenzeslaus\&action=edit\&redlink=1}{Wenzeslaus}\\
1& \myhref{https://en.wikibooks.org/w/index.php\%3ftitle=User:Wgjbeek\&action=edit\&redlink=1}{Wgjbeek}\\
1& \myhref{https://en.wikibooks.org/w/index.php\%3ftitle=User:White_gecko\&action=edit\&redlink=1}{White gecko}\\
1& \myhref{https://en.wikibooks.org/wiki/User:Whiteknight}{Whiteknight}\\
6& \myhref{https://en.wikibooks.org/wiki/User:Whym}{Whym}\\
14& \myhref{https://en.wikibooks.org/wiki/User:Wickedjargon}{Wickedjargon}\\
9& \myhref{https://en.wikibooks.org/wiki/User:Wikieditoroftoday}{Wikieditoroftoday}\\
1& \myhref{https://en.wikibooks.org/w/index.php\%3ftitle=User:Willy_james\&action=edit\&redlink=1}{Willy james}\\
1& \myhref{https://en.wikibooks.org/wiki/User:Winfree}{Winfree}\\
1& \myhref{https://en.wikibooks.org/w/index.php\%3ftitle=User:Winniehell\&action=edit\&redlink=1}{Winniehell}\\
25& \myhref{https://en.wikibooks.org/wiki/User:Withinfocus}{Withinfocus}\\
1& \myhref{https://en.wikibooks.org/w/index.php\%3ftitle=User:Wkdurfee~enwikibooks\&action=edit\&redlink=1}{Wkdurfee\~{}enwikibooks}\\
1& \myhref{https://en.wikibooks.org/w/index.php\%3ftitle=User:Wmheric\&action=edit\&redlink=1}{Wmheric}\\
1& \myhref{https://en.wikibooks.org/w/index.php\%3ftitle=User:Wn202\&action=edit\&redlink=1}{Wn202}\\
1& \myhref{https://en.wikibooks.org/w/index.php\%3ftitle=User:Wootery\&action=edit\&redlink=1}{Wootery}\\
1& \myhref{https://en.wikibooks.org/w/index.php\%3ftitle=User:Wp4bl0\&action=edit\&redlink=1}{Wp4bl0}\\
2& \myhref{https://en.wikibooks.org/w/index.php\%3ftitle=User:Writalnaie\&action=edit\&redlink=1}{Writalnaie}\\
3& \myhref{https://en.wikibooks.org/w/index.php\%3ftitle=User:Wxm29\&action=edit\&redlink=1}{Wxm29}\\
1& \myhref{https://en.wikibooks.org/w/index.php\%3ftitle=User:Wysinwygaa\&action=edit\&redlink=1}{Wysinwygaa}\\
12& \myhref{https://en.wikibooks.org/wiki/User:Xania}{Xania}\\
1& \myhref{https://en.wikibooks.org/w/index.php\%3ftitle=User:Xeracles\&action=edit\&redlink=1}{Xeracles}\\
1& \myhref{https://en.wikibooks.org/w/index.php\%3ftitle=User:Xin-Xin_W.\&action=edit\&redlink=1}{Xin-{}Xin W.}\\
1& \myhref{https://en.wikibooks.org/w/index.php\%3ftitle=User:Xnn\&action=edit\&redlink=1}{Xnn}\\
13& \myhref{https://en.wikibooks.org/wiki/User:Xonqnopp}{Xonqnopp}\\
1& \myhref{https://en.wikibooks.org/w/index.php\%3ftitle=User:Yanuzz\&action=edit\&redlink=1}{Yanuzz}\\
5& \myhref{https://en.wikibooks.org/w/index.php\%3ftitle=User:Yeshua_Saves\&action=edit\&redlink=1}{Yeshua Saves}\\
1& \myhref{https://en.wikibooks.org/w/index.php\%3ftitle=User:Yez\&action=edit\&redlink=1}{Yez}\\
1& \myhref{https://en.wikibooks.org/wiki/User:Yinweichen}{Yinweichen}\\
1& \myhref{https://en.wikibooks.org/wiki/User:Ynhockey}{Ynhockey}\\
1& \myhref{https://en.wikibooks.org/w/index.php\%3ftitle=User:Yotann\&action=edit\&redlink=1}{Yotann}\\
1& \myhref{https://en.wikibooks.org/w/index.php\%3ftitle=User:Ypey\&action=edit\&redlink=1}{Ypey}\\
2& \myhref{https://en.wikibooks.org/wiki/User:Ysangkok}{Ysangkok}\\
2& \myhref{https://en.wikibooks.org/w/index.php\%3ftitle=User:Ysnikraz\&action=edit\&redlink=1}{Ysnikraz}\\
1& \myhref{https://en.wikibooks.org/w/index.php\%3ftitle=User:YuryKirienko\&action=edit\&redlink=1}{YuryKirienko}\\
1& \myhref{https://en.wikibooks.org/w/index.php\%3ftitle=User:Zaslav\&action=edit\&redlink=1}{Zaslav}\\
3& \myhref{https://en.wikibooks.org/wiki/User:ZeroOne}{ZeroOne}\\
1& \myhref{https://en.wikibooks.org/w/index.php\%3ftitle=User:ZimbiX\&action=edit\&redlink=1}{ZimbiX}\\
1& \myhref{https://en.wikibooks.org/w/index.php\%3ftitle=User:Zrisher\&action=edit\&redlink=1}{Zrisher}\\
3& \myhref{https://en.wikibooks.org/wiki/User:Zvika}{Zvika}\\
1& \myhref{https://en.wikibooks.org/w/index.php\%3ftitle=User:Zwiebelleder\&action=edit\&redlink=1}{Zwiebelleder}\\
1& \myhref{https://en.wikibooks.org/w/index.php\%3ftitle=User:Zxx117\&action=edit\&redlink=1}{Zxx117}\\
1& \myhref{https://en.wikibooks.org/w/index.php\%3ftitle=User:Zylorian\&action=edit\&redlink=1}{Zylorian}\\
2& \myhref{https://en.wikibooks.org/w/index.php\%3ftitle=User:Zyqqh~enwikibooks\&action=edit\&redlink=1}{Zyqqh\~{}enwikibooks}\\
1& \myhref{https://en.wikibooks.org/w/index.php\%3ftitle=User:Zzo38\&action=edit\&redlink=1}{Zzo38}\\
1& \myhref{https://en.wikibooks.org/wiki/User:\%25C3\%2586var_Arnfj\%25C3\%25B6r\%25C3\%25B0_Bjarmason}{Ævar Arnfjörð Bjarmason}\\
1& \myhref{https://en.wikibooks.org/w/index.php\%3ftitle=User:\%25D0\%259F\%25D0\%25B8\%25D0\%25BA\%25D0\%25B0_\%25D0\%259F\%25D0\%25B8\%25D0\%25BA\%25D0\%25B0\&action=edit\&redlink=1}{Пика Пика}\\
1& \myhref{https://en.wikibooks.org/w/index.php\%3ftitle=User:\%25D5\%258D\%25D5\%25A1\%25D5\%25B0\%25D5\%25A1\%25D5\%25AF\&action=edit\&redlink=1}{Սահակ}\\
1& \myhref{https://en.wikibooks.org/wiki/User:\%25D7\%259C\%25D7\%25A2\%25D7\%25A8\%25D7\%2599_\%25D7\%25A8\%25D7\%2599\%25D7\%2599\%25D7\%25A0\%25D7\%2594\%25D7\%2590\%25D7\%25A8\%25D7\%2598}{לערי ריינהארט}\\
1& \myhref{https://en.wikibooks.org/w/index.php\%3ftitle=User:\%25D8\%25A7\%25D9\%2585\%25DB\%258C\%25D8\%25B1_\%25D8\%25A7\%25D8\%25B9\%25D9\%2588\%25D8\%25A7\%25D9\%2586\%25DB\%258C\&action=edit\&redlink=1}{امیر اعوانی}\\
\end{longtable}
\pagebreak
\listoffigures
\label{ListOfFigures}
\begin{itemize}
\item GFDL: Gnu Free Documentation License. \url{http://www.gnu.org/licenses/fdl.html}
\item cc-by-sa-3.0:  Creative Commons Attribution ShareAlike 3.0 License. \url{http://creativecommons.org/licenses/by-sa/3.0/} 
\item cc-by-sa-2.5:  Creative Commons Attribution ShareAlike 2.5 License. \url{http://creativecommons.org/licenses/by-sa/2.5/} 
\item cc-by-sa-2.0:  Creative Commons Attribution ShareAlike 2.0 License. \url{http://creativecommons.org/licenses/by-sa/2.0/} 
\item cc-by-sa-1.0:  Creative Commons Attribution ShareAlike 1.0 License. \url{http://creativecommons.org/licenses/by-sa/1.0/} 
\item cc-by-2.0:  Creative Commons Attribution 2.0 License.  \url{http://creativecommons.org/licenses/by/2.0/}  
\item cc-by-2.0:  Creative Commons Attribution 2.0 License. \url{http://creativecommons.org/licenses/by/2.0/deed.en}  
\item cc-by-2.5:  Creative Commons Attribution 2.5 License. \url{http://creativecommons.org/licenses/by/2.5/deed.en}  
\item cc-by-3.0:  Creative Commons Attribution 3.0 License. \url{http://creativecommons.org/licenses/by/3.0/deed.en}  
\item GPL:  GNU General Public License. \url{http://www.gnu.org/licenses/gpl-2.0.txt} 
\item LGPL:  GNU Lesser General Public License. \url{http://www.gnu.org/licenses/lgpl.html}
 \item PD: This image is in the public domain.
\item ATTR:  The copyright holder of this file allows anyone to use it for any purpose, provided that the copyright holder is properly attributed. Redistribution, derivative work, commercial use, and all other use is permitted. 
\item EURO: This is the common (reverse) face of a euro coin. The copyright on the design of the common face of the euro coins belongs to the European Commission. Authorised is reproduction in a format without relief (drawings, paintings, films) provided they are not detrimental to the image of the euro.
\item LFK: Lizenz Freie Kunst. \url{http://artlibre.org/licence/lal/de} 
\item CFR: Copyright free use. 
\item EPL: Eclipse Public License. \url{http://www.eclipse.org/org/documents/epl-v10.php} 
\end{itemize}
Copies of the GPL, the LGPL as well as a GFDL are included in chapter \mylref{Licenses}{Licenses}. Please note that images in the public domain do not require attribution. You may click on the image numbers in the following table to open the webpage of the images in your webbrower.
\pagebreak
\small
\begin{longtable}{|p{0.05\textwidth}|p{0.6\textwidth}|p{0.15\textwidth}|}
\hline
\href{https://en.wikibooks.org/wiki/File:Gummi\%200.6.1\%20screenshot.png}{1}& Gummi team&\\ \hline 
\href{https://en.wikibooks.org/wiki/File:LyX1.6.3.png}{2}& LyX developer team (see www.lyx.org)&GPL\\ \hline 
\href{https://en.wikibooks.org/wiki/File:TeXworks.png}{3}& \myhref{http://commons.wikimedia.org/wiki/User:PAC2}{PAC2}, \myhref{https://commons.wikimedia.org/wiki/User:PAC2}{PAC2}&GPL\\ \hline 
\href{https://en.wikibooks.org/wiki/File:Kile\%201.9.3.png}{4}& BotMultichill, BotMultichillT, Emijrpbot, Hazard-Bot, JarektBot, KAMiKAZOW, Paucabot, Wiso&\\ \hline 
\href{https://en.wikibooks.org/wiki/File:Jabref-2.2-screenshot.png}{5}& Emijrpbot, Hazard-Bot, JarektBot, MGA73bot2, Mwtoews, Patrick87, SieBot, Ö&\\ \hline 
\href{https://en.wikibooks.org/wiki/File:BibDesk1.3.8.jpg}{6}& Myself&\\ \hline 
\href{https://en.wikibooks.org/wiki/File:LaTeX_diagram.svg}{7}& Alessio Damato&GFDL\\ \hline 
\href{https://en.wikibooks.org/wiki/File:quote1.png}{8}& Editor at Large, Infrogmation, Itsmine, JarektBot, Jtwdog~enwikibooks, Michiel1972, Shyam, Waldir, Wst&\\ \hline 
\href{https://en.wikibooks.org/wiki/File:quote2.png}{9}& Jtwdog~enwikibooks&\\ \hline 
\href{https://en.wikibooks.org/wiki/File:quote2.png}{10}& Jtwdog~enwikibooks&\\ \hline 
\href{https://en.wikibooks.org/wiki/File:quote4.png}{11}& \myhref{http://commons.wikimedia.org/w/index.php?title=User:Tomato86\&action=edit\&redlink=1}{Tomato86}, \myhref{https://commons.wikimedia.org/w/index.php?title=User:Tomato86\&action=edit\&redlink=1}{Tomato86}&GFDL\\ \hline 
\href{https://en.wikibooks.org/wiki/File:Example\%20of\%20German\%20quotation\%20marks.png}{12}& \myhref{http://commons.wikimedia.org/w/index.php?title=User:Kscheel\&action=edit\&redlink=1}{Karl Scheel}, \myhref{https://commons.wikimedia.org/w/index.php?title=User:Kscheel\&action=edit\&redlink=1}{Karl Scheel}&CC-BY-SA-3.0\\ \hline 
\href{https://en.wikibooks.org/wiki/File:Example\%20of\%20French\%20quotation\%20marks.png}{13}& \myhref{http://commons.wikimedia.org/w/index.php?title=User:Kscheel\&action=edit\&redlink=1}{Karl Scheel}, \myhref{https://commons.wikimedia.org/w/index.php?title=User:Kscheel\&action=edit\&redlink=1}{Karl Scheel}&CC-BY-SA-3.0\\ \hline 
\href{https://en.wikibooks.org/wiki/File:Latex_quote_3.png}{14}& \myhref{http://commons.wikimedia.org/w/index.php?title=User:Thenub314\&action=edit\&redlink=1}{Thenub314}, \myhref{https://commons.wikimedia.org/w/index.php?title=User:Thenub314\&action=edit\&redlink=1}{Thenub314}&GFDL\\ \hline 
\href{https://en.wikibooks.org/wiki/File:Latex_quote_4.png}{15}& \myhref{http://commons.wikimedia.org/w/index.php?title=User:Tomato86\&action=edit\&redlink=1}{Tomato86}, \myhref{https://commons.wikimedia.org/w/index.php?title=User:Tomato86\&action=edit\&redlink=1}{Tomato86}&GFDL\\ \hline 
\href{https://en.wikibooks.org/wiki/File:LaTeX\%20sloppypar.png}{16}& Derbeth&\\ \hline 
\href{https://en.wikibooks.org/wiki/File:Latex\%20example\%20ligatures.png}{17}& Tobias Oetiker&GFDL\\ \hline 
\href{https://en.wikibooks.org/wiki/File:LaTeXSubSuperscript.png}{18}& \myhref{http://commons.wikimedia.org/w/index.php?title=User:Johannes_Bo\&action=edit\&redlink=1}{Johannes Bo}, \myhref{https://commons.wikimedia.org/w/index.php?title=User:Johannes_Bo\&action=edit\&redlink=1}{Johannes Bo}&PD\\ \hline 
\href{https://en.wikibooks.org/wiki/File:Latex\%20dashes\%20example.png}{19}& Tobias Oetiker&GFDL\\ \hline 
\href{https://en.wikibooks.org/wiki/File:Latex\%20example\%20text\%20dots.png}{20}& Tobias Oetiker&GFDL\\ \hline 
\href{https://en.wikibooks.org/wiki/File:Latex\%20ready-made\%20strings.png}{21}& Tobias Oetiker&GFDL\\ \hline 
\href{https://en.wikibooks.org/wiki/File:verbatim.svg}{22}& \myhref{http://commons.wikimedia.org/wiki/User:Dirk_H\%C3\%BCnniger}{Dirk Hünniger}, \myhref{https://commons.wikimedia.org/wiki/User:Dirk_H\%C3\%BCnniger}{Dirk Hünniger}&CC-BY-SA-3.0\\ \hline 
\href{https://en.wikibooks.org/wiki/File:alltt.svg}{23}& \myhref{http://commons.wikimedia.org/wiki/User:Dirk_H\%C3\%BCnniger}{Dirk Hünniger}, \myhref{https://commons.wikimedia.org/wiki/User:Dirk_H\%C3\%BCnniger}{Dirk Hünniger}&CC-BY-SA-3.0\\ \hline 
\href{https://en.wikibooks.org/wiki/File:LaTeX_colour_demo_1.png}{24}& \myhref{http://commons.wikimedia.org/wiki/User:ChrisHodgesUK}{ChrisHodgesUK}, \myhref{https://commons.wikimedia.org/wiki/User:ChrisHodgesUK}{ChrisHodgesUK}&PD\\ \hline 
\href{https://en.wikibooks.org/wiki/File:LaTeX\%20font\%20example.png}{25}& \myhref{http://commons.wikimedia.org/wiki/User:ChrisHodgesUK}{ChrisHodgesUK}, \myhref{https://commons.wikimedia.org/wiki/User:ChrisHodgesUK}{ChrisHodgesUK}&PD\\ \hline 
\href{https://en.wikibooks.org/wiki/File:emph.png}{26}& Jtwdog~enwikibooks&\\ \hline 
\href{https://en.wikibooks.org/wiki/File:WikibookLists.png}{27}& \myhref{http://commons.wikimedia.org/w/index.php?title=User:Johannes_Bo\&action=edit\&redlink=1}{Johannes Bo}, \myhref{https://commons.wikimedia.org/w/index.php?title=User:Johannes_Bo\&action=edit\&redlink=1}{Johannes Bo}&\\ \hline 
\href{https://en.wikibooks.org/wiki/File:nested.svg}{28}& \myhref{http://commons.wikimedia.org/wiki/User:Dirk_H\%C3\%BCnniger}{Dirk Hünniger}, \myhref{https://commons.wikimedia.org/wiki/User:Dirk_H\%C3\%BCnniger}{Dirk Hünniger}&CC-BY-SA-3.0\\ \hline 
\href{https://en.wikibooks.org/wiki/File:WikibookListsLabeling.png}{29}& \myhref{http://commons.wikimedia.org/w/index.php?title=User:Johannes_Bo\&action=edit\&redlink=1}{Johannes Bo}, \myhref{https://commons.wikimedia.org/w/index.php?title=User:Johannes_Bo\&action=edit\&redlink=1}{Johannes Bo}&\\ \hline 
\href{https://en.wikibooks.org/wiki/File:WikibooksListsInline.png}{30}& \myhref{http://commons.wikimedia.org/w/index.php?title=User:Johannes_Bo\&action=edit\&redlink=1}{Johannes Bo}, \myhref{https://commons.wikimedia.org/w/index.php?title=User:Johannes_Bo\&action=edit\&redlink=1}{Johannes Bo}&\\ \hline 
\href{https://en.wikibooks.org/wiki/File:WikibooksListsTask.png}{31}& \myhref{http://commons.wikimedia.org/w/index.php?title=User:Johannes_Bo\&action=edit\&redlink=1}{Johannes Bo}, \myhref{https://commons.wikimedia.org/w/index.php?title=User:Johannes_Bo\&action=edit\&redlink=1}{Johannes Bo}&\\ \hline 
\href{https://en.wikibooks.org/wiki/File:WikibookListsCompact.png}{32}& \myhref{http://commons.wikimedia.org/w/index.php?title=User:Johannes_Bo\&action=edit\&redlink=1}{Johannes Bo}, \myhref{https://commons.wikimedia.org/w/index.php?title=User:Johannes_Bo\&action=edit\&redlink=1}{Johannes Bo}&\\ \hline 
\href{https://en.wikibooks.org/wiki/File:WikibookListsCustom.png}{33}& \myhref{http://commons.wikimedia.org/w/index.php?title=User:Johannes_Bo\&action=edit\&redlink=1}{Johannes Bo}, \myhref{https://commons.wikimedia.org/w/index.php?title=User:Johannes_Bo\&action=edit\&redlink=1}{Johannes Bo}&\\ \hline 
\href{https://en.wikibooks.org/wiki/File:LaTeX-dingbats.png}{34}& Derbeth&\\ \hline 
\href{https://en.wikibooks.org/wiki/File:Latex\%20example\%20tabular\%20cline.svg}{35}& \myhref{http://commons.wikimedia.org/wiki/User:Dirk_H\%C3\%BCnniger}{Dirk Hünniger}, \myhref{https://commons.wikimedia.org/wiki/User:Dirk_H\%C3\%BCnniger}{Dirk Hünniger}&CC-BY-SA-3.0\\ \hline 
\href{https://en.wikibooks.org/wiki/File:Latex\%20example\%20wrapped\%20table.svg}{36}& \myhref{http://commons.wikimedia.org/wiki/User:Dirk_H\%C3\%BCnniger}{Dirk Hünniger}, \myhref{https://commons.wikimedia.org/wiki/User:Dirk_H\%C3\%BCnniger}{Dirk Hünniger}&CC-BY-SA-3.0\\ \hline 
\href{https://en.wikibooks.org/wiki/File:Latex\%20example\%20defining\%20multiple\%20columns.svg}{37}& \myhref{http://commons.wikimedia.org/wiki/User:Dirk_H\%C3\%BCnniger}{Dirk Hünniger}, \myhref{https://commons.wikimedia.org/wiki/User:Dirk_H\%C3\%BCnniger}{Dirk Hünniger}&CC-BY-SA-3.0\\ \hline 
\href{https://en.wikibooks.org/wiki/File:specifier1.svg}{38}& \myhref{http://commons.wikimedia.org/wiki/User:Dirk_H\%C3\%BCnniger}{Dirk Hünniger}, \myhref{https://commons.wikimedia.org/wiki/User:Dirk_H\%C3\%BCnniger}{Dirk Hünniger}&CC-BY-SA-3.0\\ \hline 
\href{https://en.wikibooks.org/wiki/File:specifier2.svg}{39}& \myhref{http://commons.wikimedia.org/wiki/User:Dirk_H\%C3\%BCnniger}{Dirk Hünniger}, \myhref{https://commons.wikimedia.org/wiki/User:Dirk_H\%C3\%BCnniger}{Dirk Hünniger}&CC-BY-SA-3.0\\ \hline 
\href{https://en.wikibooks.org/wiki/File:specifier3.svg}{40}& \myhref{http://commons.wikimedia.org/wiki/User:Dirk_H\%C3\%BCnniger}{Dirk Hünniger}, \myhref{https://commons.wikimedia.org/wiki/User:Dirk_H\%C3\%BCnniger}{Dirk Hünniger}&CC-BY-SA-3.0\\ \hline 
\href{https://en.wikibooks.org/wiki/File:specifier4.svg}{41}& \myhref{http://commons.wikimedia.org/wiki/User:Dirk_H\%C3\%BCnniger}{Dirk Hünniger}, \myhref{https://commons.wikimedia.org/wiki/User:Dirk_H\%C3\%BCnniger}{Dirk Hünniger}&CC-BY-SA-3.0\\ \hline 
\href{https://en.wikibooks.org/wiki/File:LaTeX_example_dcolumn.png}{42}& \myhref{http://commons.wikimedia.org/wiki/User:ChrisHodgesUK}{ChrisHodgesUK}, \myhref{https://commons.wikimedia.org/wiki/User:ChrisHodgesUK}{ChrisHodgesUK}&PD\\ \hline 
\href{https://en.wikibooks.org/wiki/File:LaTeX\%20example\%20dcolumn\%20bold.png}{43}& \myhref{http://commons.wikimedia.org/wiki/User:ChrisHodgesUK}{ChrisHodgesUK}, \myhref{https://commons.wikimedia.org/wiki/User:ChrisHodgesUK}{ChrisHodgesUK}&PD\\ \hline 
\href{https://en.wikibooks.org/wiki/File:multicolumn.svg}{44}& \myhref{http://commons.wikimedia.org/wiki/User:Dirk_H\%C3\%BCnniger}{Dirk Hünniger}, \myhref{https://commons.wikimedia.org/wiki/User:Dirk_H\%C3\%BCnniger}{Dirk Hünniger}&CC-BY-SA-3.0\\ \hline 
\href{https://en.wikibooks.org/wiki/File:multirow.svg}{45}& \myhref{http://commons.wikimedia.org/wiki/User:Dirk_H\%C3\%BCnniger}{Dirk Hünniger}, \myhref{https://commons.wikimedia.org/wiki/User:Dirk_H\%C3\%BCnniger}{Dirk Hünniger}&CC-BY-SA-3.0\\ \hline 
\href{https://en.wikibooks.org/wiki/File:multirowandcolumnexample.svg}{46}& \myhref{http://commons.wikimedia.org/wiki/User:Dirk_H\%C3\%BCnniger}{Dirk Hünniger}, \myhref{https://commons.wikimedia.org/wiki/User:Dirk_H\%C3\%BCnniger}{Dirk Hünniger}&CC-BY-SA-3.0\\ \hline 
\href{https://en.wikibooks.org/wiki/File:Latex-tables-double-dichotomy-example.svg}{47}& \myhref{http://commons.wikimedia.org/wiki/User:Dirk_H\%C3\%BCnniger}{Dirk Hünniger}, \myhref{https://commons.wikimedia.org/wiki/User:Dirk_H\%C3\%BCnniger}{Dirk Hünniger}&CC-BY-SA-3.0\\ \hline 
\href{https://en.wikibooks.org/wiki/File:LaTeXAlternateRowTable.svg}{48}& \myhref{http://commons.wikimedia.org/wiki/User:Dirk_H\%C3\%BCnniger}{Dirk Hünniger}, \myhref{https://commons.wikimedia.org/wiki/User:Dirk_H\%C3\%BCnniger}{Dirk Hünniger}&CC-BY-SA-3.0\\ \hline 
\href{https://en.wikibooks.org/wiki/File:LaTeX\%20TabWidth1.svg}{49}& \myhref{http://commons.wikimedia.org/wiki/User:Dirk_H\%C3\%BCnniger}{Dirk Hünniger}, \myhref{https://commons.wikimedia.org/wiki/User:Dirk_H\%C3\%BCnniger}{Dirk Hünniger}&CC-BY-SA-3.0\\ \hline 
\href{https://en.wikibooks.org/wiki/File:LaTeX\%20TabWidth2.svg}{50}& \myhref{http://commons.wikimedia.org/wiki/User:Dirk_H\%C3\%BCnniger}{Dirk Hünniger}, \myhref{https://commons.wikimedia.org/wiki/User:Dirk_H\%C3\%BCnniger}{Dirk Hünniger}&CC-BY-SA-3.0\\ \hline 
\href{https://en.wikibooks.org/wiki/File:LaTeX\%20TabXWidth1.svg}{51}& \myhref{http://commons.wikimedia.org/wiki/User:Dirk_H\%C3\%BCnniger}{Dirk Hünniger}, \myhref{https://commons.wikimedia.org/wiki/User:Dirk_H\%C3\%BCnniger}{Dirk Hünniger}&CC-BY-SA-3.0\\ \hline 
\href{https://en.wikibooks.org/wiki/File:LaTeX\%20TabXWidth2.svg}{52}& \myhref{http://commons.wikimedia.org/wiki/User:Dirk_H\%C3\%BCnniger}{Dirk Hünniger}, \myhref{https://commons.wikimedia.org/wiki/User:Dirk_H\%C3\%BCnniger}{Dirk Hünniger}&CC-BY-SA-3.0\\ \hline 
\href{https://en.wikibooks.org/wiki/File:LaTeX\%20tabularx_multi.svg}{53}& \myhref{http://commons.wikimedia.org/wiki/User:Dirk_H\%C3\%BCnniger}{Dirk Hünniger}, \myhref{https://commons.wikimedia.org/wiki/User:Dirk_H\%C3\%BCnniger}{Dirk Hünniger}&CC-BY-SA-3.0\\ \hline 
\href{https://en.wikibooks.org/wiki/File:Partial-vertical-line-add.svg}{54}& \myhref{http://commons.wikimedia.org/wiki/User:Dirk_H\%C3\%BCnniger}{Dirk Hünniger}, \myhref{https://commons.wikimedia.org/wiki/User:Dirk_H\%C3\%BCnniger}{Dirk Hünniger}&CC-BY-SA-3.0\\ \hline 
\href{https://en.wikibooks.org/wiki/File:Partial-vertical-line-remove.svg}{55}& \myhref{http://commons.wikimedia.org/wiki/User:Dirk_H\%C3\%BCnniger}{Dirk Hünniger}, \myhref{https://commons.wikimedia.org/wiki/User:Dirk_H\%C3\%BCnniger}{Dirk Hünniger}&CC-BY-SA-3.0\\ \hline 
\href{https://en.wikibooks.org/wiki/File:LaTeX\%20animal\%20table.svg}{56}& \myhref{http://commons.wikimedia.org/wiki/User:Dirk_H\%C3\%BCnniger}{Dirk Hünniger}, \myhref{https://commons.wikimedia.org/wiki/User:Dirk_H\%C3\%BCnniger}{Dirk Hünniger}&CC-BY-SA-3.0\\ \hline 
\href{https://en.wikibooks.org/wiki/File:LaTeX\%20animal\%20table\%20with\%20array.svg}{57}& \myhref{http://commons.wikimedia.org/w/index.php?title=User:Danroa\&action=edit\&redlink=1}{Danroa}, \myhref{https://commons.wikimedia.org/w/index.php?title=User:Danroa\&action=edit\&redlink=1}{Danroa}&\\ \hline 
\href{https://en.wikibooks.org/wiki/File:LaTeX\%20animal\%20table\%20with\%20booktabs.svg}{58}& \myhref{http://commons.wikimedia.org/wiki/User:Dirk_H\%C3\%BCnniger}{Dirk Hünniger}, \myhref{https://commons.wikimedia.org/wiki/User:Dirk_H\%C3\%BCnniger}{Dirk Hünniger}&CC-BY-SA-3.0\\ \hline 
\href{https://en.wikibooks.org/wiki/File:TitlepageWikibook.png}{59}& Johannes Bo&PD\\ \hline 
\href{https://en.wikibooks.org/wiki/File:Latex_layout.svg}{60}& \myhref{http://commons.wikimedia.org/wiki/User:Alejo2083}{Alessio Damato}, \myhref{https://commons.wikimedia.org/wiki/User:Alejo2083}{Alessio Damato}&GFDL\\ \hline 
\href{https://en.wikibooks.org/wiki/File:chick1.png}{61}& The original uploader was \myhref{http://en.wikibooks.org/wiki/User:Jtwdog}{Jtwdog} at \myhref{http://en.wikibooks.org/wiki/}{English Wikibooks}&GFDL\\ \hline 
\href{https://en.wikibooks.org/wiki/File:chick1.png}{62}& The original uploader was \myhref{http://en.wikibooks.org/wiki/User:Jtwdog}{Jtwdog} at \myhref{http://en.wikibooks.org/wiki/}{English Wikibooks}&GFDL\\ \hline 
\href{https://en.wikibooks.org/wiki/File:chick1.png}{63}& The original uploader was \myhref{http://en.wikibooks.org/wiki/User:Jtwdog}{Jtwdog} at \myhref{http://en.wikibooks.org/wiki/}{English Wikibooks}&GFDL\\ \hline 
\href{https://en.wikibooks.org/wiki/File:Latex_picture_example.png}{64}& \myhref{http://commons.wikimedia.org/wiki/User:ChrisHodgesUK}{ChrisHodgesUK}, \myhref{https://commons.wikimedia.org/wiki/User:ChrisHodgesUK}{ChrisHodgesUK}&PD\\ \hline 
\href{https://en.wikibooks.org/wiki/File:Latex\%20caption\%20example.png}{65}& \myhref{http://commons.wikimedia.org/wiki/User:Alejo2083}{Alessio Damato}, \myhref{https://commons.wikimedia.org/wiki/User:Alejo2083}{Alessio Damato}&GFDL\\ \hline 
\href{https://en.wikibooks.org/wiki/File:Latex\%20example\%20sidecap.png}{66}& \myhref{http://commons.wikimedia.org/wiki/User:Mwtoews}{User:Mwtoews}, \myhref{https://commons.wikimedia.org/wiki/User:Mwtoews}{User:Mwtoews}&GFDL\\ \hline 
\href{https://en.wikibooks.org/wiki/File:LaTeX\%20figure\%20caption\%20with\%20lof\%20entry.png}{67}& \myhref{http://commons.wikimedia.org/wiki/User:Mwtoews}{Mwtoews}, \myhref{https://commons.wikimedia.org/wiki/User:Mwtoews}{Mwtoews}&GFDL\\ \hline 
\href{https://en.wikibooks.org/wiki/File:Latex\%20example\%20wrapfig.png}{68}& \myhref{http://commons.wikimedia.org/wiki/User:Alejo2083}{Alessio Damato}, \myhref{https://commons.wikimedia.org/wiki/User:Alejo2083}{Alessio Damato}&GFDL\\ \hline 
\href{https://en.wikibooks.org/wiki/File:Latex\%20example\%20wrapfig\%20vspace.png}{69}& \myhref{http://commons.wikimedia.org/wiki/User:Alejo2083}{Alessio Damato}, \myhref{https://commons.wikimedia.org/wiki/User:Alejo2083}{Alessio Damato}&GFDL\\ \hline 
\href{https://en.wikibooks.org/wiki/File:Latex\%20example\%20subfig.png}{70}& \myhref{http://commons.wikimedia.org/wiki/User:Alejo2083}{Alessio Damato}, \myhref{https://commons.wikimedia.org/wiki/User:Alejo2083}{Alessio Damato}&GFDL\\ \hline 
\href{https://en.wikibooks.org/wiki/File:LaTeX-footnote.png}{71}& Derbeth, \myhref{https://commons.wikimedia.org/w/index.php?title=User:Jovan.Andj1996\&action=edit\&redlink=1}{Jovan.Andj1996}&\\ \hline 
\href{https://en.wikibooks.org/wiki/File:LaTeX\%20marginpar.png}{72}& \myhref{http://commons.wikimedia.org/wiki/User:Derbeth}{Derbeth}, \myhref{https://commons.wikimedia.org/wiki/User:Derbeth}{Derbeth}&GFDL\\ \hline 
\href{https://en.wikibooks.org/wiki/File:Marginnote\%20geometry\%20LaTeX\%20packages.png}{73}& \myhref{http://commons.wikimedia.org/wiki/User:Maschen}{Maschen}, \myhref{https://commons.wikimedia.org/wiki/User:Maschen}{Maschen}&PD\\ \hline 
\href{https://en.wikibooks.org/wiki/File:Latex\%20example\%20referencing\%20section.png}{74}& \myhref{http://commons.wikimedia.org/wiki/User:Alejo2083}{Alessio Damato}, \myhref{https://commons.wikimedia.org/wiki/User:Alejo2083}{Alessio Damato}&GFDL\\ \hline 
\href{https://en.wikibooks.org/wiki/File:Latex\%20example\%20figure\%20referencing.png}{75}& \myhref{http://commons.wikimedia.org/wiki/User:Alejo2083}{Alessio Damato} 13:31, 12 January 2007 (UTC), \myhref{https://commons.wikimedia.org/wiki/User:Alejo2083}{Alessio Damato} 13:31, 12 January 2007 (UTC)&GFDL\\ \hline 
\href{https://en.wikibooks.org/wiki/File:Latex\%20example\%20math\%20referencing.png}{76}& \myhref{http://commons.wikimedia.org/wiki/User:Alejo2083}{Alessio Damato}, \myhref{https://commons.wikimedia.org/wiki/User:Alejo2083}{Alessio Damato}&GFDL\\ \hline 
\href{https://en.wikibooks.org/wiki/File:Texcharbox.svg}{77}& \myhref{http://commons.wikimedia.org/wiki/User:Ambrevar}{Ambrevar}, \myhref{https://commons.wikimedia.org/wiki/User:Ambrevar}{Ambrevar}&PD\\ \hline 
\href{https://en.wikibooks.org/wiki/File:Latex\%20example\%20box\%20test.png}{78}& \myhref{http://commons.wikimedia.org/wiki/User:Alejo2083}{Alessio Damato}, \myhref{https://commons.wikimedia.org/wiki/User:Alejo2083}{Alessio Damato}&GFDL\\ \hline 
\href{https://en.wikibooks.org/wiki/File:Latex\%20example\%20box\%20test\%202.png}{79}& \myhref{http://commons.wikimedia.org/wiki/User:Alejo2083}{Alessio Damato}, \myhref{https://commons.wikimedia.org/wiki/User:Alejo2083}{Alessio Damato}&GFDL\\ \hline 
\href{https://en.wikibooks.org/wiki/File:Latex\%20example\%20rule.png}{80}& \myhref{http://commons.wikimedia.org/wiki/User:Alejo2083}{Alessio Damato}, \myhref{https://commons.wikimedia.org/wiki/User:Alejo2083}{Alessio Damato}&GFDL\\ \hline 
\href{https://en.wikibooks.org/wiki/File:LaTeX-xfrac-example.png}{81}& \myhref{http://commons.wikimedia.org/w/index.php?title=User:Tomato86\&action=edit\&redlink=1}{Tomato86}, \myhref{https://commons.wikimedia.org/w/index.php?title=User:Tomato86\&action=edit\&redlink=1}{Tomato86}&GFDL\\ \hline 
\href{https://en.wikibooks.org/wiki/File:Latex_new_squareroot.png}{82}& \myhref{http://commons.wikimedia.org/wiki/User:Alejo2083}{Alessio Damato}, \myhref{https://commons.wikimedia.org/wiki/User:Alejo2083}{Alessio Damato}&GFDL\\ \hline 
\href{https://en.wikibooks.org/wiki/File:LaTeX\%20example\%20sqrt.png}{83}& \myhref{http://commons.wikimedia.org/wiki/User:ChrisHodgesUK}{ChrisHodgesUK}, \myhref{https://commons.wikimedia.org/wiki/User:ChrisHodgesUK}{ChrisHodgesUK}&PD\\ \hline 
\href{https://en.wikibooks.org/wiki/File:Latex_example_middle.png}{84}& \myhref{http://commons.wikimedia.org/wiki/User:ChrisHodgesUK}{ChrisHodgesUK}, \myhref{https://commons.wikimedia.org/wiki/User:ChrisHodgesUK}{ChrisHodgesUK}&PD\\ \hline 
\href{https://en.wikibooks.org/wiki/File:bordermatrix.png}{85}& \myhref{http://commons.wikimedia.org/w/index.php?title=User:Winfree\&action=edit\&redlink=1}{Winfree}, \myhref{https://commons.wikimedia.org/w/index.php?title=User:Winfree\&action=edit\&redlink=1}{Winfree}&\\ \hline 
\href{https://en.wikibooks.org/wiki/File:LaTeX-smallmatrix.png}{86}& \myhref{http://commons.wikimedia.org/w/index.php?title=User:Tomato86\&action=edit\&redlink=1}{Tomato86}, \myhref{https://commons.wikimedia.org/w/index.php?title=User:Tomato86\&action=edit\&redlink=1}{Tomato86}&GFDL\\ \hline 
\href{https://en.wikibooks.org/wiki/File:Mathscr\%20(A-F).png}{87}& \myhref{http://commons.wikimedia.org/wiki/User:Waldir}{Waldir}, \myhref{https://commons.wikimedia.org/wiki/User:Waldir}{Waldir}&GFDL\\ \hline 
\href{https://en.wikibooks.org/wiki/File:LaTeX\%20Dotsc.png}{88}& \myhref{http://commons.wikimedia.org/wiki/User:Neet}{Neet}, \myhref{https://commons.wikimedia.org/wiki/User:Neet}{Neet}&\\ \hline 
\href{https://en.wikibooks.org/wiki/File:LaTeX\%20Dotsb.png}{89}& \myhref{http://commons.wikimedia.org/wiki/User:Neet}{Neet}, \myhref{https://commons.wikimedia.org/wiki/User:Neet}{Neet}&\\ \hline 
\href{https://en.wikibooks.org/wiki/File:LaTeX\%20Dotsm.png}{90}& \myhref{http://commons.wikimedia.org/wiki/User:Neet}{Neet}, \myhref{https://commons.wikimedia.org/wiki/User:Neet}{Neet}&\\ \hline 
\href{https://en.wikibooks.org/wiki/File:LaTeX\%20Dotsi.png}{91}& \myhref{http://commons.wikimedia.org/wiki/User:Neet}{Neet}, \myhref{https://commons.wikimedia.org/wiki/User:Neet}{Neet}&\\ \hline 
\href{https://en.wikibooks.org/wiki/File:LaTeX\%20Dotso.png}{92}& \myhref{http://commons.wikimedia.org/wiki/User:Neet}{Neet}, \myhref{https://commons.wikimedia.org/wiki/User:Neet}{Neet}&\\ \hline 
\href{https://en.wikibooks.org/wiki/File:LaTeX-mathclap-example.png}{93}& \myhref{http://commons.wikimedia.org/w/index.php?title=User:Tomato86\&action=edit\&redlink=1}{Tomato86}, \myhref{https://commons.wikimedia.org/w/index.php?title=User:Tomato86\&action=edit\&redlink=1}{Tomato86}&GFDL\\ \hline 
\href{https://en.wikibooks.org/wiki/File:LaTeX-mathtools-brackets.png}{94}& \myhref{http://commons.wikimedia.org/w/index.php?title=User:Tomato86\&action=edit\&redlink=1}{Tomato86}, \myhref{https://commons.wikimedia.org/w/index.php?title=User:Tomato86\&action=edit\&redlink=1}{Tomato86}&GFDL\\ \hline 
\href{https://en.wikibooks.org/wiki/File:LaTeX-mathtools-arrows.png}{95}& \myhref{http://commons.wikimedia.org/w/index.php?title=User:Tomato86\&action=edit\&redlink=1}{Tomato86}, \myhref{https://commons.wikimedia.org/w/index.php?title=User:Tomato86\&action=edit\&redlink=1}{Tomato86}&GFDL\\ \hline 
\href{https://en.wikibooks.org/wiki/File:LaTeX-mathtools-harpoons.png}{96}& \myhref{http://commons.wikimedia.org/w/index.php?title=User:Tomato86\&action=edit\&redlink=1}{Tomato86}, \myhref{https://commons.wikimedia.org/w/index.php?title=User:Tomato86\&action=edit\&redlink=1}{Tomato86}&GFDL\\ \hline 
\href{https://en.wikibooks.org/wiki/File:LaTeX\%20example\%20split\%20gather.png}{97}& \myhref{http://commons.wikimedia.org/wiki/User:ChrisHodgesUK}{ChrisHodgesUK}, \myhref{https://commons.wikimedia.org/wiki/User:ChrisHodgesUK}{ChrisHodgesUK}&PD\\ \hline 
\href{https://en.wikibooks.org/wiki/File:LaTeX\%20-\%20Indented\%20Equations.png}{98}& \myhref{http://commons.wikimedia.org/wiki/User:Inductiveload}{Inductiveload}&\\ \hline 
\href{https://en.wikibooks.org/wiki/File:LaTeX-displaybreak-in-math.png}{99}& \myhref{http://commons.wikimedia.org/w/index.php?title=User:Tomato86\&action=edit\&redlink=1}{Tomato86}, \myhref{https://commons.wikimedia.org/w/index.php?title=User:Tomato86\&action=edit\&redlink=1}{Tomato86}&GFDL\\ \hline 
\href{https://en.wikibooks.org/wiki/File:LaTeX-boxed-equation.png}{100}& \myhref{http://commons.wikimedia.org/w/index.php?title=User:Tomato86\&action=edit\&redlink=1}{Tomato86}&GFDL\\ \hline 
\href{https://en.wikibooks.org/wiki/File:LaTeX-boxed-formula-minipage.png}{101}& \myhref{http://commons.wikimedia.org/w/index.php?title=User:Tomato86\&action=edit\&redlink=1}{Tomato86}, \myhref{https://commons.wikimedia.org/w/index.php?title=User:Tomato86\&action=edit\&redlink=1}{Tomato86}&GFDL\\ \hline 
\href{https://en.wikibooks.org/wiki/File:Latex-Aboxed-example.png}{102}& \myhref{http://commons.wikimedia.org/wiki/User:ChrisHodgesUK}{ChrisHodgesUK}, \myhref{https://commons.wikimedia.org/wiki/User:ChrisHodgesUK}{ChrisHodgesUK}&PD\\ \hline 
\href{https://en.wikibooks.org/wiki/File:Latex-intertext.png}{103}& \myhref{http://commons.wikimedia.org/w/index.php?title=User:Tomato86\&action=edit\&redlink=1}{Tomato86}, \myhref{https://commons.wikimedia.org/w/index.php?title=User:Tomato86\&action=edit\&redlink=1}{Tomato86}&GFDL\\ \hline 
\href{https://en.wikibooks.org/wiki/File:Chemfig_angles.png}{104}& \myhref{http://commons.wikimedia.org/w/index.php?title=User:Pmillerrhodes\&action=edit\&redlink=1}{Pmillerrhodes}, \myhref{https://commons.wikimedia.org/w/index.php?title=User:Pmillerrhodes\&action=edit\&redlink=1}{Pmillerrhodes}&CC-BY-SA-3.0\\ \hline 
\href{https://en.wikibooks.org/wiki/File:Chemfig_bonds.png}{105}& \myhref{http://commons.wikimedia.org/w/index.php?title=User:Pmillerrhodes\&action=edit\&redlink=1}{Pmillerrhodes}, \myhref{https://commons.wikimedia.org/w/index.php?title=User:Pmillerrhodes\&action=edit\&redlink=1}{Pmillerrhodes}&CC-BY-SA-3.0\\ \hline 
\href{https://en.wikibooks.org/wiki/File:Methane_chemfig.png}{106}& \myhref{http://commons.wikimedia.org/w/index.php?title=User:Pmillerrhodes\&action=edit\&redlink=1}{Pmillerrhodes}, \myhref{https://commons.wikimedia.org/w/index.php?title=User:Pmillerrhodes\&action=edit\&redlink=1}{Pmillerrhodes}&CC-BY-SA-3.0\\ \hline 
\href{https://en.wikibooks.org/wiki/File:Butane-skeletal.png}{107}& Ben Mills&\\ \hline 
\href{https://en.wikibooks.org/wiki/File:Skeletondiagram2.png}{108}& \myhref{http://commons.wikimedia.org/w/index.php?title=User:Pmillerrhodes\&action=edit\&redlink=1}{Pmillerrhodes}, \myhref{https://commons.wikimedia.org/w/index.php?title=User:Pmillerrhodes\&action=edit\&redlink=1}{Pmillerrhodes}&CC-BY-SA-3.0\\ \hline 
\href{https://en.wikibooks.org/wiki/File:Ring_chemfig.png}{109}& \myhref{http://commons.wikimedia.org/w/index.php?title=User:Pmillerrhodes\&action=edit\&redlink=1}{Pmillerrhodes}, \myhref{https://commons.wikimedia.org/w/index.php?title=User:Pmillerrhodes\&action=edit\&redlink=1}{Pmillerrhodes}&CC-BY-SA-3.0\\ \hline 
\href{https://en.wikibooks.org/wiki/File:Ring2_chemfig.png}{110}& \myhref{http://commons.wikimedia.org/w/index.php?title=User:Pmillerrhodes\&action=edit\&redlink=1}{Pmillerrhodes}, \myhref{https://commons.wikimedia.org/w/index.php?title=User:Pmillerrhodes\&action=edit\&redlink=1}{Pmillerrhodes}&CC-BY-SA-3.0\\ \hline 
\href{https://en.wikibooks.org/wiki/File:Ring3_chemfig.png}{111}& \myhref{http://commons.wikimedia.org/w/index.php?title=User:Pmillerrhodes\&action=edit\&redlink=1}{Pmillerrhodes}, \myhref{https://commons.wikimedia.org/w/index.php?title=User:Pmillerrhodes\&action=edit\&redlink=1}{Pmillerrhodes}&CC-BY-SA-3.0\\ \hline 
\href{https://en.wikibooks.org/wiki/File:Ring4_chemfig.png}{112}& \myhref{http://commons.wikimedia.org/w/index.php?title=User:Pmillerrhodes\&action=edit\&redlink=1}{Pmillerrhodes}, \myhref{https://commons.wikimedia.org/w/index.php?title=User:Pmillerrhodes\&action=edit\&redlink=1}{Pmillerrhodes}&CC-BY-SA-3.0\\ \hline 
\href{https://en.wikibooks.org/wiki/File:Carbon\%20Lewis\%20Structure\%20PNG.png}{113}& \myhref{http://commons.wikimedia.org/w/index.php?title=User:Daviewales\&action=edit\&redlink=1}{Daviewales}, \myhref{https://commons.wikimedia.org/w/index.php?title=User:Daviewales\&action=edit\&redlink=1}{Daviewales}&\\ \hline 
\href{https://en.wikibooks.org/wiki/File:H2O\%20Lewis\%20Structure\%20PNG.png}{114}& \myhref{http://commons.wikimedia.org/w/index.php?title=User:Daviewales\&action=edit\&redlink=1}{Daviewales}, \myhref{https://commons.wikimedia.org/w/index.php?title=User:Daviewales\&action=edit\&redlink=1}{Daviewales}&\\ \hline 
\href{https://en.wikibooks.org/wiki/File:Acetate-ion2.png}{115}& \myhref{http://commons.wikimedia.org/w/index.php?title=User:Pmillerrhodes\&action=edit\&redlink=1}{Pmillerrhodes}, \myhref{https://commons.wikimedia.org/w/index.php?title=User:Pmillerrhodes\&action=edit\&redlink=1}{Pmillerrhodes}&CC-BY-SA-3.0\\ \hline 
\href{https://en.wikibooks.org/wiki/File:Acetate-ion.png}{116}& \myhref{http://commons.wikimedia.org/w/index.php?title=User:Pmillerrhodes\&action=edit\&redlink=1}{Pmillerrhodes}, \myhref{https://commons.wikimedia.org/w/index.php?title=User:Pmillerrhodes\&action=edit\&redlink=1}{Pmillerrhodes}&CC-BY-SA-3.0\\ \hline 
\href{https://commons.wikimedia.org/wiki/File:Ion-example.png}{117}& No machine-{}readable author provided. \myhref{http://commons.wikimedia.org/w/index.php?title=User:Pmillerrhodes\&action=edit\&redlink=1}{Pmillerrhodes} assumed (based on copyright claims)., No machine-{}readable author provided. \myhref{https://commons.wikimedia.org/w/index.php?title=User:Pmillerrhodes\&action=edit\&redlink=1}{Pmillerrhodes} assumed (based on copyright claims).&CC-BY-SA-3.0\\ \hline 
\href{https://en.wikibooks.org/wiki/File:Chemfig\%20Arrow\%20Examples.png}{118}& \myhref{http://commons.wikimedia.org/wiki/User:Clapsus}{Clapsus}, \myhref{https://commons.wikimedia.org/wiki/User:Clapsus}{Clapsus}&\\ \hline 
\href{https://en.wikibooks.org/wiki/File:Corticosterone\%20(1).png}{119}& The original uploader was \myhref{http://en.wikipedia.org/wiki/User:Iorsh}{Iorsh} at \myhref{http://en.wikipedia.org/wiki/}{English Wikipedia}&PD\\ \hline 
\href{https://commons.wikimedia.org/wiki/File:Latex-algorithm2e-if-else.png}{120}& \myhref{http://commons.wikimedia.org/w/index.php?title=User:Lavaka\&action=edit\&redlink=1}{Lavaka}, \myhref{https://commons.wikimedia.org/w/index.php?title=User:Lavaka\&action=edit\&redlink=1}{Lavaka}&CC-BY-SA-3.0\\ \hline 
\href{https://en.wikibooks.org/wiki/File:Latex-algorithmic-if-else.png}{121}& \myhref{http://commons.wikimedia.org/wiki/User:Nemti}{Nemti}, \myhref{https://commons.wikimedia.org/wiki/User:Nemti}{Nemti}&\\ \hline 
\href{https://en.wikibooks.org/wiki/File:LaTeX_program_package_example01.png}{122}& MyName (\myhref{http://commons.wikimedia.org/wiki/User:Gkc}{Gkc} (\myhref{http://commons.wikimedia.org/wiki/User_talk:Gkc}{talk})), MyName (\myhref{https://commons.wikimedia.org/wiki/User:Gkc}{Gkc} (\myhref{https://commons.wikimedia.org/wiki/User_talk:Gkc}{talk}))&GFDL\\ \hline 
\href{https://en.wikibooks.org/wiki/File:Latex\%20Pascal\%20Listing.png}{123}& LaTeX / GIMP&CC-BY-3.0\\ \hline 
\href{https://en.wikibooks.org/wiki/File:Listings\%20Example.svg}{124}& \myhref{http://commons.wikimedia.org/wiki/User:Ambrevar}{Ambrevar}, \myhref{https://commons.wikimedia.org/wiki/User:Ambrevar}{Ambrevar}&CC-BY-SA-3.0\\ \hline 
\href{https://en.wikibooks.org/wiki/File:Gb4e1.png}{125}& jeg&PD\\ \hline 
\href{https://en.wikibooks.org/wiki/File:Gb4e2.png}{126}& \myhref{http://commons.wikimedia.org/w/index.php?title=User:Hankjones\&action=edit\&redlink=1}{Hankjones}, \myhref{https://commons.wikimedia.org/w/index.php?title=User:Hankjones\&action=edit\&redlink=1}{Hankjones}&GFDL\\ \hline 
\href{https://en.wikibooks.org/wiki/File:Gb4e3.png}{127}& \myhref{http://commons.wikimedia.org/w/index.php?title=User:Hankjones\&action=edit\&redlink=1}{Hankjones}, \myhref{https://commons.wikimedia.org/w/index.php?title=User:Hankjones\&action=edit\&redlink=1}{Hankjones}&CC-BY-SA-3.0\\ \hline 
\href{https://en.wikibooks.org/wiki/File:Gb4e4.png}{128}& \myhref{http://commons.wikimedia.org/w/index.php?title=User:Hankjones\&action=edit\&redlink=1}{Hankjones}, \myhref{https://commons.wikimedia.org/w/index.php?title=User:Hankjones\&action=edit\&redlink=1}{Hankjones}&CC-BY-SA-3.0\\ \hline 
\href{https://en.wikibooks.org/wiki/File:Gb4e1.png}{129}& jeg&PD\\ \hline 
\href{https://en.wikibooks.org/wiki/File:Gb4e2.png}{130}& \myhref{http://commons.wikimedia.org/w/index.php?title=User:Hankjones\&action=edit\&redlink=1}{Hankjones}, \myhref{https://commons.wikimedia.org/w/index.php?title=User:Hankjones\&action=edit\&redlink=1}{Hankjones}&GFDL\\ \hline 
\href{https://en.wikibooks.org/wiki/File:Gb4e3.png}{131}& \myhref{http://commons.wikimedia.org/w/index.php?title=User:Hankjones\&action=edit\&redlink=1}{Hankjones}, \myhref{https://commons.wikimedia.org/w/index.php?title=User:Hankjones\&action=edit\&redlink=1}{Hankjones}&CC-BY-SA-3.0\\ \hline 
\href{https://en.wikibooks.org/wiki/File:Latex_qtree_simple_tree.png}{132}& \myhref{http://commons.wikimedia.org/w/index.php?title=User:PhilJohnG\&action=edit\&redlink=1}{Philip John Gorinski}, \myhref{https://commons.wikimedia.org/w/index.php?title=User:PhilJohnG\&action=edit\&redlink=1}{Philip John Gorinski}&PD\\ \hline 
\href{https://en.wikibooks.org/wiki/File:Latex_qtree_simple_tree.png}{133}& \myhref{http://commons.wikimedia.org/w/index.php?title=User:PhilJohnG\&action=edit\&redlink=1}{Philip John Gorinski}, \myhref{https://commons.wikimedia.org/w/index.php?title=User:PhilJohnG\&action=edit\&redlink=1}{Philip John Gorinski}&PD\\ \hline 
\href{https://en.wikibooks.org/wiki/File:Simple\%20sideways\%20tree.png}{134}& \myhref{http://commons.wikimedia.org/w/index.php?title=User:PhilJohnG\&action=edit\&redlink=1}{Philip John Gorinski}, \myhref{https://commons.wikimedia.org/w/index.php?title=User:PhilJohnG\&action=edit\&redlink=1}{Philip John Gorinski}&PD\\ \hline 
\href{https://en.wikibooks.org/wiki/File:Latex-tikz-simple-deptree.png}{135}& Matěj Korvas&\\ \hline 
\href{https://en.wikibooks.org/wiki/File:Latex-xyling-simple-deptree.png}{136}& \myhref{http://commons.wikimedia.org/w/index.php?title=User:Matej.korvas\&action=edit\&redlink=1}{Matej.korvas}, \myhref{https://commons.wikimedia.org/w/index.php?title=User:Matej.korvas\&action=edit\&redlink=1}{Matej.korvas}&PD\\ \hline 
\href{https://en.wikibooks.org/wiki/File:Latex-xyling-simple-text-with-deptree.png}{137}& \myhref{http://commons.wikimedia.org/w/index.php?title=User:Matej.korvas\&action=edit\&redlink=1}{Matej.korvas}, \myhref{https://commons.wikimedia.org/w/index.php?title=User:Matej.korvas\&action=edit\&redlink=1}{Matej.korvas}&PD\\ \hline 
\href{https://en.wikibooks.org/wiki/File:Latex-dependency-parse-example-with-tikz-dependency.png}{138}& Daniele Pighin&GFDL\\ \hline 
\href{https://en.wikibooks.org/wiki/File:cat-eats-cream.png}{139}& \myhref{http://commons.wikimedia.org/w/index.php?title=User:Olesh\&action=edit\&redlink=1}{Olesh}, \myhref{https://commons.wikimedia.org/w/index.php?title=User:Olesh\&action=edit\&redlink=1}{Olesh}&CC-BY-SA-3.0\\ \hline 
\href{https://en.wikibooks.org/wiki/File:Ipa1.png}{140}& \myhref{http://commons.wikimedia.org/w/index.php?title=User:Hankjones\&action=edit\&redlink=1}{Hankjones}, \myhref{https://commons.wikimedia.org/w/index.php?title=User:Hankjones\&action=edit\&redlink=1}{Hankjones}&CC-BY-SA-3.0\\ \hline 
\href{https://en.wikibooks.org/wiki/File:Ipa2.png}{141}& \myhref{http://commons.wikimedia.org/w/index.php?title=User:Hankjones\&action=edit\&redlink=1}{Hankjones}, \myhref{https://commons.wikimedia.org/w/index.php?title=User:Hankjones\&action=edit\&redlink=1}{Hankjones}&CC-BY-SA-3.0\\ \hline 
\href{https://en.wikibooks.org/wiki/File:Ipa7.png}{142}& \myhref{http://commons.wikimedia.org/wiki/User:ChrisHodgesUK}{ChrisHodgesUK}, \myhref{https://commons.wikimedia.org/wiki/User:ChrisHodgesUK}{ChrisHodgesUK}&PD\\ \hline 
\href{https://en.wikibooks.org/wiki/File:Ipa3.png}{143}& \myhref{http://commons.wikimedia.org/w/index.php?title=User:Hankjones\&action=edit\&redlink=1}{Hankjones}, \myhref{https://commons.wikimedia.org/w/index.php?title=User:Hankjones\&action=edit\&redlink=1}{Hankjones}&CC-BY-SA-3.0\\ \hline 
\href{https://en.wikibooks.org/wiki/File:Ipa4.png}{144}& \myhref{http://commons.wikimedia.org/w/index.php?title=User:Hankjones\&action=edit\&redlink=1}{Hankjones}, \myhref{https://commons.wikimedia.org/w/index.php?title=User:Hankjones\&action=edit\&redlink=1}{Hankjones}&CC-BY-SA-3.0\\ \hline 
\href{https://en.wikibooks.org/wiki/File:Ipa5.png}{145}& \myhref{http://commons.wikimedia.org/w/index.php?title=User:Hankjones\&action=edit\&redlink=1}{Hankjones}, \myhref{https://commons.wikimedia.org/w/index.php?title=User:Hankjones\&action=edit\&redlink=1}{Hankjones}&CC-BY-SA-3.0\\ \hline 
\href{https://en.wikibooks.org/wiki/File:Ipa6.png}{146}& \myhref{http://commons.wikimedia.org/w/index.php?title=User:Hankjones\&action=edit\&redlink=1}{Hankjones}, \myhref{https://commons.wikimedia.org/w/index.php?title=User:Hankjones\&action=edit\&redlink=1}{Hankjones}&CC-BY-SA-3.0\\ \hline 
\href{https://en.wikibooks.org/wiki/File:Phonrule\%20output\%20example.png}{147}& \myhref{http://commons.wikimedia.org/wiki/User:SynConlanger}{Stefano Coretta}, \myhref{https://commons.wikimedia.org/wiki/User:SynConlanger}{Stefano Coretta}&\\ \hline 
\href{https://en.wikibooks.org/wiki/File:LaTeX\%20bibliography\%20plain.svg}{148}& \myhref{http://commons.wikimedia.org/wiki/User:Dirk_H\%C3\%BCnniger}{Dirk Hünniger}, \myhref{https://commons.wikimedia.org/wiki/User:Dirk_H\%C3\%BCnniger}{Dirk Hünniger}&CC-BY-SA-3.0\\ \hline 
\href{https://en.wikibooks.org/wiki/File:LaTeX\%20bibliography\%20abbrv.svg}{149}& \myhref{http://commons.wikimedia.org/wiki/User:Dirk_H\%C3\%BCnniger}{Dirk Hünniger}, \myhref{https://commons.wikimedia.org/wiki/User:Dirk_H\%C3\%BCnniger}{Dirk Hünniger}&CC-BY-SA-3.0\\ \hline 
\href{https://en.wikibooks.org/wiki/File:LaTeX\%20bibliography\%20alpha.svg}{150}& \myhref{http://commons.wikimedia.org/wiki/User:Dirk_H\%C3\%BCnniger}{Dirk Hünniger}, \myhref{https://commons.wikimedia.org/wiki/User:Dirk_H\%C3\%BCnniger}{Dirk Hünniger}&CC-BY-SA-3.0\\ \hline 
\href{https://en.wikibooks.org/wiki/File:Literatur-Generator.jpg}{151}& \myhref{http://commons.wikimedia.org/w/index.php?title=User:Literaturgenerator\&action=edit\&redlink=1}{Literaturgenerator}, \myhref{https://commons.wikimedia.org/w/index.php?title=User:Literaturgenerator\&action=edit\&redlink=1}{Literaturgenerator}&GFDL\\ \hline 
\href{https://en.wikibooks.org/wiki/File:Jabref-2.2-screenshot.png}{152}& Emijrpbot, Hazard-Bot, JarektBot, MGA73bot2, Mwtoews, Patrick87, SieBot, Ö&\\ \hline 
\href{https://en.wikibooks.org/wiki/File:BibDesk-1.3.10-screenshot.png}{153}& Mijio&\\ \hline 
\href{https://en.wikibooks.org/wiki/File:Plainnatrefs2.png}{154}& \myhref{http://commons.wikimedia.org/w/index.php?title=User:Jimbotyson\&action=edit\&redlink=1}{Jimbotyson}, \myhref{https://commons.wikimedia.org/w/index.php?title=User:Jimbotyson\&action=edit\&redlink=1}{Jimbotyson}&GFDL\\ \hline 
\href{https://en.wikibooks.org/wiki/File:LaTeX-\%d0\%bf\%d0\%b8\%d1\%81\%d0\%bc\%d0\%be.png}{155}& \myhref{http://en.wikibooks.org/wiki/en:User:Derbeth}{Derbeth}.&GFDL\\ \hline 
156&&\\ \hline 
\href{https://en.wikibooks.org/wiki/File:Koma_env.png}{157}& Gms&\\ \hline 
\href{https://en.wikibooks.org/wiki/File:Frametitle_keyword_example.png}{158}& \myhref{http://commons.wikimedia.org/w/index.php?title=User:Neoriddle\&action=edit\&redlink=1}{Israel Buitron}, \myhref{https://commons.wikimedia.org/w/index.php?title=User:Neoriddle\&action=edit\&redlink=1}{Israel Buitron}&GFDL\\ \hline 
\href{https://en.wikibooks.org/wiki/File:Latex\%20Beamer\%20-\%20Columns\%20Example\%202.png}{159}& \myhref{http://commons.wikimedia.org/w/index.php?title=User:Flip\&action=edit\&redlink=1}{Flip}, \myhref{https://commons.wikimedia.org/w/index.php?title=User:Flip\&action=edit\&redlink=1}{Flip}&CC-BY-SA-3.0\\ \hline 
\href{https://en.wikibooks.org/wiki/File:Blocks\%20beamer\%20example.png}{160}& \myhref{http://commons.wikimedia.org/w/index.php?title=User:Neoriddle\&action=edit\&redlink=1}{Israel Buitron}, \myhref{https://commons.wikimedia.org/w/index.php?title=User:Neoriddle\&action=edit\&redlink=1}{Israel Buitron}&CC-BY-SA-3.0\\ \hline 
\href{https://en.wikibooks.org/wiki/File:Moderncv\%20banking\%20black.svg}{161}& \myhref{http://commons.wikimedia.org/wiki/User:Ambrevar}{Ambrevar}, \myhref{https://commons.wikimedia.org/wiki/User:Ambrevar}{Ambrevar}&PD\\ \hline 
\href{https://en.wikibooks.org/wiki/File:Moderncv\%20classic\%20green.svg}{162}& \myhref{http://commons.wikimedia.org/wiki/User:Ambrevar}{Ambrevar}, \myhref{https://commons.wikimedia.org/wiki/User:Ambrevar}{Ambrevar}&PD\\ \hline 
\href{https://en.wikibooks.org/wiki/File:Latex-xymatrix.png}{163}& \myhref{http://commons.wikimedia.org/wiki/User:ChrisHodgesUK}{ChrisHodgesUK}, \myhref{https://commons.wikimedia.org/wiki/User:ChrisHodgesUK}{ChrisHodgesUK}&PD\\ \hline 
\href{https://en.wikibooks.org/wiki/File:Latex_example_line_segments.png}{164}& \myhref{http://commons.wikimedia.org/wiki/User:Alejo2083}{Alessio Damato}, \myhref{https://commons.wikimedia.org/wiki/User:Alejo2083}{Alessio Damato}&GFDL\\ \hline 
\href{https://en.wikibooks.org/wiki/File:Latex_example_arrows.png}{165}& \myhref{http://commons.wikimedia.org/wiki/User:Alejo2083}{Alessio Damato}, \myhref{https://commons.wikimedia.org/wiki/User:Alejo2083}{Alessio Damato}&GFDL\\ \hline 
\href{https://en.wikibooks.org/wiki/File:Latex_example_circles.png}{166}& \myhref{http://commons.wikimedia.org/wiki/User:Alejo2083}{Alessio Damato}, \myhref{https://commons.wikimedia.org/wiki/User:Alejo2083}{Alessio Damato}&GFDL\\ \hline 
\href{https://en.wikibooks.org/wiki/File:Latex_example_text_formulas.png}{167}& \myhref{http://commons.wikimedia.org/wiki/User:Alejo2083}{Alessio Damato}, \myhref{https://commons.wikimedia.org/wiki/User:Alejo2083}{Alessio Damato}&GFDL\\ \hline 
\href{https://en.wikibooks.org/wiki/File:Latex_example_multiput.png}{168}& \myhref{http://commons.wikimedia.org/wiki/User:Alejo2083}{Alessio Damato}, \myhref{https://commons.wikimedia.org/wiki/User:Alejo2083}{Alessio Damato}&GFDL\\ \hline 
\href{https://en.wikibooks.org/wiki/File:Latex_example_ovals.png}{169}& \myhref{http://commons.wikimedia.org/wiki/User:Alejo2083}{Alessio Damato}, \myhref{https://commons.wikimedia.org/wiki/User:Alejo2083}{Alessio Damato}&GFDL\\ \hline 
\href{https://en.wikibooks.org/wiki/File:Latex_example_multiple_pics.png}{170}& \myhref{http://commons.wikimedia.org/wiki/User:Alejo2083}{Alessio Damato}, \myhref{https://commons.wikimedia.org/wiki/User:Alejo2083}{Alessio Damato}&GFDL\\ \hline 
\href{https://en.wikibooks.org/wiki/File:Latex_example_bezier.png}{171}& \myhref{http://commons.wikimedia.org/wiki/User:Alejo2083}{Alessio Damato}, \myhref{https://commons.wikimedia.org/wiki/User:Alejo2083}{Alessio Damato}&GFDL\\ \hline 
\href{https://en.wikibooks.org/wiki/File:Latex_example_catenary.png}{172}& \myhref{http://commons.wikimedia.org/wiki/User:Alejo2083}{Alessio Damato}, \myhref{https://commons.wikimedia.org/wiki/User:Alejo2083}{Alessio Damato}&GFDL\\ \hline 
\href{https://en.wikibooks.org/wiki/File:Latex_example_rapidity.png}{173}& \myhref{http://commons.wikimedia.org/wiki/User:Alejo2083}{Alessio Damato}, \myhref{https://commons.wikimedia.org/wiki/User:Alejo2083}{Alessio Damato}&GFDL\\ \hline 
\href{https://en.wikibooks.org/wiki/File:Neighbourhood\%20definition2.svg}{174}& 

\begin{myitemize}
\item{}\myhref{http://commons.wikimedia.org/wiki/File:Neighbourhood_definition2.png}{Neighbourhood\_definition2.png}: \myhref{http://commons.wikimedia.org/wiki/User:Wegmann}{Wegmann}
\item{}derivative work: \myhref{http://commons.wikimedia.org/w/index.php?title=User:Pablo_Castellanos\&action=edit\&redlink=1}{Pablo Castellanos} (\myhref{http://commons.wikimedia.org/wiki/User_talk:Pablo_Castellanos}{talk})

\end{myitemize}

, 

\begin{myitemize}
\item{}\myhref{https://commons.wikimedia.org/wiki/File:Neighbourhood_definition2.png}{Neighbourhood\_definition2.png}: \myhref{https://commons.wikimedia.org/wiki/User:Wegmann}{Wegmann}
\item{}derivative work: \myhref{https://commons.wikimedia.org/w/index.php?title=User:Pablo_Castellanos\&action=edit\&redlink=1}{Pablo Castellanos} (\myhref{https://commons.wikimedia.org/wiki/User_talk:Pablo_Castellanos}{talk})

\end{myitemize}

&GFDL\\ \hline 
\href{https://en.wikibooks.org/wiki/File:TikZ\%20Tutorial\%20-\%20straight\%20lines.svg}{175}& \myhref{http://commons.wikimedia.org/wiki/User:Nobelium}{Nobelium}, \myhref{https://commons.wikimedia.org/wiki/User:Nobelium}{Nobelium}&PD\\ \hline 
\href{https://en.wikibooks.org/wiki/File:TikZ\%20Tutorial\%20-\%20straight\%20lines\%20style\%20options.svg}{176}& \myhref{http://commons.wikimedia.org/wiki/User:Nobelium}{Nobelium}, \myhref{https://commons.wikimedia.org/wiki/User:Nobelium}{Nobelium}&PD\\ \hline 
\href{https://en.wikibooks.org/wiki/File:TikZ\%20Tutorial\%20-\%20closed\%20straight\%20lines.svg}{177}& \myhref{http://commons.wikimedia.org/wiki/User:Nobelium}{Nobelium}, \myhref{https://commons.wikimedia.org/wiki/User:Nobelium}{Nobelium}&PD\\ \hline 
\href{https://en.wikibooks.org/wiki/File:TikZ\%20Tutorial\%20-\%20two\%20straight\%20lines.svg}{178}& \myhref{http://commons.wikimedia.org/wiki/User:Nobelium}{Nobelium}&PD\\ \hline 
\href{https://en.wikibooks.org/wiki/File:Tikz\%20Tutorial\%20-\%20line\%20hori\%20vert.svg}{179}& \myhref{http://commons.wikimedia.org/w/index.php?title=User:KlasN\&action=edit\&redlink=1}{KlasN}, \myhref{https://commons.wikimedia.org/w/index.php?title=User:KlasN\&action=edit\&redlink=1}{KlasN}&\\ \hline 
\href{https://en.wikibooks.org/wiki/File:Tikz\%20Tutorial\%20-\%20line\%20vert\%20hori.svg}{180}& \myhref{http://commons.wikimedia.org/w/index.php?title=User:KlasN\&action=edit\&redlink=1}{KlasN}, \myhref{https://commons.wikimedia.org/w/index.php?title=User:KlasN\&action=edit\&redlink=1}{KlasN}&\\ \hline 
\href{https://en.wikibooks.org/wiki/File:TikZ\%20Tutorial\%20-\%20Bezier\%20curve.svg}{181}& \myhref{http://commons.wikimedia.org/wiki/User:Nobelium}{Nobelium}, \myhref{https://commons.wikimedia.org/wiki/User:Nobelium}{Nobelium}&PD\\ \hline 
\href{https://en.wikibooks.org/wiki/File:TikZ\%20Tutorial\%20-\%20bending.svg}{182}& \myhref{http://commons.wikimedia.org/wiki/User:Nobelium}{Nobelium}, \myhref{https://commons.wikimedia.org/wiki/User:Nobelium}{Nobelium}&\\ \hline 
\href{https://en.wikibooks.org/wiki/File:TikZ\%20Tutorial\%20-\%20filled\%20rectangle.svg}{183}& \myhref{http://commons.wikimedia.org/wiki/User:Nobelium}{Nobelium}, \myhref{https://commons.wikimedia.org/wiki/User:Nobelium}{Nobelium}&\\ \hline 
\href{https://en.wikibooks.org/wiki/File:TikZ\%20Tutorial\%20-\%20circles.svg}{184}& \myhref{http://commons.wikimedia.org/wiki/User:Nobelium}{Nobelium}, \myhref{https://commons.wikimedia.org/wiki/User:Nobelium}{Nobelium}&\\ \hline 
\href{https://en.wikibooks.org/wiki/File:TikZ\%20Tutorial\%20-\%20pie.svg}{185}& \myhref{http://commons.wikimedia.org/wiki/User:Nobelium}{Nobelium}, \myhref{https://commons.wikimedia.org/wiki/User:Nobelium}{Nobelium}&\\ \hline 
\href{https://en.wikibooks.org/wiki/File:TikZ\%20Tutorial\%20-\%20special\%20paths.svg}{186}& \myhref{http://commons.wikimedia.org/wiki/User:Nobelium}{Nobelium}, \myhref{https://commons.wikimedia.org/wiki/User:Nobelium}{Nobelium}&\\ \hline 
\href{https://en.wikibooks.org/wiki/File:TikZ\%20Tutorial\%20-\%20arrows.svg}{187}& \myhref{http://commons.wikimedia.org/wiki/User:Nobelium}{Nobelium}, \myhref{https://commons.wikimedia.org/wiki/User:Nobelium}{Nobelium}&\\ \hline 
\href{https://en.wikibooks.org/wiki/File:TikZ\%20Tutorial\%20-\%20foreach.svg}{188}& \myhref{http://commons.wikimedia.org/wiki/User:Nobelium}{Nobelium}, \myhref{https://commons.wikimedia.org/wiki/User:Nobelium}{Nobelium}&\\ \hline 
\href{https://en.wikibooks.org/wiki/File:TikZ\%20Tutorial\%20-\%20plots.svg}{189}& \myhref{http://commons.wikimedia.org/wiki/User:Nobelium}{Nobelium}, \myhref{https://commons.wikimedia.org/wiki/User:Nobelium}{Nobelium}&\\ \hline 
\href{https://en.wikibooks.org/wiki/File:Tikz\%20Tutorial\%20-\%203\%20nodes\%20and\%20dotted\%20lines.svg}{190}& \myhref{http://commons.wikimedia.org/w/index.php?title=User:KlasN\&action=edit\&redlink=1}{KlasN}, \myhref{https://commons.wikimedia.org/w/index.php?title=User:KlasN\&action=edit\&redlink=1}{KlasN}&\\ \hline 
\href{https://en.wikibooks.org/wiki/File:Tikz\%20Tutorial\%20-\%203\%20nodes\%20with\%20text.svg}{191}& \myhref{http://commons.wikimedia.org/w/index.php?title=User:KlasN\&action=edit\&redlink=1}{KlasN}, \myhref{https://commons.wikimedia.org/w/index.php?title=User:KlasN\&action=edit\&redlink=1}{KlasN}&\\ \hline 
\href{https://en.wikibooks.org/wiki/File:Tikz\%20Tutorial\%20-\%20nodes\%20at\%20fraction\%20position.svg}{192}& \myhref{http://commons.wikimedia.org/w/index.php?title=User:KlasN\&action=edit\&redlink=1}{KlasN}, \myhref{https://commons.wikimedia.org/w/index.php?title=User:KlasN\&action=edit\&redlink=1}{KlasN}&\\ \hline 
\href{https://en.wikibooks.org/wiki/File:Tikz\%20Tutorial\%20-\%20nodes\%20coordinates.svg}{193}& \myhref{http://commons.wikimedia.org/w/index.php?title=User:KlasN\&action=edit\&redlink=1}{KlasN}, \myhref{https://commons.wikimedia.org/w/index.php?title=User:KlasN\&action=edit\&redlink=1}{KlasN}&\\ \hline 
\href{https://en.wikibooks.org/wiki/File:Tikz\%20Tutorial\%20-\%20node\%20text\%20alignment.svg}{194}& \myhref{http://commons.wikimedia.org/w/index.php?title=User:KlasN\&action=edit\&redlink=1}{KlasN}, \myhref{https://commons.wikimedia.org/w/index.php?title=User:KlasN\&action=edit\&redlink=1}{KlasN}&\\ \hline 
\href{https://en.wikibooks.org/wiki/File:Tikz\%20Tutorial\%20-\%20clever\%20paths.svg}{195}& \myhref{http://commons.wikimedia.org/w/index.php?title=User:KlasN\&action=edit\&redlink=1}{KlasN}, \myhref{https://commons.wikimedia.org/w/index.php?title=User:KlasN\&action=edit\&redlink=1}{KlasN}&\\ \hline 
\href{https://en.wikibooks.org/wiki/File:Tikz\%20Tutorial\%20-\%20example\%201.svg}{196}& \myhref{http://commons.wikimedia.org/w/index.php?title=User:KlasN\&action=edit\&redlink=1}{KlasN}, \myhref{https://commons.wikimedia.org/w/index.php?title=User:KlasN\&action=edit\&redlink=1}{KlasN}&\\ \hline 
\href{https://en.wikibooks.org/wiki/File:Tikz\%20Tutorial\%20-\%20example\%202.svg}{197}& \myhref{http://commons.wikimedia.org/w/index.php?title=User:KlasN\&action=edit\&redlink=1}{KlasN}, \myhref{https://commons.wikimedia.org/w/index.php?title=User:KlasN\&action=edit\&redlink=1}{KlasN}&\\ \hline 
\href{https://en.wikibooks.org/wiki/File:Tikz\%20Tutorial\%20-\%20example\%203.svg}{198}& \myhref{http://commons.wikimedia.org/w/index.php?title=User:KlasN\&action=edit\&redlink=1}{KlasN}, \myhref{https://commons.wikimedia.org/w/index.php?title=User:KlasN\&action=edit\&redlink=1}{KlasN}&\\ \hline 
\href{https://en.wikibooks.org/wiki/File:Tikz\%20Tutorial\%20-\%20example\%204.svg}{199}& \myhref{http://commons.wikimedia.org/w/index.php?title=User:KlasN\&action=edit\&redlink=1}{KlasN}, \myhref{https://commons.wikimedia.org/w/index.php?title=User:KlasN\&action=edit\&redlink=1}{KlasN}&\\ \hline 
\href{https://en.wikibooks.org/wiki/File:Latex_example_xypics_basic.png}{200}& \myhref{http://commons.wikimedia.org/wiki/User:Alejo2083}{Alessio Damato}, \myhref{https://commons.wikimedia.org/wiki/User:Alejo2083}{Alessio Damato}&GFDL\\ \hline 
\href{https://en.wikibooks.org/wiki/File:Latex_example_xypics_arrows_1.png}{201}& \myhref{http://commons.wikimedia.org/wiki/User:Alejo2083}{Alessio Damato}, \myhref{https://commons.wikimedia.org/wiki/User:Alejo2083}{Alessio Damato}&GFDL\\ \hline 
\href{https://en.wikibooks.org/wiki/File:Latex_example_xypics_arrows_2.png}{202}& \myhref{http://commons.wikimedia.org/wiki/User:Alejo2083}{Alessio Damato}, \myhref{https://commons.wikimedia.org/wiki/User:Alejo2083}{Alessio Damato}&GFDL\\ \hline 
\href{https://en.wikibooks.org/wiki/File:Latex_example_xypics_arrows_3.png}{203}& \myhref{http://commons.wikimedia.org/wiki/User:Alejo2083}{Alessio Damato}, \myhref{https://commons.wikimedia.org/wiki/User:Alejo2083}{Alessio Damato}&GFDL\\ \hline 
\href{https://en.wikibooks.org/wiki/File:Latex_example_xypics_arrows_labels.png}{204}& \myhref{http://commons.wikimedia.org/wiki/User:Alejo2083}{Alessio Damato}, \myhref{https://commons.wikimedia.org/wiki/User:Alejo2083}{Alessio Damato}&GFDL\\ \hline 
\href{https://en.wikibooks.org/wiki/File:Latex_example_xypics_inarrow_labels.png}{205}& \myhref{http://commons.wikimedia.org/wiki/User:Alejo2083}{Alessio Damato}, \myhref{https://commons.wikimedia.org/wiki/User:Alejo2083}{Alessio Damato}&GFDL\\ \hline 
\href{https://en.wikibooks.org/wiki/File:Latex_example_xypics_arrow_list.png}{206}& \myhref{http://commons.wikimedia.org/wiki/User:Alejo2083}{Alessio Damato}, \myhref{https://commons.wikimedia.org/wiki/User:Alejo2083}{Alessio Damato}&GFDL\\ \hline 
\href{https://en.wikibooks.org/wiki/File:Latex_example_xypics_standard_arrow.png}{207}& \myhref{http://commons.wikimedia.org/wiki/User:Alejo2083}{Alessio Damato}, \myhref{https://commons.wikimedia.org/wiki/User:Alejo2083}{Alessio Damato}&GFDL\\ \hline 
\href{https://en.wikibooks.org/wiki/File:Latex\%20example\%20xypics\%20curved\%20arrow.png}{208}& \myhref{http://commons.wikimedia.org/wiki/User:Alejo2083}{Alessio Damato}, \myhref{https://commons.wikimedia.org/wiki/User:Alejo2083}{Alessio Damato}&GFDL\\ \hline 
\href{https://en.wikibooks.org/wiki/File:Latex\%20example\%20newenvironment.png}{209}& \myhref{http://commons.wikimedia.org/wiki/User:Alejo2083}{Alessio Damato}, \myhref{https://commons.wikimedia.org/wiki/User:Alejo2083}{Alessio Damato}&GFDL\\ \hline 
\href{https://en.wikibooks.org/wiki/File:Latex-theme-toc.svg}{210}& \myhref{http://commons.wikimedia.org/wiki/User:Ambrevar}{Ambrevar}, \myhref{https://commons.wikimedia.org/wiki/User:Ambrevar}{Ambrevar}&PD\\ \hline 
\href{https://en.wikibooks.org/wiki/File:Latex-theme-sections.svg}{211}& \myhref{http://commons.wikimedia.org/wiki/User:Ambrevar}{Ambrevar}, \myhref{https://commons.wikimedia.org/wiki/User:Ambrevar}{Ambrevar}&PD\\ \hline 
\href{https://en.wikibooks.org/wiki/File:Latex-theme-bloody.svg}{212}& \myhref{http://commons.wikimedia.org/wiki/User:Ambrevar}{Ambrevar}, \myhref{https://commons.wikimedia.org/wiki/User:Ambrevar}{Ambrevar}&PD\\ \hline 
\href{https://en.wikibooks.org/wiki/File:ESvn-texmf.png}{213}& Original uploader was \myhref{http://en.wikibooks.org/wiki/en:User:Arnehe}{Arnehe} at \myhref{http://en.wikibooks.org}{en.wikibooks}&GPL\\ \hline 
\href{https://en.wikibooks.org/wiki/File:Kdiff3-modification.png}{214}& The original uploader was \myhref{http://en.wikibooks.org/wiki/User:Arnehe}{Arnehe} at \myhref{http://en.wikibooks.org/wiki/}{English Wikibooks}&GPL\\ \hline 
\href{https://en.wikibooks.org/wiki/File:JabRef-KeyPattern.png}{215}& Original uploader was \myhref{http://en.wikibooks.org/wiki/en:User:Arnehe}{Arnehe} at \myhref{http://en.wikibooks.org}{en.wikibooks}&GPL\\ \hline 
\href{https://en.wikibooks.org/wiki/File:JabRef-GeneralFields.png}{216}& The original uploader was \myhref{http://en.wikibooks.org/wiki/User:Arnehe}{Arnehe} at \myhref{http://en.wikibooks.org/wiki/}{English Wikibooks}&GPL\\ \hline 
\href{https://en.wikibooks.org/wiki/File:JabRef-ExternalPrograms.png}{217}& The original uploader was \myhref{http://en.wikibooks.org/wiki/User:Arnehe}{Arnehe} at \myhref{http://en.wikibooks.org/wiki/}{English Wikibooks}&GPL\\ \hline 
\href{https://en.wikibooks.org/wiki/File:Soliton_2nd_order.svg}{218}& \myhref{http://commons.wikimedia.org/wiki/User:Alejo2083}{Alessio Damato}, \myhref{https://commons.wikimedia.org/wiki/User:Alejo2083}{Alessio Damato}&GFDL\\ \hline 
\href{https://en.wikibooks.org/wiki/File:Latex_showkeys_example.png}{219}& Alessio Damato&GFDL\\ \hline 

\end{longtable}
\pagebreak\pagebreak

\printindex

\KOMAoptions{fontsize=9pt,DIV=90,BCOR=0pt} 
\pagebreak
\chapter{Licenses}
\label{Licenses}
{\tiny
\section {GNU GENERAL PUBLIC LICENSE}
\begin{multicols}{4}

Version 3, 29 June 2007

Copyright © 2007 Free Software Foundation, Inc. <http://fsf.org/>

Everyone is permitted to copy and distribute verbatim copies of this license document, but changing it is not allowed.
Preamble

The GNU General Public License is a free, copyleft license for software and other kinds of works.

The licenses for most software and other practical works are designed to take away your freedom to share and change the works. By contrast, the GNU General Public License is intended to guarantee your freedom to share and change all versions of a program--to make sure it remains free software for all its users. We, the Free Software Foundation, use the GNU General Public License for most of our software; it applies also to any other work released this way by its authors. You can apply it to your programs, too.

When we speak of free software, we are referring to freedom, not price. Our General Public Licenses are designed to make sure that you have the freedom to distribute copies of free software (and charge for them if you wish), that you receive source code or can get it if you want it, that you can change the software or use pieces of it in new free programs, and that you know you can do these things.

To protect your rights, we need to prevent others from denying you these rights or asking you to surrender the rights. Therefore, you have certain responsibilities if you distribute copies of the software, or if you modify it: responsibilities to respect the freedom of others.

For example, if you distribute copies of such a program, whether gratis or for a fee, you must pass on to the recipients the same freedoms that you received. You must make sure that they, too, receive or can get the source code. And you must show them these terms so they know their rights.

Developers that use the GNU GPL protect your rights with two steps: (1) assert copyright on the software, and (2) offer you this License giving you legal permission to copy, distribute and/or modify it.

For the developers' and authors' protection, the GPL clearly explains that there is no warranty for this free software. For both users' and authors' sake, the GPL requires that modified versions be marked as changed, so that their problems will not be attributed erroneously to authors of previous versions.

Some devices are designed to deny users access to install or run modified versions of the software inside them, although the manufacturer can do so. This is fundamentally incompatible with the aim of protecting users' freedom to change the software. The systematic pattern of such abuse occurs in the area of products for individuals to use, which is precisely where it is most unacceptable. Therefore, we have designed this version of the GPL to prohibit the practice for those products. If such problems arise substantially in other domains, we stand ready to extend this provision to those domains in future versions of the GPL, as needed to protect the freedom of users.

Finally, every program is threatened constantly by software patents. States should not allow patents to restrict development and use of software on general-purpose computers, but in those that do, we wish to avoid the special danger that patents applied to a free program could make it effectively proprietary. To prevent this, the GPL assures that patents cannot be used to render the program non-free.

The precise terms and conditions for copying, distribution and modification follow.
TERMS AND CONDITIONS
0. Definitions.

“This License” refers to version 3 of the GNU General Public License.

“Copyright” also means copyright-like laws that apply to other kinds of works, such as semiconductor masks.

“The Program” refers to any copyrightable work licensed under this License. Each licensee is addressed as “you”. “Licensees” and “recipients” may be individuals or organizations.

To “modify” a work means to copy from or adapt all or part of the work in a fashion requiring copyright permission, other than the making of an exact copy. The resulting work is called a “modified version” of the earlier work or a work “based on” the earlier work.

A “covered work” means either the unmodified Program or a work based on the Program.

To “propagate” a work means to do anything with it that, without permission, would make you directly or secondarily liable for infringement under applicable copyright law, except executing it on a computer or modifying a private copy. Propagation includes copying, distribution (with or without modification), making available to the public, and in some countries other activities as well.

To “convey” a work means any kind of propagation that enables other parties to make or receive copies. Mere interaction with a user through a computer network, with no transfer of a copy, is not conveying.

An interactive user interface displays “Appropriate Legal Notices” to the extent that it includes a convenient and prominently visible feature that (1) displays an appropriate copyright notice, and (2) tells the user that there is no warranty for the work (except to the extent that warranties are provided), that licensees may convey the work under this License, and how to view a copy of this License. If the interface presents a list of user commands or options, such as a menu, a prominent item in the list meets this criterion.
1. Source Code.

The “source code” for a work means the preferred form of the work for making modifications to it. “Object code” means any non-source form of a work.

A “Standard Interface” means an interface that either is an official standard defined by a recognized standards body, or, in the case of interfaces specified for a particular programming language, one that is widely used among developers working in that language.

The “System Libraries” of an executable work include anything, other than the work as a whole, that (a) is included in the normal form of packaging a Major Component, but which is not part of that Major Component, and (b) serves only to enable use of the work with that Major Component, or to implement a Standard Interface for which an implementation is available to the public in source code form. A “Major Component”, in this context, means a major essential component (kernel, window system, and so on) of the specific operating system (if any) on which the executable work runs, or a compiler used to produce the work, or an object code interpreter used to run it.

The “Corresponding Source” for a work in object code form means all the source code needed to generate, install, and (for an executable work) run the object code and to modify the work, including scripts to control those activities. However, it does not include the work's System Libraries, or general-purpose tools or generally available free programs which are used unmodified in performing those activities but which are not part of the work. For example, Corresponding Source includes interface definition files associated with source files for the work, and the source code for shared libraries and dynamically linked subprograms that the work is specifically designed to require, such as by intimate data communication or control flow between those subprograms and other parts of the work.

The Corresponding Source need not include anything that users can regenerate automatically from other parts of the Corresponding Source.

The Corresponding Source for a work in source code form is that same work.
2. Basic Permissions.

All rights granted under this License are granted for the term of copyright on the Program, and are irrevocable provided the stated conditions are met. This License explicitly affirms your unlimited permission to run the unmodified Program. The output from running a covered work is covered by this License only if the output, given its content, constitutes a covered work. This License acknowledges your rights of fair use or other equivalent, as provided by copyright law.

You may make, run and propagate covered works that you do not convey, without conditions so long as your license otherwise remains in force. You may convey covered works to others for the sole purpose of having them make modifications exclusively for you, or provide you with facilities for running those works, provided that you comply with the terms of this License in conveying all material for which you do not control copyright. Those thus making or running the covered works for you must do so exclusively on your behalf, under your direction and control, on terms that prohibit them from making any copies of your copyrighted material outside their relationship with you.

Conveying under any other circumstances is permitted solely under the conditions stated below. Sublicensing is not allowed; section 10 makes it unnecessary.
3. Protecting Users' Legal Rights From Anti-Circumvention Law.

No covered work shall be deemed part of an effective technological measure under any applicable law fulfilling obligations under article 11 of the WIPO copyright treaty adopted on 20 December 1996, or similar laws prohibiting or restricting circumvention of such measures.

When you convey a covered work, you waive any legal power to forbid circumvention of technological measures to the extent such circumvention is effected by exercising rights under this License with respect to the covered work, and you disclaim any intention to limit operation or modification of the work as a means of enforcing, against the work's users, your or third parties' legal rights to forbid circumvention of technological measures.
4. Conveying Verbatim Copies.

You may convey verbatim copies of the Program's source code as you receive it, in any medium, provided that you conspicuously and appropriately publish on each copy an appropriate copyright notice; keep intact all notices stating that this License and any non-permissive terms added in accord with section 7 apply to the code; keep intact all notices of the absence of any warranty; and give all recipients a copy of this License along with the Program.

You may charge any price or no price for each copy that you convey, and you may offer support or warranty protection for a fee.
5. Conveying Modified Source Versions.

You may convey a work based on the Program, or the modifications to produce it from the Program, in the form of source code under the terms of section 4, provided that you also meet all of these conditions:

    * a) The work must carry prominent notices stating that you modified it, and giving a relevant date.
    * b) The work must carry prominent notices stating that it is released under this License and any conditions added under section 7. This requirement modifies the requirement in section 4 to “keep intact all notices”.
    * c) You must license the entire work, as a whole, under this License to anyone who comes into possession of a copy. This License will therefore apply, along with any applicable section 7 additional terms, to the whole of the work, and all its parts, regardless of how they are packaged. This License gives no permission to license the work in any other way, but it does not invalidate such permission if you have separately received it.
    * d) If the work has interactive user interfaces, each must display Appropriate Legal Notices; however, if the Program has interactive interfaces that do not display Appropriate Legal Notices, your work need not make them do so.

A compilation of a covered work with other separate and independent works, which are not by their nature extensions of the covered work, and which are not combined with it such as to form a larger program, in or on a volume of a storage or distribution medium, is called an “aggregate” if the compilation and its resulting copyright are not used to limit the access or legal rights of the compilation's users beyond what the individual works permit. Inclusion of a covered work in an aggregate does not cause this License to apply to the other parts of the aggregate.
6. Conveying Non-Source Forms.

You may convey a covered work in object code form under the terms of sections 4 and 5, provided that you also convey the machine-readable Corresponding Source under the terms of this License, in one of these ways:

    * a) Convey the object code in, or embodied in, a physical product (including a physical distribution medium), accompanied by the Corresponding Source fixed on a durable physical medium customarily used for software interchange.
    * b) Convey the object code in, or embodied in, a physical product (including a physical distribution medium), accompanied by a written offer, valid for at least three years and valid for as long as you offer spare parts or customer support for that product model, to give anyone who possesses the object code either (1) a copy of the Corresponding Source for all the software in the product that is covered by this License, on a durable physical medium customarily used for software interchange, for a price no more than your reasonable cost of physically performing this conveying of source, or (2) access to copy the Corresponding Source from a network server at no charge.
    * c) Convey individual copies of the object code with a copy of the written offer to provide the Corresponding Source. This alternative is allowed only occasionally and noncommercially, and only if you received the object code with such an offer, in accord with subsection 6b.
    * d) Convey the object code by offering access from a designated place (gratis or for a charge), and offer equivalent access to the Corresponding Source in the same way through the same place at no further charge. You need not require recipients to copy the Corresponding Source along with the object code. If the place to copy the object code is a network server, the Corresponding Source may be on a different server (operated by you or a third party) that supports equivalent copying facilities, provided you maintain clear directions next to the object code saying where to find the Corresponding Source. Regardless of what server hosts the Corresponding Source, you remain obligated to ensure that it is available for as long as needed to satisfy these requirements.
    * e) Convey the object code using peer-to-peer transmission, provided you inform other peers where the object code and Corresponding Source of the work are being offered to the general public at no charge under subsection 6d.

A separable portion of the object code, whose source code is excluded from the Corresponding Source as a System Library, need not be included in conveying the object code work.

A “User Product” is either (1) a “consumer product”, which means any tangible personal property which is normally used for personal, family, or household purposes, or (2) anything designed or sold for incorporation into a dwelling. In determining whether a product is a consumer product, doubtful cases shall be resolved in favor of coverage. For a particular product received by a particular user, “normally used” refers to a typical or common use of that class of product, regardless of the status of the particular user or of the way in which the particular user actually uses, or expects or is expected to use, the product. A product is a consumer product regardless of whether the product has substantial commercial, industrial or non-consumer uses, unless such uses represent the only significant mode of use of the product.

“Installation Information” for a User Product means any methods, procedures, authorization keys, or other information required to install and execute modified versions of a covered work in that User Product from a modified version of its Corresponding Source. The information must suffice to ensure that the continued functioning of the modified object code is in no case prevented or interfered with solely because modification has been made.

If you convey an object code work under this section in, or with, or specifically for use in, a User Product, and the conveying occurs as part of a transaction in which the right of possession and use of the User Product is transferred to the recipient in perpetuity or for a fixed term (regardless of how the transaction is characterized), the Corresponding Source conveyed under this section must be accompanied by the Installation Information. But this requirement does not apply if neither you nor any third party retains the ability to install modified object code on the User Product (for example, the work has been installed in ROM).

The requirement to provide Installation Information does not include a requirement to continue to provide support service, warranty, or updates for a work that has been modified or installed by the recipient, or for the User Product in which it has been modified or installed. Access to a network may be denied when the modification itself materially and adversely affects the operation of the network or violates the rules and protocols for communication across the network.

Corresponding Source conveyed, and Installation Information provided, in accord with this section must be in a format that is publicly documented (and with an implementation available to the public in source code form), and must require no special password or key for unpacking, reading or copying.
7. Additional Terms.

“Additional permissions” are terms that supplement the terms of this License by making exceptions from one or more of its conditions. Additional permissions that are applicable to the entire Program shall be treated as though they were included in this License, to the extent that they are valid under applicable law. If additional permissions apply only to part of the Program, that part may be used separately under those permissions, but the entire Program remains governed by this License without regard to the additional permissions.

When you convey a copy of a covered work, you may at your option remove any additional permissions from that copy, or from any part of it. (Additional permissions may be written to require their own removal in certain cases when you modify the work.) You may place additional permissions on material, added by you to a covered work, for which you have or can give appropriate copyright permission.

Notwithstanding any other provision of this License, for material you add to a covered work, you may (if authorized by the copyright holders of that material) supplement the terms of this License with terms:

    * a) Disclaiming warranty or limiting liability differently from the terms of sections 15 and 16 of this License; or
    * b) Requiring preservation of specified reasonable legal notices or author attributions in that material or in the Appropriate Legal Notices displayed by works containing it; or
    * c) Prohibiting misrepresentation of the origin of that material, or requiring that modified versions of such material be marked in reasonable ways as different from the original version; or
    * d) Limiting the use for publicity purposes of names of licensors or authors of the material; or
    * e) Declining to grant rights under trademark law for use of some trade names, trademarks, or service marks; or
    * f) Requiring indemnification of licensors and authors of that material by anyone who conveys the material (or modified versions of it) with contractual assumptions of liability to the recipient, for any liability that these contractual assumptions directly impose on those licensors and authors.

All other non-permissive additional terms are considered “further restrictions” within the meaning of section 10. If the Program as you received it, or any part of it, contains a notice stating that it is governed by this License along with a term that is a further restriction, you may remove that term. If a license document contains a further restriction but permits relicensing or conveying under this License, you may add to a covered work material governed by the terms of that license document, provided that the further restriction does not survive such relicensing or conveying.

If you add terms to a covered work in accord with this section, you must place, in the relevant source files, a statement of the additional terms that apply to those files, or a notice indicating where to find the applicable terms.

Additional terms, permissive or non-permissive, may be stated in the form of a separately written license, or stated as exceptions; the above requirements apply either way.
8. Termination.

You may not propagate or modify a covered work except as expressly provided under this License. Any attempt otherwise to propagate or modify it is void, and will automatically terminate your rights under this License (including any patent licenses granted under the third paragraph of section 11).

However, if you cease all violation of this License, then your license from a particular copyright holder is reinstated (a) provisionally, unless and until the copyright holder explicitly and finally terminates your license, and (b) permanently, if the copyright holder fails to notify you of the violation by some reasonable means prior to 60 days after the cessation.

Moreover, your license from a particular copyright holder is reinstated permanently if the copyright holder notifies you of the violation by some reasonable means, this is the first time you have received notice of violation of this License (for any work) from that copyright holder, and you cure the violation prior to 30 days after your receipt of the notice.

Termination of your rights under this section does not terminate the licenses of parties who have received copies or rights from you under this License. If your rights have been terminated and not permanently reinstated, you do not qualify to receive new licenses for the same material under section 10.
9. Acceptance Not Required for Having Copies.

You are not required to accept this License in order to receive or run a copy of the Program. Ancillary propagation of a covered work occurring solely as a consequence of using peer-to-peer transmission to receive a copy likewise does not require acceptance. However, nothing other than this License grants you permission to propagate or modify any covered work. These actions infringe copyright if you do not accept this License. Therefore, by modifying or propagating a covered work, you indicate your acceptance of this License to do so.
10. Automatic Licensing of Downstream Recipients.

Each time you convey a covered work, the recipient automatically receives a license from the original licensors, to run, modify and propagate that work, subject to this License. You are not responsible for enforcing compliance by third parties with this License.

An “entity transaction” is a transaction transferring control of an organization, or substantially all assets of one, or subdividing an organization, or merging organizations. If propagation of a covered work results from an entity transaction, each party to that transaction who receives a copy of the work also receives whatever licenses to the work the party's predecessor in interest had or could give under the previous paragraph, plus a right to possession of the Corresponding Source of the work from the predecessor in interest, if the predecessor has it or can get it with reasonable efforts.

You may not impose any further restrictions on the exercise of the rights granted or affirmed under this License. For example, you may not impose a license fee, royalty, or other charge for exercise of rights granted under this License, and you may not initiate litigation (including a cross-claim or counterclaim in a lawsuit) alleging that any patent claim is infringed by making, using, selling, offering for sale, or importing the Program or any portion of it.
11. Patents.

A “contributor” is a copyright holder who authorizes use under this License of the Program or a work on which the Program is based. The work thus licensed is called the contributor's “contributor version”.

A contributor's “essential patent claims” are all patent claims owned or controlled by the contributor, whether already acquired or hereafter acquired, that would be infringed by some manner, permitted by this License, of making, using, or selling its contributor version, but do not include claims that would be infringed only as a consequence of further modification of the contributor version. For purposes of this definition, “control” includes the right to grant patent sublicenses in a manner consistent with the requirements of this License.

Each contributor grants you a non-exclusive, worldwide, royalty-free patent license under the contributor's essential patent claims, to make, use, sell, offer for sale, import and otherwise run, modify and propagate the contents of its contributor version.

In the following three paragraphs, a “patent license” is any express agreement or commitment, however denominated, not to enforce a patent (such as an express permission to practice a patent or covenant not to sue for patent infringement). To “grant” such a patent license to a party means to make such an agreement or commitment not to enforce a patent against the party.

If you convey a covered work, knowingly relying on a patent license, and the Corresponding Source of the work is not available for anyone to copy, free of charge and under the terms of this License, through a publicly available network server or other readily accessible means, then you must either (1) cause the Corresponding Source to be so available, or (2) arrange to deprive yourself of the benefit of the patent license for this particular work, or (3) arrange, in a manner consistent with the requirements of this License, to extend the patent license to downstream recipients. “Knowingly relying” means you have actual knowledge that, but for the patent license, your conveying the covered work in a country, or your recipient's use of the covered work in a country, would infringe one or more identifiable patents in that country that you have reason to believe are valid.

If, pursuant to or in connection with a single transaction or arrangement, you convey, or propagate by procuring conveyance of, a covered work, and grant a patent license to some of the parties receiving the covered work authorizing them to use, propagate, modify or convey a specific copy of the covered work, then the patent license you grant is automatically extended to all recipients of the covered work and works based on it.

A patent license is “discriminatory” if it does not include within the scope of its coverage, prohibits the exercise of, or is conditioned on the non-exercise of one or more of the rights that are specifically granted under this License. You may not convey a covered work if you are a party to an arrangement with a third party that is in the business of distributing software, under which you make payment to the third party based on the extent of your activity of conveying the work, and under which the third party grants, to any of the parties who would receive the covered work from you, a discriminatory patent license (a) in connection with copies of the covered work conveyed by you (or copies made from those copies), or (b) primarily for and in connection with specific products or compilations that contain the covered work, unless you entered into that arrangement, or that patent license was granted, prior to 28 March 2007.

Nothing in this License shall be construed as excluding or limiting any implied license or other defenses to infringement that may otherwise be available to you under applicable patent law.
12. No Surrender of Others' Freedom.

If conditions are imposed on you (whether by court order, agreement or otherwise) that contradict the conditions of this License, they do not excuse you from the conditions of this License. If you cannot convey a covered work so as to satisfy simultaneously your obligations under this License and any other pertinent obligations, then as a consequence you may not convey it at all. For example, if you agree to terms that obligate you to collect a royalty for further conveying from those to whom you convey the Program, the only way you could satisfy both those terms and this License would be to refrain entirely from conveying the Program.
13. Use with the GNU Affero General Public License.

Notwithstanding any other provision of this License, you have permission to link or combine any covered work with a work licensed under version 3 of the GNU Affero General Public License into a single combined work, and to convey the resulting work. The terms of this License will continue to apply to the part which is the covered work, but the special requirements of the GNU Affero General Public License, section 13, concerning interaction through a network will apply to the combination as such.
14. Revised Versions of this License.

The Free Software Foundation may publish revised and/or new versions of the GNU General Public License from time to time. Such new versions will be similar in spirit to the present version, but may differ in detail to address new problems or concerns.

Each version is given a distinguishing version number. If the Program specifies that a certain numbered version of the GNU General Public License “or any later version” applies to it, you have the option of following the terms and conditions either of that numbered version or of any later version published by the Free Software Foundation. If the Program does not specify a version number of the GNU General Public License, you may choose any version ever published by the Free Software Foundation.

If the Program specifies that a proxy can decide which future versions of the GNU General Public License can be used, that proxy's public statement of acceptance of a version permanently authorizes you to choose that version for the Program.

Later license versions may give you additional or different permissions. However, no additional obligations are imposed on any author or copyright holder as a result of your choosing to follow a later version.
15. Disclaimer of Warranty.

THERE IS NO WARRANTY FOR THE PROGRAM, TO THE EXTENT PERMITTED BY APPLICABLE LAW. EXCEPT WHEN OTHERWISE STATED IN WRITING THE COPYRIGHT HOLDERS AND/OR OTHER PARTIES PROVIDE THE PROGRAM “AS IS” WITHOUT WARRANTY OF ANY KIND, EITHER EXPRESSED OR IMPLIED, INCLUDING, BUT NOT LIMITED TO, THE IMPLIED WARRANTIES OF MERCHANTABILITY AND FITNESS FOR A PARTICULAR PURPOSE. THE ENTIRE RISK AS TO THE QUALITY AND PERFORMANCE OF THE PROGRAM IS WITH YOU. SHOULD THE PROGRAM PROVE DEFECTIVE, YOU ASSUME THE COST OF ALL NECESSARY SERVICING, REPAIR OR CORRECTION.
16. Limitation of Liability.

IN NO EVENT UNLESS REQUIRED BY APPLICABLE LAW OR AGREED TO IN WRITING WILL ANY COPYRIGHT HOLDER, OR ANY OTHER PARTY WHO MODIFIES AND/OR CONVEYS THE PROGRAM AS PERMITTED ABOVE, BE LIABLE TO YOU FOR DAMAGES, INCLUDING ANY GENERAL, SPECIAL, INCIDENTAL OR CONSEQUENTIAL DAMAGES ARISING OUT OF THE USE OR INABILITY TO USE THE PROGRAM (INCLUDING BUT NOT LIMITED TO LOSS OF DATA OR DATA BEING RENDERED INACCURATE OR LOSSES SUSTAINED BY YOU OR THIRD PARTIES OR A FAILURE OF THE PROGRAM TO OPERATE WITH ANY OTHER PROGRAMS), EVEN IF SUCH HOLDER OR OTHER PARTY HAS BEEN ADVISED OF THE POSSIBILITY OF SUCH DAMAGES.
17. Interpretation of Sections 15 and 16.

If the disclaimer of warranty and limitation of liability provided above cannot be given local legal effect according to their terms, reviewing courts shall apply local law that most closely approximates an absolute waiver of all civil liability in connection with the Program, unless a warranty or assumption of liability accompanies a copy of the Program in return for a fee.

END OF TERMS AND CONDITIONS
How to Apply These Terms to Your New Programs

If you develop a new program, and you want it to be of the greatest possible use to the public, the best way to achieve this is to make it free software which everyone can redistribute and change under these terms.

To do so, attach the following notices to the program. It is safest to attach them to the start of each source file to most effectively state the exclusion of warranty; and each file should have at least the “copyright” line and a pointer to where the full notice is found.

    <one line to give the program's name and a brief idea of what it does.>
    Copyright (C) <year>  <name of author>

    This program is free software: you can redistribute it and/or modify
    it under the terms of the GNU General Public License as published by
    the Free Software Foundation, either version 3 of the License, or
    (at your option) any later version.

    This program is distributed in the hope that it will be useful,
    but WITHOUT ANY WARRANTY; without even the implied warranty of
    MERCHANTABILITY or FITNESS FOR A PARTICULAR PURPOSE.  See the
    GNU General Public License for more details.

    You should have received a copy of the GNU General Public License
    along with this program.  If not, see <http://www.gnu.org/licenses/>.

Also add information on how to contact you by electronic and paper mail.

If the program does terminal interaction, make it output a short notice like this when it starts in an interactive mode:

    <program>  Copyright (C) <year>  <name of author>
    This program comes with ABSOLUTELY NO WARRANTY; for details type `show w'.
    This is free software, and you are welcome to redistribute it
    under certain conditions; type `show c' for details.

The hypothetical commands `show w' and `show c' should show the appropriate parts of the General Public License. Of course, your program's commands might be different; for a GUI interface, you would use an “about box”.

You should also get your employer (if you work as a programmer) or school, if any, to sign a “copyright disclaimer” for the program, if necessary. For more information on this, and how to apply and follow the GNU GPL, see <http://www.gnu.org/licenses/>.

The GNU General Public License does not permit incorporating your program into proprietary programs. If your program is a subroutine library, you may consider it more useful to permit linking proprietary applications with the library. If this is what you want to do, use the GNU Lesser General Public License instead of this License. But first, please read <http://www.gnu.org/philosophy/why-not-lgpl.html>.
\end{multicols}

\section{GNU Free Documentation License}
\begin{multicols}{4}

Version 1.3, 3 November 2008

Copyright © 2000, 2001, 2002, 2007, 2008 Free Software Foundation, Inc. <http://fsf.org/>

Everyone is permitted to copy and distribute verbatim copies of this license document, but changing it is not allowed.
0. PREAMBLE

The purpose of this License is to make a manual, textbook, or other functional and useful document "free" in the sense of freedom: to assure everyone the effective freedom to copy and redistribute it, with or without modifying it, either commercially or noncommercially. Secondarily, this License preserves for the author and publisher a way to get credit for their work, while not being considered responsible for modifications made by others.

This License is a kind of "copyleft", which means that derivative works of the document must themselves be free in the same sense. It complements the GNU General Public License, which is a copyleft license designed for free software.

We have designed this License in order to use it for manuals for free software, because free software needs free documentation: a free program should come with manuals providing the same freedoms that the software does. But this License is not limited to software manuals; it can be used for any textual work, regardless of subject matter or whether it is published as a printed book. We recommend this License principally for works whose purpose is instruction or reference.
1. APPLICABILITY AND DEFINITIONS

This License applies to any manual or other work, in any medium, that contains a notice placed by the copyright holder saying it can be distributed under the terms of this License. Such a notice grants a world-wide, royalty-free license, unlimited in duration, to use that work under the conditions stated herein. The "Document", below, refers to any such manual or work. Any member of the public is a licensee, and is addressed as "you". You accept the license if you copy, modify or distribute the work in a way requiring permission under copyright law.

A "Modified Version" of the Document means any work containing the Document or a portion of it, either copied verbatim, or with modifications and/or translated into another language.

A "Secondary Section" is a named appendix or a front-matter section of the Document that deals exclusively with the relationship of the publishers or authors of the Document to the Document's overall subject (or to related matters) and contains nothing that could fall directly within that overall subject. (Thus, if the Document is in part a textbook of mathematics, a Secondary Section may not explain any mathematics.) The relationship could be a matter of historical connection with the subject or with related matters, or of legal, commercial, philosophical, ethical or political position regarding them.

The "Invariant Sections" are certain Secondary Sections whose titles are designated, as being those of Invariant Sections, in the notice that says that the Document is released under this License. If a section does not fit the above definition of Secondary then it is not allowed to be designated as Invariant. The Document may contain zero Invariant Sections. If the Document does not identify any Invariant Sections then there are none.

The "Cover Texts" are certain short passages of text that are listed, as Front-Cover Texts or Back-Cover Texts, in the notice that says that the Document is released under this License. A Front-Cover Text may be at most 5 words, and a Back-Cover Text may be at most 25 words.

A "Transparent" copy of the Document means a machine-readable copy, represented in a format whose specification is available to the general public, that is suitable for revising the document straightforwardly with generic text editors or (for images composed of pixels) generic paint programs or (for drawings) some widely available drawing editor, and that is suitable for input to text formatters or for automatic translation to a variety of formats suitable for input to text formatters. A copy made in an otherwise Transparent file format whose markup, or absence of markup, has been arranged to thwart or discourage subsequent modification by readers is not Transparent. An image format is not Transparent if used for any substantial amount of text. A copy that is not "Transparent" is called "Opaque".

Examples of suitable formats for Transparent copies include plain ASCII without markup, Texinfo input format, LaTeX input format, SGML or XML using a publicly available DTD, and standard-conforming simple HTML, PostScript or PDF designed for human modification. Examples of transparent image formats include PNG, XCF and JPG. Opaque formats include proprietary formats that can be read and edited only by proprietary word processors, SGML or XML for which the DTD and/or processing tools are not generally available, and the machine-generated HTML, PostScript or PDF produced by some word processors for output purposes only.

The "Title Page" means, for a printed book, the title page itself, plus such following pages as are needed to hold, legibly, the material this License requires to appear in the title page. For works in formats which do not have any title page as such, "Title Page" means the text near the most prominent appearance of the work's title, preceding the beginning of the body of the text.

The "publisher" means any person or entity that distributes copies of the Document to the public.

A section "Entitled XYZ" means a named subunit of the Document whose title either is precisely XYZ or contains XYZ in parentheses following text that translates XYZ in another language. (Here XYZ stands for a specific section name mentioned below, such as "Acknowledgements", "Dedications", "Endorsements", or "History".) To "Preserve the Title" of such a section when you modify the Document means that it remains a section "Entitled XYZ" according to this definition.

The Document may include Warranty Disclaimers next to the notice which states that this License applies to the Document. These Warranty Disclaimers are considered to be included by reference in this License, but only as regards disclaiming warranties: any other implication that these Warranty Disclaimers may have is void and has no effect on the meaning of this License.
2. VERBATIM COPYING

You may copy and distribute the Document in any medium, either commercially or noncommercially, provided that this License, the copyright notices, and the license notice saying this License applies to the Document are reproduced in all copies, and that you add no other conditions whatsoever to those of this License. You may not use technical measures to obstruct or control the reading or further copying of the copies you make or distribute. However, you may accept compensation in exchange for copies. If you distribute a large enough number of copies you must also follow the conditions in section 3.

You may also lend copies, under the same conditions stated above, and you may publicly display copies.
3. COPYING IN QUANTITY

If you publish printed copies (or copies in media that commonly have printed covers) of the Document, numbering more than 100, and the Document's license notice requires Cover Texts, you must enclose the copies in covers that carry, clearly and legibly, all these Cover Texts: Front-Cover Texts on the front cover, and Back-Cover Texts on the back cover. Both covers must also clearly and legibly identify you as the publisher of these copies. The front cover must present the full title with all words of the title equally prominent and visible. You may add other material on the covers in addition. Copying with changes limited to the covers, as long as they preserve the title of the Document and satisfy these conditions, can be treated as verbatim copying in other respects.

If the required texts for either cover are too voluminous to fit legibly, you should put the first ones listed (as many as fit reasonably) on the actual cover, and continue the rest onto adjacent pages.

If you publish or distribute Opaque copies of the Document numbering more than 100, you must either include a machine-readable Transparent copy along with each Opaque copy, or state in or with each Opaque copy a computer-network location from which the general network-using public has access to download using public-standard network protocols a complete Transparent copy of the Document, free of added material. If you use the latter option, you must take reasonably prudent steps, when you begin distribution of Opaque copies in quantity, to ensure that this Transparent copy will remain thus accessible at the stated location until at least one year after the last time you distribute an Opaque copy (directly or through your agents or retailers) of that edition to the public.

It is requested, but not required, that you contact the authors of the Document well before redistributing any large number of copies, to give them a chance to provide you with an updated version of the Document.
4. MODIFICATIONS

You may copy and distribute a Modified Version of the Document under the conditions of sections 2 and 3 above, provided that you release the Modified Version under precisely this License, with the Modified Version filling the role of the Document, thus licensing distribution and modification of the Modified Version to whoever possesses a copy of it. In addition, you must do these things in the Modified Version:

    * A. Use in the Title Page (and on the covers, if any) a title distinct from that of the Document, and from those of previous versions (which should, if there were any, be listed in the History section of the Document). You may use the same title as a previous version if the original publisher of that version gives permission.
    * B. List on the Title Page, as authors, one or more persons or entities responsible for authorship of the modifications in the Modified Version, together with at least five of the principal authors of the Document (all of its principal authors, if it has fewer than five), unless they release you from this requirement.
    * C. State on the Title page the name of the publisher of the Modified Version, as the publisher.
    * D. Preserve all the copyright notices of the Document.
    * E. Add an appropriate copyright notice for your modifications adjacent to the other copyright notices.
    * F. Include, immediately after the copyright notices, a license notice giving the public permission to use the Modified Version under the terms of this License, in the form shown in the Addendum below.
    * G. Preserve in that license notice the full lists of Invariant Sections and required Cover Texts given in the Document's license notice.
    * H. Include an unaltered copy of this License.
    * I. Preserve the section Entitled "History", Preserve its Title, and add to it an item stating at least the title, year, new authors, and publisher of the Modified Version as given on the Title Page. If there is no section Entitled "History" in the Document, create one stating the title, year, authors, and publisher of the Document as given on its Title Page, then add an item describing the Modified Version as stated in the previous sentence.
    * J. Preserve the network location, if any, given in the Document for public access to a Transparent copy of the Document, and likewise the network locations given in the Document for previous versions it was based on. These may be placed in the "History" section. You may omit a network location for a work that was published at least four years before the Document itself, or if the original publisher of the version it refers to gives permission.
    * K. For any section Entitled "Acknowledgements" or "Dedications", Preserve the Title of the section, and preserve in the section all the substance and tone of each of the contributor acknowledgements and/or dedications given therein.
    * L. Preserve all the Invariant Sections of the Document, unaltered in their text and in their titles. Section numbers or the equivalent are not considered part of the section titles.
    * M. Delete any section Entitled "Endorsements". Such a section may not be included in the Modified Version.
    * N. Do not retitle any existing section to be Entitled "Endorsements" or to conflict in title with any Invariant Section.
    * O. Preserve any Warranty Disclaimers.

If the Modified Version includes new front-matter sections or appendices that qualify as Secondary Sections and contain no material copied from the Document, you may at your option designate some or all of these sections as invariant. To do this, add their titles to the list of Invariant Sections in the Modified Version's license notice. These titles must be distinct from any other section titles.

You may add a section Entitled "Endorsements", provided it contains nothing but endorsements of your Modified Version by various parties—for example, statements of peer review or that the text has been approved by an organization as the authoritative definition of a standard.

You may add a passage of up to five words as a Front-Cover Text, and a passage of up to 25 words as a Back-Cover Text, to the end of the list of Cover Texts in the Modified Version. Only one passage of Front-Cover Text and one of Back-Cover Text may be added by (or through arrangements made by) any one entity. If the Document already includes a cover text for the same cover, previously added by you or by arrangement made by the same entity you are acting on behalf of, you may not add another; but you may replace the old one, on explicit permission from the previous publisher that added the old one.

The author(s) and publisher(s) of the Document do not by this License give permission to use their names for publicity for or to assert or imply endorsement of any Modified Version.
5. COMBINING DOCUMENTS

You may combine the Document with other documents released under this License, under the terms defined in section 4 above for modified versions, provided that you include in the combination all of the Invariant Sections of all of the original documents, unmodified, and list them all as Invariant Sections of your combined work in its license notice, and that you preserve all their Warranty Disclaimers.

The combined work need only contain one copy of this License, and multiple identical Invariant Sections may be replaced with a single copy. If there are multiple Invariant Sections with the same name but different contents, make the title of each such section unique by adding at the end of it, in parentheses, the name of the original author or publisher of that section if known, or else a unique number. Make the same adjustment to the section titles in the list of Invariant Sections in the license notice of the combined work.

In the combination, you must combine any sections Entitled "History" in the various original documents, forming one section Entitled "History"; likewise combine any sections Entitled "Acknowledgements", and any sections Entitled "Dedications". You must delete all sections Entitled "Endorsements".
6. COLLECTIONS OF DOCUMENTS

You may make a collection consisting of the Document and other documents released under this License, and replace the individual copies of this License in the various documents with a single copy that is included in the collection, provided that you follow the rules of this License for verbatim copying of each of the documents in all other respects.

You may extract a single document from such a collection, and distribute it individually under this License, provided you insert a copy of this License into the extracted document, and follow this License in all other respects regarding verbatim copying of that document.
7. AGGREGATION WITH INDEPENDENT WORKS

A compilation of the Document or its derivatives with other separate and independent documents or works, in or on a volume of a storage or distribution medium, is called an "aggregate" if the copyright resulting from the compilation is not used to limit the legal rights of the compilation's users beyond what the individual works permit. When the Document is included in an aggregate, this License does not apply to the other works in the aggregate which are not themselves derivative works of the Document.

If the Cover Text requirement of section 3 is applicable to these copies of the Document, then if the Document is less than one half of the entire aggregate, the Document's Cover Texts may be placed on covers that bracket the Document within the aggregate, or the electronic equivalent of covers if the Document is in electronic form. Otherwise they must appear on printed covers that bracket the whole aggregate.
8. TRANSLATION

Translation is considered a kind of modification, so you may distribute translations of the Document under the terms of section 4. Replacing Invariant Sections with translations requires special permission from their copyright holders, but you may include translations of some or all Invariant Sections in addition to the original versions of these Invariant Sections. You may include a translation of this License, and all the license notices in the Document, and any Warranty Disclaimers, provided that you also include the original English version of this License and the original versions of those notices and disclaimers. In case of a disagreement between the translation and the original version of this License or a notice or disclaimer, the original version will prevail.

If a section in the Document is Entitled "Acknowledgements", "Dedications", or "History", the requirement (section 4) to Preserve its Title (section 1) will typically require changing the actual title.
9. TERMINATION

You may not copy, modify, sublicense, or distribute the Document except as expressly provided under this License. Any attempt otherwise to copy, modify, sublicense, or distribute it is void, and will automatically terminate your rights under this License.

However, if you cease all violation of this License, then your license from a particular copyright holder is reinstated (a) provisionally, unless and until the copyright holder explicitly and finally terminates your license, and (b) permanently, if the copyright holder fails to notify you of the violation by some reasonable means prior to 60 days after the cessation.

Moreover, your license from a particular copyright holder is reinstated permanently if the copyright holder notifies you of the violation by some reasonable means, this is the first time you have received notice of violation of this License (for any work) from that copyright holder, and you cure the violation prior to 30 days after your receipt of the notice.

Termination of your rights under this section does not terminate the licenses of parties who have received copies or rights from you under this License. If your rights have been terminated and not permanently reinstated, receipt of a copy of some or all of the same material does not give you any rights to use it.
10. FUTURE REVISIONS OF THIS LICENSE

The Free Software Foundation may publish new, revised versions of the GNU Free Documentation License from time to time. Such new versions will be similar in spirit to the present version, but may differ in detail to address new problems or concerns. See http://www.gnu.org/copyleft/.

Each version of the License is given a distinguishing version number. If the Document specifies that a particular numbered version of this License "or any later version" applies to it, you have the option of following the terms and conditions either of that specified version or of any later version that has been published (not as a draft) by the Free Software Foundation. If the Document does not specify a version number of this License, you may choose any version ever published (not as a draft) by the Free Software Foundation. If the Document specifies that a proxy can decide which future versions of this License can be used, that proxy's public statement of acceptance of a version permanently authorizes you to choose that version for the Document.
11. RELICENSING

"Massive Multiauthor Collaboration Site" (or "MMC Site") means any World Wide Web server that publishes copyrightable works and also provides prominent facilities for anybody to edit those works. A public wiki that anybody can edit is an example of such a server. A "Massive Multiauthor Collaboration" (or "MMC") contained in the site means any set of copyrightable works thus published on the MMC site.

"CC-BY-SA" means the Creative Commons Attribution-Share Alike 3.0 license published by Creative Commons Corporation, a not-for-profit corporation with a principal place of business in San Francisco, California, as well as future copyleft versions of that license published by that same organization.

"Incorporate" means to publish or republish a Document, in whole or in part, as part of another Document.

An MMC is "eligible for relicensing" if it is licensed under this License, and if all works that were first published under this License somewhere other than this MMC, and subsequently incorporated in whole or in part into the MMC, (1) had no cover texts or invariant sections, and (2) were thus incorporated prior to November 1, 2008.

The operator of an MMC Site may republish an MMC contained in the site under CC-BY-SA on the same site at any time before August 1, 2009, provided the MMC is eligible for relicensing.
ADDENDUM: How to use this License for your documents

To use this License in a document you have written, include a copy of the License in the document and put the following copyright and license notices just after the title page:

    Copyright (C)  YEAR  YOUR NAME.
    Permission is granted to copy, distribute and/or modify this document
    under the terms of the GNU Free Documentation License, Version 1.3
    or any later version published by the Free Software Foundation;
    with no Invariant Sections, no Front-Cover Texts, and no Back-Cover Texts.
    A copy of the license is included in the section entitled "GNU
    Free Documentation License".

If you have Invariant Sections, Front-Cover Texts and Back-Cover Texts, replace the "with … Texts." line with this:

    with the Invariant Sections being LIST THEIR TITLES, with the
    Front-Cover Texts being LIST, and with the Back-Cover Texts being LIST.

If you have Invariant Sections without Cover Texts, or some other combination of the three, merge those two alternatives to suit the situation.

If your document contains nontrivial examples of program code, we recommend releasing these examples in parallel under your choice of free software license, such as the GNU General Public License, to permit their use in free software.
\end{multicols}

\section{GNU Lesser General Public License}
\begin{multicols}{4}


GNU LESSER GENERAL PUBLIC LICENSE

Version 3, 29 June 2007

Copyright © 2007 Free Software Foundation, Inc. <http://fsf.org/>

Everyone is permitted to copy and distribute verbatim copies of this license document, but changing it is not allowed.

This version of the GNU Lesser General Public License incorporates the terms and conditions of version 3 of the GNU General Public License, supplemented by the additional permissions listed below.
0. Additional Definitions.

As used herein, “this License” refers to version 3 of the GNU Lesser General Public License, and the “GNU GPL” refers to version 3 of the GNU General Public License.

“The Library” refers to a covered work governed by this License, other than an Application or a Combined Work as defined below.

An “Application” is any work that makes use of an interface provided by the Library, but which is not otherwise based on the Library. Defining a subclass of a class defined by the Library is deemed a mode of using an interface provided by the Library.

A “Combined Work” is a work produced by combining or linking an Application with the Library. The particular version of the Library with which the Combined Work was made is also called the “Linked Version”.

The “Minimal Corresponding Source” for a Combined Work means the Corresponding Source for the Combined Work, excluding any source code for portions of the Combined Work that, considered in isolation, are based on the Application, and not on the Linked Version.

The “Corresponding Application Code” for a Combined Work means the object code and/or source code for the Application, including any data and utility programs needed for reproducing the Combined Work from the Application, but excluding the System Libraries of the Combined Work.
1. Exception to Section 3 of the GNU GPL.

You may convey a covered work under sections 3 and 4 of this License without being bound by section 3 of the GNU GPL.
2. Conveying Modified Versions.

If you modify a copy of the Library, and, in your modifications, a facility refers to a function or data to be supplied by an Application that uses the facility (other than as an argument passed when the facility is invoked), then you may convey a copy of the modified version:

    * a) under this License, provided that you make a good faith effort to ensure that, in the event an Application does not supply the function or data, the facility still operates, and performs whatever part of its purpose remains meaningful, or
    * b) under the GNU GPL, with none of the additional permissions of this License applicable to that copy.

3. Object Code Incorporating Material from Library Header Files.

The object code form of an Application may incorporate material from a header file that is part of the Library. You may convey such object code under terms of your choice, provided that, if the incorporated material is not limited to numerical parameters, data structure layouts and accessors, or small macros, inline functions and templates (ten or fewer lines in length), you do both of the following:

    * a) Give prominent notice with each copy of the object code that the Library is used in it and that the Library and its use are covered by this License.
    * b) Accompany the object code with a copy of the GNU GPL and this license document.

4. Combined Works.

You may convey a Combined Work under terms of your choice that, taken together, effectively do not restrict modification of the portions of the Library contained in the Combined Work and reverse engineering for debugging such modifications, if you also do each of the following:

    * a) Give prominent notice with each copy of the Combined Work that the Library is used in it and that the Library and its use are covered by this License.
    * b) Accompany the Combined Work with a copy of the GNU GPL and this license document.
    * c) For a Combined Work that displays copyright notices during execution, include the copyright notice for the Library among these notices, as well as a reference directing the user to the copies of the GNU GPL and this license document.
    * d) Do one of the following:
          o 0) Convey the Minimal Corresponding Source under the terms of this License, and the Corresponding Application Code in a form suitable for, and under terms that permit, the user to recombine or relink the Application with a modified version of the Linked Version to produce a modified Combined Work, in the manner specified by section 6 of the GNU GPL for conveying Corresponding Source.
          o 1) Use a suitable shared library mechanism for linking with the Library. A suitable mechanism is one that (a) uses at run time a copy of the Library already present on the user's computer system, and (b) will operate properly with a modified version of the Library that is interface-compatible with the Linked Version.
    * e) Provide Installation Information, but only if you would otherwise be required to provide such information under section 6 of the GNU GPL, and only to the extent that such information is necessary to install and execute a modified version of the Combined Work produced by recombining or relinking the Application with a modified version of the Linked Version. (If you use option 4d0, the Installation Information must accompany the Minimal Corresponding Source and Corresponding Application Code. If you use option 4d1, you must provide the Installation Information in the manner specified by section 6 of the GNU GPL for conveying Corresponding Source.)

5. Combined Libraries.

You may place library facilities that are a work based on the Library side by side in a single library together with other library facilities that are not Applications and are not covered by this License, and convey such a combined library under terms of your choice, if you do both of the following:

    * a) Accompany the combined library with a copy of the same work based on the Library, uncombined with any other library facilities, conveyed under the terms of this License.
    * b) Give prominent notice with the combined library that part of it is a work based on the Library, and explaining where to find the accompanying uncombined form of the same work.

6. Revised Versions of the GNU Lesser General Public License.

The Free Software Foundation may publish revised and/or new versions of the GNU Lesser General Public License from time to time. Such new versions will be similar in spirit to the present version, but may differ in detail to address new problems or concerns.

Each version is given a distinguishing version number. If the Library as you received it specifies that a certain numbered version of the GNU Lesser General Public License “or any later version” applies to it, you have the option of following the terms and conditions either of that published version or of any later version published by the Free Software Foundation. If the Library as you received it does not specify a version number of the GNU Lesser General Public License, you may choose any version of the GNU Lesser General Public License ever published by the Free Software Foundation.

If the Library as you received it specifies that a proxy can decide whether future versions of the GNU Lesser General Public License shall apply, that proxy's public statement of acceptance of any version is permanent authorization for you to choose that version for the Library.
\end{multicols}
}
\pagebreak
\end{document}

